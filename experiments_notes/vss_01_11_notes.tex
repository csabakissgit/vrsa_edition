\documentclass{article}
\usepackage[utf8x]{inputenx}
\newcommand{\vsnum}[1]{\textbf{#1}}
\newcommand{\skt}[1]{\textit{#1}}
\newcommand{\msCa}{msCa}
\newcommand{\msCb}{msCb}
\newcommand{\msCc}{msCc}
\newcommand{\msNa}{msNa}
\newcommand{\msNb}{msNb}
\newcommand{\msNc}{msNc}
\newcommand{\msP}{P}
\newcommand{\Ed}{Ed}
\newcommand{\vo}{ X }




\begin{document}


%%%%%%%%%
\vsnum{1.1:} \skt{Pāda} a is reminiscent of, among other famous passages, Bhagavadgītā 11.19: \\ \skt{anādimadhyāntam anantavīryam \\ anantabāhuṃ śaśisūryanetram / \\ paśyāmi tvāṃ dīptahutāśavaktraṃ \\ svatejasā viśvam idaṃ tapantam //} \\ This faint reference to the Bhagavadgītā seems proper at the beginning of a work that claims to deliver a teaching based on, but also to surpass, the Mahābhārata (see following verses). See also e.g. Kūrmapurāṇa 1.11.237:\\ \skt{rūpaṃ tavośeṣakalāvihīnam [tavā? CHECK]\\ agocaraṃ nirmalam ekarūpam / \\ anādimadhyāntam anantām [anantam? CHECK] ādyaṃ \\ namāmi satyaṃ tamasaḥ parastāt} // \\ To say that a god has no beginning and no end in a temporal or spacial sense is natural (\skt{anādi°...°antam}), but to have no `middle part' (\skt{°madhya°}) in these senses is slightly less so. Thus the rather commonly occuring phrase \skt{anādimadhyāntam} is probably not much more than a fixed expression meaning `endless and/or eternal'. As to which god this stanza is referring to, it may be Śiva, his name not being listed among those who treat him as chief god, but the phrasing of the verse is vague enough to keep the question somewhat open: the impersonal Brahman might be another option, even more so if we look at 1.9--10, two verses nearby verses discussing \skt{brahmavidyā}.  \\ In \skt{pāda} b \skt{jagat-susāraṃ} is most probably not to be interpreted as \skt{jagatsu sāraṃ}. \\ Strictly speaking, \skt{pāda} c is unmetrical, but it is better to simply acknowledge here the phenomenon of `muta cum liquida', namely that syllables followed by consonant clusters such as \skt{ra, bra, hra, kra, śra, śya, śva, sva, dva} can be treated as short. Thus \skt{harīndrabrahmā°} can be treated as a regular beginning of an \skt{upajāti} (. - . - -), the syllable \skt{bra} not turning the previous syllable long. \\ The reading \skt{āsamagraṃ} in \skt{pāda} c is difficult to interpret. The most tempting of all the possible corrections I have considered (\skt{arcyam/arhyam/arghyam/īḍyam agraṃ}) seemed to be \skt{āptam agraṃ}, meaning `appointed/received/respected [by Hari, Indra, Brahmā etc.] as the foremost one'. The fact that the \skt{akṣara}s \skt{āsam} and \skt{āptam} look similar in most of the scripts used in our manuscripts supports this conjecture. \\ Note how we could percieve the end of \skt{pāda}s a and b, as well as \skt{pāda}s c and d as rhymes. \\ Is pāda d hypermetrical? It is actually a \skt{vaṃśastha} (\skt{triṣṭubh - jagatī} change). See Apte App. A p. 4.  \vsnum{1.2: }The dialouge of Janamejaya and Vaiśampāyana make up the outermost layer of the VSS (except for the introductory stanzas 1.1-3), which mostly contains general \skt{dharmaśāstric} material. \\ The hundred \skt{parvan}s of the Mahābhārata are listed in MBh 1.2.33--70.

\vsnum{1.3: }For a similar unsatisfaction or dissatisfaction with previous teachings, see Niśvāsa mūla 1.9: \\ \textit{vedāntaṃ viditaṃ deva sāṃkhyaṃ vai pañcaviṃśakam $|$na ca tṛptiṃ gamiṣyāmo hy ṛte śaivād anugrahāt $||$} \\ and Śivadharmaśāstra... CHECK. Vaiśampāyana, a ṛṣi, the disciple of Vyāsa, recited the Mahābhārata at the snake sacrifice of Janamejaya. CHECK SOURCE \\ Note how we are forced to emend \skt{pāda} c to contain a stem form proper noun (\skt{janamejaya}) to maintain the metre, and note how the manuscripts struggle with this \skt{pāda}.  \vsnum{1.4: }Note \skt{dharma} as a neuter noun in \skt{pāda} c and in the next verse.

\vsnum{1.5: }The majority of the MSS consulted include a \skt{vā} in \skt{pāda} b, distinguishing between the `secret Dharma' mentioned in 1.4c and the one taught by Vyāsa. This may or may not be the better reading. I decided to follow MS \msCb\ because I suspected that the two Dharmas hinted at are the same. Note the odd syntax here: \skt{viṣṇunā... dvijarūpadharo bhūtvā papraccha}. The agent of the active verb is in the instrumental case.

\vsnum{1.9: }The translation of this verse, and the reconstruction and interpretation of \skt{pāda} d, which is echoed in 1.10d, is slightly tentative.

\vsnum{1.10: }I interpret \skt{pāda} d, which is an echo of 1.9d, tentitvely as a compound, slightly differently from the way I did above.

\vsnum{1.11: }The word \skt{°śivā°} in \skt{pāda} b is slightly suspect, and could be the result of metathesis, from \skt{°viṣā°} (`by poison'). Nevertheless, jackals seems appropriate in this context, for they are commonly associated with human corpses, death and the cremation ground. (see e.g. Ohnuma 2019) (Reiko Ohnuma 2019 = Ohnuma, R. The Heretical, Heterodox Howl: Jackals in Pāli Buddhist Literature. Religions 2019, 10, 221.) The word \skt{kāla} has, as usual, a double meaning in this verse: a \skt{kālapāśa} is both Yama's noose, and also the limitation caused by time, as becomes clear at the discussion on the different time units in verses 1.11--31. The variant \skt{jijñāsyasi} seems to be the lectio difficilior as opposed to \skt{vijñāsyasi}, but the latter could also work fine here. 1.18d and 1.19a are problematic in the light of 1.19b, which redefines \skt{kalā} in harmony with the traditional interpretaion, see e.g. Arthaśāstra 2.20.33: \skt{trimśatkāṣṭhāḥ kalāḥ}.

\vsnum{1.22: }Note the stem form noun \skt{yuga}.

\vsnum{1.23: }The element \skt{°yugā°} seems to stand for °yuga° metri causa. If \skt{°yugā} and \skt{saṃkhyā} are to be separated, \skt{eṣā} becomes problematic to interpret.

\vsnum{1.24: }See 21.34ff. The plural form \skt{pralīyante} in \skt{pāda} a is metri causa for \skt{pralīyate}, perhaps also influencing \skt{utpadyante} (for \skt{utpadyate}) in \skt{pāda} d, which in turn is used here to avoid an iambic pattern (- - . - . - . -).

\vsnum{1.27: }Note the peculiar compound \skt{bhṛgu-r-ādi-maharṣayaḥ}.

\vsnum{1.30: }Note that \skt{samatītāni} (neuter) most probably picks up \skt{devarājāḥ} (masculine) in this verse, or rather \skt{devarājā} stands for \skt{devarājānāṃ} and \skt{samatītāni} picks up \skt{°parārdhāni}.

\vsnum{1.32: }The reading of all manuscripts consulted, \skt{vinisṛtam}, may be considered metrical if we interpret it, loosely, as \skt{vinisritam}. \\ \skt{Pāda} d is suspicious and my translation is tentative.

\vsnum{1.34: }For \skt{anta} meaning \skt{ananta}, see 1.58cd-59ab.

\vsnum{1.37: }The word \skt{prāpitaṃ} is a conjecture for \skt{cāpitaṃ}, which I find unintelligible. Another possibility could be \skt{jñāpitaṃ}. The purport of \skt{pāda}s c and d is slightly obscure to me.

\vsnum{1.38: }One would expect \skt{brahmāṇḍāni} in \skt{pāda} a instead of \skt{brahmāṇḍānāṃ}, but we should probably understand \skt{brahmāṇḍānāṃ viśeṣān prasaṃkhyātuṃ...} Note that in \skt{pāda} d \skt{mātariśvan} stands for the accusative \skt{mātariśvānaṃ} or the dative \skt{mātariśvane} or the genitive \skt{mātariśvanaḥ}. The claim that Brahmā taught Mātariśvan is confirmed in 1.64cd, again using the nominative for the accusative, dative or genitive, and also e.g. in Brahmāṇḍapurāṇa 3.4.58cd.

\vsnum{1.40: }The cruxed \skt{pāda} may have read \skt{sarveṣām eva pūjitāḥ} originally (`They are worshipped by all'). \\ In \skt{pāda} c, understand \skt{diśāṣṭānāṃ} as \skt{diśām aṣṭānāṃ} or \skt{digaṣṭakānāṃ}

\vsnum{1.41: }I chose to supply an \skt{avagraha} before \skt{sabhā} only because all the sources consulted read \skt{saṃhato} as the previous word, making the \skt{sandhi} \skt{o-s} suspicious. Note that many of the names here and in the following verses are, in the absence of any parallel passage, rather insecure. What is clear here is that the names evoke the name Sahasrākṣa, one of the appellations of Indra, the guadrian of the eastern direction. I have choosen the variant \skt{saṃyano} in \skt{pāda} c only to avoid the repetition of the name \skt{saṃyama}, and the variant \skt{yanoyanaḥ} because I suspect that most of the names here should begin with \skt{ya}. All the name forms in this verse are to be taken as tentative. The only guiding light is the presence of \skt{ya}, reinforcing their connection with Yama. Note that the reconstruction of these names are tentative. What is clear here is that the initials should be \skt{na} and \skt{ga}, probably suggesting a connection with \skt{nāga}s. We are forced to follow \Ed's readings here to make sense of this passage. Note that \skt{vṛnda} is not a number here. Elsewhere in this chapter it is the word that signifies `a billion'.

\vsnum{1.57: }Note \skt{śaṅkubhiḥ pṛthag...}: it stands for \skt{śaṅkūṣu pṛthag...} (instrumental for locative). The translation of \skt{pāda}s c and d is tentative.

\vsnum{1.61: }aṇḍānāṃ plural...: a new egg in every mahākalpa? CHECK

\vsnum{1.63: }Note the mixture of different grammatical genders and numbers here. Understand \skt{pramāṇeṣu saṃkhyāḥ kīrtitāḥ samāsataḥ}. \skt{Pāda} a should probably be analysed and interpreted as \skt{purāṇam (purāṇānām aśītisahasrāṇi śatāni ślokāni) brahmaṇā kathitam}. \\ Compare this list to Viṣṇupurāṇa 3.3.11--19: \\ \skt{dvāpare prathame vyastaḥ svayaṃ vedaḥ svayaṃbhuvā|\\ dvitīye dvāpare caiva vedavyāsaḥ prajāpati$||$\\ tṛtīye cośanā vyāsaś caturthe ca bṛhaspatiḥ$|$\\ savitā pañcame vyāsaḥ ṣaṣṭhe mṛtyuḥ smṛtaḥ prabhuḥ$||$ \\ saptame ca tathaivendro vasiṣṭhaś cāṣṭame smṛtaḥ$|$\\ sārasvataś ca navame tridhāmā daśame smṛtaḥ$||$ \\ ekādaśe tu triśikho bharadvājas tataḥ paraḥ$|$\\ trayodaśe cāntarikṣo varṇī cāpi caturdaśe$||$ \\ trayyāruṇaḥ pañcadaśe ṣoḍaśe tu dhanañjayaḥ$|$\\ kratuñjayaḥ saptadaśe tadūrdhvaṃ ca jayaḥ smṛtaḥ$||$ \\ tato vyāso bharadvājo bharadvājāc ca gautamaḥ$|$\\ gautamād uttaro vyāso haryātmā yo 'bhidhīyate$||$ \\ atha haryātmanonte ca smṛto vājaśravāmuniḥ$|$\\ somaśuṣkāyaṇas tasmāt tṛṇabindur iti smṛtaḥ$||$ \\ ṛkṣobhūdbhārgavas tasmād vālmīkir yo 'bhidhīyate$|$\\ tasmād asmatpitā śaktir vyāsas tasmād ahaṃ mune$||$ \\ jātukarṇo 'bhavan mattaḥ kṛṣṇadvaipāyanas tataḥ$|$\\ aṣṭaviṃśatir ity ete vedavyāsāḥ purātanāḥ$||$ \\} \\ Another relevant passage is Brahmāṇḍapurāṇa 3.4.58cd--67: \\ \skt{brahmā dadau śāstram idaṃ purāṇaṃ mātariśvane$||$ \\ tasmāc cośanasā prāptaṃ tasmāc cāpi bṛhaspatiḥ$|$\\ bṛhaspatis tu provāca savitre tadanantaram$||$ \\ savitā mṛtyave prāha mṛtyuś cendrāya vai punaḥ$|$\\ indraś cāpi vasiṣṭāya so 'pi sārasvatāya cai$||$ \\ sārasvatas tridhāmne 'tha tridhāmā ca śaradvate$|$\\ śaradvāṃs tu triviṣṭāya so 'ntarikṣāya dattavān$||$ \\ carṣiṇe cāntarikṣo vai so 'pi trayyāruṇāya ca$|$\\ trayyāruṇād dhanañjayaḥ sa vai prādāt kṛtañjaye$||$ \\ kṛtañjayāt tṛṇañjayo bharadvājāya so 'py atha$|$\\ gautamāya bharadvājaḥ so 'pi niryyantare punaḥ$||$ \\ niryyantaras tu provāca tathā vājaśravāya vai$|$\\ sa dadau somaśuṣmāya sa cādāt tṛṇabindave$||$ \\ tṛṇabindus tu dakṣāya dakṣaḥ provāca śaktaye$|$\\ śakteḥ parāśaraś cāpi garbhasthaḥ śrutavānidam$||$ \\ parāśarāj jātukarṇyas tasmād dvaipāyanaḥ prabhuḥ$|$\\ dvaipāyanāt punaś cāpi mayā prāptaṃ dvijottama$||$ \\ mayā caitat punaḥ proktaṃ putrāyāmitabuddhaye$|$\\ ity eva vākyaṃ brahmādiguruṇāṃ samudāhṛtam$||$ } \\  The name \skt{harmyadvata} is probably a variant or a corrupted form of \skt{harmyātman}, who appears in lists of \skt{vedavyāsa}s in the Purāṇas (see note to 1.64).

\vsnum{1.75: }Perhaps keep jatu°.


%%%%%%%%%
\vsnum{2.21: }After kāmarū°, MS \msCc\ has some folios missing and resumes only at 3.XX. CHECK Florinda's pics!

\vsnum{2.25: }Pāda c is unmetrical, or rather, a ra-vipulā with licence (tatraiva as SHORT-LONG). Note also the gender problem (\textit{bhogam akṣayas}), or rather take \textit{-m-} as a sandhi-bridge (\textit{bhoga-m-akṣayas}, for \textit{bhogo 'kṣayas}).

\vsnum{2.28: }Note the Aiśa form \textit{diśiṃ} in <ms>C<sub>45</sub></ms>.

\vsnum{2.29: }Note the Aiśa form \textit{diśiṃ} in <ms>K<sub>07</sub></ms> in pāda b. In pāda d, we may suppose the presence of a sandhi-bridge: \textit{sadya-m-iṣṭālayaḥ}.

\vsnum{2.30: }Note the Aiśa form \textit{diśiṃ} in <ms>C<sub>95</sub></ms> in pāda b.

\vsnum{2.31: }Note how \textit{vaktrasya} should refer to Śiva's Tatpuruṣa-face, given that the text lists Śiva's five faces: Īśāna, Tatpuruṣa, Aghora, Vāmadeva, Sadyojāta.

\vsnum{2.35: }Understand \skt{kṛcchrāditapa sarvāṇi} as \skt{kṛcchrāditapāṃsi sarvāṇi}.


%%%%%%%%%
\vsnum{3.1: }For the correct interpretation of \skt{pāda} a, namely to decide whether these questions focus on the bull of Dharma or Dharma itself/himself, see MBh 12.110.10--11: \\ \skt{prabhāvārthāya bhūtānāṃ dharmapravacanaṃ kṛtam|\\ yat syād ahiṃsāsaṃyuktaṃ sa dharma iti niścayaḥ$||$\\ dhāraṇād dharma ity āhur dharmeṇa vidhṛtāḥ prajāḥ|\\ yat syād dhāraṇasaṃyuktaṃ sa dharma iti niścayaḥ$||$} \\ Note the similarities with this chapter: the phrase \skt{dharma ity āhur}, the fact that the present chapter from verse 18 on is actually a chapter on \skt{ahiṃsā}, and that the etimological explanation involves the word [\skt{ā}]\skt{dhāraṇa} in both cases. These lead me to think that in \skt{pāda}s ab of this verse in the VSS, it is Dharma that is the focus of the inquiry and not the bull. Understand \skt{pāda} d as \skt{gatayas tasya kati smṛtāḥ}. I have accepted \skt{smṛtāḥ} because this plural signals that \skt{gatis} is meant to be plural, similarly to what happens in 3.6cd (\skt{tasya patnī... mahābhāgāḥ}).

\vsnum{3.3: }On a non-verbal stem being a \skt{dhātu}, see e.g. Vāyupurāṇa 3.17cd: \skt{bhāvya ity eṣa dhātur vai bhāvye kāle vibhāvyate}; Vāyupurāṇa 3.19cd (= Brahmāṇḍapurāṇa 1.38.21ab): \skt{nātha ity eṣa dhātur vai dhātujñaiḥ pālane smṛtaḥ}; Liṅgapurāṇa 2.9.19: \skt{bhaja ity eṣa dhātur vai sevāyāṃ parikīrtitaḥ}; etc.

\vsnum{3.4: }A similar image of the legs of the Bull of Dharma being the four \skt{āśrama}s is hinted at MBh 12.262.19--21: \\ \skt{dharmam{ }ekaṃ catuṣpādam{ }āśritās{ }te nararṣabhāḥ$|$\\taṃ santo vidhivat{ }prāpya gacchanti paramāṃ gatim$||$ \\gṛhebhya eva niṣkramya vanam{ }anye samāśritāḥ$|$\\gṛham{ }evābhisaṃśritya tato 'nye brahmacāriṇaḥ$||$ \\dharmam{ }etaṃ catuṣpādam{ }āśramaṃ brāhmaṇā viduḥ$|$\\ānantyaṃ brahmaṇaḥ sthānaṃ brāhmaṇā nāma niścayaḥ$||$} \\ On the more frequently quoted interpretation of the four legs, see Olivelle `Āśrama', 235: ``Dharma and truth possess all four feet and are whole during the Kṛta yuga, and people did not obtain anything unrighteously (\skt{adharmeṇa}). By obtaining, however, \skt{dharma} has lost one foot during each of the other \skt{yuga}s and righteousness (\skt{dharma}) likewise has diminished by one quarter due to theft, falsehood, and deceit. (MDh 1.81--82)'' \\ Understand \skt{pāda}s c and d as \skt{catvāri āśramāṇi kīrtitāni dharmo manīṣibhiḥ} or \skt{yo dharmaḥ kīrtitaś caturāśramāṇi manīṣibhiḥ} or \skt{yo dharmaś caturāśramaḥ kīrtito manīṣibhiḥ}.

\vsnum{3.5: }Understand \skt{gatiś} as \skt{gatayaś} and note that \skt{vijñeyāḥ} is an emendation from \skt{vijñeyaḥ} following the logic of 3.1d. \skt{tirya} seems to be an acceptable nominal stem in this text for \skt{tiryañc}. See e.g. 4.6a: \skt{devamānuṣatiryeṣu}. \skt{°ādayaḥ} in \skt{pāda} d seems superfluous.

\vsnum{3.6: }Note the use of the singular in \skt{pāda}s c and d. I have left \skt{sumadhyamāḥ} as the manuscripts transmit it: it signals the presence of the plural. And consider correcting \skt{mahābhāgā} to \skt{mahābhāgās}. In sum, understand \skt{tasya patnyo mahābhāgās trayodaśa sumadhyamāḥ}.

\vsnum{3.7: }\skt{śraddhāḍhyāḥ} in \skt{pāda} b is an attractive lectio difficilior (`they were rich in faith/devotion'), but I have finally decided to accept the easier and better-attested \skt{śraddhādyā}[\skt{ḥ}]. Again, I have chosen/applied the plural forms \skt{°ādyāḥ} and \skt{sumanoharāḥ} in \skt{pāda} b to hint at the fact that the presence of the plural is to be preferred here; thus only \skt{viśālākṣī} is problematic. As \skt{patnī} in the previous verse, it should be treated as a plural. Note the use of the singular for the plural also in \skt{pāda}s cd, especially \skt{babhūva ha} for \skt{babhūvuḥ}. \\ MMW on Dakṣa: ``daughters of whom 27 become the Moon's wives, forming the lunar asterisms, and 13 [or 17 BhP.; or 8 R.] those of Kaśyapa, becoming by this latter the mothers of gods, demons, men, and animals, while 10 are married to Dharma, Mn. ix, 128f.'' CHECK

\vsnum{3.8: }Consider emending \skt{tebhyaḥ} to the correct feminine form \skt{tābhyaḥ}. Note again the use of the singular (nominative) for the plural (accusative) in \skt{pāda}s ab. Alternatively, emend \skt{dharmapatnī} to \skt{dharmapatnīr} (plural accusative) and \skt{putras} to \skt{putrān} to make them work with \skt{śrotum icchāmi}. For Dharma's thirteen wives and their sons, see Liṅgapurāṇa 1.5.34-37 (note the similarity between the first line and VSS 3.6cd--7ab above): \\ \skt{dharmasya patnyaḥ śraddhādyāḥ kīrtitā vai trayodaśa$|$\\tāsu dharmaprajāṃ vakṣye yathākramam{ }anuttamam$||$ \\kāmo darpo 'tha niyamaḥ saṃtoṣo lobha eva ca$|$\\śrutas{ }tu daṇḍaḥ samayo bodhaś{ }caiva mahādyutiḥ$||$ \\apramādaś{ }ca vinayo vyavasāyo dvijottamāḥ$|$\\kṣemaṃ sukhaṃ yaśaś{ }caiva dharmaputrāś{ }ca tāsu vai$||$ \\dharmasya vai kriyāyāṃ tu daṇḍaḥ samaya eva ca$|$\\apramādas{ }tathā bodho buddher{ }dharmasya tau sutau$||$}  \\\skt{prasūtisambhavāḥ} is a rather bold conjecture that can be supported by two facts: firstly, the readings of the manuscripts are difficult to make sense of and thus are probably corrupt; secondly, a corruption from the name Prasūti, that of Dakṣa's wife, to \skt{ābhūti} is relatively easily to explain, \skt{sū} and \skt{bhū} being close enough in some scripts (e.g. in \msCa) to cause confusion. Another option would be to accept Ābhūti as the name of Dakṣa's wife. \\ For Prasūti being Dakṣa's wife in other sources, see e.g. Liṅgapurāṇa 1.5.20--21 (but also note the presence of the name Sambhūti...): \skt{prasūtiḥ suṣuve dakṣāc{ }caturviṃśatikanyakāḥ$|$śraddhāṃ lakṣmīṃ dhṛtiṃ puṣṭiṃ tuṣṭiṃ medhāṃ kriyāṃ tathā$||$ buddhi lajjāṃ vapuḥ śāntiṃ siddhiṃ kīrtiṃ mahātapāḥ$|$khyātiṃ śāntiś{ }ca saṃbhūtiṃ smṛtiṃ prītiṃ kṣamāṃ tathā$||$}

\vsnum{3.10: }Understand \skt{śraddhā} as a stem form noun for \skt{śraddhāyāḥ} (gen./abl.). It is tempting to emend \skt{abhayaḥ} to \skt{ubhayaḥ}, thus matching the relevant line in the Kūrmapurāṇa cited above: \skt{kriyāyāś cābhavat putro daṇḍaḥ samaya eva ca} and allotting only two sons to Kriyā, but in a number of sources Kriyā actually has three sons, see e.g. Viṣṇupurāṇa 1.7.29(ab? CHECK in book), where they are named as Daṇḍa, Naya and Vinaya: \skt{medhā śrutaṃ kriyā daṇḍaṃ nayaṃ vinayam eva ca}. Perhaps read \skt{kriyāyās tu nayaḥ putro} in pāda c? Compare Vāyupurāṇa 1.10.34cd \skt{kriyāyās tu nayaḥ prokto daṇḍaḥ samaya eva ca} with Brahmāṇḍapurāṇa 1.9.60ab: \skt{kriyāyās tanayau proktau damaś ca śama eva ca}

\vsnum{3.12: }In a very similar passages in Kūrmapurāṇa 1.8.20 ff., Apramāda is Buddhi's son and Lajjā has only one son, Vinaya. In the above verse (VSS 3.12), \skt{sudhiyaḥ} (for \skt{sudhīḥ}) may only be qualifying \skt{apramāda}, thus Lajjā may have two sons: Vinaya and the wise Apramāda.

\vsnum{3.13: }Note that \skt{sukhaṃ} in \skt{pāda} d is probably meant to be masculine (\skt{sukhaḥ}), but e.g. in the Kūrmapurāṇa passage quoted above it is also neuter. For the emendation in \skt{pāda} e, see Matsyapurāṇa 9.2cd: \skt{yāmā nāma purā devā āsan svāyambhuvāntare} and Bhāgavatapurāṇa 6.4.1: \skt{devāsuranṛṇāṃ sargo nāgānāṃ mṛgapakṣiṇām$|$sāmāsikas tvayā prokto yas tu svāyambhuve 'ntare$||$}.

\vsnum{3.14: }Note \skt{dharma} as a neuter noun and the form \skt{atīvaṃ} for \skt{atīva} metri causa. My emendation from \skt{kīrtaya} (`declare') to \skt{kartaya} (`cut') was influenced by the combination of \skt{chindhi} and \skt{saṃśaya}, often with \skt{kautūhala}, elsewhere in the VSS: 3.2b: \skt{saṃśayaṃ chindhi tattvataḥ}; 10.XXcd: \skt{kautūhalaṃ mahaj jātaṃ chindhi saṃśayakārakam}; 15.2ab: \skt{etat kautūhalaṃ chindhi saṃśayaṃ parameśvara}. The reading \skt{kīrtaya} may have been the result of the influence of \skt{kīrtitā} in 3.13b above (De Simini's convinicing observation).

\vsnum{3.15: }The reading \skt{°dvayī} in \msNc\ in \skt{pāda} a is attractive, but as Judit Törzsök has pointed out to me, it is probable that the slightly less convincing but widespread variant \skt{°dvayor} is original. To state that the Smārta tradition is connected to \skt{yama}s and \skt{niyama}s and the \skt{āśrama}s and then to discuss these at length (principally in chapters 3--8 and 11) can be seen as a clear self-identification with the Smārta tradition.

\vsnum{3.16: }\skt{Pāda} a should be understood as \skt{yamaniyamayoś caiva}, but the author of this line may have tried to avoid the metrical fault of having two short syllables in the second and third positions. Note that this is the beginning of a long section in our text that describes the \skt{yama-niyama} rules, reaching up to the end of chapter eight. The title given in the colophon of the next chapter, chapter four, namely \skt{yamavibhāga}, would fit this locus better than the beginning of that chapter, which commences with a discussion on the second of the \skt{yama}s, \skt{satya}.

\vsnum{3.17: }Note how all witnesses read \skt{mādhūrya} instead of \skt{mādhurya}. The former may have been acceptable originally in this text. Note the use of the singular in \skt{pāda}s cd referring back to the agents of the previous sentence. Most probably, °\skt{vadhyam} is to be understand as °\skt{vadham} and the form \skt{vadhyam} serves only to avoid two \skt{laghu} syllables in \skt{pāda} d.

\vsnum{3.20: }Understand \skt{bhujoraś ca} in \skt{pāda} a as \skt{bhuje urasi ca}, in this case with an instance of double sandhi: \skt{bhuje urasi ca} -- \skt{bhuja urasi ca} -- \skt{bhujorasi ca}. Alternatively, understand it as a compound: \skt{bhujorasi}. Understand \skt{vadhaḥ} in \skt{pāda} b as \skt{vadhyaḥ} metri causa. \skt{Pāda} a is unmetrical. Note how elliptical this verse is and that \skt{hiṃsakāni} is neuter although it refers to people, perhaps implying \skt{bhūtāni}. Alternatively, take \skt{°ny°} in \skt{hiṃsakāny} as rather unusual sandhi-bridge (\skt{hiṃsakā-ny-āhu}). Note also that \skt{āhu} stands for \skt{āhur} metri causa.

\vsnum{3.24: }Note \skt{dharma} as a neuter noun in \skt{pāda} a and that \skt{°vinirmuktaṃ} and \skt{°pradam} are neuter accordingly.

\vsnum{3.25: }Note that \skt{parataro} is masculine in \skt{pāda} d, picking up a neuter \skt{'yaśaḥ}. This phenomenon is probably the result of \skt{'yaśaḥ} resembling a masculine noun ending in \skt{-aḥ} and also of the metrical problem with the grammatically correct \skt{nātaḥ parataram ayaśaḥ}.

\vsnum{3.26: }\skt{Pāda} d (\skt{nātaḥ paraṃ tapodhana}) is slightly suspicious. The text may have read \skt{nātaḥ paratamo 'dhanaḥ} (`There is no bigger loss of wealth') or possibly something starting with \skt{nātaḥ paraṃ tapo ...} (`There is no greater austerity...').

\vsnum{3.34: }See Uttarottara chapter two for a similar section on meat-consumption. See a similarly phrased comparison in Manu 2.86: \\ \skt{ye pākayajñās catvāro vidhiyajñasamanvitāḥ |\\ sarve te japayajñasya kalāṃ nārhanti ṣoḍaśīm $||$}

\vsnum{3.39: }Understand \skt{phalam āhārā} as \skt{phalāhārā} (-m- is a sandhi-bridge). \skt{guṇākāśāt} in pāda c is difficult to interpret and \skt{guṇākarṣāt} is a conjecture by Judit Törzsök which fits the context well, although the polysemy of \skt{guṇa} may allow for other solutions. \\ Verses 3.40--42 may be echoing Brahmapurāṇa 216.64--66: \\ \skt{ māṃsān miṣṭataraṃ nāsti bhakṣyabhojyādikeṣu ca |\\ tasmān māṃsaṃ na bhuñjīta nāsti miṣṭaiḥ sukhodayaḥ $||$ \\ gosahasraṃ tu yo dadyād yas tu māṃsaṃ na bhakṣayet |\\ samāv etau purā prāha brahmā vedavidāṃ varaḥ $||$\\ sarvatīrtheṣu yat puṇyaṃ sarvayajñeṣu yat phalam |\\ amāṃsabhakṣaṇe viprās tac ca tac ca ca tatsamam $||$ }  \vsnum{3.41: }Pādas ab probably stand for \skt{ahiṃsako nāsti samo dānayajñasamīhaiḥ puruṣaiḥ} CHECK and are reminescent of Śivadharmaśāstra 11.92:\\ \skt{ ahiṃsaikā paro dharmaḥ śaktānāṃ parikīrtitam|\\ aśaktānām ayaṃ dharmo dānayajñādipūrvakaḥ $||$ } \\ Note the variant \skt{°dharma°} in both \msCc\ and \Ed\ in \skt{pāda} b. On \skt{padma} meaning `ten trillion', and on other words for numbers, see 1.32--35. \\ \skt{koṭīyajña} in pāda d may refer to a special kind of sacrifice, mostly known as \skt{koṭihoma} in the Purāṇas and in inscriptions (see e.g. Fleming 2010 and 2013) It probably involves a hundred fire-pits and a hundred times one thousand brāhmaṇas (hence the name `the ten-million sacrifice'). See e.g. Bhaviṣyapurāṇa uttaraparvan 4.142.54--58: \\ \skt{ śatānano daśamukho dvimukhaikamukhas tathā $|$caturvidho mahārāja koṭihomo vidhīyate $||$ kāryasya gurutāṃ jñātvā naiva kuryād aparvaṇi $|$yathā saṃkṣepataḥ kāryaḥ koṭihomas tathā śṛṇu $||$ kṛtvā kuṇḍaśataṃ divyaṃ yathoktaṃ hastasaṃmitam $|$ekaikasmiṃs tataḥ kuṇḍe śataṃ viprān niyojayet $||$ sadyaḥ pakṣe tu viprāṇāṃ sahasraṃ parikīrtitam $|$ekasthānapraṇīte 'gnau sarvataḥ paribhāvite $||$ homaṃ kuryur dvijāḥ sarve kuṇḍe kuṇḍe yathoditam $|$yathā kuṇḍabahutve 'pi rājasūye mahākratau $||$ } \\ Note that the second syllable of \skt{phalam} in \skt{pāda} d is treated as a long syllable: this happens often at word-boundaries in this text; and note how \msNc\ aims to restore the metre by inserting \skt{tv} after its \skt{phalaṃ}.


%%%%%%%%%
\vsnum{4.1: }Should we read \skt{satyalakṣaṇaṃ} in pāda d, following the rather similar Śivadharmaśāstra 11.105cd?

\vsnum{4.2: }\skt{suduḥsaham} (singular) in \skt{pāda} b picks up \skt{°ādīni} (plural) in \skt{pāda} a. The \skt{-m} in \skt{satyam} may be a sandhi-bridge and the phrase may refer to a masculine subject thus: \skt{sa ca satya -m- udāhṛtaḥ}.

\vsnum{4.7: }\skt{Pāda} d is slightly problematic because it is difficult to ascertain if some of the MSS actually read \skt{panthāna} or \skt{pasthāna} (or \skt{yasthāna}). I suspect that \skt{panthāna} is a stem form noun formed (metri causa) to stand for an irregular nominative of \skt{pathin}.

\vsnum{4.11: }Here and several times below, \skt{satye} is probably to be taken as standing for \skt{satyena}.

\vsnum{4.12: }\skt{Pāda} b, \skt{samayena priyavrataḥ}, probably stand for \skt{samayena priyavratasya} although it is unclear to me what exactly \skt{samaya} refers to here. \\ For Priyavrata's story, in which he wanted to turn nights into days by circling aroung Mount Meru in a chariot, and by this produced the seven oceans, see e.g. Bhāgavatapurāṇa 5.1.30--31:\\ \skt{yāvad avabhāsayati suragirim anuparikrāman bhagavān ādityo vasudhātalam ardhenaiva pratapaty ardhenāvacchādayati, tadā hi [priyavrataḥ] bhagavadupāsanopacitātipuruṣaprabhāvas tad anabhinandan samajavena rathena jyotirmayena rajanīm api dinaṃ kariṣyāmīti saptakṛtvas taraṇim anuparyakrāmad dvitīya iva pataṅgaḥ$|$ye vā u ha tadrathacaraṇanemikṛtaparikhātās te sapta sindhava āsan yata eva kṛtāḥ sapta bhuvo dvīpāḥ|}  \\Pādas cd: for a somewhat similar reference to the story of Mahābali, see e.g. Vāmanapurāṇa 65.66: \skt{evaṃ purā cakradhareṇa viṣṇunā baddho balir vāmanarūpadhāriṇā $|$śakrapriyārthaṃ surakāryasiddhaye hitāya viprarṣabhagodvijānām $||$} 

\vsnum{4.13: }Since \skt{śaśi} (instead of \skt{śaśin}) is a possible stem in this text, \skt{śaśir ācaraḥ} could also be possible here in pāda b (see \msNa\msNb\msNc), perhaps standing for \skt{śaśinaś caraṇam} or \skt{śaśiś carati}. My emendation (\skt{śaśinācaraḥ}) could stand for \skt{śaśinā/śaśinaś cāraḥ} metri causa. \\Pādas cd refer to the story of Agastya and the Vindhya mountain: Vindhya became jealous of the Sun's revolving around Mount Meru and when the Sun refused to him the same favour, he decided to grow higher and obstruct the Sun's movement. As a solution to this situation, Agastya asked Vidhya to bend down to make it easier for him to reach the south and to remain thus until he retured. Vindhya agreed to do what Agastya asked him to do but Agastya never returned. See Mahābhārata 3.102.1--14 (see in the word \skt{samaya} in verse 13 and compare it to VSS 4.12b): \\ \skt{ yudhiṣṭhira uvāca |\\ kimarthaṃ sahasā vindhyaḥ pravṛddhaḥ krodhamūrchitaḥ |\\ etad icchāmy ahaṃ śrotuṃ vistareṇa mahāmune $||$\\ lomaśa uvāca |\\ adrirājaṃ mahāśailaṃ meruṃ kanakaparvatam |\\ udayāstamaye bhānuḥ pradakṣiṇam avartata $||$\\ taṃ tu dṛṣṭvā tathā vindhyaḥ śailaḥ sūryam athābravīt |\\ yathā hi merur bhavatā nityaśaḥ parigamyate $||$\\ pradakṣiṇaṃ ca kriyate mām evaṃ kuru bhāskara |\\ evam uktas tataḥ sūryaḥ śailendraṃ pratyabhāṣata $||$\\ nāham ātmecchayā śaila karomy enaṃ pradakṣiṇam |\\ eṣa mārgaḥ pradiṣṭo me yenedaṃ nirmitaṃ jagat $||$\\ evam uktas tataḥ krodhāt pravṛddhaḥ sahasācalaḥ |\\ sūryācandramasor mārgaṃ roddhum icchan paraṃtapa $||$\\ tato devāḥ sahitāḥ sarva eva; sendrāḥ samāgamya mahādrirājam |\\ nivārayām āsur upāyatas taṃ; na ca sma teṣāṃ vacanaṃ cakāra $||$\\ athābhijagmur munim āśramasthaṃ; tapasvinaṃ dharmabhṛtāṃ variṣṭham |\\ agastyam atyadbhutavīryadīptaṃ; taṃ cārtham ūcuḥ sahitāḥ surās te $||$\\ devā ūcuḥ |\\ sūryācandramasor mārgaṃ nakṣatrāṇāṃ gatiṃ tathā |\\ śailarājo vṛṇoty eṣa vindhyaḥ krodhavaśānugaḥ $||$\\ taṃ nivārayituṃ śakto nānyaḥ kaś cid dvijottama |\\ ṛte tvāṃ hi mahābhāga tasmād enaṃ nivāraya $||$\\ lomaśa uvāca |\\ tac chrutvā vacanaṃ vipraḥ surāṇāṃ śailam abhyagāt |\\ so 'bhigamyābravīd vindhyaṃ sadāraḥ samupasthitaḥ $||$\\ mārgam icchāmy ahaṃ dattaṃ bhavatā parvatottama |\\ dakṣiṇām abhigantāsmi diśaṃ kāryeṇa kena cit $||$\\ yāvadāgamanaṃ mahyaṃ tāvat tvaṃ pratipālaya |\\ nivṛtte mayi śailendra tato vardhasva kāmataḥ $||$\\ evaṃ sa samayaṃ kṛtvā vindhyenāmitrakarśana |\\ adyāpi dakṣiṇād deśād vāruṇir na nivartate $||$\\ etat te sarvam ākhyātaṃ yathā vindhyo na vardhate |\\ agastyasya prabhāvena yan māṃ tvaṃ paripṛcchasi $||$ }  \vsnum{4.16: }Another way to translate \skt{ekena} in pāda a would turn the sentence into this: `If Truth is obtained by somebody, he will be one for whom Dharma is surely accomplished.' It is not inconceivable that \skt{tava} is meant to carry the sense of an ablative, as Kenji Takahashi has suggested to me: `I can't have enough of learning about Dharma from you.' Note \skt{asau} in pāda c as an accusative form.

\vsnum{4.23: }A line may have dropped out after pāda b, perhaps because a line similar to 4.22cd caused an eyeskip. Alternatively, this line may simply be elliptical. Note how \skt{stena} and \skt{steya} are used interchangeably (or chaotically) in the above passages in the MSS to denote both `thief' and 'theft/stealing'. The scribe of \msNc\ ends up writing \skt{stenya} in 4.27e.

\vsnum{4.28: }It appears that \skt{hriyate} in pāda a is to be taken as an active verb (\skt{harate}). Note also how \msCb\ and \msNc\ read the same here. Take \skt{°hariṇo} in pāda b as singular and \skt{m} in \skt{'nya-m-adhamo} as a sandhi-bridge. Understand \skt{stenastulya na mūḍham{ }asti} (the reading of \Ed!) as a `metri causa' version of \skt{stenatulyo na mūḍho 'sti}, and see a similar case of a nominative ending inside of compound in pāda c below. One major concern remains here: the accepted reading here is that of \Ed, an edition that rarely emerges as the sole transmitter of the best reading. A solution could be to emend to \skt{stenaṃtulya...}, meaning `There is no bigger foolishness than theft', but then the second part of pāda a is difficult to connect. \\ Understand \skt{prāptaḥśāsana tīvrasahyaviṣamaṃ} in pāda c as \skt{prāptaśāsanas tīvram asahyaṃ ca viṣamaṃ prāpnoti}. Alternatively, understand \skt{tīvrasahya°} as \skt{duḥsahya°} (suggested by Törzsök). \\ The actual reading of \msCa, \skt{prāptaś} (lost in the process of normalization and standing in contrast with that of all other MSS that read \skt{prāptaḥ}) may suggest a doubling of the \skt{ś} of \skt{śāsana} metri causa (suggestion by Törzsök). More likely is that a licence of having a nominative ending inside of a compound is applied here, as probably above in pāda a (also remarked by Törzsök). Note °\skt{kalpa} for °\skt{kalpaṃ} metri causa. I understand \skt{vipule} as \skt{vipulāyāṃ}, \skt{vipulā} appearing in Amarakośa 2.1.7 as a synonym of \skt{dhātrī}, `earth'. Note the switch from plural to singular in pāda d.

\vsnum{4.31: }Note \skt{pitur} and \skt{mātur} used as accusative forms in \skt{pāda} b, or alternatively understand: `who are hateful towards their fathers and mothers'.

\vsnum{4.32: }See Śakuntalā 1.1: \\ \skt{yā sṛṣṭiḥ sraṣṭur ādyā [1] vahati vidhihutaṃ yā havir [2] yā ca hotrī [3]\\ ye dve kālaṃ vidhattaḥ [4,5] śruti-viṣaya-guṇā yā [6] sthitā vyāpya viśvam |\\ yām āhuḥ sarva-bīja-prakṛtir [7] iti yayā prāṇinaḥ prāṇavantaḥ [8]\\ pratyakṣābhiḥ prapannas tanubhir avatu vas tābhir aṣṭābhir īśaḥ $||$}\\ \\ The eight \skt{tanu}s here are: [1] jala [2] agni [3] yajamāna [4,5] sūrya + candra [6] ākāśa [7] bhūmi [8] vāyu \\ For a similar interpretation of \skt{aṣṭamūrti}, see e.g. Īśānaśivagurudevapaddhati 2.29.34 (\skt{mantrapāda}; note \skt{yajamāna} for our \skt{dīkṣa}): \skt{kṣmā-vahni-yajamānārka-jala-vāyv-indu-puṣkaraiḥ}$|$\skt{aṣṭābhir mūrtibhiḥ śambhor dvitīyāvaraṇaṃ smṛtam}$||$ (For \skt{puṣkara} as `sky, atmosphere', see e.g. Amarakośa 1.2.167: \skt{dyodivau dve striyām abhraṃ vyoma puṣkaram ambaram}.) A closely related Aṣṭamūrti-hymn appears in Niśv mukha 1.30--41 (I owe thanks to Nirajan Kafle for drawing my attention to this); see Kafle 2018: 62, 63, 116, 119. Kafle notes that this hymn is closely parallel to some passages in the Prayogamañjarī (1.19--26), the Tantrasamuccaya (1.16--23), and the Īśānaśivagurudevapaddhati (kriyāpāda 26.56--63). See also TAK I s.v. \skt{aṣṭamūrti}.

\vsnum{4.40: }Not the peculiar verb forms \skt{anugaccheta} and \skt{anupūjyeta}) in this verse.

\vsnum{4.42: }Pāda b seems to awkwardly repeat what \skt{arghapādyena} in pāda a signifies. Some emendation may be required here, perhaps taking into account bathing (\skt{snāna}) or an unguent (\skt{abhyaṅga}). For the requirement that one could part with his wife or son, or his own life, for the benefit of someone else, see VSS 2.38 and the narrative in VSS chapter 12; these influenced my decision to emend \skt{°ātmano} to \skt{°ātmanā} in pāda a. The demonstrative pronoun \skt{tasya} in pāda c may refer to the guest: `he will obtain all his [i.e. the guest's] merits', hinting at some sort of karmic exchange. Nevertheless, I think that \skt{tasya} points at the merits one can obtain by rituals listed in the previous verse. This is suggested by passages such as the following: \\ Mahābhārata Supp. 13.14.379 ff.:\\ \textit{ahany ahani yo dadyāt kapilāṃ dvādaśīḥ samāḥi|\\ māsi māsi ca satreṇa yo yajeta sadā naraḥ$||$ \\ gavāṃ śatasahasraṃ ca yo dadyāj jyeṣṭhapuṣkare$|$\\ na taddharmaphalaṃ tulyam \textbf{atithir yasya tuṣyati}$||$} \\ Brahmavaivartapurāṇa 3.44--46:\\ \textit{atithiḥ pūjito yena pūjitāḥ sarvadevatāḥ|\\ \textbf{atithir yasya santuṣṭas} tasya tuṣṭo hariḥ svayam$||$\\ snānena sarvatīrtheṣu sarvadānena yat phalam|\\ sarvavratopavāsena sarvayajñeṣu dīkṣayā$||$ \\ sarvais tapobhir vividhair nityair naimittikādibhiḥ$|$\\ tad evātithisevāyāḥ kalāṃ nārhanti ṣoḍaśīm$||$}  This verse is a reference to the story related by a mongoose in MBh 14.92--93: A Brahmin who practises the vow of gleaning (\skt{uñcha}) and his family receive a guest. They feed the guest with the last morsels of the little food they have. In the end, the guest reveals that he is in fact Dharma (14.93.80cd) and as a reward the family departs to heaven. The noble act of the poor Brahmin and his family is depicted as yielding greater rewards than Yudhiṣṭhira's grandiose horse-sacrifice. (See some remarks on this story in Takahashi 2021.) \\We would be forced to accept the reading of \Ed\ in pāda d if the expression were in the masculine (\skt{saśarīro divaṃ gataḥ}). This would make sense and it would also echo expressions occuring e.g. in the Mahābhārata: 3.164.33cd: \textit{paśya puṇyakṛtāṃ lokān saśarīro divaṃ vraja}; 14.5.10cd: \textit{saṃjīvya kālam iṣṭaṃ ca saśarīro divaṃ gataḥ}. It is tempting to emend the pāda accordingly, but I have retained \skt{svaśarīraṃ divaṃ gatam} and I interpret it as referring to the Brahmin's whole family (\skt{sva}).  \vsnum{4.52: }Note \skt{kari} for \skt{karī} metri causa, and the end of pāda b (\skt{°mṛgāḥ}), which should be treated metrically as if it read \skt{°mrigāḥ}. Purūravas (double sandhi originally? purūravās ati° -- purūravā ati° -- purūravāti°). Pāda a may refer to the following passage in the Mahābhārata (1.70.16--18, 20ab): \\ \textit{purūravās tato vidvān ilāyāṃ samapadyata|\\ sā vai tasyābhavan mātā pitā ceti hi naḥ śrutam$||$\\ trayodaśa samudrasya dvīpān aśnan purūravāḥ|\\ amānuṣair vṛtaḥ sattvair mānuṣaḥ san mahāyaśāḥ$||$\\ vipraiḥ sa vigrahaṃ cakre vīryonmattaḥ purūravāḥ|\\ jahāra ca sa viprāṇāṃ ratnāny utkrośatām api$||$\\ ... \\ tato maharṣibhiḥ kruddhaiḥ śaptaḥ sadyo vyanaśyata|} \\ (``The wise Purūravas was born to Ilā. We heard that Ilā was both his mother and his father. The great Purūravas ruled over thirteen islands of the ocean and, though human, he was always surrounded by superhuman beings. Intoxicated with his power, Purūravas quarrelled with some Brahmins and robbed them of their wealth even though they were protesting. [...] Therefore, cursed be the great Ṛṣis, he perished.'') \\ See also Buddhacarita 11.15 (Aiḍa = Purūravas):\\ \textit{ aiḍaś ca rājā tridivaṃ vigāhya\\ nītvāpi devīṃ vaśam urvaśīṃ tām|\\ lobhād ṛṣibhyaḥ kanakaṃ jihīrṣur \\ jagāma nāśaṃ viṣayeṣv atṛptaḥ$||$} \\ For Daṇḍa(ka)'s story, see Rāmāyaṇa 7.71.31 ff.: Daṇḍa meets Arajā, a beautiful girl, in a forest and rapes her. As a consequence, her father, Śukra/Bhārgava, destroyes Daṇḍa's kingdom, which thus becomes the desolate Daṇḍaka-forest.  \\For two versions of the destruction of Sagara's sons, who were chasing the sacrificial horse of their father's Aśvamedha sacrifice, and by doing so disturbed Kapila's meditation, and who in turn burnt them to ashes, see Mahābhārata 3.105.9 ff. and Brahmāṇḍapurāṇa 2.52--53. \\ As for Rāvaṇa's haughtiness, especially the fact that he chose to be invincible by all creatures except humans, and its consequences, one should recall the story of the Rāmāyaṇa and Rāvaṇa's destruction brought about by Rāma therein.

\vsnum{4.57: }Saudāsa, also known as Kalmāṣapāda, hit Śakti, Vasiṣṭha's son, with a whip because the latter did not give way to him, and as a consequence Śakti cursed Saudāsa: Saudāsa had to roam the world as a Rākṣasa for twelve years. See Mahābhārata 1.166.1 ff. \\ As for the end of the Yādavas, see the short Mausalaparvan of the Mahābhārata (canto 16): cursed by the sages Viśvāmitra, Kaṇva and Nārada, and seeing menacing omens, the Yādavas take to drinking in Prabhāsa and destroy each other.  \\Most probably, \skt{atitṛṣṇā} in the MSS stand for \skt{atitṛṣṇāt} (intending \skt{atitṛṣṇayā}). The form \skt{māndhāto} in \msCb\ stands for \skt{māndhātā} (nominative of \skt{māndhātṛ}). I have corrected it in spite of the fact that the authors' knowledge about his story may come from Divyāvadāna 17, where it sometimes appears to be an a-stem noun (\skt{māndāta}). \skt{dvijavajñayā} in \skt{pāda} d stands for \skt{dvijāvajñayā} metri causa. \\ Māndhātṛ was born from his father's body who, being excessively thirsty once, had drank some decoction prepared for ritual purposes and as a result become pregnant with him. Nevertheless, Buddhacarita 11.13 suggests that Māndhātṛ himself was still unsatisfied with wordly objects even after he had obtained half of Indra's throne: \\ \textit{ devena vṛṣṭe 'pi hiraṇyavarṣe \\ dvīpān samagrāṃś caturo 'pi jitvā$|$\\ śakrasya cārdhāsanam apy avāpya \\ māndhātur āsīd viṣayeṣv atṛptiḥ$||$} \\ In fact, as Monika Zin points out (2012: 149) Māndhātṛ/Māndhāta's rise and fall is a very popular theme in the `Narrative Art of the Amaravati School': ``Statistics show that in the Amaravati School the most frequently represented narrative is the story of King Māndhātar, which appears 47 times.'' See ibid. p. 151: ''The story [e.g. \textit{Divyāvadāna} XVII, see more sources in fn. 17 of this article] relates that Māndhātar was a miraculously born \textit{cakravartin} with Seven Jewels who could cause rain to fall so that his subjects could prosper; not usual rain, but rain of coins, of grain or of cloth. By virtue of his moral strength alone, Māndhātar conquered the world - without any weapons. He conquered all the countries on earth, then Uttarakuru, Pūrvavideha and Aparagodānīya, after which he set out to conquer the heavens. When he was traversing from one abode of the gods to the next (Nāgas, Sadāmattas, Mālādharas, etc.) groups of gods pledged obeisance to him and immediately marched in front of his troops. Māndhātar reached the splendid city of the Trayastriṃśa gods atop Sumeru, where Indra, in the meeting-hall, bequeathed to him half of his own seat and half of his heavenly realm. Māndhātar then ruled together with Indra for an unimaginable period of time during which 36 Indras changed. One day, shortly after he won a battle against the Asuras, a sinful thought came to his mind: why should he rule alongside Indra? It was he, after all, who won the war, not Indra - he was better and should, therefore, rule alone. At that very moment Māndhatar fell from heaven, down to his former realm, became sick and died. Shortly before his death, he preached a sermon to his subjects in which \textit{gātha}s from the \textit{Dhammapada} (186--187) appear...''  \\ Nahuṣa was elevated to the position of Indra for a period of time and he also wanted to take Śacī, Indra's wife. Indra instructed Śacī to tell Nahuṣa to harness some Ṛsis to a vehicle and use this vehicle to take Śacī. Agastya, one of the Ṛṣis, was insulted even further by Nahuṣa, therefore he cursed Nahuṣa, who then fell from the vehicle. See Mahābhārata 12.329.35 ff. and the verse in the Buddhacarita (11.14) that follows the one about Māndhātṛ: \\ \textit{ bhuktvāpi rājyaṃ divi devatānāṃ \\ śatakratau vṛtrabhayāt pranaṣṭe|\\ darpān maharṣīn api vāhayitvā \\ kāmeṣv atṛpto nahuṣaḥ papāta$||$}  \vsnum{4.58: }Pāda a is most probably a reference to Mahābali's promises made to Vāmana that caused his fall. Arjuna: the exile? Flo Kirāṭārjunīya?? he killed Bhīṣma? Flo \\ King Nala was an expert in the game of dice and lost his kingdom to Puṣkara in a game. See e.g. Mahābhārata 3.56.1 ff. \\ As for Nṛga, see Mahābhārata 14.93.74: \\ \textit{ gopradānasahasrāṇi dvijebhyo 'dān nṛgo nṛpaḥ|\\ ekāṃ dattvā sa pārakyāṃ narakaṃ samavāptavān$||$\\  } (``King Nṛga had made gifts of thousands of cows for the twice-born. By giving away one single cow that belonged to someone else, he fell into hell.'')  \vsnum{4.59: }Note how flexible the gender of most nouns is in pāda b: \skt{svarga}, \skt{mokṣa} and \skt{dama} are usually masculine in standard Sanskrit. The majority of the witnesses suggest that pāda c ends in a stem form noun (\skt{°nāśa}). This pāda is unmetrical, or rather it applies the licence of a word-final short syllable being counted as potentially long (\skt{°dharMA°}). Note how \skt{viprā} in pāda d is probably an attempt in some MSS to restore the metre. This pāda is also unmetrical, or rather it applies the licence of a word-final short syllable being counted as potentially long (\skt{viPRA}).

\vsnum{4.64: }In pāda d, understand \skt{caraṇācara} as \skt{caraṇacara} (metri causa).

\vsnum{4.65: }Note \skt{mātā} as a stem form.

\vsnum{4.66: }One should probably understand \skt{śauṇḍe} in pāda c as \skt{śauṇḍike} (alternatively, it may be corrupted from \skt{ṣaṇḍhe}); see both in Vāsiṣṭhadharmaśāstra 14.1--3: \\ \textit{athāto bhojyābhojyaṃ ca varṇayiṣyāmaḥ$|$cikitsaka-mṛgayu-puṃścalī-ḍaṇḍika-stenābhiśastar-ṣaṇḍha-patitānām annam abhojyam$|$kadarya-dīkṣita-baddhātura-somavikrayi-takṣa-rajaka-śauṇḍika-sūcaka-vārdhuṣika-carmāvakṛntānām$||$} etc. \\ In Olivelle's translation (DhSūtras 1999: 285): ``Next we will describe food that is fit and food that is unfit to be eaten [...] The following are unfit to be eaten: food given by a physician, a hunter, a harlot, a law enforcement agent, a thief, a heinous sinner [...] a eunuch, or an outcaste; as also that given by a miser, a man consecrated for a sacrifice, a prisoner, a sick person, a man who sells Soma, a carpenter, a washerman, a liquor dealer, a spy, an usurer, a leather worker...'' \\ In support of reading \skt{ṣaṇḍhe}, see Manu 3.239: \\ \textit{cāṇḍālaś ca varāhaś ca kukkuṭaḥ śvā tathaiva ca|\\ rajasvalā ca ṣaṇḍhaś ca nekṣerann aśnato dvijān$||$}

\vsnum{4.67: }Understand \skt{kīrtir yaśo°} as \skt{kīrtiyaśo°} ('r' being an intrusive consonant here metri causa). Understand \skt{āyuṣa} as \skt{āyuṣaṃ} (metri causa).

\vsnum{4.69: }Is \skt{sambhinna} a Buddhist term? See also Dharmaputrikā 1.31.

\vsnum{4.70: }Possible direct sources for the idea that \skt{kāma} is an enemy to be defeated include Buddhacarita 11.17: \\ \textit{cīrāmbarā mūlaphalāmbubhakṣā\\ jaṭā vahanto 'pi bhujaṃgadīrghāḥ|\\ yair nānyakāryā munayo 'pi bhagnāḥ\\ kaḥ kāmasaṃjñān mṛgayeta śatrūn$||$} \\ and Bhagavadītā 3.43: \\ \textit{evaṃ buddheḥ paraṃ buddhvā saṃstabhyātmānam ātmanā|\\ jahi śatruṃ mahābāho kāmarūpaṃ durāsadam$||$}  \vsnum{4.71: }Is \skt{āyatana} just a synonym of \skt{vihāra} here or could this use of the term \skt{āyatana} for the four Buddhist \skt{brahmavihāra}s have been influenced by the following passage in the Dharmasamuccaya (date?)? \\ \textit{mokṣasyāyatanāni ṣaṭ|\\ apramādas tathā śraddhā vīryārambhas tathā dhṛtiḥ|\\ jñānābhyāsaḥ saṃtāśleṣo mokṣasyāyatanāni ṣaṭ$||$1.3$||$\\ nava śāntisamprāptihetavaḥ|</br> dānaṃ śīlaṃ damaḥ kṣāntir maitrī bhūteṣv ahiṃsatā|\\ karuṇāmuditopekṣā śāntisamprāptihetavaḥ$||$1.4$||$\\}  \vsnum{4.72: }Note the stem form \skt{dhyāna} in \skt{°dhyānādhunā} (for \skt{°dhyānam adhunā}) in pāda a. For contrast, see VSS 6.8: \\ \textit{dhyānaṃ pañcavidhaṃ caiva kīrtitaṃ hariṇā purā|\\ sūryaḥ somo 'gni sphaṭikaḥ sūkṣmaṃ tattvaṃ ca pañcamam$||$}

\vsnum{4.73: }If pāda c is indeed a reference to a 36-tattva philosophical system, it is in striking contrast with the 25-tattva system described in VSS chapter 20.

\vsnum{4.75: }Note the plural instrumental (\skt{yair}) with a singular active verb (\skt{vetti}). Note the stem form noun in pāda a (\skt{°sthāna}) metri causa, and also that this stem form noun may function as a singular noun next to a number (\skt{pañca}), a frequently seen phenomenon in this text. Note how \skt{pāda} f deviates from Manu.

\vsnum{4.78: }The translation of this verse is based on Olivelle's (Olivelle Crit Ed. p. 218).

\vsnum{4.82: }Note syntax.

\vsnum{4.83: }My emendation from \skt{°manasā dhūryaś} to \skt{°mana-mādhuryaś} is based on the fact that following the list of \skt{yama}s in 3.16cd--17ab, we need some reference to \skt{mādhurya} here and that it is easy to see how this corruption came about: \skt{°mano-mādhurya°} would be unmetrical, thus the form \skt{°mana-mādhurya}; \skt{°mana-mā°} is easily corrupted to \skt{°manasā°} (not to mention the fact that \skt{manasā} comes up in the next verse); in addition we need five items in this line because of \skt{pañcamaḥ}. As always, I correct \skt{mādhūrya} to \skt{mādhurya}, although it seems that the former is acceptable in this text. I did not correct \skt{mādhuryaś} to \skt{mādhuryaṃ} because of the corresponding \skt{pañcamaḥ}. Understand \skt{jātavedam} in pāda b as \skt{jātavedasam} or \skt{jātavedāḥ}, or rather as belonging to the compound \skt{°dānaṃ}: \skt{jātavedodānaṃ}. For pāda e, see Mahāsubhāṣitasaṃgraha 2558: \textit{amṛtāyatām iti vadet pīte bhukte kṣute ca śataṃ jīva} (`When eating or drinking, one should say: "Let it turn into nectar!"; and after sneezing: "Live for a hundred years!".') 

\vsnum{4.89: }In pāda a \skt{°pra°} does not make the previous syllable long: this is the phenomenon of `muta cum liquida', one of the hallmarks of the \skt{Vṛṣasārasaṃgraha}, that is, syllables such as \skt{tra, pra, bra, dra} do not necessarily make the previous syllable long. In pāda b, \skt{parata} most probably stands for \skt{paratra} or \skt{parataḥ} metri causa. We may correct it to \skt{paratra} (`muta cum liquida'). \skt{°malapahārī} in the MSS stands either for \skt{°malāpahārī} or \skt{°malaprahārī} metri causa. I could have choosen to emend it to \skt{°malaprahārī} (`muta cum liquida' again), but I decided not to because \skt{apahārin}, \skt{apahāra} \skt{apahāraka} are used in the text very frequently. See also 8.XX, which contains a very similar expression: \skt{sakalamalapahāre dharmapañcāśad etat}. 
%%%%%%%%%
In \skt{pāda} a, \skt{anyat} is a bit strange, but it could be echoing \skt{apara} above in 5.1d. Note [or emend?] the form \skt{śaucayīta}. 

\vsnum{5.10: }\vo For similar instructions, see a verse cited in Śaṅkara's commentary ad BhG 6.16: \skt{uktaṃ hi$|$ardhaṃ savyañjanānnasya tṛtīyam{ }udakasya ca$|$vāyoḥ saṃcaraṇārthaṃ tu caturtham{ }avaśeṣayet$||$} (``Half is for food with sauce, the third part for water, but in order to be able to move the air, he should leave the fourth part [empty].'') See also e.g. Aṣṭāṅgahṛdaya 8.46cd-47ab: \skt{annena kukṣer dvāv aṃśau pānenaikaṃ prapūrayet$||$ āśrayaṃ pavanādīnāṃ caturtham avaśeṣayet|} and Sannyāsopaniṣad 59: \skt{āhārasya ca bhāgau dvau tṛtīyam udakasya ca$|$vāyoḥ saṃcaraṇārthāya caturtham avaśeṣayet$||$} 

\vsnum{5.17: }Understant \skt{°śaivabhāratasaṃhite} as \skt{śaive bhāratasaṃhitāyāṃ}. Note the stem form adjective \skt{°jña} and noun \skt{°mānava} metri causa, the second syllable of \skt{yadi} as a long syllable at the caesure, the plural \skt{āpnuvanti} where one would expect a verb in the singular, \skt{kīrtir} metri causa for a compounded stem form (\skt{kīrti°}), and the sandhi-bridge \skt{-m-} in \skt{paratra-m-īhita°}. 


%%%%%%%%%
\vsnum{6.1: }Maybe ījyāṃ is to be accepted. No, see 5.3a. Note pañcaitat for pañcaitāni or pañcete.

\vsnum{6.3: }See Dharmasūtras, Niśv book, Kiraṇa, Svacchanda, Tantrāloka etc.

\vsnum{6.5: }Note vedādhyayana (stem form) and °saṃhitam for saṃhitāṃ metri causa.

\vsnum{6.9: }Note śaśiṃ for śaśinaṃ.

\vsnum{6.13: }\skt{tri°} in the MSS is a problem. Odd syntax plus gender.

\vsnum{6.19: }Note how a plural imperative ātmanepada form (jijñāsyantāṃ) stands for the singular (jijñāsyatāṃ) metri causa. Note also that the last syllable of dvijendra counts here as long: this phenomenon of a word-ending syllable becoming long by position is common in the VSS. Note the form janmena. Note that miśraka in pāda b stands for miśrakaṃ metri causa. ete would be better for etāni? phps no, see 6.24c.

\vsnum{6.26: }CHECK abhrāvakāśa in MBh, Manu and Śivadharmasaṃgraha.

\vsnum{6.29: }Note the stem form \skt{°pāśa} in \skt{pāda} b metri causa.


%%%%%%%%%
\vsnum{7.1: }\skt{tathety} is suspicious. Note how \skt{annaṃ}, \skt{vastraṃ}, \skt{hiraṇyaṃ} and \skt{bhūmi} (the latter treated as neuter, or given in stem form) are all meant to go with -\skt{dāna} (again, in stem form, metri causa).

\vsnum{7.8: }The intention originally may have been this: ``Even if he is a great soul, he will be avoided...''

\vsnum{7.11: }It seems that \skt{vidhena ca} stands for \skt{vidhinā ca} or rather \skt{vidhānena} metri causa in \skt{pāda} b.

\vsnum{7.15: }I suspect that \skt{aṅguli} is used here in the sense of \skt{aṅgulīya} (`finger-ring'). The form \skt{tuṭi} as a widespread variant of \skt{tuṭi}, see e.g. CHECK.

\vsnum{7.17: }I suspect that \skt{phalaṃ vṛddhir} stands for \skt{phalavṛddhir} (\skt{phalasya vṛddhiḥ}) metri causa, meaning `the increase of the reward'.

\vsnum{7.20: }I take \skt{sādhāraṇā} as one word, but it is possible that the intention of the author was \skt{sā dhāraṇā} in two words, in fact meaning \skt{sādhāraḥ} (\skt{sā ādhāraḥ}, `it is the basis').

\vsnum{7.23: }I think that \skt{guṇāguṇi}, or perhaps \skt{guṇaguṇi} (which would be unmetrical), should refer to the idea that e.g. the donation of a piece of land of 2 x 2 \skt{hasta}s would result in 4 x \skt{koṭiśata} years in heaven, \skt{guṇa} generally meaning `times'. But this is only a guess, and it needs to be supported by some similar passage. I suspect that \skt{pāda} c is an awkward attempt at saying \skt{śraddhādhikadāna(sya) phalaṃ}.

\vsnum{7.24: }See entry `Paraśurāma' in Purāṇic Enc.: \\ To atone for the sin of slaughtering even innocent Kṣatriyas, Paraśurāma gave away all his riches as gifts to brahmins. He invited all the brahmins to Samantapañcaka and conducted a great Yāga there. The chief Ṛtvik (officiating priest) of the Yāga was the sage Kaśyapa and Paraśurāma gave all the lands he conquered till that time to Kaśyapa. Then a plat- form of gold ten yards long and nine yards wide was made and Kaśyapa was installed there and worshipped. After the worship was over according to the instructions from Kaśyapa the gold platform was cut into pieces and the gold pieces were offered to brahmins. When Kaśyapa got all the lands from Paraśurāma he said thus:—“Oh Rāma, you have given me all your land and it is not now proper for you to live in my soil. You can go to the south and live somewhere on the shores of the ocean there.” Paraśurāma walked south and requested the ocean to give him some land to live. For \skt{śakyānurūpaṃ} in \skt{pāda} a understand \skt{śakyatānurūpaṃ}.

\vsnum{7.27: }I suspect that \skt{khyātiś ca tulyaṃ} in the MSS stands for \skt{khyātim atulyāṃ} (`and unequalled fame') metri causa. I have corrected those parts of this phrase that could be corrected without violating the metre. REVISE! ūrja? Note \skt{svargaṃ} as a neuter in \skt{pāda} d.

\vsnum{7.28: }Revise. 
%%%%%%%%%
Note the accusative ending of \skt{°saṃhitām} after a list consisting of words probably in the nominative. One may correct it to \skt{°saṃhitā}.

\vsnum{8.2: }Note that \skt{śaivatattvaṃ} in pāda a is the result of a conjecture and that the reading \skt{śaivapāśupatadvaye} in pāda b is based on one single manuscript (\msP). In spite of this uncertainty, I think that this form of the current half-verse is the only one that yields an appropriate meaning. In pāda d, \skt{kīrtitāni} pick up an implied \skt{tattvāni}.

\vsnum{8.4: }Note that \skt{tirya} seems to be an acceptable nominal stem in this text for \skt{tiryañc}. I understand the causative form \skt{sampraveśayet} as non-causative, and interpret °madhya° as the `human world' tentatively.

\vsnum{8.5: }Compare pāda a with 3.15c. \vsnum{8.8: }Understand \skt{parve} as \skt{parvani} (thematisation of the stem in \skt{-an}). Understand \skt{°ādīnāṃ} in pāda a as standing for the locative case. Understand \skt{°sargam} as neuter nominative (instead of \skt{°sargaḥ}) or alternatively understand pāda c with a hiatus bridge: \skt{garhitotsarga-m-ity etad}.

\vsnum{8.10: }The conjecture that changes \skt{anyonya°} to \skt{ayonya°} in pāda a involves minimal intervention and makes the sentence much more meaningful than the version transmitted. Also consider \skt{ayoni°}. The variant \skt{strī} for \skt{tāṃ} in pāda d in the \Ed\ may be an example of Naraharināth, the editor's conscious interventions.

\vsnum{8.13: }Note \skt{°viṣṭha°} for \skt{viṣṭhā} metri causa in pāda c (\skt{ma-vipulā}). Alternatively, read \skt{svaviṣṭhāmūtra bhūmīṣu}. Note the stem form \skt{sūryasoma} for \skt{sūryasomau} in pāda e. It is not entirely clear why cats would rejoice seeing the Sun and the Moon. Perhaps this remark refers to the fact that cats can be active both in the daytime and at night.

\vsnum{8.14: }Cranes are compared to ascetics here probably because of the similarity of their tendency of relaxing standing on one leg to ascetics performing penance standing on one leg (such as the ascetic depicted on the famous relief in Mahabalipuram). 

\vsnum{8.15: }CITE source on dog being Bhairava's vāhana...

\vsnum{8.16: }I prefer reading \skt{bhīma tuṣṭi°} as two separate words, the first one in stem form, to reading it as a compound because of the following \skt{caiva}, and to the reading \skt{bhīmas tuṣṭi°} because the corresponding witnesses are the ones that usually give inferior readings.

\vsnum{8.17: }While \skt{dārayanto} as an active participle in the masculine nominative is acceptable as an irregular form, the precise interpretation of pādas a and b is still problematic. Note the neuter \skt{idaṃ} picking up the normally masculine \skt{lokaṃ} in pāda c.

\vsnum{8.21: }My translation here follows the parallel verse in the MBh and is based on that of Kisari Mohan Ganguli. The syntax of the version here in the VSS is less smooth than that in the MBh, and the VSS's reading \skt{prāntarāśī} definitely required an emendation.

\vsnum{8.22: }Note \skt{°vele} for \skt{°velāyāṃ} in pāda c.

\vsnum{8.23: }The translation of \skt{anārambhasya} (`of someone who has not yet started eating') is tentative. For a detailed discussion of the categories \skt{bhakṣya, bhojya, lehya} and \skt{coṣya}, see Kafle 2020:245, n. 534. See also Śivadharmottara 8.13:\\ \skt{bhakṣyaṃ bhojyaṃ ca peyaṃ ca lehyaṃ coṣyaṃ ca picchilam} |\\ \skt{iti bhedāḥ ṣaḍannasya madhurādyāś ca ṣaḍguṇāḥ} $||$ \skt{pāruṣya} seems to be the good reading in pāda a because in the following a short section on this category is coming up. As far as the readings \skt{spṛṣṭavāg} and \skt{pṛṣṭavāg} are concerned, I suppose \skt{pṛṣṭavāg} is not inconceivable (as suggested by Judit Törzsök), for in 8.29 it is questions that are given as relevant examples. Nevertheless I conjectured \skt{tīkṣṇavāg} here, relying on the same verse, 8.29. My translation of pāda b, or rather of the whole verse, is tentative.

\vsnum{8.29: }Understand \skt{śiro} as standing for the locative (\skt{śirasi}). I take \skt{°katham} in pāda b as an alternative nominative form of \skt{°kathā} metri causa and as belonging to all the categories here thus: \skt{dyūtakathā, bhojanakathā, yuddhakathā, madyakathā, strīkathā}. Understand \skt{me} in pāda d as \skt{mayā}.

\vsnum{8.32: }The form \skt{janme} for \skt{janmani} often occurs in Śaiva tantras as a tipically Aiśa phenomenon. See XXXXX 

\vsnum{8.33: }To make sense of pāda d, we are forced to take \skt{śāstra} as a stem form noun and \skt{naraḥ} as a (regular) genitive from \skt{nṛ}. (I thank Judit Törzsök for this interpretation.) Another way of understanding the beginning of this sentence would be to separate \skt{śāstrāneka°} as \skt{śāstrān eka°}, treating the word \skt{śāstra} as masculine.

\vsnum{8.37: }Note \skt{tryāyuṣa} in the sense of the three \skt{puṇḍra}-lines on the forehead and compare with 11.28c. Understand \skt{sthitam} as \skt{sthitaḥ} or rather \skt{sthitāḥ} if we are to connect this line to the next (8.37cd). Grammatical notes on kṛtam and ātmanaḥ

\vsnum{8.38: }It is not clear which story concerning Vīrabhadra is referred to here. Is it the destruction of Dakṣa's sacrifice, after which the gods were relieved? Or, which is a less likely possibility, another in which Kaśyapa and other Ṛṣis were burnt to ashes then reanimated by Vīrabhadra in the Śokara forest? For the latter, less well-known story, see Padmapurāṇa 5.107.1--14ff:\\ \skt{śucismitovāca \\ kaśyapaṃ jamadagniṃ ca devānāṃ ca purā katham $|$\\ rarakṣa bhasma tad brahman samācakṣva mune mama $||$1 \\ dadhīca uvāca \\ kaśyapādiyutā devāḥ pūrvam abhyāgaman girim |\\ śokaraṃ nāma vikhyātaṃ girimadhye suśobhanam $||$2 \\ nānāvihaṃgasaṃkīrṇaṃ nānāmunigaṇāśrayam $|$\\ vāsudevāśrayaṃ ramyam apsarogaṇasevitam $||$3 \\ vicitravṛkṣasaṃvītaṃ sarvartukusumojjvalam |\\ tathāvidhaṃ praviśyaite giriṃ vayam athāpare $||$4 \\ stuvaṃtaḥ keśavaṃ tatra gatāḥ sma giriśeśvaram |\\ dṛṣṭvā tatra mahājvālāṃ praviṣṭāśca vayaṃ ca tām $||$5 \\ māmekaṃ tu tiraskṛtya hy adahad devatā munīn |\\ māṃ dadāha tataḥ paścād bhasmībhūtā vayaṃ śubhe $||$6\\ asmān etādṛśān dṛṣṭvā vīrabhadraḥ pratāpavān $|$\\ kenāpikāraṇenāsau gatavān parvataṃ ca tam $||$7 \\ bhasmoddhūlitasarvāṃgo mastakasthaśivaḥ śuciḥ |\\ ekākī niḥspṛhaḥ śānto hāhāśabdam athāśṛṇot $||$8 \\ atha ciṃtāparaś cāsīn mriyamāṇa śavadhvaniḥ $|$\\ śavānām iva gaṃdhaś ca dṛśyate tannirīkṣaṇe $||$9 \\ iti niścitya manasā jagāmāgnim atiprabham |\\ sa vahnir vīrabhadraṃ ca dagdhum ārabdhavān atha $||$10 \\ tṛṇāgnir iva śāṃto 'bhūd āsādya salilaṃ yathā $|$\\ tato 'parāṃ mahājvālāṃ vīrabhadras tu dṛṣṭavān $||$11 \\ khaṃ gacchaṃtīṃ mahākālo jvālāṃ nipatitām api |\\ manasā ciṃtayac cāpi vīrabhadraḥ pratāpavān $||$12 \\ sarveṣāṃ nāśinī jvālā prāṇināṃ śatakoṭiśaḥ $|$\\ tat sarvaṃ rakṣaṇārthaṃ hi pipāsuś cāpy ahaṃ tv imām $||$13 \\ prāśnāmi mahatīṃ jvālāṃ salilaṃ tṛṣito yathā $|$\\ etasminn aṃtare vīraṃ vāg āha cāśarīriṇī $||$14} \\  ``Śucismitā said:\\  1. O brāhmaṇa, O sage, tell me how formerly the sacred ash protected Kaśyapa, Jamadagni of the gods? Dadhīca said:\\  2--6. Formerly gods accompanied by Kaśyapa and others went to a well-known mountain named Śokara. In the middle of the mountain was a very beautiful (forest) which was full of many birds, which was resorted to by various hosts of sages, which was the resort of Vāsudeva, which was charming, which was resorted to by bevies of celestial nymphs, which was crowded with strange trees, which was bright with flowers of all seasons. We and others entered the best mountain (forest) like that and praising Viṣṇu went there to lord Śiva. We saw a great flame there and we entered it. Excepting me that deity (i.e. that flame) burnt (other) sages. After that it (also) burnt me. O auspicious one, we were reduced to ash. \\ 7--14. Seeing us like this, that brave Vīrabhadra went to that mountain for some reason. With his entire body smeared with sacred ash, he remaining at the top, auspicious and pure, all alone, desireless and tranquil, heard the sound of wailing. Then he was full of thought: ‘The sound of the bodies of dead men and the smell as it were of dead bodies, are being perceived.’ Deciding like this in his mind, he went to the fire of great brilliance. Then that fire also started to burn Vīrabhadra. But it went out as the fire of (i.e. burning) grass (i.e. hay) would go out on receiving (i.e. being sprinkled over with) water. Then Vīrabhadra saw a great, mighty flame, which went (up) to the sky even (like) flame falling (i.e. dropped by) Śiva (obscure!). The brave Vīrabhadra thought in his mind: ‘(This) flame is the destroyer of hundreds of crores of beings. So for the protection of all I desire to drink it. As a thirsty man drinks water, I shall consume this great flame.’ In the meanwhile a divine voice said to (Vīrabhadra) the hero [...] (translation by N.A. Deshpande, in: Padma-purāna, Delhi: MLBD, 1951)''  One could simply accept the reading of \msCc (\skt{°hetunā}) in pāda d, but all other rejected readings hint at an original \skt{hetutaḥ} (as pointed out by Judit Törzsök).

\vsnum{8.40: }The reading \skt{vvidhaṃ} in pāda b seems to be the lectio difficilior as opposed to the rejected \skt{vidhivat}. The Ṛgvedic mantra starting with \skt{āpo hi ṣṭhā} (ṚV 10.9) is traditionally associated with \skt{mārjana} (`cleaning, wiping'). According to Kane (A History of Dharmaśāstra, vol. 4, p. 120), a Brahmin ``should bathe thrice in the day, should perform \skt{mārjana} (splashing or sprinkling water on the head and other limbs by means of \skt{kuśas} dipped in water after repeating sacred mantras) with the three verses `apo hi sthā' [sic] (Ṛg. X.9.1--3) [...]'' This suggests a method of bathing that is more of a ritual than an actual bath. This version of bathing seems to be rather a kind of bathing in the holy dust raising from under the hooves of cows.

\vsnum{8.44: }Understand \skt{sarvalokānukampya} in pāda b as \skt{sarvalokān anukampya}. Understand \skt{sakalamalapahārī} in pāda c as \skt{sakala-mala-apahārī}, which would be unmetrical. Understand \skt{etan/etad} as either picking up °\skt{pahārī} or a plural corresponding to °\skt{pañcāśad}. 
%%%%%%%%%
I have included the element \skt{trai°} in the lemma in pādas ab only because \msCc\ has a slightly unusual ligature there (\skt{mtrai}) Understand \skt{adhogatis} in pāda c as a bahuvrīhi in plural (\skt{adhogatayas}).

\vsnum{9.19: }\skt{°mahiṣyāś} seems to be an equivalent of \skt{°mahiṣāś} metri causa. Understand \skt{°pūtī} in pāda a as standing for \skt{°pūti} metri causa, and note that °āmedhya° in the same pāda is an emendation. Read \skt{āmayārasa} in pāda c?


%%%%%%%%%
\vsnum{10.8: }Note \skt{bindusāraṃ} for \skt{bindusaras/°saraṃ/°sarasaṃ} metri causa. Is perhaps \skt{pūrvavat} used in the sense of \skt{pūrvaṃ} here? There seems to be only two yogic tunnel here (and in 10.20--21): Suṣumṇā and Iḍā, instead of the usual three (Iḍā, Piṅgalā, Suṣumnā). This is strikingly similar to what we see in the archaic yoga of the Niśvāsa Naya, see Goodall et al. pp. 33--34. \\ Note \Ed's attempt to make pāda a metrical, but also note how some similar passages in other texts have the same hypermetrical reading as all our manusctipts; MBh Indices 6.3A.41--44:\\ \skt{iḍā bhagavatī gaṅgā piṅgalā yamunā nadī |\\ tayor madhye tṛtīyā tu tat prayāgam anusmaret $||$\\ iḍā vai vaiṣṇavī nāḍī brahmanāḍī tu piṅgalā |\\ suṣumṇā caiśvarī nāḍī tridhā prāṇavahā smṛtā} |\\ \\ See also \skt{Haṭhayogapradīpikā} 3.110: \\ iḍā bhagavatī gaṅgā piṅgalā yamunā nadī |\\ iḍāpiṅgalayor madhye bālaraṇḍā ca kuṇḍalī $||$  \vsnum{10.23: }\skt{hṛdi} might be meant to be a nominative, as in 12.17, here compounded with \skt{madhyastham}. Understand \skt{mānasasara°} in pāda a as \skt{mānasasaro} (metri causa). Note \skt{hṛdi} as a nominative in pāda c and possibly also in pāda d (and see 10.23a).

\vsnum{10.30: }Note that \skt{°kaṇṭhora} is a conjecture based on the context: this line probably talks about sounds and the production of sounds. For this \skt{uraḥ}/\skt{ura} (`chest') seems better that \skt{ūru} (`thigh'). 
%%%%%%%%%
alpakleśa -m- anāyāsa (sandhi bridge)

\vsnum{11.3: }Understand dayā as instrumental: tava dayayā bhūteṣu na tulyaṃ paśyāmi. Context: Viśvarūpa was a son of Tvaṣṭṛ. Viśvarūpa's heads were struck off by Indra. In the Bhāgavatapurāṇa, Indra's sin are distributed among the ground, water, trees and women.

\vsnum{11.15: }Or emend to °indhana-samujjvāla°, where °samujjvāla° is metri causa for °samujjvala°?

\vsnum{11.17: }Understand: dhāraṇām adhvaryuvat kṛtvā (dhāraṇā is a stem form noun). Understand: padaṃ śāśvatam (pada is a stem form noun metri causa).

\vsnum{11.22: }On the guṇātīta state of mind, see 9.39--43. Understand guṇātītatvaṃ and nirañjanatvaṃ? hāvana = havana metri causa

\vsnum{11.37: }°mṛgākūla for °mṛgākulaḥ metri causa? Or: [For him] the gist of the Śāstras is friendship[?], self-control, compassion etc.

\vsnum{11.40: }pūrṇa-m-itihāsa°: -m- is a filler. The Śivasaṃkalpa is Ṛgvedakhila 4.11 ff: yenedam bhūtaṃ bhuvanaṃ bhaviṣyat parigṛhītam amṛtena sarvam, yena yajñas tāyate saptahotā tan me manaś śivasaṅkalpam astu, etc. See also Manu 11.251ab: sakṛt japtvāsyavāmīyaṃ śivasaṃkalpam eva ca. Gender!

\vsnum{11.43: }\msNa\ only corrects °haraṇamanitya° to °haraṇam anitya° (CHECK this), but its scribe probably meant an anusvāra at the end of °haraṇaṃ, perhaps trying to correct the metre. He tries to correct the metre also with anityaharaṇan tajñā°. The fourth line of this verse could be Naraharinātha's invention.

\vsnum{11.49: }Check if saṃyama is a technical term here. tridaṇḍa = the three staves of the Parivrājaka MMW, check. Olivelle p. 173: ``There are numerous scriptural passages cited by the Vaisnavas that prescribe the carrying of a triple staff---that is, three bamboos tied together---by renouncers.'' °kṣaram avyayam would be unmetrical, so the nominative is used here.


\vsnum{11.57: }Buddhist terms. vihita here in the sense of `devoid'. 
\end{document}


