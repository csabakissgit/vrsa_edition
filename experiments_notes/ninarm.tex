\newcommand{\dnapp}[1]{}\newcommand{\rmapp}[1]{#1}
\fejno=0\versno=0

\vers

sarvākāram aśeṣasya jagataḥ sarvadā śivam\thinspace{\dandab} \dontdisplaylinenum
	    \var{\va sarvākāram\lem \all; sarvākāmān \msKa}%

gobrāhmaṇanṛpāṇāṃ ca śivaṃ bhavatu sarvadā \veg\dontdisplaylinenum
	    \var{\vb jagataḥ\lem \msNa \msNc \msBa, jagatas \msNb \msTa, japasya \msKa}%
	    \paral{\vo Before the beginning of the work, several manuscripts add some verses. \msNa, \msNb, \msNc\ and \msBa\ add the following four pādas: namas tuṅgaśiraś*cumbi(\msNa \msNbpc \msBa, °cumba° \msNbac\msNc)candracāmaracārave{\thinspace\danda} trailokya*nagarāraṃbhe (\all, °nagarārambha° \msNa) mūlastambhāya śaṃbhave{\thinspace\ketdanda} \oo
In addition, \msBa\ adds the following four pādas, shared with \msKa, which, however, does not feature the above verse: *praṇamya (\msBa, oṃ praṇamya \msKa) śirasā devaṃ sadāśivam anāmayaṃ{\thinspace\danda} śivadharmaṃ pravakṣyāmi śivabhakti*pradīpanam (\msBa, °pradīpikān \msKa){\thinspace\ketdanda}}

śivam ādau śivaṃ madhye śivam ante ca sarvadā\thinspace{\dandab} \dontdisplaylinenum

sarveṣāṃ śivabhaktānāṃ manujānāṃ ca naḥ śivam \veg\dontdisplaylinenum
	    \var{\vc °bhaktānāṃ \lem \all; °bhaktānā \msNbac}%
	    \var{\vd  manujānāṃ ca \lem  \msN \msBa, narāṇām astu \msKa, anugānāṃ ca \msTa \oo  naḥ \lem  \msNb \msNc \msBa, vaḥ \msNa \msTa}%
