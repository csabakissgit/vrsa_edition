\documentclass{article}
\usepackage[utf8x]{inputenx}
\newcommand{\vsnum}[1]{\textbf{#1}}
\newcommand{\skt}[1]{\textit{#1}}
\begin{document}


%%%%%%%%%
\vsnum{1.1:} Having bowed to [Him] whose boundaries are limitless, who has no beginning, no middle part and no end, [to Him] who is very subtle and who is the unmanifest and fine essence of the world, [to Him] who is respected as the foremost by Hari, Indra, Brahmā and the other [gods], I shall recite [the work called] `A Compendium on the Essence of the Bull [of Dharma]'. 

\vsnum{1.2: }Having listened to the Bhāratasaṃhitā [i.e. the Mahābhārata], the supreme book of a hundred thousand [verses], a thousand chapters (\skt{adhyāya}) with all its hundred sections (\skt{parvan}),

\vsnum{1.3: }Janamejaya remained unsatisfied and what he asked Vaiśampāyana in the past, listen to that unweariedly.

\vsnum{1.4: }Janamejaya spoke: O venerable sir, O knower of the entire Dharma, O you who are well-versed in all the sciences (\skt{śāstra})! Is there a supreme and secret Dharma which liberates [us] from the ocean of mundane existence (\skt{saṃsāra})?

\vsnum{1.5: }Teach me the Dharma that emerged from [Vyāsa] Dvaipāyana's mouth, O best of Brahmins. Help me find satisfaction at all cost, O great ascetic!

\vsnum{1.6: }Vaiśampāyana spoke: Listen with great attention, O king, to this unsurpassed narration of Dharma. Hear the secret Dharma that I received by Vyāsa's favour.

\vsnum{1.7: }|F| Viṣṇu, the great Lord, assuming the form of a Brahmin, wanted to test the one who performed nonmaterial sacrifices, the one who focused on his austerities and observances, the one whose conduct was virtuous and pure, and who was intent on compassion towards all living beings, and therefore he humbly asked him a question.

\vsnum{1.9: }[Vigatarāga spoke:] How is the knowledge of Brahman to be understood if [this knowledge] is devoid of [definitions of the] form and colour [of Brahman]? [And] the syllable that is devoid of vowels and consonants: is that [its] highest [form]?

\vsnum{1.10: }Anarthayajña spoke: That syllable is not to be pronounced, is unquestionable, non-dividable, consistent, spotless, all-pervading and subtle: what could be higher than that?

\vsnum{1.11: }Vigatarāga spoke: When the body disintegrates in the ground, in water, in fire or [is torn apart] by jackals and other [animals], how is the supportless and spotless soul led [to the netherworld] by Yama's messengers?

\vsnum{1.12: }How is it bound by the nooses of death/time? And if it is bodiless, how can it move? And how does the [soul of a] virtuous [person] (\skt{bahudharmakṛt}) reach heaven if it has no body? Teach me about this doubtful matter (\skt{saṃśaya}) that I am raising (\skt{me}). I want to know the truth about it.

\vsnum{1.13: }Anarthayajña spoke: You are asking me about an extremely doubtful and problematic matter, O great Brahmin. It is difficult to understand by humans, and [even] by gods (\skt{deva}), demons (\skt{dānava}) and serpents (\skt{pannaga}).

\vsnum{1.14: }The cause of both the birth and death of the body is karma. Good and bad deeds are called the two nooses.

\vsnum{1.15: }[Man] goes to hell or heaven accordingly. Happiness and suffering, both arising from karma, are to be experienced by the body.

\vsnum{1.16: }O great Brahmin, the body is produced for humans for his reason. Now learn about that which they call the noose of time, I shall teach you, O you of great observances.

\vsnum{1.17: }[If] you don't know anything, how could you start your investigation, O Brahmin? O Brahmin king, you should know the noose of time in its entirety.

\vsnum{1.18: }Learn about time which is divided into digits (\skt{kalā}), [i.e. about] the division[s] (\skt{kalā}) of the entity [called] Time (\skt{kālatattva}). Two atomic units of time (\skt{truṭi}) is one twinkling (\skt{nimeṣa}). One digit (\skt{kalā}) is twice a twinkling.

\vsnum{1.19: }Two digits (\skt{kalā}) form one bit (3.2 seconds; \skt{kāṣṭhā}). Thirty bits (\skt{kāṣṭhā}) is one digit (1.6 minutes; \skt{kalā}). Thirty digits (\skt{kalā}) make up one section (48 minutes; \skt{muhūrta}) according to mankind, O great Brahmin.

\vsnum{1.20: }Thirty sections (\skt{muhūrta}) are known to the wise as night and day [i.e. a full day]. Thirty days and nights are taught by the wise ones to be one month.

\vsnum{1.21: }One year is twelve months [according to] people who know the entity of time. The time span of three hundred thousand plus sixty thousand years

\vsnum{1.22: }by human standards is said to be the Kali era. The Dvāpara era is known to be twice as long as the Kali era.

\vsnum{1.23: }The Tretā era is thrice [as long], the Kṛta era four [times as long as the Kali]. This is [how to add up] the number[s] related to the Four Yugas [= a \skt{mahāyuga}]. Taking [this length of four \skt{yuga}s] seventy-one [times],

\vsnum{1.24: }the knowledge about one time-span of Manu is being taught briefly [i.e.\ 71 four-fold \skt{mahāyuga}s make up a \skt{manvantara}]. One Kalpa is fourteen \skt{manvantara}s in total.

\vsnum{1.25: }Brahmā's day is made up of ten thousand Kalpas. [Brahmā's] night is of the same [length] according to the wise who know the truth.

\vsnum{1.26: }When [Brahmā's] night falls, the whole moving and unmoving universe dissolves. And when [his] daylight comes, the moving and unmoving [universe] is born.

\vsnum{1.27: }A \skt{para} times \skt{parārdha} [number of, i.e. two hundred quadrillion times a hundred quadrillion] \skt{kalpas} have passed [so far], O great Brahmin. Bhṛgu and the other sages say that the future is the same [time span].

\vsnum{1.28: }Just as the sun, the planets, the stars and the moon are percieved by us as wandering around, the wheel of time (\skt{kālacakra}) keeps spinning and we never experience its halting.

\vsnum{1.29: }Time creates living beings and time destroys them again. Everything is under the control of time. There is nothing that can bring time under control.

\vsnum{1.30: }Fourteen \skt{parārdha}s is [the number of] the kings of the gods [i.e. Indras?], O Brahmin, who passed by over time, for time is difficult to overcome.

\vsnum{1.31: }Time is [manifest] as a great yogin, as Brahmā, Viṣṇu and supreme Śiva, it is beginningless and endless, it is the creator, the great soul. Pay homage [to Time].

\vsnum{1.32: }Vigatarāga spoke: I have just heard [the term] `wheel of time' (\skt{kālacakra}) uttered from [your] lotus mouth, as well as \skt{parārdha} and \skt{para}. You have made these things appear as exciting, as things to hear.

\vsnum{1.33: }Anarthayajña spoke: One, ten, a hundred, a thousand, and ten thousand (\skt{ayuta}), a hundred thousand (\skt{prayuta}), a million (\skt{niyuta}), ten millions (\skt{koṭi}), a hundred millions (\skt{arbuda}), and a billion (\skt{vṛnda}, 10^9),

\vsnum{1.34: }ten billion (\skt{kharva}), a hundred billion (\skt{nikharva}), one trillion (\skt{śaṅku}, 10^12), and ten trillion (\skt{padma}), a hundred trillion (\skt{samudra}), one quadrillion (\skt{madhya} 10^15), ten quadrillion (\skt{[an]anta}), a hundred quadrillion (\skt{parārdha}), and two hundred quadrillion (\skt{para}).

\vsnum{1.35: }All should be known as powers of ten up to \skt{parārdha}. The number corresponding to \skt{para} is double the \skt{parārdha}.

\vsnum{1.36: }There is no higher number than \skt{para}. This is my conviction, which is based on readings of the Purāṇas and the Vedas and [which I have now] taught [to you], O great Brahmin.

\vsnum{1.37: }Vigatarāga spoke: How many eggs of Brahmā are there? And are its measurements available anywhere? From how many finger's breadths high does the sun heat the earth?

\vsnum{1.38: }Anarthayajña spoke: How could I enumerate all the eggs of Brahmā, O Brahmin? Even the gods don't know [all the details], not to mention mortals.

\vsnum{1.39: }I shall teach [these details to you] one by one, as far as I can, O great Brahmin, in the manner in which Brahmā taught Mātariśvan in the past, truthfully.

\vsnum{1.40: }The ten names [of cosmic rulers/worlds] associated with each of the eight directions in Brahmā's Egg, inside Śiva's Egg, [...], are being taught now, listen.

\vsnum{1.41: }[1] Saha, [2] Asaha, [3] Sahas, [4] Sahya, [5] Visaha, [6] Saṃhata, [7] Asabhā, [8] Prasaha, [9] Aprasaha, [10] Sānu: [these are] the ten Leaders in the East.

\vsnum{1.42: }[1] Prabhāsa, [2] Bhāsana, [3] Bhānu, [4] Pradyota, [5] Dyutima, [6] Dyuti, [7] Dīptatejas, [8] Tejas, [9] Tejā, [10] Tejavaho: [these are] the ten

\vsnum{1.43: }[leaders] in the direction of Agni [SE]. Now listen to [the names for] the direction of Yama [S], O Brahmin. [1] Yama, [2] Yamunā, [3] Yāma, [4] Saṃyama, [5] Yamuna, [6] Ayama,

\vsnum{1.44: }[7] Saṃyana, [8] Yamanoyāna, [9] Yaniyugmā, [10] Yanoyana. [1] Nagaja, [2] Naganā, [3] Nanda, [4] Nagara, [5] Naga, [6] Nandana, [7] Nagarbha, [8] Gahana, [9] Guhyo, [10] Gūḍhaja: [these are] the ten associated with [the South-West]. I shall teach you the [names] in Varuṇa's direction [in the west]. Listen, O Brahmin, learn from me.

\vsnum{1.46: }[1] Babhra, [2] Setu, [3] Bhava, [4] Udbhadra, [5] Prabhava, [6] Udbhava, [7] Bhājana, [8] Bharaṇa, [9] Bhuvana, and [10] Bhartṛ: these ten dwell in Varuṇa's direction [in the west].

\vsnum{1.47: }[1] Nṛgarbha, [2] Asuragarbha, [3] Devagarbha, [4] Mahīdhara, [5] Vṛṣabha, [6] Vṛṣagarbha, [7] Vṛṣāṅka, [8] Vṛṣabhadhvaja,

\vsnum{1.48: }and [9] Vṛsaja and [10] Vṛṣanandana: these are to be known properly as the ten leaders in Vāyu's direction [in the north-west], as I taught them, O Brahmin.

\vsnum{1.49: }[1] Sulabha, [2] Sumana, [3] Saumya, [4] Supraja, [5] Sutanu, [6] Śiva, [7] Sata, [8] Satya, [9] Laya, [10] Śambhu: [these are] the ten leaders in the north.

\vsnum{1.50: }[1] Indu, [2] Bindu, [3] Bhuva, [4] Vajra, [5] Varada, [6] Vara, [7] Varṣaṇa, [8] Ilana, [9] Valina, [10] Brahmā: [these are] the ten leaders in the Iśāna direction [in the north-east].

\vsnum{1.51: }[1] Apara, [2] Vimala, [3] Moha, [4] Nirmala, [5] Mana, [6] Mohana, [7] Akṣaya, [8] Avyaya, [9] Viṣṇu, [10] Varada: [these are] the ten [leaders] in the centre.

\vsnum{1.52: }Each of the ten deities[?] has a retinue of a hundred [deities]. Each one in [these groups of] a hundred [deities] is surrounded by a thousand.

\vsnum{1.53: }Each one in these [groups of] a thousand [deities] is surrounded by ten thousand [deities]. The ten thousand by a multitude of a hundred thousand. The hundred thousand is surrounded by a million,

\vsnum{1.54: }[that is] each one has a retinue of a million [deities] (\skt{niyuta}). [Then] each [of those] is surrounded by ten million [deities] (\skt{koṭi}), [they] by a hundred million (\skt{daśakoṭi} = \skt{arbuda}).

\vsnum{1.55: }Each one of the hundred million (\skt{daśakoṭi} = \skt{arbuda}) is surrounded by a billion (\skt{vṛnda}) bhṛta??? Each of those billion (\skt{vṛnda}) is surrounded by ten billion (\skt{kharva}) [deities].

\vsnum{1.56: }Each of those ten billion (\skt{kharva}) is surrounded by a hundred billion (\skt{daśakharva} = \skt{nikharva}). Each of those hundred billion (\skt{daśakharva} = \skt{nikharva}) is surrounded by one trillion (\skt{śaṅku}) [deities].

\vsnum{1.57: }Each of those one trillion (\skt{śaṅku}) is surrounded be ten trillion (\skt{padma}). Each of those ten trillion (\skt{padma}) is surrounded by a hundred trillion (\skt{samudra}).

\vsnum{1.58: }And each of those hundred trillion (\skt{samudra}) is surrounded by those whose number is one quadrillion (\skt{madhya}). Each of those quadrillion (\skt{madhya}) is surrounded by ten quadrillion (\skt{ananta}).

\vsnum{1.59: }Each of those ten quadrillion (\skt{ananta}) is surrounded by a hundred quadrillion (\skt{parārdha}). Each of those hundred quadrillion (\skt{parārdha}) is surrounded by two hundred quadrillion (\skt{para}). This is how it is taught, O Brahmin. [All] the possible numbers have been taught.

\vsnum{1.60: }Hear about the measurements [of the universe] briefly, O Brahmin, from me, I shall teach [you]. Listen to the extent [of the Brahmāṇḍa], O Brahmin! I shall teach it to you in concise manner. The body of the Egg is like that of the full moon at moonrise.

\vsnum{1.61: }The whole circumference of the Eggs has been declared by Brahmā to be \skt{koṭi} times a thousand \skt{koṭi} yojanas.

\vsnum{1.62: }The Sun shines from above from seven thousand and seven hundred \skt{koṭi} [height] ... twenty \skt{koṭi} gulma?? mūrdha?

\vsnum{1.63: }In brief the numbers pertaining to the measurements have been taught. The characteristics of the unmeasurable Brahmāṇḍa[s] have been taught.

\vsnum{1.64: }O true Brahmin, the Purāṇa[s of] 8,000,000 [verses] were taught by [1] Brahmā to [2] Mātariśvan [= Vāyu] in their entirety, in their true form.

\vsnum{1.65: }Vāyu abridged the verses and then gave [the Purāṇas] to [3] Uśanas. He [Uśanas] also abridged the verses, and [4] Bṛhaspati received them.

\vsnum{1.66: }Bṛhaspati taught 30,000 [verses] to [5] Sūrya [the Sun]. Divākara [= the Sun] taught 25,000 [verses] to [6] Mṛtyu [Death].

\vsnum{1.67: }Death taught 21,000 [verses] to [7] Indra. Indra taught 20,000 verses to [8] Vasiṣṭha.

\vsnum{1.68: }And he[, Vasiṣṭha taught] 18,000 [verses] to [9] Sārasvata. Sārasvata [taught] 17,000 [verses] to [10] Tridhāman.

\vsnum{1.69: }[Tridhāman] taught 16,000 verses to [11] Bharadvāja. [Bharadvāja] taught 15,000 verses to [12] Trivṛṣa.

\vsnum{1.70: }[Trivṛṣa] then [taught] 14,000 verses to [13] Antarīkṣa. [Antarīkṣa] taught 13,000 [verses] to [14] Trayyāruṇi.

\vsnum{1.71: }Trayyāruṇi, the great Brahmin, having abridged them again, taught 12,000 [verses] to [15] Dhanaṃjaya.

\vsnum{1.72: }Dhanaṃjaya, the great sage, handed [them] over to [16] Kṛtaṃjaya. [This recension was transmitted] from Kṛtaṃjaya, O great Brahmin, to [17] noble Ṛṇaṃjaya.

\vsnum{1.73: }Then from Ṛṇaṃjaya it was given to [18] Gautama, the great sage, from Gautama to [19] Bharadvāja, from him to [20] Dharmadvata.

\vsnum{1.74: }Then [21] Rājaśravas received it, then [22] Somaśuṣma. Then from Somaśuṣma [23] Tṛṇabindu received it, O Brahmin.

\vsnum{1.75: }Tṛṇabindu taught it to [24] Vṛkṣa, Vṛkṣa to [25] Śakti [the father of Parāśara]. Śakti taught it to [26] Parāśara, then [Parāśara] to [27] Jātūkarṇa.

\vsnum{1.76: }Jātukarṇa taught it to [28] [Vyāsa] Dvaipāyana, the great sage. Dvaipāyana, the great sage, gave it to Romaharṣa.

\vsnum{1.77: }He [Dvaipāyana] taught the Purāṇa[s] [consisting of] 12,000 [verses] to Romaharṣa, his brilliant son, [in the form that] has been revealed [to us] for the benefit of humankind. What else do you wish to know?


%%%%%%%%%
\vsnum{2.1: }Vigatarāga spoke: I, the Brahmin(? phps accept it) [rather: through you, a Brahmin], have listened to the concise description of the Brahmāṇḍa, it's extent, colour, form and the numbers associated with it.

\vsnum{2.2: }You mentioned the Śivāṇḍa as taught to be the receptacle of the Brahmāṇḍa [see 1.40ab]. What are its characteristics and how much is its extent?

\vsnum{2.3: }Whose dwelling/resting place is it [phps ālayana for ālaya] and [what] is the extent/proof of the one who dwells there? [maybe the number of inhabitants Flo] [Or: what is its extent and [who are its] inhabitants]? Who are the people there? And who is Prajāpati there?

\vsnum{2.4: }Anarthayajña spoke: Please don't ask me about the characteristics of the Śivāṇḍa, O Brahmin. How could even the gods have the power to really know and see...

\vsnum{2.5: }The path leading to it is not to be trodden, it is extremely secret and [...] There is no master or the opposite there, nobody to be punished and no punisher.

\vsnum{2.6: }There are no truthful or untruthful people there, no moral or immoral people, no wicked people, no hypocrisy, no thirst or envy.

\vsnum{2.7: }There is no anger or desire, no arrogance or discontent ([a]sūyaka). No envy or hatred, no cheaters and no jealousy.

\vsnum{2.8: }There is no disease, no aging, no grief and no agitation there. There are no inferior or superior people and there is nobody in-between.

\vsnum{2.9: }There are no privileged men or women there in Śiva's abode, no reproach or praise, no selfish or treacherous people.

\vsnum{2.10: }There is no pride or arrogance there, no cruelty or trickery and so on. There are no beggars and no donors there.

\vsnum{2.11: }Go without material desires (\skt{anarthin}), being there you'll be resting under a wishing tree. There is no karma there and no enemy. The era of strife [the Kali era] is not there and there is no fight.

\vsnum{2.12: }There is no Dvāpara era or Tretā or Kṛta. There are no Manvantaras (1 Manvantara = 1000 Kalpas) there and no Kalpas.

\vsnum{2.13: }No universal floods of destruction come, and there are no days and nights of Brahmā. There is no birth and death there and one never encounters catastrophes.

\vsnum{2.14: }Nobody is tied to the noose of hope and there is no passion or delusion. There are no gods and demons there and no Yakṣas, Serpents and Rākṣasas.

\vsnum{2.15: }There are no Ghosts nor Piśācas, no Gandharvas and no Ṛṣis. There are no asterisms and planets there, no Nāgas, Kiṃnaras or Garuḍa-like creatures.

\vsnum{2.16: }There is no recitation there or daily rituals, nobody performs the Agnihotra and there is no sacrificer. There are no religious observances and no austerities and no 'animal hell' [or: on animals and no hell].

\vsnum{2.17: }Nobody would be able to tell the extent of the god Īśāna's[??] powers starting with aiśvarya, not even in a hundred years.

\vsnum{2.18: }[Instead] I shall teach you all that are produced by Hara's wish one by one, excluding the gods and people, starting with the trees, the bushes and creepers.

\vsnum{2.19: }[Their?] height is two Parārdha, and [their?] width is the same. There are lovely flowers of different forms [there] and also lovely fruits.

\vsnum{2.20: }There are also golden trees and also gem trees, coral gem thickets and ruby plants.

\vsnum{2.21: }There are trees with twigs on which creepers with tasty roots reach for the tasty fruits. [REVISE] All of them can change their shapes on their own accord [just bending etc.?] and they fulfill man's desires and they whisper in a lovely way[?] [any language? maybe not].

\vsnum{2.22: }There [in the Śivāṇḍa], O Brahmin, all the subjects are the oceans of endless virtues. They are all equally beautiful and strong, and they shine like millions of suns.

\vsnum{2.23: }... is two Parārdha [yojanas] long and two Parārdha [yojanas] wide, and two Parārdha yojanas is its extension[?], O great Brahmin.

\vsnum{2.24: }Authority is not a number [cannot be expressed by a number? OR: there is no question of....?] neither is the Power of strength, O Brahmin. Down and up are no numbers [no question of going to heaven or hell?], and nobody goes to the Tiryañc [hell] [??? OR with iti: there is no horizontal extension?].

\vsnum{2.25: }I do not know the length and width of the Śivāṇḍa. Enjoyment is undecaying there, and there is no birth or death there.

\vsnum{2.26: }Inside the Śivāṇḍa, there is the dwelling-place of Īśāna's people [= Īśāna's region] [on] one and a half Para krore [yojanas? or that many people?], who shine like cow's milk [or the region shines?].

\vsnum{2.27: }They are all like the rising sun in the House of Tatpuruṣa [on] one and a half Para krore [yojanas? or that many people?] in the east.

\vsnum{2.28: }All of them are like collyrium in the southern direction, in the House of Aghora, [on] one and a half Para krore [yojanas?].

\vsnum{2.29: }In the western direction, in Sadyojāta's beloved House, [on] one and a half krore [yojanas?] they are like jasmine, the moon, like snowy rocks.

\vsnum{2.30: }In the northern direction, in Vāmadeva's House of one and a half krore [yojanas?] they are like saffron and water.

\vsnum{2.31: }Īśāna has five parts (kalā), [his Tatpuruṣa] face has four. Aghora has eight, and there are thirteen Vāmadeva[-kalā]s.

\vsnum{2.32: }Sadyojāta has eight parts. These parts, altogether thirty-eight, which liberate us from the ocean of existence, have been taught, O truest Brahmin.

\vsnum{2.33: }Those who explore the Truth should know the numbers, the colours and directions associated with each one [of Śiva's faces] in the way taught above.

\vsnum{2.34: }If one has the intention to go to the Śivāṇḍa [if he is 'pulled' towards it], one should practise Śiva yoga regularly. Without Śiva yoga, O Brahmin, it is impossible to go there.

\vsnum{2.35: }[Even] by [performing] millions of sacrifices such as the Aśvamedha, or all the difficult austerities, for a hundred Kalpas, it is impossible to get there even for the gods, O great ascetic.

\vsnum{2.36: }By [merely] bathing and performing austerities at all the sacred places such as the Gaṅgā, even the honorable Ṛṣis will not be able to get there.

\vsnum{2.37: }Or by donating the oceans of the seven islands with all their gems to a Veda expert, O Brahmin, having faith and devotion, one will not be able to go there without meditation. [This is a] certainty.

\vsnum{2.38: }He who destroys his own body and gives it without hesitation to those who are in need of it, or gives away his wife, his son and his possessions or his own head to those in need, or by [performing] other difficult deeds, will not be able to go there [by merely doing these].

\vsnum{2.39: }He who has completed the sacrifices, the pilgrimages, the austerities, the donations, the study of the Vedas, will experience those enjoyments that the Brahmāṇḍa offers, still being subject to time and death.

\vsnum{2.40: }Dharma decays with time that is sent by... Like a circle of burning coal, time goes round and round. Time is called \skt{kāla} because of the waves (kalana) of the three divisions of time [past, present, future].


%%%%%%%%%
\vsnum{3.1: }Vigatarāga spoke: Why do they call [Dharma] Dharma? And how many embodiments (\skt{mūrti}) is he known to have? He is known as a bull: how many legs does it/he have? How many are his paths?

\vsnum{3.2: }I have become curious [about these questions]. Put an end to my doubts for good. Whose son is [Dharma], O best of sages? How many children does he have?

\vsnum{3.3: }Anarthayajña spoke: Well, the root [sic!] \skt{dhṛti} (`resolution') is said to be a synonym [of \skt{dharma}]. It is called Dharma because it supports (\skt{āDHĀRaṇa}) and because it is great (\skt{MAhattva}).

\vsnum{3.4: }The four-legged Bull is the embodiment of both Śruti and Smṛti. The four \skt{āśrama}s are taught by the wise to be [the four legs of] Dharma. [or rather: ... which is Dharma as made up of the four āśramas... kīrtitaḥ!]

\vsnum{3.5: }And the paths of Dharma are five. Listen, O Brahmin: [existence as] gods, men, animals, [existence in] hell and [as] immovable things [such as plants and rocks] etc.

\vsnum{3.6: }Eternal Dharma was born after splitting Brahmā's heart. He has beautiful wives, thirteen in number, with nice waists.

\vsnum{3.7: }They are Dakṣa's daughters, [called] Śraddhā and so on. They have huge eyes and they are beautiful. and they are charming. Numerous sons and grandsons were born to him. This is the emergence of Dharma. What more do you wish to hear?

\vsnum{3.8: }Vigatarāga spoke: I would like to hear about Dharma's wives according to the truth[?] and about each one of the sons born to them. Teach me, O great ascetic.

\vsnum{3.9: }Anarthayajña spoke: [Dharma's wives are:] [1] Śraddhā (`Faith'), [2] Lakṣmī (`Prosperity'), [3] Dhṛti (`Resolution'), [4] Tuṣṭi (`Satisfaction'), [5] Puṣṭi (`Growth'), [6] Medhā (`Wisdom'), [7] Kriyā (`Ritual'), [8] Lajjā (`Modesty'), [9] Buddhi (`Intelligence'), [10] Śānti (`Tranquillity'), [11] Vapus (`Beauty'), [12] Kīrti (`Fame'), [13] Siddhi (`Success'), [all] born to Prasūti [Dakṣa's wife].

\vsnum{3.10: }Śraddhā's son is Kāma (`Desire'), Darpa (`Pride') is said to be Lakṣmī's son. Dhṛti's son is Niyama (`Rule'), Saṃtoṣa (`Satisfaction') is Tuṣṭi's son.

\vsnum{3.11: }To Puṣṭi was born a son [called] Lābha (`Profit'). Medhā's son is Śruta (`Sacred Knowledge'). Kriyā's sons are Abhaya (`Freedom from danger'), Daṇḍa (`Punishment') and Samaya (`Law').

\vsnum{3.12: }Lajjā's son is Vinaya (`Discipline'), Buddhi's son is Bodha (`Intelligence'). Lajjā has two [more] sons: Sudhiya[/Sudhī] (`Wise') and Apramāda (`Cautiousness'). [or one more son only: the wise Apramāda?]

\vsnum{3.13: }Kṣema (`Peace') is to be known as Śānti's son, Vyavasāya (`Resolution') is Vapus' son. Yaśas (`Fame') is Kīrti's son, Sukha (`Joy') was born to Siddhi. [This is how] the sons of Dharma in the era of Svāyambhuva [Manu] were known.

\vsnum{3.14: }Vigatarāga spoke: How does Dharma have two embodiments? Tell me, O great ascetic. I am extremely intrigued. Cut my doubts concerning [this] knowledge.

\vsnum{3.15: }Anarthayajña spoke: Dharma's embodiment is said to consist of Śruti and Smṛti. The characteristics of the Śrauta [tradition] are an association with a wife [i.e.\ marriage] and with the fire ritual, and sacrifice. The Smārta [tradition] [focuses on] the conduct (\skt{ācāra}) of the classes (\skt{varṇa}) and life-stages (\skt{āśrama}) which is connected to rules and regulations (\skt{yama-niyama}).

\vsnum{3.16: }Now hear the classification of both the \skt{yama} and \skt{niyama} rules. Non-violence, truthfulness, not stealing, kindness, self-restraint, the rule of taboos,

\vsnum{3.17: }virtue, carefulness, charm, honesty: these are the ten \skt{yama}s. The wise say that there are five subclasses to each.

\vsnum{3.18: }I shall teach you about non-violence and the other [\skt{yama}-rules]. Listen carefully, O Brahmin. Frightening and beating [other people], tying [someone] up, killing and the destruction of [other people's] livelihood: violence is said by the wise who see the truth to be of [these] five types.

\vsnum{3.19: }Cruel people beat [other people] with sticks, clods of earth [understand: they stone them], whips and other [objects] in the everyday world. Their bodies broken by the same blows, they receive the capital punishment.

\vsnum{3.20: }[Others,] tie up [people] at their feet and their arms and chests. [These,] bound by [with?] their hair and [on their?] necks, die without being wounded. This is the capital punishment for tying up [other people].

\vsnum{3.21: }He who frightens [other people] with the terrible danger of enemies and thieves, with lions, tigers, elephants or snakes, will be destroyed [by the above] or by other horrors.

\vsnum{3.22: }He who robs somebody's money is to be punished by the same person. He is [to be] hit by those whose livelihood got damaged by him as many times [as the victims are].

\vsnum{3.23: }[Those who kill other people] with poison, fire, arrows, swords, or by the force of magic or yoga are called murderers by the sages who see the truth, O great Brahmin[, and to be killed by the same methods].

\vsnum{3.24: }Non-violence is the highest Dharma. He who abandons it is a wicked person. It is free of pain and trouble, it yields the fruits of all [other] Dharmic teachings [in itself].

\vsnum{3.25: }There isn't a bigger fool than he [who abandons it is]. There is no bigger mental darkness [than the abandonment of non-violence]. There is no greater suffering or greater infamy.

\vsnum{3.26: }There is no greater sin or a more effective poison. There is no greater ignorance, there is nothing worse, O great ascetic.

\vsnum{3.27: }He who does not harm the four types of living beings beginning with plants is the best person, having compassion for all creatures.

\vsnum{3.28: }He who always has compassion for all creatures is the [true] Pandit. He is the [true] sacrificer, the [true] ascetic, he is the donor, the one with a firm vow CHECK.

\vsnum{3.29: }Non-violence is the supreme sacred place. Non-violence is the highest austerity. Non-violence is the highest donation. Non-violence is the highest joy.

\vsnum{3.30: }Non-violence is the supreme sacrifice. Non-violence is the supreme religious observance. Non-violence is supreme knowledge. Non-violence is the supreme ritual.

\vsnum{3.31: }Non-violence is the highest purity. Non-violence is the highest self-restraint. Non-violence is the highest profit. Non-violence is the greatest fame.

\vsnum{3.32: }Non-violence is the supreme Dharma. Non-violence is the supreme path. Non-violence is the supreme Brahman. Non-violence is supreme Śiva.

\vsnum{3.33: }One should refrain from meat-consumption. One should not even desire it mentally. He who abandons meat will receive a great reward.

\vsnum{3.34: }He who wishes to nourish his own flesh with the flesh of other [beings], outside of worshipping the ancestors and the gods, is the biggest sinner of all.

\vsnum{3.35: }During the \skt{madhuparka} offering and during a sacrifice, during rituals for the ancestors and the gods: only in these cases are animals to be slaughtered and not in any other case. [This is what] Manu taught.

\vsnum{3.36: }Should he buy it or procure it himself or should it be offered by others, if he eats meat, he will not sin if he first worships the gods and the ancestors.

\vsnum{3.37: }[People who know] the Vedas and [perform] sacrifices and austerities and [visit] sacred places, donate, [are of] good conduct, [perform] rituals and [keep] religious vows [but eat meat] will not [be able to] enjoy even a tiny portion of [such rewards that] [those] people [receive] who have given up meat.

\vsnum{3.38: }The deer and the goats, the sheep, the cows and other [animals] wander in the world happily and in great strength [just] from eating leaves and grass.

\vsnum{3.39: }Monkeys eat fruits, Rākṣasas prefer blood. The fruit-eating monkeys defeated all the Rākṣasas [as the Rāmāyaṇa tells us].

\vsnum{3.40: }Therefore one should not crave meat in the hope of gaining strength, O Brahmin,  in order to be able to draw a bow with force, or out of fear of the danger coming from the enemy.

\vsnum{3.41: }One cannot be equal to someone who refrains from violence by [merely] wishing to make donations and perform sacrifices. [He will have] fame and glory in this world and the supreme path in the other.

\vsnum{3.42: }A person who refrains from violence will gain, no doubt about it, the [same] meritorious rewards that others would get by donating the three worlds filled with jewels and gems in their entirety to an excellent Brahmin, by performing a thousand [times] ten trillion (\skt{padma}) [times] ten thousand (\skt{ayuta}) \skt{koṭīyajña} (= koṭihoma?) sacrifices, by donating the earth [to a priest] as sacrificial fee, and by bathing [at] a thousand times ten million times a million (\skt{niyuta}) sacred places at once,


%%%%%%%%%
\vsnum{4.1: }Anarthayajña spoke: The state of being real (\skt{sad-bhāva}) is called Truth (\skt{sat-ya}). Alternatively, it is also a notion that originates in perception. [Also, it is] relating things that correspond to reality. This is how Truth is discussed. REVISE

\vsnum{4.2: }He who endures severe abuse and beating etc. but keeps quiet, his self being conquered, is said to be [an example of] truth.

\vsnum{4.3: }If one is being interrogated any time with a sword lifted to strike him down, then it is not the truth that is to be spoken. [In this case,] a lie is called truth.

\vsnum{4.4: }A person who is walking on the road and is afraid of being killed, should not reply [to people who are potentially dangerous] even if they ask him. That is also called Truth.

\vsnum{4.5: }A lie does not hurt when it is connected with joking, with women, O king[!], at the time of marriage, at the departure from life and when one's entire wealth is about to be taken away. They call these five kinds of lies Truth.

\vsnum{4.6: }Since Truth is the supreme Dharma with respect to gods, humans and animals[?], Truth is the best, the most preferable. Truth is the eternal Dharma.

\vsnum{4.7: }Truth is an unmanifest ocean. Truth yields imperishable pleasures. Truth is the ship that carries you to the other world. Truth is the wide path.

\vsnum{4.8: }Truth is said to be the desired path. Truth is the supreme sacrifice. Truth is a pilgrimage place, a supreme pilgrimage place. Truth is an endless donation.

\vsnum{4.9: }Truth is morality, austerity, knowledge. Truth is purity, self-control and tranquillity. Truth is the ladder upwards. Truth is fame and glory and happiness.

\vsnum{4.10: }[When] a thousand Aśvamedha sacrifices and Truth are measured on a pair of scales, Truth indeed surpasses a thousand Aśvamedha sacrifices.

\vsnum{4.11: }The Sun shines because of Truth. The Earth stays in place by Truth. The winds blow because of Truth. Water is cooling through Truth.

\vsnum{4.12: }The oceans dwell in Truth because of their encounter[?] with Priyavrata [Manu's son]. Govinda abides in Truth because He [as Vāmana] stopped [Mahā]Bali [in spite of the fact that this was achieved by a trick].

\vsnum{4.13: }Fire burns with Truth. The Moon rises by Truth. It is because of Truth that the Vindhya mountain stands in place and that although is was growing it is not growing [anymore].

\vsnum{4.14: }The [mythical] Lokāloka mountains are located in Truth. Mount Meru stands by Truth. The Vedas abide in Truth. Dharma is rooted in Truth.

\vsnum{4.15: }The milk a cow yields is Truth. Ghee in milk is there as Truth. The soul dwells in the body in Truth. The eternal soul is Truth.

\vsnum{4.16: } If Truth alone (ekena) is obtained, Dharma is surely accomplished. By the heroism of Rāma Rāghava, Truthfulness was well-guarded, more than anything else.

\vsnum{4.17: }This is how [I] taught the rules of Truth to you, O virtuous one, to favour the whole world. What else do you wish to hear?

\vsnum{4.18: }Vigatarāga spoke: I can't have enough of learning about [this teaching of] your[s on] Dharma. Teach me further than this, O great ascetic.

\vsnum{4.19: }Anarthayajña spoke: Now listen to [my teaching about] stealing, O great Brahmin, which is taught to be of five kinds. Firstly, [listen to] theft [lit. `taking what has not been given'], then bribery, cheating with weights, cheating with scales, and the fifth kind, robbery.

\vsnum{4.20: }Theft is when somebody else's wealth is taken away through a bold/impudent crime. [A person who commits such a crime] is foolish even if he remains unnoticed [or: kept back from the crime?].

\vsnum{4.21: }O great Brahmin, listen to bribery, which defiles Dharma. A sum of money taken in order to annul a punishment [or something that is to be done, in order to become exempt from a duty] is a bribe. Therefore this [also] should be considered as such [i.e.\ as stealing because] it is committed out of greed.

\vsnum{4.22: }[Even if] somebody wants to protect families by the method of cheating with weights, that person should be considered a thief, because he takes away other people's wealth.

\vsnum{4.23: }[The case is similar] if somebody takes away somebody else's belongings by the method of cheating with scales. Other people, deceitful swindlers (\skt{kūṭa-kāpaṭika}) [can also] have the characteristics of thieves.

\vsnum{4.24: }[If] someone, by deceit or by force, snatches away the wealth of weak and honest people or children [and women and simpletons?], that morally corrupt thief is [rightly] called a thief.

\vsnum{4.25: }There is no sin equal to stealing. There is no crime (\skt{adharma}) equal to it. There is no ill-fame comparable to that of being a thief. There is no bad-conduct comparable to being a thief.

\vsnum{4.26: }There is no such ignorance as stealing. There are no bigger rouges than thieves. There is nobody as ignorant as a thief. There is not a lazy person who is comparable to a thief.

\vsnum{4.27: }There is nobody as detestable as a thief. There is nobody as much of an enemy as a thief. There is no such suffering as stealing. There is nobody more disgraced than a thief.

\vsnum{4.28: }Some [thieves] take away [other people's] wealth in disguise, some in broad daylight. Other wicked people take money from deposits, and some people steal through fraud. Some gather wealth by forging documents, others steal from stolen money??? Some people's wealth is from a purchased [child?? (\skt{krīta})]. .............. These are considered the vilest.

\vsnum{4.29: }There are no bigger idiots than thieves, who are wicked people without Dharma and Artha. As long as he lives, he trembles in fear of the king, wailing. Having received his punishment, he gets into severe and [in]tolerable difficulties, propelled by [his] karma. When his time comes, he dies and goes to hell, weeping vehemently.

\vsnum{4.30: }Having spent ten million aeons of suffering, they emerge from hell to the state of animal existence. Similarly [CHECK eka], after roaming about in animal existence for a hundred and one times ten million years, then they reach the status of human existence on earth which is full of poverty and disease. Then abandoning all one's karmas, the causes of suffering, one seeks refuge in Śiva.

\vsnum{4.31: }The one who is hostile towards the eight-formed Śiva, he who hurts his mother or father, he who is hostile towards cows or guests: these are the five types of cruel people.

\vsnum{4.32: }Śiva in his manifest form (\skt{sākṣāt}) is of eight forms, with the five elements (vyoman! NOTE), and the Sun, the Moon, and the sacrificer. [He who] disgraces [any of these] is a cruel person.

\vsnum{4.33: }The father is to be considered similar to the sky, he is the cause of one's birth. .... 

\vsnum{4.34: }The mother is more venerable than the earth. Who would not praise a mother? By that [praise], sacrifices, donations, austerities and [the study of] the Vedas, all will be completed.

\vsnum{4.35: }Cows are a sacred [auspicious/purifying Judit] blessing, they are the gods of the gods. Cows contain in themselves all the gods. That's exactly why one should not hurt them.

\vsnum{4.36: }Cows are the protectors of the world as if the world were their new-born [calf], there is no doubt about it. The collection of [the five products of the cow, the \skt{pañcagavya},] ghee, milk, curd, and [the cow's] urine and dung [is auspicious].

\vsnum{4.37: }People who drink the five products of the cow, the five nectars, the five holy and pure [substances] [or: clarified with a strainer??], will obtain the fruits of a horse sacrifice, and then reach the undecaying heavens.

\vsnum{4.38: }There is no wealth comparable to [having] a cow. They yield milk, they draw [a plough etc.]. [As] they roam under the sky, feeding on grass, they issue nectar. When given to Brahmins, they deliver the family [from \skt{saṃsāra}/the suffering experienced in hell].

\vsnum{4.39: }He who never fails to serve the cow daily [e.g. with a handful of grass], and he who tends to the cows' service, will obtain the merits of all sacrifices, austerities and donation [because] he is one who is kind to it (\skt{tām}?) [i.e. to the cow].

\vsnum{4.40: }He who looks after a guest, he who respects a guest, he who worships a guest, he who praises a guest,

\vsnum{4.41: }he who does not harm a guest, he who does not commit a fault towards a guest, he who does kind things to a guest, he who attends to the needs of a guest, he who makes a guest satisfied: his merits are endless.

\vsnum{4.42: }He should offer [the guest] a seat, water-offering, feet-washing water [or: °pātreṇa?], water for washing his feet[?], or gifts of food and clothes, or all [of these].

\vsnum{4.43: }He who worships the guest by [offering him] his own son, wife or himself with willingness and with a brave and non-hesitating mind,

\vsnum{4.44: }and does not ask [the guests about their] lineage, Vedic affiliation (\skt{caraṇa}), studies, country or birth, and imagines mentally, with devotion, that it is Dharma himself who has arrived,

\vsnum{4.45: }[will obtain all the fruits of] thousands of Aśvamedha sacrifices and hundreds of Rājasūya sacrifices, a thousand Puṇḍarīka sacrifices and the fruit of [visiting] all the pilgrimage places and [performing] all the austerities;

\vsnum{4.46: }he whose guest is satisfied [and] he who can abandon the sentiment of cruelty, will obtain all the merits of [the above], there is no doubt about it.

\vsnum{4.47: }... he who [does not] know [how to greet his] guests ... will never reach the path ... ? Therefore one should go up to the arriving guest with respectfully joined palms.

\vsnum{4.48: }By one \skt{prastha} of coarsely ground grains given to a guest, an extremely great sacrifice was performed [so to say], and his [the Brahmin's and his family members'] bodies (\skt{svaśarīraṃ}) reached heaven.

\vsnum{4.49: }The mongoose related [this story in the Mahābhārata] in the past in detail, O great Brahmin, and you've known it already. The story of the \skt{prastha} is well-known.

\vsnum{4.50: }Self-restraint of humans is in itself the collected essence of Dharma. Self-restraint is Dharma, Self-restraint is heaven, Self-restraint is fame, Self-restraint is happiness.

\vsnum{4.51: } Self-restraint is sacrifice, Self-restraint is a pilgrimage-place, Self-restraint is merit, Self-restraint is religious austerity. If one has no Self-restraint, there is no Dharma, [while] Self-restraint yields a multitude of desired objects.

\vsnum{4.52: }The elephant, the fish, the moth, the bee and the deer are without Self-restraint. The senses are the skin, the tongue, the nose, the eye and the ear.

\vsnum{4.53: }Each of these sense faculties are hard to conquer and all are known to be fatal [if unconquered]. If one masters Self-restraint, the [one with a?] lack of Self-restraint will die.???? 

\vsnum{4.54: }In the case of the deer, death comes about because of hearing [when hunters use buck grunts]. Moths die because[?] of their eyes [as they are attracted to the light of a lamp]. Bees perish because of their smelling, fish because of their tongues.

\vsnum{4.55: }The elephant perishes because of touch, not being able to tolerate being in fetters [?]. How much more true it is for those who enjoy all five [senses]! Why should death come as a surprise for them?

\vsnum{4.56: }Purūravas [perished] by excessive greed, Daṇḍaka by excessive desire, Sagara's sons by excessive pride, Rāvaṇa by excessive haughtiness,

\vsnum{4.57: }Saudāsa by excessive anger, the Yādavas by excessive drinking, Māndhātṛ by excessive desire, Nahuṣa by contempt for Brahmins,

\vsnum{4.58: }[Mahā]bali perished by excessive donations, Arjuna by excessive heroism, King Nala by excessive gambling, Nṛga by taking a cow.

\vsnum{4.59: }[For] a person who is without Self-restraint, O great Brahmin, there is no heaven, liberation or happiness. O Brahmin, people without Self-restraint are the destruction of knowledge, Dharma, family and fame.

\vsnum{4.60: }[For] a person without taboos there is neither the other world, nor this life. In the case of a person without taboos there is no Dharma or religious austerity.

\vsnum{4.61: }These five are taboo: women who are not depending on oneself, others' wealth, taking away others' lives, hurting others and [consuming] others' food.

\vsnum{4.62: }Listen, O great Brahmin, the wise should always treat women who are not dependent on oneself as taboo, [be she] a queen, a Brahmin's wife, a wandering religious mendicant, a relative or of another family.

\vsnum{4.63: }Listen further to something else with regards to others' wealth. [It may include] gaining wealth through unlawful means, when somebody takes away other people's wealth by cheating with [small] weights of an \skt{āḍha[ka]} or a \skt{prastha} and with scales.

\vsnum{4.64: }O Brahmin, the wise should regard the taking away [of others'] lives as taboo. Wild and domesticated animals, [serpents] that live in holes and those that walk on their feet [are examples of life forms not to destroy].

\vsnum{4.65: }And what is the hurting of others? Listen, O Brahmin, I'll tell you briefly. He who is hostile to the gods, Brahmins, gurus, mothers and guests [hurts others].

\vsnum{4.66: }As regards other people's food, eating together with people whose food is not to be accepted (\skt{abhojyeṣu}) is taboo, [e.g.] after birth or death [in the family], in case there are vendors of alcohol, in the case of a family having lost their caste, and in the case of a Naṭa [dancer caste?].

\vsnum{4.67: }Those people who cling to [the prohibition of] the five kinds of taboo [and thus] seek heaven, wealth and liberation, will reach eternal faultlessness in this world, embellished with fame and glory. [A person like that] will obtain wisdom, intelligence, [knowledge of] the Śruti and Smṛti traditions, and honour forever. He will be kindness itself[?] and he will obtain an extra long life, no doubt.

\vsnum{4.68: }The four cases of observing silence, [victory over] the four enemies, the four sanctuaries/planes, the four meditations, and the four legged [Dharma] are called the five ways of being virtuous[?].

\vsnum{4.69: }I shall tell you about the four cases of observing silence. Listen, be attentive. One should avoid [1] violent [words], [2] slanderous [words], [3] lies, and [4] idle [talk].

\vsnum{4.70: }The fourfold enemy, desire, anger, greed and delusion, is to be destroyed. He who destroys [these] enemies will become sinless.

\vsnum{4.71: }I shall teach you the four sanctuaries/planes. Listen, O Brahmin. Compassion, sympathy in joy, indifference, and benevolence are the four sanctuaries/planes.

\vsnum{4.72: }I shall now teach you the four meditations, which will liberate you from mundane existence (\skt{saṃsāra}). Meditation is taught to be fourfold: of the Self, \skt{vidyā}, \skt{bhava} [= Śiva?] and the subtle one.

\vsnum{4.73: }The \skt{tattva} of the Self is the \skt{ātman}. \skt{Vidyā} in the five in a fivefold way[??]. They call the thirty-sixth the imperishable one, [and] the subtle \skt{tattva} has no attributes.

\vsnum{4.74: }Dharma is said to be four-legged [as] it rests on the four \skt{āśrama}s, [those of] the householder, the chaste one, the forest-dweller and the mendicant.

\vsnum{4.75: }Virtuous are those who know these thoroughly, O great Brahmin. [They will experience] the purification of all sins and the growth of merits.

\vsnum{4.76: }One's life-span, fame and glory and happiness grow only through virtue (\skt{dhanya}). [In] a virtuous person piece, prosperity, memory/tradition? and intelligence will arise.

\vsnum{4.77: }There are five areas of negligence. I shall teach them to you, listen. Murdering a Brahmin, drinking alcohol, stealing, having sex with the guru's wife: they call these Grievous Sins. The fifth is when one is connected with them [i.e. with these sins or with people involved in these sinful acts].

\vsnum{4.78: }A lie concerning one's superiority, a slander that reaches the king's ear, and false accusations against an elder are equal to killing a Brahmin.

\vsnum{4.79: }Defaming a Brahmin or the Ṛgveda, being a false witness, murdering a friend, eating unfit or forbidden food are six [deeds that are] equal to drinking alcohol.

\vsnum{4.80: }Sexual intercourse with a female relative, with an unmarried girl, with women of the lowest castes, with the wife of a friend or of one's own son are said to be equal to violating the guru's bed.

\vsnum{4.81: }Stealing/taking away deposits, people, horses, silver, land, diamonds, or gems are said to be equal to stealing gold.

\vsnum{4.82: }If a man takes parts in these four [i.e. \skt{brahmahatyā, surāpāna, stena, gurvaṅganāgama}], that is the fifth Grievous Sin. By this all [of them] have been explained. These five kinds of negligence are to be avoided, O great Brahmin.

\vsnum{4.83: }[Charm has five types:] bodily, verbal and mental charm, [charm of] the eyes and [of one's] thoughts pañcamaḥ. Giving [others] a friendly glance [is commendable] and one should avoid cruel thoughts.

\vsnum{4.84: }One should meditate with a tranquil mind and should speak [to other people using] gentle words. [When] respectable people arrive at one's own hermitage, [one should] present them with as many gifts as one can,

\vsnum{4.85: }with gifts of fire-wood, water and fire. [If] fire-wood, fire and water are easily available [but] are not given [as gift] or [if the phrase] `Live [for a hundred years]!' is not uttered [by him] when [somebody else] sneezes, what reward could there be for him in the afterlife?

\vsnum{4.86: }The sages who see the truth praise five types of sincerity. [Sincerity] in action, in livelihood, in prosperity, in gratifying others [and ...?]. A sincere person does not rejoice in women, wealth, bribery and property.

\vsnum{4.87: }Sincerity [means] no sacrifice [performed] idly. Sincerity [means] no austerity [performed] idly. Sincerity [means] no donation [given] idly. Sincerity [means] no fires [kindled] idly.

\vsnum{4.88: }The sense faculties of a sincere person are firm even when he is delighted. The gods always live inside the body of a sincere person.

\vsnum{4.89: }Thus has been taught this section on the \skt{yama}-rules, O great Brahmin. Humans should follow them to reach happiness here and in the other world. He'll live by Śaṅkara's command with his filth of sins destroyed. He'll become a ruler of the world [that he subjugates] under one royal umbrella.


%%%%%%%%%
\vsnum{5.1: }Vigatarāga spoke: [Please] now teach me the true nature of the Niyama-rules in detail. It is comparable to a speech of ambrosia. I have become curious to hear [it]. [It was?] burnt by the fire of Prakṛti, sprinkled with the water of knowledge[?!]. There is no satisfaction [yet] in the Dharmas [for me]. ...[perhaph \skt{apara-vadam ataj-jñā... or apara[ṃ] vada me tajjña? mata-jña?}].

\vsnum{5.2: }Anarthayajña spoke: I shall teach you something else that is nice to hear, O great Brahmin: the particular part[s, for kalā; or for kalpa?] of Niyama are of five types [each]. It is the essence of Dharma, dear to Hari, Hara and the sages, O great Brahmin, the destruction of the impurity of the Kali age, generally[?] known as liberation.

\vsnum{5.3: }Purification, sacrifice, penance, donation, Vedic study and the restraint of sexual desire, religious observances, fasting, taciturnity, and bathing: these are the ten Niyamas.

\vsnum{5.4: }From among these, now I shall tell you the particulars of purification [first], and [then] the others. [1] Bodily purity, [2] [purity of] food, [3] [purity of] property[?], [4] [purity of] conduct[?], and the fifth, [5]...?

\vsnum{5.5: }He should not beat or tie or kill [any living being]. When this concerns others' wives and property, it is called bodily purity.

\vsnum{5.6: }The cleanliness of the ears, O great Brahmin, and of the anus, the loins, the mouth etc. [is also bodily purity]. The purity of the mouth [comes from] sipping water when eating, speaking,

\vsnum{5.7: }[after] the emission of urine and faeces, and [before] the worship of gods. The wise one should clean his anus and his loins with clay and water.

\vsnum{5.8: }One [portion of clay] for the loins, five for the anus, and ten for one [the left] hand. [Then] seven is to be applied for both [hands] by him who wishes cleanliness with clay.

\vsnum{5.9: }This is the purification for the householder (\skt{gṛhastha}), twice as much for the chaste one (\skt{brahmacārin}), three times as much for the forest-dweller (\skt{vānaprastha}), four times as much for the ascetic (\skt{yati}).

\vsnum{5.10: }I shall teach you the rules of purity with food. Listen, pay great attention. He should eat [as much] food [that fills] two quarters [of the stomach] and drink water [that fills] one quarter. In order to be able to practise breath-control, he should save the remaining quarter.

\vsnum{5.11: }[By] the wise one['s applying] the six soft and sweet juices, [which are] the six juices in food, the disturbances of the \skt{dhātu}s and the terrible illnesses will disappear.

\vsnum{5.12: }He should not eat foods that are forbidden and he should not drink drinks that are forbidden. He should not go where he is not allowed to and he should not say what is improper.

\vsnum{5.13: }He should avoid garlic, onion, \skt{gṛñjana} onion, mushrooms, buffalo meat? and pork, following the rules.

\vsnum{5.14: }He should not eat \skt{chattrāka} mushrooms, village hog, and cow flesh. He should also avoid sparrows, pigeons, and water-birds.

\vsnum{5.15: }He should also avoid [eating] swans, cranes, \skt{cakravāka} birds, dogs, parrots and hawks, crows, owls, \skt{balāka} cranes, fish etc.

\vsnum{5.16: }He should avoid everything that is ritually impure or polluted. He should also completely avoid those vegetables, roots and fruits that are prohibited.

\vsnum{5.17: }In the books of Manu, in the Purāṇas, in Śaiva texts, and in the Bhāratasaṃhitā (= the Mahābhārata), the practice of purity is definitely expanded in full.

\vsnum{5.18: }Now you have asked me [? about it], and I taught it [to you] in a condensed form. He who speaks the truth is pure. He who engages in yogic meditation is pure.

\vsnum{5.19: }He who avoids violence and is restrained is pure. He whose patience has become compassion is pure[???]. Of all the [ways of] purification, material purification is taught to be the highest.

\vsnum{5.20: }For he who is pure with regards to material things is truly pure, and not he who [only] uses clay and water [i.e. who performs only ordinary baths]. When purification pertains to the body, to speech and to the mind, that is purity of all things.

\vsnum{5.21: }If a person knows the rules of purity and impurity, he will surely (niścayaṃ?) gain happiness at the end of time, eternally embellished with glory and fame. He has reached here in this world all the merits that the books on true Dharma teach, i and at the end of his life he will undoubtedly reach the desired path in the other world.


%%%%%%%%%
\vsnum{6.1: }[Anarthayajña spoke:] Now I shall teach you the five types of sacrifice, O excellent Brahmin, for [your] success in Dharma and liberation. Listen carefully, O Brahmin! 

\vsnum{6.2: }Material sacrifice, sacrifice through work, sacrifice through recitation, knowledge and meditation: I shall teach you these five one by one.

\vsnum{6.3: }Material sacrifice includes the following: the worship of fire etc., the performance of the ritual of Agnihotra, oblations on the eight day after full moon, oblations offered at new and full moons, and the rituals for the ancestors.

\vsnum{6.4: }The sacrifice through work is the construction of a grove, a park, a pond or a temple with one's own hands.

\vsnum{6.5: }Next I shall teach you the sacrifice with recitation, the bestower of the fruits of heaven and liberation. One should recite the Vedas, the Śivasaṃhitā [= Śivasaṃkalpa? or rather śaivaṃ bhāratasaṃhitaṃ ca?],

\vsnum{6.6: }the epics and the Purāṇas: this is called sacrifice with recitation. He who is knowledgeable about inference CHECK and reasoning, [and knows that] ``this is [proper] action; the other is improper action'',

\vsnum{6.7: }and views [things through?] the eyes of science is called [a person performing] sacrifice through knowledge. I shall teach you concisely about sacrifice through meditation. Listen to me.

\vsnum{6.8: }Meditation was taught by Hari in the past as of five kinds. [Meditation of] the Sun, the Moon, Fire, Crystal and the subtle Tattva as fifth.

\vsnum{6.9: }First it is the Sun [that should be meditated upon], which is said to be Prakṛti Tattva. He should visualize the Moon in its centre: that is said to be Puruṣa [Tattva].

\vsnum{6.10: }In the centre of the Moon disk, he should visualise a flame, a fire. That is said to be Prabhu Tattva, the destroyer of birth and death.

\vsnum{6.11: }In the centre of the ring of fire, he should visualize a spottless crystal. That is said to be Vidyā Tattva, the never-born, imperishable Cause.

\vsnum{6.12: }In the centre of the disk of Vidyā, he should visualize the highest Tattva, never-heard, unparalleled one, undecaying and imperishable Śiva. The fifth Tattva of the sacrifice through meditation has been taught in short.

\vsnum{6.13: }Vigatarāga spoke: Teach me: what are the fruits of [reaching] each Tattva? Which worlds can be attained and how much time [can one spend there], O great ascetic?

\vsnum{6.14: }Anarthayajña spoke: The first [world to reach] is Brahmaloka, through the meditation on the first Tattva, Prakṛti. He will rejoice [there] happily like Śiva for millions of aeons.

\vsnum{6.15: }If one dies while meditating on the second Tattva, Puruṣa, one goes to Viṣṇuloka from this world, [and will live there] happily for billions of aeons.

\vsnum{6.16: }Should one die while meditating on the third Tattva, Prabhu, one can live in Śivaloka continuously for a hundred billion aeons.

\vsnum{6.17: }If he visualizes Vidyā Tattva, [i.e.] Sadāśiva [or sadā śivam?] he can reach [His] immortal, diseaseless, imperishable world [and can live there] well beyond endless aeons[?].

\vsnum{6.18: }The fifth one, the subtle Śivatattva dwells in the Self. There is no counting of time there and he will be rejoicing [there] together with Śiva.

\vsnum{6.19: }[If] he practises the five meditations, there is no rebirth and no more fetters of transmigration. O excellent Brahmin, [the Lord] should be seeked, a wishing tree of desires, [as] he burns away existence. Liberation comes within one single birth! People, why should you not strive [for it]! [This is known] as the destroyer of all impurity. [It's ascertainable] by direct perception. It is not inference. It is to be experienced by your own self.

\vsnum{6.20: }The first [type of penance] is mental penance, the second is verbal penance, the third is the bodily one, the next one[??] is the one which is both mental and verbal action. The fifth type of penance is a mixture of the bodily and the verbal.

\vsnum{6.21: }Gentleness of the mind, calmness, self-control, taciturnity and the purification of one's state of mind: mental penance comprises these five.

\vsnum{6.22: }Verbal penance is taught as speech that causes no anxiety, which is kind, true and useful, and [it include] also the practice of recitation.

\vsnum{6.23: }Bodily penance is taught as the following: honesty, harmlessness, chastity, the worship of gods, and purity as the fifth.

\vsnum{6.24: }[Penance] which is a mixture of the mental [and the verbal] is taught by the great Ṛṣis to be these five: He should speak [about things that are] agreeable, virtuous [bhāva?], auspicious, salutary and useful.

\vsnum{6.25: }[Penance] in which bodily [and verbal things] are mixed is taught by the great Ṛṣis to be these five: the worship of the guest and the guru by asking about their well-being, celebrating them and blessing them.[??]

\vsnum{6.26: }[Being] a [so-called] frog-yogin in the winter, or one with the five fires, or one who has the clouds [i.e. the open sky] for shelter in the rainy season: this kind of penance is called \skt{sādhana}.

\vsnum{6.27: }Carving out his own flesh as a donation, or [offering his own] hand, feet and head, ... puṣpa as blood? All these kinds of penance is \skt{sādhana},

\vsnum{6.28: }[such as also] the Painful penance and the Extremely Paniful one, [eating only] at night, the Hot and Painful and [the one in which only food obtained] without solicitation [can be eaten], the Cāndrāyaṇa and Parāka penances, the Sāṃtapana etc. 

\vsnum{6.29: }A person who performs with a well-disposed mind this penance that puts an end to the suffering caused by mundane existence, abandoning the trap of hope, with a spotless mind, giving up the lowest rewards [such as] wishing for heaven, being a king and having enjoyments for the senses, can bring that ultimate [? \skt{sarvāntika}] reward which stems from it [i.e. from \skt{tapas}] to [this] home of eternal births and deaths.


%%%%%%%%%
\vsnum{7.1: }In the past the wise declared that there were five kinds of donation ... CHECK Donation of food, clothes, gold, land and the fifth, donation of cows.

\vsnum{7.2: }From food [comes] energy, memory, the vital breath, growth, body, happiness. From food arise grace and beauty, heroism, strength.

\vsnum{7.3: }Living beings live on food. Food always satisfies. From food arise desire, rapture, pride and valour.

\vsnum{7.4: }Food drives away hunger and thirst and disease instantly. From donations of food arise happiness, fame and glory.

\vsnum{7.5: }He who donates food donates life. He who donates life donates everything. Therefore nothing is equal to the donation of food, nothing was, nothing will be.

\vsnum{7.6: }... ? A person without clothes may not be respected by his wife, son, friends etc.

\vsnum{7.7: }Be it a learned person from a good family or an intelligent and virtuous one, a person without clothes is subdued and humiliated on every occasion

\vsnum{7.8: }because a person without clothes receives contempt and disrespect. Even a great soul will try to avoid [him] at the court, among women, in an assembly.

\vsnum{7.9: }Therefore the wise praise donations of clothes. One should not give away old, torn or dirty clothes.

\vsnum{7.10: }[Clothes] should be donated [only if they are] new, not worn, soft, delicate and beautiful, well-washed, and [if] accompanied by willingness and devotion.

\vsnum{7.11: }They say that the reward [of donation/generosity] is in every case dependent on the particular [donor's] willingness and character, the choice of place and time, and on the particular recipient and material.

\vsnum{7.12: }The reward received will be similar to the clothes donated. By donating old clothes, one would receive old clothes [as a reward]. By donating beautiful clothes, one would receive beautiful clothes [as a reward].

\vsnum{7.13: }Should one bestow very beautiful clothes on a Brahmin [lit. on a person who is first among the twice-born] in an auspicious time, respectfully. he [i.e. the donor] will receive unequalled happiness and a beautiful appearance. When he departs, he will be given hundreds of millions of items of nice clothes, no doubt about that. Therefore do donate clothes often. It is the way up to the other world.

\vsnum{7.14: }O great Brahmin, now I shall teach you about the donation of gold in a concise manner. It is pure, auspicious and meritorious [act] and it washes off all sins.

\vsnum{7.15: }Should one hand over [to someone] a golden bracelet or ring, O Brahmin, he will be freed of all sins, just as the moon is freed from [the demon] Rāhu.

\vsnum{7.16: }If a person donates gold to Brahmins or gods, O excellent Brahmin, even if it is only in a minute quantity, he will be freed of all sins.

\vsnum{7.17: }[The amount can be just] one \skt{rakti}, a \skt{māṣaka}, a \skt{karṣa}, half a \skt{pala} or a \skt{pala}: this is exactly how the increase in the [size of the corresponding] reward will be, in proportion to the kind [i.e.\ amount] of the donation.

\vsnum{7.18: }The wise praise the donation of land as the basis of everything [else]. Food, clothes, gold etc.: all of these originate in the land.

\vsnum{7.19: }O Brahmin, one can obtain all the rewards of donation be donating land. If there is anything that equals the donation of land, O Brahmin, you should really tell me.

\vsnum{7.20: }[Humans] have the earth as their abode as soon as they get out of their mother's womb. Land is taught as common to all that is mobile and immobile.

\vsnum{7.21: }Be it [only a land of] one forearm, two forearms, fifty or a hundred, a thousand, ten thousand, a hundred thousand, donations of land are held in great esteem.

\vsnum{7.22: }Should he donate a piece of land of [only] one forearm to an excellent Brahmin, he will enjoy a billion divine years in heaven.

\vsnum{7.23: }Thus in case of many forearms [of land], the reward is said to be [proportional to the dimensions of the land, i.e.] ... O Brahmin, I have taught you about the rewards of donation that is made willingly.

\vsnum{7.24: }[Paraśu]rāma, the son of Jamadagni, having donated land to the Brahmin [Kaśyapa], obtained eternal life in this very world, O excellent Brahmin.

\vsnum{7.25: }[A cow] with golden horns, silver hooves, garment and bell, O Brahmin, when given to a Veda-knowing Brahmin, [produces] rewards that are said to be endless.

\vsnum{7.26: }Always rejoicing in the practice of giving as far as his capacities go ... ? one should give food, clothes, gold and silver, water, cows, sesamum [oil?], land, sandals, parasols, seats, jars, cups or anything else. Making the [deed of] giving willingly (\skt{śraddhādāna}) something done with an uninterrupted facial expression of affection, one's mind becomes spotless.

\vsnum{7.27: }Glory and fortune that makes us happy come about only by donations, and one can gain unequalled fame. The reproach of the enemy will give pleasure and happiness only because of donations[?]. Being invincible comes from donation and also unequalled graciousness. One can reach happiness thought donations. Endless enjoyments surely come only from donations, and heaven is [reached] also because of it.

\vsnum{7.28: }The unequalled world of Śakra [i.e. Indra] [can be reached] only by donations. Donations make people happy. Samrāj enjoyed the whole earth in the world only because of donations. CHECK Skanda (\skt{candrānana}) is seen as a handsome and fortunate one with a [good] family[? CHECK] only because of donations. One can reach happiness that lasts countless births only through donations, there is no doubt about that.


%%%%%%%%%
\vsnum{8.1: }Five kinds of study are to be pursued by those who wish to be happy in this life and in the other: [one has to study the] Śaiva [teachings], Sāṃkhya [philosophy], the Purāṇa[s], the Smārta [tradition] and the \skt{Bhāratasaṃhitā} [i.e. the \skt{Mahābhārata}].

\vsnum{8.2: }He should reflect on the Śaiva truth in both Śaiva and Pāśupata [teachings]. In those teachings the whole essence of truth is taught extensively.

\vsnum{8.3: }Those who reflect on the truth (\skt{tattva}) can grasp the truth (\skt{tattva}) of enumeration (\skt{saṃkhyā}) [of ontological principles/reality levels] from Sāṃkhya [texts]. The great sages taught [those twenty-five] \skt{tattva}s [of Sāṃkhya] as being in groups of five.

\vsnum{8.4: }In the Purāṇas it is the sheaths of the world that are described extensively. One can definitely enter [the realm] of the lower [world, i.e. hell], the upper [world, i.e. heaven], and middle [world, i.e. the human world], and the horizontal [world, i.e. of animals by studying the Purāṇas].

\vsnum{8.5: }The Smārta [tradition] deals with the conduct of the classes (\skt{varṇa}) and the conduct in the life-stages (\skt{āśrama}), and with the activities of Dharma and legal proceedings. Good conduct is to be gathered from that [source] without hesitation, with trust.

\vsnum{8.6: }A man who studies the epics (\skt{itihāsa}) will become omniscient. [All his] doubts about Dharma, Artha, Kāma and Mokṣa will be eliminated.

\vsnum{8.7: }Listen with great attention, O Brahmin, to the five types of sexual restraint [concerning the following:] women, forbidden ejaculation, and masturbation are mentioned [in this context, as well as] offence while sleeping, O Brahmin, and daydreaming as the fifth.

\vsnum{8.8: }A woman is not to be approached sexually in daytime and on the four days of the changes of the Moon (\skt{parvan}), even if she is one's lawful wife. One should not have sex with a woman who is taboo or with one of those who have lost their class (\skt{varṇa}) or are [of a] superior [\skt{jāti} than oneself].

\vsnum{8.9: }Intercourse with goats, sheep, cows, mares, buffaloes is called forbidden ejaculation, which is to be avoided at all cost.

\vsnum{8.10: }Rubbing himself against something else than a female sexual organ or rubbing his anus, are called masturbation, therefore these are to be avoided.

\vsnum{8.11: }Offence while sleeping, O best of Brahmins, has always been [considered] undesirable by the learned. [If] one enjoys women while sleeping, his semen gets spilt.

\vsnum{8.12: }Daydreaming [about women] should always be avoided by those who are intent on Dharma. Women are called `the bolts [that block the gate to] the path to heaven'.

\vsnum{8.13: }[Hear about] the five religious observances [called] the cat, the crane, the dog, the cow, and the earth. <sep/>He buries his own urine and faeces in the ground, O truest Brahmin. He rejoices [seeing] the sun and the moon when performing the cat observance.

\vsnum{8.14: }O great ascetic, one should suppress all of his senses like a crane, and should cultivate the peace of the mind, focusing on achieving liberation.

\vsnum{8.15: }He does not bury his urine and faeces in the ground, and he barks constantly. Lord Śarva [i.e. Śiva] is satisfied when one practises the dog observance.

\vsnum{8.16: }A person practising the Cow Vow should never hold back his urine and faeces. He is terrifying and he gives satisfaction, [as] stated in the Purāṇas.

\vsnum{8.17: }CHECK Digging [the earth] with spades and collecting [? the soil] with wedges: Goddess Earth bears [this] patiently. This is exactly how one can practise the earth vow.

\vsnum{8.18: }He who practises these five religious observances with his senses subdued will, without doubt, reach this superior world (i.e. Śiva's heaven).

\vsnum{8.19: }Eating leftovers, [not] eating in-between [breakfast and dinner], eating [only] at night, eating food obtained without solicitation, and fasting: listen, I shall teach you these five.

\vsnum{8.20: }[He who eats] the leftovers belonging to all the gods, to guests, and to the ancestors, he who eats the leftovers (śeṣāśin) of servants, sons and wives is the one who consumes the remains of food (\skt{vighasāśana}).

\vsnum{8.21: }He will be regarded as one that is always fasting if he never eats between breakfast and dinner.

\vsnum{8.22: }One should not eat in the daytime or in the evening, and should eat [only] at midnight if he wishes to follow the order of [eating only at] night.

\vsnum{8.23: }He should eat only the unsolicited food of someone who has not yet started eating [this food]. He who eats [only] that which has been given by others [without asking them for it] is called [one who eats] unsolicited [food].

\vsnum{8.24: }Chewable and unchewable food (\skt{bhakṣyaṃ bhojyaṃ ca}), food to be sipped or sucked or drunk, as the fifth [category]: if one does not long for and does not consume [any of the above], that is called fasting (\skt{upavāsa}).

\vsnum{8.25: }One should keep these five types of taciturnity, always dwelling in religious observances: [in situations where silence is best instead of] deceitful speech, envious speech, abuse, harsh speech, bragging.

\vsnum{8.26: }Fictitious [speech], [speech on] unknown [things], [speech about things] outside the range of Dharma, meaningless and unfriendly speech: these are called lying.

\vsnum{8.27: }One who does not rejoice in others' fortune or in others' power, one who would like to see something disadvantageous [for others] is called envious [and he should rather remain silent].

\vsnum{8.28: }[May your] mother and father be dead! [This is] how a ruined state will befall [you]! Enjoy the love of unclean [women]! [These are] called abuse.

\vsnum{8.29: }Won't you burst in your heart, stupid? Will your head not split into two? [If one utters] these or similar [curses], he is said to be one of harsh speech.

\vsnum{8.30: }Relating fancy stories about gambling, enjoyments, fights, drinking and women are the five types of bragging, as I teach them, O excellent Brahmin.

\vsnum{8.31: }Taciturnity should always be practised by those who prefer the beauty of speech. One should always speak without abuse and without idle talk.

\vsnum{8.32: }He who does not practise taciturnity is defiled and he is the black sheep of the family. For a number of rebirths, [his mouth] will stink and he will become mute.

\vsnum{8.33: }Therefore the speech of a person who always keeps the observance of taciturnity firmly, with resolution, will be impossible to ignore and he will make the community rejoice. The fragrance of lotuses and [other kinds of] strong fragrances will blow from his mouth. Thousands of faultless \skt{śāstra}s will be declared in the words of this person.

\vsnum{8.34: }I shall teach you the five kinds of bathing as they really are: Fire bath, water bath, Vedic bath, wind bath and divine bath.

\vsnum{8.35: }Fire bath is [performed] with ashes. Its fruits are a hundred times bigger than [those of] a water [bath]. [Things] purified with ashes are holy. Ashes destroy sin.

\vsnum{8.36: }Therefore one should use ashes for it purifies humans of their defilement. Ashes produce peace for everyone. Ashes are the ultimate protectors.

\vsnum{8.37: }Drawing [the sectarian marks on their foreheads while reciting] the Tryāyuṣa [mantra], remaining in chastity, all the Ṛṣis purified themselves with ashes.

\vsnum{8.38: }The gods, afflicted by their fear of Vīrabhadra, were set free with the help of ashes. Seeing the glory of ashes, Brahmā consented [to the use of this otherwise impure substance].

\vsnum{8.39: }[Thus] the Pāśupata observance was created, which is above [the system of] the four \skt{āśrama}s. Therefor the Pāśupata [observance] is the best because it involves carrying ashes [on one's body].

\vsnum{8.40: }A water bath (\skt{vāruṇa}) is to be performed with water by people in various ways in the water of rivers, water tanks, streams and ponds.

\vsnum{8.41: }The wise know the Vedic bath as [the one performed with the Vedic mantra beginning] \skt{āpo hi ṣṭhā} [ṚV 10.9.1--3], O excellent Brahmin. It is to be performed at the three junctures of the day (dawn, noon, evening). It is called the Vedic bath.

\vsnum{8.42: }He should go where, on the paths where cows roam, dust is rising, and he should sit down there. This is called [a kind of] bath, [namely the \skt{vāyavya} or wind-bath].

\vsnum{8.43: }One should immerse his own body in the water-showers of rain water. The one and only great Lord (\skt{maheśvara}) of the universe calls it heavenly bath.

\vsnum{8.44: }Thus have I taught you the section on the Niyama-rules [see Chapters 5--8] in divisions of five [sub-categories] because you asked me to, favouring the whole world. [These Niyama-rules] wipe off all the defilement, these fifty Dharma [teachings, i.e. 10 main topics/rules × 5 subcategories]. There will not be rebirth [for one who keeps these rules], not even in millions of aeons.


%%%%%%%%%
\vsnum{9.1: }The whole universe with its moving and unmoving elements is divided by the three [divisions of] time and the [three] \skt{guṇa}s [or guṇa not tech term here?]. Therefore the whole world is bound by the fetters of the three \skt{guṇa}s.

\vsnum{9.2: }Vigatarāga spoke: What does the term `the three divisions of time' mean for the soul in the three worlds[?]? Talk about it in a somewhat more extended manner, O great ascetic.

\vsnum{9.3: }Anarthayajña spoke: The three [divisions of] time are the three \skt{guṇa}s. It[?] is pervading and born from Prakṛti. They support each other, they serve each other.

\vsnum{9.4: }Sattva, Rajas and Tamas; Rajas, Sattva and Tamas; Tamas, Sattva and Rajas; they are each other's pairs.

\vsnum{9.5: }Lord Viṣṇu is Sattvic. [Brahmā], the one who was born on a lotus, is Rājasa. Lord Īśa is Tāmasa, the limbless is all ... [?]

\vsnum{9.6: }Sattva is of the colour of jasmine and the moon. Rajas is of the colour of ruby. Tamas is of the colour of lamp-black ... śaila. [This is what] the wise teach.

\vsnum{9.7: }Sattva is water, Rajas is charcoal, Tamas is full of smoke. All souls are constructed/suffer (\skt{pacyante}) as bound by these \skt{guṇa}s.

\vsnum{9.8: }Vigatarāga spoke: By what sorts of noose of \skt{guṇa}s is [the soul] bound? Teach me the signs connected to them one by one, O great ascetic.

\vsnum{9.9: }Anarthayajña spoke: The souls are bound in many ways and by many conditions by the fetters of the \skt{guṇa}s. Those who are deluded do not recognize [them]. The Śivayogins do recognize [them].

\vsnum{9.10: }He who is always established in Sattva goes upwards. He who is covered with Rajas goes in the middle. Those lowest of men in the state of Tamas go downward.

\vsnum{9.11: }These three kinds of \skt{guṇa}s are to be acknowledged even in heaven, O great ascetic, and among humans and also among animals.

\vsnum{9.12: }The ten superior Sattva [beings] are: Brahmā, Viṣṇu, Rudra, Dharma, Indra, Prajāpati, Soma, Agni, Varuṇa and Sūrya.

\vsnum{9.13: }...

\vsnum{9.14: }...

\vsnum{9.15: }... ...

\vsnum{9.16: }... ...

\vsnum{9.17: }... ...

\vsnum{9.18: }These are the ten superior Tāmasa [animals]: cows, elephants, Gayal oxen, horses, deer, Yaks, Kiṃnaras, lions, tigers, wild boar.

\vsnum{9.19: }The ten middle ranking Tāmasa [beings] are: rams, sheep, buffaloes, mice, mongooses etc., camels, Raṅku deer, hares, rhinoceroses. [only 9!]

\vsnum{9.20: }The ten low-ranking Tāmasa [beings] are: bears, alligators, deer, horned animals[?], cranes, apes, donkeys, boar, dogs and frogs.

\vsnum{9.21: }...

\vsnum{9.22: }... ...

\vsnum{9.23: }Cuckoos, owls, ... , doves [only 4! kiñjalka is wrong, not an animal] Makaras, cow-killing alligators and ... are of Tamas-Sattva. Tortoises, ... , crocodiles of the Ganges, frogs are of Tamas-Rajas. ...

\vsnum{9.25: }... ...

\vsnum{9.26: }The ten Tamas-Rajas [trees] are: Citron trees, bread-fruit trees, hog-plum trees, pomegranate trees, jujube trees, ratan trees, Neemb trees, Kadamba trees and ...

\vsnum{9.27: }... ...

\vsnum{9.28: }... ...

\vsnum{9.29: }[These words describe] the people who are the best among the Sāttvika [type]: compassion, truthfulness, self-control, purity, knowledge, taciturnity, penance, patience, integrity, lack of self-conceit.

\vsnum{9.30: }[These words describe] the people who are the best among the Rājasa [type]: desire, thirst, pleasure, gambling, arrogance, fight, intoxication, delight, cruel, quarrelling.

\vsnum{9.31: }[These words describe] the people who are the best among the Tāmasa [type]: harming, envious, incompassionate, stupid, sleepy, lazy, cowardly, idle, angry, greedy, cheating.

\vsnum{9.32: }The Sāttvika can be characterised as follows: light, joyful, bright, always eager for yoga meditation, wise, intelligent and dispassionate.

\vsnum{9.33: }The Rājasa can be characterised as follows: childish, skilful, passionate, proud, arrogant, greedy, desirous, jealous and chattering.

\vsnum{9.34: }The Tāmasa can be characterised as follows: anxious, lazy, deluded, cruel, a pitiless robber, angry, wicked and sleepy.

\vsnum{9.35: }Vigatarāga spoke: By what signs can the food of all humans be recognized? [?] Teach me about the three \skt{guṇa}s, O great ascetic.

\vsnum{9.36: }Anarthayajña spoke: The Sāttvikas prefer food that yields [long] life, fame, happiness, joy, which increases strength and health, which is savoury and which tastes nice, and which is soft.

\vsnum{9.37: }The best food for the Rājasas is rather warm, acidic, salty, hard, hot and pungent. It gives you pain, a burning sensation and indigestion.

\vsnum{9.38: }Tāmasas prefer food that is prohibited, impure and foul-smelling,  ... and tasteless. 

\vsnum{9.39: }Vigatarāga spoke: How can one recognize [the state of getting] beyond the \skt{guṇa}s, which leads one to the other shore of [the ocean] of mundane existence? Tell me truly about the liberation of those who are [initially] bound by the noose of the \skt{guṇa}s.

\vsnum{9.40: }Anarthayajña spoke: Well, he who looks at all living beings in the correct way, as his own Self, O Brahmin, is to be known as one beyond the \skt{guṇa}s, as one who has departed to the other shore of [the ocean of] mundane existence.

\vsnum{9.41: }He who treats envy and hate[?], happiness and sorrow, praise and reproach as equal is called `one who is beyond the \skt{guṇa}s'.

\vsnum{9.42: }He who is indifferent to pleasant and unpleasant things, to enemy or friend, to respect and contempt is called `one who is beyond the \skt{guṇa}s'.

\vsnum{9.43: }O Brahmin, thus has the exposition of the essence of the \skt{guṇa}s been taught to you. Those who are connected with the \skt{guṇa}s are mundane (\skt{saṃsārin}), those beyond the \skt{guṇa}s are on the supreme path.


%%%%%%%%%
\vsnum{10.1: }Vigatarāga spoke: Which pilgrimage place do the wise consider the best of all? Tell me, O best of sages, if there is one in the world that fulfills [all] desires.

\vsnum{10.2: }Anarthayajña spoke: This question [that I have been] asked is an extremely deep secret. Out of fondness, O excellent Brahmin, I'll teach you an ancient legend that Nandi told me.

\vsnum{10.3: }Nandikeśvara spoke: On a beautiful peak of Mount Kailāsa, which is frequented by Siddhas and celestial singers (\skt{cāraṇa}), there was Śiva himself there, seated, and Devī spoke to him thus:

\vsnum{10.4: }Devī spoke: O Lord, Lord of the chiefs of the gods, O ruler of all beings and all the world, I would like to ask you about one thing that concerns the eternal and secret Dharma,

\vsnum{10.5: }the transcendental and highly secret pilgrimage place by which one can be liberated from Saṃsāra. O Maheśvara, teach me the truth for the benefit of mankind.

\vsnum{10.6: }Maheśvara spoke: Who else would ask me that question if not you, O Sundarī? Listen, I'll expound that question which is difficult to grasp even for the gods.

\vsnum{10.7: }[If one] gets to know Kurukṣetra, Prayāga, Vārāṇasī, Gaṅgā, Agni, Somatīrtha, Sūrya, Puṣkara, Mānasa,

\vsnum{10.8: }Naimiṣa, Bindusaras, Setubandha, Surahrada, Ghaṇṭikeśvara, and Vāgīśa, he'll certainly be able to destroy his sins.

\vsnum{10.9: }Umā spoke: This and other [related] things, O Mahādeva, have been [just] taught to me [by you] as previously. Among these[?] the pilgrimage place that yields all enjoyments, O Suranāyaka.

\vsnum{10.10: }[But] how is one liberated from mundane existence merely be knowledge, O Īśvara? Cut [this] great curiosity arising [in me] that causes doubt.

\vsnum{10.11: }Rudra spoke: How could I not know that pilgrimage place which is both easy and difficult to reach? It is easy to reach for those who serve their guru and difficult to reach should one abandon it [i.e. the service of the guru].

\vsnum{10.12: }\skt{Kuru} [in \skt{kurukṣetra}] is to be known as the soul (\skt{puruṣa}), \skt{kṣetra} as the body. Kurukṣetra [which] is in the body yields the fruits of all pilgrimage places.

\vsnum{10.13: }[And there will be] the obtaining of the fruits of all sacrifices, the fruits of all [possible] donations, and all the fruits of all religious observances and penance observed.

\vsnum{10.14: }In the same manner [one will obtain] the fruits of those fifteen pilgrimage places [from Kurukṣetra to Vāgīśa, cf. 10.7--8, by only knowing the bodily Kurukṣetra]. ... [this] great pilgrimage place is extremely auspicious and pleasant.

\vsnum{10.15: }Devī spoke: I am extremely thrilled, O Tridaśeśvara. Hearing this which is easy to obtain, easy to perform and is subtle, I am filled with satisfaction.

\vsnum{10.16: }Teach me on, teach me the remaining fourteen pleasant [pilgrimage places], Prayāga and the others, one by one, as they are, O Sureśvara.

\vsnum{10.17: }The Suṣumnā[-tube] is the Honourable Gaṅgā, Iḍā[-tube] is the river Yamunā. The right nostril is [the river] Vāruṇī, the left nostril is known as [the river] Asi. Because [it is] at the confluence of Vāruṇā and Asi, [the city there] is known as Vārāṇasī.

\vsnum{10.19: }She is called the ethereal Gaṅgā [because] the nectar of immortality issues from her day and night uninterruptedly. That's why she is called Gaṅgā (perhaps: `ever-goer').

\vsnum{10.20: }Somatīrtha is the tube Iḍā. It is characterised by the ringing of small bells. Upon hearing that [ringing], all of one's sins will be destroyed.

\vsnum{10.21: }Somatīrtha is the [tube] Suṣumnā .... By merely hearing about it one is liberated, even if one has a huge heap of sins.

\vsnum{10.22: }Agnitīrtha is the Arjuna tube[??]. It is charming because of the hum of Veda recitation. Upon hearing this or that syllable, one will become immortal.

\vsnum{10.23: }Puṣkara is [a lotus] with eight petals and a pericarp in the centre of the heart. One should visualize the Subtle One in its centre [and] it'll destroy birth and death.

\vsnum{10.24: }In the centre of the Mānasa lake on a lotus with [the syllables] HAṂ-SA, ...

\vsnum{10.25: }Listen to Naimiṣa, O Deveśī. It presents proof in a moment. One can observe one's own or others' shadow properly[?].

\vsnum{10.26: }... When he has seen the proof thus, he is called the knower of Naimiṣa.

\vsnum{10.27: }Listen O Sundarī, I shall teach you the pilgrimage place called Bindusaras. The heart is to be known to be located in the centre of the body. In the centre of the heart, there is a lotus.

\vsnum{10.28: }There is a pericarp in the centre of the lotus, and the subtle sonic matter (\skt{bindu}) in the centre of the pericarp. In the centre of the subtle sonic matter (\skt{bindu}), there is the subtle sound (\skt{nāda}). How is that subtle sound (\skt{nāda}) divided?

\vsnum{10.29: }Divided by the sound U and the sound MA, the subtle sound (\skt{nāda}) departs. Realizing that [subtle sound], O Viśālākṣi, one can obtain immortality.

\vsnum{10.30: }I shall teach you Setubandha, [which sports] a current whose water of subtle sound (\skt{nāda}) cleanses you of the dirt of your sins. The banks [of this river] are the tongue, the throat and the chest, its sandy beaches are the host of gods, it roars with whirlpools and is wavy. It's full of the roar of Ganges crocodiles and full of fish, ten types of sea-monsters [also: makāra?], terrifying alligators and with \skt{visarga}[?] Go to Setubandha, [the pilgrimage place that] tastes like the pleasure of intoxication in the deep ... 

\vsnum{10.31: }O Moon-faced goddess, listen to [Surahrada], the reaching of the cessation of all sorrow, located in the centre of the seven islands. It is frequented by Īśāna, it's a spotless lake in the heart full of the cool water of sound (\skt{nāda}). There is a lotus arising, with Prakṛti as its petals, and divided by its Śakti filaments. It is praised by the five voids, it is the path to the supreme level, and it is to be served if one wishes to obtain [heaven].


%%%%%%%%%
\vsnum{11.1: }The Goddess spoke: O Paraśreṣṭha, O Surottama! Is there another [form of] universal sacrifice, which is free of pain, which is easy, and which does not require an abundance of materials, O Īśvara?

\vsnum{11.2: }For the benefit of mankind, teach me, O Suraśreṣṭha, how one obtains the fruits of [this] universal sacrifice, which [process] is praised even by the gods.

\vsnum{11.3: }Maheśvara spoke: I cannot see anything comparable to your compassion towards living beings, O Bhāminī. What else could I teach concerning which there is no compassion [in you towards living beings]?

\vsnum{11.4: }I heard [this] previously from Sadāśiva's mouth, O Varasundarī. Listen, O Goddess, I shall teach you the ultimate essence of Dharma.

\vsnum{11.5: }Immaterial sacrifice satisfies all desires. It is undecaying and imperishable, and it removes all sins.

\vsnum{11.6: }Material things present many kinds of obstacle and [their acquisition causes] great fatigue, similarly to Indra's murder of the Brahmin [Viśvarūpa], which yielded fruits that were distributed [among trees, lands etc.].

\vsnum{11.7: }Material sacrifice can be purified by the five purifications, O Varānanā. If it is purified, then the fruits will also be pure. If it is not purified, there is no fruit.

\vsnum{11.8: }The Goddess spoke: I am not sure about the five purifications, O Suraśreṣṭha. Please teach [them to] me one by one, I want to hear [them] as [they] really [are].

\vsnum{11.9: }Rudra spoke: The first is the purification of the mind, then comes the purification of the substances. The third is the purification of the mantras. The next one is the purification of the ritual. The fifth is the purification of Sattva. The purification of the sacrifice is [thus] fivefold.

\vsnum{11.10: }The purification of the mind is [achived] by mentally creating what is not wrong. The purification of the substances is [achieved] by [using] substances that were not obtained by unlawful means.

\vsnum{11.11: }The purification of the mantras is [achived] by [properly] joining vowels to consonants. The purification of the ritual is [achived] by not altering the proper sequence. The purification of Sattva is [achived] by the non-prevalence of Rajas and Tamas.

\vsnum{11.12: }When he has purified the ritual (\skt{vidhi}) thus and performs the sacrifice, he will obtain the fruits of the sacrifice, and will not experience birth and death [again].

\vsnum{11.13: }But he who performs immaterial sacrifice, O Varasundarī, will not obtain [only] its fruits, [but] of all sacrifices, without exception.

\vsnum{11.14: }His sacrificial ground is Kurukṣetra, he has made his abode in the house of Truth/Sattva. His great altar is the withdrawal of the senses. His seat of kuśa grass is self-control.

\vsnum{11.15: }The injunction is the various .. . He lights the fire of meditation which is flaring up by the fuel of the firewood of yoga and is abounding in the smoke of penance.

\vsnum{11.16: }The placing down of the chalice is knowledge about Śiva. [The oblation of] boiled rice is [when] he becomes Śiva [?!]. The continuous oblation of clarified butter is poured with the ladle of Lambaka [uvula, lambikā?].

\vsnum{11.17: }Transforming concentration into an Adhvaryu [priest], breath control will be the [other] priests. Samādhi which involves Tarka and which is long is the burning of the oblation[? vayas-tāpana?].

\vsnum{11.18: }The sacrificial post is made up of the knowledge about Brahman. The tying of the sacrificial animal is [the mental state called] Manonmanas. His wife is Faith, O Viśālākṣī. His sacrificial ritual intention/declaration is the eternal abode.

\vsnum{11.19: }Rice oblation is the consumption of the nectar of immortality that arises from the victory over the five senses. The great mantra is Brahmā's sound. Expiation is the victory over breath.

\vsnum{11.20: }The consumption of Soma is complete knowledge. The commencement [of the reading of the Veda] is the four yama-rules[?]. The ritual water-bath is [the reading of] the epics. His garment is made of [his readings of] the Purāṇas.

\vsnum{11.21: }Ritual bathing and sipping water once are [to be performed] at the confluence of the Iḍā and the Suṣumnā. Having honoured Contentment as a guest, he salutes the Brahmin as Compassion.

\vsnum{11.22: }The Brahmakūrca [penance] is the Guṇātīta [state of mind], the scent of the sacrifice is the Nirañjana [state of mind]. [His] sacred thread is the three Tattvas. For a shaven head he has enlightenment/teaching.

\vsnum{11.23: }The four Vedas are Nivṛtti etc. His seat is the four Prakaraṇas. He should always perform a sacrifice donating the priestly fee of providing being[s] with freedom from danger.

\vsnum{11.24: }The attainment of immaterial sacrifice has been taught to you, O Varānanā. [The sacrificer] will in any case obtain the fruits of upto a thousand [ordinary] sacrifices.

\vsnum{11.25: }The first life-stage [life option] has been taught to you, O Varānanā, the true Dharma, which is revered by Sadāśiva and also by the [other] gods.

\vsnum{11.26: }[Now] learn about brahmacarya. Listen with attention, O Śubhā. [This is] the second life-stage, O Devī, the destroyer of all sins.

\vsnum{11.27: }[Here] religious observance is [now] meditation on Brahman. The Sāvitrī [hymn] is absorption in Prakṛti. The Brahmanical cord is the subtle syllable. His girdle is now contained in the three guṇas.

\vsnum{11.28: }His staff is self-restraint, his bowl compassion. Begging/alms? is liberation from saṃsāra. The tryāyuṣa [mantra] is the one beyond the two syllables[?]. It[?] is embellished with the ashes of knowledge.

\vsnum{11.29: }The bath-vow is speaking the truth always. It is accompanied by the purity of moral conduct. Sacrifice to Agni is the three tattvas[?]. Recitation is the sound at the aperture of Brahmā.

\vsnum{11.30: }[This is] the second life-stage as Lord Śiva taught it, O Devī. I have also taught [it to] you[,] the destruction of birth and death.

\vsnum{11.31: }Listen, O Long-eyed goddess, I shall teach you the forest-dweller's way of life, which is revered by the Ṛṣis and the gods, as I heard it, as it [really] is.

\vsnum{11.32: }Having taken to the forest of indifference, he should take residence in the Āśrama of niyama-rules, within walls that have the stone-strong gate of moral conduct, with his sense faculties conquered.

\vsnum{11.33: }The spiritual substratum of material objects [adhibhūta?] is his mother, the supreme spirit is his father. the divine realm is his teacher, determination his brothers.

\vsnum{11.34: }His wives are Śruti and Smṛti, his son is Wisdom, his younger brother Patience. His relative is Benevolence, his twisted hair is his bow, Compassion his sacred thread.

\vsnum{11.35: }Sympathy is the four ways of taciturnity. All his teendők... are Endurance. He has the yama-rules for a garment made of bark, and he wears Penance for the skin of a black antelope.

\vsnum{11.36: }He is seated on the highest level of non-attachment, and the firm observance is his yoga-belt. The sound of murmuring the Vedas is noise[??]. Fire sacrifice is breath-control.

\vsnum{11.37: }He is full of[??] conquered breaths for a deer[??]. [For him] sacrifice is resolution, ritual is recitation. His companion from among all the collected teachings[??!] of the Śāstras is self-control, compassion etc.

\vsnum{11.38: }He should perform sacrifice to Śiva [with/as?] the worship of the eight [yogic?] practices. He is purified by the water of the five Brahma[-mantras] in the auspicious pool on the sacred banks of truthfulness.

\vsnum{11.39: }Having bathed and having sipped water [there], he should take refuge at [or rather upāsayet?] the three junctures of the day. His rosary is the meaning of the Purāṇas. The pacification of mantras? is? recitation day and night.

\vsnum{11.40: }His jar of epics is filled with the water of knowledge. [Tentatively:] The actions of the five [medical] procedures are suicide. The five kinds of pleasure are recitation.[?]

\vsnum{11.41: }The Śivasaṃkalpa [hymn] is practice (sādhana), which yields fruits of yoga accomplishments. His food is the fruit of Contentment. He conquered lust and anger.

\vsnum{11.42: }His practice is the victory over the trap of hope. He prefers the joy of yoga meditation. The forest-dweller should observe his vow by providing his guests with fearlessness. This is how the Dharma of the forest-dweller has been taught in the past.

\vsnum{11.43: }[If the yogin] follows, with faith and self-control, the supreme Dharma, which delivers him from Saṃsāra, removes transient existence, uproots ignorance, increases wisdom, which is fruitful, which delivers cross him from the flood of affliction, removes birth, disease and burns bad karma, he will really become a living Śiva.

\vsnum{11.44: }Here follows the a wandering religious mendicant's Dharma. Listen, I shall teach you about it. Making joy and pain equal, he gets rid of greed and folly.

\vsnum{11.45: }He should avoid honey and meat, as well as others' wives. He should avoid staying [in a place] for long and also staying at others' places.

\vsnum{11.46: }He should avoid food that has been thrown away and he should avoid a single alms round[?? the same food?]. He should always refrain from accumulating wealth and from self-conceit.

\vsnum{11.47: }Meditating on the subtle he can put his feet into the pure.[??] He should not get angry when [food] in not available, and when it is, he should not rejoice.

\vsnum{11.48: }He should not be agitated with regards to thirst for material things or to violent anger. He should take praise and reproach equal, as well as pleasant and unpleasant things.

\vsnum{11.49: }His garment is the Niyama-rules, and he is girded by the girdle of self-control. He makes his mind supportless, his intellect spotless,

\vsnum{11.50: }his self Earth, the Manonmana ether[?], his three staffs the three guṇas, his bowl the imperishable syllable.

\vsnum{11.51: }He should throw away [the distinction between?] Dharma and Adharma, and should avoid envy and hatred. He is indifferent to the opposites [such as cold and heat, good and bad], dwells always in truthfulness, unselfish, humble.

\vsnum{11.52: }He should go on his alms round visiting seven houses at the eighth part of the day. He should not sit down, he should not stay, and he should not say `Give me!'.

\vsnum{11.53: }He should live on what is available, on[?] eight bites a day. He should not stick to items of clothes, food or a bed for long.

\vsnum{11.54: }He should nor rejoice in death, he should not rejoice in life. Having conquered his senses, having killed his desire, firm in his observances,

\vsnum{11.55: }the Bhikṣu should never think about the past or the future. The wandering mendicant should always avoid anger, self-conceit, intoxication and pride.

\vsnum{11.56: }Making indifference a bow which is strung by the strings of breath-control, he should kill the beast the sense-faculty which is the mind with the sharp-pointed arrow of concentration.

\vsnum{11.57: }He should stab the enemy that is Saṃsāra with the extremely sharp knife of friendliness. He should defeat the rutting elephant of anger with the whirling discus of compassion.

\vsnum{11.58: }His body is clad in the armour of sympathy, his quiver is full of indifference. He should constantly recall the unutterable syllable which is supreme Brahman, O Brahmin.

\vsnum{11.59: }Brahmā's heart is Viṣṇu. Viṣṇu's heart is Śiva. Śiva's heart is the Junctures of the day. Therefore he should worship the Junctures.

\vsnum{11.60: }[Śiva] is deliverance from the ocean of mundane existence, the path to happiness, the Brahman, the junctures, the [sacred] syllable. [the yogin] should always, unweariedly, meditate on matchless Śiva, who is to be recognized as the manifested soul. He should take refuge in Hara, who is devoid[!] of form, colour, qualities etc., who is the supreme aim which is difficult to aim at, ... , the divine guru, who removes all pain.


\end{document}