\documentclass[11pt]{book}


% PACKAGES
\usepackage{fontspec} % Font selection for XeLaTeX; see fontspec.pdf for documentation
\defaultfontfeatures{Mapping=tex-text} % to support TeX conventions like ``---''
\usepackage{xunicode} % Unicode support for LaTeX character names
\usepackage{polyglossia}
\usepackage{lscape}
\usepackage[total={10.7cm,20cm},
top=4cm, left=5cm, headsep=1.1cm, footskip=1.7cm, footnotesep=1cm]{geometry}

\newfontfamily\devanagarifont[Script=Devanagari]{Pedantic Indic}

\begin{document}

\begin{landscape}
\devanagarifont

अनादिमध्यान्तमनन्तपारं
सुसूक्ष्ममव्यक्तजगत्सुसारम् ।
हरीन्द्रब्रह्मादिभिराप्तमग्रं
प्रणम्य वक्ष्ये वृषसारसंग्रहम् ।।१.१ ।।
शतसाहस्रिकं ग्रन्थं सहस्राध्यायमुत्तमम् ।
पर्व चास्य शतं पूर्णं श्रुत्वा भारतसंहिताम् ।।१.२ ।।
अतृप्तः पुन पप्रच्छ वैशम्पायनमेव हि ।
जनमेजय यत्पूर्वं तच्छृणु त्वमतन्द्रितः ।।१.३ ।।
जनमेजय उवाच ।
भगवन्सर्वधर्मज्ञ सर्वशास्त्रविशारद ।
अस्ति धर्मं परं गुह्यं संसारार्णवतारणम् ।।१.४ ।।
द्वैपायनमुखोद्गीर्णं धर्मं यत्तद्द्विजोत्तम ।
कथयस्व हि मे तृप्तिं कुरु यत्नात्तपोधन ।।१.५ ।।
वैशम्पायन उवाच ।
शृणु राजन्नवहितो धर्माख्यानमनुत्तमम् ।
व्यासानुग्रहसम्प्राप्तं गुह्यधर्मं शृणोतु मे ।।१.६ ।।
अनर्थयज्ञकर्तारं तपोव्रतपरायणम् ।
शीलशौचसमाचारं सर्वभूतदयापरम् ।।१.७ ।।
जिज्ञासनार्थं प्रश्नैकं विष्णुना प्रभविष्णुना ।
द्विजरूपधरो भूत्वा पप्रच्छ विनयान्वितः ।।१.८ ।।
[विगतराग उवाच ।]
ब्रह्मविद्या कथं ज्ञेया रूपवर्णविवर्जिता ।
स्वरव्यञ्जननिर्मुक्तमक्षरं किमु तत्परम् ।।१.९ ।।
अनर्थयज्ञ उवाच ।
अनुच्चार्यमसन्दिग्धमविच्छिन्नमनाकुलम् ।
निर्मलं सर्वगं सूक्ष्ममक्षरं किमु तत्परम् ।।१.१० ।।
विगतराग उवाच ।
देही देहे क्षयं याते भूजलाग्निशिवादिभिः ।
यमदूतैः कथं नीतो निरालम्बो निरञ्जनः ।।१.११ ।।
कालपाशैः कथं बद्धो निर्देहश्च कथं व्रजेत् ।
स्वर्गं वा स कथं याति निर्देहो बहुधर्मकृत् ।
एतन्मे संशयं ब्रूहि ज्ञातुमिच्छामि तत्त्वतः ।।१.१२ ।।
अनर्थयज्ञ उवाच ।
अतिसंशयकष्टं ते पृष्टो ऽहं द्विजसत्तम ।
दुर्विज्ञेयं मनुष्यैस्तु देवदानवपन्नगैः ।।१.१३ ।।
कर्महेतुः शरीरस्य उत्पत्तिर्निधनं च यत् ।
सुकृतं दुष्कृतं चैव पाशद्वयमुदाहृतम् ।।१.१४ ।।
तेनैव सह संयाति नरकं स्वर्गमेव वा ।
सुखदुःखं शरीरेण भोक्तव्यं कर्मसम्भवम् ।।१.१५ ।।
हेतुनानेन विप्रेन्द्र देहः सम्भवते नृणाम् ।
यं कालपाशमित्याहुः शृणु वक्ष्यामि सुव्रत ।।१.१६ ।।
न त्वया विदितं किञ्चिज्जिज्ञास्यसि कथं द्विज ।
कालपाशं च विप्रेन्द्र सकलं वेत्तुमर्हसि ।।१.१७ ।।
कलाकलितकालं च कालतत्त्वकलां शृणु ।
त्रुटिद्वयं निमेषस्तु निमेषद्विगुणा कला ।।१.१८ ।।
कलाद्विगुणिता काष्ठा काष्ठा वै त्रिंशतिः कला ।
त्रिंशत्कला मुहूर्तश्च मानुषेन द्विजोत्तम ।।१.१९ ।।
मुहूर्तत्रिंशकेनैव अहोरात्रं विदुर्बुधाः ।
अहोरात्रं पुनस्त्रिंशन्मासमाहुर्मनीषिणः ।।१.२० ।।
समा द्वादश मासाश्च कालतत्त्वविदो जनाः ।
शतं वर्षसहस्राणि त्रीणि मानुषसंख्यया ।।१.२१ ।।
षष्टिं चैव सहस्राणि कालः कलियुगः स्मृतः ।
द्विगुणः कलिसंख्यातो द्वापरो युग संज्ञितः ।।१.२२ ।।
त्रेता तु त्रिगुणा ज्ञेया चतुः कृतयुगः स्मृतः ।
एषा चतुर्युगासंख्या कृत्वा वै ह्येकसप्ततिः ।।१.२३ ।।
मन्वन्तरस्य चैकस्य ज्ञानमुक्तं समासतः ।
कल्पो मन्वन्तराणां तु चतुर्दश तु संख्यया ।।१.२४ ।।
दश कल्पसहस्राणि ब्रह्माहः परिकल्पितम् ।
रात्रिरेतावती प्रोक्ता मुनिभिस्तत्त्वदर्शिभिः ।।१.२५ ।।
रात्र्यागमे प्रलीयन्ते जगत्सर्वं चराचरम् ।
अहागमे तथैवेह उत्पद्यन्ते चराचरम् ।।१.२६ ।।
परार्धपरकल्पानि अतीतानि द्विजोत्तम ।
अनागतं तथैवाहुर्भृगुरादिमहर्षयः ।।१.२७ ।।
यथार्कग्रहतारेन्दु भ्रमतो दृश्यते त्विह ।
कालचक्रं भ्रमत्वैव विश्रमं न च विद्महे ।।१.२८ ।।
कालः सृजति भूतानि कालः संहरते पुनः ।
कालस्य वशगाः सर्वे न कालवशकृत्क्वचित् ।।१.२९ ।।
चतुर्दशपरार्धानि देवराजा द्विजोत्तम ।
कालेन समतीतानि कालो हि दुरतिक्रमः ।।१.३० ।।
एष कालो महायोगी ब्रह्मा विष्णुः परः शिवः ।
अनादिनिधनो धाता स महात्मा नमस्कुरु ।।१.३१ ।।
विगतराग उवाच ।
श्रुतं वै कालचक्रं तु मुखपद्मविनिःसृतम् ।
परार्धं च परं चैव श्रोतुं वः प्रतिदीपितम् ।।१.३२ ।।
अनर्थयज्ञ उवाच ।
एकं दशं शतं चैव सहस्रमयुतं तथा ।
प्रयुतं नियुतं कोटिमर्बुदं वृन्दमेव च ।।१.३३ ।।
खर्वं चैव निखर्वं च शङ्कुः पद्मं तथैव च ।
समुद्रो मध्यमन्तं च परार्धं च परं तथा ।।१.३४ ।।
सर्वे दशगुणा ज्ञेयाः परार्धं यावदेव हि ।
परार्धद्विगुणेनैव परसंख्या विधीयते ।।१.३५ ।।
परात्परतरं नास्ति इति मे निश्चिता मतिः ।
पुराणवेदपठिता मयाख्याता द्विजोत्तम ।।१.३६ ।।

---- ब्रह्माण्डम् ----

विगतराग उवाच ।
ब्रह्माण्डं कति विज्ञेयं प्रमाणं प्रापितं क्वचित् ।
कति चाङ्गुलिमूर्धेषु सूर्यस्तपति वै महीम् ।।१.३७ ।।
अनर्थयज्ञ उवाच ।
ब्रह्माण्डानां प्रसंख्यातुं मया शक्यं कथं द्विज ।
देवास्ते ऽपि न जानन्ति मानुषाणां च का कथा ।।१.३८ ।।
पर्यायेण तु वक्ष्यामि यथाशक्यं द्विजोत्तम ।
ब्रह्मणा यत्पुराख्यातो मातरिश्वा यथा तथा ।।१.३९ ।।
शिवाण्डाभ्यन्तरेणैव सर्वेषामिव भूरिताः ।
दशनाम दिशाष्टानां ब्रह्माण्डे कीर्तितं शृणु ।।१.४० ।।

---- दश नामानि दिगष्टकानाम् ----

सहासह सहः सह्यो विसहः संहतो ऽसभा ।
प्रसहो ऽप्रसहः सानुः पूर्वतो दश नायकाः ।।१.४१ ।।
प्रभासो भासनो भानुः प्रद्योतो द्युतिमो द्युतिः ।
दीप्ततेजाश्च तेजाश्च तेजा तेजवहो दश ।।१.४२ ।।
आग्नेये त्वेतदाख्यातं याम्ये शृण्वथ भो द्विज ।
यमो ऽथ यमुना यामः संयमो यमुनो ऽयमः ।।१.४३ ।।
संयनो यमनोयानो यनियुग्मा यनोयनः ।
नगजो नगना नन्दो नगरो नग नन्दनः ।।१.४४ ।।
नगर्भो गहनो गुह्यो गूड्हजो दश तत्परः ।
वारुणेन प्रवक्ष्यामि शृणु विप्र निबोध मे ।।१.४५ ।।
बभ्रः सेतुर्भवोद्भद्रः प्रभवोद्भवभाजनः ।
भरणो भुवनो भर्ता दशैते वरुणालयाः ।।१.४६ ।।
नृगर्भो ऽसुरगर्भश्च देवगर्भो महीधरः ।
वृषभो वृषगर्भश्च वृषाङ्को वृषभध्वजः ।।१.४७ ।।
ज्ञातव्यश्च तथा सम्यग्वृषजो वृषनन्दनः ।
नायका दश वायव्ये कीर्तिता ये मया द्विज ।।१.४८ ।।
सुलभः सुमनः सौम्यः सुप्रजः सुतनुः शिवः ।
सत सत्य लयः शम्भुर्दशनायकमुत्तरे ।।१.४९ ।।
इन्दु बिन्दु भुवो वज्र वरदो वर वर्षणः ।
इलनो वलिनो ब्रह्मा दशेशानेषु नायकाः ।।१.५० ।।
अपरो विमलो मोहो निर्मलो मन मोहनः ।
अक्षयश्चाव्ययो विष्णुर्वरदो मध्यमे दश ।।१.५१ ।।
सर्वेषां दशमीशानां परिवारशतं शतं ।
शतानां पृथगेकैकं सहस्रैः परिवारितम् ।।१.५२ ।।
सहस्रेषु च एकैकमयुतैः परिवारितम् ।
अयुतं प्रयुतैर्वृन्दैः प्रयुतं नियुतैर्वृतः ।।१.५३ ।।
एकैकस्य परीवारो नियुतः पृथगेव च ।
कोटिभिर्दशकोट्येन एकैकः परिवारितः ।।१.५४ ।।
दशकोटिषु एकैकं वृन्दवृन्दभृतैर्वृतम् ।
वृन्दवर्गेषु एकैकं खर्वभिः परिवारितम् ।।१.५५ ।।
खर्ववर्गेषु एकैकं दशखर्वगणैर्वृतम् ।
दशखर्वेषु एकैकं शङ्कुभिः परिवारितम् ।।१.५६ ।।
शङ्कुभिः पृथगेकैकं पद्मेन परिवारितम् ।
पद्मवर्गेषु एकैकं समुद्रैः परिवारितम् ।।१.५७ ।।
समुद्रेषु तथैकैकं मध्यसंख्यैस्तु तैर्वृतम् ।
मध्यसंख्येषु एकैकमनन्तैः परिवारितम् ।।१.५८ ।।
अनन्तेषु च एकैकं परार्धपरिवारितम् ।
परार्धेषु च एकैकं परेण परिवारितम् ।
एष वै कथितो विप्र शक्यं सांख्यमुदीरितम् ।।१.५९ ।।

---- प्रमाणम् ----

प्रमाणं शृणु मे विप्र संक्षेपाद्ब्रुवतो मम ।
चन्द्रोदये पूर्णमास्यां वपुरण्डस्य तादृशम् ।।१.६० ।।
कोटिकोटिसहस्रं तु योजनानां समन्ततः ।
अण्डानां च परीमाणं ब्रह्मणा परिकीर्तितम् ।।१.६१ ।।
सप्तकोटिसहस्राणि सप्तकोटिशतानि च ।
विंशकोटिषु गुल्मेषु ऊर्ध्वतस्तपते रविः ।।१.६२ ।।
प्रमाणं नाम संख्या च कीर्तितानि समासतः ।
ब्रह्माण्डं चाप्रमेयाणां लक्षणं परिकीर्तितम् ।।१.६३ ।।

---- व्यासाः ----

पुराणाशीसहस्राणि शतानि द्विजसत्तम ।
ब्रह्मणा कथितं पूर्णं मातरिश्वा यथातथम् ।।१.६४ ।।
वायुना पाद संक्षिप्य प्राप्तं चोशनसं पुरा ।
तेनापि पाद संक्षिप्य प्राप्तवांश्च बृहस्पतिः ।।१.६५ ।।
बृहस्पतिस्तु प्रोवाच सूर्यं त्रिंशत्सहस्रिकम् ।
पञ्चविंशत्सहस्राणि मृत्युं प्राह दिवाकरः ।।१.६६ ।।
एकविंशत्सहस्राणि मृत्युनेन्द्राय कीर्तितम् । 
इन्द्रेणाह वसिष्ठाय विंशत्श्लोकसहस्रिकम् ।।१.६७ ।।
अष्टादशसहस्राणि तेन सारस्वताय तु ।
सारस्वतस्त्रिधामाय सहस्रदश सप्त च ।।१.६८ ।।
षोडशानां सहस्राणि भरद्वाजाय वै ततः ।
दश पञ्चसहस्राणि त्रिवृषाय अभाषत ।।१.६९ ।।
चतुर्दशसहस्राणि अन्तरीक्षाय वै ततः ।
त्रय्यारुणिं सहस्राणि त्रयोदश अभाषत ।।१.७० ।।
त्रय्यारुणिस्तु विप्रेन्द्रो धनंजयमभाषत ।
द्वादशानि सहस्राणि संक्षिप्य पुनरब्रवीत् ।।१.७१ ।।
कृतंजयाय सम्प्राप्तो धनंजयमहामुनिः ।
कृतंजयाद्द्विजश्रेष्ठ ऋणंजयमहात्मने ।।१.७२ ।।
ऋणञ्जयात्पुनः प्राप्तो गौतमाय महर्षिणे ।
गौतमाच्च भरद्वाजस्तस्माद्धर्यद्वताय तु ।।१.७३ ।।
राजश्रवास्ततः प्राप्तः सोमशुष्माय वै ततः ।
सोमशुष्मात्ततः प्राप्तस्तृणबिन्दुस्तु भो द्विज ।।१.७४ ।।
तृणबिन्दुस्तु वृक्षाय वृक्षः शक्तिमभाषत ।
शक्तिः पराशरं प्राह जातूकर्णाय वै ततः ।।१.७५ ।।
द्वैपायनं तु प्रोवाच जातूकर्णो महर्षिणम् ।
रोमहर्षाय सम्प्राप्तो द्वैपायनमहामुनिः ।।१.७६ ।।
रोमहर्षाय प्रोवाच पुत्रायामितबुद्धये ।
दशद्वे च सहस्राणि पुराणं सम्प्रकाशितम् ।
मानुषाणां हितार्थाय किं भूयः श्रोतुमिच्छसि ।।१.७७ ।।

 ।।इति वृषसारसंग्रहे ब्रह्माण्डसंख्या नामाध्यायः प्रथमः ।।




विगतराग उवाच ।
श्रुतं मया जनाग्रेण ब्रह्माण्डस्य तु निर्णयम् ।
प्रमाणं वर्णरूपं च संख्या तस्य समासतः ।।२.१ ।।
शिवाण्डेति त्वया प्रोक्तो ब्रह्माण्डालयकीर्तितः ।
कीदृशं लक्षणं ज्ञेयं प्रमाणं तस्य वा कति ।।२.२ ।।
कस्य वालयनं ज्ञेयं प्रमाणं वात्र वासिनः ।
का वा तत्र प्रजा ज्ञेया को वा तत्र प्रजापतिः ।।२.३ ।।

---- शिवाण्डसंख्या ----

अनर्थयज्ञ उवाच ।
शिवाण्डलक्षणं विप्र न त्वं प्रष्टुमिहार्हसि ।
दैवतैरपि का शक्तिर्ज्ञातुं द्रष्टुं च तत्त्वतः ।।२.४ ।।
अगम्यगमनं गुह्यं गुह्यादपि समुद्धितम् ।
न प्रभुर्नेतरस्तत्र न दण्ड्यो न च दण्डकः ।।२.५ ।।
न सत्यो नानृतस्तत्र सुशीलो नो दुःशीलवान् ।
नानृजुर्न च दम्भित्वं न तृष्णा न च ईर्ष्यता ।।२.६ ।।
न क्रोधो न च लोभो ऽस्ति न मानो ऽस्ति न सूयकः ।
ईर्ष्या द्वेषो न तत्रास्ति न शठो न च मत्सरः ।।२.७ ।।
न व्याधिर्न जरा तत्र न शोको ऽस्ति न विक्लवः ।
नाधमः पुरुषस्तत्र नोत्तमो न च मध्यमः ।।२.८ ।।
नोत्कृष्टो मानवस्तस्मिन्स्त्रियश्चैव शिवालये ।
न निन्दा न प्रशंसास्ति मत्सरी पिशुनो न च ।।२.९ ।।
गर्वदर्पं न तत्रास्ति क्रूरमायादिकं तथा ।
याचमानो न तत्रास्ति दाता चैव न विद्यते ।।२.१० ।।
अनर्थी व्रज तत्रस्थः कल्पवृक्षसमाश्रितः ।
न कर्म नाप्रियस्तत्र न कलिः कलहो न च ।।२.११ ।।
द्वापरो न च न त्रेता कृतं चापि न विद्यते ।
मन्वन्तरं न तत्रास्ति कल्पश्चैव न विद्यते ।।२.१२ ।।
आहूतसम्प्लवं नास्ति ब्रह्मरात्रिदिनं तथा ।
न जन्ममरणं तत्र आपदं नाप्नुयात्क्वचित् ।।२.१३ ।।
न चाशापाशबद्धो ऽस्ति रागमोहं न विद्यते ।
न देवा नासुरास्तत्र न यक्षोरगराक्षसाः ।।२.१४ ।।
न भूता न पिशाचाश्च गन्धर्वा ऋषयस्तथा ।
तारा ग्रहं न तत्रास्ति नागकिंनरगारुडम् ।।२.१५ ।।
न जपो नाह्निकस्तत्र नाग्निहोत्री न यज्ञकृत् ।
न व्रतं न तपश्चैव न तिर्यं नरकं तथा ।।२.१६ ।।
तस्येशानस्य देवस्य ऐश्वर्यगुणविस्तरम् ।
अपि वर्षशतेनापि शक्यं वक्तुं न केनचित् ।।२.१७ ।।
हरेच्छाप्रभवाः सर्वे पर्यायेण ब्रवीमि ते ।
देवमानुषवर्ज्यानि वृक्षगुल्मलतादयः ।।२.१८ ।।
परार्धद्विगुणोत्सेधा विस्तारश्च तथाविधः ।
अनेकाकारपुष्पाणि फलानि च मनोहरम् ।।२.१९ ।।
अन्ये काञ्चनवृक्षाणि मणिवृक्षाण्यथापरे ।
प्रवालमणिषण्डाश्च पद्मरागरुहानि च ।।२.२० ।।
स्वादुमूलफलास्कन्दलताविटपपादपाः ।
कामरूपाश्च ते सर्वे कामदाः कामभाषिणः ।।२.२१ ।।
तत्र विप्र प्रजाः सर्वे अनन्तगुणसागराः ।
तुल्यरूपबलाः सर्वे सूर्यायुतसमप्रभाः ।।२.२२ ।।
परार्धद्वयविस्तारं परार्धद्वयमायतम् ।
परार्धद्वयविक्षेपा योजनानां द्विजोत्तम ।।२.२३ ।।
ऐश्वर्यत्वं न संख्यास्ति बलशक्तिश्च भो द्विज ।
अधोर्ध्वो न च संख्यास्ति न तिर्यञ्चैति कश्चन ।।२.२४ ।।
शिवाण्डस्य च विस्तारमायामं च न वेद्म्यहम् ।
भोगमक्षयस्तत्रैव जन्ममृत्युर्न विद्यते ।।२.२५ ।।
शिवाण्डमध्यमाश्रित्य गोक्षीरसदृशप्रभाः ।
परार्धपरकोटीनामीशानानां स्मृतालयः ।।२.२६ ।।
बालसूर्यप्रभाः सर्वे ज्ञेयास्तत्पुरुषालये ।
परार्धपरकोटीनां पूर्वस्यां दिशमाश्रिताः ।।२.२७ ।।
भिन्नाञ्जनप्रभाः सर्वे दक्षिणां दिशमाश्रिताः ।
परार्धपरकोटीनामघोरालयमाश्रिताः ।।२.२८ ।।
कुन्देन्दुहिमशैलाभाः पश्चिमां दिशमाश्रिताः ।
परार्धपरकोटीनां सद्यमिष्टालयः स्मृतः ।।२.२९ ।।
कुङ्कुमोदकसंकाशा उत्तरां दिशमाश्रिताः ।
परार्धपरकोतीनां वामदेवालयः स्मृतः ।।२.३० ।।
ईशानस्य कलाः पञ्च वक्त्रस्यापि चतुष्कलाः ।
अघोरस्य कला अष्टौ वामदेवास्त्रयोदश ।।२.३१ ।।
सद्यश्चाष्टौ कला ज्ञेयाः संसारार्णवतारकाः ।
अष्टत्रिंशत्कला ह्येताः कीर्तिता द्विजसत्तम ।।२.३२ ।।
संख्या वर्णा दिशश्चैव एकैकस्य पृथक्पृथक् ।
पूर्वोक्तेन विधानेन बोधव्यास्तत्त्वचिन्तकैः ।।२.३३ ।।
शिवाण्डगमनाकृष्ट्या शिवयोगं सदाभ्यसेत् ।
शिवयोगं विना विप्र तत्र गन्तुं न शक्यते ।।२.३४ ।।
अश्वमेधादियज्ञानां कोट्यायुतशतानि च ।
कृच्छ्रादितप सर्वाणि कृत्वा कल्पशतानि च ।
तत्र गन्तुं न शक्येत देवैरपि तपोधन ।।२.३५ ।।
गङ्गादिसर्वतीर्थेषु स्नात्वा तप्त्वा च वै पुनः ।
तत्र गन्तुं न शक्येत ऋषिभिर्वा महात्मभिः ।।२.३६ ।।
सप्तद्वीपसमुद्राणि रत्नपूर्णानि भो द्विज ।
दत्त्वा वा वेदविदुषे श्रद्धाभक्तिसमन्वितः ।
तत्र गन्तुं न शक्येत विना ध्यानेन निश्चयः ।।२.३७ ।।
स्वदेहान्मांसमुद्धृत्य दत्त्वार्थिभ्यश्च निश्चयात् ।
स्वदारपुत्रसर्वस्वं शिरो ऽर्थिभ्यश्च यो ददेत् ।
न तत्र गन्तुं शक्येत अन्यैर्वापि सुदुष्करैः ।।२.३८ ।।
यज्ञतीर्थतपोदानवेदाध्ययनपारगः ।
ब्रह्माण्डान्तस्य भोगांस्तु भुङ्क्ते कालवशानुगः ।।२.३९ ।।
कालेन समप्रेष्येण धर्मो याति परिक्षयम् ।
अलातचक्रवत्सर्वं कालो याति परिभ्रमन् ।
त्रैकाल्यकलनात्कालस्तेन कालः प्रकीर्तितः ।।२.४० ।।

 ।।इति वृषसारसंग्रहे शिवाण्डसंख्या नामाध्यायो द्वितीयः ।।





---- धर्मप्रवचनम् ----

विगतराग उवाच ।
किमर्थं धर्ममित्याहुः कतिमूर्तिश्च कीर्त्यते ।
कतिपादवृषो ज्ञेयो गतिस्तस्य कति स्मृताः ।।३.१ ।।
कौतूहलं ममोत्पन्नं संशयं छिन्धि तत्त्वतः ।
कस्य पुत्रो मुनिश्रेष्ठ प्रजास्तस्य कति स्मृताः ।।३.२ ।।
अनर्थयज्ञ उवाच ।
धृतिरित्येष धातुर्वै पर्यायः परिकीर्तितः ।
आधारणान्महत्त्वाच्च धर्म इत्यभिधीयते ।। ३.३ ।।
श्रुतिस्मृतिद्वयोर्मूर्तिश्चतुष्पादवृषः स्थितः ।
चतुराश्रम यो धर्मः कीर्तितानि मनीषिभिः ।।३.४ ।।
गतिश्च पञ्च विज्ञेयाः शृणु धर्मस्य भो द्विज ।
देवमानुषतिर्यं च नरकस्थावरादयः ।।३.५ ।।
ब्रह्मणो हृदयं भित्त्वा जातो धर्मः सनातनः ।
तस्य पत्नी महाभागा त्रयोदश सुमध्यमाः ।।३.६ ।।
दक्षकन्या विशालाक्षी श्रद्धाद्याः सुमनोहराः ।
तस्य पुत्राश्च पौत्राश्च अनेकाश्च बभूव ह ।
एष धर्मनिसर्गो ऽयं किं भूयः श्रोतुमिच्छसि ।।३.७ ।।
विगतराग उवाच ।
धर्मपत्नी विशेषेण पुत्रस्तेभ्यः पृथक्पृथक् ।
श्रोतुमिच्छामि तत्त्वेन कथयस्व तपोधन ।।३.८ ।।
अनर्थयज्ञ उवाच ।
श्रद्धा लक्ष्मीर्धृतिस्तुष्टिः पुष्टिर्मेधा क्रिया लज्जा ।
बुद्धिः शान्तिर्वपुः कीर्तिः सिद्धिः प्रसूतिसम्भवाः ।।३.९ ।।
श्रद्धा कामः सुतो जातो दर्पो लक्ष्मीसुतः स्मृतः ।
धृत्यास्तु नियमः पुत्रः संतोषस्तुष्टिजः स्मृतः ।।३.१० ।।
पुष्ट्या लाभः सुतो जातो मेधापुत्रः श्रुतस्तथा ।
क्रियायास्त्वभयः पुत्रो दण्डः समय एव च ।।३.११ ।।
लज्जाया विनयः पुत्रो बुद्ध्या बोधः सुतः स्मृतः ।
लज्जायाः सुधियः पुत्र अप्रमादश्च तावुभौ ।।३.१२ ।।
क्षेमः शान्तिसुतो विन्द्याद्व्यवसायो वपोः सुतः ।
यशः कीर्तिसुतो ज्ञेयः सुखं सिद्धेर्व्यजायत ।
स्वायम्भुवे ऽन्तरे त्वासन्कीर्तिता धर्मसूनवः ।।३.१३ ।।
विगतराग उवाच ।
मूर्तिद्वयं कथं धर्मं कथयस्व तपोधन ।
कौतूहलमतीवं मे कर्तय ज्ञानसंशयम् ।।३.१४ ।।
अनर्थयज्ञ उवाच ।
श्रुतिस्मृतिद्वयोर्मूर्तिर्धर्मस्य परिकीर्तिता ।
दाराग्निहोत्रसम्बन्धमिज्या श्रौतस्य लक्षणम् ।
स्मार्तो वर्णाश्रमाचारो यमैश्च नियमैर्युतः ।।३.१५ ।।

---- यमनियमभेदः ----

यमश्च नियमश्चैव द्वयोर्भेदमतः शृणु ।
अहिंसा सत्यमस्तेयमानृशंस्यं दमो घृणा ।।३.१६ ।।
धन्याप्रमादो माधुर्यमार्जवं च यमा दश ।
एकैकस्य पुनः पञ्चभेदमाहुर्मनीषिणः ।।३.१७ ।।

---- यमेष्व{ }हिंसा (१) ----

अहिंसादि प्रवक्ष्यामि शृणुष्वावहितो द्विज ।
त्रासनं ताडनं बन्धो मारणं वृत्तिनाशनम् ।
हिंसां पञ्चविधामाहुर्मुनयस्तत्त्वदर्शिनः ।।३.१८ ।।
काष्ठलोष्टकशाद्यैस्तु ताडयन्तीह निर्दयाः ।
तत्प्रहारविभिन्नाङ्गो मृतवध्यमवाप्नुयात् ।।३.१९ ।।
बद्ध्वा पादौ भुजोरश्च शिरोरुक्कण्ठपाशिताः ।
अनाहता म्रियन्त्येवं वधो बन्धनजः स्मृतः ।।३.२० ।।
शत्रुचौरभयैर्घोरैः सिंहव्याघ्रगजोरगैः ।
त्रासनाद्वधमाप्नोति अन्यैर्वापि सुदुःसहैः ।।३.२१ ।।
यस्य यस्य हरेद्वित्तं तस्य तस्य वधः स्मृतः ।
वृत्तिजीवाभिभूतानां तद्वारान्निहतः स्मृतः ।।३.२२ ।।
विषवह्निशरशस्त्रैर्मायायोगबलेन वा ।
हिंसकान्याहु विप्रेन्द्र मुनयस्तत्त्वदर्शिनः ।।३.२३ ।।
अहिंसा परमं धर्मं यस्त्यजेत्स दुरात्मवान् ।
क्लेशायासविनिर्मुक्तं सर्वधर्मफलप्रदम् ।।३.२४ ।।
नातः परतरो मूर्खो नातः परतरं तमः ।
नातः परतरं दुःखं नातः परतरो ऽयशः ।।३.२५ ।।
नातः परतरं पापं नातः परतरं विषम् ।
नातः परतराविद्या नातः परं तपोधन ।।३.२६ ।।
यो हिनस्ति न भूतानि उद्भिज्जादि चतुर्विधम् ।
स भवेत्पुरुषः श्रेष्ठः सर्वभूतदयान्वितः ।।३.२७ ।।
सर्वभूतदयां नित्यं यः करोति स पण्डितः ।
स यज्वा स तपस्वी च स दाता स दृड्हव्रतः ।।३.२८ ।।
अहिंसा परमं तीर्थमहिंसा परमं तपः ।
अहिंसा परमं दानमहिंसा परमं सुखम् ।।३.२९ ।।
अहिंसा परमो यज्ञः अहिंसा परमं व्रतम् ।
अहिंसा परमं ज्ञानमहिंसा परमा क्रिया ।।३.३० ।।
अहिंसा परमं शौचमहिंसा परमो दमः ।
अहिंसा परमो लाभः अहिंसा परमं यशः ।।३.३१ ।।
अहिंसा परमो धर्मः अहिंसा परमा गतिः ।
अहिंसा परमं ब्रह्म अहिंसा परमः शिवः ।।३.३२ ।।
मांसाशनान्निवर्तेत मनसापि न काङ्क्षयेत् ।
स महत्फलमाप्नोति यस्तु मांसं विवर्जयेत् ।।३.३३ ।।
स्वमांसं परमांसेन यो वर्धयितुमिच्छति ।
अनभ्यर्च्य पितॄन्देवान्न ततो ऽन्यो ऽस्ति पापकृत् ।।३.३४ ।।
मधुपर्के च यज्ञे च पितृदैवतकर्मणि ।
अत्रैव पशवो हिंस्या नान्यत्र मनुरब्रवीत् ।।३.३५ ।।
क्रीत्वा स्वयं वाप्युत्पाद्य परोपहृतमेव वा ।
देवान्पितॄंश्चार्चयित्वा खादन्मांसं न दोषभाक् ।।३.३६ ।।
वेदयज्ञतपस्तीर्थदानशीलक्रियाव्रतैः ।
मांसाहारनिवृत्तानां षोडशांशं न पूर्यते ।।३.३७ ।।
मृगाः पर्णतृणाहारादजमेषगवादिभिः ।
सुखिनो बलवन्तश्च विचरन्ति महीतले ।।३.३८ ।।
वानराः फल{ ।}म{ ।}ाहारा राक्षसा रुधिरप्रियाः ।
निहता राक्षसाः सर्वे वानरैः फलभोजिभिः ।।३.३९ ।।
तस्मान्मांसं न हीहेत बलकामेन भो द्विज ।
बलेन च गुणाकर्षात्परतो भयभीरुणा ।।३.४० ।।
अहिंसकसमो नास्ति दानयज्ञसमीहया ।
इह लोके यशः कीर्तिः परत्र च परा गतिः ।।३.४१ ।।
त्रैलोक्यं मणिरत्नपूर्णमखिलं दत्त्वोत्तमे ब्राह्मणे
कोटीयज्ञसहस्रपद्ममयुतं दत्त्वा महीं दक्षिणाम् ।
तीर्थानां च सहस्रकोटिनियुतं स्नात्वा सकृन्मानवः
एतत्पुण्यफलमहिंसकजनः प्राप्नोति निःसंशयः ।।३.४२ ।।

 ।।इति वृषसारसंग्रहे अहिंसाप्रशंसा नामाध्यायस्तृतीयः ।।





---- यमेषु सत्यम् (२) ----

अनर्थयज्ञ उवाच ।
सद्भावः सत्यमित्याहुर्दृष्टप्रत्ययमेव वा ।
यथाभूतार्थकथनं तत्सत्यकथनं स्मृतम् ।।४.१ ।।
आक्रोशताडनादीनि यः सहेत सुदुःसहम् ।
क्षमते यो जितात्मा तु स च सत्यमुदाहृतम् ।।४.२ ।।
वधार्थमुद्यतः शस्त्रं यदि पृच्छेत कर्हिचित् ।
न तत्र सत्यं वक्तव्यमनृतं सत्यमुच्यते ।।४.३ ।।
वधार्हः पुरुषः कश्चिद्व्रजेत्पथि भयातुरः ।
पृच्छतो ऽपि न वक्तव्यं सत्यं तद्वापि उच्यते ।।४.४ ।।
न नर्मयुक्तमनृतं हिनस्ति
न स्त्रीषु राजन्न विवाहकाले ।
प्राणात्यये सर्वधनापहारे
पञ्चानृतं सत्यमुदाहरन्ति ।।४.५ ।।
देवमानुषतिर्येषु सत्यं धर्मः परो यतः ।
सत्यं श्रेष्ठं वरिष्ठं च सत्यं धर्मः सनातनः ।।४.६ ।।
सत्यं सागरमव्यक्तं सत्यमक्षयभोगदम् ।
सत्यं पोतः परत्रार्थं सत्यं पन्थान विस्तरम् ।।४.७ ।।
सत्यमिष्टगतिः प्रोक्तं सत्यं यज्ञमनुत्तमम् ।
सत्यं तीर्थं परं तीर्थं सत्यं दानमनन्तकम् ।।४.८ ।।
सत्यं शीलं तपो ज्ञानं सत्यं शौचं दमः शमः ।
सत्यं सोपानमूर्ध्वस्य सत्यं कीर्तिर्यशः सुखम् ।।४.९ ।।
अश्वमेधसहस्रं च सत्यं च तुलया धृतम् ।
अश्वमेधसहस्राद्धि सत्यमेव विशिष्यते ।।४.१० ।।
सत्येन तपते सूर्यः सत्येन पृथिवी स्थिता ।
सत्येन वायवो वान्ति सत्ये तोयं च शीतलम् ।।४.११ ।।
तिष्ठन्ति सागराः सत्ये समयेन प्रियव्रतः ।
सत्ये तिष्ठति गोविन्दो बलिबन्धनकारणात् ।।४.१२ ।।
अग्निर्दहति सत्येन सत्येन शशिनाचरः ।
सत्येन विन्ध्यास्तिष्ठन्ति वर्धमानो न वर्धते ।।४.१३ ।।
लोकालोकः स्थितः सत्ये मेरुः सत्ये प्रतिष्ठितः ।
वेदास्तिष्ठन्ति सत्येषु धर्मः सत्ये प्रतिष्ठति ।।४.१४ ।।
सत्यं गौः क्षरते क्षीरं सत्यं क्षीरे घृतं स्थितम् ।
सत्ये जीवः स्थितो देहे सत्यं जीवः सनातनः ।।४.१५ ।।
सत्यमेकेन सम्प्राप्तो धर्मसाधननिश्चयः ।
रामराघववीर्येण सत्यमेकं सुरक्षितम् ।।४.१६ ।।
एतत्सत्यविधानस्य कीर्तितं तव सुव्रत ।
सर्वलोकहितार्थाय किमन्यच्छ्रोतुमिच्छसि ।।४.१७ ।।
विगतराग उवाच ।
न हि तृप्तिं विजानामि श्रुत्वा धर्मं तवाप्यहम् ।
उपरिष्टादतो भूयः कथयस्व तपोधन ।।४.१८ ।।

---- यमेष्व{ }स्तेयम् (३) ----

अनर्थयज्ञ उवाच ।
स्तेयं शृण्वथ विप्रेन्द्र पञ्चधा परिकीर्तितम् ।
अदत्तादानमादौ तु उत्कोचं च ततः परम् ।
प्रस्थव्याजस्तुलाव्याजः प्रसह्यस्तेय पञ्चमम् ।।४.१९ ।।
धृष्टदुष्टप्रभावेन परद्रव्यापकर्षणम् ।
वार्यमाणो ऽपि दुर्बुद्धिरदत्तादानमुच्यते ।।४.२० ।।
उत्कोचं शृणु विप्रेन्द्र धर्मसंकरकारकम् ।
मूल्यं कार्यविनाशार्थमुत्कोचः परिगृह्यते ।
तेन चासौ विजानीयाद्द्रव्यलोभबलात्कृतम् ।।४.२१ ।।
प्रस्थव्याज ।उपायेन कुटुम्बं त्रातुमिच्छति ।
तं च स्तेनं विजानीयात्परद्रव्यापहारकम् ।।४.२२ ।।
तुलाव्याज ।उपायेन परस्वार्थं हरेद्यदि ।
चौरलक्षणकाश्चान्ये कूटकापटिका नराः ।।४.२३ ।।
दुर्बलार्जवबालेषु च्छद्मना वा बलेन वा ।
अपहृत्य धनं मूड्हः स चोरश्चोर उच्यते ।।४.२४ ।।
नास्ति स्तेयसमं पापं नास्त्यधर्मश्च तत्समः ।
नास्ति स्तेनसमाकीर्तिर्नास्ति स्तेनसमो ऽनयः ।।४.२५ ।।
नास्ति स्तेयसमाविद्या नास्ति स्तेनसमः खलः ।
नास्ति स्तेनसम अज्ञो नास्ति स्तेनसमो ऽलसः ।।४.२६ ।।
नास्ति स्तेनसमो द्वेष्यो नास्ति स्तेनसमो ऽप्रियः ।
नास्ति स्तेयसमं दुःखं नास्ति स्तेनसमो ऽयशः ।।४.२७ ।।
प्रच्छन्नो ह्रियते ऽर्थमन्यपुरुषः प्रत्यक्षमन्यो हरेत
निक्षेपाद्धनहारिणो ऽन्य{ ।}म{ ।}धमो व्याजेन चान्यो हरेत् ।
अन्ये लेख्यविकल्पनाहृतधना अन्यो हृताद्वै हृता
अन्यः क्रीतधनो ऽपरो धयहृत एते जघन्याः स्मृताः ।।४.२८ ।।
स्तेनस्तुल्य न मूड्हमस्ति पुरुषो धर्मार्थहीनो ऽधमः
यावज्जीवति शङ्कया नरपतेः संत्रस्यमानो रटन् ।
प्राप्तःशासन तीव्रसह्यविषमं प्राप्नोति कर्मेरितः
कालेन म्रियते स याति निरयमाक्रन्दमानो भृशम् ।।४.२९ ।।
नीत्वा दुर्गतिकोटिकल्प निरयात्तिर्यत्वमायान्ति ते
तिर्यत्वे च तथैवमेकशतिकं प्रभ्रम्य वर्षार्बुदम् ।
मानुष्यं तदवाप्नुवन्ति विपुले दारिद्र्यरोगाकुलम
तस्माद्दुर्गतिहेतु कर्म सकलं त्यक्त्वा शिवं चाश्रयेत् ।।४.३० ।।

---- यमेष्व{ }ानृशंस्यम् (४) ----

अष्टमूर्तिशिवद्वेष्टा पितुर्मातुश्च यो द्विषेत् ।
गवां वा अतिथेर्द्वेष्टा नृशंसाः पञ्च एव ते ।।४.३१ ।।
अष्टमूर्तिः शिवः साक्षात्पञ्चव्योमसमन्वितः ।
सूर्यः सोमश्च दीक्षश्च दूषकः तन्नृशंसकः ।।४.३२ ।।
पिताकाशसमो ज्ञेयो जन्मोत्पत्तिकरः पिता ।
पितृदैवतमादित्यमानृशंस तमन्वितः ।।४.३३ ।।
पृथ्व्या गुरुतरी माता को न वन्देत मातरम् ।
यज्ञदानतपोवेदास्तेन सर्वं कृतं भवेत् ।।४.३४ ।।
गावः पवित्रं मङ्गल्यं देवतानां च देवताः ।
सर्वदेवमया गावस्तस्मादेव न हिंसयेत् ।।४.३५ ।।
जातमात्रस्य लोकस्य गावस्त्राता न संशयः ।
घृतं क्षीरं दधि मूत्रं शकृत्कर्षणमेव च ।।४.३६ ।।
पञ्चामृतं पञ्चपवित्रपूतं
ये पञ्चगव्यं पुरुषाः पिबन्ति ।
ते वाजिमेधस्य फलं लभन्ति
तदक्षयं स्वर्गमवाप्नुवन्ति ।।४.३७ ।।
गोभिर्न तुल्यं धनमस्ति किंचिद
दुह्यन्ति वाह्यन्ति बहिश्चरन्ति ।
तृणानि भुक्त्वा अमृतं स्रवन्ति
विप्रेषु दत्ताः कुलमुद्धरन्ति ।।४.३८ ।।
गवाह्निकं यश्च करोति नित्यं
शुश्रूषणं यः कुरुते गवां तु ।
अशेषयज्ञतपदानपुण्यं
लभत्यसौ तमनृशंसकर्ता ।।४.३९ ।।
अतिथिं यो ऽनुगच्छेत अतिथिं यो ऽनुमन्यते ।
अतिथिं यो ऽनुपूज्येत अतिथिं यः प्रशंसते ।।४.४० ।।
अतिथिं यो न पीड्येत अतिथिं यो न दुष्यति ।
अतिथिप्रियकर्ता यः अतिथेः परिचारकः ।
अतिथेः कृतसंतोषस्तस्य पुण्यमनन्तकम् ।।४.४१ ।।
आसनेनार्घपाद्येन पादशौचजलेन च ।
अन्नवस्त्रप्रदानैर्वा सर्वं वापि निवेदयेत् ।।४.४२ ।।
पुत्रदारात्मना वापि यो ऽतिथिमनुपूजयेत् ।
श्रद्धया चाविकल्पेन अक्लीबमानसेन च ।।४.४३ ।।
न पृच्छेद्गोत्रचरणं स्वाध्यायं देशजन्मनी ।
चिन्तयेन्मनसा भक्त्या धर्मः स्वयमिहागतः ।।४.४४ ।।
अश्वमेधसहस्राणि राजसूयशतानि च ।
पुण्डरीकसहस्रं च सर्वतीर्थतपःफलम् ।।४.४५ ।।
अतिथिर्यस्य तुष्येत नृशंसमतमुत्सृजेत् ।
स तस्य सकलं पुण्यं प्राप्नुयान्नात्र संशयः ।।४.४६ ।।
न गतिमतिथिज्ञस्य गतिमाप्नोति कर्हिचित् ।
तस्मादतिथिमायान्तमभिगच्छेत्कृताञ्जलिः ।।४.४७ ।।
सक्तुप्रस्थेन चैकेन यज्ञ आसीन्महाद्भुतः ।
अतिथिप्राप्तदानेन स्वशरीरं दिवं गतम् ।।४.४८ ।।
नकुलेन पुराधीतं विस्तरेण द्विजोत्तम ।
विदितं च त्वया पूर्वं प्रस्थवार्त्ता च कीर्तिता ।।४.४९ ।।

---- यमेषु दमः (५) ----

दम एव मनुष्याणां धर्मसारसमुच्चयः ।
दमो धर्मो दमः स्वर्गो दमः कीर्तिर्दमः सुखम् ।।४.५० ।।
दमो यज्ञो दमस्तीर्थं दमः पुण्यं दमस्तपः ।
दमहीन{ ।}म{ ।}धर्मश्च दमः कामकुलप्रदः ।।४.५१ ।।
निर्दमः करि मीनश्च पतङ्गभ्रमरमृगाः ।
त्वग्जिह्वा च तथा घ्राणा चक्षुः श्रवणमिन्द्रियाः ।।४.५२ ।।
दुर्जयेन्द्रियमेकैकं सर्वे प्राणहराः स्मृताः ।
दमं यो जयते सम्यग्निर्दमो निधनं व्रजेत् ।।४.५३ ।।
मृगे श्रोत्रवशान्मृत्युः पतङ्गाश्चक्षुषोर्मृताः ।
घ्राणया भ्रमरो नष्टो नष्टो मीनश्च जिह्वया ।।४.५४ ।।
स्पर्शेन च करी नष्टो बन्धनावासदुःसहः ।
किं पुनः पञ्चभुक्तानां मृत्युस्तेभ्यः किमद्भुतम् ।।४.५५ ।।
पुरूरवो ऽतिलोभेन अतिकामेन दण्डकः ।
सागराश्चातिदर्पेण अतिमानेन रावणः ।।४.५६ ।।
अतिक्रोधेन सौदास अतिपानेन यादवाः ।
अतितृष्णाच्च मान्धाता नहुषो द्विजवज्ञया ।।४.५७ ।।
अतिदानाद्बलिर्नष्ट अतिशौर्येण अर्जुनः ।
अतिद्यूतान्नलो राजा नृगो गोहरणेन तु ।।४.५८ ।।
दमेन हीनः पुरुषो द्विजेन्द्र
स्वर्गं च मोक्षं च सुखं च नास्ति ।
विज्ञानधर्मकुलकीर्तिनाश
भवन्ति विप्र दमया विहीनाः ।।४.५९ ।।

---- यमेषु घृणा (६) ----

निर्घृणो न परत्रास्ति निर्घृणो न इहास्ति वै ।
निर्घृणे न च धर्मो ऽस्ति निर्घृणे न तपो ऽस्ति वै ।।४.६० ।।
परस्त्रीषु परार्थेषु परजीवापकर्षणे ।
परनिन्दापरान्नेषु घृणां पञ्चसु कारयेत् ।।४.६१ ।।
परस्त्री शृणु विप्रेन्द्र घृणीकार्या सदा बुधैः ।
राज्ञी विप्री परिव्राजा स्वयोनिपरयोनिषु ।।४.६२ ।।
परार्थे शृणु भूयो ऽन्य अन्यायार्थ{ ।}म{ ।}ुपार्जनम् ।
आड्हप्रस्थतुलाव्याजैः परार्थं यो ऽपकर्षति ।।४.६३ ।।
जीवापकर्षणे विप्र घृणीकुर्वीत पण्डितः ।
वनजावनजा जीवा विलगाश्चरणाचराः ।।४.६४ ।।
परनिन्दा च का विप्र शृणु वक्ष्ये समासतः ।
देवानां ब्राह्मणानां च गुरुमातातिथिद्विषः ।।४.६५ ।।
परान्नेषु घृणा कार्या अभोज्येषु च भोजनम् ।
सूतके मृतके शौण्डे वर्णभ्रष्टकुले नटे ।।४.६६ ।।
एते पञ्चघृणासु सक्तपुरुषाः स्वर्गार्थमोक्षार्थिनः
लोके ऽनिन्दनमाप्नुवन्ति सततं कीर्तिर्यशोऽलंकृतम् ।
प्रज्ञाबोधश्रुतिं स्मृतिं च लभते मानं च नित्यं लभेत
दाक्षिण्यं स भवेत्स आयुष परं प्राप्नोति निःसंशयः ।।४.६७ ।।

---- यमेषु पञ्चविधो धन्यः (७) ----

चतुर्मौनश्चतुःशत्रुश्चतुरायतनं तथा ।
चतुर्ध्यानं चतुष्पादं पञ्चधन्यविधोच्यते ।।४.६८ ।।
चतुर्मौनस्य वक्ष्यामि शृणुष्वावहितो भव ।
पारुष्यपिशुनामिथ्यासम्भिन्नानि च वर्जयेत् ।।४.६९ ।।
कामः क्रोधश्च लोभश्च मोहश्चैव चतुर्विधः । 
चतुःशत्रुर्निहन्तव्यः सो ऽरिहा वीतकल्मषः ।।४.७० ।।
चतुरायतनं विप्र कथयिष्यामि तच्छृणु ।
करुणा मुदितोपेक्षा मैत्री चायतनं स्मृतम् ।।४.७१ ।।
चतुर्ध्यानाधुना वक्ष्ये संसारार्णवतारणम् ।
आत्मविद्याभवः सूक्ष्मं ध्यानमुक्तं चतुर्विधम् ।।४.७२ ।।
आत्मतत्त्वः स्मृतो धर्मो विद्या पञ्चसु पञ्चधा ।
षट्त्रिंशाक्षरमित्याहुः सूक्ष्मतत्त्वमलक्षणम् ।।४.७३ ।।
चतुष्पादः स्मृतो धर्मश्चतुराश्रममाश्रितः ।
गृहस्थो ब्रह्मचारी च वानप्रस्थो ऽथ भैक्षुकः ।।४.७४ ।।
धन्यास्ते यैरिदं वेत्ति निखिलेन द्विजोत्तम ।
पावनं सर्वपापानां पुण्यानां च प्रवर्धनम् ।।४.७५ ।।
आयुः कीर्तिर्यशः सौख्यं धन्यादेव प्रवर्धते ।
शान्तिः पुष्टिः स्मृतिर्मेधा जायते धन्यमानवे ।।४.७६ ।।

---- यमेष्व{ }प्रमादः (८) ----

प्रमादस्थान पञ्चैव कीर्तयिष्यामि तच्छृणु ।
ब्रह्महत्या सुरापानं स्तेयो गुर्वङ्गनागमम् ।
महापातकमित्याहुस्तत्संयोगी च पञ्चमः ।।४.७७ ।।
अनृतं च समुत्कर्षे राजगामी च पैशुनः ।
गुरोश्चालीकनिर्बद्धः समानि ब्रह्महत्यया ।।४.७८ ।।
ब्रह्मो ऋग्वेदनिन्दा च कूटसाक्षी सुहृद्वधः ।
गर्हितानाद्ययोर्जग्धिः सुरापानसमानि षट् ।।४.७९ ।।
रेतोत्सेकः स्वयोन्यासु कुमारीष्वन्त्यजासु च ।
सख्युः पुत्रस्य च स्त्रीषु गुरुतल्पसमः स्मृतः ।।४.८० ।।
निक्षेपस्यापहरणं नराश्वरजतस्य च ।
भूमिवज्रमणीनां च रुक्मस्तेयसमः स्मृतः ।।४.८१ ।।
चत्वार एते सम्भूय यत्पापं कुरुते नरः ।
महापातकपञ्चैतन्तेन सर्वं प्रकाशितम् ।
पञ्चप्रमादमेतानि वर्जनीयं द्विजोत्तम ।।४.८२ ।।

---- यमेषु माधुर्यम् (९) ----

कायवाङ्मनमाधुर्यं चक्षुर्बुद्धिश्च पञ्चमः ।
सौम्यदृष्टिप्रदानं च क्रूरबुद्धिं च वर्जयेत् ।।४.८३ ।।
प्रसन्नमनसा ध्यायेत्प्रियवाक्यमुदीरयेत् ।
यथाशक्तिप्रदानं च स्वाश्रमाभ्यागतो गुरुः ।।४.८४ ।।
इन्धनोदकदानं च जातवेदमथापि वा ।
सुलभानि न दत्तानि इन्धनाग्न्युदकानि च ।
क्षुते जीवेति वा नोक्तं तस्य किं परतः फलम् ।।४.८५ ।।

---- यमेष्व{ }ार्जवम् (१०) ----

पञ्चार्जवाः प्रशंसन्ति मुनयस्तत्त्वदर्शिनः ।
कर्मवृत्त्याभिवृद्धिं च पारतोषिकमेव च ।
स्त्रीधनोत्कोचवित्तं च आर्जवो नाभिनन्दति ।।४.८६ ।।
आर्जवो न वृथा यज्ञ आर्जवो न वृथा तपः ।
आर्जवो न वृथा दानमार्जवो न वृथाग्नयः ।।४.८७ ।।
आर्जवस्येन्द्रियग्रामः सुप्रसन्नो ऽपि तिष्ठति ।
आर्जवस्य सदा देवाः काये तस्य चरन्ति ते ।।४.८८ ।।
इति यमप्रविभागः कीर्तितो ऽयं द्विजेन्द्र
इह परत सुखार्थं कारयेत्तं मनुष्यः ।
दुरितमलपहारी शङ्करस्याज्ञयास्ते
भवति पृथिविभर्ता ह्येकछत्रप्रवर्ता ।।४.८९ ।।

 ।।इति वृषसारसंग्रहे यमविभागो नामाध्यायश्चतुर्थः ।।





---- नियमाः ----

विगतराग उवाच ।
कथय नियमतत्त्वं साम्प्रतं त्वं विशेषाद
अमृतवदनतुल्यं श्रोतुकामो गतो ऽस्मि ।
प्रकृतिदहनदग्धं ज्ञानतोयैर्निषिक्तम
अपर वद मतज्ञा नास्ति धर्मेषु तृप्तिः ।।५.१ ।।
अनर्थयज्ञ उवाच ।
श्रवणसुखमतो ऽन्यत्कीर्तयिष्ये द्विजेन्द्र
नियमकलविशेषः पञ्च पञ्च प्रकारः ।
हरिहरमुनिभीष्टं धर्मसारं द्विजेन्द्र
कलिकलुषविनाशं प्रायमोक्षप्रसिद्धम् ।।५.२ ।।
शौचमिज्या तपो दानं स्वाध्यायोपस्थनिग्रहः ।
व्रतोपवासमौनं च स्नानं च नियमा दश ।।५.३ ।।

---- नियमेषु शौचम् (१) ----

तत्र शौचादिनिर्देशं वक्ष्यामीह द्विजोत्तम ।
शारीरशौचमाहारो मात्रा भावश्च पञ्चमः ।।५.४ ।।
ताडयेन्न च बन्धेत न च प्राणैर्वियोजयेत् ।
परस्त्रीपरद्रव्येषु शौचं कायिकमुच्यते ।।५.५ ।।
श्रोत्रशौचं द्विजश्रेष्ठ गुदोपस्थमुखादयः ।
मुखस्याचमनं शौचमाहारवचनेषु च ।।५.६ ।।
मूत्रविष्टासमुत्सर्गे देवताराधनेषु च ।
मृत्तोयैस्तु गुदोपस्थं शौचयीत विचक्षणः ।।५.७ ।।
एकोपस्थे गुदे पञ्च तथैकत्र करे दश ।
उभयोः सप्त दातव्या मृदः शुद्धिं समीहता ।।५.८ ।।
एतच्छौचं गृहस्थानां द्विगुणं ब्रह्मचारिणाम् ।
वानप्रस्थस्य त्रिगुणं यतीनां तु चतुर्गुणम् ।।५.९ ।।
आहारशौचं वक्ष्यामि शृणुष्वावहितो भव ।
भागद्वयं तु भुञ्जीत भागमेकं जलं पिबेत् ।
वायुसंचारदानार्थं चतुर्थमवशेषयेत् ।।५.१० ।।
स्निग्धस्वादुरसैः षड्भिराहारषड्रसैर्बुधः ।
धातुवैषम्यनाशो ऽस्ति न च रोगाः सुदारुणाः ।।५.११ ।।
अभक्ष्यं च न भक्षेत अपेयं न च पाययेत् ।
अगम्यं न च गम्येत अवाच्यं न च भाषयेत् ।।५.१२ ।।
लशुनं च पलाण्डुं च गृञ्जनं कचकानि च ।
गौरं च शूकरं मांसं वर्जयेच्च विधानतः ।।५.१३ ।।
छत्त्राकं विड्वराहं च गोमांसं च न भक्षयेत् ।
चटकं च कपोतं च जालपादांश्च वर्जयेत् ।।५.१४ ।।
हंससारसचक्राह्वकुक्कुटान्शुकश्येनकान् ।
काकोलूकं बलाकं च मत्स्यादींश्चापि वर्जयेत् ।।५.१५ ।।
अमेध्यांश्चापवित्रांश्च सर्वानेव विवर्जयेत् ।
शाकमूलफलानां च अभक्ष्यं परिवर्जयेत् ।।५.१६ ।।
मानवेषु पुराणेषु शैवभारतसंहिते ।
कीर्तितानि विशेषेण शौचाचारमशेषतः ।।५.१७ ।।
त्वया जिज्ञासितो ऽस्म्यद्य संक्षिप्तः कथितो मया ।
सत्यवादी शुचिर्नित्यं ध्यानयोगरतः शुचिः ।।५.१८ ।।
अहिंसकः शुचिर्दान्तो दयाभूतक्षमा शुचिः ।
सर्वेषामेव शौचानामर्थशौचं परं स्मृतम् ।।५.१९ ।।
यो ऽर्थे हि शुचिः स शुचिर्न मृद्वारिशुचिः शुचिः ।
कायवाङ्मनसां शौचं स शुचिः सर्ववस्तुषु ।।५.२० ।।
शौचाशौचविधिज्ञ मानव यदि कालक्षये निश्चयः
सौभाग्यत्वमवाप्नुवन्ति सततं कीर्तिर्यशोऽलङ्कृतः ।
प्राप्तं तेन इहैव पुण्यसकलं सद्धर्मशास्त्रेरितम
जीवान्ते च परत्र{ ।}म{ ।}ीहितगतिं प्राप्नोति निःसंशयम् ।।५.२१ ।।

 ।।इति वृषसारसंग्रहे शौचाचारविधिर्नामाध्यायः पञ्चमः ।।





---- नियमेषु इज्या (२) ----

[अनर्थयज्ञ उवाच ।]
अथ पञ्चविधामिज्यां प्रवक्ष्यामि द्विजोत्तम ।
धर्ममोक्षप्रसिद्ध्यर्थं शृणुष्वावहितो द्विज ।।६.१ ।।
अर्थयज्ञः क्रियायज्ञो जपयज्ञस्तथैव च ।
ज्ञानं ध्यानं च पञ्चैतत्प्रवक्ष्यामि पृथक्पृथक् ।।६.२ ।।
अग्न्युपासनकर्मादि अग्निहोत्रक्रतुक्रिया ।
अष्टकाः पार्वणी श्राद्धं द्रव्ययज्ञः स उच्यते ।।६.३ ।।
आरामोद्यानवापीषु देवतायतनेषु च ।
स्वहस्तकृतसंस्कारः क्रियायज्ञ स उच्यते ।।६.४ ।।
जपयज्ञं ततो वक्ष्ये स्वर्गमोक्षफलप्रदम् ।
वेदाध्ययन कर्तव्यं शिवसंहितमेव च ।।६.५ ।।
इतिहासपुराणं च जपयज्ञः स उच्यते ।
इदं कर्म अकर्मेदमूहापोहविशारदः ।।६.६ ।।
शास्त्रचक्षुः समालोक्य ज्ञानयज्ञः स उच्यते ।
ध्यानयज्ञं समासेन कथयिष्यामि ते शृणु ।।६.७ ।।
ध्यानं पञ्चविधं चैव कीर्तितं हरिणा पुरा ।
सूर्यः सोमो ऽग्नि स्फटिकः सूक्ष्मं तत्त्वं च पञ्चमम् ।।६.८ ।।
सूर्यमण्डलमादौ तु तत्त्वं प्रकृतिरुच्यते ।
तस्य मध्ये शशिं ध्यायेत्तत्त्वं पुरुष उच्यते ।।६.९ ।।
चन्द्रमण्डलमध्ये तु ज्वालामग्निं विचिन्तयेत् ।
प्रभुतत्त्वः स विज्ञेयो जन्ममृत्युविनाशनः ।।६.१० ।।
अग्निमण्डलमध्ये तु ध्यायेत्स्फटिक निर्मलम् ।
विद्यातत्त्वः स विज्ञेयः कारणमजमव्ययम् ।।६.११ ।।
विद्यामण्डलमध्ये तु ध्यायेत्तत्त्वमनुत्तमम् ।
अकीर्तितमनौपम्यं शिवमक्षयमव्ययम् ।
पञ्चमं ध्यानयज्ञस्य तत्त्वमुक्तं समासतः ।।६.१२ ।।
विगतराग उवाच ।
एकैकस्य हि तत्त्वस्य फलं कीर्तय कीदृशम् ।
कानि लोकाः प्रपद्यन्ते कालं वास्य तपोधन ।।६.१३ ।।
अनर्थयज्ञ उवाच ।
ब्रह्मलोकं तु प्रथमं तत्त्वं प्रकृतिचिन्तया ।
कल्पकोटिसहस्राणि शिववन्मोदते सुखी ।।६.१४ ।।
द्वितीयं तत्त्व पुरुषं ध्यायमानो मृतो यदि ।
विष्णुलोकमितो याति कल्पकोट्ययुतं सुखी ।।६.१५ ।।
प्रभुतत्त्वं तृतीयं तु ध्यायमानो मरिष्यति ।
शिवलोके वसेन्नित्यं कल्पकोत्ययुतं शतम् ।।६.१६ ।।
विद्यातत्त्वामृतं ध्यायेत्सदाशिवमनामयम् ।
अक्षयं लोकमाप्नोति कल्पानान्तपरं तथा ।। ६.१७ ।।
पञ्चमं शिवतत्त्वं तु सूक्ष्मं चात्मनि संस्थितम् ।
न कालसंख्या तत्रास्ति शिवेन सह मोदते ।।६.१८ ।।
पञ्चध्यानाभियुक्तो भवति च न पुनर्जन्मसंस्कारबन्धः
जिज्ञास्यन्तां द्विजेन्द्र भवदहनकरः प्रार्थनाकल्पवृक्षः ।
जन्मेनैकेन मुक्तिर्भवति किमु न वा मानवाः साधयन्तु
प्रत्यक्षान्नानुमानं सकलमलहरं स्वात्मसंवेदनीयम् ।।६.१९ ।।

---- नियमेषु तपः (३) ----

मानसं तप आदौ तु द्वितीयं वाचिकं तपः ।
कायिकं च तृतीयं तु मनोवाक्कर्म तत्परम् ।
कायिकं वाचिकं चैव तपो मिश्रक पञ्चमम् ।।६.२० ।।
मनःसौम्यं प्रसादश्च आत्मनिग्रहमेव च ।
मौनं भावविशुद्धिश्च पञ्चैतत्तप मानसम् ।।६.२१ ।।
अनुद्वेगकरा वाणी प्रियं सत्यं हितं च यत् ।
स्वाध्यायाभ्यसनं चैव वाचिकं तप उच्यते ।।६.२२ ।।
आर्जवं च अहिंसा च ब्रह्मचर्यं सुरार्चनम् ।
शौचं पञ्चममित्येतत्कायिकं तप उचयते ।।६.२३ ।।
इष्टं कल्याणभावं च धन्यं पथ्यं हितं वदेत् ।
मनोमिश्रक पञ्चैतत्तप उक्तं महर्षिभिः ।।६.२४ ।।
स्वस्तिमङ्गलमाशीर्भिरतिथिगुरुपूजनम् ।
कायमिश्रक पञ्चैतत्तप उक्तं महात्मभिः ।।६.२५ ।।
मण्डूकयोगी हेमन्ते ग्रीष्मे पञ्चतपास्तथा ।
अभ्रावकाशे वर्षासु तपः साधनमुच्यते ।।६.२६ ।।
स्वमांसोद्धृत्य दानं च हस्तपादशिरस्तथा ।
पुष्पमुत्पाद्य दानं च सर्वे ते तप साधनाः ।।६.२७ ।।
कृच्छ्रातिकृच्छ्रं नक्तं च तप्तकृच्छ्रमयाचितम् ।
चान्द्रायणं पराकं च तपः सांतपनादयः ।।६.२८ ।।
येनेदं तप तप्यते सुमनसा संसारदुःखच्छिदम
आशापाश विमुच्य निर्मलमतिस्त्यक्त्वा जघन्यं फलम् ।
स्वर्गाकाङ्क्ष्यनृपत्वभोगविषयं सर्वान्तिकं तत्फलम
जन्तुः शाश्वतजन्ममृत्युभवने तन्निष्ठसाध्यं वहेत् ।।६.२९ ।।

 ।।इति वृषसारसंग्रहे षष्ठो ऽध्यायः ।।





---- नियमेषु दानम् (४) ----

दानानि च तथेत्याहुः पञ्चधा मुनिभिः पुरा ।
अन्नं वस्त्रं हिरण्यं च भूमि गोदान पञ्चमम् ।।७.१ ।।
अन्नात्तेजः स्मृतिः प्राणः अन्नात्पुष्टिर्वपुः सुखम् ।
अन्नाच्छ्रीः कान्ति वीर्यं च अन्नात्सत्त्वं च जायते ।।७.२ ।।
अन्नाज्जीवन्ति भूतानि अन्नं तुष्टिकरं सदा ।
आन्नात्कामो मदो दर्पः अन्नाच्छौर्यं च जायते ।।७.३ ।।
अन्नं क्षुधातृषाव्याधीन्सद्य एव विनाशयेत् ।
अन्नदानाच्च सौभाग्यं ख्यातिः कीर्तिश्च जायते ।।७.४ ।।
अन्नदः प्राणदश्चैव प्राणदश्चापि सर्वदः ।
तस्मादन्नसमं दानं न भूतं न भविष्यति ।।७.५ ।।
वस्त्राभावान्मनुष्यस्य श्रियादपि परित्यजेत् ।
वस्त्रहीनो न पूज्येत भार्यापुत्रसखादिभिः ।।७.६ ।।
विद्यावान्सुकुलीनो ऽपि ज्ञानवान्गुणवानपि ।
वस्त्रहीनः पराधीनः परिभूतः पदे पदे ।।७.७ ।।
अपमानमवज्ञां च वस्त्रहीनो ह्यवाप्नुयात् ।
जुगुप्सति महात्मापि सभास्त्रीजनसंसदि ।।७.८ ।।
तस्माद्वस्त्रप्रदानानि प्रशंसन्ति मनीषिणः ।
न जीर्णं स्फुटितं दद्याद्वस्त्रं कुत्सितमेव वा ।।७.९ ।।
नवं पुराणरहितं मृदु सूक्ष्मं सुशोभनम् ।
सुसंस्कृत्य प्रदातव्यं श्रद्धाभक्तिसमन्वितम् ।।७.१० ।।
श्रद्धासत्त्वविशेषेण देशकालविधेन च ।
पात्रद्रव्यविशेषेण फलमाहुः पृथक्पृथक् ।।७.११ ।।
यादृशं दीयते वस्त्रं तादृशं प्राप्यते फलम् ।
जीर्णवस्त्रप्रदानेन जीर्णवस्त्रमवाप्नुयात् ।
शोभनं दीयते वस्त्रं शोभनं वस्त्रमाप्नुयात् ।।७.१२ ।।
दद्याद्वस्त्र सुशोभनं द्विजवरे काले शुभे सादरम
सौभाग्यमतुलं लभेत स नरो रूपं तथा शोभनम् ।
तस्मिन्याति सुवस्त्रकोटि शतशः प्राप्नोति निःसंशयम
तस्मात्त्वं कुरु वस्त्रदानमसकृत्पारत्रिकोत्कर्षणम् ।।७.१३ ।।
सुवर्णदानं विप्रेन्द्र संक्षिप्य कथयाम्यहम् ।
पवित्रं मङ्गलं पुण्यं सर्वपातकनाशनम् ।।७.१४ ।।
धारयेत्सततं विप्र सुवर्णकटकाङ्गुलिम् ।
मुच्यते सर्वपापेभ्यो राहुना चन्द्रमा यथा ।।७.१५ ।।
दत्त्वा सुवर्णं विप्रेभ्यो देवेभ्यश्च द्विजर्षभ ।
तुटिमात्रे ऽपि यो दद्यात्सर्वपापैः प्रमुच्यते ।।७.१६ ।।
रक्तिमाषककर्षं वा पलार्धं पलमेव वा ।
एवमेव फलं वृद्धिर्ज्ञेया दानविशेषतः ।।७.१७ ।।
सर्वाधारं महीदानं प्रशंसन्ति मनीषिणः ।
अन्नवस्त्रहिरण्यादि सर्वं वै भूमिसम्भवम् ।।७.१८ ।।
भूमिदानेन विप्रेन्द्र सर्वदानफलं लभेत् ।
भूमिदानसमं विप्र यद्यस्ति वद तत्त्वतः ।।७.१९ ।।
मातृकुक्षिविमुक्तस्तु धरणीशरणो भवेत् ।
चराचराणां सर्वेषां भूमिः साधारणा स्मृता ।।७.२० ।।
एकहस्तं द्विहस्तं वा पञ्चाशच्छतमेव वा ।
सहस्रायुतलक्षं वा भूमिदानं प्रशस्यते ।।७.२१ ।।
एकहस्तां च यो भूमिं दद्याद्द्विजवराय तु ।
वर्षकोटिशतं दिव्यं स्वर्गलोके महीयते ।।७.२२ ।।
एवं बहुषु हस्तेषु गुणागुणि फलं स्मृतम् ।
श्रद्धाधिकं फलं दानं कथितं ते द्विजोत्तम ।।७.२३ ।।
जामदग्न्येन रामेण भूमिं दत्त्वा द्विजाय वै ।
आयुरक्षयमाप्तं तु इहैव च द्विजोत्तम ।।७.२४ ।।
हेमशृङ्गां रौप्यखुरां चैलघण्टां द्विजोत्तम ।
विप्राय वेदविदुषे दत्त्वानन्तफलं स्मृतम् ।।७.२५ ।।
दानाभ्यासरतः प्रवर्तनभवां शक्यानुरूपं सदा
अन्नं वस्त्रहिरण्यरौप्यमुदकं गावस्तिलान्मेदिनीम् ।
दद्यात्पादुकछत्त्रपीठकलशं पात्राद्यमन्यच्च वा
श्रद्धादानमभिन्नरागवदनं कृत्वा मनो निर्मलम् ।।७.२६ ।।
दानादेव यशः श्रियः सुखकराः ख्यातिं च तुल्यां लभेत
दानादेव निगर्हणं रिपुगणे आनन्ददं सौख्यदम् ।
दानाद्दुर्जयता प्रसादमतुलं सौभाग्य दानाल्लभेत
दानादेव अनन्तभोग नियतं स्वर्गं च तस्माद्भवेत् ।।७.२७ ।।
दानादेव च शक्रलोकसकलं दानाज्जनानन्दनम
दानादेव महीं समस्त बुभुजे सम्राड्महीमण्डले ।
दानादेव सुरूपयोनिसुभगश्चन्द्राननो वीक्ष्यते
दानादेव अनेकसम्भवसुखं प्राप्नोति निःसंशयम् ।।७.२८ ।।

 ।।इति वृषसारसंग्रहे दानप्रशंसाध्यायः सप्तमः ।।





---- नियमेषु स्वाध्यायः (५) ----

पञ्चस्वाध्यायनं कार्यमिहामुत्र सुखार्थिना ।
शैवं सांख्यं पुराणं च स्मार्तं भारतसंहिताम् ।।८.१ ।।
शैवतत्त्वं विचिन्तेत शैवपाशुपतद्वये ।
अत्र विस्तरतः प्रोक्तं तत्त्वसारसमुच्चयम् ।।८.२ ।।
संख्यातत्त्वं तु सांख्येषु बोद्धव्यं तत्त्वचिन्तकैः ।
पञ्चतत्त्वविभागेन कीर्तितानि महर्षिभिः ।।८.३ ।।
पुराणेषु महीकोषो विस्तरेण प्रकीर्तितः ।
अधोर्ध्वमध्यतिर्यं च यत्नतः सम्प्रवेशयेत् ।।८.४ ।।
स्मार्तं वर्णाश्रमाचारं धर्मन्यायप्रवर्तनम् ।
शिष्टाचारो ऽविकल्पेन ग्राह्यस्तत्र अशङ्कितः ।।८.५ ।।
इतिहासमधीयानः सर्वज्ञः स नरो भवेत् ।
धर्मार्थकाममोक्षेषु संशयस्तेन छिद्यते ।।८.६ ।।

---- नियमेष्व{ }ुपस्थनिग्रहः (६) ----

शृणुष्वावहितो विप्र पञ्चोपस्थविनिग्रहम् ।
स्त्रियो वा गर्हितोत्सर्गः स्वयंमुक्तिश्च कीर्त्यते ।
स्वप्नोपघातं विप्रेन्द्र दिवास्वप्नं च पञ्चमः ।।८.७ ।।
अगम्या स्त्री दिवा पर्वे धर्मपत्न्यपि वा भवेत् ।
विरुद्धस्त्री न सेवेत वर्णभ्रष्टाधिकासु च ।।८.८ ।।
अजमेषगवादीनां वडवा महिषीषु च ।
गर्हितोत्सर्गमित्येतद्यत्नेन परिवर्जयेत् ।।८.९ ।।
अन्योन्यकषणा वापि अपानकषणापि वा ।
स्वयंमुक्तिरियं ज्ञेया तस्मात्तां परिवर्जयेत् ।।८.१० ।।
स्वप्नघातं द्विजश्रेष्ठ अनिष्टं पण्डितैः सदा ।
स्वप्ने स्त्रीषु रमन्ते च रेतः प्रक्षरते ततः ।।८.११ ।।
दिवाशयं न कर्तव्यं नित्यं धर्मपरेण तु । 
स्वर्गमार्गार्गला ह्येताः स्त्रियो नाम प्रकीर्तिताः ।।८.१२ ।।

---- नियमेषु व्रतपञ्चकम् (७) ----

मार्जारकबकश्वानगोमहीव्रतपञ्चकम् ।
स्वविष्ठमूत्रं भूमीषु छादयेद्द्विजसत्तम ।
सूर्यसोमानुमोदन्ति मार्जारव्रतिकेषु च ।।८.१३ ।।
बकवच्चेन्द्रियग्रामं सुनियम्य तपोधन ।
साधयेच्च मनस्तुष्टिं मोक्षसाधनतत्परः ।।८.१४ ।।
मूत्रविष्ठे न भूमीषु कुरुते श्वनदं सदा ।
तुष्यते भगवान्शर्वः श्वानव्रतचरो यदि ।।८.१५ ।।
मूत्रवर्चो न रुध्येत सदा गोव्रतिको नरः ।
भीमतुष्टिकरश्चैव पुराणेषु निगद्यते ।।८.१६ ।।
कुद्दालैर्दारयन्तो ऽपि कीलकोटिशतैश्चितः ।
क्षमते पृथिवी देवी एवमेव महीव्रतः ।।८.१७ ।।
व्रतपञ्चकमित्येतद्यश्चरेत जितेन्द्रियः ।
स चोत्तममिदं लोकं प्राप्नोति न च संशयः ।।८.१८ ।।

---- नियमेष्व{ }ुपवासः (८) ----

शेषान्नमन्तरान्नं च नक्तायाचितमेव च ।
उपवासं च पञ्चैतत्कथयिष्यामि तच्छृणु ।।८.१९ ।।
वैश्वदेवातिथिशेषं पितृशेषं च यद्भवेत् ।
भृत्यपुत्रकलत्रेभ्यः शेषाशी विघसाशनः ।।८.२० ।।
अन्तरा प्रातराशी च सायमाशी तथैव च ।
सदोपवासी भवति यो न भुङ्क्ते कदाचन ।।८.२१ ।।
न दिवा भोजनं कार्यं रात्रौ नैव च भोजयेत् ।
नक्तवेले च भोक्तव्यं नक्तधर्मं समीहता ।।८.२२ ।।
अनारम्भस्य आहारं कुर्यान्नित्यमयाचितम् ।
परैर्दत्तं तु यो भुङ्क्ते तमयाचितमुच्यते ।।८.२३ ।।
भक्ष्यं भोज्यं च लेह्यं च चोष्यं पेयं च पञ्चमम् ।
न काङ्क्षेन्नोपयुञ्जीत उपवासः स उच्यते ।।८.२४ ।।

---- नियमेषु मौनव्रतम् (९) ----

मिथ्यापिशुनपारुष्यस्पृष्टवागप्रलापनम् ।
मौनपञ्चकमित्येतद्धारयेन्नियतव्रतः ।।८.२५ ।।
असम्भूतमदृष्टं च धर्माच्चापि बहिष्कृतम् ।
अनर्थाप्रियवाक्यं यत्तन्मिथ्यावचनं स्मृतम् ।।८.२६ ।।
परश्रीं नाभिनन्दन्ति परस्यैश्वर्यमेव च ।
अनिष्टदर्शनाकाङ्क्षी पिशुनः समुदाहृतः ।।८.२७ ।।
मृतमाता पिता चैव हानिस्थानं कथं भवेत् ।
भुक्त्वा कामममृष्टानां पारुष्यं समुदाहृतम् ।।८.२८ ।।
हृदि न स्फुटसे मूड्ह शिरो वा न विदार्यसे ।
एवमादीन्यनेकानि तीक्ष्णवादी स उच्यते ।।८.२९ ।।
द्यूतभोजनयुद्धं च मद्यस्त्रीकर्षमेव च ।
असत्प्रलापः पञ्चैतत्कीर्तितं मे द्विजोत्तम ।।८.३० ।।
मौनमेव सदा कार्यं वाक्यसौभाग्यमिच्छता ।
अपारुष्यमसम्भिन्नं वाक्यं सत्यमुदीरयेत् ।।८.३१ ।।
यस्तु मौनस्य नो कर्ता दूषितः स कुलाधमः ।
जन्मे जन्मे च दुर्गन्धो मूकश्चैवोपजायते ।।८.३२ ।।
तस्मान्मौनव्रतं सदैव सुदृड्हं कुर्वीत यो निश्चितम
वाचा तस्य अलङ्घ्यता च भवति सर्वां सभां नन्दति ।
वक्त्राच्चोत्पलगन्धमस्य सततं वायन्ति गन्धोत्कटाः
शास्त्रानेकसहस्रशो गिरिनरः प्रोच्चार्यते निर्मलः ।।८.३३ ।।

---- नियमेषु स्नानम् (१०) ----

स्नानं पञ्चविधं चैव प्रवक्ष्यामि यथातथम् ।
आग्नेयं वारुणं ब्राह्म्यं वायव्यं दिव्यमेव च ।।८.३४ ।।
आग्नेयं भस्मना स्नानं तोयाच्छतगुणं फलम् ।
भस्मपूतं पवित्रं च भस्म पापप्रणाशनम् ।।८.३५ ।।
तस्माद्भस्म प्रयुञ्जीत देहिनां तु मलापहम् ।
सर्वशान्तिकरं भस्म भस्म रक्षकमुत्तमम् ।।८.३६ ।।
भस्मना त्र्यायुषं कृत्वा ब्रह्मचर्यव्रते स्थितम् ।
भस्मना ऋषयः सर्वे पवित्रीकृतमात्मनः ।।८.३७ ।।
भस्मना विबुधा मुक्ता वीरभद्रभयार्दिताः ।
भस्मानुसंसंदृष्ट्यैव ब्रह्मणानुमता कृतः ।।८.३८ ।।
चतुराश्रमतो ऽधिक्यं व्रतं पाशुपतं कृतम् ।
तस्मात्पाशुपतं श्रेष्ठं भस्मधारणहेतवः ।।८.३९ ।।
वारुणं सलिलं स्नानं कर्तव्यं विविधं नरैः ।
नदीतोयतडागेषु प्रस्रवेषु ह्रदेषु च ।।८.४० ।।
ब्रह्मस्नानं च विप्रेन्द्र आपोहिष्ठं विदुर्बुधाः ।
त्रिसंध्यमेव कर्तव्यं ब्रह्मस्नानं तदुच्यते ।।८.४१ ।।
गोषु संचारमार्गेषु यत्र गोधूलिसम्भवः ।
तत्र गत्वावसीदेत स्नानमुक्तं मनीषिभिः ।।८.४२ ।।
वर्षतोयाम्बुधाराभिः प्लावयित्वा स्वकां तनुम् ।
स्नानं दिव्यं वदत्येव जगदादिमहेश्वरः ।।८.४३ ।।
इति नियमविभागः पञ्चभेदेन विप्र
निगदित तव पृष्टः सर्वलोकानुकम्प ।
सकलमलपहारी धर्मपञ्चाशदेतन
न भवति पुनजन्म कल्पकोट्यायुते ऽपि ।।८.४४ ।।

 ।।इति वृषसारसंग्रहे नियमप्रशंसा नामाध्यायो ऽष्टमः ।।





---- त्रैगुण्यम् ----

[अनर्थयज्ञ उवाच ।]
त्रिकालगुणभेदेन भिन्नं सर्वचराचरम् ।
तस्मात्त्रिगुणबन्धेन वेष्टितं निखिलं जगत् ।।९.१ ।।
विगतराग उवाच ।
त्रैकाल्यमिति किं ज्ञेयं त्रैधातुकशरीरिणः ।
किंचिद्विस्तरमेवेह कथयस्व तपोधन ।।९.२ ।।
अनर्थयज्ञ उवाच ।
त्रैकाल्यं त्रिगुणं ज्ञेयं व्यापी प्रकृतिसम्भवः ।
अन्योन्यमुपजीवन्ति अन्योन्यमनुवर्तिनः ।।९.३ ।।
सत्त्वं रजस्तमश्चैव रजः सत्त्वं तमस्तथा ।
तमः सत्त्वं रजश्चैव अन्योन्यमिथुनाः स्मृताः ।।९.४ ।।
सात्त्विको भगवान्विष्णु राजसः लोद्भवः ।
तामसो भगवानीशः सकलं विकलेश्वरः ।।९.५ ।।
सत्त्वं कुन्देन्दुवर्णाभं पद्मरागनिभं रजः ।
तमश्चाञ्जनशैलाभं कीर्तितानि मनीषिभिः ।।९.६ ।।
सत्त्वं जलं रजो ऽङ्गारं तमो धूमसमाकुलम् ।
एतद्गुणमयैर्बद्धाः पच्यन्ते सर्वदेहिनः ।।९.७ ।।
विगतराग उवाच ।
केन केन प्रकारेण गुणपाशेन बध्यते ।
चिह्नमेषां पृथक्त्वेन कथयस्व तपोधन ।।९.८ ।।
अनर्थयज्ञ उवाच ।
अनेकाकारभावेन बध्यन्ते गुणबन्धनैः ।
मोहिता नाभिजानन्ति जानन्ति शिवयोगिनः ।।९.९ ।।
ऊर्ध्वंगो नित्यसत्त्वस्थो मध्यगो रजसावृतः ।
अधोगतिस्तमोऽवस्था भवन्ति पुरुषाधमाः ।।९.१० ।।
स्वर्गे ऽपि हि त्रयो वैते भावनीयास्तपोधन ।
मानुषेषु च तिर्येषु गुणभेदास्त्रयस्त्रयः ।।९.११ ।।
ब्रह्मा विष्णुश्च रुद्रश्च धर्म इन्द्रः प्रजापतिः ।
सोमो ऽग्नि वरुणः सूर्यो दश सत्त्वोत्तमाः स्मृताः ।।९.१२ ।।
रुद्रादित्या वसुसाध्याः विश्वेशमरुतो ध्रुवः ।
ऋषयः पितरश्चैव दशैते सत्त्वमध्यमाः ।।९.१३ ।।
तारा ग्रहा सुरा यक्षा गन्धर्वाः किंनरोरगाः ।
रक्षोभूतपिशाचाश्च दशैते सात्त्विकाधमाः ।।९.१४ ।।
ऋत्विक्पुरोहिताचार्ययज्वानो ऽतिथिविज्ञनी ।
राजमन्त्री व्रती वेदी दशैते राजसोत्तमाः ।।९.१५ ।।
सूतो ऽम्बष्टवणिक्चोग्रः शिल्पकारुकमागधाः ।
वेणवैदेहकामात्या दशैते रजमध्यमाः ।।९.१६ ।।
चर्मकृत्कुम्भकृत्कोली लोहकृत्त्रपुनीलिकाः ।
नटमुष्टिकचण्डाला दशैते रजसाधमाः ।।९.१७ ।।
गोगजगवया अश्वमृगचामरकिंनराः ।
सिंहव्याघ्रवराहाश्च दशैते तमसोत्तमाः ।।९.१८ ।।
अजमेषमहिष्याश्च मूषिकानकुलादयः ।
उष्ट्ररङ्कुशशगण्डा दशैते तममध्यमाः ।।९.१९ ।।
ऋक्षगोधामृगशृङ्गिबकवानरगर्दभाः ।
सूकरश्वानगोमायुर्दशैते तामसाधमाः ।।९.२० ।।
क्रौञ्चहंसशुकश्येनभासवारुण्डसारसाः ।
चक्राह्वशुकमायूरा दशैते तमसात्त्विकाः ।।९.२१ ।।
वलाकाः कुक्कुटाः काकाश्चिल्ललावकतित्तिराः ।
गृध्रकङ्कबकश्येन दशैते तमराजसाः ।।९.२२ ।।
कोकिलोलूककिञ्जल्ककपोताः पञ्च एव च ।
शारिकाश्च कुलिङ्गाश्च दशैते तमसाधमाः ।।९.२३ ।।
मकरगोहनक्राश्च ऋषा च तमसात्त्विकाः ।
कच्छपशुशुकुम्भीरमण्डुकास्तमराजसाः ।
शङ्खशुक्तिकशम्बूककबन्ध्यास्तमतामसाः ।।९.२४ ।।
चन्दनागरुपद्मं च प्लक्षोदुम्बरपिप्पलाः ।
वटदारुशमीबिल्वा दशैते तमसात्त्विकाः ।।९.२५ ।।
जाम्बीरलकुचाम्रातदाडिमाकोलवेतसाः ।
निम्बिनीपो धुवावश्च दशैते तमराजसाः ।।९.२६ ।।
वृक्षवल्लीलतावेणुत्वक्सारतृणभूरुहाः ।
मीरजा च शिलाशस्या दशैते तमसात्त्विकाः ।।९.२७ ।।
भ्रमरादिपतङ्गाश्च क्रिमिकीटजलौकसः ।
यूकोद्दंशमशानां च विष्टजास्तमसात्त्विकाः ।।९.२८ ।।
दया सत्यं दमः शौचं ज्ञानं मौनं तपः क्षमा ।
शिलं च नाभिमानं च सात्त्विकाश्चोत्तमा जनाः ।।९.२९ ।।
कामतृष्णारतिद्यूतमानो युद्धं मदः स्पृहा ।
निर्घृणाः कलिकर्तारो राजसेषूत्तमा जनाः ।।९.३० ।।
हिंसासूयाघृणामूड्हनिद्रातन्द्रीभयालसाः ।
क्रोधो मत्सरमायी च तामसेषूत्तमा जनाः ।।९.३१ ।।
लघुप्रीतिप्रकाशी च ध्यानयोगे सदोत्सुकः ।
प्रज्ञाबुद्धिविरागी च सात्त्विकं गुणलक्षणम् ।।९.३२ ।।
बालको निपुणो रागी मानो दर्पश्च लोभकः ।
स्पृहा ईर्ष्या प्रलापी च राजसं गुणलक्षणम् ।।९.३३ ।।
उद्वेग आलसो मोहः क्रूरस्तस्करनिर्दयः ।
क्रोधः पिशुननिद्रा च तामसं गुणलक्षणम् ।।९.३४ ।।
विगतराग उवाच ।
केन चिह्नेन विज्ञेय आहारः सर्वदेहिनाम् ।
त्रैगुण्यस्य पृथक्त्वेन कथयस्व तपोधन ।।९.३५ ।।
अनर्थयज्ञ उवाच ।
आयुः कीर्तिः सुखं प्रीतिर्बलारोग्यविवर्धनम् ।
हृद्यस्वादुरसं स्निग्ध आहारः सात्त्विकप्रियः ।।९.३६ ।।
अत्युष्णमाम्ललवणं रूक्षं तीक्ष्णं विदाहि च ।
राजसश्रेष्ठ आहारो दुःखशोकामयप्रदः ।।९.३७ ।।
अभक्ष्यामेध्यपूती च पूति पर्युषितं च यत् ।
आयामरसविस्वाद आहारस्तामसप्रियः ।।९.३८ ।।
विगतराग उवाच ।
गुणातीतं कथं ज्ञेयं संसारपरपारगम् ।
गुणपाशनिबद्धानां मोक्षं कथय तत्त्वतः ।।९.३९ ।।
अनर्थयज्ञ उवाच ।
आत्मवत्सर्वभूतानि सम्यक्पश्येत भो द्विज ।
गुणातीतः स विज्ञेयः संसारपरपारगः ।।९.४० ।।
ईर्षाद्वेषसमो यस्तु सुखदुःखसमाश्च ये ।
स्तुतिनिन्दासमा ये च गुणातीतः स उच्यते ।।९.४१ ।।
तुल्यः प्रियाप्रियो यश्च अरिमित्रसमस्तथा ।
मानापमानयोस्तुल्यो गुणातीतः स उच्यते ।। ९.४२ ।।
एष ते कथितो विप्र गुणसद्भावनिर्णयः ।
गुणयुक्तस्तु संसारी गुणातीतः पराङ्गतिः ।।९.४३ ।।

 ।।इति वृषसारसंग्रहे त्रैगुण्यविशेषनीयो नामाध्यायो नवमः ।।





---- कायतीर्थोपवर्णनम् ----

विगतराग उवाच ।
कतमं सर्वतीर्थानां श्रेष्ठमाहुर्मनीषिनः ।
कथयस्व मुनिश्रेष्ठ यद्यस्ति भुवि कामदम् ।।१०.१ ।।
अनर्थयज्ञ उवाच ।
अतिगुह्यमिदं प्रश्नं पृष्टः स्नेहाद्द्विजोत्तम ।
ब्रवीमि वः पुरावृत्तं नन्दिना कथितो ऽस्म्यहम् ।।१०.२ ।।
नन्दिकेश्वर उवाच ।
कैलासशिखरे रम्ये सिद्धचारणसेविते ।
तत्रासीनं शिवं साक्षाद्देवी वचनमब्रवीत् ।।१०.३ ।।
देव्युवाच ।
भगवन्देवदेवेश सर्वभूतजगत्पते ।
प्रष्टुमिच्छाम्यहं त्वेकं धर्मगुह्यं सनातनम् ।।१०.४ ।।
अतितीर्थं परं गुह्यं संसाराद्येन मुच्यते ।
मनुष्याणां हितार्थाय ब्रूहि तत्त्वं महेश्वर ।।१०.५ ।।
महेश्वर उवाच ।
को मां पृच्छति तं प्रश्नं मुक्त्वा त्वामेव सुन्दरि ।
शृणु वक्ष्यामि तत्प्रश्नं देवैरपि सुदुर्लभम् ।।१०.६ ।।
कुरुक्षेत्रं प्रयागं च वाराणसीमतः परम् ।
गङ्गाग्निं सोमतीर्थं च सूर्यपुष्करमानसम् ।।१०.७ ।।
नैमिषं बिन्दुसारं च सेतुबन्धं सुरह्रदम् ।
घण्टिकेश्वरवागीशं ज्ञात्वा निश्चयपापहा ।।१०.८ ।।
उमोवाच ।
एवमादि महादेव पूर्ववत्कथितास्म्यहम् ।
स्वर्गभोगप्रदं तीर्थमेतेषां सुरनायक ।।१०.९ ।।
कथं मुच्येत संसाराज्ज्ञानमात्रेण ईश्वर ।
कौतूहलं महज्जातं छिन्धि संशयकारकम् ।।१०.१० ।।
रुद्र उवाच ।
किं न जानामि तत्तीर्थं सुलभं दुर्लभं च यत् ।
सुलभं गुरुसेवीनां दुर्लभं तद्विवर्जयेत् ।।१०.११ ।।
कुरुः पुरुष विज्ञेयः शरीरं क्षेत्र उच्यते ।
शरीरस्थं कुरुक्षेत्रं सर्वतीर्थफलप्रदम् ।।१०.१२ ।।
सर्वयज्ञफलावाप्तिः सर्वदानफलानि च ।
सर्वव्रततपश्चीर्णं तत्फलं सकलं भवेत् ।।१०.१३ ।।
एवमेव फलं तेषां तीर्थपञ्चदशेषु च ।
अनघानं महापुण्यं महातीर्थं महासुखम् ।।१०.१४ ।।
देव्युवाच ।
अतीव रोमहर्षो मे जातो ऽस्ति त्रिदशेश्वर ।
सुलभं सुकरं सूक्ष्मं श्रुत्वा तुष्टिश्च मे गता ।।१०.१५ ।।
चतुर्दश परो भूयः कथयस्व मनोहरम् ।
प्रयागादि पृथक्त्वेन तत्त्वतस्तु सुरेश्वर ।।१०.१६ ।।
रुद्र उवाच ।
सुषुम्ना भगवती गङ्गा इडा च यमुना नदी ।
एता स्रोतवहा नद्यः प्रयागः स विधीयते ।।१०.१७ ।।
दक्षिणा वारुणी नासा वामनासा असि स्मृता ।
वारुणा ।असिमध्येन तेन वाराणसी स्मृता ।।१०.१८ ।।
आकाशगङ्गा विख्याता तस्याः स्रवति चामृतम् ।
अहोरात्रमविच्छिन्नं गङ्गा सा तेन उच्यते ।।१०.१९ ।।
सोमतीर्थमिडा नाडी किङ्किणीरवचिह्निता ।
तं तु श्रुत्वा न संदेहः सर्वपापक्षयो भवेत् ।।१०.२० ।।
सूर्यतीर्थं सुषुम्ना च नीरवारवसंयुता ।
श्रुतिमात्राद्विमुच्येत पापराशिर्महानपि ।।१०.२१ ।।
अग्नितीर्थार्जुना नाडी ब्रह्मघोषमनोरमा ।
तत्तदक्षरमाकर्ण्य अमृतत्वाय कल्पते ।।१०.२२ ।।
पुष्करं हृदि मध्यस्थमष्टपत्त्रं सकर्णिकम् ।
चिन्तयेत्सूक्ष्म तन्मध्ये जन्ममृत्युविनाशनम् ।।१०.२३ ।।
मानससरमध्यस्थं सहंसकमलोपरि ।
सलीलो लीलयाचारी परतः परपारगः ।।१०.२४ ।।
नैमिषं शृणु देवेशि निमिषा प्रत्ययो भवेत् ।
सम्यग्छायां निरीक्षेत आत्मानो वा परस्य वा ।।१०.२५ ।।
आयतप्यङ्गुलीमात्रं निमिषाक्षि स पश्यति ।
दृष्ट्वा प्रत्ययमेवं हि नैमिषज्ञः स उच्यते ।।१०.२६ ।।
तीर्थं बिन्दुसरं नाम शृणु वक्ष्यामि सुन्दरि ।
देहमध्ये हृदि ज्ञेयं हृदिमध्ये तु पङ्कजम् ।।१०.२७ ।।
कर्णिका पद्ममध्ये तु बिन्दुः कर्णिकमध्यतः ।
बिन्दुमध्ये स्थितो नादः स नादः केन भिद्यते ।।१०.२८ ।।
उकारं च मकारं च भित्त्वा नादो विनिर्गतः ।
तं विदित्वा विशालाक्षि सो ऽमृतत्वं लभेत च ।।१०.२९ ।।
वक्ष्ये ते सेतुबन्धं दुरितमलहरं नादतोयप्रवाहम
जिह्वाकण्ठोरकूला स्वरगणपुलिनावर्तघोषा तरङ्गा ।
कुम्भीराघोषमीना दशगणमकरा भीमनक्रा विसर्गा
सानुस्वारे गभीरे मदसुखरसनं सेतुबन्धं व्रजस्व ।।१०.३० ।।
सप्तद्वीपान्तमध्ये शृणु शशिवदने सर्वदुःखान्तलाभम
ईशानेनाभिजुष्टं हृदि ह्रद विमलं नादशीताम्बुपूर्णम् ।
तत्रैकं जातपद्मं प्रकृतिदलयुतं केशरशक्तिभिन्नं
पञ्चव्योमप्रशस्तं गतिपरमपदं प्राप्तुकामेन सेव्यम् ।।१०.३१ ।।
नाड्यैकासङ्गतानि निपतितममृतं घण्टिकापारकेण
तृप्यन्ते तेन नित्यं हृदि कमलपुटं स्थानुभूतान्तरात्मा ।
यं पश्यन्तीशभक्ता कलिकलुषहरं व्यापिनं निष्प्रपञ्चम
देवेशं घण्टिकेशमरभवमभवन्तीर्थमाकाशबिन्दुम् ।।१०.३२ ।।
मीमांसारत्नकूला क्रमपदपुलिना शैवशास्त्रार्थतोया
मीनौघा पञ्चरात्रं श्रुतिकुटिलगतिस्मार्तवेगा तरङ्गा ।
योगावर्तातिशोभा उपनिषदिवहा भारतावर्तफेना
पञ्चाशद्व्योमरूपी रसभवननदी तीर्थवागीश्वरीयम् ।।१०.३३ ।।
यस्तं वेत्ति स वेत्ति वेदनिखिलं संसारदुःखच्छिदम
जन्मव्याधिवियोगतापमरणं क्लेशार्णवं दुःसहम् ।
गर्भावासमतीव सह्यविषयं दुस्तीर्यदुःखालयम
प्राप्तं तेन न संशयः शिवपदं दुष्प्राप्य देवैरपि ।।१०.३४ ।।

 ।।इति वृससारसंग्रहे कायतीर्थोपवर्णनो नामाध्यायो दशमः ।।





---- चतुराश्रमधर्मविधानः ----

देव्युवाच ।
सर्वयज्ञः परश्रेष्ठ अस्ति अन्यः सुरोत्तम । 
अल्पक्लेशमनायास अर्थप्रायं विनेश्वर ।।११.१ ।।
सर्वयज्ञफलावाप्ति दैवतैश्चापि पूजितम् ।
कथयस्व सुरश्रेष्ठ मानुषाणां हिताय वै ।।११.२ ।।
महेश्वर उवाच ।
न तुल्यं तव पश्यामि दया भूतेषु भामिनि ।
किमन्यत्कथयिष्यामि दया यत्र न विद्यते ।।११.३ ।।
सदाशिवमुखात्पूर्वं श्रुतं मे वरसुन्दरि ।
शृणु देवि प्रवक्ष्यामि धर्मसारमनुत्तमम् ।।११.४ ।।

---- गृहस्थः(?) ----

विनार्थेन तु यो यज्ञः स यज्ञः सार्वकामिकः ।
अक्षयश्चाव्ययश्चैव सर्वपातकनाशनः ।।११.५ ।।
बहुविघ्नकरो ह्यर्थो बह्वायासकरस्तथा ।
ब्रह्महत्या इवेन्द्रस्य प्रविभागफला स्मृता ।।११.६ ।।
पञ्चशोध्येन शोध्येत अर्थयज्ञो वरानने ।
शोधिते तु फलं शुद्धमशुद्धे निष्फलं भवेत् ।।११.७ ।।
देव्युवाच ।
पञ्चशोध्ये सुरश्रेष्ठ संशयो ऽत्र भवेन्मम ।
कथयस्व विभागेन श्रोतुमिच्छामि तत्त्वतः ।।११.८ ।।
रुद्र उवाच ।
मनःशुद्धिस्तु प्रथमं द्रव्यशुद्धिरतः परम् ।
मन्त्रशुद्धिस्तृतीया तु कर्मशुद्धिरतः परम् ।
पञ्चमी सत्त्वशुद्धिस्तु क्रतुशुद्धिश्च पञ्चधा ।।११.९ ।।
मनःशुद्धिर्नाम अविपरीतभावनया । 
द्रव्यशुद्धिर्नाम अनन्यायोपार्जितद्रव्येन ।। ११.१० ।।
मन्त्रशुद्धिर्नाम स्वरव्यञ्जनयुक्ततया । 
क्रियाशुद्धिर्नाम यथाक्रमाविपरीततया । 
सत्त्वशुद्धिर्नाम रजस्तम ।अप्रधानतया ।। ११.११ ।।
विधिमेवं यदा शुध्येद्यदि यज्ञं करोति हि ।
तस्य यज्ञफलावाप्तिर्जन्ममृत्युश्च नो भवेत् ।।११.१२ ।।
विनार्थेन तु यो यज्ञं करोति वरसुन्दरि ।
न तस्य तत्फलावाप्तिः सर्वयज्ञेष्वशेषतः ।।११.१३ ।।
यज्ञवाट कुरुक्षेत्रं सत्त्वावासकृतालयः ।
प्रत्याहार महावेदिः कुशप्रस्तरसंयमः ।।११.१४ ।।
विधि नियमविस्तारो ध्यानवह्निप्रदीपितः ।
योगेन्धनसमिज्ज्वालतपोधूमसमाकुलः ।।११.१५ ।।
पात्रन्यास शिवज्ञानं स्थालीपाक शिवात्मकः ।
आज्याहुतिमविच्छिन्नं लम्बकश्रुवपातितः ।।११.१६ ।।
धारणाध्वर्युवत्कृत्वा प्राणायामश्च ऋत्विजः ।
तर्कयुक्तः सविस्तारः समाधिर्वयतापनः ।।११.१७ ।।
ब्रह्मविद्यामयो यूपः पशुबन्धो मनोन्मनः ।
श्रद्धा पत्नी विशालाक्षि संकल्पः पद शाश्वतम् ।।११.१८ ।।
पञ्चेन्द्रियजयोत्पन्नः पुरोडाशो ऽमृताशनः ।
ब्रह्मनादो महामन्त्रः प्रायश्चित्तानिलो जयः ।।११.१९ ।।
सोमपान परिज्ञानमुपाकर्म चतुर्यमः ।
इतिहास जलस्नानं पुराणकृत{ ।}म{ ।}म्बरः ।।११.२० ।।
इडासुषुम्नासंवेद्ये स्नानमाचमनं सकृत् ।
संतोषातिथिमादृत्य दयाभूतद्विजार्चितः ।।११.२१ ।।
ब्रह्मकूर्च गुणातीत हविर्गन्ध निरञ्जनः ।
ब्रह्मसूत्रं त्रयस्तत्त्वं बोधना मुण्डितं शिरः ।।११.२२ ।।
निवृत्त्यादि चतुर्वेदश्चतुःप्रकरणासनः ।
दक्षिणामभयं भूते दत्त्वा यज्ञं यजेत्सदा ।।११.२३ ।।
विनार्थं यज्ञसम्प्राप्तिः कथिता ते वरानने ।
आसहस्रस्य यज्ञानां फलं प्राप्नोति नित्यशः ।।११.२४ ।।
आश्रमः प्रथमस्तुभ्यं कथितो ऽस्ति वरानने ।
सदाशिवेन सद्धर्मं दैवतैरपि पूजितम् ।।११.२५ ।।

---- ब्रह्मचर्यम् ----

ब्रह्मचर्यं निबोधेदं शृणुष्वावहिता शुभे ।
द्वितीयमाश्रमं देवि सर्वपापविनाशनम् ।।११.२६ ।।
व्रतं ब्रह्मपरं ध्यानं सावित्री प्रकृतौ लयः ।
ब्रह्मसूत्राक्षरं सूक्ष्मं त्रिगुणालय मेखलम् ।।११.२७ ।।
दम दण्ड दया पात्रं भिक्षा संसारमोचनम् ।
त्र्यायुषं द्व्यक्षरातीतं ज्ञानभष्म ।अलङ्कृतम् ।।११.२८ ।।
स्नानव्रतं सदासत्यं शीलशौचसमन्वितम् ।
अग्निहोत्र त्रयस्तत्त्वं जप ब्रह्मबिलस्वरः ।।११.२९ ।।
द्वितीय आश्रमो देवि यथाह भगवान्शिवः ।
मयापि कथितं तुभ्यं जन्ममृत्युविनाशनम् ।।११.३० ।।

---- वानप्रस्थः ----

वानप्रस्थविधिं वक्ष्ये शृणुष्वायतलोचने ।
यथाश्रुतं यथातथ्यमृषिदैवतपूजितम् ।।११.३१ ।।
वैराग्यवनमाश्रित्य नियमाश्रममाहरेत् ।
शीलशैलदृड्हद्वारे प्राकारे विजितेन्द्रियः ।।११.३२ ।।
अधिभूतः स्मृतो माता अध्यात्मश्च पिता तथा ।
अधिदैविक{ ।}म{ ।}ाचार्यो व्यवसायाश्च भ्रातरः ।।११.३३ ।।
श्रुतिः स्मृतिः स्मृता भार्या प्रज्ञा पुत्रः क्षमानुजः ।
मैत्री बन्धुर्जटा चापं करुणा सुपवित्रकम् ।।११.३४ ।।
मुदिता मौन चत्वारः सर्वकार्यमुपेक्षका ।
यमवल्कलसंवीतस्तपःकृष्णाजिनाधरः ।।११.३५ ।।
 उत्तरासङ्गमासीनो योगपट्टदृड्हव्रतः ।
वेदघोषेण घोषेण प्राणायामो ऽग्निहावनम् ।।११.३६ ।।
जितप्राणमृगाकूलो धृति यज्ञः क्रिया जपः ।
अर्थसंग्रह शास्त्रेषु सखा दमदयादयः ।।११.३७ ।।
शिवयज्ञं प्रयुञ्जीत साधनाष्टकपूजनम् ।
पञ्चब्रह्मजलैः पूतः सत्यतीर्थशिवह्रदे ।।११.३८ ।।
स्नानमाचमनं कृत्वा संध्यात्रयमुपाश्रयेत् ।
अक्षमाला पुराणार्थं जप शान्तं दिवानिशम् ।।११.३९ ।।
ज्ञानसलिलसम्पूर्णमितिहासकमण्डलुः ।
पञ्चकर्मक्रियोत्क्रान्ति जप पञ्चविधः सुखम् ।।११.४० ।।
साधनं शिवसंकल्पो योगसिद्धिफलप्रदः ।
संतोषफलमाहारः कामक्रोधपराजितः ।।११.४१ ।।
आशापाशजयाभ्यासो ध्यानयोगरतिप्रियः ।
अतिथिभ्यो ऽभयं दत्त्वा वानप्रस्थश्चरेद्व्रतम् ।
वानप्रस्थमयं धर्मं यत्पूर्वमवधारितम् ।।११.४२ ।।
! संसारोद्धरणमनित्यहरणमज्ञाननिर्मूलनम् 
! प्रज्ञावृद्धिकरममोघकरणं क्लेशार्णवोत्तारणम् ।
! जन्मव्याधिहरमकर्मदहनं सेवेत्स धर्मोत्तमम
? श्रद्धापूर्वकमेव यः सनियमं साक्षाच्च जीवन्शिवः ।।११.४३ ।।

---- परिव्राजकः ----

परिव्राजकधर्मो ऽयं कीर्तयिष्यामि तच्छृणु ।
सुखदुःखं समं कृत्वा लोभमोहविवर्जितः ।।११.४४ ।।
वर्जयेन्मधु मांसानि परदारांश्च वर्जयेत् ।
वर्जयेच्चिरवासं च परवासं च वर्जयेत् ।।११.४५ ।।
वर्जयेत्सृष्टभोज्यानि भिक्षामेकां च वर्जयेत् ।
वर्जयेत्संग्रहं नित्यमभिमानं च वर्जयेत् ।।११.४६ ।।
सुसूक्ष्मं मनसा ध्यात्वा शुचौ पादं विनिक्षिपेत् ।
न कुप्येत अनालाभे लाभे वापि न हर्षयेत् ।।११.४७ ।।
अर्थतृष्णास्वनुद्विग्नो रोषे वापि सुदारुणे ।
स्तुतिनिन्दा समं कृत्वा प्रियं वाप्रियमेव वा ।।११.४८ ।।
नियमास्तु परीधानं संयमावृतमेखलः ।
निरालम्बं मनः कृत्वा बुद्धिं कृत्वा निरञ्जनाम् ।।११.४९ ।।
आत्मानं पृथिवीं कृत्वा खं च कृत्वा मनोन्मनम् ।
त्रिदण्डं त्रिगुणं कृत्वा पात्रं कृत्वाक्षरो ऽव्ययः ।।११.५० ।।
न्यसेद्धर्ममधर्मं च ईर्ष्याद्वेषं परित्यजेत् ।
निर्द्वन्द्वो नित्यसत्यस्थो निर्ममो निरहंकृतः ।।११.५१ ।।
दिवसस्याष्टमे भागे भिक्षां सप्तगृहं चरेत् ।
न चासीत न तिष्ठेत न च देहीति वा वदेत् ।।११.५२ ।।
यथालाभेन वर्तेत अष्टौ पिण्डान्दिने दिने ।
वस्त्रभोजनशय्यासु न प्रसज्येत विस्तरम् ।।११.५३ ।।
नाभिनन्देत मरणं नाभिनन्देत जीवितम् ।
इन्द्रियाणि वशंकृत्वा कामं हत्वा यतव्रतः ।।११.५४ ।।
अतीतं च भविष्यं च न भिक्षुश्चिन्तयेत्सदा ।
! क्रोधमानमददर्पान्परिव्राड्वर्जयेत्सदा ।।११.५५ ।।
विरागं तु धनुः कृत्वा प्राणायामगुणैर्युतम् ।
धारणाशरतीक्ष्णेन मृगं हत्वा मनेन्द्रियम् ।।११.५६ ।।
मैत्रीखड्गसुतीक्ष्णेन संसारारिं निकृन्तयेत् ।
करुणावर्तचक्रेण क्रोधमत्तगजं जयेत् ।।११.५७ ।।
मुदितावर्मबद्धाङ्गस्तूणं पूर्णमुपेक्षया ।
अनक्षरं परं ब्रह्म चिन्तयेत्सततं द्विज ।।११.५८ ।।
ब्रह्मणो हृदयं विष्णुर्विष्णोश्च हृदयं शिवः ।
शिवस्य हृदयं संध्या तस्मात्संध्यामुपासयेत् ।।११.५९ ।।
संसारार्णवतारणं शुभगतिः स ब्रह्म संध्याक्षरं
ध्यायेन्नित्यमतन्द्रितो ह्यनुपमं व्यक्तात्मवेद्यं शिवम् ।
रूपैर्वर्णगुणादिभिश्च विहितं दुर्लक्ष्यलक्ष्योत्तमं
यत्नोद्धृत्य समाश्रयेत्सुरगुरुं सर्वार्तिहर्ता हरम् ।।११.६० ।।

 ।।इति वृषसारसंग्रहे चतुराश्रमधर्मविधानो नामाध्याय एकादशमः ।।





---- आतिथ्यधर्मः ---- 

देव्युवाच ।
अहिंसा परमो धर्मः सततं परिकीर्त्यते ।
आतिथ्यकानां धर्मं च कथयस्व यदुत्तमम् ।।१२.१ ।।
महेश्वर उवाच ।
अहिंसातिथ्यकानां च शृणु धर्मं यदुत्तमम् ।
त्रैलोक्यमखिलं देवि रत्नपूर्णं सुलोचने ।।१२.२ ।।
चतुर्वेदविदे दानं न तत्तुल्यमहिंसकः ।
शृणु धर्ममतिथ्यानां कीर्तयिष्यामि सुन्दरि ।।१२.३ ।।

---- विपुलोपाख्यानम् ----

आसीद्वृत्तं पुराख्यानं नगरे कुसुमाह्वये ।
कपिलस्य सुतो विद्वान्विपुलो नाम विश्रुतः ।।१२.४ ।।
धर्मनित्यो जितक्रोधः सत्यवादी जितेन्द्रियः ।
ब्रह्मण्यश्च कृतज्ञश्च मद्भक्तः कृतनिश्चयः ।।१२.५ ।।
धनाड्ह्यो ऽतिथिपूज्यश्च दाता दान्तो दयालुकः ।
न्यायार्जितधनो नित्यमन्यायपरिवर्जितः ।।१२.६ ।।
भार्या च रूपिणी तस्य चन्द्रबिम्बशुभानना ।
पीनोत्तुङ्गस्तनी कान्ता सकलानन्दकारिणी ।
पतिव्रता पतिरता पतिशुश्रूषणे रता ।।१२.७ ।।
अथ केनापि कालेन सूर्यरागमभूत्ततः ।
ग्रस्तभागत्रयस्त्वासीत्कृष्णमाधवमासिके ।।१२.८ ।।
स्नातुकामावतीर्यन्ते सर्वे पौरनृपादयः ।
देवाश्च पितरश्चैव तर्प्यन्ते विधिवत्तथा ।।१२.९ ।।
केचिज्जुह्वति तत्राग्निं केचिद्विप्रांश्च तर्पयेत् ।
केचिद्दानोपतिष्ठन्ति केचित्स्तुन्वन्ति देवताम् ।।१२.१० ।।
ध्यानयोगरताः केचित्केचित्पञ्चतपे रताः ।
एवं प्रवर्तमानेषु राजनादिषु सर्वशः ।।१२.११ ।।
विपुलो ऽपि च तत्रैव गङ्गागण्डकिसंगमे ।
भार्यया सह तत्रैव स्नात्वा क्षोमविभूषणः ।।१२.१२ ।।
देवतागुरुविप्राणामन्येषां तर्पणे रतः ।
तत्रावसरसम्प्राप्तो ब्राह्मणो ऽतिथिरागतः ।।१२.१३ ।।
भार्या तस्यातिरूपेण मोहिता ब्रह्मणस्तदा ।
ब्राह्मणो ऽपि तथैवेह रूपेणाप्रतिमो भवेत् ।।१२.१४ ।।
अन्योन्यदृष्टिसंसक्तौ जातौ तौ तु परस्परम् ।
विपुलेनाञ्जलिं कृत्वा ब्राह्मण संशितव्रत ।।१२.१५ ।।
आज्ञापय द्विजश्रेष्ठ अद्य मे ऽनुग्रहं कुरु ।
भार्याभृत्यपशुग्राम रत्नानि विविधानि च ।।१२.१६ ।।
विपुलेनैवमुक्तस्तु गृहीतो ब्राह्मणो ऽब्रवीत् ।
यदि सत्यं प्रदातासि सुप्रसन्नं मनस्तव ।।१२.१७ ।।
विपुल उवाच ।
सुप्रसन्नं मनो मे ऽद्य सुप्रसन्नं तपःफलम् ।
शीघ्रमाज्ञापय विप्र यच्चाभिलषितं तव ।
अदेयं नास्ति विप्रस्य स्वशिरःप्रभृति द्विज ।।१२.१८ ।।
ब्राह्मण उवाच ।
यद्येवं वदसे भद्र भार्यां मे देहि रूपिणीम् ।
स्वस्ति भवतु भद्रं वः कल्याणं भव शाश्वतम् ।।१२.१९ ।।
विपुल उवाच ।
प्रतीच्छ भार्यां सुश्रोणीं रूपयौवनशालिनीम् ।
अकुत्सितां विशालाक्षीं पूर्णचन्द्रनिभाननाम् ।।१२.२० ।।
भार्योवाच ।
परित्याज्या कथं नाथ अपापां त्यजसे कथम् ।
अतीव हि प्रियां भार्यां निर्दोषां स कथं त्यजेः ।।१२.२१ ।।
सखा भार्या मनुष्याणामिह लोके परत्र च ।
दानं वा सुमहद्दत्त्वा यज्ञो वा सुबहुः कृतः ।।१२.२२ ।।
अपुत्रो नाप्नुयात्स्वर्गं तपोभिर्वा सुदुष्करैः ।
श्रुतो मे पितृभिः प्रोक्तो ब्राह्मणैश्च ममान्तिके ।।१२.२३ ।।
अपुत्रो नाप्नुयात्स्वर्गं श्रुतं मे बहुशः पुरा ।
मन्दपालो द्विजश्रेष्ठो गतः स्वर्गं तपोबलात् ।।१२.२४ ।।
दानानि च बहून्दत्त्वा यज्ञांश्च विविधांस्तथा ।
वेदांश्च जप यज्ञांश्च कृत्वा तद्द्विजसत्तमः ।।१२.२५ ।।
प्राप्तद्वारो ऽपि यस्यापि देवदूतैर्निवारितः ।
अपुत्रो नाप्नुयात्स्वर्गं यदि यज्ञशतैरपि ।।१२.२६ ।।
इत्युक्तस्तु च्युतः स्वर्गान्मन्दपालो महानृषिः ।
पुत्रानुत्पादयामास शारङ्गाश्चतुरो द्विजः ।।१२.२७ ।।
तेन पुण्यप्रभावेण स्वर्गं प्राप्तो ह्यवारितः ।
कुलत्राणां कलत्रास्मि भरणाद्भार्य एव च ।।१२.२८ ।।
दारसंग्रह पुत्रार्थे क्रियते शास्त्रदर्शनात् ।
यानि सन्ति गृहे द्रव्यं ग्रामघोषगृहाणि च ।।१२.२९ ।।
दातुमर्हसि विप्राय न मां दातुमिहार्हसि ।
भार्याया वचनं श्रुत्वा विपुलः पुनरब्रवीत् ।।१२.३० ।।
विपुल उवाच ।
साधु भामिनि जानामि साधु साधु पतिव्रते ।
जितो ऽस्म्यनेन वाक्येन अनेनास्मि हि तोषितः ।।१२.३१ ।।
अद्य ग्रहणकाले च द्विज आगत्य याचते ।
ददामीति प्रतिज्ञाय अदत्त्वा नरकं व्रजे ।।१२.३२ ।।
नरकं यदि गच्छामि कुलेन सह सुन्दरि ।
कल्पकोटिसहस्रे ऽपि नरकस्थाद्यशस्विनि ।।१२.३३ ।।
मुक्तिमेव न पश्यामि जन्मकोटिशतैरपि ।
अदानाच्चाशुभं देवि पश्यामि वरवर्णिनि ।।१२.३४ ।।
दानेन तु शुभं पश्ये स्वर्गलोके यदक्षयम् ।
नोक्तं मयानृतं पूर्वं नित्यं सत्यव्रते स्थितः ।।१२.३५ ।।
सत्यधर्ममतिक्रम्य नान्यधर्मं समाचरे ।
भार्या धर्मसखेत्येवं त्वयि पूर्वमुदाहृतम् ।।१२.३६ ।।
यदि धर्मसखायासि सो ऽद्य काल इहागतः ।
द्विजरूपधरो धर्मः स्वयमेव इहागतः ।।१२.३७ ।।
जिज्ञासार्थमहं भद्रे न विघ्नं कर्तुमर्हसि ।
माताव्यक्तः पिता ब्रह्मा बुद्धिर्भार्या दमः सखा ।।१२.३८ ।।
पुत्रो धर्मः क्रियाचार्य इत्येते मम बान्धवाः ।
कालश्रेष्ठो ग्रहः सूर्यो गङ्गा श्रेष्ठा नदीषु च ।।१२.३९ ।।
चन्द्रक्षये दिनं श्रेष्ठं नरश्रेष्ठो द्विजोत्तमः ।
शुश्रूषणार्थं विप्रस्य मया दत्तासि सुन्दरि ।
सर्वस्वं ब्राह्मणे दत्त्वा वनमेवाश्रयाम्यहम् ।।१२.४० ।।
शङ्कर उवाच ।
तूष्णीम्भूता ततो भार्या अश्रुपूर्णाकुलेक्षणा ।
करे गृह्य विशालाक्षी ब्राह्मणाय निवेदिता ।।१२.४१ ।।
यानि सन्ति गृहे द्रव्यं हिरण्यं पशवस्तथा ।
ददामि ते द्विजश्रेष्ठ ग्रामघोषगृहादिकम् ।।१२.४२ ।।
मुक्ता वैडूर्यवासांसि दिव्याण्याभरणानि च ।
सर्वान्गृहाण विप्रेन्द्र श्रद्धया दत्तसत्कृताम् ।।१२.४३ ।।
प्रीयतां भगवान्धर्मः प्रीयतां च महेश्वरः ।
प्रीयन्तां पितरः सर्वे यद्यस्ति सुकृतं फलम् ।।१२.४४ ।।
रुद्र उवाच ।
विपुलस्य वचः श्रुत्वा ब्राह्मणेन तपस्विना ।
आशीः सुविपुलं दत्त्वा विपुलाय महात्मने ।।१२.४५ ।।
वसेत्तत्र गृहे रम्ये भार्यामादाय तस्य च ।
विपुलस्तु नमस्कृत्वा कृत्वा चापि प्रदक्षिणम् ।।१२.४६ ।।
ब्राह्मणमभिवाद्यैवं गतः शीघ्रं वनान्तरम् ।
वने मूलफलाहारो विचरेत महीतले ।।१२.४७ ।।
एकाकी विजने शून्ये चिन्तया च परिप्लुतः ।
क्व गच्छामि क्व भोक्ष्यामि कुत्र वा किं करोम्यहम् ।।१२.४८ ।।
न पथं विषयं वेद्मि ग्रामं वा नगराणि वा ।
खेटखर्वटदेशं वा जानामीह न कंचन ।।१२.४९ ।।
अमुं सुशैलं पश्यामि विपुलोदरकन्दरम् ।
तमारुह्य निरीक्ष्यामि ग्रामं नगरपत्तनम् ।।१२.५० ।।
एवमुक्त्वा तु विपुलः शनैः पर्वतमारुहत् ।
वृक्षच्छायां समालोक्य निषसाद श्रमान्वितः ।।१२.५१ ।।
एतस्मिन्नेव काले तु वृक्षशाखावतार्य च ।
अपूर्वं च सुरूपं च सुगन्धत्वं च शोभनम् ।।१२.५२ ।।
फलं गृह्य विचित्रं च हृदयानन्दनं शुभम् ।
विपुलस्याग्रतः कृत्वा पुनर्वृक्षं समारुहत् ।।१२.५३ ।।
विपुलश्चित्रवद्दृष्ट्वा विस्मयं परमं गतः ।
अहो वा स्वप्नभूतो ऽस्मि अहो वा तपसः फलम् ।।१२.५४ ।।
न पश्यामि न जिघ्रामि न च स्वादं च वेद्म्यहम् ।
वार्तापि न च मे श्रोत्रा प्रतिजानामि कंचन ।।१२.५५ ।।
एवमुक्त्वा ह्यनेकानि फलं गृह्य मनोरमम् ।
सुनिरीक्ष्य पुनर्जिघ्रं पुनर्जिघ्रं निरीक्ष्य च ।।१२.५६ ।।
फलं चात्र निरूप्यन्तो देशं वाप्यवलोकयन् ।
पाथेयरहितश्चास्मि देवदत्तं फलं मम ।।१२.५७ ।।
तत्फलं प्रतिगृह्यैव नगरं प्रविशाम्यहम् ।
प्रार्थयित्वा च यत्किंचिज्जीवनार्थं चराम्यहम् ।।१२.५८ ।।
ततः शैलमतिक्रम्य नगरं प्रविवेश ह ।
पथि कश्चिज्जनः पृष्ठः किंनाम नगरं त्विदम् ।।१२.५९ ।।
स होवाच पथी केन किमपूर्वमिहागतः ।
दक्षिणापथदेशो ऽयं नरवीरपुरं त्वदः ।।१२.६० ।।
राजा सिंहजटो नाम राज्ञी तस्य च केकयी ।
अतिवृद्धो जराग्रस्तः केकयी च तथैव च ।।१२.६१ ।।
दाता सर्वकलाज्ञश्च युद्धे वीर्यबलान्वितः ।
ब्रह्मण्यो वत्सलो लोके सर्वशास्त्रविशारदः ।।१२.६२ ।।
विपुल उवाच ।
अत्र श्रेष्ठिमुपास्यामि नाम वा तस्य किं वद ।
कतमो देशस्तद्वासः कथयस्व न संशयः ।।१२.६३ ।।
विपुलेनैवमुक्तस्तु पथिकोवाच तं पुनः ।
मम भीमबलो नाम श्रेष्ठिकस्य गृहागतः ।।१२.६४ ।।
श्रेष्ठिकः पुण्डको नाम ख्यातः श्रेष्ठिक उच्यते ।
कौतुकं तव यद्यस्ति तदागच्छ मया सह ।।१२.६५ ।।
एवमस्त्विति तेनोक्तो विपुलेन महात्मना ।
तेनैव सह निर्यातः श्रेष्ठिकस्य गृहं प्रति ।।१२.६६ ।।
श्रेष्ठिकः स्वगृहासीनो दृष्टः स विपुलेन तु ।
तस्यान्तिकमुपागम्य तत्फलं स निवेदितः ।।१२.६७ ।।
अहो फलमिदं श्रेष्ठमहो फलमिहानितम् ।
अहो रूपमहो गन्धमहो फलं सुशोभनम् ।।१२.६८ ।।
तत्फलं न महीजातं न मेरौ न च कन्दरे ।
देवलोकिक सुव्यक्तं न मर्त्य उपजायते ।।१२.६९ ।।
अहो ऽस्मि सफलं भोक्ता राजार्हश्च न संशयः ।
ड्हौकयित्वा फलं दिव्यं राजानं तोषयाम्यहम् ।।१२.७० ।।
ततस्त्वरित गत्वैव फलं गृह्य मनोहरम् ।
आदरेणोपसृत्यैव राजानं स फलं ददौ ।।१२.७१ ।।
राजा च स फलं दृष्ट्वा विस्मयं परमं गतः ।
कुतः श्रेष्ठि त्वया नीतं फलं सर्वमनोहरम् ।।१२.७२ ।।
स्वादुमूलफलकन्दं दृष्ट्वा पूर्वं न तादृशम् ।
रूपगन्धगुणोपेतं हृदयानन्दकारकम् ।।१२.७३ ।।
सद्य एवोपभुञ्जामि त्वया दत्तमिदं फलम् ।
कीदृशं स्वाद विज्ञातुमिच्छामि कुरु माचिरम् ।।१२.७४ ।।
ततः स भक्षयामास फलं चामृतसंनिभम् ।
अमृतोपमसुस्वादं सर्वं च बुभुजे नृपः ।।१२.७५ ।।
सद्य षोडशवर्षस्य यौवनं समपद्यत ।
न वलीपलितं सद्यो न जरा न च दुर्बलः ।।१२.७६ ।।
केशदन्तनखस्निग्धो दृड्हदन्तो दृड्हेन्द्रियः ।
तेजश्चक्षुर्बलप्राणान्सद्य सर्वानवाप्तवान् ।।१२.७७ ।।
मन्त्री पुरोहितामात्य सर्वे भृत्यजनास्तथा ।
पौरस्त्री बालवृद्धाश्च सर्वे ते विस्मयं गताः ।।१२.७८ ।।
राजा सिंहजटो नाम तुष्टिमेव परां गतः ।
प्रहर्षमतुलं चैव प्राप्तवान्स नरेश्वरः ।।१२.७९ ।।
उवाच राजा तं श्रेष्ठिं स्वार्थतत्परनिर्दयः ।
कुरु भीमबलस्त्वेवं फलमानय अद्य वै ।।१२.८० ।।
पुनर्मे यौवनप्राप्तिस्त्वत्प्रसादान्नरोत्तम ।
केकयीं दुर्बलां वृद्धां पुनः प्रापय यौवनम् ।।१२.८१ ।।
स राज्ञा एवमुक्तस्तु श्रेष्ठी भीमबलस्तथा ।
प्रत्युवाच ह राजानं प्राञ्जलिः प्रणतः स्थितः ।।१२.८२ ।।
न फलेदं वने राजन्न वाणिज्यकृषेण वा ।
केनापि कुलपुत्रेण तव दर्शनकांक्षया ।।१२.८३ ।।
दत्तो ऽस्मि तव राजेन्द्र मया दत्तो ऽसि भूपते ।
न ते शक्नोम्यहं राजन्वक्तुं वैदेशिनं नरम् ।।१२.८४ ।।
श्रुत्वा भीमबलं वाक्यं प्रत्युवाच ततः पुनः ।
अमात्यकुलपुत्रस्त्वं ब्रूहि मद्वचनं पुनः ।।१२.८५ ।।
यदि नास्ति किमेतत्तं मया वा प्रार्थितो भवान् ।
यत्र ह्येको बहवो ऽत्र जायन्ते नात्र संशयः ।।१२.८६ ।।
आगमोपायमार्गं च तेनैव स तु गम्यताम् ।
अवश्यं तेन गन्तव्यं तेन मार्गेण मार्गय ।।१२.८७ ।।
अदत्त्वा फलमन्यच्च शिरश्छेद्यामि दुर्मतेः ।
छेद्य चण्डविचण्डाभ्यां रक्षभीमबलाधमः ।।१२.८८ ।।
ततो भीमबलः क्रुद्धः खड्गं गृह्य शशिप्रभम् ।
अलङ्घ्य वचनं राज्ञः कुलपुत्र व्रज त्वरम् ।।१२.८९ ।।
मा रुष कुलपुत्र त्वं मया वध्यो भविष्यसि ।
यद्यस्ति फलमन्यद्वा देहि राजानमद्य वै ।।१२.९० ।।
यत्र प्राप्तं फलं दिव्यं तत्र वा देशय तव ।
तत्फलेन विना भद्र दुर्लभं तव जीवितम् ।।१२.९१ ।।
विपुल उवाच ।
जीविताशामहं प्राप्तो वैदेशि भवनं तव ।
कृतकर्ता कथं वध्यः प्राप्नुयामहमद्य वै ।।१२.९२ ।।
फलं वा न पुनस्त्वन्यद्दातुं शक्यं न केनचित् ।
सह्य पर्वतशैलाग्रे आशीनः श्रान्तमानसः ।।१२.९३ ।।
वानरस्तत्फलं गृह्य मम दत्त्वा पुनर्गतः ।
मया दत्तमिदं तुभ्यं त्वयापि च नराधिपे ।।१२.९४ ।।
तत्र गच्छाव भो श्रेष्ठि दृश्यते यदि वानरः ।
त्वया मया च गत्वैव यो वासः प्लवगाधिपः ।।१२.९५ ।।
श्रेष्ठिना च तथेत्याह गच्छामः सहिता वयम् ।
यत्र प्राप्तं फलं तुभ्यं मोक्षयामो न संशयः ।।१२.९६ ।।
रुद्र उवाच ।
तमारुह्य गिरिं सह्यं मार्गमाणः समन्ततः ।
विपुलेन ततो दृष्टो वानरः प्लवगाधिपः ।।१२.९७ ।।
अयं स वानरश्रेष्ठो वृक्षच्छायासमाश्रितः ।
मम पुण्यबलेनैव दृश्यते ऽद्यापि वानरः ।।१२.९८ ।।
वानर कुरु मित्रार्थं सद्योमृत्युर्भवेन्मम ।
पूर्वदत्तं फलमन्यद्देहि वानर जीवय ।।१२.९९ ।।
वानर उवाच ।
गन्धर्वेण मम दत्तं फलं दत्तं तु ते मया ।
पुनरन्यत्कथं दास्ये तत्र गच्छ यदीच्छसि ।।१२.१०० ।।
विपुल उवाच ।
अदत्त्वा तत्फलं तुभ्यं जीवितुं संशयो भवेत् ।
अथवा तत्र गच्छामो यत्र चित्ररथः स्वयम् ।।१२.१०१ ।।
वानरः पुनरेवाह एवं कुर्वामहे वयम् ।
ततश्चित्ररथावासमुपगम्येदमब्रवीत् ।।१२.१०२ ।।
गन्धर्वराज कार्यार्थी त्वं ह्यहं पुनरागतः ।
पूर्वदत्तफलं त्वन्यद्देहि मां यदि शक्यते ।। १२.१०३ ।।
गन्धर्वराजोवाच ।
सूर्यलोकगतश्चास्मि तेन दत्तं फलोत्तमम् ।
मया दत्तं फलं तुभ्यमत्यन्तसुहृदो ऽसि मे ।।१२.१०४ ।।
कुतो ऽन्यत्फलमादास्ये मम नास्ति प्लवङ्गम ।
सूर्यलोकं गमिष्यामस्तत्र याचस्व भास्करम् ।।१२.१०५ ।।
गन्धर्वेनैवमुक्तस्तु तथेत्याह प्लवङ्गमः ।
सूर्यलोकं ततः प्राप्ता गन्धर्वादय सर्वशः ।।१२.१०६ ।।
गन्धर्व उवाच ।
कार्यार्थेन पुनः प्राप्तस्त्वत्सकाशं खगेश्वर ।
पूर्वदत्तफलं त्वन्यद्देहि जीवमनाशय ।।१२.१०७ ।।
सूर्य उवाच ।
सोमलोकगतश्चास्मि तेन दत्तं फलोत्तमम् ।
सफलं दत्तमेवासि सुहृदत्वान्मया तव ।।१२.१०८ ।।
अन्यद्दातुं न शक्नोमि गच्छ सोमपुराद्य वै ।
तं प्रार्थयाविकल्पेन अत्रिपुत्रं ग्रहेश्वरम् ।।१२.१०९ ।।
रुद्र उवाच ।
गतः सूर्याग्रतः कृत्वा सोमलोकं तथैव हि ।
उवाच सूर्यः सोमाय कारणापेक्षया शशिम् ।।१२.११० ।।
सोम उवाच ।
किमर्थमागतो भूयः कर्तव्यं तत्र भास्कर ।
फलं दातुं पुनस्त्वन्यन्मुक्त्वा त्वन्यत्करोम्यहम् ।।१२.१११ ।।
सूर्य उवाच ।
यदि शक्यं फलं देहि अन्यन्न प्रार्थयाम्यहम् ।
न दत्तासि फलमन्यन्मया वद्ध्यो भविष्यसि ।।१२.११२ ।।
सोम उवाच ।
आगमं तस्य वक्ष्यामि शृणुष्वावहितो भव ।
इन्द्रेणास्मि फलं दत्तं सफलं दत्त मे भवान् ।।१२.११३ ।।
गत्वैवेन्द्रसदस्त्वन्यत्प्रार्थयामः सहैव तु ।
एवं कुर्म इति प्राह गत्वेन्द्रसदनं प्रति ।।१२.११४ ।।
सोमेनेन्द्रमुवाचेदं फलकामा इहागताः ।
पूर्वदत्तफलमन्यद्देहि शक्र ममाद्य वै ।।१२.११५ ।।
इन्द्र उवाच ।
यदर्थमिह सम्प्राप्तः स च नास्ति निशाकर ।
विष्णुहस्तान्मया प्राप्तमेकमेव फलं शुभम् ।।१२.११६ ।।
सर्व एव हि गच्छामो विष्णुलोकं ग्रहेश्वर ।
सर्व एवोपजग्मुस्ते फलार्थं मधुसूदनम् ।।१२.११७ ।।
एवमुक्त्वा गताः सर्वे देवराजपुरस्कृताः ।
मुहूर्तेनैव सम्प्राप्ता विष्णुलोकं यशस्विनि ।।१२.११८ ।।
उपसृत्य तत इन्द्रः प्रणिपत्य जनार्दनम् ।
सर्वेषामुपरोधेन प्रार्थयामि यशोधर ।।१२.११९ ।।
विष्णुरुवाच ।
पूर्वदत्तफलस्यार्थे तच्च सर्वमिहागताः ।
न शक्नोमि फलं दातुं किं वा त्वन्यत्करोम्यहम् ।।१२.१२० ।।
इन्द्र उवाच ।
ब्रह्माण्डमपि भेत्तुं त्वं शक्नोषि गरुडध्वज ।
अशक्यं तव नास्तीति जानामि पुरुषोत्तम ।।१२.१२१ ।।
एवमुक्त्वा पुनर्विष्णुः प्रत्युवाच पुरन्दरम् ।
फलमेकं परित्यज्य सर्वं शक्नोमि कौशिक ।।१२.१२२ ।।
उपायो ऽत्र प्रवक्ष्यामि आगमं शृणु गोपते ।
ब्रह्मणा च मम दत्तं तत्फलैकं पुरन्दर ।।१२.१२३ ।।
मया दत्तफलं त्वेकं किमन्यद्दातुमिच्छसि ।
प्रार्थयामो ऽत्र गत्वैकं परमेष्ठिप्रजापतिम् ।। १२.१२४ ।।
तवोपराधाद्देवेन्द्र प्रार्थयामि पितामहम् ।
एवमुक्त्वा गताः सर्वे पुरस्कृत्य जनार्दनम् ।।१२.१२५ ।।
इन्द्रः सोमश्च सूर्यश्च गन्धर्वो वानरस्तथा ।
विपुलः श्रेष्ठिकश्चैव राजदूतद्वयं तथा ।।१२.१२६ ।।
ब्रह्मलोकं मुहूर्तेन प्राप्तवान्सुरसुन्दरि ।
दृष्ट्वा ब्रह्मसदो रम्यं सर्वकामपरिच्छदम् ।।१२.१२७ ।।
अनेकानि विचित्राणि रत्नानि विविधानि च ।
मन्दारतरुशोभानि वैदूर्यमणिकुट्टिमम् ।।१२.१२८ ।।
प्रवालमणिस्तम्भानि वज्रकाञ्चनवेदिकाम् ।
प्रवालस्फाटिको जाल इन्द्रनीलगवाक्षकः ।।१२.१२९ ।।
दृश्यते विपुलस्तत्र नानावृक्ष मनोरमाः ।
पुष्पानामितवृक्षाग्राः फलानामितका भवेत् ।।१२.१३० ।।
सर्वे रत्नमया वृक्षाः सर्वे रत्नमयं जलम् ।
वृक्षगुल्मलतावल्ली कन्दमूलफलानि च ।।१२.१३१ ।।
सर्वे रत्नमया दृष्टा विपुलो विपुलेक्षणः ।
अनेकभौमं प्रासादं मुक्तादामविभूषितम् ।।१२.१३२ ।।
अप्सरोगणकोटीभिः सर्वाभरणभूषितम् ।
विमानकोटिकोटीशं सर्वकामसमन्वितम् ।।१२.१३३ ।।
ब्रह्मलोकसभा रम्या सूर्यकोटिसमप्रभा ।
तत्र ब्रह्मा सुखासीनो नानारत्नोपशोभिते ।।१२.१३४ ।।
चतुर्मूर्तिश्चतुर्वक्त्रश्चतुर्बाहुश{}्चतुर्भुजः ।
चतुर्वेदधरो देवश्चतुराश्रमनायकः ।।१२.१३५ ।।
चतुर्वेदावृतस्तत्र मूर्तिमन्तमुपासते ।
गायत्री वेदमाता च सावित्री च सुरूपिणी ।।१२.१३६ ।।
व्याहृतिः प्रणवश्चैव मूर्तिमान्समुपासते ।
वौषट्कारो वषट्कारो नमस्कारः स मूर्तिमान् ।।१२.१३७ ।।
श्रुतिः स्मृतिश्च नीतिश्च धर्मशास्त्रं समूर्तिमान् ।
इतिहासः पुराणं च सांख्ययोगः पतञ्जलम् ।।१२.१३८ ।।
आयुर्वेदो धनुर्वेदो वेदो गान्धर्व{ ।}म{ ।}ेव च ।
अर्थवेदो ऽन्यवेदाश्च मूर्तिमान्समुपासिते ।।१२.१३९ ।।
ततो ब्रह्मा समुत्थाय अभिगम्य जनार्दिनम् ।
गां च अर्घं च दत्त्वैवमास्यतामिति चाब्रवीत् ।।१२.१४० ।।
मणिरत्नमये दिव्ये आसने गरुडध्वजः ।
देवराजो रविः सोमो गन्धर्वः प्लवगेश्वरः ।।१२.१४१ ।।
विपुलश्च महासत्त्व आस्यतां रत्न ।आसने ।
साधु भो विपुलश्रेष्ठ साधु भो विपुलं तपः ।।१२.१४२ ।।
साधु भो विपुलप्राज्ञ साधु भो विपुलश्रिय ।
तोषिताः स्म वयं सर्वे ब्रह्मविष्णुमहेश्वराः ।।१२.१४३ ।।
आदित्या वसवो रुद्रा साध्याश्विनौ मरुत्तथा ।
भुङ्क्ष्व भोगान्यथोत्साहं मम लोके यथासुखम् ।।१२.१४४ ।।
इयं विमानकोटीणां तवार्थायोपकल्पिता ।
सहस्राणां सहस्राणि अप्सरा कामरूपिणी ।।१२.१४५ ।।
तवार्थीयोपसर्पन्ति सर्वालंकारभूषिताः ।
यावत्कल्पसहस्राणि परार्धानि तपोधन ।
यत्र यत्र प्रयासित्वं तत्र तत्रोपभुज्यताम् ।।१२.१४६ ।।
महेश्वर उवाच ।
इति श्रुत्वा वचस्तस्य विपुलो विपुलेक्षणः ।
वेपमानो भयत्रस्त अश्रुपूर्णाकुलेक्षणः ।।१२.१४७ ।।
प्रणम्य शिरसा भूमौ प्रणिपत्य पुनः पुनः ।
उवाच मधुरं वाक्यं ब्रह्मलोके पितामहम् ।।१२.१४८ ।।
विपुल उवाच ।
भगवन्सर्वलोकेश सर्वलोकपितामह ।
स्वप्नभूतमिवाश्चर्यं पश्यामि त्रिदशेश्वर ।।१२.१४९ ।।
स्मृतिभ्रंशश्च मे जातो बुद्धिर्जातान्धचेतना ।
मूड्हो ऽहं त्वां कथं स्तौमि ज्ञानातीतं परात्परम् ।।१२.१५० ।।
तुभ्यं त्रैलोक्यबन्धो भव मम शरणं त्राहि संसारघोरम
भीतो ऽहं गर्भवासाज्जरमरणभयात्त्राहि मां मोहबन्धात् ।
! नित्यं रागाधिवासमनियतवपुषं त्राहि मां कालपाशात
तिर्यं चान्योन्यभक्षं बहुयुगशतशस्त्राहि मोहान्धकारात् ।।१२.१५१ ।।
श्रुत्वैवोवाच ब्रह्मा विपुलमति पुनर्मानयित्वा यथावत
! आहूत सम्प्लवन्ते भविष्यसि तव मे जन्मलोभो न भूयः ।
गर्भावासन्नचत्वन्न च पुनमरणं क्लेशमायासपूर्णम
छित्त्वा मोहान्धशत्रुं व्रजसि च परमं ब्रह्मभूयत्वमेसि ।।१२.१५२ ।।
महेश्वर उवाच ।
ब्रह्मणा एवमुक्तस्तु विष्णुना प्रभविष्णुना ।
एवं भवतु भद्रं वो यथोवाच पितामहः ।।१२.१५३ ।।
इन्द्रेण रविणा चैव सोमेन च पुनः पुनः ।
साध्यादित्यैर्मरुद्रुद्रैर्विश्वेभिर्वसवैस्तथा ।।१२.१५४ ।।
अहो तपः फलं दिव्यं विपुलस्य महात्मनः ।
स्वशरीरं दिवं प्राप्तं श्रद्धया तिथिपूजया ।।१२.१५५ ।।
एवमादीन्यनेकानि विपुले परिकीर्तितम् ।
ब्रह्माणं पुनरेवाह विष्णुर्विश्वजगत्प्रभुः ।।१२.१५६ ।।

 ।।इति वृषसारसंग्रहे विपुलोपाख्यानो नामाध्यायो द्वादशमः ।।




देव्युवाच ।
अहिंसातिथ्यकानां च श्रुतो धर्मः सुविस्तरः ।
किं न कुर्वन्ति मनुजाः सुखोपायं महत्फलम् ।।१३.१ ।।
स्वशरीरस्थितो यज्ञः स्वशरीरे स्थितं तपः ।
स्वशरीरे स्थितं तीर्थं श्रुतो विस्तरतो मया ।।१३.२ ।।
किमर्थं भगवन्ब्रूहि सुखोपायं महत्फलम् ।
किं निवृत्तास्तु देवेश ऋषिदैवतमानुषाः ।।१३.३ ।।
महादेव उवाच ।
अद्य पृष्टेन कथितं गोपितं ऋषि सुन्दरि ।
मानुषाणां हितार्थाय तव च वरवर्णिनि ।।१३.४ ।।
अद्यप्रभृति देवेशि ख्यातिर्लोके भविष्यति ।
धन्या एवं चरिष्यन्ति अधन्या न रमन्ति तम् ।।१३.५ ।।
त्रिगुणेन तु बन्धेन बद्धा पाशदृड्हेन तु ।
तेनार्थेन रमन्त्यत्र जानन्तो ऽपि विमोहिताः ।।१३.६ ।।
देव्युवाच ।
किं वा त्रिगुणबन्धेति ब्रूहि संशयछेदक ।
अद्यापि मम देवेश मोहोत्पन्नस्त्रिबन्धनैः ।।१३.७ ।।
भगवानुवाच ।
प्राकृतं वैकृतं चैव दक्षिणाबन्धमेव च ।
एतेनैव तु बन्धेन बद्धाः वर्णाश्रमाः सदा ।।१३.८ ।।
ज्ञानहीना निवर्तन्ते परमं प्राप्य तत्परम् ।
इष्टस्त्रीणा निवर्तन्ते धनधान्यसमुच्चये ।
स्नेहादाकृष्य मनसां बन्धः प्राकृत उच्यते ।।१३.९ ।।
योगयुक्तेन मनसा यद्यदैश्वर्यमाप्यते ।
तच्च वैकृतबन्धस्तु यदि तत्रानुरज्यते ।।१३.१० ।।
आरामोद्यानवापीषु दानक्रतुफलेषु च ।
आशक्तमनसा वाचा दक्षिणाबन्धः कथ्यते ।।१३.११ ।।
अनेनैव तु पाशेन बद्धावानरवद्यथा ।
मोक्षितं न च शक्नोति इतश्चेतश्च धावति ।।१३.१२ ।।
देवासुरमनुष्येषु तिर्येषु नरकेषु च ।
भ्रमन्ते चक्रयन्त्रेव ? यावत्तत्त्वं न विन्दति ।।१३.१३ ।।
गर्भवासपरिक्लेशौ जन्ममृत्यु पुनः पुनः ।
व्याधिः शोकभयायास चिन्तया जरया हतः ।।१३.१४ ।।
देव्युवाच ।
गर्भोत्पत्तिः कथं देव योगी लभति कीदृशीम् ।
कीदृशं लभते गर्भः श्रोतुं नः प्रत्युदीर्यताम् ।।१३.१५ ।।
भगवानुवाच ।
शृणु देवि प्रवक्ष्यामि गर्भोत्पत्तिर्यथाक्रमम् ।
यथा संशयविच्छेदं भविष्यसि वरानने ।।१३.१६ ।।
अक्षरात्प्रभवो ब्रह्मा कर्मबद्धसमुद्भवम् ।
कर्मतो यज्ञप्रभवो यज्ञतो धूमसम्भवः ।।१३.१७ ।।
पर्जन्यादन्नमुत्पत्तिरन्नाद्भूतानि जज्ञिरे ।
अन्नाद्रससमुत्पत्ति रसाच्छोणितसम्भवः ।।१३.१८ ।।
शोणितात्  । मांस ।म् ।उत्पत्ति मांसाद्मेदसमुद्भवः ।
मेदसो ऽस्थीनि जायन्ते अस्थिभ्यो मज्जसम्भवः ।।१३.१९ ।।
मज्जायास्तु भवेच्छुक्रं नरः शुक्रसमुद्भवः ।
शुक्रशोणितसंयोगाद्गर्भोत्पत्तिस्ततः स्मृतः ।।१३.२० ।।
अग्निसोमात्मकं देवि शरीरद्वयधातुतः ।
सोमधातुस्मृतं शुक्रमग्निधातुरजस्मृतम् ।
अग्निसोमाश्रयं देवि शरीरमिति संज्ञितम् ।।१३.२१ ।।
मासी मासी ऋतुः स्त्रीणां भवतीह न संशयः ।
ऋतुकाले प्रसर्प्येत न सुखार्थं वरानने ।।१३.२२ ।।
पुत्रकामप्रयुञ्जीत धर्मार्थश्च यशस्विनि ।
पुमान्स्त्रीपुं प्रयुञ्जीत अरणी बहुताशनः ।।१३.२३ ।।
पुमान्शुक्राधिको ज्ञेयः कन्या रक्ताधिका भवेत् ।
समशुक्रे च रक्ते च स च जायेन्नपुंसकः ।।१३.२४ ।।

---- द्वियमा त्रियमा च गुर्विणी ----

देव्युवाच ।
द्वियमा त्रियमा चैव कथं जायेत गुर्विणी ।
कथं स्त्रीद्वियमा जायेत्कथं वा पुरुषद्वयम् ।।१३.२५ ।।
भगवानुवाच ।
रक्ताधिका स्मृता कन्या जायते वरवर्णिनि ।
वायुना च द्विधा भिन्ना कन्यकद्वियमा स्मृता ।।१३.२६ ।।
शुक्राधिकास्तु पुरुष द्विधा भिन्नानिलेन तु ।
द्वियमा पुरुषो ज्ञेया त्रियमास्तु त्रिधा कृते ।।१३.२७ ।।
ऋतुस्नाता यदा नारी यदि गर्भादि गृह्यति ।
प्रथमे च द्वितीये च तृतीये च स जीवति ।।१३.२८ ।।
समेषु जनयेत्पुत्रः कन्यका विषमे दिने ।
षष्ट्याष्टमौ च दशमी द्वादशी च पुमान्भवेत् ।।१३.२९ ।।
पञ्चमी सप्तमी चैव नवमेकादशी स्त्रियः ।
समरक्ते च शुक्रे च श्यामः संजायते पुमान् ।।१३.३० ।।
रुधिरं त्वेकरात्रेण कललं प्रतिपद्यते ।
कललं पञ्चरात्रेण अर्बुदत्वं प्रजायते ।।१३.३१ ।।
अर्बुदः सप्तरात्रेण मांसपेशी समुद्भवः ।
द्वितीयं सप्तरात्रेण तत्सर्वं मांसशोणितम् ।।१३.३२ ।।
तृतीयं सप्तरात्रेण हृदयं जायते ततः ।
ततः सर्वाणि गात्राणि शिरश्चैवोपजायते ।।१३.३३ ।।
हृदये जायमाने तु मूर्च्छान्तन्द्रिररोचकः ।
स्त्रियाः धर्दिः प्रशेकश्च दौर्बल्यं चोपजायते ।।१३.३४ ।।
तस्या हि हृदयं नारी यदि भक्ष्यति किंचन ।
भक्ष्यं लोह्यं तथा पेयमुपभोगास्तथाययत् ।।१३.३५ ।।
शयनासनयानानि वस्त्राण्याभरणानि च ।
यद्यदाकांक्षते किंचित्तत्तदास्यै प्रदापयेत् ।।१३.३६ ।।
नाया संकारयेच्चास्या न चैवमवमानयेत् ।
मुखमापाण्डुरं स्निग्धं कपोलस्तनकेशयोः ।।१३.३७ ।।
शरीरश्च श्रिया जष्टुं पीनोरुश्रोणि वक्षसम् ।
लिङ्गेरेभिर्विजानीयां गर्भे जीवं प्रतिष्ठितम् ।।१३.३८ ।।
चतुर्थे सप्तरात्रेण शिरश्चैवोपजायते ।
पञ्चमसप्तरात्रेण ग्रीवा तत्रोपजायते ।।१३.३९ ।।
षष्ठमसप्तरात्रेण स्कन्धगात्रं प्रजायते ।
सप्तमसप्तरात्रेण पृष्ठवंश प्रजायते ।।१३.४० ।।
अष्टमसप्तरात्रेण पाणी जायते चोभयौ ।
सप्तरात्रं नव प्राप्य जायते हृदि पञ्जरम् ।।१३.४१ ।।
दशमे सप्तरात्रे च पादौ जायते चोभौ ।
उदरश्चोपजायेत सप्तैकादशरात्रिके ।।१३.४२ ।।
द्वादशसप्तरात्रेण कुक्षिपार्श्वेः प्रजायते ।
सप्तत्रैदशरात्रेण कुटिसुत्रोपजायते ।।१३.४३ ।।
नवत्यष्टमरातेण जायते सूत्रविंशति ।
सप्तपञ्चदशाहेन सर्वमेदः प्रजायते ।।१३.४४ ।।
षोडशसप्तरात्रेण अथिसर्वाणि जायते ।
सप्तसप्तदशाहेन जायते स्नायुबन्धनम् ।।१३.४५ ।।
सप्तमाष्टादशाहेन जायते मुखमण्डलम् ।
सप्तोनविंशरात्रेण घ्राणवंशः प्रजायते ।।१३.४६ ।।
सप्तविंशतिरात्रेण नैत्रनालिं प्रजायते ।
सप्तैकविंशरात्रेण कर्णयुग्मं प्रजायते ।।१३.४७ ।।
द्वाविंशसप्तरात्रेण जायते द्वौ भ्रुवौ ततः ।
सप्तत्रिविंशरात्रेण गण्डयुग्मं प्रजायते ।।१३.४८ ।।
चतुर्विंशतिसप्ताहे ओष्ठयुग्मं प्रजायते ।
पञ्चविंशतिसप्ताहे जिह्वा जायते सुन्दरि ।।१३.४९ ।।
षड्विंशसप्तरात्रेण दन्तपङ्क्ति प्रजायते ।
उनविंशतिसप्ताहे जायते च त्वगेव च ।।१३.५० ।।
त्रिंशतसप्तरात्रेण जायते नाभिमण्डलम् ।
सप्तैकत्रिंशरात्रेण सर्वरन्ध्रं प्रजायते ।।१३.५१ ।।
द्वात्रिंशसप्तरात्रेण नखविंशति जायते ।
त्रित्रिंशसप्तरात्रेण सर्वे सन्धिः प्रजायते ।।१३.५२ ।।
पञ्चत्रिंशति सप्ताहे सर्वमर्म प्रजायते ।
षड्त्रिंशसप्तरात्रेण वेदना चोपजायते ।।१३.५३ ।।
सप्तत्रिंशतिसप्ताहे ईर्ष्याद्वेषः प्रजायते ।
अष्टत्रिंशतिसप्ताहे पञ्चात्मकसमन्वितम् ।।१३.५४ ।।
सर्वाङ्गमङ्गसम्पूर्णः परिपक्व(ः) स तिष्ठति ।
मातुस्वाशितपीतश्च नाभिसूत्रागनेन तु ।।१३.५५ ।।
अजातस्योपधार्यन्ते गर्भस्थस्यैव जन्तवः ।
ततः प्रविशते देहे निद्रास्वप्न यथा तथा ।।१३.५६ ।।
नोपलभ्यति सूक्ष्मत्वादरण्यग्निर्यथा तथा ।
गर्भोदकेन सिक्ताङ्गजराया परिवेष्टितः ।।१३.५७ ।।
जाति स्मरति तत्रस्थो जन्तुश्चेतःसमन्वितः ।
मृतश्चाहं पुनर्जातो भूयश्चैव पुनर्मृतः ।।१३.५८ ।।
स्थावराणां सहस्रेषु जातो ऽस्मि विविधेषु च ।
चतुर्वर्णविवर्णेषु मानुषेषु सहस्रशः ।।१३.५९ ।।
साम्प्रतं च पुनर्गर्भः क्लेशः प्राप्तः सुदुःसहः ।
इदानीं जातमात्रो ऽहं संस्कारैश्चापि संस्कृतः ।।१३.६० ।।
योगमेवाभिसेवामि सा[ं]ख्यं वा पञ्चविंशकम् ।
यत्र जन्मजरा नास्ति यत्र मृत्युश्च नास्ति वै ।।१३.६१ ।।
यत्र ब्रह्म परं वेद्यं चरिष्यामि यतव्रतः ।
एवमादीन्यनेकानि चिन्तयित्वा पुनः पुनः ।।१३.६२ ।।
यावत्तिष्ठति गर्भस्थो जाति स्मरति पूर्विकाम् ।
ततो जायति कष्टेन महाक्लेशेन मानवः ।।१३.६३ ।।
योनियन्त्रसुतीव्रेण पीड्यमानसुदुःखितः ।
जातमात्रोस्मृतिभ्रंशो भवतीह अचेतनेः ।।१३.६४ ।।
मायामुद्गरतीव्रेण हतः किं शुभमाचरेत् ।
एष गर्भसमुत्पत्तिः कथितो ऽस्मि वरानने ।
दुःखसंसारप्रशमं किं भूयः श्रोतुमिच्छसि ।।१३.६५ ।।

 ।।इति वृषसारसंग्रहे गर्भोत्पत्तिर्नाम त्रयदशो ! ऽध्यायः ।।




देव्युवाच ।
अतिदीर्घातिह्रस्वश्च पुमान्केनोपजायते ।
अतिगौरो ऽतिकृष्णश्च नरो भवति किं प्रभो ।।१४.१ ।।
भगवानुवाच ।
गृहीतगर्भा या नारी नित्यमुत्तानशालिनी ।
प्रसारितविमुक्तात्मा सो ऽतिदीर्घः प्रजायते ।।१४.२ ।।
गृहीतगर्भा या नारी शेते संकुचिता सदा ।
रसो ऽन्नादीनि कटुकं सेवनाः ह्रस्व जायते ।।१४.३ ।।
गृहीतगर्भा या नारी नित्यं क्षीरोपसेविता ।
वरकोद्रवशाली च भुक्ता चापि यवोदनम् ।।१४.४ ।।
शुक्लवस्त्रस्रजा युक्ता सातिगौरं प्रजायते ।
गृहीतगर्भा या नारी बालधान्यानि सेवते ।।१४.५ ।।
कृष्णकोद्रवतैलादि माषकृष्णयवोदनम् ।
कृष्णवस्त्रस्रजादीनि तस्याः कृष्णः प्रजायते ।।१४.६ ।।
देव्युवाच ।
जात्यन्धो जायते कस्मान्षण्ड्होभीरुर्हतेन्द्रियः ।
कुजो वा वामनो वापि पङ्गवः स्थूलशिरः कथम् ।।१४.७ ।।
भगवानुवाच ।
गृहीतगर्भा या नारी तीक्ष्णोष्णान्युपसेवते ।
लशुनानिपलाण्डूनि करञ्जमूलकानि च ।।१४.८ ।।
पिप्पलीशृङ्गवेरं च सर्षपान्मरिचानि च ।
आसवश्च परिक्लिष्टा ये चान्ये कटुतिक्तकाः ।।१४.९ ।।
तीक्ष्णं तु सेवमाना या जात्यन्धो जायते सुतः ।
मिथ्यापचाराः स्त्रीपुंसो व्यापन्ने शुक्रशोणिते ।
यदा गर्भाशये रक्तं स्त्रियाः पूर्वं निषिच्यते ।।१४.१० ।।
पश्चाच्छुक्रं रक्तकाले तदाषण्डः प्रजायते ।
त्रस्तोद्विग्नो यदा भीतस्त्रीपुंसांसूपजायते ।।१४.११ ।।
तत्र यो जायते गर्भभिरुः क्रन्दनको भवेत् ।
निसर्गकाले शुक्रस्य विघ्न उत्पद्यते यदा ।।१४.१२ ।।
इन्द्रियावर्तविघ्ने तु तदा जायेदतिन्द्रियः । 
गृहीतगर्भा या नारी वातलान्युपसेवते ।।१४.१३ ।।
कटुकानि कषायानि तिक्तानि च विशेषतः ।
वातः प्रकुपितस्तस्या गर्भमातुह्य तिष्ठति ।।१४.१४ ।।
कुब्जस्तु जायते तस्माद्गर्भाद्वातनिपीडनात् ।
नित्यसासवशीलाया तथा चोत्कटुकाशना ।।१४.१५ ।।
तस्या संहन्यते गर्भो वामनस्तेन जायते ।
अतिव्यायामशीला तु य नारी विषमासनी ।।१४.१६ ।।
गर्भः संक्षुभ्यते तस्याः पषण्डस्तेनोपजायते ।
गृहीतगर्भा या नारी रूक्षधान्यानि सेवते ।।१४.१७ ।।
वातश्लेष्मशिरस्थो वै तस्या गर्भस्य कुप्यते ।
ततः स्थूलशिरास्तेन पुमान्जायत्यसंशयः ।।१४.१८ ।।
देव्युवाच ।
करालाङ्गा हनुः पङ्गूर्मूको गद्गदभाषकः ।
विकृताक्षस्त्वनक्षो वा भवद्रस्वगुदः कथम् ।।१४.१९ ।।
भगवानुवाच ।
करालस्तेन दोषेण जायते मानवस्तथा ।
अथ करालं कुरुते नारी लम्बोतिचूचुका ।
तस्मादनेन दोषेण करालो जायते पुमान् ।।१४.२० ।।
गृहीतगर्भा या नारी रक्तपित्तामयार्दिता ।
गोहनुं जनयेत्येषा रक्तपित्तप्रकोपितः ।।१४.२१ ।।
गृहीतगर्भा या नारी वातशूलैरुपद्रुता ।
शुक्रो दावर्तनी चापि पङ्गू जनयते सुतम् ।।१४.२२ ।।
क्षुधार्ता वेदनार्ता च सततश्चोपवासिनी ।
मूकं जनयते बालं दौहृदश्च विमानिता ।।१४.२३ ।।
गृहीतगर्भा या नारी विसृजेत्  । मास मासिकम् ।
अनक्षो जायते तस्या गर्भशोणितसंक्षयात् ।।१४.२४ ।।
अथ ग्रस्ता यदा नारी वातो दावर्तपीडिता ।
गृहीतगर्भा रुक्षाणि वातलान्युपसेवते ।।१४.२५ ।।
वातस्थानन्ततस्तस्या गर्भस्यापीडितं भवेत् ।
अगुदो जायते तस्माज्जातश्चापि न जीवति ।।१४.२६ ।।
देव्युवाच ।
हीनाङ्गो जायते कस्मादधिकाङ्गो ऽपि वा कथम् ।
श्वेतपिङ्गेक्षणः कस्मात्कथं लोहितलोचनः ।।१४.२७ ।।
भगवानुवाच ।
गर्भस्य जायमानस्य  ।  ।  ।  जायते निलः ।
वाताभ्यां श्लेष्मणात्  ।  ।  । तदङ्गं परिहीयते ।।१४.२८ ।।
हीनाङ्गो जायते तस्मात्पुमान्वातप्रकोपतः ।
गृहीतगर्भा या नारी मधुराण्युपसेवते ।।१४.२९ ।।
शृङ्गाटककलोत्यानि शालूकानि विशानि च ।
मोचं तालफलं चैव नारिकेलफलं तथा ।।१४.३० ।।
अतिक्ष्णं सेवमाना तु अधिकाङ्गंप्रसूयते ।
पिङ्गाक्षः श्लेष्मपित्ताभ्यां श्वेताक्षः श्लेष्मणा भवेत् ।।१४.३१ ।।
देव्युवाच ।
कथं वा जायते पुत्रः कन्यका केन जायते ।
अपुमान्केन जायेत द्वियमा त्रियमा तथा ।।१४.३२ ।।
भगवानुवाच ।
शुक्राधिकः पुमान्ज्ञेयः कन्या रक्ताधिका भवेत् ।
रक्तशुक्रसमत्वेन जायते स नपुंसकः ।।१४.३३ ।।
पिण्डीभूतो यदा गर्भ मारुतौ विभवेद्द्विधा ।
एवं ते द्वियमा ज्ञेयास्त्रियमा च त्रिधा कृते ।।१४.३४ ।।
देव्युवाच ।
शोणितं मांस मेदश्च अस्थि मज्जा च पञ्चमी ।
शरीरस्थानि दृश्यन्ते शुक्रस्थानं न दृश्यते ।।१४.३५ ।।
तस्योत्पत्तिश्च स्थानं च ज्ञातुमिच्छामि तत्त्वतः ।
कथयस्व त्रिलोकेश च्छेत्तुमर्हसि संशयः ।।१४.३६ ।।
भगवानुवाच ।
मनः शुक्रस्य प्रभवं घ्राणं श्रोत्रं तथाक्षिणी ।
स्थानं तु सर्वाङ्गसमस्पर्शात्स्पर्शः प्रवर्तते ।।१४.३७ ।।
यथा निषिक्तं क्षीरं तु पयसाद्दधि जायते ।
प्रमथ्यमानदध्नस्तु सर्पिसो ऽपि तथागमः ।।१४.३८ ।।
एवं शरीर निर्गच्चेत्  । शुक्रं शुक्रवहा शिराः ।
पूरयित्वानुपूर्वेण अस्थयो प्रतिपद्यते ।।१४.३९ ।।
ततस्तु ताः शुक्रवहा मेड्ह्रनाभीमनुसृताः ।
नाशुक्रं तत्तु सिञ्चन्ति तस्माद्गर्भस्य सम्भवः ।।१४.४० ।।
देव्युवाच ।
कथं वेदयते जाति कथं जातिस्मरो भवेत् ।
एतस्मिन्संशयं मे ऽद्य छेत्तुमर्हसि शङ्कर ।।१४.४१ ।।
भगवानुवाच ।
भावितात्मां च यो जन्तुर्देवि भोगाधिकं च यत् ।
ब्रह्मविद्ज्ञानसंयुक्तः स जातिं स्मरते पुमान् ।।१४.४२ ।।
देव्युवाच ।
कथं सद्यो गृहीतस्य लिङ्गगर्भस्य दृश्यते ।
एतत्कथय देवेश रहः काले महेश्वर ।।१४.४३ ।।
भगवानुवाच ।
पिपाशारोमहर्षं च वेपनं गात्रसीदनम् ।
निद्रास्वेदं च तन्द्रा च मुहूर्तमुपजायते ।।१४.४४ ।।
निक्लेदत्वं खरत्वं च योन्यात्समुपजायते ।
न चार्द्रवंवै दृश्येत शुक्रस्य रजसो ऽपि वा । 
सद्योगृहीतगर्भाया लिङ्गान्येतानि तत्त्वतः ।।१४.४५ ।।
देव्युवाच ।
केन लिङ्गेन विज्ञेयं पुत्रजन्म महेश्वर ।
कन्यका केन लिङ्गेन ज्ञायते कथयस्व मे ।।१४.४६ ।।
भगवानुवाच ।
पादोरुजङ्घपार्श्वश्च दक्षिणं यदि ह्युन्नतः ।
दक्षिणं विपुलं तत्र तदा पुत्रः प्रजायते ।।१४.४७ ।।
वामश्चैव यदा पश्येत्तदा जायेत कन्यका ।
उन्नतं मध्यमस्थाश्च तदा जायेत्  । नपुंसकम् ।।१४.४८ ।।
देव्युवाच ।
पुंसा कपोलरोमानि खलितं केन जायते ।
कथं स्त्रीणां न जायेत रोमाणि खलितं तथा ।।१४.४९ ।।
भगवानुवाच ।
तथा वृषणगा जन्तोर्यस्य रेतोवहा शिरः ।
निबद्धा मस्तके तालु कपोलास्तु समाश्रिताः ।।१४.५० ।।
तैः कपोलेषु रोमाणि जायन्ते अन्तरेतसः ।
खलितं शुक्रदोषेण नराणामुपजायते ।।१४.५१ ।।
शिरा शुक्रवहा स्त्रीणां न शून्यस्यान्न जायते ।
यात्मापालो च कास्त्वग्नि दृष्टिमण्डलसंश्रितः ? ।।१४.५२ ।।
शोणितै सोक्तिकोष्टस्थन्निशोषयति तत्त्वतः ।
निबद्धन्त्यक्षिपक्ष्माणि तेन रोमाणि च भ्रुवोः ।।१४.५३ ।।
अशुक्रत्वाच्च नारीणां खलितं नोपजायते ।
छायाव्यपगतस्नेहा रुक्षागात्रशिरोरुहा ।
ग्रसतोस्माभजठरा मृतगर्भः प्रजायते ।।१४.५४ ।।
देव्युवाच ।
सोमधातु कथं ज्ञेया अग्निधातुस्तथेश्वर ।
पृथग्भागविशेषेण कथयस्व महेश्वर ।।१४.५५ ।।
भगवानुवाच ।
श्लेष्ममेदस्तथा स्नायुः अस्थिदन्तनखानि च ।
स्त्रियास्तन्यश्च शुक्रश्च यच्च श्वेतं तथाक्षिषु ।।१४.५६ ।।
एतेषां सौम्यभागत्वाच्छ्वेतत्वमुपजायते ।
आग्नेयभावाद्रक्तत्वं कृष्णत्वं चापि गच्छति ।।१४.५७ ।।
त्वग्मांसरुधिरं मज्जादृष्टिरोम तथैव च ।
आग्नेयधातुसोमश्च कथितो ऽस्मि वरानने ।
ब्रूहि ब्रूहि विशालाक्षि यद्यस्ति तव संशयः ।।१४.५८ ।।

 ।।इति वृषसारसंग्रहे प्रश्नव्याकरणो नामश्चतुर्दशो ऽध्यायः ।।





---- जीववर्णनम् ----

देव्युवाच ।
जीवभूतेति यत्प्रोक्तं लक्षणं कीदृशं भवेत् ।
स्थानमस्य न जानामि रूपं वर्णं च ईश्वर ।।१५.१ ।।
एतत्कौतूहलं छिन्धि संशयं परमेश्वर ।
न चान्यदेव पश्यामि जीवनिर्णय कीर्तय ।।१५.२ ।।
ईश्वर उवाच ।
जीवस्य लक्षणं देवि कथितुं केन शक्यते ।
न रूपवर्णं जीवस्य विद्यते स्थानमेव च ।।१५.३ ।।
व्यापि सर्वगतं सूक्ष्मं सर्वमाश्रित्य तिष्ठति ।
निरालम्बमनाधारमनौपम्यं निरञ्जनम् ।।१५.४ ।।
अरणिस्थो यथा वह्निः काष्ठेषु नोपलभ्यते ।
तद्वज्जीवो न पश्येत शरीरस्थो ऽपि सुन्दरि ।।१५.५ ।।
दधिवच्च यथा सर्पिर्दृश्यते न च दृश्यते ।
तद्वज्जीवः शरीरस्थो दृश्यते न च दृश्यते ।।१५.६ ।।
देव्युवाच ।
अदृष्टप्रत्ययो ह्यस्ति नास्ति प्रत्ययदर्शनम् ।
व्यापी कथं महादेव सर्वत्रावस्थितः कथम् ।।१५.७ ।।
महेश्वर उवाच ।
असंशयो महादेवि व्यापी सर्वगतः शिवः ।
दृश्यतेन्द्रियसंयोगाज्जीवप्रत्ययदर्शनम् ।।१५.८ ।।
यथाकाशस्थितो वायुः शब्दस्पर्शगुणान्वितः ।
तद्वद्देही विजानीयाद्गुणचेष्टेन नान्यथा ।।१५.९ ।।
देव्युवाच ।
व्यापीति कथितः पूर्वं जीवः सर्वगतो ऽपि च ।
तं वृथा कथितो ऽस्यद्य म्रियते केन हेतुना ।।१५.१० ।।
ईश्वर उवाच ।
न जीवो म्रियते देवि सर्वेषां सुरसुन्दरि ।
घटान्तस्थो यथाकाशो बहिराकाशवद्यथा ।।१५.११ ।।
घटभिन्ने विशालाक्षि विशेषो नोपलक्ष्यते ।
देहभिन्ने यदा देवि विनाशो नोपलभ्यते ।।१५.१२ ।।
सुसूक्ष्मः सर्वगो व्यापी परमात्मानमव्ययः ।
बहिरन्तश्च भूतानामचरश्चर एव सः ।।१५.१३ ।।
अप्रमेयो ऽविनाशी च अप्रपञ्चः प्रपञ्चकः ।
सर्वेन्द्रियगुणाभासः सर्वेन्द्रियविवर्जितः ।।१५.१४ ।।
एवमेष महादेवि जीवस्य वरवर्णिनि ।
कथितो ऽस्मि समासेन किमन्यच्छ्रोतुमिच्छसि ।।१५.१५ ।।

---- सारश्रेष्ठम् ----

देव्युवाच ।
सारश्रेष्ठं महादेव कथयेशान ईश्वर ।
श्रोतुमिच्छामि देवेश मानुषाणां हितं वद ।।१५.१६ ।।
ईश्वर उवाच ।
आश्रमाणां गृही श्रेष्ठो वर्णश्रेष्ठा द्विजातयः ।
अश्वमेधः क्रतुश्रेष्ठो जपश्रेष्ठो ऽघमर्षणः ।।१५.१७ ।।
देवतानां हरिः श्रेष्ठः श्रेष्ठा गङ्गा नदीषु च ।
अनाशनस्तपःश्रेष्ठस्तीर्थश्रेष्ठः सुरह्रदः ।।१५.१८ ।।
क्षोमं वस्त्रेषु च श्रेष्ठं यशः श्रेष्ठं विभूषणम् ।
भारतं श्रुतिषु श्रेष्ठं व्रतश्रेष्ठो दयापरः ।।१५.१९ ।।
दानेषु चाभयं श्रेष्ठं मनः श्रेष्ठेन्द्रियेषु च ।
विद्या संग्रहषु श्रेष्ठा सत्यं श्रेष्ठं वचःसु च ।।१५.२० ।।
आयुधानां धनुः श्रेष्ठं बान्धवेषु च मातरः ।
ज्ञानमौषधिषु श्रेष्ठं वैद्यश्रेष्ठः शिवाक्षरः ।।१५.२१ ।।
अकारश्चाक्षरः श्रेष्ठो धर्मश्रेष्ठो ह्यहिंसकः ।
पशुषु सौरभी श्रेष्ठा नरेषु च नराधिपः ।।१५.२२ ।।
मासि मार्गशिरः श्रेष्ठं कृतः श्रेष्ठश्चतुर्युगे ।
वसन्त ऋतुषु श्रेष्ठः श्रेष्ठं चायनमुत्तरम् ।।१५.२३ ।।
अमावास्या दिनश्रेष्ठा ग्रहश्रेष्ठो दिवाकरः ।
स्त्रीषु लक्ष्मीर्धृतिः श्रेष्ठा वसुश्रेष्ठो हुताशनः ।।१५.२४ ।।
ऋषिषु उषणा श्रेष्ठः कान्तिश्रेष्ठो निशाकरः ।
नक्षत्रेष्वभिजित्श्रेष्ठः कालः श्रेष्ठः कलेषु च  ।।१५.२५ ।।
वेदेषु च वरं साम स्थावरेषु हिमालयः ।
अश्वत्थो वट वृक्षेषु भूतेषु वर चेतनः ।।१५.२६ ।।
अध्यात्म सर्वविद्यासु वाक्य सत्य वर स्मृतः ।
प्रह्लादो वर दैत्येषु यक्षरक्षो धनेश्वरः ।।१५.२७ ।।
मरीचिर्वर वातेषु हरिः श्रेष्ठो मृगेषु च ।
साध्य नारायणः श्रेष्ठः पितॄणां च पितामहः ।।१५.२८ ।।
एतत्समासतो देवि कथितो ऽसि वरानने ।
सर्वसारं समुद्धृत्य किं भूयः कथयाम्यहम् ।।१५.२९ ।।

 ।।इति वृषसारसंग्रहे जीवनिर्णयो नामाध्यायः पञ्चदशमः ।।





---- योगसद्भावनिर्णयः ----

देव्युवाच ।
अधुना श्रोतुमिच्छामि योगसद्भावनिर्णयम् ।
करणं च यथान्यायं कथयस्व सुरेश्वर ।।१६.१ ।।
ईश्वर उवाच ।
शृणु देवि प्रवक्ष्यामि योगसद्भावमुत्तमम् ।
यं विदित्वा न पश्यन्ति जनाः संसारबन्धनम् ।।१६.२ ।।
ब्रह्महा गुरुतल्पी वा सुरापस्तेय एव वा ।
अथवा संकरे जातस्तत्सर्वमपनोदति ।।१६.३ ।।
मुहूर्तार्धे मुहूर्ते वा प्राणायामपरायणः ।
ध्येयं चिन्तयमानस्य तत्पापं क्षीयते नरात् ।।१६.४ ।।
न यमो नान्तकः क्रुद्धो न मृत्युर्भीमविग्रहः ।
नाविशन्ति महात्मानो योगिनो बलवत्तराः ।।१६.५ ।।
यथा वै सर्वधातूनां दोषा दह्यन्ति धाम्यताम् ।
तथा पापाः प्रदह्यन्ते ध्रुवं प्राणस्य निग्रहात् ।।१६.६ ।।
अश्वमेधसहस्रं च राजसूयशतं तथा ।
प्राणायामशतं चैव न तत्तुल्यं कदाचन ।।१६.७ ।।
यज्ञेन देवानाप्नोति राज्यं वै तपसः फलम् ।
संन्यासाद्ब्रह्मणः स्थानं वैराग्यात्प्रकृतौ लयम् ।।१६.८ ।।
ज्ञानात्प्राप्नोति कैवल्यं परं ब्रह्म सनातनम् ।
इत्येता गतयः पञ्च विधिवत्परिकीर्तिताः ।।१६.९ ।।
मुहूर्तार्धं मुहूर्तं वा योगं युञ्जीत योगवित् ।
निस्तरेत्सर्वपापानि अमृतत्वं च गच्छति ।।१६.१० ।।
युञ्जानो ऽपि प्रयत्नेन यावत्तत्त्वं न विन्दति ।
ब्रह्मलोके ध्रुवं वासो विष्णुलोके च सुन्दरि ।।१६.११ ।।
भुक्त्वा कर्मसहस्राणि सर्वकामसमन्वितः ।
क्षीणपुण्ये ततो मर्त्ये जायते विपुले कुले ।।१६.१२ ।।
योगमेवाभिसेवेत पूर्वजातिस्मरो नरः ।
संसारार्णवमुत्तीर्य स शिवत्वमवाप्नुयात् ।।१६.१३ ।।

---- योगविधिः ----

देव्युवाच ।
योगस्य विधिमिच्छामि श्रोतुं मे पुरुषोत्तम ।
ध्यानधारणसिद्धीनां कथयस्व सुरेश्वर ।।१६.१४ ।।
महेश्वर उवाच ।
शृणु योगविधिं वक्ष्ये भवपाशनिकृन्तनम् ।
शुचिरेकाग्रचित्तस्तु जनशब्दविवर्जिते ।
तत्रासीनासने योगी परमात्मान चिन्तयेत् ।।१६.१५ ।।
पद्मकं स्वस्तिकं चैव निष्कलमञ्जलिस्तथा ।
अर्धचन्द्रं च दण्डं च पर्यङ्कं भद्रमेव च ।।१६.१६ ।।
एतदासनबन्धेन बद्ध्वा योगं समभ्यसेत् ।
समं कायशिरोग्रीवं धारयन्नचलस्थितः ।।१६.१७ ।।
प्रत्याहारस्तथा ध्यानं प्राणायामश्च धारणा ।
तर्कश्चैव समाधिश्च षडङ्गो योग उच्यते ।।१६.१८ ।।
विषयासक्तचित्तानामिन्द्रियाणां प्रति प्रति ।
मनसाकर्षयेद्यस्तु प्रत्याहारः स उच्यते ।।१६.१९ ।।
शब्दादिविषयान्देवि वर्तुलीकृत्य धारयेत् ।
वीतरागः समाधिस्थो ध्येये वस्तुनि योजयेत् ।।१६.२० ।।
आत्मा ध्याता मनो ध्यानं ध्येयः शुद्धः परः शिवः ।
यत्परं परमैश्वर्यमेकं तत्र प्रयोजनम् ।।१६.२१ ।।
पूरकः कुम्भकश्चैव रेचकस्तदनन्तरम् ।
प्रशान्तश्चेति विख्यातः प्राणायामश्चतुर्विधः ।।१६.२२ ।।
पूरके स्थापयेद्वह्निं पादाङ्गुष्ठेन बुद्धिमान् ।
कुम्भकेन विरुध्येत दह्यमानं विचिन्तयेत् ।।१६.२३ ।।
भस्मीभूतं तथात्मानं रेचकेन विचिन्तयेत् ।
शुद्धदेहस्ततश्चात्मा शुद्धस्फटिकनिर्मलः ।।१६.२४ ।।
तालशब्दास्तु निर्वाणं दश द्वे च प्रकीर्तितः ।
प्राणायामान्न संदेहो द्विगुणा धारणा स्मृता ।।१६.२५ ।।
योगे तु त्रिगुणा प्रोक्ता संक्रमे च चतुर्गुणा ।
! तथोत्क्रान्तौ पञ्चगुणा योगसिद्धिस्तु षड्गुणा ।।१६.२६ ।।
षडङ्गेन समायुक्तो योगयुक्तस्तु नित्यशः ।
मानसो यौगपद्यश्च द्विरूपो योग उच्यते ।।१६.२७ ।।
अकृत्वा प्राणसंरोधं मनसैकेन केवलम् ।
ध्यायेत परमं सूक्ष्मं स योगो मानसः स्मृतः ।।१६.२८ ।।
संयम्य मनसा प्राणं प्राणायामान्मनस्तथा ।
एवं ध्यायेत्परं सूक्ष्मं यौगपद्यः स उच्यते ।।१६.२९ ।।

---- सिद्धिलक्षणम् ----

सिद्धिलक्षण योगस्य शृणु वक्ष्यामि सुन्दरि ।
शङ्खभेरीमृदङ्गं च वेणुदुन्दुभिमेव च ।
ताडितं न च विन्देत यदा तन्मयतां गतः ।।१६.३० ।।
शीतोष्णं सुखदुःखं च तृष्णाभुक्षं तथैव च ।
वेदनां नैव जानाति योगसिद्धस्तु सुन्दरि ।।१६.३१ ।।
एष योगविधिर्देवि तव पृष्टेन सुन्दरि ।
कथितो ऽस्मि समासेन किमन्यत्कथयाम्यहम् ।।१६.३२ ।।
देव्युवाच ।
विना योगेन देवेश संसारतारणं मम ।
कथयस्व महादेव निर्विकल्पकरं मनः ।।१६.३३ ।।
महेश्वर उवाच ।
सदाशिवस्तु निश्वास ऊर्ध्वश्वासः परः शिवः ।
तयोर्मध्ये तु विज्ञेयः परमात्मा शिवो ऽव्ययः ।।१६.३४ ।।
ध्यानयोगं न तस्यास्ति करणं च न विद्यते ।
ज्ञातमात्रेण मुच्यन्ते किमन्यत्परिपृच्छसि ।।१६.३५ ।।

---- पञ्च शास्त्राणि ----

ज्ञानमन्यत्प्रवक्ष्यामि शृणु देवि निबोध मे ।
शास्त्रपञ्चसु यत्प्रोक्तं शृणु संक्षेप निर्णयम् ।
सांख्ये योगे पञ्चरात्रे शैवे वेदे च निर्मितम् ।।१६.३६ ।।
यत्सांख्यसिद्धं कथयाम्यहं ते
संसारघोरार्णवयोगसारम् ।
योगेषु सारेष्वथ पञ्चरात्रे
वेदेषु शैवेषु च निश्चयस्ते ।।१६.३७ ।।
घ्राणेन्द्रियाद्येषु च यत्समस्तम
मनश्च लीनं भवतीव यस्य ।
! बुद्ध्या नियम्य सकलान्हि भावान
स लब्धलक्ष्यः शिवमभ्युपैति ।।१६.३८ ।।
श्रोत्रादिसर्वेन्द्रियनिश्चलत्वे
एकाग्रचित्तं मनसा नियम्य ।
स्वदेहशून्यः स भवेच्चिरेण
संयोगसिद्धिं प्रवदन्ति तज्ज्ञाः ।।१६.३९ ।।
आदावेव मनः शनैरुपरमेत्कृत्वा च वश्येन्द्रियं
यावत्तल्लयतां व्रजेत मनसा निःसंज्ञदेहस्तथा ।
एतद्ध्यानसमाधियोगसकलं प्राप्नोति निःसंशयं
किं तच्छास्त्रसहस्रकोटिपठितं सारं न यो ऽन्विष्यति ।।१६.४० ।।
आत्मारामजितः समाधिनिरतो वैराग्यमप्याश्रितः
चित्तं यस्य परिक्षयो यदि भवेत्तिष्ठेत्तनुत्वं यथा ।
तज्ज्ञेयं गतिमुत्तमं शिवपदं संसारदुःखच्छिदं
वेदान्तेषु च निष्ठ एष कथितः किं शास्त्रमन्यद्विशेत् ।।१६.४१ ।।
हृत्पद्मे कर्णिकायामुपरि रविरवद्योतयन्तो ऽन्तरालम
यत्तेजस्तेजमार्गैर्बहलतमघनैर्द्योतनाद्दीप्तदीपम् ।
भित्त्वा यत्तालुदेशे मुखमुपरिगतं तालुदेशेन मूर्ध्नि
! मूर्ध्नि द्वारान्तरेण शिवपरमपदं यान्ति योगेन युक्ताः ।।१६.४२ ।।
कृष्णः कृष्णतमोत्तमो ऽतिमहतो यस्तेजतेजात्मकः
लोकालोकधराधरः श्रियपतिः प्राणप्रविष्टालयः ।
कर्ता कारणमव्ययो ऽव्ययमसौ व्यापी विभक्ताविदम
विष्णुर्भावमयो विभक्तविषयैर्विश्वेश्वरो विश्ववित् ।।१६.४३ ।।
! एष तत्त्ववरः परापरमयस्तेजः परस्थानदः
बुद्ध्या भावनभावयेन्द्रियमनो देहान्तरालोकयन् ।
हृत्पद्मायतनस्थितः स पुरुषो निश्वासमुच्छ्वासदः
नादस्तस्य सदा सदा नदति तं नादोपरिष्ठा हरः ।।१६.४४ ।।
यस्तेजस्तेजते ऽजो बहुनिविडघनो ग्रन्थिमालोपगूड्हः
मूर्तिर्मूर्तानुसारी बहुकरणभृतं कारणाद्देहबन्धः ।
भित्त्वा ग्रन्थिं सपाशं विषमिव विषयं त्यक्तसङ्गैकभावाः
पश्यन्त्येते तमीशं गुणकलरहितं निर्विकारं प्रकाशम् ।।१६.४५ ।।
यो ऽसौ तेजान्तरात्मा कमलपुटकुटीसंकटस्थानलीनः
इन्दोर्भासानुरूपी विमलदलसदाच्छादितः कर्णिकायाम् ।
तत्र स्थाने स्थितो ऽसौ त्रिभुवननिलयः सर्वभूताधिवासः
आकाशादूर्ध्वतत्त्वस्थितविकसकलासंहतो मुक्तबन्धः ।।१६.४६ ।।
एतानि तत्त्वान्यखिलानि देवि
! संक्षेपतः कीर्तितः पञ्चभेदः ।
श्रोतुं किमन्यद्विजिगीषितार्थम
संसारमोक्षेण च तत्परो ऽस्ति ।।१६.४७ ।।
देव्युवाच ।
तुष्टास्मि देव मम संशयमद्य नष्टम
अद्य प्रसन्नपरमेश्वर ईश्वर त्वम् ।
अद्य श्रुतं त्वयि च पुण्यफलप्रभावम
पूर्णानि चाद्य मम इष्टमनोरथानि ।।१६.४८ ।।
अज्ञानपङ्कघनमध्यनिलीयमानाम
उत्तारयेश सकलार्तिविनाशनाय ।
सर्वेश तत्त्वपरमार्थ नमो नमस्ते
अद्यापि तृप्तिरिह नास्ति ममापि शम्भो ।।१६.४९ ।।
पीत्वामृतं चोत्तमवक्त्रजातम
आख्याहि दानं फलधर्मसारम् ।
संसारपारं परमं नयस्व ।।१६.५० ।।

 ।।इति वृषसारसंग्रहे ऽध्यात्मनिर्णयो नामाध्यायः षोडशमः ।।




देव्युवाच ।
पृथग्दानस्य इच्छामि श्रोतुं मां दातुमर्हसि ।
अन्नवस्त्रहिरण्यानां गोभूमिकनकस्य च ।।१७.१ ।।
भगवानुवाच ।
! सुसंस्कृतमन्नमतिप्रदद्यात
! घृतप्रभूतमवदंशयुक्तम् ।
घृतप्रपक्वं सुकृतं च पूपं
सितेन खण्डेन गुडेन युक्तम् ।।१७.२ ।।
मार्गं खगश्चोदकजङ्गमश्च
दद्याद्वटं नागरवंशमूलम् ।
शाकं फलं चाम्लमधूरतिक्तम
पानं पयः शीतसुगन्धतोयम् ।।१७.३ ।।
दधि प्रदद्याद्गुडमिश्रितं च
मृणालशालूकवनालका च ।
सदक्षिणालेपपवित्रपुष्पम
श्रद्धान्वितः सत्कृतया प्रणम्य ।।१७.४ ।।
प्रयाति लोकं जगदीश्वरस्य
विमानयानैः सहितो ऽप्सरोभिः ।
एकैकसिष्टस्य सहस्रवर्षम
अन्नप्रदो मोदति देवलोके ।।१७.५ ।।
च्युतश्च मर्त्ये स भवेद्धनाड्ह्यः
कुलोद्गतः सर्वगुणोपपन्नः ।
यशः श्रियं सर्वकलज्ञता च
भवेत्स भोगी सकलत्रपुत्रः ।।१७.६ ।।
दद्याद्दरिद्रः कृपणार्तदीनो
बालागदत्वातुरमागतानाम् ।
तृष्णाबुभुक्षागतिकागतानाम
दत्त्वा सधर्मस्य फलं कनिष्ट ।।१७.७ ।।
वाणिज्यधर्मादिफलाश्रितानाम
धर्मो हि तस्य न च निर्मलो ऽस्ति ।
तोयं च दद्याल्लघुपूर्णकम्भम
शीतं सुगन्धं परिवारितं च ।।१७.८ ।।
स याति लोकं सलिलेश्वरस्य
न तस्य जन्मानितृषाभिभूतः ।
उपानहं यो ददति द्विजाय
सुशोभनं तैलसुदी सुरपितं च ।।१७.९ ।।
ते यान्ति लोकममराधिपस्य
यमालयं कष्टपथान यान्ति ।
प्रक्षीणपुण्या पुनरत्र लोके
जातो भवेद्दिव्यकुलोपपन्नः ।।१७.१० ।।
धनैः समृद्धोधोपतित्वताश्च
रथाश्च नागा प्रभवन्ति तस्य ।
वस्त्रप्रदानेन भवन्ति देवि
रूपोत्तमसर्वकलज्ञतां च ।।१७.११ ।।
समृद्धिसौभाग्यगुणान्विताश्च
स्वर्गच्युतास्ते पुरुषा भवन्ति ।
वस्त्रप्रदानाभिरतस्य पुंसः
अन्यत्प्रवक्ष्यामि ततः प्रशस्ताम् ।।१७.१२ ।।
वस्त्रं तु लोकेष्वतिपूजनीयम
वस्त्रं नराणां त्वतिमाननीयम् ।
वस्त्रं तु भूयो न च मानलाभः
पराभवश्चाति जुगुप्सनश्च ।।१७.१३ ।।
तस्माद्धि वस्त्रं सततं प्रदेयम
यशः श्रियः स्वर्गसमान्तलाभम् ।
यावन्ति सूत्राणि भवन्ति वस्त्रे
तावद्युगं गच्छन्ति सोमलोकम् ।।१७.१४ ।।
पुण्यक्षयाज्जायति मृत्युलोके
वस्त्रप्रभूते धनधान्यकीर्णो ? ।
सुरूपसौभाग्ययशशिवनश्च
विद्याधरो लोकप्रभुत्वताश्च ।।१७.१५ ।।
द्विजेभ्यच्छत्रं सुकृतं प्रदद्यात
वर्षातपत्रं दृड्हशोभनं च ।
अङ्गारवर्षत्रषु खड्गमाद्यम
असंशयं त्रायति याम्यमार्गे ।।१७.१६ ।।
स्वर्गं च यान्ति ग्रहनायकश्च
स वर्षकोट्यायुतमन्तकाले ।
जायन्ति ते मानुषमर्त्यलोके
गृहोत्तमे भोगपतिर्भवन्ति ।।१७.१७ ।।
कृत्वा मठं शोभनविप्रदाता
द्रव्येण शुद्धेन तु पूजयित्वा ।
स याति देवेन्द्रसदं यथेष्टम
सवर्षकोटिशतदिव्यसंख्यैः ।।१७.१८ ।।
तदन्तकाले यदि मानुषत्वम
जायन्ति ते सप्तमहीप्रभोक्ता ।
स सप्तरथ्यत्रयसम्प्रयुक्ता
बलाधिको यज्ञसहस्रकर्ता ।।१७.१९ ।।
भूमिप्रदाता द्विजहीनदीनम
संमृद्धसस्यो जलसंनिकृष्त ।
स याति लोकममराधिपस्य !
विमानयानेन मनोहरेण ।।१७.२० ।।
मन्वन्तरं यावदभुक्तभोगान
तदन्तकाले च्युतमर्त्यलोके ।
स जवमुखण्डाधिपतिर्भवेत
वीर्यान्वितो राजसहस्रनाथः ।।१७.२१ ।।
स चैलघण्टां कनकाग्रशृङ्गाम
दोग्धीं सवत्सां पयसां द्विजानाम् ।
दत्त्वा द्विजेभ्यः समलङ्कृतानाम
प्रयान्ति लोकं सुरभीसुतानाम् ।।१७.२२ ।।
यावन्ति रोमाणि भवन्ति गावः
तावद्युगानामनुभूयभोगान् ।
तस्माच्च्युता मर्त्यमहीभुजास्ते
सहस्रराजानुगतो महात्मा ।।१७.२३ ।।
सुवर्णकांस्यायसरौप्यदाता
ताम्रप्रवालामणिमौक्तिकाद्यान् ।
दत्त्वा द्विजेभ्यो वसुसाध्यलोके
प्राप्नोति वर्षं दशपञ्चकोट्यो !  ।।१७.२४ ।।
भुक्त्वा यथेष्टं क्रमदेवलोकान
च्युतं च मर्त्ये स भवेन्नरेन्द्रः ।
सुदुर्जयः शक्रसहस्रजेता
सुदीर्घमायुश्च पराक्रमश्च ।।१७.२५ ।।
यत्प्रेक्षणं दर्शयितुं प्रदाता
सुरूपसौभाग्य फलं लभेत ।
तृणाशनामूलफलाशनेन
लभेत राज्यानि कण्टकानि ।।१७.२६ ।।
लभेतपर्णाशनस्वर्गवासम
पयः प्रयोगेन च देवलोके ।
शुश्रूषणो यो गुरवे च नित्यम
विद्याधरो जायति मर्त्यलोके ।।१७.२७ ।।
दद्याद्गवां धासतृणस्य मुष्टिः
गवाड्ह्यतां जायति मर्त्यलोके ।
श्राद्धं च दत्त्वा प्रयतो द्विजाय
समृद्धसन्तान भवेद्युगान्ते ।।१७.२८ ।।
अहिंसको जायति दीर्घमायुः
कुलोत्तमं जायति दीक्षितेन ।
कालत्रयं स्नानकृतेन राज्यं
पीत्वा च वायुस्त्रिदशाधिपत्वम् ।।१७.२९ ।।
अनश्नतायाः फलमीशलोके
तृप्तिर्भवेत्तोयप्रदानशीलः ।
अन्नप्रदाता पुरुषः समृद्धः
स सर्वकामा लभतीह लोके ।।१७.३० ।।
श्रद्धामतिर्यः प्रविशेद्धुतासनं !
स याति लोकं प्रपितामहस्य ।
सत्यं वदेद्यो ऽपि च धर्मशीलो
मोदत्यसौ देवि सहाप्सरोभिः ।।१७.३१ ।।
रसास्तु षड्यो परिवर्जयन्ति
अतीव सौभाग्य लभेत साध्वी ।
दानेन भोगानतुल्यं लभेत
चिरायुतां याति हि ब्रह्मचर्यात् ।।१७.३२ ।।
धनाड्ह्यतां यान्ति हि पुण्यकर्मान
मौनेन  । आज्ञा लभते अलङ्घ्याम् ।
प्राप्नोति कामं तपसः सुतप्तं
कीर्तिर्यशः स्वर्गमनन्तभोगम् ।।१७.३३ ।।
आयुः श्रियारोग्यधनप्रभुत्वं ।
ज्ञानादिलाभं तपसा लभेत ।।१७.३४ ।।
त्रैलोक्याधिपतित्वशक्रमगमत्कृत्वा तपो दुष्करम
यक्षेशो ऽपि तपः प्रभावगुरुणा गुह्याधिपत्वं महत् ।
रक्षेशो ऽपि बिभीषणस्त्वमरतां प्राप्तस्तपस्यैव तु
रुद्राराधनतत्परास्तपफलात्नन्दीगणत्वं गतः ।।१७.३५ ।।
ज्ञानं द्विजान्तपसो आह विष्णुः
क्षत्रं तपोरक्षणमाह सूर्य ।
वैश्यं तपश्चाञ्जनमाह वायुः
शूद्रं हि शिल्पं तप आह इन्द्रः ।।१७.३६ ।।
रणोत्सहं क्षत्रिययज्ञमिष्टं
वैश्यं हविर्यज्ञमुदाहरन्ति ।
शूद्रस्य यज्ञः परिचर्यमिष्टं
यज्ञं द्विजानां जपमुक्तमोक्षम् ।।१७.३७ ।।
देव्युवाच ।
स्वमांसरुधिरं दानं दानं पुत्रकलत्रयोः ।
किं प्रशस्यं महादेव तत्त्वं वक्तुमिहार्हसि ।।१७.३८ ।।
महेश्वर उवाच ।
स्वमांसरुधिरं दानं प्रशंसन्ति मनीषिणः ।
श्रूयतां पूर्ववृत्तानि संक्षिप्य कथयाम्यहम् ।।१७.३९ ।।
उशीनरस्तु राजर्षिः कयो ?तार्थे स्वकान्तन्तु?  ।
त्यक्त्वा स्वर्गमनुप्राप्तः परार्थे परतत्परः ।।१७.४० ।।
पुत्रमांसं स्वयं छित्वा अग्निदत्तं पुरानघे ।
तेन दानप्रभावेन अलर्कस्त्रिदिवं गतः ।।१७.४१ ।।
स्वदानदानेन मुदा स पुत्र
अपुत्रभूतस्य च पुत्र जातः ।
स्वर्गे स्वयं चोक्वय भोगलाभं
प्राप्तो महद्दानय?्ल प्रभावात् ।।१७.४२ ।।
यादवश्चार्जनो देवि दत्त्वा खण्डवभाजनम् ।।१७.४३ ।।
तपनस्य प्रसादेन सप्तद्वीपेश्वरो भवेत् ।
हरिणा च शिरो भित्वा दत्तं मे रुधिरं पुरा ।।१७.४४ ।।
प्रतीच्छितं कपालेन ब्रह्मसम्भवजेन मे ।
दिव्यवर्षसहस्राणि धारा तस्य न छिद्यते ।।१७.४५ ।।
परितुष्टो ऽस्मि तेनाहं कर्मणानेन सुन्दरि ।
वरं दत्तं मया देवि पुराणपुरुषो ऽव्ययः ।।१७.४६ ।।
अक्षयं वलमूर्जं च अजरामरमेव च ।
ममाधिकं भवेद्विष्णुर्माम यित्वम्विजेष्यसि ।।१७.४७ ।।
एवमादीन्यनेकानि मयोक्तानि जनार्दने ।
निष्कम्प निश्चलमनः स्थाणुभूत इव स्थितः ।।१७.४८ ।।
द?चिः स्वतनुं दत्त्वा विबुधानां वरानने ।
भुक्त्वा लोकान्क्रमात्सर्वान्शिवलोके प्रतिष्ठितः ।।१७.४९ ।।
जामदग्निर्महीं दत्त्वा काश्यपाय महात्मने ।
इहैव स यालं भोक्ता देवराज्यमवाप्स्यति ।। १७.५० ।।
दत्त्वा गो सकलं देवि व्यासस्यामिततेजसः ।
युधिष्ठिर महीयास देहस्त्रिदिवद्भतः ।।१७.५१ ।।
सत्यनामः ? (भीमः?) स्वकं भर्ता दत्त्वा नारादसत्कृतम् ।
दानस्यास्य प्रभावेन अक्षयं त्रिदिवद्भतः ? ।।१७.५२ ।।
चतुःषष्ठिसहस्ताणि गवां दत्त्वा द्विजन्मने ।
दुर्योधनमहीया?ओ गतः स्वर्गमनन्तकम् ।।१७.५३ ।।
वासुकिस्सर्पराजेन्द्रो दत्त्वा विप्रसुसंस्कृतम् ।
रत्कारुश्च ? साभान्या सर्वे नागविमोक्षिताः ।।१७.५४ ।।
गोभूमिकनकादीनां दानं कन्यसमुच्यते ।
भृत्यपुत्रकलत्राणां दानं मध्यममुच्यते ।।१७.५५ ।।
स्वदेहं पिसितादीनां दानमुत्तममुच्यते ।
एतत्सर्वं यदा दानं तद्दानमुत्तमोत्तमम् ।।१७.५६ ।।
जावज्जन्मसहस्राणि भोक्ता भवति कन्यसः ।
शतजन्मसहस्राणि भोक्ता भवति मध्यमः ।।१७.५७ ।।
उत्तमः पलभोक्ता (फल?) वि ? जन्मकोटिशतत्रयम् । 
परार्धद्वयजन्मानां भोक्ता वै चोत्तमोत्तमः ।।१७.५८ ।।
भूतानामनुकम्पया यदि धनं दाता सदान्वर्षिने ।
दीनान्वकृयणेष्वनाथमलिनेश्वानादिनि?? च ।।१७.५९ ।।
यद्येव कुरुते सदार्तिहरणं श्रद्धान्वितौ भक्तिमान् ।
तस्यानन्तयालं वदन्ति विबुधांस्स यस्य सन्दर्शनात् ।।१७.६० ।।

 ।।इति वृषसारसंग्रहे दानधर्मविशेषं नाम सप्तादशमो ऽध्यायः ।।




देव्युवाच ।
भुक्त्वा तु भोगान्सुचिरं यथेष्टं
पुण्यक्षयान्मर्त्यमुपागतानाम् ।
चिह्नानि तेषां कथयस्व मे ऽद्य
यथाक्रमं कर्मफलं विशेषात् ।।१८.१ ।।
महेश्वर उवाच ।
सदान्नदाता कृपणार्तिदीनां
स वर्षकोट्यायुतमीशलोके ।
भुक्त्वा च भोगान्सममप्सरोभिः
प्रक्षीणपुण्यः पुनरेति मर्त्यम् ।।१८.२ ।।
जायन्ति दिव्येषु कुलेषु पुंसः
सस्त्रीसमृद्धे बहुभृत्य ।
पूर्णे गौरव? श्वरन्नादि धना
कुलेषु ऋषो ?ज्ज्वलकान्तिसमायुतं च ।। १८.३ ।।
वस्त्रं सुसत्कृत्य द्विजस्य दानात
स्वर्गेषु मोदन्ति स वर्षकोट्यः ।
पुनश्च ते मर्त्यमुपागताश्च
चिह्न?आह?क्रीयवमाप्नुवन्ति ।।१८.४ ।।
कूपप्रयापुष्करणी प्रदाता
स लोकमाप्नोति जलेश्वरस्य ।
ततस्स तस्माच्च्युतिमाप्य लोका
अखीसुतृप्तेषु कुलेषु जायेत् ।।१८.५ ।।
रन्निप्रमाणादपि हेमदानात
सुरेन्द्रलोकं समवाप्नुवन्ति ।
तस्माच्च्युतो मर्त्यमुपागतानं
चिह्न?? (्सज?) द्वि? नधान्यलक्ष्याः ।।१८.६ ।।
अदूष्य भूमीवरविप्रदानात
स लोकमाप्नोति सुरेश्वरस्य ।
भुक्त्वा तु भोगान्च्युत मर्त्यलोके
चिह्नं लभेद्वै विषयाधिपत्वम् ।।१८.७ ।।
द्विजस्य सत्कृत्य तिलप्रदाता स
लोकमाप्नोति च केशवस्य ।
भ्रष्टस्ततो मर्त्यमुपागतस्तु
चिह्नं लभेदक्षयमर्थलाभम् ।।१८.८ ।।
गदा ? स्व?अयां विधिवद्द्विजानाम
दत्त्वा च गोकोलमवाप्नुवन्ति ।
कप्लावसाने समुपेत्य मर्त्ये
चिह्नङ्सवाड्ह्यं शतगोयुतं च ।।१८.९ ।।
स्वर्गं सतानां पुरुषस्य चिह्नं
वनाड्ह्यता श्री मुखभोगलाभम् ।
आयुर्यशोरूपकलत्रपुत्रम
सम्यङ्विभूति कुलकीर्तिमर्थम् ।।१८.१० ।।
दाना?(ष्ट?)भूञ्चो?त्तमकीर्तनन्ते
चिह्नं च लोकं च समासतो मे ।
शृणोतु देवी निरयागतानां
चिह्नं च कर्मं च विपाकतां च ।।१८.११ ।।
हत्वा च विप्रं मनसा च वाचा
स याति पारं निरयस्य घोरम् ।
अशीतिकल्पं निरये क्रमेण
भुक्त्वा पुनस्तिर्य शतायुतानाम् ।।१८.१२ ।।
जयन्ति ते मानुषहीनविद्या
प्रत्यन्तवामाः कुलवित्तहीनाः ।
नित्यं च तस्याक्षयरोगपीडा
इदन्तु चिह्नं द्विजजीवहर्तुः ।।१८.१३ ।।
पीत्वा च मद्यं द्विजः ? कामतो वा
आघ्राति गध्वं स्वमनीषिकेण ।
स याति घोरं नरकमसह्यं
यावच्च कल्पं दश अत्र भुक्त्वा ।।१८.१४ ।।
तीर्यं च सर्वमनुभूय??
स्वं स कष्टकष्टेन मनुष्यजन्वा ।
चण्डालशौनश्वयचन्वमेति
श्यामं च ताल भवतीह चिह्नम् ।।१८.१५ ।।
निन्दन्ति ये वेदसस्नूय जिह्वा
यः कूटसाक्षी स च खल्वला?औ ।
सुहृद्वधामृत्युशतं हि गर्भे
गर्हाशनोच्छिष्टभुजो भवन्ति ।।१८.१६ ।।
स्तैन्यस्तु यैः कुर्वति पापसत्त्वम
ते पापदोषान्नरकं व्रजन्ति ।
मन्वन्तरादीन्यनुभूयदुःखम
पुनश्च तिर्यक्शतशो ऽनुभूयात् ।।१८.१७ ।।
मानुष्यजन्मेषु च दुःखभागी
स्तेनेयमायाति पुनश्च मूड्हः ।
सुवर्णचौरकुनखत्वचिह्नम
विशीर्णगात्रो रजतापहारी ।।१८.१८ ।।
ताम्रापहारि स्फटिताग्रपाणीर
लोहापहारी भुजच्छेदचिह्नं ।
कांसापहारी करभग्नचिह्नम
हृत्वा चरीति त्रपुसीसकानाम् ।।१८.१९ ।।
नासौष्ठकर्णश्रवणस्य छेदः
चिह्नं नृणां वस्त्रहरं कुचेलः ।
धान्यापहारी भवत्येङ्गहीनः
दीपोपहारी भवत्यन्धचिह्नम् ।।१८.२० ।।
निर्वापहा काण भवेत चिह्नम
यः स्त्री हरेत्सो ऽपि जितः स्त्रिया स्यात् ।
सस्यापहारी भवतेन्नहीनः
हृत्वायुधयन्त्रहतत्वचिह्नं ।।१८.२१ ।।
अन्नापहारी परदत्तभोक्ता
हृत्वा तु गावः स भवेत्दरिद्रः ।
हरिहरेत्तद्धरिणा दहन्ति
हृत्वा तु मेषानजगर्दभश्च ।।१८.२२ ।।
स भारभृज्जीवमुदाहरन्ति
रत्नापहारी अनपत्यता च ।
छत्रापहारी अपवित्रता च
हृत्वा च बीजं स भवेदबीजः ।।१८.२३ ।।
गोधूमशालियवमुद्गमाषान
हृत्वा मसूरं विलयं व्रजन्ति ।
कामातुरो मातरमातृपुत्री
मातृश्वसाङ्गच्छति मातुलानीम् ।।१८.२४ ।।
राजाङ्गनां पुत्रसुतां स्नुषां च
प्रव्राजिनीं ब्राह्मणीमन्त्यजां च । 
अजाश्वमेषसुरभीसुताश्च
यत्कामयेत्तेषु विमूड्हचेतः ।।१८.२५ ।।
स याति कृच्छ्रं नरकं सुघोरं
स वर्षकोटीशतशो भ्रमित्वा ।
तीर्यञ्च भूयः शतशोव्यतीत्य
कष्टेन वै जायति मानुषत्वम् ।।१८.२६ ।।
हीनाङ्गतादीनशरीरताश्च
यो मातृगामी स भवेदलिङ्गः ।
मातृस्वसातल्पगवानलिङ्गा
लिङ्गे ऽपरोधः सुतपुत्रिकामः ।।१८.२७ ।।
स्नुषां च यः सेवति रक्तमेही
दौः चर्मताश्च द्विजसुन्दरीषु ।
राजाङ्गनायासु च लिङ्गच्छेदः
प्रव्राजिनी कामुकमूत्रकृच्छ्रम् ।।१८.२८ ।।
सव्याधिलिङ्ग लभतेन्त्यजासु
विलीनलिङ्गः पशुयोनिगामी ।
जायन्ति ते मूषिकधान्यचौरी
क्षीरं हरेद्वायसतां प्रयाति ।।१८.२९ ।।
हंसापहारी स भवेन्निहंसः
श्वानत्वमायाति रसापहारी ।
हृत्वा च सूचीन्तु भवेत्स दंशः
हृत्वा तु सर्पिर्वृषतां प्रयाति ।।१८.३० ।।
मांसं तु हृत्वा स भवेत गृध्रः
तैलापहारी खगतां प्रयाति ।
गुडं च हृत्वा गुडिका भवन्ति
शाकापहारी स भवेन्मयूरम् ।।१८.३१ ।।
हृत्वा पशुं पङ्गुरजायतेहः
चित्रत्वमायाति सुवस्त्रहारी ।
हृत्वा दुकूलं स च सारसत्त्वं
क्षौमं च हृत्वा स च दुर्बलत्वम् ।।१८.३२ ।।
ऊर्नानि वस्त्राण्यपहृत्य मेषः
छुछुन्दरी जायति गन्धहारी ।
ब्रह्मस्वमल्पमपहृत्य भोक्ता
स गृध्र उच्छिष्टभुजो भवन्ति ।।१८.३३ ।।
पादेन यः स्पर्शयते द्विजाङ्घ्रिं
तच्छीतरक्तं चरणौ भवेत ।
पादेन यः स्पर्शयते च गावः
स पादरोगान्विविधांल्लभेत ।।१८.३४ ।।
यो मातरः ताडयते पादेन
पादे तदीये कृमयः पतन्ति ।
पादात्पृशेद्यः पितरं दुरात्मा
सूनोन्नपादः स भवेत्परत्र ।।१८.३५ ।।
पदात्पृशेत्तोयमनादरेण
सश्लीपदीपादयुगे भवेत ।
पादेन य स्पर्शयते हुताशं
स चाग्निपादः सततं भवेत ।।१८.३६ ।।
पादेन यश्चार्यमुपस्पृशेत
स पादच्छेदं बहुशो लभेत ।
ग्रन्थापहारी स भवेत मूकः
दुर्गन्धवक्त्रः परिछिद्रवादी ।।१८.३७ ।।
पैशुन्यवादी स च पूतिनासाम
अनम्रवक्त्रस्त्वनृतापवादी ।
पारुष्यवक्ता मुखपाकरागी
असत्प्रलापी स च दन्तरोगः ।।१८.३८ ।।
स्तीक्ष्णप्रदायी स च वक्रनास
सम्भिन्नवक्ता स च कण्ठरोगी ।
क्रुद्धेक्षणः पश्यति यस्तु विप्रं
तीव्राक्षिरोगी स तु जायते हि ।।१८.३९ ।।
प्रद्वेषयालोकयते ऽतिथीन्य
उत्पादिताक्षिस्स भवेत्परत्र ।
वैरूप्य चक्षुस्त्वतिसूक्ष्मचक्षुः
स जायते केकरपिङ्गयक्षुः ।।१८.४० ।।
गर्ताक्षिकादीनि विपाण्डुराणि
नेत्रामयान्येव च पापदोषात् ।
शृण्वन्ति ये पापकथां प्रशस्तां
तां कर्णसर्पिः परिपीडियेत ।।१८.४१ ।।
शृण्वन्ति निन्दां हरिशर्वयोर्यः
स कर्णशूलेन तु जीवती वा ।
मातापितॄणां शृणुते ऽपवादः
स कर्णसाफेन विनाशमेति ।।१८.४२ ।।
शृणोति निन्दां गुरुविप्रजा यः
स कर्णपूयं स्रवते सरक्तम् ।
विरूप्यदारिध्रकुलाधमेषु
अनिष्टकर्मभृतिजीवनाश्च ।।१८.४३ ।।
अकीर्तनं दर्शनवर्जनं च
श्वापाकतो श्वादिषु जायते सः ।
एतानि चिह्नं निरयागतानां
मानुष्यलोके कुकृतस्य दृष्टम् ।।१८.४४ ।।
समासतः कीर्तित एव देवि ।
यथैव मुक्तिस्त्विह कर्मभङ्गः ।।१८.४५ ।।
मातापित्रोघतो यासुतदुहितृवहा भ्रातृगम्भीरवेगा
भार्यावर्ता विवर्ता कुटिलगतिवधुर्बान्धवोर्मीतरङ्गा ।
कामक्रोधोभकूला करिमकरझषा ग्राहकामा भयन्ते
मृत्योराख्यार्णवे ऽस्मिन्न शरणविवशाकालदृष्टो प्रयाति ।।१८.४६ ।।
नित्यं येन विना न याति दिवसं पञ्चत्वमापद्यते
त्यक्त्वा देह वनान्तरेषु विषमे श्वानश्रिगालाकुले ।
बन्धुः सर्वनिवर्तते गतदया धर्मैक तत्र स्थितः
तस्माद्धर्मपरो न चान्यः सुहृदः सेवेत्परत्रार्थिनः ।।१८.४७ ।।

 ।।इति वृषसारसंग्रहे पूर्वकर्मविपाकचिह्नाष्टादशो ऽध्यायः ।।




विगतराग उवाच ।
क्रियासूक्ष्मो महाधर्मः कर्मणा केन प्राप्यते ।
अल्पोपायं नरार्थाय पृच्छामि कथयस्व मे ।।१९.१ ।।
अनर्थयज्ञ उवाच ।
अल्पोपायं महाधर्मं कथयामि द्विजोत्तम ।
सुखेन लभते स्वर्गं कर्मणा येन तच्छृणु ।।१९.२ ।।
लोकानं मातरो गावो गोभिः सर्वं जगद्धृतम् ।
गोमयममृतं सर्वं जातं सर्वशिवेच्छया ।।१९.३ ।।
सर्वदेवमयी गावः सर्वदेवमयो द्विजः ।
सर्वदेवमयो भूमिः सर्वदेवमयः शिवः ।।१९.४ ।।
तस्माद्गावः सदा सेव्या धर्ममोक्षार्थसिद्धिदा ।
परिचर्या यथाशक्त्या ग्रासवासजलादिभिः ।।१९.५ ।।
ताडयेन्नातिवेगेन वाचयेन्मृदुनाचरेत् ।
पालयन्तर्पनाद्येषु भग्नोद्विग्नेषु यत्नतः ।।१९.६ ।।
व्याधिवनपरिक्लेश ओषधोपक्रमश्चरेत् ।
कण्डूयनं च कर्तव्यं यथासौख्यं भवेद्गवाम् ।।१९.७ ।।
गवां प्रदक्षिणं कृत्वा श्रद्धाभक्तिसमन्वितः ।
सागरान्ता मही सर्वा न्प्रदक्षिणीकृता भवेत् ।।१९.८ ।।
पृष्टसंस्पर्शनाद्यञ्च श्रद्धया यदि मानवः ।
अहोरात्रकृतं पापं नश्यते नात्रसंशयः ।।१९.९ ।।
लाङ्गूलेनोद्धृतं तोयं मूर्द्ध्ना गृह्णाति यो नरः ।
यावज्जीव कृतं पापं नश्यते नात्र संशयः ।।१९.१० ।।
विधिवत्स्नापयेद्गांश्च मन्त्रयुक्तेन वारिणा ।
तेनाम्भसा स्वयं स्नात्वा सर्वपापक्षयो भवेत् ।।१९.११ ।।
व्याधिविघ्नमलक्ष्मीत्वं नश्यते सद्य एव च ।
मृतापत्याश्च गावाश्च स्नानमेव प्रशस्यते ।।१९.१२ ।।
गवां शृङ्गोदकं गृह्य मूर्ध्नि यो धारयेन्नरः ।
स सर्वतीर्थस्नानस्य फलं प्राप्नोति मानवः ।।१९.१३ ।।
ग्रासमुष्टिप्रदानेन गोषु भक्तिसमन्वितः ।
अग्निहोत्रं हुतं तेन सर्वदेवाः सुतर्पिताः ।।१९.१४ ।।
चत्वारः स्तनधारास्तु यस्तु मूर्ध्ना प्रतीच्छति ।
स चतुःसागरं गत्वा स्नानपुण्यफलं लभेत् ।।१९.१५ ।।
गवार्थं यस्त्यजेत्प्राणान्गोग्रहेषु द्विजोत्तम ।
कल्पकोटिशतं दिव्यं शिवलोके महीयते ।।१९.१६ ।।
च्युतभग्नादिसंस्कारं सर्वं यः कुरुते नरः ।
भार्याकोटिशतं दानं यत्फलं परिकीर्तितम् ।। १९.१७ ।।
तत्फलं लभते मर्त्यः शिवलोकं च गच्छति ।
शिवलोकपरिभ्रष्टः पृथिव्यामेकराड्भवेत् ।।१९.१८ ।।
समासतः समाख्यातं यथातत्त्वं द्विजोत्तम ।
न शक्यं विस्तराद्वक्तिउं गोमहात्म्यसमुत्तमम् ।।१९.१९ ।।
विगतराग उवाच ।
देवाः रष्टविधाः प्रोक्ताः तिर्यक्पञ्चविधः स्मृतः ।
मानुष्यमेकमेवाहुश्चातुर्वर्ण्यः कथं भवेत् ।।१९.२० ।।
अनर्थयज्ञ उवाच ।
पूर्वकल्पसृजत्येष विष्णुना प्रभविष्णुना ।
एवं वर्णा द्विजश्चासीत्सर्वकल्पाग्रमग्रतः ।।१९.२१ ।।
सर्ववेदविदो विप्राः सर्ववेदविदस्तथा ।
तथा विप्रसहस्राणां यज्ञोत्साहमनो भवेत् ।।१९.२२ ।।
वृद्धविप्रसहस्राणां मतमाश्रित्य ब्राह्मणैः ।
कर्तुं कर्म समारब्धकर्मश्चापि विभज्यते ।।१९.२३ ।।
ऋत्वजत्वे स्थिताः केचित्केचित्संरक्षणे स्थिताः ।
अर्थोपार्जनयुक्तान्ये अन्ये शिल्पे नियोजिताः ।।१९.२४ ।।
एवं यज्ञविधानेन कर्तुमरेभिरे पुरा ।
यथोद्दिष्टेन कर्मेण यज्ञोत्साहमवर्तत ।।१९.२५ ।।
आगता ऋषयः सर्वे देवताः पितरस्तथा ।
अन्योन्यमब्रुवन्तत्र देवर्षिपितृदेवताः ।।१९.२६ ।।
यज्ञार्तमसृजद्वर्णं विधिना पातुहेतवः ।
एवमेव प्रवर्तन्तु भवतिर्द्विजसत्तमाः ।।१९.२७ ।।
इज्याध्याध्ययनसम्पन्ना ब्रह्मणा यत्र कल्पिताः ।
सुविप्रा विप्रतां यान्तु षड्कर्मानिरताः सदा ।।१९.२८ ।।
रक्षणार्तं तु ये विप्राः कल्पिताः शस्त्रपाणयः ।
कृतत्राणाय विप्राणां नित्यं क्षात्रव्रतोद्भवाः ।।१९.२९ ।।
अर्थोपार्जनमुद्दिश्य कल्पिता ये द्विजातयः ।
ते तु वैश्यत्वमायान्तु वार्तो आपणतोद्भवाः ।
वधबन्धनकर्मेषु शिल्पस्थानवधेषु च ।।१९.३० ।।
कल्पिता ये द्विजातीनां सर्वे शूद्रा भवन्तु ते ।
प्राजापत्यं ब्राह्मणानामीज्याध्ययनतत्पराम् ।।१९.३१ ।।
स्थानमैन्द्रं क्षत्रियाणां प्रजापालनतत्परम् ।
वैश्यानां वासवस्थानं वाणिज्यं कृषिजीविनाम् ।।१९.३२ ।।
शूद्राणां मरुतः स्थानं शुश्रूषानिरतात्मनाम् ।
महर्षिपितृदेवानां मतमाज्ञाय निश्चितः ।
एष संकल्पितो ब्रह्मा पद्मयोनिः पितामहः ।।१९.३३ ।।
संकल्पप्रभवाः सर्वे देवदानवमानवाः ।
पशुपक्षिमृगामुख्या यावन्ति जगसम्भवाः ।।१९.३४ ।।
भूतसंकल्पकर्ता य कल्पमासीद्द्विजोत्तम ।
कीर्तितानि समासेन किमन्यच्छ्रोतुमिच्छसि ।।१९.३५ ।।
विगतराग उवाच ।
किं तपः सर्ववर्णानां वृत्तिर्वापि तपोधन ।
यज्ञाश्चैव पृथक्त्वेन श्रोतुमिच्छामि तत्त्वतः ।।१९.३६ ।।
अनर्थयज्ञ उवाच ।
ब्राह्मणस्य तपो यज्ञाः  । तपः क्षात्रस्य रक्षणम् ।
वैश्यश्च तप वाणिज्य तपः शूद्रस्य सेवनम् ।।१९.३७ ।।
प्रतिग्रह धनो विप्रः क्षत्रियस्य धनुर्धनम् ।
कृषिर्धनं तथा वैश्यः शूद्रः शुश्रूषणं धनम् ।।१९.३८ ।।
आरम्भयज्ञः क्षत्रस्य हविर्यज्ञो विशस्तथा ।
शूद्रः परिचरो यज्ञो जपयज्ञो द्विजातयः ।।१९.३९ ।।
सत्य तीर्थ द्विजातीनां रण तीर्थं तु क्षत्रियाः ।
आर्या तीर्थं तु वैशानां ! शूद्रतीर्थं तु वै द्विजाः ।।१९.४० ।।
नास्ति विद्यासमो मित्रो नास्ति दानसमः सखा ।
नास्ति ज्ञानसमो बन्दुर्नास्ति यज्ञो जपः समः ।।१९.४१ ।।
धर्महीनो मृतस्तुल्यो देवतुल्यो जितेन्द्रियः ।
यज्ञतुल्यो ऽभयं दाता शिवतुल्यओ मनोन्मनः ।।१९.४२ ।।
विगतराग उवाच ।
दान यज्ञस्तपस्तीर्थं संन्यासं योग एव च ।
एतेषु कतमः श्रेष्ठः श्रोतुमिच्छामि कीर्तय ।।१९.४३ ।।
अनर्थयज्ञ उवाच ।
दानधर्मसहस्रेभ्यः यज्ञयाजी विशिष्यते ।
यज्ञयाजीसहस्रेभ्यस्तीर्थयात्री विशिष्यते ।।१९.४४ ।।
तीर्थयात्रिसहस्रेभ्यस्तपनिष्टो विशिष्यते ।
तपनिष्ठसहस्रेभ्यः श्रेष्ठः संन्यासिकः स्मृतः ।।१९.४५ ।।
संन्यासीनां सहस्रेभ्यः श्रेष्ठो यच्य जितेन्द्रियः ।
जितेन्द्रियसहस्रेभ्यः योगयुक्तो विशिष्यते ।।१९.४६ ।।
योगयुक्तसहस्रेभ्यः श्रेष्ठो लीनमनः स्मृतः ।
तस्मात्सर्वप्रयत्नेन आदौ मन विशोधयेत् ।।१९.४७ ।।
निगृहीतेन्द्रियग्रामः स्वर्गमोक्षौ तु साधनम् ।
विशिष्ठे त्विन्द्रियग्रामे तिर्यन्नरकसाधनम् ।।१९.४८ ।।
विगतराग उवाच ।
चराचराणां भूतानां कतमः श्रेष्ठ उच्यते ।
कथयस्व ममाद्य त्वं छेत्तुमर्हसि संशयम् ।।१९.४९ ।।
अनर्थयज्ञ उवाच ।
चराचराणां भूतानां तत्र श्रेष्ठो  । चराः स्मृताः ।
चराणां चैव सर्वेषां बुद्धिमान्श्रेष्ठ उच्यते ।।१९.५० ।।
बुद्धिमान्षु ! च सर्वेषु ततः श्रेष्ठ नराः स्मृताः ।
नराणां चैव सर्वेषां ब्राह्मणः श्रेष्ठ उच्यते ।।१९.५१ ।।
विद्वर्स्वपि च सर्वेषु कृतबुद्धिर्विशिष्यते ।
कृतबुद्धिषु सर्वेषु श्रेष्ठः कर्ता स उच्यते ।।१९.५२ ।।
कर्तृष्वपि च सर्वेषु ब्रह्मवेदी विशिष्यते ।
ब्रह्मवेदि परं ! विप्रः नान्यं वेद्मि परंतपः ।
स विप्रः स तपस्वी च स योगी स शिवः स्मृतः ।।१९.५३ ।।

 ।।इति वृषसारसंग्रहे दानयज्ञविशेषो नाम उनविंशतितमो ऽध्यायः ।।




विगतराग उवाच ।
पञ्चविंशति यत्तत्त्वं ज्ञातुमिच्छामि तत्त्वतः ।
कथयस्व ममाद्य त्वं छिद्यते येन संशयः ।।२०.१ ।।

---- तत्त्वनिर्णयम् ----

अनर्थयज्ञ उवाच ।
सर्वप्रत्यक्षदर्शित्वं कथं मां प्रष्टुमर्हसि ।
पृष्टेन कथनीयो ऽस्मि एष मे कृतनिश्चयः ।
शृणु ते सम्प्रवक्ष्यामि तत्त्वसद्भावमुत्तमम् ।।२०.२ ।।

---- पुरुषः-शिवः-ब्रह्मा (२५) ----

नादिमध्यं न चान्तं च यन्न वेद्यं सुरैरपि ।
अतिसूक्ष्मो ह्यतिस्थूलो निरालम्बो निरञ्जनः ।।२०.३ ।।
अचिन्त्यश्चाप्रमेयश्च अक्षराक्षरवर्जितः ।
सर्वः सर्वगतो व्यापी सर्वमावृत्य तिष्ठति ।।२०.४ ।।
सर्वेन्द्रियगुणाभासः सर्वेन्द्रियविवर्जितः ।
अजरामरजः शान्तः परमात्मा शिवो ऽव्ययः ।।२०.५ ।।
अलक्ष्यलक्षणः स्वस्थो ब्रह्मा पुरुषसंज्ञितः ।
पञ्चविंशः स विज्ञेयो जन्ममृत्युहरः प्रभुः ।।२०.६ ।।
कलाकलङ्कनिर्मुक्तो व्योमपञ्चाशवर्जितः ।
जलपक्षी यथा तोयैर्न लिप्येत जले चरन् ।
तद्वद्दोषैर्न लिप्येत पापकर्मशतैरपि ।।२०.७ ।।

---- प्रकृतिः (२४) ----

चतुर्विंशति यत्तत्त्वं प्रकृतिं विद्धि निश्चयम् ।
विकृतिश्च स विज्ञेयस्तत्त्वतः स मनीषिभिः ।।२०.८ ।।
प्रकृतिप्रभवाः सर्वे बुद्ध्यहंकार ।आदयः ।
विकृतिं प्रतिलीयन्ते भूम्यादि क्रमशस्तु वै ।।२०.९ ।।

---- मतिः-बुद्धिः (२३) ----

मतितत्त्व त्रयोविंश धर्मादिगुणसंयुतः ।
सत्त्वाधिकसमुत्पन्नबोद्धारं विद्धि देहिनः ।।२०.१० ।।

---- अहंकारः (२२) ----

द्वाविंशति अहंकारस्तत्त्वमुक्तं मनीषिभिः ।
भूतादि मम पञ्चाह रजाधिकसमुद्भवम् ।।२०.११ ।।

---- आकाशः (सुषिरत्वं) शब्दश{ }्च (२१-२०) ----

एकविंशति यत्तत्त्वं सुषिरं विद्धि भो द्विज ।
शब्दातीतं सुषिरत्वं सशब्दगुणलक्षणम् ।।२०.१२ ।।
सप्तस्वरास्त्रयो ग्रामा मूर्छनास्त्वेकविंशतिः ।
ताना{ ।}म{ ।}ेकोनपञ्चाशच्छब्दभेदस्तदादयः ।।२०.१३ ।।
एवमादीन्यनेकानि स्वरभेदा द्विजोत्तम ।
गान्धर्वस्वरतत्त्वज्ञैर्मुनिभिः समुदाहृतम् ।।२०.१४ ।।
वेणुमुरजतन्त्रीणां दुन्दुभीनां स्वनानि च ।
शङ्खकाहलकांस्यानां शब्दानि विविधानि च ।।२०.१५ ।।
आकाशधातु विप्रेन्द्र शृणु वक्ष्यामि ते दश ।
पायूपस्थोदर कण्ठ शङ्खलौ मुख नासिकौ ।।२०.१६ ।।
हृदिं च दशमं ज्ञेयं देह आकाशसम्भवः ।
पुनरन्यत्प्रवक्ष्यामि तच्छृणुष्व द्विजोत्तम ।।२०.१७ ।।
दश धातुगुणा ज्ञेयाः पञ्चभूतः पृथक्पृथक् ।
आकाशस्य गुणाः शब्दो व्यापित्वं छिद्रतापि च ।।२०.१८ ।।
अनाश्रयनिरालम्बमव्यक्तमविकारिता ।
अप्रतीघातिता चैव भूतत्वं प्रकृतानि च ।।२०.१९ ।।

---- वायुः स्पर्शश{ }्च (१९-१८) ----

आकाशधातोर्विप्रेन्द्र ततो वायुसमुद्भवः ।
शब्दपूर्वगुणं गृह्य वायोः स्पर्शगुणः स्मृतः ।।२०.२० ।।
शब्द पूर्वं मयाख्यातं शृणु स्पर्शं द्विजोत्तम ।
कठिनश्चिक्कणः श्लक्ष्णो मृदुस्निग्धखरद्रवाः ।।२०.२१ ।।
कर्कशः परुषस्तीक्ष्णः शीतोष्ण दश च द्वयम् ।
इष्टानिष्टद्वयस्पर्श वपुषा परिगृह्यते ।।२०.२२ ।।
प्राणो ऽपानः समानश्च उदानो व्यान एव च ।
नागकूर्मो ऽथ कृकरो देवदत्तो धनंजयः ।।२०.२३ ।।
दश वायुप्रधानैते कीर्तिता द्विजसत्तम ।
धनंजयो भवेद्घोषो देवदत्तो विजृम्भकः ।।२०.२४ ।।
कृकरः क्षुधकृन्नित्यं कूर्मोन्मीलितलोचनः ।
नाग उद्घाटनं पुष्यं करोति सततं द्विज ।।२०.२५ ।।
प्राणः श्वसति भूतानां निश्वसन्ति च नित्यशः ।
प्रयाणं कुरुते यस्मात्तस्मात्प्राण इति स्मृतः ।।२०.२६ ।।
अपनयत्यपानस्तु आहारं मनुजामधः ।
शुक्रमूत्रवहो वायुरपानस्तेन कीर्तितः ।।२०.२७ ।।
पीतभक्षितमाघ्रातं रक्तपित्तकफानिलम् ।
समं नयति गात्रेषु समानो नाम मारुतः ।।२०.२८ ।।
स्पन्दयत्यधरं वक्त्रं नेत्रगात्रप्रकोपनम् ।
उद्वेजयति मर्माणि उदानो नाम मारुतः ।।२०.२९ ।।
व्यानो विनामयत्यङ्गं व्यङ्गो व्याधिप्रकोपनः ।
प्रीतिविनाशकथितं वार्धिक्यं व्यान उच्यते ।।२०.३० ।।
दशवायुविभागे च कीर्तितो मे द्विजोत्तम ।
दशवायुगुणांश्चान्यां छृणु कीर्तयतो मम ।।२०.३१ ।।
वायोरनियम स्पर्शो वादस्थानं स्वतन्त्रता ।
बलं शीघ्रं च मोक्षं च चेष्टा कर्मात्मना भवः ।।२०.३२ ।।

---- तेजो रूपश{ }्च (१७-१६) ----

वायुनापि सृजस्तेजस्तद्रूपं गुणमुच्यते ।
शब्दस्पर्शसम ज्योतिस्त्रिगुणं समुदाहृतम् ।।२०.३३ ।।
शब्दः स्पर्शः पुरा प्रोक्तः शृणु रूपगुणं ततः ।
ह्रस्वं दीर्घमणु स्थूलं वृत्तमण्डलमेव च ।।२०.३४ ।।
चतुरस्रं द्विरस्रं च त्र्यस्रं चैव षडस्रकम् ।
शुक्लः कृष्णस्तथा रक्तो नीलः पीतो ऽरुणस्तथा ।।२०.३५ ।।
श्यामः पिङ्गल बभ्रुश्च नव रङ्गाः प्रकीर्तिताः ।
नवधा नवरङ्गानामेकाशीति गुणाः स्मृताः ।।२०.३६ ।।
तेजोधातु दश ब्रूमः शृणुष्वावहितो भव ।
कामस्तेजो क्षणः क्रोधो जठराग्निश्च पञ्चमः ।।२०.३७ ।।
ज्ञानं योगस्तपो ध्यानं विश्वाग्निर्दशमः स्मृतः ।
दश तेजोगुणांश्चान्यान्प्रवक्ष्यामि द्विजोत्तम ।।२०.३८ ।।
अग्नेर्दुर्धर्षताप्नोति तापपाकप्रकाशनः ।
शौचं रागो लघुस्तैक्ष्ण्यं दशमं चोर्ध्वभागिता ।।२०.३९ ।।

---- आपो रसश{ }्च (१५-१४) ----

ज्योतिसो ऽपि सृजश्चापः सरसो गुणसंयुतः ।
चतुर्गुणाः स्मृता आपः विज्ञेया च मनीषिभिः ।।२०.४० ।।
शब्दः स्पर्शश्च रूपं च रसश्च स चतुर्गुणः ।
रूपादिगुण पूर्वोक्त अधुनाथ रसं शृणु ।।२०.४१ ।।
कटुतिक्तकषायाश्च लवणाम्लस्तथैव च ।
मधुरश्च रसान्षड्वै प्रवदन्ति मनीषिणः ।।२०.४२ ।।
षड्रसाः षड्विभेदेन षट्त्रिंशगुण उच्यते ।
आपधातु दश त्वन्यान्शृणु कीर्तयतो मम ।।२०.४३ ।।
लाला सिङ्घाणिका श्लेष्मा रक्तः पित्तः कफस्तथा ।
स्वेदमश्रु रसश्चैव मेदश्च दशमः स्मृतः ।।२०.४४ ।।
दश आपगुणाश्चान्ये कीर्तयिष्यामि तान्शृणु ।
अद्भ्य शैत्यं रस क्लेदो द्रवत्वं स्नेहसौम्यता ।।२०.४५ ।।
जिह्वा विष्यन्दिनी चैव भौमान्यश्रवणाधमः ।

---- भूमिर{ }्गन्धश{ }्च (१३-१२) ----

आपश्चाप्यसृजद्भूमिस्तस्या गन्धगुणः स्मृतः ।।२०.४६ ।।
चतुरापगुणान्गृह्य भूमेर्गन्धगुणः स्मृतः ।
शब्दः स्पर्शश्च रूपं च रसो गन्धश्च पञ्चमः ।।२०.४७ ।।
आपःपूर्वगुणाः प्रोक्ता भूमेर्गन्धगुणं शृणु ।
इष्टानिष्टद्वयोर्गन्धः सुरभिर्दुरभिस्तथा ।।२०.४८ ।।
कर्पूरः कस्तुरीकं च चन्दनागरुमेव च ।
कुङ्कुमादिसुगन्धानि घ्राणमिष्टं प्रकीर्तितम् ।।२०.४९ ।।
विङ्मूत्रस्वेदगन्धानि वक्त्रगन्धं च दुःसहम् ।
जीर्णस्फोटितगन्धानि अनिष्टानीति कीर्तितम् ।।२०.५० ।।
भूमेर्धातु दश त्वन्यान्कथयिष्यामि तच्छृणु ।
त्वचं मांसं च मेदं च स्नायु मज्जा सिरा तथा ।।२०.५१ ।।
नखदन्तरुहाश्चैव केशश्च दशमस्तथा ।
दश त्वन्यान्प्रवक्ष्यामि शृणु भूमिगुणान्द्विज ।।२०.५२ ।।
भूमेः स्थैर्यं रजस्त्वं च काठिन्यं प्रसवात्मकम् ।
गन्धो गुरुश्च शक्तिश्च नीहारस्थापनाकृतिः ।।२०.५३ ।।
गुणधातुविशेषश्च उत्पत्तिश्च द्विजोत्तम ।
यथा श्रुतं मया पूर्वं कीर्तितं निखिलेन तु ।।२०.५४ ।।

---- बुद्धीन्द्रियाणि कर्मेन्द्रियाणि च (११-२) ----

वैकारिकमहंकारं सत्त्वोद्रिक्तात्तु सात्त्विकः ।
श्रोत्रं त्वक्चक्षुषी जिह्वा नासिका चैव पञ्चमी ।।२०.५५ ।।
बुद्धीन्द्रियाणि पञ्चैव कीर्तितानि द्विजोत्तम ।
हस्तपादस्तथा पायुरुपस्थो वाक्च पञ्चमः ।।२०.५६ ।।
श्रोत्रेण गृह्यते शब्दो विविधस्तु द्विजोत्तम ।
वेणुवीणास्वनानां च तन्त्रीशब्दमनेकधा ।।२०.५७ ।।
मुरज सौन्द पणवभेरीपटहनिस्वनम् ।
शङ्खकाहलशब्दं च शब्दं डिण्डिमगोमुखम् ।।२०.५८ ।।
कांसिकाहलमिश्रं च गीतानि विविधानि च ।
त्वचया गृह्यते स्पर्शः सुखदुःखसमन्वितः ।।२०.५९ ।।
मृदुसूक्ष्मसुखं स्पर्शः वस्त्रशय्यासनादयः ।
तीक्ष्णशस्त्रजलं शैत्य उष्णतप्तक्षतेक्षरः ।।२०.६० ।।
एवमादीन्यनेकानि ज्ञेयानीष्टं द्विजोत्तम ।
चक्षुषा गृह्यते रूपं सहस्राणि शतानि च ।।२०.६१ ।।
देवरूपविकाराणि नक्षत्रग्रहतारकाः ।
मानुषानां विकाराणि ग्रामं नगरपत्तनम् ।।२०.६२ ।।
वृक्षगुल्मलतानां च पशुपक्षिशरीसृपां ।
कृमिकीटपतङ्गानां जलजानामनेकधा ।।२०.६३ ।।
शैलदारवरोमाणि रूपाणि विविधानि च ।
धातुद्रव्यविकाराणि रूपाणि द्विजसत्तम ।।२०.६४ ।।
जिह्वया गृह्यते स्वादो हृद्याहृद्यो द्विजोत्तम ।
फलमूलानि शाकानि कन्दानि पिशितानि च ।।२०.६५ ।।
पक्वापक्व विशेषाणि दधिक्षीरघृतानि च ।
व्रीह्योषधिरसानां च मिश्रामिश्रमनेकधा ।।२०.६६ ।।
षट्कर्मप्रतिभेदेन रसभेदशत स्मृतम् ।
घ्राणेन गृह्यते गन्ध इष्टानिष्टो द्विजर्षभः ।।२०.६७ ।।
गुडाज्यं गुग्गुलुर्भष्मचन्दनागरुकं तथा ।
कस्तूरिकुङ्कुमादीनामिष्टो गन्धो मनोहरः ।।२०.६८ ।।
व्रणमूत्रपुरीषाणां मांसपर्युषितानि च ।
वातकर्मादिदुर्गन्ध अनिष्टः समुदाहृतः ।।२०.६९ ।।
हस्तेन कुरुते कर्म विविधानि द्विजोत्तम ।
माहेन्द्रं वारुणं चैव वायव्याग्नेयमेव च ।।२०.७० ।।
आग्नेयपवनादीनि कांस्यो लोहस्त्रपुस्तथा ।
अग्निकर्माण्यनेकानि यज्ञहोमक्रियास्तथा ।।२०.७१ ।।
सूर्यव्यजनवातेन मुखवातेन वै तथा ।
चमरचर्मवातेन वातयन्त्रं च वायवम् ।।२०.७२ ।।
वारुणं तोयकर्माणि कुरुते विविधानि च ।
रसोपरसकर्माणि तस्य पोषणकर्म च ।।२०.७३ ।।
स्नानाचमनकर्माणि वस्त्रशौचादयस्तथा ।
कायशौचं च कुरुते तृषानाशनमेव च ।।२०.७४ ।।
नवमानि ह्यनेकानि वारुणं कर्म उच्यते ।
माहेन्द्रं पार्थिवं कर्म अनेकानि द्विजोत्तम ।।२०.७५ ।।
कुलालकर्मभूकर्म कर्म पाषाणमेव च ।
दारुदन्तिमशृङ्गादि कर्म पार्थिवमुच्यते ।।२०.७६ ।।
चतुष्कर्म समासेन हस्ततः परिकीर्तितम् ।
पादाभ्यां गमनं कर्म दिशश्च विदिशस्तथा ।।२०.७७ ।।
निम्नोन्नतसमे देशे शिलासंकटकोटरे ।
तोयकर्दमसंघाते बहुकण्टकसंकुले ।।२०.७८ ।।
पायुकर्म विसर्गं तु कठिनद्रवपिच्छिलम् ।
सरक्तफेनिलादीनि पायुशक्ति प्रमुञ्चते ।।२०.७९ ।।
उपस्थकर्म आनन्दं करोति जननं प्रजा ।
स्त्रीपुंनपुंसकं चैव उपस्थं कुरुते द्विज ।।२०.८० ।।
वाचा तु कुरुते कर्म नवधा द्विजपुङ्गव ।
स्तुतिनिन्दा प्रशंसा च आक्रोशः प्रिय एव सः ।।२०.८१ ।।
प्रश्नो ऽनुज्ञा तथाख्यानमाशीश्च विधयो नव ।
एता नवविधा वाणी कीर्तितो मे द्विजोत्तम ।।२०.८२ ।।

---- मनश{ }्चोन्मनश{ }्च (१) ----

अधुना कथयिष्यामि मनसो नव वै गुणान् ।
चलोपपत्तिः स्थैरं च विसर्गकल्पनाक्षमा ।
सदसच्चाशुता चैव मनसो नव वै गुणाः ।।२०.८३ ।।
इष्टानिष्टविकल्पश्च व्यवसायः समाधिता ।
मनसो द्विविधं रूपं मनश्चोन्मन एव च ।।२०.८४ ।।
मनस्त्विन्द्रियभावत्वे उन्मनस्त्वमतीन्द्रिये ।
निगृहीता विसृष्तं च बन्धमोक्षौ तु साधनम् ।।२०.८५ ।।
निगृहीतेन्द्रियग्रामः स्वर्गमोक्षौ तु साधनम् ।
विसृष्टे इन्द्रियग्रामे दुःखसंसारसाधनम् ।।२०.८६ ।।
सकलं निष्कलं चैव मन एव विदुर्बुधाः ।
सकलं मननानात्वे एकत्वे मननिष्कलम् ।।२०.८७ ।।

---- उन्मनः ----

विगतराग उवाच ।
मनः स्ववेद्यं लोकानामुन्मनस्तु न विद्यते ।
उन्मनः कथयास्माकं कीदृशं लक्षणं भवेत् ।।२०.८८ ।।
अनर्थयज्ञ उवाच ।
उन्मनस्त्वं गते विप्र निबोध दशलक्षणम् ।
न शब्दं शृणुते श्रोत्रं शङ्खभेरीस्वनादपि ।।२०.८९ ।।
त्वचः स्पर्शं न जानाति शीतोष्णमपि दुःसहम् ।
रूपं पश्यति नो चक्षुः पर्वताभ्यधिको ऽपि वा ।।२०.९० ।।
जिह्वा रसं न विन्देत मधुराम्लवतो ऽपि वा ।
गन्धं जिघ्रति न घ्राणा तीक्ष्णं वाप्यशुचीन्यपि ।।२०.९१ ।।
उन्मनस्तव मे ख्यातं सर्वद्वैतविनाशनम् ।
भवपारगसुव्यक्तं निष्कलं शिवमव्ययम् ।।२०.९२ ।।
स शिवः स परो ब्रह्मा स विष्णुः स परो ऽक्षरः ।
स सूक्ष्मः स परो हंसः सो ऽक्षरः क्षरवर्जितः ।।२०.९३ ।।
एष उन्मन जानीहि शिवश्च द्विजपुङ्गव ।
कीर्तितो ऽस्मि समासेन किमन्यत्परिपृच्छसि ।।२०.९४ ।।

 ।।इति वृषसारसंग्रहे पञ्चविंशतितत्त्वनिर्णयो नाम विंशतिमो ऽध्यायः ।।





----   ----

विगतराग उवाच ।
अहो मतिमतां श्रेष्ठ अहो धर्मभृतां वर ।
अहो दम शमः सत्य अहो यज्ञ अहो तपः ।।२१.१ ।।
अनेनामृतवाक्येन विस्मयो मे परो गतः ।
प्रीतो ऽस्मि च तपाधारज्ञानाद्भुतरसेन च ।।२१.२ ।।
किं ददामि वरं ब्रूहि दातास्मि तव चेप्सितम् ।
एतच्छ्रुत्वा ततस्तेन प्रत्युवाच शुभां गिरम् ।।२१.३ ।।
[अनर्थयज्ञ उवाच ।]
को भवान्वरदश्रेष्ठ देवदानवराक्षसाः ।
अथवा भगवान्विष्णुर्मम जिज्ञासुरागतः ।।२१.४ ।।
व्यक्तं त्वां पुरुषश्रेष्ठ जानामि पुरुषोत्तम ।
रूपं दर्शय गोविन्द यद्यस्ति तपसः फलम् ।।२१.५ ।।
ततस्तु पुण्डरीकाक्षो दर्शयामास स्वां तनुम् ।
शङ्खचक्रगदापाणिः पीताम्बरधरो हरिः ।।२१.६ ।।
अनर्थयज्ञस्तं दृष्ट्वा विस्मयं परमं गतः ।
प्रहर्षमतुलं लब्ध्वा अश्रुपूर्णाकुलेक्षणः ।।२१.७ ।।
वेपमानस्वरेणात्र उवाच च जनार्दनम् ।
अद्य मे सफलं जन्म अद्य मे सफलं तपः ।।२१.८ ।।
नमो नमस्ते ऽस्तु जनादिसम्भवे
नमो नमस्ते ऽस्तु च विश्वरूपिणे ।
नमो नमस्ते ऽस्तु जनाभिसम्भवे
नमो नमस्ते ऽस्तु पितामहोद्भवे ।।२१.९ ।।
नमो नमस्ते ऽस्तु सहस्रशीर्षिणे
नमो नमस्ते ऽस्तु सहस्रचक्षुषे ।
नमो नमस्ते ऽस्तु सहस्रलिङ्गिने
नमो नमस्ते ऽस्तु सहस्रवक्षसे ।।२१.१० ।।
नमो नमस्ते ऽस्तु सहस्रमूर्तये
नमो नमस्ते ऽस्तु सहस्रबाहवे ।
नमो नमस्ते ऽस्तु सहस्रवक्त्रिणे
नमो नमस्ते ऽस्तु सहस्रमायिने ।।२१.११ ।।
नमो नमस्ते ऽस्तु वराहरूपिणे
नमो नमस्ते ऽस्तु महीसमुद्धृते ।
नमो नमस्ते ऽस्तु च भूतसृष्टिने
नमो नमस्ते चतुराश्रमाश्रये ।।२१.१२ ।।
नमो नमस्ते नरसिंहरूपिणे
नमो नमस्ते दितिजोरदारिणे ।
नमो नमस्ते ऽसुरचक्रसूदने
नमो नमस्ते ऽसुरदर्पनाशने ।।२१.१३ ।।
नमो नमस्ते दितिपुत्रदामने
नमो नमस्ते बलियज्ञसूदने ।
नमो नमस्ते ऽस्तु षडर्धविक्रमे
नमो नमस्ते त्रिदशार्तिनाशने ।।२१.१४ ।।
नमो नमस्ते ऽस्तु अनन्त अच्युते
नमो नमस्ते जगदर्तिनाशने ।
नमो नमस्ते मधुकैटनाशने
नमो नमस्ते ऽस्तु त्रिलोकबान्धवे ।।२१.१५ ।।
नमो नमस्ते त्रिदशाभिनन्दने
नमो नमस्ते ऽस्तु च दिव्यचक्षुषे ।
नमो नमस्ते ऽस्तु भवान्तपारगे
नमो नमस्ते ऽस्तु त्रिलोकपूजिते ।।२१.१६ ।।
नमो नमस्ते ऽस्तु गदाग्रपाणये
नमो नमस्ते वरचक्रपाणये ।
नमो नमस्ते ऽस्तु च शङ्खपाणये
नमो नमस्ते ऽस्तु च कम्बुपाणये ।।२१.१७ ।।
नमो नमस्ते ऽस्तु जलौघशायिने
नमो नमस्ते हरमर्दरूपिणे ।
नमो नमस्ते खगराजकेतवे
नमो नमस्ते शशिसूर्यलोचने ।।२१.१८ ।।
नमो नमस्ते उरगारिवाहने
नमो नमस्ते ऽद्भुतरूपदर्शिने ।
नमो नमस्ते ऽयुतसूर्यतेजसे
नमो नमस्ते ऽमृतमन्थनध्रुवे ।।२१.१९ ।।
नमो नमस्ते ऽमरलोकसंस्तुते
नमो नमस्ते जगमण्डपाश्रये ।
नमो नमस्ते जगदेकवत्सले
नमो नमस्ते शिवसर्वदे नमः ।।२१.२० ।।
क्षमस्व गोविन्द ममापराधम
अतीव पृष्टेन दुरात्मनेन ।
मयेद सर्वं कथितं स्मयेन
दयां कुरु त्वं त्रिदशेश्वरेण ।।२१.२१ ।।
वैशम्पायन उवाच ।
स्तोत्रेणानेन संतुष्टः केशवः परवीरहा ।
प्रत्युवाच महासेनो गिरया निरुपस्पृहा ।।२१.२२ ।।
स्तोत्रेणानेन मे तात तुष्टो ऽस्मि भृशमेजितः ।
दुर्लभान्यपि त्रैलोक्ये ददामि वरमीप्सितम् ।।२१.२३ ।।
अनेन मां स्तौति निराश्रितेन
त्वयोक्तवेदार्थमनोहरेण ।
यावन्ति तत्राक्षरसंख्यमस्ति
तावन्ति कल्पान्दिवि ते वसन्ति ।।२१.२४ ।।
त्वं चापि मे ब्रूहि वरं यथेष्टं
त्रैलोक्यराज्यादपि निर्विशङ्कम् ।
ददामि किं सप्तमहीश्वरत्वम
अथार्थराशिं बहुकन्यकां वा ।।२१.२५ ।।
श्रुत्वैव दिव्यं वरमच्युतस्य
प्रणम्य पादद्वयपङ्कजे तु ।
विज्ञाय विष्णुं वरदं वरेण्यं
? प्रहृ चेतः पुकान्चितो ऽतो ऽब्रवीत् ।।२१.२६ ।।
अनर्थयज्ञ उवाच ।
न कामये ऽन्यप्रवरं तु देव
असंशयं बन्धनसारमेकम् ।
विमुक्तबन्धो भवतः प्रसादाद
भवामि गोविन्द रतश्च धर्मे ।।२१.२७ ।।
भगवानुवाच ।
यथैव चित्तं तव सुप्रसन्नं
महर्षिदेवैरपि नैव दृष्टम् ।
अकल्मषं दुःखविवर्जितत्वम
भवार्णवस्तीर्णमसंशयेन ।।२१.२८ ।।
गच्छाम भो साम्प्रत श्वेतद्वीपम
अगम्य देवैरपि दुर्निरीक्ष्यम् ।
मद्भक्तिपूतमनसा प्रयाति
घोरार्णवे नैव पुनश्चरन्ति ।।२१.२९ ।।
वैशम्पायन उवाच ।
एवमुक्त्वा हरिस्तत्र करे गृह्य तपोधनम् ।
ततः सो ऽन्तर्हितस्तत्र तेनैव सह केशवः ।।२१.३० ।।
एवं हि धर्मस्त्वधिकप्रभावाद
गतः स लोकं पुरुषोत्तमस्य ।
अशेषभूतप्रभवाव्ययस्य
सनातनं शाश्वतमक्षरस्य ।।२१.३१ ।।
त्वमेव भक्तिं कुरु केशवस्य
जनार्दनस्यामितविक्रमस्य ।
यथा हि तस्यैव द्विजर्षभस्य
गतिं लभस्व पुरुषोत्तमस्य ।।२१.३२ ।।
किमन्य भूयः कथयामि राजन
यदस्ति कौतूहलमन्यशेषम् ।
पृच्छस्व मां तात यथेप्सितं ते
भविष्यभूतं भवतो यथेष्टम् ।।२१.३३ ।।
जनमेजय उवाच ।
कियन्ति कल्पानि गतानि पूर्वम
भविष्यकल्पानि कियन्ति विप्र ।
एकैककल्पं कियदिन्द्रमुक्तम
प्रवर्तमानादपि कीर्तयस्व ।।२१.३४ ।।
वैशम्पायन उवाच ।
परार्धकल्पं गत पूर्व राज्यम
चतुर्दशैवेन्द्र नरेन्द्र कल्पम् ।
तथैव मन्वन्तर कल्पमेकम
भविष्यकल्पं च परार्धमेव ।।२१.३५ ।।
वराहकल्पः प्रथमो बभूव
गताश्च मन्वन्तर षट्नरेन्द्र ।
चतुर्युगं सप्तति एकयुक्तं
मन्वन्तरा संख्यमुदाहरन्ति ।।२१.३६ ।।
मन्वन्तराणां च चतुर्दशैव
कल्पस्य संख्या मुनयो वदन्ति ।
कल्पायुतश्चाह पितामहस्य
तथा च रात्रिं प्रवदन्ति तज्ज्ञाः ।।२१.३७ ।।
षड्लक्षकल्पेन तु मासमाहुस
तद्द्वादशा वर्षमुदाहरन्ति ।।२१.३८ ।।
तेनाब्देन परार्धकल्पगुणितं ब्रह्मायुरित्युच्यते
त्रैलोक्याधिपतिः प्रधानपुरुषो ब्रह्माप्यनित्यः स्मृतः ।
शेषं भूतचतुर्विधस्य नियतं जीवस्य किं शोच्यते
तस्मान्नास्ति जगत्सुसारविमलं मुक्त्वा शिवं शाश्वतम् ।।२१.३९ ।।

 ।।इति वृषसारसंग्रहे कल्पनिर्णयो नामैकविंशतिमो ऽध्यायः ।।




जनमेजय उवाच ।
श्रुतो ऽथाब्जमुखाद्धर्मसारसंग्रहमुत्तमम् ।
मधुरश्लक्ष्णवाणीभिः सम्यग्वेदार्थसंयुतम् ।।२२.१ ।।
न्याययुक्तं महासारं गुह्यज्ञानमनुत्तरम् ।
तृप्तो ऽस्मीहामृतं पीत्वा जन्ममृत्युरुजापहम् ।।२२.२ ।।
प्रश्नमेकान्य पृच्छामि नामहेतुं तपोधन ।
वर्णगोत्राश्रमं तस्माच्छ्रोतुमिच्छामि ते पुनः ।।२२.३ ।।
वैशम्पायन उवाच ।
शृणु राजन्नवहितो योगेन्द्रस्य महात्मनः ।
आश्रमं वर्णजातीनां वक्ष्याम्येव नराधिप ।।२२.४ ।।
हिमवद्दक्षिणे पार्श्वे मृगेन्द्रशिखरे नृप ।
महेन्द्रपथगा नाम नदीतीरे नराधिप ।।२२.५ ।।
तत्राश्रमपदं तस्य पुलिने सुमनोरमे ।
वसति स्म महाभागस्तत्त्वपारगनिस्पृहः ।।२२.६ ।।
शीलशौचसमाचारो जितद्वन्द्वो जितश्रमः ।
जितमानभयक्रोधो जितसर्वपरिग्रहः ।।२२.७ ।।
सोमवंशप्रसूतास्ते क्षत्रिया द्विजतां गताः ।
तपसा विनयाचारैर्विष्णुना द्विजकल्पिताः ।।२२.८ ।।
अजिता नाम तत्पूर्वं कामक्रोधजितेन तु ।
संकल्पस्तस्य राजेन्द्र कथयिष्यामि तच्छृणु ।।२२.९ ।।
अध्यात्मनगरस्फीतः अधिभूतजनाकुलः ।
अधिदैवतसांनिध्यं दशायतन पञ्च च ।।२२.१० ।।
दशयज्ञव्रतं चीर्णं दशकामपराजितः ।
नियमान्दश संश्रित्य दश वायव ऋत्विजः ।।२२.११ ।।
दशाक्षरेण मन्त्रेण दशधर्मक्रियापदः ।
दशसंयमदीप्ताग्नौ जिह्वातेजोदशेन्द्रियः ।।२२.१२ ।।
दशयोगासनासीनो दशध्यानपरायणः ।
बुद्धिर्वेदी मनो यूपः सोमपानो ऽमृताक्षरः ।।२२.१३ ।।
दक्षिणाभय भूतेभ्यः पशुबन्ध स्वयंकृतः ।
विनार्थं यज्ञमिष्ट्वा तु कालं च क्षपयत्यसौ ।
अनर्थयज्ञं तं प्राहुर्मुनयस्तत्त्वदर्शिनः ।।२२.१४ ।।
जनमेजय उवाच ।
दशयज्ञमहं श्रोतुं देहि मां द्विजसत्तम ।
दशकामदशध्यानं दशयोगदशाक्षरम् ।।२२.१५ ।।
वैशम्पायन उवाच ।
ब्रह्मदेवपितृयज्ञो यज्ञो भूतातिथेश्च ह ।
जपो योगस्तपो ध्यानं स्वाध्यायश्च दश स्मृतः ।।२२.१६ ।।
पत्नीपुत्रपशुभृत्यधनधान्ययशःश्रियः ।
मान भोग दश राजन्दशकाम उदाहृतः ।।२२.१७ ।।
मानसो यौगपद्यश्च संक्षिप्तश्च विशाम्पते ।
विशाला नाम योगश्च ततो द्विकरणः स्मृतः ।।२२.१८ ।।
रविः सोमो हुताशश्च स्फटिकाम्बरमेव च ।
दशयोगासनासीनो नित्यमेव तपोधनः ।।२२.१९ ।।
अनिरोधमनाः सूक्ष्मं ध्यायेद्योगः स मानसः ।
प्राणायामैर्मनो रुद्ध्वा यौगपद्यः स उच्यते ।।२२.२० ।।
ब्रह्मादिस्तम्बपर्यन्तं सर्वं स्थावरजङ्गमम् ।
प्रलीयमानं ध्यायेत क्रमात्सूक्ष्मं विचिन्तयेत् ।।२२.२१ ।।
संक्षिप्त एष आख्यातो विशालां छृणु तत्त्वतः ।
ब्रह्मादिसूक्ष्मपर्यन्तं चिन्तयीत विचक्षणः ।।२२.२२ ।।
संक्षिप्तां च विशालां च चिन्तयीत परस्परम् ।
एषा द्विकरणी नाम योगस्य विधिरुच्यते ।।२२.२३ ।।
देहमध्ये हृदि ज्ञेयं हृदिमध्ये तु पङ्कजम् ।
पङ्कजस्य च मध्ये तु कर्णिकां विद्धि गोपते ।।२२.२४ ।।
कर्णिकायास्तु मध्ये तु पञ्चबिन्दुं विदुर्बुधाः ।
रविसोमशिखां चैव स्फटिकाम्बरमेव च ।।२२.२५ ।।
रविमण्डलमध्ये तु भावयेच्चन्द्रमण्डलम् ।
तस्य मध्ये शिखां ध्यायेन्निर्धूमज्वलनप्रभाम् ।।२२.२६ ।।
अग्निमध्ये मणिं ध्यायेच्छुद्धधाराजलप्रभम् ।
तस्य मध्ये ऽम्बरं ध्यायेत्सुसूक्ष्मं शिवमव्ययम् ।।२२.२७ ।।
दशयोगमिदं राजन्कथितं च मया तव ।
दशध्यानं समासेन कीर्तितं शृणु तद्यथा ।।२२.२८ ।।
घोषणी पिङ्गला चैव वैद्युती चन्द्रमालिनी ।
चन्द्रा मनोऽनुगा चैव सुकृता च तथापरा ।।२२.२९ ।।
सौम्या निरञ्जना चैव निरालम्बा च कीर्तिता ।
सुपिषित्वाङ्गुलौ श्रोत्रे ध्वनिमाकर्णयेन्नरः ।।२२.३० ।।
तत्तदक्षरमाकर्ण्य अमृतत्वाय कल्प्यते ।
पिङ्गलां तु शिखाधूमां ध्यायेन्नित्यमतन्द्रितः ।।२२.३१ ।।
विमुक्तः सर्वपापेभ्यो निर्द्वन्द्वपदमाप्नुयात् ।
वैद्युती तु निशामध्ये लक्षते ऽजमनामयम् ।।२२.३२ ।।
पञ्चमाससदाभ्यासाद्दिव्यचक्षुर्भवेन्नरः ।
बिन्दुमालां ततः पश्येत्तरुच्छायासमाश्रिताम् ।।२२.३३ ।।
जात्यस्फटिकसंकाशं दृष्ट्वा मुच्यति बन्धनैः ।
ध्यायेन्मनोऽनुगा नाम पक्ष्मीरापीड्य लोचने ।।२२.३४ ।।
श्वेतपीतारुणं बिन्दुं दृष्ट्वा भूयो न जायते ।
मनोऽनुगादि षट्त्वेते ध्यानमुक्तं मया तव ।।२२.३५ ।।

---- परमाणुः ----

अधुनान्यत्प्रवक्ष्यामि परमाणु चतुर्विधम् ।
पार्थिवादिचतुर्भूतं यैर्व्याप्तं निखिलं जगत् ।
लक्षणं तस्य राजेन्द्र शृणु वक्ष्यामि साम्प्रतम् ।।२२.३६ ।।
पार्थिवोर्ध्वगतिः सूक्ष्मः परमाणु नराधिप ।
प्रत्यक्षदर्शनं ध्यानं लक्षयेन्नियतं शुचिः ।।२२.३७ ।।
मुच्यते सर्वपापेभ्यो राहुना चन्द्रमा यथा ।
तेन यो ऽभ्यसते नित्यं स योगी भुवनेश्वरः ।।२२.३८ ।।
अधोगति महाराज परमाणु जलोद्भवः ।
अभ्यसेद्यदिदं राजन्सर्वपातकनाशनम् ।।२२.३९ ।।
आग्नेयपरमाणूनि तिर्यगूर्ध्वगतिः स्मृता ।
य इदं ध्यायते नित्यमुत्तमां गतिमाप्नुयात् ।।२२.४० ।।
वायव्यपरमाणूनि अधोर्ध्वतिर्यगास्मृता ।
न स मुह्यति तं दृष्ट्वा वायुसम्भव भूपते ।।२२.४१ ।।
चत्वार एते राजेन्द्र परमाणु निरीक्षते ।
तेन सर्वमखैरिष्टं तेन तप्तं तपस्तथा ।।२२.४२ ।।
तेन दत्ता मही कृत्स्ना सप्तसागरसंवृता ।
सर्वतीर्थाभिषेकश्च सर्वव्रतक्रिया तथा ।।२२.४३ ।।
अनेनैव विधानेन दशध्यानं नराधिप ।
कुरुते अव्यवच्छिन्नं सर्वकामफलप्रदम् ।।२२.४४ ।।

---- दशाक्षरमन्त्रः ----

दशाक्षरमहाराज योगीन्द्रस्य महात्मनः ।
कथयामि समासेन शृणुष्वावहितो भव ।।२२.४५ ।।
प्रणवादिस्वरा त्रीणि दीर्घबिन्दुसमायुतम् ।
पञ्च पञ्च चवर्गे तु वायुबीजमधस्थितम् ।।२२.४६ ।।
त्रयोदशस्वरायुक्तं पञ्चम परिकीर्तितम् ।
पञ्चविंशतिमः षष्ठ अक्षरः परिकीर्तितः ।।२२.४७ ।।
यादृशं पञ्चमः प्रोक्तं सप्तमे च प्रयोजयेत् ।
अकारस्वरसंयुक्तं सर्वपातकनाशनम् ।।२२.४८ ।।
प्रथमं पञ्चमे वर्गे तृतीयस्वरयोजितम् ।
उक्तरेकारसंयुक्तं नवमं परिकीर्तितम् ।।२२.४९ ।।
दशमः पुनरोंकारः मन्त्रश्रेष्ठो दशाक्षरः ।
जपतो ध्यायते वापि पार्थिवादि क्रमेण तु ।।२२.५० ।।
मुच्यते सो ऽपि संसारे संशयो नास्ति भूपते ।
आचारमूलो धर्मस्तु धर्ममूलो जनार्दनः ।
तेन सर्वजगद्व्याप्तं त्रैलोक्यं स चराचरं ।।२२.५१ ।।

---- आचारविधिः ----

आचाराल्लभतीह आयुरतुलमैश्वर्यवित्तं तथा
आचारात्सुतमीप्सितं च लभते श्रीकीर्तिप्रज्ञायशः ।
आचाराल्लभते च लक्ष्मिमतुलं ख्यातिं तथैवोत्तमम
आचारादिह मन्त्रधर्मपरमं प्राप्नोति निःसंशयम् ।।२२.५२ ।।
जनमेजय उवाच ।
आचारात्प्रभवानुसङ्गकथितं सुश्लिष्टधर्माकरम
आचारात्कतिधाङ्ग कीर्तय पुनस्तृप्तिर्न मे जायते ।
सर्वज्ञः त्वमहं शृणोमि वरदं किञ्चिन्न मे शाश्वरम
तन्मे कीर्तय धर्मसारशुभदमाचारमूलाश्रयम् ।।२२.५३ ।।
वैशम्पायन उवाच ।
नित्यं नम्रशिरोद्विजातिगुरुषु शुश्रूषणं दैवतम
तिष्ठेनाचमनेन चाशनकरं वामास्थिमानादरम् ।
सूर्याग्निशशिबन्धुरार्यपुरतः कुर्यान्न चावश्यकम
शस्ये भस्मनि गोव्रजेद्विजजलं कुर्यान्न चार्कं नरः ।।२२.५४ ।।
पादेनाग्निजलं स्पृशेन्न च गुरुं पादेन पादं तथा
शौचं कार्य जलादिना च नियतं नाधो जलं कारयेत् ।
कुर्यान्नित्यभिवादनं द्विजगुरोर्मातापितृर्दैवतम
एताचारविधिः समासनियमस्तुभ्यं मया कीर्तितम् ।।२२.५५ ।।

---- स्त्रियः ----

जनमेजय उवाच ।
स्त्रीणां किं प्रियमस्ति तद्वद विभो संसारसारस्त्रियाम
किं सद्भाव न वेद्मि तस्य विषये किं द्वेष्य किं तात्प्रियम् ।
पश्यामि न च तस्य गर्भकलया प्राप्नोति निःसंशयम
मायाजालसहस्रगापि युवती कुर्वन्ति भर्ता प्रियम् ।।२२.५६ ।।
वैशम्पायन उवाच ।
राजन्किं प्रियमस्ति अर्थपरतः पश्यामि नान्यन्नृपे
पुत्रार्थैकप्रयोजनं युवतयः स्वायम्भुवोक्तामरैः ।
कान्ता नित्यकला प्रवर्तनकरी धर्मसखाया सती
माया वापि करोति सद्य मनुजात्यक्तान्य वा सेवते ।।२२.५७ ।।
स्त्रीसङ्गं परिवर्जयेन्नरपते आयासदं दुःखदम
मृत्युद्वारभयाकरं विषगृहमापत्सुघोरालयम् ।
अग्निर्मारुतमत्तवारणसम तस्यानुगामी सदा
स्त्रीहेतोर्हतरावणस्त्रिदशप इन्द्रो ऽपि विस्थापितः ।
स्त्रीहेतोरपि चन्द्रमास्त्रिभुवने धिक्तां गतश्चामरो ।
दण्डक्षो हतराष्ट्रपौरसहितः किं भूय वक्ष्याम्यहम् ।।२२.५८ ।।

---- विप्रमुनिभिक्षुनिर्ग्रन्थिपरिव्राजकर्ष्यादयः ----

जनमेजय उवाच ।
विप्रे कीदृशलक्षणं भवति भो कीदृग्मुनिश्चोच्यते
तेनार्थेन भवेत भिक्षु भगवन्निग्रन्थि को वा द्विज ।
केनार्थेन भवेद्द्विजेन्द्र भगवन्ज्ञेयः परिव्राजकः
! ज्ञेयाः किमृषयश्च लक्षण मुनेरिच्छामि ज्ञातुं पुनः ।।२२.५९ ।।
वैशम्पायन उवाच ।
सत्यं शौचमहिंसता दमशमौ भूतानुकम्पी सदा
आत्मारामजितो स्वधर्मनिरतः सत्त्वस्थ नित्यं मनः ।
कामक्रोधयमस्वदारनिरतः संत्यज्य लोभः शनैः
एवं यः कुरुते द्विजातिसुवरः शूद्रो ऽपि यः संयमी ।।२२.६० ।।
तस्माच्छद्मकवर्जितः स भगवान्संसारभीभिद्यकः
यत्तत्स्थानपरं व्रजन्ति पुरुषाः तस्मात्परिव्राजकः ।
ग्रन्थिदारसुतं धनंश्च विरति निर्ग्रन्थिक सोच्यते
रम्यन्ते ऋषिराश्रमे धृतिमनस्तस्मादृषिः सोच्यते ।।२२.६१ ।।
कायवाङ्मनदण्डतत्परतरस्ते दण्डिकरूच्यते
सद्धर्मश्रवणं वदन्ति श्रवणः सद्धर्मब्रह्माक्षरः ।
पाशप्रक्षिपतं पशुत्वसकलं पाशूपतास्ते स्मृताः
विप्रे पाशुपतादिभिक्षुसकलं पृष्टो ऽस्म्यहं लक्षणम् ।।२२.६२ ।।
सर्वं तत्कथितो ऽसि लक्षण मया सन्धिश्वनिर्नाशनम
प्रज्ञासंग्रहशीतवर्धनपरं संसारनिर्मूलनम् ।
 
एतज्ज्ञानपरं प्रबोधमतुलं नित्यं शिवं धार्यते ।।२२.६३ ।।

 ।।इति वृषसारसंग्रहे द्वाविंशतितमो ऽध्यायः ।।




जनमेजय उवाच ।
देवानां दानवानां च उत्तरारणिमेव च ।
विद्विषन्ति च ते ऽन्योन्यं कारणं तस्य कीर्तय ।।२३.१ ।।
वैशम्पायन उवाच ।
पापपुण्यस्वभावाभ्यां देवदैत्यस्य भूपते ।
धर्मपक्षस्मृतो देवो दानवो ऽधर्मपक्षतः ।।२३.२ ।।
हेतुना तेन राजेन्द्र अन्योन्यं विद्विषन्ति ते ।
देवद्वेष्टासुराः सर्वे विबुधाश्चासुरद्विषः ।।२३.३ ।।

---- धर्माधर्मविपक्षता ----

धर्माधर्मविपक्षतां शृणु परां भूतानुकम्पोदयाम
सत्यं शौचमहिंसता दमशमो निर्मानमीर्ष्यारुषा ।
तृष्णालोभरतस्य कामविषयः सर्वेन्द्रियाणां जयः
आध्यात्मेषु रतिः प्रसन्नमनसो निर्द्वन्द्वसर्वालयः ।।२३.४ ।।
पापोपेक्षणशश्वपुण्यमुदितो दीनेषु कारुण्यता
दानं शीलधृतिक्षमाजपतपः स्वाध्यायमौने रतिः ।
योगाभ्यासरतिर्दिवौकसगणे ज्ञाने च सांख्ये तथा
अक्रोधार्जवतेजयज्ञमभयं संतोष अद्रोहता ।।२३.५ ।।
त्यागो मार्दवह्रीरचापलरतिर्न्यासाभिमानो ग्रहात
मैत्रीभावसदारपैशुनमतिर्ब्राह्मण्यश्रद्धान्वितः ।
एताचार सदा नरेन्द्र विबुधाः संक्षेपतः कीर्तिताः
दैत्यानां शृणु कीर्तये स्ववहितो ऽसम्भाव्य तेषां निजम् ।।२३.६ ।।
दैत्याः पापरतिस्वभावचपला निर्लज्जदर्पालसाः
कामक्रोधवशाः सुदुष्टमनसस्तृष्णाधिका निर्दयाः ।
शौचाचारविवर्जिता गुरुगिरान्नानित्य कुर्युः क्रियाः
जीवाकर्षणजीवनः प्रतिदिनं मोहान्धरागान्विताः ।।२३.७ ।।
निद्रा नित्य दिवा प्रसक्तमशुचिः सूर्योदये सुप्यते
आशापाशशतैर्निबद्धहृदयो हृत्वा परस्वं पुनः ।
मात्सर्यात्परपाकभेदनिरतो मूलस्य दुष्पूरता
! नास्तीकत्वपराङ्गनास्वभिरत उत्कोचकामः सदा ।।२३.८ ।।
देवब्राह्मण विद्विषन्ति सततं लोभाच्च कार्यक्रिया
धर्मं दूषयते च मूड्हमनसा आर्यं च तीर्थं तथा ।
हन्तव्यान्यहताश्च मन्यबहवो विस्फूर्जितमद्रुवन
दैत्यानां कथितं च चिह्न कतिचित्सद्भावतः कीर्तितम् ।।२३.९ ।।
मर्त्येष्वेव नरेन्द्र मानुषमभूद्देवासुराणां नृपः
यो यं प्रोक्तः स्वभावतामुभयतो मानुष्यलोके तथा ।
यन्मे पृच्छितवान्नरेन्द्र कथितं यत्त्वं पुरा गोपितम
विद्वेषोभयकारणं नरपते किं भूय वक्ष्याम्यहम् ।।२३.१० ।।

---- निद्रोत्त्पत्तिः ----

जनमेजय उवाच ।
अस्ति कौतूहलं चान्यं पृच्छामि त्वां द्विजोत्तम ।
कथं निद्रा समुत्पन्ना सर्वभूतविमोहनी ।।२३.११ ।।
रात्रौ प्रजायते कस्माद्दिवा कस्मान्न जायते ।
कस्माच्च कुरुते जन्तोर्निद्रा नेत्रप्रमीलनम् ।
एतन्मे संशयं छिन्धि सर्वज्ञो ऽसि द्विजोत्तम ।।२३.१२ ।।
वैशम्पायन उवाच ।
देवी ह्येषा महाभागा निद्रा नेत्राश्रया नृणाम् ।
तस्या वशं गतं सर्वं जगत्स्थावरजङ्गमम् ।।२३.१३ ।।
सदेवदानवगणा गन्धर्वोरगराक्षसाः ।
यक्षभूतपिशाचाश्च पशुपक्षिसरीसृपाः ।।२३.१४ ।।
गुह्यकाश्च मृगा नागा किंनरा जलजोरगाः ।
निद्रावशगताः सर्वे पाप्मना त्वभिलङ्घिताः ।।२३.१५ ।।
देवदानवकर्मान्ते तस्मिन्नमृतसम्भवे ।
मन्दरोत्थापने विष्णुर्देवासुरसमागमे ।।२३.१६ ।।
जायते विग्रहे त्वेषा कृते ह्यमृतमन्थने ।
रजस्तमश्चासुरं वै सत्त्वं देवकृतैः शुभैः ।।२३.१७ ।।
ततः सत्त्वमयी देवी रजस्तमनिवासिनी ।
क्रोधजा वै स्थिता मध्ये देवदानवपक्षयोः ।।२३.१८ ।।
तामद्भुतमयीं दृष्ट्वा विस्मिता देवदानवाः ।
तस्याः प्रभावाभिहता दुद्रुवस्ते दिशो दश ।।२३.१९ ।।
तत्र पीताम्बरधरो विष्णुरेकस्तु तिष्ठति ।
साभिगत्वा विशालाक्षी नारायणमथाब्रवीत् ।।२३.२० ।।
देवदानवनाथस्त्वं त्वयि सर्वं प्रतिष्ठितम् ।
देहि देव ममावासं यत्राहं निवसे सुखम् ।।२३.२१ ।।
ततो नारायणस्तुष्टस्तां देवीं प्रत्यभाषत ।
शरीरे मम वस्तव्यं विष्णुरेनामथाब्रवीत् ।।२३.२२ ।।
ततस्तां वैष्णवं तेजः पाप्मना समतिष्ठत ।
ततः शेते स वैकुण्ठः पाप्मना त्वभिलङ्घितः ।।२३.२३ ।।
तस्मिन्शयाने वित्रस्ता देवासुरगणास्तथा ।
ऊचुस्ते परमोद्विग्नाः शयानं विष्णुमच्युतम् ।।२३.२४ ।।
त्रातारं नाभिगच्छाम उत्तिष्ठोत्तिष्ठ केशव ।
ततः शङ्खगदापाणिरुत्तिष्ठत महाभुजः ।।२३.२५ ।।
उत्थितश्च विशालाक्षः पाप्मना तस्य पृष्ठतः ।
ततः सा विग्रहवती स्थिता नारायणालये ।।२३.२६ ।।
विष्णुर्देवासुरगणानिदं वचनमब्रवीत् ।
अस्माकं वै शरीरेषु इयं पाप्मा विनिःसृता ।।२३.२७ ।।
एषाभिसत्त्वारसता सत्येन भगिनी मम ।
विश्रुतां त्रिषु लोकेषु तां पूजयथ मां यथा ।।२३.२८ ।।
ततो देवासुरगणाः सप्तलोकाः समानुषाः ।
विभक्ता वैष्णवी पाप्मा तेषु सर्वेषु देवता ।।२३.२९ ।।
पर्वतेष्वथ वृक्षेषु सागरेषु सरित्सु च ।
ततो निद्रावशगतं जगत्स्थावरजङ्गमम् ।।२३.३० ।।
एषोत्पत्तिश्च निद्राया यथा वसति तच्छृणु ।
त्रीणि स्थानानि यस्या वै शरीरेषु शरीरिणाम् ।।२३.३१ ।।
श्लेष्मपित्तानिलस्थाने त्रीणि पक्षाणि वासिनः ।
तमः श्लेष्माश्रया निद्रा रजोनिद्रा तु वातिका ।।२३.३२ ।।
पित्ताश्रयां स्मृतां निद्रां सात्त्विकां विद्धि भूपते ।
आदित्यप्रभवं तेजस्तस्मिन्सत्त्वं प्रतिष्ठति ।।२३.३३ ।।
निद्रा दिवा न भवति तस्मात्सत्त्वगुणात्मिका ।
यस्मात्सोमोद्भवा निद्रा तमांसि च रजांसि च ।।२३.३४ ।।
तस्माद्रात्रौ भवेन्निद्रा तामसी हरजात्मिका ।
यदा हि सर्वाङ्गगतौ श्रोतांसि प्रतिपद्यते ।।२३.३५ ।।
रजस्तमश्च नियतस्तदा निद्रा प्रवर्तते ।
तमस्यूर्ध्वगतश्रोतो ह्यक्षिपक्ष्मासमाश्रिता ।।२३.३६ ।।
तमः प्रवर्तते जन्तोस्ततस्त्वक्ष्नोर्निमीलनम् ।
नासाक्षिकर्णश्रोतांसि प्रयुज्यन्ते कफेन तु ।।२३.३७ ।।
हृदयं मुह्यते चापि तमसा चावृतं मनः ।
स्पर्शं न वेदयत्येव न शृणोति न पश्यति ।।२३.३८ ।।
नोच्छ्वासयति नासाभ्यां विवृताक्षिमुखो नरः ।
एषा नृणामन्तकरी निद्रा वै तामसी स्मृता ।।२३.३९ ।।
अकर्मण्यप्रवृत्तिश्च मृतवत्स्वपते क्षितौ ।
निद्रोत्पत्तिं विकारं च कथितो ऽस्मि नराधिप ।
तस्मान्निद्रां न सेवेत तमोमोहप्रवर्धनीम् ।।२३.४० ।।

 ।।इति वृषसारसंग्रहे निद्रोत्पत्तिस्त्रयोविंशतिमो ऽध्यायः ।।




जनमेजय उवाच ।
देवानां दानवानां च वैषम्यानि श्रुतानि मे ।
निद्रासम्भवमाश्चर्यं त्वत्प्रसादेन वेदितम् ।।२४.१ ।।
त्रैलोक्यविस्तरायामं श्रोतुमिच्छामि भो द्विज ।
कस्मिंश्चिन्नरकं ज्ञेयं पातालं च द्विजोत्तम ।।२४.२ ।।
सप्तद्वीपं समिच्छामि सप्तसागरमेव च ।
मेरुमूर्धं च विप्रेन्द्र देवालयं निबोध माम् ।।२४.३ ।।

---- त्रैलोक्यं नरकाणि च ----

वैशम्पायन उवाच ।
शृणु संक्षेपतो राजन्त्रैलोक्यायामविस्तरम् ।
कालाग्निः प्रथमो ज्ञेयः सर्वाधस्तान्नरेश्वर ।।२४.४ ।।
तस्योपरि नृपश्रेष्ठ ज्ञेया नरककोटयः ।
रौरवादि अवीच्यन्तं यातनास्थानमुच्यते ।।२४.५ ।।

---- सप्त पातालाः ----

उपरिष्टात्तु विज्ञेयाः पातालाः सप्त एव तु ।
आभासतालः प्रथमः स्वतालश्च ततः परम् ।।२४.६ ।।
शीतलश्च गभस्तिश्च शर्करश्च शिलातलम् ।
सप्तमं तु महातालं शेषनागकृतालयः ।।२४.७ ।।
बलिश्च दैत्यराजेन्द्रो राक्षसश्च विशंखणः ।
इत्येवमादयः सर्वे नागदानवराक्षसाः ।।२४.८ ।।

---- सप्त द्वीपाः प्रियव्रतसुताश{ }्च ----

सप्त द्वीपास्ततो ज्ञेयाः सप्तसागरसंवृताः ।
प्रियव्रतस्य पुत्रो ऽभूद्दश राजपराक्रमः ।। २४.९ ।।
अग्नीध्रश्चाग्निबाहुश्च मेधा मेधातिथिर्वसुः ।
ज्योतिष्मान्द्युतिमान्हव्यः सवनः पत्र एव च ।।२४.१० ।।
अग्निबाहुश्च मेधा च पत्रश्चैव त्रयो जनाः ।
संसारभयभीतेन मोक्षमार्गसमाश्रिताः ।।२४.११ ।।
अग्नीध्रं प्रथमद्वीपे अभ्यषिञ्चत्प्रियव्रतः ।
प्लक्षद्वीपेश्वरं चक्रे नाम्ना मेधातिथिं तथा ।।२४.१२ ।।
वसुश्च शाल्मलीद्वीपे अभिषिक्तो महीपतिः ।
ज्योतिष्मन्तं कुशद्वीपे राजानमभिषेचयेत् ।।२४.१३ ।।
क्रौञ्चद्वीपेश्वरं चक्रे द्युतिमन्तं नरेश्वर ।
शाकद्वीपेश्वरं हव्यं पुष्करे सवनः स्मृतः ।।२४.१४ ।।
मध्ये पुष्करद्वीपस्य पर्वतो मानसोत्तरः ।
लोकपालाः स्थितास्तत्र चतुर्भिश्चतुरो दिशः ।।२४.१५ ।।
महावीतः स्मृतो वर्षो धातकी च नराधिप ।
तस्य बाह्यः समुद्रो ऽभूत्स्वादूदक इति स्मृतः ।।२४.१६ ।।
चतुःषष्टि स्मृतो लक्षो योजनानां नराधिप ।
पुष्करद्वीपमन्तश्च क्षीरोदो नाम सागरः ।।२४.१७ ।।
द्वात्रिंशल्लक्षविस्तारः शाकद्वीपबहिर्वृतः ।
जलदश्च कुमारश्च सुकुमारमणीचकः ।।२४.१८ ।।
कुसुमोत्तरमोदश्च सप्तमं च महाद्रुमम् ।
हव्यपुत्राः स्मृताः सप्त वर्षनाम तथा स्मृतः ।।२४.१९ ।।
द्वीपान्तं दधिमण्डोदक्षीरोदार्धं विनिर्दिशेत् ।
क्रौञ्चद्वीपसमुद्रान्ते सप्त वर्षास्तु ते स्मृताः ।।२४.२० ।।
कुशलो मनोनुगश्चोष्णः यावनश्चान्धकारकः ।
मुनिश्च दुन्दुभिश्चैव सुता द्युतिमतस्तु वै ।।२४.२१ ।।
दध्यर्धे घृतमण्डोदः कुशद्वीपसमावृतः ।
तत्रापि सप्तवर्षे च नामतः शृणु भारत ।।२४.२२ ।।
उद्भिमान्वेणुमांश्चैव स्वैरन्नालम्बनो धृतिः ।
षष्ठः प्रभाकरश्चैव कपिलः सप्तमः स्मृतः ।।२४.२३ ।।
घृतमण्डस्तदर्धेन तस्यान्ते मदिरोदधिः ।
समन्ताच्छाल्मलीद्वीपो वर्षाः सप्तैव कीर्तिताः ।।२४.२४ ।।
श्वेतश्च हरितश्चैव जीमूतो रोहितस्तथा ।
वैद्युतो मानसश्चैव सुप्रभः सप्तमः स्मृतः ।।२४.२५ ।।
मदिरोदधितो ऽर्धेन ज्ञेयस्त्विक्षुरसोदधिः ।
प्लक्षद्वीपो वृतस्तेन सप्तवर्षसमन्वितः ।।२४.२६ ।।
शान्तश्च शिशिरश्चैव सुखदानन्द एव च ।
शिवक्षेमो ध्रुवश्चैव सप्त मेधातिथेः सुताः ।।२४.२७ ।।
लवणोदस्तु तस्यान्ते जम्बूद्वीपसमावृतः ।
लक्षयोजनविस्तार उपद्वीपसमन्वितः ।।२४.२८ ।।
अङ्गद्वीपो यवद्वीपो मलयद्वीप एव च ।
शङ्खद्वीपकमुद्वीपो वराहद्वीप एव च ।।२४.२९ ।।
सिंह बर्हिणद्वीपं च पद्मश्चक्रस्तथैव च ।
वज्ररत्नाकरद्वीपो हंसकः कुमुदस्तथा ।।२४.३० ।।
लाङ्गलो वृषद्वीपश्च द्वीपो भद्राकरस्तथा ।
चन्द्रद्वीपश्च सिन्धुश्च चन्दनद्वीप एव च ।
उपद्वीपसहस्राणि एवमादीनि कीर्तितम् ।।२४.३१ ।।

---- अग्नीध्रपुत्रा जम्बुद्वीपे ----

अग्नीध्रो नववर्षेषु नवपुत्रानसिञ्चयत् ।
नाभिः किंपुरुषश्चैव हरिवर्ष इलावृतः ।।२४.३२ ।।
पञ्चमं रम्यकं वर्षं षष्ठं चैव हिरण्मयम् ।
कुरवः सप्तमो ज्ञेयो भद्राश्वश्चाष्टमः स्मृतः ।।२४.३३ ।।
नवमः केतुमालो ऽभून्नववर्षाः प्रकीर्तिताः ।
हिमवद्दक्षिणे पार्श्वे वर्षो भारतसंज्ञितः ।।२४.३४ ।।
अत्रापि नवभेदो ऽभूद्भारतात्मजसम्भवः ।
इन्द्रद्वीपः कशेरुश्च ताम्रवर्णो गभस्तिमान् ।।२४.३५ ।।
नागद्वीपस्तथा सौम्यो गान्धर्वश्चाथ वारुणः ।
अयं च नवमो द्वीपः कुमारीद्वीपसंज्ञितः ।
दक्षिणे हेमकूटस्य वर्षः किंपुरुषः स्मृतः ।।२४.३६ ।।
निषधो दक्षिणपार्श्वे हरिवर्ष इति स्मृतः ।
मेरुमूले तु राजेन्द्र ज्ञेयो वर्ष इलावृतः ।।२४.३७ ।।
उत्तरणेण (उत्तरेण?) तु नीलस्य वर्ष रम्यक उच्यते ।
श्वेत ।उत्तरतो ज्ञेयो वर्षरम्यहिरण्मयः ।।२४.३८ ।।
तस्य उत्तरतो ज्ञेयस्त्रिशृङ्गवरपर्वतः ।
तस्य चोत्तरपार्श्वे तु वर्षः कुरुवले स्मृतः ।।२४.३९ ।।
पूर्वं भद्राश्वतो ज्ञेयः केतुमालस्तु पश्चिमे ।
हिमंवान्हेमकूटश्च निषधो नील एव च ।।२४.४० ।।
श्वेतश्च शृङ्गवन्तश्च षडेते वर्षपर्वताः ।
अशीतिनवतीलक्षः  । वर्षपर्वतमायतम् ।।२४.४१ ।।
हिमवान्हेमकूटश्च निषधश्चेति दक्षिण ।
श्वेतश्चैवत्रिशृङ्गश्च नीलश्चैव तथोत्तरे ।।२४.४२ ।।
निषधो नीलमध्ये तु मेरुः शैलमनोरमः ।
प्रविष्टषोडशाधस्तां चतुराशीतिमुच्छृतः ।।२४.४३ ।।
योजनानां सहस्राणि द्वात्रिंशदूर्ध ! विस्तृतः ।
ब्रह्मामनोवती नाम पुरेव सतिमध्यमे ।।२४.४४ ।।
देवराजो ऽमरावत्यामग्निस्तेजोवती पुरे ।।२४.४५ ।।
यमः संयमनी नाम नित्यं वसति भूपते ।
नैऋतिर्वसति नित्यं रम्ये शुद्धवती पुरे ।।२४.४६ ।।
वरुणो भोगवत्यां तु वायोर्गन्धवती पुरी ।
महोदयापुरी रम्या सोमस्यालयरं स्मृतम् ।।२४.४७ ।।
यशोवती पुरी रम्यान्नित्यमास्ते त्रिशूलिनः ।
तत्रगङ्गा चतुःभिन्ना निपतन्ती महीतले ।।२४.४८ ।।
उत्तरे पश्चिमे चैव पूर्वदक्षिणतस्तथा ।
पूर्वं गङ्गा स्रवत्याच्चालकानन्दा च दक्षिणे ।।२४.४९ ।।
शीता पश्चिमगा गङ्गा भद्रसोमा तथोत्तरे ।
षष्टियोजनसाहस्रं निरालम्बा निपत्य च ।।२४.५० ।।
भद्राश्वं प्लावयित्वा तु वनान्युपवनानि च ।
द्रोणस्थली गिरीणां च अतिक्रम्यार्णवं गता ।।२४.५१ ।।
तथैवालकनन्दा च गताशैलेननिम्नगा ।
गङ्गा भारतवर्षं च प्रविष्टालवणो दधिम् ।।२४.५२ ।।
प्लावयित्वा स्थलीन्सर्वान्मानुषाकलुषापहा ।
पश्चिमेन गतागङ्गा सीतानामा च भारतः ।।२४.५३ ।।
प्लावयेत्केतुमालां च क्षेत्रशैववनस्थलीम् ।
अतिक्रम्यार्णवगता स्थलीद्रोणी च निम्नगा ।।२४.५४ ।।
भद्रसोमनदीत्येवं प्लावयित्वोत्तरं कुरुन् ।
स्थली प्रस्रवणद्रोणीमतिक्रम्यार्णवं गता ।।२४.५५ ।।
मेरो वै दक्षिणे पार्श्वे जम्बूवृक्षः सनातनः ।
तेन नामाङ्कितो राजन्जम्बूद्वीप इति श्रुतम् ।।२४.५६ ।।
कोटीषोडशभिश्चैव अयुतानि त्रयोदश ।
अधोर्धयाम राजेन्द्र क्षित्यावरणमन्ततः ।।२४.५७ ।।
नवलक्षाधिकं राजन्पञ्चकोटी मही स्मृता ।
योजनानां तु विज्ञेयः पृथिव्यायामविस्तरात् ।।२४.५८ ।।
स्वादूदकस्य च बहिर्लोकालोको महागिरिः ।
कञ्चनिद्विगुणाभूमि तस्माद्गिरिबहि स्मृतः ।।२४.५९ ।।
तस्माद्बाह्यः समुद्रो भूद्गर्भादेति समुद्रराट् ।
अष्टाविंशतिकं लक्षं शतलक्षाणि विस्तरम् ।।२४.६० ।।
एतद्भूर्लोकविस्तारो ह्यत ऊर्ध्व भुवः स्मृतः ।
स्वर्लोकास्यपरेणैव महर्लोकमतः परम् ।।२४.६१ ।।
जनलोकस्तपः सत्यं क्रमशः परिकीर्तितम् ।
ब्रह्मलोकः स्मृतः सत्यं विष्णुलोकमतः परम् ।।२४.६२ ।।

---- शिवलोकः ----

तस्मात्परेण बोधव्यं दिव्यध्यानपुरं महत् ।
सहस्रभौमप्रासादं वैदूर्यमणितोरणम् ।।२४.६३ ।।
नानारत्नविचित्राणि नानाभूतगणाकुलम् ।
सर्वकामसमृद्धानि पूर्णं तत्र मनोहरैः ।।२४.६४ ।।
तत्र सिंहासने दिव्ये सर्वरत्नविभूषिते ।
तत्रास्ते भगवान्रुद्रः सोमाङ्कितजटाधरः ।।२४.६५ ।।
त्र्यक्षत्रिभुवनश्रेष्ठस्त्रिशूली त्रिदशाधिपः ।
देव्या सह महाभागो गणैश्च परिवारितः ।।२४.६६ ।।
स्कन्दनन्दिपुरोगश्च गणकोटिशताकुलः ।
अनेकरुद्रकन्याभि रूपिणीभिरलङ्कितः ।।२४.६७ ।।
तत्र पुण्यनदी सप्त सर्वपापापनोदनी ।
सुवर्णवालुकादिव्या रत्नपाषाणशोभिता ।।२४.६८ ।।
पावनी च वरेण्या च वरार्हावरदा वरा ।
वरेशावरभद्रा च सुप्रसन्ना जलाशिवा ।।२४.६९ ।।
अनेककुसुमारामा रत्नपुष्पफलद्रुमाः ।
अनेकरत्नप्राकारा योजनायुतमुच्छ्रिताः ।।२४.७० ।।
अहिंसासत्यनिरताः कामक्रोधविवर्जिताः ।
ध्यानयोगरतानित्यं तत्र मोदन्ति ते नराः ।। २४.७१ ।।
तत्र गोमातरस्सर्वा निवसन्ति यतव्रताः ।
गोलोकः शिवलोकश्च एक एव विधीयते ।।२४.७२ ।।
तस्मादूर्धं परं ज्ञेयं स्थानत्रयमनुत्तमम् ।
कन्दगौरी महेशानं नित्यशुद्धं परं शिवम् ।।२४.७३ ।।
दिनकृत्कोटिसङ्कासमनोपम्यं सनातनम् ।
आदित्याद ! शिवान्तश्च द्विस्थेणोर्ध्वक्रमैः मृस्तः (स्मृतः) ।।२४.७४ ।।

---- शास्त्रवर्णना ----

अभ्यन्तरे तत्कथितो ऽद्य सारं
किमन्य राजन्कथयामि सारम् ।
ज्ञानार्णवं कीर्तित धर्मसारम
पुराणवेदोपनिषत्सुसारम् ।।२४.७५ ।।
यथा हि राजा परिवारमध्ये
यथान्तवर्ती बहिवर्तिनेव ।
भुञ्जन्ति भोगान्सततान्तवर्ती
क्लेशाधिकं नित्य बहिःस्थितानाम् ।।२४.७६ ।।
यथैव राजा करिणो ऽन्तदन्तम
भुञ्जन्ति भोगान्सततं नरेन्द्र ।
युध्येत राजा बहिर्दन्तभोगैर
यदन्तरं पश्य समानजातम् ।।२४.७७ ।।
न दानतुल्यं त्वभयप्रदस्य
न यज्ञतुल्यं जित ।इन्द्रियस्य ।
न चार्थतुल्यं जितकामिनश्च
न धर्मतुल्यं दमकामितस्य ।।२४.७८ ।।
बह्वन्तरं नैव हि धर्मयोश्च
क्लेशाधिकं बाह्यफलाल्पसारम् ।
यदत्र धर्मं फलनैष्ठिकस्य
न तुल्य कोटीशतयाजिनापि ।।२४.७९ ।।
एतत्पवित्रं परमं सधर्मम
पुरा यथोक्तं परमेश्वरेण ।
मयापि तुल्यं कथितं यथावत
पुराणवेदोपनिषत्सुसारम् ।।२४.८० ।।
सदोजसौभाग्यमतीव मेधा
निरुत्सुकः सौम्यमनुत्तमं च ।
सुपुत्रपौत्रं न विछिन्नगोत्रम
भवन्ति विद्याधरलोकपूज्यम् ।।२४.८१ ।।
यशश्रियं कीर्तिरतीव तेजो
जनप्रियो धान्यधनायुवृद्धिम् ।
प्रबोधप्रज्ञारुजधर्मवृद्धिम
भवन्ति तं शास्त्रसदाभियोगी ।।२४.८२ ।।
यशस्विनी आर्यसुवर्णशृङ्गी
वेदान्तविप्रद्विजगायनेषु ।
दत्त्वा फलं तीर्थमनुत्तमेषु
शृण्वन्ति ये तस्य भवेत्सपुण्यम् ।।२४.८३ ।।
दशाधिकं वाचयितुश्च पुण्यम
शताधिकं यः पठति प्रभाते ।
सहस्रशः पुस्तकृतस्य पुण्यम
परे ऽभ्यस्ते कीर्तयते ऽयुतानि ।।२४.८४ ।।
अधीत्य यस्योरगतं सुशास्त्रम
समस्तमध्यायमनुक्रमेन ।
दशायुताङ्गो ददतुश्च पुण्यम
लभत्यसंदिग्धयथादिनैकं ।।२४.८५ ।।
येनेदं शास्त्रसारमविकलमनसा यो ऽभ्यसेत्तत्प्रयत्नात
व्यक्तो ऽसौ सिद्धयोगी भवति च नियतं यस्तु चित्तप्रसन्नः ।
पित्र्यं यो गीतपूर्वं प्रतिदिन शतश उद्ध्रियन्ते च सर्वे
आत्मानं निर्विकल्पं शिवपदमसमं प्राप्नुवन्तीह सर्वे ।।२४.८६ ।।

 ।।इति वृषसारसंग्रहे शास्त्रवर्णना नाम चतुर्विंशतितमो ऽध्यायः समाप्तः ।।

 ।।वृषसारसंग्रहः समाप्त इति ।।
 
 
\end{landscape}


\end{document}
