\renewcommand{\dnapp}[1]{}\renewcommand{\rmapp}[1]{#1}
\fejno=0\versno=0

\vers

\vers

\vers

\vers

\vers

\vers
\szam\bek\versno=0\fejno=4



\alfejezet{\textbf{caturtho 'dhyāyaḥ}}\jump\jump

\alalfejezet{yamaḥ 2: satyam}
\vers

anarthayajña uvāca~{\dandab}\dontdisplaylinenum 

sadbhāvaḥ satyam ity āhur d\textsubring{r}ṣṭapratyakṣam eva vā\thinspace{\danda} \dontdisplaylinenum
            \var{\va sadbhāvaḥ\lem  \msCa; sadbhāva° \Ed\oo
                 satyam ity āhur\lem  \Ed; satyam \uncl{i}ty āhu \msCa}%
            \var{\vb °pratyakṣam\lem  \Ed; °pratyayam \msCa}%

yathābhūtārthakathana\.m tat satyakathana\.m sm\textsubring{r}tam \veg\dontdisplaylinenum

ākrośatāḍanādīni yaḥ saheta suduḥsaham\thinspace{\dandab} \dontdisplaylinenum

kṣamate yo jitātmā tu sa ca satyam udāh\textsubring{r}tam \veg\dontdisplaylinenum
            \var{\vd satyam udāh\textsubring{r}tam\lem  \Ed; 
                        \uncl{satya}m u\uncl{dā}h\textsubring{r}tam \msCa}%

vadhārtham udyataḥ śastra\.m yadi p\textsubring{r}ccheta karhicit\thinspace{\dandab} \dontdisplaylinenum
            \var{\va śastra\.m\lem  \msCa; satya \Ed}%

na tatra satya\.m vaktavyam an\textsubring{r}ta\.m satyam ucyate \veg\dontdisplaylinenum
            \var{\vc satya\.m\lem  \msCa; satya \Ed\oo
                 vaktavyam\lem  \Ed; vaktayā \msCa}%

vadhārhaḥ puruṣaḥ kaścid vrajet pathi bhayāturaḥ\thinspace{\dandab} \dontdisplaylinenum

p\textsubring{r}cchato 'pi na vaktavya\.m satya\.m tad vāpi ucyate \veg\dontdisplaylinenum
            \var{\vc p\textsubring{r}cchato\lem  \msCa; p\textsubring{r}cchate \Ed}%

\ujvers\nemsloka 
! na narmayuktam an\textsubring{r}ta\.m hinasti
\dontdisplaylinenum
    \var{\va hinasti\lem  \msCa; hi nāsti \Ed}%

\nemslokab 
na strīṣu rājan na vivāhakāle \danda\dontdisplaylinenum

\nemslokac 
prāṇātyaye sarvadhanāpahāre
\dontdisplaylinenum

\nemslokad 
pañcān\textsubring{r}ta\.m satyam udāharanti \veg\dontdisplaylinenum
    \paral{\textit{\vo {\normalfont cf.\ \MBh\ 1.77.16: } na narmayukta\.m vacana\.m hinasti na strīṣu rājan na vivāhakāle{\thinspace\danda}
        prāṇātyaye sarvadhanāpahāre pañcān\textsubring{r}tāny āhur apātakāni{\thinspace\ketdanda};
        {\normalfont \MBh\ 12.159.28: } na narmayukta\.m vacana\.m hinasti na strīṣu rājan na vivāhakāle{\thinspace\danda}
        na gurvarthe nātmano jīvitārthe pañcān\textsubring{r}tāny āhur apātakāni{\thinspace\ketdanda};
     {\normalfont \MP\ 31.16: } na narmayukta\.m vacana\.m hinasti na strīṣu rājan na vivāhakāle{\thinspace\danda}
             prāṇātyaye sarvadhanāpahāre pañcān\textsubring{r}tāny āhur apātakāni{\thinspace\ketdanda};
{\normalfont Kauṇḍinya's commentary ad \PS\ 1.9: }
        gobrāhmaṇārthe 'vacana\.m himasti na strīṣu rājan na vivāhakāle{\thinspace\danda}
        prāṇātyaye sarvadhanāpahāre pañcān\textsubring{r}tāni āhur apātakāni{\thinspace\ketdanda};
{\normalfont Abhidharmakośabhāṣya  24114--24117 }:
        na narmayuktam an\textsubring{r}ta\.m hi nāsti na strīṣu rājan na vivāhakāle{\thinspace\danda}
        prāṇātyaye sarvadhanāpahāre pañcān\textsubring{r}tāñ ? āhur apātakāni{\thinspace\ketdanda} }}

\vers

devamānuṣatiryeṣu satyadharmaparāyaṇaḥ\thinspace{\dandab} \dontdisplaylinenum
            \var{\vb satyadharmaparāyaṇaḥ\lem  \Ed; satya\.m dharmaḥ payataḥ \msCa}%

satya\.m śreṣṭha\.m variṣṭha\.m ca satya\.m dharmaḥ sanātanaḥ \veg\dontdisplaylinenum
            \var{\vc śreṣṭha\.m\lem  \msCa; śreṣṭha \Ed}%
            \var{\vd dharmaḥ\lem  \msCa; dharma \Ed}%

satya\.m sāgaram avyakta\.m satyam akṣayabhogadam\thinspace{\dandab} \dontdisplaylinenum
            \var{\vb akṣayabhogadam\lem  \msCa; akṣayate nara\.m \Ed}%

satya\.m potaḥ paratrārtha\.m satya\.m yaj jñānavistaram \veg\dontdisplaylinenum
            \var{\vc potaḥ\lem  \msCa; proktaḥ \Ed}%

satyam iṣṭagatiḥ prokta\.m satya\.m yajñam anuttamam\thinspace{\dandab} \dontdisplaylinenum

satya\.m tīrthāt para\.m tīrtha\.m satya\.m dānam anantakam \veg\dontdisplaylinenum
            \var{\vc tīrthāt\lem  \Ed; tīrtha\.m \msCa}%

satya\.m śīla\.m tapo jñāna\.m satya\.m śauca\.m damaḥ śamaḥ\thinspace{\dandab} \dontdisplaylinenum

satya\.m sopānam ūrdhvasya satya\.m kīrtir yaśaḥ sukham \veg\dontdisplaylinenum
            \var{\vd sukham\lem  \msCa; sukhaḥ \Ed}%

aśvamedhasahasra\.m ca satya\.m ca tulayā dh\textsubring{r}tam\thinspace{\dandab} \dontdisplaylinenum
            \paral{\textit{\vo {\normalfont  \kb\ Mārkaṇḍeyapurāṇa 8.42: }
                aśvamedhasahasra\.m ca satya\.m ca tulayā dh\textsubring{r}tam{\thinspace\danda}
                aśvamedhasahasrād dhi satyam eva viśiṣyate{\thinspace\ketdanda}}}

aśvamedhasahasrād dhi satyam eva viśiṣyate \veg\dontdisplaylinenum
            \var{\vd eva\lem  \msCa; eva\.m \Ed}%
            \paral{\textit{\vcd {\normalfont  = MBh 1.69.22cd and 13.74.29cd }}}

satyena tapate sūryaḥ satyena p\textsubring{r}thivī sthitā\thinspace{\dandab} \dontdisplaylinenum
            \var{\vab sūryaḥ satyena p\textsubring{r}thivī sthitā\lem  \corr;
                sū\uncl{ryaḥ sa}tyena p\textsubring{r}thi sthitāḥ \msCa;
                sūryaḥ satyena p\textsubring{r}thivī sthitāḥ \Ed}%

satyena vāyavo vānti satyāt toya\.m ca śītalam \veg\dontdisplaylinenum
            \var{\vd satyāt\lem  \Ed; satyo \msCa}%

tiṣṭhanti sāgarāḥ satye satyena ca priyavrataḥ\thinspace{\dandab} \dontdisplaylinenum
            \var{\vb satyena ca\lem  \Ed; samayena \msCa}%

satye tiṣṭhati govindo balibandhanakāraṇāt \veg\dontdisplaylinenum

agnir dahati satyena satyena śaśibhāṣkaraḥ\thinspace{\dandab} \dontdisplaylinenum
            \var{\vb śaśibhāṣkaraḥ\lem  \Ed; saśi\uncl{bhācaraḥ} \msCa}%

satyena vindhyās tiṣṭhante vardhamāno na vardhate \veg\dontdisplaylinenum
            \var{\vc vindhyās tiṣṭhante\lem  \msCa; tiṣṭhate vindhyo \Ed}%

lokālokaḥ sthitaḥ satya\.m meruḥ satye pratiṣṭhitaḥ\thinspace{\dandab} \dontdisplaylinenum
            \var{\va °lokaḥ\lem  \Ed; °loka \msCa}%
            \var{\vb meruḥ\lem  \msCa; meru \Ed}%

vedās tiṣṭhanti satyeṣu dharmaḥ satye pratiṣṭhati \veg\dontdisplaylinenum
            \var{\vc vedās\lem  \msCa; vedā \Ed}%

satya\.m gauḥ kṣarate kṣīra\.m satya\.m kṣīra\.m gh\textsubring{r}ta\.m sthitam\thinspace{\dandab} \dontdisplaylinenum
            \var{\vb kṣīre gh\textsubring{r}ta\.m sthitam\lem  \msCa; kṣīra\.m sthita\.m gh\textsubring{r}tam \Ed}%

satye jīvaḥ sthito dehe satya\.m jīvaḥ sanātanaḥ \veg\dontdisplaylinenum
            \var{\vc satye jīvaḥ\lem  \msCa; satya\.m jīva \Ed}%

satyam ekena samprāpto dharmaḥ sādhananiścayaḥ\thinspace{\dandab} \dontdisplaylinenum
            \var{\vb dharmaḥ\lem  \msCa; dharma \Ed\oo
                 °niścayaḥ\lem  \Ed; °niścaḥ \msCa}%

rāmarāghavavīryeṇa satyam eka\.m surakṣitam \veg\dontdisplaylinenum

etat satyavidhānasya kīrtita\.m tava suvrata\thinspace{\dandab} \dontdisplaylinenum
            \var{\vb suvrata\lem  \msCa; suvrata\.m \Ed}%

sarvalokahitārthāya kim anyac chrotum icchasi \veg\dontdisplaylinenum

vigatarāga uvāca~{\dandab}\dontdisplaylinenum 

na hi t\textsubring{r}pti\.m vijānāmi dharma\.m śrutvā tathāpy aham\thinspace{\danda} \dontdisplaylinenum
            \var{\vb dharma\.m śrutvā tathāpy aham\lem  \Ed; 
                        śru dharman tavāmy aham \msCa}%

upariṣṭād ato bhūyaḥ kathayasva tapodhana \veg\dontdisplaylinenum
            \var{\vd °dhana\lem  \Ed; °dhūna \msCa}%


\alalfejezet{yamaḥ 3: asteyam}
anarthayajña uvāca~{\dandab}\dontdisplaylinenum 

steya\.m ś\textsubring{r}ṇv atha viprendra pañcadhā parikīrtitam\thinspace{\danda} \dontdisplaylinenum

adattādānam ādau tu utkoca\.m ca tataḥ param \veg\dontdisplaylinenum
            \var{\vd ca tataḥ\lem  \msCa; cān\textsubring{r}taḥ \Ed}%

prasthavyājas tulāvyājaḥ prasahyastena pañcamam\thinspace{\dandab} \dontdisplaylinenum
            \var{\va tulāvyājaḥ\lem  \Ed; tulāvyāja \msCa}%
            \var{\vb °stena\lem  \msCa; °steya \Ed\oo
                 pañcamam\lem  \msCa; pañcamaḥ \Ed}%

dh\textsubring{r}taduṣṭaprabhāvena paradravyāpakarṣaṇam \veg\dontdisplaylinenum
            \var{\vc dh\textsubring{r}ta°\lem  \msCa; dh\textsubring{r}ṣṭa° \Ed}%

vāryamāṇo 'pi durbuddhir adattādānam ucyate\thinspace{\dandab} \dontdisplaylinenum

utkoca\.m ś\textsubring{r}ṇu viprendra dharmasa\.mkarakārakam \veg\dontdisplaylinenum
            \var{\vc utkoca\.m\lem  \Ed; utkoca \msCa}%
            \var{\vd °sa\.mkara°\lem  \eme; °śaṅkara° \msCa, °sa\.mhāra° \Ed}%

mūlakāryavināśārtham utkocaḥ parig\textsubring{r}hyate\thinspace{\dandab} \dontdisplaylinenum
            \var{\vb utkocaḥ\lem  \msCa; utkoca \Ed}%

tena cāsau vijānīyād dravyalobhabalāt k\textsubring{r}tam \veg\dontdisplaylinenum

prasthavyāja upāyena kuṭumba\.m trātum icchati\thinspace{\dandab} \dontdisplaylinenum

ta\.m ca stenam vijānīyāt paradravyāpahārakam \veg\dontdisplaylinenum
            \var{\vc ta\.m ca stenam\lem  \msCa; so 'pi tena \Ed}%

tulāvyāja upāyena parasvārtha\.m hared yadi\thinspace{\dandab} \dontdisplaylinenum
            \var{\vb parasvārtha\.m\lem  \msCa; parasyārtha\.m \Ed}%

cauralakṣaṇakāś cānye kūṭakāryaṭikā narāḥ \veg\dontdisplaylinenum
            \var{\vd kūṭakāryaṭikā\lem  \Ed; \uncl{ku}ṭakāyaṭikā \msCa}%

durbalārjavabāleṣu cchadmanā vā balena vā\thinspace{\dandab} \dontdisplaylinenum
            \var{\vb cchadmanā\lem  \Ed; cchanmanā \msCa}%

apah\textsubring{r}tya dhana\.m mūḍhaḥ sa coraś cora ucyate \veg\dontdisplaylinenum

nāsti stenasama\.m pāpa\.m nāsty adharmaś ca tatsamaḥ\thinspace{\dandab} \dontdisplaylinenum
            \var{\vo \om\ \Ed}%
            \var{\va stenasamo\lem  \eme; tena samam \msCa}%

nāsti stenasamo 'kīrtir nāsti stenasamo 'nayaḥ \veg\dontdisplaylinenum

nāsti stenasamo 'vidyā nāsti steyasamaḥ khalaḥ\thinspace{\dandab} \dontdisplaylinenum
            \var{\va stena°\lem  \msCa; steya° \Ed}%
            \var{\vb stena°\lem  \msCa; tena \Ed}%

nāsti stenasama ajño nāsti stenasamo 'lasaḥ \veg\dontdisplaylinenum
            \var{\vc stena°\lem  \msCa; steya° \Ed\oo
                 ajño\lem  \eme; ajña{\il} \msCa, ajñaḥ \Ed}%
            \var{\vd stena°\lem  \msCa; tena \Ed}%

nāsti stenasamo dveṣyo nāsti steyasamo 'priyaḥ\thinspace{\dandab} \dontdisplaylinenum
            \var{\va stena°\lem  \msCa; tena \Ed}%

nāsti stenasama\.m duḥkha\.m nāsti stenasamo 'yaśaḥ \veg\dontdisplaylinenum
            \var{\vc stena°\lem  \msCa; tena \Ed}%
            \var{\vd stena°\lem  \msCa; tena \Ed}%

\ujvers\nemsloka 
pracchanno hriyate ca vittam athavā pratyakṣyam anyo haret
\dontdisplaylinenum
            \var{\va ca\lem  \Ed; \om\ \msCa\oo
                 athavā\lem  \Ed; \om\ \msCa\oo
                 anyo\lem  \msCa; anye \Ed}%

\nemslokab 
nikṣepād dhanahāriṇo 'nyavidhayo vyājena cānyo haret \danda\dontdisplaylinenum
            \var{\vb nikṣepād dhana\lem  \msCa; nikṣepātraya° \Ed\oo
                'nyavidhayo\lem  \Ed; 'nyamadhamo \msCa\oo
                cānyo\lem  \msCa; cānyā \Ed}%

\nemslokac 
anyo lekhyavikalpanāh\textsubring{r}tadhanā anyo h\textsubring{r}tād vai h\textsubring{r}tā
\dontdisplaylinenum
            \var{\vc anyo lekhya\lem  \msCa; anyollekhya \Ed}%

\nemslokad 
! anyaḥ krītadhano paro dhayah\textsubring{r}ta ete jaghanyāḥ sm\textsubring{r}tāḥ \veg\dontdisplaylinenum
            \var{\vd anyaḥ krītadhano\lem  \msCa; anāśrītadhana\.m \Ed\oo
                 paro dhayah\textsubring{r}ta\lem  \msCa; madā hy apah\textsubring{r}ta\.m \Ed\oo
                 jaghanyāḥ\lem  \msCa; jaghanyaḥ \Ed}%

\ujvers\nemsloka 
stena\.m tulya na mūḍham asti puruṣo dharmārthahīno 'dhamaḥ
\dontdisplaylinenum
            \var{\va stenam\lem  \msCa; stenas \Ed}%

\nemslokab 
yāvaj jīvati śaṅkayā narapateḥ sa\.mtrasyamāno śaṭhaḥ \danda\dontdisplaylinenum
            \var{\vb yāvaj jīvati\lem  \msCa; yāvat taj jīvati \Ed\oo
                 °pateḥ\lem  \conj; °patis \msCa\Ed\oo
                 sa\.mtrasyamāno\lem  \msCa; sa\.mtrāsyamāno \Ed\oo
                 śaṭhaḥ\lem  \Ed; raṭan \msCa}%

\nemslokac 
prāptaḥ śāsanatīvrasadyaviṣamaḥ prāpnoti karmeritaḥ
\dontdisplaylinenum
            \var{\vc °sahya°\lem  \msCa; °sadya° \Ed\oo
                 karmeritaḥ\lem  \Ed; karme\uncl{rita} \msCa}%

\nemslokad 
! kālena mriyate sa yāti nirayam ākrandamāno bh\textsubring{r}śam \veg\dontdisplaylinenum
            \var{\vd nirayam ākrandamāno\lem  \msCa; niyamam ākrandramāno \Ed}%

\ujvers\nemsloka 
nītvā durgatikoṭikalpanirayān tiryaktvam āyānti te
\dontdisplaylinenum
            \var{\va tiryaktvam\lem  \eme; tiryatvam \msCa tiryaktvā \Ed}%

\nemslokab 
tiryaktve ca tathaikam ekaśatika\.m prabhramya varṣāmbudaḥ \danda\dontdisplaylinenum
            \var{\vb tiryaktve\lem  \corr; tiryaktva\.m \Ed, tiryatve \msCa\oo
                °śatika\.m\lem  \msCa; °sakika\.m \Ed\oo
                varṣāmbudaḥ\lem  \Ed; varṣāmbudam \msCa}%

\nemslokac 
mānuṣya\.m tad avāpnuvanti vipula\.m dāridryarogākulam
\dontdisplaylinenum
            \var{\v vipula\.m\lem  \Ed; vipule \msCa\oo
                dāridrya°\lem  \msCa; dāridhra° \Ed}%

\nemslokad 
tasmād durgatihetukarma sakala\.m tyaktvā śiva\.m cāśrayet \veg\dontdisplaylinenum


\alalfejezet{yamaḥ 4: ān\textsubring{r}śa\.msyam}
\vers

aṣṭamūrtiśivadveṣṭā pitur mātuś ca yo dviṣet\thinspace{\dandab} \dontdisplaylinenum

gavā\.m vā atither dveṣṭā n\textsubring{r}śa\.msāḥ pañca eva te \veg\dontdisplaylinenum
            \var{\vd n\textsubring{r}śa\.msāḥ\lem  \msCa; n\textsubring{r}śa\.msā \Ed}%

aṣṭamūrtiḥ śivaḥ sākṣāt pañcavyomasamanvitaḥ\thinspace{\dandab} \dontdisplaylinenum
            \var{\va °mūrtiḥ\lem  \msCa; °mūrti° \Ed}%

sūryaḥ somaś ca dīkṣaś ca dūṣakaḥ sa n\textsubring{r}śa\.msakaḥ \veg\dontdisplaylinenum
            \var{\vc sūryaḥ\lem  \msCa; sūrya° \Ed\oo
                 dīkṣaś\lem  \msCa; dīkṣuś \Ed}%

pitākāśasamo jñeyo janmotpattikaraḥ pitā\thinspace{\dandab} \dontdisplaylinenum

pit\textsubring{r}daivatam ādityam ān\textsubring{r}śa\.msa tato 'nvitaḥ \veg\dontdisplaylinenum
            \var{\vc ādityam\lem  \Ed; ādiścam \msCa}%
            \var{\vd ān\textsubring{r}śa\.msa tato 'nvitaḥ\lem  \Ed; ān\textsubring{r}śa\.mśatamanvitaḥ \msCa}%

p\textsubring{r}thvyā\.m gurutarī mātā ko na vandeta mātaram\thinspace{\dandab} \dontdisplaylinenum
            \var{\va p\textsubring{r}thvyā\.m\lem  \Ed; p\textsubring{r}thvyā \msCa}%

yajñadānatapo vedās tena sarvak\textsubring{r}ta\.m bhavet \veg\dontdisplaylinenum

gāvaḥ pavitra\.m maṅgalya\.m devatānā\.m ca devatāḥ\thinspace{\dandab} \dontdisplaylinenum
            \var{\va maṅgalya\.m\lem  \msCa; māṅgalya\.m \Ed\oo
                 devatāḥ\lem  \msCa; devatā \Ed}%

sarvadevamayā gāvas tasmād eva na hi\.msayet \veg\dontdisplaylinenum
            \var{\vd eva\lem  \msCa; gāva\.m \Ed}%

jātamātrasya lokasya gāvas trātā na sa\.mśayaḥ\thinspace{\dandab} \dontdisplaylinenum

! gh\textsubring{r}ta\.m kṣīra\.m dadhi mūtra\.m śak\textsubring{r}t karṣaṇam eva ca \veg\dontdisplaylinenum

\ujvers\nemsloka 
pañcām\textsubring{r}ta\.m pañcapavitrapūtam
\dontdisplaylinenum
            \var{\va °pūtam\lem  \Ed; °pūtana \msCa}%

\nemslokab 
ye pañcagavya\.m puruṣāḥ pibanti \danda\dontdisplaylinenum
            \var{\vb puruṣāḥ\lem  \msCa; puruṣaḥ \Ed}%

\nemslokac 
te vājimedhasya phala\.m labhanti
\dontdisplaylinenum

\nemslokad 
tad akṣaya\.m svargam avāpnuvanti \veg\dontdisplaylinenum

\ujvers\nemsloka 
na gāvatulya\.m dhanam asti ki\.mcid
\dontdisplaylinenum
            \var{\va gāva°\lem  \Ed; gobhis \msCa\ \unmetr}%

\nemslokab 
duhyanti vāhyanti bahiścaranti \danda\dontdisplaylinenum

\nemslokac 
t\textsubring{r}ṇāni bhuktvā am\textsubring{r}ta\.m sravanti
\dontdisplaylinenum

\nemslokad 
vipreṣu dattāḥ kulam uddharanti \veg\dontdisplaylinenum
            \var{\vd dattāḥ\lem  \msCa; dattā \Ed}%

\ujvers\nemsloka 
gavāhnika\.m yaḥ prakaroti nityam
\dontdisplaylinenum
            \var{\va gavāhnika\.m\lem  \Ed; gavā\.mhnika\.m \msCa\oo
                 prakaroti\lem  \Ed; ca karoti \msCa}%

\nemslokab 
śuśrūṣaṇa\.m yaḥ kurute gavānām \danda\dontdisplaylinenum
            \var{\vb gavānām\lem  \Ed; gavān tu \msCa}%

\nemslokac 
! aśeṣayajñatapadānapuṇyam
\dontdisplaylinenum
            \var{\vc °tapa°\lem  \msCa; °japa° \Ed}%

\nemslokad 
bhavaty asau dharmam aśeṣakartā \veg\dontdisplaylinenum
            \var{\vd bhavaty asau dharmam aśeṣakartā\lem  \Ed; 
                bhaty asau bhaman\textsubring{r}śa\.msakartā \msCa}%

\vers

atithi\.m yo 'nugaccheta atithi\.m yo 'numanyate\thinspace{\dandab} \dontdisplaylinenum

atithi\.m yo 'nupūjyeta atithi\.m yaḥ praśa\.msate \veg\dontdisplaylinenum

atithi\.m yo na pīḍyeta atithi\.m yo na duṣyati\thinspace{\dandab} \dontdisplaylinenum

atithipriyakartā yaḥ atitheḥ paricārakaḥ \veg\dontdisplaylinenum
            \var{\vc atithi°\lem  \msCa; atithi\.m \Ed\oo
                 yaḥ\lem  \Ed; yar \msCa}%

atithik\textsubring{r}tasa\.mtoṣas tasya puṇyam anantakam\thinspace{\dandab} \dontdisplaylinenum
            \var{\va atithi°\lem  \msCa; atithi\.m \Ed}%

āsanenārghyapādyena pādaśaucajalena ca \veg\dontdisplaylinenum
            \var{\vc °ārghya°\lem  \msCa; °ārdhya° \Ed}%

annavastrapradānair vā sarva\.m vāpi pradāpayet\thinspace{\dandab} \dontdisplaylinenum

putradārātmako vāpi yo 'tithim anupūjayet \veg\dontdisplaylinenum
            \var{\vc °dārātmako\lem  \Ed; °\uncl{dārā}tmano \msCa}%

śraddhāyā cāvikalpena aklībamānasena ca\thinspace{\dandab} \dontdisplaylinenum
            \var{\va cāvikalpena\lem  \Ed; cāpi kalpena \msCa}%

na p\textsubring{r}cched gotravaraṇa\.m svādhyāya\.m deśam eva vā \veg\dontdisplaylinenum
            \var{\vc °caraṇa\.m\lem  \msCa; °pravara\.m \Ed}%
            \var{\vd deśam eva vā\lem  \eme; deśajanmanā \msCa, deśajanmanī \Ed}%
            \paral{\textit{ {\normalfont \vcd cf.\ \MBh\ 13.62.18ab:
                } na p\textsubring{r}cched gotracaraṇa\.m svādhyāya\.m deśam eva vā}}

cintayen manasā bhaktyā dharmaḥ svayam ihāgataḥ\thinspace{\dandab} \dontdisplaylinenum

aśvamedhasahasrāṇi rājasūyaśatāni ca \veg\dontdisplaylinenum

puṇḍarīkasahasra\.m ca sarvatīrthatapaḥphalam\thinspace{\dandab} \dontdisplaylinenum

atithir yasya tuṣyeta n\textsubring{r}śa\.msam atam uts\textsubring{r}jet \veg\dontdisplaylinenum
            \var{\vd n\textsubring{r}śa\.msam atam uts\textsubring{r}jet\lem  \msCa;
                na sa\.mśaya samaśnute \Ed}%

sa tasya sakala\.m puṇya\.m prāpnuyān nātra sa\.mśayaḥ\thinspace{\dandab} \dontdisplaylinenum

na gatim atithijñasya gatim āpnoti karhicit \veg\dontdisplaylinenum
            \var{\vc na gatim\lem  \msCa; na tithim \Ed}%

tasmād atithim āyāntam abhigacchet k\textsubring{r}tāñjaliḥ\thinspace{\dandab} \dontdisplaylinenum

śaṅkuprasthena caikena yajña āsīn mahadbhutaḥ \veg\dontdisplaylinenum
            \var{\vc śaṅku°\lem  \msCa; śakti° \Ed}%
            \var{\vd °bhutaḥ\lem  \msCa; °bhutam \Ed}%

atithiprāptadānena svaśarīra\.m diva\.mgatam\thinspace{\dandab} \dontdisplaylinenum
            \var{\vb sva°\lem  \msCa; sa° \Ed}%

nakulena purādhīta\.m vistareṇa dvijottama \veg\dontdisplaylinenum
            \var{\vd dvijottama\lem  \msCa; dvijottamaḥ \Ed}%

vidita\.m ca tvayā pūrva\.m prasthavārtā ca kīrtitāḥ\thinspace{\dandab} \dontdisplaylinenum


\alalfejezet{yamaḥ 5: damaḥ}
dama eva manuṣyāṇā\.m dharmasārasamuccayaḥ \veg\dontdisplaylinenum
            \var{\vd dharmasāra°\lem  \eme; dharmabhāra° \msCa; dharmabhāra° \Ed}%

damo dharmo damaḥ svargaḥ damaḥ kīrtir damaḥ sukham\thinspace{\dandab} \dontdisplaylinenum

damo yajño damas tīrtha\.m damaḥ puṇya\.m damaḥ tapaḥ \veg\dontdisplaylinenum

damahīnam adharmaś ca damaḥ kāmakulapradaḥ\thinspace{\dandab} \dontdisplaylinenum
            \var{\vb damaḥ\lem  \msCa; dama\.m \Ed}%

nirdamaḥ karimīnaś ca pataṅgabhramaram\textsubring{r}gāḥ \veg\dontdisplaylinenum

tvagjihvā ca tathā ghrāṇā cakṣuḥ śravaṇam indriyāḥ\thinspace{\dandab} \dontdisplaylinenum
            \var{\vb indriyāḥ\lem  \msCa; indriyaḥ \Ed}%

durjayendriyam ekaika\.m sarve prāṇaharā sm\textsubring{r}tāḥ \veg\dontdisplaylinenum

dama\.m yo jayate samyak nirdamo nidhana\.m vrajet\thinspace{\dandab} \dontdisplaylinenum

m\textsubring{r}ge śrotravaśān m\textsubring{r}tyuḥ pataṅgāś cakṣuṣor m\textsubring{r}tāḥ \veg\dontdisplaylinenum
            \var{\vc m\textsubring{r}ge\lem  \msCa; m\textsubring{r}go \Ed}%
            \var{\vd pataṅgāś\lem  \msCa; pataṅgā \Ed}%

ghrāṇayā bhramaro naṣṭo naṣṭo mīnaś ca jihvayā\thinspace{\dandab} \dontdisplaylinenum

sparśena ca karī naṣṭo bandhanāvāsaduḥsahaḥ \veg\dontdisplaylinenum

ki\.m punaḥ pañcabhuktānā\.m m\textsubring{r}tyus tebhyaḥ kim adbhutam\thinspace{\dandab} \dontdisplaylinenum
            \var{\va punaḥ\lem  \msCapcorr\Ed; puna \msCaacorr}%
            \var{\vb tebhyaḥ\lem  \msCa; tebhya \Ed}%

purūravātilobhena atikāmena puṇḍakaḥ \veg\dontdisplaylinenum
            \var{\vc purūravā°\lem  \corr; purūravo \msCa; pururavā° \Ed\oo
                 °tilobhena atikāmena\lem  \msCa; °tikāmena atilobhena \Ed}%

sagaraś cātidarpeṇa atimānena rāvaṇaḥ\thinspace{\dandab} \dontdisplaylinenum

atikrodhena saudāsa atipāpena yādavāḥ \veg\dontdisplaylinenum
            \var{\vd atipāpena\lem  \Ed; atiyānena \msCa}%

atit\textsubring{r}ṣṇā ca mānāc ca nahuṣo dvijavajñayā\thinspace{\dandab} \dontdisplaylinenum
            \var{\va atit\textsubring{r}ṣṇā ca mānāc ca\lem  \conj;
                atit\textsubring{r}ṣṇā ca māndāto \msCa,
                atit\textsubring{r}ṣṇā ca mānāc ca ca \Ed}%

atidānād balir naṣṭa atiśauryeṇa arjunaḥ \veg\dontdisplaylinenum

atidyūtān nalo rājā n\textsubring{r}go goharaṇena tu\thinspace{\dandab} \dontdisplaylinenum
            \var{\va atidyūtān\lem  \msCa; atikhyātān \Ed}%
            \var{\vb n\textsubring{r}go\lem  \Ed; n\textsubring{r}gaṅ \msCa}%

tasmād dama\.m sadā rakṣet ati sarvatra varjayet \veg\dontdisplaylinenum
            \var{\vcd \om\ \msCa}%
            \var{\vc dama\.m sadā rakṣet\lem  \corr; dama sadā sa rakṣet \Ed}%

\ujvers\nemsloka 
damena hīnaḥ puruṣo dvijendra
\dontdisplaylinenum
            \var{\va hīnaḥ puruṣo dvijendra\lem  \msCa;
                hīna\.m puruṣa\.m dvijendraḥ \Ed}%

\nemslokab 
svarga\.m ca mokṣa\.m ca sukha\.m ca nāsti \danda\dontdisplaylinenum

\nemslokac 
! vijñānadharmakulakīrtināśo
\dontdisplaylinenum
            \var{\vc °nāśo\lem  \Ed; °nāma \msCa}%

\nemslokad 
! bhavanti viprā damayā vihīnāḥ \veg\dontdisplaylinenum
            \var{\vd viprā\lem  \conj; vipra \msCa\Ed}%


\alalfejezet{yamaḥ 6: gh\textsubring{r}ṇā}
\vers

nirgh\textsubring{r}ṇo na paratrāsti nirgh\textsubring{r}ṇo na ihāsti vai\thinspace{\dandab} \dontdisplaylinenum
            \var{\va nirgh\textsubring{r}ṇo\lem  \msCa; nirgh\textsubring{r}ṇe \Ed}%
            \var{\vb nirgh\textsubring{r}ṇo\lem  \msCa; nirgh\textsubring{r}ṇe \Ed}%

nirgh\textsubring{r}ṇe na ca dharmo 'sti nirgh\textsubring{r}ṇe na tapo 'sti vai \veg\dontdisplaylinenum

parastrīṣu parārtheṣu parajīvopakarṣaṇe\thinspace{\dandab} \dontdisplaylinenum

paranindāparānneṣu gh\textsubring{r}ṇā\.m pañcasu kārayet \veg\dontdisplaylinenum
            \var{\vc paranindā°\lem  \Ed; \uncl{paranind}ā° \msCa}%
            \var{\vd gh\textsubring{r}ṇā\.m\lem  \msCa; gh\textsubring{r}ṇā \Ed}%

parastrī ś\textsubring{r}ṇu viprendra gh\textsubring{r}ṇīkāryā sadā budhaiḥ\thinspace{\dandab} \dontdisplaylinenum

rājñī viprī parivrājā svayoniparayoniṣu \veg\dontdisplaylinenum
            \var{\vc °vrājā\lem  \msCa; °vrājyā \Ed}%

parārthe ś\textsubring{r}ṇu bhūyo 'nya anyāyārtham upārjanam\thinspace{\dandab} \dontdisplaylinenum

āḍhaprasthatulāvyājaiḥ parārtha\.m yo 'pakarṣati \veg\dontdisplaylinenum

jīvāpakarṣaṇe vipra gh\textsubring{r}ṇīkurvīta paṇḍitaḥ\thinspace{\dandab} \dontdisplaylinenum
            \var{\vb gh\textsubring{r}ṇī°\lem  \msCa; gh\textsubring{r}ṇā\.m \Ed}%

vanajā vanajā jīvā vihagācaraṇācarāḥ \veg\dontdisplaylinenum
            \var{\vd vihagācaraṇācarāḥ\lem  \conj; vilagācaraṇācarāḥ \msCa;
                                vilagocaragocaraḥ \Ed}%

paranindā ca kā vipra ś\textsubring{r}ṇu vakṣye samāsataḥ\thinspace{\dandab} \dontdisplaylinenum
            \var{\vb vakṣye\lem  \msCa; vakṣyā \Ed}%

devānā\.m brāhmaṇānā\.m ca gurumātātithidviṣaḥ \veg\dontdisplaylinenum

parānneṣu gh\textsubring{r}ṇā kāryā abhojyeṣu ca bhojanam\thinspace{\dandab} \dontdisplaylinenum

sūtake m\textsubring{r}take śauṇḍe varṇabhraṣṭakule naṭe \veg\dontdisplaylinenum
            \var{\vc śauṇḍe\lem  \conj; sauṇḍye \msCa; sauṇḍo \Ed}%

\ujvers\nemsloka 
ete pañcagh\textsubring{r}ṇāsu saktapuruṣaḥ svargārthamokṣārthinām
\dontdisplaylinenum

\nemslokab 
loke 'nindanam āpnuvanti satata\.m kīrtir yaśo'la\.mk\textsubring{r}tam \danda\dontdisplaylinenum
            \var{\vb 'nindanam āpnuvanti\lem  \msCa; nandanavāyuvānti \Ed}%

\nemslokac 
prajñābodhaśrutism\textsubring{r}ti\.m ca labhate māna\.m ca nitya\.m labhet
\dontdisplaylinenum

\nemslokad 
dākṣiṇya\.m sa bhavet sa mānuṣapara\.m prāpnoti niḥsa\.mśayaḥ \veg\dontdisplaylinenum
            \var{\vd mānuṣa°\lem  \Ed; māyuṣa° \msCa}%


\alalfejezet{yamaḥ 7: pañcadhanyavidhiḥ}
\vers

caturmaunaś catuḥśatruś catur āyatana\.m tathā\thinspace{\dandab} \dontdisplaylinenum
            \var{\va śatruḥ\lem  \msCa; śatru \Ed}%

catur dhyāna\.m catuṣpāda\.m pañcadhanyavidhocyate \veg\dontdisplaylinenum
            \var{\vd pañcadhanya°\lem  \msCa; dhanyapañca° \Ed}%

caturmaunasya vakṣyāmi ś\textsubring{r}ṇuṣvāvahito bhava\thinspace{\dandab} \dontdisplaylinenum

pāruṣyapiśunāmithyāsambhinnāni ca varjayet \veg\dontdisplaylinenum
            \var{\vc °piśunā°\lem  \msCa; °piṇḍānā° \Ed}%

kāmaḥ krodhaś ca lobhaś ca mohaś caiva caturvidhaḥ\thinspace{\dandab} \dontdisplaylinenum 

catuḥśatrur nihantavyaḥ sarvathā vītakalmaṣaḥ \veg\dontdisplaylinenum
            \var{\vd sarvathā\lem  \Ed; sorithā \msCa}%

caturāyatana\.m vipra kathayiṣyāmi tac ch\textsubring{r}ṇu\thinspace{\dandab} \dontdisplaylinenum

karuṇāmuditopekṣāmaitrī cāyātana\.m sm\textsubring{r}tam \veg\dontdisplaylinenum
            \var{\vc mudito°\lem  \msCa; muditau° \Ed}%
            \var{\vd cāyātana\.m\lem  \Ed; cāyātana \msCa}%

catur dhyānādhunā vakṣye sa\.msārārṇavatāraṇam\thinspace{\dandab} \dontdisplaylinenum

ātmavidyābhava\.m sūkṣma\.m dhyānam ukta\.m caturvidham \veg\dontdisplaylinenum
            \var{\vc °bhava\.m\lem  \Ed; °bhava \msCa}%
            \var{\vd dhyānam ukta\.m\lem  \corr;
                        dhyānam uktaś \msCa; dhyānayajñaś \Ed}%

ātmatattvaḥ sm\textsubring{r}to dharmo vidyāpañcasu pañcadhā\thinspace{\dandab} \dontdisplaylinenum
            \var{\va sm\textsubring{r}to\lem  \msCa; sm\textsubring{r}tā \Ed\oo
                dharmo\lem  \msCa; dhanyā \Ed}%

ṣaṭtri\.mśākṣaram ityāhuḥ sūkṣmatattvam alakṣaṇam \veg\dontdisplaylinenum
            \var{\vcd āhuḥ sū°\lem  \Ed; ā{\il}{\il} \msCa}%

catuṣpādaḥ sm\textsubring{r}to dharmaś caturāśramam āśritaḥ\thinspace{\dandab} \dontdisplaylinenum

g\textsubring{r}hastho brahmacārī ca vānaprastho 'tha bhaikṣukaḥ \veg\dontdisplaylinenum
            \var{\vd bhaikṣukaḥ\lem  \msCa; bhakṣakaḥ \Ed}%
            \paral{\textit{\vcd {\normalfont  = MBh 12.234.13ab \kb\ MBh 14.4513ab etc. } }}

dhanyās te yair ida\.m vetti nikhilena dvijottama\thinspace{\dandab} \dontdisplaylinenum

pāvana\.m sarvapāpānā\.m puṇyānā\.m ca pravardhanam \veg\dontdisplaylinenum
            \var{\vd pravardhanam\lem  \msCa; pravardhanaḥ \Ed}%

āyuḥ kīrtir yaśaḥ saukhya\.m dharmād eva pravardhate\thinspace{\dandab} \dontdisplaylinenum


\alalfejezet{yamaḥ 8: apramādaḥ}
śāntiḥ puṣṭiḥ sm\textsubring{r}tir medhā jāyate dhanyamānavaḥ \veg\dontdisplaylinenum 
            \var{\vc puṣṭiḥ\lem  \Ed; {\il}ṣṭiḥ \msCa}%

pramādasthāna pañcaiva\.m kīrtayiṣyāmi tac ch\textsubring{r}ṇu\thinspace{\dandab} \dontdisplaylinenum
            \var{\va °sthāna\lem  \msCa; °sthāna\.m \Ed}%

brahmahatyā surāpāna\.m steyo gurvaṅganāgamam \veg\dontdisplaylinenum
            \paral{\textit{\vcd {\normalfont cf.\ \MBh\ Indeces 12.30:}
                    brahmahatyā\.m surāpāna\.m steya\.m gurvaṅganāgamam{\thinspace\danda}
                    mahānti pātakāny āhuḥ sa\.myoga\.m caiva taiḥ saha{\thinspace\ketdanda}
                    {\normalfont  cf.\ also Manu 11.54: }
                    brahmahatyā surāpāna\.m steya\.m gurvaṅganāgamaḥ{\thinspace\danda}
                    mahānti pātakāny āhuḥ sa\.msargaś cāpi taiḥ saha{\thinspace\ketdanda}}}

mahāpātakam ity āhus tatsa\.myogī ca pañcamaḥ\thinspace{\dandab} \dontdisplaylinenum

an\textsubring{r}ta\.m ca samutkarṣa\.m rājagāmī ca paiśunaḥ \veg\dontdisplaylinenum
            \var{\vc samutkarṣa\.m\lem  \msCa; samutkarṣa \Ed}%
            \var{\vd rāja°\lem  \msCa; rājñī° \Ed}%

guroś cālīka nirbaddhas samāni brahmahatyayā\thinspace{\dandab} \dontdisplaylinenum
            \var{\va nirbaddhas\lem  \msCa; nibaddhas \msCa}%
            \var{\vb brahmahatyayā\lem  \Ed; bra{\il}{\il}{\il}yā \msCa}%
            \paral{\textit{\vab \kb\ {\normalfont  MBh 5.40.3cd: } guroś cālīkanirbandhaḥ samāni brahmahatyayā}}

brahmo \textsubring{r}gvedanindā ca kūṭasākṣī sak\textsubring{r}d budhaḥ \veg\dontdisplaylinenum
            \var{\vc brahmo\lem  \msCa; brahma \Ed}%
            \var{\vd sak\textsubring{r}d budhaḥ\lem  \Ed; suh\textsubring{r}d badhaḥ \msCa}%

garhitānnaś ca yo vipraḥ surāpānasamāniṣaṭ\thinspace{\dandab} \dontdisplaylinenum
            \var{\va °ānnaś ca yo vipraḥ\lem  \Ed; °ānnañ ca yojagvis \msCa}%

retotsekaḥ svayonyāsu kumārīṣv antyajāsu ca \veg\dontdisplaylinenum

sakhyaputrasya ca strīṣu gurutalpasamaḥ sm\textsubring{r}taḥ\thinspace{\dandab} \dontdisplaylinenum
            \var{\va °putrasya ca strīṣu\lem  \msCa; °putrīṣu cāstrīṣu \Ed}%
            \var{\vb °samaḥ\lem  \msCa; °sama \Ed}%

nikṣepasyāpaharaṇa\.m narāśvarajatasya ca \veg\dontdisplaylinenum

bhūmivajramaṇīnā\.m ca h\textsubring{r}tasteyasamaḥ sm\textsubring{r}taḥ\thinspace{\dandab} \dontdisplaylinenum
            \var{\vb h\textsubring{r}tasteya°\lem  \Ed; \uncl{rūgya}{\il}ya° \msCa\oo
                 °samaḥ\lem  \msCa; °sama \Ed}%

catvāra ete sa\.mbhūya yat pāpa\.m kurute naraḥ \veg\dontdisplaylinenum
            \var{\vc ete\lem  \msCa; eva \Ed}%

mahāpātakapañcaitan tena sarva\.m prakāśitam\thinspace{\dandab} \dontdisplaylinenum

pañcapramādam etāni varjanīya\.m dvijottama \veg\dontdisplaylinenum
            \var{\vc °mādam\lem  \msCa; °māda \Ed}%


\alalfejezet{yamaḥ 9: mādhuryam}
kāyavāṅmanasā pūryaś cakṣurbuddhiś ca pañcamaḥ\thinspace{\dandab} \dontdisplaylinenum
            \var{\va pūryaś\lem  \msCa; bhūyaś \Ed}%

saumyad\textsubring{r}ṣṭipradāna\.m ca krūrabuddhi\.m ca varjayet \veg\dontdisplaylinenum
            \var{\vc °dāna\.m\lem  \msCa; °dānaś \Ed}%
            \var{\vd °buddhi\.m\lem  \msCa; °d\textsubring{r}ṣṭi\.m \Ed}%

prasannamanasā dhyāyet priyavākyam udīrayet\thinspace{\dandab} \dontdisplaylinenum

yathā śaktipradāna\.m ca svāśramābhyāgato guruḥ \veg\dontdisplaylinenum
            \var{\vc yathā\lem  \msCa; yasya \Ed\oo
                 °dāna\.m\lem  \msCa; °dātaś \Ed}%

indhanodakadāna\.m ca jātavedam athāpi vā\thinspace{\dandab} \dontdisplaylinenum

sulabhāni na dattāni indhanāgnyudakāni ca \veg\dontdisplaylinenum
            \var{\vc sulabhāni na\lem  \msCa; surabhāni ca \Ed}%

kṣuta\.m jīveti vā nokta\.m tasya ki\.m parataḥ phalam\thinspace{\dandab} \dontdisplaylinenum
            \var{\va kṣuta\.m\lem  \msCa; śata\.m \Ed}%


\alalfejezet{yamaḥ 10: ārjavam}
pañcārjavā praśa\.msanti munayas tattvadarśinaḥ \veg\dontdisplaylinenum
            \var{\vc pañcārjavāḥ\lem  \msCa; pañcārjavā \Ed\oo
         praśa\.msanti\lem  \msCa; praśasanti \Ed}%

karmav\textsubring{r}ttyābhiv\textsubring{r}ddhi\.m ca pāratoṣikam eva ca\thinspace{\dandab} \dontdisplaylinenum
            \var{\va karma°\lem  \Ed; {\il}rmma° \Ed\oo
                 °v\textsubring{r}ddhi\.m\lem  \msCa; °v\textsubring{r}ttiś \Ed}%

strīdhanotkocavitta\.m ca ārjavo nābhinandati \veg\dontdisplaylinenum
            \var{\vc strīdhanotkoca°\lem  \msCa; strīdhanaṅgo ca \Ed}%
            \var{\vd ārjavo\lem  \msCa; ārjave \Ed}%

ārjavo na v\textsubring{r}thā yajña ārjavo na v\textsubring{r}thā tapaḥ\thinspace{\dandab} \dontdisplaylinenum
            \var{\vab yajña ārjavo\lem  \msCa; yajñaś cārjavo \Ed}%

ārjavo na v\textsubring{r}thā dānam ārjavo na v\textsubring{r}thāgnayaḥ \veg\dontdisplaylinenum
            \var{\vcd \om\ \Ed}%

ārjavasyendriyagrāmaḥ suprasanno 'pi tiṣṭhati\thinspace{\dandab} \dontdisplaylinenum
            \var{\vab \om\ \Ed}%

ārjavasya sadā devāḥ kāye tasya ramanti te \veg\dontdisplaylinenum
            \var{\vd tasya ramanti\lem  \Ed; {\il}{\il}{\il}nti \msCa}%

\ujvers\nemsloka 
iti yamapravibhāgaḥ kīrtito 'ya\.m dvijendra
\dontdisplaylinenum
            \var{\va yamapravibhāgaḥ\lem  \msCa; niyamaparibhāgaḥ \Ed\oo
                 dvijendra\lem  \msCa; narendra \Ed}%

\nemslokab 
iha parata sukhārtha\.m kārayet tanmanuṣyaḥ \danda\dontdisplaylinenum

\nemslokac 
duritamalaprahārī śaṅkarasyājñayāste
\dontdisplaylinenum
            \var{\vc °prahārī\lem  \conj; °pahārī \msCa\Ed\oo
                 durita°\lem  \msCa; irita° \Ed}%

\nemslokad 
bhavati p\textsubring{r}thivibhartā hy ekachatraprav\textsubring{r}ttā \veg\dontdisplaylinenum
            \var{\vd °v\textsubring{r}ttā\lem  \msCa; °v\textsubring{r}ttāḥ \Ed}%

\vers


\jump
\begin{center}
\ketdanda iti v\textsubring{r}ṣasārasa\.mgrahe yamavibhāgo nāmādhyāyaś caturthaḥ\ketdanda
\end{center}
\dontdisplaylinenum\vers 
            \var{{\normalfont Colophon: } nāmādhyāyaś caturthaḥ\lem  \msCa;
                                 nāmaś caturtho 'dhyāyaḥ \Ed}%

\vers

\vers

\vers

\vers

\vers

\vers

\vers

\vers

\vers

\vers

\vers

\vers

\vers

\vers

\vers

\vers

\vers

\vers

\vers

\vers

\vers

\vers

\vers

\vers
\bekveg\szamveg\vfill\phpspagebreak\szam\bek\versno=0\fejno=15



\alfejezet{\textbf{pañcadaśamo 'dhyāyaḥ}}\jump\jump

\alalfejezet{jīvavarṇanam}
devy uvāca~{\dandab}\dontdisplaylinenum 

jīvabhūteti yat prokta\.m lakṣaṇa\.m kīd\textsubring{r}śa\.m bhavet\thinspace{\danda} \dontdisplaylinenum
            \var{\vb lakṣaṇa\.m kī°\lem  \msNa\msNb\msNc\Ed; lakṣaṇāṅ kī° \msCa, laṇa\.m kī° \msCb}%

sthānam asya na jānāmi rūpa\.m varṇa\.m ca īśvara \veg\dontdisplaylinenum
            \var{\vc sthānam asya\lem  \msCb\msNa\msNb\msNc\Ed; {\il}\uncl{na}m asya \msCa}%
            \var{\vd rūpa\.m varṇa\.m\lem  \msCa\msCb\msNa\Ed; rūpavarṇa\.m \msNb\msNc}%
            \paral{\textit{{\normalfont Testimonia for this chapter: \msCa\ ff.\thinspace 219r--220r, 
                                             \msCb\ ff.\thinspace 222v--223v, 
                                             \msCc\ is not available for this chapter,
                                             \msNa\ ff.\thinspace 26r--27r, 
                                             \msNb\ ff.\thinspace 230v--231r, 
                                             \msNc\ ff.\thinspace 234r--235r;
                                               \mssCaCbCc\ = \msCa + \msCb + \msCc }}}

etat kautūhala\.m chindhi sa\.mśaya\.m parameśvara\thinspace{\dandab} \dontdisplaylinenum
            \var{\va etat kautūhala\.m\lem  \msCa\msCb\msNa\msNb\Ed; etat kautūla\.m \msNc\oo
                 chindhi\lem  \msCa\msCb\msNa\msNb\Ed; chitvāndhi \msNc}%
            \var{\vb sa\.mśaya\.m\lem  \msCa\msCb\msNa\msNc\Ed; sa\.mśaya \msNb}%

na cānyad eva paśyāmi jīvanirṇaya kīrtaya \veg\dontdisplaylinenum

īśvara uvāca~{\dandab}\dontdisplaylinenum 
            \var{\vo īśvara\lem  \msCa\msCb\msNa\msNb\msNc; bhagavān \Ed}%

jīvasya lakṣaṇa\.m devi kathitu\.m kena śakyate\thinspace{\danda} \dontdisplaylinenum
            \var{\va lakṣaṇa\.m\lem  \msCb\msNa\msNb\msNc\Ed; kathita\.m \msCa}%

na rūpavarṇa\.m jīvasya vidyate sthānam eva ca \veg\dontdisplaylinenum
            \var{\vc °varṇa\.m\lem  \msCb\msNa\msNc; °varṇa \msCa\msNb\Ed}%

vyāpi sarvagata\.m sūkṣma\.m sarvam āśritya tiṣṭhati\thinspace{\dandab} \dontdisplaylinenum
            \var{\va vyāpi\lem  \msCb\msNa\msNb\msNc; vyā\uncl{pi} \msCa, vyāpī \Ed}%
            \var{\va °śritya\lem  \msCb; °ś\textsubring{r}tya \msCa\msNa\msNb, °śrutya \msNc, °v\textsubring{r}tya \Ed}%

nirālambam anādhāram anaupamya\.m nirañjanam \veg\dontdisplaylinenum
            \var{\vd °pamya\.m\lem  \msCa\msCb\msNa\msNc\Ed; °pamya \msNb}%

araṇistho yathā vahniḥ kāṣṭheṣu nopalabhyate\thinspace{\dandab} \dontdisplaylinenum

tadvaj jīvo na paśyeta śarīrastho 'pi sundari \veg\dontdisplaylinenum
            \var{\vc jīvo na\lem  \msCb\msNa\msNb\msNc; jīvon na \msCa, jīva\.m na \Ed}%
            \var{\vd 'pi\lem  \msCa\msCb\msNa\msNc\Ed; hi \msNb}%

dadhivac ca yathā sarpir d\textsubring{r}śyate na ca d\textsubring{r}śyate\thinspace{\dandab} \dontdisplaylinenum

tadvaj jīvaḥ śarīrastho d\textsubring{r}śyate na ca d\textsubring{r}śyate \veg\dontdisplaylinenum
            \var{\vc tadvaj jīvaḥ\lem  \msCa\msCb\msNa\msNb; tadva jīvaḥ \msNc, tadvaj jīva \Ed}%

devy uvāca~{\dandab}\dontdisplaylinenum 

ad\textsubring{r}ṣṭapratyayo hy asti nāsti pratyayadarśanam\thinspace{\danda} \dontdisplaylinenum

vyāpī katha\.m mahādeva sarvatrāvasthitaḥ katham \veg\dontdisplaylinenum
            \var{\vd °sthitaḥ\lem  \msCb\msNc\Ed; °sthita\.m \msCa\msNa, °sthita \msNb}%

maheśvara uvāca~{\dandab}\dontdisplaylinenum 
            \var{\vo maheśvara\lem  \msCa\msCb\msNb\msNc; mahādeva \msNa, bhagavān \Ed}%

asa\.mśayo mahādevi vyāpī sarvagataḥ śivaḥ\thinspace{\danda} \dontdisplaylinenum

d\textsubring{r}śyatendriyasa\.myogāj jīvapratyayadarśanam \veg\dontdisplaylinenum
            \var{\vc d\textsubring{r}śyate°\lem  \msCb\msNa\msNb\msNc; d\textsubring{r}śyete° \msCa, d\textsubring{r}śyante \Ed}%
            \var{\vd °jīva°\lem  \msCa\msCb\msNa\msNb\Ed; °jī° \msNc}%

yathākāśasthito vāyuḥ śabdasparśaguṇānvitaḥ\thinspace{\dandab} \dontdisplaylinenum
            \var{\vcd vāyuḥ śabda°\lem  \msCb\msNa\msNb\msNc\Ed; vāyu\uncl{śśa}{\il} \msCa}%
            \var{\vd °nvitaḥ\lem  \msCa\msNa\msNb\msNc\Ed; °nvitam \msCb}%

tadvad dehī vijānīyād guṇaceṣṭena nānyathā \veg\dontdisplaylinenum
            \var{\vd °ceṣṭena\lem  \msCa\msCb\msNa\msNb; °veṣṭana \msNc, °veṣṭena \Ed}%

devy uvāca~{\dandab}\dontdisplaylinenum 

vyāpīti kathitaḥ pūrva\.m jīvaḥ sarvagato 'pi ca\thinspace{\danda} \dontdisplaylinenum
            \var{\va kathitaḥ\lem  \msCa\msNa\msNcpcorr\Ed; kathita\.m \msCb\msNb, kathatiḥ \msNc}%

ta\.m v\textsubring{r}thā kathito 'sy adya mriyate kena hetunā \veg\dontdisplaylinenum
            \var{\vc v\textsubring{r}thā\lem  \msCa\msCb\msNa\msNb\Ed; vyathā \msNc\oo
                 'sy adya\lem  \msCa\msCb\msNc; smy adya \msNa\Ed, sy a{\il} \msNb}%

īśvara uvāca~{\dandab}\dontdisplaylinenum 
            \var{\vo īśvara\lem  \msCa\msCb\msNb\msNc; bhagavān \msNa\Ed}%

na jīvo mriyate devi sarveṣā\.m surasundari\thinspace{\danda} \dontdisplaylinenum

ghaṭāntastho yathākāśo bahirākāśavad yathā \veg\dontdisplaylinenum

ghaṭabhinne viśālākṣi viśeṣo nopalakṣyate\thinspace{\dandab} \dontdisplaylinenum
            \var{\vb nopalakṣyate\lem  \msCa\msCb\msNb\msNc\Ed; nopalabhyate \msNa}%

dehabhinne yadā devi vināśo nopalabhyate \veg\dontdisplaylinenum
            \var{\vc deha°\lem  \msCa\msNa\msNb\msNc\Ed; dehe \msCb\oo
                 yadā devi\lem  \msCa\msCb\msNa\msNb\msNc; tathā dehī \Ed}%
            \paral{\textit{\vo {\normalfont cf.\ Bhāgavatapurāṇa 12.5.5: }
                         ghaṭe bhinne ghaṭākāśa ākāśaḥ syād yathā purā{\thinspace\danda}
                         eva\.m dehe m\textsubring{r}te jīvo brahma sampadyate punaḥ{\thinspace\ketdanda}}}

susūkṣmaḥ sarvago vyāpī paramātmānam avyayaḥ\thinspace{\dandab} \dontdisplaylinenum
            \var{\va susūkṣmaḥ\lem  \msCa\msCb\msNa\msNb; susūkṣma \msNc, sa sūkṣmaḥ \Ed}%

bahir antaś ca bhūtānām acaraś cara eva saḥ \veg\dontdisplaylinenum
            \var{\vd °caraś ca°\lem  \msCa\msCb\msNa\msNb\msNc; °caran ca° \Ed\oo
                 saḥ\lem  \msCa\msCb\msNa\msNb\msNc; sa \Ed}%

aprameyo 'vināśī ca aprapañcaḥ prapañcakaḥ\thinspace{\dandab} \dontdisplaylinenum
            \var{\vab \om\ \msNb}%

sarvendriyaguṇābhāsaḥ sarvendriyavivarjitaḥ \veg\dontdisplaylinenum

evam eṣa mahādevi jīvasya varavarṇini\thinspace{\dandab} \dontdisplaylinenum

kathito 'smi samāsena kim anyac chrotum icchasi \veg\dontdisplaylinenum
            \var{\vd icchasi\lem  \msCb\msNa\msNb\msNc\Ed; icchati \msCa}%


\alalfejezet{sāraśreṣṭham}
devy uvāca~{\dandab}\dontdisplaylinenum 

sāraśreṣṭha\.m mahādeva kathayeśāna īśvara\thinspace{\danda} \dontdisplaylinenum
            \var{\va °śreṣṭha\.m\lem  \msCb\msNa\Ed; °śreṣṭha \msCa\msNb\msNc}%  

śrotum icchāmi deveśa mānuṣāṇā\.m hita\.m vada \veg\dontdisplaylinenum
            \var{\vd vada\lem  \msCa\msCb\msNa\msNb; vadaḥ \msNc\Ed}%

īśvara uvāca~{\dandab}\dontdisplaylinenum 
            \var{\vo īśvara\lem  \msCa\msCb\msNa\msNb\msNc; bhagavān \Ed}%

āśramāṇā\.m g\textsubring{r}hī śreṣṭho varṇaśreṣṭhā dvijātayaḥ\thinspace{\danda} \dontdisplaylinenum
            \var{\va āśramāṇā\.m\lem  \msCa\msCb\msNa\msNc\Ed; āśramāṇā \msNb\oo
                 g\textsubring{r}hī\lem  \msCb\msNa\msNb\msNc\Ed; g\textsubring{r}\uncl{hī} \msCa}%
            \var{\vb °śreṣṭhā\lem  \msCa\msCb\msNc; °śreṣṭo \msNa\msNb\Ed}%

aśvamedhaḥ kratuśreṣṭho japaśreṣṭho 'ghamarṣaṇaḥ \veg\dontdisplaylinenum
            \var{\vd japa°\lem  \msCapcorr\msNa\msNb\msNc\Ed; ja° \msCaacorr, 'japa° \msCb\oo
                 'ghamarṣaṇaḥ\lem  \msCb\msNa\msNb\msNc\Ed; rghamarṣaṇaḥ \msCa}%

devatānā\.m hariḥ śreṣṭhaḥ śreṣṭhā gaṅgā nadīṣu ca\thinspace{\dandab} \dontdisplaylinenum
            \var{\vab śreṣṭhaḥ śreṣṭhā gaṅgā\lem  \msCa\msNa\msNb\msNc\Ed; śreṣṭhā gaṅgāṇāñ ca \msCb}%

anāśanas tapaḥśreṣṭhas tīrthaśreṣṭhaḥ surahradaḥ \veg\dontdisplaylinenum
            \var{\vc anāśana°\lem  \msCa\msCb\msNa\msNb\Ed; anaśana° \msNc}%
            \var{\vd °rthaśreṣṭhaḥ\lem  \msCa\msCb\msNa\msNb\Ed; °rthaśreṣṭha \msNc}%

kṣoma\.m vastreṣu ca śreṣṭha\.m yaśaḥ śreṣṭha\.m vibhūṣaṇam\thinspace{\dandab} \dontdisplaylinenum
            \var{\va kṣauma\.m\lem  \msNc\Ed; kṣoma\.m \msCa\msCb\msNa, kṣoma \msNb}%
            \var{\vb śreṣṭha\.m\lem  \msCa\msCb\msNa\msNc\Ed; śreṣṭha \msNb\oo
                 °bhūṣaṇam\lem  \msCa\msNa\msNb\msNc\Ed; °bhūṣiṇam \msCb}%

bhārata\.m śrutiṣu śreṣṭha\.m vrataśreṣṭho dayāparaḥ \veg\dontdisplaylinenum
            \var{\vd °śreṣṭho\lem  \msCa\msCb\msNa\msNc\Ed; śreṣṭha\.m \msNb\oo
                 dayāparaḥ\lem  \msCb\msNa\msNb\msNc\Ed; \uncl{dayāpa}raḥ \msCa}%

dāneṣu cābhaya\.m śreṣṭha\.m manaḥ śreṣṭhendriyeṣu ca\thinspace{\dandab} \dontdisplaylinenum

vidyā sa\.mgrahaṣu śreṣṭhā satya\.m śreṣṭha\.m vacaḥsu ca \veg\dontdisplaylinenum
            \var{\vc sa\.mgrahaṣu\lem  \msCa\msCb\msNa\msNb\Ed; sa\.mgraheṣu \msNc\ \unmetr\oo
                 śreṣṭhā\lem  \msCa\msCb\msNa\msNb\msNc; śreṣṭho \Ed}% 

āyudhānā\.m dhanuḥ śreṣṭha\.m bāndhaveṣu ca mātaraḥ\thinspace{\dandab} \dontdisplaylinenum
            \var{\va śreṣṭha\.m\lem  \msCa\msCb\msNa\msNc\Ed; śreṣṭha \msNb}%
            \var{\vb bāndhaveṣu ca mātaraḥ\lem  \msCa\msCb\msNaacorr\msNc\Ed; bāndhaveṣu ca mātara\.m \msNapcorr,
                grahaśreṣṭho divākaraḥ \msNb\ \eyeskip{to 15.24b}}%

jñānam auṣadhiṣu śreṣṭha\.m vaidyaśreṣṭhaḥ śivākṣaraḥ \veg\dontdisplaylinenum
            \var{\vcd \om\ \msNb}%
            \var{\vc jñānam oṣadhiṣu\lem  \msNc; jñānam auṣadhiṣu \msCa\msCb\msNa\msNb\Ed}%
            \var{\vd vaidya°\lem  \msCa\msCb\msNa; \om\ \msNb, vaidyaḥ \msNc, vaidyo \Ed\oo
                 °śreṣṭhaḥ\lem  \msCb\msNa\msNc\Ed; °śreṣṭha \msCa, \om\ \msNb}%

akāraś cākṣaraḥ śreṣṭho dharmaśreṣṭho hy ahi\.msakaḥ\thinspace{\dandab} \dontdisplaylinenum

paśuṣu saurabhī śreṣṭhā nareṣu ca narādhipaḥ \veg\dontdisplaylinenum
            \var{\vo \om\ \msNb}%

māsi mārgaśiraḥ śreṣṭha\.m k\textsubring{r}taḥ śreṣṭhaś caturyuge\thinspace{\dandab} \dontdisplaylinenum
            \var{\vo \om\ \msNb}%
            \var{\va māsi\lem  \msCa\msCb\msNa\msNc; \om\ \msNb, māsī \Ed\oo
                 °śiraḥ\lem  \msCa\msCb\msNa\msNb\Ed; °śira \msNc}%
            \var{\vb śreṣṭhaś caturyuge\lem  \msCa\msNa\Ed; śreṣṭha\.m caturyuge \msCb, \om\ \msNb, śreṣṭhaś caryuge \msNc}%

vasanta \textsubring{r}tuṣu śreṣṭhaḥ śreṣṭha\.m cāyanam uttaram \veg\dontdisplaylinenum
            \var{\vd śreṣṭha\.m cā°\lem  \msCa\msCb\msNc\Ed; śreṣṭhaś cā° \msNa, \om\ \msNb\oo
                 °ttaram\lem  \msCa\msNa\msNc\Ed; °tta\uncl{me}m \msCb, \om\ \msNb}%

amāvāsyā dinaśreṣṭhā grahaśreṣṭho divākaraḥ\thinspace{\dandab} \dontdisplaylinenum
            \var{\va amāvāsyā dinaśreṣṭhā\lem  \msCa\msCb\msNc\Ed; \om\ \msNb, amāvāsyā dinaśreṣṭho \msNa}%
            \var{\vb grahaśreṣṭho divākaraḥ\lem  \msCa\msCb\msNa\msNb; grahaḥ śreṣṭho divākaraḥ \msNc, vasuśreṣṭho hutāśanaḥ \Ed}%

strīṣu lakṣmīr dh\textsubring{r}tiḥ śreṣṭhā vasuśreṣṭho hutāśanaḥ \veg\dontdisplaylinenum
            \var{\vcd \om\ \Ed}%
            \var{\vc strīṣu\lem  \msCa\msNa\msNb\msNc\Ed; strī \msCb\oo
                 lakṣmīr dh\textsubring{r}tiḥ\lem  \msCa; lakṣmīdh\textsubring{r}tiḥ \msCb\msNa\msNb\msNc, \om\ \Ed}%

\textsubring{r}ṣiṣu uṣaṇā śreṣṭhaḥ kāntiśreṣṭho niśākaraḥ\thinspace{\dandab} \dontdisplaylinenum
            \var{\va uṣaṇā\lem  \corr; uśanāḥ \msCa\msCb\msNa\msNb\msNc, uśanaḥ \Ed}%
            \var{\vb kānti°\lem  \msCb\msNa\msNb\Ed; kāntiḥ \msNc, kā{\il} \msCa}%

nakṣatreṣv abhijit śreṣṭhaḥ kālaḥ śreṣṭhaḥ kaleṣu ca  \veg\dontdisplaylinenum
            \var{\vc °bhijit śre°\lem  \Ed; °bhijiḥ śre° \msCa\msCb\msNa\msNbpcorr\msNc, °bhiji \msNbacorr}%
            \var{\vd kālaḥ\lem  \msCa\msCb\msNa\msNb\msNc; kaliḥ \Ed}%

vedeṣu ca vara\.m sāma sthāvareṣu himālayaḥ\thinspace{\dandab} \dontdisplaylinenum

aśvattho vaṭa v\textsubring{r}kṣeṣu bhūteṣu vara cetanaḥ \veg\dontdisplaylinenum
            \var{\vc vaṭa\lem  \msCa\msCb\msNa\msNb; vara \msNc\Ed}%
            \var{\vd vara cetanaḥ\lem  \msCb\Ed; varaś cetanaḥ \msCa\msNa\msNc\ \unmetr, vaś cetanaḥ \msNb}%

adhyātma sarvavidyāsu vākya satya vara sm\textsubring{r}taḥ\thinspace{\dandab} \dontdisplaylinenum
            \var{\va adhyātma\lem  \msCb\msNb\Ed; adhyātmā \msCa\msNc, ādhyātma\.m \msNa\oo
                 sarvavidyāsu\lem  \msCa\msNa\msNb\msNc; sarvavidyānā\.m \msCb, varavidyāsu \Ed}%
            \var{\vb vākya\lem  \msCb; vāhu \msCa\msNa\msNb\msNc, vācaḥ \Ed\oo
                 vara\lem  \msCa\msCb\Ed; va\uncl{ra}ḥ \msNa, varaḥ \msNb\msNc}%

prahlādo vara daityeṣu yakṣarakṣo dhaneśvaraḥ \veg\dontdisplaylinenum
            \var{\vc prahlādo\lem  \msCa\msCb\msNa\Ed; prahrādo \msNb\msNc}%
            \var{\vd °śvaraḥ\lem  \msCa\msCb\msNa\msNc\Ed; °śvara \msNb}%

marīcir vara vāteṣu hariḥ śreṣṭho m\textsubring{r}geṣu ca\thinspace{\dandab} \dontdisplaylinenum
            \var{\va marīcir vara\lem  \msNc; marīci vara \msCb\msNa\msNb\Ed, ma{\il}{\il}{\il}{\il} \msCa}%
            \var{\vb hariḥ\lem  \msCa\msCb\msNb\msNc\Ed; hari \msNa}%

sādhya nārāyaṇaḥ śreṣṭhaḥ pit\textsubring{\=r}ṇā\.m ca pitāmahaḥ \veg\dontdisplaylinenum

etat samāsato devi kathito 'si varānane\thinspace{\dandab} \dontdisplaylinenum
            \var{\vb 'si\lem  \msCa\msCb\msNa\msNb; smi \msNc\Ed}%

sarvasāra\.m samuddh\textsubring{r}tya ki\.m bhūyaḥ kathayāmy aham \veg\dontdisplaylinenum
            \var{\vd ki\.m\lem  \msCb\msNa\msNb\msNc\Ed; ki \msCa}%


\jump
\begin{center}
\ketdanda iti v\textsubring{r}ṣasārasa\.mgrahe jīvanirṇayo nāmādhyāyaḥ pañcadaśamaḥ\ketdanda
\end{center}
\dontdisplaylinenum\vers 
            \var{{\normalfont Colophon:} nāmādhyāyaḥ pañcadaśamaḥ\lem  \msCa\msCb\msNa; nāmādhyāyaḥ pañcamaḥ \msNb,
                                nāmādhyāyaḥ pañcadaśama \msNc, nāma pañcadaśo 'dhyāyaḥ \Ed}%

\vers

\nemslokalong


\nemslokanormal


\vers

\vers

\vers

\vers

\vers

\vers

\vers

\vers

\vers

\vers

\vers

\vers

\vers

\vers

\vers

\vers

\vers

\vers

\vers
\bekveg\szamveg\vfill\phpspagebreak\szam\bek\versno=0\fejno=22



\alfejezet{\textbf{dvāvi\.mśo 'dhyāyaḥ}}\jump\jump
\vers

janamejaya uvāca~{\dandab}\dontdisplaylinenum 

śruto 'thābjamukhād dharmasārasa\.mgraham uttamam\thinspace{\danda} \dontdisplaylinenum
            \var{\va śruto 'thābjamukhād dharma°\lem  \eme; 
        śruto vābjamukhād dharmaḥ \msCa, śruto vābjamukhod dharmaḥ \msCb, śruto vābjamukhā dharmaḥ \msCc, 
        śruto cābjamukhād dharmaḥ \msNa\msL, śruto cābdamukhā dharmaḥ \msNb, śrutvā vābjamukhād dharmaḥ \msNc,
                        śruto vā tvanmukhād dharmaḥ \Ed}%
            \paral{\textit{{\normalfont Testimonia for this chapter: \msCa\ ff.\thinspace 232r--234v, 
                                             \msCb\ ff.\thinspace 233v--235r, 
                                             \msCc\ ff.\thinspace 314r--317r,
                                             \msNa\ ff.\thinspace 39r--41v,
                                             \msNb\ ff.\thinspace 241v--243v, 
                                             \msNc\ ff.\thinspace 247v--250r;
                                                \mssCaCbCc\ = \msCa + \msCb + \msCc }}}

madhuraślakṣṇavāṇībhiḥ samyagvedārthasa\.myutam \veg\dontdisplaylinenum
            \var{\vc °ślakṣṇavāṇī°\lem  \msCb\msCc\msNa\msNb\msNc;
                       ślakṣṇaṇī° \msCa, °ślakṣyavānī° \msL, °ślakṣṇāvāṇī° \Ed}%

nyāyayukta\.m mahāsāra\.m guhyajñānam anuttaram\thinspace{\dandab} \dontdisplaylinenum
            \var{\va nyāyayukta\.m mahāsāra\.m\lem  \msCa\msCc\msNb\msNc\Ed; nyāyam ukta\.m mahat sāra\.m \msCb,
                                        nyāyayukta\.m mahat sāra\.m \msNa\msL}%
            \var{\vb guhya°\lem  \mssCaCbCc\msNa\msNb\msNc\msL; guhya\.m \Ed\oo
                 °nuttaram\lem  \msCa\msNa\msNb\msL; °nuttamam \msCb\msCc\msNc, °nantaram \Ed}%

t\textsubring{r}pto 'smīhām\textsubring{r}ta\.m pītvā janmam\textsubring{r}tyurujāpaham \veg\dontdisplaylinenum
            \var{\vcd pītvā janma°\lem  \msCb\msCc\msNa\msNb\msNc\msL\Ed; \uncl{pī}{\lost}{\lost}nma \msCa}%
            \var{\vd  °rujā°\lem  \msCa\msCc\msNa\msNb\msNc\msL\Ed; °mujā° \msCb}%

praśnam ekānya p\textsubring{r}cchāmi nāmahetu\.m tapodhana\thinspace{\dandab} \dontdisplaylinenum
            \var{\va praśna°\lem  \mssCaCbCc\msNa\msNc\msL\Ed; prasta° \msNb\oo
                 °kānya\lem  \mssCaCbCc\msNb\msNc; °kānyat \msNa\ \unmetr, 
                                   °kā\.mnyat \msL\ \unmetr, °konya \Ed}%
            \var{\vb nāma°\lem  \mssCaCbCc\msNa\msNb\msL\Ed; nāya° \msNc\oo
                 °hetu\.m\lem  \msCa\msCb\msNa\msL; °hetu \msCc\msNb\msNc\Ed\oo
                 °dhana\lem  \mssCaCbCc\msNb\msNc\Ed; °dhanam \msNa\msL}%

varṇagotrāśrama\.m tasmāc chrotum icchāmi te punaḥ \veg\dontdisplaylinenum
            \var{\vc varṇa°\lem  \mssCaCbCc\msNa\msNb\msNc\msBod\msL; varṇa\.m \Ed}%

vaiśampāyana uvāca~{\dandab}\dontdisplaylinenum 
            \var{\vo uvāca\lem  \mssCaCbCc\msNa\msNb\msL\Ed; {\lost}{\lost}{\lost} \msNc}%

ś\textsubring{r}ṇu rājann avahito yogendrasya mahātmanaḥ\thinspace{\danda} \dontdisplaylinenum
            \var{\va rājann a°\lem  \msCb\msCc\msNa\msNc\msL\Ed; rājan a° \msCa\msNb}%
            \var{\vab °vahito yogendrasya\lem  \mssCaCbCc\msNapcorr\msNb\msNc\Ed; °vahito yogendra \msNaacorr, 
                                                °hito yogandrasya \msL}%

āśrama\.m varṇajātīnā\.m vakṣyāmy eva narādhipa \veg\dontdisplaylinenum
            \var{\vd vakṣyāmy eva\lem  \msCa\msCc\msNa\msNb\Ed; vakṣyām eva \msCb\msNc\msL\oo
                 °pa\lem  \mssCaCbCc\msNa\msNb\msNc\msL; °paḥ \Ed}%

himavaddakṣiṇe pārśve m\textsubring{r}gendraśikhare n\textsubring{r}pa\thinspace{\dandab} \dontdisplaylinenum
            \var{\vb m\textsubring{r}gendra°\lem  \msCb\msCc\msNa\msNb\msNc\msL\Ed; \uncl{m\textsubring{r}}{\lost}ndra° \msCa\oo
                 n\textsubring{r}pa\lem  \mssCaCbCc\msNa\msNb; n\textsubring{r}paḥ \msNc\msL\Ed}%

mahendrapathagā nāma nadītīre narādhipa \veg\dontdisplaylinenum
            \var{\vc mahendra°\lem  \mssCaCbCc\msNa\msNc\Ed; m\textsubring{r}gendra° \msNb, mahindra° \msL}%
            \var{\vd °pa\lem  \mssCaCbCc\msNa\msNb\msNc\msL; °paḥ \Ed}%

tatrāśramapada\.m tasya puline sumanorame\thinspace{\dandab} \dontdisplaylinenum
            \var{\vb puline su°\lem  \msCa\msCb\msNa; pulineṣu \msCc\msNb\msNc\Ed, puline pu° \msL}%

vasati sma mahābhāgas tattvapāraganisp\textsubring{r}haḥ \veg\dontdisplaylinenum
            \var{\vc vasati\lem  \mssCaCbCc\msNa\msNb\msNc\Ed; vasanti \msL}%
            \var{\vd °pāraga°\lem  \msCa\msCc\msNa\msNb\msNc\msL\Ed; °pāra° \msCb\oo
                 °sp\textsubring{r}haḥ\lem  \mssCaCbCc\msNa\msNb\msNc\msL; °sp\textsubring{r}hāḥ \Ed}%

śīlaśaucasamācāro jitadvandvo jitaśramaḥ\thinspace{\dandab} \dontdisplaylinenum

jitamānabhayakrodho jitasarvaparigrahaḥ \veg\dontdisplaylinenum
            \var{\vd jita°\lem  \msCa\msCc\msNa\msNb\msNc\msL\Ed; jija° \msCb}%

somava\.mśaprasūtās te kṣatriyā dvijatā\.m gatāḥ\thinspace{\dandab} \dontdisplaylinenum
            \var{\va soma°\lem  \mssCaCbCc\msNa\msNb\msNc\Ed; soya° \msL\oo
                 prasūtās te\lem  \msCb\msCc\msNb\msNc\Ed; pra{\lost}{\lost}{\lost} \msCa, prasūtas te \msNa\msL}%
            \var{\vb kṣatriyā\lem  \mssCaCbCc\msNb; kṣatriyo \msNa\msNc\msL\Ed\oo
                 gatāḥ\lem  \mssCaCbCc\msNb\Ed; gataḥ \msNa\msNc\msL}% 

tapasā vinayācārair viṣṇunā dvijakalpitāḥ \veg\dontdisplaylinenum
            \var{\vc °cārair vi°\lem  \msCa\msCb\msNa\msNb\msNc\msL\Ed; °cārai vi° \msCc}%
            \var{\vd dvijakalpitāḥ\lem  \Ed; dvijaḥ kalpitaḥ \mssCaCbCc\msNc\ \unmetr,
                                                dvijakalpitaḥ \msNa\msNb\msL}%

ajitā nāma tat pūrva\.m kāmakrodhajitena tu\thinspace{\dandab} \dontdisplaylinenum
            \var{\va pūrva\.m\lem  \mssCaCbCc\msNb\msNc\Ed; pūrva \msNa\msL}%

sa\.mkalpas tasya rājendra kathayiṣyāmi tac ch\textsubring{r}ṇu \veg\dontdisplaylinenum
            \var{\vc sa\.mkalpas ta\lem  \mssCaCbCc\msNa\msNb\msNc\Ed; sa\.mkalpa ta \msL}%

adhyātmanagarasphītaḥ adhibhūtajanākulaḥ\thinspace{\dandab} \dontdisplaylinenum
            \var{\vab °sphītaḥ adhi°\lem  \msCb\msCc\msNa\msNb\msNc\msL\Ed; °sphītaradhi° \msCa}%

adhidaivatasā\.mnidhya\.m daśāyatana pañca ca \veg\dontdisplaylinenum
            \var{\vc °sā\.mnidhya\.m\lem  \msCa\Ed; sānaidhya\.m \msCb\msCc\msNa\msNb\msL, sānnaidhya\.m \msNc}%
            \var{\vd daśā°\lem  \mssCaCbCc\msNa\msNb\msNc\msL; deśā° \Ed}%
            \paral{\textit{\vo {\normalfont Cf.\ 4.72: } caturāyatana\.m vipra kathayiṣyāmi tac ch\textsubring{r}ṇu{\thinspace\danda}
                                 karuṇāmuditopekṣāmaitrī cāyātana\.m sm\textsubring{r}tam{\thinspace\ketdanda}}}

daśayajñavrata\.m cīrṇa\.m daśakāmaparājitaḥ\thinspace{\dandab} \dontdisplaylinenum
            \var{\va daśayajñavrata\.m cīrṇa\.m\lem  \msNa\msNb\msNc\msL; da\uncl{śayajña\.m} {\lost}{\lost}ñ cīrṇan \msCa, 
                        daśayajñavratacīrṇan \msCb\msCc, daśayajña\.m vrata\.m cīrṇa° \Ed}%
            \var{\vb °parājitaḥ\lem  \msCa\msCc\msNa\msNb\msNc\msL\Ed; °paparājitaḥ \msCb}%

niyamān daśa sa\.mśritya daśa vāyava \textsubring{r}tvijaḥ \veg\dontdisplaylinenum
            \var{\vc niyamān daśa\lem  \mssCaCbCc\msNa\msNb\msNc\Ed; nimāyā daśa \msLacorr, niyamā daśa \msLpcorr}%
            \paral{\textit{\vd {\normalfont cf.\ 11.17ab: } dhāraṇādhvaryuvat k\textsubring{r}tvā prāṇāyāmaś ca \textsubring{r}tvijaḥ}}

daśākṣareṇa mantreṇa daśadharmakriyāpadaḥ\thinspace{\dandab} \dontdisplaylinenum
            \var{\vb °dharmakriyāpadaḥ\lem  \msCa\msCb\msNa\msNb\msNc\msL\Ed; °dharmaḥ kripadaḥ \msCc}%

daśasa\.myamadīptāgnau jihvātejodaśendriyaḥ \veg\dontdisplaylinenum
            \var{\vc °sa\.myama°\lem  \mssCaCbCc\msNa\msNb\msNc\Ed; °sa\.mśaya° \msL\oo
                 °dīptā°\lem  \mssCaCbCc\msNa\msNc\msL; °dīpto \msNb, °dīpā° \Ed}%
            \var{\vd °daśe°\lem  \mssCaCbCc\msNa\msNb\msL; °jite° \msNc\Ed}%

daśayogāsanāsīno daśadhyānaparāyaṇaḥ\thinspace{\dandab} \dontdisplaylinenum
            \var{\va °sanāsīno\lem  \mssCaCbCc\msNa\msNb\msNc\msL; samāsīnā \Ed}%
            \var{\vb °yaṇaḥ\lem  \mssCaCbCc\msNb\msNc\Ed; °yaṇāḥ \msNa\msL}%

buddhir vedī mano yūpaḥ somapāno 'm\textsubring{r}tākṣaraḥ \veg\dontdisplaylinenum
            \var{\vc buddhir vedī\lem  \mssCaCbCc\msNa\msNb\msL; buddhi vedī \msNc, buddhir vedi \Ed}%
            \var{\vd °pāno 'm\textsubring{r}tākṣaraḥ\lem  \msCb\msNa\msNb\msNc\msL; {\lost}{\lost}{\lost}{\lost}{\lost}{\lost} \msCa, 
                                              °pānam\textsubring{r}tākṣaraḥ \msCc, °dānam\textsubring{r}tākṣaraḥ \Ed}%

dakṣiṇābhaya bhūtebhyaḥ paśubandha svaya\.mk\textsubring{r}taḥ\thinspace{\dandab} \dontdisplaylinenum
            \var{\va °bhaya\lem  \mssCaCbCc\msNa\msNb\msNc\msL; °gnaya \Ed}%

vinārtha\.m yajñam iṣṭvā tu kāla\.m ca kṣapayaty asau \danda\dontdisplaylinenum
            \var{\va °rtha\.m\lem  \msCa\msCb\Ed; °rtha° \msCc\msNa\msNb\msNc\msL}%
            \var{\vb kāla\.m\lem  \mssCaCbCc\msNa\msNb\msNc\msL; kālāñ \Ed\oo
                 kṣapayaty asau\lem  \mssCaCbCc\msNa\msNc\msL;
                        \uncl{kṣapayaty asau} \msNb, kṣapayaty asauḥ \Ed}%

anarthayajña\.m ta\.m prāhur munayas tattvadarśinaḥ \veg\dontdisplaylinenum
            \var{\vcd °yajña\.m ta\.m prāhur munayas ta°\lem  \msCa\msCb\msNb\msNc\Ed;
                °yajña ta\.m prāhu munayas ta° \msCc, °yajñan ta\.m prāhur munaya ta° \msNa,
                °yajña\.m prāhur munaya ta° \msL}%

janamejaya uvāca~{\dandab}\dontdisplaylinenum 

daśayajñam aha\.m śrotu\.m dehi mā\.m dvijasattama\thinspace{\danda} \dontdisplaylinenum
            \var{\va °yajñam aha\.m\lem  \mssCaCbCc\msNa\msNb\msNc\msLpcorr; °yajñam ida\.m \Ed}%
            \var{\vb mā\.m\lem  \msCa\msCb\msNa\msNb\msNc\msL\Ed; mā \msCc\oo
                 °ttama\lem  \mssCaCbCc\msNb\msNc\Ed; °ttamaḥ \msNa\msL}%

daśakāmadaśadhyāna\.m daśayogadaśākṣaram \veg\dontdisplaylinenum
            \var{\vc °daśadhyāna\.m\lem  \msCa\msCb\msNa\msNb\msNc; °daśadhyāna° \msCc\Ed, °datadhyānan \msL}%
            \var{\vd °kṣaram\lem  \msCb\msNb\msNc; °kṣara{\lost} \msCa, °kṣaraḥ \msCc\msNa\msL\Ed}%

vaiśampāyana uvāca~{\dandab}\dontdisplaylinenum 
            \var{\vo vaiśampāyana uvāca\lem  \msCb\msCc\msNa\msNb\msNc\msL\Ed; {\lost}{\lost}{\lost}{\lost}{\lost}{\lost}vāca \msCa}%

brahmadevapit\textsubring{r}yajño yajño bhūtātitheś ca ha\thinspace{\danda} \dontdisplaylinenum
            \var{\va °deva°\lem  \msCa\msCc\msNa\msNb\msNc\msL\Ed; °daiva° \msCb\oo
                 °yajño\lem  \msCa\msCb\msNa\msNb\Ed; °yojño \msNc, °yajña \msCc\msL}%
            \var{\vb yajño\lem  \msCa\msCb\msNa\msL; yajña° \msCc\msNb\msNc\Ed\oo
                 °titheś ca ha\lem  \msCb; °tithiś ca ha \msCa\msCc\msNa\msNb\msNc\msL, °tithiñ ca yaḥ \Ed}%
            \paral{\textit{\vb {\normalfont cf.\ Śatapathabrāhmana 11.5.6: } aharaharbhūtebhyo bali\.m haret tathaitam bhūtayajña\.m\oo
                {\normalfont Garuḍapurāṇa 1.50.71cd: } bhūtayajñaḥ sa vai jñeyo bhūtebhyo yastvaya\.m baliḥ}} 

japo yogas tapo dhyāna\.m svādhyāyaś ca daśa sm\textsubring{r}taḥ \veg\dontdisplaylinenum
            \var{\vc yogas tapo dhyāna\.m\lem  \mssCaCbCc\msNb\msNc\Ed; yoga{\lost}{\lost}\uncl{dhāna\.m} \msNa,
                                                yoga \gap\gap\ pāna\.m \msL}%
            \var{\vd svādhyāyaś ca\lem  \mssCaCbCc\msNb\msNc\Ed; \uncl{sādhyā}yaś ca \msNa, 
                                                sādhutapaś ca \msL}%

patnīputrapaśubh\textsubring{r}tyadhanadhānyayaśaḥśriyaḥ\thinspace{\dandab} \dontdisplaylinenum
            \var{\va °yaśaḥ°\lem  \msCa\msCb\msNa\msNb\msNc\msL; °yaśa° \msCc\Ed}%

māna bhoga daśa rājan daśakāma udāh\textsubring{r}taḥ \veg\dontdisplaylinenum
            \var{\vc °bhoga\lem  \mssCaCbCc\msNa\msNb\msNc\msL; °bhoga\.m \Ed}%
            \var{\vd °h\textsubring{r}taḥ\lem  \msCa\msCc\msNa\msNb\msNc\msL\Ed; °h\textsubring{r}tam \msCb}%

mānaso yaugapadyaś ca sa\.mkṣiptaś ca viśāmpate\thinspace{\dandab} \dontdisplaylinenum
            \var{\va yaugapadyaś ca\lem  \corr; yaugapadyañ ca \msCa\msCb\msNb,
                        yogapadya\.m ca \msCc\msNa\msNc\msL, yogapadyaś ca \Ed\
                 °kṣiptaś ca\lem  \Ed; °ksipta\.m ca \mssCaCbCc\msNa\msNb\msNc\msL}%
             \paral{\textit{\vo {\normalfont cf.\ Dharmaputrikā 1.56: } sa\.mkṣiptā prathamā jñeyā viśālā samanantaram{\thinspace\ketdanda}
                                        tato dvikaraṇī ceti trividho yoga ucyate{\thinspace\danda}}}

viśālā nāma yogaś ca tato dvikaraṇaḥ sm\textsubring{r}taḥ \veg\dontdisplaylinenum
            \var{\vc viśālā nāma yogaś ca\lem  \Ed; vi{\lost}{\lost}{\lost}{\lost} yogañ ca \msCa, 
                                        viśālā nāma yoga\.m ca \msCb\msCc\msNa\msNb\msNc\msL}%
            \var{\vd dvikaraṇaḥ\lem  \msCa\msCb\msNa\msL; vikaraṇaḥ \msCc\Ed, dvikaraṇī \msNb, dvikaraṇa \msNc}%

raviḥ somo hutāśaś ca sphaṭikāmbaram eva ca\thinspace{\dandab} \dontdisplaylinenum
            \var{\va raviḥ\lem  \msCa; ravi° \msCb\msCc\msNa\msNb\msNc\msL\Ed}%
            \var{\vb sphaṭikāmbara°\lem  \mssCaCbCc\msNb\msNc\Ed; sphaṭikā\.m{\lost}ra° \msNa, sphaṭikā\.msata° \msL}%
            \paral{\textit{\vab {\normalfont cf. Dharmaputikā 4:5cd: } sūryacandrahutāśārciḥsphāṭikāmbarasannibhāḥ}}

daśayogāsanāsīno nityam eva tapodhanaḥ \veg\dontdisplaylinenum
            \var{\vc daśayogāsanāsīno\lem  \msCa\msCc\msNa\msNb\msNc; daśayogasamāsīno \msCb,
                                       devayogāsatāsīno \msL, daśayogāsanāsīnau \Ed}%
            \var{\vd °dhanaḥ\lem  \msCa\msCb\msNa\msL; °dhana \msCc\msNb\msNc\Ed}%

anirodhamanāḥ sūkṣma\.m dhyāyed yogaḥ sa mānasaḥ\thinspace{\dandab} \dontdisplaylinenum
            \var{\va anirodha°\lem  \mssCaCbCc\msNa\msNb\msNc\msL; anilādha° \Ed\oo
                 °manāḥ\lem  \mssCaCbCc\msNa\msNc\msL\Ed; °manā \msNb}%
            \var{\vb dhyāyed yo°\lem  \msCa\msCb\msNa\msNb\msNc\msL; dhyāyo° \msCc, dhyāna\.m yo° \Ed}%
            \paral{\textit{\vab {\normalfont cf.\ Dharmaputrikā 1.54: } ak\textsubring{r}tvā prāṇasa\.mrodha\.m manasaikena kevalam{\thinspace\danda}
                                dhyāyeta parama\.m sūkṣma\.m sa yogo mānasaḥ sm\textsubring{r}taḥ{\thinspace\ketdanda}}}

prāṇāyāmair mano ruddhvā yaugapadyaḥ sa ucyate \veg\dontdisplaylinenum
            \var{\vc °yāmair ma°\lem  \msCa\msNa\msNb\msNcpcorr\msL\Ed;
                                 °yāmai ma° \msCb, °yāmai mma° \msCc, °yāmer ma° \msNcacorr\oo
                 ruddhvā\lem  \mssCaCbCc\msNa\msNb\msNc\msL; ruddhā \Ed}%
            \var{\vd yauga°\lem  \msCa\msCb\msNa\msNcpcorr\msL; yoga° \msCc\msNb\msNcacorr\Ed}%
            \paral{\textit{\vcd {\normalfont cf.\ Dharmaputrikā 1.55:} sa\.myamya manasā prāṇa\.m prāṇāyāmair manas tathā{\thinspace\danda}
                                        eva\.m dhyāyet para\.m sūkṣma\.m yaugapadyaḥ sa ucyate{\thinspace\ketdanda}}}

brahmādistambaparyanta\.m sarva\.m sthāvarajaṅgamam\thinspace{\dandab} \dontdisplaylinenum
            \var{\va °stamba°\lem  \mssCaCbCc\msNa\msNc\Ed; \om\ \msNb, °sta\.mbha° \msL\oo
                 °paryanta\.m\lem  \msCb\msCc\msNa\msL; °\uncl{dviya}{\lost}° \msCa, \om\ \msNb, °paryanta° \msNc\Ed}%
            \var{\vb sarva\.m\lem  \msCb\msNa; {\lost}{\lost} \msCa, sarva° \msCc\msNc\msL\Ed, \om\ \msNb}%
            \paral{\textit{\vab {\normalfont \kb\ Dharmaputrikā 1.57cd: } brahmādistambhaparyantāḥ sarve sthāvarajaṅgamāḥ}}

pralīyamāna\.m dhyāyeta kramāt sūkṣma\.m vicintayet \veg\dontdisplaylinenum
            \var{\vo \om\ \msNb}%
            \var{\vc pralīya°\lem  \mssCaCbCc\msNa\msNc\Ed; \om\ \msNb, praṇīya° \msL}%
            \var{\vd kramāt sū°\lem  \msCa\msCb\msNa\msNc\msL\Ed; kramā sū° \msCc, \om\ \msNb}%
            \paral{\textit{\vcd {\normalfont \kb\ Dharmaputrikā 1.59ab: } pralīyamānan dhyāyeta kramāc chūnya\.m bhavej jagat}}

sa\.mkṣipta eṣa ākhyāto viśālā\.m ch\textsubring{r}ṇu tattvataḥ\thinspace{\dandab} \dontdisplaylinenum
            \var{\va sa\.mkṣipta\lem  \mssCaCbCc\msNa\msNc\Ed; \om\ \msNb, sa\.mkṣiptaḥ \msL\oo
                 eṣa\lem  \mssCaCbCc\msNa\msNc\msL; \om\ \msNb, eva \Ed\oo
                 ākhyāto\lem  \msCb\msNc; ākhyātaḥ \msCa\msCc\msNa\msL\Ed, \om\ \msNb}%
            \paral{\textit{\vab {\normalfont cf.\ Dharmaputrikā 1.60ab: } eṣa yogavidhiḥ proktaḥ sa\.mkṣipto nāma nāmataḥ}}

brahmādisūkṣmaparyanta\.m cintayīta vicakṣaṇaḥ \veg\dontdisplaylinenum
            \var{\vo \om\ \msNb}%
            \var{\vc °sūkṣma°\lem  \mssCaCbCc\msNc\Ed; °sta\.mba° \msNa, \om\ \msNb, tava \msL\oo
                 °paryanta\.m\lem  \mssCaCbCc\msNa\msL; \om\ \msNb, °paryanta \msNc\Ed}%
            \var{\vd cintayīta\lem  \msCa\msCbpcorr\msCc\msNa\msNc\msL\Ed; \om\ \msNb, ciyīta \msCbacorr}%

sa\.mkṣiptā\.m ca viśālā\.m ca cintayīta parasparam\thinspace{\dandab} \dontdisplaylinenum
            \var{\va sa\.mkṣiptā\.m\lem  \msCb\msNc; sa\.mkṣiptā \msCapcorr\msCc\msNa\msL\Ed, \om\ \msCaacorr\msNb\oo
                 viśālā\.m\lem  \msCapcorr\msCb\msNc; \om\ \msCaacorr, viśālā \msCc\msNa\msL\Ed, \om\ \msNb}%

eṣā dvikaraṇī nāma yogasya vidhir ucyate \veg\dontdisplaylinenum
            \var{\vo \om\ \msNb}%
            \var{\vc dvi°\lem  \msCa\msCb\msNa\msNc\msL; vi° \msCc\Ed, \om\ \msNb}%
            \paral{\textit{\vo {\normalfont \kb\ Dharmaputrikā 1.62cd--63ab: } etau sa\.mhāravargau dvau pāramparyeṇa cintayet{\thinspace\ketdanda}
                                                       eṣā dvikaraṇī nāma yogasya vidhir iṣyate{\thinspace\danda}}}

dehamadhye h\textsubring{r}di jñeya\.m h\textsubring{r}dimadhye tu paṅkajam\thinspace{\dandab} \dontdisplaylinenum
            \var{\va jñeya\.m\lem  \msCa\msCb\msNa\Ed\msNc; jñeya \msCc\msL, jñe \msNbacorr; jñe{\lost} \msNbpcorr}%
            \var{\vb tu paṅkajam\lem  \msCb\msCc\msNa\msNb\msNc\msL\Ed; \uncl{tu} pa{\lost}{\lost} \msCa}%

paṅkajasya ca madhye tu karṇikā\.m viddhi gopate \veg\dontdisplaylinenum
            \var{\vc paṅkajasya ca\lem  \msCb\msCc\msNa\msNc\Ed; {\lost}ṅkajasya ca \msCa,
                                                        kaṅkasya tu \msNb, pankaja\.msya ca \msL}%
            \var{\vd karṇikā\.m viddhi gopate\lem  \msCa\msCb\msNa\msNb\msNc\msL; karṇiddhiddhi gopate \msCc, 
                                                               karṇikā\.m ca vi\.mśāpate \Ed}%

karṇikāyās tu madhye tu pañcabindu\.m vidur budhāḥ\thinspace{\dandab} \dontdisplaylinenum
            \var{\vb °bindu\.m\lem  \msCa\msNc; °bindu \msCb\msCc\msNa\msNb\msL\Ed}%

ravisomaśikhā\.m caiva sphaṭikāmbaram eva ca \veg\dontdisplaylinenum
            \var{\vc °śikhā\.m\lem  \msCa\msNa\msL; °śikhā \msCb\msCc\msNb\msNc\Ed}%
            \var{\vd sphaṭi°\lem  \msCa\msCc\msNa\msNb\msNc\msL\Ed; sphāṭi° \msCb}%
    \paral{\textit{\vcd {\normalfont cf.\ Dharmaputrikā 4.5cd: } sūryacandraprakā\-śārcisphāṭikāmbarasannibhāḥ}}

ravimaṇḍalamadhye tu bhāvayec candramaṇḍalam\thinspace{\dandab} \dontdisplaylinenum
            \var{\vb bhāvayec candramaṇḍalam\lem  \msCa\msCb\msNa\msNb\msNc\msL\Ed; bhāvaye candramaṇḍalaḥ \msCc}%

tasya madhye śikhā\.m dhyāyen nirdhūmajvalanaprabhām \veg\dontdisplaylinenum
            \var{\vc °śikhā\.m\lem  \msCa\msCb\msNa\msNb\msNc\msL; °śikhā \msCc\Ed}%

agnimadhye maṇi\.m dhyāyec chuddhadhārājalaprabham\thinspace{\dandab} \dontdisplaylinenum
            \var{\vab maṇi\.m dhyāyec chuddha°\lem  \msCb\msNa\msNb\msNc\msL\Ed; {\lost}{\lost}{\lost}{\lost}{\lost}{\lost} \msCa, 
                                                mani\.m dhyāyec chuddha° \msCc}%
            \var{\vb °dhārā°\lem  \msCa\msCb\msNa\msNb\msNc\msL; °dhāra° \msCc\Ed\oo
                  °prabham\lem  \msCc\msNa\msNb\msNc\msL\Ed; °prabhām \msCa\msCb}%

tasya madhye 'mbara\.m dhyāyet susūkṣma\.m śivam avyayam \veg\dontdisplaylinenum
            \var{\vc 'mbara\.m\lem  \msCa\msCb\msNa\msNb\msNc; 'mbara \msCc, bara\.m \msL, 'kṣara\.m \Ed\
             \vd susūkṣma\.m\lem  \msCc\msNa\msNc\msL; sūkṣma\.m \msCa, susūkṣma° \msCb, 
                                        \uncl{sva}sūkṣma° \msNb, sasūkṣma\.m \Ed}%

daśayogam ida\.m rājan kathita\.m ca mayā tava\thinspace{\dandab} \dontdisplaylinenum

daśadhyāna\.m samāsena kīrtita\.m ś\textsubring{r}ṇu tad yathā \veg\dontdisplaylinenum
            \var{\vc °dhyāna\.m\lem  \mssCaCbCc\msNa\msNc; °dhyāna \msNb\msL\Ed}%

ghoṣaṇī piṅgalā caiva vaidyutī candramālinī\thinspace{\dandab} \dontdisplaylinenum
            \var{\va ghoṣaṇī\lem  \mssCaCbCc\msNa\msNb\msNc\msL; ghoṣaṇā \Ed}%
            \var{\vb vaidyutī\lem  \msCa\msCb\msNa\msNb\msNc\msL\Ed; vidyuta \msCc, vidyutī \Ed}%

candrā mano'nugā caiva suk\textsubring{r}tā ca tathāparā \veg\dontdisplaylinenum
            \var{\vc candrā mano'nugā\lem  \msCb\msNa\msNb\msNc\msL; 
                                candrā manānugā \msCa, candramanonugā \msCc, candro mano'nugā \Ed}%
            \var{\vd suk\textsubring{r}tā ca tathāparā\lem  \msCa\msCc\msNa\msNc\msL; suk\textsubring{r}tā tathāparā \msCb, \om\ \msNb,
                                                                suk\textsubring{r}tā ca tathāpara \Ed}%

saumyā nirañjanā caiva nirālambā ca kīrtitā\thinspace{\dandab} \dontdisplaylinenum
            \var{\va saumyā nirañjanā caiva\lem  \msCb\msCc\msNa\msL\Ed; saumyā nirañjanā {\lost}{\lost} \msCa, \om\ \msNb,
                                                                saumyā ṇirañjanā caiva \msNc}%
            \var{\vb kīrtitā\lem  \mssCaCbCc\msNa\msNb\msNc\Ed; kīrtitāḥ \msL}%

supiṣitvāṅgulau śrotre dhvanim ākarṇayen naraḥ \veg\dontdisplaylinenum
            \var{\vc supiṣitvāṅgulau\lem  \msCa\msCb\msNa\msNb\msNc; su{\lost}{i}{}ṣicāṅgulau \msCc,
                                               supithitvāṅgulau \msL, suśiṣi cāṅgulau \Ed}%
            \var{\vd °karṇaye°\lem  \msNb; °karṣaye° \mssCaCbCc\msNa\msNc\Ed, °karṣaya° \msL}%

tat tad akṣaram ākarṇya am\textsubring{r}tatvāya kalpyate\thinspace{\dandab} \dontdisplaylinenum
            \var{\va °karṇya\lem  \mssCaCbCc\msNb\msNc\msL\Ed; °kaṇṇya \msNa}%

piṅgalā\.m tu śikhādhūmā\.m dhyāyen nityam atandritaḥ \veg\dontdisplaylinenum
            \var{\vc piṅgalā\.m tu śikhādhūmā\.m\lem  \msCa\msCb\msNb\msL; piṅgalā tu śikhādhūma\.m \msCc\Ed,
                                        piṅgalā\.mn tu śikhādhūmā\.m \msNa, piṅgalān tu śikhādhūmā \msNc}%
            \var{\vd °tandritaḥ\lem  \mssCaCbCc\msNa\msNb\msNc\Ed; °tendritaḥ \msL}%

vimuktaḥ sarvapāpebhyo nirdvandvapadam āpnuyāt\thinspace{\dandab} \dontdisplaylinenum
            \var{\va vimuktaḥ\lem  \msCa\msCb\msNa\msNb\msNc\msL\Ed; vimukta \msCc}%
            \var{\vb nirdvandva°\lem  \mssCaCbCc\msNc; nidvanda° \msNa\msNb\msL, nirdvanda° \Ed}%

vaidyutī tu niśāmadhye lakṣate 'jam anāmayam \veg\dontdisplaylinenum
            \var{\vc vaidyutī tu\lem  \mssCaCbCc\msNa\msNb\msNc\Ed; vaidyutīnta \msL}%
            \var{\vd lakṣate 'jam a°\lem  \msCc\Ed; lakṣye teja a° \msCa\msCb, lakṣyateja a° \msNa\msNb\msL,
                                                                                 lakṣateja a° \msNc}%

pañcamāsasadābhyāsād divyacakṣur bhaven naraḥ\thinspace{\dandab} \dontdisplaylinenum
            \var{\va pañcamāsasadā°\lem  \msCb\msNa\msNb\msL; \uncl{pa}{\lost}{\lost}sasadā° \msCa, pañcamāsassadā° \msCc, 
                                                      pañcamāsasamā° \Ed, pañcamāsa\.m sadā° \msNc}%
            \var{\vab °sād di\lem  \mssCaCbCc\msNa\msNb\msL\Ed; °sā di° \msNc}%
            \var{\vb °kṣur bhaven na°\lem  \msCa\msCb\msNa\Ed; °kṣur bhave na° \msCc,
                                        °kṣu bhaven na° \msNb\msL, °rkṣu bhaven na \msNc}%

bindumālā\.m tataḥ paśyet tarucchāyāsamāśritām \veg\dontdisplaylinenum
            \var{\vc tataḥ paśyet\lem  \mssCaCbCc\msNa\msNb\msNc\msL; tu yaḥ paśyen \Ed}%
            \var{\vd tarucchāyā°\lem  \mssCaCbCc\msNa\msNb\msNc\msL; naracchāyā\.m \Ed\oo
                 °śritām\lem  \mssCaCbCc\msNb; °śritāḥ \msNa\msL, °śritam \msNc\Ed}%

jātyasphaṭikasa\.mkāśa\.m d\textsubring{r}ṣṭvā mucyati bandhanaiḥ\thinspace{\dandab} \dontdisplaylinenum
            \var{\va °kasa\.mkāśa\.m\lem  \mssCaCbCc\msNa\msNb\msNc\msLpcorr\Ed; °sa\.mkakāśa\.m \msLpcorr}%
            \var{\vb bandhanaiḥ\lem  \msCa\msNa\msNc; bandhavaiḥ \msCb, bandhanāt \msCc\msNb\Ed,
                                                                                va\.mcanaiḥ \msL}%

dhyāyen mano'nugā nāma pakṣmīr āpīḍya locane \veg\dontdisplaylinenum
            \var{\vd pakṣmī°\lem  \mssCaCbCc\msNa\msL; yakṣmī \msNb, yakṣmo° \msNc, pakṣī° \Ed\oo
                 locane\lem  \msCa\msCb\msNa\msL; locanaḥ \msNb, locanaiḥ \msCc\Ed, locanai \msNc}%

śvetapītāruṇa\.m bindu\.m d\textsubring{r}ṣṭvā bhūyo na jāyate\thinspace{\dandab} \dontdisplaylinenum

mano'nugādi ṣaṭ tv ete dhyānam ukta\.m mayā tava \veg\dontdisplaylinenum
            \var{\vc °ṣaṭ tv ete\lem  \msCa\msNa\msNb\msNc\msL; °ṣaṭ tv etā \msCb, °ṣaṭkena \msCc\Ed}%
            \var{\vd °kta\.m mayā tava\lem  \msCc\msNa\msNc\msL\Ed; \uncl{ka}{\lost}{\lost} tava \msCa, °kta\.m samāsataḥ \msCb,
                                                                                °kta mayā tava \msNb}%


\alalfejezet{paramāṇuḥ}
adhunānyat pravakṣyāmi paramāṇu caturvidham\thinspace{\dandab} \dontdisplaylinenum
            \var{\vb °vidham\lem  \msCa\msCc\msNa\msNb\Ed; °vidhaḥ \msCb}%

pārthivādicaturbhūta\.m yair vyāpta\.m nikhila\.m jagat \danda\dontdisplaylinenum
            \var{\vcd °bhūta\.m yair vyāpta\.m\lem  \msNa; °bhūta\.m yair vyāptin \msCa, °bhūta\.m yai vyāpta\.m \msCb\msCc\msNb, °bhūtair yair vyāpta\.m \Ed}%

lakṣaṇa\.m tasya rājendra ś\textsubring{r}ṇu vakṣyāmi sāmpratam \veg\dontdisplaylinenum

pārthivordhvagatiḥ sūkṣmaḥ paramāṇu narādhipa\thinspace{\dandab} \dontdisplaylinenum
            \var{\va pārthivordhva°\lem  \mssCaCbCc\msNa\msNb; pārthivorddha° \Ed}%
            \var{\vb paramāṇu narādhipa\lem  \msCa\msCb\msNapcorr; paramāṇu narādhipaḥ \msCc,
                                                paramāṇu narādhinarādhipa \msNaacorr, paramānu narādhipa \msNb, paramāṇur narādhipa \Ed}%

pratyakṣadarśana\.m dhyāna\.m lakṣayen niyata\.m śuciḥ \veg\dontdisplaylinenum
            \var{\vc pratyakṣadarśana\.m\lem  \mssCaCbCc\msNb\Ed; pratyakṣa\.m darśana\.m \msNa}%
            \var{\vd lakṣayen niyata\.m\lem  \msCa\msNa\msNb; lakṣayen niyataḥ \msCb, lakṣayen niyata \msCc, lakṣayan niyataḥ \Ed}%

mucyate sarvapāpebhyo rāhunā candramā yathā\thinspace{\dandab} \dontdisplaylinenum
            \var{\va sarvapāpebhyo\lem  \msCb\msCc\msNa\msNb\Ed; \uncl{sarvapāpebhyo} \msCa}%
            \var{\vb rāhunā\lem  \msCb\msCc\msNa\msNb\Ed; {\il}{\il}nā \msCa}%

tena yo 'bhyasate nitya\.m sa yogī bhuvaneśvaraḥ \veg\dontdisplaylinenum
            \var{\vc 'bhyasate\lem  \msCa\msCc\msNa\msNb\Ed; labhyate \msCb}%
            \var{\vd °śvaraḥ\lem  \mssCaCbCc\msNa\msNb; °śvara \Ed}%

adhogati mahārāja paramāṇu jalodbhavaḥ\thinspace{\dandab} \dontdisplaylinenum

abhyased yad ida\.m rājan sarvapātakanāśanam \veg\dontdisplaylinenum
            \var{\vc abhyased yad ida\.m\lem  \msCa\msCb\msNa\msNb\Ed; abhyased ida\.m \msCc}%

āgneyaparamāṇūni tiryagūrdhvagatiḥ sm\textsubring{r}tā\thinspace{\dandab} \dontdisplaylinenum
            \var{\va āgneya°\lem  \mssCaCbCc\msNa\Ed; agneya° \msNb\oo
                 °paramāṇūni\lem  \mssCaCbCc\msNa; °paramānūni \msNb, paramāṇuś ca \Ed}%
            \var{\vb tiryagūrdhva°\lem  \mssCaCbCc\msNa\msNb; tiryagūrddha° \Ed\oo
                 °gatiḥ\lem  \mssCaCbCc\msNa\Ed; °mitiḥ \msNb\oo
                 sm\textsubring{r}tā\lem  \msCa\msNa; sm\textsubring{r}tāḥ \msCb\msCc\msNb\Ed}%

ya ida\.m dhyāyate nityam uttamā\.m gatim āpnuyāt \veg\dontdisplaylinenum
            \var{\vd gatim āpnu°\lem  \msCa\msCc\msNa\msNb\Ed; phalam āpnu° \msCb}%

vāyavyaparamāṇūni adhordhvatiryag āsm\textsubring{r}tā\thinspace{\dandab} \dontdisplaylinenum
            \var{\va vāyavyaparamāṇūni\lem  \msCb; vāya{\il}{\il}ramāṇūni \msCa, vāyavya\.m paramāṇūni \msCc\msNa,
                                                        vāyavyā paramāṇūni \msNb, vāyavya\.m paramāṇuś ca \Ed}%
            \var{\vb °rdhvatirya°\lem  \mssCaCbCc\msNb\Ed; °rdhvantirya° \msNa}%

na sa muhyati ta\.m d\textsubring{r}ṣṭvā vāyusambhava bhūpate \veg\dontdisplaylinenum

catvāra ete rājendra paramāṇu nirīkṣate\thinspace{\dandab} \dontdisplaylinenum
            \var{\vb paramāṇu nirīkṣate\lem  \msCa\msCc\msNa\msNb; paramāṇur rīkṣate \msCb, paramāṇur nirīkṣate \Ed}%

tena sarvamakhair iṣṭa\.m tena tapta\.m tapas tathā \veg\dontdisplaylinenum
            \var{\vc °makhair i°\lem  \msCa\msCb\msNa\msNb\Ed; mayair i° \msCc}%
            \var{\vd tapas tathā\lem  \msCa\msCb\msNa\msNb; tapan tathā \msCc, tapta\.m tathā \Ed}%

tena dattā mahī k\textsubring{r}tsnā saptasāgarasa\.mv\textsubring{r}tā\thinspace{\dandab} \dontdisplaylinenum

sarvatīrthābhiṣekaś ca sarvavratakriyā tathā \veg\dontdisplaylinenum
            \var{\vc °bhiṣekaś ca\lem  \mssCaCbCc\msNb\Ed; °bhiṣeka \msNaacorr, °bhiṣeka\.m ca \msNapcorr}%

anenaiva vidhānena daśadhyāna\.m narādhipa\thinspace{\dandab} \dontdisplaylinenum
            \var{\va anenaiva vidhānena\lem  \msCb\msCc\msNa\msNb\Ed; a{\il}{\lost}{\lost}{\lost}dhānena \msCa}%

kurute avyavacchinna\.m sarvakāmaphalapradam \veg\dontdisplaylinenum
            \var{\vc °cchinna\.m\lem  \mssCaCbCc\msNa\Ed; °cchinna \msNb}%


\alalfejezet{daśākṣaramantraḥ}
daśākṣaramahārāja yogīndrasya mahātmanaḥ\thinspace{\dandab} \dontdisplaylinenum

kathayāmi samāsena ś\textsubring{r}ṇuṣvāvahito bhava \veg\dontdisplaylinenum

praṇavādisvarā trīṇi dīrghabindusamāyutam\thinspace{\dandab} \dontdisplaylinenum

pañca pañca cavarge tu vāyubījam adhasthitam \veg\dontdisplaylinenum

trayodaśasvarāyukta\.m pañcama parikīrtitam\thinspace{\dandab} \dontdisplaylinenum

pañcavi\.mśatimaḥ ṣaṣṭha akṣaraḥ parikīrtitaḥ \veg\dontdisplaylinenum

yād\textsubring{r}śa\.m pañcamaḥ prokta\.m saptame ca prayojayet\thinspace{\dandab} \dontdisplaylinenum

akārasvarasa\.myukta\.m sarvapātakanāśanam \veg\dontdisplaylinenum

prathama\.m pañcame varge t\textsubring{r}tīyasvarayojitam\thinspace{\dandab} \dontdisplaylinenum

uktarekārasa\.myukta\.m navama\.m parikīrtitam \veg\dontdisplaylinenum

daśamaḥ punar o\.mkāraḥ mantraśreṣṭho daśākṣaraḥ\thinspace{\dandab} \dontdisplaylinenum

japato dhyāyate vāpi pārthivādi krameṇa tu \veg\dontdisplaylinenum

mucyate so 'pi sa\.msāre sa\.mśayo nāsti bhūpate\thinspace{\dandab} \dontdisplaylinenum

ācāramūlo dharmas tu dharmamūlo janārdanaḥ \danda\dontdisplaylinenum

tena sarvajagad vyāpta\.m trailokya\.m sa carācara\.m \veg\dontdisplaylinenum


\alalfejezet{ācāravidhiḥ}
\ujvers\nemsloka 
ācārāl labhatīha āyur atulam aiśvaryavitta\.m tathā
\dontdisplaylinenum

\nemslokab 
ācārāt sutam īpsita\.m ca labhate śrīkīrtiprajñāyaśaḥ \danda\dontdisplaylinenum

\nemslokac 
ācārāl labhate ca lakṣmim atula\.m khyāti\.m tathaivottamam
\dontdisplaylinenum

\nemslokad 
ācārād iha mantradharmaparama\.m prāpnoti niḥsa\.mśayam \veg\dontdisplaylinenum

\vers

janamejaya uvāca~{\dandab}\dontdisplaylinenum 

\nemsloka 
ācārāt prabhavānusaṅgakathita\.m suśliṣṭadharmākaram
\dontdisplaylinenum

\nemslokab 
ācārāt katidhāṅga kīrtaya punas t\textsubring{r}ptir na me jāyate \danda\dontdisplaylinenum

\nemslokac 
sarvajñaḥ tvam aha\.m ś\textsubring{r}ṇomi varada\.m kiñcin na me śāśvaram
\dontdisplaylinenum

\nemslokad 
tan me kīrtaya dharmasāraśubhadam ācāramūlāśrayam \veg\dontdisplaylinenum

\vers

vaiśampāyana uvāca~{\dandab}\dontdisplaylinenum 

\nemsloka 
nitya\.m namraśirodvijātiguruṣu śuśrūṣaṇa\.m daivatam
\dontdisplaylinenum

\nemslokab 
tiṣṭhenācamanena cāśanakara\.m vāmāsthimānādaram \danda\dontdisplaylinenum

\nemslokac 
sūryāgniśaśibandhur āryapurataḥ kuryān na cāvaśyakam
\dontdisplaylinenum

\nemslokad 
śasye bhasmani govrajedvijajala\.m kuryān na cārka\.m naraḥ \veg\dontdisplaylinenum

\ujvers\nemsloka 
pādenāgnijala\.m sp\textsubring{r}śen na ca guru\.m pādena pāda\.m tathā
\dontdisplaylinenum

\nemslokab 
śauca\.m kārya jalādinā ca niyata\.m nādho jala\.m kārayet \danda\dontdisplaylinenum

\nemslokac 
kuryān nityabhivādana\.m dvijaguror mātāpit\textsubring{r}r daivatam
\dontdisplaylinenum

\nemslokad 
etācāravidhiḥ samāsaniyamas tubhya\.m mayā kīrtitam \veg\dontdisplaylinenum


\alalfejezet{striyaḥ}
\vers

janamejaya uvāca~{\dandab}\dontdisplaylinenum 

\nemsloka 
strīṇā\.m ki\.m priyam asti tad vada vibho sa\.msārasārastriyām
\dontdisplaylinenum

\nemslokab 
ki\.m sadbhāva na vedmi tasya viṣaye ki\.m dveṣya ki\.m tātpriyam \danda\dontdisplaylinenum

\nemslokac 
paśyāmi na ca tasya garbhakalayā prāpnoti niḥsa\.mśayam
\dontdisplaylinenum

\nemslokad 
māyājālasahasragāpi yuvatī kurvanti bhartā priyam \veg\dontdisplaylinenum

\vers

vaiśampāyana uvāca~{\dandab}\dontdisplaylinenum 

\nemsloka 
rājan ki\.m priyam asti arthaparataḥ paśyāmi nānyan n\textsubring{r}pe
\dontdisplaylinenum

\nemslokab 
putrārthaikaprayojana\.m yuvatayaḥ svāyambhuvoktāmaraiḥ \danda\dontdisplaylinenum

\nemslokac 
kāntā nityakalā pravartanakarī dharmasakhāyā satī
\dontdisplaylinenum

\nemslokad 
māyā vāpi karoti sadya manujātyaktānya vā sevate \veg\dontdisplaylinenum

\ujvers\nemsloka 
strīsaṅga\.m parivarjayen narapate āyāsada\.m duḥkhadam
\dontdisplaylinenum

\nemslokab 
m\textsubring{r}tyudvārabhayākara\.m viṣag\textsubring{r}ham āpat sughorālayam \danda\dontdisplaylinenum

\nemslokac 
agnir mārutamattavāraṇasama tasyānugāmī sadā
\dontdisplaylinenum

\nemslokab 
strīhetor hatarāvaṇas tridaśapa indro 'pi visthāpitaḥ \danda\dontdisplaylinenum

\nemslokab 
strīhetor api candramāstribhuvane dhiktā\.m gataś cāmaro \danda\dontdisplaylinenum

\nemslokad 
daṇḍakṣo hatarāṣṭrapaurasahitaḥ ki\.m bhūya vakṣyāmy aham \veg\dontdisplaylinenum


\alalfejezet{vipramunibhikṣunirgranthiparivrājakarṣādayaḥ}
\vers

janamejaya uvāca~{\dandab}\dontdisplaylinenum 

\nemsloka 
vipre kīd\textsubring{r}śalakṣaṇa\.m bhavati bho kīd\textsubring{r}g muniś cocyate
\dontdisplaylinenum

\nemslokab 
tenārthena bhaveta bhikṣu bhagavan nigranthi ko vā dvija \danda\dontdisplaylinenum

\nemslokac 
kenārthena bhaved dvijendra bhagavan jñeyaḥ parivrājakaḥ
\dontdisplaylinenum

\nemslokad 
! jñeyāḥ kim \textsubring{r}ṣayaś ca lakṣaṇa muner icchāmi jñātu\.m punaḥ \veg\dontdisplaylinenum

\vers

vaiśampāyana uvāca~{\dandab}\dontdisplaylinenum 

\nemsloka 
satya\.m śaucam ahi\.msatā damaśamau bhūtānukampī sadā
\dontdisplaylinenum

\nemslokab 
ātmārāmajito svadharmanirataḥ sattvastha nitya\.m manaḥ \danda\dontdisplaylinenum

\nemslokac 
kāmakrodhayamasvadāranirataḥ sa\.mtyajya lobhaḥ śanaiḥ
\dontdisplaylinenum

\nemslokad 
eva\.m yaḥ kurute dvijātisuvaraḥ śūdro 'pi yaḥ sa\.myamī \veg\dontdisplaylinenum

\ujvers\nemsloka 
tasmāc chadmakavarjitaḥ sa bhagavān sa\.msārabhībhidyakaḥ
\dontdisplaylinenum

\nemslokab 
yat tat sthānapara\.m vrajanti puruṣāḥ tasmāt parivrājakaḥ \danda\dontdisplaylinenum

\nemslokac 
granthidārasuta\.m dhana\.mś ca virati nirgranthika socyate
\dontdisplaylinenum

\nemslokad 
ramyante \textsubring{r}ṣir āśrame dh\textsubring{r}timanas tasmād \textsubring{r}ṣiḥ socyate \veg\dontdisplaylinenum

\ujvers\nemsloka 
kāyavāṅmanadaṇḍatatparataras te daṇḍikarūcyate
\dontdisplaylinenum

\nemslokab 
saddharmaśravaṇa\.m vadanti śravaṇaḥ saddharmabrahmākṣaraḥ \danda\dontdisplaylinenum

\nemslokac 
pāśaprakṣipata\.m paśutvasakala\.m pāśūpatās te sm\textsubring{r}tāḥ
\dontdisplaylinenum

\nemslokad 
vipre pāśupatādibhikṣusakala\.m p\textsubring{r}ṣṭo 'smy aha\.m lakṣaṇam \veg\dontdisplaylinenum

\ujvers\nemsloka 
sarva\.m tat kathito 'si lakṣaṇa mayā sandhiśvanirnāśanam
\dontdisplaylinenum

\nemslokab 
prajñāsa\.mgrahaśītavardhanapara\.m sa\.msāranirmūlanam \danda\dontdisplaylinenum

\nemslokac 
 
\dontdisplaylinenum

\nemslokad 
etaj jñānapara\.m prabodham atula\.m nitya\.m śiva\.m dhāryate \veg\dontdisplaylinenum

\vers


\jump
\begin{center}
\ketdanda iti v\textsubring{r}ṣasārasa\.mgrahe dvāvi\.mśatitamo 'dhyāyaḥ\ketdanda
\end{center}
\dontdisplaylinenum\vers 

\vers

\vers

\vers

\vers
