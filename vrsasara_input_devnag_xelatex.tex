\fejno=0\versno=0
\centerline{\Huge\devanagarifontbold वृषसारसंग्रहः  }

 \versno=0\fejno=19
\thispagestyle{empty}

\fancyhead[CO]{{\footnotesize\devanagarifont वृषसारसंग्रहे  }}
\fancyhead[CE]{{\footnotesize\devanagarifont एकोनविंशतिमो ऽध्यायः  }}
\fancyhead[LE]{}
\fancyhead[RE]{}
\fancyhead[LO]{}
\fancyhead[RO]{}
\szam\bek

\centerline{\Large\devanagarifontbold [   एकोनविंशतिमो ऽध्यायः  ]}{\vrule depth10pt width0pt} 

\alalfejezet{गावः}
\vers


{\devanagarifont विगतराग उवाच {\dandab}\dontdisplaylinenum  }%
 
{\devanagarifont क्रियासूक्ष्मो महाधर्मः कर्मणा केन प्राप्यते \thinspace{\danda} \dontdisplaylinenum }%
 
%Verse 19:1

{\devanagarifont अल्पोपायं नरार्थाय पृच्छामि कथयस्व मे {॥१९:१॥} \veg\dontdisplaylinenum }%
 
{\devanagarifont अनर्थयज्ञ उवाच {\dandab}\dontdisplaylinenum  }%
 
{\devanagarifont अल्पोपायं महाधर्मं कथयामि द्विजोत्तम \thinspace{\danda} \dontdisplaylinenum }%
     \var{{\devanagarifont \numemph\va\textbf{॰धर्मं}\lem \msCa\Ed, ॰धर्म \msNa}}% 

%Verse 19:2

{\devanagarifont सुखेन लभते स्वर्गं कर्मणा येन तच्छृणु {॥१९:२॥} \veg\dontdisplaylinenum }%
 
{\devanagarifont लोकानां मातरो गावो गोभिः सर्वं जगद्धृतम् \thinspace{\dandab} \dontdisplaylinenum }%
 
%Verse 19:3

{\devanagarifont गोमयममृतं सर्वं जातं सर्वं शिवेच्छया {॥१९:३॥} \veg\dontdisplaylinenum }%
     \var{{\devanagarifont \numemph\vd\textbf{सर्वं}\lem \msCa, सर्व॰ \msNa\Ed}}% 

{\devanagarifont सर्वदेवमया गावः सर्वदेवमयो द्विजः \thinspace{\dandab} \dontdisplaylinenum }%
     \var{{\devanagarifont \numemph\va\textbf{॰मया}\lem \msCa\msNa, ॰मयी \Ed}}% 

%Verse 19:4

{\devanagarifont सर्वदेवमयी भूमिः सर्वदेवमयः शिवः {॥१९:४॥} \veg\dontdisplaylinenum }%
     \var{{\devanagarifont \numnoemph\vc\textbf{॰मयी भूमिः}\lem \msCa, ॰मयी भूमि \msNa, ॰मयो भूमिः \Ed}}% 

{\devanagarifont तस्माद्गावः सदा सेव्या धर्ममोक्षार्थसिद्धिदाः \thinspace{\dandab} \dontdisplaylinenum }%
     \var{{\devanagarifont \numemph\vb\textbf{॰दाः}\lem \msCa\msNa, ॰दा \Ed}}% 

%Verse 19:5

{\devanagarifont परिचर्या यथाशक्त्या ग्रासवासजलादिभिः {॥१९:५॥} \veg\dontdisplaylinenum }%
 
{\devanagarifont ताडयेन्नातिवेगेन वाचयेन्मृदुनाचरेत् \thinspace{\dandab} \dontdisplaylinenum }%
 
%Verse 19:6

{\devanagarifont पालयेत घनाढ्येषु भग्नोद्विग्नेषु यत्नतः {॥१९:६॥} \veg\dontdisplaylinenum }%
     \var{{\devanagarifont \numemph\vc \lem \msCa\msNa, पालयन्तर्प्पनाद्येषु \Ed}}% 

{\devanagarifont व्याधिव्रणपरिक्लेश ओषधोपक्रमं चरेत् \thinspace{\dandab} \dontdisplaylinenum }%
     \var{{\devanagarifont \numemph\va\textbf{॰व्रण॰}\lem \msCa, ॰वन॰ \Ed}}% 
    \var{{\devanagarifont \numnoemph\vb\textbf{॰क्रमं च}\lem \msCa, ॰क्रमश्च॰ \Ed}}% 

%Verse 19:7

{\devanagarifont कण्डूयनं च कर्तव्यं यथासौख्यं भवेद्गवाम् {॥१९:७॥} \veg\dontdisplaylinenum }%
 
{\devanagarifont गवां प्रदक्षिणं कृत्वा श्रद्धाभक्तिसमन्वितः \thinspace{\dandab} \dontdisplaylinenum }%
     \var{{\devanagarifont \numemph\vb\textbf{॰न्वितः}\lem \msNa\Ed, ॰न्वित\lk\ \msCa}}% 

%Verse 19:8

{\devanagarifont सागरान्ता मही सर्वा प्रदक्षिणीकृता भवेत् {॥१९:८॥} \veg\dontdisplaylinenum }%
     \var{{\devanagarifont \numnoemph\vcd\textbf{सर्वा प्र}\lem \msCa\msNa, सर्वा न्प्र \Ed}}% 

{\devanagarifont स्पृष्टसंस्पर्शनाद्ये च श्रद्धया यदि मानवः \thinspace{\dandab} \dontdisplaylinenum }%
     \var{{\devanagarifont \numemph\va \lem \msCa\msNa, पृष्टसंस्पर्शनाद्यञ्च \Ed}}% 

%Verse 19:9

{\devanagarifont अहोरात्रकृतं पापं नश्यते नात्र संशयः {॥१९:९॥} \veg\dontdisplaylinenum }%
 
{\devanagarifont लाङ्गूलेनोद्धृतं तोयं मूर्ध्ना गृह्णाति यो नरः \thinspace{\dandab} \dontdisplaylinenum }%
 
%Verse 19:10

{\devanagarifont यावज्जीवकृतं पापं नश्यते नात्र संशयः {॥१९:१०॥} \veg\dontdisplaylinenum }%
 
{\devanagarifont विधिवत्स्नापयेद्गांश्च मन्त्रयुक्तेन वारिणा \thinspace{\dandab} \dontdisplaylinenum }%
     \var{{\devanagarifont \numemph\va \lem \msNa, विधिवच्छापयेद्गां च \msCa, 
विधिवत्स्नापयेद्गांश्च \Ed}}% 

%Verse 19:11

{\devanagarifont तेनाम्भसा स्वयं स्नात्वा सर्वपापक्षयो भवेत् {॥१९:११॥} \veg\dontdisplaylinenum }%
 
{\devanagarifont व्याधिर्विघ्नमलक्ष्मीत्वं नश्यते सद्य एव च \thinspace{\dandab} \dontdisplaylinenum }%
     \var{{\devanagarifont \numemph\va\textbf{व्याधिर्वि॰}\lem \msCa, व्याधिवि॰ \msNa\Ed}}% 

%Verse 19:12

{\devanagarifont मृतापत्यानपत्याश्च स्नानमेव प्रशस्यते {॥१९:१२॥} \veg\dontdisplaylinenum }%
     \var{{\devanagarifont \numnoemph\vc \lem \msCa\msNapcorr, मृत्यपत्यानपत्याश्च, 
मृतापत्याश्च गावाश्च \Ed}}% 

{\devanagarifont गवां शृङ्गोदकं गृह्य मूर्ध्नि यो धारयेन्नरः \thinspace{\dandab} \dontdisplaylinenum }%
     \var{{\devanagarifont \numemph\va\textbf{॰दकं}\lem \msCa\Ed, ॰दक \msNa}}% 

%Verse 19:13

{\devanagarifont स सर्वतीर्थस्नानस्य फलं प्राप्नोति मानवः {॥१९:१३॥} \veg\dontdisplaylinenum }%
     \var{{\devanagarifont \numnoemph\vc\textbf{॰स्नानस्य}\lem \msCa\msNapcorr\Ed, ॰स्नान \msNaacorr}}% 

{\devanagarifont ग्रासमुष्टिप्रदानेन गोषु भक्तिसमन्वितः \thinspace{\dandab} \dontdisplaylinenum }%
 
%Verse 19:14

{\devanagarifont अग्निहोत्रं हुतं तेन सर्वदेवाः सुतर्पिताः {॥१९:१४॥} \veg\dontdisplaylinenum }%
     \var{{\devanagarifont \numemph\vc\textbf{हुतं}\lem \msCa\Ed, फलं \msNa}}% 

{\devanagarifont चत्वारः स्तनधारास्तु यस्तु मूर्ध्ना प्रतीच्छति \thinspace{\dandab} \dontdisplaylinenum }%
 
%Verse 19:15

{\devanagarifont स चतुःसागरं गत्वा स्नानपुण्यफलं लभेत् {॥१९:१५॥} \veg\dontdisplaylinenum }%
 
{\devanagarifont गवार्थं यस्त्यजेत्प्राणान्गोग्रहेषु द्विजोत्तम \thinspace{\dandab} \dontdisplaylinenum }%
 
%Verse 19:16

{\devanagarifont कल्पकोटिशतं दिव्यं शिवलोके महीयते {॥१९:१६॥} \veg\dontdisplaylinenum }%
 
{\devanagarifont च्युतभग्नादिसंस्कारं सर्वं यः कुरुते नरः \thinspace{\dandab} \dontdisplaylinenum }%
 
%Verse 19:17

{\devanagarifont भार्याकोटिशतं दानं यत्फलं परिकीर्तितम् {॥१९:१७॥} \veg\dontdisplaylinenum  }%
     \var{{\devanagarifont \numemph\vc\textbf{भार्या॰}\lem \msNa\Ed, आर्या \msCa}}% 

{\devanagarifont तत्फलं लभते मर्त्यः शिवलोकं च गच्छति \thinspace{\dandab} \dontdisplaylinenum }%
     \var{{\devanagarifont \numemph\vb\textbf{॰लोकं च गच्छति}\lem \msCa\Ed, ॰लोके वगच्छति \msNa}}% 

%Verse 19:18

{\devanagarifont शिवलोकपरिभ्रष्टः पृथिव्यामेकराड्भवेत् {॥१९:१८॥} \veg\dontdisplaylinenum }%
 
{\devanagarifont समासतः समाख्यातं यथातत्त्वं द्विजोत्तम \thinspace{\dandab} \dontdisplaylinenum }%
 
%Verse 19:19

{\devanagarifont न शक्यं विस्तराद्वक्तुं गोमहाभाग्यमुत्तमम् {॥१९:१९॥} \veg\dontdisplaylinenum }%
     \var{{\devanagarifont \numemph\vc\textbf{विस्तराद्व॰}\lem \Ed, विस्तरान्व॰ \msCa, विस्तरां व॰ \msNa}}% 
    \var{{\devanagarifont \numnoemph\vd \lem \msCa, गोमहात्म्यस॰ \Ed}}% 


\alalfejezet{चातुर्वर्ण्यम्}
{\devanagarifont विगतराग उवाच {\dandab}\dontdisplaylinenum  }%
 
{\devanagarifont देवा अष्टविधाः प्रोक्तास्तिर्यक्पञ्चविधः स्मृतः \thinspace{\danda} \dontdisplaylinenum }%
     \var{{\devanagarifont \numemph\va\textbf{देवा अष्ट॰}\lem \mssCaCbCc\msNa\msNb\msNc, देवाःरष्ट॰ \Ed}}% 
    \var{{\devanagarifont \numnoemph\vb\textbf{॰र्यक्प॰}\lem \msCapcorr\msCb\msNa\msNb\msNc\Ed, ॰क्प॰ \msCaacorr\oo 
\textbf{स्मृतः}\lem \msCa\msNa\msNb\msNc\Ed, स्मृताः \msCb}}% 

%Verse 19:20

{\devanagarifont मानुषमेकमेवाहुश्चातुर्वर्णः कथं भवेत् {॥१९:२०॥} \veg\dontdisplaylinenum }%
     \var{{\devanagarifont \numnoemph\vc\textbf{मानुष॰}\lem \mssCaCbCc\msNa\msNb\msNc, मानुष्य॰ \Ed}}% 
    \var{{\devanagarifont \numnoemph\vd\textbf{॰वर्णः}\lem \msCa\msCb\msNa\msNb, ॰वर्ण्ण \msNc, ॰व्वर्ण्यः \Ed}}% 

{\devanagarifont अनर्थयज्ञ उवाच {\dandab}\dontdisplaylinenum  }%
 
{\devanagarifont पूर्वकल्पसृजस्त्वेष विष्णुना प्रभविष्णुना \thinspace{\danda} \dontdisplaylinenum }%
     \var{{\devanagarifont \numemph\va\textbf{॰सृजस्त्वेष}\lem \msCa\msNa\msNb\msNc, ॰सृज\lk\lk  \msCb, ॰सृजत्येष \Ed}}% 

%Verse 19:21

{\devanagarifont एकवर्णो द्विजश्चासीत्सर्वकल्पाग्रमग्रतः {॥१९:२१॥} \veg\dontdisplaylinenum }%
     \var{{\devanagarifont \numnoemph\vc\textbf{एकवर्णो द्विजश्चासी॰}\lem \msNc, ए{\il}{\il}\uncl{र्ण्णो}{\il}{\il}{\il}श्चासी॰ \msCa, 
एकवर्ण्णा द्विजश्चासी॰ \msCb, 
एकवर्ण्ण द्विजश्चासी॰ \msNb, 
एकव\uncl{र्ण्णो} द्विजश्चासी॰ \msNa, एवं वर्णा द्विजश्चासी॰ \Ed}}% 

{\devanagarifont सर्ववेदविदो विप्राः सर्वयज्ञविदस्तथा \thinspace{\dandab} \dontdisplaylinenum }%
     \var{{\devanagarifont \numemph\vb\textbf{॰यज्ञ॰}\lem \msCa\msCb\msNa\msNb\msNc, ॰वेद॰ \Ed}}% 

%Verse 19:22

{\devanagarifont तेषां विप्रसहस्राणां यज्ञोत्साहमनो भवेत् {॥१९:२२॥} \veg\dontdisplaylinenum }%
     \var{{\devanagarifont \numnoemph\vc\textbf{तेषां}\lem \msCa\msCb\msNa\msNc, तथा \Ed}}% 
    \var{{\devanagarifont \numnoemph\vd\textbf{यज्ञोत्सा॰}\lem \msCa\msCb\msNa\Ed, यज्ञोच्छाह॰ \msNb, यज्ञोतसाह॰ \msNc\ \unmetr}}% 

{\devanagarifont वृद्धविप्रसहस्राणां मतमाज्ञाय ब्राह्मणैः \thinspace{\dandab} \dontdisplaylinenum }%
     \var{{\devanagarifont \numemph\vb\textbf{॰ज्ञाय}\lem \msNa\msCb\msNb\msNc, ॰\uncl{ज्ञाय} \msCa, ॰श्रित्य \Ed\oo 
\textbf{ब्राह्मणैः}\lem \msCa\msNa\msNb\msNc\Ed, ब्राह्मणेः \msCb}}% 

%Verse 19:23

{\devanagarifont कर्तुं कर्म समारब्धं कर्म चापि विभज्यते {॥१९:२३॥} \veg\dontdisplaylinenum }%
     \var{{\devanagarifont \numnoemph\vc\textbf{कर्तुं}\lem \msCa\msCb\msNc\Ed, कर्तु \msNa\msNb\oo 
\textbf{समारब्धं}\lem \msCa\msNa\msNb\msNc, समारन्धं \msCb, समारब्ध \Ed}}% 
    \var{{\devanagarifont \numnoemph\vd\textbf{कर्म चापि}\lem \msCa\msCb\msNb\msNc, कर्मं चापि \msNa, कर्मश्चापि \Ed}}% 

{\devanagarifont ऋत्विजत्वे स्थिताः केचित्केचित्संरक्षणे स्थिताः \thinspace{\dandab} \dontdisplaylinenum }%
     \var{{\devanagarifont \numemph\vb\textbf{॰रक्षणे स्थिताः}\lem \msNa\msNb\msNc\Ed, ॰रक्ष\lac\  \msCa, \uncl{रक्षणे स्थि}\lk  \msCb}}% 

%Verse 19:24

{\devanagarifont अर्थोपार्जनयुक्तान्ये अन्ये शिल्पे नियोजिताः {॥१९:२४॥} \veg\dontdisplaylinenum }%
     \var{{\devanagarifont \numnoemph\vc\textbf{॰युक्तान्ये}\lem \msCa\msNb\msNc\Ed, ॰युक्ता\uncl{न्ये} \msCb, ॰युक्तात्मे \msNa}}% 
    \var{{\devanagarifont \numnoemph\vd\textbf{अन्ये}\lem \msCapcorr\msCb\msNa\msNb\msNc\Ed, \om\ \msCaacorr}}% 

{\devanagarifont एवं यज्ञविधानेन कर्तुमारेभिरे पुरा \thinspace{\dandab} \dontdisplaylinenum }%
     \var{{\devanagarifont \numemph\vb\textbf{॰मारेभिरे}\lem \msCa\msCb\msNa\msNb\msNc\Ed, ॰म आरेमिरे \msNa}}% 

%Verse 19:25

{\devanagarifont यथोद्दिष्टेन कर्मेण यज्ञोत्साहमवर्तत {॥१९:२५॥} \veg\dontdisplaylinenum }%
     \var{{\devanagarifont \numnoemph\vc\textbf{यथोद्दि॰}\lem \Ed, यथोदि॰ \msCa\msCb\msNa\msNb\msNc}}% 
    \var{{\devanagarifont \numnoemph\vd\textbf{॰वर्तत}\lem \msCa\msCb\msNa\msNb\Ed, ॰वर्त्ततः \msNc}}% 

{\devanagarifont आगता ऋषयः सर्वे देवताः पितरस्तथा \thinspace{\dandab} \dontdisplaylinenum }%
 
%Verse 19:26

{\devanagarifont अन्योन्यमब्रुवन्तत्र देवर्षिपितृदेवताः {॥१९:२६॥} \veg\dontdisplaylinenum }%
 
{\devanagarifont यज्ञार्थमसृजद्वर्णं विधिना क्रतुहेतवः \thinspace{\dandab} \dontdisplaylinenum }%
     \var{{\devanagarifont \numemph\vab\textbf{॰सृजद्वर्णं वि॰}\lem \msCa\msNb\msNc\Ed, ॰सृजद्वर्ण्णान्वि॰ \msCb, ॰सृद्वण्णन्विवि॰ \msNa}}% 
    \var{{\devanagarifont \numnoemph\vb\textbf{॰धिना}\lem \msCa\msCb\msNa\msNc\Ed, ॰धानां \msNb\oo 
\textbf{क्रतुहेतवः}\lem \msNa\msNb\msNc, \uncl{क्रतु}\lac  तवः \msCa, क्रतुहेतु\uncl{त} \msCb, 
पातुहेतवः \Ed}}% 

%Verse 19:27

{\devanagarifont एवमेव प्रवर्तन्तु भवद्भिर्द्विजसत्तमाः {॥१९:२७॥} \veg\dontdisplaylinenum }%
     \var{{\devanagarifont \numnoemph\vd\textbf{भवद्भिर्द्वि॰}\lem \msCapcorr\msCb, भगवद्भिर्द्वि॰ \msCaacorr, भवद्भि द्वि॰ \msNa\msNb\msNc, 
भवतिद्वि॰ \Ed\oo 
\textbf{॰सत्तमाः}\lem \msCa\msCb\msNa\msNc\Ed, ॰सत्तमः \msNb}}% 

{\devanagarifont इज्याध्ययनसम्पन्ना ब्राह्मणा ये ऽत्र कल्पिताः \thinspace{\dandab} \dontdisplaylinenum }%
     \var{{\devanagarifont \numemph\vb\textbf{ये ऽत्र}\lem \msCa\msCb\msNa\msNb\msNc, यत्र \Ed\oo 
\textbf{कल्पिताः}\lem \msCa\msCb\msNa\msNc\Ed, कल्पिता \msNb}}% 

%Verse 19:28

{\devanagarifont सुविप्रा विप्रतां यान्तु षट्कर्मनिरताः सदा {॥१९:२८॥} \veg\dontdisplaylinenum }%
     \var{{\devanagarifont \numnoemph\vc\textbf{सुविप्रा}\lem \msCa\msCb\msNa\msNc\Ed, सुविप्र \msNb\oo 
\textbf{यान्तु}\lem \msCa\msCb\msNa\msNc\Ed, यातु \msNb}}% 
    \var{{\devanagarifont \numnoemph\vd\textbf{षट्कर्म॰}\lem \msCa\msCb\msNa\msNb\msNc, षड्कर्मा॰ \Ed\oo 
\textbf{सदा}\lem \msCa\msCb\msNa\Ed, सजा \msNb, सदाः \msNc}}% 

{\devanagarifont रक्षणार्थं तु ये विप्राः कल्पिताः शस्त्रपाणयः \thinspace{\dandab} \dontdisplaylinenum }%
     \var{{\devanagarifont \numemph\va\textbf{विप्राः}\lem \msCa\msNa\msNb\msNc\Ed, विप्रा \msCb}}% 
    \var{{\devanagarifont \numnoemph\vb\textbf{शस्त्र॰}\lem \msCa\msNa\msNc\Ed, शास्त्र॰ \msCb\msNb}}% 

%Verse 19:29

{\devanagarifont क्षतत्राणाय विप्राणां नित्यक्षत्रव्रतोद्भवाः {॥१९:२९॥} \veg\dontdisplaylinenum }%
     \var{{\devanagarifont \numnoemph\vc\textbf{क्षत॰}\lem \msCa\msNa\msNb, क्षत्र॰ \msCb\msNc, कृत॰ \Ed\oo 
\textbf{विप्राणां}\lem \msCa\msCb\msNa\Ed, विप्राणा \msNb}}% 
    \var{{\devanagarifont \numnoemph\vd\textbf{नित्यक्षत्र॰}\lem \msCa\msCb\msNc, नित्यं क्षत्र॰ \msNa\msNb, नित्यं क्षात्र॰ \Ed\oo 
\textbf{॰व्रतोद्भवाः}\lem \msCa\msCb\msNa\msNc\Ed, ॰व्रतोत्तमः \msNb}}% 
    \paral{{\devanagarifontsmall \vcd {\englishfont cf.\ MBh 12.59.128ab:}
                  ब्राह्मणानां क्षतत्राणात्ततः क्षत्रिय उच्यते }}

{\devanagarifont अर्थोपार्जनमुद्दिश्य कल्पिता ये द्विजातयः \thinspace{\dandab} \dontdisplaylinenum }%
     \var{{\devanagarifont \numemph\vb\textbf{ये}\lem \msCa\msCb\msNa\msNc\Ed, यो \msNb}}% 

%Verse 19:30

{\devanagarifont ते तु वैश्यत्वमायान्तु वार्त्तोपायरतोद्भवाः {॥१९:३०॥} \veg\dontdisplaylinenum }%
     \var{{\devanagarifont \numnoemph\vd\textbf{वार्त्तोपायरतो॰}\lem \msCa\msCb\msNa\msNb\msNc, वार्त्तो आपणतोद्भवाः \Ed}}% 

{\devanagarifont वधबन्धनकर्मसु शिल्पस्थानविधेषु च \thinspace{\dandab} \dontdisplaylinenum }%
     \var{{\devanagarifont \numemph\va\textbf{वधबन्धनकर्मसु}\lem \msCa\msNa, वधवन्धनकर्मेषु \msCb\msNb\msNc\Ed}}% 
    \var{{\devanagarifont \numnoemph\vb\textbf{॰विधेषु}\lem \msCa\msCb\msNa\msNb\msNc, ॰वधेषु \Ed}}% 

%Verse 19:31

{\devanagarifont कल्पिता ये द्विजातीनां सर्वे शूद्रा भवन्तु ते {॥१९:३१॥} \veg\dontdisplaylinenum }%
     \var{{\devanagarifont \numnoemph\vc\textbf{॰जातीनां}\lem \msCa\msCb\msNa\msNb\Ed, ॰जातीना \msNc}}% 

{\devanagarifont प्राजापत्यं ब्राह्मणानामिज्याध्ययनतत्परात् \thinspace{\dandab} \dontdisplaylinenum }%
     \var{{\devanagarifont \numemph\vab\textbf{प्राजापत्यं ब्राह्मणानामिज्याध्ययनतत्परात्}\lem \msCa\msCb, 
प्राजापत्यं ब्राह्मणानांमिज्याध्ययनतत्परात् \msNa, 
प्राजापत्य  ब्राह्मणामीज्याध्ययनतत्परात् \msNb, 
प्राजापत्यं ब्राह्मणामीज्याध्ययनतत्परात् \msNc, 
प्राजापत्यं ब्राह्मणानामीज्याध्ययनतत्परां \Ed}}% 

%Verse 19:32

{\devanagarifont स्थानमैन्द्रं क्षत्रियाणां प्रजापालनतत्परात् {॥१९:३२॥} \veg\dontdisplaylinenum }%
     \var{{\devanagarifont \numnoemph\vc\textbf{॰न्द्रं}\lem \msCa\msCb\msNa\Ed, ॰न्द्र \msNb\msNc}}% 
    \var{{\devanagarifont \numnoemph\vd\textbf{॰त्परात्}\lem \msNa\msCb\msNb\msNc, ॰\uncl{त्परात्} \msCa, ॰त्परं \Ed}}% 
    \paral{{\devanagarifontsmall \vo {\englishfont cf.\ Vāyupurāṇa 1.8.166:}
                 प्राजापत्यं ब्राह्मणानां स्मृतं स्थानं क्रियावताम्\thinspace{\devanagarifontsmall ।}
                 स्थानम् ऐन्द्रं क्षत्रियाणां संग्रामेष्वपलायिनाम्\thinspace{\devanagarifontsmall ॥}
                    {\englishfont \similar\ Bhaviṣyapurāṇa 2.1.34, etc.} }}

{\devanagarifont वैश्यानां वासवस्थानं वाणिज्यकृषिजीविनाम् \thinspace{\dandab} \dontdisplaylinenum }%
     \var{{\devanagarifont \numemph\vb\textbf{वाणिज्य॰}\lem \msCa\msCb\msNb, वाणिज्यं \msNa\Ed\oo 
\textbf{॰जीविनाम्}\lem \msCa\msCb\msNa\msNc\Ed, ॰जीविनम् \msNb}}% 

%Verse 19:33

{\devanagarifont शूद्राणां मरुतः स्थानं शुश्रूषानिरतात्मनाम् {॥१९:३३॥} \veg\dontdisplaylinenum }%
     \var{{\devanagarifont \numnoemph\vcd\textbf{शूद्राणां मरुतः स्थानं शुश्रूषानिरतात्मनाम्}\lem \msCa\msCb\msNc\Ed, \om\ \msNaacorr, 
शूद्राणां मरुतस्थानं शुश्रूषानिरतात्मनाम् \msNapcorr 
शूद्राणा मरुतः स्थानं शुश्रूषानिरतात्मनाम् \msNb}}% 
    \paral{{\devanagarifontsmall \vo {\englishfont cf.\ Vāyupurāṇa 1.8.167--168ab:}
                 वैश्यानां मारुतं स्थानं स्वधर्ममुपजीविनाम्\thinspace{\devanagarifontsmall ।} 
                 गान्धर्वं शूद्रजातीनां प्रतिचारेण तिष्ठताम्\thinspace{\devanagarifontsmall ॥}
                 स्थानान्येतानि वर्णानां व्यत्याचारवतां स्वयम्\thinspace{\devanagarifontsmall ।} }}

{\devanagarifont महर्षिपितृदेवानां मतमाज्ञाय निश्चितः \thinspace{\dandab} \dontdisplaylinenum }%
     \var{{\devanagarifont \numemph\va\textbf{॰देवानां म॰}\lem \msCa\msNa\msNb\Ed, ॰देवाना म॰ \msCb}}% 
    \var{{\devanagarifont \numnoemph\vb\textbf{मत॰}\lem \msCa\msCb\msNa\msNb\Ed, मन॰ \msNc\oo 
\textbf{निश्चितः}\lem \msCa\msCb\msNb\msNc\Ed, निश्चिताः \msNa}}% 

%Verse 19:34

{\devanagarifont एष संकल्पितो ब्रह्मा पद्मयोनिः पितामहः {॥१९:३४॥} \veg\dontdisplaylinenum }%
     \var{{\devanagarifont \numnoemph\vc\textbf{॰कल्पितो ब्रह्मा}\lem \msCa\msCb\msNa\Ed, ॰कल्पिता ब्राह्मा \msNb}}% 

{\devanagarifont संकल्पप्रभवाः सर्वे देवदानवमानवाः \thinspace{\dandab} \dontdisplaylinenum }%
     \var{{\devanagarifont \numemph\vb\textbf{देवदानवमानवाः}\lem \msCa\msCb\msNa\msNc\Ed, देवदेदानमानवः \msNbacorr, 
देवदानमानवः \msNbpcorr}}% 

%Verse 19:35

{\devanagarifont पशुपक्षिमृगा मुख्या यावन्ति जगसम्भवाः {॥१९:३५॥} \veg\dontdisplaylinenum }%
     \var{{\devanagarifont \numnoemph\vc\textbf{॰मृगा}\lem \msCb\msNa\msNb\Ed, \lk गा \msCa}}% 
    \var{{\devanagarifont \numnoemph\vd\textbf{जग॰}\lem \msCa\msCb\Ed, जंगम॰ \msNa, जगे \msNb}}% 

{\devanagarifont भूतसंकल्पकं नाम कल्पमासीद्द्विजोत्तम \thinspace{\dandab} \dontdisplaylinenum }%
     \var{{\devanagarifont \numemph\va\textbf{भूतसंकल्पकं नाम}\lem \msCa\msCb\msNa\msNb\msNc, भूतसंकल्पकर्ता य \Ed}}% 

%Verse 19:36

{\devanagarifont कीर्तितानि समासेन किमन्यच्छ्रोतुमिच्छसि {॥१९:३६॥} \veg\dontdisplaylinenum }%
 
{\devanagarifont विगतराग उवाच {\dandab}\dontdisplaylinenum  }%
 
{\devanagarifont किं तपः सर्ववर्णानां वृत्तिर्वापि तपोधन \thinspace{\danda} \dontdisplaylinenum }%
 
%Verse 19:37

{\devanagarifont यज्ञाश्चैव पृथक्त्वेन श्रोतुमिच्छामि तत्त्वतः {॥१९:३७॥} \veg\dontdisplaylinenum }%
 
{\devanagarifont अनर्थयज्ञ उवाच {\dandab}\dontdisplaylinenum  }%
 
{\devanagarifont ब्राह्मणस्य तपो यज्ञाः - तपः क्षात्रस्य रक्षणम् \thinspace{\danda} \dontdisplaylinenum }%
 
%Verse 19:38

{\devanagarifont वैश्यश्च तप वाणिज्य तपः शूद्रस्य सेवनम् {॥१९:३८॥} \veg\dontdisplaylinenum }%
 
{\devanagarifont प्रतिग्रहधनो विप्रः क्षत्रियस्य धनुर्धनम् \thinspace{\dandab} \dontdisplaylinenum }%
 
%Verse 19:39

{\devanagarifont कृषिर्धनं तथा वैश्यः शूद्रः शुश्रूषणं धनम् {॥१९:३९॥} \veg\dontdisplaylinenum }%
 
{\devanagarifont आरम्भयज्ञः क्षत्रस्य हविर्यज्ञो विशस्तथा \thinspace{\dandab} \dontdisplaylinenum }%
     \paral{{\devanagarifontsmall \vab {\englishfont  \similar\ MBh 12.224.61ab and 12.230.12ab }  }}

%Verse 19:40

{\devanagarifont शूद्रः परिचरो यज्ञो जपयज्ञो द्विजातयः {॥१९:४०॥} \veg\dontdisplaylinenum }%
 
{\devanagarifont सत्य तीर्थ द्विजातीनां रण तीर्थं तु क्षत्रियाः \thinspace{\dandab} \dontdisplaylinenum }%
 
%Verse 19:41

{\devanagarifont आर्या तीर्थं तु वैशानां ! शूद्रतीर्थं तु वै द्विजाः {॥१९:४१॥} \veg\dontdisplaylinenum }%
 
{\devanagarifont नास्ति विद्यासमो मित्रो नास्ति दानसमः सखा \thinspace{\dandab} \dontdisplaylinenum }%
 
%Verse 19:42

{\devanagarifont नास्ति ज्ञानसमो बन्दुर्नास्ति यज्ञो जपः समः {॥१९:४२॥} \veg\dontdisplaylinenum }%
 
{\devanagarifont धर्महीनो मृतस्तुल्यो देवतुल्यो जितेन्द्रियः \thinspace{\dandab} \dontdisplaylinenum }%
 
%Verse 19:43

{\devanagarifont यज्ञतुल्यो ऽभयं दाता शिवतुल्यो मनोन्मनः {॥१९:४३॥} \veg\dontdisplaylinenum }%
 
{\devanagarifont विगतराग उवाच {\dandab}\dontdisplaylinenum  }%
 
{\devanagarifont दान यज्ञस्तपस्तीर्थं संन्यासं योग एव च \thinspace{\danda} \dontdisplaylinenum }%
 
%Verse 19:44

{\devanagarifont एतेषु कतमः श्रेष्ठः श्रोतुमिच्छामि कीर्तय {॥१९:४४॥} \veg\dontdisplaylinenum }%
 
{\devanagarifont अनर्थयज्ञ उवाच {\dandab}\dontdisplaylinenum  }%
 
{\devanagarifont दानधर्मसहस्रेभ्यः यज्ञयाजी विशिष्यते \thinspace{\danda} \dontdisplaylinenum }%
 
%Verse 19:45

{\devanagarifont यज्ञयाजीसहस्रेभ्यस्तीर्थयात्री विशिष्यते {॥१९:४५॥} \veg\dontdisplaylinenum }%
 
{\devanagarifont तीर्थयात्रिसहस्रेभ्यस्तपनिष्टो विशिष्यते \thinspace{\dandab} \dontdisplaylinenum }%
 
%Verse 19:46

{\devanagarifont तपनिष्ठसहस्रेभ्यः श्रेष्ठः संन्यासिकः स्मृतः {॥१९:४६॥} \veg\dontdisplaylinenum }%
 
{\devanagarifont संन्यासीनां सहस्रेभ्यः श्रेष्ठो यच्य जितेन्द्रियः \thinspace{\dandab} \dontdisplaylinenum }%
 
%Verse 19:47

{\devanagarifont जितेन्द्रियसहस्रेभ्यः योगयुक्तो विशिष्यते {॥१९:४७॥} \veg\dontdisplaylinenum }%
 
{\devanagarifont योगयुक्तसहस्रेभ्यः श्रेष्ठो लीनमनः स्मृतः \thinspace{\dandab} \dontdisplaylinenum }%
 
%Verse 19:48

{\devanagarifont तस्मात्सर्वप्रयत्नेन आदौ मन विशोधयेत् {॥१९:४८॥} \veg\dontdisplaylinenum }%
 
{\devanagarifont निगृहीतेन्द्रियग्रामः स्वर्गमोक्षौ तु साधनम् \thinspace{\dandab} \dontdisplaylinenum }%
 
%Verse 19:49

{\devanagarifont विशिष्ठे त्विन्द्रियग्रामे तिर्यन्नरकसाधनम् {॥१९:४९॥} \veg\dontdisplaylinenum }%
 
{\devanagarifont विगतराग उवाच {\dandab}\dontdisplaylinenum  }%
 
{\devanagarifont चराचराणां भूतानां कतमः श्रेष्ठ उच्यते \thinspace{\danda} \dontdisplaylinenum }%
 
%Verse 19:50

{\devanagarifont कथयस्व ममाद्य त्वं छेत्तुमर्हसि संशयम् {॥१९:५०॥} \veg\dontdisplaylinenum }%
 
{\devanagarifont अनर्थयज्ञ उवाच {\dandab}\dontdisplaylinenum  }%
 
{\devanagarifont चराचराणां भूतानां तत्र श्रेष्ठो - चराः स्मृताः \thinspace{\danda} \dontdisplaylinenum }%
 
%Verse 19:51

{\devanagarifont चराणां चैव सर्वेषां बुद्धिमान्श्रेष्ठ उच्यते {॥१९:५१॥} \veg\dontdisplaylinenum }%
 
{\devanagarifont बुद्धिमान्षु ! च सर्वेषु ततः श्रेष्ठ नराः स्मृताः \thinspace{\dandab} \dontdisplaylinenum }%
 
%Verse 19:52

{\devanagarifont नराणां चैव सर्वेषां ब्राह्मणः श्रेष्ठ उच्यते {॥१९:५२॥} \veg\dontdisplaylinenum }%
 
{\devanagarifont विद्वर्स्वपि च सर्वेषु कृतबुद्धिर्विशिष्यते \thinspace{\dandab} \dontdisplaylinenum }%
 
%Verse 19:53

{\devanagarifont कृतबुद्धिषु सर्वेषु श्रेष्ठः कर्ता स उच्यते {॥१९:५३॥} \veg\dontdisplaylinenum }%
 
{\devanagarifont कर्तृष्वपि च सर्वेषु ब्रह्मवेदी विशिष्यते \thinspace{\dandab} \dontdisplaylinenum }%
 
{\devanagarifont ब्रह्मवेदि परं ! विप्रः नान्यं वेद्मि परं तपः  \danda\dontdisplaylinenum }%
 
%Verse 19:54

{\devanagarifont स विप्रः स तपस्वी च स योगी स शिवः स्मृतः {॥१९:५४॥} \veg\dontdisplaylinenum }%
 
{\devanagarifont 
\jump
\begin{center}
\ketdanda\ इति वृषसारसंग्रहे दानयज्ञविशेषो नाम उनविंशतितमो ऽध्यायः \ketdanda
\end{center}
\dontdisplaylinenum\vers  }%
 