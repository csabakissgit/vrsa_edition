\fejno=0\versno=0
\centerline{\Huge\devanagarifontbold वृषसारसंग्रहः  }

 
{\vrule depth10pt width0pt}
\versno=0\fejno=9
\thispagestyle{empty}

\centerline{\Large\devanagarifontbold [   नवमो ऽध्यायः  ]}{\vrule depth10pt width0pt} \fancyhead[CO]{{\footnotesize\devanagarifont वृषसारसंग्रहे  }}
\fancyhead[CE]{{\footnotesize\devanagarifont नवमो ऽध्यायः  }}
\fancyhead[LE]{}
\fancyhead[RE]{}
\fancyhead[LO]{}
\fancyhead[RO]{}
\szam\bek



\alalfejezet{त्रैगुण्यम्}
\vers


{\devanagarifont [अनर्थयज्ञ उवाच {\dandab}\dontdisplaylinenum  ] }%
 
{\devanagarifont त्रिकालगुणभेदेन भिन्नं सर्वचराचरम् \thinspace{\danda} \dontdisplaylinenum }%
     \var{{\devanagarifont \numemph\va\textbf{त्रिकाल॰}\lem \mssALL, त्रिष्काल॰ \msCc\oo 
\textbf{॰भेदेन}\lem \mssALL, ॰भेन \msNbacorr}}% 
    \var{{\devanagarifont \numnoemph\vb\textbf{भिन्नं}\lem \mssALL, भिन्न \msNb}}% 
    \lacuna{\devanagarifontsmall {\englishfont Witnesses used for this chapter: \msCa\ ff.\thinspace 205v--207r, 
                                              \msCb\ ff.\thinspace 211v--212v, 
                                              \msCc\ ff.\thinspace 282r--283v,
                                              \msNa\ ff.\thinspace 13r--14v, 
                                              \msNb\ exp.\thinspace 54 (lower) -- 55 (lower),
                                              \msNc\ ff.\thinspace 221r--222v,
                                              \Ed\ pp.\thinspace 606--609; 
                                              \mssCaCbCc~= \msCa + \msCb + \msCc} }%
  
%Verse 9:1

{\devanagarifont तस्मात्त्रिगुणबन्धेन वेष्टितं निखिलं जगत् {॥ ९:१॥} \veg\dontdisplaylinenum }%
     \var{{\devanagarifont \numnoemph\vc\textbf{तस्मात्त्रि॰}\lem \mssALL, तस्मा त्रि॰ \msCc\msNc}}% 

{\devanagarifont विगतराग उवाच {\dandab}\dontdisplaylinenum  }%
 
{\devanagarifont त्रैकाल्यमिति किं ज्ञेयं त्रैधातुकशरीरिणः \thinspace{\danda} \dontdisplaylinenum }%
     \var{{\devanagarifont \numemph\va\textbf{॰काल्यम्}\lem \mssALL, ॰कालम् \msCa\msNc}}% 
    \var{{\devanagarifont \numnoemph\vab\textbf{किं ज्ञेयं त्रै॰}\lem \msCa\msNc, 
विज्ञेयं त्रै॰ \msCb\msNa\msNb\Ed, कि ज्ञेयम्त्रै॰ \msCc}}% 
    \var{{\devanagarifont \numnoemph\vb\textbf{॰धातुक॰}\lem \mssALL, ॰धायुक्त॰ \Ed}}% 

%Verse 9:2

{\devanagarifont किंचिद्विस्तरमेवेह कथयस्व तपोधन {॥ ९:२॥} \veg\dontdisplaylinenum }%
     \var{{\devanagarifont \numnoemph\vc\textbf{किंचि॰}\lem \mssALL, 
सात्त्विको भगव् विष्णु राजसः कमलोद्भवः\thinspace{\devanagarifont ।} 
तामसो भगवानीशः सकलं विक किञ्चि॰ \msCbacorr\ 
{\englishfont (eyeskip to 9.5)}\oo 
\textbf{॰वेह}\lem \mssALL, ॰तद्धि \Ed}}% 
    \var{{\devanagarifont \numnoemph\vd\textbf{कथयस्व}\lem \mssALL, क\lk\lk \lk\ \msCa}}% 

{\devanagarifont अनर्थयज्ञ उवाच {\dandab}\dontdisplaylinenum  }%
 
{\devanagarifont त्रैकाल्यं त्रिगुणं ज्ञेयं व्यापी प्रकृतिसम्भवः \thinspace{\danda} \dontdisplaylinenum }%
     \var{{\devanagarifont \numemph\va\textbf{॰काल्यं}\lem \mssALL, ॰काल्य \msCc\oo 
\textbf{॰गुणं}\lem \mssALL, ॰गुण \msCc}}% 

%Verse 9:3

{\devanagarifont अन्योन्यमुपजीवन्ति अन्योन्यमनुवर्तिनः {॥ ९:३॥} \veg\dontdisplaylinenum }%
     \paral{{\devanagarifontsmall \vcd {\englishfont \similar\ \BRAHMANDAPUR\ 1.4.9--10:}
                         एत एव त्रयो लोका एत एव त्रयो गुणाः\thinspace{\devanagarifontsmall ।}  
                         एत एव त्रयो वेदा एत एव त्रजो ऽग्नयः\thinspace{\devanagarifontsmall ॥}
                         परस्परान्वया ह्येते परस्परमनुव्रताः\thinspace{\devanagarifontsmall ।}
                         परस्परेण वर्तन्ते प्रेरयन्ति परस्परम्\thinspace{\devanagarifontsmall ॥}
                      {\englishfont \similar\ \VAYUP\ 1.5.16--17ab \similar\ \LINPU\ 1.70.78--79} }}

{\devanagarifont सत्त्वं रजस्तमश्चैव रजः सत्त्वं तमस्तथा \thinspace{\dandab} \dontdisplaylinenum }%
     \var{{\devanagarifont \numemph\va\textbf{सत्त्वं}\lem \mssALL, सत्व \msNb\oo 
\textbf{रजस्त॰}\lem \mssALL, रजत॰ \Ed}}% 
    \var{{\devanagarifont \numnoemph\vb\textbf{रजः}\lem \msCa\msCb\msNa\msNc, रज॰ \msCc\msNb\Ed\oo 
\textbf{सत्त्वं तमस्तथा}\lem \msCa\msNa\msNc, सत्त्वं तमन्तथा \msCb, 
सत्वस्तमस्तथा \msCc\msNb, सत्त्वतमस्तथा \Ed}}% 

%Verse 9:4

{\devanagarifont तमः सत्त्वं रजश्चैव अन्योन्यमिथुनाः स्मृताः {॥ ९:४॥} \veg\dontdisplaylinenum }%
     \var{{\devanagarifont \numnoemph\vc\textbf{तमः सत्त्वं}\lem \msCa\msCb\msNa\msNc, तमसत्व॰ \msCc, तमः सत्व॰ \msNb\Ed\oo 
\textbf{रजश्चैव}\lem \mssALL, रजःश्चैव \msCb}}% 
    \var{{\devanagarifont \numnoemph\vd\textbf{स्मृताः}\lem \mssALL, \om\ \msCc}}% 
    \paral{{\devanagarifontsmall \vd {\englishfont \similar\ \BRAHMANDAPUR\ 1.4.11ab:}
                         अन्योन्यं मिथुनं ह्येते अन्योन्यमुपजीविनः
                     {\englishfont \similar\ \VAYUP\ 1.5.17cd \similar\ \LINPU\ 1.70.80ab} }}

{\devanagarifont सात्त्विको भगवान्विष्णू राजसः कमलोद्भवः \thinspace{\dandab} \dontdisplaylinenum }%
     \var{{\devanagarifont \numemph\va\textbf{॰ष्णू}\lem \corr, ॰ष्णु \mssCaCbCc\msNa\msNb\msNc\Ed}}% 
    \var{{\devanagarifont \numnoemph\vb \lem \mssALL, 
\uncl{राज}\lk\lk\lk\lk\lk\lk\ \msCa}}% 
    \paral{{\devanagarifontsmall \vo {\englishfont \compare\ \BRAHMANDAPUR\ 1.4.6cd:}
                 सत्त्वं विष्णू रजो ब्रह्मा तमो रुद्रः प्रजापतिः }}

%Verse 9:5

{\devanagarifont तामसो भगवानीशः सकलंविकलेश्वरः {॥ ९:५॥} \veg\dontdisplaylinenum }%
     \var{{\devanagarifont \numnoemph\vcd\textbf{तामसो भगवानीशः सकलं}\lem \mssALL, 
\lk\lk \lk\lk \lk\lk \lk\lk \uncl{सकलम्} \msCa}}% 

{\devanagarifont सत्त्वं कुन्देन्दुवर्णाभं पद्मरागनिभं रजः \thinspace{\dandab} \dontdisplaylinenum }%
     \var{{\devanagarifont \numemph\va\textbf{सत्त्वं}\lem \mssALL, सत्व \msCc\msNc\oo 
\textbf{॰वर्णाभं}\lem \mssALL, ॰वर्ण्णाभ \msCc, ॰वण्णाभं \msNa}}% 

%Verse 9:6

{\devanagarifont तमश्चाञ्जनशैलाभं कीर्तितानि मनीषिभिः {॥ ९:६॥} \veg\dontdisplaylinenum }%
     \var{{\devanagarifont \numnoemph\vc\textbf{॰भं}\lem \mssALL, ॰भा \Ed}}% 

{\devanagarifont सत्त्वं जलं रजो ऽङ्गारं तमो धूमसमाकुलम् \thinspace{\dandab} \dontdisplaylinenum }%
     \var{{\devanagarifont \numemph\va\textbf{जलं}\lem \mssALL, रजं \msCc, ज्वाल \msNb\oo 
\textbf{रजो ऽङ्गारं}\lem \mssALL, 
र\uncl{ङ्गो}ङ्गारन् \msCc, रजोङ्गरन् \Ed}}% 

%Verse 9:7

{\devanagarifont एतद्गुणमयैर्बद्धाः पच्यन्ते सर्वदेहिनः {॥ ९:७॥} \veg\dontdisplaylinenum }%
     \var{{\devanagarifont \numnoemph\vd\textbf{॰देहिनः}\lem \mssALL, ॰देहिना \msCb}}% 

{\devanagarifont विगतराग उवाच {\dandab}\dontdisplaylinenum  }%
 
{\devanagarifont केन केन प्रकारेण गुणपाशेन बध्यते \thinspace{\danda} \dontdisplaylinenum }%
     \var{{\devanagarifont \numemph\vb\textbf{गुण॰}\lem \mssALL, \om\ \msCa}}% 

%Verse 9:8

{\devanagarifont चिह्नमेषां पृथक्त्वेन कथयस्व तपोधन {॥ ९:८॥} \veg\dontdisplaylinenum }%
     \var{{\devanagarifont \numnoemph\vc\textbf{॰षां पृथक्त्वेन}\lem \mssALL, ॰षा पृथकेन \msNc}}% 

{\devanagarifont अनर्थयज्ञ उवाच {\dandab}\dontdisplaylinenum  }%
 
{\devanagarifont अनेकाकारभावेन बध्यन्ते गुणबन्धनैः \thinspace{\danda} \dontdisplaylinenum }%
 
%Verse 9:9

{\devanagarifont मोहिता नाभिजानन्ति जानन्ति शिवयोगिनः {॥ ९:९॥} \veg\dontdisplaylinenum }%
     \var{{\devanagarifont \numemph\vc\textbf{॰भिजानन्ति}\lem \mssALL, ॰भिजानान्ति \msCc}}% 
    \var{{\devanagarifont \numnoemph\vd\textbf{जानन्ति}\lem \mssALL, \om\ \msCbacorr}}% 

{\devanagarifont ऊर्ध्वंगो नित्यसत्त्वस्थो मध्यगो रजसावृतः \thinspace{\dandab} \dontdisplaylinenum }%
     \var{{\devanagarifont \numemph\va\textbf{ऊर्ध्वंगो नित्य}\lem \conj, 
ऊर्ध्वाङ्गो नित्य॰ \mssCaCbCc\msNapcorr\Ed, 
ऊर्ध्वाङ्गा नत्य॰ \msNaacorr, 
ऊर्ध्वगो सित्य॰ \msNbacorr, 
ऊर्ध्वगो सत्य॰ \msNbpcorr, 
उर्ध्वाङ्गो नित्य॰ \msNc\oo 
\textbf{॰सत्त्व॰}\lem \msCa\msCb\msNa\msNc, ॰सत्य॰ \msCc\Ed, ॰नित्य॰ \msNb}}% 
    \var{{\devanagarifont \numnoemph\vb\textbf{मध्यगो}\lem \mssALL, मध्यमो \Ed\oo 
\textbf{॰वृतः}\lem \mssALL, ॰वृतम् \Ed}}% 

%Verse 9:10

{\devanagarifont अधोगतिस्तमोऽवस्था भवन्ति पुरुषाधमाः {॥ ९:१०॥} \veg\dontdisplaylinenum }%
     \var{{\devanagarifont \numnoemph\vc\textbf{॰गतिस्तमो॰}\lem \mssALL, ॰गतितमो॰ \msCb\msCc}}% 

{\devanagarifont स्वर्गे ऽपि हि त्रयो वैते भावनीयास्तपोधन \thinspace{\dandab} \dontdisplaylinenum }%
 
%Verse 9:11

{\devanagarifont मानुषेषु च तिर्येषु गुणभेदास्त्रयस्त्रयः {॥ ९:११॥} \veg\dontdisplaylinenum }%
     \var{{\devanagarifont \numemph\vc\textbf{मानुषेषु}\lem \mssALL, मनुष्येषु \msCb, मानुष्येषु \msNc\oo 
\textbf{तिर्येषु}\lem \mssALL, तीर्येषु \Ed}}% 
    \var{{\devanagarifont \numnoemph\vd\textbf{॰स्त्रयः}\lem \mssALL, ॰स्त्रः \msCbacorr}}% 


\alalalfejezet{सात्त्विकोत्तमाः}

{\devanagarifont ब्रह्मा विष्णुश्च रुद्रश्च धर्म इन्द्रः प्रजापतिः \thinspace{\dandab} \dontdisplaylinenum }%
     \var{{\devanagarifont \numemph\vb\textbf{धर्म इन्द्रः}\lem \mssALL, इर्म इन्द्र \msCb, धर्मरिन्द्र॰ \Ed}}% 

%Verse 9:12

{\devanagarifont सोमो ऽग्निर्वरुणः सूर्यो दश सत्त्वोत्तमाः स्मृताः {॥ ९:१२॥} \veg\dontdisplaylinenum }%
     \var{{\devanagarifont \numnoemph\vc\textbf{ग्निर्वरुणः}\lem \msCa\msNa\msNc, ग्नि वरुण \msCb\msCc\msNb\Ed}}% 
    \var{{\devanagarifont \numnoemph\vd\textbf{दश}\lem \mssALL, दशः \Ed\oo 
\textbf{सत्त्वोत्तमाः}\lem \mssALL, सत्वत्तमाः \msCb, सत्तोतमाः \msNc}}% 


\alalalfejezet{सात्त्विकमध्यमाः}

{\devanagarifont रुद्रादित्या वसुसाध्या विश्वेशमरुतो ध्रुवः \thinspace{\dandab} \dontdisplaylinenum }%
     \var{{\devanagarifont \numemph\vab\textbf{॰दित्या वसुसाध्या}\lem \msCb\msNa\msNb\msNc, ॰दित्या वसुसा\lk\ \msCa, ॰दित्य वसुसाध्या \msCc, 
॰दित्य वसुसाध्याः वि॰ \Ed}}% 
    \var{{\devanagarifont \numnoemph\vb\textbf{विश्वेश॰}\lem \mssALL, \lk श्वेश \msCa, विश्वेशि॰ \msCc}}% 

%Verse 9:13

{\devanagarifont ऋषयः पितरश्चैव दशैते सत्त्वमध्यमाः {॥ ९:१३॥} \veg\dontdisplaylinenum }%
     \var{{\devanagarifont \numnoemph\vd\textbf{दशैते}\lem \mssALL, दशैतेते \msCbacorr}}% 


\alalalfejezet{सात्त्विकाधमाः}

{\devanagarifont तारा ग्रहाः सुरा यक्षा गन्धर्वाः किंनरोरगाः \thinspace{\dandab} \dontdisplaylinenum }%
     \var{{\devanagarifont \numemph\va\textbf{ग्रहाः सुरा}\lem \mssALL, ग्रहास्वराः \msCc, ग्रहाऽसुरा \Ed}}% 
    \var{{\devanagarifont \numnoemph\vb\textbf{गन्धर्वाः}\lem \msCa\msNb\msNc\Ed, गन्धर्वा \msCb\msNa, गन्धर्व्वाः गन्धर्व्वा \msCc}}% 

%Verse 9:14

{\devanagarifont रक्षोभूतपिशाचाश्च दशैते सात्त्विकाधमाः {॥ ९:१४॥} \veg\dontdisplaylinenum }%
     \var{{\devanagarifont \numnoemph\vc\textbf{॰पिशाचाश्च}\lem \mssALL, ॰पिशाश्चाश्च \msNc}}% 
    \var{{\devanagarifont \numnoemph\vd\textbf{दशैते}\lem \mssALL, दशेते \msCb\oo 
\textbf{सात्त्विका॰}\lem \mssALL, सत्वका॰ \msCb}}% 


\alalalfejezet{राजसोत्तमाः}

{\devanagarifont ऋत्विक्पुरोहिताचार्ययज्वानो ऽतिथि विज्ञनी \thinspace{\dandab} \dontdisplaylinenum }%
     \var{{\devanagarifont \numemph\vb\textbf{॰विज्ञनी}\lem \mssALL, ॰विज्ञकौ \Ed}}% 

%Verse 9:15

{\devanagarifont राजा मन्त्री व्रती वेदी दशैते राजसोत्तमाः {॥ ९:१५॥} \veg\dontdisplaylinenum }%
     \var{{\devanagarifont \numnoemph\vc\textbf{राजा}\lem \eme, राज॰ \mssCaCbCc\msNa\msNb\msNc\Ed\oo 
\textbf{॰मन्त्री व्रती}\lem \mssALL, ॰मन्त्रि व्रतो \Ed}}% 
    \var{{\devanagarifont \numnoemph\vd\textbf{राजसो॰}\lem \mssALL, रामसो \msCb}}% 


\alalalfejezet{राजसमध्यमाः}

{\devanagarifont सूतो ऽम्बष्ठवणिश्चोग्रः शिल्पिकारुकमागधाः \thinspace{\dandab} \dontdisplaylinenum }%
     \var{{\devanagarifont \numemph\va\textbf{सूतो ऽम्बष्ठ॰}\lem \corr, सूतो \lk ष्ट॰ \msCa, सूत\uncl{म्बष्ट}॰ \msCb, 
सूतोन्वष्ठ॰ \msCc, 
सूतोत्वष्टा॰ \msNa, सूतोत्वष्ट॰ \msNb\msNc, 
सूतो ऽम्बष्ट॰ \Ed\oo 
\textbf{॰वणिश्चो॰}\lem \mssALL, ॰वणिश्वो॰ \Ed}}% 
    \var{{\devanagarifont \numnoemph\vb\textbf{शिल्पि॰}\lem \msNb, शिल्प॰ \mssCaCbCc\msNa\msNc\Ed\oo 
\textbf{मागधाः}\lem \mssALL, मागधा \msCc}}% 

%Verse 9:16

{\devanagarifont वेणवैदेहकामात्या दशैते रजमध्यमाः {॥ ९:१६॥} \veg\dontdisplaylinenum }%
     \var{{\devanagarifont \numnoemph\vc \lem \msCa\msCc\msNa\msNb, वैणवेदेहकामात्या \msCb, 
वेनवैदेहकामात्या \msNc, वेणवैदेचकौ मात्या \Ed}}% 


\alalalfejezet{राजसाधमाः}

{\devanagarifont चर्मकृत्कुम्भकृत्कोली लोहकृत्त्रपुनीलिकाः \thinspace{\dandab} \dontdisplaylinenum }%
     \var{{\devanagarifont \numemph\va\textbf{॰कृत्कोली}\lem \mssALL, ॰ककोली \msNa, ॰कृत्काली \Ed}}% 
    \var{{\devanagarifont \numnoemph\vb\textbf{॰नीलिकाः}\lem \mssALL, ॰तीलिका \Ed}}% 

%Verse 9:17

{\devanagarifont नटमुष्टिकचण्डाला दशैते रजसाधमाः {॥ ९:१७॥} \veg\dontdisplaylinenum }%
     \var{{\devanagarifont \numnoemph\vc\textbf{॰मुष्टिक॰}\lem \mssALL, ॰मौष्टिक॰ \msCc\oo 
\textbf{॰चण्डाला}\lem \mssALL, ॰चाण्डालः \Ed}}% 
    \var{{\devanagarifont \numnoemph\vd\textbf{दशैते}\lem \mssALL, दशेते \msCb}}% 
    \paral{{\devanagarifontsmall \vc {\englishfont = \UMS\ 2.10a, 2.20a = \UUMS\ 2.31c} }}


\alalalfejezet{तामसोत्तमाः}

{\devanagarifont गोगजगवया अश्वमृगचामरकिंनराः \thinspace{\dandab} \dontdisplaylinenum }%
     \var{{\devanagarifont \numemph\va\textbf{॰गवया}\lem \mssALL, ॰गवय \msNb, ॰गवयो \Ed}}% 
    \var{{\devanagarifont \numnoemph\vb\textbf{॰चामर॰}\lem \msCa\msCb\msNa\msNc, ॰वानर॰ \msCc\Ed, ॰\uncl{वा}नर॰ \msNb}}% 

%Verse 9:18

{\devanagarifont सिंहव्याघ्रवराहाश्च दशैते तामसोत्तमाः {॥ ९:१८॥} \veg\dontdisplaylinenum }%
     \var{{\devanagarifont \numnoemph\vc\textbf{॰वराहा॰}\lem \mssALL, ॰वराह॰ \msNb\Ed}}% 
    \var{{\devanagarifont \numnoemph\vd\textbf{तामसोत्तमाः}\lem \mssALL, तामशोत्तमः \msCb, 
तमसोत्तमाः \Ed}}% 


\alalalfejezet{तामसमध्यमाः}

{\devanagarifont अजमेषमहिष्याश्च मूषिकानकुलादयः \thinspace{\dandab} \dontdisplaylinenum }%
     \var{{\devanagarifont \numemph\va\textbf{॰महिष्याश्च}\lem \mssALL, ॰महिंष्या च \msNb}}% 

%Verse 9:19

{\devanagarifont उष्ट्ररङ्कुशशगण्डा दशैते तममध्यमाः {॥ ९:१९॥} \veg\dontdisplaylinenum }%
     \var{{\devanagarifont \numnoemph\vc\textbf{उष्ट्र॰}\lem \mssALL, उष्ट॰ \msCc, दंष्ट्रि॰ \Ed\oo 
\textbf{॰शशगण्डा}\lem \mssALL, ॰शगण्डाश्च \Ed}}% 
    \var{{\devanagarifont \numnoemph\vd\textbf{तममध्यमाः}\lem \mssALL, तमध्यमाः \msCa}}% 


\alalalfejezet{तामसाधमाः}

{\devanagarifont ऋक्षगोधामृगशृङ्गिबकवानरगर्दभाः \thinspace{\dandab} \dontdisplaylinenum }%
     \var{{\devanagarifont \numemph\vb\textbf{॰गर्दभाः}\lem \mssALL, ॰गर्दभः \Ed}}% 

%Verse 9:20

{\devanagarifont सूकरश्वानगोमायुर्दशैते तामसाधमाः {॥ ९:२०॥} \veg\dontdisplaylinenum }%
     \var{{\devanagarifont \numnoemph\vc\textbf{सूकर॰}\lem \mssALL, सुखर॰ \msCb}}% 
    \var{{\devanagarifont \numnoemph\vcd\textbf{॰गोमायुर्द॰}\lem \mssALL, ॰गोमायु द॰ \msNa\msNb}}% 
    \var{{\devanagarifont \numnoemph\vd\textbf{॰शैते}\lem \mssALL, ॰शेते \msCb}}% 


\alalalfejezet{तमसात्त्विकाः}

{\devanagarifont क्रौञ्चहंसशुकश्येनभासबारुण्डसारसाः \thinspace{\dandab} \dontdisplaylinenum }%
     \var{{\devanagarifont \numemph\va\textbf{क्रौञ्च॰}\lem \Ed, क्रोञ्च॰ \mssCaCbCc\msNa\msNb\msNc}}% 
    \var{{\devanagarifont \numnoemph\vb\textbf{॰सारसाः}\lem \mssALL, ॰सारसा \msNc}}% 

%Verse 9:21

{\devanagarifont चक्राह्वशुकमायूरा दशैते तमसात्त्विकाः {॥ ९:२१॥} \veg\dontdisplaylinenum }%
     \var{{\devanagarifont \numnoemph\vc\textbf{॰ह्वशुकमायूरा}\lem \mssALL, 
॰\uncl{ङ्ग}\lk\lk \lk यूरा \msCa, ॰ङ्गशुकमायूरा \Ed}}% 
    \var{{\devanagarifont \numnoemph\vd\textbf{दशैते}\lem \mssALL, दशेते \msCb\oo 
\textbf{तमसात्त्विकाः}\lem \msCc\msNc\Ed, तमस्सात्त्विकाः \msCa\msNb\ \unmetr, 
नमः सात्विकाः \msCb\ \unmetr, 
तमःसात्विकाः \msNa\ \unmetr}}% 


\alalalfejezet{तमराजसाः}

{\devanagarifont बलाकाः कुक्कुटाः काकाश्चिल्ललावकतित्तिराः \thinspace{\dandab} \dontdisplaylinenum }%
     \var{{\devanagarifont \numemph\va\textbf{बलाकाः}\lem \corr, वलाका \msCa\msNa\msNc, वलाक॰ \msCb\msCc\msNb\Ed}}% 
    \var{{\devanagarifont \numnoemph\vab\textbf{कुक्कुटाः काकाश्चि॰}\lem \corr, कुक्कुटकाकाश्चि॰ \msCa\msCb\ \unmetr, कुर्कुटा काकाश्चि॰ \msCc\msNc, 
कुर्कुटकाकाश्चि \msNa\msNb, कुक्कुटो काका चि॰ \Ed}}% 
    \var{{\devanagarifont \numnoemph\vb\textbf{॰तित्तिराः}\lem \mssALL, ॰तित्तराः \msNc, ॰तित्तिरिः \Ed}}% 

%Verse 9:22

{\devanagarifont गृध्रकङ्कबकश्येन दशैते तमराजसाः {॥ ९:२२॥} \veg\dontdisplaylinenum }%
     \var{{\devanagarifont \numnoemph\vc\textbf{गृध्र॰}\lem \mssALL, गृध॰ \msNc}}% 


\alalalfejezet{तामसाधमादि}

{\devanagarifont कोकिलोलूककञ्जल्यकपोताः पञ्च एव च \thinspace{\dandab} \dontdisplaylinenum }%
     \var{{\devanagarifont \numemph\va\textbf{कोकिलो॰}\lem \mssALL, कौकिलो॰ \msCb\oo 
\textbf{॰कञ्जल्य॰}\lem \eme, ॰किञ्जल्य॰ \msCa\msCc\msNa, ॰किञ्जल्क॰ \msCb\msNb\msNc\Ed}}% 
    \var{{\devanagarifont \numnoemph\vb\textbf{च}\lem \mssALL, चः \msNc}}% 

%Verse 9:23

{\devanagarifont शारिकाश्च कुलिङ्गाश्च दशैते तमसाधमाः {॥ ९:२३॥} \veg\dontdisplaylinenum }%
     \var{{\devanagarifont \numnoemph\vc\textbf{शारिकाश्च}\lem \corr, शारिका च \mssCaCbCc\msNa\msNb\msNc, शालिका च \Ed\oo 
\textbf{कुलिङ्गाश्च}\lem \corr, कुलिङ्गा च \msCa\msNb\Ed, कुलिङ्का च \msCb\msCc\msNc, 
कुलिकां च \msNa}}% 

{\devanagarifont मकरगोहनक्राश्च ऋक्षाश्च तमसात्त्विकाः \thinspace{\dandab} \dontdisplaylinenum }%
     \var{{\devanagarifont \numemph\va\textbf{॰गोहनक्राश्च}\lem \mssALL, 
॰गोहनक्रा च \msCc, ॰ग्रोहनक्राश्च \msNb}}% 
    \var{{\devanagarifont \numnoemph\vb\textbf{ऋक्षाश्च}\lem \conj, ऋषा च \mssCaCbCc\msNa\msNb\msNc\Ed\oo 
\textbf{तमसात्त्विकाः}\lem \Ed, तम\uncl{स्सा}\lk\lk\ \msCa, 
तमःसात्विकाः \msCb\msCc\msNa\msNb\ \unmetr, तसमात्विकाः \msNc}}% 

{\devanagarifont कच्छपशिशुकुम्भीरमण्डूकास्तमराजसाः  \danda\dontdisplaylinenum }%
     \var{{\devanagarifont \numnoemph\vc\textbf{॰शिशु॰}\lem \eme, ॰शुशु॰ \mssCaCbCc\msNa\msNb\msNc\Ed\oo 
\textbf{॰कुम्भीर॰}\lem \mssALL, ॰कम्भीरा \msCc\Ed}}% 
    \var{{\devanagarifont \numnoemph\vd\textbf{॰मण्डूका॰}\lem \mssALL, ॰मण्डूक॰ \msNb, ॰मण्डुका॰ \Ed}}% 

%Verse 9:24

{\devanagarifont शङ्खशुक्तिकशम्बूकाः कवय्यस्तमतामसाः {॥ ९:२४॥} \veg\dontdisplaylinenum }%
     \var{{\devanagarifont \numnoemph\ve\textbf{शम्बूकाः}\lem \corr, ॰शम्बूका \mssCaCbCc\msNa\msNb\Ed, ॰\uncl{स}म्बूकाः \msNc}}% 
    \var{{\devanagarifont \numnoemph\vf\textbf{॰कवय्य॰}\lem \conj, ॰कबन्ध्या॰ \mssCaCbCc\msNa\msNbpcorr\msNc\Ed, ॰कबन॰ \msNbacorr\oo 
\textbf{॰मतामसाः}\lem \msCb\Ed, ॰मस्तामसाः \msCa\msCc\msNc\ \unmetr, ॰मःतामसाः \msNa\msNb\ \unmetr}}% 

{\devanagarifont चन्दनागरुपद्मं च प्लक्षोदुम्बरपिप्पलाः \thinspace{\dandab} \dontdisplaylinenum }%
     \var{{\devanagarifont \numemph\va\textbf{॰गरु॰}\lem \mssALL, ॰गुरु॰ \Ed}}% 

%Verse 9:25

{\devanagarifont वटदारुशमीबिल्वा दशैते तमसात्त्विकाः {॥ ९:२५॥} \veg\dontdisplaylinenum }%
     \var{{\devanagarifont \numnoemph\vc\textbf{॰बिल्वा}\lem \msCa\msCb\msNa\Ed, ॰बिल्व \msCc\msNb\msNc}}% 
    \var{{\devanagarifont \numnoemph\vd\textbf{दशैते}\lem \mssALL, दशै \msCc\oo 
\textbf{तमसात्त्विकाः}\lem \Ed, तमस्सात्विकाः \msCa\ \unmetr, तमःसात्विकाः \msCb\msCc\msNa\msNb\msNc\ \unmetr}}% 

{\devanagarifont जाम्बीरलकुचाम्रातदाडिमाकोलवेतसाः \thinspace{\dandab} \dontdisplaylinenum }%
     \var{{\devanagarifont \numemph\va\textbf{जाम्बीर॰}\lem \mssALL, जम्बीर॰ \msCc}}% 
    \var{{\devanagarifont \numnoemph\vb\textbf{॰दाडिमा॰}\lem \mssALL, ॰द्राडिमा॰ \msCc, ॰द्राडि\uncl{हा}॰ \msNa}}% 

%Verse 9:26

{\devanagarifont निम्बनीपो †ध्रवावश्च† दशैते तमराजसाः {॥ ९:२६॥} \veg\dontdisplaylinenum }%
     \var{{\devanagarifont \numnoemph\vc\textbf{॰नीपो}\lem \mssALL, ॰नीपौ \msNc\oo 
\textbf{ध्रवावश्च}\lem \mssALL, 
धवावश्च \msCapcorr, धुवावश्च \Ed}}% 
    \var{{\devanagarifont \numnoemph\vd\textbf{दशैते}\lem \mssALL, \lk\lk\lk\ \msCa}}% 

{\devanagarifont वृक्षवल्लीलतावेणुत्वक्सारतृणभूरुहाः \thinspace{\dandab} \dontdisplaylinenum }%
     \var{{\devanagarifont \numemph\va\textbf{वृक्षवल्ली॰}\lem \mssALL, \uncl{वृक्षवल्ली} \msNb}}% 
    \var{{\devanagarifont \numnoemph\vb\textbf{॰त्वक्सारतृण॰}\lem \msCa\msCb\msNa\msNb, 
॰त्वक्सारस्तृण॰ \msCc\Ed, ॰त्वकसारतृण॰ \msNc\ \unmetr}}% 

%Verse 9:27

{\devanagarifont मीरजाश्च शिलाशस्या दशैते तमसात्त्विकाः {॥ ९:२७॥} \veg\dontdisplaylinenum }%
     \var{{\devanagarifont \numnoemph\vc\textbf{मीरजाश्च}\lem \corr, मीरजा च \msCa\msCc\msNa\msNb\msNc\Ed, मीनजा च \msCb}}% 
    \var{{\devanagarifont \numnoemph\vd\textbf{तमसात्त्विकाः}\lem \msNc\Ed, तमस्सात्विकाः \msCa, 
तमःसात्विकाः \msCb\msCc\msNa\ \unmetr, तमःसाधिकाः \msNb\ \unmetr}}% 

{\devanagarifont भ्रमरादिपतङ्गाश्च क्रिमिकीटजलौकसः \thinspace{\dandab} \dontdisplaylinenum }%
     \var{{\devanagarifont \numemph\va\textbf{पतङ्गाश्च}\lem \mssALL, पतङ्गानां \Ed}}% 
    \var{{\devanagarifont \numnoemph\vb \lem \mssCaCbCc\msNa, क्रिमिकीटजलोकसः \msNb, 
क्रिमिकीटजलौक\uncl{साः} \msNc, किमिकीटजलौकसां \Ed}}% 

%Verse 9:28

{\devanagarifont यूकोद्दंशमशानां च विष्ठाजास्तमसात्त्विकाः {॥ ९:२८॥} \veg\dontdisplaylinenum }%
     \var{{\devanagarifont \numnoemph\vc \lem \msCa, 
यूकोदंशमशानाञ्च \msCb\msNa, 
यूकोदंशमसकानाञ्च \msCc\ \unmetr, 
यूकोदंशमसानान्तु \msNb, 
\uncl{यूकोद्दं}\lk\lk \lk\lk \lk\  \msNc, 
युक्तोदंशमशानाश्च \Ed}}% 
    \var{{\devanagarifont \numnoemph\vd \lem \corr, 
विष्टजास्तमस्सात्विकाः \msCa\ \unmetr, 
विष्टजास्तमःसात्विकाः \msCb\msCc\msNa\ \unmetr, 
विष्टजास्तमःसाधिकाः \msNb\ \unmetr, 
\lk\lk \uncl{जा}तमस्साधिकाः \msNc\ \unmetr, 
विष्टजा तमसात्त्विकाः \Ed}}% 

{\devanagarifont दया सत्यं दमः शौचं ज्ञानं मौनं तपः क्षमा \thinspace{\dandab} \dontdisplaylinenum }%
     \var{{\devanagarifont \numemph\vb\textbf{ज्ञानं}\lem \msCa\msCc\msNb\Ed, ज्ञान \msCb\msNc, ज्ञा\uncl{नं} \msNa\oo 
\textbf{मौनं}\lem \mssALL, मौन \msNa\oo 
\textbf{क्षमा}\lem \mssALL, क्षमाः \msCb\msNb}}% 

%Verse 9:29

{\devanagarifont शीलं च नाभिमानं च सात्त्विकाश्चोत्तमा जनाः {॥ ९:२९॥} \veg\dontdisplaylinenum }%
     \var{{\devanagarifont \numnoemph\vc\textbf{शीलं च}\lem \mssALL, नीलञ्च \msNb, शिलं च \Ed\oo 
\textbf{नाभिमानं}\lem \mssALL, नाभिमानां \Ed}}% 

{\devanagarifont कामतृष्णारतिद्यूतमानो युद्धं मदः स्पृहा \thinspace{\dandab} \dontdisplaylinenum }%
     \var{{\devanagarifont \numemph\va\textbf{॰मानो}\lem \mssALL, ॰मनो \msCc}}% 
    \var{{\devanagarifont \numnoemph\vb\textbf{युद्धं}\lem \mssALL, युद्ध॰ \Ed\oo 
\textbf{स्पृहा}\lem \mssALL, स्मृत \msNb}}% 

%Verse 9:30

{\devanagarifont निर्घृणाः कलिकर्तारो राजसेषूत्तमा जनाः {॥ ९:३०॥} \veg\dontdisplaylinenum }%
     \var{{\devanagarifont \numnoemph\vc\textbf{निर्घृणाः}\lem \mssCaCbCc, निर्घृणा \msNa\Ed, निघृणाः \msNb\msNc}}% 
    \var{{\devanagarifont \numnoemph\vd\textbf{राजसेषूत्तमा}\lem \mssALL, 
राजसेसूतमा \msCc, राजसे ह्युत्तमो \Ed}}% 

{\devanagarifont हिंसासूयाघृणामूढनिद्रातन्द्रीभयालसाः \thinspace{\dandab} \dontdisplaylinenum }%
     \var{{\devanagarifont \numemph\va\textbf{॰सूया॰}\lem \mssALL, ॰स\uncl{यू}॰ \msNb\oo 
\textbf{॰मूढ॰}\lem \mssALL, ॰मूढा॰ \msCb\msNb}}% 
    \var{{\devanagarifont \numnoemph\vb\textbf{॰तन्द्री॰}\lem \mssALL, ॰तन्त्री॰ \Ed}}% 

%Verse 9:31

{\devanagarifont क्रोधो मत्सरमायी च तामसेषूत्तमा जनाः {॥ ९:३१॥} \veg\dontdisplaylinenum }%
     \var{{\devanagarifont \numnoemph\vc\textbf{क्रोधो}\lem \mssALL, क्रोध॰ \Ed}}% 
    \var{{\devanagarifont \numnoemph\vd\textbf{तामसेषूत्तमा}\lem \mssALL, 
तामसेसूतमा \msCc, तामसे ह्युत्तमो \Ed}}% 

{\devanagarifont लघुप्रीतिप्रकाशी च ध्यानयोगे सदोत्सुकः \thinspace{\dandab} \dontdisplaylinenum }%
     \var{{\devanagarifont \numemph\vb\textbf{॰योगे}\lem \mssALL, ॰\uncl{योगे} \msCa}}% 

%Verse 9:32

{\devanagarifont प्रज्ञाबुद्धिविरागी च सात्त्विकं गुणलक्षणम् {॥ ९:३२॥} \veg\dontdisplaylinenum }%
     \var{{\devanagarifont \numnoemph\vc\textbf{॰विरागी च}\lem \mssALL, ॰विरागी \msNa, ॰विराङ्क्री च \msNc}}% 

{\devanagarifont बालको निपुणो रागी मानो दर्पश्च लोभकः \thinspace{\dandab} \dontdisplaylinenum }%
     \var{{\devanagarifont \numemph\va\textbf{बालको}\lem \mssALL, चालको \msNc\oo 
\textbf{निपुणो}\lem \Ed, निपुनो \mssCaCbCc\msNa\msNb, निपुणे \msNc}}% 

%Verse 9:33

{\devanagarifont स्पृहा ईर्षा प्रलापी च राजसं गुणलक्षणम् {॥ ९:३३॥} \veg\dontdisplaylinenum }%
     \var{{\devanagarifont \numnoemph\vc\textbf{ईर्षा}\lem \mssALL, ईर्ष्या \msCb\Ed\oo 
\textbf{प्रलापी}\lem \mssALL, च लापी \msCc}}% 
    \var{{\devanagarifont \numnoemph\vd\textbf{राजसं}\lem \mssALL, तामसं \Ed}}% 

{\devanagarifont उद्वेग आलसो मोहः क्रूरस्तस्करनिर्दयः \thinspace{\dandab} \dontdisplaylinenum }%
     \var{{\devanagarifont \numemph\va\textbf{आलसो}\lem \mssALL, अलसो \msCb}}% 
    \var{{\devanagarifont \numnoemph\vb\textbf{क्रूरस्त॰}\lem \msCa\msCb\msNa, क्रूरत॰ \msCc\msNc\Ed, कूरस्त॰ \msNb\oo 
\textbf{॰निर्दयः}\lem \mssALL, ॰निर्दयाः \msNc}}% 

%Verse 9:34

{\devanagarifont क्रोधः पिशुन निद्रा च तामसं गुणलक्षणम् {॥ ९:३४॥} \veg\dontdisplaylinenum }%
     \var{{\devanagarifont \numnoemph\vc\textbf{क्रोधः}\lem \mssALL, क्रोध॰ \msCb\oo 
\textbf{पिशुन}\lem \Ed, पिशुनो \mssCaCbCc\msNa\msNb\msNc\oo 
\textbf{च}\lem \mssALL, \om\ \msNb}}% 
    \var{{\devanagarifont \numnoemph\vd\textbf{गुण॰}\lem \mssALL, गु॰ \msCbacorr}}% 


\alalalfejezet{आहारस्त्रैगुण्ये}

{\devanagarifont विगतराग उवाच {\dandab}\dontdisplaylinenum  }%
 
{\devanagarifont केन चिह्नेन विज्ञेय आहारः सर्वदेहिनाम् \thinspace{\danda} \dontdisplaylinenum }%
     \var{{\devanagarifont \numemph\vab \lem \mssALL, 
\lk\lk\lk\lk\lk\lk\lk\lk\lk\lk\lk\lk\lk  देहिनाम् \msCa, 
केन चिह्नेन विज्ञेय आहार सर्वदेहिनाम् \msNb}}% 

%Verse 9:35

{\devanagarifont त्रैगुण्यस्य पृथक्त्वेन कथयस्व तपोधन {॥ ९:३५॥} \veg\dontdisplaylinenum }%
     \var{{\devanagarifont \numnoemph\vc\textbf{पृथक्त्वेन}\lem \mssALL, पृथक्केण \msNc}}% 
    \var{{\devanagarifont \numnoemph\vd\textbf{॰धन}\lem \mssALL, ॰धनः \msNc}}% 

{\devanagarifont अनर्थयज्ञ उवाच {\dandab}\dontdisplaylinenum  }%
 
{\devanagarifont आयुः कीर्तिः सुखं प्रीतिर्बलारोग्यविवर्धनम् \thinspace{\danda} \dontdisplaylinenum }%
     \var{{\devanagarifont \numemph\va\textbf{कीर्तिः}\lem \mssALL, किर्तिः \Ed\oo 
\textbf{सुखं प्रीतिर्ब॰}\lem \msNc, सुखं प्रीतिब॰ \msCa\msCb\msNa\msNb, 
सुखप्रीति ब॰ \msCc, सुखं प्रितिव॰ \Ed}}% 
    \var{{\devanagarifont \numnoemph\vb\textbf{॰रोग्य॰}\lem \mssALL, ॰रोग्यं \msCb}}% 

%Verse 9:36

{\devanagarifont हृद्यस्वादुरसं स्निग्ध आहारः सात्त्विकप्रियः {॥ ९:३६॥} \veg\dontdisplaylinenum }%
     \var{{\devanagarifont \numnoemph\vc\textbf{हृद्य॰}\lem \mssALL, हृद॰ \Ed\oo 
\textbf{॰रसं}\lem \msCa\msCb\msNa, ॰रस \msCc, ॰\uncl{रस} \msNb, ॰रसां \msNc, ॰रसा \Ed\oo 
\textbf{स्निग्ध}\lem \mssALL, स्निग्धं \msNa, \uncl{सन्दिग्ध} \msNb}}% 
    \var{{\devanagarifont \numnoemph\vd\textbf{आहारः}\lem \msCapcorr\msNb\msNc\Ed, आहार \msCaacorr\msCb\msCc\msNa\oo 
\textbf{सात्त्विकप्रियः}\lem \msCa\msCb\msNa\msNc, 
सात्विकप्रिया \msCc, सात्विकप्रिय \msNb, सात्विकः कियाः \Ed}}% 

{\devanagarifont अत्युष्णमाम्ललवणं रूक्षं तीक्ष्णं विदाहि च \thinspace{\dandab} \dontdisplaylinenum }%
     \var{{\devanagarifont \numemph\va\textbf{॰म्ल॰}\lem \mssALL, ॰ल्ल॰ \Ed\oo 
\textbf{॰लवणं}\lem \mssALL, ॰लक्षणं \msCb}}% 
    \var{{\devanagarifont \numnoemph\vb\textbf{तीक्ष्णं}\lem \mssALL, ती\uncl{क्ष्ण} \msCa, स्तीक्षं \Ed\oo 
\textbf{विदाहि च}\lem \msCb\msNa\msNb\msNc, \lk \uncl{दाहि च} \msCa, 
विदाहिक \msCcpcorr, विदाहिकः \msCcacorr\Ed}}% 

%Verse 9:37

{\devanagarifont राजसश्रेष्ठ-आहारो दुःखशोकामयप्रदः {॥ ९:३७॥} \veg\dontdisplaylinenum }%
     \var{{\devanagarifont \numnoemph\vcd \lem \msCb\msNa\msNc, 
\lk\lk \lk\lk \lk\lk \lk\lk \lk\lk \lk\lk \lk\lk \lk\lk\  \msCa, 
राजसश्रेष्ठ आहारो दुःखशोकामयः प्रदः \msCc, 
राजसः श्रेष्ठ आहारो दुःखशोकामयप्रदः \msNb, 
राजसे श्रेष्ठमाहारो दुःखशोकाभयप्रदः \Ed}}% 

{\devanagarifont अभक्ष्यामेध्यपूती च पूति पर्युषितं च यत् \thinspace{\dandab} \dontdisplaylinenum }%
     \var{{\devanagarifont \numemph\va \lem \eme, अभक्ष्यमेध्यपूती च \mssCaCbCc\msNa, 
अभक्षमेध्यपूती च \msNb, अभक्षामेध्यपूती च \msNc, अभक्षमद्यपूती वै \Ed}}% 

%Verse 9:38

{\devanagarifont आमयारसविस्वाद आहारस्तामसप्रियः {॥ ९:३८॥} \veg\dontdisplaylinenum }%
     \var{{\devanagarifont \numnoemph\vc\textbf{आमया॰}\lem \conj, आयाम॰ \mssCaCbCc\msNa\msNb\msNc, आयास॰ \Ed}}% 
    \var{{\devanagarifont \numnoemph\vd\textbf{॰मस॰}\lem \mssALL, ॰मसः \msCc\Ed\oo 
\textbf{॰प्रियः}\lem \mssALL, ॰प्रियाः \msCc}}% 


\alalalfejezet{गुणातीतम्}

{\devanagarifont विगतराग उवाच {\dandab}\dontdisplaylinenum  }%
 
{\devanagarifont गुणातीतं कथं ज्ञेयं संसारपरपारगम् \thinspace{\danda} \dontdisplaylinenum }%
     \var{{\devanagarifont \numemph\va\textbf{॰तीतं}\lem \mssALL, ॰तीत \msCc\msNb}}% 
    \var{{\devanagarifont \numnoemph\vb\textbf{॰गम्}\lem \mssALL, ॰गः \msCc}}% 

%Verse 9:39

{\devanagarifont गुणपाशनिबद्धानां मोक्षं कथय तत्त्वतः {॥ ९:३९॥} \veg\dontdisplaylinenum }%
     \var{{\devanagarifont \numnoemph\vc\textbf{॰बद्धानां}\lem \mssALL, ॰वर्द्धानां \msCb, ॰बद्धामो \Ed}}% 

{\devanagarifont अनर्थयज्ञ उवाच {\dandab}\dontdisplaylinenum  }%
 
{\devanagarifont आत्मवत्सर्वभूतानि सम्यक्पश्येत भो द्विज \thinspace{\danda} \dontdisplaylinenum }%
     \var{{\devanagarifont \numemph\va\textbf{॰भूतानि}\lem \mssALL, ॰भूतां \msNa}}% 
    \var{{\devanagarifont \numnoemph\vb\textbf{सम्यक्प॰}\lem \mssALL, सम्यत्प॰ \msNa}}% 
    \paral{{\devanagarifontsmall \vab {\englishfont \similar\ \PADMAP\ 1.19.337ab:} 
                         आत्मवत्सर्वभूतानि यः पश्यति स पश्यति }}

%Verse 9:40

{\devanagarifont गुणातीतः स विज्ञेयः संसारपरपारगः {॥ ९:४०॥} \veg\dontdisplaylinenum }%
     \var{{\devanagarifont \numnoemph\vc\textbf{॰तीतः}\lem \msCa\msCb\msNa\msNb, ॰तीत \msCc\msNc, ॰तीतं \Ed}}% 
    \paral{{\devanagarifontsmall \vo {\englishfont \compare\ \BHG\ 6.32:}
                 आत्मौपम्येन सर्वत्र समं पश्यति यो ऽर्जुन\thinspace{\devanagarifontsmall ।}
                 सुखं वा यदि वा दुःखं स योगी परमो मतः\thinspace{\devanagarifontsmall ॥} }}

{\devanagarifont ईर्षाद्वेषसमो यस्तु सुखदुःखसमाश्च ये \thinspace{\dandab} \dontdisplaylinenum }%
     \var{{\devanagarifont \numemph\va\textbf{ईर्षा॰}\lem \mssALL, ईर्ष्या॰ \msNc\Ed}}% 
    \var{{\devanagarifont \numnoemph\vb\textbf{॰समाश्च ये}\lem \mssALL, ॰समाश्रये \msNb}}% 
    \paral{{\devanagarifontsmall \vab {\englishfont \compare\ \VSS\ 11.51ab:}
                     न्यसेद्धर्ममधर्मं च ईर्ष्याद्वेषं परित्यजेत
                     {\englishfont \compare\ \BHG\ 14.25:}
                         मानापमानयोस्तुल्यस्तुल्यो मित्रारिपक्षयोः\thinspace{\devanagarifontsmall ।}
                         सर्वारम्भपरित्यागी गुणातीतः स उच्यते\thinspace{\devanagarifontsmall ॥}
                    {\englishfont \compare\ \BHG\ 12.13:}
                 अद्वेष्टा सर्वभूतानां मैत्रः करुण एव च\thinspace{\devanagarifontsmall ।}
                 निर्ममो निरहंकारः समदुःखसुखः क्षमी\thinspace{\devanagarifontsmall ॥} }}

%Verse 9:41

{\devanagarifont स्तुतिनिन्दासमा ये च गुणातीतः स उच्यते {॥ ९:४१॥} \veg\dontdisplaylinenum }%
     \var{{\devanagarifont \numnoemph\vd\textbf{॰तीतः}\lem \mssALL, ॰तीत \msNb}}% 

{\devanagarifont तुल्यप्रियाप्रियो यश्च अरिमित्रसमस्तथा \thinspace{\dandab} \dontdisplaylinenum }%
     \var{{\devanagarifont \numemph\va\textbf{तुल्य॰}\lem \Ed, तुल्यः \mssCaCbCc\msNa\msNb\msNc}}% 
    \var{{\devanagarifont \numnoemph\vb\textbf{॰सम॰}\lem \mssALL, ॰समा॰ \msCc}}% 

%Verse 9:42

{\devanagarifont मानापमानयोस्तुल्यो गुणातीतः स उच्यते {॥ ९:४२॥} \veg\dontdisplaylinenum  }%
     \paral{{\devanagarifontsmall \vo {\englishfont \compare\ \BHG\ 14.24cd--25:}
                         तुल्यप्रियाप्रियो धीरस्तुल्यनिन्दात्मसंस्तुतिः\thinspace{\devanagarifontsmall ॥}
                         मानावमानयोस्तुल्यस्तुल्यो मित्रारिपक्षयोः\thinspace{\devanagarifontsmall ।}
                         सर्वारम्भपरित्यागी गुणातीतः स उच्यते\thinspace{\devanagarifontsmall ॥} }}

{\devanagarifont एष ते कथितो विप्र गुणसद्भावनिर्णयः \thinspace{\dandab} \dontdisplaylinenum }%
     \var{{\devanagarifont \numemph\va\textbf{ते}\lem \mssALL, तो \msNb}}% 
    \var{{\devanagarifont \numnoemph\vb\textbf{॰सद्भाव॰}\lem \mssALL, ॰मद्भाव॰ \Ed}}% 

%Verse 9:43

{\devanagarifont गुणयुक्तस्तु संसारी गुणातीतः पराङ्गतिः {॥ ९:४३॥} \veg\dontdisplaylinenum }%
     \var{{\devanagarifont \numnoemph\vd\textbf{गुणातीतः}\lem \msCa\msCc\msNa, गुणातीत \msCb\msNb\msNc\Ed\oo 
\textbf{पराङ्गतिः}\lem \Ed, पराङ्गतिम् \mssCaCbCc\msNa\msNb\msNc}}% 

{\devanagarifont 
\jump
\begin{center}
\ketdanda~इति वृषसारसंग्रहे त्रैगुण्यविशेषणीयो नामाध्यायो नवमः~\ketdanda
\end{center}
\dontdisplaylinenum\vers  }%
     \var{{\devanagarifont \numnoemph{\englishfont \Colo:}\textbf{॰विशेषणीयो}\lem \corr, ॰विशेषनीयो \mssCaCbCc\msNa\msNb\msNc\Ed\oo 
\textbf{नामाध्यायो नवमः}\lem \mssALL, नाम नवमो ऽध्यायः \Ed}}% 
