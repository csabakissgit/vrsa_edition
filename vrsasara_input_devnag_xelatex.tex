\fejno=0\versno=0
\centerline{\Huge\devanagarifontbold वृषसारसंग्रहः  }

 \versno=0\fejno=17
\thispagestyle{empty}

\fancyhead[CO]{{\footnotesize\devanagarifont वृषसारसंग्रहे  }}
\fancyhead[CE]{{\footnotesize\devanagarifont सप्तदशमो ऽध्यायः  }}
\fancyhead[LE]{}
\fancyhead[RE]{}
\fancyhead[LO]{}
\fancyhead[RO]{}
\szam\bek

\centerline{\Large\devanagarifontbold [   सप्तदशमो ऽध्यायः  ]}{\vrule depth10pt width0pt} 

\alalfejezet{दानधर्मविशेषः }
 
\vers


{\devanagarifontbold देव्युवाच {\dandab}\dontdisplaylinenum  }%
     \lacuna{\devanagarifont {\englishfont Testimonia for this chapter---\msCa: f.~222r line 2 -- f.~224r line 4; 
                                               \msCb: f.~225r line 3 -- f.~226v line 6;
                                               \msCc: f.;
                                               \msNa: f.;
                                               \msM:  f.;
                                               \msPaperA:;
                                               \Ed: pp.~646--649 (it breaks down after 17.38)} }%
  
{\devanagarifontbold पृथग्दानस्य इच्छामि श्रोतुं मां दातुमर्हसि \thinspace{\danda} \dontdisplaylinenum }%
     \var{{\devanagarifont \numemph\vb \lem \msCa\msCb\msNa; 
माहात्म्यं वक्तुमर्हसि \Ed}}% 

%Verse 17:1

{\devanagarifontbold अन्नवस्त्रहिरण्यानां गोभूमिकनकस्य च {॥१७:१॥} \veg\dontdisplaylinenum }%
     \var{{\devanagarifont \numnoemph\vc \textbf{॰वस्त्र॰}\lem \msCa\msCb\Ed; ॰वस्त्रं \msNa}}% 


\alalalfejezet{अन्नप्रदानम् }
 

{\devanagarifontbold भगवानुवाच {\dandab}\dontdisplaylinenum  }%
 
\nemsloka 
{\devanagarifontbold सुसंस्कृतमन्नमतिप्रदद्याद् }%
  \dontdisplaylinenum

\nemslokab

{\devanagarifontbold घृतप्रभूतमवदंशयुक्तम्  \danda\dontdisplaylinenum }%
     \var{{\devanagarifont \numemph\vb \textbf{॰भूत॰}\lem \msCa\msCb\Ed; ॰सूत॰ \msNa}}% 

\nemslokac

{\devanagarifontbold घृतप्रपक्वं सुकृतं च पूपं }%
  \dontdisplaylinenum    \var{{\devanagarifont \numnoemph\vc \textbf{सुकृतं च पूपं}\lem \msCa\msCb\msNapcorr; सुकृतं पूपं \msNaacorr, सुकृतम्मपूपं \Ed}}% 


\nemslokad

{\devanagarifontbold सितेन खण्डेन गुडेन युक्तम् {॥१७:२॥} \veg\dontdisplaylinenum }%
 
\ujvers\nemsloka {
{\devanagarifontbold मार्गं खगं चोदकजङ्गलं च }%
  \dontdisplaylinenum}    \var{{\devanagarifont \numemph\va \textbf{मार्गं}\lem \msCa\msCb\msNa; मार्ग॰ \Ed\ \unmetr\oo 
\textbf{खगं चो॰}\lem \msCb\msNa; खञ्चो॰\msCa, खगश्चो॰ \Ed\oo 
\textbf{॰जङ्गलं च}\lem \msCa\msCb\msNa; ॰जङ्गमश्च \Ed}}% 


\nemslokab

{\devanagarifontbold दद्याद्वटं नागरवंशमूलम्  \danda\dontdisplaylinenum }%
     \var{{\devanagarifont \numnoemph\vb \textbf{वटं}\lem \msCa\msCb\msNa; वट \Ed\ \unmetr}}% 

\nemslokac

{\devanagarifontbold शाकं फलं चाम्ल मधूरतिक्तं }%
  \dontdisplaylinenum

\nemslokad

{\devanagarifontbold पानं पयः शीतसुगन्धतोयम् {॥१७:३॥} \veg\dontdisplaylinenum }%
 
\ujvers\nemsloka {
{\devanagarifontbold दधि प्रदद्याद्गुडमिश्रितं च }%
  \dontdisplaylinenum}

\nemslokab

{\devanagarifontbold मृणाल शालूक च नालका च  \danda\dontdisplaylinenum }%
     \var{{\devanagarifont \numemph\vb \textbf{॰शालूक च}\lem \conj; ॰शालूक व \msCa\msNa\Ed, ॰क व \msCb}}% 

\nemslokac

{\devanagarifontbold सदक्षिणालेपपवित्रपुष्पं }%
  \dontdisplaylinenum

\nemslokad

{\devanagarifontbold श्रद्धान्वितः सत्कृतया प्रणम्य {॥१७:४॥} \veg\dontdisplaylinenum }%
     \var{{\devanagarifont \numnoemph\vd \textbf{सत्कृतया}\lem \msCa\msCb\msNa; सक्ततया \Ed}}% 

\ujvers\nemsloka {
{\devanagarifontbold प्रयान्ति लोकं जगदीश्वरस्य }%
  \dontdisplaylinenum}    \var{{\devanagarifont \numemph\va \textbf{प्रयान्ति}\lem \msCa\msCb\msNa; प्रयाति \Ed}}% 


\nemslokab

{\devanagarifontbold विमानयानैः सहितो ऽप्सरोभिः  \danda\dontdisplaylinenum }%
 
\nemslokac

{\devanagarifontbold एकैकसिक्थस्य सहस्रवर्षम् }%
  \dontdisplaylinenum    \var{{\devanagarifont \numnoemph\vc \textbf{॰सिक्थस्य}\lem \msCa\msCb\msNa; ॰सिष्टस्य \Ed}}% 


\nemslokad

{\devanagarifontbold अन्नप्रदो मोदति देवलोके {॥१७:५॥} \veg\dontdisplaylinenum }%
 
\ujvers\nemsloka {
{\devanagarifontbold च्युतश्च मर्त्ये स भवेद्धनाढ्यः }%
  \dontdisplaylinenum}

\nemslokab

{\devanagarifontbold कुलोद्गतः सर्वगुणोपपन्नः  \danda\dontdisplaylinenum }%
 
\nemslokac

{\devanagarifontbold यशः श्रियं सर्वकलाज्ञता च }%
  \dontdisplaylinenum    \var{{\devanagarifont \numemph\vc \textbf{॰कला॰}\lem \eme; ॰कल॰ \msCa\msCb\msNa\Ed}}% 


\nemslokad

{\devanagarifontbold भवेत्स भोगी सकलत्रपुत्रः {॥१७:६॥} \veg\dontdisplaylinenum }%
     \var{{\devanagarifont \numnoemph\vd \textbf{॰कलत्रपुत्रः}\lem \msCa\msNa\Ed; ॰कलत्रः \msCb}}% 

\ujvers\nemsloka {
{\devanagarifontbold दद्याद्दरिद्रः कृपणार्तदीना }%
  \dontdisplaylinenum}    \var{{\devanagarifont \numemph\va \textbf{॰रिद्रः}\lem \msCa\Ed; ॰रिद्र \msCb\msNa\oo 
\textbf{॰दीना}\lem \msCa\msCb\msNa; ॰दीनो \Ed}}% 


\nemslokab

{\devanagarifontbold कालागतत्वातुरमागतानाम्  \danda\dontdisplaylinenum }%
     \var{{\devanagarifont \numnoemph\vb \textbf{कालागतत्वा॰}\lem \msCa\msCb\msNa; वालाग दत्वा॰ \Ed}}% 

\nemslokac

{\devanagarifontbold तृष्णाबुभुक्षागतिकागतानाम् }%
  \dontdisplaylinenum    \var{{\devanagarifont \numnoemph\vc \textbf{तृष्णा॰}\lem \msNa\Ed; तृष्णां \msCa\msCb\oo 
\textbf{बुभुक्षा॰}\lem \msCb\msNa\Ed; भुभुक्ता॰ \msCa}}% 


\nemslokad

{\devanagarifontbold दत्त्वा स धर्मफलमाश्रयेत {॥१७:७॥} \veg\dontdisplaylinenum }%
     \var{{\devanagarifont \numnoemph\vd \textbf{॰श्रयेत}\lem \msCa\msNa\Ed; ॰श्रयेत् \msCb}}% 

\ujvers\nemsloka {
{\devanagarifontbold देशे च काले च तथा च पात्रे }%
  \dontdisplaylinenum}    \var{{\devanagarifont \numemph\va \textbf{पात्रे}\lem \msCa\msCb; यात्रे \msNa}}% 


\nemslokab

{\devanagarifontbold दानादिधर्मस्य फलं कनिष्टम्  \danda\dontdisplaylinenum }%
     \var{{\devanagarifont \numnoemph\vb \textbf{दानादि॰}\lem \msCa\msNa; दानानि \msCb}}% 
    \lacuna{\devanagarifont \vab {\englishfont missing in \Ed} }%
  
\nemslokac

{\devanagarifontbold वाणिज्यधर्मा हि फलाश्रितानां }%
  \dontdisplaylinenum    \var{{\devanagarifont \numnoemph\vc \textbf{वाणिज्य॰}\lem \msCa\msCb\Ed; वाणि \msNaacorr, वणिज्यं \msNapcorr\oo 
\textbf{॰धर्मा हि॰}\lem \msCa\msCb\msNa; ॰धर्मादि \Ed}}% 


\nemslokad

{\devanagarifontbold धर्मो हि तस्य न च निर्मलो ऽस्ति {॥१७:८॥} \veg\dontdisplaylinenum }%
     \var{{\devanagarifont \numnoemph\vd \textbf{हि}\lem \msCa\msCb\Ed; स्ति \msNa}}% 

\ujvers\nemsloka {
{\devanagarifontbold तोयं च दद्याल्लघुपूर्णकुम्भं }%
  \dontdisplaylinenum}

\nemslokab

{\devanagarifontbold शीतं सुगन्धं परिवासितं च  \danda\dontdisplaylinenum }%
 
\nemslokac

{\devanagarifontbold स याति लोकं सलिलेश्वरस्य }%
  \dontdisplaylinenum

\nemslokad

{\devanagarifontbold न सप्तजन्मानि तृषाभिभूतः {॥१७:९॥} \veg\dontdisplaylinenum }%
     \var{{\devanagarifont \numemph\vd \textbf{सप्त॰}\lem \msCa\msCb\msNa; तस्य \Ed}}% 


\alalfejezet{वस्त्रादिप्रदानम् }
 
\ujvers\nemsloka {
{\devanagarifontbold उपानहं यो ददति द्विजाय }%
  \dontdisplaylinenum}    \var{{\devanagarifont \numemph\va \textbf{यो}\lem \msCa\msNa\Ed; ये \msCb}}% 


\nemslokab

{\devanagarifontbold सुशोभनं तैलसुदीपितं च  \danda\dontdisplaylinenum }%
     \var{{\devanagarifont \numnoemph\vb \textbf{॰दीपितं च}\lem \msCa\msCb\msNapcorr; ॰दीपितं \msNaacorr, 
॰दीसुरपितञ्च \Ed\ \hypermetr}}% 

\nemslokac

{\devanagarifontbold ते यान्ति लोकममराधिपस्य }%
  \dontdisplaylinenum    \var{{\devanagarifont \numnoemph\vc \textbf{लोकममरा॰}\lem \msCa\msCb\Ed; लोकं समरा॰ \msNa}}% 


\nemslokad

{\devanagarifontbold यमालयं कष्टपथा न यान्ति {॥१७:१०॥} \veg\dontdisplaylinenum }%
 
\ujvers\nemsloka {
{\devanagarifontbold प्रक्षीणपुण्यः पुनरत्र लोके }%
  \dontdisplaylinenum}    \var{{\devanagarifont \numemph\va \textbf{॰पुण्यः}\lem \msCa; \om\ \msNaacorr, ॰पुण्य \msNapcorr, ॰पुण्या \Ed\oo 
\textbf{पुनरत्र लोके}\lem \msCa\msNa\Ed; पुनरभ्युपेति \msCb}}% 


\nemslokab

{\devanagarifontbold जातो भवेद्दिव्यकुलोपपन्नः  \danda\dontdisplaylinenum }%
 
\nemslokac

{\devanagarifontbold धनैः समृद्धो ऽधिपतित्वतां च }%
  \dontdisplaylinenum    \var{{\devanagarifont \numnoemph\vc \textbf{॰पतित्वतां च}\lem \msCa\msCb\msNa; ॰पतित्वताश्च \Ed}}% 


\nemslokad

{\devanagarifontbold रथाश्वनागासनगा भवन्ति {॥१७:११॥} \veg\dontdisplaylinenum }%
     \var{{\devanagarifont \numnoemph\vd \lem \msCa\msCb\msNa; 
रथाश्च नागा प्रभवन्ति तस्य \Ed}}% 

\ujvers\nemsloka {
{\devanagarifontbold वस्त्रप्रदानेन भवन्ति देवि }%
  \dontdisplaylinenum}

\nemslokab

{\devanagarifontbold रूपोत्तमाः सर्वकलाज्ञताश्च  \danda\dontdisplaylinenum }%
     \var{{\devanagarifont \numemph\vb \textbf{॰त्तमाः}\lem \msCa\msNa; ॰त्तमा \msCb, ॰त्तम॰ \Ed\oo 
\textbf{॰कलाज्ञता च}\lem \eme; ॰कलज्ञताश्च \msCa\msNa; ॰कलज्ञता च \msCb, ॰कलज्ञताञ्च \Ed}}% 

\nemslokac

{\devanagarifontbold समृद्धिसौभाग्यगुणान्विताश्च }%
  \dontdisplaylinenum

\nemslokad

{\devanagarifontbold स्वर्गच्युतास्ते पुरुषा भवन्ति {॥१७:१२॥} \veg\dontdisplaylinenum }%
 
\ujvers\nemsloka {
{\devanagarifontbold वस्त्रप्रदानाभिरतस्य पुंसः }%
  \dontdisplaylinenum}

\nemslokab

{\devanagarifontbold अन्यां प्रवक्ष्यामि ततः प्रशंसाम्  \danda\dontdisplaylinenum }%
     \var{{\devanagarifont \numemph\vb \textbf{अन्यां प्र॰}\lem \msCa\msNa; अन्यत्प्र॰ \msCb\Ed\oo 
\textbf{॰शंसाम्}\lem \msCa\msNa; ॰शसाम् \msCb, ॰शस्तां \Ed}}% 

\nemslokac

{\devanagarifontbold वस्त्रं तु लोकेष्वभिपूजनीयं }%
  \dontdisplaylinenum    \var{{\devanagarifont \numnoemph\vc \textbf{॰भिपूज॰}\lem \msCa\msCb\msNa; ॰तिपूज॰ \Ed}}% 


\nemslokad

{\devanagarifontbold वस्त्रं नराणां त्वतिमाननीयम् {॥१७:१३॥} \veg\dontdisplaylinenum }%
 
\ujvers\nemsloka {
{\devanagarifontbold वस्त्रं तु भूयो न च मानलाभः }%
  \dontdisplaylinenum}

\nemslokab

{\devanagarifontbold पराभवश्चातिजुगुप्सनं च  \danda\dontdisplaylinenum }%
     \var{{\devanagarifont \numemph\vb \textbf{॰जुगुप्सनं च}\lem \msCa\msCb; ॰जुप्सितं च \msNa, जुगुप्सनश्च \Ed}}% 

\nemslokac

{\devanagarifontbold तस्माद्धि वस्त्रं सततं प्रदेयं }%
  \dontdisplaylinenum

\nemslokad

{\devanagarifontbold यशः श्रियः स्वर्गमनन्तलाभम् {॥१७:१४॥} \veg\dontdisplaylinenum }%
     \var{{\devanagarifont \numnoemph\vd \textbf{श्रियः}\lem \msCa\msCb\Ed; श्रियंः \msNa\oo 
\textbf{स्वर्गमनन्त॰}\lem \msCa\msCb\msNa; ॰स्वर्गसमन्तआभं \Ed}}% 

\ujvers\nemsloka {
{\devanagarifontbold यावन्ति सूत्राणि भवन्ति वस्त्रे }%
  \dontdisplaylinenum}

\nemslokab

{\devanagarifontbold तावद्युगं गच्छति सोमलोकम्  \danda\dontdisplaylinenum }%
     \var{{\devanagarifont \numemph\vb \textbf{तावद्यु॰}\lem \msCa\msNa\Ed; ताव यु॰ \msCb\oo 
\textbf{गच्छति}\lem \msCapcorr\msCb\msNa\Ed; गति \msCaacorr}}% 

\nemslokac

{\devanagarifontbold पुण्यक्षयाज्जायति मर्त्यलोके }%
  \dontdisplaylinenum    \var{{\devanagarifont \numnoemph\vc \textbf{मर्त्य॰}\lem \msCa\msCb; मार्त्य॰ \msNa, मृत्यु॰ \Ed}}% 


\nemslokad

{\devanagarifontbold वस्त्रप्रभूते धनधान्यकीर्णे {॥१७:१५॥} \veg\dontdisplaylinenum }%
     \var{{\devanagarifont \numnoemph\vd \textbf{॰कीर्णे}\lem \msCa\msCb\msNa; ॰कीर्णो \Ed}}% 

\ujvers\nemsloka {
{\devanagarifontbold सुरूपसौभाग्ययशशिवनश्च }%
  \dontdisplaylinenum}

\nemslokad

{\devanagarifontbold विद्याधरो लोकप्रभुत्वताश्च {॥१७:१६॥} \veg\dontdisplaylinenum }%
 
\ujvers\nemsloka {
{\devanagarifontbold द्विजेभ्यच्छत्रं सुकृतं प्रदद्यात् }%
  \dontdisplaylinenum}

\nemslokab

{\devanagarifontbold वर्षातपत्रं दृढशोभनं च  \danda\dontdisplaylinenum }%
 
\nemslokac

{\devanagarifontbold अङ्गारवर्षत्रषु खड्गमाद्यम् }%
  \dontdisplaylinenum

\nemslokad

{\devanagarifontbold असंशयं त्रायति याम्यमार्गे {॥१७:१७॥} \veg\dontdisplaylinenum }%
 
\ujvers\nemsloka {
{\devanagarifontbold स्वर्गं च यान्ति ग्रहनायकश्च }%
  \dontdisplaylinenum}

\nemslokab

{\devanagarifontbold स वर्षकोट्यायुतमन्तकाले  \danda\dontdisplaylinenum }%
 
\nemslokac

{\devanagarifontbold जायन्ति ते मानुषमर्त्यलोके }%
  \dontdisplaylinenum

\nemslokad

{\devanagarifontbold गृहोत्तमे भोगपतिर्भवन्ति {॥१७:१८॥} \veg\dontdisplaylinenum }%
 
\ujvers\nemsloka {
{\devanagarifontbold कृत्वा मठं शोभनविप्रदाता }%
  \dontdisplaylinenum}

\nemslokab

{\devanagarifontbold द्रव्येण शुद्धेन तु पूजयित्वा  \danda\dontdisplaylinenum }%
 
\nemslokac

{\devanagarifontbold स याति देवेन्द्रसदं यथेष्टम् }%
  \dontdisplaylinenum

\nemslokad

{\devanagarifontbold सवर्षकोटिशतदिव्यसंख्यैः {॥१७:१९॥} \veg\dontdisplaylinenum }%
 
\ujvers\nemsloka {
{\devanagarifontbold तदन्तकाले यदि मानुषत्वम् }%
  \dontdisplaylinenum}

\nemslokab

{\devanagarifontbold जायन्ति ते सप्तमहीप्रभोक्ता  \danda\dontdisplaylinenum }%
 
\nemslokac

{\devanagarifontbold स सप्तरथ्यत्रयसम्प्रयुक्ता }%
  \dontdisplaylinenum

\nemslokad

{\devanagarifontbold बलाधिको यज्ञसहस्रकर्ता {॥१७:२०॥} \veg\dontdisplaylinenum }%
 

\alalfejezet{भूमिप्रदानम् }
 
\ujvers\nemsloka {
{\devanagarifontbold भूमिप्रदाता द्विजहीनदीनम् }%
  \dontdisplaylinenum}

\nemslokab

{\devanagarifontbold संमृद्धसस्यो जलसंनिकृष्त  \danda\dontdisplaylinenum }%
 
\nemslokac

{\devanagarifontbold स याति लोकममराधिपस्य ! }%
  \dontdisplaylinenum

\nemslokad

{\devanagarifontbold विमानयानेन मनोहरेण {॥१७:२१॥} \veg\dontdisplaylinenum }%
 
\ujvers\nemsloka {
{\devanagarifontbold मन्वन्तरं यावदभुक्तभोगान् }%
  \dontdisplaylinenum}

\nemslokab

{\devanagarifontbold तदन्तकाले च्युतमर्त्यलोके  \danda\dontdisplaylinenum }%
 
\nemslokac

{\devanagarifontbold स जवमुखण्डाधिपतिर्भवेत् }%
  \dontdisplaylinenum

\nemslokad

{\devanagarifontbold वीर्यान्वितो राजसहस्रनाथः {॥१७:२२॥} \veg\dontdisplaylinenum }%
 
\ujvers\nemsloka {
{\devanagarifontbold स चैलघण्टां कनकाग्रशृङ्गाम् }%
  \dontdisplaylinenum}

\nemslokab

{\devanagarifontbold दोग्धीं सवत्सां पयसां द्विजानाम्  \danda\dontdisplaylinenum }%
 
\nemslokac

{\devanagarifontbold दत्त्वा द्विजेभ्यः समलङ्कृतानाम् }%
  \dontdisplaylinenum

\nemslokad

{\devanagarifontbold प्रयान्ति लोकं सुरभीसुतानाम् {॥१७:२३॥} \veg\dontdisplaylinenum }%
 
\ujvers\nemsloka {
{\devanagarifontbold यावन्ति रोमाणि भवन्ति गावः }%
  \dontdisplaylinenum}    \var{{\devanagarifont \numemph\va \textbf{यावन्ति}\lem \Ed; प्रयान्ति \msCa}}% 


\nemslokab

{\devanagarifontbold तावद्युगानामनुभूयभोगान्  \danda\dontdisplaylinenum }%
 
\nemslokac

{\devanagarifontbold तस्माच्च्युता मर्त्यमहीभुजास्ते }%
  \dontdisplaylinenum

\nemslokad

{\devanagarifontbold सहस्रराजानुगतो महात्मा {॥१७:२४॥} \veg\dontdisplaylinenum }%
 
\ujvers\nemsloka {
{\devanagarifontbold सुवर्णकांस्यायसरौप्यदाता }%
  \dontdisplaylinenum}

\nemslokab

{\devanagarifontbold ताम्रप्रवालामणिमौक्तिकाद्यान्  \danda\dontdisplaylinenum }%
 
\nemslokac

{\devanagarifontbold दत्त्वा द्विजेभ्यो वसुसाध्यलोके }%
  \dontdisplaylinenum

\nemslokad

{\devanagarifontbold प्राप्नोति वर्षं दशपञ्चकोट्यो !  {॥१७:२५॥} \veg\dontdisplaylinenum }%
 
\ujvers\nemsloka {
{\devanagarifontbold भुक्त्वा यथेष्टं क्रमदेवलोकान् }%
  \dontdisplaylinenum}

\nemslokab

{\devanagarifontbold च्युतं च मर्त्ये स भवेन्नरेन्द्रः  \danda\dontdisplaylinenum }%
 
\nemslokac

{\devanagarifontbold सुदुर्जयः शक्रसहस्रजेता }%
  \dontdisplaylinenum

\nemslokad

{\devanagarifontbold सुदीर्घमायुश्च पराक्रमश्च {॥१७:२६॥} \veg\dontdisplaylinenum }%
 
\ujvers\nemsloka {
{\devanagarifontbold यत्प्रेक्षणं दर्शयितुं प्रदाता }%
  \dontdisplaylinenum}

\nemslokab

{\devanagarifontbold सुरूपसौभाग्य फलं लभेत  \danda\dontdisplaylinenum }%
 
\nemslokac

{\devanagarifontbold तृणाशनामूलफलाशनेन }%
  \dontdisplaylinenum

\nemslokad

{\devanagarifontbold लभेत राज्यानि कण्टकानि {॥१७:२७॥} \veg\dontdisplaylinenum }%
 
\ujvers\nemsloka {
{\devanagarifontbold लभेत पर्णाशनस्वर्गवासम् }%
  \dontdisplaylinenum}

\nemslokab

{\devanagarifontbold पयः प्रयोगेन च देवलोके  \danda\dontdisplaylinenum }%
 
\nemslokac

{\devanagarifontbold शुश्रूषणो यो गुरवे च नित्यम् }%
  \dontdisplaylinenum

\nemslokad

{\devanagarifontbold विद्याधरो जायति मर्त्यलोके {॥१७:२८॥} \veg\dontdisplaylinenum }%
 
\ujvers\nemsloka {
{\devanagarifontbold दद्याद्गवां धासतृणस्य मुष्टिः }%
  \dontdisplaylinenum}

\nemslokab

{\devanagarifontbold गवाढ्यतां जायति मर्त्यलोके  \danda\dontdisplaylinenum }%
 
\nemslokac

{\devanagarifontbold श्राद्धं च दत्त्वा प्रयतो द्विजाय }%
  \dontdisplaylinenum

\nemslokad

{\devanagarifontbold समृद्धसन्तान भवेद्युगान्ते {॥१७:२९॥} \veg\dontdisplaylinenum }%
 
\ujvers\nemsloka {
{\devanagarifontbold अहिंसको जायति दीर्घमायुः }%
  \dontdisplaylinenum}

\nemslokab

{\devanagarifontbold कुलोत्तमं जायति दीक्षितेन  \danda\dontdisplaylinenum }%
 
\nemslokac

{\devanagarifontbold कालत्रयं स्नानकृतेन राज्यं }%
  \dontdisplaylinenum

\nemslokad

{\devanagarifontbold पीत्वा च वायुस्त्रिदशाधिपत्वम् {॥१७:३०॥} \veg\dontdisplaylinenum }%
 
\ujvers\nemsloka {
{\devanagarifontbold अनश्नतायाः फलमीशलोके }%
  \dontdisplaylinenum}

\nemslokab

{\devanagarifontbold तृप्तिर्भवेत्तोयप्रदानशीलः  \danda\dontdisplaylinenum }%
 
\nemslokac

{\devanagarifontbold अन्नप्रदाता पुरुषः समृद्धः }%
  \dontdisplaylinenum

\nemslokad

{\devanagarifontbold स सर्वकामा लभतीह लोके {॥१७:३१॥} \veg\dontdisplaylinenum }%
 
\ujvers\nemsloka {
{\devanagarifontbold श्रद्धामतिर्यः प्रविशेद्धुतासनं ! }%
  \dontdisplaylinenum}

\nemslokab

{\devanagarifontbold स याति लोकं प्रपितामहस्य  \danda\dontdisplaylinenum }%
 
\nemslokac

{\devanagarifontbold सत्यं वदेद्यो ऽपि च धर्मशीलो }%
  \dontdisplaylinenum

\nemslokad

{\devanagarifontbold मोदत्यसौ देवि सहाप्सरोभिः {॥१७:३२॥} \veg\dontdisplaylinenum }%
 
\ujvers\nemsloka {
{\devanagarifontbold रसास्तु षड्यो परिवर्जयन्ति }%
  \dontdisplaylinenum}

\nemslokab

{\devanagarifontbold अतीव सौभाग्य लभेत साध्वी  \danda\dontdisplaylinenum }%
 
\nemslokac

{\devanagarifontbold दानेन भोगानतुल्यं लभेत }%
  \dontdisplaylinenum

\nemslokad

{\devanagarifontbold चिरायुतां याति हि ब्रह्मचर्यात् {॥१७:३३॥} \veg\dontdisplaylinenum }%
 
\ujvers\nemsloka {
{\devanagarifontbold धनाढ्यतां यान्ति हि पुण्यकर्मान् }%
  \dontdisplaylinenum}

\nemslokab

{\devanagarifontbold मौनेन - आज्ञा लभते अलङ्घ्याम्  \danda\dontdisplaylinenum }%
 
\nemslokac

{\devanagarifontbold प्राप्नोति कामं तपसः सुतप्तं }%
  \dontdisplaylinenum

\nemslokab

{\devanagarifontbold कीर्तिर्यशः स्वर्गमनन्तभोगम्  \danda\dontdisplaylinenum }%
 
\nemslokae

{\devanagarifontbold आयुः श्रियारोग्यधनप्रभुत्वं }%
  \dontdisplaylinenum

\nemslokad

{\devanagarifontbold ज्ञानादिलाभं तपसा लभेत {॥१७:३४॥} \veg\dontdisplaylinenum }%
 
\ujvers\nemsloka {
{\devanagarifontbold त्रैलोक्याधिपतित्वशक्रमगमत्कृत्वा तपो दुष्करम् }%
  \dontdisplaylinenum}

\nemslokab

{\devanagarifontbold यक्षेशो ऽपि तपः प्रभावगुरुणा गुह्याधिपत्वं महत्  \danda\dontdisplaylinenum }%
 
\nemslokac

{\devanagarifontbold रक्षेशो ऽपि बिभीषणस्त्वमरतां प्राप्तस्तपस्यैव तु }%
  \dontdisplaylinenum

\nemslokad

{\devanagarifontbold रुद्राराधनतत्परास्तपफलात् नन्दीगणत्वं गतः {॥१७:३५॥} \veg\dontdisplaylinenum }%
 
\ujvers\nemsloka {
{\devanagarifontbold ज्ञानं द्विजान्तपसो आह विष्णुः }%
  \dontdisplaylinenum}

\nemslokab

{\devanagarifontbold क्षत्रं तपो रक्षणमाह सूर्य  \danda\dontdisplaylinenum }%
 
\nemslokac

{\devanagarifontbold वैश्यं तपश्चाञ्जनमाह वायुः }%
  \dontdisplaylinenum

\nemslokad

{\devanagarifontbold शूद्रं हि शिल्पं तप आह इन्द्रः {॥१७:३६॥} \veg\dontdisplaylinenum }%
 
\ujvers\nemsloka {
{\devanagarifontbold रणोत्सहं क्षत्रिययज्ञमिष्टं }%
  \dontdisplaylinenum}

\nemslokab

{\devanagarifontbold वैश्यं हविर्यज्ञमुदाहरन्ति  \danda\dontdisplaylinenum }%
 
\nemslokac

{\devanagarifontbold शूद्रस्य यज्ञः परिचर्यमिष्टं }%
  \dontdisplaylinenum

\nemslokad

{\devanagarifontbold यज्ञं द्विजानां जपमुक्तमोक्षम् {॥१७:३७॥} \veg\dontdisplaylinenum }%
 

\alalfejezet{स्वमांसरुधिरदानम् }
 
\vers


{\devanagarifontbold देव्युवाच {\dandab}\dontdisplaylinenum  }%
 
{\devanagarifontbold स्वमांसरुधिरं दानं दानं पुत्रकलत्रयोः \thinspace{\danda} \dontdisplaylinenum }%
 
%Verse 17:38

{\devanagarifontbold किं प्रशस्यं महादेव तत्त्वं वक्तुमिहार्हसि {॥१७:३८॥} \veg\dontdisplaylinenum }%
     \lacuna{\devanagarifont \vo {\englishfont \Ed\ breaks down after 17.38, and resumes only at 18.16c.} }%
  
{\devanagarifontbold महेश्वर उवाच {\dandab}\dontdisplaylinenum  }%
 
{\devanagarifontbold स्वमांसरुधिरं दानं प्रशंसन्ति मनीषिणः \thinspace{\danda} \dontdisplaylinenum }%
 
%Verse 17:39

{\devanagarifontbold श्रूयतां पूर्ववृत्तानि संक्षिप्य कथयाम्यहम् {॥१७:३९॥} \veg\dontdisplaylinenum }%
 
{\devanagarifontbold उशीनरस्तु राजर्षिः कयो ?तार्थे स्वकान्तन्तु?  \thinspace{\dandab} \dontdisplaylinenum }%
 
%Verse 17:40

{\devanagarifontbold त्यक्त्वा स्वर्गमनुप्राप्तः परार्थे परतत्परः {॥१७:४०॥} \veg\dontdisplaylinenum }%
 
{\devanagarifontbold पुत्रमांसं स्वयं छित्वा अग्निदत्तं पुरानघे \thinspace{\dandab} \dontdisplaylinenum }%
 
%Verse 17:41

{\devanagarifontbold तेन दानप्रभावेन अलर्कस्त्रिदिवं गतः {॥१७:४१॥} \veg\dontdisplaylinenum }%
 
\ujvers\nemsloka {
{\devanagarifontbold स्वदानदानेन मुदा स पुत्र }%
  \dontdisplaylinenum}

\nemslokab

{\devanagarifontbold अपुत्रभूतस्य च पुत्र जातः  \danda\dontdisplaylinenum }%
 
\nemslokac

{\devanagarifontbold स्वर्गे स्वयं चोक्वय भोगलाभं }%
  \dontdisplaylinenum

\nemslokad

{\devanagarifontbold प्राप्तो महद्दानय?ल प्रभावात् {॥१७:४२॥} \veg\dontdisplaylinenum }%
 
\vers


%Verse 17:42

{\devanagarifontbold यादवश् चार्जनो देवि दत्त्वा खण्डवभाजनम् {॥१७:४२॥} \veg\dontdisplaylinenum }%
 
{\devanagarifontbold तपनस्य प्रसादेन सप्तद्वीपेश्वरो भवेत् \thinspace{\dandab} \dontdisplaylinenum }%
 
%Verse 17:43

{\devanagarifontbold हरिणा च शिरो भित्वा दत्तं मे रुधिरं पुरा {॥१७:४३॥} \veg\dontdisplaylinenum }%
 
{\devanagarifontbold प्रतीच्छितं कपालेन ब्रह्मसम्भवजेन मे \thinspace{\dandab} \dontdisplaylinenum }%
 
%Verse 17:44

{\devanagarifontbold दिव्यवर्षसहस्राणि धारा तस्य न छिद्यते {॥१७:४४॥} \veg\dontdisplaylinenum }%
 
{\devanagarifontbold परितुष्टो ऽस्मि तेनाहं कर्मणानेन सुन्दरि \thinspace{\dandab} \dontdisplaylinenum }%
 
%Verse 17:45

{\devanagarifontbold वरं दत्तं मया देवि पुराणपुरुषो ऽव्ययः {॥१७:४५॥} \veg\dontdisplaylinenum }%
 
{\devanagarifontbold अक्षयं वलमूर्जं च अजरामरमेव च \thinspace{\dandab} \dontdisplaylinenum }%
 
%Verse 17:46

{\devanagarifontbold ममाधिकं भवेद्विष्णुर्माम यित्वम् विजेष्यसि {॥१७:४६॥} \veg\dontdisplaylinenum }%
 
{\devanagarifontbold एवमादीन्यनेकानि मयोक्तानि जनार्दने \thinspace{\dandab} \dontdisplaylinenum }%
 
%Verse 17:47

{\devanagarifontbold निष्कम्प निश्चलमनः स्थाणुभूत इव स्थितः {॥१७:४७॥} \veg\dontdisplaylinenum }%
 
{\devanagarifontbold द?चिः स्वतनुं दत्त्वा विबुधानां वरानने \thinspace{\dandab} \dontdisplaylinenum }%
 
%Verse 17:48

{\devanagarifontbold भुक्त्वा लोकान् क्रमात्सर्वान् शिवलोके प्रतिष्ठितः {॥१७:४८॥} \veg\dontdisplaylinenum }%
 
{\devanagarifontbold जामदग्निर्महीं दत्त्वा काश्यपाय महात्मने \thinspace{\dandab} \dontdisplaylinenum }%
 
%Verse 17:49

{\devanagarifontbold इहैव स यालं भोक्ता देवराज्यमवाप्स्यति {॥१७:४९॥} \veg\dontdisplaylinenum  }%
 
{\devanagarifontbold दत्त्वा गो सकलं देवि व्यासस्यामिततेजसः \thinspace{\dandab} \dontdisplaylinenum }%
 
%Verse 17:50

{\devanagarifontbold युधिष्ठिर महीयास देहस्त्रिदिवद्भतः {॥१७:५०॥} \veg\dontdisplaylinenum }%
 
{\devanagarifontbold सत्यनामः ? (भीमः?) स्वकं भर्ता दत्त्वा नारादसत्कृतम् \thinspace{\dandab} \dontdisplaylinenum }%
 
%Verse 17:51

{\devanagarifontbold दानस्यास्य प्रभावेन अक्षयं त्रिदिवद्भतः ? {॥१७:५१॥} \veg\dontdisplaylinenum }%
 
{\devanagarifontbold चतुःषष्ठिसहस्ताणि गवां दत्त्वा द्विजन्मने \thinspace{\dandab} \dontdisplaylinenum }%
 
%Verse 17:52

{\devanagarifontbold दुर्योधनमहीया?ओ गतः स्वर्गमनन्तकम् {॥१७:५२॥} \veg\dontdisplaylinenum }%
 
{\devanagarifontbold वासुकिस्सर्पराजेन्द्रो दत्त्वा विप्रसुसंस्कृतम् \thinspace{\dandab} \dontdisplaylinenum }%
 
%Verse 17:53

{\devanagarifontbold रत्कारुश्च ? साभान्या सर्वे नागविमोक्षिताः {॥१७:५३॥} \veg\dontdisplaylinenum }%
 
{\devanagarifontbold गोभूमिकनकादीनां दानं कन्यसमुच्यते \thinspace{\dandab} \dontdisplaylinenum }%
 
%Verse 17:54

{\devanagarifontbold भृत्यपुत्रकलत्राणां दानं मध्यममुच्यते {॥१७:५४॥} \veg\dontdisplaylinenum }%
 
{\devanagarifontbold स्वदेहं पिसितादीनां दानमुत्तममुच्यते \thinspace{\dandab} \dontdisplaylinenum }%
 
%Verse 17:55

{\devanagarifontbold एतत्सर्वं यदा दानं तद्दानमुत्तमोत्तमम् {॥१७:५५॥} \veg\dontdisplaylinenum }%
     \var{{\devanagarifont \numemph\vd \textbf{॰ओत्तमम्}\lem \msCapcorr; ॰ओत्त \msCaacorr}}% 

{\devanagarifontbold जावज्जन्मसहस्राणि भोक्ता भवति कन्यसः \thinspace{\dandab} \dontdisplaylinenum }%
 
%Verse 17:56

{\devanagarifontbold शतजन्मसहस्राणि भोक्ता भवति मध्यमः {॥१७:५६॥} \veg\dontdisplaylinenum }%
 
{\devanagarifontbold उत्तमः पलभोक्ता (फल?) वि ? जन्मकोटिशतत्रयम् \thinspace{\dandab} \dontdisplaylinenum  }%
 
%Verse 17:57

{\devanagarifontbold परार्धद्वयजन्मानां भोक्ता वै चोत्तमोत्तमः {॥१७:५७॥} \veg\dontdisplaylinenum }%
 
{\devanagarifontbold भूतानामनुकम्पया यदि धनं दाता सदान्वर्षिने \thinspace{\dandab} \dontdisplaylinenum }%
 
%Verse 17:58

{\devanagarifontbold दीनान्वकृयणेष्वनाथमलिनेश्वानादिनि?? च {॥१७:५८॥} \veg\dontdisplaylinenum }%
 
{\devanagarifontbold यद्येव कुरुते सदार्तिहरणं श्रद्धान्वितौ भक्तिमान् \thinspace{\dandab} \dontdisplaylinenum }%
 
%Verse 17:59

{\devanagarifontbold तस्यानन्तयालं वदन्ति विबुधांस् स यस्य सन्दर्शनात् {॥१७:५९॥} \veg\dontdisplaylinenum }%
 
\vers


{\devanagarifontbold 
\jump
\begin{center}
\ketdanda\ इति वृषसारसंग्रहे दानधर्मविशेषं नाम सप्तादशमो ऽध्यायः \ketdanda
\end{center}
\dontdisplaylinenum\vers  }%
 