\fejno=0\versno=0
\centerline{\Huge\devanagarifontbold वृषसारसंग्रहः  }

 
{\vrule depth10pt width0pt}
\versno=0\fejno=6
\thispagestyle{empty}

\centerline{\Large\devanagarifontbold [   षष्ठो ऽध्यायः  ]}{\vrule depth10pt width0pt} \fancyhead[CO]{{\footnotesize\devanagarifont वृषसारसंग्रहे  }}
\fancyhead[CE]{{\footnotesize\devanagarifont षष्ठो ऽध्यायः  }}
\fancyhead[LE]{}
\fancyhead[RE]{}
\fancyhead[LO]{}
\fancyhead[RO]{}
\szam\bek


\nemslokanormal



\alalfejezet{नियमेष्विज्या (२)}
\vers


{\devanagarifont अथ पञ्चविधामिज्यां प्रवक्ष्यामि द्विजोत्तम \thinspace{\dandab} \dontdisplaylinenum }%
     \var{{\devanagarifont \numemph\va\textbf{॰मिज्यां}\lem \msCb, ॰मीज्यां \msCa\msCc\msNa\msNb\msNc\Ed}}% 
    \var{{\devanagarifont \numnoemph\vb\textbf{॰त्तम}\lem \mssCaCbCc\msNa\Ed, ॰त्तमः \msNb\msNc}}% 
    \lacuna{\devanagarifontsmall {\englishfont Witnesses used for this chapter: \msCa\ ff.\thinspace 202r--203r, 
                                              \msCb\ ff.\thinspace 209r--209v, 
                                              \msCc\ ff.\thinspace 278r--279r,
                                              \msNa\ ff.\thinspace 9v--10v, 
                                              \msNb\ exp.\thinspace 51 (lower--upper) -- 52 (lower),
                                              \msNc\ ff.\thinspace 218r--218v,
                                              \Ed\ pp.\thinspace 599--601;  
                                              \mssCaCbCc\ = \msCa + \msCb + \msCc} }%
  
%Verse 6:1

{\devanagarifont धर्ममोक्षप्रसिद्ध्यर्थं शृणुष्वावहितो द्विज {॥ ६:१॥} \veg\dontdisplaylinenum }%
     \var{{\devanagarifont \numnoemph\vc\textbf{॰मोक्षप्रसिद्ध्यर्थं}\lem \mssCaCbCc\msNc, ॰मोक्षप्रसिद्ध्यर्थ \msNa\msNb, 
॰मोक्षेशसिद्ध्यअर्थं \Ed}}% 
    \var{{\devanagarifont \numnoemph\vd\textbf{द्विज}\lem \mssCaCbCc\msNa\msNb\msNc, भव \Ed}}% 

{\devanagarifont अर्थयज्ञः क्रियायज्ञो जपयज्ञस्तथैव च \thinspace{\dandab} \dontdisplaylinenum }%
     \var{{\devanagarifont \numemph\va\textbf{अर्थयज्ञः}\lem \msCa\msCc\msNa, अनर्थयज्ञः \msCb, अर्थयज्ञ \msNb\msNc, अर्थयज्ञ॰ \Ed}}% 

%Verse 6:2

{\devanagarifont ज्ञानं ध्यानं च पञ्चैतत्प्रवक्ष्यामि पृथक्पृथक् {॥ ६:२॥} \veg\dontdisplaylinenum }%
     \var{{\devanagarifont \numnoemph\vc\textbf{ज्ञानं}\lem \msCa\msCb\msNa\msNb\Ed, ज्ञान \msCc\msNc}}% 


\alalalfejezet{अर्थयज्ञः}

{\devanagarifont अग्न्युपासनकर्मादि अग्निहोत्रक्रतुक्रिया \thinspace{\dandab} \dontdisplaylinenum }%
     \var{{\devanagarifont \numemph\vb\textbf{अग्नि॰}\lem \msCb\msCc\msNa\msNc\Ed, \uncl{अ}\lac\  \msCa, \lk\lk\ \msNb\oo 
\textbf{॰क्रिया}\lem \msCa\msNa\msNb\msNc\Ed, ॰क्रियाः \msCb\msCc}}% 

%Verse 6:3

{\devanagarifont अष्टका पार्वणी श्राद्धं द्रव्ययज्ञः स उच्यते {॥ ६:३॥} \veg\dontdisplaylinenum }%
     \var{{\devanagarifont \numnoemph\vc\textbf{पार्वणी}\lem \msCa\msCc\msNa\msNc\Ed, पर्वणी \msCb, \uncl{पर्वणी} \msNb}}% 
    \var{{\devanagarifont \numnoemph\vd\textbf{॰यज्ञः}\lem \msCa\msCb\msNa\msNc\Ed, ॰यज्ञ \msCc, \lk\lk\ \msNb}}% 


\alalalfejezet{क्रियायज्ञः}

{\devanagarifont आरामोद्यानवापीषु देवतायतनेषु च \thinspace{\dandab} \dontdisplaylinenum }%
     \var{{\devanagarifont \numemph\vb\textbf{॰यतनेषु}\lem \msCb\msCc\Ed, ॰लयनेषु \msCa\msNa\msNc, ॰यत\lk\lk\ \msNb}}% 

%Verse 6:4

{\devanagarifont स्वहस्तकृतसंस्कारः क्रियायज्ञ स उच्यते {॥ ६:४॥} \veg\dontdisplaylinenum }%
     \var{{\devanagarifont \numnoemph\vc\textbf{॰हस्त॰}\lem \mssCaCbCc\msNa\msNc, \lk\lk\ \msNb, ॰हस्तैः \Ed}}% 


\alalalfejezet{जपयज्ञः}

{\devanagarifont जपयज्ञं ततो वक्ष्ये स्वर्गमोक्षफलप्रदम् \thinspace{\dandab} \dontdisplaylinenum }%
     \var{{\devanagarifont \numemph\va\textbf{॰यज्ञं ततो}\lem \msCa\msNa\msNb\msNc\Ed, ॰यज्ञं तपो \msCb ॰यज्ञस्ततो \msCc}}% 

{\devanagarifont वेदाध्ययन कर्तव्यं शिवसंहितमेव च  \danda\dontdisplaylinenum }%
     \var{{\devanagarifont \numnoemph\vc\textbf{वेदा॰}\lem \mssCaCbCc\msNa\msNc\Ed, अदा॰ \msNb}}% 

%Verse 6:5

{\devanagarifont इतिहासपुराणं च जपयज्ञः स उच्यते {॥ ६:५॥} \veg\dontdisplaylinenum }%
     \var{{\devanagarifont \numnoemph\ve\textbf{॰पुराणं च}\lem \mssCaCbCc\msNa\msNb\msNc, ॰पुराणश्च \Ed}}% 
    \var{{\devanagarifont \numnoemph\vf\textbf{॰यज्ञः}\lem \msCa\msCb\msNa\msNb\msNc\Ed, ॰यज्ञ \msCc}}% 


\alalalfejezet{ज्ञानयज्ञः}

{\devanagarifont इदं कर्म अकर्मेदमूहापोहविशारदः \thinspace{\dandab} \dontdisplaylinenum }%
     \var{{\devanagarifont \numemph\va\textbf{कर्म}\lem \mssCaCbCc\msNa\msNb\msNc, क्रमम् \Ed}}% 

%Verse 6:6

{\devanagarifont शास्त्रचक्षुः समालोक्य ज्ञानयज्ञः स उच्यते {॥ ६:६॥} \veg\dontdisplaylinenum }%
     \var{{\devanagarifont \numnoemph\vc\textbf{॰चक्षुः}\lem \msCa\msCb\msNa\msNb\msNc\Ed, ॰चक्षु \msCc}}% 
    \var{{\devanagarifont \numnoemph\vd\textbf{॰यज्ञः}\lem \msCa\msCb\msNa\msNc\Ed, ॰यज्ञ \msCc, ॰\uncl{यज्ञस्} \msNb}}% 


\alalalfejezet{ध्यानयज्ञः}

{\devanagarifont ध्यानयज्ञं समासेन कथयिष्यामि ते शृणु \thinspace{\dandab} \dontdisplaylinenum }%
     \var{{\devanagarifont \numemph\va\textbf{॰यज्ञं}\lem \msCa\msCb\msNa\msNc\Ed, ॰यज्ञ \msCc\msNb}}% 

{\devanagarifont ध्यानं पञ्चविधं चैव कीर्तितं हरिणा पुरा  \danda\dontdisplaylinenum }%
     \var{{\devanagarifont \numnoemph\vc\textbf{ध्यानं}\lem \mssCaCbCc\msNb\Ed, ध्यान \msNa\msNc}}% 

%Verse 6:7

{\devanagarifont सूर्यः सोमो ऽग्नि स्फटिकः सूक्ष्मं तत्त्वं च पञ्चमम् {॥ ६:७॥} \veg\dontdisplaylinenum }%
     \var{{\devanagarifont \numnoemph\ve\textbf{सोमो}\lem \msCa\msCc\msNa\msNc, सोमा॰ \msCb\msNb\Ed}}% 
    \var{{\devanagarifont \numnoemph\vf\textbf{सूक्ष्मं तत्त्वं च पञ्चमम्}\lem \msCb, 
सूक्ष्मं त\uncl{त्व}\lac  ञ्चमम् \msCa, 
सूक्ष्मतत्त्वं च पञ्चमः \msCc\msNa\msNb, 
सूक्ष्मं तत्त्वञ्च पञ्चमः \msNc, 
सूक्ष्मां तत्त्वश्च पञ्चमम् \Ed}}% 

{\devanagarifont सूर्यमण्डलमादौ तु तत्त्वं प्रकृतिरुच्यते \thinspace{\dandab} \dontdisplaylinenum }%
 
%Verse 6:8

{\devanagarifont तस्य मध्ये शशिं ध्यायेत्तत्त्वं पुरुष उच्यते {॥ ६:८॥} \veg\dontdisplaylinenum }%
     \var{{\devanagarifont \numemph\vc\textbf{शशिं}\lem \mssCaCbCc\msNa\Ed, शशि \msNb, शशिंन् \msNc}}% 
    \var{{\devanagarifont \numnoemph\vcd\textbf{ध्यायेत्त॰}\lem \msCa\msCb\msNa\msNb\msNc\Ed, ध्याये त॰ \msCc}}% 

{\devanagarifont चन्द्रमण्डलमध्ये तु ज्वालामग्निं विचिन्तयेत् \thinspace{\dandab} \dontdisplaylinenum }%
     \var{{\devanagarifont \numemph\vb\textbf{ज्वालामग्निं}\lem \mssCaCbCc\msNa\msNb\Ed, जालामग्नि \msNc}}% 

%Verse 6:9

{\devanagarifont प्रभुतत्त्वः स विज्ञेयो जन्ममृत्युविनाशनः {॥ ६:९॥} \veg\dontdisplaylinenum }%
     \var{{\devanagarifont \numnoemph\vc\textbf{॰तत्त्वः}\lem \mssCaCbCc\msNc, ॰तत्व \msNa, ॰तत्वं \msNb\Ed}}% 
    \var{{\devanagarifont \numnoemph\vd\textbf{॰नाशनः}\lem \msCa\msCb\msNa\msNb\msNc, ॰नाशनम् \msCc\Ed}}% 

{\devanagarifont अग्निमण्डलमध्ये तु ध्यायेत्स्फटिक निर्मलम् \thinspace{\dandab} \dontdisplaylinenum }%
     \var{{\devanagarifont \numemph\vb\textbf{ध्यायेत्स्फटिक}\lem \msCapcorr\msCb\msNa\msNb\msNc, ध्यायेत्स्फटि \msCaacorr, 
ध्याये स्फटिक \msCc\Ed\oo 
\textbf{॰मलम्}\lem \mssCaCbCc\msNb\Ed, ॰मलः \msNa, ॰\uncl{मलः} \msNc}}% 

%Verse 6:10

{\devanagarifont विद्यातत्त्वः स विज्ञेयः कारणमजमव्ययम् {॥ ६:१०॥} \veg\dontdisplaylinenum }%
     \var{{\devanagarifont \numnoemph\vc\textbf{तत्त्वः स}\lem \msCb\msNa\msNb\msNc, त\uncl{त्वन्}\lac\  \msCa, तत्व स \msCc, तत्वं स \Ed}}% 
    \var{{\devanagarifont \numnoemph\vd\textbf{॰जमव्ययम्}\lem \msCa\msCb\msNa\msNb\msNc\Ed, ॰मव्ययं \msCc}}% 

{\devanagarifont विद्यामण्डलमध्ये तु ध्यायेत्तत्त्वमनुत्तमम् \thinspace{\dandab} \dontdisplaylinenum }%
     \var{{\devanagarifont \numemph\vab\textbf{ध्यायेत्त॰}\lem \msCa\msCb\msNa\msNb\msNc\Ed, ध्याये त॰ \msCc}}% 

{\devanagarifont अकीर्तितमनौपम्यं शिवमक्षयमव्ययम्  \danda\dontdisplaylinenum }%
     \paral{{\devanagarifontsmall \vcd {\englishfont \DHARMP\ 4.14ab: } अकीर्तितमनौपम्यं पञ्चमं शिवमण्डलम् }}

%Verse 6:11

{\devanagarifont पञ्चमं ध्यानयज्ञस्य तत्त्वमुक्तं समासतः {॥ ६:११॥} \veg\dontdisplaylinenum }%
     \var{{\devanagarifont \numnoemph\ve\textbf{॰यज्ञस्य}\lem \msCa\msCb\msNa\msNb\msNc, ॰यज्ञञ्च \msCc\Ed}}% 
    \var{{\devanagarifont \numnoemph\vf\textbf{समासतः}\lem \mssCaCbCc\msNa\msNb\msNc, सनातनः \Ed}}% 

{\devanagarifont विगतराग उवाच {\dandab}\dontdisplaylinenum  }%
 
{\devanagarifont एकैकस्य तु तत्त्वस्य फलं कीर्तय कीदृशम् \thinspace{\danda} \dontdisplaylinenum }%
     \var{{\devanagarifont \numemph\va\textbf{तु}\lem \conj, त्रि॰ \mssCaCbCc\msNa\msNb\msNc, हि \Ed}}% 

%Verse 6:12

{\devanagarifont कानि लोकाः प्रपद्यन्ते कालं वास्य तपोधन {॥ ६:१२॥} \veg\dontdisplaylinenum }%
     \var{{\devanagarifont \numnoemph\vc\textbf{लोकाः}\lem \msCa\msNa\msNc, लोका \msCb\msCc\msNb\Ed\oo 
\textbf{प्रपद्यन्ते}\lem \msCb\msCc\msNa\msNb\msNc\Ed, प्र\lk\lk\lk\ \msCa}}% 
    \var{{\devanagarifont \numnoemph\vd\textbf{॰धन}\lem \msCa\msCc\msNa\msNb\Ed, ॰धनः \msCb\msNc}}% 

{\devanagarifont अनर्थयज्ञ उवाच {\dandab}\dontdisplaylinenum  }%
 
{\devanagarifont ब्रह्मलोकं तु प्रथमं तत्त्वप्रकृतिचिन्तया \thinspace{\danda} \dontdisplaylinenum }%
     \var{{\devanagarifont \numemph\vab\textbf{प्रथमं तत्त्व॰}\lem \mssCaCbCc\msNapcorr\msNb\msNc, 
\om\ \msNaacorr, प्रथमं तत्त्वं \Ed\oo 
\textbf{प्रकृतिचिन्तया}\lem \mssCaCbCc\msNa\msNb\msNc, च कृतिचिन्तय \Ed}}% 

%Verse 6:13

{\devanagarifont कल्पकोटिसहस्राणि शिववन्मोदते सुखी {॥ ६:१३॥} \veg\dontdisplaylinenum }%
     \var{{\devanagarifont \numnoemph\vd\textbf{सुखी}\lem \mssCaCbCc\msNa\msNb\msNc, सुखम् \Ed}}% 

{\devanagarifont द्वितीयं तत्त्व पुरुषं ध्यायमानो मृतो यदि \thinspace{\dandab} \dontdisplaylinenum }%
 
%Verse 6:14

{\devanagarifont विष्णुलोकमितो याति कल्पकोट्ययुतं सुखी {॥ ६:१४॥} \veg\dontdisplaylinenum }%
     \var{{\devanagarifont \numemph\vc\textbf{याति}\lem \mssCaCbCc\msNa\msNb\msNc, यान्ति \Ed}}% 

{\devanagarifont प्रभुतत्त्वं तृतीयं तु ध्यायमानो मरिष्यति \thinspace{\dandab} \dontdisplaylinenum }%
     \var{{\devanagarifont \numemph\va\textbf{॰तत्त्वं}\lem \msCa\msCb\msNa\msNb\msNc\Ed, ॰तत्व \msCc\oo 
\textbf{तृतीयं}\lem \mssCaCbCc\msNa\msNb\msNc, तृतीयस् \Ed}}% 
    \var{{\devanagarifont \numnoemph\vb\textbf{ध्यायमानो मरिष्यति}\lem \msCb\msCc\msNa\msNb\msNc, ध्याय\lk\lk \lk रिष्यति \msCa, 
धयायामानो मरिष्यति \Ed}}% 

%Verse 6:15

{\devanagarifont शिवलोके वसेन्नित्यं कल्पकोट्ययुतं शतम् {॥ ६:१५॥} \veg\dontdisplaylinenum }%
     \var{{\devanagarifont \numnoemph\vc\textbf{शिवलोके}\lem \msCa\msCc\msNa\msNb\msNc, शिवलोक \msCb, रुद्रलोके \Ed\oo 
\textbf{वसेन्नि॰}\lem \msCa\msCb\msNa\msNb\msNc\Ed, वसे नि॰ \msCc}}% 
    \var{{\devanagarifont \numnoemph\vd\textbf{॰युतं}\lem \mssCaCbCc\msNa\msNc\Ed, ॰युत \msNb}}% 

{\devanagarifont विद्यातत्त्वामृतं ध्यायेत्सदाशिवमनामयम् \thinspace{\dandab} \dontdisplaylinenum }%
     \var{{\devanagarifont \numemph\va\textbf{॰तत्त्वामृतं}\lem \msCa\msCb\msNa\msNb\msNc, ॰तत्वमृतन् \msCc, ॰तत्त्वामतं \Ed}}% 

%Verse 6:16

{\devanagarifont अक्षयं लोकमाप्नोति कल्पानान्तपरं तथा {॥ ६:१६॥} \veg\dontdisplaylinenum  }%
     \var{{\devanagarifont \numnoemph\vc\textbf{अक्षयं}\lem \mssCaCbCc\msNa\msNb\msNc, अक्षय॰ \Ed}}% 

{\devanagarifont पञ्चमं शिवतत्त्वं तु सूक्ष्मं चात्मनि संस्थितम् \thinspace{\dandab} \dontdisplaylinenum }%
 
%Verse 6:17

{\devanagarifont न कालसंख्या तत्रास्ति शिवेन सह मोदते {॥ ६:१७॥} \veg\dontdisplaylinenum }%
 
\ujvers\nemsloka {
{\devanagarifont पञ्चध्यानाभियुक्तो भवति च न पुनर्जन्मसंस्कारबन्धः }%
  \dontdisplaylinenum}    \var{{\devanagarifont \numemph\va\textbf{॰युक्तो}\lem \msCb\msCc\msNa\msNb\msNc, ॰यु\lk\ \msCa\ \toplost, ॰युक्तौ \Ed\oo 
\textbf{च}\lem \msCa\msCc\msNa\msNb\msNc, \om\ \msCb\Ed\oo 
\textbf{पुनर्जन्म॰}\lem \msCb\msNa\msNb\msNc\Ed, पुन\uncl{ज}न्म॰ \msCa\ \toplost, पुनजन्म॰ \msCc}}% 


\nemslokab

{\devanagarifont जिज्ञास्यन्तां द्विजेन्द्र भवदहनकरः प्रार्थनाकल्पवृक्षः  \danda\dontdisplaylinenum }%
     \var{{\devanagarifont \numnoemph\vb\textbf{जिज्ञास्यन्तां}\lem \msCa\msNb\msNc\Ed, जिज्ञास्यतां \msCb\msNa\ \unmetr, जिज्ञास्यन्ता \msCc}}% 

\nemslokac

{\devanagarifont जन्मेनैकेन मुक्तिर्भवति किमु न वा मानवाः साधयन्तु }%
  \dontdisplaylinenum    \var{{\devanagarifont \numnoemph\vc\textbf{जन्मेनैकेन}\lem \msCb\msNb\msNc\Ed, जन्मनैकेन \msCa\msCc\msNa\ \unmetr\oo 
\textbf{मुक्तिरभ्॰}\lem \msCa\msCb\msNa\msNb\msNc\Ed, मुक्ति भ्॰ \msCc\oo 
\textbf{न वा}\lem \mssCaCbCc\msNb\msNc\Ed, भवा \msNa\oo 
\textbf{मानवाः}\lem \msCa\msNa\msNb\msNc, मानमानवाः \msCb, मानवा \msCc, मानव \Ed}}% 

%Verse 6:18


\nemslokad

{\devanagarifont प्रत्यक्षान्नानुमानं सकलमलहरं स्वात्मसंवेदनीयम् {॥ ६:१८॥} \veg\dontdisplaylinenum }%
     \var{{\devanagarifont \numnoemph\vd\textbf{प्रत्यक्षा॰}\lem \mssCaCbCc\msNb\msNc\Ed, प्रत्यक्ष॰ \msNa\oo 
\textbf{॰वेदनीयम्}\lem \msCb\msNa\msNb, ॰वेदनीयः \msCa\msCc\msNc, ॰वेदनीय \Ed}}% 

\vers



\alalfejezet{नियमेषु तपः (३)}
{\devanagarifont मानसं तप आदौ तु द्वितीयं वाचिकं तपः \thinspace{\dandab} \dontdisplaylinenum }%
     \var{{\devanagarifont \numemph\va\textbf{॰तप}\lem \mssCaCbCc\msNa\msNb\msNc, ॰तपम् \Ed}}% 

{\devanagarifont कायिकं च तृतीयं तु मनोवाक्कर्म तत्परम्  \danda\dontdisplaylinenum }%
     \var{{\devanagarifont \numnoemph\vc\textbf{कायिकं च तृतीयं तु}\lem \mssCaCbCc\msNa\msNc\Ed, मानसं तप आदौ तु \msNb\ {\englishfont (eyeskip)}}}% 
    \var{{\devanagarifont \numnoemph\vd\textbf{मनोवाक्कर्म}\lem \msCa\msNc\Ed, मनोक्कर्म \msCb, म्मनोवाकर्म॰ \msCc, मनोवाक्काय॰ \msNa\msNb\oo 
\textbf{॰परम्}\lem \msCc, ॰परः \msCa\msCb\msNa\msNb\msNc\Ed}}% 

%Verse 6:19

{\devanagarifont कायिकं वाचिकं चैव तपो मिश्रक पञ्चमम् {॥ ६:१९॥} \veg\dontdisplaylinenum }%
     \var{{\devanagarifont \numnoemph\ve\textbf{कायिकं}\lem \mssCaCbCc\msNb\msNc\Ed, कायिक \msNa}}% 

{\devanagarifont मनःसौम्यं प्रसादश्च आत्मनिग्रहमेव च \thinspace{\dandab} \dontdisplaylinenum }%
     \var{{\devanagarifont \numemph\va\textbf{॰सौम्यं}\lem \msNc, ॰सौम्य॰ \msCa\msCb\msNa\msNb\Ed, ॰सौम्\uncl{य}॰ \msCc\ \toplost\oo 
\textbf{प्रसादश्च}\lem \msCa\msCc\msNa\msNc, प्रसादं च \msCb\Ed, प्रदानश्च \msNb}}% 

%Verse 6:20

{\devanagarifont मौनं भावविशुद्धिश्च पञ्चैतत्तप मानसम् {॥ ६:२०॥} \veg\dontdisplaylinenum }%
     \var{{\devanagarifont \numnoemph\vc\textbf{मौनं}\lem \mssCaCbCc\msNa\msNb\msNc, मौन\lk  \Ed\oo 
\textbf{॰शुद्धिश्च}\lem \msCa\msCb\msNa\msNb\msNc, ॰शुद्धिं च \msCc\Ed}}% 
    \var{{\devanagarifont \numnoemph\vd\textbf{पञ्चैतत्}\lem \msCa\msNb\msNc, पञ्चैते \msCb\msNa, पञ्चेतत् \msCc, पञ्चैतन् \Ed}}% 
    \paral{{\devanagarifontsmall \vo {\englishfont \similar\ \MBH\ 6.39.16 (\BHG\ 17.16):}
                 मनःप्रसादः सौम्यत्वं मौनमात्मविनिग्रहः\thinspace{\devanagarifontsmall ।}
                 भावसंशुद्धिरित्येतत्तपो मानसमुच्यते\thinspace{\devanagarifontsmall ॥} }}

{\devanagarifont अनुद्वेगकरा वाणी प्रियं सत्यं हितं च यत् \thinspace{\dandab} \dontdisplaylinenum }%
 
%Verse 6:21

{\devanagarifont स्वाध्यायाभ्यसनं चैव वाचिकं तप उच्यते {॥ ६:२१॥} \veg\dontdisplaylinenum }%
     \var{{\devanagarifont \numemph\vc\textbf{॰भ्यसनं चैव}\lem \msCb\msCc\msNa\msNc\Ed, ॰भ्यसन\lk\lk\ \msCa, ॰भ्यस\uncl{नं} चैव \msNb}}% 
    \paral{{\devanagarifontsmall \vcd {\englishfont \similar\ \MBH\ 6.39.15cd (\BHG\ 17.15):}
                                  अनुद्वेगकरं वाक्यं सत्यं प्रियहितं च यत्\thinspace{\devanagarifontsmall ।}
                                  स्वाध्यायाभ्यसनं चैव वाङ्मयं तप उच्यते\thinspace{\devanagarifontsmall ॥} }}

{\devanagarifont आर्जवं च अहिंसा च ब्रह्मचर्यं सुरार्चनम् \thinspace{\dandab} \dontdisplaylinenum }%
     \var{{\devanagarifont \numemph\va\textbf{आर्जवं च अहिंसा च}\lem \mssCaCbCc\msNa\msNb\msNc, 
आर्जवत्वमहिंसाश्च \Ed}}% 
    \var{{\devanagarifont \numnoemph\vb\textbf{॰चर्यं}\lem \msCa\msCb\msNa\msNb\msNc, ॰चर्य \msCc\Ed}}% 

%Verse 6:22

{\devanagarifont शौचं पञ्चममित्येतत्कायिकं तप उच्यते {॥ ६:२२॥} \veg\dontdisplaylinenum }%
     \var{{\devanagarifont \numnoemph\vc\textbf{शौचं}\lem \mssCaCbCc\msNa\msNb\msNc, शौच \Ed}}% 
    \paral{{\devanagarifontsmall \vo {\englishfont \compare\ \MBH\ 6.39.14 (\BHG\ 17.14):}
                          देवद्विजगुरुप्राज्ञपूजनं शौचमार्जवम्\thinspace{\devanagarifontsmall ।}
                          ब्रह्मचर्यमहिंसा च शारीरं तप उच्यते\thinspace{\devanagarifontsmall ॥} }}

{\devanagarifont इष्टं कल्याणभावं च धन्यं पथ्यं हितं वदेत् \thinspace{\dandab} \dontdisplaylinenum }%
     \var{{\devanagarifont \numemph\va\textbf{इष्टं}\lem \msCa\msCb\msNa\msNc\Ed, इष्ट \msCc\msNb\oo 
\textbf{॰भावं}\lem \mssCaCbCc\msNa\msNb\msNc, ॰भावश् \Ed}}% 
    \var{{\devanagarifont \numnoemph\vb\textbf{पथ्यं}\lem \mssCaCbCc\msNa\msNb\msNc, सत्यं \Ed}}% 

%Verse 6:23

{\devanagarifont मनोमिश्रक पञ्चैतत्तप उक्तं महर्षिभिः {॥ ६:२३॥} \veg\dontdisplaylinenum }%
     \var{{\devanagarifont \numnoemph\vc\textbf{मनो॰}\lem \mssCaCbCc\msNa\msNb\msNc, मन॰ \Ed\oo 
\textbf{पञ्चैतत्}\lem \mssCaCbCc\msNa\msNb, पञ्चेतत् \msNc, पञ्चैतान् \Ed}}% 
    \var{{\devanagarifont \numnoemph\vd\textbf{तप उक्तं महर्षिभिः}\lem \mssCaCbCc\msNa\msNb\msNc, तपमुक्तं महिर्षिभिः \Ed}}% 

{\devanagarifont स्वस्ति मङ्गलमाशीर्भिरतिथिगुरुपूजनम् \thinspace{\dandab} \dontdisplaylinenum }%
     \var{{\devanagarifont \numemph\va\textbf{॰शीर्भि॰}\lem \msCa\Ed, ॰शीभि॰ \msCb\msCc\msNa\msNb\msNc}}% 
    \var{{\devanagarifont \numnoemph\vb\textbf{॰तिथि॰}\lem \mssCaCbCc\msNa\msNb\msNc, ॰तिथिं \Ed}}% 
    \paral{{\devanagarifontsmall \vab {\englishfont \compare\ \SDHS\ 11.79:}
                 नमस्काराभिवादेषु स्वस्तिमङ्गलवाचकैः\thinspace{\devanagarifontsmall ।}
                 शिवं भवतु सर्वत्र प्रब्रूयात्सर्वकर्मसु\thinspace{\devanagarifontsmall ॥} }}

%Verse 6:24

{\devanagarifont कायमिश्रक पञ्चैतत्तप उक्तं महात्मभिः {॥ ६:२४॥} \veg\dontdisplaylinenum }%
     \var{{\devanagarifont \numnoemph\vc\textbf{॰मिश्रक}\lem \msCc\msNa\msNb\msNc\Ed, ॰\lk\lk क \msCa, ॰मित्यश्रक \msCb\oo 
\textbf{पञ्चैतत्}\lem \mssCaCbCc\msNa\msNb\msNc, पञ्चैतन् \Ed}}% 
    \var{{\devanagarifont \numnoemph\vd\textbf{तप उक्तं}\lem \mssCaCbCc\msNa\msNb\msNc, तपमुक्तं \Ed}}% 

{\devanagarifont मण्डूकयोगी हेमन्ते ग्रीष्मे पञ्चतपास्तथा \thinspace{\dandab} \dontdisplaylinenum }%
     \var{{\devanagarifont \numemph\vb\textbf{ग्रीष्मे}\lem \mssCaCbCc\msNa\msNb\msNc, गृष्मे \Ed}}% 
    \paral{{\devanagarifontsmall \vab {\englishfont \similar\ \MBH\ Appendices 15.801:}
                                 मण्डूकशायी हेमन्ते ग्रीष्मे पञ्चतपा भवेत
                     {\englishfont \similar\ \UMS\ 6.26ab:}मण्डूकयोगो हेमन्ते ग्रीष्मे पञ्चतपास्तथा;
                     {\englishfont \compare\ \SDHSAMGR\ 9.32ab:}
                         अभ्रावकाश्यं शीतोष्णे पञ्चाग्निर्जलशायिता }}

%Verse 6:25

{\devanagarifont अभ्रावकाशो वर्षासु तपः साधनमुच्यते {॥ ६:२५॥} \veg\dontdisplaylinenum }%
     \var{{\devanagarifont \numnoemph\vc\textbf{॰वकाशो}\lem \eme, ॰वकाशे \mssCaCbCc\msNa\msNb\msNc\Ed}}% 
    \var{{\devanagarifont \numnoemph\vd\textbf{तपः}\lem \msCa\msCb\msNa\msNb\msNc\Ed, तप \msCc\oo 
\textbf{साधनमु॰}\lem \msCa\msNa\msNc\Ed, साधन उ॰ \msCb\msCc\msNb}}% 

{\devanagarifont स्वमांसोद्धृत्य दानं च हस्तपादशिरस्तथा \thinspace{\dandab} \dontdisplaylinenum }%
     \var{{\devanagarifont \numemph\va\textbf{दानं}\lem \mssCaCbCc\msNa\msNc, \uncl{दान} \msNb\ \toplost, दानश् \Ed}}% 

%Verse 6:26

{\devanagarifont पुष्पमुत्पाद्य दानंच सर्वे ते तपसाधनाः {॥ ६:२६॥} \veg\dontdisplaylinenum }%
     \var{{\devanagarifont \numnoemph\vc\textbf{दानं}\lem \mssCaCbCc\msNa\msNb\msNc, दानश् \Ed}}% 
    \var{{\devanagarifont \numnoemph\vd\textbf{तप}\lem \Ed, तपः \mssCaCbCc\msNa\msNb\msNc\ \unmetr}}% 

{\devanagarifont कृच्छ्रातिकृच्छ्रं नक्तं च तप्तकृच्छ्रमयाचितम् \thinspace{\dandab} \dontdisplaylinenum }%
     \var{{\devanagarifont \numemph\va\textbf{कृच्छ्रातिकृच्छ्रं}\lem \msCa\msCb\msNa\Ed, कृच्छ्रादिकृच्छ्र \msCc, कृच्छ्रातिकृच्छ्र \msNb, कृच्छातिकृच्छं \msNc}}% 
    \var{{\devanagarifont \numnoemph\vb\textbf{॰याचितम्}\lem \mssCaCbCc\msNa\msNb\msNc, ॰याचितः \Ed}}% 

%Verse 6:27

{\devanagarifont चान्द्रायणं पराकं च तपः सांतपनादयः {॥ ६:२७॥} \veg\dontdisplaylinenum }%
     \var{{\devanagarifont \numnoemph\vc\textbf{चान्द्रायणं पराकं}\lem \msCa\msCc\msNb\msNc, चान्द्रायनं पराकं \msCb, 
चन्द्रायणं पराकं \msNa, चान्द्रायणवराकश् \Ed}}% 
    \var{{\devanagarifont \numnoemph\vd\textbf{तपः सांतपनादयः}\lem \msCa\msCb\msNa\msNb\msNc, तपसान्तपनादयः \msCc\Ed}}% 

\ujvers\nemsloka {
{\devanagarifont येनेदं तप तप्यते सुमनसा संसारदुःखच्छिदम् }%
  \dontdisplaylinenum}    \var{{\devanagarifont \numemph\va\textbf{तप त॰}\lem \Ed, तपस्त॰ \mssCaCbCc\msNa\msNb\msNc\ \unmetr\oo 
\textbf{॰मनसा}\lem \eme, ॰मनसः \mssCaCbCc\msNa\msNb\msNc\Ed}}% 


\nemslokab

{\devanagarifont आशापाश विमुच्य निर्मलमतिस्त्यक्त्वा जघन्यं फलम्  \danda\dontdisplaylinenum }%
     \var{{\devanagarifont \numnoemph\vb\textbf{निर्मलमति॰}\lem \msCa\msCc\msNa\msNb\msNc\Ed, निर्मलर्मति॰ \msCb\oo 
\textbf{जघन्यं}\lem \mssCaCbCc\msNa\msNb\msNc, जगत्यं \Ed}}% 

\nemslokac

{\devanagarifont स्वर्गाकाङ्क्ष्यनृपत्वभोगविषयं सर्वान्तिकं तत्फलं }%
  \dontdisplaylinenum    \var{{\devanagarifont \numnoemph\vc\textbf{॰काङ्क्ष्य॰}\lem \mssCaCbCc\msNa\msNb\msNc, ॰कांक्ष॰ \Ed\oo 
\textbf{सर्वान्तिकं}\lem \msCa\msCc\msNa\msNb\msNc\Ed, सर्वार्त्तिकं \msCb}}% 

%Verse 6:28


\nemslokad

{\devanagarifont जन्तुः शाश्वतजन्ममृत्युभवने तन्निष्ठसाध्यं वहेत् {॥ ६:२८॥} \veg\dontdisplaylinenum }%
     \var{{\devanagarifont \numnoemph\vd\textbf{॰भवने}\lem \mssCaCbCc\msNa\msNb\Ed, ॰भवेने \msNc\oo 
\textbf{॰साध्यं वहेत्}\lem \msCc\msNa\msNb\msNc, ॰\uncl{साध्यम्}\lk\lk\ \msCa, 
॰साध्य वहेत् \msCb, ॰साध्यं वदेत् \Ed}}% 

\vers


{\devanagarifont 
\jump
\begin{center}
\ketdanda~इति वृषसारसंग्रहे षष्ठो ऽध्यायः~\ketdanda
\end{center}
\dontdisplaylinenum\vers  }%
 