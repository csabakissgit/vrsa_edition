\fejno=0\versno=0
\centerline{\Huge\devanagarifontbold वृषसारसंग्रहः  }

 
{\vrule depth10pt width0pt}
\versno=0\fejno=10
\thispagestyle{empty}

\centerline{\Large\devanagarifontbold [   दशमो ऽध्यायः  ]}{\vrule depth10pt width0pt} \fancyhead[CO]{{\footnotesize\devanagarifont वृषसारसंग्रहे  }}
\fancyhead[CE]{{\footnotesize\devanagarifont दशमो ऽध्यायः  }}
\fancyhead[LE]{}
\fancyhead[RE]{}
\fancyhead[LO]{}
\fancyhead[RO]{}
\szam\bek



\alalfejezet{कायतीर्थोपवर्णनम्}
\vers


{\devanagarifont विगतराग उवाच {\dandab}\dontdisplaylinenum  }%
 
{\devanagarifont कतमं सर्वतीर्थानां श्रेष्ठमाहुर्मनीषिनः \thinspace{\danda} \dontdisplaylinenum }%
     \var{{\devanagarifont \numemph\va\textbf{कतमं सर्व॰}\lem \mssALL, 
कतमसर्व॰ \msNb, कथमन्सर्व॰ \msNc}}% 
    \var{{\devanagarifont \numnoemph\vab\textbf{॰तीर्थानां श्रेष्ठ॰}\lem \mssALL, ॰तीर्था\lk\lk ष्ठ॰ \msCa}}% 
    \var{{\devanagarifont \numnoemph\vb\textbf{मनीषिनः}\lem \mssALL, मनीषिभिः \Ed}}% 
    \lacuna{\devanagarifontsmall {\englishfont Witnesses used for this chapter: \msCa\ ff.\thinspace 207r--208v, 
                                              \msCb\ ff.\thinspace 212v--214r, 
                                              \msCc\ ff.\thinspace 283v--285v,
                                              \msNa\ ff.\thinspace 14v--15v, 
                                              \msNb\ exp.\thinspace 55 (lower) -- 56 (lower),
                                              \msNc\ ff.\thinspace 222v--223v,
                                              \Ed\ pp.\thinspace 610--613; 
                                              \mssCaCbCc\ = \msCa + \msCb + \msCc} }%
  
%Verse 10:1

{\devanagarifont कथयस्व मुनिश्रेष्ठ यद्यस्ति भुवि कामदम् {॥ १०:१॥} \veg\dontdisplaylinenum }%
     \var{{\devanagarifont \numnoemph\vd\textbf{भुवि}\lem \mssALL, भूरि \Ed\oo 
\textbf{॰दम्}\lem \mssALL, ॰दः \msNa}}% 

{\devanagarifont अनर्थयज्ञ उवाच {\dandab}\dontdisplaylinenum  }%
 
{\devanagarifont अतिगुह्यमिदं प्रश्नं पृष्टः स्नेहाद्द्विजोत्तम \thinspace{\danda} \dontdisplaylinenum }%
     \var{{\devanagarifont \numemph\vb\textbf{स्नेहाद्द्वि॰}\lem \mssALL, स्नेहा द्वि॰ \msCc}}% 

%Verse 10:2

{\devanagarifont ब्रवीमि वः पुरावृत्तं नन्दिना कथितो ऽस्म्यहम् {॥ १०:२॥} \veg\dontdisplaylinenum }%
     \var{{\devanagarifont \numnoemph\vd\textbf{ऽस्म्यहम्}\lem \mssALL, स्मृहम् \msCc}}% 

{\devanagarifont नन्दिकेश्वर उवाच {\dandab}\dontdisplaylinenum  }%
     \var{{\devanagarifont \numemph\vo\textbf{नन्दि॰}\lem \mssALL, नन्दी॰ \msCb}}% 

{\devanagarifont कैलासशिखरे रम्ये सिद्धचारणसेविते \thinspace{\danda} \dontdisplaylinenum }%
     \var{{\devanagarifont \numnoemph\va\textbf{कैलास॰}\lem \mssALL, कैलाशे \Ed}}% 
    \paral{{\devanagarifontsmall \vab {\englishfont  \compare\ MBh 12.327.18cd:} मेरौ गिरिवरे रम्ये सिद्धचारणसेविते  }}

%Verse 10:3

{\devanagarifont तत्रासीनं शिवं साक्षाद्देवी वचनमब्रवीत् {॥ १०:३॥} \veg\dontdisplaylinenum }%
 
{\devanagarifont देव्युवाच {\dandab}\dontdisplaylinenum  }%
 
{\devanagarifont भगवन्देवदेवेश सर्वभूतजगत्पते \thinspace{\danda} \dontdisplaylinenum }%
     \var{{\devanagarifont \numemph\va\textbf{॰देवेश}\lem \mssALL, ॰देश \msCb}}% 
    \var{{\devanagarifont \numnoemph\vb\textbf{॰पते}\lem \mssALL, ॰पतिम् \msNaacorr}}% 

%Verse 10:4

{\devanagarifont प्रष्टुमिच्छाम्यहं त्वेकं धर्मगुह्यं सनातनम् {॥ १०:४॥} \veg\dontdisplaylinenum }%
     \var{{\devanagarifont \numnoemph\vc\textbf{धर्म॰}\lem \mssALL, ध\uncl{र्मं} \msNa}}% 

{\devanagarifont अतितीर्थं परं गुह्यं संसाराद्येन मुच्यते \thinspace{\dandab} \dontdisplaylinenum }%
     \var{{\devanagarifont \numemph\va\textbf{॰तीर्थं}\lem \mssALL, ॰तीर्थ \msNb\Ed}}% 
    \var{{\devanagarifont \numnoemph\vab\textbf{गुह्यं संसाराद्येन मुच्यते}\lem \mssALL, 
\uncl{ग}\lac  \uncl{सं}साराद्येन मुच्यते \msNb}}% 

%Verse 10:5

{\devanagarifont मनुष्याणां हितार्थाय ब्रूहि तत्त्वं महेश्वर {॥ १०:५॥} \veg\dontdisplaylinenum }%
     \var{{\devanagarifont \numnoemph\vd\textbf{॰श्वर}\lem \mssALL, ॰श्वरः \msCc}}% 

{\devanagarifont महेश्वर उवाच {\dandab}\dontdisplaylinenum  }%
 
{\devanagarifont को मां पृच्छति तं प्रश्नं मुक्त्वा त्वामेव सुन्दरि \thinspace{\danda} \dontdisplaylinenum }%
     \var{{\devanagarifont \numemph\va\textbf{तं प्रश्नं}\lem \msNa\msNb, तत्प्रश्न \msCa\msCb, तत्प्रश्नं \msCc\Ed, 
तं प्रश्न \msNc}}% 
    \var{{\devanagarifont \numnoemph\vb\textbf{मुक्त्वा}\lem \mssALL, मुक्ता \Ed}}% 

%Verse 10:6

{\devanagarifont शृणु वक्ष्यामि तं प्रश्नं देवैरपि सुदुर्लभम् {॥ १०:६॥} \veg\dontdisplaylinenum }%
     \var{{\devanagarifont \numnoemph\vc\textbf{तं प्रश्नं}\lem \msNc, तत्प्रश्नं \mssCaCbCc\msNa\msNb\Ed}}% 

{\devanagarifont कुरुक्षेत्रं प्रयागं च वाराणसीमतः परम् \thinspace{\dandab} \dontdisplaylinenum }%
 
%Verse 10:7

{\devanagarifont गङ्गाग्निं सोमतीर्थं च सूर्यपुष्करमानसम् {॥ १०:७॥} \veg\dontdisplaylinenum }%
     \var{{\devanagarifont \numemph\vc\textbf{गङ्गाग्निं}\lem \msCa\msCb, गङ्गाग्नि \msCc\msNa\msNb\msNc, गङ्गाऽग्नि॰ \Ed}}% 

{\devanagarifont नैमिषं बिन्दुसारं च सेतुबन्धं सुरद्रहम् \thinspace{\dandab} \dontdisplaylinenum }%
     \var{{\devanagarifont \numemph\va\textbf{नैमिषं}\lem \mssALL, नेमिस \msNc}}% 
    \var{{\devanagarifont \numnoemph\vb\textbf{॰बन्धं}\lem \mssALL, ॰बन्ध॰ \Ed\oo 
\textbf{॰द्रहम् }\lem \mssALL, ॰ह्रदं \Ed}}% 

%Verse 10:8

{\devanagarifont घण्टिकेश्वरवागीशं ज्ञात्वा निश्चयपापहा {॥ १०:८॥} \veg\dontdisplaylinenum }%
     \var{{\devanagarifont \numnoemph\vc\textbf{॰वागीशं}\lem \mssALL, \lac \uncl{गीश} \msNb}}% 
    \var{{\devanagarifont \numnoemph\vd\textbf{निश्चयपापहा}\lem \mssALL, 
निश्च\uncl{य}\lk\lk\lk\  \msCa}}% 

{\devanagarifont उमोवाच {\dandab}\dontdisplaylinenum  }%
 
{\devanagarifont एवमादि महादेव पूर्ववत्कथितास्म्यहम् \thinspace{\danda} \dontdisplaylinenum }%
     \var{{\devanagarifont \numemph\vb\textbf{कथिता॰}\lem \msCa\msCc\msNa\msNc, कथितो \msCb\msNb\Ed}}% 

%Verse 10:9

{\devanagarifont स्वर्गभोगप्रदं तीर्थमेतेषां सुरनायक {॥ १०:९॥} \veg\dontdisplaylinenum }%
     \var{{\devanagarifont \numnoemph\vcd\textbf{तीर्थमे॰}\lem \mssALL, तीर्थंमे॰ \msCc}}% 
    \var{{\devanagarifont \numnoemph\vd\textbf{सुरनायक}\lem \msCapcorr\msNa\msNc, सुरनाक \msCaacorr, सुरनायकम् \msCb\msCc\msNb\Ed}}% 

{\devanagarifont कथं मुच्येत संसाराज्ज्ञानमात्रेण ईश्वर \thinspace{\dandab} \dontdisplaylinenum }%
     \var{{\devanagarifont \numemph\va\textbf{कथं}\lem \mssALL, कथ \msCb}}% 
    \var{{\devanagarifont \numnoemph\vb\textbf{ज्ञान॰}\lem \mssALL, ज्ञात॰ \msCb\oo 
\textbf{ईश्वर}\lem \mssALL, चेश्वर \msNa}}% 

%Verse 10:10

{\devanagarifont कौतूहलं महज्जातं छिन्धि संशयकारकम् {॥ १०:१०॥} \veg\dontdisplaylinenum }%
     \var{{\devanagarifont \numnoemph\vc \lem \mssCaCbCc\Ed, कौतूहलम्म\uncl{हो}ज्जातं \msNa, 
कौहलम्महज्जातं \msNbacorr, 
कौ\uncl{तू}हलम्महज्जातं \msNbpcorr, 
कोतूहलं महज्जातं \msNc}}% 
    \var{{\devanagarifont \numnoemph\vd\textbf{॰कारकम्}\lem \Ed, ॰कारक \mssCaCbCc\msNb\msNc, ॰कारकः \msNa}}% 

{\devanagarifont रुद्र उवाच {\dandab}\dontdisplaylinenum  }%
 
{\devanagarifont किं न जानामि तत्तीर्थं सुलभं दुर्लभं च यत् \thinspace{\danda} \dontdisplaylinenum }%
     \var{{\devanagarifont \numemph\va\textbf{जानामि}\lem \mssCaCbCc\msNb, जाना\uncl{मि} \msNaacorr, जाना\uncl{सि} \msNapcorr, 
जानासि \msNc\Ed}}% 
    \var{{\devanagarifont \numnoemph\vb\textbf{दुर्लभं च}\lem \msCa\msNa\msNb\Ed, दुलभञ्च \msCb\msNc, दुल्लभञ्च \msCc}}% 

%Verse 10:11

{\devanagarifont सुलभं गुरुसेवीनां दुर्लभं तद्विवर्जयेत् {॥ १०:११॥} \veg\dontdisplaylinenum }%
     \var{{\devanagarifont \numnoemph\vc \lem \mssALL, 
\lk\lk \lk\lk \lk\lk वीनां \msCa}}% 
    \var{{\devanagarifont \numnoemph\vd\textbf{॰वर्जयेत्}\lem \mssALL, ॰वर्जये \msNa, ॰वर्जनात् \Ed}}% 


\alalalfejezet{कुरुक्षेत्रम्}

{\devanagarifont कुरुः पुरुष विज्ञेयः शरीरं क्षेत्र उच्यते \thinspace{\dandab} \dontdisplaylinenum }%
     \var{{\devanagarifont \numemph\va\textbf{कुरुः}\lem \mssALL, गुरुः \msNb\oo 
\textbf{पुरुष}\lem \Ed, पुरुषः \mssCaCbCc\msNa\msNb\ \unmetr, पुरुषो \msNc\ \unmetr}}% 
    \var{{\devanagarifont \numnoemph\vb\textbf{शरीरं}\lem \mssALL, शरी\uncl{र} \msCa\oo 
\textbf{क्षेत्र उच्यते}\lem \mssALL, क्षेत्रमुच्यते \msNa}}% 
    \paral{{\devanagarifontsmall \vb {\englishfont \compare\ \BHG\ 13.1:}
                         इदं शरीरं कौन्तेय क्षेत्रमित्यभिधीयते\thinspace{\devanagarifontsmall ।}
                         एतद्यो वेत्ति तं प्राहुः क्षेत्रज्ञ इति तद्विदः\thinspace{\devanagarifontsmall ॥} }}

%Verse 10:12

{\devanagarifont शरीरस्थं कुरुक्षेत्रं सर्वतीर्थफलप्रदम् {॥ १०:१२॥} \veg\dontdisplaylinenum }%
     \var{{\devanagarifont \numnoemph\vc\textbf{॰स्थं}\lem \mssALL, ॰स्थ \msNc\oo 
\textbf{॰क्षेत्रं}\lem \mssALL, ॰क्षेत्र \msNc}}% 

{\devanagarifont सर्वयज्ञफलावाप्तिः सर्वदानफलानि च \thinspace{\dandab} \dontdisplaylinenum }%
     \paral{{\devanagarifontsmall \vab {\englishfont \similar\ \UMS\ 21.48cd:}
                                 सर्वयज्ञफलावाप्तिः सर्वदानफलं लभेत् }}

%Verse 10:13

{\devanagarifont सर्वव्रततपश्चीर्णं तत्फलं सकलं भवेत् {॥ १०:१३॥} \veg\dontdisplaylinenum }%
     \var{{\devanagarifont \numemph\vd\textbf{तत्फलं}\lem \mssALL, तत्फल \msNc}}% 

{\devanagarifont एवमेव फलं तेषां तीर्थपञ्चदशेषु च \thinspace{\dandab} \dontdisplaylinenum }%
     \var{{\devanagarifont \numemph\vb\textbf{तीर्थपञ्चदशेषु}\lem \mssALL, तीर्थम्पंचदशैषु \msCb}}% 

%Verse 10:14

{\devanagarifont अनघानं महापुण्यं महातीर्थं महासुखम् {॥ १०:१४॥} \veg\dontdisplaylinenum }%
     \var{{\devanagarifont \numnoemph\vc \lem \msCb\msNc, \lk\lk \lk\lk \lk\lk पुण्य \msCa, 
अनप्याम्महापुण्यं \msCc\ \hypermetr, 
अनध्यानं महापुण्यं \msNa, अध्वानन्तु महापुण्यं \msNb, 
स्नानध्यानं महापुण्यं \Ed}}% 

{\devanagarifont देव्युवाच {\dandab}\dontdisplaylinenum  }%
 
{\devanagarifont अतीव रोमहर्षो मे जातो ऽस्ति त्रिदशेश्वर \thinspace{\danda} \dontdisplaylinenum }%
     \var{{\devanagarifont \numemph\va\textbf{अतीव}\lem \mssALL, अवीव \msCb}}% 
    \var{{\devanagarifont \numnoemph\vb\textbf{ऽस्ति}\lem \mssALL, स्मि \msNb\oo 
\textbf{त्रिदशेश्वर}\lem \mssALL, त्रिदशेश्वरः \msCc, त्रि\lac  शेश्वर \msNb}}% 

%Verse 10:15

{\devanagarifont सुलभं सुकरं सूक्ष्मं श्रुत्वा तुष्टिश्च मे गता {॥ १०:१५॥} \veg\dontdisplaylinenum }%
     \var{{\devanagarifont \numnoemph\vd\textbf{तुष्टिश्च}\lem \mssALL, तुष्टिञ्च \msCc\oo 
\textbf{गता}\lem \mssALL, गताः \msCb}}% 

{\devanagarifont चतुर्दश परो भूयः कथयस्व मनोहरम् \thinspace{\dandab} \dontdisplaylinenum }%
 
%Verse 10:16

{\devanagarifont प्रयागादि पृथक्त्वेन तत्त्वतस्तु सुरेश्वर {॥ १०:१६॥} \veg\dontdisplaylinenum }%
     \var{{\devanagarifont \numemph\vd\textbf{तत्त्वतस्तु}\lem \mssALL, तत्वत \msNaacorr}}% 


\alalalfejezet{प्रयागो वाराणसी च}

{\devanagarifont रुद्र उवाच {\dandab}\dontdisplaylinenum  }%
 
{\devanagarifont सुषुम्ना भगवती गङ्गा इडा च यमुना नदी \thinspace{\danda} \dontdisplaylinenum }%
     \var{{\devanagarifont \numemph\va\textbf{सुषुम्ना}\lem \mssALL, सुषुम्णा \Ed\oo 
\textbf{भगवती गङ्गा}\lem \mssALL, 
भगवती ग\lk\ \msCa, भवती गङ्गा \Ed}}% 

%Verse 10:17

{\devanagarifont एताः स्रोतोवहा नद्यः प्रयागः स विधीयते {॥ १०:१७॥} \veg\dontdisplaylinenum }%
     \var{{\devanagarifont \numnoemph\vc\textbf{एताः स्रोतोवहा}\lem \eme, एता श्रोतवहा \msCa\msNc\Ed, 
एते श्रोतावहा \msCb\msCc, एता श्रोत्रवहा \msNa\msNb}}% 

{\devanagarifont दक्षिणा वारुणी नासा वामनासा असि स्मृता \thinspace{\dandab} \dontdisplaylinenum }%
     \var{{\devanagarifont \numemph\va\textbf{दक्षिणा}\lem \mssALL, दक्षि\uncl{णं} \msCa, दक्षिणं \msCc\oo 
\textbf{वारुणी}\lem \msNapcorr\msNc\Ed, वरुणी \msCa\msCc\msNaacorr\msNb, वरुणा \msCb}}% 
    \var{{\devanagarifont \numnoemph\vb\textbf{॰नासा}\lem \mssALL, ॰ना \msCb\msNb}}% 

%Verse 10:18

{\devanagarifont वारुणा-असिमध्येन तेन वाराणसी स्मृता {॥ १०:१८॥} \veg\dontdisplaylinenum }%
     \var{{\devanagarifont \numnoemph\vc \lem \Ed, वरुणा असिमध्येन \msCa\msCb\msNa\msNc, 
वारुणन्नासमध्येत \msCc, 
वरुण असिमध्येन \msNb}}% 


\alalalfejezet{गङ्गा}

{\devanagarifont आकाशगङ्गा विख्याता तस्याः स्रवति चामृतम् \thinspace{\dandab} \dontdisplaylinenum }%
     \var{{\devanagarifont \numemph\vb\textbf{तस्याः}\lem \mssALL, तस्मा \msCc, तस्या \msNb}}% 

%Verse 10:19

{\devanagarifont अहोरात्रमविच्छिन्नं गङ्गा सा तेन उच्यते {॥ १०:१९॥} \veg\dontdisplaylinenum }%
     \var{{\devanagarifont \numnoemph\vd\textbf{तेन}\lem \mssALL, ते \msCc}}% 


\alalalfejezet{सोमतीर्थम्}

{\devanagarifont सोमतीर्थमिडा नाडी किङ्किणीरवचिह्निता \thinspace{\dandab} \dontdisplaylinenum }%
     \var{{\devanagarifont \numemph\va\textbf{॰तीर्थमिडा}\lem \mssALL, ॰तीर्थ इडा \msCb}}% 
    \var{{\devanagarifont \numnoemph\vb\textbf{किङ्किणी॰}\lem \mssALL, चिञ्चिनी॰ \msCc\oo 
\textbf{॰रव॰}\lem \mssALL, ॰रवि॰ \msCbacorr, ॰राव॰ \Ed\oo 
\textbf{॰चिह्निता}\lem \mssALL, ॰चिह्निका \msCc, ॰चिह्नता \msNb}}% 

%Verse 10:20

{\devanagarifont तं तु श्रुत्वा न संदेहः सर्वपापक्षयो भवेत् {॥ १०:२०॥} \veg\dontdisplaylinenum }%
     \var{{\devanagarifont \numnoemph\vc\textbf{तं तु}\lem \corr, \uncl{तन्तु} \msCa, तन्तु \msCb\msCc\msNa\msNc\Ed, 
त\uncl{त्तु} \msNb\oo 
\textbf{न संदेहः}\lem \mssALL, वरारोहेः \msCc}}% 


\alalalfejezet{सूर्यतीर्थम्}

{\devanagarifont सूर्यतीर्थं सुषुम्ना च नीरवारवसंयुता \thinspace{\dandab} \dontdisplaylinenum }%
     \var{{\devanagarifont \numemph\va\textbf{॰तीर्थं}\lem \mssALL, ॰तीर्थ \msNb\oo 
\textbf{सुषुम्ना}\lem \mssALL, सुषुम्णा \Ed}}% 
    \var{{\devanagarifont \numnoemph\vb\textbf{नीरवा॰}\lem \Ed, वीरवा॰ \msCa\msCc, चीरवा॰ \msCb\msNa\msNb\msNc\oo 
\textbf{॰युता}\lem \msCa\msNa\msNc\Ed, ॰युतम् \msCb\msCc, ॰युतां \msNb}}% 

%Verse 10:21

{\devanagarifont श्रुतिमात्राद्विमुच्येत पापराशिर्महानपि {॥ १०:२१॥} \veg\dontdisplaylinenum }%
     \var{{\devanagarifont \numnoemph\vc\textbf{॰मात्रा॰}\lem \mssALL, ॰माता॰ \msCc}}% 


\alalalfejezet{अग्नितीर्थम्}

{\devanagarifont अग्नितीर्थार्जुना नाडी ब्रह्मघोषमनोरमा \thinspace{\dandab} \dontdisplaylinenum }%
     \var{{\devanagarifont \numemph\va\textbf{॰र्जुना}\lem \mssALL, ॰जुना \msCc, ॰र्जुनं \Ed}}% 
    \var{{\devanagarifont \numnoemph\vb\textbf{॰रमा}\lem \mssALL, ॰रमाः \msNc\Ed}}% 

%Verse 10:22

{\devanagarifont तत्तदक्षरमाकर्ण्य अमृतत्वाय कल्पते {॥ १०:२२॥} \veg\dontdisplaylinenum }%
     \var{{\devanagarifont \numnoemph\vc\textbf{॰कर्ण्य}\lem \mssALL, ॰र्ण्य \msCb}}% 
    \var{{\devanagarifont \numnoemph\vd\textbf{कल्पते}\lem \msCb\msNc\Ed, क\lk \lac\  \msCa, कल्प्यते \msCc\msNa\msNb}}% 


\alalalfejezet{पुष्करम्}

{\devanagarifont पुष्करं हृदि मध्यस्थमष्टपत्त्रं सकर्णिकम् \thinspace{\dandab} \dontdisplaylinenum }%
     \var{{\devanagarifont \numemph\vb\textbf{॰पत्त्रं}\lem \msCb\msNa\msNc\Ed, \lk\lk\ \msCa, ॰पत्र \msCc\msNb\oo 
\textbf{॰कर्णिकम्}\lem \mssALL, \lk\lk\lk\  \msCa, ॰कर्णिकाम् \Ed}}% 

%Verse 10:23

{\devanagarifont चिन्तयेत्सूक्ष्म तन्मध्ये जन्ममृत्युविनाशनम् {॥ १०:२३॥} \veg\dontdisplaylinenum }%
     \var{{\devanagarifont \numnoemph\vc\textbf{सूक्ष्म}\lem \mssALL, \uncl{सूक्ष्म} \msCa, सूक्ष्मं \Ed}}% 


\alalalfejezet{मानसम्}

{\devanagarifont मानससरमध्यस्थं स हंसः कमलोपरि \thinspace{\dandab} \dontdisplaylinenum }%
     \var{{\devanagarifont \numemph\va\textbf{मानस॰}\lem \msCb\msNa, \uncl{मानस} \msCa, मानसं \msCc\msNb\msNc\Ed}}% 
    \var{{\devanagarifont \numnoemph\vb\textbf{स हंसः}\lem \conj, सहंस॰ \msCa\msCc\msNa\msNb\msNc\Ed, सहसं \msCb}}% 

%Verse 10:24

{\devanagarifont सलीलो लीलयाचारी परतः परपारगः {॥ १०:२४॥} \veg\dontdisplaylinenum }%
     \var{{\devanagarifont \numnoemph\vc\textbf{सलीलो}\lem \mssALL, सलीला \Ed}}% 
    \var{{\devanagarifont \numnoemph\vd\textbf{परतः}\lem \mssALL, परत \msNb}}% 


\alalalfejezet{नैमिषम्}

{\devanagarifont नैमिषं शृणु देवेशि निमिषा प्रत्ययो भवेत् \thinspace{\dandab} \dontdisplaylinenum }%
     \var{{\devanagarifont \numemph\vb \lem \mssALL, निमि प्रत्ययो भवेत् \msCb, 
नि\lac \uncl{षो} प्रत्ययो \uncl{भवेत्} \msNb}}% 

%Verse 10:25

{\devanagarifont सम्यग्छायां निरीक्षेत आत्मानो वा परस्य वा {॥ १०:२५॥} \veg\dontdisplaylinenum }%
     \var{{\devanagarifont \numnoemph\vd\textbf{आत्मनो}\lem \mssALL, \lk न्मनो \msCa, स्वात्मानो \Ed\oo 
\textbf{परस्य वा}\lem \mssALL, परस्य च \Ed}}% 

{\devanagarifont आयतमङ्गुलीमात्रं निमिषाक्षिः स पश्यति \thinspace{\dandab} \dontdisplaylinenum }%
     \var{{\devanagarifont \numemph\va\textbf{आयतमङ्गुली॰}\lem \conj, आयतप्यङ्गुली॰ \mssCaCbCc\msNa\msNb, 
आयातप्यङ्गुली॰ \msNc\Ed\oo 
\textbf{॰मात्रं}\lem \mssALL, ॰मात्र \msNc, ॰मध्ये \Ed}}% 
    \var{{\devanagarifont \numnoemph\vb\textbf{॰क्षिः}\lem \eme, ॰क्षि \mssCaCbCc\msNa\msNb\msNc\Ed}}% 

%Verse 10:26

{\devanagarifont दृष्ट्वा प्रत्ययमेवं हि नैमिषज्ञः स उच्यते {॥ १०:२६॥} \veg\dontdisplaylinenum }%
     \var{{\devanagarifont \numnoemph\vd\textbf{नैमिषज्ञः}\lem \mssALL, नैमिसंज्ञः \msCb, नैमिषज्ञ \msCc}}% 


\alalalfejezet{बिन्दुसरः}

{\devanagarifont तीर्थं बिन्दुसरं नाम शृणु वक्ष्यामि सुन्दरि \thinspace{\dandab} \dontdisplaylinenum }%
     \var{{\devanagarifont \numemph\va\textbf{तीर्थं बिन्दु॰}\lem \mssALL, तीर्थमिन्दु॰ \Ed}}% 

%Verse 10:27

{\devanagarifont देहमध्ये हृदि ज्ञेयं हृदिमध्ये तु पङ्कजम् {॥ १०:२७॥} \veg\dontdisplaylinenum }%
     \var{{\devanagarifont \numnoemph\vc\textbf{हृदि ज्ञेयं}\lem \mssALL, \om\ \msCb}}% 
    \paral{{\devanagarifontsmall \vo {\englishfont \compare\ \NISVK\ 5.55:}
                 एतेषां नादमध्ये तु शिवं तत्र व्यवस्थितः\thinspace{\devanagarifontsmall ।}
                 हृदयं देहमध्ये तु तत्र पद्मं व्यवस्थितम्\thinspace{\devanagarifontsmall ॥} }}

{\devanagarifont कर्णिका पद्ममध्ये तु बिन्दुः कर्णिकमध्यतः \thinspace{\dandab} \dontdisplaylinenum }%
     \var{{\devanagarifont \numemph\va\textbf{॰मध्ये}\lem \mssALL, ॰ध्ये \msCa, ॰पध्ये \msNa}}% 

%Verse 10:28

{\devanagarifont बिन्दुमध्ये स्थितो नादः स नादः केन भिद्यते {॥ १०:२८॥} \veg\dontdisplaylinenum }%
     \var{{\devanagarifont \numnoemph\vc\textbf{बिन्दुमध्ये}\lem \mssALL, \uncl{बिन्दु}\lk\lk\ \msCa}}% 
    \var{{\devanagarifont \numnoemph\vd\textbf{भिद्यते}\lem \mssALL, \uncl{वि}द्यते \msCa, विद्यते \msCc}}% 
    \paral{{\devanagarifontsmall \vo {\englishfont \compare\ \NISVK\ 5.56:}
                 कर्णिका पद्ममध्ये तु अकारं तस्य मध्यतः\thinspace{\devanagarifontsmall ।}
                 तस्य मध्ये विनिष्क्रान्तं नादं परमदुर्लभम्\thinspace{\devanagarifontsmall ॥} }}

{\devanagarifont उकारं च मकारं च भित्त्वा नादो विनिर्गतः \thinspace{\dandab} \dontdisplaylinenum }%
     \var{{\devanagarifont \numemph\va\textbf{उकारं च मकारं}\lem \mssALL, उकारश्च मकारश् \Ed}}% 
    \paral{{\devanagarifontsmall \vab {\englishfont = \NISVK\ 5.57ab} }}

%Verse 10:29

{\devanagarifont तं विदित्वा विशालाक्षि सो ऽमृतत्वं लभेत च {॥ १०:२९॥} \veg\dontdisplaylinenum }%
     \var{{\devanagarifont \numnoemph\vd\textbf{सो ऽमृतत्वं}\lem \mssALL, सोम्यतत्वं \msCc, सोमतत्वं \Ed\oo 
\textbf{च}\lem \mssALL, वा \Ed}}% 


\alalalfejezet{सेतुबन्धम्}

\nemslokalong


\ujvers\nemsloka {
{\devanagarifont वक्ष्ये ते सेतुबन्धं दुरितमलहरं नादतोयप्रवाहं }%
  \dontdisplaylinenum}    \var{{\devanagarifont \numemph\va\textbf{ते}\lem \mssALL, \om\ \msCaacorr, हं \msCc\oo 
\textbf{॰बन्धं}\lem \mssALL, ॰बन्धूं \msCb\oo 
\textbf{॰तोय॰}\lem \mssALL, ॰तोयं \msNb}}% 


\nemslokab

{\devanagarifont जिह्वाकण्ठोरकूला स्वरगणपुलिनावर्तघोषा तरङ्गा  \danda\dontdisplaylinenum }%
     \var{{\devanagarifont \numnoemph\vb\textbf{॰कण्ठोर॰}\lem \conj, ॰कण्ठोरु॰ \mssCaCbCc\msNa\msNb\msNc\Ed\oo 
\textbf{स्वर॰}\lem \mssALL, सुर॰ \msCc\Ed}}% 

\nemslokac

{\devanagarifont कुम्भीराघोषमीना दशगणमकरा भीमनक्रा विसर्गा }%
  \dontdisplaylinenum    \var{{\devanagarifont \numnoemph\vc\textbf{॰मीना}\lem \mssALL, ॰माना \Ed\oo 
\textbf{दश॰}\lem \mssALL, \lk\lk\ \msCa\oo 
\textbf{विसर्गा}\lem \mssCaCbCc, विसर्गाः \msNa\msNb\msNc\Ed}}% 

%Verse 10:30


\nemslokad

{\devanagarifont सानुस्वारे गभीरे मदसुखरसनं सेतुबन्धं व्रजस्व {॥ १०:३०॥} \veg\dontdisplaylinenum }%
     \var{{\devanagarifont \numnoemph\vd\textbf{॰स्वारे}\lem \msCa\msCb\msNc\Ed, ॰सारे \msCc, 
॰स्वारो \msNa, ॰स्वा\uncl{रेण} \msNb\ \unmetr\oo 
\textbf{गभीरे}\lem \msCa\msCb\msNc, गम्भीरे \msCc\msNb\Ed, \uncl{गं}भीरे \msNa\oo 
\textbf{॰रसनं}\lem \mssALL, ॰रमणं \Ed\oo 
\textbf{॰बन्धं}\lem \mssALL, ॰बन्ध \msCb\oo 
\textbf{व्रजस्व}\lem \mssALL, रमस्व \Ed}}% 


\alalalfejezet{सुरद्रहः}

\nemslokalong


\ujvers\nemsloka {
{\devanagarifont सप्तद्वीपान्तमध्ये शृणु शशिवदने सर्वदुःखान्तलाभम् }%
  \dontdisplaylinenum}    \var{{\devanagarifont \numemph\va\textbf{॰द्वीपा॰}\lem \mssALL, ॰दीपा॰ \msNc}}% 


\nemslokab

{\devanagarifont ईशानेनाभिजुष्टं हृदि ह्रद विमलं नादशीताम्बुपूर्णम्  \danda\dontdisplaylinenum }%
     \var{{\devanagarifont \numnoemph\vb\textbf{ईशानेनाभिजुष्टं}\lem \msCc\msNa\msNc\Ed, ईशानेनाभिदुष्टं \msCa\msNb, 
ईशानेभिदुष्टं \msCbacorr, ईशानेभि\lac  दुष्टं \msCbpcorr\oo 
\textbf{विमलं नादशीता॰}\lem \mssALL, 
विमलान्नादशीता॰ \msNb, विमलं नामशिता॰ \Ed}}% 

\nemslokac

{\devanagarifont तत्रैकं जातपद्मं प्रकृतिदलयुतं केशरं शक्तिभिन्नं }%
  \dontdisplaylinenum    \var{{\devanagarifont \numnoemph\vc\textbf{केशरं}\lem \msCb\Ed, केशर॰ \msCa\msCc\msNa\msNc\ \unmetr, केश्वर॰ \msNb\ \unmetr}}% 

%Verse 10:31


\nemslokad

{\devanagarifont पञ्चव्योमप्रशस्तं गतिपरमपदं प्राप्तुकामेन सेव्यम् {॥ १०:३१॥} \veg\dontdisplaylinenum }%
     \var{{\devanagarifont \numnoemph\vd\textbf{॰व्योम॰}\lem \mssALL, ॰व्यो\uncl{मं} \msNa\oo 
\textbf{॰शस्तं ग॰}\lem \mssALL, ॰शस्वङ्ग॰ \msCc\oo 
\textbf{॰परम॰}\lem \mssALL, ॰परमं \msNa\ \unmetr\oo 
\textbf{सेव्यम्}\lem \mssALL, सर्वम् \Ed}}% 


\alalalfejezet{घण्टिकेश्वरम्}

\nemslokalong


\ujvers\nemsloka {
{\devanagarifont †नाड्यैकासङ्गतानि† निपतितममृतं घण्टिकापारकेण }%
  \dontdisplaylinenum}    \var{{\devanagarifont \numemph\va\textbf{निपतितममृतं}\lem \mssALL, निपतितममृत॰ \msNa\ \unmetr, 
नि\lac  तममृतं \msNb\oo 
\textbf{॰पारकेण}\lem \msCa\msCb\msNa\msNc, ॰याङ्करेण \msCc\Ed, ॰\uncl{पारकेन} \msNb}}% 


\nemslokab

{\devanagarifont तृप्यन्ते तेन नित्यं हृदि कमलपुटं स्थाणुभूतान्तरात्मा  \danda\dontdisplaylinenum }%
     \var{{\devanagarifont \numnoemph\vb\textbf{॰पुटं}\lem \mssALL, ॰पुट \msCb\oo 
\textbf{स्थाणु॰}\lem \conj, स्थानु॰ \mssCaCbCc\msNa\msNc, 
\uncl{स्थान}॰ \msNb, स्थान॰ \Ed}}% 

\nemslokac

{\devanagarifont यं पश्यन्तीशभक्ताः कलिकलुषहरं व्यापिनं निष्प्रपञ्चं }%
  \dontdisplaylinenum    \var{{\devanagarifont \numnoemph\vc\textbf{यं पश्यन्तीशभक्ताः}\lem \msNa, यं पश्यन्तीशभक्ता \msCa\msNb, 
यं पश्यन्तीशभर्त्ताः \msCb, यं पस्यन्तीसभक्त्या \msCc, 
यत्पश्यन्तीशभक्त्या \msNc, यं पश्यन्नीशमक्षा \Ed\oo 
\textbf{॰प्रपञ्चम्}\lem \msCa\msNa\msNb\msNc, ॰प्रपञ्च \msCb\msCc\Ed}}% 

%Verse 10:32


\nemslokad

{\devanagarifont देवेशं घण्टिकेशामरभवमभवं तीर्थमाकाशबिन्दुम् {॥ १०:३२॥} \veg\dontdisplaylinenum }%
     \var{{\devanagarifont \numnoemph\vd\textbf{देवेशं}\lem \msCb\msNb\Ed, देव्येशं \msCa\msCc\msNa, देव्येश \msNc\oo 
\textbf{घण्टिकेशामर॰}\lem \msCc, घण्टिकेशमर॰ \msCa\msCb\msNb\msNc, 
घण्टिकेशं मर॰ \msNa, घाण्टकेशामर॰ \Ed\oo 
\textbf{॰भवं तीर्थम्}\lem \eme, ॰भवन्तीर्थम् \msCb\msCc\msNa\msNb\msNc\Ed, भव\lk\lk र्थम् \msCa\oo 
\textbf{॰बिन्दुम्}\lem \mssALL, ॰बिन्दु \msCc}}% 


\alalalfejezet{वागीश्वरतीर्थम्}

\nemslokalong


\ujvers\nemsloka {
{\devanagarifont मीमांसारत्नकूला क्रमपदपुलिना शैवशास्त्रार्थतोया }%
  \dontdisplaylinenum}    \var{{\devanagarifont \numemph\va\textbf{शैव॰}\lem \mssALL, शर्व॰ \Ed}}% 


\nemslokab

{\devanagarifont मीनौघा पञ्चरात्रं श्रुतिकुटिलगतिः स्मार्तवेगा तरङ्गा  \danda\dontdisplaylinenum }%
     \var{{\devanagarifont \numnoemph\vb\textbf{मीनौघा॰}\lem \msNa\msNb\Ed,  मीनोघा॰ \mssCaCbCc\msNc\oo 
\textbf{पञ्चरात्रं}\lem \mssALL, पञ्चशत्रं \Ed\oo 
\textbf{॰गतिः}\lem \corr, ॰गति \mssCaCbCc\msNa\msNb\msNc\Ed\oo 
\textbf{॰स्मार्तवेगा तरङ्गा}\lem \mssALL, ॰स्मा\lac  \uncl{वेगा तरङ्गा} \msNb, 
॰स्मार्तवेगास्तरङ्गा \Ed}}% 

\nemslokac

{\devanagarifont योगावर्तातिशोभा उपनिषदिवहा भारतावर्तफेना }%
  \dontdisplaylinenum    \var{{\devanagarifont \numnoemph\vc\textbf{॰वहा भारता॰}\lem \mssALL, महाभारता॰ \msNb}}% 

%Verse 10:33


\nemslokad

{\devanagarifont पञ्चाशद्व्योमरूपी रसभवननदी तीर्थ वागीश्वरीयम् {॥ १०:३३॥} \veg\dontdisplaylinenum }%
     \var{{\devanagarifont \numnoemph\vd\textbf{॰शद्व्योम॰}\lem \mssALL, ॰शव्योम॰ \msNa, ॰सद्व्योम॰ \Ed}}% 

\nemslokalong


\ujvers\nemsloka {
{\devanagarifont यस्तं वेत्ति स वेत्ति वेदनिखिलं संसारदुःखच्छिदं }%
  \dontdisplaylinenum}    \var{{\devanagarifont \numemph\va\textbf{यस्तं}\lem \mssALL, यस्त॰ \msCa\msCb\oo 
\textbf{स वेत्ति}\lem \mssALL, \uncl{न} वेत्ति \msNc}}% 


\nemslokab

{\devanagarifont जन्मव्याधिवियोगतापमरणं क्लेशार्णवं दुःसहम्  \danda\dontdisplaylinenum }%
     \var{{\devanagarifont \numnoemph\vb\textbf{॰मरणं}\lem \mssALL, ॰मरण \msNc\oo 
\textbf{॰र्णवं}\lem \mssALL, ॰ण्णवं \msNa, ॰र्णव \Ed}}% 

\nemslokac

{\devanagarifont गर्भावासमतीव सह्यविषयं दुस्तीर्यदुःखालयं }%
  \dontdisplaylinenum    \var{{\devanagarifont \numnoemph\vc\textbf{गर्भावासम्}\lem \mssALL, गर्भोवासम् \Ed\oo 
\textbf{॰विषयं}\lem \msCa\msCb\msNb, ॰विषमं \msCc\msNa\msNc\Ed\oo 
\textbf{॰लयम्}\lem \mssALL, ॰लय\uncl{ः} \msNa\oo 
\textbf{दुस्तीर्य॰}\lem \mssALL, दुस्तीर्यः \msNc}}% 

%Verse 10:34


\nemslokad

{\devanagarifont प्राप्तं तेन न संशयः शिवपदं दुष्प्राप्य देवैरपि {॥ १०:३४॥} \veg\dontdisplaylinenum }%
     \var{{\devanagarifont \numnoemph\vd \lem \msCa\msCbpcorr\msNa\msNc, 
प्राप्तं तेन न संशयः   शिवदं दुष्प्राप्य देवैरपि \msCbacorr, 
प्राप्तं तेन न संशयं शिवपदं दुष्प्राप्य देवैरपि \msCc\Ed, 
प्रा\lac  \uncl{यः शिव} \lk\lk \lk\lk  \uncl{य देवैरपि} \msNb}}% 

\vers


{\devanagarifont 
\jump
\begin{center}
\ketdanda~इति वृषसारसंग्रहे कायतीर्थोपवर्णनो नामाध्यायो दशमः~\ketdanda
\end{center}
\dontdisplaylinenum\vers  }%
     \var{{\devanagarifont \numnoemph{\englishfont \Colo:}\textbf{कायतीर्थोपवर्णनो}\lem \mssALL, 
कायती\lk\lk \lk र्ण्णनो \msCa\oo 
\textbf{नामाध्यायो दशमः}\lem \mssALL, नाम दशमो ऽध्यायः \Ed}}% 

\nemslokanormal

