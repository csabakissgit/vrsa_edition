\fejno=0\versno=0
\centerline{\Huge\devanagarifontbold वृषसारसंग्रहः  }

 \versno=0\fejno=18
\thispagestyle{empty}

\fancyhead[CO]{{\footnotesize\devanagarifont वृषसारसंग्रहे  }}
\fancyhead[CE]{{\footnotesize\devanagarifont अष्टादशमो ऽध्यायः  }}
\fancyhead[LE]{}
\fancyhead[RE]{}
\fancyhead[LO]{}
\fancyhead[RO]{}
\szam\bek

\centerline{\Large\devanagarifontbold [   अष्टादशमो ऽध्यायः  ]}{\vrule depth10pt width0pt} 

\alalfejezet{स्वर्गान्मर्त्यमुपागतानां चिह्नानि }
 
\vers


{\devanagarifontbold देव्युवाच {\dandab}\dontdisplaylinenum  }%
     \lacuna{\devanagarifont {\englishfont Testimonia for this chapter---\msCa: f.~224r line 4 -- f.~226r line 4; 
                              \msCb: f.~226v line 6 -- f.~228r line 6;
                              \msCc: f.~306r line 1 -- f.~306v line 5;
                              \msNa: f.~31v line 1 --  f.~33r line 6;
                              \msM:  f.~33v line 2 -- f.~35v line 4;
                              \msPaperA: f.~231r line 2 -- f.~232v line 7;
                              \Ed: pp.~649--651} }%
  
\nemsloka 
{\devanagarifontbold भुक्त्वा तु भोगान्सुचिरं यथेष्टं }%
  \dontdisplaylinenum    \lacuna{\devanagarifont \vo {\englishfont \msCc\ broke off in chapter 14 and resumes below at 18.28b.}
                    {\englishfont The gap in \Ed\ that started at 17.39 continues up to 18.16c.} }%
      \var{{\devanagarifont \numemph\va तु\lem \mssALL; \om\ \msNa, च \msM}}% 


\nemslokab

{\devanagarifontbold पुण्यक्षयान्मर्त्यमुपागतानाम्  \danda\dontdisplaylinenum }%
     \var{{\devanagarifont \numnoemph\vb ॰क्षयान्म॰\lem \mssALL; ॰क्षया म॰ \msM}}% 

\nemslokac

{\devanagarifontbold चिह्नानि तेषां कथयस्व मे ऽद्य }%
  \dontdisplaylinenum    \var{{\devanagarifont \numnoemph\vc चिह्नानि\lem \mssALL; किञ्चिह्न \msM}}% 


\nemslokad

{\devanagarifontbold यथाक्रमं कर्मफलं विशेषात् {॥१८:१॥} \veg\dontdisplaylinenum }%
     \var{{\devanagarifont \numnoemph\vd विशेषात्\lem \mssALL; विशेषः \msM}}% 


\alalfejezet{दानाष्टकम् }
 
\vers


{\devanagarifontbold महेश्वर उवाच {\dandab}\dontdisplaylinenum  }%
     \var{{\devanagarifont \numemph\vo महेश्वर\lem \msCa\msCb; भगवान् \msNa\msM\msPaperA}}% 

\nemsloka 
{\devanagarifontbold सदान्नदाता कृपणार्तिदीनां }%
  \dontdisplaylinenum    \var{{\devanagarifont \numnoemph\va ॰दीनां\lem \mssALL; ॰दीना \msM}}% 


\nemslokab

{\devanagarifontbold स वर्षकोट्यायुतमीशलोके  \danda\dontdisplaylinenum }%
     \var{{\devanagarifont \numnoemph\vb ॰युतमीशलोके\lem \mssALL; 
॰युतमीनशलोके \msCaacorr, ॰युतस्वर्गलोके \msM}}% 

\nemslokac

{\devanagarifontbold भुक्त्वा च भोगान्सममप्सरोभिः }%
  \dontdisplaylinenum    \var{{\devanagarifont \numnoemph\vc ॰न्सममप्सरोभिः\lem \mssALL; ॰न्सहरप्सरोभिः \msM, ॰न्सरोभिः \msPaperA}}% 


\nemslokad

{\devanagarifontbold प्रक्षीणपुण्यः पुनरेति मर्त्यम् {॥१८:२॥} \veg\dontdisplaylinenum }%
     \var{{\devanagarifont \numnoemph\vd ॰पुण्यः पुनरेति मर्त्यम्\lem \mssALL; ॰पुण्यः पुनरेति मर्त्ये \msNa, 
॰पुण्य पुनः मर्त्यलोके \msM}}% 

\ujvers\nemsloka {
{\devanagarifontbold जायन्ति दिव्येषु कुलेषु पुंसः }%
  \dontdisplaylinenum}    \var{{\devanagarifont \numemph\va \lem \msCa\msNa; 
\lac\ \msCb, 
जायन्ति ते दिव्यकुलेषु पुन्साम् \msM, 
जायन्ति ते दिव्ये कुलेषु पुंसः \msPaperA\ \unmetr}}% 


\nemslokab

{\devanagarifontbold सस्त्रीसमृद्धे बहुभृत्यपूर्णे  \danda\dontdisplaylinenum }%
 
\nemslokac

{\devanagarifontbold गौरश्वरत्नादिधनाकुलेषु }%
  \dontdisplaylinenum    \var{{\devanagarifont \numnoemph\vc गौर॰\lem \msCapcorr\msNa\msPaperA; गौरव॰ \msCaacorr, गोर॰ \msCb\msM\oo 
॰रत्नादि॰\lem \msNa\msM; ॰रन्नादि॰ \msCa\msPaperA, ॰रत्नाति॰ \msCb\oo 
॰धना॰\lem \mssALL; ॰समा॰ \msM}}% 


\nemslokad

{\devanagarifontbold रूपोज्ज्वलः कान्तिसमायुतश्च {॥१८:३॥} \veg\dontdisplaylinenum  }%
     \var{{\devanagarifont \numnoemph\vd रूपोज्ज्वलः\lem \eme; रूपोज्ज्वल॰ \msCa\msCb\msNa, रूपोज्वलः \msM, रूपर्ज्वल॰ \msPaperA\oo 
॰समायुतश्च\lem \msCapcorr\msNa; ॰समायुतञ्च \msCaacorr\msCb\msM}}% 

\ujvers\nemsloka {
{\devanagarifontbold वस्त्रं सुसत्कृत्य द्विजस्य दानात् }%
  \dontdisplaylinenum}    \var{{\devanagarifont \numemph\va वस्त्रं सुसत्कृत्य\lem \msCa\msCb; वस्त्रं सुसंस्कृत्य \msNa, 
सुवस्त्र सत्कृत्य \msM, 
वस्त्रं सुसंकृत्य \msPaperA}}% 


\nemslokab

{\devanagarifontbold स्वर्गेषु मोदन्ति स वर्षकोट्यः  \danda\dontdisplaylinenum }%
     \var{{\devanagarifont \numnoemph\vb ॰कोट्यः\lem \mssALL; ॰कोट्या \msM}}% 

\nemslokac

{\devanagarifontbold पुनश्च ते मर्त्यमुपागताश्च }%
  \dontdisplaylinenum    \var{{\devanagarifont \numnoemph\vc \lem \msCa\msCb\msNa\msPaperA; 
पुनश्च्युता मर्त्यमुपा$\-$गतानां \msM}}% 


\nemslokad

{\devanagarifontbold चिह्नं महच्छ्रीपदमाप्नुवन्ति {॥१८:४॥} \veg\dontdisplaylinenum }%
     \var{{\devanagarifont \numnoemph\vd चिह्नं म॰\lem \mssALL; चिह्न\uncl{म्म}॰ \msCa, चिह्नं न॰ \msM}}% 

\ujvers\nemsloka {
{\devanagarifontbold कूपप्रपापुष्करिणीप्रदाता }%
  \dontdisplaylinenum}    \var{{\devanagarifont \numemph\va कूप॰\lem \mssALL; कूपं \msM\oo 
॰पुष्करिणी॰\lem \msNa\msPaperA; ॰पुष्करणी॰ \msCa, ॰पुष्किरणी॰ \msCb, ॰पुष्किरिणी॰ \msM}}% 


\nemslokab

{\devanagarifontbold स लोकमाप्नोति जलेश्वरस्य  \danda\dontdisplaylinenum }%
     \var{{\devanagarifont \numnoemph\vb जलेश्वरस्य\lem \msCa\msCb\msNa\msPaperA; जलेःस्वरस्य \msM}}% 

\nemslokac

{\devanagarifontbold ततः स तस्माच्च्युतिमाप्य लोकात् }%
  \dontdisplaylinenum    \var{{\devanagarifont \numnoemph\vc \lem \msCa\msPaperA; 
ततः स तस्माच्च्युतिमप्य लोकात् \msCb, 
ततः स तस्माच्च्युतिमाप्य लोके \msNa, 
तस्मात्स लोकाश्च्युत मर्त्यलोके \msM}}% 


\nemslokad

{\devanagarifontbold सुखी सुतृप्तेषु कुलेषु जायेत् {॥१८:५॥} \veg\dontdisplaylinenum }%
     \var{{\devanagarifont \numnoemph\vd ॰तृप्तेषु\lem \mssALL; ॰तृ\lac\ \msCb, ॰तप्तेषु \msNa\oo 
कुलेषु जायेत्\lem \mssALL; च जायते सः \msM}}% 

\ujvers\nemsloka {
{\devanagarifontbold रत्निप्रमाणादपि हेमदानात् }%
  \dontdisplaylinenum}    \var{{\devanagarifont \numemph\va रत्नि॰\lem \msCa\msNa; रति॰ \msCb, रत्ति॰ \msM, रत्न॰ \msPaperA\oo 
॰प्रमाणा॰\lem \mssALL; ॰प्रमाना॰ \msM\oo 
॰दानात्\lem \mssALL; ॰दाता \msM}}% 


\nemslokab

{\devanagarifontbold सुरेन्द्रलोकं समवाप्नुवन्ति  \danda\dontdisplaylinenum }%
     \var{{\devanagarifont \numnoemph\vb समवाप्नुवन्ति\lem \mssALL; समुवाप्नुवन्ति \msCb, चमवाप्नुवन्ति \msM}}% 

\nemslokac

{\devanagarifontbold तस्माच्च्युतो मर्त्यमुपागतानां }%
  \dontdisplaylinenum    \var{{\devanagarifont \numnoemph\vc तस्माच्च्युतो\lem \mssALL; तस्माच्च्युते \msM}}% 


\nemslokad

{\devanagarifontbold चिह्नं समृद्धिर्धनधान्यलक्ष्म्याः {॥१८:६॥} \veg\dontdisplaylinenum }%
     \var{{\devanagarifont \numnoemph\vd चिह्नं\lem \mssALL; चिह्न \msCa\oo 
॰लक्ष्म्याः\lem \msNa; लक्ष्याः \msCa\msM, लक्ष्मः \msCb, ल\uncl{क्ष्या} \msPaperA\oo 
समृद्धिर्ध॰\lem \mssALL; समृद्ध्यध॰ \msM}}% 

\ujvers\nemsloka {
{\devanagarifontbold अदूष्यभूमीवरविप्रदानात् }%
  \dontdisplaylinenum}    \var{{\devanagarifont \numemph\va ॰दानात्\lem \mssALL; ॰दाता \msM}}% 


\nemslokab

{\devanagarifontbold स लोकमाप्नोति सुरेश्वरस्य  \danda\dontdisplaylinenum }%
     \var{{\devanagarifont \numnoemph\vb लोकमाप्नोति\lem \mssALL; लोक प्राप्नोति \msPaperA\oo 
सुरे॰\lem \mssALL; स्वरे॰ \msM}}% 

\nemslokac

{\devanagarifontbold भुक्त्वा तु भोगान्च्युत मर्त्यलोके }%
  \dontdisplaylinenum    \var{{\devanagarifont \numnoemph\vc भुक्त्वा भोगान्च्युत\lem \msCa\msCb\msNa; 
स भुक्तभोगां च्युत \msM, 
भुक्त्वा भोगानच्युत \msPaperA}}% 


\nemslokad

{\devanagarifontbold चिह्नं लभेद्वै विषयाधिपत्वम् {॥१८:७॥} \veg\dontdisplaylinenum }%
     \var{{\devanagarifont \numnoemph\vd लभेद्वै\lem \msCa\msPaperA; लभेद्वैद् \msCb, भवेद्वै \msNa, भवेतद् \msM}}% 

\ujvers\nemsloka {
{\devanagarifontbold द्विजस्य सत्कृत्य तिलप्रदाता }%
  \dontdisplaylinenum}    \var{{\devanagarifont \numemph\va तिल॰\lem \mssALL; तिलः \msM}}% 


\nemslokab

{\devanagarifontbold स लोकमाप्नोति च केशवस्य  \danda\dontdisplaylinenum }%
     \var{{\devanagarifont \numnoemph\vb केशवस्य\lem \mssALL; वासवेभ्यः \msM}}% 

\nemslokac

{\devanagarifontbold भ्रष्टस्ततो मर्त्यमुपागतस्तु }%
  \dontdisplaylinenum    \var{{\devanagarifont \numnoemph\vc ॰गतस्तु\lem \mssALL; ॰गतस्य \msM}}% 


\nemslokad

{\devanagarifontbold चिह्नं लभेदक्षयमर्थलाभम् {॥१८:८॥} \veg\dontdisplaylinenum }%
     \var{{\devanagarifont \numnoemph\vd लभेद॰\lem \mssALL; भवेद॰ \msNa, नृणाञ्चो \msM}}% 

\ujvers\nemsloka {
{\devanagarifontbold गवां सुरूपां विधिवद्द्विजानां }%
  \dontdisplaylinenum}    \var{{\devanagarifont \numemph\va गवां सुरूपां\lem \msPaperA; गवां स्वरूपां \msCa\msNa, गवां स्वरूपं \msCb, गवा सुरूपा \msM}}% 


\nemslokab

{\devanagarifontbold दत्त्वा च गोलोकमवाप्नुवन्ति  \danda\dontdisplaylinenum }%
     \var{{\devanagarifont \numnoemph\vb च\lem \mssALL; स \msM}}% 

\nemslokac

{\devanagarifontbold कल्पावसाने समुपेत्य मर्त्ये }%
  \dontdisplaylinenum    \var{{\devanagarifont \numnoemph\vc समुपेत्य मर्त्ये\lem \mssALL; पुनः मर्त्यलोके \msM}}% 


\nemslokad

{\devanagarifontbold चिह्नं गवाढ्यं शतगोयुतं च {॥१८:९॥} \veg\dontdisplaylinenum }%
     \var{{\devanagarifont \numnoemph\vd चिह्नं\lem \mssALL; चिह्न \msPaperA}}% 

\ujvers\nemsloka {
{\devanagarifontbold स्वर्गं गतानां पुरुषस्य चिह्नं }%
  \dontdisplaylinenum}    \var{{\devanagarifont \numemph\va स्वर्गं गतानां\lem \mssALL; स्वर्गागतानां \msM}}% 


\nemslokab

{\devanagarifontbold धनाढ्यता श्री सुखभोगलाभम्  \danda\dontdisplaylinenum }%
     \var{{\devanagarifont \numnoemph\vb ॰ढ्यता\lem \mssALL; ॰ढ्यतां \msM}}% 

\nemslokac

{\devanagarifontbold आयुर्यशोरूपकलत्रपुत्रं }%
  \dontdisplaylinenum    \var{{\devanagarifont \numnoemph\vc ॰र्यशो॰\lem \mssALL; ॰र्यषे॰ \msM}}% 


\nemslokad

{\devanagarifontbold सम्पद्विभूतिकुलकीर्तिमर्थम् {॥१८:१०॥} \veg\dontdisplaylinenum }%
     \var{{\devanagarifont \numnoemph\vd ॰मर्थम्\lem \mssALL; ॰मत्वम् \msM}}% 


\alalfejezet{निरयान्मर्त्यमुपागतानां चिह्नानि }
 
\ujvers\nemsloka {
{\devanagarifontbold दानाष्टकं चोत्तम कीर्तितं ते }%
  \dontdisplaylinenum}    \var{{\devanagarifont \numemph\va चोत्तम\lem \mssALL; चोतम \msM\oo 
कीर्तितं ते\lem \eme; कीर्तनं ते \msCa\msNa, \uncl{कीर्तितन्ते} \msCb, कीर्त्तितो यम् \msM, 
कीर्त्तिनन्ते \msPaperA}}% 


\nemslokab

{\devanagarifontbold चिह्नं च लोकं च समासतो मे  \danda\dontdisplaylinenum }%
     \var{{\devanagarifont \numnoemph\vb चिह्नं च\lem \mssALL; चिंह्नं स \msM\oo 
समासतो\lem \mssALL; समागतो \msM}}% 

\nemslokac

{\devanagarifontbold शृणोतु देवी निरयागतानां }%
  \dontdisplaylinenum    \var{{\devanagarifont \numnoemph\vc शृणोतु देवी\lem \mssALL; \uncl{शृण्वन्तु देवो} \msM}}% 


\nemslokad

{\devanagarifontbold चिह्नं च कर्मं च विपाकतां च {॥१८:११॥} \veg\dontdisplaylinenum }%
     \var{{\devanagarifont \numnoemph\vd चिह्नं च\lem \mssALL; चिंह्नं स्व॰ \msM\oo 
विपाकतां च\lem \mssALL; विपाकतानाम् \msM}}% 

\ujvers\nemsloka {
{\devanagarifontbold हत्वा च विप्रं मनसा च वाचा }%
  \dontdisplaylinenum}    \var{{\devanagarifont \numemph\va विप्रं\lem \mssALL; विप्र \msM\msPaperA}}% 


\nemslokab

{\devanagarifontbold स याति पारं निरयस्य घोरम्  \danda\dontdisplaylinenum }%
     \var{{\devanagarifont \numnoemph\vb \lem \msCa\msCb\msNa; 
स यान्ति पारं णिरयं सुघोरम् \msM, 
स याति पारं निरयश्च घोरम् \msPaperA}}% 

\nemslokac

{\devanagarifontbold अशीतिकल्पं निरये क्रमेण }%
  \dontdisplaylinenum    \var{{\devanagarifont \numnoemph\vc निरये\lem \mssALL; निरयः \msM}}% 


\nemslokad

{\devanagarifontbold भुक्त्वा पुनस्तिर्य शतायुतानाम् {॥१८:१२॥} \veg\dontdisplaylinenum }%
     \var{{\devanagarifont \numnoemph\vd पुनस्तिर्य\lem \mssALL; पुनः तिय \msM}}% 

\ujvers\nemsloka {
{\devanagarifontbold जायन्ति ते मानुष हीनविद्याः }%
  \dontdisplaylinenum}    \var{{\devanagarifont \numemph\va ॰विद्याः\lem \msCb\msM; ॰विद्या \msCa\msNa\msPaperA}}% 


\nemslokab

{\devanagarifontbold प्रत्यन्तवासाः कुलवित्तहीनाः  \danda\dontdisplaylinenum }%
     \var{{\devanagarifont \numnoemph\vb ॰वासाः\lem \mssALL; ॰वासा \msNa, ॰वासी \msM\oo 
हीनाः\lem \mssALL; ॰हीना \msM}}% 

\nemslokac

{\devanagarifontbold नित्यं च तस्याक्षयरोगपीडा }%
  \dontdisplaylinenum    \paral{{\devanagarifont \vb {\englishfont \compare\ \YAJNS\ 3.209a:}
                 ब्रह्महा क्षयरोगी स्यात् }}


\nemslokad

{\devanagarifontbold इदं तु चिह्नं द्विजजीवहर्तुः {॥१८:१३॥} \veg\dontdisplaylinenum }%
     \var{{\devanagarifont \numnoemph\vd इदं तु चिह्नं\lem \mssALL; चिंह्नञ्च मे त \msM}}% 

\ujvers\nemsloka {
{\devanagarifontbold पीत्वा च मद्यं द्विज कामतो वा }%
  \dontdisplaylinenum}    \var{{\devanagarifont \numemph\va मद्यं\lem \mssALL; मद्य \msM\oo 
द्विज\lem \msNa\msM; द्विजः \msCa\msCb\msPaperA\ \unmetr\oo 
वा\lem \mssALL; वे \msPaperA}}% 


\nemslokab

{\devanagarifontbold आघ्राति गन्धं स्वमनीषिकेण  \danda\dontdisplaylinenum }%
     \var{{\devanagarifont \numnoemph\vb आघ्राति\lem \mssALL; माघ्राति \msM\oo 
॰मनीषिकेण\lem \mssALL; ॰मनीशिकेन \msM, ॰मणीषकेण \msPaperA}}% 

\nemslokac

{\devanagarifontbold स याति घोरं नरकमसह्यं }%
  \dontdisplaylinenum    \var{{\devanagarifont \numnoemph\vc \lem \mssALL; 
स यान्ति घोरा नरकर्मसह्यं \msM}}% 


\nemslokad

{\devanagarifontbold यावच्च कल्पं दश अत्र भुक्त्वा {॥१८:१४॥} \veg\dontdisplaylinenum }%
     \var{{\devanagarifont \numnoemph\vd कल्पं दश अत्र\lem \mssALL; कल्पा दषमन्त्र \msM}}% 

\ujvers\nemsloka {
{\devanagarifontbold तिर्यं च सर्वमनुभूय दुःखं }%
  \dontdisplaylinenum}    \var{{\devanagarifont \numemph\va सर्वमनुभूय दुःखं\lem \mssALL; सर्वमनुभूय दुःख \msNa, 
सर्वं मनुभूय दुःखा \msM}}% 


\nemslokab

{\devanagarifontbold स कष्टकष्टेन मनुष्यजन्म  \danda\dontdisplaylinenum }%
     \var{{\devanagarifont \numnoemph\vb कष्टकष्टेन\lem \mssALL; ष्टकष्टेन \msCb, कष्टकष्टेन \msM\oo 
॰जन्म\lem \mssALL; ॰जन्मम् \msM}}% 

\nemslokac

{\devanagarifontbold चण्डालशौनश्वपचत्वमेति }%
  \dontdisplaylinenum    \var{{\devanagarifont \numnoemph\vc चण्डाल॰\lem \mssALL; चाण्डाल॰ \msM\msPaperA\oo 
॰त्वमेति\lem \mssALL; ॰त्वमे \msCb}}% 


\nemslokad

{\devanagarifontbold श्यामं च तालु भवतीह चिह्नम् {॥१८:१५॥} \veg\dontdisplaylinenum }%
     \var{{\devanagarifont \numnoemph\vd श्यामं\lem \mssALL; श्यामः \msPaperA\oo 
भवतीह\lem \mssALL; भवतीति ह \msPaperA\oo 
चिह्नम्\lem \mssALL; चिंह्नं \msM}}% 
    \paral{{\devanagarifont \vb {\englishfont \compare\ \YAJNS\ 3.209b:}
                         सुरापः श्यावदन्तकः }}

\ujvers\nemsloka {
{\devanagarifontbold निन्दन्ति ये वेद {\englishfont †}सम्भूय{\englishfont †} जिह्वा }%
  \dontdisplaylinenum}    \var{{\devanagarifont \numemph\va \lem \msCa\msNa\msPaperA; 
निन्दन्ति ये वेद \uncl{सम्भूय जिह्वा} \msCb, 
निन्दन्ति ये वेद सस्त्रपजिह्वा \msNb, 
निन्दन्ति ये वद शस्त्राय जिह्वा \msNc, 
निन्दन्ति यो वेद स भूय जिह्वा \msM}}% 
    \paral{{\devanagarifont \vo \compare\ {\englishfont \MANU\ 11.57:}
                 ब्रह्मोज्झता वेदनिन्दा कौटसाक्ष्यं सुहृद्वधः\thinspace{\devanagarifont ।}
                 गर्हितानाद्ययोर्जग्धिः सुरापान$\-$समानि षट्\thinspace{\devanagarifont ॥} }}


\nemslokab

{\devanagarifontbold यः कूटसाक्षी स च खल्वलान्धौ  \danda\dontdisplaylinenum }%
     \var{{\devanagarifont \numnoemph\vb यः कूटसाक्षी\lem \mssALL; यः कूटसाक्ष \msCb, यो कूटसा\lk क्षी \msM\oo 
खल्वलान्धौ\lem \msCa\msCb\msNa\msPaperA; ख\uncl{स्पृ}लत्वौ \msM}}% 

\nemslokac

{\devanagarifontbold सुहृद्वधा मृत्युशतं हि गर्भे }%
  \dontdisplaylinenum    \lacuna{\devanagarifont \vc {\englishfont \Ed\ resumes here on p.~649 with} महो {\englishfont (for} मृत्यु{\englishfont ; about two folio sides seem to be missing)}. }%
      \var{{\devanagarifont \numnoemph\vc मृत्यु॰\lem \mssALL; महो॰ \Ed}}% 


\nemslokad

{\devanagarifontbold गर्हाशनोच्छिष्टभुजो भवन्ति {॥१८:१६॥} \veg\dontdisplaylinenum }%
     \var{{\devanagarifont \numnoemph\vd गर्हा॰\lem \mssALL; गर्भा॰ \msM\oo 
भवन्ति\lem \mssALL; भ्वन्ति भुङ्क्त्वा महद्दुःख \Ed}}% 

\ujvers\nemsloka {
{\devanagarifontbold स्तैन्यं तु यः कुर्वति पापसत्त्वं }%
  \dontdisplaylinenum}    \var{{\devanagarifont \numemph\va स्तैन्यं तु\lem \corr; स्तैन्यस्तु \msCa\msNa\msPaperA, 
\gap न्य\uncl{स्तु} \msCb, 
स्तैन्यञ्च \msM, 
दैत्यस्तु \Ed\oo 
यः\lem \msCb\msM\Ed; ये \msCapcorr\msNa\msPaperA, यैः \msCaacorr\oo 
॰सत्त्वम्\lem \mssALL; ॰स\uncl{त्वन्} \msCa}}% 


\nemslokab

{\devanagarifontbold ते पापदोषान्नरकं व्रजन्ति  \danda\dontdisplaylinenum }%
     \var{{\devanagarifont \numnoemph\vb ते\lem \mssALL; {\lost} \msCa, स \msM\oo 
॰दोषान्न॰\lem \mssALL; ॰दोषा न॰ \msCb\msM\oo 
व्रजन्ति\lem \mssALL; प्रयान्ति \msM}}% 

\nemslokac

{\devanagarifontbold मन्वन्तरादीन्यनुभूय दुःखं }%
  \dontdisplaylinenum    \var{{\devanagarifont \numnoemph\vc ॰न्तरादीन्य॰\lem \mssALL; ॰न्तराद्धीम॰ \msM\oo 
दुःखं\lem \mssALL; दुःख \msPaperA}}% 


\nemslokad

{\devanagarifontbold पुनश्च तिर्यं शतशो ऽनुभूयात् {॥१८:१७॥} \veg\dontdisplaylinenum }%
     \var{{\devanagarifont \numnoemph\vd तिर्यं श॰\lem \mssALL; तिय स॰ \msM, तिर्यक् श॰ \msPaperA\Ed\oo 
नुभूयात्\lem \mssALL; नुभुक्त्वा \msM}}% 

\ujvers\nemsloka {
{\devanagarifontbold मानुष्यजन्मेषु च दुःखभागी }%
  \dontdisplaylinenum}    \var{{\devanagarifont \numemph\va मानुष्य॰\lem \mssALL; मनुष्य॰ \msNa\msM}}% 


\nemslokab

{\devanagarifontbold स्तेनत्वमायाति पुनश्च मूढः  \danda\dontdisplaylinenum }%
     \var{{\devanagarifont \numnoemph\vb \lem \msCa\msCb\msNa\msPaperA; 
स्ते\uncl{यु}त्वमायांति पुनश्च मूढाः \msM, 
स्तेने ऽयम{  }आयाति पुनश्च मूढाः \Ed}}% 

\nemslokac

{\devanagarifontbold सुवर्णचोरी कुनखत्व चिह्नं }%
  \dontdisplaylinenum    \var{{\devanagarifont \numnoemph\vc ॰चोरी\lem \msCa; ॰चौ\lk\ \msCb, ॰चौरी \msNa\msM\msPaperA, ˚चौर \Ed\oo 
कुनखत्व चिह्नम्\lem \msCa\msNa\msPaperA\Ed; \uncl{कनखत्व चिह्न} \msCb, 
कुनत्व चिंह्नं \msMacorr, 
कुनक्षत्व चिंह्नं \msMpcorr}}% 
    \paral{{\devanagarifont \vb {\englishfont \compare\ \YAJNS\ 3.209c:}
                         हेमहारी तु कुनखी }}


\nemslokad

{\devanagarifontbold विशीर्णगात्रो रजतापहारी {॥१८:१८॥} \veg\dontdisplaylinenum }%
     \var{{\devanagarifont \numnoemph\vd विशीर्ण॰\lem \mssALL; \uncl{विस्तीर्ण}॰ \msCb\oo 
॰गात्रो\lem \mssALL; ॰सूत्रद् \msM}}% 

\ujvers\nemsloka {
{\devanagarifontbold ताम्रापहारी स्फुटिताग्रपाणिर् }%
  \dontdisplaylinenum}    \var{{\devanagarifont \numemph\va ताम्रापहारी\lem \mssALL; त्रांम्रापहारी \msM, ताम्रापहारि \Ed\oo 
स्फुटिता॰\lem \mssALL; स्फटिता॰ \Ed\oo 
॰पाणिर्\lem \mssALL; ॰पाणि \msM\msPaperA, ॰पाणीर् \Ed}}% 


\nemslokab

{\devanagarifontbold लोहापहारी भुजछेद चिह्नम्  \danda\dontdisplaylinenum }%
     \var{{\devanagarifont \numnoemph\vb लोहापहारी\lem \mssALL; लोह\lac हारी \msCa\oo 
चिह्नम्\lem \mssALL; चिंह्नम् \msM}}% 

\nemslokac

{\devanagarifontbold कांसापहारी करभग्न चिह्नं }%
  \dontdisplaylinenum    \var{{\devanagarifont \numnoemph\vc कांसा॰\lem \mssALL; कांसो॰ \msPaperA\oo 
कर॰\lem \mssALL; क॰ \msCaacorr\oo 
चिह्नं\lem \mssALL; चिह्णम् \msMpcorr, ह्णम् \msMacorr}}% 


\nemslokad

{\devanagarifontbold हृत्वा च रीति-त्रपु-सीसकानाम् {॥१८:१९॥} \veg\dontdisplaylinenum }%
     \var{{\devanagarifont \numnoemph\vd हृत्वा च रीति॰\lem \msCa\msPaperA\Ed; हृत्वापचारीति \msCb, हृत्वा च रीतिं \msNa, 
हृत्वा च रीती॰ \msM}}% 

\ujvers\nemsloka {
{\devanagarifontbold नासोष्ठकर्णश्रवणस्य छेदश् }%
  \dontdisplaylinenum}    \var{{\devanagarifont \numemph\va नासो॰\lem \mssALL; नासौ॰ \Ed}}% 


\nemslokab

{\devanagarifontbold चिह्नं नृणां वस्त्रहरः कुचैलः  \danda\dontdisplaylinenum }%
     \var{{\devanagarifont \numnoemph\vb चिह्नं\lem \mssALL; चिंह्नं \msM\oo 
॰हरः\lem \mssALL; ॰हरं \Ed\oo 
कुचैलः\lem \mssALL; कुचेलः \msPaperA\Ed}}% 

\nemslokac

{\devanagarifontbold धान्यापहारी भवते ऽङ्गहीनो }%
  \dontdisplaylinenum    \var{{\devanagarifont \numnoemph\vc भवते ऽङ्ग॰\lem \msNa\msM\msPaperA; 
भवदेङ्ग॰ \msCa, 
\uncl{भवत्यङ्ग}॰ \msCb, 
भवत्येङ्ग॰ \Ed}}% 


\nemslokad

{\devanagarifontbold दीपापहारी भवते ऽन्ध चिह्नम् {॥१८:२०॥} \veg\dontdisplaylinenum }%
     \var{{\devanagarifont \numnoemph\vd दीपा॰\lem \mssALL; दिपो \Ed\oo 
भवते ऽन्ध\lem \msCa\msCb\msNa\msM\msPaperA; भवत्यन्ध \Ed}}% 

\ujvers\nemsloka {
{\devanagarifontbold निर्वापहा काण भवेत चिह्नं }%
  \dontdisplaylinenum}    \var{{\devanagarifont \numemph\va चिह्नं\lem \mssALL; \uncl{चिह्नं} \msCb, चिंह्नं \msM}}% 


\nemslokab

{\devanagarifontbold यः स्त्रीं हरेत्सो ऽपि जितः स्त्रिया स्यात्  \danda\dontdisplaylinenum }%
     \var{{\devanagarifont \numnoemph\vb \lem \corr; 
यः स्त्री हरेत्सो ऽपि जितस्स्त्रिया स्यात् \msCa, 
यः स्त्री हरे सो ऽपि जितः स्त्रियायात् \msCb, 
यः स्त्री हरेत्सो ऽपि जित स्त्रिया स्यात् \msNa, 
यः स्त्री हरे स्त्रीभिः जिता भवन्ति \msM, 
यः स्त्री हरेत्सो ऽपि जितः स्त्रिया स्यात् \msPaperA\Ed}}% 
    \paral{{\devanagarifont \vo {\englishfont \compare\ Mitākṣara ad Yājñavalkya 3.216cd:}
          ... न्यासापहारी च काणः, स्त्रीपण्योपजीवी षण्ढः, कौमारदारत्यागी दुर्भगः... }}

\nemslokac

{\devanagarifontbold सस्यापहारी भवते ऽन्नहीनो }%
  \dontdisplaylinenum    \var{{\devanagarifont \numnoemph\vc भवते ऽन्नहीनो\lem \mssALL; भवते ऽन्नहीवः \msCb, 
भवेतन्नहीनः \msPaperA}}% 


\nemslokad

{\devanagarifontbold हृत्वायुधमस्त्रहतत्व चिह्नम् {॥१८:२१॥} \veg\dontdisplaylinenum }%
     \var{{\devanagarifont \numnoemph\vd ॰युधमस्त्र॰\lem \mssALL; ॰युधयन्त्र \Ed}}% 

\ujvers\nemsloka {
{\devanagarifontbold अन्नापहारी परदत्तभोक्ता }%
  \dontdisplaylinenum}

\nemslokab

{\devanagarifontbold हृत्वा तु गावः स भवेद्दरिद्रः  \danda\dontdisplaylinenum }%
     \var{{\devanagarifont \numemph\vb भवेद्दरिद्रः\lem \mssALL; भवे दरिद्रंः \msM}}% 

\nemslokac

{\devanagarifontbold हरिं हरेत्तद्धरिणा दहन्ति }%
  \dontdisplaylinenum    \var{{\devanagarifont \numnoemph\vc हरिं हरेत्तद्धरिणा\lem \mssALL; हरिन्भवे त हरिणा \msM, 
हरिहरेत्तद्धरिणा \Ed}}% 


\nemslokad

{\devanagarifontbold हृत्वा तु मेषान् अजगर्दभं वा {॥१८:२२॥} \veg\dontdisplaylinenum }%
     \var{{\devanagarifont \numnoemph\vd \lem \msCa\msCb\msPaperA; 
हृत्वा च मेषानजगर्दभम्च \msNa, 
हृत्वा च मेषामजगर्दभञ्च \msM, 
हृत्वा तु मेषानजगर्दभश्च \Ed}}% 

\ujvers\nemsloka {
{\devanagarifontbold स भारभृज्जीव्यमुदाहरन्ति }%
  \dontdisplaylinenum}    \var{{\devanagarifont \numemph\va ॰ज्जीव्य॰\lem \mssALL; ॰ज्जीवा॰ \msM, ॰ज्जीव॰ \Ed}}% 


\nemslokab

{\devanagarifontbold रत्नापहारी अनपत्यता च  \danda\dontdisplaylinenum }%
     \var{{\devanagarifont \numnoemph\vb अनपत्यता\lem \mssALL; \lac त्यता \msCa}}% 
    \paral{{\devanagarifont \vb {\englishfont \compare\ Mitākṣara ad Yājñavalkya 3.216cd:}
                        ... गौतमो ऽपि क्वचिद्विशेषमाह\thinspace{\devanagarifont ।} ... 
                        न्यासापहार्यनपत्यः, रत्नापहार्यत्यन्तदरिद्रः...  }}

\nemslokac

{\devanagarifontbold छत्रापहारी अपवित्रता च }%
  \dontdisplaylinenum    \var{{\devanagarifont \numnoemph\vc अपवित्रता\lem \msCb\msPaperA\Ed; अपरित्रता \msCa\msNa\msM}}% 


\nemslokad

{\devanagarifontbold हृत्वा च बीजं स भवेदबीजः {॥१८:२३॥} \veg\dontdisplaylinenum }%
     \var{{\devanagarifont \numnoemph\vd \lem \msCa\msNa\msM\Ed; 
\uncl{हृत्वा नृजीवः स भवेदजीवः} \msCb, 
हृत्वा च बीजं स भवेदजीवः \msPaperA}}% 

\ujvers\nemsloka {
{\devanagarifontbold गोधूमशालियवमुद्गमाषान् }%
  \dontdisplaylinenum}    \var{{\devanagarifont \numemph\va ॰मुद्गमाषान्\lem \msCa\msNa\Ed; ॰मुद्गमाषा \msCb, ॰माषमुजान् \msM, ॰मुद्गमाषां \msPaperA}}% 


\nemslokab

{\devanagarifontbold हृत्वा मसूरं विलयं व्रजन्ति  \danda\dontdisplaylinenum }%
     \var{{\devanagarifont \numnoemph\vb मसूरं\lem \mssALL; मूत्रं \msMacorr, मूरं \msMpcorr}}% 

\nemslokac

{\devanagarifontbold कामातुरो मातर मातृपुत्रीं }%
  \dontdisplaylinenum    \var{{\devanagarifont \numnoemph\vc मातृपुत्रीं\lem \msCa\msCb\msNa; मात्रपुत्री \msM\msPaperA, मातृपुत्री \Ed}}% 


\nemslokad

{\devanagarifontbold मातृस्वसां गच्छति मातुलानीम् {॥१८:२४॥} \veg\dontdisplaylinenum }%
     \var{{\devanagarifont \numnoemph\vd ॰स्वसां\lem \mssALL; ॰स्वसा \msM\oo 
मातुलानीम्\lem \mssALL; मातुलानी \msNa, मातुलानीः \msM}}% 

\ujvers\nemsloka {
{\devanagarifontbold राजाङ्गनां पुत्रसुतां स्नुषां च }%
  \dontdisplaylinenum}    \var{{\devanagarifont \numemph\va राजाङ्गनां\lem \mssALL; राजाङ्गणा \msCb\msM\oo 
॰सुतां\lem \msNa\Ed; ॰सुत \msCa\msCb\msM, ॰सुता \msPaperA\oo 
स्नुषां\lem \mssALL; स्नुसा \msM}}% 


\nemslokab

{\devanagarifontbold प्रव्राजिनीं ब्राह्मणिमन्त्यजां च  \danda\dontdisplaylinenum  }%
     \var{{\devanagarifont \numnoemph\vb ॰व्राजिनीं\lem \msCa\msCb\Ed; ॰व्राजिनी \msNa, ॰व्रजनी \msM, ॰व्रजिनां \msPaperA\oo 
ब्राह्मणिमन्त्यजां च\lem \msCb\msNa; ब्राह्मणिमन्त्य\lk ञ्च \msCa, 
ब्राह्मणि चान्त्यजा च \msM, 
ब्रह्मणिमन्त्यजाञ्च \msPaperA, 
ब्राह्मणीमन्त्यजां च \Ed}}% 

\nemslokac

{\devanagarifontbold अजाश्वमेषं सुरभीसुतां च }%
  \dontdisplaylinenum    \var{{\devanagarifont \numnoemph\vc ॰मेषं\lem \msCa\msCb\msNa; ॰मेष॰ \msM\msPaperA\Ed\oo 
॰सुतां च\lem \msPaperA; ॰सुतं च \msCa\msCb\msNa; ॰सुतश्च \msM, ॰सुताश्च \Ed}}% 


\nemslokad

{\devanagarifontbold यत्कामयेत्तेषु विमूढचेताः {॥१८:२५॥} \veg\dontdisplaylinenum }%
     \var{{\devanagarifont \numnoemph\vd यत्का॰\lem \mssALL; यः का॰ \msM\oo 
॰चेताः\lem \mssALL; ॰चेतः \Ed}}% 

\ujvers\nemsloka {
{\devanagarifontbold स याति कृच्छ्रं नरकं सुघोरं }%
  \dontdisplaylinenum}    \var{{\devanagarifont \numemph\va याति\lem \mssALL; यान्ति \msM}}% 


\nemslokab

{\devanagarifontbold स वर्षकोटी शतशो भ्रमित्वा  \danda\dontdisplaylinenum }%
 
\nemslokac

{\devanagarifontbold तिर्यं च भूयः शतशो व्यतीत्य }%
  \dontdisplaylinenum    \var{{\devanagarifont \numnoemph\vc तिर्यं च\lem \mssALL; तियञ्च \msM, तीर्यञ्च \Ed\oo 
व्यतीत्य\lem \mssALL; व्यतित्य \msPaperA}}% 


\nemslokad

{\devanagarifontbold कष्टेन वै जायति मानुषत्वम् {॥१८:२६॥} \veg\dontdisplaylinenum }%
     \var{{\devanagarifont \numnoemph\vd वै जायति\lem \mssALL; वै \uncl{नेत्यद्मि} \msCb, प्राप्नोति स \msM}}% 

\ujvers\nemsloka {
{\devanagarifontbold हीनाङ्गता दीनशरीरताश्च }%
  \dontdisplaylinenum}    \var{{\devanagarifont \numemph\va \lem \msNa\Ed; 
हीनाङ्गतान्दीनशरीरताञ्च \msCa, 
हीनाङ्गतान्दीनशरीरवाञ्च \msCb, 
हीनाङ्गता दीनशरीरताञ्च \msPaperA, 
हीनाङ्ग दीनकुशरीरता च \msM}}% 


\nemslokab

{\devanagarifontbold यो मातृगामी स भवेदलिङ्गः  \danda\dontdisplaylinenum }%
     \var{{\devanagarifont \numnoemph\vb ॰लिङ्गः\lem \mssALL; ॰लि\lk\ \msCa}}% 

\nemslokac

{\devanagarifontbold मातृस्वसातल्पगवातलिङ्गा }%
  \dontdisplaylinenum    \var{{\devanagarifont \numnoemph\vc ॰वातलिङ्गा\lem \msCa\msNa; ॰वात॰ \msCb, ॰वातलिङ्गः \msM, ॰वानलि\lk\ \msPaperA, ॰वानलिङ्गा \Ed}}% 


\nemslokad

{\devanagarifontbold लिङ्गापरोधः सुतपुत्रिकामः {॥१८:२७॥} \veg\dontdisplaylinenum }%
     \var{{\devanagarifont \numnoemph\vd \lem \msNa\msPaperA; 
लिङ्गापरोधः सुतपुत्रिकाम \msCa, 
लिङ्गापरोधः सुतपुत्रिकामा \msCb, 
लिङ्गापरोधः सुत्रपुत्रिकामा \msM, 
लिङ्गेपरोधः सुतपुत्रिकामः \Ed}}% 

\ujvers\nemsloka {
{\devanagarifontbold स्नुषां च यः सेवति रक्तमेही }%
  \dontdisplaylinenum}

\nemslokab

{\devanagarifontbold दौश्चर्मतां च द्विजसुन्दरीषु  \danda\dontdisplaylinenum }%
     \var{{\devanagarifont \numemph\vb \lem \msCa\msCc; 
\lac ञ्च  द्विजसुन्दरीषु \msCb, 
दौचर्मतां च द्विजसुन्दरीषु \msNa, 
दौचर्म्मता याति द्विजाङ्गनाम्च \msM, 
दौःचर्म्मतांच द्विजसुन्दरीषु \msPaperA, 
दौः चर्मराश् च द्विजसुन्दरीषु \Ed}}% 
    \lacuna{\devanagarifont \vb {\englishfont \msCc\ resumes here in f.~306r with} चर्मताश्च द्विजसुन्दरीषु }%
      \paral{{\devanagarifont \vb {\englishfont \compare\ \YAJNS\ 3.209d:}
                         दुश्चर्मा गुरुतल्पगः }}

\nemslokac

{\devanagarifontbold राजाङ्गनायासु च लिङ्गच्छेदः }%
  \dontdisplaylinenum    \var{{\devanagarifont \numnoemph\vc लिङ्ग॰\lem \mssALL; लिङ्गः \msNa}}% 


\nemslokad

{\devanagarifontbold प्रव्राजिनीकामुक मूत्रकृच्छ्रम् {॥१८:२८॥} \veg\dontdisplaylinenum }%
     \var{{\devanagarifont \numnoemph\vd ॰व्राजिनी॰\lem \mssALL; ॰वाजिनी॰ \msCc, ॰व्रजिनी॰ \msPaperA\oo 
॰कृच्छ्रम्\lem \mssALL; ॰कृच्छ्रः \msM, ॰कृच्छ्र \msPaperA}}% 

\ujvers\nemsloka {
{\devanagarifontbold सव्याधिलिङ्गं लभते ऽन्त्यजासु }%
  \dontdisplaylinenum}    \var{{\devanagarifont \numemph\va सव्याधिलिङ्गं लभते\lem \mssALL; अत्याधिलिङ्गा भवते \msM, सव्याधिलिङ्ग लभते \msPaperA\Ed}}% 


\nemslokab

{\devanagarifontbold विलीनलिङ्गः पशुयोनिगामी  \danda\dontdisplaylinenum }%
     \var{{\devanagarifont \numnoemph\vb विलीनलिङ्गः\lem \mssALL; विलीनः \msCb\oo 
॰योनिगामी\lem \mssALL; ॰यो\lac मी \msCa}}% 

\nemslokac

{\devanagarifontbold जायन्ति ते मूषिक धान्यचौरी }%
  \dontdisplaylinenum    \var{{\devanagarifont \numnoemph\vc ॰चौरी\lem \msCb\msNa\Ed; ॰चोरी \msCa\msCc\msM\msPaperA}}% 
    \paral{{\devanagarifont \vcd {\englishfont for these pādas and the next verse, \compare\ \MANU\ 12.62:}
                 धान्यं हृत्वा भवत्याखुः कांस्यं हंसो जलं प्लवः\thinspace{\devanagarifont ।}
                 मधु दंशः पयः काको रसं श्वा नकुलो घृतम्\thinspace{\devanagarifont ॥};
                 {\englishfont \compare\ also \YAJNS\ 3.214:}
                         मूषको धान्यहारी स्याद्यानम् उष्ट्रः कपिः फलम्\thinspace{\devanagarifont ।}
                         जलं प्लवः पयः काको गृहकारी ह्युपस्करम्\thinspace{\devanagarifont ॥} }}


\nemslokad

{\devanagarifontbold क्षीरं हरेद्वायसतां प्रयाति {॥१८:२९॥} \veg\dontdisplaylinenum }%
     \var{{\devanagarifont \numnoemph\vd ॰यसतां\lem \mssALL; ॰यता \msNaacorr, ॰यसता \msNapcorr}}% 

\ujvers\nemsloka {
{\devanagarifontbold कांसापहारी स भवेत्तु हंसः }%
  \dontdisplaylinenum}    \var{{\devanagarifont \numemph\va कांसा॰\lem \eme; हंसा॰ \mssCaCbCc\msNa\msPaperA\Ed, हान्सा॰ \msM\oo 
भवेत्तु\lem \conj; भवेन्नि॰ \mssCaCbCc\msNa\msPaperA\Ed, भवेत \msM}}% 


\nemslokab

{\devanagarifontbold श्वानत्वमायाति रसापहारी  \danda\dontdisplaylinenum }%
     \var{{\devanagarifont \numnoemph\vb श्वानत्व॰\lem \mssALL; श्वातत्व॰ \msCb, श्वनत्व॰ \msPaperA\oo 
रसा॰\lem \mssALL; रषा॰ \msNa}}% 

\nemslokac

{\devanagarifontbold हृत्वा च सूचीं तु भवेत्स दंशः }%
  \dontdisplaylinenum    \var{{\devanagarifont \numnoemph\vc सूचीं तु भवेत्स\lem \msCa\msCb; सूची तु भवेत्स \msNa\msPaperA, माध्वी करसं स \msM, 
सूचीन्तु भवेत्स \msCc\Ed}}% 


\nemslokad

{\devanagarifontbold हृत्वा तु सर्पिर्वृकतां प्रयाति {॥१८:३०॥} \veg\dontdisplaylinenum }%
     \var{{\devanagarifont \numnoemph\vd सर्पिर्वृकतां प्रयाति\lem \msCa, 
सर्प्पि \uncl{वृ}कता प्रयाति \msCb, 
सर्प्पिर्वृकृतां प्रयान्ति \msCc, 
सर्पि वृ\uncl{क}तां प्रयाति \msNa, 
सर्प्पिर्वृकृतां प्रयाति \msM, 
सर्प्पि वृषतां प्रयाति \msPaperA, 
सर्पिर्वृषतां प्रयाति \Ed}}% 
    \paral{{\devanagarifont \vo {\englishfont \compare\ \YAJNS\ 3.215:}
                         मधु दंशः पलं गृध्रो गां गोधाग्निं बकस्तथा\thinspace{\devanagarifont ।}
                         श्वित्री वस्त्रं श्वा रसं तु चीरी लवणहारकः\thinspace{\devanagarifont ॥} }}

\ujvers\nemsloka {
{\devanagarifontbold मांसं तु हृत्वा स भवेत गृध्रस् }%
  \dontdisplaylinenum}    \var{{\devanagarifont \numemph\va मांसं\lem \mssALL; मान्सा \msM}}% 
    \paral{{\devanagarifont \vo {\englishfont \compare\ \MANU\ 12.63ab:}
                 मांसं गृध्रो वसां मद्गुस्तैलं तैलपकः खगः }}


\nemslokab

{\devanagarifontbold तैलापहारी खगतां प्रयाति  \danda\dontdisplaylinenum }%
     \var{{\devanagarifont \numnoemph\vb ॰हारी\lem \mssALL; ॰हारा \msM\oo 
खगतां\lem \mssALL; खशतां \msPaperA\oo 
॰याति\lem \mssALL; ॰या\lk\ \msCa}}% 
    \paral{{\devanagarifont \vb {\englishfont \compare\ \YAJNS\ 3.211c:}
                     तैलहृत् तैलपायी }}

\nemslokac

{\devanagarifontbold गुडं च हृत्वा गुडिका भवन्ति }%
  \dontdisplaylinenum    \var{{\devanagarifont \numnoemph\vc गुडं च\lem \mssALL; \lk डञ्च \msCa, गुडन्तु \msM\oo 
भवन्ति\lem \mssALL; \uncl{भवन्तिम्} \msCb}}% 
    \paral{{\devanagarifont \vc {\englishfont \compare\ \MANU\ 12.64d:} गोधा गां वाग्गुदो गुडम् }}


\nemslokad

{\devanagarifontbold शाकापहारी स भवेन्मयूरः {॥१८:३१॥} \veg\dontdisplaylinenum }%
     \var{{\devanagarifont \numnoemph\vd ॰पहारी\lem \mssALL; ॰प्रहारी \msM\oo 
भवेन्मयूरः\lem \msCa\msNa; भवेत्मयूरः \msCb\msCc\msPaperA, 
भवे मयूरः \msM, 
भवेन्मयूरम् \Ed}}% 
    \paral{{\devanagarifont \vd {\englishfont \compare\ \MANU\ 12.65b:} पत्रशाकं तु बर्हिणः;
                    {\englishfont \compare\ \YAJNS\ 3.213c:} पत्रशाकं शिखी हत्वा }}

\ujvers\nemsloka {
{\devanagarifontbold हृत्वा पशुं पङ्गुर जायते ह }%
  \dontdisplaylinenum}    \var{{\devanagarifont \numemph\va \lem \msCc\msPaperA; हृत्वा य पशुं पङ्गुर जायते ह \msCa, 
\uncl{हृत्वा पशुं पङ्गुनु} जायते हः \msCb, 
हृत्वा पशु पङ्गुर जायते ह \msNa, 
हृत्वा पशु पंगुक जायतीहः \msM, 
हृत्वा पशुं पङ्गुर जायते हः \Ed}}% 


\nemslokab

{\devanagarifontbold श्वित्रत्वमायाति सुवस्त्रहारी  \danda\dontdisplaylinenum }%
     \var{{\devanagarifont \numnoemph\vb श्वित्रत्व॰\lem \mssALL; श्वैत्रत्व॰ \msM, चित्रत्व॰ \Ed\oo 
॰वस्त्र॰\lem \mssALL; ॰व॰ \msNaacorr\oo 
॰हारी\lem \mssALL; ॰हर्त्ता \msM}}% 

\nemslokac

{\devanagarifontbold हृत्वा दुकूलं स च सारसत्त्वं }%
  \dontdisplaylinenum    \var{{\devanagarifont \numnoemph\vc दुकूलं\lem \mssALL; ऽकूलं \msPaperA}}% 


\nemslokad

{\devanagarifontbold क्षौमं च हृत्वा स च दर्दुरत्वम् {॥१८:३२॥} \veg\dontdisplaylinenum }%
     \var{{\devanagarifont \numnoemph\vd क्षौमं च\lem \msCa\msM\msPaperA\Ed; क्षोमं च \msCb\msCc\msNa\oo 
दर्दुरत्वम्\lem \msCa\msCb\msNa\msM; दुर्दुरत्वम् \msCc, दुर्द्दलत्वं \msPaperA, दुर्व्वलत्वम् \Ed}}% 

\ujvers\nemsloka {
{\devanagarifontbold और्णानि वस्त्राण्यपहृत्य मेषः }%
  \dontdisplaylinenum}    \var{{\devanagarifont \numemph\va और्णानि वस्त्राण्य॰\lem \msCc\msNa\msPaperA; 
ओर्णानि वस्त्राण्य॰ \msCa\msCb, 
उर्ण्णञ्च वस्त्रम॰  \msM, 
ऊर्णानि वस्त्राण्य॰ \Ed\oo 
मेषः\lem \mssALL; मेसं \msM}}% 


\nemslokab

{\devanagarifontbold छुच्छुन्दरी जायति गन्धहारी  \danda\dontdisplaylinenum }%
     \var{{\devanagarifont \numnoemph\vb छुच्छुन्दरी\lem \mssALL; छुंछुन्दरी \msNa}}% 
    \paral{{\devanagarifont \vb {\englishfont \compare\ \YAJNS\ 3.213d:}
                         गन्धान्छुच्छुन्दरी शुभान् }}

\nemslokac

{\devanagarifontbold ब्रह्मस्वमल्पमपहृत्य भोक्ता }%
  \dontdisplaylinenum    \var{{\devanagarifont \numnoemph\vc ब्रह्मस्वमल्प॰\lem \msCb\msCc\msNa\msPaperA\Ed; \uncl{ब्रह्म}\lac मल्प॰ \msCa, 
ब्रह्मश्वद्रव्य॰ \msM}}% 


\nemslokad

{\devanagarifontbold स गृध्र उच्छिष्टभुजो भवन्ति {॥१८:३३॥} \veg\dontdisplaylinenum }%
     \var{{\devanagarifont \numnoemph\vd ॰भुजो\lem \msCc\msNa\msM\Ed; ॰भुजे \msCa, ॰\uncl{भुजा} \msCb, ॰भुजा \msPaperA}}% 

\ujvers\nemsloka {
{\devanagarifontbold पादेन यः स्पर्शयते द्विजाङ्घ्रिं }%
  \dontdisplaylinenum}    \var{{\devanagarifont \numemph\va \lem \mssALL; 
\lac\ \msCb, 
पादेन य स्पर्शयते द्विजानां \msM}}% 


\nemslokab

{\devanagarifontbold तद्वातरक्तं चरणे भवेत  \danda\dontdisplaylinenum }%
     \var{{\devanagarifont \numnoemph\vb \lem \msCa\msCc\msNa\msPaperA; 
\uncl{तद्वात}रक्तञ्चरणे भवेत् \msCb\ \unmetr, 
स वातरक्त चरणा भवन्ति \msM, 
तच्छीतरक्तं चरणौ भवेत \Ed}}% 

\nemslokac

{\devanagarifontbold पादेन यः स्पर्शयते च गावः }%
  \dontdisplaylinenum    \var{{\devanagarifont \numnoemph\vc पादेन यः\lem \mssALL; पादा हि सं॰ \msM}}% 


\nemslokad

{\devanagarifontbold स पादरोगान्विविधान्लभेत {॥१८:३४॥} \veg\dontdisplaylinenum }%
     \var{{\devanagarifont \numnoemph\vd \lem \msCa\msCc\Ed; स पादरोगान्विविधान्लभेत् \msCb\msNa\ \unmetr, 
स पादरोगा विविधा भवन्ति \msM, 
स पादरोगा विधा लभेत \msPaperA}}% 

\ujvers\nemsloka {
{\devanagarifontbold यो मातरं ताडयते पदेन }%
  \dontdisplaylinenum}    \var{{\devanagarifont \numemph\va \lem \mssALL; 
पादेन यो ताडयतीह माताः \msM, 
यो मातरः ताडयते पदेन \Ed}}% 


\nemslokab

{\devanagarifontbold पादे तदीये कृमयः पतन्ति  \danda\dontdisplaylinenum }%
     \var{{\devanagarifont \numnoemph\vb पादे तदीये कृमयः\lem \mssALL; पादेषु तस्य तृमियः \msM}}% 

\nemslokac

{\devanagarifontbold पदा स्पृशेद्यः पितरं दुरात्मा }%
  \dontdisplaylinenum    \var{{\devanagarifont \numnoemph\vc पदा स्पृशेद्यः\lem \msCa\msCb\msNa; पादा शेद्यः \msCc, पादेन पृष्त \msM, 
पादात्पृशेद्यः \msPaperA\Ed\oo 
पितरं\lem \mssALL; पित\lk\ \msCa}}% 


\nemslokad

{\devanagarifontbold सूनोन्नपादः स भवेत्परत्र {॥१८:३५॥} \veg\dontdisplaylinenum }%
     \var{{\devanagarifont \numnoemph\vd सूनोन्नपादः\lem \msNa\msPaperA\Ed; सूनोन्नपाद \mssCaCbCc\ \unmetr, सूनोनपाद \msM\oo 
॰त्र\lem \mssALL; ॰त्रः \msM}}% 

\ujvers\nemsloka {
{\devanagarifontbold पदा स्पृशेत्तोयमनादरेण }%
  \dontdisplaylinenum}    \var{{\devanagarifont \numemph\va पदा स्पृशेत्तो॰\lem \msCa\msCb\msNa\msPaperA; पादा स्पृशे तो॰ \msCc, 
\uncl{पादे} स्पृशे तो॰ \msM, 
पदात्पृशेत्तो॰ \Ed}}% 


\nemslokab

{\devanagarifontbold स श्लीपदी पादयुगे भवेत  \danda\dontdisplaylinenum }%
     \var{{\devanagarifont \numnoemph\vb पादयुगे भवेत\lem \msCa\msCb\msPaperA\Ed; पादयुगे भवेत् \msCc\msNa\ \unmetr, पाद महद्भवन्ति \msM}}% 

\nemslokac

{\devanagarifontbold पादेन यः स्पर्शयते हुताशं }%
  \dontdisplaylinenum    \var{{\devanagarifont \numnoemph\vc \lem \msCa\msCc\msPaperA; \lac\ \msCb, 
पादेन यः स्पर्शयते हुताशनं \msNa\ \unmetr, 
\lac\ \msM, 
पादेन य स्पर्शयते हुताशं \Ed}}% 


\nemslokad

{\devanagarifontbold स चाग्निपादः सततं भवेत {॥१८:३६॥} \veg\dontdisplaylinenum }%
     \var{{\devanagarifont \numnoemph\vd \lem \msCa\msNa\msPaperA\Ed; 
स चाग्निपादः सततं भवेत् \lk\ \msCb\msCc, 
तथाग्निपादा सततम्भवन्ति \msM}}% 

\ujvers\nemsloka {
{\devanagarifontbold पादेन यश्चार्यमुपस्पृशेत }%
  \dontdisplaylinenum}    \var{{\devanagarifont \numemph\va \lem \mssALL; 
\lac य\lk श्चार्यमु\uncl{पस्पृशेत} \msCb, 
पादेन ये चायम्मुपस्पृशन्ति \msM}}% 


\nemslokab

{\devanagarifontbold स पादछेदं बहुशो लभेत  \danda\dontdisplaylinenum }%
     \var{{\devanagarifont \numnoemph\vb स\lem \mssALL; ते \msM\oo 
॰छेदं\lem \mssALL; ॰च्छेद \msM\oo 
लभेत\lem \msCa\msNa\msPaperA\Ed; लभेत् \msCb\msCc\ \unmetr, भवन्ति \msM}}% 

\nemslokac

{\devanagarifontbold ग्रन्थापहारी स भवेत मूकः }%
  \dontdisplaylinenum    \var{{\devanagarifont \numnoemph\vc भवेत\lem \mssALL; भवे ह \msM}}% 
    \paral{{\devanagarifont \vc {\englishfont \compare\ \YAJNS\ 3.210d:} मूको वागपहारकः }}


\nemslokad

{\devanagarifontbold दुर्गन्धवक्त्रः परछिद्रवादी {॥१८:३७॥} \veg\dontdisplaylinenum }%
     \var{{\devanagarifont \numnoemph\vd दुर्गन्धवक्त्रः\lem \mssALL; \lk र्ग्गन्धवक्त्रः \msCa, दुर्गन्धवक्त्र \msM}}% 
    \paral{{\devanagarifont \vd {\englishfont \compare\ \YAJNS\ 3.211d:}
                         स्यात्पूतिवक्त्रस्तु सूचकः }}

\ujvers\nemsloka {
{\devanagarifontbold पैशुन्यवादी स च पूतिनासो }%
  \dontdisplaylinenum}    \var{{\devanagarifont \numemph\va पैशु॰\lem \msM; पैशू॰ \mssCaCbCc\msNa\msPaperA\Ed\oo 
॰नासो\lem \mssCaCbCc\msNa; ॰नासा \msPaperA\Ed}}% 
    \paral{{\devanagarifont \va {\englishfont \compare\ \YAJNS\ 3.211b:}
                         पिशुनः पूतिनासिकः }}


\nemslokab

{\devanagarifontbold नृ नम्रवक्त्रस्त्वनृतापवादी  \danda\dontdisplaylinenum }%
     \var{{\devanagarifont \numnoemph\vb नृ नम्र॰\lem \msCa\msCb\msNa; ननम्र॰ \msCc, तृ नम्र \msM, ननृम्र॰ \msPaperA, मनम्र॰ \Ed\oo 
॰पवादी\lem \mssALL; ॰प्रवादी \msM}}% 

\nemslokac

{\devanagarifontbold पारुष्यवक्ता मुखपाकरोगी }%
  \dontdisplaylinenum    \var{{\devanagarifont \numnoemph\vc ॰वक्ता\lem \mssALL; ॰वक्त्रा \msM}}% 


\nemslokad

{\devanagarifontbold असत्प्रलापी स च दन्तरोगः {॥१८:३८॥} \veg\dontdisplaylinenum }%
     \var{{\devanagarifont \numnoemph\vd ॰रोगः\lem \mssALL; ॰रोगीः \msCc}}% 

\ujvers\nemsloka {
{\devanagarifontbold तीक्ष्णप्रदायी स च वक्रनासः }%
  \dontdisplaylinenum}    \var{{\devanagarifont \numemph\va तीक्ष्ण॰\lem \mssALL; क्ष्ण॰ \msPaperA, स्तीक्ष्ण॰ \Ed\oo 
स च\lem \mssALL; भव \msM\oo 
॰नासः\lem \mssALL; ॰नास \Ed}}% 


\nemslokab

{\devanagarifontbold सम्भिन्नवक्ता स च कण्ठरोगी  \danda\dontdisplaylinenum }%
     \var{{\devanagarifont \numnoemph\vb \lem \mssALL; संभिनं वक्ता सद कण्ठरोगः \msM}}% 

\nemslokac

{\devanagarifontbold क्रुद्धेक्षणः पश्यति यस्तु विप्रं }%
  \dontdisplaylinenum    \var{{\devanagarifont \numnoemph\vc \lem \mssALL; 
क्रुद्धेक्षणः पश्यति यस्तु विप्रः \msCc, 
क्रोधेन यः पश्यति विप्र मूढा \msM}}% 


\nemslokad

{\devanagarifontbold तीव्राक्षिरोगी स तु जायते हि {॥१८:३९॥} \veg\dontdisplaylinenum }%
     \var{{\devanagarifont \numnoemph\vd ॰रोगी स तु जायते हि\lem \mssALL; ॰रोगातुर जायतीहः \msM}}% 

\ujvers\nemsloka {
{\devanagarifontbold प्रद्वेषयालोकयते ऽतिथीन्य }%
  \dontdisplaylinenum}    \var{{\devanagarifont \numemph\va तिथीन्य\lem \mssALL; तिथिश्च \msM, तिथीन्य \msPaperA}}% 


\nemslokab

{\devanagarifontbold उत्पाटिताक्षिः स भवेत्परत्र  \danda\dontdisplaylinenum }%
     \var{{\devanagarifont \numnoemph\vb \lem \mssALL; 
स चाक्षिमुत्पाटयते परत्र \msM, 
उत्पादिताक्षिः स भवेत्परत्र \msPaperA\Ed}}% 

\nemslokac

{\devanagarifontbold वैरूप्यचक्षुस्त्वतिसूक्ष्मचक्षुः }%
  \dontdisplaylinenum    \var{{\devanagarifont \numnoemph\vc ॰तिसूक्ष्म॰\lem \mssALL; ॰निमृ\lk\ \msM\oo 
॰चक्षुः\lem \mssALL; ॰चक्षु \msCc}}% 


\nemslokad

{\devanagarifontbold स जायते केकरपिङ्गचक्षुः {॥१८:४०॥} \veg\dontdisplaylinenum }%
     \var{{\devanagarifont \numnoemph\vd स जायते\lem \mssALL; जायन्ति ते \msM\oo 
केकर॰\lem \mssALL; केककर॰ \msCc}}% 

\ujvers\nemsloka {
{\devanagarifontbold गर्ताक्षिकादीनि विपण्डुलानि }%
  \dontdisplaylinenum}    \var{{\devanagarifont \numemph\va ॰क्षिकादीनि\lem \mssALL; ॰क्षि\uncl{सासाभि} \msM\oo 
विपण्डुलानि\lem \mssALL; विषण्डुलानि \msPaperA, विपाण्डुरानि \Ed}}% 


\nemslokab

{\devanagarifontbold नेत्रामयान्येव च पापदोषात्  \danda\dontdisplaylinenum }%
     \var{{\devanagarifont \numnoemph\vb \lem \mssALL; 
नेत्रामयानेव च पापदोषात् \msCc, 
भवन्ति नेत्रामय पापदोसा \msM}}% 

\nemslokac

{\devanagarifontbold शृण्वन्ति ये पापकथां प्रशस्तां }%
  \dontdisplaylinenum    \var{{\devanagarifont \numnoemph\vc ये\lem \mssALL; यो \msPaperA\oo 
॰कथां प्रशस्तां\lem \msCb\msCc\msNa\msPaperA\Ed; ॰कथां \lac स्तान् \msCa, ॰कथा प्रस्तं \msM}}% 


\nemslokad

{\devanagarifontbold तान्कर्णसर्पिः परिपीडयेत {॥१८:४१॥} \veg\dontdisplaylinenum }%
     \var{{\devanagarifont \numnoemph\vd तान्कर्णसर्पिः\lem \mssALL; ता कर्णसर्पिः \msCb, स कर्ण्णसर्प्प \msM\oo 
॰पीडयेत\lem \msCapcorr\msCc\msNa; ॰पीडियेत \msCaacorr\Ed, ॰पीडयेत् \msCb\ \unmetr, ॰पीडयन्ति \msM, 
॰पीडयोत \msPaperA}}% 

\ujvers\nemsloka {
{\devanagarifontbold शृणोति निन्दां हरिशर्वयोर्यः }%
  \dontdisplaylinenum}    \var{{\devanagarifont \numemph\va \lem \mssALL; 
शृण्वंति ये निन्द हरीश्वराभ्याम् \msM, 
शृण्वन्ति निन्द्रा हरिशर्व्वयोर्य्यः \msPaperA, 
शृण्वन्ति निन्दां हरिशर्वयोर्यः \Ed}}% 


\nemslokab

{\devanagarifontbold स कर्णशूलेन तु जीवतीव  \danda\dontdisplaylinenum }%
     \var{{\devanagarifont \numnoemph\vb जीवतीव\lem \mssALL; जीतीव \msCb, जीवनिष्टं \msM, जीवतीं वा \Ed}}% 

\nemslokac

{\devanagarifontbold मातापितॄणां शृणुते ऽपवादं }%
  \dontdisplaylinenum    \var{{\devanagarifont \numnoemph\vc ॰पितॄणां\lem \mssALL; ॰पितॄणा \msCb, ॰पितृभ्यां \msM\oo 
॰वादं\lem \mssALL; ॰वादः \msM}}% 


\nemslokad

{\devanagarifontbold स कर्णशोफेन विनाशमेति {॥१८:४२॥} \veg\dontdisplaylinenum }%
     \var{{\devanagarifont \numnoemph\vd शोफेन\lem \msMacorr\msPaperA; ॰सोहेन \mssCaCbCc\msNa, शोलेन \msMpcorr, ॰साफेन \Ed\oo 
विनाशमेति\lem \mssALL; विनासयन्ति \msM}}% 

\ujvers\nemsloka {
{\devanagarifontbold शृणोति निन्दां गुरुविप्रजां यः }%
  \dontdisplaylinenum}    \var{{\devanagarifont \numemph\va \lem \msCc\msNa\msPaperA; शृणोति निन्दां गुरुविप्रजा यः \msCa\Ed, 
शृणोति निन्दा गुरुविप्रजां यः \msCb, 
शृण्वन्ति ये निन्द गुरुः द्विजा वां \msM}}% 


\nemslokab

{\devanagarifontbold स कर्णपूयं स्रवते सरक्तम्  \danda\dontdisplaylinenum }%
     \var{{\devanagarifont \numnoemph\vb ॰पूयं\lem \mssALL; ॰पूय \msCc\msM}}% 

\nemslokac

{\devanagarifontbold विरूपदारिद्र्यकुलाधमेषु }%
  \dontdisplaylinenum    \var{{\devanagarifont \numnoemph\vc विरूपदारिद्र्य॰\lem \msCb; विरूपदारि\uncl{द्र्य}॰ \msCa, 
विरूप्यदारिद्र्य \msCc\msPaperA, 
विरूपदारिद्र॰ \msNa\msM, 
विरूप्यदारिध्र॰ \Ed\oo 
॰कुलाधमेषु\lem \mssALL; \lac मेषु \msCa}}% 


\nemslokab

{\devanagarifontbold अनिष्टकर्मभृतिजीवनं च  \danda\dontdisplaylinenum }%
     \var{{\devanagarifont \numnoemph\vd ॰भृतिजीवनं च\lem \mssALL; ॰भृतजीवनञ्च \msCc, ॰भृतिजीवनाश्च \Ed}}% 

\nemslokae

{\devanagarifontbold अकीर्तनं दर्शनवर्जनं च }%
  \dontdisplaylinenum    \var{{\devanagarifont \numnoemph\ve अकीर्तनं\lem \mssALL; अकीर्त्तनी \msM\oo 
॰वर्जनं च\lem \mssALL; ॰वर्जितञ्च \msM}}% 


\nemslokad

{\devanagarifontbold श्वपाकडोम्बादिषु जायते सः {॥१८:४३॥} \veg\dontdisplaylinenum }%
     \var{{\devanagarifont \numnoemph\vf श्वपाकडोम्बादिषु\lem \mssALL; श्वापाकतोम्बादिषु \msPaperA, श्वापाकतोश्वादिषु \Ed}}% 

\ujvers\nemsloka {
{\devanagarifontbold एतानि चिह्नं निरयागतानां }%
  \dontdisplaylinenum}    \var{{\devanagarifont \numemph\va चिह्नं\lem \mssALL; चिह्ना \msCb, चिंह्ना \msM\oo 
निरयागतानां\lem \mssALL; निररागताना \msPaperA}}% 


\nemslokab

{\devanagarifontbold मानुष्यलोके कुकृतस्य दृष्टम्  \danda\dontdisplaylinenum }%
     \var{{\devanagarifont \numnoemph\vb मानुष्य॰\lem \mssALL; मनुष्य॰ \msM\oo 
दृष्टम्\lem \mssALL; निष्ठे \msMacorr, दृष्ठे \msM}}% 

\nemslokac

{\devanagarifontbold समासतः कीर्तित एव देवि }%
  \dontdisplaylinenum    \var{{\devanagarifont \numnoemph\vc कीर्तित एव\lem \mssALL; कीर्तित एष \msNa, कीर्तितमेष \msM}}% 


\nemslokad

{\devanagarifontbold यथैव मुक्तस्त्विह कर्मभङ्गः {॥१८:४४॥} \veg\dontdisplaylinenum }%
     \var{{\devanagarifont \numnoemph\vd यथैव\lem \mssALL; यथाव \msM}}% 

\nemslokalong


\ujvers\nemsloka {
{\devanagarifontbold मातापित्रोघतोया सुतदुहितृवहा भ्रातृगम्भीरवेगा }%
  \dontdisplaylinenum}    \var{{\devanagarifont \numemph\va घतो या\lem \mssALL; पघातो \msNa, घतो याः \msM\oo 
सुत॰\lem \mssALL; सुतृ॰ \msCc\oo 
॰वहा\lem \mssALL; ॰वही \msM\oo 
॰वेगा\lem \mssALL; ॰वेगात् \msM}}% 


\nemslokab

{\devanagarifontbold भार्यावर्ता विवर्ता कुटिलगतिवधू बान्धवोर्मीतरङ्गा  \danda\dontdisplaylinenum }%
     \var{{\devanagarifont \numnoemph\vb ॰वधू बा॰\lem \mssALL; ॰वधुर्बा॰ \msPaperA\Ed\oo 
॰तरङ्गा\lem \mssALL; ॰तरङ्गाः \msM}}% 

\nemslokac

{\devanagarifontbold कामक्रोधोभकूला करिमकरझषा ग्राहकामा भयन्ते }%
  \dontdisplaylinenum    \var{{\devanagarifont \numnoemph\vc ॰कूला क॰\lem \mssALL; ॰कूलात्क॰ \msM\oo 
॰मकर॰\lem \mssALL; ॰मरण॰ \msCa\oo 
भयन्ते\lem \mssALL; भषन्तः \msMacorr, हयन्ते \msPaperA}}% 


\nemslokad

{\devanagarifontbold मृत्योराख्यार्णवे ऽस्मिन्न शरण विवशा कालदष्टा प्रयाति {॥१८:४५॥} \veg\dontdisplaylinenum }%
     \var{{\devanagarifont \numnoemph\vd स्मिन्न\lem \mssALL; स्मि\uncl{न्सु}॰ \msM\oo 
॰दष्टा\lem \msCa; ॰दुष्टा \msCb, ॰दुष्ट \msCc, ॰द्रष्ट्रा \msNa, ॰दष्ट \msM, ॰दृष्टा \msPaperA, ॰दृष्टो \Ed}}% 

\ujvers\nemsloka {
{\devanagarifontbold नित्यं येन विनाश याति दिवसं पञ्चत्वमापद्यते }%
  \dontdisplaylinenum}    \var{{\devanagarifont \numemph\va विनाश याति\lem \msM; विना न याति \msCa\msCc\msNa\msPaperA\Ed, विना\uncl{स याति} \msCb}}% 


\nemslokab

{\devanagarifontbold त्यक्त्वा देह वनान्तरेषु विषमे श्वानशृगालाकुले  \danda\dontdisplaylinenum }%
     \var{{\devanagarifont \numnoemph\vb श्वान॰\lem \mssALL; श्वानः \msM}}% 

\nemslokac

{\devanagarifontbold बन्धुः सर्व निवर्तते गतदया धर्मैक तत्र स्थितः }%
  \dontdisplaylinenum    \var{{\devanagarifont \numnoemph\vc बन्धुः\lem \mssALL; बन्धु \msM\ \unmetr\oo 
॰दया धर्मैक\lem \msM\msPaperA\Ed; ॰दया धर्मैकस् \msCa\ \unmetr, 
॰दयो धर्मैकस् \msCb\msNa\ \unmetr}}% 
    \lacuna{\devanagarifont \vc {\englishfont \msCc\ breaks down after reading} सर्व्व. }%
  

\nemslokad

{\devanagarifontbold तस्माद्धर्मपरो न चान्यसुहृदः सेवेत्परत्रार्थिनः {॥१८:४६॥} \veg\dontdisplaylinenum }%
     \var{{\devanagarifont \numnoemph\vd चान्य॰\lem \eme; चान्यः \msCa\msCb\msNa\msM\msPaperA\Ed\ \unmetr\oo 
सुहृदः सेवेत्प॰\lem \mssALL; सुहृतः सेवे प॰ \msM, सुहृदः सवस्प॰ \msPaperA}}% 

\vers


{\devanagarifontbold 
\jump
\begin{center}
\ketdanda\ इति वृषसारसंग्रहे पूर्वकर्मविपाकचिह्नाष्टादशमो ऽध्यायः \ketdanda
\end{center}
\dontdisplaylinenum\vers  }%
     \var{{\devanagarifont \numnoemph{\englishfont \Colo:} ॰विपाकचिह्नाष्टादशमो ऽध्यायः\lem \msCa\msCb\msNa\msPaperA; 
॰चिंह्नाध्यायः अष्टादशमः \msM, 
॰विपाकचिह्नाष्टादशो ऽध्यायः \Ed}}% 

\nemslokanormal

\versno=19
\versno=60
