\fejno=0\versno=0
\centerline{\Huge\devanagarifontbold वृषसारसंग्रहः  }

 
{\vrule depth10pt width0pt}
\versno=0\fejno=12
\thispagestyle{empty}

\centerline{\Large\devanagarifontbold [   द्वादशमो ऽध्यायः  ]}{\vrule depth10pt width0pt} \fancyhead[CO]{{\footnotesize\devanagarifont वृषसारसंग्रहे  }}
\fancyhead[CE]{{\footnotesize\devanagarifont द्वादशमो ऽध्यायः  }}
\fancyhead[LE]{}
\fancyhead[RE]{}
\fancyhead[LO]{}
\fancyhead[RO]{}
\szam\bek



\alalfejezet{आतिथ्यधर्मः}
\vers


{\devanagarifont देव्युवाच {\dandab}\dontdisplaylinenum  }%
 
{\devanagarifont अहिंसा परमो धर्मः सततं परिकीर्त्यते \thinspace{\danda} \dontdisplaylinenum }%
     \var{{\devanagarifont \numemph\vab\textbf{धर्मः स॰}\lem \mssALL, धर्मोस्स॰ \msCc}}% 
    \lacuna{\devanagarifontsmall {\englishfont Witnesses used for this chapter: \msCa\ ff.\thinspace 210r--215r, 
                                              \msCb\ ff.\thinspace 215v--219v, 
                                              \msCc\ ff.\thinspace 287v--283v 
                                                        (f.\thinspace 291 is missing),
                                              \msNa\ ff.\thinspace 17v--22r, 
                                              \msNb\ exp.\thinspace 58 (lower) -- 62 (lower),
                                              \msNc\ ff.\thinspace 225v--230r,
                                              \Ed\ pp.\thinspace 617--628; 
                                              \mssCaCbCc\ = \msCa + \msCb + \msCc} }%
  
%Verse 12:1

{\devanagarifont आतिथ्यकानां धर्मं च कथयस्व यदुत्तमम् {॥ १२:१॥} \veg\dontdisplaylinenum }%
     \var{{\devanagarifont \numnoemph\vc\textbf{आतिथ्य॰}\lem \mssALL, अतिथ्य॰ \msCb\msNb\oo 
\textbf{धर्मं च}\lem \mssALL, 
धर्मश्च \msCc, धर्मानां \msNb}}% 

{\devanagarifont महेश्वर उवाच {\dandab}\dontdisplaylinenum  }%
     \var{{\devanagarifont \numemph\vo\textbf{महेश्वर}\lem \mssALL, भगवान् \msNa}}% 

{\devanagarifont अहिंसातिथ्यकानां च शृणु धर्मं यदुत्तमम् \thinspace{\danda} \dontdisplaylinenum }%
     \var{{\devanagarifont \numnoemph\vb\textbf{शृणु}\lem \mssALL, \lac  णु \msCa\oo 
\textbf{धर्मं}\lem \mssALL, धर्म \msCc\Ed\oo 
\textbf{॰त्तमम्}\lem \mssALL, ॰त्तमां \Ed}}% 

%Verse 12:2

{\devanagarifont त्रैलोक्यमखिलं देवि रत्नपूर्णं सुलोचने {॥ १२:२॥} \veg\dontdisplaylinenum }%
     \var{{\devanagarifont \numnoemph\vd\textbf{॰पूर्णं}\lem \mssALL, पूर्ण्ण \msCc, ॰पूर्णां \Ed\oo 
\textbf{॰लोचने}\lem \mssALL, ॰लोचनं \msCb}}% 

{\devanagarifont चतुर्वेदविदे दानं न तत्तुल्यमहिंसकः \thinspace{\dandab} \dontdisplaylinenum }%
     \var{{\devanagarifont \numemph\va\textbf{दानं}\lem \mssALL, नानं \msCb}}% 

%Verse 12:3

{\devanagarifont शृणु धर्ममतिथ्यानां कीर्तयिष्यामि सुन्दरि {॥ १२:३॥} \veg\dontdisplaylinenum }%
 

\alalfejezet{विपुलोपाख्यानम्}
{\devanagarifont आसीद्वृत्तं पुराख्यानं नगरे कुसुमाह्वये \thinspace{\dandab} \dontdisplaylinenum }%
     \var{{\devanagarifont \numemph\va\textbf{आसीद्वृत्तं}\lem \msCa\msNa\Ed, आशीदत्तं \msCb, आसीद्वृतम् \msCc, आसी वृत्तं \msNb, आसीद्वृत्त \msNc\oo 
\textbf{॰ख्यानं}\lem \mssALL, ॰ख्यातं \Ed}}% 
    \var{{\devanagarifont \numnoemph\vb \lem \mssALL, 
नगरं कुसुमाह्वयम् \msCc\msNb}}% 

%Verse 12:4

{\devanagarifont कपिलस्य सुतो विद्वान्विपुलो नाम विश्रुतः {॥ १२:४॥} \veg\dontdisplaylinenum }%
 
{\devanagarifont धर्मनित्यो जितक्रोधः सत्यवादी जितेन्द्रियः \thinspace{\dandab} \dontdisplaylinenum }%
     \paral{{\devanagarifontsmall \vb {\englishfont  = \MBH\ 12.218.13b} }}

%Verse 12:5

{\devanagarifont ब्रह्मण्यश्च कृतज्ञश्च मद्भक्तः कृतनिश्चयः {॥ १२:५॥} \veg\dontdisplaylinenum }%
     \var{{\devanagarifont \numemph\vc\textbf{ब्रह्मण्य॰}\lem \msCb\msNa\msNb\Ed, ब्राह्मण्य॰ \msCa\msCc\msNc\oo 
\textbf{॰ज्ञश्च}\lem \mssALL, 
॰ज्ञ \msCb, ॰ज्ञश्च \msNb}}% 
    \var{{\devanagarifont \numnoemph\vd\textbf{॰भक्तः}\lem \mssALL, ॰भक्त॰ \Ed}}% 

{\devanagarifont धनाढ्यो ऽतिथिपूज्यश्च दाता दान्तो दयालुकः \thinspace{\dandab} \dontdisplaylinenum }%
     \var{{\devanagarifont \numemph\va\textbf{॰पूज्यश्च}\lem \msCa\msCc\msNapcorr\msNc\Ed, 
॰पूज्य \msCb\msNaacorr, ॰पूजश्च \msNb}}% 
    \var{{\devanagarifont \numnoemph\vb\textbf{दान्तो}\lem \msCbacorr\msNc\Ed, 
दान्त \msCa\msCc\msNa, दान्तोम{\englishfont (?)} \msCbpcorr, 
दान्त \msNb}}% 

%Verse 12:6

{\devanagarifont न्यायार्जितधनो नित्यमन्यायपरिवर्जितः {॥ १२:६॥} \veg\dontdisplaylinenum }%
     \var{{\devanagarifont \numnoemph\vc\textbf{न्याया॰}\lem \msCc\msNa\msNc\Ed, न्यायो॰ \msCa\msCb\msNb}}% 
    \var{{\devanagarifont \numnoemph\vcd\textbf{नित्यम॰}\lem \mssALL, नित्यंम॰ \msNb}}% 
    \var{{\devanagarifont \numnoemph\vd\textbf{॰वर्जितः}\lem \mssALL, ॰वर्जयेत् \msNb}}% 

{\devanagarifont भार्या च रूपिणी तस्य चन्द्रबिम्बशुभानना \thinspace{\dandab} \dontdisplaylinenum }%
     \var{{\devanagarifont \numemph\vb\textbf{॰बिम्ब॰}\lem \mssALL, ॰बिं\uncl{बा} \msNa\oo 
\textbf{॰शुभानना}\lem \mssALL, ॰निभानना \msNb}}% 

{\devanagarifont पीनोत्तुङ्गस्तनी कान्ता सकलानन्दकारिणी  \danda\dontdisplaylinenum }%
     \var{{\devanagarifont \numnoemph\vd\textbf{सकला॰}\lem \mssALL, \lac\  \msCa}}% 

%Verse 12:7

{\devanagarifont पतिव्रता पतिरता पतिशुश्रूषणे रता {॥ १२:७॥} \veg\dontdisplaylinenum }%
     \var{{\devanagarifont \numnoemph\ve\textbf{पतिव्रता}\lem \mssALL, प्रतिव्रता \msCb\oo 
\textbf{पतिरता}\lem \mssALL, प्रतिरता \msCb\msNb}}% 
    \var{{\devanagarifont \numnoemph\vf\textbf{पतिशुश्रूषणे}\lem \mssALL, प्रतिशुश्रूषणे \msNb}}% 
    \paral{{\devanagarifontsmall \vef {\englishfont \compare\ \BrahmaVP\ 4.27.174cd:}
                          पतिव्रते पतिरते पतिं देहि नमो ऽस्तु ते }}

{\devanagarifont अथ केनापि कालेन सूर्यरागमभूत्ततः \thinspace{\dandab} \dontdisplaylinenum }%
     \var{{\devanagarifont \numemph\vb\textbf{॰भूत्ततः}\lem \mssALL, ॰भूततः \msCc}}% 

%Verse 12:8

{\devanagarifont ग्रस्तभागत्रयस्त्वासीत्कृष्णमाधवमासिके {॥ १२:८॥} \veg\dontdisplaylinenum }%
 
{\devanagarifont स्नातुकामावतीर्यन्ते सर्वे पौरनृपादयः \thinspace{\dandab} \dontdisplaylinenum }%
     \var{{\devanagarifont \numemph\va\textbf{॰वतीर्यन्ते}\lem \mssALL, च तीर्थन्ते \Ed}}% 

%Verse 12:9

{\devanagarifont देवाश्च पितरश्चैव तर्प्यन्ते विधिवत्तथा {॥ १२:९॥} \veg\dontdisplaylinenum }%
     \var{{\devanagarifont \numnoemph\vc\textbf{देवाश्च}\lem \mssALL, देवश्च \msCc}}% 
    \var{{\devanagarifont \numnoemph\vd\textbf{तर्प्यन्ते}\lem \mssALL, तप्यन्ते \msCb\msNb}}% 

{\devanagarifont केचिज्जुह्वति तत्राग्निं केचिद्विप्रांश्च तर्पयेत् \thinspace{\dandab} \dontdisplaylinenum }%
     \var{{\devanagarifont \numemph\va\textbf{॰चिज्जुह्वति}\lem \mssALL, 
॰चिज्जुति \msCb, ॰चि\uncl{ज्व}ह्वति \msCc}}% 
    \var{{\devanagarifont \numnoemph\vb\textbf{विप्रांश्च}\lem \mssALL, विप्राश्च \msCb}}% 

%Verse 12:10

{\devanagarifont केचिद्दानोपतिष्ठन्ति केचित्स्तुवन्ति देवताम् {॥ १२:१०॥} \veg\dontdisplaylinenum }%
     \var{{\devanagarifont \numnoemph\vc\textbf{दानो॰}\lem \mssALL, ध्यानो॰ \Ed}}% 
    \var{{\devanagarifont \numnoemph\vd\textbf{केचित्स्तुवन्ति}\lem \msCa\msCb\msNc, केचिद्वन्ति \msCc, 
केचि स्तुवन्ति \msNa\msNb, 
केचित्स्तुन्वन्ति \Ed\oo 
\textbf{देवताम्}\lem \mssALL, देवता \msCb\msNc}}% 

{\devanagarifont ध्यानयोगरताः केचित्केचित्पञ्चतपे रताः \thinspace{\dandab} \dontdisplaylinenum }%
     \var{{\devanagarifont \numemph\va\textbf{॰रताः}\lem \mssALL, ॰रता \msNb}}% 

%Verse 12:11

{\devanagarifont एवं प्रवर्तमानेषु राजनादिषु सर्वशः {॥ १२:११॥} \veg\dontdisplaylinenum }%
     \var{{\devanagarifont \numnoemph\vd\textbf{राजना॰}\lem \mssALL, राजाना॰ \Ed}}% 

{\devanagarifont विपुलो ऽपि हि तत्रैव गङ्गागण्डकिसंगमे \thinspace{\dandab} \dontdisplaylinenum }%
     \var{{\devanagarifont \numemph\va\textbf{ऽपि हि}\lem \msCa\msCc\msNapcorr\msNb\msNc, 
पि \msCb, हि न \msNaacorr, पि च \Ed}}% 

%Verse 12:12

{\devanagarifont भार्यया सह तत्रैव स्नात्वा क्षोमविभूषणः {॥ १२:१२॥} \veg\dontdisplaylinenum }%
     \var{{\devanagarifont \numnoemph\vc\textbf{भार्यया}\lem \msCapcorr\msCb\msNa\msNb\msNc, भार्याया \msCaacorr\msCc\Ed}}% 
    \var{{\devanagarifont \numnoemph\vd\textbf{॰भूषणः}\lem \mssALL, 
॰भूष\uncl{णैः} \msCc, ॰भूषितः \msNa}}% 

{\devanagarifont देवतागुरुविप्राणामन्येषां तर्पणे रतः \thinspace{\dandab} \dontdisplaylinenum }%
     \var{{\devanagarifont \numemph\vab \lem \msCb\msNapcorr\msNb\msNc, 
देवतागुरुवि\lac  णामन्येषां तर्पणे रतः \msCa, 
देवतागुरुविप्राणामन्येषां तर्पणे रताः \msCc, 
\om\ \msNaacorr, 
देवतागुरुविप्राणामन्येषां तर्पणा रतः \Ed}}% 

%Verse 12:13

{\devanagarifont तत्रावसरसम्प्राप्तो ब्राह्मणो ऽतिथिरागतः {॥ १२:१३॥} \veg\dontdisplaylinenum }%
 
{\devanagarifont भार्या तस्यातिरूपेण मोहिता ब्रह्मणस्तदा \thinspace{\dandab} \dontdisplaylinenum }%
     \var{{\devanagarifont \numemph\vb\textbf{मोहिता}\lem \mssALL, मोहितो \msCb\oo 
\textbf{ब्रह्मणस्तदा}\lem \msCa\msCb\msNc, ब्राह्मणास्तथा \msCc, 
ब्राह्मणस्तदा \msNa\msNb, ब्राह्मणस्य च \Ed}}% 

%Verse 12:14

{\devanagarifont ब्राह्मणो ऽपि तथैवेह रूपेणाप्रतिमो भवेत् {॥ १२:१४॥} \veg\dontdisplaylinenum }%
     \var{{\devanagarifont \numnoemph\vc\textbf{ब्राह्मणो}\lem \mssALL, ब्रह्मणो \msCb\oo 
\textbf{तथैवेह}\lem \msCb\msNa\msNb\Ed, 
त\uncl{थे}वेह \msCa, तथेवेह \msCc\msNc}}% 
    \var{{\devanagarifont \numnoemph\vd\textbf{रूपेणा॰}\lem \msCa\msNa\msNb\msNc, रूपेना॰ \msCb, रूपेण \msCc, रूपिणा॰ \Ed}}% 

{\devanagarifont अन्योन्यदृष्टिसंसक्तौ जातौ तौ तु परस्परम् \thinspace{\dandab} \dontdisplaylinenum }%
     \var{{\devanagarifont \numemph\va\textbf{॰संसक्तौ}\lem \Ed, ॰संशक्तौ \msCa\msNa\msNc, 
॰शक्तौ \msCb, ॰संसक्तो \msCc\msNb}}% 
    \var{{\devanagarifont \numnoemph\vb\textbf{जातौ तौ}\lem \mssALL, 
जातो तौ तौ \msCc, जातौ \uncl{ता} \msNc}}% 

%Verse 12:15

{\devanagarifont विपुलेनाञ्जलिं कृत्वा ब्राह्मण संशितव्रत {॥ १२:१५॥} \veg\dontdisplaylinenum }%
     \var{{\devanagarifont \numnoemph\vd\textbf{ब्राह्मण}\lem \msCb\msCc, ब्राह्मणः \msCa\msNa\msNb\msNc\Ed\oo 
\textbf{॰शित॰}\lem \eme, ॰श्रित॰ \mssCaCbCc\msNa\msNb\msNc\Ed\oo 
\textbf{॰व्रत}\lem \conj, ॰व्र\lk\ \msCa, ॰व्रतः \msCb\msCc\msNa\msNb\msNc\Ed}}% 
    \paral{{\devanagarifontsmall \vd {\englishfont  = MBh 12.213.18d and 12.347.1d } }}

{\devanagarifont आज्ञापय द्विजश्रेष्ठ अद्य मे ऽनुग्रहं कुरु \thinspace{\dandab} \dontdisplaylinenum }%
     \var{{\devanagarifont \numemph\vb\textbf{॰ग्रहं}\lem \mssALL, ॰ग्रह \msCb}}% 

%Verse 12:16

{\devanagarifont भार्याभृत्यपशुग्राम रत्नानि विविधानि च {॥ १२:१६॥} \veg\dontdisplaylinenum }%
     \var{{\devanagarifont \numnoemph\vc\textbf{॰भृत्य॰}\lem \mssALL, ॰भृत्या॰ \msCc}}% 

{\devanagarifont विपुलेनैवमुक्तस्तु गृहीतो ब्राह्मणो ऽब्रवीत् \thinspace{\dandab} \dontdisplaylinenum }%
     \var{{\devanagarifont \numemph\vb\textbf{ब्राह्मणो ऽब्रवीत्}\lem \mssALL, 
भ्राह्मणस्तथा \msCc}}% 

%Verse 12:17

{\devanagarifont यदि सत्यं प्रदातासि सुप्रसन्नं मनस्तव {॥ १२:१७॥} \veg\dontdisplaylinenum }%
     \var{{\devanagarifont \numnoemph\vc \lem \mssALL, \om\ \msCc}}% 
    \var{{\devanagarifont \numnoemph\vd \lem \msCa\msCb\msNa\msNc, \om\ \msCc, 
सुप्रसन्नमनस्तव \msNb\Ed}}% 

{\devanagarifont विपुल उवाच {\dandab}\dontdisplaylinenum  }%
 
{\devanagarifont सुप्रसन्नं मनो मे ऽद्य सुप्रसन्नं तपःफलम् \thinspace{\danda} \dontdisplaylinenum }%
     \var{{\devanagarifont \numemph\va\textbf{॰प्रसन्नं मनो}\lem \mssALL, 
॰प्रसन्नमनो \msCc\msNb}}% 
    \var{{\devanagarifont \numnoemph\vb\textbf{सुप्रसन्नं तपः॰}\lem \mssALL, 
सुप्रसन्नतपः॰ \msNb}}% 

{\devanagarifont शीघ्रमाज्ञापय विप्र यच्चाभिलषितं तव  \danda\dontdisplaylinenum }%
     \var{{\devanagarifont \numnoemph\vc\textbf{शीघ्र॰}\lem \mssALL, श्रीघ्र॰ \msNb}}% 

%Verse 12:18

{\devanagarifont अदेयं नास्ति विप्रस्य स्वशिरःप्रभृति द्विज {॥ १२:१८॥} \veg\dontdisplaylinenum }%
     \var{{\devanagarifont \numnoemph\ve\textbf{अदेयं}\lem \mssALL, अदेय \msNb}}% 
    \var{{\devanagarifont \numnoemph\vf\textbf{स्वशिरः॰}\lem \mssALL, शरीर॰ \msNa\oo 
\textbf{॰भृति}\lem \mssALL, ॰भृतिर् \Ed}}% 

{\devanagarifont ब्राह्मण उवाच {\dandab}\dontdisplaylinenum  }%
     \var{{\devanagarifont \numemph\vo\textbf{ब्राह्मण}\lem \mssALL, 
ब्राह्मणा \msCaacorr, ब्रह्म \msNb}}% 

{\devanagarifont यद्येवं वदसे भद्र भार्यां मे देहि रूपिणीम् \thinspace{\danda} \dontdisplaylinenum }%
     \var{{\devanagarifont \numnoemph\vb\textbf{भार्यां}\lem \mssALL, भार्या \msNb\msNc}}% 

%Verse 12:19

{\devanagarifont स्वस्ति भवतु भद्रं वः कल्याणं भव शाश्वतम् {॥ १२:१९॥} \veg\dontdisplaylinenum }%
     \var{{\devanagarifont \numnoemph\vc\textbf{स्वस्ति}\lem \mssALL, स्वस्तिं \msNb, स्वस्तिर् \Ed}}% 
    \var{{\devanagarifont \numnoemph\vd\textbf{कल्याणं}\lem \mssALL, कल्या\uncl{ण} \msCc\oo 
\textbf{भव}\lem \mssALL, तव \Ed}}% 

{\devanagarifont विपुल उवाच {\dandab}\dontdisplaylinenum  }%
     \var{{\devanagarifont \numemph\vo\textbf{विपुल}\lem \mssALL, विप्र \Ed}}% 

{\devanagarifont प्रतीच्छ भार्यां सुश्रोणीं रूपयौवनशालिनीम् \thinspace{\danda} \dontdisplaylinenum }%
     \var{{\devanagarifont \numnoemph\va\textbf{भार्यां}\lem \mssALL, भार्या \msNb\oo 
\textbf{॰श्रोणीं}\lem \msCa\msCb\msNapcorr\msNc\Ed, ॰श्रोणि \msCc\msNaacorr\msNb}}% 
    \var{{\devanagarifont \numnoemph\vb\textbf{॰शालिनीम्}\lem \mssALL, ॰शालिनी \msNb, ॰शीलिनीं \msNc}}% 

%Verse 12:20

{\devanagarifont अकुत्सितां विशालाक्षीं पूर्णचन्द्रनिभाननाम् {॥ १२:२०॥} \veg\dontdisplaylinenum }%
     \var{{\devanagarifont \numnoemph\vc \lem \mssALL, 
अकुत्सि\uncl{ता} विशालाक्षि \msCc, 
अकुत्सिता विशालाक्सी \msNb}}% 
    \var{{\devanagarifont \numnoemph\vd\textbf{॰निभाननाम्}\lem \mssALL, ॰निभानना \msNb}}% 

{\devanagarifont भार्योवाच {\dandab}\dontdisplaylinenum  }%
 
{\devanagarifont परित्याज्या कथं नाथ अपापां त्यजसे कथम् \thinspace{\danda} \dontdisplaylinenum }%
     \var{{\devanagarifont \numemph\va\textbf{॰त्याज्या}\lem \msCa\msNa\msNc\Ed, 
॰त्याज्य \msCb\msNb, ॰त्या\uncl{ज्य} \msCc}}% 

%Verse 12:21

{\devanagarifont अतीव हि प्रियां भार्यां निर्दोषां च कथं त्यजेः {॥ १२:२१॥} \veg\dontdisplaylinenum }%
     \var{{\devanagarifont \numnoemph\vc\textbf{प्रियां}\lem \mssALL, प्रियं \msCc\msNb}}% 
    \var{{\devanagarifont \numnoemph\vd\textbf{निर्दोषां}\lem \mssALL, निर्दोष \msCc\oo 
\textbf{त्यजेः}\lem \msCa\msNa\msNc, त्यज्येत् \msCb\msCc, त्यजेत् \msNb\Ed\oo 
\textbf{च}\lem \conj, स \mssCaCbCc\msNa\msNb\msNc\Ed}}% 

{\devanagarifont सखा भार्या मनुष्याणामिह लोके परत्र च \thinspace{\dandab} \dontdisplaylinenum }%
     \var{{\devanagarifont \numemph\vab\textbf{मनुष्याणामिह}\lem \mssALL, 
मनुष्याणांमिह \msCc}}% 

%Verse 12:22

{\devanagarifont दानं वा सुमहद्दत्त्वा यज्ञो वा सुबहुः कृतः {॥ १२:२२॥} \veg\dontdisplaylinenum }%
     \var{{\devanagarifont \numnoemph\vd\textbf{॰बहुः}\lem \eme, ॰बहु \mssCaCbCc\msNa\msNc\ \unmetr, 
॰बहुं \msNb, ॰बहून् \Ed\oo 
\textbf{कृतः}\lem \mssALL, कृतम् \msCc}}% 

{\devanagarifont अपुत्रो नाप्नुयात्स्वर्गं तपोभिर्वा सुदुष्करैः \thinspace{\dandab} \dontdisplaylinenum }%
     \var{{\devanagarifont \numemph\vab\textbf{स्वर्गं तपोभिर्वा}\lem \mssALL, 
स्व\uncl{र्ग्गन्} \lac  र्व्वा \msCa}}% 

%Verse 12:23

{\devanagarifont श्रुतो मे पितृभिः प्रोक्तो ब्राह्मणैश्च ममान्तिके {॥ १२:२३॥} \veg\dontdisplaylinenum }%
     \var{{\devanagarifont \numnoemph\vd\textbf{॰न्तिके}\lem \mssALL, ॰न्तिकैः \msCb}}% 

{\devanagarifont अपुत्रो नाप्नुयात्स्वर्गं श्रुतं मे बहुशः पुरा \thinspace{\dandab} \dontdisplaylinenum }%
     \var{{\devanagarifont \numemph\va\textbf{स्वर्गं}\lem \msCa\msNa\msNc\Ed, स्वर्ग \msCb\msCc\msNb}}% 

%Verse 12:24

{\devanagarifont मन्दपालो द्विजश्रेष्ठो गतः स्वर्गं तपोबलात् {॥ १२:२४॥} \veg\dontdisplaylinenum }%
     \var{{\devanagarifont \numnoemph\vc\textbf{॰पालो}\lem \msNc\Ed, ॰पाल \mssCaCbCc\msNa\msNb}}% 

{\devanagarifont दानानि च बहून्दत्त्वा यज्ञांश्च विविधांस्तथा \thinspace{\dandab} \dontdisplaylinenum }%
     \var{{\devanagarifont \numemph\va\textbf{बहून्द॰}\lem \mssALL, बहू द॰ \msNc}}% 
    \var{{\devanagarifont \numnoemph\vb \lem \msCa\msCc\msNa\msNb, 
यत्वा यज्ञांश्च विविधां तथा \msCb, 
यज्ञांश्च विविधाम्तथा \msNc, 
स्यज्ञाश्च विविधास्तथा \Ed}}% 

%Verse 12:25

{\devanagarifont वेदांश्च जपयज्ञांश्च कृत्वा स द्विजसत्तमः {॥ १२:२५॥} \veg\dontdisplaylinenum }%
     \var{{\devanagarifont \numnoemph\vc \lem \msCa\msCc\msNa\msNc, 
वेदाश्च जपयज्ञांश्च \msCb, वेदांश्च जपयज्ञाश्च \msNb, 
वेदाश्च जपयज्ञाश्च \Ed}}% 
    \var{{\devanagarifont \numnoemph\vd\textbf{स द्वि॰}\lem \conj, तद्द्वि॰ \mssCaCbCc\msNa\Ed, तद्द्वि॰ \msNb, सद्द्वि॰ \msNc\oo 
\textbf{॰सत्तमः}\lem \mssALL, ॰सत्तम \msNa}}% 

{\devanagarifont प्राप्तद्वारो ऽपि यस्यापि देवदूतैर्निवारितः \thinspace{\dandab} \dontdisplaylinenum }%
     \var{{\devanagarifont \numemph\va\textbf{॰द्वारो}\lem \mssALL,   ॰द्वारे \msNb}}% 
    \var{{\devanagarifont \numnoemph\vab\textbf{यस्यापि दे॰}\lem \mssALL, यस्यापि द्दे॰ \msNb, 
यस्याहि दे॰ \Ed}}% 
    \var{{\devanagarifont \numnoemph\vb\textbf{॰दूतैर्नि॰}\lem \mssALL, ॰दूतै न्नि॰ \msNb, 
॰दूतै नि॰ \msNc}}% 

%Verse 12:26

{\devanagarifont अपुत्रो नाप्नुयात्स्वर्गं यदि यज्ञशतैरपि {॥ १२:२६॥} \veg\dontdisplaylinenum }%
     \var{{\devanagarifont \numnoemph\vc\textbf{॰यात्स्वर्गं}\lem \mssALL, 
॰यात्स्वर्ग्ग \msCc}}% 
    \var{{\devanagarifont \numnoemph\vd\textbf{॰शतैरपि}\lem \mssALL, करोति यः \msCc}}% 

{\devanagarifont इत्युक्तस्तु च्युतः स्वर्गान्मन्दपालो महानृषिः \thinspace{\dandab} \dontdisplaylinenum }%
     \var{{\devanagarifont \numemph\va\textbf{॰क्तस्तु च्युतः}\lem \mssALL, 
॰क्तस्तु\uncl{म्च्यु}तः \msCc}}% 

%Verse 12:27

{\devanagarifont पुत्रानुत्पादयामास शारङ्गांश्चतुरो द्विजः {॥ १२:२७॥} \veg\dontdisplaylinenum }%
     \var{{\devanagarifont \numnoemph\vc\textbf{पुत्रानु॰}\lem \mssALL, पुत्रमु॰ \msCc}}% 
    \var{{\devanagarifont \numnoemph\vd\textbf{शारङ्गांश्च}\lem \msNa\msNc, शारङ्गाश्च \msCa, शारङ्गंश्च \msCb, 
शारङ्गश्च \msCc\msNb, शारङ्गाच्च \Ed\oo 
\textbf{द्विजः}\lem \mssALL, द्विज \msCc}}% 

{\devanagarifont तेन पुण्यप्रभावेण स्वर्गं प्राप्तो ह्यवारितः \thinspace{\dandab} \dontdisplaylinenum }%
     \var{{\devanagarifont \numemph\vb\textbf{स्वर्गं}\lem \mssALL, स्वर्ग्ग \msCc\oo 
\textbf{॰वारितः}\lem \mssALL, ॰वरितः \msNb}}% 

%Verse 12:28

{\devanagarifont कुलत्राणात्कलत्रास्मि भरणाद्भार्य एव च {॥ १२:२८॥} \veg\dontdisplaylinenum }%
     \var{{\devanagarifont \numnoemph\vc\textbf{कुल॰}\lem \msCb, कल॰ \msCa\msCc\msNa\msNb\msNc\Ed\oo 
\textbf{॰त्राणात्क॰}\lem \msNb, ॰त्राणां क॰ \mssCaCbCc\msNa\Ed, ॰त्राणा क॰ \msNc\oo 
\textbf{॰स्मि}\lem \mssALL, ॰स्मिं \msNb}}% 
    \var{{\devanagarifont \numnoemph\vd\textbf{॰आद्भार्य एव}\lem \msCa\msNa\msNc\Ed, 
॰आद्भार्यमेव \msCb, ॰आ भार्य एव \msCc\msNb}}% 

{\devanagarifont दारसंग्रह पुत्रार्थे क्रियते शास्त्रदर्शनात् \thinspace{\dandab} \dontdisplaylinenum }%
     \var{{\devanagarifont \numemph\va\textbf{॰ग्रह}\lem \msCc\msNb\msNc\Ed, ॰ग्रहः \msCa\msCb\msNa\oo 
\textbf{पुत्रा॰}\lem \mssALL, पात्रा॰ \Ed}}% 
    \var{{\devanagarifont \numnoemph\vb\textbf{क्रियते}\lem \mssALL, क्रियाते \msCb}}% 

%Verse 12:29

{\devanagarifont यानि सन्ति गृहे द्रव्यं ग्रामघोषगृहाणि च {॥ १२:२९॥} \veg\dontdisplaylinenum }%
 
{\devanagarifont दातुमर्हसि विप्राय न मां दातुमिहार्हसि \thinspace{\dandab} \dontdisplaylinenum }%
 
%Verse 12:30

{\devanagarifont भार्याया वचनं श्रुत्वा विपुलः पुनरब्रवीत् {॥ १२:३०॥} \veg\dontdisplaylinenum }%
     \var{{\devanagarifont \numemph\vc\textbf{वचनं}\lem \mssALL, वचन \msNc}}% 
    \var{{\devanagarifont \numnoemph\vd\textbf{॰ब्रवीत्}\lem \mssALL, 
॰ब्रवीत्\thinspace{\devanagarifont ।} विपुल उवाच\thinspace{\devanagarifont ।} \msCcpcorr\Ed}}% 

{\devanagarifont साधु भामिनि जानामि साधु साधु पतिव्रते \thinspace{\dandab} \dontdisplaylinenum }%
     \var{{\devanagarifont \numemph\va\textbf{जानामि}\lem \msCb\msCc\msNa\Ed, जानासि \msCa\msNb\msNc}}% 
    \var{{\devanagarifont \numnoemph\vb\textbf{पति॰}\lem \mssALL, प्रति॰ \msNb}}% 

%Verse 12:31

{\devanagarifont जितो ऽस्म्यनेन वाक्येन अनेनास्मि हि तोषितः {॥ १२:३१॥} \veg\dontdisplaylinenum }%
     \var{{\devanagarifont \numnoemph\vd\textbf{तोषितः}\lem \mssALL, तोर्षिनः \msNc}}% 

{\devanagarifont अद्य ग्रहणकाले च द्विज आगत्य याचते \thinspace{\dandab} \dontdisplaylinenum }%
 
%Verse 12:32

{\devanagarifont ददामीति प्रतिज्ञाय अदत्त्वा नरकं व्रजे {॥ १२:३२॥} \veg\dontdisplaylinenum }%
     \var{{\devanagarifont \numemph\vd\textbf{व्रजे}\lem \msCa\msNapcorr\msNc, व्रजेत् \msCb\msCc\msNb\Ed, 
व्रजे\lk\ \msNaacorr}}% 

{\devanagarifont नरकं यदि गच्छामि कुलेन सह सुन्दरि \thinspace{\dandab} \dontdisplaylinenum }%
     \var{{\devanagarifont \numemph\va\textbf{यदि}\lem \mssALL, ययदि \msNc}}% 

{\devanagarifont कल्पकोटिसहस्रे ऽपि नरकस्थो यशस्विनि  \danda\dontdisplaylinenum }%
     \var{{\devanagarifont \numnoemph\vc\textbf{॰सहस्रे ऽपि}\lem \mssALL, ॰सहस्राणि \msCc\Ed}}% 
    \var{{\devanagarifont \numnoemph\vd\textbf{॰स्थो य॰}\lem \msNc\Ed, ॰स्थाद्य॰ \msCa\msCc\msNa\msNb, स्था य॰ \msCb}}% 

%Verse 12:33

{\devanagarifont मुक्तिमेव न पश्यामि जन्मकोटिशतैरपि {॥ १२:३३॥} \veg\dontdisplaylinenum }%
     \var{{\devanagarifont \numnoemph\ve\textbf{मुक्तिमेव}\lem \mssALL, मुक्तिमेवन् \Ed}}% 

{\devanagarifont अदानाच्चाशुभं देवि पश्यामि वरवर्णिनि \thinspace{\dandab} \dontdisplaylinenum }%
     \var{{\devanagarifont \numemph\va\textbf{अदानाच्चा॰}\lem \mssALL, अदाना चा॰ \msCc}}% 

%Verse 12:34

{\devanagarifont दानेन तु शुभं पश्ये स्वर्गलोके यदक्षयम् {॥ १२:३४॥} \veg\dontdisplaylinenum }%
     \var{{\devanagarifont \numnoemph\vd\textbf{॰लोके}\lem \mssALL, 
\om\ \msNaacorr, ॰लोकं \Ed}}% 

{\devanagarifont नोक्तं मयानृतं पूर्वं नित्यं सत्यव्रते स्थितः \thinspace{\dandab} \dontdisplaylinenum }%
     \var{{\devanagarifont \numemph\va\textbf{नोक्तं}\lem \mssALL, नोक्ता \msNcacorr}}% 
    \var{{\devanagarifont \numnoemph\vb\textbf{॰व्रते}\lem \mssALL, ॰व्रत॰ \Ed}}% 

%Verse 12:35

{\devanagarifont सत्यधर्ममतिक्रम्य नान्यधर्मं समाचरे {॥ १२:३५॥} \veg\dontdisplaylinenum }%
     \var{{\devanagarifont \numnoemph\vd\textbf{॰चरे}\lem \mssALL, ॰चरेत् \msNb\Ed}}% 

{\devanagarifont भार्या धर्मसखेत्येवं त्वया पूर्वमुदाहृतम् \thinspace{\dandab} \dontdisplaylinenum }%
     \var{{\devanagarifont \numemph\va\textbf{धर्म॰}\lem \mssALL, धर्मं \msNa}}% 
    \var{{\devanagarifont \numnoemph\vb\textbf{त्वया}\lem \eme, त्वयि \mssCaCbCc\msNa\msNb\msNc\Ed}}% 

%Verse 12:36

{\devanagarifont यदि धर्मसखायासि सो ऽद्य काल इहागतः {॥ १२:३६॥} \veg\dontdisplaylinenum }%
     \var{{\devanagarifont \numnoemph\vc\textbf{॰सखाया॰}\lem \mssALL, ॰सखा॰ \msCb}}% 

{\devanagarifont द्विजरूपधरो धर्मः स्वयमेव इहागतः \thinspace{\dandab} \dontdisplaylinenum }%
     \var{{\devanagarifont \numemph\va\textbf{॰धरो}\lem \mssALL, ॰परो \msCb}}% 

%Verse 12:37

{\devanagarifont जिज्ञासार्थमहं भद्रे न विघ्नं कर्तुमर्हसि {॥ १२:३७॥} \veg\dontdisplaylinenum }%
     \var{{\devanagarifont \numnoemph\vc\textbf{॰र्थमहं}\lem \mssALL, 
॰र्थम्महं \msNb, ॰र्थमह \msNc}}% 

{\devanagarifont माताव्यक्तः पिता ब्रह्मा बुद्धिर्भार्या दमः सखा \thinspace{\dandab} \dontdisplaylinenum }%
     \var{{\devanagarifont \numemph\va\textbf{॰व्यक्तः}\lem \mssALL, 
॰व्यक्त \msCc, ॰व्यक्त\uncl{ऽ} \msNc}}% 
    \var{{\devanagarifont \numnoemph\vb\textbf{बुद्धिर्भा॰}\lem \msCa\msCb\msNb, बुद्धि भा॰ \msCc\msNa\msNc\Ed\oo 
\textbf{दमः}\lem \mssALL, दम \msNb\ \unmetr\oo 
\textbf{सखा}\lem \mssALL, समा \msCa}}% 

%Verse 12:38

{\devanagarifont पुत्रो धर्मः क्रियाचार्य इत्येते मम बान्धवाः {॥ १२:३८॥} \veg\dontdisplaylinenum }%
 
{\devanagarifont कालश्रेष्ठो ग्रहः सूर्यो गङ्गा श्रेष्ठा नदीषु च \thinspace{\dandab} \dontdisplaylinenum }%
     \var{{\devanagarifont \numemph\va\textbf{॰श्रेष्थो}\lem \msCb\msNa\msNcpcorr, ॰श्रेष्ठ॰ \msCa\msCc\msNb, ॰श्रेष्ठा \msNcacorr, ॰श्रेष्ठः \Ed}}% 
    \var{{\devanagarifont \numnoemph\vb\textbf{श्रेष्ठा}\lem \mssALL, श्रेष्ठो \msNa, श्रेष्ठ \msNb}}% 
    \paral{{\devanagarifontsmall \vb {\englishfont \similar\ 15.18b:} श्रेष्ठा गङ्गा नदीषु च }}

%Verse 12:39

{\devanagarifont चन्द्रक्षये दिनं श्रेष्ठं नरश्रेष्ठो द्विजोत्तमः {॥ १२:३९॥} \veg\dontdisplaylinenum }%
     \var{{\devanagarifont \numnoemph\vc\textbf{दिनं}\lem \msCa\msCb\msNa\msNc, दिन॰ \msCc\msNb\Ed}}% 
    \var{{\devanagarifont \numnoemph\vd\textbf{॰त्तमः}\lem \mssALL, ॰त्तम \msCc}}% 

{\devanagarifont शुश्रूषणार्थं विप्रस्य मया दत्तासि सुन्दरि \thinspace{\dandab} \dontdisplaylinenum }%
     \var{{\devanagarifont \numemph\va\textbf{॰र्थं}\lem \mssALL, ॰र्थ \msCb}}% 

%Verse 12:40

{\devanagarifont सर्वस्वं ब्राह्मणे दत्त्वा वनमेवाश्रयाम्यहम् {॥ १२:४०॥} \veg\dontdisplaylinenum }%
 
{\devanagarifont शङ्कर उवाच {\dandab}\dontdisplaylinenum  }%
     \var{{\devanagarifont \numemph\vo\textbf{शङ्कर}\lem \mssALL, महेश्वर \Ed}}% 

{\devanagarifont तूष्णीम्भूता ततो भार्या अश्रुपूर्णाकुलेक्षणा \thinspace{\danda} \dontdisplaylinenum }%
     \var{{\devanagarifont \numnoemph\va\textbf{तूष्णीम्भूता}\lem \msCa, तूष्णीभूत्वा \msCb, तुष्णीभूत \msCc, तूष्णीभूता \msNa\msNb, 
तुष्णीम्भूती \msNc, तूष्णीभूतां \Ed\oo 
\textbf{भार्या}\lem \mssALL, भार्यां \Ed}}% 
    \var{{\devanagarifont \numnoemph\vb\textbf{॰क्षणा}\lem \msCa\msCb\msNa\msNc, ॰क्षणः \msCc, ॰क्षणाः \msNb, ॰क्षणाम् \Ed}}% 

%Verse 12:41

{\devanagarifont करे गृह्य विशालाक्षी ब्राह्मणाय निवेदिता {॥ १२:४१॥} \veg\dontdisplaylinenum }%
     \var{{\devanagarifont \numnoemph\vc\textbf{॰क्षी}\lem \mssALL, ॰क्षीं \Ed}}% 
    \var{{\devanagarifont \numnoemph\vd \lem \mssALL, 
ब्राह्मय दिवेदिता \msCb}}% 

{\devanagarifont यानि सन्ति गृहे द्रव्यं हिरण्यं पशवस्तथा \thinspace{\dandab} \dontdisplaylinenum }%
     \var{{\devanagarifont \numemph\vb\textbf{हिरण्यं}\lem \mssALL, हिरण्य॰ \msNa\Ed}}% 

%Verse 12:42

{\devanagarifont ददामि ते द्विजश्रेष्ठ ग्रामघोषगृहादिकम् {॥ १२:४२॥} \veg\dontdisplaylinenum }%
     \var{{\devanagarifont \numnoemph\vc\textbf{ददामि}\lem \mssALL, ददानि \msCb\oo 
\textbf{ते द्विज॰}\lem \mssALL, \lac  ज॰ \msCa, त द्विज॰ \msNc}}% 

{\devanagarifont मुक्तावैडूर्यवासांसि दिव्याण्याभरणानि च \thinspace{\dandab} \dontdisplaylinenum }%
     \var{{\devanagarifont \numemph\va\textbf{॰वैडूर्य॰}\lem \msCa\msCb\msNb\msNc, ॰वैभार्य॰ \msCc, ॰वैर्य॰ \msNaacorr, 
॰वैदूर्य॰ \msNapcorr\Ed\oo 
\textbf{॰वासांसि}\lem \mssALL, ॰वासासि \msNc}}% 

%Verse 12:43

{\devanagarifont सर्वान्गृहाण विप्रेन्द्र श्रद्धया दत्तसत्कृतान् {॥ १२:४३॥} \veg\dontdisplaylinenum }%
     \var{{\devanagarifont \numnoemph\vc\textbf{सर्वान्गृहाण}\lem \msCa\msCb\msNa\Ed, सर्वान्तान्गृह्ण \msCc, 
सर्वान्गृहान् \msNb, 
सर्वां गृहाण \msNc}}% 
    \var{{\devanagarifont \numnoemph\vd\textbf{॰सत्कृतान्}\lem \eme, ॰सत्कृताम् \mssCaCbCc\msNa\msNc\Ed, ॰सत्कृतम् \msNb}}% 

{\devanagarifont प्रीयतां भगवान्धर्मः प्रीयतां च महेश्वरः \thinspace{\dandab} \dontdisplaylinenum }%
     \var{{\devanagarifont \numemph\vb\textbf{प्रीय॰}\lem \mssALL, प्रीन॰ \msNcacorr}}% 

%Verse 12:44

{\devanagarifont प्रीयन्तां पितरः सर्वे यद्यस्ति सुकृतं फलम् {॥ १२:४४॥} \veg\dontdisplaylinenum }%
     \var{{\devanagarifont \numnoemph\vc\textbf{प्रीयन्तां}\lem \msCa, प्रीयतां \msCb\msCc\msNa\msNc\Ed, प्रीयता \msNb\oo 
\textbf{पितरः}\lem \mssALL, पितर \msNa}}% 
    \var{{\devanagarifont \numnoemph\vd\textbf{अस्ति}\lem \mssALL, असि \msCa}}% 

{\devanagarifont रुद्र उवाच {\dandab}\dontdisplaylinenum  }%
     \var{{\devanagarifont \numemph\vo\textbf{रुद्र}\lem \mssALL, महेश्वर \Ed}}% 

{\devanagarifont विपुलस्य वचः श्रुत्वा ब्राह्मणेन तपस्विना \thinspace{\danda} \dontdisplaylinenum }%
     \var{{\devanagarifont \numnoemph\va\textbf{वचः श्रुत्वा}\lem \mssALL, 
वच\uncl{श्श्रु}\lac\  \msCa}}% 
    \var{{\devanagarifont \numnoemph\vb\textbf{तपस्विना}\lem \mssALL, तपस्विनाम् \msNb}}% 

%Verse 12:45

{\devanagarifont आशीः सुविपुलं दत्त्वा विपुलाय महात्मने {॥ १२:४५॥} \veg\dontdisplaylinenum }%
 
{\devanagarifont वसेत्तत्र गृहे रम्ये भार्यामादाय तस्य च \thinspace{\dandab} \dontdisplaylinenum }%
     \var{{\devanagarifont \numemph\va\textbf{वसेत्तत्र गृहे}\lem \msCb\msNa, वस तत्र गृहे \msCa\msCc\msNb, 
वस्\uncl{एन्त}त्र गृहे \msNc, 
वसते च गृहं \Ed}}% 

%Verse 12:46

{\devanagarifont विपुलस्तु नमस्कृत्वा कृत्वा चापि प्रदक्षिणम् {॥ १२:४६॥} \veg\dontdisplaylinenum }%
     \var{{\devanagarifont \numnoemph\vc\textbf{विपुलस्तु}\lem \mssALL, विपुलस्य \msNb}}% 
    \var{{\devanagarifont \numnoemph\vd\textbf{कृत्वा चापि}\lem \mssALL, \lk\lk \lk\lk\ \msNc, 
कृत्वा च वि॰ \Ed}}% 

{\devanagarifont ब्राह्मणमभिवाद्यैवं गतः शीघ्रं वनान्तरम् \thinspace{\dandab} \dontdisplaylinenum }%
     \var{{\devanagarifont \numemph\va\textbf{ब्राह्मण॰}\lem \mssALL, ब्राह्मणा॰ \msNb\oo 
\textbf{॰द्यैवं}\lem \eme, ॰द्येवं \msCa\msCc\msNa\msNb\Ed, ॰द्येनं \msCb, 
॰द्यवं \msNc}}% 
    \var{{\devanagarifont \numnoemph\vb\textbf{शीघ्रं}\lem \mssALL, श्रीघ्रं \msNb}}% 

%Verse 12:47

{\devanagarifont वने मूलफलाहारो विचरेत महीतले {॥ १२:४७॥} \veg\dontdisplaylinenum }%
     \var{{\devanagarifont \numnoemph\vc\textbf{॰फलाहारो}\lem \mssALL, 
॰फाहारो \msNcacorr}}% 

{\devanagarifont एकाकी विजने शून्ये चिन्तया च परिप्लुतः \thinspace{\dandab} \dontdisplaylinenum }%
     \var{{\devanagarifont \numemph\va\textbf{एकाकी}\lem \mssALL, 
ए\uncl{का}\lac\  \msCa}}% 
    \var{{\devanagarifont \numnoemph\vb\textbf{परि॰}\lem \mssALL, पलि॰ \msNc}}% 

%Verse 12:48

{\devanagarifont क्व गच्छामि क्व भोक्ष्यामि कुत्र वा किं करोम्यहम् {॥ १२:४८॥} \veg\dontdisplaylinenum }%
     \var{{\devanagarifont \numnoemph\vc\textbf{क्व गच्छामि}\lem \mssALL, क्ष गच्छामि \msNc\oo 
\textbf{क्व भोक्ष्यामि}\lem \msCa, क्व भोज्यामि \msCb\msNa\msNb, क्व भोक्ष्यानि \msCc, 
क्व भोक्षामि \msNc, किं भोक्ष्यामि \Ed\ \unmetr}}% 

{\devanagarifont न पथं विषयं वेद्मि ग्रामं वा नगराणि वा \thinspace{\dandab} \dontdisplaylinenum }%
     \var{{\devanagarifont \numemph\va\textbf{विषयं वेद्मि}\lem \msCa\msNa\msNb\Ed, विषमं वेद्मि \msCb\msCc, वियषं वे\uncl{श्मि} \msNc}}% 
    \var{{\devanagarifont \numnoemph\vb\textbf{वा}\lem \mssALL, च \msCb\msNa}}% 

%Verse 12:49

{\devanagarifont खेटखर्वटदेशं वा जानामीह न कंचन {॥ १२:४९॥} \veg\dontdisplaylinenum }%
     \var{{\devanagarifont \numnoemph\vc\textbf{खेट॰}\lem \mssALL, क्षेत्र॰ \msCc\oo 
\textbf{॰खर्वट॰}\lem \Ed, ॰कर्पट॰ \mssCaCbCc\msNa\msNb\msNc}}% 
    \var{{\devanagarifont \numnoemph\vd\textbf{कंचन}\lem \eme, कश्चन \mssCaCbCc\msNa\msNb\msNc\Ed}}% 

{\devanagarifont अमुं सुशैलं पश्यामि विपुलोदरकन्दरम् \thinspace{\dandab} \dontdisplaylinenum }%
     \var{{\devanagarifont \numemph\va\textbf{सुशैलं}\lem \mssALL, सुशेलं \msNc}}% 
    \var{{\devanagarifont \numnoemph\vb\textbf{विपुलो॰}\lem \mssALL, विलो॰ \msNb}}% 

%Verse 12:50

{\devanagarifont तमारुह्य निरीक्ष्यामि ग्रामं नगरपत्तनम् {॥ १२:५०॥} \veg\dontdisplaylinenum }%
     \var{{\devanagarifont \numnoemph\vc\textbf{निरीक्ष्यामि}\lem \mssALL, निरीक्षामि \msNc}}% 

{\devanagarifont एवमुक्त्वा तु विपुलः शनैः पर्वतमारुहत् \thinspace{\dandab} \dontdisplaylinenum }%
     \var{{\devanagarifont \numemph\va\textbf{एवमु॰}\lem \mssALL, एकं उ॰ \msCb}}% 
    \var{{\devanagarifont \numnoemph\vb\textbf{॰रुहत्}\lem \Ed, ॰रुहेत् \mssCaCbCc\msNa\msNb\msNc}}% 

%Verse 12:51

{\devanagarifont वृक्षच्छायां समालोक्य निषसाद श्रमान्वितः {॥ १२:५१॥} \veg\dontdisplaylinenum }%
     \var{{\devanagarifont \numnoemph\vc\textbf{॰च्छायां}\lem \mssALL, ॰च्छाया \msNc}}% 

{\devanagarifont एतस्मिन्नेव काले तु वृक्षशाखावतार्य च \thinspace{\dandab} \dontdisplaylinenum }%
     \var{{\devanagarifont \numemph\va\textbf{एतस्मिन्नेव}\lem \mssALL, एतस्मिंनैव \msCc, एतस्मिन्नैव \msNc\oo 
\textbf{काले तु}\lem \msCa\msCb\msNa\msNb, कालेन \msCc\Ed, कालेनु \msNc}}% 
    \var{{\devanagarifont \numnoemph\vb\textbf{वृक्ष॰}\lem \mssALL, वृक्षा॰ \msNa\msNcacorr}}% 

%Verse 12:52

{\devanagarifont अपूर्वं च सुरूपं च सुगन्धत्वं च शोभनम् {॥ १२:५२॥} \veg\dontdisplaylinenum }%
     \var{{\devanagarifont \numnoemph\vc\textbf{सुरूपं}\lem \mssALL, स्वरूपं \msCb\msNa}}% 

{\devanagarifont फलं गृह्य विचित्रं च हृदयानन्दनं शुभम् \thinspace{\dandab} \dontdisplaylinenum }%
 
%Verse 12:53

{\devanagarifont विपुलस्याग्रतः कृत्वा पुनर्वृक्षं समारुहत् {॥ १२:५३॥} \veg\dontdisplaylinenum }%
     \var{{\devanagarifont \numemph\vd \lem \mssALL, 
पुन वृक्ष समारुहम् \msCc, 
पुनर्वृक्ष समारुहं \msNb}}% 

{\devanagarifont विपुलश्चित्रवद्दृष्ट्वा विस्मयं परमं गतः \thinspace{\dandab} \dontdisplaylinenum }%
     \var{{\devanagarifont \numemph\va\textbf{॰त्रवद्दृष्ट्वा}\lem \mssALL, ॰त्रव दृष्ट्वा \msCc}}% 

%Verse 12:54

{\devanagarifont अहो वा स्वप्नभूतो ऽस्मि अहो वा तपसः फलम् {॥ १२:५४॥} \veg\dontdisplaylinenum }%
     \var{{\devanagarifont \numnoemph\vcd\textbf{॰भूतो ऽस्मि अहो}\lem \mssALL, ॰संभूतो \uncl{स्म्य}हो \msNa}}% 

{\devanagarifont न पश्यामि न जिघ्रामि न च स्वादं च वेद्म्यहम् \thinspace{\dandab} \dontdisplaylinenum }%
     \var{{\devanagarifont \numemph\va\textbf{जिघ्रामि}\lem \mssALL, च घ्रामि \msCb}}% 

%Verse 12:55

{\devanagarifont वार्त्तापि न च मे श्रोता प्रतिजानामि कंचन {॥ १२:५५॥} \veg\dontdisplaylinenum }%
     \var{{\devanagarifont \numnoemph\vc\textbf{श्रोता}\lem \mssALL, श्रोत्रा \msCa}}% 
    \var{{\devanagarifont \numnoemph\vd\textbf{कंचन}\lem \eme, कश्चन \mssCaCbCc\msNa\msNb\msNc\Ed}}% 

{\devanagarifont एवमुक्त्वा ह्यनेकानि फलं गृह्य मनोरमम् \thinspace{\dandab} \dontdisplaylinenum }%
     \var{{\devanagarifont \numemph\va\textbf{॰मुक्त्वा}\lem \mssALL, ॰मुक्ता \msCc}}% 
    \var{{\devanagarifont \numnoemph\vb\textbf{गृह्य}\lem \mssALL, गृह \msNc}}% 

%Verse 12:56

{\devanagarifont सुनिरीक्ष्य पुनर्जिघ्रन् पुनर्जिघ्रन्निरीक्ष्य च {॥ १२:५६॥} \veg\dontdisplaylinenum }%
     \var{{\devanagarifont \numnoemph\vc\textbf{॰निरीक्ष्य}\lem \mssALL, ॰निरीक्ष \msNc}}% 
    \var{{\devanagarifont \numnoemph\vcd\textbf{पुनर्जिघ्रन्पुनर्जिघ्रन्}\lem \msCa\msCb\msNa\Ed, 
मुन जिघ्रं पुन जिघ्रं \msCc, 
पुनर्जिघ्र पुनर्जिघ्रं \msNb, 
पुनर्जिघ्र पुनर्जिघ्र \msNc}}% 
    \var{{\devanagarifont \numnoemph\vd\textbf{निरीक्ष्य}\lem \mssALL, निरीक्ष \msNc}}% 

{\devanagarifont फलं चात्र निरूप्यन्तो देशं वाप्यवलोकयन् \thinspace{\dandab} \dontdisplaylinenum }%
     \var{{\devanagarifont \numemph\va\textbf{चात्र}\lem \mssALL, 
चा \msCaacorr, चा\uncl{त्र} \msCapcorr\oo 
\textbf{निरूप्यन्तो}\lem \Ed, निरूप्यान्ति \msCa, निरूप्यां चा \msCb, 
निरूप्यन्ति \msCc\msNa\msNb\msNc}}% 
    \var{{\devanagarifont \numnoemph\vb\textbf{॰लोकयन्}\lem \mssALL, ॰लोकयत् \msCb}}% 

%Verse 12:57

{\devanagarifont पाथेयरहितश्चास्मि देवदत्तं फलं मम {॥ १२:५७॥} \veg\dontdisplaylinenum }%
     \var{{\devanagarifont \numnoemph\vc\textbf{पाथेय॰}\lem \mssALL, पथेय॰ \msNb\oo 
\textbf{॰रहितश्चा॰}\lem \mssALL, ॰रहिते चा॰ \msCc}}% 
    \var{{\devanagarifont \numnoemph\vd\textbf{॰दत्तं}\lem \msCa\msNa\msNc, ॰दत्त॰ \msCb\msCc\msNb\Ed\oo 
\textbf{फलं}\lem \mssALL, \om\ \msNc}}% 

{\devanagarifont तत्फलं प्रतिगृह्यैव नगरं प्रविशाम्यहम् \thinspace{\dandab} \dontdisplaylinenum }%
     \var{{\devanagarifont \numemph\va\textbf{॰गृह्यैव}\lem \msCb\msNb\Ed, ॰गृह्येव \msCa\msNc, गृहे च \msCc, ॰गृह्यैवं \msNa}}% 

%Verse 12:58

{\devanagarifont प्रार्थयित्वा तु यत्किंचिज्जीवनार्थं चराम्यहम् {॥ १२:५८॥} \veg\dontdisplaylinenum }%
     \var{{\devanagarifont \numnoemph\vc\textbf{तु}\lem \mssALL, च \Ed}}% 
    \var{{\devanagarifont \numnoemph\vcd\textbf{यत्किंचिज्जी॰}\lem \mssALL, 
यत्किंजि जी॰ \msCc}}% 

{\devanagarifont ततः शैलमतिक्रम्य नगरं प्रविवेश ह \thinspace{\dandab} \dontdisplaylinenum }%
 
%Verse 12:59

{\devanagarifont पथि कश्चिज्जनः पृष्ठः किंनाम नगरं त्विदम् {॥ १२:५९॥} \veg\dontdisplaylinenum }%
     \var{{\devanagarifont \numemph\vd\textbf{नगरं त्विदम्}\lem \msCa\msNa\msNc\Ed, 
नगर त्विदम् \msCb\msCc, नगरं त्विह \msNb}}% 

{\devanagarifont स होवाच पथीकेन किमपूर्वमिहागतः \thinspace{\dandab} \dontdisplaylinenum }%
     \var{{\devanagarifont \numemph\va\textbf{स हो॰}\lem \mssALL, अहो॰ \msCb\msNb\oo 
\textbf{पथीकेन}\lem \mssALL, पथीको न \msNc}}% 
    \var{{\devanagarifont \numnoemph\vb\textbf{॰गतः}\lem \mssALL, ॰तवः \msNb}}% 

%Verse 12:60

{\devanagarifont दक्षिणापथदेशो ऽयं नरवीरपुरं त्वदः {॥ १२:६०॥} \veg\dontdisplaylinenum }%
     \var{{\devanagarifont \numnoemph\vc\textbf{॰पथ॰}\lem \mssALL, ॰पथे \msCb}}% 
    \var{{\devanagarifont \numnoemph\vd\textbf{॰पुरं त्वदः}\lem \msCb, ॰पुरं त्वयः \msCa, ॰पुरं त्वयं \msCc\msNa\msNb, 
पुरन्दरः \msNc, ॰पुरं स्वयम् \Ed}}% 
