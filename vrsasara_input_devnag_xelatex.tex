\fejno=0\versno=0
\centerline{\Huge\devanagarifontbold वृषसारसंग्रहः  }

 
{\vrule depth10pt width0pt}
\versno=0\fejno=1
\thispagestyle{empty}

\centerline{\Large\devanagarifontbold [   प्रथमो ऽध्यायः  ]}{\vrule depth10pt width0pt} \fancyhead[CE]{{\footnotesize\devanagarifont वृषसारसंग्रहे  }}
\fancyhead[CO]{{\footnotesize\devanagarifont प्रथमो ऽध्यायः  }}
\fancyhead[LE]{}
\fancyhead[RE]{}
\fancyhead[LO]{}
\fancyhead[RO]{}
\szam\bek


\vers



\alalfejezet{स्तुतिः}
\ujvers\nemsloka {
{\devanagarifont अनादिमध्यान्तमनन्तपारं }%
  \dontdisplaylinenum}    \var{{\devanagarifontvar\numemph\va ॰न्तमनन्त॰\lem \mssALL,\hskip.2em plus .9em 
॰न्तमन्त॰ \msCbacorr\oo 
 ॰पारं\lem \mssCaCbCc\msNc\msParis\msM\msKOb\msPaperA\msPaperC\Ed,\hskip.2em plus .9em 
॰पारगं \msNa\msNb\msNd\msKOa}}% 
    \paral{{\devanagarifontsmall \va {\englishfont \compare\ \SDHU\ 10.6:}
                 आदिमध्यान्तनिर्मुक्तः स्वभावविमलः प्रभुः\thinspace{\devanagarifontsmall ।}
                 सर्वज्ञः परिपूर्णश्च शिवो ज्ञेयः शिवागमे\thinspace{\devanagarifontsmall ॥} }}


\nemslokab

{\devanagarifont सुसूक्ष्ममव्यक्तजगत्सुसारम्  \danda\dontdisplaylinenum }%
     \var{{\devanagarifontvar\numnoemph\vb सुसूक्ष्म॰\lem \mssALL,\hskip.2em plus .9em 
शुसुक्ष्म॰ \msCc\oo 
 ॰व्यक्त॰\lem \mssALL,\hskip.2em plus .9em 
॰व्य॰ \msKOa\oo 
 ॰जगत्सुसारम्\lem \msCa\msCb\msNa\msNc\msParis\msM\msKOa\msKOb\msPaperA\msPaperC\Ed,\hskip.2em plus .9em 
॰जगशुसारं \msCc,\hskip.5em plus .9em 
॰जगत्सुरासुरं \msNb,\hskip.5em plus .9em 
॰जगतसुसारम् \msNd}}% 

\nemslokac

{\devanagarifont हरीन्द्रब्रह्मादिभिरासमग्रं }%
  \dontdisplaylinenum    \var{{\devanagarifontvar\numnoemph\vc हरी॰\lem \mssALL,\hskip.2em plus .9em 
हरीं \msKOa\oo 
 ॰भिरासमग्रं\lem \mssALL,\hskip.2em plus .9em 
॰भिर्यत्समग्रं \msM\ \unmetr,\hskip.5em plus .9em 
॰भिरोसमग्रं \msPaperC}}% 

%Verse 1:1


\nemslokad

{\devanagarifont प्रणम्य वक्ष्ये वृषसारसंग्रहम् {॥ १:\hspace{.11em}१॥} \veg\dontdisplaylinenum }%
     \var{{\devanagarifontvar\numnoemph\vd वृष॰\lem \mssALL,\hskip.2em plus .9em 
॰वृषो \msCaacorr}}% 
    \lacuna{\devanagarifontsmall {\englishfont Witnesses used for this chapter:       \msCa\ ff.\thinspace 193v--195v,
                                                        \msCb\ ff.\thinspace 201v--203v,
                                                        \msCc\ ff.\allowbreak\thinspace 267r--270r,
                                                        \msNa\ ff.\thinspace 1v--3v,
                                                        \msNb\ exp.\thinspace 44, 43 lower and then upper leaf
                                                                              (1.62cd--2.22 are missing),
                                                        \msNc\  ff.\thinspace 209v--211v,
                                                        \msNd\  ff.\thinspace 227v--229v 
                                                                                (collated only up to 1.15ab),
                                                        \msParis\ ff.\thinspace 212v--213v (after which there is a lacuna),
                                                        \msM\   ff.\thinspace 1r--3v,
                                                        \msKOa\ ff.\thinspace 1v--4r 
                                                                                (collated only up to 1.16),
                                                        \msKOb\ ff.\thinspace 210v--212v 
                                                                                (collated only up to 1.16),
                                                        \msPaperA\ ff.\thinspace 204r--206r,
                                                        \msPaperC\ ff.\thinspace 206r--209r 
                                                                                (collated only up to 1.15),
                                                        \Ed\ pp.\thinspace 580--585;
                                                        \mssCaCbCc\ = \msCa + \msCb + \msCc} }%
  
\pend
\endnumbering
\vfill\pagebreak\beginnumbering\pstart
\vers


\alalfejezet{जनमेजयवैशम्पायनसंवादः}
\vers


{\devanagarifont शतसाहस्रिकं ग्रन्थं सहस्राध्यायमुत्तमम् \thinspace{\dandab} \dontdisplaylinenum }%
     \var{{\devanagarifontvar\numemph\va ॰स्रिकं\lem \mssALL,\hskip.2em plus .9em 
॰स्रकं \msPaperA\oo 
 ग्रन्थं\lem \mssALL,\hskip.2em plus .9em 
ग्रंथ \msKOa}}% 
    \var{{\devanagarifontvar\numnoemph\vb सहस्राध्यायमु॰\lem \mssALL,\hskip.2em plus .9em 
सहश्रध्यायमु॰ \msCc,\hskip.5em plus .9em 
सहस्राध्यायरु॰ \Ed}}% 

%Verse 1:2

{\devanagarifont पर्व चास्य शतं पूर्णं श्रुत्वा भारतसंहिताम् {॥ १:\hspace{.11em}२॥} \veg\dontdisplaylinenum }%
     \var{{\devanagarifontvar\numnoemph\vc पर्व चास्य\lem \msCa\msNa\msNb\msNc\msParis\msMpcorr\msKOb,\hskip.2em plus .9em पर्वञ्चास्य \msCb,\hskip.5em plus .9em 
पर्वमस्य \msCc\msNd\msMacorr\msPaperA\msPaperC\Ed,\hskip.5em plus .9em पूर्व चास्य \msKOa\oo 
 शतं पूर्णं\lem \mssALL,\hskip.2em plus .9em 
त \msCc,\hskip.5em plus .9em शतं पूर्ण्ण \msKOa}}% 
    \var{{\devanagarifontvar\numnoemph\vd श्रुत्वा\lem \mssALL,\hskip.2em plus .9em 
श्रद्धा \msCb\oo 
 भारतसंहिताम्\lem \msCa\msCb\msNa\msNb\msNc\msM\msKOa,\hskip.2em plus .9em 
भारसंहिता \msCc,\hskip.5em plus .9em भारतसंहितं \msNd\msParis\msKOb,\hskip.5em plus .9em 
नारदसंहिताम् \msPaperA\msPaperC\Ed}}% 
    \paral{{\devanagarifontsmall \vc {\englishfont \compare\ \MBH\ 1.2.70ab:} एतत्पर्वशतं पूर्णं व्यासेनोक्तं महात्मना }}

\vers


{\devanagarifont अतृप्तः पुन पप्रच्छ वैशम्पायनमेव हि \thinspace{\dandab} \dontdisplaylinenum }%
     \var{{\devanagarifontvar\numemph\va अतृप्तः पुन पप्रच्छ\lem \eme,\hskip.2em plus .9em 
अ\uncl{तृप्तः पु}\lk\lk प्रच्छ \msCa,\hskip.5em plus .9em 
अतृप्तः पुनः पप्रच्छ \msCb\msNa\msNb\msNc\msParis\msKOb\ \unmetr,\hskip.5em plus .9em 
अतृप्तः पुनरप्रच्छे \msCc,\hskip.5em plus .9em 
अतृप्तः पुन पःप्रच्छ \msNd,\hskip.5em plus .9em 
अतृप्तः पुनः पपृच्छ \msM,\hskip.5em plus .9em 
पप्रच्छ पुनरतृप्तो \msKOa,\hskip.5em plus .9em 
अतृप्ताः पुनः पप्रेच्छ \msPaperA,\hskip.5em plus .9em 
अतृप्त पुनः पप्रच्छ \msPaperC,\hskip.5em plus .9em 
अतृप्ता पुनः पप्रच्छ \Ed}}% 
    \var{{\devanagarifontvar\numnoemph\vb वैशम्पायन॰\lem \mssALL,\hskip.2em plus .9em 
वेसम्पायन॰ \msCc}}% 

%Verse 1:3

{\devanagarifont जनमेजयेन यत्पूर्वं तच्छृणु त्वमतन्द्रितम् {॥ १:\hspace{.11em}३॥} \veg\dontdisplaylinenum }%
     \var{{\devanagarifontvar\numnoemph\vc जनमेजयेन यत्पूर्वं\lem \msCapcorr\msCb\msNc\msNd\msParispcorr\msKOb\msPaperA\msPaperC\Ed,\hskip.2em plus .9em 
जनमेजये यत्पूर्वं \msCaacorr,\hskip.5em plus .9em 
जन्मेजयेन यम्पूर्वं \msCc,\hskip.5em plus .9em 
जनमेजयेन यत्पूर्व \msNa,\hskip.5em plus .9em 
जनमेजयेन यत्पू\uncl{र्व} \msNb,\hskip.5em plus .9em 
जनमेजनयेन यत्पूर्वं \msParisacorr,\hskip.5em plus .9em 
जन्मेजयेण यत्पूर्वं \msM,\hskip.5em plus .9em 
जन्मेजयेन य\lacwithnum{2}\ \msKOa}}% 
    \var{{\devanagarifontvar\numnoemph\vd तच्छृणु त्वम॰\lem \msCa\msCb\msNa\msNc\msParis\msM\msKOb\msPaperA\msPaperC\Ed,\hskip.2em plus .9em 
तच्छृण त्वम॰ \msCc,\hskip.5em plus .9em \lacwithnum{5}  \msNb,\hskip.5em plus .9em तच्छृणु स्वम॰ \msNd,\hskip.5em plus .9em 
त शृणु त्वम॰ \msKOa\oo 
 ॰तन्द्रितम्\lem \msCa\msCb\msNc\msNd\msM\msKOa\msKOb\msPaperA\msPaperC\Ed,\hskip.2em plus .9em ॰तन्द्रितः \msCc\msNa,\hskip.5em plus .9em 
\lacwithnum{3}  \msNb,\hskip.5em plus .9em ॰तन्द्रि\uncl{तं} \msParis}}% 

{\devanagarifont जनमेजय उवाच {\dandab}\dontdisplaylinenum  }%
     \var{{\devanagarifontvar\numemph\vo जनमेजय उवाच\lem \mssALL,\hskip.2em plus .9em 
जन्मेजय उवाच \msCc,\hskip.5em plus .9em \lacwithnum{4} य उवाय \msParis}}% 

{\devanagarifont भगवन्सर्वधर्मज्ञ सर्वशास्त्रविशारद \thinspace{\danda} \dontdisplaylinenum }%
     \var{{\devanagarifontvar\numnoemph\va भगवन्स॰\lem \msCa\msCb\msNa\msNb\msNc\msParis\msKOa\msKOb\msPaperA\msPaperC\Ed,\hskip.2em plus .9em 
भचावं स॰ \msCc,\hskip.5em plus .9em भगव स॰ \msNd,\hskip.5em plus .9em 
भगवं स॰ \msM\oo 
 ॰धर्मज्ञ\lem \mssALL,\hskip.2em plus .9em 
॰ज्ञ \msNa,\hskip.5em plus .9em ॰धर्मज्ञः \msNd}}% 
    \var{{\devanagarifontvar\numnoemph\vb ॰विशारद\lem \msCa\msNb\msNc\msNd\msParis\msKOb\msPaperA,\hskip.2em plus .9em 
॰विसारदः \msCb\msCc\msNa\msKOa\msPaperC\Ed,\hskip.5em plus .9em ॰विशारदम् \msM}}% 
    \paral{{\devanagarifontsmall \vab {\englishfont = \MBH\ 13.112.9ab} }}

%Verse 1:4

{\devanagarifont अस्ति धर्मं परं गुह्यं संसारार्णवतारणम् {॥ १:\hspace{.11em}४॥} \veg\dontdisplaylinenum }%
     \var{{\devanagarifontvar\numnoemph\vc अस्ति धर्मं\lem \msCa\msNa\msNb\msNc\msParis\msKOb\msPaperA\msPaperC\Ed,\hskip.2em plus .9em अस्ति धर्मः \msCb,\hskip.5em plus .9em 
अस्ति धर्म \msCc\msM\msKOa,\hskip.5em plus .9em अधर्म \msNd\oo 
 परं गुह्यं\lem \msCa\msNb\msNd\msParis\msM\msKOa\msKOb\msPaperA\msPaperC\Ed,\hskip.2em plus .9em 
परो गुह्य \msCb,\hskip.5em plus .9em परं गुह्य \msCc\msNa,\hskip.5em plus .9em 
परगुह्यं \msNc}}% 
    \var{{\devanagarifontvar\numnoemph\vd ॰तारणम्\lem \mssALL,\hskip.2em plus .9em 
॰तारणा \msKOa}}% 

{\devanagarifont द्वैपायनमुखोद्गीर्णं धर्मं वा यद्द्विजोत्तम \thinspace{\dandab} \dontdisplaylinenum }%
     \var{{\devanagarifontvar\numemph\va द्वैपायन॰\lem \mssALL,\hskip.2em plus .9em 
द्वेपायन॰ \msCc,\hskip.5em plus .9em 
वैसांपायन॰ \msKOa\oo 
 ॰मुखोद्गीर्णं\lem \msCa\msCb\msNa\msNb\msNc\msParis\msKOb\msPaperA\msPaperC,\hskip.2em plus .9em 
॰मुखोद्गीर्ण \msCc\msKOa,\hskip.5em plus .9em 
॰मुद्गीर्ण्ण \msNd,\hskip.5em plus .9em 
मुखं गीर्ण्णं \msMacorr,\hskip.5em plus .9em 
मु\uncl{खां} गीर्ण्णं \msMpcorr,\hskip.5em plus .9em 
मुखाद्गीर्णं \Ed}}% 
    \var{{\devanagarifontvar\numnoemph\vb धर्मं वा यद्द्वि॰\lem \msCa\msNa\msNb\msNc\msParis\msKOb\msPaperA\msPaperC\Ed,\hskip.2em plus .9em 
धर्मं यत्तद्द्वि॰ \msCb,\hskip.5em plus .9em 
धर्मवत्य द्वि॰ \msCc\msKOa,\hskip.5em plus .9em धर्म वा यद्द्वि॰ \msNd,\hskip.5em plus .9em 
धर्मवाक्यं द्वि॰ \msM\oo 
 ॰त्तम\lem \mssALL,\hskip.2em plus .9em 
॰त्तमः \msCc,\hskip.5em plus .9em ॰तमः \msM}}% 

%Verse 1:5

{\devanagarifont कथयस्व हि मे तृप्तिं कुरु यत्नात्तपोधन {॥ १:\hspace{.11em}५॥} \veg\dontdisplaylinenum }%
     \var{{\devanagarifontvar\numnoemph\vc हि मे तृप्तिं\lem \mssCaCbCc\msNa\msNb\msNc\msParis\msKOb\msPaperA\msPaperC\Ed,\hskip.2em plus .9em 
हि मे तृप्ति \msNd\msKOa,\hskip.5em plus .9em 
प्रसादेन \msM}}% 
    \var{{\devanagarifontvar\numnoemph\vd यत्नात्तपोधन\lem \msCb\msNa\msNb\msNc\msParis\msKOb\msPaperA\msPaperC\Ed,\hskip.2em plus .9em 
यन्नात्त\lk\lk न \msCa,\hskip.5em plus .9em 
यत्ना तपोधनः \msCc,\hskip.5em plus .9em यत्ना तपोधन \msNd,\hskip.5em plus .9em 
यत्नन्तपोधन \msM,\hskip.5em plus .9em यंनात्त॰ \msKOa}}% 

{\devanagarifont वैशम्पायन उवाच {\dandab}\dontdisplaylinenum  }%
     \var{{\devanagarifontvar\numemph\vo वैशम्पायन उवाच\lem \mssALL,\hskip.2em plus .9em 
\om\ \msMacorr,\hskip.5em plus .9em वै\thinspace{\devanagarifont ॥} वैशम्पायन \msPaperC}}% 

{\devanagarifont शृणु राजन्नवहितो धर्माख्यानमनुत्तमम् \thinspace{\danda} \dontdisplaylinenum }%
     \var{{\devanagarifontvar\numnoemph\va राजन्न॰\lem \mssALL,\hskip.2em plus .9em 
राजंन॰ \msNd,\hskip.5em plus .9em राजन॰ \msM\oo 
 ॰हितो\lem \mssALL,\hskip.2em plus .9em 
॰हितं \msPaperA}}% 
    \var{{\devanagarifontvar\numnoemph\vb ॰ख्यानमनुत्तमम्\lem \msCa\msNa\msNb\msNc\msParis\msM\msKOb\Ed,\hskip.2em plus .9em ॰ख्यानमुत्तमम् \msCb,\hskip.5em plus .9em 
॰ख्यानमुतमम् \msCc,\hskip.5em plus .9em ॰धर्मव्याख्यानमुत्तमं \msNd\ \hypermetr,\hskip.5em plus .9em 
॰ख\lac मनुत्तमं \msKOa,\hskip.5em plus .9em 
॰ख्यानमनुत्तमः \msPaperA,\hskip.5em plus .9em 
॰ख्यानमुत्तमः \msPaperC}}% 

%Verse 1:6

{\devanagarifont व्यासानुग्रहसम्प्राप्तं गुह्यधर्मं शृणोतु मे {॥ १:\hspace{.11em}६॥} \veg\dontdisplaylinenum }%
     \var{{\devanagarifontvar\numnoemph\vc ॰प्राप्तं\lem \mssALL,\hskip.2em plus .9em 
॰प्राप्त \msCc}}% 
    \var{{\devanagarifontvar\numnoemph\vd ॰धर्मं\lem \mssALL,\hskip.2em plus .9em 
॰र्मं \msCc,\hskip.5em plus .9em ॰धर्म \msKOa\oo 
 शृणोतु\lem \mssALL,\hskip.2em plus .9em 
शृणोत \msCc,\hskip.5em plus .9em शृणोत्त \msParis\oo 
 मे\lem \mssALL,\hskip.2em plus .9em 
मै \msCb}}% 

{\devanagarifont अनर्थयज्ञकर्तारं तपोव्रतपरायणम् \thinspace{\dandab} \dontdisplaylinenum }%
     \var{{\devanagarifontvar\numemph\va ॰कर्तारं\lem \mssALL,\hskip.2em plus .9em 
॰कर्त्तन्तं \msNb,\hskip.5em plus .9em \lacwithnum{3} \msParis}}% 
    \var{{\devanagarifontvar\numnoemph\vb ॰व्रत॰\lem \mssALL,\hskip.2em plus .9em 
॰प्रत॰ \msM\oo 
 ॰यणम्\lem \msCa\msCb\msNb\msParis\msM\msKOa\msKOb\msPaperA\msPaperC\Ed,\hskip.2em plus .9em 
॰यन \msCc,\hskip.5em plus .9em ॰यणः \msNa,\hskip.5em plus .9em 
॰यनं \msNc,\hskip.5em plus .9em ॰\uncl{यणं} \msNd}}% 

%Verse 1:7

{\devanagarifont शीलशौचसमाचारं सर्वभूतदयापरम् {॥ १:\hspace{.11em}७॥} \veg\dontdisplaylinenum }%
     \var{{\devanagarifontvar\numnoemph\vc ॰चारं\lem \mssALL,\hskip.2em plus .9em 
॰चार \msKOa}}% 
    \var{{\devanagarifontvar\numnoemph\vd ॰परम्\lem \msCa\msCb\msNa\msNc\msParis\msM\msKOb\msPaperA\msPaperC\Ed,\hskip.2em plus .9em ॰न्वितम् \msCc\msNd\msKOa,\hskip.5em plus .9em 
॰\uncl{प}रं \msNb}}% 

{\devanagarifont जिज्ञासनार्थं प्रश्नैकं विष्णुना प्रभविष्णुना \thinspace{\dandab} \dontdisplaylinenum }%
     \var{{\devanagarifontvar\numemph\va ॰र्थं प्रश्नैकं\lem \msCb\msNa\msNb\msNc\msParis,\hskip.2em plus .9em ॰र्थं प्रश्नेकं \msCa\msNd,\hskip.5em plus .9em 
॰र्थप्रश्नेकम् \msCc\msPaperA\msPaperC\Ed,\hskip.5em plus .9em ॰र्थप्रश्चैकं \msM,\hskip.5em plus .9em 
॰थप्रश्नैक \msKOa,\hskip.5em plus .9em 
॰र्थ\lk प्रश्नैकं \msKObacorr,\hskip.5em plus .9em ॰र्थप्रश्नैकं \msKObpcorr}}% 
    \var{{\devanagarifontvar\numnoemph\vb प्रभविष्णुना\lem \msCa\msCb\msNa\msNb\msNd\msParis\msM\msKOa\msKObpcorr\msPaperA\msPaperC\Ed,\hskip.2em plus .9em 
प्रभुविष्णुना \msCc,\hskip.5em plus .9em प्राभविष्णुना \msNc,\hskip.5em plus .9em \om\ \msKObacorr}}% 

%Verse 1:8

{\devanagarifont द्विजरूपधरो भूत्वा पप्रच्छ विनयान्वितः {॥ १:\hspace{.11em}८॥} \veg\dontdisplaylinenum }%
     \var{{\devanagarifontvar\numnoemph\vc ॰धरो\lem \mssALL,\hskip.2em plus .9em 
॰\lk रो \msCa,\hskip.5em plus .9em ॰धरा \msNb}}% 
    \var{{\devanagarifontvar\numnoemph\vd ॰न्वितः\lem \msCa\msCb\msNa\msNb\msNc\msParis\msKOa\msKOb\msPaperA\msPaperC\Ed,\hskip.2em plus .9em 
॰न्वितं \msCc\msNd\msM}}% 


\alalfejezet{ब्रह्मविद्या}
{\devanagarifont [विगतराग उवाच {\dandab}\dontdisplaylinenum  ] }%
 
{\devanagarifont ब्रह्मविद्या कथं ज्ञेया रूपवर्णविवर्जिता \thinspace{\danda} \dontdisplaylinenum }%
     \var{{\devanagarifontvar\numemph\va कथं\lem \mssALL,\hskip.2em plus .9em 
कथ \msKOa\oo 
 ज्ञेया\lem \msCa\msNa\msNb\msNc\msParis\msM\msKOa\msKOb\msPaperA\msPaperC,\hskip.2em plus .9em 
ज्ञेयं \msCb\msCc,\hskip.5em plus .9em ज्ञेय \msNd,\hskip.5em plus .9em भूयो \Ed}}% 
    \var{{\devanagarifontvar\numnoemph\vb ॰वर्ण॰\lem \mssALL,\hskip.2em plus .9em 
॰वर्णा॰ \Ed\oo 
 ॰वर्जिता\lem \msCa\msCb\msNa\msNb\msNd\msParis\msM\msKOb\msPaperA\msPaperC\Ed,\hskip.2em plus .9em 
॰वर्जितं \msCc,\hskip.5em plus .9em ॰वर्जिताः \msNc,\hskip.5em plus .9em \lacwithnum{2} ता \msKOa}}% 

%Verse 1:9

{\devanagarifont स्वरव्यञ्जननिर्मुक्तमक्षरं किमु तत्परम् {॥ १:\hspace{.11em}९॥} \veg\dontdisplaylinenum }%
     \var{{\devanagarifontvar\numnoemph\vc ॰व्यञ्जन॰\lem \mssALL,\hskip.2em plus .9em 
॰व्यज्जन॰ \Ed}}% 
    \var{{\devanagarifontvar\numnoemph\vcd ॰मुक्तमक्ष॰\lem \msCa\msCc\msNa\msNb\msNc\msParis\msKOb\msPaperC\Ed,\hskip.2em plus .9em ॰मुक्त अक्ष॰ \msCb\msKOa,\hskip.5em plus .9em 
॰मुक्तं अख॰ \msNd,\hskip.5em plus .9em ॰मुक्तं अक्ष॰ \msM,\hskip.5em plus .9em ॰म्मुक्तंमक्ष॰ \msPaperA}}% 
    \var{{\devanagarifontvar\numnoemph\vd किमु तत्परम्\lem \msCa\msNa\msNc\msParis\msKOa\msKOb\msPaperA\msPaperC\Ed,\hskip.2em plus .9em 
किमतः परम् \msCb\msCc,\hskip.5em plus .9em 
किमतत्परं \msNb\msNd\msM}}% 

{\devanagarifont अनर्थयज्ञ उवाच {\dandab}\dontdisplaylinenum  }%
 
{\devanagarifont अनुच्चार्यमसन्दिग्धमविच्छिन्नमनाकुलम् \thinspace{\danda} \dontdisplaylinenum }%
     \var{{\devanagarifontvar\numemph\va अनुच्चार्य॰\lem \msCa\msCb\msNa\msNb\msParis\msM\msKOb\msPaperA\msPaperC\Ed,\hskip.2em plus .9em 
अनुचार्य॰ \msCc\msNc\msNd,\hskip.5em plus .9em 
अन्त्रचाय॰ \msKOa}}% 
    \var{{\devanagarifontvar\numnoemph\vab ॰सन्दिग्धमविच्छिन्नमनाकुलम्\lem \msCa\msCb\msNa\msNc\msNd\msParis\msM\msKOb\msPaperA\msPaperC\Ed,\hskip.2em plus .9em 
॰विच्छिन्नसन्दिग्धमनाकुन \msCc,\hskip.5em plus .9em ॰सन्दिग्धमनच्छिन्नमनाकुलम् \msNb,\hskip.5em plus .9em 
॰सन्दिग्धमविच्छिनमनाकुलं \msKOa}}% 

%Verse 1:10

{\devanagarifont निर्मलं सर्वगं सूक्ष्ममक्षरं किमतः परम् {॥ १:\hspace{.11em}१०॥} \veg\dontdisplaylinenum }%
     \var{{\devanagarifontvar\numnoemph\vc निर्मलं सर्वगं\lem \mssALL,\hskip.2em plus .9em 
\lacwithnum{4} वगं \msParis,\hskip.5em plus .9em निर्मलं सर्वग \msKOa}}% 
    \var{{\devanagarifontvar\numnoemph\vc ॰क्षरं किमतः परम्\lem \msCb\msM,\hskip.2em plus .9em ॰क्षरं किमु तत्परम् \msCa\msNa\msNb\msNc\msParis\msKOb\Ed,\hskip.5em plus .9em 
॰क्षरं किमतत्परं \msCc\msNd\msPaperC,\hskip.5em plus .9em 
॰क्षर किमतः परं \msKOa,\hskip.5em plus .9em 
॰क्षराङ्कमतत्परं \msPaperA}}% 

\pend
\endnumbering
\vfill\pagebreak\beginnumbering\pstart
\vers


\alalfejezet{कालपाशः}
{\devanagarifont विगतराग उवाच {\dandab}\dontdisplaylinenum  }%
     \var{{\devanagarifontvar\numemph\vo ॰राग उवाच\lem \mssALL,\hskip.2em plus .9em 
॰रागोवाच \msNd}}% 

{\devanagarifont देही देहे क्षयं याते भूजलाग्निशिवादिभिः \thinspace{\danda} \dontdisplaylinenum }%
     \var{{\devanagarifontvar\numnoemph\va देहे क्ष॰\lem \msCa\msCc\msNc\msParis\msKOb,\hskip.2em plus .9em देहात्क्ष॰ \msCb,\hskip.5em plus .9em 
देहक्ष॰ \msNa\msNb\msNd\msM\msKOa\msPaperA\msPaperC\Ed\oo 
 याते\lem \mssALL,\hskip.2em plus .9em यान्ते \msNd}}% 
    \var{{\devanagarifontvar\numnoemph\vb ॰जलाग्निशिवादिभिः\lem \msCa\msCb\msNa\msNb\msNc\msParis\msM\msKOb\msPaperA\msPaperC\Ed,\hskip.2em plus .9em 
॰जलाग्निशिवादिभि \msCc,\hskip.5em plus .9em 
॰जलाग्निं शि\lk दिभि \msNd,\hskip.5em plus .9em ॰जालादिशिवादिभिः \msKOa}}% 
    \paral{{\devanagarifontsmall \vb {\englishfont \compare\ \KURMP\ 2.23.74:} 
                 अथ कश्चित्प्रमादेन म्रियते ऽग्निविषादिभिः\thinspace{\devanagarifontsmall ।} 
                 तस्याशौचं विधातव्यं कार्यं चैवोदकादिकम्\thinspace{\devanagarifontsmall ॥} }}

%Verse 1:11

{\devanagarifont यमदूतैः कथं नीतो निरालम्बो निरञ्जनः {॥ १:\hspace{.11em}११॥} \veg\dontdisplaylinenum }%
     \var{{\devanagarifontvar\numnoemph\vc ॰दूतैः\lem \mssALL,\hskip.2em plus .9em 
॰दूते \msCc\msNd\oo 
 कथं\lem \mssALL,\hskip.2em plus .9em 
कथ \msKOa\oo 
 नीतो\lem \msCa\msCb\msNa\msNb\msNc\msNd\msParis\msKOb,\hskip.2em plus .9em नीत्वा \msCc,\hskip.5em plus .9em नीतः \msM,\hskip.5em plus .9em नीते \msKOa,\hskip.5em plus .9em 
नीता \msPaperA\msPaperC\Ed}}% 
    \var{{\devanagarifontvar\numnoemph\vd निरालम्बो\lem \mssALL,\hskip.2em plus .9em 
निरोलया \msPaperA,\hskip.5em plus .9em निरोरैन्वो \msPaperC\oo 
 निरञ्जनः\lem \mssALL,\hskip.2em plus .9em 
निरञ्जन \msCc,\hskip.5em plus .9em 
निरञ्ज\lk\ \msKOa}}% 

{\devanagarifont कालपाशैः कथं बद्धो निर्देहश्च कथं व्रजेत् \thinspace{\dandab} \dontdisplaylinenum }%
     \var{{\devanagarifontvar\numemph\va ॰पाशैः\lem \mssALL,\hskip.2em plus .9em 
॰पाशे \msCc,\hskip.5em plus .9em ॰पाशै \msNd\oo 
 बद्धो\lem \mssALL,\hskip.2em plus .9em 
ब\uncl{द्धो} \msCb,\hskip.5em plus .9em बद्ध \msNd}}% 
    \var{{\devanagarifontvar\numnoemph\vb निर्देहश्च\lem \msCa\msCb\msNa\msNb\msNc\msParis\msMpcorr\msKOb\msPaperA\msPaperC\Ed,\hskip.2em plus .9em 
निर्दहः स \msCc,\hskip.5em plus .9em निर्देहस्य \msNd,\hskip.5em plus .9em 
निर्देहन्म \msMacorr,\hskip.5em plus .9em निदेहश्च \msKOa\oo 
 व्रजेत्\lem \mssALL,\hskip.2em plus .9em भवेत् \msNb}}% 

{\devanagarifont स्वर्गं वा स कथं याति निर्देहो बहुधर्मकृत्  \danda\dontdisplaylinenum }%
     \var{{\devanagarifontvar\numnoemph\vc स्वर्गं\lem \msCa\msCb\msNa\msNb\msNc\msParis\msKOb\msPaperA\msPaperC\Ed,\hskip.2em plus .9em 
स्वर्ग \msCc\msNd\msM,\hskip.5em plus .9em स्वागं \msKOa\oo 
 स\lem \mssALL,\hskip.2em plus .9em 
सं \msNb\msM\oo 
 याति\lem \msNa\msNb\msNc\msNd\msParis\msM\msKOa\msKOb\msPaperA\msPaperC,\hskip.2em plus .9em 
यान्ति \mssCaCbCc\Ed}}% 
    \var{{\devanagarifontvar\numnoemph\vd निर्देहो\lem \mssALL,\hskip.2em plus .9em 
निदेहो \msKOa}}% 

%Verse 1:12

{\devanagarifont एतन्मे संशयं ब्रूहि ज्ञातुमिच्छामि तत्त्वतः {॥ १:\hspace{.11em}१२॥} \veg\dontdisplaylinenum }%
     \var{{\devanagarifontvar\numnoemph\ve एतन्मे संशयं\lem \mssCaCbCc\msNc\msParis\msM\msPaperA\msPaperC\Ed,\hskip.2em plus .9em 
एतन्मे संशये \msNa,\hskip.5em plus .9em एतन्मे संशयो \msNb\msNd,\hskip.5em plus .9em 
एवं विस्मयसंसय \msKOa,\hskip.5em plus .9em एतंत्मे संशयं \msKOb}}% 
    \var{{\devanagarifontvar\numnoemph\vf ॰तुमि$\-$च्छामि\lem \mssALL,\hskip.2em plus .9em 
॰तुमि \msCb}}% 

{\devanagarifont अनर्थयज्ञ उवाच {\dandab}\dontdisplaylinenum  }%
     \var{{\devanagarifontvar\numemph\vo अनर्थयज्ञ उवाच\lem \mssALL,\hskip.2em plus .9em 
\om\ \msNaacorr,\hskip.5em plus .9em अनर्थयज्ञ \uncl{उवाच} \msParis}}% 

{\devanagarifont अतिसंशयकष्टं ते पृष्टो ऽहं द्विजसत्तम \thinspace{\danda} \dontdisplaylinenum }%
     \var{{\devanagarifontvar\numnoemph\va अतिसंशयकष्टं ते\lem \msCb\msNa\msNb\msNc\msParis\msMpcorr\msKOb\msPaperC,\hskip.2em plus .9em 
अतिशंस$\-$\uncl{य}कष्टन्ते \msCa,\hskip.5em plus .9em 
अतिशंसयक$\-$ष्टम्मे \msCc\msMacorr\Ed,\hskip.5em plus .9em 
अतिसंशयकष्टो मो \msNd,\hskip.5em plus .9em 
अतिसंसयकष्टञ्च \msKOa,\hskip.5em plus .9em 
अतिसंसयकष्ट\lk न्ते पा \msPaperA}}% 
    \var{{\devanagarifontvar\numnoemph\vb द्विजसत्तम\lem \msCa\msCb\msNa\msNb\msNc\msParis\msM\msKOb\msPaperA\msPaperC\Ed,\hskip.2em plus .9em 
च द्विजोत्तमः \msCc\msKOa,\hskip.5em plus .9em द्विजसत्तमः \msNd}}% 

%Verse 1:13

{\devanagarifont दुर्विज्ञेयं मनुष्यैस्तु देवदानवपन्नगैः {॥ १:\hspace{.11em}१३॥} \veg\dontdisplaylinenum }%
     \var{{\devanagarifontvar\numnoemph\vc ॰ज्ञेयं\lem \msCa\msCb\msNa\msNc\msParis\msKOb,\hskip.2em plus .9em ॰ज्ञेय \msCc\msNb\msNd\msM\msKOa\msPaperA\msPaperC\Ed\oo 
 मनुष्यैस्तु\lem \msCa\msNa\msNb\msNc\msParis\msM\msKOa\msKOb\msPaperA\msPaperC\Ed,\hskip.2em plus .9em 
मनुषैश्च \msCb,\hskip.5em plus .9em मणुक्षे\uncl{प्तु} \msCc,\hskip.5em plus .9em 
मनुष्येस्तु \msNd}}% 

{\devanagarifont कर्महेतु शरीरस्य उत्पत्ति निधनं च यत् \thinspace{\dandab} \dontdisplaylinenum }%
     \var{{\devanagarifontvar\numemph\va कर्म॰\lem \msCa\msCb\msNa\msNb\msNc\msNd\msParis\msM\msKOa\msKOb,\hskip.2em plus .9em 
अनर्थयज्ञ उवाच\thinspace{\devanagarifont ॥} कर्म॰ \msCc\msPaperA\msPaperC\Ed\oo 
 ॰हेतु\lem \mssALL,\hskip.2em plus .9em 
॰हेतुः \msCb,\hskip.5em plus .9em ॰हेंतु \msCc\oo 
 शरीरस्य\lem \mssALL,\hskip.2em plus .9em 
शरीरस्यं \msCc,\hskip.5em plus .9em 
स\lac \uncl{स्य} \msKOa}}% 
    \var{{\devanagarifontvar\numnoemph\vb उत्पत्ति नि॰\lem \msCa\msCb\msNa\msNb\msNc\msParis\msKOa\msKOb\msPaperA\msPaperC\Ed,\hskip.2em plus .9em 
उत्पतिनि॰ \msCc\msNd,\hskip.5em plus .9em उत्पत्तिर्नि॰ \msM\oo 
 च यत्\lem \mssALL,\hskip.2em plus .9em 
च यः \msNb,\hskip.5em plus .9em यत् \msNd}}% 

%Verse 1:14

{\devanagarifont सुकृतं दुष्कृतं चैव पाशद्वयमुदाहृतम् {॥ १:\hspace{.11em}१४॥} \veg\dontdisplaylinenum }%
     \var{{\devanagarifontvar\numnoemph\vc सुकृतं\lem \mssALL,\hskip.2em plus .9em 
सुकृतकृतन् \msCc,\hskip.5em plus .9em सुकृत \msNd\oo 
 चैव\lem \mssALL,\hskip.2em plus .9em वापि \msNd\msKOa}}% 
    \var{{\devanagarifontvar\numnoemph\vd पाश॰\lem \mssALL,\hskip.2em plus .9em पासा॰ \msKOa\oo 
 ॰हृतम्\lem \mssALL,\hskip.2em plus .9em 
॰हृतः \msCc}}% 

{\devanagarifont तेनैव सह संयाति नरकं स्वर्गमेव वा \thinspace{\dandab} \dontdisplaylinenum }%
     \var{{\devanagarifontvar\numemph\va तेनैव\lem \mssALL,\hskip.2em plus .9em 
तेनेव \msCc\msNd\oo 
 सह संयाति\lem \msCa\msCb\msNa\msNb\msNc\msParis\msKOb\msPaperC\Ed,\hskip.2em plus .9em 
सह सा यान्ति \msCc\msNd,\hskip.5em plus .9em सह सा याति \msM,\hskip.5em plus .9em 
सह संयान्ति \msKOa,\hskip.5em plus .9em सहं स याति \msPaperA}}% 
    \var{{\devanagarifontvar\numnoemph\vb नरकं स्वर्ग॰\lem \mssALL,\hskip.2em plus .9em 
नरकदुर्ग्ग॰ \msKOa\oo 
 वा\lem \mssCaCbCc\msNb\msNc\msParis\msM\msKOb\msPaperA\msPaperC\Ed,\hskip.2em plus .9em च \msNa\msNd\msKOa}}% 

%Verse 1:15

{\devanagarifont सुखदुःखं शरीरेण भोक्तव्यं कर्मसम्भवम् {॥ १:\hspace{.11em}१५॥} \veg\dontdisplaylinenum }%
     \var{{\devanagarifontvar\numnoemph\vc सुख॰\lem \mssALL,\hskip.2em plus .9em सुखं \msM\oo 
 ॰दुःखं\lem \msCa\msCb\msNa\msNc\msParis\msM\msKOb,\hskip.2em plus .9em ॰दुःख \msCc\msNb\msKOa\msPaperA\msPaperC\Ed}}% 
    \var{{\devanagarifontvar\numnoemph\vd भोक्तव्यं\lem \mssALL,\hskip.2em plus .9em 
भोक्तव्य \msKOa\oo 
 ॰सम्भवम्\lem \msCa\msCb\msNa\msNb\msNc\msParis\msM\msKOb,\hskip.2em plus .9em 
॰सम्भवः \msCc\msPaperA\msPaperC\Ed,\hskip.5em plus .9em ॰संभावात् \msKOa}}% 

{\devanagarifont हेतुनानेन विप्रेन्द्र देहः सम्भवते नृणाम् \thinspace{\dandab} \dontdisplaylinenum }%
     \var{{\devanagarifontvar\numemph\va हेतुनानेन\lem \mssALL,\hskip.2em plus .9em 
हेतुना तेन \msKOa,\hskip.5em plus .9em हेतुनाने \msPaperCacorr\oo 
 ॰न्द्र\lem \mssALL,\hskip.2em plus .9em ॰न्द्रः \msNb\msKOb}}% 
    \var{{\devanagarifontvar\numnoemph\vb देहः\lem \msCa\msCb\msNa\msNc\msParis\msKOb\Ed,\hskip.2em plus .9em देहे \msCc,\hskip.5em plus .9em देह \msNb\msM\msKOa\msPaperA,\hskip.5em plus .9em 
देहं \msPaperC\oo 
 नृणाम्\lem \mssALL,\hskip.2em plus .9em नृणा \msCb\msCc}}% 

%Verse 1:16

{\devanagarifont यं कालपाशमित्याहुः शृणु वक्ष्यामि सुव्रत {॥ १:\hspace{.11em}१६॥} \veg\dontdisplaylinenum }%
     \var{{\devanagarifontvar\numnoemph\vc यं कालपाशमित्याहुः\lem \eme,\hskip.2em plus .9em यं कालपाशमित्याह \msCa\msCb\msNa,\hskip.5em plus .9em 
कालपासेति सत्वाह \msCc,\hskip.5em plus .9em यं कालपाशमित्याहु \msNb\msNc\msParis\msKOb\msPaperA\Ed,\hskip.5em plus .9em 
कालपाषेति \uncl{पस्त्वे}ह \msM,\hskip.5em plus .9em 
यां कालपासमित्याहु \msKOa}}% 
    \var{{\devanagarifontvar\numnoemph\vd ॰व्रत\lem \msCa\msNa\msNb\msNc\msParis\msM\msKOb\msPaperA\Ed,\hskip.2em plus .9em ॰व्रतः \msCb\msCc\msKOa}}% 

{\devanagarifont न त्वया विदितं किञ्चिज्जिज्ञास्यसि कथं द्विज \thinspace{\dandab} \dontdisplaylinenum }%
     \var{{\devanagarifontvar\numemph\va विदितं\lem \mssALL,\hskip.2em plus .9em विदित \msCc}}% 
    \var{{\devanagarifontvar\numnoemph\vab किञ्चिज्जि॰\lem \msCb\msM,\hskip.2em plus .9em किञ्चिद्वि॰ \msCapcorr\msNa\msNb\msNc\msParis\msKOb\msPaperA\Ed,\hskip.5em plus .9em 
किद्वि॰ \msCaacorr,\hskip.5em plus .9em 
किञ्चि जि॰ \msCc}}% 
    \var{{\devanagarifontvar\numnoemph\vb कथं द्विज\lem \msCa\msCb\msNa\msNb\msNc\msM\msPaperA\Ed,\hskip.2em plus .9em 
\lk\lk\lk\lk\lk\lk\lk\lk\lk  \uncl{म त्वया विदितं किञ्चिद्विज्ञास्यसि} 
\cancelled\ कथं द्विज \msCc,\hskip.5em plus .9em 
कथं द्विजः \msParis\msKOb}}% 

%Verse 1:17

{\devanagarifont कालपाशं च विप्रेन्द्र सकलं वेत्तुमर्हसि {॥ १:\hspace{.11em}१७॥} \veg\dontdisplaylinenum }%
     \var{{\devanagarifontvar\numnoemph\vc कालपाशं च\lem \mssALL,\hskip.2em plus .9em कालपाषेति \msM}}% 
    \var{{\devanagarifontvar\numnoemph\vd वेत्तुमर्हसि\lem \mssCaCbCc\msNa\msNb\msParis\msKOb,\hskip.2em plus .9em 
वेत्तुमूहसि \msNc,\hskip.5em plus .9em वक्तुमर्हसि \msM\msPaperA\Ed}}% 

{\devanagarifont कलाकलितकालं च कालतत्त्वकलां शृणु \thinspace{\dandab} \dontdisplaylinenum }%
     \var{{\devanagarifontvar\numemph\va कला॰\lem \mssALL,\hskip.2em plus .9em काला॰ \msCc\msNaacorr\oo 
 ॰कलित॰\lem \mssALL,\hskip.2em plus .9em ॰\uncl{कन्मित}॰ \msPaperA\oo 
 ॰कालं च\lem \mssALL,\hskip.2em plus .9em ॰कालश्च \msM\Ed}}% 
    \var{{\devanagarifontvar\numnoemph\vb ॰कलां\lem \msCa\msCc\msNb\msParis\msPaperA\Ed,\hskip.2em plus .9em ॰कला \msCb\msNc\msKOb,\hskip.5em plus .9em ॰विधिं \msNa,\hskip.5em plus .9em ॰कलाः \msM}}% 

%Verse 1:18

{\devanagarifont त्रुटिद्वयं निमेषस्तु निमेषद्विगुणा कला {॥ १:\hspace{.11em}१८॥} \veg\dontdisplaylinenum }%
     \var{{\devanagarifontvar\numnoemph\vc त्रुटिद्वयं\lem \msCa\msCc\msNc\msParis\Ed,\hskip.2em plus .9em तुटिद्वय \msCb\msNb\msKOb,\hskip.5em plus .9em तुटिद्वयं \msNa\msM,\hskip.5em plus .9em 
त्रुविद्वयं \msPaperA\oo 
 ॰मेषस्तु\lem \mssALL,\hskip.2em plus .9em 
॰मेवस्तु \msCa,\hskip.5em plus .9em ॰मेषद्वि॰ \msNa}}% 
    \var{{\devanagarifontvar\numnoemph\vd निमेषद्वि॰\lem \mssALL,\hskip.2em plus .9em निमेषाद्वि॰ \msM,\hskip.5em plus .9em नि\lk षस्तु दिव्॰ \msKOb}}% 

{\devanagarifont कलाद्विगुणिता काष्ठा काष्ठा वै त्रिंशतिः कला \thinspace{\dandab} \dontdisplaylinenum }%
     \var{{\devanagarifontvar\numemph\va ॰गुणिता काष्ठा\lem \mssALL,\hskip.2em plus .9em ॰गुणितं काष्ठा \msM,\hskip.5em plus .9em 
॰गुणितं काष्ठी \msPaperA}}% 
    \var{{\devanagarifontvar\numnoemph\vb काष्ठा वै त्रिंशतिः\lem \msCa\msNa\msNb\msNc\msParis\msKOb\msPaperA\Ed,\hskip.2em plus .9em वै त्रिंशता \msCb,\hskip.5em plus .9em 
काष्ठा वै त्रिंशति \msCc,\hskip.5em plus .9em काष्ठान्वै त्रिंशति \msM}}% 

%Verse 1:19

{\devanagarifont त्रिंशत्कला मुहूर्तश्च मानुषेन द्विजोत्तम {॥ १:\hspace{.11em}१९॥} \veg\dontdisplaylinenum }%
     \var{{\devanagarifontvar\numnoemph\vc मुहूर्तश्च\lem \mssALL,\hskip.2em plus .9em 
मुहूर्त्त \msCb,\hskip.5em plus .9em मुहूर्तञ्च \Ed}}% 
    \var{{\devanagarifontvar\numnoemph\vd मानुषेन\lem \mssALL,\hskip.2em plus .9em मानु\uncl{षश्च} \msCc,\hskip.5em plus .9em 
मानुषेणण \msKOb\oo 
 ॰त्तम\lem \mssCaCbCc\msNa\msNcpcorr\msParis\msKOb\msPaperA\Ed,\hskip.2em plus .9em ॰तमः \msNb\msM,\hskip.5em plus .9em ॰त्तमः \msNcacorr}}% 

{\devanagarifont मुहूर्तत्रिंशकेनैव अहोरात्रं विदुर्बुधाः \thinspace{\dandab} \dontdisplaylinenum }%
     \var{{\devanagarifontvar\numemph\va मुहूर्त॰\lem \mssALL,\hskip.2em plus .9em मुहूर्त्ता \msM,\hskip.5em plus .9em मुहूर्तं \Ed}}% 
    \var{{\devanagarifontvar\numnoemph\vb ॰धाः\lem \mssALL,\hskip.2em plus .9em ॰धा \msPaperA}}% 

%Verse 1:20

{\devanagarifont अहोरात्रं पुनस्त्रिंशन्मासमाहुर्मनीषिणः {॥ १:\hspace{.11em}२०॥} \veg\dontdisplaylinenum }%
     \var{{\devanagarifontvar\numnoemph\vc ॰रात्रं\lem \mssALL,\hskip.2em plus .9em ॰रात्र \msM}}% 
    \var{{\devanagarifontvar\numnoemph\vd ॰नीषिणः\lem \mssALL,\hskip.2em plus .9em ॰नीषिन \msM}}% 

{\devanagarifont समा द्वादश मासाश्च कालतत्त्वविदो जनाः \thinspace{\dandab} \dontdisplaylinenum }%
     \var{{\devanagarifontvar\numemph\va समा\lem \mssALL,\hskip.2em plus .9em मास \msCc,\hskip.5em plus .9em समा समाया \msPaperA\oo 
 ॰मासाश्च\lem \msCa\msCb\msNa\msNb\msNc\msParis\msKOb\msPaperA,\hskip.2em plus .9em ॰मासश्च \msCc\Ed,\hskip.5em plus .9em मासाहुः \msM}}% 
    \var{{\devanagarifontvar\numnoemph\vb कालतत्त्व॰\lem \mssCaCbCc\msNa\msNb\msM\msPaperA\Ed,\hskip.2em plus .9em कलातत्त्व॰ \msNc,\hskip.5em plus .9em कालन्तत्व॰ \msParis\msKOb}}% 

{\devanagarifont शतं वर्षसहस्राणि त्रीणि मानुषसंख्यया  \danda\dontdisplaylinenum }%
     \var{{\devanagarifontvar\numnoemph\vc शतं\lem   \mssALL,\hskip.2em plus .9em        शत॰ \msPaperA\Ed}}% 
    \var{{\devanagarifontvar\numnoemph\vd त्रीणि\lem \mssALL,\hskip.2em plus .9em \om\ \msKObacorr\oo 
 मानुष॰\lem \mssALL,\hskip.2em plus .9em माणुष्य॰ \msCb\msCc\ \unmetr}}% 

%Verse 1:21

{\devanagarifont षष्टिं चैव सहस्राणि कालः कलियुगः स्मृतः {॥ १:\hspace{.11em}२१॥} \veg\dontdisplaylinenum }%
     \var{{\devanagarifontvar\numnoemph\ve षष्टिं चैव\lem \mssCaCbCc\msNc\msParis\msM\msKOb,\hskip.2em plus .9em षष्टिं वर्ष॰ \msNa\msPaperA,\hskip.5em plus .9em षष्टिश्चैव \Ed}}% 
    \var{{\devanagarifontvar\numnoemph\vf ॰युगः\lem       \mssALL,\hskip.2em plus .9em ॰युग \msM\Ed}}% 
    \lacuna{\devanagarifontsmall \vo {\englishfont \msNb\ omits verses 21ef--24ab} }%
  
{\devanagarifont द्विगुणः कलिसंख्यातो द्वापरो युग संज्ञितः \thinspace{\dandab} \dontdisplaylinenum }%
     \var{{\devanagarifontvar\numemph\va द्विगुणः कलिसंख्यातो\lem \mssCaCbCc\msNa\msNc\msParis\msKOb,\hskip.2em plus .9em कलिसंख्यास्तु द्विगुणो \msM,\hskip.5em plus .9em 
द्विगुर्णः कलिसंख्यातो \msPaperA,\hskip.5em plus .9em 
द्विगुणा कलिसंख्यातो \Ed}}% 
    \var{{\devanagarifontvar\numnoemph\vb द्वापरो युग संज्ञितः\lem \mssALL,\hskip.2em plus .9em 
द्वापरः युगः संज्ञिकम् \msM,\hskip.5em plus .9em 
द्वापरे युग संज्ञितः \Ed}}% 

%Verse 1:22

{\devanagarifont त्रेता तु त्रिगुणा ज्ञेया चतुः कृतयुगः स्मृतः {॥ १:\hspace{.11em}२२॥} \veg\dontdisplaylinenum }%
     \var{{\devanagarifontvar\numnoemph\vc त्रेता\lem   \msCa\msCb\msNa\msParis\msKOb\msPaperA\Ed,\hskip.2em plus .9em              तेत्रा \msCc\msM,\hskip.5em plus .9em त्रेत्रा \msNc\oo 
 त्रिगुणा\lem \mssALL,\hskip.2em plus .9em  तृगुणो \msM\oo 
 ज्ञेया\lem   \mssALL,\hskip.2em plus .9em  ज्ञेयः \msM}}% 
    \var{{\devanagarifontvar\numnoemph\vd ॰युगः\lem  \mssALL,\hskip.2em plus .9em ॰युग \Ed}}% 

{\devanagarifont एषा चतुर्युगासंख्या कृत्वा वै ह्येकसप्ततिः \thinspace{\dandab} \dontdisplaylinenum }%
     \var{{\devanagarifontvar\numemph\vb ह्ये॰\lem   \mssALL,\hskip.2em plus .9em   हे॰ \msNc\oo 
 ॰सप्ततिः\lem \mssALL,\hskip.2em plus .9em ॰सप्तति \msM}}% 

%Verse 1:23

{\devanagarifont मन्वन्तरस्य चैकस्य ज्ञानमुक्तं समासतः {॥ १:\hspace{.11em}२३॥} \veg\dontdisplaylinenum }%
     \var{{\devanagarifontvar\numnoemph\vc मन्वन्तरस्य\lem \mssALL,\hskip.2em plus .9em मन्वन्तरन्तस्य \msParis\oo 
 चैकस्य\lem \mssALL,\hskip.2em plus .9em \om\ \msNaacorr\msMacorr}}% 
    \var{{\devanagarifontvar\numnoemph\vd ॰क्तं\lem    \mssALL,\hskip.2em plus .9em                ॰क्त \msM}}% 

{\devanagarifont कल्पो मन्वन्तराणां तु चतुर्दश तु संख्यया \thinspace{\dandab} \dontdisplaylinenum }%
     \var{{\devanagarifontvar\numemph\va कल्पो\lem \msCb,\hskip.2em plus .9em कल्प \msCa\msCc\msNa\msNc\msParis\msM\msKOb\msPaperA\Ed\oo 
 मन्वन्त॰\lem \mssALL,\hskip.2em plus .9em 
न्वन्त॰ \msMacorr,\hskip.5em plus .9em मंन्वन्त॰ \msMpcorr}}% 
    \var{{\devanagarifontvar\numnoemph\vb ॰दश\lem     \mssALL,\hskip.2em plus .9em ॰दशं \msCb\oo 
 संख्यया\lem \mssALL,\hskip.2em plus .9em      शंक्षया \msM}}% 

{\devanagarifont दश कल्पसहस्राणि ब्रह्माहः परिकल्पितम्  \danda\dontdisplaylinenum }%
     \var{{\devanagarifontvar\numnoemph\vd ॰आहः\lem \mssALL,\hskip.2em plus .9em ॰आह \msCa\oo 
 परिकल्पितम्\lem \msCa\msNc\msParis\msKOb,\hskip.2em plus .9em करिकल्पितम् \msCb,\hskip.5em plus .9em परिकल्पितः \msCc\msNb\msM\msPaperA\Ed,\hskip.5em plus .9em 
परिकीर्तिताः \msNa}}% 

%Verse 1:24

{\devanagarifont रात्रिरेतावती प्रोक्ता मुनिभिस्तत्त्वदर्शिभिः {॥ १:\hspace{.11em}२४॥} \veg\dontdisplaylinenum }%
     \var{{\devanagarifontvar\numnoemph\vf ॰दर्शिभिः\lem \mssALL,\hskip.2em plus .9em ॰दर्शिभि \msM}}% 

{\devanagarifont रात्र्यागमे प्रलीयन्ते जगत्सर्वं चराचरम् \thinspace{\dandab} \dontdisplaylinenum }%
     \var{{\devanagarifontvar\numemph\va ॰गमे\lem      \mssALL,\hskip.2em plus .9em         ॰गम \msPaperA\oo 
 प्रलीयन्ते\lem \mssALL,\hskip.2em plus .9em प्रलीयते \msCb}}% 
    \var{{\devanagarifontvar\numnoemph\vb सर्वं च॰\lem \mssALL,\hskip.2em plus .9em 
सर्वश्च॰ \msM,\hskip.5em plus .9em सर्व\uncl{ञ्ज}॰ \msKOb}}% 

%Verse 1:25

{\devanagarifont अहागमे तथैवेह उत्पद्यन्ते चराचरम् {॥ १:\hspace{.11em}२५॥} \veg\dontdisplaylinenum }%
     \var{{\devanagarifontvar\numnoemph\vc अहागमे\lem \mssCaCbCc\msNa\msNc\msParis\msKOb,\hskip.2em plus .9em अहाग\lacwithnum{1}  \msNb,\hskip.5em plus .9em 
अहरागमे \msM\ \unmetr,\hskip.5em plus .9em अहागम \msPaperA,\hskip.5em plus .9em अह्नागमे \Ed}}% 
    \var{{\devanagarifontvar\numnoemph\vd ॰पद्यन्ते\lem \mssALL,\hskip.2em plus .9em ॰पद्यंति \msM}}% 

{\devanagarifont परार्धपरकल्पानि अतीतानि द्विजोत्तम \thinspace{\dandab} \dontdisplaylinenum }%
     \var{{\devanagarifontvar\numemph\va ॰र्ध॰\lem \mssALL,\hskip.2em plus .9em ॰र्धं \msNb,\hskip.5em plus .9em ॰ध॰ \msPaperA}}% 

%Verse 1:26

{\devanagarifont अनागतं तथैवाहुर्भृगुरादिमहर्षयः {॥ १:\hspace{.11em}२६॥} \veg\dontdisplaylinenum }%
     \var{{\devanagarifontvar\numnoemph\vcd ॰वाहुर्भृ॰\lem \msCa\msCb\msNa\msNc\msParis\msKOb\msPaperA\Ed,\hskip.2em plus .9em 
॰वाहु भृ॰ \msCc\msNb\msM}}% 
    \var{{\devanagarifontvar\numnoemph\vd ॰महर्षयः\lem    \mssCaCbCc\msNapcorr\msNb\msParis\msKOb\msPaperA\Ed,\hskip.2em plus .9em 
॰महयः \msNaacorr,\hskip.5em plus .9em ॰मर्हषयः \msNc,\hskip.5em plus .9em 
॰महर्षिभिः \msM}}% 

{\devanagarifont यथार्कग्रहतारेन्दु भ्रमतो दृश्यते त्विह \thinspace{\dandab} \dontdisplaylinenum }%
     \var{{\devanagarifontvar\numemph\va ॰आर्क॰\lem   \mssALL,\hskip.2em plus .9em ॰आर्का॰ \msMacorr\oo 
 ॰तारेन्दु\lem \mssALL,\hskip.2em plus .9em          ॰तारैन्दु \msM}}% 
    \var{{\devanagarifontvar\numnoemph\vb भ्रमतो\lem \mssALL,\hskip.2em plus .9em                भुमनो \msPaperA\oo 
 दृश्यते त्विह\lem \msCa\msNa\msNb\msNc\msParis\msKOb\msPaperA\Ed,\hskip.2em plus .9em 
दृश्यन्दिह \msCb,\hskip.5em plus .9em दृस्यते त्विहः \msCc,\hskip.5em plus .9em 
दृश्यते त्विहः \msM}}% 

%Verse 1:27

{\devanagarifont कालचक्रं भ्रमित्वैव विश्रमं न च विद्महे {॥ १:\hspace{.11em}२७॥} \veg\dontdisplaylinenum }%
     \var{{\devanagarifontvar\numnoemph\vc भ्रमित्वैव\lem \corr,\hskip.2em plus .9em भ्रमत्वैव \msCa\msNa\msNc\msKOb\Ed,\hskip.5em plus .9em 
भ्रमत्वेव  \msCb\msNb\msParis\msM,\hskip.5em plus .9em भ्रमत्वेह \msCc,\hskip.5em plus .9em 
भ्रमत्यैव \msPaperA}}% 
    \var{{\devanagarifontvar\numnoemph\vd ॰श्रमं\lem \mssCaCbCc\msNapcorr\msNc\msPaperA\Ed,\hskip.2em plus .9em 
॰श्रमो \msNaacorr,\hskip.5em plus .9em ॰श्रामन् \msNb,\hskip.5em plus .9em ॰श्रमेन् \msParis\msKOb,\hskip.5em plus .9em ॰श्रामो \msM\oo 
 विद्महे\lem \mssALL,\hskip.2em plus .9em विग्रहे \msCb,\hskip.5em plus .9em विद्यते \msM}}% 

{\devanagarifont कालः सृजति भूतानि कालः संहरते पुनः \thinspace{\dandab} \dontdisplaylinenum }%
     \var{{\devanagarifontvar\numemph\vb कालः\lem \mssALL,\hskip.2em plus .9em काल \Ed}}% 
    \paral{{\devanagarifontsmall \vab {\englishfont \similar\ \UMS\ 12.34cd:}
                         कालः पचति भूतानि कालः संहरते प्रजाः }}

%Verse 1:28

{\devanagarifont कालस्य वशगाः सर्वे न कालवशकृत्क्वचित् {॥ १:\hspace{.11em}२८॥} \veg\dontdisplaylinenum }%
     \var{{\devanagarifontvar\numnoemph\vc कालस्य\lem     \mssALL,\hskip.2em plus .9em कालःस्य \msMacorr\oo 
 वशगाः\lem     \mssALL,\hskip.2em plus .9em         वशगा \Ed}}% 
    \var{{\devanagarifontvar\numnoemph\vd कालवशकृ॰\lem \mssALL,\hskip.2em plus .9em          कालो वशकृ॰ \msM}}% 
    \paral{{\devanagarifontsmall \vo \similar\ {\englishfont \KURMP\ 1.11.32:}
                 कालः सृजति भूतानि कालः संहरते प्रजाः\thinspace{\devanagarifontsmall ।}
                 सर्वे कालस्य वशगा न कालः कस्यचिद्वशे\thinspace{\devanagarifontsmall ॥} }}

{\devanagarifont चतुर्दश परार्धानि देवराजा द्विजोत्तम \thinspace{\dandab} \dontdisplaylinenum }%
     \var{{\devanagarifontvar\numemph\vb देवराजा\lem \mssALL,\hskip.2em plus .9em     देवराज \msM\Ed\oo 
 ॰त्तम\lem   \mssALL,\hskip.2em plus .9em ॰त्तमः \msM}}% 

%Verse 1:29

{\devanagarifont कालेन समतीतानि कालो हि दुरतिक्रमः {॥ १:\hspace{.11em}२९॥} \veg\dontdisplaylinenum }%
     \paral{{\devanagarifontsmall \vd {\englishfont = \MBH\ 12.220.41d = \GARPUR\ 1.108.7d} }}

{\devanagarifont एष कालो महायोगी ब्रह्मा विष्णुः परः शिवः \thinspace{\dandab} \dontdisplaylinenum }%
     \var{{\devanagarifontvar\numemph\va कालो\lem \msCa\msCb\msNa,\hskip.2em plus .9em      काल \msCc\msNb\msNc\msParis\msM\msKOb\msPaperA\Ed\oo 
 महायोगी\lem \mssALL,\hskip.2em plus .9em मयोयोगी \msParis}}% 
    \var{{\devanagarifontvar\numnoemph\vb ब्रह्मा विष्णुः परः\lem \msCb,\hskip.2em plus .9em ब्रह्मविष्णुपरः \msCa\msNc\msM\msPaperA,\hskip.5em plus .9em 
ब्रह्मा विष्णु परः \msCc\msNa\msNb,\hskip.5em plus .9em 
ब्रह्म विष्णुः परः \msParis\msKOb,\hskip.5em plus .9em 
ब्रह्मविष्णुपर \Ed\ \unmetr}}% 

%Verse 1:30

{\devanagarifont अनादिनिधनो धाता स महात्मा नमस्कुरु {॥ १:\hspace{.11em}३०॥} \veg\dontdisplaylinenum }%
 

\alalfejezet{परार्धादि}
{\devanagarifont विगतराग उवाच {\dandab}\dontdisplaylinenum  }%
 
{\devanagarifont श्रुतं वै कालचक्रं तु मुखपद्मविनिःसृतम् \thinspace{\danda} \dontdisplaylinenum }%
     \var{{\devanagarifontvar\numemph\va श्रुतं वै\lem \mssALL,\hskip.2em plus .9em श्रुतो वः \msM\oo 
 ॰चक्रं तु\lem \mssALL,\hskip.2em plus .9em ॰चक्रस्य \msCc,\hskip.5em plus .9em ॰चक्रत्तु \msM}}% 
    \var{{\devanagarifontvar\numnoemph\vb विनिःसृतम्\lem \corr,\hskip.2em plus .9em विनिसृतम् \mssCaCbCc\msNa\msNb\msNc\msParis\msM\msKOb\msPaperA\Ed\ \unmetr}}% 

%Verse 1:31

{\devanagarifont परार्धं च परं चैव श्रोतुं वः प्रतिदीपितम् {॥ १:\hspace{.11em}३१॥} \veg\dontdisplaylinenum }%
     \var{{\devanagarifontvar\numnoemph\vc परार्धं च\lem \msCb\msCc\msNa\msNb\msNc\msParis\msKOb\msPaperA\Ed,\hskip.2em plus .9em 
\uncl{प}रार्द्धं च \msCa,\hskip.5em plus .9em 
पराधञ्च \msMacorr,\hskip.5em plus .9em 
परार्धंञ्चे \msMpcorr\oo 
 परं चैव\lem \mssALL,\hskip.2em plus .9em                पराञ्चैव \msM\msPaperA}}% 
    \var{{\devanagarifontvar\numnoemph\vd वः\lem         \mssCaCbCc\msNa\msNb\msNc\msParisacorr\msMacorr\msKOb\msPaperA,\hskip.2em plus .9em 
नो \msParispcorr,\hskip.5em plus .9em 
नः \msMpcorr,\hskip.5em plus .9em यः \Ed\oo 
 ॰दीपितम्\lem    \mssALL,\hskip.2em plus .9em      ॰दीयतां \msM}}% 

{\devanagarifont अनर्थयज्ञ उवाच {\dandab}\dontdisplaylinenum  }%
     \var{{\devanagarifontvar\numemph\vo अनर्थयज्ञ उवाच\lem \mssALL,\hskip.2em plus .9em \om\ \msNaacorr}}% 

{\devanagarifont एकं दश शतं चैव सहस्रमयुतं तथा \thinspace{\danda} \dontdisplaylinenum }%
     \var{{\devanagarifontvar\numnoemph\va दश\lem \msCa\msKOb,\hskip.2em plus .9em दशं \msCb\msCc\msNa\msNb\msNc\msM\msPaperA\Ed,\hskip.5em plus .9em \uncl{दश} \msParis}}% 
    \var{{\devanagarifontvar\numnoemph\vb सहस्र॰\lem \mssALL,\hskip.2em plus .9em साहस्र॰ \msM\oo 
 ॰युतं\lem   \mssALL,\hskip.2em plus .9em  ॰तन् \msNb}}% 

%Verse 1:32

{\devanagarifont प्रयुतं नियुतं कोटिमर्बुदं वृन्दमेव च {॥ १:\hspace{.11em}३२॥} \veg\dontdisplaylinenum }%
     \var{{\devanagarifontvar\numnoemph\vc प्र॰\lem        \mssALL,\hskip.2em plus .9em      प॰ \msPaperA}}% 
    \var{{\devanagarifontvar\numnoemph\vcd कोटिम॰\lem \mssALL,\hskip.2em plus .9em  कोटिर॰ \msNc}}% 
    \var{{\devanagarifontvar\numnoemph\vd ॰र्बुदं\lem     \mssALL,\hskip.2em plus .9em ॰बुदं \msNc}}% 

{\devanagarifont खर्वं चैव निखर्वं च शङ्कु पद्मं तथैव च \thinspace{\dandab} \dontdisplaylinenum }%
     \var{{\devanagarifontvar\numemph\va निखर्वं च\lem \mssALL,\hskip.2em plus .9em       निखर्वं तु \msNb,\hskip.5em plus .9em निसर्वञ्च \msM}}% 
    \var{{\devanagarifontvar\numnoemph\vb शङ्कु\lem        \mssALL,\hskip.2em plus .9em शंख \Ed\oo 
 पद्मं\lem       \mssALL,\hskip.2em plus .9em  पद्म \msM}}% 
    \lacuna{\devanagarifontsmall \vab {\englishfont After these two pādas, \msPaperA\ inserts this:}
                                वृन्दञ्चैव महावृन्द द्विपरो नन्तनेव च }%
      \paral{{\devanagarifontsmall \vab {\englishfont  = \BRAHMANDAPUR\ 3.2.101 }  }}

%Verse 1:33

{\devanagarifont समुद्रो मध्यमन्तं च परार्धं च परं तथा {॥ १:\hspace{.11em}३३॥} \veg\dontdisplaylinenum }%
     \var{{\devanagarifontvar\numnoemph\vc समुद्रो\lem            \mssALL,\hskip.2em plus .9em समुद्र॰ \msM\oo 
 मध्यमन्तं च\lem \mssCaCbCc\msNaacorr\msParis\msM\msKOb\msPaperA,\hskip.2em plus .9em         मध्यमान्तं च \msNapcorr,\hskip.5em plus .9em 
मध्य\uncl{मन्तञ्च} \msNb,\hskip.5em plus .9em 
मध्यमन्तश्च \msNc}}% 
    \var{{\devanagarifontvar\numnoemph\vd परार्धं च परं तथा\lem \mssALL,\hskip.2em plus .9em  परार्द्धपरद्वेगुणाम् \msM}}% 
    \lacuna{\devanagarifontsmall \vcd {\englishfont \Ed\ omits 34cd--35 and then inserts this:}
                                वृन्दञ्चैव महावृन्द द्विपरानन्तमेव च }%
  
{\devanagarifont सर्वे दशगुणा ज्ञेयाः परार्धं यावदेव हि \thinspace{\dandab} \dontdisplaylinenum }%
     \var{{\devanagarifontvar\numemph\va सर्वे\lem     \mssALL,\hskip.2em plus .9em सर्वं \msPaperA}}% 
    \var{{\devanagarifontvar\numnoemph\vb परार्धं\lem \msNc\msParis,\hskip.2em plus .9em                        परा\uncl{र्ध} \msCa,\hskip.5em plus .9em 
परार्ध \msCb\msCc\msNa\msNb\msM\msKOb\msPaperA\oo 
 यावदेव\lem \mssALL,\hskip.2em plus .9em दशदव \msPaperA}}% 

%Verse 1:34

{\devanagarifont परार्धद्विगुणेनैव परसंख्या विधीयते {॥ १:\hspace{.11em}३४॥} \veg\dontdisplaylinenum }%
     \var{{\devanagarifontvar\numnoemph\vc परार्ध॰\lem \mssALL,\hskip.2em plus .9em   परार्धं \msNc}}% 
    \var{{\devanagarifontvar\numnoemph\vd ॰संख्या\lem  \mssALL,\hskip.2em plus .9em ॰सख्या \msM}}% 

{\devanagarifont परात्परतरं नास्ति इति मे निश्चिता मतिः \thinspace{\dandab} \dontdisplaylinenum }%
     \var{{\devanagarifontvar\numemph\vab परात्परतरं नास्ति इति मे निश्चिता मतिः\lem \mssCaCbCc\msNb\msNcpcorr\msParis\msKOb\msPaperA\Ed,\hskip.2em plus .9em 
परात्परतरं नास्ति इति मे निश्चिता मति \msNa\msNcacorr,\hskip.5em plus .9em 
परापरतरन्नास्ति इति मे निश्चिता मति \msM}}% 

%Verse 1:35

{\devanagarifont पुराणवेदपठिता मयाख्याता द्विजोत्तम {॥ १:\hspace{.11em}३५॥} \veg\dontdisplaylinenum }%
     \var{{\devanagarifontvar\numnoemph\vc ॰वेद॰\lem \msCa\Ed,\hskip.2em plus .9em ॰वेदे \msCb\msCc\msNb\msNc\msParis\msKOb\msPaperA,\hskip.5em plus .9em 
॰वेदा \msNa,\hskip.5em plus .9em ॰वेदैः \msM}}% 
    \var{{\devanagarifontvar\numnoemph\vd ॰ख्याता\lem \msCa\msCb\msNa\msParis\msKOb,\hskip.2em plus .9em ॰ख्यातं \msCc\msNb\msNc\msM\msPaperA\Ed\oo 
 ॰त्तम\lem \mssALL,\hskip.2em plus .9em ॰तम \msM}}% 


\alalfejezet{ब्रह्माण्डम्}
{\devanagarifont विगतराग उवाच {\dandab}\dontdisplaylinenum  }%
 
{\devanagarifont ब्रह्माण्डं कति विज्ञेयं प्रमाणं ज्ञापितं क्वचित् \thinspace{\danda} \dontdisplaylinenum }%
     \var{{\devanagarifontvar\numemph\va ब्रह्माण्डं\lem \mssALL,\hskip.2em plus .9em ब्रह्माण्ड \msCc}}% 
    \var{{\devanagarifontvar\numnoemph\vb प्रमाणं ज्ञापितं क्वचित्\lem \conj,\hskip.2em plus .9em प्रमाणं चापितं क्वचित् \mssCaCbCc\msNa\msNb\msParis\msKOb\msPaperA\Ed,\hskip.5em plus .9em 
प्रमाञ्चापितत् क्वचित् \msNc,\hskip.5em plus .9em प्रमाणञ्चापितां कति \msM}}% 

%Verse 1:36

{\devanagarifont कति चाङ्गुलिमूर्ध्वेषु सूर्यस्तपति वै महीम् {॥ १:\hspace{.11em}३६॥} \veg\dontdisplaylinenum }%
     \var{{\devanagarifontvar\numnoemph\vc ॰र्ध्वेषु\lem \eme,\hskip.2em plus .9em ॰र्धेषु \mssCaCbCc\msNa\msNb\msNc\msParis\msM\msKOb\msPaperA\Ed}}% 
    \var{{\devanagarifontvar\numnoemph\vd सूर्यस्त॰\lem \mssALL,\hskip.2em plus .9em र्यो \msMacorr,\hskip.5em plus .9em शूर्यो \msMpcorr\oo 
 महीम्\lem \msCb\msCc\msNa\msParis\msM\msKOb\msPaperA,\hskip.2em plus .9em मही\uncl{म् } \msCa,\hskip.5em plus .9em मही \msNb\msNc\Ed}}% 

{\devanagarifont अनर्थयज्ञ उवाच {\dandab}\dontdisplaylinenum  }%
 
{\devanagarifont ब्रह्माण्डानां प्रसंख्यातुं मया शक्यं कथं द्विज \thinspace{\danda} \dontdisplaylinenum }%
     \var{{\devanagarifontvar\numemph\va ब्रह्मा॰\lem \mssALL,\hskip.2em plus .9em ब्रह्म॰ \msM\oo 
 प्रसंख्यातुं\lem \mssALL,\hskip.2em plus .9em प्रसंसा तु \msNb,\hskip.5em plus .9em च संख्यातुं \Ed}}% 
    \var{{\devanagarifontvar\numnoemph\vb शक्यं क॰\lem \msNa\msNb\msPaperApcorr\Ed,\hskip.2em plus .9em शक्या क॰ \mssCaCbCc\msNc\msParis\msKOb,\hskip.5em plus .9em सक्याङ्क॰ \msM,\hskip.5em plus .9em 
ह्यक्यं क॰ \msPaperAacorr}}% 

%Verse 1:37

{\devanagarifont देवास्ते ऽपि न जानन्ति मानुषाणां च का कथा {॥ १:\hspace{.11em}३७॥} \veg\dontdisplaylinenum }%
     \var{{\devanagarifontvar\numnoemph\vc देवास्ते ऽपि न\lem   \mssALL,\hskip.2em plus .9em देवतापि न \msM,\hskip.5em plus .9em 
देवास्ते ऽपि \msKObacorr}}% 
    \var{{\devanagarifontvar\numnoemph\vd मानुषाणां च\lem \mssALL,\hskip.2em plus .9em मानुषार्नञ्च \msMacorr,\hskip.5em plus .9em 
मानुषानाञ्च \msMpcorr}}% 

{\devanagarifont पर्यायेण तु वक्ष्यामि यथाशक्यं द्विजोत्तम \thinspace{\dandab} \dontdisplaylinenum }%
 
%Verse 1:38

{\devanagarifont ब्रह्मणा यत्पुराख्यातो मातरिश्वा यथा तथा {॥ १:\hspace{.11em}३८॥} \veg\dontdisplaylinenum }%
     \var{{\devanagarifontvar\numemph\vc यत्पुराख्यातो\lem \mssCaCbCc\msNa\msNb\msNc\msParis\msKOb,\hskip.2em plus .9em यत्पुराख्यातं \msM,\hskip.5em plus .9em 
यत्प्रयात्परायाख्यातो \msPaperA,\hskip.5em plus .9em 
यत्ममाख्यातो \Ed}}% 
    \paral{{\devanagarifontsmall \vcd {\englishfont \compare\ \BRAHMANDAPUR\ 3.4.58cd:} 
                         ब्रह्मा ददौ शास्त्रमिदं पुराणं मातरिश्वने }}

{\devanagarifont शिवाण्डाभ्यन्तरेणैव सर्वेषामिव भूभृताम् \thinspace{\dandab} \dontdisplaylinenum }%
     \var{{\devanagarifontvar\numemph\va शिवाण्डा॰\lem \mssALL,\hskip.2em plus .9em शिवाण्ड॰ \msMacorr,\hskip.5em plus .9em शिवाण्डे॰ \msMpcorr}}% 
    \var{{\devanagarifontvar\numnoemph\vb सर्वेषामिव भूभृताम्\lem \conj,\hskip.2em plus .9em सर्वेषामिव भूरिताः \msCa\msCb\msNc\msParis,\hskip.5em plus .9em 
सर्वेषामेव भूरिताः \msCc,\hskip.5em plus .9em 
सर्वेषामिव भूरिता \msNa,\hskip.5em plus .9em सर्वेषामेव भूरिणाम् \msNb,\hskip.5em plus .9em 
स\uncl{र्षपा} इव भाविता \msM,\hskip.5em plus .9em 
सर्वेषामिव भू\uncl{रि}ताः \msKOb,\hskip.5em plus .9em 
सर्वेषामेव भूरिनाः \msPaperA,\hskip.5em plus .9em 
सर्वेषामेव भूरिमां \Ed}}% 

%Verse 1:39

{\devanagarifont दश नाम दिशाष्टानां ब्रह्माण्डे कीर्तितं शृणु {॥ १:\hspace{.11em}३९॥} \veg\dontdisplaylinenum }%
     \var{{\devanagarifontvar\numnoemph\vc दश\lem \mssALL,\hskip.2em plus .9em दशा॰ \msKOb\oo 
 दिशाष्टाणां\lem         \mssALL,\hskip.2em plus .9em  शिवाष्टाणां \msNb,\hskip.5em plus .9em 
दिशाष्टाणा \msParis}}% 
    \var{{\devanagarifontvar\numnoemph\vd ब्रह्माण्डे\lem     \mssALL,\hskip.2em plus .9em ब्रह्मण्डा \msM\oo 
 कीर्तितं शृणु\lem \mssALL,\hskip.2em plus .9em य च कीर्तितम् \msCb,\hskip.5em plus .9em 
कीर्त्तिता शृणु \msM}}% 


\alalfejezet{भूभृतां नामानि}

\alalalfejezet{पूर्वतः}

{\devanagarifont सहासहः सहः सह्यो विसहः संहतो ऽसभा \thinspace{\dandab} \dontdisplaylinenum }%
     \var{{\devanagarifontvar\numemph\va सहासहः\lem   \msNc,\hskip.2em plus .9em                     साहासह \mssCaCbCc\msNa\msNb\msParis\msM\msKOb\msPaperA\Ed\oo 
 सहः सह्यो\lem \msCa\msCc\msNa\msNb\msNc\msParis\msKOb,\hskip.2em plus .9em सहः सज्ञा \msCb,\hskip.5em plus .9em 
सहो सह्यः \msM,\hskip.5em plus .9em 
सहः सज्ञो \msPaperA\Ed}}% 
    \var{{\devanagarifontvar\numnoemph\vb विसहः\lem     \msCa\msCb\msNa\msNb\msNc\msParis\msKOb\Ed,\hskip.2em plus .9em विसह \msCc\msM,\hskip.5em plus .9em विंसहः \msPaperA\oo 
 ऽसभा\lem      \msCa\msCc\msNa\msNb\msNc\msParis\msKOb,\hskip.2em plus .9em    सभाः \msCb,\hskip.5em plus .9em सहा \msM,\hskip.5em plus .9em सता \msPaperA\Ed}}% 

%Verse 1:40

{\devanagarifont प्रसहो ऽप्रसहः सानुः पूर्वतो दश नायकाः {॥ १:\hspace{.11em}४०॥} \veg\dontdisplaylinenum }%
     \var{{\devanagarifontvar\numnoemph\vc प्रसहो\lem  \mssALL,\hskip.2em plus .9em  प्रसहेः \Ed\oo 
 प्रसहः\lem \mssALL,\hskip.2em plus .9em 
प्रस\uncl{वः} \msCc,\hskip.5em plus .9em सप्रहः \Ed\oo 
 सानुः\lem    \mssCaCbCc\msNa\msNb\msParis\msKOb\msPaperA,\hskip.2em plus .9em            सानु \msNc\msM\Ed}}% 
    \var{{\devanagarifontvar\numnoemph\vd पूर्वतो\lem  \mssALL,\hskip.2em plus .9em  पर्वतो \Ed}}% 


\alalalfejezet{आग्नेये}

{\devanagarifont प्रभासो भासनो भानुः प्रद्योतो द्युतिमो द्युतिः \thinspace{\dandab} \dontdisplaylinenum }%
     \var{{\devanagarifontvar\numemph\va भासनो\lem \msCa\msCb\msNa\msNb\msNc\msParis\msM\msKOb,\hskip.2em plus .9em        भास\lacwithnum{1}  \msCc,\hskip.5em plus .9em 
भांसतो \msPaperA,\hskip.5em plus .9em 
भासतो \Ed\oo 
 भानुः\lem  \mssALL,\hskip.2em plus .9em भानु \msCb\msM}}% 
    \var{{\devanagarifontvar\numnoemph\vb द्युतिमो\lem \mssCaCbCc\msNa\msNb\msParis\msM\msKOb,\hskip.2em plus .9em             द्युतिनो \msNc\msPaperA\Ed}}% 

{\devanagarifont दीप्ततेजाश्च तेजाश्च तेजा तेजवहो दश  \danda\dontdisplaylinenum }%
     \var{{\devanagarifontvar\numnoemph\vc दीप्ततेजाश्च तेजाश्च\lem \msCa\msCc\msNa\msNb\msNc\msParis\msKOb\msPaperA,\hskip.2em plus .9em दीप्ततेजाश्च तेजश्च \msCb,\hskip.5em plus .9em 
दीप्ततेजस् तेजश्च \msM\ \unmetr,\hskip.5em plus .9em 
दीप्ततेजश्च तेजाश्च \Ed}}% 
    \var{{\devanagarifontvar\numnoemph\vd तेजा तेजवहो\lem \mssALL,\hskip.2em plus .9em      तेजतेजयह \msM}}% 

%Verse 1:41

{\devanagarifont आग्नेये त्वेतदाख्यातं याम्ये शृण्वथ भो द्विज {॥ १:\hspace{.11em}४१॥} \veg\dontdisplaylinenum }%
     \var{{\devanagarifontvar\numnoemph\ve आग्नेये\lem         \mssCaCbCc\msNa\msNb\msParis\msKOb\Ed,\hskip.2em plus .9em              आग्नेय \msNc\msPaperA,\hskip.5em plus .9em 
आग्नेर्ये \msM\oo 
 त्वेतदा॰\lem \mssALL,\hskip.2em plus .9em त्वेचमा \msM}}% 
    \var{{\devanagarifontvar\numnoemph\vf शृण्वथ\lem    \mssALL,\hskip.2em plus .9em शृणुथ \msM\oo 
 द्विज\lem          \mssALL,\hskip.2em plus .9em  द्विजः \msNb}}% 


\alalalfejezet{याम्ये}

{\devanagarifont यमो ऽथ यमुना यामः संयमो यमुनो ऽयमः \thinspace{\dandab} \dontdisplaylinenum }%
     \var{{\devanagarifontvar\numemph\va यमो\lem    \mssALL,\hskip.2em plus .9em यमा \msPaperA}}% 
    \var{{\devanagarifontvar\numnoemph\vb संयमो\lem \mssALL,\hskip.2em plus .9em     संयम \msM,\hskip.5em plus .9em 
संयमा \msPaperA\oo 
 यमुनो\lem  \msCa\msCb\msNb\msParis\msKOb\msPaperA,\hskip.2em plus .9em                यमनो \msCc\msNc,\hskip.5em plus .9em 
युमुना \msNa,\hskip.5em plus .9em 
यमतो \msM,\hskip.5em plus .9em यमुना॰ \Ed\oo 
 यमः\lem   \mssALL,\hskip.2em plus .9em     यन \msM,\hskip.5em plus .9em 
यामः \msPaperA\ \unmetr}}% 

%Verse 1:42

{\devanagarifont संयनो यमनोयानो यनियुग्मा यनोयनः {॥ १:\hspace{.11em}४२॥} \veg\dontdisplaylinenum }%
     \var{{\devanagarifontvar\numnoemph\vc संयनो यमनोयानो\lem \msNa,\hskip.2em plus .9em संयमो यमनोयानो \msCa\msCc\Ed,\hskip.5em plus .9em 
संयमो यमुनोयानो \msCb\msNb\msParis\msKOb,\hskip.5em plus .9em 
संयमा यमनो यामो \msNc,\hskip.5em plus .9em 
यमियुग्मा यनो यानः \msM,\hskip.5em plus .9em 
संयमा यमनो यानो \msPaperA}}% 
    \var{{\devanagarifontvar\numnoemph\vd यनियुग्मा यनोयनः\lem \msNb,\hskip.2em plus .9em यनियुग्मा नयो यनः \msCa\msCc\msNa\msParis,\hskip.5em plus .9em 
यनियुग्मा नयो नयः \msCb\msKOb\msPaperA,\hskip.5em plus .9em 
यनियुग्मा नयो यमः \msNc,\hskip.5em plus .9em 
दशमा याम्यमाशृता \msM,\hskip.5em plus .9em 
यनियुग्मा नयोनय \Ed}}% 


\alalalfejezet{नै\char"0930\char"094D\char"090Bते}

{\devanagarifont नगजो नगना नन्दो नगरो नग नन्दनः \thinspace{\dandab} \dontdisplaylinenum }%
     \var{{\devanagarifontvar\numemph\va नगना नन्दो\lem        \msCa\msCc\msNa\msNb\msNc\msParis\msKOb,\hskip.2em plus .9em 
नगजा नन्दो \msCb,\hskip.5em plus .9em 
नगनागेन्द्र \msM,\hskip.5em plus .9em 
नगनो नदो \msPaperA\Ed}}% 
    \var{{\devanagarifontvar\numnoemph\vb नगरो नगनन्दनः\lem \msNb\msMacorr\msPaperA,\hskip.2em plus .9em नगरोरगनन्दनः \msCa\msNc,\hskip.5em plus .9em 
नगरो\uncl{नगनन्द}नः \msCb,\hskip.5em plus .9em 
नग\uncl{रो}\lacwithnum{2}  नन्दनः \msCc,\hskip.5em plus .9em 
नगरोगरनन्दनः \msNa,\hskip.5em plus .9em 
नगरोरगनन्दनः \msParis\msKOb,\hskip.5em plus .9em 
नगरो नननन्दनः \msMpcorr,\hskip.5em plus .9em 
नगरोन्नगनन्दनः \Ed}}% 

%Verse 1:43

{\devanagarifont नगर्भो गहनो गुह्यो गूढजो दश तत्परः {॥ १:\hspace{.11em}४३॥} \veg\dontdisplaylinenum }%
     \var{{\devanagarifontvar\numnoemph\vc नगर्भो\lem     \mssALL,\hskip.2em plus .9em      नृगभो \msNb,\hskip.5em plus .9em नगर्भ \msM\oo 
 गहनो गुह्यो\lem \mssALL,\hskip.2em plus .9em 
गुहनो गुह्य \msM,\hskip.5em plus .9em गहनो गुह्ये \Ed}}% 
    \var{{\devanagarifontvar\numnoemph\vd गूढजो\lem      \mssALL,\hskip.2em plus .9em गुडजो \msM\oo 
 तत्परः\lem     \mssALL,\hskip.2em plus .9em तत्परम् \msM}}% 


\alalalfejezet{वारुणे}

{\devanagarifont वारुणेन प्रवक्ष्यामि शृणु विप्र निबोध मे \thinspace{\dandab} \dontdisplaylinenum }%
     \var{{\devanagarifontvar\numemph\va वारुणेन\lem \mssALL,\hskip.2em plus .9em वारुणे च \Ed}}% 
    \var{{\devanagarifontvar\numnoemph\vb शृणु\lem      \msNb\msParis\msM,\hskip.2em plus .9em  शृङ्गे \msCa\msCb\msNa\msNc,\hskip.5em plus .9em 
शृ\uncl{ङ्गे} \msCc,\hskip.5em plus .9em 
शृङ्गे \msKOb,\hskip.5em plus .9em 
मृद्धे \uncl{पाप्त} \cancelled\ \msPaperA,\hskip.5em plus .9em 
मृद्धे \Ed}}% 

{\devanagarifont बभ्रुः सेतुर्भवोद्भद्रः प्रभवोद्भवभाजनः  \danda\dontdisplaylinenum }%
     \var{{\devanagarifontvar\numnoemph\vc बभ्रुः सेतुर्भवो॰\lem \corr,\hskip.2em plus .9em  बभ्रुं सेतुर्भवो॰ \msCa\msCb,\hskip.5em plus .9em 
बभ्रुं सेतु भवो॰ \msCc,\hskip.5em plus .9em 
बभ्रुः सेतु भवो॰ \msNa,\hskip.5em plus .9em 
बभ्रं सोतुर्भवो॰ \msNb,\hskip.5em plus .9em 
बभ्रु सेर्तुभवो॰ \msNc,\hskip.5em plus .9em 
बभ्रुं सेतुर्भवे॰ \msParis\msKOb,\hskip.5em plus .9em 
बभ्रू सेतु भवो॰ \msM,\hskip.5em plus .9em 
बभ्रून्सेतुर्भवो॰ \msPaperA,\hskip.5em plus .9em 
बभ्रून्सतुर्भवो॰ \Ed}}% 
    \var{{\devanagarifontvar\numnoemph\vd प्रभवोद्भव॰\lem \mssALL,\hskip.2em plus .9em   प्रभवोभव॰ \msM\oo 
 ॰भाजनः\lem       \mssALL,\hskip.2em plus .9em ॰भाजन \Ed}}% 

%Verse 1:44

{\devanagarifont भरणो भुवनो भर्ता दशैते वरुणालयाः {॥ १:\hspace{.11em}४४॥} \veg\dontdisplaylinenum }%
     \var{{\devanagarifontvar\numnoemph\ve भरणो\lem \msCb\msNc,\hskip.2em plus .9em भरण \msCa\msNa,\hskip.5em plus .9em भरणां \msCc\msPaperA\Ed,\hskip.5em plus .9em 
भरणा \msNb,\hskip.5em plus .9em भरणे \msParis\msKOb,\hskip.5em plus .9em भरणः \msM}}% 
    \var{{\devanagarifontvar\numnoemph\vf दशैते\lem \mssALL,\hskip.2em plus .9em    दशेते \msNc,\hskip.5em plus .9em दशैता \msM\oo 
 ॰लयाः\lem  \mssALL,\hskip.2em plus .9em ॰लया \msM\Ed}}% 


\alalalfejezet{वायव्ये}

{\devanagarifont नृगर्भो ऽसुरगर्भश्च देवगर्भो महीधरः \thinspace{\dandab} \dontdisplaylinenum }%
     \var{{\devanagarifontvar\numemph\va नृगर्भो\lem      \mssALL,\hskip.2em plus .9em नृगभा \msM\oo 
 ॰गर्भश्च\lem \msCa\msCb\msNb\msNc\msParis\msKOb\msPaperA,\hskip.2em plus .9em               ॰गर्भाश्च \msCc\msNa\msM\Ed}}% 
    \var{{\devanagarifontvar\numnoemph\vb देवगर्भो\lem    \mssALL,\hskip.2em plus .9em देवगर्भ \msM}}% 

%Verse 1:45

{\devanagarifont वृषभो वृषगर्भश्च वृषाङ्को वृषभध्वजः {॥ १:\hspace{.11em}४५॥} \veg\dontdisplaylinenum }%
     \var{{\devanagarifontvar\numnoemph\vc ॰गर्भश्च\lem \mssCaCbCc\msNb\msNc\msParis\msKOb\Ed,\hskip.2em plus .9em ॰गर्भाश्च \msNa,\hskip.5em plus .9em ॰गर्भोश्च \msM,\hskip.5em plus .9em 
॰शभश्च \msPaperA}}% 
    \var{{\devanagarifontvar\numnoemph\vd वृषाङ्को\lem  \mssALL,\hskip.2em plus .9em     वृषांगो \msM\oo 
 वृषभ॰\lem \mssALL,\hskip.2em plus .9em वृष\lk ॰ \msCc}}% 

{\devanagarifont ज्ञातव्यश्च तथा सम्यग् वृषजो वृषनन्दनः \thinspace{\dandab} \dontdisplaylinenum }%
     \var{{\devanagarifontvar\numemph\va ज्ञातव्यश्च तथा सम्यग्\lem \mssCaCbCc\msNa\msNb\msNc,\hskip.2em plus .9em 
ज्ञातव्यश्च यथा सम्यग् \msParis\msKOb,\hskip.5em plus .9em 
वृषञ्जवृषनन्दश्च \msM,\hskip.5em plus .9em 
ज्ञानवाञ्च तथा सम्य \msPaperA,\hskip.5em plus .9em 
ज्ञानवाञ्च तथा सत्य॰ \Ed}}% 
    \var{{\devanagarifontvar\numnoemph\vb वृषजो वृषनन्दनः\lem \mssALL,\hskip.2em plus .9em वृषनन्दनः \msNa,\hskip.5em plus .9em 
दशनायक वायवे \msM}}% 

%Verse 1:46

{\devanagarifont नायका दश वायव्ये कीर्तिता ये मया द्विज {॥ १:\hspace{.11em}४६॥} \veg\dontdisplaylinenum }%
     \var{{\devanagarifontvar\numnoemph\vcd नायका दश वायव्ये कीर्तिता ये मया द्विज\lem \msCa\msCb\msNa\msParis\msKOb\msPaperA\Ed,\hskip.2em plus .9em 
नायका दश वायव्ये कीर्तिता ये मया द्विजः \msCc\msNb,\hskip.5em plus .9em 
नायका दश वायव्ये कीर्तिता य मया द्विज \msNc,\hskip.5em plus .9em 
कीर्तितो यं मया द्विप्र यथा तथ्येन सुव्रतः \msM}}% 


\alalalfejezet{उत्तरे}

{\devanagarifont सुलभः सुमनः सौम्यः सुप्रजः सुतनुः शिवः \thinspace{\dandab} \dontdisplaylinenum }%
     \var{{\devanagarifontvar\numemph\va सुलभः\lem \mssALL,\hskip.2em plus .9em सुरभः \msPaperA\Ed\oo 
 सुमनः\lem  \mssCaCbCc\msNa\msNb\Ed,\hskip.2em plus .9em सुमनाः \msNc,\hskip.5em plus .9em सुसमः \msParis\msKOb,\hskip.5em plus .9em सुमनो \msM,\hskip.5em plus .9em सुमन \msPaperA\oo 
 सौम्यः\lem  \mssALL,\hskip.2em plus .9em सोम्य \msM}}% 

%Verse 1:47

{\devanagarifont सतः सत्य लयः शम्भुर्दश नायकमुत्तरे {॥ १:\hspace{.11em}४७॥} \veg\dontdisplaylinenum }%
     \var{{\devanagarifontvar\numnoemph\vc सतः सत्य\lem \corr,\hskip.2em plus .9em सत सत्य \mssCaCbCc\msNc\msParis\msKOb\msPaperA,\hskip.5em plus .9em सत्यसत्य \msNa,\hskip.5em plus .9em सुत सत्य \msNb,\hskip.5em plus .9em 
सुतः सत्य \msM,\hskip.5em plus .9em सत सत्या॰ \Ed\oo 
 लयः\lem \mssALL,\hskip.2em plus .9em लयं \msNc}}% 
    \var{{\devanagarifontvar\numnoemph\vcd शम्भुर्द॰\lem \msCa\msCb\msNb\msParis\msKOb\msPaperA\Ed,\hskip.2em plus .9em शम्भु द॰ \msCc\msNa\msNc,\hskip.5em plus .9em 
शम्\uncl{भुं} द॰ \msM}}% 
    \var{{\devanagarifontvar\numnoemph\vd ॰नायकमु॰\lem \mssALL,\hskip.2em plus .9em ॰नायक उ॰ \Ed}}% 


\alalalfejezet{ईशाने}

{\devanagarifont इन्दु बिन्दु भुवो वज्र वरदो वर वर्षणः \thinspace{\dandab} \dontdisplaylinenum }%
     \var{{\devanagarifontvar\numemph\va वज्र\lem \mssALL,\hskip.2em plus .9em व्रजः \msM}}% 
    \var{{\devanagarifontvar\numnoemph\vb ॰वर्षणः\lem \mssCaCbCc\msNa\msNb\msParis\msM\msKOb,\hskip.2em plus .9em ॰\lk \uncl{र्शणम्} \msNc,\hskip.5em plus .9em 
॰दर्प्पणः \msPaperA,\hskip.5em plus .9em 
॰दर्य्य च \Ed}}% 

%Verse 1:48

{\devanagarifont इलिनो वलिनो ब्रह्मा दशेशानेषु नायकाः {॥ १:\hspace{.11em}४८॥} \veg\dontdisplaylinenum }%
     \var{{\devanagarifontvar\numnoemph\vc ब्रह्मा\lem \mssALL,\hskip.2em plus .9em ब्रह्मः \msM}}% 
    \var{{\devanagarifontvar\numnoemph\vd दशे॰\lem               \msCa\msNa\msNc\msParis\msKOb\msPaperA\Ed,\hskip.2em plus .9em  दशै॰ \msCb\msCc\msNb,\hskip.5em plus .9em दिशै॰ \msM\oo 
 नायकाः\lem             \mssALL,\hskip.2em plus .9em नायका \msM}}% 


\alalalfejezet{मध्यमे}

{\devanagarifont अपरो विमलो मोहो निर्मलो मन मोहनः \thinspace{\dandab} \dontdisplaylinenum }%
     \var{{\devanagarifontvar\numemph\va अपरो विमलो मोहो\lem \mssALL,\hskip.2em plus .9em अपरः विमला मोहा \msM}}% 
    \var{{\devanagarifontvar\numnoemph\vb निर्मलो म॰\lem \eme,\hskip.2em plus .9em निमलो म॰ \msCa,\hskip.5em plus .9em निर्मलोन्म॰ \msCb\msNc\msKOb\msPaperA,\hskip.5em plus .9em 
निर्मलोत्म॰ \msCc\Ed,\hskip.5em plus .9em निमलोर्म॰ \msNa\msNb,\hskip.5em plus .9em निमलोत्म॰ \msParis,\hskip.5em plus .9em निर्मलोन्म॰ \msM}}% 

%Verse 1:49

{\devanagarifont अक्षयश्चाव्ययो विष्णुर्वरदो मध्यमे दश {॥ १:\hspace{.11em}४९॥} \veg\dontdisplaylinenum }%
     \var{{\devanagarifontvar\numnoemph\vc अक्षयश्चाव्ययो\lem \msCa\msCb\msNa\msNb\msNc\msParis\msKOb\msPaperA,\hskip.2em plus .9em अक्षयाश्चाव्ययो \msCc,\hskip.5em plus .9em 
अक्षयश्चाव्ययं \msM,\hskip.5em plus .9em अक्षयञ्चाव्ययो \Ed}}% 
    \var{{\devanagarifontvar\numnoemph\vcd विष्णुर्व॰\lem \msCa\msCb\msNc\msPaperA\Ed,\hskip.2em plus .9em विष्णु व॰ \msCc\msNa\msParis\msM\msKOb,\hskip.5em plus .9em र्विष्णुर्व॰ \msNb}}% 
    \var{{\devanagarifontvar\numnoemph\vd मध्यमे दश\lem \msCa\msCb\msNc\msPaperA,\hskip.2em plus .9em मध्यमो दश \msCc\msNa\msParis\msKOb,\hskip.5em plus .9em 
वरवर्षणः \msNb,\hskip.5em plus .9em मध्यमो दशः \msM,\hskip.5em plus .9em मध्यमे दशः \Ed}}% 


\alalalfejezet{परिवाराः}

{\devanagarifont सर्वेषां दशमीशानां परिवारशतं शतम् \thinspace{\dandab} \dontdisplaylinenum }%
     \var{{\devanagarifontvar\numemph\va सर्वेषां\lem       \mssALL,\hskip.2em plus .9em   सर्वेषा \msNc\oo 
 दशमीशानां\lem \mssALL,\hskip.2em plus .9em दशरीशानां \Ed}}% 
    \var{{\devanagarifontvar\numnoemph\vb परिवार॰\lem      \mssALL,\hskip.2em plus .9em   परि॰ \msCb,\hskip.5em plus .9em परिवारं \msNa}}% 

%Verse 1:50

{\devanagarifont शतानां पृथगेकैकं सहस्रैः परिवारितम् {॥ १:\hspace{.11em}५०॥} \veg\dontdisplaylinenum }%
     \var{{\devanagarifontvar\numnoemph\vd सहस्रैः\lem \mssALL,\hskip.2em plus .9em  सहस्रै \msM\oo 
 ॰वारितम्\lem  \msCa\msCb\msCcpcorr\msNa\msNb\msNc\msParis\msKOb\msPaperA,\hskip.2em plus .9em ॰वारिता \msCcacorr,\hskip.5em plus .9em 
॰वारितः \msM,\hskip.5em plus .9em ॰वारिताः \Ed}}% 

{\devanagarifont सहस्रेषु च एकैकमयुतैः परिवारितम् \thinspace{\dandab} \dontdisplaylinenum }%
     \var{{\devanagarifontvar\numemph\vab एकैकम॰\lem \msCa\msCb\msNb\msNc\msParis\msKOb\msPaperA\Ed,\hskip.2em plus .9em          एकैकं म॰ \msCc\msNa\msM}}% 
    \var{{\devanagarifontvar\numnoemph\vb परिवारितम्\lem  \mssALL,\hskip.2em plus .9em परिवारितः \msM,\hskip.5em plus .9em परिवारितमाः \Ed}}% 

%Verse 1:51

{\devanagarifont अयुतं प्रयुतैर्वृन्दैः प्रयुतं नियुतैर्वृतम् {॥ १:\hspace{.11em}५१॥} \veg\dontdisplaylinenum }%
     \var{{\devanagarifontvar\numnoemph\vc अयुतं\lem \Ed,\hskip.2em plus .9em अयुतैः \mssCaCbCc\msNa\msNc\msParis\msM\msKOb\msPaperA,\hskip.5em plus .9em अयुतै \msNb\oo 
 प्रयुतैर्वृन्दैः\lem \mssALL,\hskip.2em plus .9em प्रयुतै वृन्दैः \msNc,\hskip.5em plus .9em 
प्रयुतैर्भृत्य \msM}}% 
    \var{{\devanagarifontvar\numnoemph\vd प्रयुतं नियुतैर्वृतम्\lem \corr,\hskip.2em plus .9em प्रयुतैर्नियुतैर्वृतः \msCa\msCb\msNa\msNc\msParis,\hskip.5em plus .9em 
प्रयुतेर्नियुतैर्वृतः \msCc,\hskip.5em plus .9em प्रयुतै नियुतै वृतः \msNb,\hskip.5em plus .9em 
प्रयुतः नियुतैः वृतः \msM,\hskip.5em plus .9em प्रयुतै नियुतैर्वृतः \msKOb,\hskip.5em plus .9em 
प्रयुते नियुतैर्वृतः \msPaperA,\hskip.5em plus .9em प्रयुतं नियुतैर्वृतः \Ed}}% 

{\devanagarifont एकैकस्य परीवारो नियुतः पृथगेव च \thinspace{\dandab} \dontdisplaylinenum }%
     \var{{\devanagarifontvar\numemph\va परीवारो\lem \mssALL,\hskip.2em plus .9em             परिवार \msM\ \unmetr,\hskip.5em plus .9em 
परिवारो \Ed\ \unmetr}}% 
    \var{{\devanagarifontvar\numnoemph\vb नियुतः\lem  \mssALL,\hskip.2em plus .9em      नियुत \msCc\oo 
 च\lem       \mssALL,\hskip.2em plus .9em चः \msNcacorr}}% 

%Verse 1:52

{\devanagarifont कोटिभिर्दशकोट्येन एकैकः परिवारितः {॥ १:\hspace{.11em}५२॥} \veg\dontdisplaylinenum }%
     \var{{\devanagarifontvar\numnoemph\vc कोटिभिर्दशकोट्येन\lem \msCa\msCc\msParispcorr\msPaperA\Ed,\hskip.2em plus .9em कोटिभि दशकोट्येन \msCb,\hskip.5em plus .9em 
कोटिभिर्दशकोट्योन \msNa\msNc\msParisacorr\msKOb,\hskip.5em plus .9em कोटिभिर्दशकोट्येनः \msNb,\hskip.5em plus .9em 
कोटिभिः परिवाराणि कोटिभि दशकोटिकम् \msM}}% 
    \var{{\devanagarifontvar\numnoemph\vd एकैकः परिवारितः\lem \msCb\msNa\Ed,\hskip.2em plus .9em एकैकः परिवरि\uncl{तः} \msCa,\hskip.5em plus .9em 
एकैकपरिवारितः \msCc\msNb\msNc\msParis\msKOb,\hskip.5em plus .9em एकैकपरिवाराणां \msM,\hskip.5em plus .9em 
एकैकः परिवारितं \msPaperA}}% 

{\devanagarifont दशकोटिषु एकैकं वृन्दवृन्दभृतैर्वृतम् \thinspace{\dandab} \dontdisplaylinenum }%
     \var{{\devanagarifontvar\numemph\va दशकोटिषु एकैकं\lem \msCb\msCc\msNb\msPaperA\Ed,\hskip.2em plus .9em दशकोटीषु एकैकं \msCa\msNa\msNc\msParis\msKOb,\hskip.5em plus .9em 
दशकोट्येषु एककं \msM}}% 
    \var{{\devanagarifontvar\numnoemph\vb वृन्दवृन्दभृतैर्वृतम्\lem \mssCaCbCc\msNb\msParis\msKOb,\hskip.2em plus .9em वृन्दवृन्दवृतैर्वृतं \msNa,\hskip.5em plus .9em 
वृन्दवृन्दभृतै वृतं \msNc,\hskip.5em plus .9em 
वृन्द्रवृन्देषु एकैकं \msM,\hskip.5em plus .9em 
वृन्दवृन्दवृतैर्वृत \msPaperA,\hskip.5em plus .9em 
वृन्दवृन्दं वृतैर्वृतः \Ed}}% 

%Verse 1:53

{\devanagarifont वृन्दवर्गेषु एकैकं खर्वभिः परिवारितम् {॥ १:\hspace{.11em}५३॥} \veg\dontdisplaylinenum }%
     \var{{\devanagarifontvar\numnoemph\vc वृन्दवर्गेषु\lem \mssALL,\hskip.2em plus .9em वृन्दवर्गेभिः तै वृतम् \msM}}% 
    \var{{\devanagarifontvar\numnoemph\vd खर्वभिः परिवारितम्\lem \mssCaCbCc\msNa\msNb\msParis\msKOb,\hskip.2em plus .9em खर्वर्भिः परिवारितम् \msNc,\hskip.5em plus .9em 
खर्वाभिः परिवाराणि \msM,\hskip.5em plus .9em 
खर्वर्भिः परिवारित \msPaperA,\hskip.5em plus .9em 
खर्वर्भिः परिवारितः \Ed}}% 

{\devanagarifont खर्ववर्गेषु एकैकं दशखर्वगणैर्वृतम् \thinspace{\dandab} \dontdisplaylinenum }%
     \var{{\devanagarifontvar\numemph\va खर्ववर्गेषु एकैकं\lem    \mssALL,\hskip.2em plus .9em 
खर्ववर्गेव एककम् \msM}}% 
    \var{{\devanagarifontvar\numnoemph\vb दशखर्वगणैर्वृतम्\lem \msCa\msCc\msNa\msNb\msParis\msKOb\msPaperA,\hskip.2em plus .9em दशखर्वगणै वृतम् \msCb,\hskip.5em plus .9em 
दशखर्वगणे वृत्तं \msNc,\hskip.5em plus .9em 
दशखर्वेषु एकैकं दशखर्वगणैर्वृतम् \msM,\hskip.5em plus .9em 
दशखर्वगणैर्वृतः \Ed}}% 

%Verse 1:54

{\devanagarifont दशखर्वेषु एकैकं शङ्कुभिः परिवारितम् {॥ १:\hspace{.11em}५४॥} \veg\dontdisplaylinenum }%
     \var{{\devanagarifontvar\numnoemph\vc ॰खर्वेषु\lem   \mssALL,\hskip.2em plus .9em ॰गर्वेषु \msNc}}% 
    \var{{\devanagarifontvar\numnoemph\vd परिवारितम्\lem \mssALL,\hskip.2em plus .9em परिवारित \msPaperA,\hskip.5em plus .9em परिवारितः \Ed}}% 

{\devanagarifont शङ्कुभिः पृथगेकैकं पद्मेन परिवारितम् \thinspace{\dandab} \dontdisplaylinenum }%
     \var{{\devanagarifontvar\numemph\va पृथगेकैकं\lem \eme,\hskip.2em plus .9em पृथगेनैव \msCa\msCc\msNa\msNb\msNc\msParis\msM\msPaperA\Ed,\hskip.5em plus .9em 
पृथगैनैव \msCb,\hskip.5em plus .9em 
पृथगे\uncl{नै}व \msKOb}}% 
    \var{{\devanagarifontvar\numnoemph\vb ॰वारितम्\lem \msNapcorr\msM,\hskip.2em plus .9em ॰वारितः \mssCaCbCc\msNb\msNc\msParis\msKOb\msPaperA\Ed,\hskip.5em plus .9em ॰तं \msNaacorr}}% 

%Verse 1:55

{\devanagarifont पद्मवर्गेषु एकैकं समुद्रैः परिवारितम् {॥ १:\hspace{.11em}५५॥} \veg\dontdisplaylinenum }%
     \var{{\devanagarifontvar\numnoemph\vd समुद्रैः\lem \mssALL,\hskip.2em plus .9em 
समुदैः \msCa,\hskip.5em plus .9em दमु\uncl{दैः} \msCb\oo 
 ॰वारितम्\lem  \mssALL,\hskip.2em plus .9em ॰वारितः \Ed}}% 

{\devanagarifont समुद्रेषु तथैकैकं मध्यसंख्यैस्तु तैर्वृतम् \thinspace{\dandab} \dontdisplaylinenum }%
     \var{{\devanagarifontvar\numemph\va तथै॰\lem \mssALL,\hskip.2em plus .9em तथे॰ \msCc}}% 
    \var{{\devanagarifontvar\numnoemph\vb मध्यसंख्यैस्तु तैर्वृतम्\lem \mssCaCbCc\msNa\msParis\msM\msKOb\msPaperA,\hskip.2em plus .9em 
मध्यसख्यैस्तु तै वृतम् \msNb,\hskip.5em plus .9em 
मध्यसख्यैस्तु तेर्वृतं \msNc,\hskip.5em plus .9em 
मध्ये शङ्ख्यायुतैर्वृतः \Ed}}% 

%Verse 1:56

{\devanagarifont मध्यसंख्येषु एकैकमनन्तैः परिवारितम् {॥ १:\hspace{.11em}५६॥} \veg\dontdisplaylinenum }%
     \var{{\devanagarifontvar\numnoemph\vc मध्यसंख्येषु\lem     \mssALL,\hskip.2em plus .9em 
मध्यसांखो च \msM,\hskip.5em plus .9em मध्ये शंखेषु \Ed}}% 
    \var{{\devanagarifontvar\numnoemph\vcd एकैकमनन्तैः\lem \mssALL,\hskip.2em plus .9em 
एकैकं मनतैः \msNc,\hskip.5em plus .9em एकैकं अनन्तै \msM}}% 
    \var{{\devanagarifontvar\numnoemph\vd ॰वारितम्\lem            \mssALL,\hskip.2em plus .9em ॰वारितः \Ed}}% 

{\devanagarifont अनन्तेषु च एकैकं परार्धपरिवारितम् \thinspace{\dandab} \dontdisplaylinenum }%
     \var{{\devanagarifontvar\numemph\va अनन्तेषु च एकैकं\lem \mssALL,\hskip.2em plus .9em अनन्तेषु न एकैकं \msKObpcorr,\hskip.5em plus .9em 
\om\ \msKObacorr}}% 
    \var{{\devanagarifontvar\numnoemph\vb परार्धपरिवारितम्\lem \msCa\msCb\msNa\msNb\msNc\msKObpcorr\msPaperA,\hskip.2em plus .9em परार्ध\lacwithnum{3}  रितम् \msCc,\hskip.5em plus .9em 
परार्धै परिवारितम्\thinspace{\devanagarifont ।} अनन्तेषु च एकैकं परार्धपरिवारितं \msM,\hskip.5em plus .9em \om\ \msKObacorr,\hskip.5em plus .9em परार्धैः परिवारितः \Ed}}% 
    \lacuna{\devanagarifontsmall \vab {\englishfont omitted in \msParis\ and written in the top margin in \msKObpcorr} }%
  
{\devanagarifont परार्धेषु च एकैकं परेण परिवारितम्  \danda\dontdisplaylinenum }%
     \var{{\devanagarifontvar\numnoemph\vd ॰वारितम्\lem \mssALL,\hskip.2em plus .9em ॰वारिवारितं \msNb,\hskip.5em plus .9em ॰वारितः \Ed}}% 

%Verse 1:57

{\devanagarifont एष वै कथितो विप्र शक्यं सांख्यमुदीरितम् {॥ १:\hspace{.11em}५७॥} \veg\dontdisplaylinenum }%
     \var{{\devanagarifontvar\numnoemph\ve कथितो\lem \mssALL,\hskip.2em plus .9em \uncl{कथितो} \msNb,\hskip.5em plus .9em कथिता \Ed}}% 
    \var{{\devanagarifontvar\numnoemph\vf शक्यं\lem  \mssALL,\hskip.2em plus .9em  शक्य \msCc,\hskip.5em plus .9em संख्यां शक्यं \msPaperA\oo 
 सांख्यमु॰\lem \msCa\msCc\msNb\msParis\msM\msKObacorr,\hskip.2em plus .9em साख्यमु॰ \msCb,\hskip.5em plus .9em स्यख्यमु॰ \msNa,\hskip.5em plus .9em 
संख्यमु \msNc\msKObpcorr,\hskip.5em plus .9em संख्यामु॰ \msPaperA\Ed}}% 


\alalfejezet{प्रमाणम्}
{\devanagarifont प्रमाणं शृणु मे विप्र संक्षेपाद्ब्रुवतो मम \thinspace{\dandab} \dontdisplaylinenum }%
     \var{{\devanagarifontvar\numemph\va प्रमाणं\lem \msCc\msNa\msNc\msParis\msM\msKOb\msPaperA\Ed,\hskip.2em plus .9em प्रणामं \msCa\msCb,\hskip.5em plus .9em प्रमाण \msNb}}% 
    \var{{\devanagarifontvar\numnoemph\vb संक्षेपाद्ब्रुवतो\lem \msCa\msCc\msNa\msNb\msKOb\msPaperA\Ed,\hskip.2em plus .9em संक्षेपाद्वदतो \msCb,\hskip.5em plus .9em 
संख्येपाद्ब्रुवतो \msNc,\hskip.5em plus .9em 
संक्षेप ब्रुवतो \msM}}% 
    \lacuna{\devanagarifontsmall \vb {\englishfont After} संक्षेपा॰ {\englishfont \msParis\ f. 213v breaks off and resumes only at 2.21c. One folio
                containing 1.58cd--2.21ab is missing.} }%
  
%Verse 1:58

{\devanagarifont चन्द्रोदये पूर्णमास्यां वपुरण्डस्य तादृशम् {॥ १:\hspace{.11em}५८॥} \veg\dontdisplaylinenum }%
 
{\devanagarifont कोटिकोटिसहस्रं तु योजनानां समन्ततः \thinspace{\dandab} \dontdisplaylinenum }%
     \var{{\devanagarifontvar\numemph\va कोटिकोटि॰\lem \mssALL,\hskip.2em plus .9em कोटीकोटि॰ \msM}}% 
    \var{{\devanagarifontvar\numnoemph\vb योज॰\lem     \mssALL,\hskip.2em plus .9em      याज॰ \msPaperA}}% 

%Verse 1:59

{\devanagarifont अण्डानां च परीमाणं ब्रह्मणा परिकीर्तितम् {॥ १:\hspace{.11em}५९॥} \veg\dontdisplaylinenum }%
     \var{{\devanagarifontvar\numnoemph\vc च परीमाणं\lem \mssALL,\hskip.2em plus .9em 
च परिमाणं \msCb\ \unmetr,\hskip.5em plus .9em 
परिमाणञ्च \msM}}% 
    \var{{\devanagarifontvar\numnoemph\vd ब्रह्मणा\lem \mssALL,\hskip.2em plus .9em \lacwithnum{3}  \msCc\oo 
 ॰कीर्तितम्\lem \msCa\msCb\msNb\msNc\msKOb\msPaperA\Ed,\hskip.2em plus .9em ॰कीर्ति\uncl{ताः} \msCc,\hskip.5em plus .9em ॰कीर्तितः \msNa\msM}}% 

{\devanagarifont सप्तकोटिसहस्राणि सप्तकोटिशतानि च \thinspace{\dandab} \dontdisplaylinenum }%
     \var{{\devanagarifontvar\numemph\va ॰स्राणि\lem \mssALL,\hskip.2em plus .9em ॰स्रणि \msPaperA}}% 

%Verse 1:60

{\devanagarifont विंशकोटिष्वङ्गुलीषु ऊर्ध्वतस्तपते रविः {॥ १:\hspace{.11em}६०॥} \veg\dontdisplaylinenum }%
     \var{{\devanagarifontvar\numnoemph\vc विंशकोटिष्वङ्गुलीषु\lem \conj,\hskip.2em plus .9em विंशकोटिषु गुल्मेषु \mssCaCbCc\msNa\msNb\msNc\msKOb\msPaperA\Ed,\hskip.5em plus .9em 
विंशकोटि विना गुल्मे \msM}}% 
    \var{{\devanagarifontvar\numnoemph\vd ऊर्ध्वतस्त॰\lem \mssCaCbCc\msNa\msNc\msKOb\Ed,\hskip.2em plus .9em ऊर्ध्व\lacwithnum{2}  \msNb,\hskip.5em plus .9em ऊर्द्ध्वतो त॰ \msM,\hskip.5em plus .9em 
उद्धतस्त॰ \msPaperA\oo 
 रविः\lem \mssALL,\hskip.2em plus .9em रवि \Ed}}% 
    \lacuna{\devanagarifontsmall \vcd {\englishfont The folio in \msNb\ ends with } ऊर्ध्व॰, {\englishfont and the folios 
                                that may have contained verses 1.60d--2.22 are missing.} }%
  
{\devanagarifont प्रमाणं नाम संख्या च कीर्तितानि समासतः \thinspace{\dandab} \dontdisplaylinenum }%
     \var{{\devanagarifontvar\numemph\va प्रमाणं नाम संख्या च\lem \msCa\msCc\msNa\msNc\msM\Ed,\hskip.2em plus .9em प्रणामं नाम संख्या च \msCb\msKOb,\hskip.5em plus .9em 
प्रमाणेनाणञ्चम संख्या\lk त च \msPaperA}}% 
    \var{{\devanagarifontvar\numnoemph\vb कीर्तितानि\lem \mssALL,\hskip.2em plus .9em कीर्त्तियानानि \msPaperA}}% 

%Verse 1:61

{\devanagarifont ब्रह्माण्डं चाप्रमेयाणां लक्षणं परिकीर्तितम् {॥ १:\hspace{.11em}६१॥} \veg\dontdisplaylinenum }%
     \var{{\devanagarifontvar\numnoemph\vc ब्रह्माण्डं चा॰\lem \msNa,\hskip.2em plus .9em ब्रह्माण्डश्च \msCa\msCb\msNc\msM\msKOb\msPaperA,\hskip.5em plus .9em \uncl{ब्रह्माण्डाश्चा}॰ \msCc,\hskip.5em plus .9em 
ब्रह्माण्डाश्चा \Ed\oo 
 ॰मेयाणां\lem \msCa\msNa\msM\msKOb\msPaperA\Ed,\hskip.2em plus .9em ॰मेयाणा \msCb\msCc\msNc}}% 
    \var{{\devanagarifontvar\numnoemph\vd ॰कीर्तितम्\lem \mssALL,\hskip.2em plus .9em ॰कीर्तिताः \msCc,\hskip.5em plus .9em ॰कीर्त्तितः \msM}}% 


\alalfejezet{पुराणम्}
{\devanagarifont पुराणाशीसहस्राणि शतानि द्विजसत्तम \thinspace{\dandab} \dontdisplaylinenum }%
     \var{{\devanagarifontvar\numemph\vb ॰सत्तम\lem \mssALL,\hskip.2em plus .9em \lacwithnum{2}  मः  \msCc}}% 

%Verse 1:62

{\devanagarifont ब्रह्मणा कथितं पूर्णं मातरिश्वा यथातथम् {॥ १:\hspace{.11em}६२॥} \veg\dontdisplaylinenum }%
     \var{{\devanagarifontvar\numnoemph\vc पूर्णं\lem    \msCa\msCc\msNa\msKOb\msPaperA\Ed,\hskip.2em plus .9em              पूर्वे \msCb,\hskip.5em plus .9em पूर्ण्ण \msNc,\hskip.5em plus .9em पूर्वं \msM}}% 
    \var{{\devanagarifontvar\numnoemph\vd मातरिश्वा\lem \mssALL,\hskip.2em plus .9em  मातरिश्व \msM\oo 
 ॰तथम्\lem   \mssALL,\hskip.2em plus .9em ॰तथा \msCc\msM}}% 

{\devanagarifont वायुना पाद संक्षिप्य प्राप्तं चोशनसं पुरा \thinspace{\dandab} \dontdisplaylinenum }%
     \var{{\devanagarifontvar\numemph\va संक्षिप्य\lem \mssALL,\hskip.2em plus .9em संक्षिप्यः \msM}}% 
    \var{{\devanagarifontvar\numnoemph\vb प्राप्तं चोशनसं\lem \msCb\msNa\msNc,\hskip.2em plus .9em प्राप्तं चौसनसं \msCa\msKOb\msPaperA,\hskip.5em plus .9em प्राप्त\lk औसनसं \msCc,\hskip.5em plus .9em 
प्राप्ताश्चोशनसम \msM\ \unmetr,\hskip.5em plus .9em प्राप्तश्चोशनसं \Ed}}% 

%Verse 1:63

{\devanagarifont तेनापि पाद संक्षिप्य प्राप्तवांश्च बृहस्पतिः {॥ १:\hspace{.11em}६३॥} \veg\dontdisplaylinenum }%
     \var{{\devanagarifontvar\numnoemph\vc संक्षिप्य\lem \mssALL,\hskip.2em plus .9em संक्षिप्यः \msM}}% 
    \var{{\devanagarifontvar\numnoemph\vd प्राप्तवांश्च बृहस्पतिः\lem \mssALL,\hskip.2em plus .9em प्राप्तधञ्च वृहस्पति \msM}}% 

{\devanagarifont बृहस्पतिस्तु प्रोवाच सूर्यं त्रिंशत्सहस्रिकम् \thinspace{\dandab} \dontdisplaylinenum }%
     \var{{\devanagarifontvar\numemph\vb सूर्यं\lem \msCc\Ed,\hskip.2em plus .9em सूर्यस् \msCa\msNa\msNc\msKOb\msPaperA,\hskip.5em plus .9em सूर्य \msCb\msM\oo 
 त्रिंशत्स॰\lem \mssALL,\hskip.2em plus .9em त्रिंशस॰ \msCc\msM}}% 

%Verse 1:64

{\devanagarifont पञ्चविंशत्सहस्राणि मृत्युं प्राह दिवाकरः {॥ १:\hspace{.11em}६४॥} \veg\dontdisplaylinenum }%
     \var{{\devanagarifontvar\numnoemph\vc ॰विंशत्सहस्राणि\lem \corr,\hskip.2em plus .9em ॰विंशहस्राणि \msCa,\hskip.5em plus .9em 
॰विंशसहस्राणि \msCb\msCc\msNa\msNc\msM\msKOb\msPaperA,\hskip.5em plus .9em ॰विशत्सहस्राणि \Ed}}% 
    \var{{\devanagarifontvar\numnoemph\vd मृत्युं प्राह\lem \mssALL,\hskip.2em plus .9em मृत्यु प्राहः \msM}}% 

{\devanagarifont एकविंशत्सहस्राणि मृत्युनेन्द्राय कीर्तितम् \thinspace{\dandab} \dontdisplaylinenum  }%
     \var{{\devanagarifontvar\numemph\va ॰विंशत्॰\lem \Ed,\hskip.2em plus .9em ॰विंश॰ \mssCaCbCc\msNa\msNc\msM\msKOb\msPaperA}}% 
    \var{{\devanagarifontvar\numnoemph\vb कीर्तितम्\lem \Ed,\hskip.2em plus .9em कीर्तितः \msCa\msCb\msNa\msNcpcorr\msM\msKOb,\hskip.5em plus .9em कीर्तिताः \msCc,\hskip.5em plus .9em 
कीर्त्तित \msNcacorr,\hskip.5em plus .9em कीर्तितंः \msPaperA}}% 

%Verse 1:65

{\devanagarifont इन्द्रेणाह वसिष्ठाय विंशत्श्लोकसहस्रिकम् {॥ १:\hspace{.11em}६५॥} \veg\dontdisplaylinenum }%
     \var{{\devanagarifontvar\numnoemph\vc इन्द्रे॰\lem \mssALL,\hskip.2em plus .9em इन्दे॰ \msPaperA}}% 
    \var{{\devanagarifontvar\numnoemph\vc वसिष्ठाय\lem \mssALL,\hskip.2em plus .9em विशिष्ठाय \msCb,\hskip.5em plus .9em वहिष्ठाय \msNc}}% 
    \var{{\devanagarifontvar\numnoemph\vd विंशत्श्लो॰\lem \corr,\hskip.2em plus .9em विंशश्लो॰ \msCa\msCc\msNa\msNc\msKOb\msPaperA\Ed,\hskip.5em plus .9em विशश्लो॰ \msCb,\hskip.5em plus .9em त्रिंशश्लो॰ \msM}}% 

{\devanagarifont अष्टादशसहस्राणि तेन सारस्वताय तु \thinspace{\dandab} \dontdisplaylinenum }%
     \var{{\devanagarifontvar\numemph\va अष्टादशसहस्राणि\lem \mssALL,\hskip.2em plus .9em 
आष्टादशसहस्राणि \msNc,\hskip.5em plus .9em वसिष्ठेदशसहस्रं \msM}}% 

%Verse 1:66

{\devanagarifont सारस्वतस्त्रिधामाय सहस्रदश सप्त च {॥ १:\hspace{.11em}६६॥} \veg\dontdisplaylinenum }%
     \var{{\devanagarifontvar\numnoemph\vc सारस्वतस्त्रि॰\lem \eme,\hskip.2em plus .9em सारस्वता त्रि॰ \msCa\msCc\msNa\msNc\msKOb\msPaperA\Ed,\hskip.5em plus .9em सारस्वतास्त्रि॰ \msCb,\hskip.5em plus .9em 
सारस्वत तृ॰ \msM\oo 
 ॰धामाय\lem \mssALL,\hskip.2em plus .9em \om\ \msNaacorr}}% 
    \var{{\devanagarifontvar\numnoemph\vd सहस्रदश\lem \mssALL,\hskip.2em plus .9em सहस्रादश \msM}}% 

{\devanagarifont षोडशानां सहस्राणि भरद्वाजाय वै ततः \thinspace{\dandab} \dontdisplaylinenum }%
     \var{{\devanagarifontvar\numemph\vb भर॰\lem \mssALL,\hskip.2em plus .9em भार॰ \msCc,\hskip.5em plus .9em सन॰ \msM}}% 

%Verse 1:67

{\devanagarifont दश पञ्चसहस्राणि त्रिवृषाय अभाषत {॥ १:\hspace{.11em}६७॥} \veg\dontdisplaylinenum }%
     \var{{\devanagarifontvar\numnoemph\vd अभाषत\lem \msCa\msCb\msNa\msKOb\msPaperA,\hskip.2em plus .9em अ\uncl{भाषत} \msCc,\hskip.5em plus .9em अभाषतः \msNc\Ed,\hskip.5em plus .9em मभासतः \msM}}% 

{\devanagarifont चतुर्दशसहस्राणि अन्तरीक्षाय वै ततः \thinspace{\dandab} \dontdisplaylinenum }%
     \var{{\devanagarifontvar\numemph\vb अन्तरी॰\lem \mssALL,\hskip.2em plus .9em अन्तरि॰ \msM}}% 

%Verse 1:68

{\devanagarifont त्रय्यारुणिं सहस्राणि त्रयोदश अभाषत {॥ १:\hspace{.11em}६८॥} \veg\dontdisplaylinenum }%
     \var{{\devanagarifontvar\numnoemph\vc त्रय्यारुणिं\lem \corr,\hskip.2em plus .9em त्र्यैयारुणि \msCa\msCb\msNa\msM\msKOb\msPaperA,\hskip.5em plus .9em त्रैयारुणि \msCc\Ed,\hskip.5em plus .9em 
त्र्यैयारूपिनि \msNc}}% 
    \var{{\devanagarifontvar\numnoemph\vd अभाषत\lem \msCa\msCc\msNc\msKOb\msPaperA,\hskip.2em plus .9em अभाषतः \msCb,\hskip.5em plus .9em स्वभावत \msNa,\hskip.5em plus .9em मभासतः \msM,\hskip.5em plus .9em 
ह्यभाषत \Ed}}% 

{\devanagarifont त्रय्यारुणिस्तु विप्रेन्द्रो धनंजयमभाषत \thinspace{\dandab} \dontdisplaylinenum }%
     \var{{\devanagarifontvar\numemph\va त्रय्यारुणि॰\lem \corr,\hskip.2em plus .9em त्र्यैयारुणि॰ \mssCaCbCc\msNc\msKOb\msPaperA,\hskip.5em plus .9em त्रैयारुणि॰ \msNa\Ed,\hskip.5em plus .9em त्र्यैर्यारुणि॰ \msM\oo 
 विप्रेन्द्रो\lem \mssALL,\hskip.2em plus .9em विप्रेन्द \msCc\msM}}% 
    \var{{\devanagarifontvar\numnoemph\vb धनंजय॰\lem \mssALL,\hskip.2em plus .9em धन॰ \msNaacorr\oo 
 ॰भाषत\lem \msCa\msCc\msNa\msNc\msKOb\msPaperA,\hskip.2em plus .9em ॰भाषतः \msCb\msM\Ed}}% 

%Verse 1:69

{\devanagarifont द्वादशानि सहस्राणि संक्षिप्य पुनरब्रवीत् {॥ १:\hspace{.11em}६९॥} \veg\dontdisplaylinenum }%
 
{\devanagarifont कृतंजयाय सम्प्राप्तो धनंजयमहामुनिः \thinspace{\dandab} \dontdisplaylinenum }%
     \var{{\devanagarifontvar\numemph\vb ॰मुनिः\lem \mssALL,\hskip.2em plus .9em ॰मुणि \msM}}% 

%Verse 1:70

{\devanagarifont कृतंजयाद्द्विजश्रेष्ठ ऋणंजयमहात्मने {॥ १:\hspace{.11em}७०॥} \veg\dontdisplaylinenum }%
     \var{{\devanagarifontvar\numnoemph\vc कृतंजयाद्द्वि॰\lem \msCa\msNa\msKOb\msPaperA\Ed,\hskip.2em plus .9em कृतंजया द्वि॰ \msCb\msCc\msNc,\hskip.5em plus .9em धनञ्जय द्वि॰ \msM\oo 
 ॰श्रेष्ठ\lem \mssALL,\hskip.2em plus .9em ॰श्रेष्ठो \Ed}}% 
    \var{{\devanagarifontvar\numnoemph\vd ऋणंजय॰\lem \mssALL,\hskip.2em plus .9em ऋणंजाय॰ \msCb,\hskip.5em plus .9em रणञ्जय॰ \msKOb\oo 
 ॰महात्मने\lem \mssALL,\hskip.2em plus .9em ॰मभाशतः \msM}}% 

{\devanagarifont ऋणञ्जयात्पुनः प्राप्तो गौतमाय महर्षिणे \thinspace{\dandab} \dontdisplaylinenum }%
     \var{{\devanagarifontvar\numemph\va प्राप्तो\lem \mssALL,\hskip.2em plus .9em प्राप्तः \msM,\hskip.5em plus .9em प्राप्तौ \Ed}}% 
    \var{{\devanagarifontvar\numnoemph\vb गौतमाय\lem \mssALL,\hskip.2em plus .9em गोतमाय \msKOb\oo 
 महर्षिणे\lem \mssALL,\hskip.2em plus .9em महर्षिणः \msM}}% 

%Verse 1:71

{\devanagarifont गौतमाच्च भरद्वाजस्तस्माद्धर्यद्वताय तु {॥ १:\hspace{.11em}७१॥} \veg\dontdisplaylinenum }%
     \var{{\devanagarifontvar\numnoemph\vc गौतमाच्च\lem \mssCaCbCc\msNa\Ed,\hskip.2em plus .9em गौतमाश्च \msNc\msPaperA,\hskip.5em plus .9em गौतमेन \msM,\hskip.5em plus .9em गोतमाच्च \msKOb}}% 
    \var{{\devanagarifontvar\numnoemph\vcd भरद्वाजस्तस्माद्धर्यद्वताय\lem \msCa\msCc\msNa\msNc\msKOb,\hskip.2em plus .9em 
भरद्वारस्तस्माद्धर्यद्वताय \msCb,\hskip.5em plus .9em 
भरद्वाज तस्मा हर्यद्वताय \msM,\hskip.5em plus .9em 
भरद्वाजस्तस्माद्धर्यद्वनाय \msPaperA,\hskip.5em plus .9em 
भरद्वाजस्तस्माद्दम्याद्दमाय \Ed}}% 

{\devanagarifont राजश्रवास्ततः प्राप्तः सोमशुष्माय वै ततः \thinspace{\dandab} \dontdisplaylinenum }%
     \var{{\devanagarifontvar\numemph\va राजश्रवास्त॰\lem \eme,\hskip.2em plus .9em राजश्रव त॰ \mssCaCbCc\msNa\msKOb\msPaperA\Ed,\hskip.5em plus .9em राजश्रवे त॰ \msNc,\hskip.5em plus .9em 
राजर्षव त॰ \msM}}% 
    \var{{\devanagarifontvar\numnoemph\vab प्राप्तः सोम॰\lem \mssALL,\hskip.2em plus .9em प्राप्त साम॰ \msPaperA}}% 

%Verse 1:72

{\devanagarifont सोमशुष्मात्ततः प्राप्तस्तृणबिन्दुस्तु भो द्विज {॥ १:\hspace{.11em}७२॥} \veg\dontdisplaylinenum }%
     \var{{\devanagarifontvar\numnoemph\vc ॰शुष्मात्त॰\lem \mssALL,\hskip.2em plus .9em ॰शुष्मा त॰ \msNa}}% 
    \var{{\devanagarifontvar\numnoemph\vcd प्राप्तस्तृणबिन्दुस्तु\lem \mssALL,\hskip.2em plus .9em 
प्रा\uncl{प्त तृ}णबिन्दुस्तु \msCc,\hskip.5em plus .9em 
प्राप्तस्तृणविन्दुन्तु \msPaperA}}% 
    \var{{\devanagarifontvar\numnoemph\vd भो\lem \mssALL,\hskip.2em plus .9em \om\ \msCb}}% 

{\devanagarifont तृणबिन्दुस्तु वृक्षाय वृक्षः शक्तिमभाषत \thinspace{\dandab} \dontdisplaylinenum }%
     \var{{\devanagarifontvar\numemph\vb वृक्षः\lem \mssALL,\hskip.2em plus .9em वृक्ष \msM\msKOb\oo 
 ॰भाषत\lem \msCa\msCb\msNa\msNc\msKOb\msPaperA,\hskip.2em plus .9em ॰भाषतः \msCc\msM\Ed}}% 

%Verse 1:73

{\devanagarifont शक्तिः पराशरं प्राह जतुकर्णाय वै ततः {॥ १:\hspace{.11em}७३॥} \veg\dontdisplaylinenum }%
     \var{{\devanagarifontvar\numnoemph\vc शक्तिः पराशरं\lem \mssALL,\hskip.2em plus .9em 
शपरासर \msMacorr,\hskip.5em plus .9em शक्ति परासर \msMpcorr}}% 
    \var{{\devanagarifontvar\numnoemph\vd जतुकर्णाय\lem \msCa\msCc\msNa\msNc\msPaperA\Ed,\hskip.2em plus .9em तुकर्ण्णाय \msCb,\hskip.5em plus .9em जंतुकर्ण्णाय \msM,\hskip.5em plus .9em 
जतुवर्ण्णाय \msKOb}}% 

{\devanagarifont द्वैपायनं तु प्रोवाच जतुकर्णो महर्षिणम् \thinspace{\dandab} \dontdisplaylinenum }%
     \var{{\devanagarifontvar\numemph\va द्वैपायनं तु\lem \eme,\hskip.2em plus .9em द्वैपायनस्तु \mssCaCbCc\msNa\msNc\msM\msKOb\msPaperA,\hskip.5em plus .9em 
द्वैपायनाय \Ed\ \unmetr}}% 
    \var{{\devanagarifontvar\numnoemph\vb जतुकर्णो महर्षिणम्\lem \msCa\msCb\msNapcorr\msNc,\hskip.2em plus .9em जतुकर्णा महर्षिणः \msCc,\hskip.5em plus .9em 
जकर्णो महर्षिणं \msNaacorr,\hskip.5em plus .9em जंतुकर्ण्णमहर्षिणा \msM,\hskip.5em plus .9em जतुवर्ण्णो महर्षिणम् \msKOb,\hskip.5em plus .9em 
जतुकर्णा महषिण \msPaperA,\hskip.5em plus .9em जतुकर्णमहर्षिणा \Ed}}% 

%Verse 1:74

{\devanagarifont रोमहर्षाय सम्प्राप्तो द्वैपायनमहामुनिः {॥ १:\hspace{.11em}७४॥} \veg\dontdisplaylinenum }%
     \var{{\devanagarifontvar\numnoemph\vd ॰मुनिः\lem \mssALL,\hskip.2em plus .9em ॰मुनि \msM\Ed}}% 

{\devanagarifont रोमहर्षेण प्रोवाच पुत्रायामितबुद्धये \thinspace{\dandab} \dontdisplaylinenum }%
     \var{{\devanagarifontvar\numemph\va ॰हर्षेण\lem \msM,\hskip.2em plus .9em ॰हर्षाय \mssCaCbCc\msNa\msNc\msKOb\msPaperA,\hskip.5em plus .9em ॰हर्षणाय \Ed}}% 
    \var{{\devanagarifontvar\numnoemph\vb ॰बुद्धये\lem \mssALL,\hskip.2em plus .9em ॰बुद्धयः \msM}}% 
    \paral{{\devanagarifontsmall \vab \similar\ {\englishfont \BRAHMANDAPUR\ 3.4.67ab:}
                 मया चैतत्पुनः प्रोक्तं पुत्रायामितबुद्धये }}

{\devanagarifont दश द्वे च सहस्राणि पुराणं सम्प्रकाशितम्  \danda\dontdisplaylinenum }%
     \var{{\devanagarifontvar\numnoemph\vd पुराणं सम्प्रकाशितम्\lem \mssALL,\hskip.2em plus .9em 
पुराण सम्प्रकाशितां \msCc}}% 

%Verse 1:75

{\devanagarifont मानुषाणां हितार्थाय किं भूयः श्रोतुमिच्छसि {॥ १:\hspace{.11em}७५॥} \veg\dontdisplaylinenum }%
     \var{{\devanagarifontvar\numnoemph\ve मानुषाणां\lem \mssALL,\hskip.2em plus .9em मनुषाणां \msCb,\hskip.5em plus .9em मानुषाना \msM\oo 
 हितार्थाय\lem \mssALL,\hskip.2em plus .9em हित्यथाय \msM,\hskip.5em plus .9em हिताथयि \msPaperA}}% 
    \var{{\devanagarifontvar\numnoemph\vf भूयः\lem \mssALL,\hskip.2em plus .9em भूय \msM\Ed}}% 

{\devanagarifont 
\jump
\begin{center}
\ketdanda~इति वृषसारसंग्रहे ब्रह्माण्डसंख्या नामाध्यायः प्रथमः~\ketdanda
\end{center}
\dontdisplaylinenum\vers  }%
     \var{{\devanagarifontvar\numnoemph{\englishfont \Colo:} वृषसार॰\lem \mssALL,\hskip.2em plus .9em वृषार॰ \msKOb\oo 
 नामाध्यायः प्रथमः\lem \mssALL,\hskip.2em plus .9em नामाध्यायः प्रथमः श्लोक ७७ \msM,\hskip.5em plus .9em 
नाम प्रथमो ऽध्याय \Ed}}% 
\bekveg\szamveg
\vfill
\phpspagebreak

\versno=0\fejno=2
\thispagestyle{empty}

\centerline{\Large\devanagarifontbold [   द्वितीयो ऽध्यायः  ]}{\vrule depth10pt width0pt} \fancyhead[CE]{{\footnotesize\devanagarifont वृषसारसंग्रहे  }}
\fancyhead[CO]{{\footnotesize\devanagarifont द्वितीयो ऽध्यायः  }}
\fancyhead[LE]{}
\fancyhead[RE]{}
\fancyhead[LO]{}
\fancyhead[RO]{}
\szam\bek


\vers


{\devanagarifont विगतराग उवाच {\dandab}\dontdisplaylinenum  }%
 
{\devanagarifont श्रुतं मया जनाग्रेण ब्रह्माण्डस्य तु निर्णयम् \thinspace{\danda} \dontdisplaylinenum }%
     \var{{\devanagarifontvar\numemph\va जनाग्रेण\lem \mssALL,\hskip.2em plus .9em जना\lacwithnum{2}  \msCa}}% 

%Verse 2:1

{\devanagarifont प्रमाणं वर्णरूपं च संख्या तस्य समासतः {॥ २:\hspace{.11em}१॥} \veg\dontdisplaylinenum }%
     \lacuna{\devanagarifontsmall {\englishfont Witnesses used for this chapter: \msCa\ ff.\thinspace 195v--197r, 
                                             \msCb\ ff.\thinspace 203v--204v,
                                             \msCc\ ff.\allowbreak\thinspace 270r--270v 
                                                (it breaks off at 2.21 and resumes at 3.30b),
                                             \msNa\ ff.\thinspace 3v--4v, 
                                             \msNb\ exp.\thinspace 43 and 42 
                                                (sic!; it broke off at 1.60d and resumes at 2.23),
                                             \msNc\ ff.\thinspace 211v--213r,
                                             \msParis\ f.\thinspace 215 (only from 2.19cd),
                                             \msKOb\ ff.\thinspace 212v--213v,
                                             \Ed\ pp.\thinspace 585--588;
                                             \mssCaCbCc\ = \msCa + \msCb + \msCc } }%
  
{\devanagarifont शिवाण्डेति त्वया प्रोक्तो ब्रह्माण्डालयकीर्तितः \thinspace{\dandab} \dontdisplaylinenum }%
     \var{{\devanagarifontvar\numemph\vb ब्रह्माण्डा॰\lem \mssALL,\hskip.2em plus .9em ब्रह्माण्ड \Ed}}% 

%Verse 2:2

{\devanagarifont कीदृशं लक्षणं ज्ञेयं प्रमाणं तस्य वा कति {॥ २:\hspace{.11em}२॥} \veg\dontdisplaylinenum }%
     \var{{\devanagarifontvar\numnoemph\vc ज्ञेयं\lem \mssALL,\hskip.2em plus .9em ज्ञेया \msCc}}% 
    \var{{\devanagarifontvar\numnoemph\vd कति\lem \mssALL,\hskip.2em plus .9em कतिः \msCc}}% 

{\devanagarifont कस्य वा लयनं ज्ञेयं प्रमाणं वात्र वासिनः \thinspace{\dandab} \dontdisplaylinenum }%
     \var{{\devanagarifontvar\numemph\va लयनं ज्ञेयं\lem \mssALL,\hskip.2em plus .9em लयनं \msCb,\hskip.5em plus .9em लक्षणं ज्ञेयं \Ed}}% 
    \var{{\devanagarifontvar\numnoemph\vb वासिनः\lem \mssALL,\hskip.2em plus .9em वासिरानः \msCb}}% 

%Verse 2:3

{\devanagarifont का वा तत्र प्रजा ज्ञेया को वा तत्र प्रजापतिः {॥ २:\hspace{.11em}३॥} \veg\dontdisplaylinenum }%
     \var{{\devanagarifontvar\numnoemph\vc का\lem \eme,\hskip.2em plus .9em को \mssCaCbCc\msNa\msNc\msKOb,\hskip.5em plus .9em किं \Ed\oo 
 प्रजा ज्ञेया\lem \mssALL,\hskip.2em plus .9em प्र\uncl{जा}\lacwithnum{1}  या \msCa}}% 


\alalfejezet{शिवाण्डसंख्या}
{\devanagarifont अनर्थयज्ञ उवाच {\dandab}\dontdisplaylinenum  }%
 
{\devanagarifont शिवाण्डलक्षणं विप्र न त्वं प्रष्टुमिहार्हसि \thinspace{\danda} \dontdisplaylinenum }%
     \var{{\devanagarifontvar\numemph\va विप्र\lem \mssALL,\hskip.2em plus .9em विप्रं \msKOb}}% 
    \var{{\devanagarifontvar\numnoemph\vb न त्वं\lem \mssALL,\hskip.2em plus .9em तत्वं \Ed\oo 
 ॰र्हसि\lem \mssALL,\hskip.2em plus .9em ॰हसि \msNc}}% 

%Verse 2:4

{\devanagarifont दैवतैरपि का शक्तिर्ज्ञातुं द्रष्टुं च तत्त्वतः {॥ २:\hspace{.11em}४॥} \veg\dontdisplaylinenum }%
     \var{{\devanagarifontvar\numnoemph\vc दैवतै॰\lem \msCa\msCb\msNa\msKOb,\hskip.2em plus .9em देवतै॰ \msCc\msNc\Ed\oo 
 शक्तिर्\lem \msCa,\hskip.2em plus .9em शक्ति \msCb\msCc\msNa\msNc\msKOb\Ed}}% 

\pend
\endnumbering
\vfill\pagebreak\beginnumbering\pstart
\vers

{\devanagarifont अगम्यगमनं गुह्यं गुह्यादपि समुद्धितम् \thinspace{\dandab} \dontdisplaylinenum }%
     \var{{\devanagarifontvar\numemph\va अगम्यगमनं\lem \mssALL,\hskip.2em plus .9em अगम्यगगहनं \msCc,\hskip.5em plus .9em अगम्यगगमनं \msNc}}% 
    \var{{\devanagarifontvar\numnoemph\vb गुह्या॰\lem \msNc\Ed,\hskip.2em plus .9em गुहा॰ \mssCaCbCc\msNa\msKOb\oo 
 समुद्धितं\lem \mssALL,\hskip.2em plus .9em सम्रद्धितं \msNc,\hskip.5em plus .9em समृद्धिदम् \Ed}}% 
    \paral{{\devanagarifontsmall \vab {\englishfont \compare\ \LINPU\ 1.21.71ab:} नमो गुण्याय गुह्याय अगम्यगमनाय च }}

%Verse 2:5

{\devanagarifont न प्रभुर्नेतरस्तत्र न दण्ड्यो न च दण्डकः {॥ २:\hspace{.11em}५॥} \veg\dontdisplaylinenum }%
     \var{{\devanagarifontvar\numnoemph\vc प्रभुर्ने॰\lem \mssALL,\hskip.2em plus .9em प्रने॰ \msCc}}% 
    \var{{\devanagarifontvar\numnoemph\vd दण्ड्यो\lem \msCc\msNa\msNc\msKOb,\hskip.2em plus .9em दण्डो \msCa\msCb,\hskip.5em plus .9em दण्ड्या \Ed\oo 
 दण्डकः\lem \mssALL,\hskip.2em plus .9em ण्डकः \msCbacorr,\hskip.5em plus .9em पण्डकः \msCbpcorr}}% 

{\devanagarifont न सत्यो नानृतस्तत्र सुशीलो नो दुःशीलवान् \thinspace{\dandab} \dontdisplaylinenum }%
     \var{{\devanagarifontvar\numemph\va सत्यो\lem \mssALL,\hskip.2em plus .9em सत्यौ \Ed\oo 
 तत्र\lem \mssALL,\hskip.2em plus .9em तत्रा \Ed}}% 
    \var{{\devanagarifontvar\numnoemph\vb नो\lem \mssALL,\hskip.2em plus .9em \lacwithnum{1}  \msCa}}% 

%Verse 2:6

{\devanagarifont नानृजुर्न च दम्भित्वं न तृष्णा न च ईर्ष्यता {॥ २:\hspace{.11em}६॥} \veg\dontdisplaylinenum }%
     \var{{\devanagarifontvar\numnoemph\vc नानृजुर्न\lem \eme,\hskip.2em plus .9em नाऋजुर्न्न \msCa\msKOb\Ed,\hskip.5em plus .9em नाऋजुर्न \msCb\msNc,\hskip.5em plus .9em 
\uncl{नाऋजु न} \msCc,\hskip.5em plus .9em नाऋजुन्न \msNa}}% 
    \var{{\devanagarifontvar\numnoemph\vd न तृष्णा न च\lem \mssALL,\hskip.2em plus .9em  न च तृष्णा न \msNa\oo 
 ईर्ष्यता\lem \mssALL,\hskip.2em plus .9em ईर्ष्यताः \msCc,\hskip.5em plus .9em इर्ष्यता \Ed}}% 

{\devanagarifont न क्रोधो न च लोभो ऽस्ति न मानो ऽस्ति न सूयकः \thinspace{\dandab} \dontdisplaylinenum }%
     \var{{\devanagarifontvar\numemph\va क्रोधो\lem \mssALL,\hskip.2em plus .9em क्रोधौ \msCc}}% 
    \var{{\devanagarifontvar\numnoemph\vb सूयकः\lem \mssALL,\hskip.2em plus .9em सूचकः \msCb,\hskip.5em plus .9em स्तेयकः \Ed\ \unmetr}}% 

%Verse 2:7

{\devanagarifont ईर्ष्या द्वेषो न तत्रास्ति न शठो न च मत्सरः {॥ २:\hspace{.11em}७॥} \veg\dontdisplaylinenum }%
     \var{{\devanagarifontvar\numnoemph\vd शठो\lem \mssALL,\hskip.2em plus .9em षठो \msCc,\hskip.5em plus .9em शठे \Ed\oo 
 मत्सरः\lem \mssALL,\hskip.2em plus .9em मत्सराः \Ed}}% 

{\devanagarifont न व्याधिर्न जरा तत्र न शोको ऽस्ति न विक्लवः \thinspace{\dandab} \dontdisplaylinenum }%
     \var{{\devanagarifontvar\numemph\va व्याधिर्न\lem \mssALL,\hskip.2em plus .9em व्याधि न \msCc\msNc\oo 
 जरा तत्र\lem \msCb\msNc,\hskip.2em plus .9em जरास्तत्र \msCa\msCc\msNa\msKOb\Ed}}% 
    \var{{\devanagarifontvar\numnoemph\vb विक्लवः\lem \mssALL,\hskip.2em plus .9em विक्लव \Ed}}% 

%Verse 2:8

{\devanagarifont नाधमः पुरुषस्तत्र नोत्तमो न च मध्यमः {॥ २:\hspace{.11em}८॥} \veg\dontdisplaylinenum }%
 
{\devanagarifont नोत्कृष्टो मानवस्तस्मिन्स्त्रियश्चैव शिवालये \thinspace{\dandab} \dontdisplaylinenum }%
     \var{{\devanagarifontvar\numemph\va मानव॰\lem \mssALL,\hskip.2em plus .9em मा\lacwithnum{1}  व॰ \msCa}}% 

%Verse 2:9

{\devanagarifont न निन्दा न प्रशंसास्ति मत्सरी पिशुनो न च {॥ २:\hspace{.11em}९॥} \veg\dontdisplaylinenum }%
     \var{{\devanagarifontvar\numnoemph\vc प्रशंसास्ति\lem \mssALL,\hskip.2em plus .9em प्रंसास्ति \msKOb,\hskip.5em plus .9em प्रशंसाश्च \Ed}}% 

{\devanagarifont गर्वदर्पं न तत्रास्ति क्रूरमायादिकं तथा \thinspace{\dandab} \dontdisplaylinenum }%
 
%Verse 2:10

{\devanagarifont याचमानो न तत्रास्ति दाता चैव न विद्यते {॥ २:\hspace{.11em}१०॥} \veg\dontdisplaylinenum }%
     \var{{\devanagarifontvar\numemph\vc तत्रास्ति\lem \mssALL,\hskip.2em plus .9em तत्रा \msNaacorr}}% 

\pend
\endnumbering
\vfill\pagebreak\beginnumbering\pstart
\vers

{\devanagarifont अनर्थी व्रज तत्रस्थः कल्पवृक्षसमाश्रितः \thinspace{\dandab} \dontdisplaylinenum }%
     \var{{\devanagarifontvar\numemph\va व्रज त॰\lem \mssALL,\hskip.2em plus .9em व्रजस्त॰ \msNc}}% 

%Verse 2:11

{\devanagarifont न कर्म नाप्रियस्तत्र न कलिः कलहो न च {॥ २:\hspace{.11em}११॥} \veg\dontdisplaylinenum }%
     \var{{\devanagarifontvar\numnoemph\vc कर्म ना॰\lem \eme,\hskip.2em plus .9em कर्म न \mssCaCbCc\msNa\msNc\msKOb,\hskip.5em plus .9em कर्मणा \Ed}}% 
    \var{{\devanagarifontvar\numnoemph\vd कलिः\lem \mssALL,\hskip.2em plus .9em कलि \msNcacorr\Ed}}% 

{\devanagarifont द्वापरो न च न त्रेता कृतं चापि न विद्यते \thinspace{\dandab} \dontdisplaylinenum }%
     \var{{\devanagarifontvar\numemph\va च न त्रेता\lem \msCc\msNa\msNc\Ed,\hskip.2em plus .9em च न त्रेत्रा \msCa,\hskip.5em plus .9em च त्रेता न \msCb,\hskip.5em plus .9em 
च तत्रेता \msKOb}}% 
    \var{{\devanagarifontvar\numnoemph\vb कृतं चा॰\lem \msCc\msNa,\hskip.2em plus .9em कृतश्चा॰ \msCa\msCb\msNc\msKOb\Ed}}% 

%Verse 2:12

{\devanagarifont मन्वन्तरं न तत्रास्ति कल्पश्चैव न विद्यते {॥ २:\hspace{.11em}१२॥} \veg\dontdisplaylinenum }%
     \var{{\devanagarifontvar\numnoemph\vc मन्वन्तरं न तत्रास्ति\lem \mssALL,\hskip.2em plus .9em मन्वन्तत्रास्ति \msCc,\hskip.5em plus .9em 
मन्वन्तरनन्त तत्रास्ति \msNc}}% 
    \var{{\devanagarifontvar\numnoemph\vd कल्पश्चैव\lem \mssALL,\hskip.2em plus .9em कल्पं चैव \msNa}}% 

{\devanagarifont आहूतसम्प्लवं नास्ति ब्रह्मरात्रिदिनं तथा \thinspace{\dandab} \dontdisplaylinenum }%
     \var{{\devanagarifontvar\numemph\va आहूत॰\lem \mssALL,\hskip.2em plus .9em आभूत॰ \Ed}}% 
    \var{{\devanagarifontvar\numnoemph\vb ब्रह्मरात्रिदिनं\lem \mssALL,\hskip.2em plus .9em ब्रह्मरात्रिदिवस् \Ed}}% 

%Verse 2:13

{\devanagarifont न जन्ममरणं तत्र आपदं नाप्नुयात्क्वचित् {॥ २:\hspace{.11em}१३॥} \veg\dontdisplaylinenum }%
     \var{{\devanagarifontvar\numnoemph\vc जन्ममरणं तत्र\lem \msCc\msNa\msKOb\Ed,\hskip.2em plus .9em जन्मरणं तत्र \msCa\msCb,\hskip.5em plus .9em 
जन्ममरणन्त्रत \msNc}}% 
    \var{{\devanagarifontvar\numnoemph\vd आपदं\lem \mssALL,\hskip.2em plus .9em अपदं \Ed}}% 

{\devanagarifont न चाशापाशबद्धो ऽस्ति रागमोहं न विद्यते \thinspace{\dandab} \dontdisplaylinenum }%
     \var{{\devanagarifontvar\numemph\va चाशापाश॰\lem \msCb\msNcpcorr,\hskip.2em plus .9em च सायाश॰ \msCa\msCc\msNa\msNcacorr\msKOb\Ed\oo 
 ॰बद्धो\lem \mssALL,\hskip.2em plus .9em ॰द्धो \msCc,\hskip.5em plus .9em ॰वृद्धो \Ed}}% 
    \var{{\devanagarifontvar\numnoemph\vb ॰मोहं\lem \mssALL,\hskip.2em plus .9em ॰मोहो \msCa}}% 

%Verse 2:14

{\devanagarifont न देवा नासुरास्तत्र न यक्षोरगराक्षसाः {॥ २:\hspace{.11em}१४॥} \veg\dontdisplaylinenum }%
     \var{{\devanagarifontvar\numnoemph\vc देवा नासुरास्त॰\lem \mssALL,\hskip.2em plus .9em देवो नासुरास्त॰ \msCb,\hskip.5em plus .9em 
देवो नासुरस्त॰ \msKOb}}% 

{\devanagarifont न भूता न पिशाचाश्च गन्धर्वा ऋषयस्तथा \thinspace{\dandab} \dontdisplaylinenum }%
     \var{{\devanagarifontvar\numemph\va भूता\lem \mssALL,\hskip.2em plus .9em च भूता \msKObacorr}}% 
    \var{{\devanagarifontvar\numnoemph\vb गन्धर्वा\lem \mssALL,\hskip.2em plus .9em गन्धर्वो \Ed}}% 

%Verse 2:15

{\devanagarifont ताराग्रहं न तत्रास्ति नागकिंनरगारुडम् {॥ २:\hspace{.11em}१५॥} \veg\dontdisplaylinenum }%
 
{\devanagarifont न जपो नाह्निकस्तत्र नाग्निहोत्री न यज्ञकृत् \thinspace{\dandab} \dontdisplaylinenum }%
     \var{{\devanagarifontvar\numemph\va जपो\lem \mssALL,\hskip.2em plus .9em जयो \msCa\oo 
 नाह्निकस्त॰\lem \mssALL,\hskip.2em plus .9em नाह्निक त॰ \msCb}}% 

%Verse 2:16

{\devanagarifont न व्रतं न तपश्चैव न तिर्यङ्नरकं तथा {॥ २:\hspace{.11em}१६॥} \veg\dontdisplaylinenum }%
     \var{{\devanagarifontvar\numnoemph\vd न तिर्यङ्नरकं\lem \eme,\hskip.2em plus .9em नातिर्यन्नरकस् \msCa\msCc\msNa,\hskip.5em plus .9em 
नातिर्यनरकन् \msCb,\hskip.5em plus .9em नात्रिर्यं नरकस् \msNc,\hskip.5em plus .9em 
नातिर्यन्नरकन् \msKOb,\hskip.5em plus .9em न तीर्थन्नरकन् \Ed}}% 
    \paral{{\devanagarifontsmall \vd {\englishfont \compare\ \VSS\ 19.49cd:} विसृष्टे त्विन्द्रियग्रामे तिर्यङ्नरकसाधनम् }}

\pend
\endnumbering
\vfill\pagebreak\beginnumbering\pstart
\vers

{\devanagarifont तस्येशानस्य देवस्य ऐश्वर्यगुणविस्तरम् \thinspace{\dandab} \dontdisplaylinenum }%
     \paral{{\devanagarifontsmall \vb {\englishfont \compare\ \MBH\ Suppl. 14.4.2743:} ऐश्वर्यगुणसंपन्नाः क्रीडन्ति च यथासुखम्, 
                       {\englishfont and \BRAHMANDAPUR\ 1.26.1:} महादेवस्य महात्म्यं प्रभुत्वं च महात्मनः\thinspace{\devanagarifontsmall ।}  
                                             श्रोतुमिच्छामहे सम्यगैश्वर्यगुणविस्तरम्\thinspace{\devanagarifontsmall ॥} }}

%Verse 2:17

{\devanagarifont अपि वर्षशतेनापि शक्यं वक्तुं न केनचित् {॥ २:\hspace{.11em}१७॥} \veg\dontdisplaylinenum }%
 
{\devanagarifont हरेच्छाप्रभवाः सर्वे पर्यायेण ब्रवीमि ते \thinspace{\dandab} \dontdisplaylinenum }%
     \var{{\devanagarifontvar\numemph\va हरेच्छाप्रभवाः\lem \msNc,\hskip.2em plus .9em हरेच्छप्रभवाः \mssCaCbCc\msNa\msKOb,\hskip.5em plus .9em हरेच्छाप्रभवा \Ed}}% 

%Verse 2:18

{\devanagarifont देवमानुषवर्ज्यानि वृक्षगुल्मलतादयः {॥ २:\hspace{.11em}१८॥} \veg\dontdisplaylinenum }%
     \var{{\devanagarifontvar\numnoemph\vc वर्ज्यानि\lem \mssALL,\hskip.2em plus .9em वज्ज्ञानि \Ed}}% 

{\devanagarifont परार्धद्विगुणोत्सेधो विस्तारश्च तथाविधः \thinspace{\dandab} \dontdisplaylinenum }%
     \var{{\devanagarifontvar\numemph\va ॰गुणोत्सेधो\lem \conj,\hskip.2em plus .9em ॰गुणोच्छेधा \msCa\msCb\msNa\msNc\msKOb,\hskip.5em plus .9em 
॰गुणेच्छेधा \msCc,\hskip.5em plus .9em ॰गुणाच्छ्रेधा \Ed}}% 
    \var{{\devanagarifontvar\numnoemph\vb विस्तारश्च\lem \msNc,\hskip.2em plus .9em विस्तारं च \mssCaCbCc\msNa\msKOb\Ed\oo 
 ॰विधः\lem \msNc,\hskip.2em plus .9em ॰विधा \mssCaCbCc\msNa\msKOb\Ed}}% 

%Verse 2:19

{\devanagarifont अनेकाकारपुष्पाणि फलानि च मनोहरम् {॥ २:\hspace{.11em}१९॥} \veg\dontdisplaylinenum }%
     \var{{\devanagarifontvar\numnoemph\vc अनेकाकार॰\lem \mssALL,\hskip.2em plus .9em अनेकार॰ \msCa,\hskip.5em plus .9em काकार॰ \msParis}}% 
    \lacuna{\devanagarifontsmall \vcd {\englishfont \msParis\ resumes here with verses 2.19cd--21ab written
                 in the top margin in red and in a different hand.} }%
  
{\devanagarifont अन्ये काञ्चनवृक्षाणि मणिवृक्षाण्यथापरे \thinspace{\dandab} \dontdisplaylinenum }%
     \var{{\devanagarifontvar\numemph\va अन्ये\lem \mssALL,\hskip.2em plus .9em बहु॰ \Ed}}% 

%Verse 2:20

{\devanagarifont प्रवालमणिषण्डाश्च पद्मरागरुहाणि च {॥ २:\hspace{.11em}२०॥} \veg\dontdisplaylinenum }%
     \var{{\devanagarifontvar\numnoemph\vc षण्डाश्च\lem \mssALL,\hskip.2em plus .9em घण्टाश्च \Ed}}% 
    \var{{\devanagarifontvar\numnoemph\vd ॰रुहाणि\lem \msCc,\hskip.2em plus .9em ॰रुहानि \msCa\msCb\msNa\msNc\msParis\msKOb,\hskip.5em plus .9em ॰सहानि \Ed}}% 

{\devanagarifont स्वादुमूलफलाः स्कन्धलताविटपपादपाः \thinspace{\dandab} \dontdisplaylinenum }%
     \var{{\devanagarifontvar\numemph\va स्वादु॰\lem \mssALL,\hskip.2em plus .9em स्वाधु॰ \msCa\oo 
 ॰मूल॰\lem \mssALL,\hskip.2em plus .9em ॰मूला \msNa\oo 
 ॰फलाः\lem \conj,\hskip.2em plus .9em ॰फला \mssCaCbCc\msNa\msNc\msParis\msKOb\Ed}}% 
    \var{{\devanagarifontvar\numnoemph\vb स्कन्ध॰\lem \conj,\hskip.2em plus .9em स्कन्द॰ \mssCaCbCc\msNa\msNc\Ed,\hskip.5em plus .9em स्क\lk\ \msParis,\hskip.5em plus .9em स्कन्दा॰ \msKOb\oo 
 ॰पाः\lem \mssALL,\hskip.2em plus .9em ॰पा \msParis}}% 

%Verse 2:21

{\devanagarifont कामरूपाश्च ते सर्वे कामदाः कामभाषिणः {॥ २:\hspace{.11em}२१॥} \veg\dontdisplaylinenum }%
     \lacuna{\devanagarifontsmall \vc {\englishfont After }कामरू॰, {\englishfont \msCc\ has two folios missing (ff.\ 271--272) and resumes only at 3.30b} }%
  
{\devanagarifont तत्र विप्र प्रजाः सर्वे अनन्तगुणसागराः \thinspace{\dandab} \dontdisplaylinenum }%
 
%Verse 2:22

{\devanagarifont तुल्यरूपबलाः सर्वे सूर्यायुतसमप्रभाः {॥ २:\hspace{.11em}२२॥} \veg\dontdisplaylinenum }%
     \var{{\devanagarifontvar\numemph\vc ॰बलाः\lem \mssALL,\hskip.2em plus .9em ॰वराः \Ed}}% 

\pend
\endnumbering
\vfill\pagebreak\beginnumbering\pstart
\vers

{\devanagarifont परार्धद्वयविस्तारं परार्धद्वयमायतम् \thinspace{\dandab} \dontdisplaylinenum }%
     \var{{\devanagarifontvar\numemph\vb ॰द्वय॰\lem \mssALL,\hskip.2em plus .9em ॰यद्वय॰ \msParisacorr}}% 

%Verse 2:23

{\devanagarifont परार्धद्वयविक्षेपं योजनानां द्विजोत्तम {॥ २:\hspace{.11em}२३॥} \veg\dontdisplaylinenum }%
     \var{{\devanagarifontvar\numnoemph\vc ॰द्वय॰\lem \mssALL,\hskip.2em plus .9em ॰द्व॰ \msNaacorr,\hskip.5em plus .9em ॰य॰ \msParis\oo 
 विक्षेपं\lem \eme,\hskip.2em plus .9em विक्षेपा \msCa\msCb\msNa\msNb\msNc\msParis\msKOb,\hskip.5em plus .9em विज्ञेया \Ed}}% 
    \var{{\devanagarifontvar\numnoemph\vd ॰त्तम\lem \mssALL,\hskip.2em plus .9em ॰त्तमः \msNa}}% 

{\devanagarifont ऐश्वर्यत्वं न संख्यास्ति बलशक्तिश्च भो द्विज \thinspace{\dandab} \dontdisplaylinenum }%
     \var{{\devanagarifontvar\numemph\vb बलशक्तिश्च भो द्विज\lem \mssALL,\hskip.2em plus .9em 
\om\ \msNaacorr,\hskip.5em plus .9em तव शक्तिश्च भो द्विज \Ed}}% 

%Verse 2:24

{\devanagarifont अधोर्ध्वो न च संख्यास्ति न तिर्यञ्चैति कश्चन {॥ २:\hspace{.11em}२४॥} \veg\dontdisplaylinenum }%
     \var{{\devanagarifontvar\numnoemph\vc अधोर्ध्वो न च संख्यास्ति\lem \mssALL,\hskip.2em plus .9em 
\om\ \msNaacorr}}% 
    \var{{\devanagarifontvar\numnoemph\vd न तिर्यञ्चैति कश्चन\lem \msNapcorr\msNc\msParis\msKOb,\hskip.2em plus .9em 
न तिर्यञ्चेति कश्चन \msCa\msCb\msNb\Ed,\hskip.5em plus .9em 
न तिर्यं चेति कश्चन \msNaacorr}}% 

{\devanagarifont शिवाण्डस्य च विस्तारमायामं च न वेद्म्यहम् \thinspace{\dandab} \dontdisplaylinenum }%
 
%Verse 2:25

{\devanagarifont भोगमक्षय तत्रैव जन्ममृत्युर्न विद्यते {॥ २:\hspace{.11em}२५॥} \veg\dontdisplaylinenum }%
     \var{{\devanagarifontvar\numemph\vc भोगमक्षय त॰\lem \eme,\hskip.2em plus .9em 
भोगमक्षयस्त॰ \msCa\msCb\msNa\msNb\msNc\msParis\msKOb\ \unmetr,\hskip.5em plus .9em 
भोगमयास्तु त॰ \Ed}}% 
    \var{{\devanagarifontvar\numnoemph\vd ॰मृत्युर्न\lem \mssALL,\hskip.2em plus .9em ॰मृत्यु न \msNb}}% 

{\devanagarifont शिवाण्डमध्यमाश्रित्य गोक्षीरसदृशप्रभाः \thinspace{\dandab} \dontdisplaylinenum }%
     \var{{\devanagarifontvar\numemph\vb प्रभाः\lem \mssALL,\hskip.2em plus .9em प्रभा \Ed}}% 

%Verse 2:26

{\devanagarifont परार्धपरकोटीनामीशानानां स्मृतालयः {॥ २:\hspace{.11em}२६॥} \veg\dontdisplaylinenum }%
     \var{{\devanagarifontvar\numnoemph\vd ॰शानानां\lem \mssALL,\hskip.2em plus .9em ॰शानाना \msNb,\hskip.5em plus .9em ॰गानानां \msNc\oo 
 स्मृतालयः\lem \msCa\msNb\msNc\msParis\msKOb,\hskip.2em plus .9em स्मृतालय \msCb,\hskip.5em plus .9em स्मृतालयं \msNa,\hskip.5em plus .9em स्मृतालया \Ed}}% 

{\devanagarifont बालसूर्यप्रभाः सर्वे ज्ञेयास्तत्पुरुषालये \thinspace{\dandab} \dontdisplaylinenum }%
     \var{{\devanagarifontvar\numemph\va ॰भाः\lem \mssALL,\hskip.2em plus .9em ॰भा \Ed}}% 
    \var{{\devanagarifontvar\numnoemph\vb ज्ञेयास्त॰\lem \mssALL,\hskip.2em plus .9em ज्ञेया त॰ \msNa\Ed\oo 
 ॰आलये\lem \mssALL,\hskip.2em plus .9em ॰आलयं \Ed}}% 

%Verse 2:27

{\devanagarifont परार्धपरकोटीनां पूर्वस्यां दिशमाश्रिताः {॥ २:\hspace{.11em}२७॥} \veg\dontdisplaylinenum }%
     \var{{\devanagarifontvar\numnoemph\vd दिश॰\lem \mssALL,\hskip.2em plus .9em दिशि॰ \msNb}}% 

{\devanagarifont भिन्नाञ्जनप्रभाः सर्वे दक्षिणां दिशमाश्रिताः \thinspace{\dandab} \dontdisplaylinenum }%
     \var{{\devanagarifontvar\numemph\va ॰ञ्जनप्रभाः\lem \mssALL,\hskip.2em plus .9em ॰ञ्जनः प्रभास् \msParisacorr,\hskip.5em plus .9em 
॰ञ्जनप्रभा \Ed}}% 
    \var{{\devanagarifontvar\numnoemph\vb दक्षिणां\lem \mssALL,\hskip.2em plus .9em दक्षिण॰ \Ed\oo 
 दिशम्\lem \mssALL,\hskip.2em plus .9em दिशिम् \msCb\Ed}}% 

%Verse 2:28

{\devanagarifont परार्धपरकोटीनामघोरालयमाश्रिताः {॥ २:\hspace{.11em}२८॥} \veg\dontdisplaylinenum }%
     \var{{\devanagarifontvar\numnoemph\vd ॰घोरा॰\lem \mssALL,\hskip.2em plus .9em ॰धोरा॰ \Ed\oo 
 ॰श्रिताः\lem \mssALL,\hskip.2em plus .9em ॰श्रिता \Ed}}% 

\pend
\endnumbering
\vfill\pagebreak\beginnumbering\pstart
\vers

{\devanagarifont कुन्देन्दुहिमशैलाभाः पश्चिमां दिशमाश्रिताः \thinspace{\dandab} \dontdisplaylinenum }%
     \var{{\devanagarifontvar\numemph\vb पश्चिमां\lem \mssALL,\hskip.2em plus .9em पश्चिमा \msCb\oo 
 दिश॰\lem \mssALL,\hskip.2em plus .9em दिशि॰ \msNc\oo 
 ॰श्रिताः\lem \mssALL,\hskip.2em plus .9em ॰श्रिता \Ed}}% 

%Verse 2:29

{\devanagarifont परार्धपरकोटीनां सद्यमिष्टालयः स्मृतः {॥ २:\hspace{.11em}२९॥} \veg\dontdisplaylinenum }%
     \var{{\devanagarifontvar\numnoemph\vd सद्यमिष्टा॰\lem \mssALL,\hskip.2em plus .9em सद्यमिष्ट्वा॰ \msNa\oo 
 स्मृतः\lem \mssALL,\hskip.2em plus .9em स्मृताः \msCb}}% 

{\devanagarifont कुङ्कुमोदकसंकाशा उत्तरां दिशमाश्रिताः \thinspace{\dandab} \dontdisplaylinenum }%
     \var{{\devanagarifontvar\numemph\vb उत्तरां\lem \mssALL,\hskip.2em plus .9em उत्तरा \msCb\oo 
 दिशम्\lem \mssALL,\hskip.2em plus .9em दिशिम् \msCa}}% 

%Verse 2:30

{\devanagarifont परार्धपरकोतीनां वामदेवालयः स्मृतः {॥ २:\hspace{.11em}३०॥} \veg\dontdisplaylinenum }%
     \var{{\devanagarifontvar\numnoemph\vd ॰लयः\lem \mssALL,\hskip.2em plus .9em ॰लय \msNc}}% 

{\devanagarifont ईशानस्य कलाः पञ्च वक्त्रस्यापि चतुष्कलाः \thinspace{\dandab} \dontdisplaylinenum }%
     \var{{\devanagarifontvar\numemph\va कलाः\lem \mssALL,\hskip.2em plus .9em कला \Ed}}% 
    \var{{\devanagarifontvar\numnoemph\vb चतुष्कलाः\lem \mssALL,\hskip.2em plus .9em चतुष्तके \Ed}}% 

%Verse 2:31

{\devanagarifont अघोरस्य कला अष्टौ वामदेवास्त्रयोदश {॥ २:\hspace{.11em}३१॥} \veg\dontdisplaylinenum }%
     \var{{\devanagarifontvar\numnoemph\vd वामदेवा॰\lem \mssALL,\hskip.2em plus .9em वामदेव॰ \msNb\oo 
 ॰दश\lem \mssALL,\hskip.2em plus .9em ॰दशः \msKOb}}% 
    \paral{{\devanagarifontsmall \vo \compare\ {\englishfont Ātmārthapūjāpaddhati 276cd--277ab:}
                                 ईशानस्य कलाः पञ्च पुरुषस्य चतुष्कलाः\thinspace{\devanagarifontsmall ॥} 
                                 अघोरास्य कलाश् चाष्टौ वामदेवेन त्रयोदश\thinspace{\devanagarifontsmall ।} }}

{\devanagarifont सद्यश्चाष्टौ कला ज्ञेयाः संसारार्णवतारकाः \thinspace{\dandab} \dontdisplaylinenum }%
     \var{{\devanagarifontvar\numemph\va ज्ञेयाः\lem \mssALL,\hskip.2em plus .9em ज्ञेया \Ed}}% 
    \var{{\devanagarifontvar\numnoemph\vb संसारार्णव॰\lem \mssALL,\hskip.2em plus .9em संसार्ण्णव॰ \msCbacorr,\hskip.5em plus .9em 
संसारव॰ \msKOb}}% 
    \paral{{\devanagarifontsmall \vo \compare\ {\englishfont Ātmārthapūjāpaddhati 277cd:}
                                 अष्टौ सद्यकला ज्ञेया मकुटादिक्रमान् न्यसेत् }}

%Verse 2:32

{\devanagarifont अष्टत्रिंशत्कला ह्येताः कीर्तिता द्विजसत्तम {॥ २:\hspace{.11em}३२॥} \veg\dontdisplaylinenum }%
     \var{{\devanagarifontvar\numnoemph\vc ॰त्रिंशत्क॰\lem \msParis\msKOb,\hskip.2em plus .9em ॰त्रिंशक॰ \msCa\msCb\msNa\msNb\msNc\Ed\oo 
 ह्येताः\lem \mssALL,\hskip.2em plus .9em ज्ञेयाः \Ed}}% 
    \var{{\devanagarifontvar\numnoemph\vd ॰सत्तम\lem \msCa\msCb\msNa\msNc\msParispcorr\msKOb,\hskip.2em plus .9em ॰सत्तमः \msNb\msParisacorr\Ed}}% 

{\devanagarifont संख्या वर्णा दिशश्चैव एकैकस्य पृथक्पृथक् \thinspace{\dandab} \dontdisplaylinenum }%
     \var{{\devanagarifontvar\numemph\va संख्या वर्णा\lem \msCb\msNc\msParis\msKOb,\hskip.2em plus .9em 
संख्या वर्ण्णो \msCa\msNb,\hskip.5em plus .9em संख्या वण्णा \msNa,\hskip.5em plus .9em संध्या वर्णा \Ed}}% 
    \var{{\devanagarifontvar\numnoemph\vb एकैकस्य\lem \mssALL,\hskip.2em plus .9em ऐकैकस्य \msCb\msNa}}% 

%Verse 2:33

{\devanagarifont पूर्वोक्तेन विधानेन बोधव्यास्तत्त्वचिन्तकैः {॥ २:\hspace{.11em}३३॥} \veg\dontdisplaylinenum }%
     \var{{\devanagarifontvar\numnoemph\vd बोधव्यास्त॰\lem \eme,\hskip.2em plus .9em बोधव्या त॰ \msCa\msCb\msNa\msNb\msNc\msParis\msKOb\Ed}}% 

{\devanagarifont शिवाण्डगमनाकृष्ट्या शिवयोगं सदाभ्यसेत् \thinspace{\dandab} \dontdisplaylinenum }%
     \var{{\devanagarifontvar\numemph\va ॰कृष्ट्या\lem \msCa\msCb\msNb\msKOb\Ed,\hskip.2em plus .9em ॰कृष्टा \msNa\msNc,\hskip.5em plus .9em ॰\uncl{कृष्त्या} \msParis}}% 
    \var{{\devanagarifontvar\numnoemph\vb शिवयोगं सदाभ्यसेत्\lem \mssALL,\hskip.2em plus .9em 
शिवयोग समभ्यसेत् \msNb,\hskip.5em plus .9em शिवायोगं समभ्यसेत् \msKObacorr}}% 

%Verse 2:34

{\devanagarifont शिवयोगं विना विप्र तत्र गन्तुं न शक्यते {॥ २:\hspace{.11em}३४॥} \veg\dontdisplaylinenum }%
     \var{{\devanagarifontvar\numnoemph\vc ॰योगं\lem \mssALL,\hskip.2em plus .9em ॰योग \Ed}}% 

\pend
\endnumbering
\vfill\pagebreak\beginnumbering\pstart
\vers

{\devanagarifont अश्वमेधादियज्ञानां कोट्यायुतशतानि च \thinspace{\dandab} \dontdisplaylinenum }%
 
{\devanagarifont कृच्छ्रादितप सर्वाणि कृत्वा कल्पशतानि च  \danda\dontdisplaylinenum }%
     \var{{\devanagarifontvar\numemph\vc ॰तप\lem \Ed,\hskip.2em plus .9em ॰तपः \msCa\msCb\msNa\msNb\msNc\msParis\msKOb\ \unmetr}}% 

%Verse 2:35

{\devanagarifont तत्र गन्तुं न शक्येत देवैरपि तपोधन {॥ २:\hspace{.11em}३५॥} \veg\dontdisplaylinenum }%
     \var{{\devanagarifontvar\numnoemph\ve शक्येत\lem \mssALL,\hskip.2em plus .9em शक्यैत \msCb,\hskip.5em plus .9em शक्येते \Ed}}% 
    \var{{\devanagarifontvar\numnoemph\vf देवै॰\lem \mssALL,\hskip.2em plus .9em देवे॰ \msNc\oo 
 ॰धन\lem \mssALL,\hskip.2em plus .9em ॰धनम् \msCb}}% 

{\devanagarifont गङ्गादिसर्वतीर्थेषु स्नात्वा तप्त्वा च वै पुनः \thinspace{\dandab} \dontdisplaylinenum }%
 
%Verse 2:36

{\devanagarifont तत्र गन्तुं न शक्येत ऋषिभिर्वा महात्मभिः {॥ २:\hspace{.11em}३६॥} \veg\dontdisplaylinenum }%
     \var{{\devanagarifontvar\numemph\vc गन्तुं\lem \mssALL,\hskip.2em plus .9em गन्तु \msNb\msNc\oo 
 शक्येत\lem \mssALL,\hskip.2em plus .9em शक्यन्ते \Ed}}% 

{\devanagarifont सप्तद्वीपसमुद्राणि रत्नपूर्णानि भो द्विज \thinspace{\dandab} \dontdisplaylinenum }%
     \var{{\devanagarifontvar\numemph\va ॰द्वीप॰\lem \mssALL,\hskip.2em plus .9em ॰दीप॰ \msNc\oo 
 ॰समुद्राणि\lem \mssALL,\hskip.2em plus .9em ॰समुद्राय \msNb}}% 
    \paral{{\devanagarifontsmall \vab {\englishfont \compare\ \SDHU\ 2.104:} त्रिः प्रदत्वा महीं पूर्णां{\englishfont ...} }}

{\devanagarifont दत्त्वा वा वेदविदुषे श्रद्धाभक्तिसमन्वितः  \danda\dontdisplaylinenum }%
     \var{{\devanagarifontvar\numnoemph\vc ॰विदुषे\lem \mssALL,\hskip.2em plus .9em ॰विदुषेण \msKObacorr}}% 

%Verse 2:37

{\devanagarifont तत्र गन्तुं न शक्येत विना ध्यानेन निश्चयः {॥ २:\hspace{.11em}३७॥} \veg\dontdisplaylinenum }%
     \var{{\devanagarifontvar\numnoemph\ve गन्तुं\lem \mssALL,\hskip.2em plus .9em गन्तु \msNb,\hskip.5em plus .9em गंन्तु \msNc\oo 
 शक्येत\lem \mssALL,\hskip.2em plus .9em शक्यन्ते \Ed}}% 

{\devanagarifont स्वदेहान्मांसमुद्धृत्य दत्त्वार्थिभ्यश्च निश्चयात् \thinspace{\dandab} \dontdisplaylinenum }%
     \var{{\devanagarifontvar\numemph\va स्वदेहान्मांस॰\lem \msCa\msCb\msNa\msNb,\hskip.2em plus .9em स्वदेहात्मांस॰ \msNc\msParis\msKOb,\hskip.5em plus .9em 
स्वदेहात्मां स॰ \Ed}}% 

{\devanagarifont स्वदारपुत्रसर्वस्वं शिरो ऽर्थिभ्यश्च यो ददेत्  \danda\dontdisplaylinenum }%
     \var{{\devanagarifontvar\numnoemph\vc ॰स्वं\lem \mssALL,\hskip.2em plus .9em ॰स्व \msNb}}% 

%Verse 2:38

{\devanagarifont न तत्र गन्तुं शक्येत अन्यैर्वापि सुदुष्करैः {॥ २:\hspace{.11em}३८॥} \veg\dontdisplaylinenum }%
     \var{{\devanagarifontvar\numnoemph\ve न तत्र गन्तुं\lem \mssALL,\hskip.2em plus .9em न तत्र गन्तुं न \msCb}}% 
    \var{{\devanagarifontvar\numnoemph\vf ॰दुष्करैः\lem \mssALL,\hskip.2em plus .9em ॰दुष्कृतः \msNb}}% 

{\devanagarifont यज्ञतीर्थतपोदानवेदाध्ययनपारगः \thinspace{\dandab} \dontdisplaylinenum }%
     \var{{\devanagarifontvar\numemph\va ॰दान॰\lem \mssALL,\hskip.2em plus .9em ॰दानं \msNa,\hskip.5em plus .9em ॰दानै \msNb}}% 
    \var{{\devanagarifontvar\numnoemph\vb ॰पारगः\lem \mssALL,\hskip.2em plus .9em ॰पारगाः \msCa\msNb}}% 

%Verse 2:39

{\devanagarifont ब्रह्माण्डान्तस्य भोगांस्तु भुङ्क्ते कालवशानुगः {॥ २:\hspace{.11em}३९॥} \veg\dontdisplaylinenum }%
     \var{{\devanagarifontvar\numnoemph\vc ब्रह्माण्डान्तस्य भोगांस्तु\lem \mssALL,\hskip.2em plus .9em 
ब्रह्माण्डान्तस्य भोगास्तु \msNb,\hskip.5em plus .9em 
ब्रह्माण्डात्तस्य भोगास्तु \Ed}}% 
    \var{{\devanagarifontvar\numnoemph\vd भुङ्क्ते\lem \mssALL,\hskip.2em plus .9em \uncl{भुङ्क्ते} \msNc,\hskip.5em plus .9em भुक्त्वा \Ed\oo 
 ॰गः\lem \mssALL,\hskip.2em plus .9em ॰गाः \msNaacorr\msKObacorr}}% 

\pend
\endnumbering
\vfill\pagebreak\beginnumbering\pstart
\vers

{\devanagarifont कालेन समप्रेष्येण धर्मो याति परिक्षयम् \thinspace{\dandab} \dontdisplaylinenum }%
     \var{{\devanagarifontvar\numemph\vb धर्मो\lem \mssALL,\hskip.2em plus .9em धर्मे \msNc}}% 

{\devanagarifont अलातचक्रवत्सर्वं कालो याति परिभ्रमन्  \danda\dontdisplaylinenum }%
 
%Verse 2:40

{\devanagarifont त्रैकाल्यकलनात्कालस्तेन कालः प्रकीर्तितः {॥ २:\hspace{.11em}४०॥} \veg\dontdisplaylinenum }%
     \var{{\devanagarifontvar\numnoemph\ve ॰कलनात्काल॰\lem \mssALL,\hskip.2em plus .9em ॰कलना काल॰ \msNb}}% 
    \var{{\devanagarifontvar\numnoemph\vf प्र॰\lem \mssALL,\hskip.2em plus .9em प्ररि॰ \msKOb}}% 

{\devanagarifont 
\jump
\begin{center}
\ketdanda~इति वृषसारसंग्रहे शिवाण्डसंख्या नामाध्यायो द्वितीयः~\ketdanda
\end{center}
\dontdisplaylinenum\vers  }%
     \var{{\devanagarifontvar\numnoemph{\englishfont \Colo:} वृषसार॰\lem \mssALL,\hskip.2em plus .9em वृषार॰ \msKOb\oo 
 नामाध्यायो द्वितीयः\lem \mssALL,\hskip.2em plus .9em 
नामाध्याय द्वितीयः \msNb,\hskip.5em plus .9em 
नाम द्वितीयो ऽध्यायः \Ed}}% 
\bekveg\szamveg
\vfill
\phpspagebreak

\versno=0\fejno=3
\thispagestyle{empty}

\centerline{\Large\devanagarifontbold [   तृतीयो ऽध्यायः  ]}{\vrule depth10pt width0pt} \fancyhead[CE]{{\footnotesize\devanagarifont वृषसारसंग्रहे  }}
\fancyhead[CO]{{\footnotesize\devanagarifont तृतीयो ऽध्यायः  }}
\fancyhead[LE]{}
\fancyhead[RE]{}
\fancyhead[LO]{}
\fancyhead[RO]{}
\szam\bek



\alalfejezet{धर्मप्रवचनम्}
\vers


{\devanagarifont विगतराग उवाच {\dandab}\dontdisplaylinenum  }%
 
{\devanagarifont किमर्थं धर्ममित्याहुः कतिमूर्तिश्च कीर्त्यते \thinspace{\danda} \dontdisplaylinenum }%
     \var{{\devanagarifontvar\numemph\va आहुः\lem \mssALL,\hskip.2em plus .9em आहु \Ed}}% 
    \lacuna{\devanagarifontsmall {\englishfont Witnesses used for this chapter:
                                             \msCa\ ff.\thinspace 197r--198v, 
                                             \msCb\ ff.\thinspace 204v--206r, 
                                             \msCc\ ff.\allowbreak\thinspace 273r--273v (broke off at 2.21 and resumes at 3.30b),
                                             \msNa\ ff.\thinspace 4v--6r, 
                                             \msNb\ exp.\thinspace 42, 47 (upper), 48 (lower),
                                             \msNc\ ff.\thinspace 213r--214v,
                                             \msParis\ exp.\thinspace 215r--215v 
                                                   (breaks off after 3.14d and resumes at 4.8a),
                                             \msKOb\ ff.\thinspace 213v--214v,
                                             \msTub\ f.\thinspace 272 (only),
                                             \Ed\ pp.\thinspace 588--591;
                                        \mssCaCbCc\ = \msCa + \msCb + \msCc } }%
  
%Verse 3:1

{\devanagarifont कतिपादवृषो ज्ञेयो गतिस्तस्य कति स्मृताः {॥ ३:\hspace{.11em}१॥} \veg\dontdisplaylinenum }%
     \var{{\devanagarifontvar\numnoemph\vd स्मृताः\lem \mssALL,\hskip.2em plus .9em स्मृता \msCb,\hskip.5em plus .9em स्मृतः \Ed}}% 

{\devanagarifont कौतूहलं ममोत्पन्नं संशयं छिन्धि तत्त्वतः \thinspace{\dandab} \dontdisplaylinenum }%
     \var{{\devanagarifontvar\numemph\va कौतूहलं\lem \mssALL,\hskip.2em plus .9em कौतुहल \Ed\oo 
 ममोत्पन्नं\lem \mssALL,\hskip.2em plus .9em समोत्पन्नं \msNc}}% 
    \var{{\devanagarifontvar\numnoemph\vb संशयं\lem \mssALL,\hskip.2em plus .9em सशयं \msCa\oo 
 छिन्धि\lem \mssALL,\hskip.2em plus .9em च्छित्वि \msKOb}}% 

%Verse 3:2

{\devanagarifont कस्य पुत्रो मुनिश्रेष्ठ प्रजास्तस्य कति स्मृताः {॥ ३:\hspace{.11em}२॥} \veg\dontdisplaylinenum }%
 
{\devanagarifont अनर्थयज्ञ उवाच {\dandab}\dontdisplaylinenum  }%
 
{\devanagarifont धृतिरित्येष धातुर्वै पर्यायः परिकीर्तितः \thinspace{\danda} \dontdisplaylinenum }%
 
%Verse 3:3

{\devanagarifont आधारणान्महत्त्वाच्च धर्म इत्यभिधीयते {॥ ३:\hspace{.11em}३॥} \veg\dontdisplaylinenum  }%
     \var{{\devanagarifontvar\numemph\vc आधारणान्म॰\lem \msCa\msNb\msParis\msKOb,\hskip.2em plus .9em 
आधारणात्प॰ \msCb,\hskip.5em plus .9em आधारणात्म॰ \msNa\msNc,\hskip.5em plus .9em आधारेण म॰ \Ed}}% 
    \var{{\devanagarifontvar\numnoemph\vd इत्यभिधीयते\lem \msCa\msNa\msNc\msKOb\Ed,\hskip.2em plus .9em 
इत्यविधीयते \msCb\msNb,\hskip.5em plus .9em 
इ\uncl{त्यभिधीयते} \msParis}}% 
    \paral{{\devanagarifontsmall \vcd {\englishfont \compare\ \LINPU\ 1.10.12cd--13ab:}
                         धारणार्थे महान्ह्येष धर्मशब्दः प्रकीर्तितः\thinspace{\devanagarifontsmall ॥}
                         अधारणे ऽमहत्त्वे च अधर्म इति चोच्यते\thinspace{\devanagarifontsmall ।}
                \vo\ {\englishfont \compare\ \BRAHMANDAPUR\ 1.32.29:}
                         धारणार्थो धृतिश्चैव धातुः शब्दे प्रकीर्तितः\thinspace{\devanagarifontsmall ।}
                         अधारणामहत्त्वे च अधर्म इति चोच्यते\thinspace{\devanagarifontsmall ॥}
                     {\englishfont \compare\ \VAYUP\ 1.59.28:}
                         धारणा धृतिरित्यर्थाद्धातोर्धर्मः प्रकीर्तितः\thinspace{\devanagarifontsmall ।}
                         अधारणे ऽमहत्त्वे च अधर्म इति चोच्यते\thinspace{\devanagarifontsmall ॥}
                     {\englishfont \compare\ \MATSP\ 145.27:}  
                         धर्मेति धारणे धातुर्महत्वे चैव उच्यते\thinspace{\devanagarifontsmall ।}
                         आधारणे महत्त्वे वा धर्मः स तु निरुच्यते\thinspace{\devanagarifontsmall ।} }}

\pend
\endnumbering
\vfill\pagebreak\beginnumbering\pstart
\vers

{\devanagarifont श्रुतिस्मृतिद्वयोर्मूर्तिश्चतुष्पादवृषः स्थितः \thinspace{\dandab} \dontdisplaylinenum }%
     \var{{\devanagarifontvar\numemph\vab ॰स्मृतिद्वयोर्मूर्तिश्च॰\lem \msCa,\hskip.2em plus .9em ॰स्मृतिद्वयो मूर्त्तिश्च॰ \msCb\msNb\msParis\msKOb,\hskip.5em plus .9em 
॰स्मृतिद्वयो मूर्त्ति च॰ \msNa\msNc,\hskip.5em plus .9em 
॰स्मृतिर्द्वयो मूर्तिश्च \Ed}}% 
    \var{{\devanagarifontvar\numnoemph\vb ॰वृषः\lem \mssALL,\hskip.2em plus .9em ॰वृष \msNc}}% 

%Verse 3:4

{\devanagarifont चतुराश्रम यो धर्मः कीर्तितानि मनीषिभिः {॥ ३:\hspace{.11em}४॥} \veg\dontdisplaylinenum }%
     \var{{\devanagarifontvar\numnoemph\vc चतुरा॰\lem \msCb\msNa\msNb\msParispcorr\Ed,\hskip.2em plus .9em चातुरा॰ \msCa\msNc\msParisacorr\msKOb}}% 
    \paral{{\devanagarifontsmall \vo {\englishfont \compare\ \VSS\ 4.74 below:}
                 चतुष्पादः स्मृतो धर्मश्चतुराश्रममाश्रितः\thinspace{\devanagarifontsmall ।}
                 गृहस्थो ब्रह्मचारी च वानप्रस्थो ऽथ भैक्षुकः\thinspace{\devanagarifontsmall ॥} }}

{\devanagarifont गतिश्च पञ्च विज्ञेयः शृणु धर्मस्य भो द्विज \thinspace{\dandab} \dontdisplaylinenum }%
     \var{{\devanagarifontvar\numemph\va विज्ञेयः\lem \mssALL,\hskip.2em plus .9em \om\ \msCb}}% 
    \var{{\devanagarifontvar\numnoemph\vb द्विज\lem \mssALL,\hskip.2em plus .9em द्विजः \msKOb}}% 
    \lacuna{\devanagarifontsmall \vab {\englishfont The first available folio of the \VSS\ in \msTub\ (f.~272) starts here.
                        \msCb\ reads here } गतिश्च पौत्राश्च अनेकाश्च बभूव ह,
                        {\englishfont skipping to 3.7cd, omitting 3.5--7ab.} }%
  
%Verse 3:5

{\devanagarifont देवमानुषतिर्यं च नरकस्थावरादयः {॥ ३:\hspace{.11em}५॥} \veg\dontdisplaylinenum }%
     \var{{\devanagarifontvar\numnoemph\vc ॰मानुष॰\lem \mssALL,\hskip.2em plus .9em ॰मानुषि॰ \msParis}}% 

{\devanagarifont ब्रह्मणो हृदयं भित्त्वा जातो धर्मः सनातनः \thinspace{\dandab} \dontdisplaylinenum }%
     \var{{\devanagarifontvar\numemph\va ब्रह्मणो\lem \mssALL,\hskip.2em plus .9em \om\ \msCb,\hskip.5em plus .9em ब्राह्मणो \Ed\oo 
 भित्त्वा\lem \mssALL,\hskip.2em plus .9em वित्त्वा \msNb}}% 
    \var{{\devanagarifontvar\numnoemph\vb धर्मः\lem \mssALL,\hskip.2em plus .9em धर्म \msNb\oo 
 सना॰\lem \mssALL,\hskip.2em plus .9em शवा॰ \msTub}}% 
    \paral{{\devanagarifontsmall \vab {\englishfont \compare\ \DEVIP\ 4.59cd:} ब्रह्मणो हृदयाज्जातः पुत्रो धर्म इति स्मृतः \oo 
                     {\englishfont \compare\ also \MBH\ 1.60.40ab:} ब्रह्मणो हृदयं भित्त्वा निःसृतो भगवान्भृगुः }}

%Verse 3:6

{\devanagarifont तस्य पत्नी महाभागा त्रयोदश सुमध्यमाः {॥ ३:\hspace{.11em}६॥} \veg\dontdisplaylinenum }%
     \var{{\devanagarifontvar\numnoemph\vd ॰मध्यमाः\lem \mssALL,\hskip.2em plus .9em \om\ \msCb}}% 

{\devanagarifont दक्षकन्या विशालाक्षी श्रद्धाद्या सुमनोहराः \thinspace{\dandab} \dontdisplaylinenum }%
     \var{{\devanagarifontvar\numemph\va ॰आक्षी\lem \mssALL,\hskip.2em plus .9em \om\ \msCb,\hskip.5em plus .9em ॰आक्षि \Ed}}% 
    \var{{\devanagarifontvar\numnoemph\vb ॰आद्या\lem ॰आद्या \msNb\msNc\msParis\msKOb\msTub\Ed,\hskip.2em plus .9em ॰आढ्या \msCa,\hskip.5em plus .9em \om\ \msCb,\hskip.5em plus .9em ॰आढ्याः \msNa\oo 
 ॰हराः\lem \msNb\Ed,\hskip.2em plus .9em ॰हरा \msCa\msNc\msParis\msKOb,\hskip.5em plus .9em  \om\ \msCb,\hskip.5em plus .9em ॰\lk \uncl{माः} \msNa,\hskip.5em plus .9em 
॰हरात् \msTub}}% 

{\devanagarifont तस्य पुत्राश्च पौत्राश्च अनेकाश्च बभूव ह  \danda\dontdisplaylinenum }%
     \var{{\devanagarifontvar\numnoemph\vcd तस्य पुत्राश्च पौत्राश्च अनेकाश्च बभूव ह\lem \msCa\msNb\msParis,\hskip.2em plus .9em 
गतिश्च पौत्राश्च अनेकाश्च बभूव ह {\englishfont (eyeskip to 3.5a)} \msCb,\hskip.5em plus .9em 
तस्य पुत्राश्च योत्राश्च अनेकाश्च बभूव ह \msNa\msNc,\hskip.5em plus .9em 
तस्य पुत्राश्च पौत्राश्च अनेकाश्च बहूव ह \msKOb,\hskip.5em plus .9em 
तस्य पुत्राश्च पौत्राश्च अनेकाश्च \lac\ \msTub,\hskip.5em plus .9em 
तस्य पुत्रा अनेकाश्च तथा पौत्रा बभूवहः \Ed}}% 

%Verse 3:7

{\devanagarifont एष धर्मनिसर्गो ऽयं किं भूयः श्रोतुमिच्छसि {॥ ३:\hspace{.11em}७॥} \veg\dontdisplaylinenum }%
     \var{{\devanagarifontvar\numnoemph\vef ॰निसर्गो\lem \mssALL,\hskip.2em plus .9em ॰विसर्गो \msTub}}% 

\pend
\endnumbering
\vfill\pagebreak\beginnumbering\pstart
\vers

{\devanagarifont विगतराग उवाच {\dandab}\dontdisplaylinenum  }%
     \var{{\devanagarifontvar\numemph\vo विगतराग उवाच\lem \msCb\msNapcorr\msNc\msKOb\msTub\Ed,\hskip.2em plus .9em 
विगतराग उ \msCa\msNb\msParis,\hskip.5em plus .9em \om\ \msNaacorr}}% 

{\devanagarifont धर्मपत्नी विशेषेण पुत्रस्तेभ्यः पृथक्पृथक् \thinspace{\danda} \dontdisplaylinenum }%
 
%Verse 3:8

{\devanagarifont श्रोतुमिच्छामि तत्त्वेन कथयस्व तपोधन {॥ ३:\hspace{.11em}८॥} \veg\dontdisplaylinenum }%
 
{\devanagarifont अनर्थयज्ञ उवाच {\dandab}\dontdisplaylinenum  }%
 
{\devanagarifont श्रद्धा लक्ष्मीर्धृतिस्तुष्टिः पुष्टिर्मेधा क्रिया लज्जा \thinspace{\danda} \dontdisplaylinenum }%
     \var{{\devanagarifontvar\numemph\va लक्ष्मीर्धृतिस्तुष्टिः\lem \msCa,\hskip.2em plus .9em 
लक्ष्मीर्धृतिस्तुष् \msCb,\hskip.5em plus .9em 
लक्ष्मी द्धृतिर्द्धृतिस्तुष्टिः \msNaacorr,\hskip.5em plus .9em 
लक्ष्मीर्द्धृतिस्तुष्टिः \msNapcorr,\hskip.5em plus .9em 
लक्ष्मीं धृति तुष्टिः \msNb,\hskip.5em plus .9em 
लक्ष्मी धृतिस्तुष्टिः \msNc\msParis\msKOb\msTub,\hskip.5em plus .9em 
लक्ष्मी धृतिस्तुष्टी \Ed}}% 
    \var{{\devanagarifontvar\numnoemph\vb पुष्टिर्मे॰\lem \mssALL,\hskip.2em plus .9em पुष्टि मे॰ \msTub\Ed\oo 
 लज्जा\lem \mssALL,\hskip.2em plus .9em लजा \msNa}}% 

%Verse 3:9

{\devanagarifont बुद्धिः शान्तिर्वपुः कीर्तिः सिद्धिः प्रसूतिसम्भवाः {॥ ३:\hspace{.11em}९॥} \veg\dontdisplaylinenum }%
     \var{{\devanagarifontvar\numnoemph\vc बुद्धिः\lem \mssALL,\hskip.2em plus .9em बुद्धि \msCa}}% 
    \var{{\devanagarifontvar\numnoemph\vd सिद्धिः प्रसूतिसम्भवाः\lem \conj,\hskip.2em plus .9em 
सिद्धिश्चाभूतिसम्भवा \msCa\msNa\msNb\msNc,\hskip.5em plus .9em 
सिद्धिश्चातिसम्भवा \msCb,\hskip.5em plus .9em 
सिद्धिश्चाभूतिसम्भवाः \msParis\msKOb,\hskip.5em plus .9em 
सि\lac संभवा \msTub,\hskip.5em plus .9em 
सिद्धिश्च भूतिसम्भवा \Ed}}% 

{\devanagarifont श्रद्धा कामः सुतो जातो दर्पो लक्ष्मीसुतः स्मृतः \thinspace{\dandab} \dontdisplaylinenum }%
     \var{{\devanagarifontvar\numemph\va कामः\lem \msNa,\hskip.2em plus .9em काम॰ \msCa\msCb\msNb\msNc\msParis\msKOb\msTub,\hskip.5em plus .9em धर्म॰ \Ed\oo 
 जातो\lem \mssALL,\hskip.2em plus .9em \om\ \msKObacorr}}% 

%Verse 3:10

{\devanagarifont धृत्यास्तु नियमः पुत्रः संतोषस्तुष्टिजः स्मृतः {॥ ३:\hspace{.11em}१०॥} \veg\dontdisplaylinenum }%
     \var{{\devanagarifontvar\numnoemph\vcd नियमः पुत्रः संतोषस्तुष्टिजः\lem \mssALL,\hskip.2em plus .9em 
निय\lacwithnum{4} स्सन्तोषस्\uncl{तुष्टि}जः \msKOb}}% 
    \paral{{\devanagarifontsmall \vo {\englishfont See a passage similar to \VSS\ 3.10--13,
         e.g., in \KURMP\ 1.8.20 ff.:}
         श्रद्धाया आत्मजः कामो दर्पो लक्ष्मीसुतः स्मृतः\thinspace{\devanagarifontsmall ।}
         धृत्यास्तु नियमः पुत्रस्तुष्ट्याः संतोष उच्यते\thinspace{\devanagarifontsmall ॥} 
         पुष्ट्या लाभः सुतश्चापि मेधापुत्रः श्रुतस्तथा\thinspace{\devanagarifontsmall ।} 
         क्रियायाश्चाभवत्पुत्रो दण्डः समय एव च\thinspace{\devanagarifontsmall ॥}  
         बुद्ध्या बोधः सुतस्तद्वदप्रमादो व्यजायत\thinspace{\devanagarifontsmall ।} 
         लज्जाया विनयः पुत्रो वपुषो व्यवसायकः\thinspace{\devanagarifontsmall ॥}  
         क्षेमः शान्तिसुतश्चापि सुखं सिद्धिरजायत\thinspace{\devanagarifontsmall ।}
         यशः कीर्तिसुतस्तद्वदित्येते धर्मसूनवः\thinspace{\devanagarifontsmall ॥}   
         कामस्य हर्षः पुत्रो ऽभूद्देवानन्दो व्यजायत\thinspace{\devanagarifontsmall ।}  
         इत्येष वै सुखोदर्कः सर्गो धर्मस्य कीर्तितः\thinspace{\devanagarifontsmall ॥} }}

{\devanagarifont पुष्ट्या लाभः सुतो जातो मेधापुत्रः श्रुतस्तथा \thinspace{\dandab} \dontdisplaylinenum }%
     \var{{\devanagarifontvar\numemph\va लाभः\lem \mssALL,\hskip.2em plus .9em लाभ॰ \msNa\Ed\oo 
 जातो\lem \mssALL,\hskip.2em plus .9em \om\ \msParis}}% 
    \var{{\devanagarifontvar\numnoemph\vb ॰पुत्रः\lem \msKObpcorr,\hskip.2em plus .9em ॰पुत्र \msCa\msCb\msNa\msNb\msNc\msParis\msKObacorr\msTub\Ed\oo 
 श्रुत॰\lem \msCa\msNa\msNb\msNc\msTub\Ed,\hskip.2em plus .9em श्रत॰ \msCb,\hskip.5em plus .9em श्रुति॰ \msParis\msKOb}}% 

%Verse 3:11

{\devanagarifont क्रियायास्त्वभवत्पुत्रो दण्डः समय एव च {॥ ३:\hspace{.11em}११॥} \veg\dontdisplaylinenum }%
     \var{{\devanagarifontvar\numnoemph\vc त्वभवत्पुत्रो\lem \eme,\hskip.2em plus .9em त्वभयः पुत्रो \msCa\msCb\msNa\msNb\msNc\msParis\msTub,\hskip.5em plus .9em 
त्व\uncl{भ}यः पुत्रो \msKOb,\hskip.5em plus .9em 
तूभयः पुत्रौ \Ed}}% 
    \var{{\devanagarifontvar\numnoemph\vd दण्डः\lem \corr,\hskip.2em plus .9em दण्डे \msCa\msNaacorr दण्ड॰ \msNapcorr\msNb\msNc\msParis\msKOb\msTub\Ed,\hskip.5em plus .9em 
दण्डो \msCb\oo 
 च\lem \mssALL,\hskip.2em plus .9em तु \msTub\Ed}}% 
    \paral{{\devanagarifontsmall \vcd {\englishfont \similar\ \LINPU\ 1.70.295ab:}क्रियायामभवत्पुत्रो दण्डः समय एव च;
                     {\englishfont \similar\ \KURMP\ 1.8.22cd:   }क्रियायाश्चाभवत्पुत्रो दण्डः समय एव च;
                     {\englishfont \compare\ \LINPU\ 1.5.37:     }धर्मस्य वै क्रियायां तु दण्डः समय एव च }}

{\devanagarifont लज्जाया विनयः पुत्रो बुद्ध्या बोधःसुतः स्मृतः \thinspace{\dandab} \dontdisplaylinenum }%
     \var{{\devanagarifontvar\numemph\va लज्जाया विनयः\lem \mssALL,\hskip.2em plus .9em लज्जायाः विनय॰ \Ed}}% 
    \var{{\devanagarifontvar\numnoemph\vb सुतः स्मृतः\lem \mssALL,\hskip.2em plus .9em 
सुतः \lk\lk\ \msCa,\hskip.5em plus .9em सुतःस्तथा \msCb}}% 

%Verse 3:12

{\devanagarifont लज्जायाः सुधियः पुत्र अप्रमादश्च तावुभौ {॥ ३:\hspace{.11em}१२॥} \veg\dontdisplaylinenum }%
     \var{{\devanagarifontvar\numnoemph\vc लज्जायाः\lem \mssALL,\hskip.2em plus .9em \lac\ \msTub\oo 
 सुधियः\lem \msKObpcorr\msTub\Ed,\hskip.2em plus .9em सुधिय \msCa\msCb\msNa\msNb\msNc\msParis\msKObacorr\oo 
 पुत्र\lem \mssALL,\hskip.2em plus .9em पुत्रः \Ed}}% 
    \var{{\devanagarifontvar\numnoemph\vd अप्रमाद॰\lem \mssALL,\hskip.2em plus .9em अप्रमादा॰ \msNa}}% 

{\devanagarifont क्षेमः शान्तिसुतो विन्द्याद्व्यवसायो वपोः सुतः \thinspace{\dandab} \dontdisplaylinenum }%
     \var{{\devanagarifontvar\numemph\vb वपोः\lem \mssALL,\hskip.2em plus .9em वपो \msNa}}% 

{\devanagarifont यशः कीर्तिसुतो ज्ञेयः सुखं सिद्धेर्व्यजायत  \danda\dontdisplaylinenum }%
     \var{{\devanagarifontvar\numnoemph\vd सुखं सिद्धेर्व्यजायत\lem \msCb\msNa,\hskip.2em plus .9em 
सुखं सिद्धि व्यजायत \msCa\msParis,\hskip.5em plus .9em 
सुखं सिद्धेर्व्यजायते \msNb,\hskip.5em plus .9em 
सुखं सिद्धि व्यजायतः \msNc,\hskip.5em plus .9em 
\lacwithnum{4} व्यजा\uncl{य}\lk\ \msKOb,\hskip.5em plus .9em 
सुखं सिद्धि व्यजायते \msTub\Ed}}% 

%Verse 3:13

{\devanagarifont स्वायम्भुवे ऽन्तरे त्वासन्कीर्तिता धर्मसूनवः {॥ ३:\hspace{.11em}१३॥} \veg\dontdisplaylinenum }%
     \var{{\devanagarifontvar\numnoemph\ve स्वायम्भुवे\lem \msCa\msNa\msNc\msParis\msKOb\msTub,\hskip.2em plus .9em स्वायम्भुवो \msCb,\hskip.5em plus .9em स्वयम्भुवे \msNb\Ed\oo 
 ऽन्तरे त्वासन्\lem \conj,\hskip.2em plus .9em ऽन्तरे त्वासि \msCa\msCb\msNa\msParis\msKOb\msTub,\hskip.5em plus .9em 
ऽन्तरे त्वासीत् \msNb,\hskip.5em plus .9em ऽन्तरे त्वासं \msNc,\hskip.5em plus .9em ऽन्तरेवासि \Ed}}% 

{\devanagarifont विगतराग उवाच {\dandab}\dontdisplaylinenum  }%
 
{\devanagarifont मूर्तिद्वयं कथं धर्मं कथयस्व तपोधन \thinspace{\danda} \dontdisplaylinenum }%
     \var{{\devanagarifontvar\numemph\va धर्मं\lem \mssALL,\hskip.2em plus .9em द्धर्म \msNc,\hskip.5em plus .9em धर्मः \Ed}}% 
    \var{{\devanagarifontvar\numnoemph\vb ॰धन\lem \mssALL,\hskip.2em plus .9em ॰धनः \msKObacorr\msTub}}% 

%Verse 3:14

{\devanagarifont कौतूहलमतीवं मे कर्तय ज्ञानसंशयम् {॥ ३:\hspace{.11em}१४॥} \veg\dontdisplaylinenum }%
     \var{{\devanagarifontvar\numnoemph\vc कौतूहलमतीवं मे\lem \mssALL,\hskip.2em plus .9em 
कोतूहलमतीव मे \msCb,\hskip.5em plus .9em 
\lac लम्मतीवम्मे \msTub}}% 
    \var{{\devanagarifontvar\numnoemph\vd कर्तय\lem \eme,\hskip.2em plus .9em कीर्तय \msCa\msCb\msNa\msNb\msNc\msKOb\Ed,\hskip.5em plus .9em कीर्तिय \msTub\oo 
 ॰संशयम्\lem \msCa\msNa\msNc\Ed,\hskip.2em plus .9em ॰संशयः \msCb\msNb\msTub,\hskip.5em plus .9em ॰सञ्चयम् \msKOb}}% 
    \lacuna{\devanagarifontsmall \vc {\englishfont In \msParis, folio 215v ends with} कौतूहलमती {\englishfont and the next available 
                      folio side (217r) starts with} त्यमिष्टगतिः प्रोक्तं {\englishfont  in 4.8a. Thus one folio (f. 216), 
                      containing 3.14d--4.7, is missing.} }%
  
{\devanagarifont अनर्थयज्ञ उवाच {\dandab}\dontdisplaylinenum  }%
 
{\devanagarifont श्रुतिस्मृतिद्वयोर्मूर्तिर्धर्मस्य परिकीर्तिता \thinspace{\danda} \dontdisplaylinenum }%
     \var{{\devanagarifontvar\numemph\va श्रुति॰\lem \mssALL,\hskip.2em plus .9em श्रुतिः \msCb\Ed}}% 
    \var{{\devanagarifontvar\numnoemph\vab ॰द्वयोर्मूर्तिर्ध॰\lem \msCa\msKOb,\hskip.2em plus .9em ॰द्वयो मूर्ति ध॰ \msCb\msNa\msNb\msTub,\hskip.5em plus .9em 
॰द्वयी मूर्ति ध॰ \msNc,\hskip.5em plus .9em ॰द्वयोर्मूर्ति ध॰ \Ed}}% 
    \var{{\devanagarifontvar\numnoemph\vb ॰कीर्तिता\lem \msCa\msCb\msNa\msKOb\Ed,\hskip.2em plus .9em ॰कीर्त्तितः \msNb,\hskip.5em plus .9em कीर्त्तिताः \msNc\msTub}}% 

{\devanagarifont दाराग्निहोत्रसम्बन्ध इज्या श्रौतस्य लक्षणम्  \danda\dontdisplaylinenum }%
     \var{{\devanagarifontvar\numnoemph\vcd ॰बन्ध इ॰\lem \msNb\Ed,\hskip.2em plus .9em ॰बद्ध इ॰ \msCa\msCb\msNa\msNc\msKOb\msTub}}% 
    \var{{\devanagarifontvar\numnoemph\vd श्रौतस्य\lem \msKOb,\hskip.2em plus .9em श्रोतस्य \msCa\msCb\msNc\msTub,\hskip.5em plus .9em 
श्रौत्रस्य \msNa,\hskip.5em plus .9em स्रोत्रस्य \msNb,\hskip.5em plus .9em श्रुतस्य \Ed}}% 
    \paral{{\devanagarifontsmall \vcd {\englishfont \compare\ \MANU\ 3.171ab:}दाराग्निहोत्रसंयोगं कुरुते यो ऽग्रजे स्थिते; 
                         {\englishfont and also \MATSP\ 142.41:} 
                         दाराग्निहोत्रसम्बन्धमृग्यजुःसामसंहिताः\thinspace{\devanagarifontsmall ।}
                         इत्यादिबहुलं श्रौतं धर्मं सप्तर्षयो ऽब्रुवन्\thinspace{\devanagarifontsmall ॥}\hspace{2em} }}

%Verse 3:15

{\devanagarifont स्मार्तो वर्णाश्रमाचारो यमैश्च नियमैर्युतः {॥ ३:\hspace{.11em}१५॥} \veg\dontdisplaylinenum }%
     \var{{\devanagarifontvar\numnoemph\ve स्मार्तो\lem \eme,\hskip.2em plus .9em स्मार्त \msCa\msCb\msNa\msNb\msNc\msTub\Ed,\hskip.5em plus .9em स्मा\lac\ \msKOb}}% 
    \paral{{\devanagarifontsmall \vcdef {\englishfont \similar\ \MBH\ Suppl. 1.36.10: 
                                 }दानाग्निहोत्रमिज्या च श्रौतस्यैतद्धि लक्षणम्\thinspace{\devanagarifontsmall ।}
                                 स्मार्तो वर्णाश्रमाचारो यमैश्च नियमैर्युतः\thinspace{\devanagarifontsmall ॥}
                          \similar\ {\englishfont \MATSP\ 145.30cd--31ab:
                                 }दाराग्निहोत्रसम्बन्धमिज्या श्रौतस्य लक्षणम्\thinspace{\devanagarifontsmall ।}
                                 स्मार्तो वर्णाश्रमाचारो यमैश्च नियमैर्युतः\thinspace{\devanagarifontsmall ॥}
                          \similar\ {\englishfont \BRAHMANDAPUR\ 1.32.33cd--34ab:}
                                 दाराग्निहोत्रसम्बन्धाद् द्विधा श्रौतस्य लक्षणम्\thinspace{\devanagarifontsmall ।}
                                 स्मार्तो वर्णाश्रमाचारैर्यमैः स नियमैः स्मृतः\thinspace{\devanagarifontsmall ॥} }}


\alalfejezet{यमनियमभेदः}
{\devanagarifont यमश्च नियमश्चैव द्वयोर्भेदमतः शृणु \thinspace{\dandab} \dontdisplaylinenum }%
     \var{{\devanagarifontvar\numemph\va नियम॰\lem \mssALL,\hskip.2em plus .9em नियमै॰ \msNa}}% 
    \var{{\devanagarifontvar\numnoemph\vb द्वयोर्भेदमतः\lem \mssALL,\hskip.2em plus .9em 
द्वयो\lac मतः \msTub}}% 

{\devanagarifont अहिंसा सत्यमस्तेयमानृशंस्यो दमो घृणा  \danda\dontdisplaylinenum }%
     \var{{\devanagarifontvar\numnoemph\vd ॰मानृशंस्यो\lem \msKOb,\hskip.2em plus .9em ॰मनृशंस्यो \msCa\msCb\msNa\msNb\msTub\Ed,\hskip.5em plus .9em 
॰मानृशंस्या \msNc}}% 
    \paral{{\devanagarifontsmall \vcd {\englishfont \similar\ \MBH\ 12.8.17ab:} अहिंसा सत्यवचनमानृशंस्यं दमो घृणा
                 \vo {\englishfont \similar\ \VDHU\ 3.233.203: 
                         }आनृशंस्यं क्षमा सत्यमहिंसा च दमः स्पृहा\thinspace{\devanagarifontsmall ।}
                         ध्यानं प्रसादो माधुर्यं चार्जवं च यमा दश\thinspace{\devanagarifontsmall ॥} }}

%Verse 3:16

{\devanagarifont धन्याप्रमादो माधुर्यमार्जवं च यमा दश {॥ ३:\hspace{.11em}१६॥} \veg\dontdisplaylinenum }%
     \var{{\devanagarifontvar\numnoemph\ve धन्या॰\lem \Ed,\hskip.2em plus .9em धन्यः \msCa\msCb\msNb\msNc\msKOb,\hskip.5em plus .9em ध्यन्यं \msNa,\hskip.5em plus .9em 
धन्य \msTub\oo 
 माधुर्य॰\lem \Ed,\hskip.2em plus .9em माधूर्य॰ \msCa\msCb\msNa\msNb\msNc\msKOb\msTub}}% 
    \var{{\devanagarifontvar\numnoemph\vf ॰र्जवं च\lem \mssALL,\hskip.2em plus .9em 
॰र्जव\uncl{श्चा} \msTub,\hskip.5em plus .9em 
॰र्जवश्च \Ed}}% 

{\devanagarifont एकैकस्य पुनः पञ्चभेदमाहुर्मनीषिणः \thinspace{\dandab} \dontdisplaylinenum }%
     \var{{\devanagarifontvar\numemph\vb ॰माहुर्म॰\lem \mssALL,\hskip.2em plus .9em ॰माहु म॰ \msNc}}% 

%Verse 3:17

{\devanagarifont अहिंसादि प्रवक्ष्यामि शृणुष्वावहितो द्विज {॥ ३:\hspace{.11em}१७॥} \veg\dontdisplaylinenum }%
     \var{{\devanagarifontvar\numnoemph\vd शृणुष्वा॰\lem \mssALL,\hskip.2em plus .9em शृणुष्व॰ \msNa\msNb}}% 

\pend
\endnumbering
\vfill\pagebreak\beginnumbering\pstart
\vers


\alalfejezet{यमेष्वहिंसा (१)}

\alalalfejezet{पञ्चविधा हिंसा}

{\devanagarifont त्रासनं ताडनं बन्धो मारणं वृत्तिनाशनम् \thinspace{\dandab} \dontdisplaylinenum }%
     \var{{\devanagarifontvar\numemph\va बन्धो\lem \msCa\msCb\msNa\msNc,\hskip.2em plus .9em बद्धो \msNb\msKOb\msTub,\hskip.5em plus .9em बन्ध \Ed}}% 
    \var{{\devanagarifontvar\numnoemph\vb मारणं\lem \mssALL,\hskip.2em plus .9em मा\lac\ \msKOb}}% 

%Verse 3:18

{\devanagarifont हिंसां पञ्चविधामाहुर्मुनयस्तत्त्वदर्शिनः {॥ ३:\hspace{.11em}१८॥} \veg\dontdisplaylinenum }%
     \var{{\devanagarifontvar\numnoemph\vc हिंसां\lem \msCa\msNa\msNc\msKOb,\hskip.2em plus .9em हिंसा \msCb\msNb\msTub\Ed}}% 
    \var{{\devanagarifontvar\numnoemph\vcd ॰विधामाहुर्मु॰\lem \msCb\msNa\msNc\msKOb,\hskip.2em plus .9em ॰विधमाहुर्मु॰ \msCa,\hskip.5em plus .9em 
॰विधान्याहुर्मु॰ \msNb,\hskip.5em plus .9em ॰विधा\lac \msTub,\hskip.5em plus .9em 
॰विध प्राहुर्मु॰ \Ed}}% 

{\devanagarifont काष्ठलोष्टकशाद्यैस्तु ताडयन्तीह निर्दयाः \thinspace{\dandab} \dontdisplaylinenum }%
     \var{{\devanagarifontvar\numemph\va काष्ठलोष्ट॰\lem \mssALL,\hskip.2em plus .9em 
का\uncl{ष्ठ}\lacwithnum{2}  \msNb\oo 
 ॰द्यैस्तु\lem \mssALL,\hskip.2em plus .9em ॰द्यैश्च \msKOb}}% 
    \var{{\devanagarifontvar\numnoemph\vb निर्दयाः\lem \mssALL,\hskip.2em plus .9em निर्दया \Ed}}% 

%Verse 3:19

{\devanagarifont तत्प्रहारविभिन्नाङ्गो मृतवध्यमवाप्नुयात् {॥ ३:\hspace{.11em}१९॥} \veg\dontdisplaylinenum }%
     \var{{\devanagarifontvar\numnoemph\vc ॰भिन्नाङ्गो\lem \mssALL,\hskip.2em plus .9em ॰भिन्नाङ्गा \Ed}}% 
    \var{{\devanagarifontvar\numnoemph\vd ॰वध्यमवा॰\lem \mssALL,\hskip.2em plus .9em ॰वध्यववा॰ \msCa}}% 

{\devanagarifont बद्ध्वा पादौ भुजोरश्च शिरोरुक्कण्ठपाशिताः \thinspace{\dandab} \dontdisplaylinenum }%
     \var{{\devanagarifontvar\numemph\va भुजोरश्च\lem \mssALL,\hskip.2em plus .9em भुजौरश्च \msNa\Ed}}% 
    \var{{\devanagarifontvar\numnoemph\vb शिरोरुक्कण्ठ॰\lem \eme,\hskip.2em plus .9em शिरोरुकण्ठ॰ \msCa\msCb\msNa\msNb\msNc\msKOb,\hskip.5em plus .9em 
शिरोरुः कण्ठ॰ \msTub\Ed\oo 
 ॰पाशिताः\lem \mssALL,\hskip.2em plus .9em ॰पासिनः \msTub}}% 

%Verse 3:20

{\devanagarifont अनाहता म्रियन्त्येवं वधो बन्धनजः स्मृतः {॥ ३:\hspace{.11em}२०॥} \veg\dontdisplaylinenum }%
     \var{{\devanagarifontvar\numnoemph\vc अनाहता म्रियन्त्येवं\lem \mssALL,\hskip.2em plus .9em 
अनाहत म्रियंत्येष \msNb,\hskip.5em plus .9em अनाहतो म्रियन्त्येव \msKOb}}% 
    \var{{\devanagarifontvar\numnoemph\vd वधो बन्धनजः स्मृतः\lem \conj,\hskip.2em plus .9em वधो बन्धनजाः स्मृताः \msCa\msCb\msNa\msNb,\hskip.5em plus .9em 
वधो बन्धनजाः स्मृता \msNc,\hskip.5em plus .9em 
वधबन्धनजाः स्मृताः \msKOb\msTub,\hskip.5em plus .9em 
॰नज स्मृतः \Ed}}% 

{\devanagarifont शत्रुचौरभयैर्घोरैः सिंहव्याघ्रगजोरगैः \thinspace{\dandab} \dontdisplaylinenum }%
     \var{{\devanagarifontvar\numemph\va ॰चौरभयैर्घोरैः\lem \mssALL,\hskip.2em plus .9em 
॰चोरभयै घोरै \msNb,\hskip.5em plus .9em 
॰चौरभयै घोरैः \msTub}}% 
    \var{{\devanagarifontvar\numnoemph\vb ॰गजोरगैः\lem \mssALL,\hskip.2em plus .9em ॰ग\lac\ \msKOb}}% 

%Verse 3:21

{\devanagarifont त्रासनाद्वधमाप्नोति अन्यैर्वापि सुदुःसहैः {॥ ३:\hspace{.11em}२१॥} \veg\dontdisplaylinenum }%
     \var{{\devanagarifontvar\numnoemph\vc त्रासनाद्वध॰\lem \mssALL,\hskip.2em plus .9em \lac द्वध॰ \msTub}}% 
    \var{{\devanagarifontvar\numnoemph\vd अन्यैर्वापि\lem \mssALL,\hskip.2em plus .9em अन्ये चापि \msNc,\hskip.5em plus .9em अन्यै वापि \msKOb}}% 

{\devanagarifont यस्य यस्य हरेद्वित्तं तस्य तस्य वधः स्मृतः \thinspace{\dandab} \dontdisplaylinenum }%
     \var{{\devanagarifontvar\numemph\va हरेद्वि॰\lem \mssALL,\hskip.2em plus .9em हरे वि॰ \msNb}}% 
    \var{{\devanagarifontvar\numnoemph\vb वधः\lem \mssALL,\hskip.2em plus .9em वध \Ed}}% 

%Verse 3:22

{\devanagarifont वृत्तिजीवाभिभूतानां तद्द्वारा निहतः स्मृतः {॥ ३:\hspace{.11em}२२॥} \veg\dontdisplaylinenum }%
     \var{{\devanagarifontvar\numnoemph\vc ॰भिभूतानां\lem \mssALL,\hskip.2em plus .9em ॰विभूतानां \msNb,\hskip.5em plus .9em 
॰भिभूताना \msTub}}% 
    \var{{\devanagarifontvar\numnoemph\vd तद्द्वारा नि॰\lem \conj,\hskip.2em plus .9em तद्वारान्नि॰ \msCa\msCb\msNa\msNb\msNc\msKOb\msTub,\hskip.5em plus .9em 
तद्द्वारान्नि॰ \Ed}}% 

{\devanagarifont विषवह्निशरशस्त्रैर्मायायोगबलेन वा \thinspace{\dandab} \dontdisplaylinenum }%
     \var{{\devanagarifontvar\numemph\vab ॰शस्त्रैर्माया॰\lem \mssALL,\hskip.2em plus .9em 
॰शस्त्रै मा॰ \msNc,\hskip.5em plus .9em ॰शस्त्रैर्म्मया॰ \Ed}}% 

%Verse 3:23

{\devanagarifont हिंसकान्याहु विप्रेन्द्र मुनयस्तत्त्वदर्शिनः {॥ ३:\hspace{.11em}२३॥} \veg\dontdisplaylinenum }%
     \var{{\devanagarifontvar\numnoemph\vc हिंसकान्याहु वि॰\lem \msCb\msNb\msNc\msKOb,\hskip.2em plus .9em 
हिंसकान्याहुर्वि॰ \msCa\msNa\msTub\ \unmetr,\hskip.5em plus .9em हिंसकेत्याहु वि॰ \Ed}}% 


\alalalfejezet{अहिंसाप्रशंसा}

{\devanagarifont अहिंसा परमं धर्मं यस्त्यजेत्स दुरात्मवान् \thinspace{\dandab} \dontdisplaylinenum }%
     \var{{\devanagarifontvar\numemph\va परमं धर्मं\lem \mssALL,\hskip.2em plus .9em 
परमं धर्म \msNb,\hskip.5em plus .9em परमो धर्मं \msNc}}% 
    \var{{\devanagarifontvar\numnoemph\vb त्यजेत्स दुरात्मवान्\lem \msCb\msNc\msKOb\Ed,\hskip.2em plus .9em 
त्यजेच्छ दुरात्म\lk\ \msCa,\hskip.5em plus .9em त्यजेत्सुदुरात्मवान् \msNa,\hskip.5em plus .9em 
त्यजेत्स दुरात्मनम् \msNb,\hskip.5em plus .9em त्य\lac त्मवान् \msTub}}% 

%Verse 3:24

{\devanagarifont क्लेशायासविनिर्मुक्तं सर्वधर्मफलप्रदम् {॥ ३:\hspace{.11em}२४॥} \veg\dontdisplaylinenum }%
     \var{{\devanagarifontvar\numnoemph\vc क्लेशायासविनि॰\lem \mssALL,\hskip.2em plus .9em क्लेशा\lac\ \msKOb}}% 

{\devanagarifont नातः परतरो मूर्खो नातः परतरं तमः \thinspace{\dandab} \dontdisplaylinenum }%
     \var{{\devanagarifontvar\numemph\vb ॰तरं\lem \mssALL,\hskip.2em plus .9em ॰तन् \msCbacorr\Ed}}% 

%Verse 3:25

{\devanagarifont नातः परतरं दुःखं नातः परतरो ऽयशः {॥ ३:\hspace{.11em}२५॥} \veg\dontdisplaylinenum }%
 
{\devanagarifont नातः परतरं पापं नातः परतरं विषम् \thinspace{\dandab} \dontdisplaylinenum }%
 
%Verse 3:26

{\devanagarifont नातः परतराविद्या नातः परतरो ऽधनः {॥ ३:\hspace{.11em}२६॥} \veg\dontdisplaylinenum }%
     \var{{\devanagarifontvar\numemph\vd परतरो ऽधनः\lem \conj,\hskip.2em plus .9em परं तपोधन \msCa\msCb\msNa\msNb\msNc\msKOb,\hskip.5em plus .9em 
परतरो धन \msTub,\hskip.5em plus .9em पर तपोद्यमाः \Ed}}% 

{\devanagarifont यो हिनस्ति न भूतानि उद्भिज्जादि चतुर्विधम् \thinspace{\dandab} \dontdisplaylinenum }%
     \var{{\devanagarifontvar\numemph\va यो हिनस्ति न भूतानि\lem \msCa\msCb\msNa\msNc\msKOb,\hskip.2em plus .9em 
यो न हिन्सन्ति भूतानि \msNb,\hskip.5em plus .9em 
यो हिनस्ति \lac\ \msTub,\hskip.5em plus .9em 
यो हि नास्ति न भूतानि \Ed}}% 
    \var{{\devanagarifontvar\numnoemph\vb उद्भिज्जादि\lem \eme,\hskip.2em plus .9em उद्भिजादि \msCa\msCb\msNb\msNc\msKOb\msTub\Ed,\hskip.5em plus .9em उद्भिजानि \msNa\oo 
 ॰विधम्\lem \mssALL,\hskip.2em plus .9em ॰विधिं \msNc}}% 

%Verse 3:27

{\devanagarifont स भवेत्पुरुषः श्रेष्ठः सर्वभूतदयान्वितः {॥ ३:\hspace{.11em}२७॥} \veg\dontdisplaylinenum }%
     \var{{\devanagarifontvar\numnoemph\vc पुरुषः\lem \mssALL,\hskip.2em plus .9em पुरुष॰ \Ed}}% 
    \var{{\devanagarifontvar\numnoemph\vd सर्वभूतदया॰\lem \mssALL,\hskip.2em plus .9em \lac\ \msKOb}}% 

{\devanagarifont सर्वभूतदयां नित्यं यः करोति स पण्डितः \thinspace{\dandab} \dontdisplaylinenum }%
     \var{{\devanagarifontvar\numemph\va ॰दयां नित्यं\lem \msCa\msNa\msKOb\Ed,\hskip.2em plus .9em ॰दया नित्यं \msCb\msNb\msTub,\hskip.5em plus .9em ॰दया नित्य \msNc}}% 

%Verse 3:28

{\devanagarifont स यज्वा स तपस्वी च स दाता स दृढव्रतः {॥ ३:\hspace{.11em}२८॥} \veg\dontdisplaylinenum }%
     \var{{\devanagarifontvar\numnoemph\vc यज्वा\lem \mssALL,\hskip.2em plus .9em यज्या \msNb}}% 
    \var{{\devanagarifontvar\numnoemph\vd ॰व्रतः\lem \mssALL,\hskip.2em plus .9em ॰व्रताः \msTub}}% 

{\devanagarifont अहिंसा परमं तीर्थमहिंसा परमं तपः \thinspace{\dandab} \dontdisplaylinenum }%
     \var{{\devanagarifontvar\numemph\vab परमं तीर्थमहिंसा\lem \mssALL,\hskip.2em plus .9em परन्तीथमहिसा \msCb,\hskip.5em plus .9em 
परमं तीर्थ अहिंसा \msKOb}}% 

%Verse 3:29

{\devanagarifont अहिंसा परमं दानमहिंसा परमं सुखम् {॥ ३:\hspace{.11em}२९॥} \veg\dontdisplaylinenum }%
     \paral{{\devanagarifontsmall \vo {\englishfont This and the following verses are similar to \MBH\ 13.117.37--38} }}
    \lacuna{\devanagarifontsmall \vd {\englishfont \msCc\ resumes here in exp.\ 189, f. 273r (sic!) with} रमं सुखम.
                    {\englishfont \msTub\ breaks off here, in pāda d, after} ॰हिंसा प॰. }%
  
{\devanagarifont अहिंसा परमो यज्ञः अहिंसा परमं व्रतम् \thinspace{\dandab} \dontdisplaylinenum }%
     \var{{\devanagarifontvar\numemph\va यज्ञः\lem \msCb\msCc\msNb\msKOb\Ed,\hskip.2em plus .9em यज्ञर् \msCa,\hskip.5em plus .9em यज्ञ \msNa\msNc}}% 
    \var{{\devanagarifontvar\numnoemph\vb परमं व्रतम्\lem \mssALL,\hskip.2em plus .9em परमो व्रतम् \msKOb}}% 

%Verse 3:30

{\devanagarifont अहिंसा परमं ज्ञानमहिंसा परमा क्रिया {॥ ३:\hspace{.11em}३०॥} \veg\dontdisplaylinenum }%
     \var{{\devanagarifontvar\numnoemph\vc परमं\lem \mssALL,\hskip.2em plus .9em परमो \Ed}}% 
    \var{{\devanagarifontvar\numnoemph\vd परमा\lem \mssALL,\hskip.2em plus .9em परमां \msNb}}% 

{\devanagarifont अहिंसा परमं शौचमहिंसा परमो दमः \thinspace{\dandab} \dontdisplaylinenum }%
     \lacuna{\devanagarifontsmall \vab {\englishfont \om\ \Ed} }%
  
%Verse 3:31

{\devanagarifont अहिंसा परमो लाभ अहिंसा परमं यशः {॥ ३:\hspace{.11em}३१॥} \veg\dontdisplaylinenum }%
     \var{{\devanagarifontvar\numemph\vc लाभ\lem \msCa\msCb\msNa\msNb\Ed,\hskip.2em plus .9em लाभो \msCc,\hskip.5em plus .9em लाभः \msNc,\hskip.5em plus .9em लोभ \msKOb}}% 
    \var{{\devanagarifontvar\numnoemph\vd परमं\lem \mssALL,\hskip.2em plus .9em परमा \msNa}}% 
    \lacuna{\devanagarifontsmall \vcd {\englishfont After pādas cd, \Ed\ inserts this: }अहिंसा परमा कीर्ति अहिंसा परमो दमः,
                 {\englishfont which is not to be found in \mssCaCbCc\msNa\msNb\msNc\ (or in 
                 paper MS \msPaperA)} }%
  
{\devanagarifont अहिंसा परमो धर्म अहिंसा परमा गतिः \thinspace{\dandab} \dontdisplaylinenum }%
     \var{{\devanagarifontvar\numemph\va धर्म\lem \msCa\msCb\msKOb\Ed,\hskip.2em plus .9em धर्मो \msCc,\hskip.5em plus .9em धर्मः \msNa\msNc,\hskip.5em plus .9em ध\lacwithnum{1}\  \msNb}}% 
    \var{{\devanagarifontvar\numnoemph\vb अहिंसा परमा गतिः\lem \mssALL,\hskip.2em plus .9em \lacwithnum{8}  \msNb,\hskip.5em plus .9em 
अहिंसा परमो गतिः \Ed}}% 

%Verse 3:32

{\devanagarifont अहिंसा परमं ब्रह्म अहिंसा परमः शिवः {॥ ३:\hspace{.11em}३२॥} \veg\dontdisplaylinenum }%
     \var{{\devanagarifontvar\numnoemph\vc अहिंसा परमं ब्रह्म\lem \mssALL,\hskip.2em plus .9em 
\uncl{अहिंसा परमं ब्रह्म} \msNb,\hskip.5em plus .9em अहिंसा परंमं ब्रह्म \msNc}}% 


\alalalfejezet{मांसाहारः}

{\devanagarifont मांसाशनान्निवर्तेत मनसापि न काङ्क्षयेत् \thinspace{\dandab} \dontdisplaylinenum }%
     \var{{\devanagarifontvar\numemph\va मांसाशनान्नि॰\lem \msCa\msCb\Ed,\hskip.2em plus .9em मान्साशन नि॰ \msCc,\hskip.5em plus .9em 
मांसाशनन्नि॰ \msNa\msKOb,\hskip.5em plus .9em मन्सासनन्नि॰ \msNb,\hskip.5em plus .9em \uncl{मांसशानान्नि}॰ \msNc}}% 
    \var{{\devanagarifontvar\numnoemph\vb मनसापि\lem \mssALL,\hskip.2em plus .9em मनसपि \msKOb}}% 

%Verse 3:33

{\devanagarifont स महत्फलमाप्नोति यस्तु मांसं विवर्जयेत् {॥ ३:\hspace{.11em}३३॥} \veg\dontdisplaylinenum }%
     \var{{\devanagarifontvar\numnoemph\vd मांसं\lem \mssCaCbCc\msNa,\hskip.2em plus .9em मांस \msNb\Ed,\hskip.5em plus .9em मासं \msNc\msKOb}}% 

\pend
\endnumbering
\vfill\pagebreak\beginnumbering\pstart
\vers

{\devanagarifont स्वमांसं परमांसेन यो वर्धयितुमिच्छति \thinspace{\dandab} \dontdisplaylinenum }%
     \var{{\devanagarifontvar\numemph\va ॰मांसेन\lem \mssALL,\hskip.2em plus .9em ॰मासेन \msNc}}% 
    \var{{\devanagarifontvar\numnoemph\vb वर्धयितु॰\lem \mssALL,\hskip.2em plus .9em वर्द्धयति \msNb}}% 
    \paral{{\devanagarifontsmall \vab {\englishfont  = \MBH\ 13.116.14ab and 13.116.34ab \similar\ \UUMS\ 2.48cd:
                          }स्वमांसं परमांसेन यो देहे वृद्धिमिच्छति }}

%Verse 3:34

{\devanagarifont अनभ्यर्च्य पितॄन्देवान्न ततो ऽन्यो ऽस्ति पापकृत् {॥ ३:\hspace{.11em}३४॥} \veg\dontdisplaylinenum }%
     \var{{\devanagarifontvar\numnoemph\vc पितॄन्\lem \msCa\msCb\msNa\msNc\msKOb,\hskip.2em plus .9em पितृन् \msCc\Ed,\hskip.5em plus .9em \uncl{पितॄन्} \msNb}}% 
    \var{{\devanagarifontvar\numnoemph\vd ततो ऽन्यो\lem \mssALL,\hskip.2em plus .9em तदन्यो \Ed}}% 
    \paral{{\devanagarifontsmall \vo {\englishfont \similar\ \MANU\ 5.52 (Olivelle's edition):} 
                 स्वमांसं परमांसेन यो वर्धयितुमिच्छति\thinspace{\devanagarifontsmall ।}
                 अनभ्यर्च्य पितॄन्देवान्न ततो ऽन्यो स्त्यपुण्यकृत्\thinspace{\devanagarifontsmall ॥} }}

{\devanagarifont मधुपर्के च यज्ञे च पितृदैवतकर्मणि \thinspace{\dandab} \dontdisplaylinenum }%
     \var{{\devanagarifontvar\numemph\vb ॰दैवत॰\lem \mssALL,\hskip.2em plus .9em ॰देवत॰ \msCc\msNb}}% 

%Verse 3:35

{\devanagarifont अत्रैव पशवो हिंस्या नान्यत्र मनुरब्रवीत् {॥ ३:\hspace{.11em}३५॥} \veg\dontdisplaylinenum }%
     \var{{\devanagarifontvar\numnoemph\vc अत्रैव पशवो हिंस्या\lem \msCa\msCc\msNc\msKOb\Ed,\hskip.2em plus .9em 
अत्रैव पशवो हिंसा \msCb,\hskip.5em plus .9em अत्रैव पशवो हिंस्यान् \msNa,\hskip.5em plus .9em 
\lacwithnum{8}  \msNb}}% 
    \var{{\devanagarifontvar\numnoemph\vd नान्यत्र मनुरब्रवीत्\lem \mssALL,\hskip.2em plus .9em 
\lacwithnum{2} \uncl{त्र मनुरब्रवीत्} \msNb}}% 
    \paral{{\devanagarifontsmall \vo {\englishfont \similar\ \MANU\ 5.41 (Olivelle's edition):}
                         मधुपर्के च यज्ञे च पितृदैवतकर्मणि\thinspace{\devanagarifontsmall ।}
                         अत्रैव पशवो हिंस्या नान्यत्रेत्यब्रवीन्मनुः\thinspace{\devanagarifontsmall ॥} }}

{\devanagarifont क्रीत्वा स्वयं वाप्युत्पाद्य परोपहृतमेव वा \thinspace{\dandab} \dontdisplaylinenum }%
     \var{{\devanagarifontvar\numemph\va क्रीत्वा\lem \mssALL,\hskip.2em plus .9em कृत्वा \Ed\oo 
 ॰प्युत्पाद्य\lem \mssALL,\hskip.2em plus .9em ॰प्युत्पाद्या॰ \Ed}}% 
    \var{{\devanagarifontvar\numnoemph\vb ॰हृत॰\lem \mssALL,\hskip.2em plus .9em ॰हित॰ \Ed\oo 
 वा\lem \mssALL,\hskip.2em plus .9em च \Ed}}% 

%Verse 3:36

{\devanagarifont देवान्पितॄंश्चार्चयित्वा खादन्मांसं न दोषभाक् {॥ ३:\hspace{.11em}३६॥} \veg\dontdisplaylinenum }%
     \var{{\devanagarifontvar\numnoemph\vc पितॄंश्चार्चयित्वा\lem \mssALL,\hskip.2em plus .9em पितॄश्चार्चयित्वा \msNb,\hskip.5em plus .9em पितृश्चार्पयित्वा \Ed}}% 
    \var{{\devanagarifontvar\numnoemph\vd मांसं\lem \mssALL,\hskip.2em plus .9em मासं \msNc\msKOb}}% 
    \paral{{\devanagarifontsmall \vo {\englishfont = \MANU\ 5.32 (in Olivelle's critical edition; other editions read}
                          परोपकृत॰ {\englishfont in pāda b)} }}

{\devanagarifont वेदयज्ञतपस्तीर्थदानशीलक्रियाव्रतैः \thinspace{\dandab} \dontdisplaylinenum }%
     \var{{\devanagarifontvar\numemph\va ॰तपस्तीर्थ॰\lem \mssALL,\hskip.2em plus .9em॰svayaṃstīrtha॰ \msKOb}}% 
    \var{{\devanagarifontvar\numnoemph\vb ॰शील॰\lem \mssALL,\hskip.2em plus .9em ॰शल॰ \msCc\oo 
 ॰व्रतैः\lem \mssALL,\hskip.2em plus .9em ॰व्र\uncl{तः} \msCb}}% 

%Verse 3:37

{\devanagarifont मांसाहारनिवृत्तानां षोडशांशं न पूर्यते {॥ ३:\hspace{.11em}३७॥} \veg\dontdisplaylinenum }%
     \var{{\devanagarifontvar\numnoemph\vc ॰वृत्तानां\lem \mssCaCbCc\msNa\msNc,\hskip.2em plus .9em ॰वृत्ताना \msNb,\hskip.5em plus .9em ॰वृ\lac\ \msKOb,\hskip.5em plus .9em ॰वृत्तीनां \Ed}}% 
    \var{{\devanagarifontvar\numnoemph\vd षोडशांशं न\lem \mssALL,\hskip.2em plus .9em षोडशांशन्त \msCb,\hskip.5em plus .9em \lac\ शांशन्न \msKOb}}% 

{\devanagarifont मृगाः पर्णतृणाहारादजमेषगवादिभिः \thinspace{\dandab} \dontdisplaylinenum }%
     \var{{\devanagarifontvar\numemph\va पर्ण॰\lem \mssCaCbCc\msNb\msNc,\hskip.2em plus .9em पण्ण॰ \msNa,\hskip.5em plus .9em पर्णा॰ \msKOb\Ed}}% 
    \var{{\devanagarifontvar\numnoemph\vab ॰हाराद॰\lem \msCa\msCc\msNbpcorr\msNc\Ed,\hskip.2em plus .9em ॰हारा अ॰ \msCb\msNa\msKOb,\hskip.5em plus .9em ॰हाद॰ \msNbacorr}}% 
    \var{{\devanagarifontvar\numnoemph\vb ॰गवा॰\lem \mssALL,\hskip.2em plus .9em ॰गडा॰ \msKOb}}% 

%Verse 3:38

{\devanagarifont सुखिनो बलवन्तश्च विचरन्ति महीतले {॥ ३:\hspace{.11em}३८॥} \veg\dontdisplaylinenum }%
     \var{{\devanagarifontvar\numnoemph\vd ॰चरन्ति\lem \mssALL,\hskip.2em plus .9em ॰चरन्ती \msKObacorr}}% 

\pend
\endnumbering
\vfill\pagebreak\beginnumbering\pstart
\vers

{\devanagarifont वानराः फलमाहारा राक्षसा रुधिरप्रियाः \thinspace{\dandab} \dontdisplaylinenum }%
     \var{{\devanagarifontvar\numemph\vab ॰हारा रा॰\lem \msCb\msNa\msNb,\hskip.2em plus .9em ॰हाराद्रा॰ \msCa\msCc\msNc\msKOb\Ed}}% 

%Verse 3:39

{\devanagarifont निहता राक्षसाः सर्वे वानरैः फलभोजिभिः {॥ ३:\hspace{.11em}३९॥} \veg\dontdisplaylinenum }%
     \var{{\devanagarifontvar\numnoemph\vd ॰भोजिभिः\lem \mssALL,\hskip.2em plus .9em ॰भोगिभिः \Ed}}% 

{\devanagarifont तस्मान्मांसं न हीहेत बलकामेन भो द्विज \thinspace{\dandab} \dontdisplaylinenum }%
     \var{{\devanagarifontvar\numemph\va मांसं\lem \mssALL,\hskip.2em plus .9em मासं \msNc\msKOb}}% 
    \var{{\devanagarifontvar\numnoemph\vb हीहेत\lem \mssALL,\hskip.2em plus .9em हीयेत \msNa\msNb}}% 

%Verse 3:40

{\devanagarifont बलेन च गुणाकर्षात्परतो भयभीरुणा {॥ ३:\hspace{.11em}४०॥} \veg\dontdisplaylinenum }%
     \var{{\devanagarifontvar\numnoemph\vc गुणाकर्षा॰\lem \conjTorzsok,\hskip.2em plus .9em गुणाकाशा॰ \mssCaCbCc\msNa\msNb\msNc\msKOb,\hskip.5em plus .9em गुणा कुर्या॰ \Ed}}% 
    \var{{\devanagarifontvar\numnoemph\vd भयभीरुणा\lem \mssALL,\hskip.2em plus .9em भ\lac णा \msKOb}}% 

{\devanagarifont अहिंसकसमो नास्ति दानयज्ञसमीहया \thinspace{\dandab} \dontdisplaylinenum }%
     \var{{\devanagarifontvar\numemph\va अहिंसक॰\lem \mssALL,\hskip.2em plus .9em अहिंसका॰ \msKOb}}% 
    \var{{\devanagarifontvar\numnoemph\vb ॰यज्ञसमीहया\lem \msCa\msCb\msNa\msNb\msKOb,\hskip.2em plus .9em ॰धर्मसमीहया \msCc,\hskip.5em plus .9em 
॰यज्ञसमीहयाः \msNc,\hskip.5em plus .9em ॰धर्मसमीहय \Ed}}% 

%Verse 3:41

{\devanagarifont इह लोके यशः कीर्तिः परत्र च परा गतिः {॥ ३:\hspace{.11em}४१॥} \veg\dontdisplaylinenum }%
     \var{{\devanagarifontvar\numnoemph\vc यशः\lem \mssALL,\hskip.2em plus .9em य\uncl{शं} \msCc}}% 
    \var{{\devanagarifontvar\numnoemph\vd परा गतिः\lem \msCc\msNa\msNc\msKOb,\hskip.2em plus .9em \uncl{परा गतिः} \msCa,\hskip.5em plus .9em 
पराङ्गतिम् \msCb\msNb,\hskip.5em plus .9em परां गतिः \Ed}}% 

\ujvers\nemsloka {
{\devanagarifont त्रैलोक्यं मणिरत्नपूर्णमखिलं दत्त्वोत्तमे ब्राह्मणे }%
  \dontdisplaylinenum}    \var{{\devanagarifontvar\numemph\va त्रैलोक्यं\lem \mssALL,\hskip.2em plus .9em त्रैलोक्य \msNb\oo 
 अखिलं दत्त्वोत्तमे ब्राह्मणे\lem \mssALL,\hskip.2em plus .9em 
अ\uncl{खिलं}\lk\lk \lk\lk \lk\lk \lk\ \msCa,\hskip.5em plus .9em अखिलं दत्तोत्तमे ब्राह्मणे \msNa}}% 
    \paral{{\devanagarifontsmall \va \compare\ {\englishfont \SDHS\ 11.91:}
                    त्रैलोक्यमपि यो दद्यादखिलं रत्नपूरितम्\thinspace{\devanagarifontsmall ।}
                    चरेत्तपांसि सर्वाणि न तत्तुल्यमहिंसया\thinspace{\devanagarifontsmall ॥} }}


\nemslokab

{\devanagarifont कोटीयज्ञसहस्रपद्ममयुतं दत्त्वा महीं दक्षिणाम्  \danda\dontdisplaylinenum }%
     \var{{\devanagarifontvar\numnoemph\vb कोटीयज्ञसहस्रपद्मम्\lem \mssALL,\hskip.2em plus .9em \lk\lk \lk\lk \lk\lk \lk\lk \lk\  \msCa\oo 
 महीं\lem \mssALL,\hskip.2em plus .9em मही \msCc}}% 

\nemslokac

{\devanagarifont तीर्थानां च सहस्रकोटिनियुतं स्नात्वा सकृन्मानव }%
  \dontdisplaylinenum    \var{{\devanagarifontvar\numnoemph\vc ॰कोटि॰\lem \mssALL,\hskip.2em plus .9em ॰कोटी॰ \Ed\ \unmetr\oo 
 स्नात्वा\lem \mssALL,\hskip.2em plus .9em स्ना ऽ \msCb}}% 

%Verse 3:42


\nemslokad

{\devanagarifont एतत्पुण्यफलमहिंसकजनः प्राप्नोति निःसंशयम् {॥ ३:\hspace{.11em}४२॥} \veg\dontdisplaylinenum }%
     \var{{\devanagarifontvar\numnoemph\vd ॰फलमहिंसकजनः\lem \mssALL,\hskip.2em plus .9em ॰फलं त्वहिंसकजनः \msNc,\hskip.5em plus .9em 
फलमहिंस\lac\ \msKOb\oo 
 निःसंशयम्\lem \msKOb\Ed,\hskip.2em plus .9em \lk\lk \lk\lk\ \msCa,\hskip.5em plus .9em निःसंशय\lk\ \msCb,\hskip.5em plus .9em निःसंशयः \msCc\msNa\msNb\msNc}}% 

\vers


{\devanagarifont 
\jump
\begin{center}
\ketdanda~इति वृषसारसंग्रहे अहिंसाप्रशंसा नामाध्यायस्तृतीयः~\ketdanda
\end{center}
\dontdisplaylinenum\vers  }%
     \var{{\devanagarifontvar\numnoemph{\englishfont \Colo:} वृषसार॰\lem \mssALL,\hskip.2em plus .9em वृषार॰ \msKOb\oo 
 नामाध्यायस्तृतीयः\lem \mssALL,\hskip.2em plus .9em नामाध्यायस्तृतीय \msNc,\hskip.5em plus .9em 
नामस्तृतीयो ऽध्यायः \Ed}}% 
\bekveg\szamveg
\vfill
\phpspagebreak

\versno=0\fejno=4
\thispagestyle{empty}

\centerline{\Large\devanagarifontbold [   चतुर्थो ऽध्यायः  ]}{\vrule depth10pt width0pt} \fancyhead[CE]{{\footnotesize\devanagarifont वृषसारसंग्रहे  }}
\fancyhead[CO]{{\footnotesize\devanagarifont चतुर्थो ऽध्यायः  }}
\fancyhead[LE]{}
\fancyhead[RE]{}
\fancyhead[LO]{}
\fancyhead[RO]{}
\szam\bek



\alalfejezet{यमेषु सत्यम् (२)}
\vers


{\devanagarifont अनर्थयज्ञ उवाच {\dandab}\dontdisplaylinenum  }%
     \lacuna{\devanagarifontsmall {\englishfont Witnesses used for this chapter: \msCa\ ff.\thinspace 198v--201v, 
                                              \msCb\ ff.\thinspace 206r--208v, 
                                              \msCc\ ff.\allowbreak\thinspace 273v--277r,
                                              \msNa\ ff.\thinspace 6r--9r, 
                                              \msNb\ exp.\thinspace 48--50 (lower--upper),
                                              \msNc\ ff.\thinspace 214v--217r,
                                              \msKOb\ ff.\thinspace 214v--217v,
                                              \Ed\ pp.\thinspace 591--597;
                                        \mssCaCbCc\ = \msCa + \msCb + \msCc} }%
  
{\devanagarifont सद्भावः सत्यमित्याहुर्दृष्टप्रत्ययमेव वा \thinspace{\danda} \dontdisplaylinenum }%
     \var{{\devanagarifontvar\numemph\va सद्भावः\lem \mssALL,\hskip.2em plus .9em सद्भाव॰ \msNb\Ed}}% 
    \var{{\devanagarifontvar\numnoemph\vab सत्यमित्याहुर्दृ॰\lem \msCb\msNa\msNc\msKOb\Ed,\hskip.2em plus .9em सत्य\uncl{मि}$\-$त्याहु दृ॰ \msCa,\hskip.5em plus .9em 
सत्यमित्याहु दृ॰ \msCc,\hskip.5em plus .9em सत्यामित्याहुर्दृ॰ \msNb}}% 
    \var{{\devanagarifontvar\numnoemph\vb ॰प्रत्यय॰\lem \msCa\msCb\msNa\msNb\msKOb,\hskip.2em plus .9em ॰प्रत्य॰ \msCc,\hskip.5em plus .9em ॰प्रत्येय॰ \msNc,\hskip.5em plus .9em प्रत्यक्ष॰ \Ed}}% 
    \paral{{\devanagarifontsmall \va {\englishfont \similar\ \MBH\ 12.288.45d:} 
                         सद्भावः सत्यमुच्यते 
                    {\englishfont \compare\  also \BRAHMANDAPUR\ 3.3.86ab:}
                         असद्भावो ऽनृतं ज्ञेयं सद्भावः सत्यमुच्यते  }}

%Verse 4:1

{\devanagarifont यथाभूतार्थकथनं तत्सत्यकथनं स्मृतम् {॥ ४:\hspace{.11em}१॥} \veg\dontdisplaylinenum }%
     \var{{\devanagarifontvar\numnoemph\vc यथाभूतार्थकथनं\lem \mssALL,\hskip.2em plus .9em 
यथाभूतार्थ \msCcacorr,\hskip.5em plus .9em 
यथाभूतार्थनं क्त \msCcpcorr}}% 
    \var{{\devanagarifontvar\numnoemph\vd तत्सत्यकथनं\lem \msCa\msNa\msNb\msNc\Ed,\hskip.2em plus .9em 
तत्सत्यकथकं \msCb,\hskip.5em plus .9em 
कथनं स्मृतं \msCcacorr,\hskip.5em plus .9em 
\uncl{सत्यक ज}कथनं स्मृतं \msCcpcorr,\hskip.5em plus .9em 
\lacwithnum{2} त्यकथनं \msKObpcorr,\hskip.5em plus .9em 
\om\ \msKObacorr}}% 
    \paral{{\devanagarifontsmall \vcd {\englishfont \compare\ \SDHS\ 11.105:} 
                 स्वानुभूतं स्वदृष्टं च यः पृष्टार्थं न गूहति\thinspace{\devanagarifontsmall ।}
                 यथाभूतार्थकथनमित्येतत्सत्यलक्षणम्\thinspace{\devanagarifontsmall ॥} }}

{\devanagarifont आक्रोशताडनादीनि यः सहेत सुदुःसहम् \thinspace{\dandab} \dontdisplaylinenum }%
     \var{{\devanagarifontvar\numemph\va ॰ताडना॰\lem \mssALL,\hskip.2em plus .9em ॰नाडना॰ \msCb}}% 
    \var{{\devanagarifontvar\numnoemph\vb सुदुःसहम्\lem \mssALL,\hskip.2em plus .9em सुदुसहं \msCc}}% 

%Verse 4:2

{\devanagarifont क्षमते यो जितात्मा तु स च सत्यमुदाहृतम् {॥ ४:\hspace{.11em}२॥} \veg\dontdisplaylinenum }%
     \var{{\devanagarifontvar\numnoemph\vd सत्यमुदाहृतम्\lem \mssALL,\hskip.2em plus .9em 
\uncl{सत्य}मु\uncl{दा}हृतम् \msCa,\hskip.5em plus .9em 
स\lac तं \msKOb}}% 
    \paral{{\devanagarifontsmall \vo {\englishfont \compare\ \SDHS\ 11.82:}
                 आक्रुष्टस्ताडितो वापि यो नाक्रोशेन्न ताडयेत्\thinspace{\devanagarifontsmall ।}
                 वागाद्यविकृतः स्वस्थं क्षान्तिरेषा सुनिर्मला\thinspace{\devanagarifontsmall ॥} }}

{\devanagarifont वधार्थमुद्यतः शस्त्रं यदि पृच्छेत कर्हिचित् \thinspace{\dandab} \dontdisplaylinenum }%
     \var{{\devanagarifontvar\numemph\va ॰द्यतः\lem \mssALL,\hskip.2em plus .9em ॰द्यत \msNa\oo 
 शस्त्रं\lem \msCa\msNa\msNb\msNc\msKOb,\hskip.2em plus .9em सत्य \msCb\Ed,\hskip.5em plus .9em शस्त्र \msCc}}% 
    \var{{\devanagarifontvar\numnoemph\vb कर्हिचित्\lem \mssCaCbCc\msKOb\Ed,\hskip.2em plus .9em कर्हचित् \msNa\msNb\msNc}}% 

%Verse 4:3

{\devanagarifont न तत्र सत्यं वक्तव्यमनृतं सत्यमुच्यते {॥ ४:\hspace{.11em}३॥} \veg\dontdisplaylinenum }%
     \var{{\devanagarifontvar\numnoemph\vc सत्यं\lem \mssALL,\hskip.2em plus .9em सत्य \msCb\Ed}}% 

\pend
\endnumbering
\vfill\pagebreak\beginnumbering\pstart
\vers

{\devanagarifont वधार्हः पुरुषः कश्चिद्व्रजेत्पथि भयातुरः \thinspace{\dandab} \dontdisplaylinenum }%
     \var{{\devanagarifontvar\numemph\vb ॰तुरः\lem \mssALL,\hskip.2em plus .9em ॰तुर \msCb}}% 

%Verse 4:4

{\devanagarifont पृच्छतो ऽपि न वक्तव्यं सत्यं तद्वापि उच्यते {॥ ४:\hspace{.11em}४॥} \veg\dontdisplaylinenum }%
     \var{{\devanagarifontvar\numnoemph\vc पृच्छतो\lem \mssALL,\hskip.2em plus .9em पृच्छते \Ed}}% 
    \var{{\devanagarifontvar\numnoemph\vd तद्वापि\lem \mssALL,\hskip.2em plus .9em तदपि \msNb}}% 

\ujvers\nemsloka {
{\devanagarifont न नर्मयुक्तमनृतं हिनस्ति }%
  \dontdisplaylinenum}    \var{{\devanagarifontvar\numemph\va हिनस्ति\lem \msCa\msCb\msNb\msNc\msKOb,\hskip.2em plus .9em हि नास्ति \msCc\msNa\Ed}}% 


\nemslokab

{\devanagarifont न स्त्रीषु राजन्न विवाहकाले  \danda\dontdisplaylinenum }%
     \var{{\devanagarifontvar\numnoemph\vb राजन्न\lem \mssALL,\hskip.2em plus .9em राज न \msCc,\hskip.5em plus .9em राज्यं न \msNa}}% 

\nemslokac

{\devanagarifont प्राणात्यये सर्वधनापहारे }%
  \dontdisplaylinenum    \var{{\devanagarifontvar\numnoemph\vc ॰त्यये\lem \mssALL,\hskip.2em plus .9em ॰त्यजे \msNb\oo 
 ॰पहारे\lem \mssALL,\hskip.2em plus .9em ॰प्रहारे \msCc\msNb}}% 

%Verse 4:5


\nemslokad

{\devanagarifont पञ्चानृतं सत्यमुदाहरन्ति {॥ ४:\hspace{.11em}५॥} \veg\dontdisplaylinenum }%
     \paral{{\devanagarifontsmall \vo {\englishfont \similar\ \MBH\ 1.77.16:} न नर्मयुक्तं वचनं हिनस्ति न स्त्रीषु राजन्न विवाहकाले\thinspace{\devanagarifontsmall ।}
                                                प्राणात्यये सर्वधनापहारे पञ्चानृतान्याहुरपातकानि\thinspace{\devanagarifontsmall ॥};
                            {\englishfont \MBH\ 12.159.28:} न नर्मयुक्तं वचनं हिनस्ति न स्त्रीषु राजन्न विवाहकाले\thinspace{\devanagarifontsmall ।}
                                                न गुर्वर्थे नात्मनो जीवितार्थे पञ्चानृतान्याहुरपातकानि\thinspace{\devanagarifontsmall ॥};
                              {\englishfont \MATSP\ 31.16:} न नर्मयुक्तं वचनं हिनस्ति न स्त्रीषु राजन्न विवाहकाले\thinspace{\devanagarifontsmall ।}
         {\englishfont Abhidharmakośabhāṣya 24114--24117 (introduced by } मोहजो मृषावादो यथाह{\englishfont ):}
                                                न नर्मयुक्तमनृतं हि नास्ति न स्त्रीषु राजन्न विवाहकाले\thinspace{\devanagarifontsmall ।}
                                                प्राणात्यये सर्वधनापहारे पञ्चानृतान्याहुरपातकानि\thinspace{\devanagarifontsmall ॥} {\englishfont etc.} }}

\vers


{\devanagarifont देवमानुषतिर्येषु सत्यं धर्मः परो यतः \thinspace{\dandab} \dontdisplaylinenum }%
     \var{{\devanagarifontvar\numemph\vb ॰मानुष॰\lem \mssALL,\hskip.2em plus .9em ॰मानुष्य॰ \msNc\oo 
 सत्यं धर्मः परो यतः\lem \msCb\msCc\msKOb,\hskip.2em plus .9em सत्यं धर्मः पयतः \msCa,\hskip.5em plus .9em 
सत्यं धर्म परो यतः \msNa\msNc,\hskip.5em plus .9em सत्यधर्म परो यतः \msNb,\hskip.5em plus .9em सत्यधर्मपरायणः \Ed}}% 

%Verse 4:6

{\devanagarifont सत्यं श्रेष्ठं वरिष्ठं च सत्यं धर्मः सनातनः {॥ ४:\hspace{.11em}६॥} \veg\dontdisplaylinenum }%
     \var{{\devanagarifontvar\numnoemph\vc श्रेष्ठं\lem \mssALL,\hskip.2em plus .9em श्रेष्ठ \msNb\Ed\oo 
 वरिष्ठं च\lem \mssALL,\hskip.2em plus .9em वरिष्ठम्वरिष्ठम्वञ्च \msCbacorr}}% 
    \var{{\devanagarifontvar\numnoemph\vd सत्यं\lem \mssALL,\hskip.2em plus .9em सत्य॰ \msCb\msNb\oo 
 धर्मः\lem \msCa\msCb\msNa\msNb\msNc\msKObpcorr,\hskip.2em plus .9em धर्म \msCc\msKObacorr\Ed}}% 

{\devanagarifont सत्यं सागरमव्यक्तं सत्यमक्षयभोगदम् \thinspace{\dandab} \dontdisplaylinenum }%
     \var{{\devanagarifontvar\numemph\va सत्यं सागरमव्यक्तं\lem \mssALL,\hskip.2em plus .9em 
सत्य सागरमव्यक्तं \msCc,\hskip.5em plus .9em 
सत्यं सारमवव्यक्तं \msKOb}}% 
    \var{{\devanagarifontvar\numnoemph\vb सत्यमक्षयभोगदम्\lem \msCa\msNa\msNb\msNc\msKOb,\hskip.2em plus .9em सत्यंमक्षयभोगदम् \msCb\msCc,\hskip.5em plus .9em 
सत्यमक्षयते नरं \Ed}}% 

%Verse 4:7

{\devanagarifont सत्यं पोतः परत्रार्थं सत्यं पन्थान विस्तरम् {॥ ४:\hspace{.11em}७॥} \veg\dontdisplaylinenum }%
     \var{{\devanagarifontvar\numnoemph\vc पोतः\lem \mssALL,\hskip.2em plus .9em पोत \msNa,\hskip.5em plus .9em प्रोक्तः \Ed}}% 
    \var{{\devanagarifontvar\numnoemph\vd पन्थान विस्तरम्\lem \mssALL,\hskip.2em plus .9em यज्ज्ञानविस्तरम् \Ed}}% 

{\devanagarifont सत्यमिष्टगतिः प्रोक्तं सत्यं यज्ञमनुत्तमम् \thinspace{\dandab} \dontdisplaylinenum }%
     \var{{\devanagarifontvar\numemph\va ॰ष्टगतिः\lem \mssALL,\hskip.2em plus .9em ॰\uncl{ष्टा}गतिः \msNb}}% 
    \var{{\devanagarifontvar\numnoemph\vb सत्यं\lem \mssALL,\hskip.2em plus .9em सत्य \msKOb}}% 

%Verse 4:8

{\devanagarifont सत्यं तीर्थं परं तीर्थं सत्यं दानमनन्तकम् {॥ ४:\hspace{.11em}८॥} \veg\dontdisplaylinenum }%
     \var{{\devanagarifontvar\numnoemph\vc तीर्थं\lem \mssCaCbCc\msNa\msKOb,\hskip.2em plus .9em तीर्थ \msNb\msNc,\hskip.5em plus .9em तीर्थात् \Ed}}% 
    \var{{\devanagarifontvar\numnoemph\vd सत्यं दान॰\lem \mssALL,\hskip.2em plus .9em सत्यज्ञान॰ \msKOb}}% 

{\devanagarifont सत्यं शीलं तपो ज्ञानं सत्यं शौचं दमः शमः \thinspace{\dandab} \dontdisplaylinenum }%
     \var{{\devanagarifontvar\numemph\va सत्यं\lem \mssALL,\hskip.2em plus .9em सत्य \msCb}}% 
    \var{{\devanagarifontvar\numnoemph\vb शमः\lem \mssALL,\hskip.2em plus .9em शमम् \msNb}}% 

%Verse 4:9

{\devanagarifont सत्यं सोपानमूर्ध्वस्य सत्यं कीर्तिर्यशः सुखम् {॥ ४:\hspace{.11em}९॥} \veg\dontdisplaylinenum }%
     \var{{\devanagarifontvar\numnoemph\vc सत्यं\lem \mssALL,\hskip.2em plus .9em संत्यं \msCb,\hskip.5em plus .9em सत्य \msNc}}% 
    \var{{\devanagarifontvar\numnoemph\vd सुखम्\lem \mssALL,\hskip.2em plus .9em सुखः \Ed}}% 
    \paral{{\devanagarifontsmall \vc {\englishfont \similar\ \VARP\ 193.36cd:} सत्यं स्वर्गस्य सोपानं पारावारस्य नौरिव }}

{\devanagarifont अश्वमेधसहस्रं च सत्यं च तुलया धृतम् \thinspace{\dandab} \dontdisplaylinenum }%
     \var{{\devanagarifontvar\numemph\va ॰सहस्रं च\lem \mssALL,\hskip.2em plus .9em ॰सहस्रस्य \msCc}}% 
    \var{{\devanagarifontvar\numnoemph\vb तुलया\lem \mssALL,\hskip.2em plus .9em तुल्यया \msCc}}% 

%Verse 4:10

{\devanagarifont अश्वमेधसहस्राद्धि सत्यमेव विशिष्यते {॥ ४:\hspace{.11em}१०॥} \veg\dontdisplaylinenum }%
     \var{{\devanagarifontvar\numnoemph\vc ॰सहस्राद्धि\lem \mssALL,\hskip.2em plus .9em ॰सहस्रा हि \msCc}}% 
    \var{{\devanagarifontvar\numnoemph\vd एव\lem \mssALL,\hskip.2em plus .9em एवं \msCc\Ed}}% 
    \paral{{\devanagarifontsmall \vo {\englishfont  = \MBH\ 1.69.22 = \MBH\ Suppl. 13.20.330 = \MARKP\ 8.42 = \VDHU\ 3.265.7
                        \similar\ \MBH\ 12.156.26 (pāda d reads } सत्यमेवातिरिच्यते{\englishfont ) \similar\ \VDH\ 55.6 
                            (pāda d reads} सत्यमेतद्विशिष्यते{\englishfont )};
                    {\englishfont \compare\ \SDHS\ 11.107:}
                         अश्वमेधायुतं पूर्णं सत्यञ्च तुलितं पुरा\thinspace{\devanagarifontsmall ।}
                         अश्वमेधायुतात्सत्यमधिकं बहुभिर्गुणैः\thinspace{\devanagarifontsmall ॥} }}

{\devanagarifont सत्येन तपते सूर्यः सत्येन पृथिवी स्थिता \thinspace{\dandab} \dontdisplaylinenum }%
     \var{{\devanagarifontvar\numemph\vab सूर्यः सत्येन पृथिवी स्थिता\lem \msNa\msNc\msKOb,\hskip.2em plus .9em सू\uncl{र्यः स}त्येन पृथि स्थिताः \msCa,\hskip.5em plus .9em 
सूर्यः सत्यैन पृथिवी स्थिता \msCb,\hskip.5em plus .9em सूर्य सत्येन पृथिवी स्थिताः \msCc,\hskip.5em plus .9em 
सूर्य \uncl{सत्ये} \lacwithnum{3}  वी स्थिता \msNb,\hskip.5em plus .9em सूर्यः सत्येन पृथिवी स्थिताः \Ed}}% 

%Verse 4:11

{\devanagarifont सत्येन वायवो वान्ति सत्ये तोयं च शीतलम् {॥ ४:\hspace{.11em}११॥} \veg\dontdisplaylinenum }%
     \var{{\devanagarifontvar\numnoemph\vc वायवो\lem \mssALL,\hskip.2em plus .9em वात्यवो \msNb\oo 
 वान्ति\lem \mssALL,\hskip.2em plus .9em यान्ति \msKOb}}% 
    \var{{\devanagarifontvar\numnoemph\vd सत्ये\lem \mssALL,\hskip.2em plus .9em सत्यात् \Ed}}% 
    \paral{{\devanagarifontsmall \vo {\englishfont \similar\ \VARP\ 193.37:} 
                         सूर्यस्तपति सत्येन वातः सत्येन वाति च\thinspace{\devanagarifontsmall ।}  
                         अग्निर्दहति सत्येन सत्येन पृथिवी स्थिता\thinspace{\devanagarifontsmall ॥} 
                    {\englishfont \similar\ \VDHU\ 3.265.4cd--5ab:}
                         सत्येन वायुरभ्येति सत्येनाभासते रविः\thinspace{\devanagarifontsmall ॥} 
                         सत्येन चाग्निर्दहति स्वर्गं सत्येन गच्छति\thinspace{\devanagarifontsmall ।}  }}

{\devanagarifont तिष्ठन्ति सागराः सत्ये समयेन प्रियव्रतः \thinspace{\dandab} \dontdisplaylinenum }%
     \var{{\devanagarifontvar\numemph\va सागराः\lem \mssALL,\hskip.2em plus .9em सागरा \msCc}}% 
    \var{{\devanagarifontvar\numnoemph\vb समयेन\lem \mssALL,\hskip.2em plus .9em सत्येन च \Ed}}% 

%Verse 4:12

{\devanagarifont सत्ये तिष्ठति गोविन्दो बलिबन्धनकारणात् {॥ ४:\hspace{.11em}१२॥} \veg\dontdisplaylinenum }%
 
\pend
\endnumbering
\vfill\pagebreak\beginnumbering\pstart
\vers

{\devanagarifont अग्निर्दहति सत्येन सत्येन शशिनश्चरः \thinspace{\dandab} \dontdisplaylinenum }%
     \var{{\devanagarifontvar\numemph\vab सत्येन सत्येन\lem \mssALL,\hskip.2em plus .9em सत्येन \msNaacorr\msNc}}% 
    \var{{\devanagarifontvar\numnoemph\vb शशिनश्चरः\lem \conj,\hskip.2em plus .9em सशि\uncl{भाचरः} \msCa,\hskip.5em plus .9em 
श\uncl{सि}\lk चरः \msCb,\hskip.5em plus .9em 
स शिरा वरः \msCc,\hskip.5em plus .9em 
शशिराचरः \msNa\msNb\msNc\msKOb,\hskip.5em plus .9em 
शशिभाष्करः \Ed}}% 
    \paral{{\devanagarifontsmall \vc {\englishfont \similar\ \VARP\ 193.37cd:} 
                 अग्निर्दहति सत्येन सत्येन पृथिवी स्थिता }}
    \paral{{\devanagarifontsmall \vd {\englishfont \compare\ \VARP\ 155.30cd:}
                         सत्येन सूर्यस्तपति सोमः सत्येन राजते;
                  {\englishfont \compare\ \LAKSMINARS\  1.345.50ab:}
                         सत्येन सूर्यस्तपति चन्द्रः सत्येन वर्धते\thinspace{\devanagarifontsmall ।}
                 {\englishfont \compare\ \MBH\ Suppl. 13.587:}
                         मुचुकुन्देन मान्धात्रा हरिश्चन्द्रेण चाभिभो\thinspace{\devanagarifontsmall ।}
                         सत्यं वदत मासत्यं सत्यं धर्मः सनातनः\thinspace{\devanagarifontsmall ।}
                         हरिश्चन्द्रश्चरति वै दिवि सत्येन चन्द्रवत्\thinspace{\devanagarifontsmall ॥} }}

%Verse 4:13

{\devanagarifont सत्येन विन्ध्यास्तिष्ठन्ति वर्धमानो न वर्धते {॥ ४:\hspace{.11em}१३॥} \veg\dontdisplaylinenum }%
     \var{{\devanagarifontvar\numnoemph\vc विन्ध्यास्तिष्ठन्ति\lem \msCa\msNa\msNc\msKOb,\hskip.2em plus .9em 
विन्ध्यस्तिष्ठन्ति \msCb\msNb,\hskip.5em plus .9em विन्ध्या तिष्ठन्ति \msCc,\hskip.5em plus .9em तिष्ठते विन्ध्यो \Ed}}% 

{\devanagarifont लोकालोकः स्थितः सत्ये मेरुः सत्ये प्रतिष्ठितः \thinspace{\dandab} \dontdisplaylinenum }%
     \var{{\devanagarifontvar\numemph\va ॰लोकः\lem \Ed,\hskip.2em plus .9em ॰लोक \mssCaCbCc\msNa\msNb\msNc\msKOb\oo 
 स्थितः\lem \mssALL,\hskip.2em plus .9em स्थिः \msNc\oo 
 सत्ये\lem \mssALL,\hskip.2em plus .9em सत्यं \Ed}}% 
    \var{{\devanagarifontvar\numnoemph\vb मेरुः\lem \msCa\msCb\msNa\msNb\msNc\msKObpcorr,\hskip.2em plus .9em मेरु \msCc\msKObacorr\Ed}}% 

%Verse 4:14

{\devanagarifont वेदास्तिष्ठन्ति सत्येषु धर्मः सत्ये प्रतिष्ठति {॥ ४:\hspace{.11em}१४॥} \veg\dontdisplaylinenum }%
     \var{{\devanagarifontvar\numnoemph\vc वेदास्ति॰\lem \mssALL,\hskip.2em plus .9em देवास्ति॰ \msCb,\hskip.5em plus .9em वेदा ति॰ \Ed}}% 
    \var{{\devanagarifontvar\numnoemph\vd सत्ये\lem \mssALL,\hskip.2em plus .9em धर्मे \msCc\oo 
 प्रतिष्ठति\lem \mssALL,\hskip.2em plus .9em प्रतिष्ठिति \msNcacorr,\hskip.5em plus .9em प्रतिष्ठितः \msNcpcorr}}% 

{\devanagarifont सत्यं गौः क्षरते क्षीरं सत्यं क्षीरे घृतं स्थितम् \thinspace{\dandab} \dontdisplaylinenum }%
     \var{{\devanagarifontvar\numemph\va गौः\lem \mssALL,\hskip.2em plus .9em गौ \msCc\msNb\oo 
 क्षरते\lem \mssALL,\hskip.2em plus .9em क्षर \msKOb}}% 
    \var{{\devanagarifontvar\numnoemph\vab क्षीरं सत्यं\lem \mssALL,\hskip.2em plus .9em 
क्षीत्यं \msCbacorr,\hskip.5em plus .9em क्सी\lk  नित्यं \msCbpcorr}}% 
    \var{{\devanagarifontvar\numnoemph\vb क्षीरे घृतं स्थितम्\lem \msCa\msCb\msNa\msNc\msKOb,\hskip.2em plus .9em क्षीरं घृतं स्थितम् \msCc,\hskip.5em plus .9em क्षीरे घृत स्थितम् \msNb,\hskip.5em plus .9em 
क्षीरं स्थितं घृतम् \Ed}}% 

%Verse 4:15

{\devanagarifont सत्ये जीवः स्थितो देहे सत्यं जीवः सनातनः {॥ ४:\hspace{.11em}१५॥} \veg\dontdisplaylinenum }%
     \var{{\devanagarifontvar\numnoemph\vc सत्ये जीवः\lem \mssALL,\hskip.2em plus .9em सत्ये जीव \msNc,\hskip.5em plus .9em सत्यं जीव \Ed}}% 
    \var{{\devanagarifontvar\numnoemph\vd जीवः\lem \mssALL,\hskip.2em plus .9em जीव \msCc}}% 

{\devanagarifont सत्यमेकेन सम्प्राप्तो धर्मसाधननिश्चयः \thinspace{\dandab} \dontdisplaylinenum }%
     \var{{\devanagarifontvar\numemph\va सत्यमेकेन\lem \mssALL,\hskip.2em plus .9em सत्यमेकैन \msCb,\hskip.5em plus .9em सत्येमेकेन \msNb}}% 
    \var{{\devanagarifontvar\numnoemph\vb धर्म॰\lem \Ed,\hskip.2em plus .9em धर्मः \mssCaCbCc\msNa\msNb\msNc\msKOb\oo 
 ॰निश्चयः\lem \mssALL,\hskip.2em plus .9em ॰निश्चः \msCa}}% 

%Verse 4:16

{\devanagarifont रामराघववीर्येण सत्यमेकं सुरक्षितम् {॥ ४:\hspace{.11em}१६॥} \veg\dontdisplaylinenum }%
     \var{{\devanagarifontvar\numnoemph\vd सत्यमेकं\lem \mssALL,\hskip.2em plus .9em सत्येमेकं \msNb\oo 
 सुरक्षितम्\lem \mssALL,\hskip.2em plus .9em सुरिक्षितम् \msCb,\hskip.5em plus .9em सुरक्षितः \msNa}}% 

{\devanagarifont एवं सत्यविधानस्य कीर्तितं तव सुव्रत \thinspace{\dandab} \dontdisplaylinenum }%
     \var{{\devanagarifontvar\numemph\va एवं सत्य॰\lem \msCb,\hskip.2em plus .9em एतत्सत्य॰ \msCa\msCc\msNa\msNb\msNc\msKOb\Ed}}% 
    \var{{\devanagarifontvar\numnoemph\vb सुव्रत\lem \msCa\msNa\msNc\msKOb,\hskip.2em plus .9em सुव्रते \msCb\msNb,\hskip.5em plus .9em सुव्र\uncl{तः} \msCc,\hskip.5em plus .9em सुव्रतं \Ed}}% 

%Verse 4:17

{\devanagarifont सर्वलोकहितार्थाय किमन्यच्छ्रोतुमिच्छसि {॥ ४:\hspace{.11em}१७॥} \veg\dontdisplaylinenum }%
 

\alalfejezet{यमेष्वस्तेयम् (३)}
{\devanagarifont विगतराग उवाच {\dandab}\dontdisplaylinenum  }%
 
{\devanagarifont न हि तृप्तिं विजानामि श्रुत्वा धर्मं तवाप्यहम् \thinspace{\danda} \dontdisplaylinenum }%
     \var{{\devanagarifontvar\numemph\va तृप्तिं\lem \mssALL,\hskip.2em plus .9em तृप्ति \msCc\oo 
 विजानामि\lem \mssALL,\hskip.2em plus .9em विनामि \msNb}}% 
    \var{{\devanagarifontvar\numnoemph\vb श्रुत्वा धर्मं तवाप्यहम्\lem \mssALL,\hskip.2em plus .9em 
श्रु धर्मन्तवाप्यहम् \msCa,\hskip.5em plus .9em 
धर्मं श्रुत्वा तथाप्यहम् \Ed}}% 

%Verse 4:18

{\devanagarifont उपरिष्टादतो भूयः कथयस्व तपोधन {॥ ४:\hspace{.11em}१८॥} \veg\dontdisplaylinenum }%
     \var{{\devanagarifontvar\numnoemph\vd ॰धन\lem \msCc\msNa\msNb\msKOb\Ed,\hskip.2em plus .9em ॰धून \msCa,\hskip.5em plus .9em ॰धनः \msCb\msNc}}% 

{\devanagarifont अनर्थयज्ञ उवाच {\dandab}\dontdisplaylinenum  }%
 
{\devanagarifont स्तेयं शृण्वथ विप्रेन्द्र पञ्चधा परिकीर्तितम् \thinspace{\danda} \dontdisplaylinenum }%
     \var{{\devanagarifontvar\numemph\vb ॰कीर्तितम्\lem \mssALL,\hskip.2em plus .9em ॰कीर्त्तिताम् \msCb}}% 

{\devanagarifont अदत्तादानमादौ तु उत्कोचं च ततः परम्  \danda\dontdisplaylinenum }%
     \var{{\devanagarifontvar\numnoemph\vd उत्कोचं च ततः\lem \mssALL,\hskip.2em plus .9em त्कोच ततः \msCb,\hskip.5em plus .9em उत्कोचं चानृतः \Ed}}% 

%Verse 4:19

{\devanagarifont प्रस्थव्याजस्तुलाव्याजः प्रसह्यस्तेय पञ्चमम् {॥ ४:\hspace{.11em}१९॥} \veg\dontdisplaylinenum }%
     \var{{\devanagarifontvar\numnoemph\ve तुलाव्याजः\lem \msCb\msNc\msKOb\Ed,\hskip.2em plus .9em तुलाव्याज \msCa\msCc\msNa\msNb}}% 
    \var{{\devanagarifontvar\numnoemph\vf ॰सह्य॰\lem \mssALL,\hskip.2em plus .9em ॰सह्ये \msNb\oo 
 ॰स्तेय\lem \msCb\msCc\msNa\msNb\Ed,\hskip.2em plus .9em ॰स्तेन \msCa\msNc\msKOb\oo 
 पञ्चमम्\lem \mssALL,\hskip.2em plus .9em पञ्चमः \msCc\Ed}}% 

{\devanagarifont धृष्टदुष्टप्रभावेन परद्रव्यापकर्षणम् \thinspace{\dandab} \dontdisplaylinenum }%
     \var{{\devanagarifontvar\numemph\va धृष्टदुष्ट॰\lem \msCa\msNa\msNc\msKOb\Ed,\hskip.2em plus .9em धृष्टदुम्न॰ \msCb,\hskip.5em plus .9em धृतदुष्ट॰ \msCc,\hskip.5em plus .9em दृष्टदुष्ट॰ \msNb}}% 
    \var{{\devanagarifontvar\numnoemph\vb ॰कर्षणम्\lem \mssALL,\hskip.2em plus .9em ॰कर्षण \msNa}}% 

%Verse 4:20

{\devanagarifont वार्यमाणो ऽपि दुर्बुद्धिरदत्तादानमुच्यते {॥ ४:\hspace{.11em}२०॥} \veg\dontdisplaylinenum }%
     \var{{\devanagarifontvar\numnoemph\vc वार्यमाणो ऽपि\lem \mssALL,\hskip.2em plus .9em वार्यमानो वि॰ \msCb}}% 
    \var{{\devanagarifontvar\numnoemph\vd ॰दत्तादान॰\lem \mssALL,\hskip.2em plus .9em ॰दत्तान॰ \msKObacorr}}% 

{\devanagarifont उत्कोचं शृणु विप्रेन्द्र धर्मसंकरकारकम् \thinspace{\dandab} \dontdisplaylinenum }%
     \var{{\devanagarifontvar\numemph\va उत्कोचं\lem \mssALL,\hskip.2em plus .9em उत्कोच \msCa\oo 
 विप्रेन्द्र\lem \mssALL,\hskip.2em plus .9em विद्रेन्द्र \msNb}}% 
    \var{{\devanagarifontvar\numnoemph\vb ॰संकर॰\lem \msCc\msNa\msKOb,\hskip.2em plus .9em ॰शङ्कर॰ \msCa\msCb\msNb,\hskip.5em plus .9em ॰सकर॰ \msNc,\hskip.5em plus .9em ॰संहार॰ \Ed\oo 
 ॰कारकम्\lem \mssALL,\hskip.2em plus .9em ॰कारकः \msNa}}% 

{\devanagarifont मूल्यं कार्यविनाशार्थमुत्कोचः परिगृह्यते  \danda\dontdisplaylinenum }%
     \var{{\devanagarifontvar\numnoemph\vc मूल्यं\lem \conj,\hskip.2em plus .9em मूल \mssCaCbCc\msNa\msNb\msNc\msKOb\Ed\oo 
 ॰विनाशार्थमु॰\lem \mssALL,\hskip.2em plus .9em ॰विनार्थमु॰ \msNaacorr,\hskip.5em plus .9em 
॰विनाशाथ उ॰ \msKOb}}% 
    \var{{\devanagarifontvar\numnoemph\vd ॰त्कोचः\lem \mssALL,\hskip.2em plus .9em ॰त्कोचं \msNb,\hskip.5em plus .9em ॰त्कोच \Ed}}% 

%Verse 4:21

{\devanagarifont तेन चासौ विजानीयाद्द्रव्यलोभबलात्कृतम् {॥ ४:\hspace{.11em}२१॥} \veg\dontdisplaylinenum }%
     \var{{\devanagarifontvar\numnoemph\vef विजानीयाद्द्र॰\lem \mssALL,\hskip.2em plus .9em विजानीया द्र॰ \msCc}}% 

\pend
\endnumbering
\vfill\pagebreak\beginnumbering\pstart
\vers

{\devanagarifont प्रस्थव्याज-उपायेन कुटुम्बं त्रातुमिच्छति \thinspace{\dandab} \dontdisplaylinenum }%
 
%Verse 4:22

{\devanagarifont तं च स्तेनं विजानीयात्परद्रव्यापहारकम् {॥ ४:\hspace{.11em}२२॥} \veg\dontdisplaylinenum }%
     \var{{\devanagarifontvar\numemph\vc तं च स्तेनं\lem \msCa\msKOb,\hskip.2em plus .9em तञ्च स्तेन \msCb,\hskip.5em plus .9em 
सो ऽपि तेन \msCc\Ed,\hskip.5em plus .9em तं च स्तेयं \msNa,\hskip.5em plus .9em तञ्च तेय \msNb,\hskip.5em plus .9em तञ्च तेन \msNc}}% 
    \var{{\devanagarifontvar\numnoemph\vd ॰हारकम्\lem \msCa\msCb\msNapcorr\msNc\msKOb\Ed,\hskip.2em plus .9em ॰हारकः \msCc,\hskip.5em plus .9em ॰हारका \msNaacorr ॰हारकाः \msNb}}% 

{\devanagarifont तुलाव्याज-उपायेन परस्वार्थं हरेद्यदि \thinspace{\dandab} \dontdisplaylinenum }%
     \var{{\devanagarifontvar\numemph\va परस्वार्थं\lem \msCa\msCc\msNa\msNc,\hskip.2em plus .9em परस्वार्थ \msCb\msNb\msKOb,\hskip.5em plus .9em परस्यार्थं \Ed\oo 
 हरेद्यदि\lem \mssALL,\hskip.2em plus .9em हरेद्यति \msCb}}% 

%Verse 4:23

{\devanagarifont चौरलक्षणकाश्चान्ये कूटकापटिका नराः {॥ ४:\hspace{.11em}२३॥} \veg\dontdisplaylinenum }%
     \var{{\devanagarifontvar\numnoemph\vd कूटकापटिका\lem \msNb\msKOb,\hskip.2em plus .9em \uncl{कु}टकायटिका \msCaacorr,\hskip.5em plus .9em \uncl{कू}टकायटिका \msCapcorr,\hskip.5em plus .9em 
कूटकायटिका \msCb\msCc\msNaacorr\msNc,\hskip.5em plus .9em 
कूटकार्यटिका \msNapcorr\Ed}}% 
    \paral{{\devanagarifontsmall \vcd {\englishfont \compare\ \UMS\ 8.3cd:} कूटकापटिकाश्चैव सत्यार्जवविवर्जिताः }}

{\devanagarifont दुर्बलार्जवबालेषु च्छद्मना वा बलेन वा \thinspace{\dandab} \dontdisplaylinenum }%
     \var{{\devanagarifontvar\numemph\va ॰र्जव॰\lem \mssALL,\hskip.2em plus .9em ॰जव॰ \msNb}}% 
    \var{{\devanagarifontvar\numnoemph\vb च्छद्मना\lem \Ed,\hskip.2em plus .9em च्छन्मना \mssCaCbCc\msNa\msNb\msKOb,\hskip.5em plus .9em च्छत्माना \msNc}}% 

%Verse 4:24

{\devanagarifont अपहृत्य धनं मूढः स चौरश्चोर उच्यते {॥ ४:\hspace{.11em}२४॥} \veg\dontdisplaylinenum }%
     \var{{\devanagarifontvar\numnoemph\vcd मूढः स\lem \mssALL,\hskip.2em plus .9em मूढास्स \msNb}}% 
    \var{{\devanagarifontvar\numnoemph\vd चौरश्चोर\lem \msNc\msKOb,\hskip.2em plus .9em चोरश्चोर \msCa\msCc\msNb\Ed,\hskip.5em plus .9em चौर चोर \msCb,\hskip.5em plus .9em चौरश्चौर \msNa}}% 

{\devanagarifont नास्ति स्तेयसमं पापं नास्त्यधर्मश्च तत्समः \thinspace{\dandab} \dontdisplaylinenum }%
     \var{{\devanagarifontvar\numemph\va स्तेय॰\lem \msNa\msNc,\hskip.2em plus .9em तेन \msCa,\hskip.5em plus .9em स्तेन॰ \msCb\msCc\msNb\msKOb}}% 
    \var{{\devanagarifontvar\numnoemph\vb ॰समः\lem \mssALL,\hskip.2em plus .9em ॰समं \msCc}}% 
    \lacuna{\devanagarifontsmall \vo {\englishfont This verse is missing in \Ed.} }%
  
%Verse 4:25

{\devanagarifont नास्ति स्तेनसमाकीर्तिर्नास्ति स्तेनसमो ऽनयः {॥ ४:\hspace{.11em}२५॥} \veg\dontdisplaylinenum }%
     \var{{\devanagarifontvar\numnoemph\vc स्तेन॰\lem \mssALL,\hskip.2em plus .9em तेन \msCc,\hskip.5em plus .9em स्तेय॰ \msNc\oo 
 ॰समा॰\lem \msCb\msCc\msNb\msKOb,\hskip.2em plus .9em ॰समो \msCa\msNa\msNc}}% 
    \var{{\devanagarifontvar\numnoemph\vd स्तेन॰\lem \mssALL,\hskip.2em plus .9em स्तेय॰ \msNa\msNc}}% 

{\devanagarifont नास्ति स्तेयसमाविद्या नास्ति स्तेनसमः खलः \thinspace{\dandab} \dontdisplaylinenum }%
     \var{{\devanagarifontvar\numemph\va स्तेय॰\lem \msNa\msNc\Ed,\hskip.2em plus .9em स्तेन॰ \mssCaCbCc\msNb\msKOb\oo 
 ॰समा\lem \msCc\msNb\msKOb,\hskip.2em plus .9em ॰समो \msCa\msCb\msNa\msNc\Ed}}% 
    \var{{\devanagarifontvar\numnoemph\vb स्तेन॰\lem \mssCaCbCc\msNb\msKOb,\hskip.2em plus .9em स्तेय॰ \msNa\msNc,\hskip.5em plus .9em तेन \Ed}}% 

%Verse 4:26

{\devanagarifont नास्ति स्तेनसम अज्ञो नास्ति स्तेनसमो ऽलसः {॥ ४:\hspace{.11em}२६॥} \veg\dontdisplaylinenum }%
     \var{{\devanagarifontvar\numnoemph\vc स्तेन॰\lem \msCa\msCb\msNb\msNc\msKOb,\hskip.2em plus .9em स्तेय॰ \msCc\msNa\Ed\oo 
 ॰सम\lem \mssALL,\hskip.2em plus .9em ॰समं \msNb\oo 
 अज्ञो\lem \msCb\msKOb,\hskip.2em plus .9em अज्ञ\lk\ \msCa,\hskip.5em plus .9em अज्ञ \msCc\msNa\msNb\msNc,\hskip.5em plus .9em अज्ञः \Ed}}% 
    \var{{\devanagarifontvar\numnoemph\vd स्तेन॰\lem \msCa\msCb\msNb\msKOb,\hskip.2em plus .9em स्तेय॰ \msCc\msNa\msNc,\hskip.5em plus .9em तेन \Ed}}% 

\pend
\endnumbering
\vfill\pagebreak\beginnumbering\pstart
\vers

{\devanagarifont नास्ति स्तेनसमो द्वेष्यो नास्ति स्तेनसमो ऽप्रियः \thinspace{\dandab} \dontdisplaylinenum }%
     \var{{\devanagarifontvar\numemph\va स्तेन॰\lem \msCa\msCb\msNb,\hskip.2em plus .9em स्तेय॰ \msCc\msNa\msNc\msKOb,\hskip.5em plus .9em तेन \Ed}}% 
    \var{{\devanagarifontvar\numnoemph\vb स्तेन॰\lem \msNb,\hskip.2em plus .9em स्तेय॰ \mssCaCbCc\msNa\msNc\msKOb\Ed}}% 

%Verse 4:27

{\devanagarifont नास्ति स्तेयसमं दुःखं नास्ति स्तेयसमो ऽयशः {॥ ४:\hspace{.11em}२७॥} \veg\dontdisplaylinenum }%
     \var{{\devanagarifontvar\numnoemph\vc स्तेय॰\lem \msCc,\hskip.2em plus .9em स्तेन॰ \msCa\msCb\msNa\msNb\msKOb,\hskip.5em plus .9em स्तेन्य॰ \msNc,\hskip.5em plus .9em तेन \Ed}}% 
    \var{{\devanagarifontvar\numnoemph\vd स्तेय॰\lem \msCc\msNc,\hskip.2em plus .9em स्तेन॰ \msCa\msCb\msNa\msNb\msKOb,\hskip.5em plus .9em तेन \Ed}}% 

\nemslokalong


\ujvers\nemsloka {
{\devanagarifont प्रच्छन्नो ह्रियते ऽर्थमन्यपुरुषः प्रत्यक्षमन्यो हरेत् }%
  \dontdisplaylinenum}    \var{{\devanagarifontvar\numemph\va प्रच्छन्नो\lem \mssALL,\hskip.2em plus .9em प्रस्थन्नो \msCb\oo 
 ऽर्थमन्यपुरुषः\lem \msCb\msNc,\hskip.2em plus .9em 
वित्तम् \msCa\msNaacorr\msNb,\hskip.5em plus .9em 
चित्त \msCc,\hskip.5em plus .9em च वित्तमथवा \msNapcorr\Ed,\hskip.5em plus .9em चित्तं \msKOb\oo 
 प्रत्यक्षमन्यो\lem \mssALL,\hskip.2em plus .9em प्रत्यक्षमनो \msCb,\hskip.5em plus .9em 
प्रत्यक्ष्यमन्ये \Ed}}% 


\nemslokab

{\devanagarifont निक्षेपाद्धनहारिणो ऽन्यमधमो व्याजेन चान्यो हरेत्  \danda\dontdisplaylinenum }%
     \var{{\devanagarifontvar\numnoemph\vb निक्षेपाद्धन॰\lem \msCa\msCb\msNa\msKOb,\hskip.2em plus .9em निक्षेपा धन॰ \msCc\msNb\msNc,\hskip.5em plus .9em निक्षेपात्रय॰ \Ed\oo 
 ॰हारिणो\lem \mssALL,\hskip.2em plus .9em ॰हारिण्यो \msCb,\hskip.5em plus .9em ॰हारिणा \msNb\oo 
 ऽन्यमधमो\lem \mssALL,\hskip.2em plus .9em ऽन्यमधनो \msCc,\hskip.5em plus .9em ऽन्यविधयो \Ed\oo 
 चान्यो\lem \mssALL,\hskip.2em plus .9em चान्या \Ed\oo 
 हरेत्\lem \mssALL,\hskip.2em plus .9em हरे \msNa}}% 

\nemslokac

{\devanagarifont अन्ये लेख्यविकल्पनाहृतधना †अन्यो हृताद्वै हृता† }%
  \dontdisplaylinenum    \var{{\devanagarifontvar\numnoemph\vc अन्ये लेख्य॰\lem \corr,\hskip.2em plus .9em अन्या लेख॰ \msCb\msCc,\hskip.5em plus .9em 
अन्यो ले\uncl{ख्य}॰ \msCa,\hskip.5em plus .9em 
अन्यो लेख्य॰ \msNa\msNb\msNc\msKOb,\hskip.5em plus .9em 
अन्योल्लेख्य \Ed\oo 
 ॰धना अन्यो\lem \mssALL,\hskip.2em plus .9em ॰धन्यो \msCb,\hskip.5em plus .9em धनो अन्यो \msKOb\oo 
 हृताद्वै\lem \mssCaCbCc\msNc\Ed,\hskip.2em plus .9em हृतद्वै \msNa,\hskip.5em plus .9em हृताद्वे \msNb\msKOb}}% 

%Verse 4:28


\nemslokad

{\devanagarifont अन्यः क्रीतधनो ऽपरो धयहृत एते जघन्याः स्मृताः {॥ ४:\hspace{.11em}२८॥} \veg\dontdisplaylinenum }%
     \var{{\devanagarifontvar\numnoemph\vd अन्यः क्रीतधनो\lem \mssALL,\hskip.2em plus .9em अन्य क्रीतधनो \msNc,\hskip.5em plus .9em अनाश्रीतधनं \Ed\oo 
 ऽपरो धयहृत\lem \msCa\msCc\msNb,\hskip.2em plus .9em परो धयह्यत \msCb,\hskip.5em plus .9em परो धन\uncl{हृत} \msNa,\hskip.5em plus .9em 
परोधप्रहृत \msNc,\hskip.5em plus .9em परो ध॰ \msKOb,\hskip.5em plus .9em मदा ह्यपहृतं \Ed\oo 
 जघन्याः\lem \mssALL,\hskip.2em plus .9em जघन्यः \Ed}}% 

\ujvers\nemsloka {
{\devanagarifont स्तेनतुल्य न मूढमस्ति पुरुषो धर्मार्थहीनो ऽधमः }%
  \dontdisplaylinenum}    \var{{\devanagarifontvar\numemph\va स्तेनतुल्य\lem \msCa\msCb\msNc\msKOb\ \unmetr,\hskip.2em plus .9em स्तेयस्तुल्य \msCc,\hskip.5em plus .9em 
स्तेयतुल्य \msNa\ \unmetr,\hskip.5em plus .9em तेन तुल्य \msNb\ \unmetr,\hskip.5em plus .9em स्तेनस्तुल्य \Ed}}% 


\nemslokab

{\devanagarifont यावज्जीवति शङ्कया नरपतेः संत्रस्यमानो रटन्  \danda\dontdisplaylinenum }%
     \var{{\devanagarifontvar\numnoemph\vb यावज्जीवति\lem \mssALL,\hskip.2em plus .9em यावत्तज्जीवति \Ed\oo 
 ॰पतेः\lem \msCb\msNb\msNc,\hskip.2em plus .9em ॰पतिः \msCa\msCc\msNa\msKOb\Ed\oo 
 संत्रस्यमानो रटन्\lem \mssALL,\hskip.2em plus .9em संत्रास्यमानो शठः \Ed}}% 
    \lacuna{\devanagarifontsmall \vo {\englishfont The lower folio side in exposure 49 in \msNb\ is rather blurred and seems to be partly erased,
                        therefore all the readings in this MS for verses 4.29--46 are rather uncertain,
                        even if not indicated explicitly.} }%
  
\pend
\endnumbering
\vfill\pagebreak\beginnumbering\pstart
\vers

\nemslokac

{\devanagarifont प्राप्तःशासन तीव्रसह्यविषमं प्राप्नोति कर्मेरितः }%
  \dontdisplaylinenum    \var{{\devanagarifontvar\numnoemph\vc प्राप्तः॰\lem \mssALL,\hskip.2em plus .9em प्राप्त॰ \msNa\oo 
 ॰सह्य॰\lem \mssALL,\hskip.2em plus .9em \lacwithnum{2}  \msNb,\hskip.5em plus .9em ॰सद्य॰ \Ed\oo 
 ॰विषमं\lem \eme,\hskip.2em plus .9em ॰विषमः \mssCaCbCc\msNa\msNc\msKOb\Ed,\hskip.5em plus .9em \lacwithnum{3}  \msNb\oo 
 कर्मेरितः\lem \mssALL,\hskip.2em plus .9em कर्मे\uncl{रित} \msCa,\hskip.5em plus .9em 
\lacwithnum{2} \uncl{रितः} \msNb}}% 

%Verse 4:29


\nemslokad

{\devanagarifont कालेन म्रियते स याति निरयमाक्रन्दमानो भृशम् {॥ ४:\hspace{.11em}२९॥} \veg\dontdisplaylinenum }%
     \var{{\devanagarifontvar\numnoemph\vd निरयमाक्रन्दमानो\lem \mssCaCbCc\msNa\msKOb,\hskip.2em plus .9em \uncl{निर}यमाक्रन्दमा\uncl{नो} \msNb,\hskip.5em plus .9em 
निरयं स क्रन्दमानो \msNc,\hskip.5em plus .9em नियममाक्रन्द्रमानो \Ed}}% 

\nemslokalong


\ujvers\nemsloka {
{\devanagarifont नीत्वा दुर्गतिकल्पकोटि निरयात्तिर्यत्वमायान्ति ते }%
  \dontdisplaylinenum}    \var{{\devanagarifontvar\numemph\va ॰कल्पकोटि\lem \msCc\msNc\msKOb,\hskip.2em plus .9em ॰कोटिकल्प \msCa\msCb\msNa\msNb\Ed\oo 
 निरयात्तिर्यत्वमायान्ति ते\lem \msCb\msNa,\hskip.2em plus .9em 
निरयान्तिर्यत्वमायान्ति ते \msCa,\hskip.5em plus .9em 
निरया तिर्यत्वमायान्ति ते \msCc,\hskip.5em plus .9em 
नि\uncl{रयात्तिर्यत्व}मायान्ति ते \msNb,\hskip.5em plus .9em 
निरयान्तिर्यक्षमायान्ति ते \msNc,\hskip.5em plus .9em 
निरयान्तिर्यत्वंमायान्ति प्रते \msKOb,\hskip.5em plus .9em 
निरयान्तिर्यक्त्वमायान्ति ते \Ed}}% 


\nemslokab

{\devanagarifont तिर्यत्वे च तथैवमेकशतिकं प्रभ्रम्य वर्षार्बुदम्  \danda\dontdisplaylinenum }%
     \var{{\devanagarifontvar\numnoemph\vb तिर्यत्वे\lem \mssALL,\hskip.2em plus .9em \uncl{तिर्यत्वे} \msNb,\hskip.5em plus .9em तिर्यक्त्वं \Ed\oo 
 तथैवमेकशतिकं\lem \msCb\msKOb,\hskip.2em plus .9em तथैकमेकशतिकं \msCa\msNa\msNc,\hskip.5em plus .9em 
तथैकमेकशतिक \msCc,\hskip.5em plus .9em \uncl{तथै}कमेकशतिकं \msNb,\hskip.5em plus .9em तथैकमेकसकिकं \Ed\oo 
 ॰भ्रम्य॰\lem \mssALL,\hskip.2em plus .9em ॰भ्राम्य \msNa,\hskip.5em plus .9em \lacwithnum{1}  म्य \msNb\oo 
 वर्षार्बुदम्\lem \msNcpcorr,\hskip.2em plus .9em वर्षाम्बुदम् \msCa\msCb\msNa\msNb\msNcacorr,\hskip.5em plus .9em वर्षाम्बुदः \msCc\Ed,\hskip.5em plus .9em 
व\uncl{र्षाबु}कं \msKOb}}% 

\nemslokac

{\devanagarifont मानुष्यं तदवाप्नुवन्ति विपुले दारिद्र्यरोगाकुलं }%
  \dontdisplaylinenum    \var{{\devanagarifontvar\numnoemph\vc मानुष्यं\lem \mssALL,\hskip.2em plus .9em 
मानुष्य \msCb\ \unmetr,\hskip.5em plus .9em \uncl{मानुष्य} \msNb\ \toplost\oo 
 विपुले\lem \mssALL,\hskip.2em plus .9em विपु\uncl{ल} \msNb\ \toplost,\hskip.5em plus .9em विपुलं \Ed\oo 
 दारिद्र्यरोगा॰\lem \mssCaCbCc\msNa\msNc,\hskip.2em plus .9em \uncl{दारिद्र्यरोगा}॰ \msNb,\hskip.5em plus .9em दारिद्र्यरामा॰ \msKOb,\hskip.5em plus .9em 
दारिध्ररोगा॰ \Ed}}% 

%Verse 4:30


\nemslokad

{\devanagarifont तस्माद्दुर्गतिहेतु कर्म सकलं त्यक्त्वा शिवं चाश्रयेत् {॥ ४:\hspace{.11em}३०॥} \veg\dontdisplaylinenum }%
     \var{{\devanagarifontvar\numnoemph\vd तस्माद्दु॰\lem \mssALL,\hskip.2em plus .9em तस्मा दु॰ \msCc,\hskip.5em plus .9em \uncl{तस्मा दु}॰ \msNb\oo 
 चाश्रयेत्\lem \mssALL,\hskip.2em plus .9em चाश्रत् \msNa}}% 

\nemslokanormal



\alalfejezet{यमेष्वानृशंस्यम् (४)}
\vers


{\devanagarifont अष्टमूर्तिशिवद्वेष्टा पितुर्मातुश्च यो द्विषेत् \thinspace{\dandab} \dontdisplaylinenum }%
     \var{{\devanagarifontvar\numemph\va ॰शिव॰\lem \mssALL,\hskip.2em plus .9em ॰शिवं \msNc}}% 
    \var{{\devanagarifontvar\numnoemph\vb पितुर्मा॰\lem \mssALL,\hskip.2em plus .9em पितु मा॰ \msKOb}}% 

%Verse 4:31

{\devanagarifont गवां वा अतिथेर्द्वेष्टा नृशंसाः पञ्च एव ते {॥ ४:\hspace{.11em}३१॥} \veg\dontdisplaylinenum }%
     \var{{\devanagarifontvar\numnoemph\vc गवां वा\lem \mssALL,\hskip.2em plus .9em अवाम्वा \msCb,\hskip.5em plus .9em \lk\lk \uncl{म्वा} \msNb\oo 
 अतिथेर्द्वे॰\lem \mssALL,\hskip.2em plus .9em अतिथिद्वे॰ \msCc,\hskip.5em plus .9em अतिथे द्वे॰ \msNa}}% 
    \var{{\devanagarifontvar\numnoemph\vd नृशंसाः\lem \msCa\msCc\msNa\msNb\msKOb,\hskip.2em plus .9em नृशंसा \msCb\msNc\Ed}}% 

\pend
\endnumbering
\vfill\pagebreak\beginnumbering\pstart
\vers

{\devanagarifont अष्टमूर्तिः शिवः साक्षात्पञ्चव्योमसमन्वितः \thinspace{\dandab} \dontdisplaylinenum }%
     \var{{\devanagarifontvar\numemph\va ॰मूर्तिः\lem \mssALL,\hskip.2em plus .9em ॰मूर्ति॰ \Ed}}% 
    \var{{\devanagarifontvar\numnoemph\vb ॰न्वितः\lem \mssALL,\hskip.2em plus .9em ॰न्विताः \msCc\msNb}}% 

%Verse 4:32

{\devanagarifont सूर्यः सोमश्च दीक्षश्च दूषकः स नृशंसकः {॥ ४:\hspace{.11em}३२॥} \veg\dontdisplaylinenum }%
     \var{{\devanagarifontvar\numnoemph\vc सूर्यः\lem \mssCaCbCc\msNa\msKOb,\hskip.2em plus .9em \uncl{सूर्य}॰ \msNb\msNc,\hskip.5em plus .9em सूर्य॰ \Ed\oo 
 दीक्ष॰\lem \mssALL,\hskip.2em plus .9em \uncl{दी}\lk\ \msNb,\hskip.5em plus .9em दीक्षु॰ \Ed}}% 
    \paral{{\devanagarifontsmall \vo {\englishfont \compare\ \SDHS\ 12.17:}
                 मूर्तयो याः शिवस्याष्टौ तासु निन्दां विवर्जयेत्\thinspace{\devanagarifontsmall ।}
                 गुरोश्च शिवभक्तानां नृपसाधुतपस्विनां\thinspace{\devanagarifontsmall ॥} }}

{\devanagarifont पिताकाशसमो ज्ञेयो जन्मोत्पत्तिकरः पिता \thinspace{\dandab} \dontdisplaylinenum }%
     \var{{\devanagarifontvar\numemph\vb ॰करः पिता\lem \mssALL,\hskip.2em plus .9em ॰करपिताः \msCc,\hskip.5em plus .9em ॰\uncl{करः पिता} \msNb}}% 

%Verse 4:33

{\devanagarifont पितृदैवत†मादिश्चमानृशंस तमन्वितः† {॥ ४:\hspace{.11em}३३॥} \veg\dontdisplaylinenum }%
     \var{{\devanagarifontvar\numnoemph\vc ॰दैवत॰\lem \mssALL,\hskip.2em plus .9em ॰देवत॰ \msCb,\hskip.5em plus .9em \lk वत॰ \msNb}}% 
    \var{{\devanagarifontvar\numnoemph\vcd ॰दिश्चमानृशंस तमन्वितः\lem \msCa\msCb,\hskip.2em plus .9em 
॰दित्यमनृशंस तमन्वितः \msCc\msNb,\hskip.5em plus .9em 
॰दिश्च अनृशंस तमान्वितः \msNa,\hskip.5em plus .9em 
॰दिश्चमनृशंस तमान्वितः \msNc,\hskip.5em plus .9em 
॰दिश्चमानृशश तमान्वितः \msKOb,\hskip.5em plus .9em 
॰दित्यम्मानृशंस ततो ऽन्वितः \Ed}}% 

{\devanagarifont पृथ्व्या गुरुतरी माता को न वन्देत मातरम् \thinspace{\dandab} \dontdisplaylinenum }%
     \var{{\devanagarifontvar\numemph\va पृथ्व्या\lem \msCa\msCb\msNc,\hskip.2em plus .9em \uncl{पृथ्व्या} \msCc\msNa\msKOb,\hskip.5em plus .9em पृथ्वी \msNb,\hskip.5em plus .9em 
पृथ्व्यां \Ed}}% 
    \var{{\devanagarifontvar\numnoemph\vb वन्देत\lem \mssALL,\hskip.2em plus .9em वन्देन वन्देत \msCb,\hskip.5em plus .9em वन्द्येत \msCc}}% 

%Verse 4:34

{\devanagarifont यज्ञदानतपोवेदास्तेन सर्वं कृतं भवेत् {॥ ४:\hspace{.11em}३४॥} \veg\dontdisplaylinenum }%
     \var{{\devanagarifontvar\numnoemph\vd सर्वं\lem \msKOb,\hskip.2em plus .9em सर्व \mssCaCbCc\msNa\msNb\msNc\Ed}}% 

{\devanagarifont गावः पवित्रं मङ्गल्यं देवतानां च देवताः \thinspace{\dandab} \dontdisplaylinenum }%
     \var{{\devanagarifontvar\numemph\va पवित्रं\lem \mssALL,\hskip.2em plus .9em \uncl{पवित्र} \msNb\oo 
 मङ्गल्यं\lem \msCa\msCb\msNa\msKOb,\hskip.2em plus .9em माङ्गल्यं \msCc\msNc\Ed,\hskip.5em plus .9em \uncl{मङ्गल्यं} \msNb\oo 
 देवताः\lem \mssCaCbCc\msNc\msKOb,\hskip.2em plus .9em दैवताः \msNa,\hskip.5em plus .9em \uncl{देवताः} \msNb,\hskip.5em plus .9em देवता \Ed}}% 
    \paral{{\devanagarifontsmall \va {\englishfont \similar\ \VISNUS\ 23.57c:} गावः पवित्रमङ्गल्यं (गोषु लोकाः प्रतिष्ठिता)\oo
                 {\englishfont \compare\ also \MBH\ Suppl. 13.15.33:} गावः पवित्रं परमं गोषु लोकाः प्रतिष्ठिताः 
                 {\englishfont and \AGNIP\ 291.1cd:} गावः पवित्रा माङ्गल्या गोषु लोकाः प्रतिष्ठिताः }}

%Verse 4:35

{\devanagarifont सर्वदेवमया गावस्तस्मादेव न हिंसयेत् {॥ ४:\hspace{.11em}३५॥} \veg\dontdisplaylinenum }%
     \var{{\devanagarifontvar\numnoemph\vd ॰स्मादेव\lem \mssALL,\hskip.2em plus .9em ॰स्मादुव \msCb,\hskip.5em plus .9em ॰स्माद्गावं \Ed}}% 
    \paral{{\devanagarifontsmall \vc {\englishfont = \VDHU\ 3.291.25c} }}

{\devanagarifont जातमात्रस्य लोकस्य गावस्त्राता न संशयः \thinspace{\dandab} \dontdisplaylinenum }%
     \var{{\devanagarifontvar\numemph\va जातमात्रस्य लोकस्य\lem \msCa\msCc\msNa\msNc\msKOb\Ed,\hskip.2em plus .9em सतसातस्य \msCbacorr,\hskip.5em plus .9em 
सतसातस्य नोकस्य \msCbpcorr,\hskip.5em plus .9em 
जातमात्र\uncl{स्य लोकस्य} \msNb}}% 

%Verse 4:36

{\devanagarifont घृतं क्षीरं दधि मूत्रं शकृत्कर्षणमेव च {॥ ४:\hspace{.11em}३६॥} \veg\dontdisplaylinenum }%
     \var{{\devanagarifontvar\numnoemph\vd शकृत्क॰\lem \mssALL,\hskip.2em plus .9em क्षत्क॰ \msCb,\hskip.5em plus .9em \uncl{शकृत्क}॰ \msNb}}% 
    \paral{{\devanagarifontsmall \vo {\englishfont \compare\ \SDHU\ 12.92ff} }}

\ujvers\nemsloka {
{\devanagarifont पञ्चामृतं पञ्चपवित्रपूतं }%
  \dontdisplaylinenum}    \var{{\devanagarifontvar\numemph\va ॰पवित्रपूतम्\lem \msCc\msNa\Ed,\hskip.2em plus .9em ॰पवित्रपूतन \msCa\ \unmetr,\hskip.5em plus .9em 
॰पवित्रं \msCb,\hskip.5em plus .9em ॰पवित्रपूत \msNb,\hskip.5em plus .9em 
॰पवित्रपूतंनं \msNc\ \unmetr,\hskip.5em plus .9em ॰पवित्रं पूतनं \msKOb}}% 


\nemslokab

{\devanagarifont ये पञ्चगव्यं पुरुषाः पिबन्ति  \danda\dontdisplaylinenum }%
     \var{{\devanagarifontvar\numnoemph\vb ॰गव्यं\lem \mssALL,\hskip.2em plus .9em ॰गव्या \msCc,\hskip.5em plus .9em ॰\uncl{गव्यां} \msNb\oo 
 पुरुषाः\lem  \mssALL,\hskip.2em plus .9em पुरुषा \msCc,\hskip.5em plus .9em पुरुषः \Ed\oo 
 पिबन्ति\lem  \mssALL,\hskip.2em plus .9em विवन्ति \msCc}}% 

\nemslokac

{\devanagarifont ते वाजिमेधस्य फलं लभन्ति }%
  \dontdisplaylinenum    \var{{\devanagarifontvar\numnoemph\vc लभन्ति\lem \mssALL,\hskip.2em plus .9em भवन्ति \msCc}}% 

%Verse 4:37


\nemslokad

{\devanagarifont तदक्षयं स्वर्गमवाप्नुवन्ति {॥ ४:\hspace{.11em}३७॥} \veg\dontdisplaylinenum }%
     \var{{\devanagarifontvar\numnoemph\vd स्वर्ग॰\lem \mssALL,\hskip.2em plus .9em स्व॰ \msCb}}% 

\ujvers\nemsloka {
{\devanagarifont गोभिर्न तुल्यं धनमस्ति किंचिद् }%
  \dontdisplaylinenum}    \var{{\devanagarifontvar\numemph\va गोभिर्न तुल्यं ध॰\lem \msNc,\hskip.2em plus .9em न गोभिस्तुल्यं ध॰ \mssCaCbCc\msNa\msNb\ \unmetr,\hskip.5em plus .9em 
न गोभिस्तुन्ध॰ \msKOb,\hskip.5em plus .9em न गावतुल्यं ध॰ \Ed}}% 
    \paral{{\devanagarifontsmall \va {\englishfont = \SDHU\ 12.102d, 103d, 104d; } 
                    {\englishfont \compare\ \MBH\ 13.51.26cd:} गोभिस्तुल्यं न पश्यामि धनं किंचिदिहाच्युत }}


\nemslokab

{\devanagarifont दुह्यन्ति वाह्यन्ति बहिश्चरन्ति  \danda\dontdisplaylinenum }%
 
\nemslokac

{\devanagarifont तृणानि भुक्त्वा अमृतं स्रवन्ति }%
  \dontdisplaylinenum
%Verse 4:38


\nemslokad

{\devanagarifont विप्रेषु दत्ताः कुलमुद्धरन्ति {॥ ४:\hspace{.11em}३८॥} \veg\dontdisplaylinenum }%
     \var{{\devanagarifontvar\numnoemph\vd दत्ताः\lem \mssALL,\hskip.2em plus .9em \uncl{दत्ता} \msCc,\hskip.5em plus .9em दत्ता \Ed}}% 
    \paral{{\devanagarifontsmall \vo {\englishfont \compare\ \SDHU\ 12.92:}
                         तृणानि खादन्ति वसन्त्यरण्ये पिबन्ति तोयान्यपरिग्रहाणि\thinspace{\devanagarifontsmall ।}
                         दुह्यन्ति वाह्यन्ति पुनन्ति पापं गवां रसैर्जीवति जीवलोकः\thinspace{\devanagarifontsmall ॥} }}

\ujvers\nemsloka {
{\devanagarifont गवाह्निकं यश्च करोति नित्यं }%
  \dontdisplaylinenum}    \var{{\devanagarifontvar\numemph\va गवाह्निकं\lem \mssALL,\hskip.2em plus .9em गवांह्निकं \msCa\oo 
 यश्च करोति\lem \mssALL,\hskip.2em plus .9em यः प्रकरोति \Ed}}% 


\nemslokab

{\devanagarifont शुश्रूषणं यः कुरुते गवां तु  \danda\dontdisplaylinenum }%
     \var{{\devanagarifontvar\numnoemph\vb गवां तु\lem \msCb\msNc\msKOb,\hskip.2em plus .9em गवान्तु \msCa\msCc\msNa\msNb,\hskip.5em plus .9em गवानाम् \Ed}}% 

\nemslokac

{\devanagarifont अशेषयज्ञतपदानपुण्यं }%
  \dontdisplaylinenum    \var{{\devanagarifontvar\numnoemph\vc ॰तप॰\lem \mssALL,\hskip.2em plus .9em ॰\uncl{तप}॰ \msNb,\hskip.5em plus .9em ॰जप॰ \Ed}}% 

%Verse 4:39


\nemslokad

{\devanagarifont लभत्यसौ तामनृशंसकर्ता {॥ ४:\hspace{.11em}३९॥} \veg\dontdisplaylinenum }%
     \var{{\devanagarifontvar\numnoemph\vd लभत्यसौ तामनृशंसकर्ता\lem \eme,\hskip.2em plus .9em 
लभत्यसौ तमनृशंसकर्ता \msCb\msNa\msNb\msNc,\hskip.5em plus .9em 
लभत्यसौ भमनृशंसकर्त्ता \msCa\msKOb,\hskip.5em plus .9em 
लभत्यसौ तमनृतं स कर्त्ता \msCc,\hskip.5em plus .9em 
भवत्यसौ धर्ममशेषकर्ता \Ed}}% 

\vers


{\devanagarifont अतिथिं यो ऽनुगच्छेत अतिथिं यो ऽनुमन्यते \thinspace{\dandab} \dontdisplaylinenum }%
 
%Verse 4:40

{\devanagarifont अतिथिं यो ऽनुपूज्येत अतिथिं यः प्रशंसते {॥ ४:\hspace{.11em}४०॥} \veg\dontdisplaylinenum }%
     \var{{\devanagarifontvar\numemph\vd प्रशंसते\lem \mssALL,\hskip.2em plus .9em प्रशंस्यते \msCc}}% 

{\devanagarifont अतिथिं यो न पीड्येत अतिथिं यो न दुष्यति \thinspace{\dandab} \dontdisplaylinenum }%
     \var{{\devanagarifontvar\numemph\va न पीड्येत\lem \msCa\msCb\msNa\msKOb\Ed,\hskip.2em plus .9em न गच्छेत ({\englishfont eyeskip to 4.40c}) \msCc,\hskip.5em plus .9em 
\uncl{न पी}\lk\lk\ \msNb,\hskip.5em plus .9em निपीड्येत \msNc}}% 
    \var{{\devanagarifontvar\numnoemph\vb अतिथिं\lem \mssALL,\hskip.2em plus .9em अतिं \msCc,\hskip.5em plus .9em \lk\lk\lk\ \msNb\oo 
 न दुष्यति\lem \mssALL,\hskip.2em plus .9em नुदुष्यति \msCb,\hskip.5em plus .9em \lk\ दुष्यति \msNb}}% 

{\devanagarifont अतिथिप्रियकर्ता यः अतिथेः परिचारकः  \danda\dontdisplaylinenum }%
     \var{{\devanagarifontvar\numnoemph\vc अतिथि॰\lem \msCa\msNa,\hskip.2em plus .9em अतिथिं \msCb\msCc\msNc\msKOb\Ed,\hskip.5em plus .9em अति\uncl{थिं} \msNb\oo 
 ॰प्रिय॰\lem \mssALL,\hskip.2em plus .9em प्रियः \msCc\oo 
 यः\lem \msCb\msCc\msNb\msNc\Ed,\hskip.2em plus .9em यर् \msCa,\hskip.5em plus .9em य \msNa,\hskip.5em plus .9em या \msKOb}}% 

%Verse 4:41

{\devanagarifont अतिथिकृतसंतोषस्तस्य पुण्यमनन्तकम् {॥ ४:\hspace{.11em}४१॥} \veg\dontdisplaylinenum }%
     \var{{\devanagarifontvar\numnoemph\ve अतिथि॰\lem \msCa\msNa\msNb\msKOb,\hskip.2em plus .9em अतिथेः \msCb\msCc\msNc,\hskip.5em plus .9em अतिथिं \Ed}}% 
    \var{{\devanagarifontvar\numnoemph\vef ॰संतोषस्तस्य\lem \mssALL,\hskip.2em plus .9em ॰संता यस्य \msCb}}% 
    \var{{\devanagarifontvar\numnoemph\vf पुण्य॰\lem \mssALL,\hskip.2em plus .9em पून॰ \msNc}}% 

{\devanagarifont आसनेनार्घपात्रेण पादशौचजलेन च \thinspace{\dandab} \dontdisplaylinenum }%
     \var{{\devanagarifontvar\numemph\va ॰आर्घ॰\lem \mssALL,\hskip.2em plus .9em ॰आर्ध्य॰ \Ed\oo 
 ॰पात्रेण\lem \conj,\hskip.2em plus .9em ॰पाद्येन \mssCaCbCc\msNa\msNb\msNc\msKOb\Ed}}% 

%Verse 4:42

{\devanagarifont अन्नवस्त्रप्रदानैर्वा सर्वं वापि निवेदयेत् {॥ ४:\hspace{.11em}४२॥} \veg\dontdisplaylinenum }%
     \var{{\devanagarifontvar\numnoemph\vc अन्नव॰\lem \mssALL,\hskip.2em plus .9em अन्नम्व॰ \msCc,\hskip.5em plus .9em \uncl{अन्न}व॰ \msNb}}% 
    \var{{\devanagarifontvar\numnoemph\vd निवेदयेत्\lem \mssALL,\hskip.2em plus .9em प्रदापयेत् \Ed}}% 

{\devanagarifont पुत्रदारात्मनो वापि यो ऽतिथिमनुपूजयेत् \thinspace{\dandab} \dontdisplaylinenum }%
     \var{{\devanagarifontvar\numemph\va ॰दारात्मनो\lem \mssALL,\hskip.2em plus .9em ॰\uncl{दारा}त्मनो \msCa,\hskip.5em plus .9em ॰दारात्मको \Ed}}% 
    \var{{\devanagarifontvar\numnoemph\vb ॰पूजयेत्\lem \msCa\msNa\msKOb\Ed,\hskip.2em plus .9em ॰पूज्यते \msCb\msCc\msNb,\hskip.5em plus .9em ॰पूजते \msNc}}% 

%Verse 4:43

{\devanagarifont श्रद्धया चाविकल्पेन अक्लीबमानसेन च {॥ ४:\hspace{.11em}४३॥} \veg\dontdisplaylinenum }%
     \var{{\devanagarifontvar\numnoemph\vc श्रद्धया\lem \mssALL,\hskip.2em plus .9em श्रद्धाया \msCc\oo 
 चाविकल्पेन\lem \mssALL,\hskip.2em plus .9em चापि कल्पेन \msCa}}% 

{\devanagarifont न पृच्छेद्गोत्रचरणं स्वाध्यायं देशजन्मनी \thinspace{\dandab} \dontdisplaylinenum }%
     \var{{\devanagarifontvar\numemph\va ॰चरणं\lem \mssALL,\hskip.2em plus .9em ॰प्रवरं \Ed}}% 
    \var{{\devanagarifontvar\numnoemph\vb देशजन्मनी\lem \mssALL,\hskip.2em plus .9em देशजन्मना \msCa}}% 
    \paral{{\devanagarifontsmall  {\englishfont \vab = \UUMS\ 10.7ab = \UMS\ 6.11ab \similar\ \MBH\ 13.62.18ab:
                 }न पृच्छेद्गोत्रचरणं स्वाध्यायं देशमेव वा }}

%Verse 4:44

{\devanagarifont चिन्तयेन्मनसा भक्त्या धर्मः स्वयमिहागतः {॥ ४:\hspace{.11em}४४॥} \veg\dontdisplaylinenum }%
     \var{{\devanagarifontvar\numnoemph\vc चिन्तयेन्म॰\lem \msCa\msCc\msNa\msNb\Ed,\hskip.2em plus .9em चित्तयेत्म॰ \msCb\msKOb,\hskip.5em plus .9em चिन्तयेत्म॰ \msNc}}% 
    \var{{\devanagarifontvar\numnoemph\vd ॰गतः\lem \mssALL,\hskip.2em plus .9em ॰गताः \msCc,\hskip.5em plus .9em ग\uncl{तम्} \msNb}}% 
    \paral{{\devanagarifontsmall \vcd {\englishfont \compare\ \VSS\ 12.37cd: }द्विजरूपधरो धर्मः स्वयमेव इहागतः }}

{\devanagarifont अश्वमेधसहस्राणि राजसूयशतानि च \thinspace{\dandab} \dontdisplaylinenum }%
     \var{{\devanagarifontvar\numemph\vb ॰सूय॰\lem \msCa\msNa\msNc\msKOb\Ed,\hskip.2em plus .9em ॰सूर्य॰ \msCb\msCc,\hskip.5em plus .9em ॰सू\uncl{र्य}॰ \msNb}}% 

%Verse 4:45

{\devanagarifont पुण्डरीकसहस्रं च सर्वतीर्थतपःफलम् {॥ ४:\hspace{.11em}४५॥} \veg\dontdisplaylinenum }%
     \var{{\devanagarifontvar\numnoemph\vd ॰तपः॰\lem \mssALL,\hskip.2em plus .9em ॰तप॰ \msNc\ \unmetr}}% 

{\devanagarifont अतिथिर्यस्य तुष्येत नृशंसमतमुत्सृजेत् \thinspace{\dandab} \dontdisplaylinenum }%
     \var{{\devanagarifontvar\numemph\vb नृशंसमतमुत्सृजेत्\lem \msCa\msNa\msNc\msKOb,\hskip.2em plus .9em नृशंसमत उत्सृजेत् \msCb,\hskip.5em plus .9em 
नृशंसकमममुत्सृजेत् \msCc,\hskip.5em plus .9em नृससमतमुत्सृजेत् \msNb,\hskip.5em plus .9em न संशय समश्नुते \Ed}}% 

%Verse 4:46

{\devanagarifont स तस्य सकलं पुण्यं प्राप्नुयान्नात्र संशयः {॥ ४:\hspace{.11em}४६॥} \veg\dontdisplaylinenum }%
 
{\devanagarifont †न गतिमतिथिज्ञस्य† गतिमाप्नोति कर्हिचित् \thinspace{\dandab} \dontdisplaylinenum }%
     \var{{\devanagarifontvar\numemph\va न गतिम॰\lem \msCa\msCb\msNb\msNc\msKOb,\hskip.2em plus .9em न तिथिम॰ \msCc\Ed,\hskip.5em plus .9em न गति ना॰ \msNa}}% 
    \var{{\devanagarifontvar\numnoemph\vb कर्हिचित्\lem \msCa\msKOb\Ed,\hskip.2em plus .9em कर्हचित् \mssALL\msCb\msCc\msNa\msNb\msNc}}% 

%Verse 4:47

{\devanagarifont तस्मादतिथिमायान्तमभिगच्छेत्कृताञ्जलिः {॥ ४:\hspace{.11em}४७॥} \veg\dontdisplaylinenum }%
     \var{{\devanagarifontvar\numnoemph\vc ॰यान्त॰\lem \mssALL,\hskip.2em plus .9em ॰यान्ति॰ \msCc}}% 
    \paral{{\devanagarifontsmall \vcd {\englishfont = \VAYUP\ 2.17.8 = \BRAHMANDAPUR\ 2.15.8 
                         \similar\ \SDHU\ 4.44ab:}
                         तस्मादतिथिमायान्तमनुगच्छेत्कृताञ्जलिः }}

{\devanagarifont सक्तुप्रस्थेन चैकेन यज्ञ आसीन्महाद्भुतः \thinspace{\dandab} \dontdisplaylinenum }%
     \var{{\devanagarifontvar\numemph\va सक्तु॰\lem \eme,\hskip.2em plus .9em शन्कु॰ \msCa\msCb,\hskip.5em plus .9em शंक्तु॰ \msCc\msKOb,\hskip.5em plus .9em शक्तु॰ \msNa\msNc,\hskip.5em plus .9em शक्थु॰ \msNb,\hskip.5em plus .9em शक्ति॰ \Ed\oo 
 चैकेन\lem \mssALL,\hskip.2em plus .9em चेकेन \msNc}}% 
    \var{{\devanagarifontvar\numnoemph\vb आसीन्महाद्भुतः\lem \corr,\hskip.2em plus .9em आसीन्महद्भुतः \msCa\msCb\msNa\msNb,\hskip.5em plus .9em आसी महद्भुतः \msCc,\hskip.5em plus .9em 
आसीत्महाद्भुतः \msNc,\hskip.5em plus .9em आसीत् महद्भुतम् \msKOb,\hskip.5em plus .9em आसीन्महद्भुतम् \Ed}}% 

%Verse 4:48

{\devanagarifont अतिथिप्राप्तदानेन स्वशरीरं दिवं गतम् {॥ ४:\hspace{.11em}४८॥} \veg\dontdisplaylinenum }%
     \var{{\devanagarifontvar\numnoemph\vc ॰दानेन\lem \mssALL,\hskip.2em plus .9em ॰प्रादानेन \msCc}}% 
    \var{{\devanagarifontvar\numnoemph\vd स्व॰\lem \mssALL,\hskip.2em plus .9em \uncl{स}॰ \msNc,\hskip.5em plus .9em स॰ \Ed\oo 
 ॰गतम्\lem \mssALL,\hskip.2em plus .9em ॰गतः \msCc}}% 

{\devanagarifont नकुलेन पुराधीतं विस्तरेण द्विजोत्तम \thinspace{\dandab} \dontdisplaylinenum }%
     \var{{\devanagarifontvar\numemph\va ॰धीतं\lem \mssALL,\hskip.2em plus .9em ॰धीत \msKOb}}% 
    \var{{\devanagarifontvar\numnoemph\vb ॰त्तम\lem \mssALL,\hskip.2em plus .9em ॰त्तमम् \msCc,\hskip.5em plus .9em ॰त्तमः \Ed}}% 

%Verse 4:49

{\devanagarifont विदितं च त्वया पूर्वं प्रस्थवार्त्ता च कीर्तिता {॥ ४:\hspace{.11em}४९॥} \veg\dontdisplaylinenum }%
     \var{{\devanagarifontvar\numnoemph\vd प्रस्थवा॰\lem \mssALL,\hskip.2em plus .9em प्रस्थम्वा \msKOb\oo 
 कीर्तिता\lem \mssALL,\hskip.2em plus .9em कीर्तितम् \msCc,\hskip.5em plus .9em कीर्तिताः \Ed}}% 


\alalfejezet{यमेषु दमः (५)}
{\devanagarifont दम एव मनुष्याणां धर्मसारसमुच्चयः \thinspace{\dandab} \dontdisplaylinenum }%
     \var{{\devanagarifontvar\numemph\vb धर्मसार॰\lem \eme,\hskip.2em plus .9em धर्मः सार॰ \mssCaCbCc\msNa\msNb\msNc\msKOb,\hskip.5em plus .9em धर्मभार॰ \Ed}}% 
    \paral{{\devanagarifontsmall \vb {\englishfont \compare, e.g., \MBH\ Suppl. 14.4.2477: }श्रोतुमिच्छामि कार्त्स्न्येन धर्मसारसमुच्चयम् }}

%Verse 4:50

{\devanagarifont दमो धर्मो दमः स्वर्गो दमः कीर्तिर्दमः सुखम् {॥ ४:\hspace{.11em}५०॥} \veg\dontdisplaylinenum }%
     \var{{\devanagarifontvar\numnoemph\vc स्वर्गो\lem \mssALL,\hskip.2em plus .9em स्वर्ग \msCc}}% 
    \var{{\devanagarifontvar\numnoemph\vd कीर्तिर्द॰\lem \msCa\msCb\msNb\msKOb\Ed,\hskip.2em plus .9em कीर्ति द॰ \msCc\msNa\msNc}}% 

{\devanagarifont दमो यज्ञो दमस्तीर्थं दमः पुण्यं दमस्तपः \thinspace{\dandab} \dontdisplaylinenum }%
     \var{{\devanagarifontvar\numemph\va दमस्ती॰\lem \mssALL,\hskip.2em plus .9em दम ती॰ \msCb}}% 

%Verse 4:51

{\devanagarifont दमहीनमधर्मश्च दमः कामकुलप्रदः {॥ ४:\hspace{.11em}५१॥} \veg\dontdisplaylinenum }%
     \var{{\devanagarifontvar\numnoemph\vd दमः\lem \mssALL,\hskip.2em plus .9em दम \msCc,\hskip.5em plus .9em दमं \Ed\oo 
 काम॰\lem \mssALL,\hskip.2em plus .9em कामं \msNc}}% 

{\devanagarifont निर्दमः करि मीनश्च पतङ्गभ्रमरमृगाः \thinspace{\dandab} \dontdisplaylinenum }%
     \var{{\devanagarifontvar\numemph\va ॰दमः\lem \mssALL,\hskip.2em plus .9em ॰दम \msCc}}% 
    \var{{\devanagarifontvar\numnoemph\vb ॰भ्रमर॰\lem \mssALL,\hskip.2em plus .9em ॰भ्रम\uncl{रा}॰ \msNc}}% 

%Verse 4:52

{\devanagarifont त्वग्जिह्वा च तथा घ्राणा चक्षुः श्रवणमिन्द्रियाः {॥ ४:\hspace{.11em}५२॥} \veg\dontdisplaylinenum }%
     \var{{\devanagarifontvar\numnoemph\vc घ्राणा\lem \msCa\msNa\msNb\msNc\Ed,\hskip.2em plus .9em घ्राणं \msCb,\hskip.5em plus .9em घ्राण \msCc,\hskip.5em plus .9em घ्राणाः \msKOb}}% 
    \var{{\devanagarifontvar\numnoemph\vd चक्षुः\lem \mssALL,\hskip.2em plus .9em चक्षु \msKOb\oo 
 ॰न्द्रियाः\lem \mssALL,\hskip.2em plus .9em ॰न्द्रियः \Ed}}% 

{\devanagarifont दुर्जयेन्द्रियमेकैकं सर्वे प्राणहराः स्मृताः \thinspace{\dandab} \dontdisplaylinenum }%
     \var{{\devanagarifontvar\numemph\vb सर्वे\lem \mssALL,\hskip.2em plus .9em सर्व॰ \msCb\oo 
 ॰हराः\lem \mssALL,\hskip.2em plus .9em ॰हरा \Ed}}% 

%Verse 4:53

{\devanagarifont दमं यो जयते ऽसम्यग्निर्दमो निधनं व्रजेत् {॥ ४:\hspace{.11em}५३॥} \veg\dontdisplaylinenum }%
     \var{{\devanagarifontvar\numnoemph\vd व्रजेत्\lem \mssALL,\hskip.2em plus .9em व्रजे\lacwithnum{1}\  \msCa}}% 

{\devanagarifont मृगे श्रोत्रवशान्मृत्युः पतङ्गाश्चक्षुषोर्मृताः \thinspace{\dandab} \dontdisplaylinenum }%
     \var{{\devanagarifontvar\numemph\va मृगे\lem \mssALL,\hskip.2em plus .9em मृगो \msNb\Ed\oo 
 श्रोत्र॰\lem \msCa\msCb\msNa\msNb\Ed,\hskip.2em plus .9em शोत्र॰ \msCc,\hskip.5em plus .9em श्रोत॰ \msNc\msKOb\oo 
 ॰वशा॰\lem \mssALL,\hskip.2em plus .9em ॰वचशा॰ \msCb}}% 
    \var{{\devanagarifontvar\numnoemph\vb पतङ्गाश्च॰\lem \mssALL,\hskip.2em plus .9em पतङ्गा च॰ \Ed\oo 
 ॰षोर्मृताः\lem \mssALL,\hskip.2em plus .9em ॰सो मृताः \msCc,\hskip.5em plus .9em ॰षो मृताः \msNc}}% 

%Verse 4:54

{\devanagarifont घ्राणया भ्रमरो नष्टो नष्टो मीनश्च जिह्वया {॥ ४:\hspace{.11em}५४॥} \veg\dontdisplaylinenum }%
     \var{{\devanagarifontvar\numnoemph\vc घ्राणया\lem \mssALL,\hskip.2em plus .9em घ्रातया \msCb}}% 
    \var{{\devanagarifontvar\numnoemph\vcd नष्टो नष्टो\lem \mssALL,\hskip.2em plus .9em नष्टो \msCb}}% 
    \paral{{\devanagarifontsmall \vo {\englishfont \compare\ \BUDDHACARITA\ 11.35:} 
                गीतैर्ह्रियन्ते हि मृगा वधाय रूपार्थमग्नौ शलभाः पतन्ति\thinspace{\devanagarifontsmall ।} 
                मत्स्यो गिरत्यायसमामिषार्थी तस्मादनर्थं विषयाः फलन्ति\thinspace{\devanagarifontsmall ॥} }}

{\devanagarifont स्पर्शेन च करी नष्टो बन्धनावासदुःसहः \thinspace{\dandab} \dontdisplaylinenum }%
     \var{{\devanagarifontvar\numemph\vb ॰सदुःसहः\lem \mssALL,\hskip.2em plus .9em ॰सदुःसह \msCb,\hskip.5em plus .9em ॰सुदुस्सहः \msNb}}% 

%Verse 4:55

{\devanagarifont किं पुनः पञ्चभुक्तानां मृत्युस्तेभ्यः किमद्भुतम् {॥ ४:\hspace{.11em}५५॥} \veg\dontdisplaylinenum }%
     \var{{\devanagarifontvar\numnoemph\vc पुनः\lem \mssALL,\hskip.2em plus .9em पुन \msCaacorr}}% 
    \var{{\devanagarifontvar\numnoemph\vd तेभ्यः\lem \mssALL,\hskip.2em plus .9em तेभ्य \Ed}}% 

{\devanagarifont पुरूरवो ऽतिलोभेन अतिकामेन दण्डकः \thinspace{\dandab} \dontdisplaylinenum }%
     \var{{\devanagarifontvar\numemph\va पुरूरवो\lem \mssALL,\hskip.2em plus .9em पुरोरवे \msCc,\hskip.5em plus .9em पुरुरवा॰ \Ed}}% 
    \var{{\devanagarifontvar\numnoemph\vab तिलोभेन अतिकामेन\lem \mssALL,\hskip.2em plus .9em तिकामेन अतिलोभेन \Ed}}% 
    \var{{\devanagarifontvar\numnoemph\vb दण्डकः\lem \mssALL,\hskip.2em plus .9em पुण्डकः \Ed}}% 

%Verse 4:56

{\devanagarifont सागराश्चातिदर्पेण अतिमानेन रावणः {॥ ४:\hspace{.11em}५६॥} \veg\dontdisplaylinenum }%
     \var{{\devanagarifontvar\numnoemph\vc सागरा॰\lem \eme,\hskip.2em plus .9em सगर॰ \msCa\msCb\msNa\msNb\msNc\msKOb\Ed,\hskip.5em plus .9em सागर॰ \msCc}}% 
    \paral{{\devanagarifontsmall \vd {\englishfont \compare\ \MAHASUBHS\ 563cd:}
                         विनष्टो रावणो लौल्यादति सर्वत्र वर्जयेत् }}

{\devanagarifont अतिक्रोधेन सौदास अतिपानेन यादवाः \thinspace{\dandab} \dontdisplaylinenum }%
     \var{{\devanagarifontvar\numemph\vb ॰पानेन\lem \mssALL,\hskip.2em plus .9em ॰पादेन \msKOb,\hskip.5em plus .9em ॰पापेन \Ed}}% 

%Verse 4:57

{\devanagarifont अतितृष्णाच्च मान्धाता नहुषो द्विजवज्ञया {॥ ४:\hspace{.11em}५७॥} \veg\dontdisplaylinenum }%
     \var{{\devanagarifontvar\numnoemph\vc अतितृष्णाच्च मान्धाता\lem \conj,\hskip.2em plus .9em 
अतितृष्णा च मान्दातो \msCa,\hskip.5em plus .9em 
अतितृष्णा च मान्धातो \msCb\msCc\msNa\msNc\msKOb,\hskip.5em plus .9em 
अतितृष्णा च मन्धातो \msNb,\hskip.5em plus .9em 
अतितृष्णा च मानाच्च च \Ed}}% 
    \var{{\devanagarifontvar\numnoemph\vd नहुषो\lem \mssALL,\hskip.2em plus .9em नघुषो \msNb}}% 

{\devanagarifont अतिदानाद्बलिर्नष्ट अतिशौर्येण अर्जुनः \thinspace{\dandab} \dontdisplaylinenum }%
     \var{{\devanagarifontvar\numemph\va ॰र्नष्ट\lem \mssALL,\hskip.2em plus .9em ॰र्नष्टो \msCb,\hskip.5em plus .9em नष्टो \msCc}}% 
    \paral{{\devanagarifontsmall \va {\englishfont \compare\ \MAHASUBHS\ 563ab:}
                         अतिदानाद्बलिर्बद्धो नष्टो मानात्सुयोधनः }}

%Verse 4:58

{\devanagarifont अतिद्यूतान्नलो राजा नृगो गोहरणेन तु {॥ ४:\hspace{.11em}५८॥} \veg\dontdisplaylinenum }%
     \var{{\devanagarifontvar\numnoemph\vc ˚द्यूतान्नलो\lem \msCa\msCc\msNb\msNc\msKOb,\hskip.2em plus .9em ॰द्यूतान्नरो \msCb\msNa,\hskip.5em plus .9em ॰ख्यातान्नलो \Ed}}% 
    \var{{\devanagarifontvar\numnoemph\vd नृगो गो॰\lem \Ed,\hskip.2em plus .9em नृगङ्गो॰ \msCa\msCc\msNb\msNc\msKOb,\hskip.5em plus .9em नृगं गो॰ \msCb\msNa}}% 
    \lacuna{\devanagarifontsmall \vo {\englishfont After this verse, \Ed\ adds:} 
                        तस्माद्दम सदा स रक्षेत् अति सर्वत्र वर्जयेत्   
                {\englishfont (understand:} तस्माद्दमं सदा रक्षेद् अति सर्वत्र वर्जयेत् {\englishfont )};
                {\englishfont \compare\ \MAHASUBHS\ 563cd:}
                        विनष्टो रावणो लौल्यादति सर्वत्र वर्जयेत्  }%
  
\ujvers\nemsloka {
{\devanagarifont दमेन हीनः पुरुषो द्विजेन्द्र }%
  \dontdisplaylinenum}    \var{{\devanagarifontvar\numemph\va हीनः पुरुषो द्विजेन्द्र\lem \mssALL,\hskip.2em plus .9em 
हीन पुरुषो द्विजेन्द्र \msNb,\hskip.5em plus .9em हीनं पुरुषं द्विजेन्द्रः \Ed}}% 


\nemslokab

{\devanagarifont स्वर्गं च मोक्षं च सुखं च नास्ति  \danda\dontdisplaylinenum }%
 
\nemslokac

{\devanagarifont विज्ञानधर्मकुलकीर्तिनाश }%
  \dontdisplaylinenum    \var{{\devanagarifontvar\numnoemph\vc ॰नाश\lem \msCb\msKOb,\hskip.2em plus .9em ॰नाम \msCa\msCc\msNa,\hskip.5em plus .9em ॰नश्च \msNb,\hskip.5em plus .9em ॰नागा \msNc,\hskip.5em plus .9em ॰नाशो \Ed}}% 

%Verse 4:59


\nemslokad

{\devanagarifont भवन्ति विप्र दमया विहीनाः {॥ ४:\hspace{.11em}५९॥} \veg\dontdisplaylinenum }%
     \var{{\devanagarifontvar\numnoemph\vd विप्र\lem \mssALL,\hskip.2em plus .9em विप्रा \msNapcorr\msNc\oo 
 दमया\lem \mssALL,\hskip.2em plus .9em दया \msCbacorr}}% 


\alalfejezet{यमेषु घृणा (६)}
\vers


{\devanagarifont निर्घृणो न परत्रास्ति निर्घृणो न इहास्ति वै \thinspace{\dandab} \dontdisplaylinenum }%
     \var{{\devanagarifontvar\numemph\va निर्घृणो\lem \msCa\msCb\msNb\msKOb,\hskip.2em plus .9em निघृणो \msCc\msNc,\hskip.5em plus .9em निर्घृण \msNaacorr,\hskip.5em plus .9em 
निर्घृ\uncl{णे} \msNapcorr,\hskip.5em plus .9em निर्घृणे \Ed}}% 
    \var{{\devanagarifontvar\numnoemph\vb निर्घृणो\lem \msCa\msCb\msNaacorr\msNb\msKOb,\hskip.2em plus .9em निघृणो \msCc\msNc,\hskip.5em plus .9em निर्घृणे \msNapcorr\Ed}}% 

%Verse 4:60

{\devanagarifont निर्घृणे न च धर्मो ऽस्ति निर्घृणे न तपो ऽस्ति वै {॥ ४:\hspace{.11em}६०॥} \veg\dontdisplaylinenum }%
     \var{{\devanagarifontvar\numnoemph\vc निर्घृणे\lem \msCa\msCb\msNb\msKOb\Ed,\hskip.2em plus .9em निघृणे \msCc\msNa\msNc}}% 
    \var{{\devanagarifontvar\numnoemph\vd निर्घृणे\lem \mssALL,\hskip.2em plus .9em निघृणे \msCc\msNc}}% 

{\devanagarifont परस्त्रीषु परार्थेषु परजीवापकर्षणे \thinspace{\dandab} \dontdisplaylinenum }%
     \var{{\devanagarifontvar\numemph\vb ॰जीवापकर्षणे\lem \mssALL,\hskip.2em plus .9em ॰जीवापर्कणे \msCb,\hskip.5em plus .9em ॰जीवोपकर्षणे \Ed}}% 

%Verse 4:61

{\devanagarifont परनिन्दापरान्नेषु घृणां पञ्चसु कारयेत् {॥ ४:\hspace{.11em}६१॥} \veg\dontdisplaylinenum }%
     \var{{\devanagarifontvar\numnoemph\vc परनिन्दा॰\lem \mssALL,\hskip.2em plus .9em परनि$\-$न्द\lk\ \msCa\oo 
 ॰परान्नेषु\lem \mssALL,\hskip.2em plus .9em ॰परांनेषु \msNb}}% 
    \var{{\devanagarifontvar\numnoemph\vd घृणां\lem \msCa\msCb\msNa\msNc\msKOb,\hskip.2em plus .9em घृणा \msCc\msNb\Ed}}% 

\pend
\endnumbering
\vfill\pagebreak\beginnumbering\pstart
\vers

{\devanagarifont परस्त्री शृणु विप्रेन्द्र घृणीकार्या सदा बुधैः \thinspace{\dandab} \dontdisplaylinenum }%
     \var{{\devanagarifontvar\numemph\va घृणी॰\lem \mssALL,\hskip.2em plus .9em घृणा \msCb}}% 

%Verse 4:62

{\devanagarifont राज्ञी विप्री परिव्राजा स्वयोनिपरयोनिषु {॥ ४:\hspace{.11em}६२॥} \veg\dontdisplaylinenum }%
     \var{{\devanagarifontvar\numnoemph\vc ॰व्राजा\lem \mssCaCbCc\msNc\msKOb,\hskip.2em plus .9em ॰व्राजी \msNa\msNb,\hskip.5em plus .9em ॰व्राज्या \Ed}}% 
    \var{{\devanagarifontvar\numnoemph\vd ॰पर॰\lem \mssALL,\hskip.2em plus .9em ॰पशु॰ \msNb}}% 

{\devanagarifont परार्थे शृणु भूयो ऽन्य अन्यायार्थमुपार्जनम् \thinspace{\dandab} \dontdisplaylinenum }%
     \var{{\devanagarifontvar\numemph\va भूयो ऽन्य\lem \mssALL,\hskip.2em plus .9em भूयो \msKObacorr}}% 
    \var{{\devanagarifontvar\numnoemph\vb अन्याया॰\lem \mssALL,\hskip.2em plus .9em अन्यया॰ \msNb\oo 
 ॰र्जनम्\lem \mssALL,\hskip.2em plus .9em ॰र्ज्जवम् \msNb}}% 
    \paral{{\devanagarifontsmall \vb {\englishfont \compare\ \BHG\ 16.12:}
                 आशापाशशतैर्बद्धाः कामक्रोधपरायणाः\thinspace{\devanagarifontsmall ।}
                 ईहन्ते कामभोगार्थमन्यायेनार्थसंचयान्\thinspace{\devanagarifontsmall ॥} }}

%Verse 4:63

{\devanagarifont आढप्रस्थतुलाव्याजैः परार्थं यो ऽपकर्षति {॥ ४:\hspace{.11em}६३॥} \veg\dontdisplaylinenum }%
     \var{{\devanagarifontvar\numnoemph\vc ॰तुला॰\lem \mssALL,\hskip.2em plus .9em ॰तुल॰ \msNb}}% 
    \var{{\devanagarifontvar\numnoemph\vd ॰र्थं\lem \msCa\msCb\msNa\msKOb\Ed,\hskip.2em plus .9em ॰र्थ \msCc,\hskip.5em plus .9em ॰\uncl{र्थ} \msNb,\hskip.5em plus .9em ॰र्थे \msNc}}% 

{\devanagarifont जीवापकर्षणे विप्र घृणीकुर्वीत पण्डितः \thinspace{\dandab} \dontdisplaylinenum }%
     \var{{\devanagarifontvar\numemph\va विप्र\lem \mssALL,\hskip.2em plus .9em वि\uncl{प्र} \msCa,\hskip.5em plus .9em विप्रे \msCc}}% 
    \var{{\devanagarifontvar\numnoemph\vb घृणी॰\lem \mssALL,\hskip.2em plus .9em घृणां \Ed}}% 

%Verse 4:64

{\devanagarifont वनजावनजा जीवा विलगाश्चरणाचराः {॥ ४:\hspace{.11em}६४॥} \veg\dontdisplaylinenum }%
     \var{{\devanagarifontvar\numnoemph\vc वनजावनजा\lem \msCa\msCc\msNa\msNb\msKOb\Ed,\hskip.2em plus .9em 
वनजाव\lk जा \msCbacorr,\hskip.5em plus .9em वनजा व\uncl{नि}जा \msCbpcorr,\hskip.5em plus .9em वनज विनजा \msNc}}% 
    \var{{\devanagarifontvar\numnoemph\vd विलगाश्चरणाचराः\lem \corr,\hskip.2em plus .9em 
विलगाचरणाचराः \msCa\msCb\msNc\msKOb,\hskip.5em plus .9em विलगोचरगोचरः \msCc\Ed,\hskip.5em plus .9em विलगोचरगोचराः \msNa,\hskip.5em plus .9em 
\uncl{विलगाचर}णाचराः \msNb}}% 

{\devanagarifont परनिन्दा च का विप्र शृणु वक्ष्ये समासतः \thinspace{\dandab} \dontdisplaylinenum }%
     \var{{\devanagarifontvar\numemph\vb वक्ष्ये\lem \mssALL,\hskip.2em plus .9em वक्ष्या \Ed}}% 

%Verse 4:65

{\devanagarifont देवानां ब्राह्मणानां च गुरुमातातिथिद्विषः {॥ ४:\hspace{.11em}६५॥} \veg\dontdisplaylinenum }%
     \lacuna{\devanagarifontsmall \vcd {\englishfont These two pādas are illegible in \msNb} }%
  
{\devanagarifont परान्नेषु घृणा कार्या अभोज्येषु च भोजनम् \thinspace{\dandab} \dontdisplaylinenum }%
     \var{{\devanagarifontvar\numemph\vb अभोज्येषु\lem \mssALL,\hskip.2em plus .9em अभोज्ये \msCb}}% 

%Verse 4:66

{\devanagarifont सूतके मृतके शौण्डे वर्णभ्रष्टकुले नटे {॥ ४:\hspace{.11em}६६॥} \veg\dontdisplaylinenum }%
     \var{{\devanagarifontvar\numnoemph\vc शौण्डे\lem \msNa,\hskip.2em plus .9em सौण्ड्ये \msCa\msCc\msNc\msKOb,\hskip.5em plus .9em शोण्ड्ये \msCb,\hskip.5em plus .9em \uncl{सौण्डे} \msNb,\hskip.5em plus .9em 
सौण्डो \Ed}}% 
    \lacuna{\devanagarifontsmall \vo {\englishfont This verse is mostly illegible in \msNb} }%
  
\pend
\endnumbering
\vfill\pagebreak\beginnumbering\pstart
\vers

\nemslokalong


\ujvers\nemsloka {
{\devanagarifont एते पञ्चघृणासु सक्तपुरुषाः स्वर्गार्थमोक्षार्थिनो }%
  \dontdisplaylinenum}    \var{{\devanagarifontvar\numemph\va ॰पुरुषाः\lem \msNc,\hskip.2em plus .9em ॰पुरुषः \mssCaCbCc\msNa\msNb\msKOb\Ed\oo 
 ॰र्थिनो\lem \eme,\hskip.2em plus .9em ॰र्थिनः \msNcpcorr,\hskip.5em plus .9em ॰र्थिनां \mssCaCbCc\msNa\msNb\msKOb\Ed,\hskip.5em plus .9em 
॰र्थिना \msNcacorr}}% 


\nemslokab

{\devanagarifont लोके ऽनिन्दनमाप्नुवन्ति सततं कीर्तिर्यशोऽलंकृतम्  \danda\dontdisplaylinenum }%
     \var{{\devanagarifontvar\numnoemph\vb ऽनिन्दनमाप्नुवन्ति\lem \mssALL,\hskip.2em plus .9em 
ऽनिन्दनवाप्नुवन्ति \msCc,\hskip.5em plus .9em नन्दनवायुवान्ति \Ed}}% 

\nemslokac

{\devanagarifont प्रज्ञाबोधश्रुतिं स्मृतिं च लभते मानं च नित्यं लभेद् }%
  \dontdisplaylinenum    \var{{\devanagarifontvar\numnoemph\vc ॰श्रुतिं\lem \msNc,\hskip.2em plus .9em ॰श्रुति॰ \mssCaCbCc\msNa\msNb\msKOb\Ed\oo 
 नित्यं\lem \mssALL,\hskip.2em plus .9em नित्य \msCb}}% 

%Verse 4:67


\nemslokad

{\devanagarifont दाक्षिण्यं सभवेत्स आयुष परं प्राप्नोति निःसंशयः {॥ ४:\hspace{.11em}६७॥} \veg\dontdisplaylinenum }%
     \var{{\devanagarifontvar\numnoemph\vd स आयुष\lem \eme,\hskip.2em plus .9em समायुष \mssCaCbCc\msNc,\hskip.5em plus .9em समायुषः \msNa\msKOb\ \unmetr,\hskip.5em plus .9em 
\uncl{समायुष} \msNb,\hskip.5em plus .9em स मानुष \Ed\oo 
 निःसंशयः\lem \mssALL,\hskip.2em plus .9em निसंशयः \msNa}}% 

\nemslokanormal



\alalfejezet{यमेषु धन्यः (७)}
\vers


{\devanagarifont चतुर्मौनश्चतुःशत्रुश्चतुरायतनं तथा \thinspace{\dandab} \dontdisplaylinenum }%
     \var{{\devanagarifontvar\numemph\va चतुर्मौन॰\lem \mssALL,\hskip.2em plus .9em चतुर्मोण॰ \msCc,\hskip.5em plus .9em 
\uncl{चतुर्मौन}॰ \msNb}}% 
    \var{{\devanagarifontvar\numnoemph\vab ॰तुःशत्रुश्च॰\lem \mssALL,\hskip.2em plus .9em 
॰तुशत्रु च॰ \msCc,\hskip.5em plus .9em ॰तुःशत्रु च॰ \Ed}}% 
    \var{{\devanagarifontvar\numnoemph\vb ॰तुरायतनं\lem \mssALL,\hskip.2em plus .9em ॰\uncl{तु}रायतनं \msCa,\hskip.5em plus .9em 
॰\uncl{तुरायतनम्} \msNb}}% 

%Verse 4:68

{\devanagarifont चतुर्ध्यानं चतुष्पादं पञ्चधन्यविधोच्यते {॥ ४:\hspace{.11em}६८॥} \veg\dontdisplaylinenum }%
     \var{{\devanagarifontvar\numnoemph\vc ॰पादं\lem \mssALL,\hskip.2em plus .9em ॰पादः \msNa,\hskip.5em plus .9em \lk\lk\ \msNb}}% 
    \var{{\devanagarifontvar\numnoemph\vd पञ्चधन्य॰\lem \mssALL,\hskip.2em plus .9em धन्यपञ्च॰ \Ed}}% 

{\devanagarifont चतुर्मौनस्य वक्ष्यामि शृणुष्वावहितो भव \thinspace{\dandab} \dontdisplaylinenum }%
     \var{{\devanagarifontvar\numemph\va ॰मौनस्य\lem \mssALL,\hskip.2em plus .9em ॰मोनस्य \msCb}}% 

%Verse 4:69

{\devanagarifont पारुष्यपिशुनामिथ्या सम्भिन्नानि च वर्जयेत् {॥ ४:\hspace{.11em}६९॥} \veg\dontdisplaylinenum }%
     \var{{\devanagarifontvar\numnoemph\vc पारुष्य॰\lem \mssALL,\hskip.2em plus .9em पारुष्यं \msNa\oo 
 ॰पिशुना॰\lem \mssALL,\hskip.2em plus .9em ॰पिण्डाना॰ \Ed}}% 
    \paral{{\devanagarifontsmall \vcd {\englishfont \compare\ \DIVYAV\ 186.21:}
                     आर्य, किमेभिः कर्म कृतम्येनैवंविधानि दुःखानि प्रत्यनुभवन्तीति? 
                     स कथयति\thinspace{\devanagarifontsmall ।} एते प्राणातिपातिका अदत्तादायिकाः काममिथ्याचारिका मृषावादिकाः पैशुनिकाः पारुषिकाः 
                     संभिन्नप्रलापिका अभिध्यालवो व्यापन्नचित्ता मिथ्यादृष्टिकाः\thinspace{\devanagarifontsmall ।};
                     {\englishfont \compare\ \DHARMP\ 1.31cd--32ab:}
                         मिथ्या पिशुनसम्भिन्नपारुष्यवचनानि च\thinspace{\devanagarifontsmall ॥}
                         जल्पतः सम्भवन्त्येते तस्मान्मौनं प्रशस्यते\thinspace{\devanagarifontsmall ।} }}

{\devanagarifont कामः क्रोधश्च लोभश्च मोहश्चैव चतुर्विधः \thinspace{\dandab} \dontdisplaylinenum  }%
 
%Verse 4:70

{\devanagarifont चतुःशत्रुर्निहन्तव्यः सो ऽरिहा वीतकल्मषः {॥ ४:\hspace{.11em}७०॥} \veg\dontdisplaylinenum }%
     \var{{\devanagarifontvar\numemph\vc चतुःशत्रुर्निह॰\lem \msCa\msCb\Ed,\hskip.2em plus .9em चतुशत्रु निह॰ \msCc\msNa\msNb\msNc,\hskip.5em plus .9em 
चतुःशत्रु निर्ह॰ \msKOb}}% 
    \var{{\devanagarifontvar\numnoemph\vd सो ऽरिहा\lem \mssALL,\hskip.2em plus .9em स्रोरिहा \msCb,\hskip.5em plus .9em सर्वथा \Ed\oo 
 वीत॰\lem \mssALL,\hskip.2em plus .9em तीत॰ \msKOb}}% 

{\devanagarifont चतुरायतनं विप्र कथयिष्यामि तच्छृणु \thinspace{\dandab} \dontdisplaylinenum }%
 
%Verse 4:71

{\devanagarifont करुणा मुदितोपेक्षा मैत्री चायतनं स्मृतम् {॥ ४:\hspace{.11em}७१॥} \veg\dontdisplaylinenum }%
     \var{{\devanagarifontvar\numemph\vc मुदितो॰\lem \mssALL,\hskip.2em plus .9em मुदितौ॰ \Ed}}% 
    \var{{\devanagarifontvar\numnoemph\vd चायतनं\lem \mssALL,\hskip.2em plus .9em चायतन \msCa,\hskip.5em plus .9em चायत\uncl{न} \msCb}}% 

{\devanagarifont चतुर्ध्यानाधुना वक्ष्ये संसारार्णवतारणम् \thinspace{\dandab} \dontdisplaylinenum }%
 
%Verse 4:72

{\devanagarifont आत्मविद्याभवः सूक्ष्मं ध्यानमुक्तं चतुर्विधम् {॥ ४:\hspace{.11em}७२॥} \veg\dontdisplaylinenum }%
     \var{{\devanagarifontvar\numemph\vc ॰भवः\lem \msCb\msCcpcorr\msNa\msNb\msNc\msKOb,\hskip.2em plus .9em ॰भव \msCa\msCcacorr,\hskip.5em plus .9em ॰भवं \Ed}}% 
    \var{{\devanagarifontvar\numnoemph\vcd सूक्ष्मं ध्या॰\lem \msCa\msNa\msNc\msKOb\Ed,\hskip.2em plus .9em 
सूक्ष्मा\uncl{न्या}॰ \msCb,\hskip.5em plus .9em 
सू\uncl{क्ष्म}ध्या॰ \msCc,\hskip.5em plus .9em सूक्ष्मध्यान॰ \msNb}}% 
    \var{{\devanagarifontvar\numnoemph\vd ॰नमुक्तं चतुर्विधम्\lem \msCc\msNb\msKOb,\hskip.2em plus .9em ॰नमुक्तश्चतुर्विधम् \msCa,\hskip.5em plus .9em 
॰नमुक्तश्चतुर्विधः \msCb\msNa,\hskip.5em plus .9em 
॰नमुक्तं चतुर्विधिं \msNc,\hskip.5em plus .9em ॰नयज्ञश्च \Ed}}% 

{\devanagarifont आत्मतत्त्वः स्मृतो धर्मो विद्या पञ्चसु पञ्चधा \thinspace{\dandab} \dontdisplaylinenum }%
     \var{{\devanagarifontvar\numemph\va स्मृतो\lem \mssALL,\hskip.2em plus .9em स्मृता \msCc\Ed\oo 
 धर्मो\lem \mssALL,\hskip.2em plus .9em धन्या \Ed}}% 

%Verse 4:73

{\devanagarifont षट्त्रिंशाक्षरमित्याहुः सूक्ष्मतत्त्वमलक्षणम् {॥ ४:\hspace{.11em}७३॥} \veg\dontdisplaylinenum }%
     \var{{\devanagarifontvar\numnoemph\vc षट्त्रिंशा॰\lem \mssALL,\hskip.2em plus .9em षञ्चत्रिंशा॰ \msKOb}}% 
    \var{{\devanagarifontvar\numnoemph\vcd आहुः सू॰\lem \mssALL,\hskip.2em plus .9em आ\lk\lk\ \msCa}}% 

{\devanagarifont चतुष्पादः स्मृतो धर्मश्चतुराश्रममाश्रितः \thinspace{\dandab} \dontdisplaylinenum }%
     \var{{\devanagarifontvar\numemph\vab धर्मश्च॰\lem \mssALL,\hskip.2em plus .9em धर्म च॰ \msCc\msNb}}% 
    \var{{\devanagarifontvar\numnoemph\vb ॰श्रितः\lem \mssALL,\hskip.2em plus .9em ॰श्रिताः \msNc}}% 

%Verse 4:74

{\devanagarifont गृहस्थो ब्रह्मचारी च वानप्रस्थो ऽथ भैक्षुकः {॥ ४:\hspace{.11em}७४॥} \veg\dontdisplaylinenum }%
     \var{{\devanagarifontvar\numnoemph\vd भैक्षुकः\lem \mssALL,\hskip.2em plus .9em भक्षकः \Ed}}% 
    \paral{{\devanagarifontsmall \vcd {\englishfont  = \MBH\ 12.234.13ab \similar\ \MBH\ 14.4513ab etc. }
                 \vo {\englishfont \compare\ 3.4 above:}
                 श्रुतिस्मृतिद्वयोर्मूर्तिश्चतुष्पादवृषः स्थितः\thinspace{\devanagarifontsmall ।}
                 चतुराश्रम यो धर्मः कीर्तितानि मनीषिभिः\thinspace{\devanagarifontsmall ॥} }}

{\devanagarifont धन्यास्ते यैरिदं वेत्ति निखिलेन द्विजोत्तम \thinspace{\dandab} \dontdisplaylinenum }%
     \var{{\devanagarifontvar\numemph\va यैरिदं\lem \mssALL,\hskip.2em plus .9em येरिदं \msCb\msCc\oo 
 वेत्ति\lem \mssALL,\hskip.2em plus .9em वेति \msCc}}% 

%Verse 4:75

{\devanagarifont पावनं सर्वपापानां पुण्यानां च प्रवर्धनम् {॥ ४:\hspace{.11em}७५॥} \veg\dontdisplaylinenum }%
     \var{{\devanagarifontvar\numnoemph\vd प्रवर्धनम्\lem \mssALL,\hskip.2em plus .9em प्रवर्धनः \Ed}}% 

{\devanagarifont आयुः कीर्तिर्यशः सौख्यं धन्यादेव प्रवर्धते \thinspace{\dandab} \dontdisplaylinenum }%
     \var{{\devanagarifontvar\numemph\vb धन्यादेव\lem \mssALL,\hskip.2em plus .9em धर्मादेव \Ed}}% 

%Verse 4:76

{\devanagarifont शान्तिः पुष्टिः स्मृतिर्मेधा जायते धन्यमानवे {॥ ४:\hspace{.11em}७६॥} \veg\dontdisplaylinenum }%
     \var{{\devanagarifontvar\numnoemph\vc पुष्टिः\lem \mssALL,\hskip.2em plus .9em \lk ष्टिः \msCa\oo 
 स्मृतिर्मेधा\lem \mssALL,\hskip.2em plus .9em स्मृति मेधा \msCc\msNa}}% 
    \var{{\devanagarifontvar\numnoemph\vd ॰मानवे\lem \eme,\hskip.2em plus .9em ॰मानवः \mssCaCbCc\msNa\msNb\msNc\msKOb\Ed}}% 

\pend
\endnumbering
\vfill\pagebreak\beginnumbering\pstart
\vers


\alalfejezet{यमेष्वप्रमादः (८)}
{\devanagarifont प्रमादस्थान पञ्चैव कीर्तयिष्यामि तच्छृणु \thinspace{\dandab} \dontdisplaylinenum }%
     \var{{\devanagarifontvar\numemph\va ॰स्थान\lem \msCa\msCc\msNa\msNb,\hskip.2em plus .9em ॰स्थानं \msCb\msNc\msKOb\Ed\ \unmetr\oo 
 पञ्चैव\lem \mssALL,\hskip.2em plus .9em पञ्चैवं \Ed}}% 
    \var{{\devanagarifontvar\numnoemph\vb कीर्तयिष्यामि\lem \mssALL,\hskip.2em plus .9em कीर्तियिष्यामि \msNb}}% 

{\devanagarifont ब्रह्महत्या सुरापानं स्तेयो गुर्वङ्गनागमम्  \danda\dontdisplaylinenum }%
     \var{{\devanagarifontvar\numnoemph\vd ॰गमम्\lem \mssALL,\hskip.2em plus .9em ॰मगम् \msKOb}}% 
    \paral{{\devanagarifontsmall \vcdef {\englishfont \similar\ \MBH\ Suppl. 12.30:}
                     ब्रह्महत्यां सुरापानं स्तेयं गुर्वङ्गनागमम्\thinspace{\devanagarifontsmall ।}
                     महान्ति पातकान्याहुः संयोगं चैव तैः सह\thinspace{\devanagarifontsmall ॥}
                     {\englishfont  \similar\ \MANU\ 11.55 (in Olivelle's edition):}
                     ब्रह्महत्या सुरापानं स्तेयं गुर्वङ्गनागमः\thinspace{\devanagarifontsmall ।}
                     महान्ति पातकान्याहुः संसर्गश्चापि तैः सह\thinspace{\devanagarifontsmall ॥}
                 {\englishfont \compare\ also \YAJNS\ 3.228:}
                         ब्रह्महा मद्यपः स्तेनस्तथैव गुरुतल्पगः\thinspace{\devanagarifontsmall ।}
                         एते महापातकिनो यश्च तैः सह संवसेत्\thinspace{\devanagarifontsmall ॥}  }}

%Verse 4:77

{\devanagarifont महापातकमित्याहुस्तत्संयोगी च पञ्चमः {॥ ४:\hspace{.11em}७७॥} \veg\dontdisplaylinenum }%
 
{\devanagarifont अनृतं च समुत्कर्षे राजगामी च पैशुनः \thinspace{\dandab} \dontdisplaylinenum }%
     \var{{\devanagarifontvar\numemph\va समुत्कर्षे\lem \eme,\hskip.2em plus .9em समुत्कर्षं \msCa\msNa\msKOb,\hskip.5em plus .9em 
समुत्क\uncl{र्ष} \msCb,\hskip.5em plus .9em 
समुत्कर्ष \msCc\msNb\msNc\Ed}}% 
    \var{{\devanagarifontvar\numnoemph\vb राज॰\lem \mssALL,\hskip.2em plus .9em राज्ञी॰ \Ed\oo 
 च\lem \mssALL,\hskip.2em plus .9em \om\ \msKOb}}% 

%Verse 4:78

{\devanagarifont गुरोश्चालीकनिर्बन्धः समानि ब्रह्महत्यया {॥ ४:\hspace{.11em}७८॥} \veg\dontdisplaylinenum }%
     \var{{\devanagarifontvar\numnoemph\vc ॰निर्बन्धः\lem \eme,\hskip.2em plus .9em ॰निर्बद्धः \msCb\msNc,\hskip.5em plus .9em निबद्धस् \msCa\msCc\msNa\msNb\msKOb,\hskip.5em plus .9em 
निर्वद्धस् \Ed}}% 
    \var{{\devanagarifontvar\numnoemph\vd ब्रह्महत्यया\lem \mssALL,\hskip.2em plus .9em ब्र\lk\lk \lk या \msCa}}% 
    \paral{{\devanagarifontsmall \vo \similar\ {\englishfont \MBH\ 5.40.3 and \MANU\ 11.56:}
                  अनृतं च समुत्कर्षे राजगामि च पैशुनम्\thinspace{\devanagarifontsmall ।}
                  गुरोश्चालीकनिर्बन्धः समानि ब्रह्महत्यया\thinspace{\devanagarifontsmall ॥}
                 {\englishfont \similar\ \VISNUS\ 37.1--4 \similar\ \AGNIP\ 168.25} }}

{\devanagarifont ब्रह्मोज्झं वेदनिन्दा च कूटसाक्षी सुहृद्वधः \thinspace{\dandab} \dontdisplaylinenum }%
     \var{{\devanagarifontvar\numemph\va ब्रह्मोज्झं\lem \eme,\hskip.2em plus .9em ब्रह्मो ऋग्॰ \mssCaCbCc\msNa\msNb\msNc\msKOb,\hskip.5em plus .9em ब्रह्म ऋग्॰ \Ed}}% 
    \var{{\devanagarifontvar\numnoemph\vb सुहृद्वधः\lem \mssALL,\hskip.2em plus .9em सकृद्बुधः \Ed}}% 

%Verse 4:79

{\devanagarifont गर्हितानाद्ययोर्जग्धिः सुरापानसमानि षट् {॥ ४:\hspace{.11em}७९॥} \veg\dontdisplaylinenum }%
     \var{{\devanagarifontvar\numnoemph\vc ॰नाद्ययोर्जग्धिः\lem \eme,\hskip.2em plus .9em ॰न्नञ्च यो जग्धिस् \msCa,\hskip.5em plus .9em ॰न्नञ्च यो जग्धि \msCb,\hskip.5em plus .9em 
॰न्नञ्च योद्विग्नः \msCc,\hskip.5em plus .9em ॰न्नं च यो जग्धिः \msNa,\hskip.5em plus .9em ॰न्नं च यो जग्धिः \msNb\msKOb,\hskip.5em plus .9em 
॰न्नञ्च यो जवे \msNc,\hskip.5em plus .9em ॰न्नश्च यो विप्रः \Ed}}% 
    \paral{{\devanagarifontsmall \vo \similar\ {\englishfont \MANU\ 11.57:}
                         ब्रह्मोज्झता वेदनिन्दा कौटसाक्ष्यं सुहृद्वधः\thinspace{\devanagarifontsmall ।}
                         गर्हितानाद्ययोर्जग्धिः सुरापानसमानि षट्\thinspace{\devanagarifontsmall ॥}
                 {\englishfont \compare\ \YAJNS\ 3.229:}
                         गुरूणामध्यधिक्षेपो वेदनिन्दा सुहृद्वधः\thinspace{\devanagarifontsmall ।}
                         ब्रह्महत्यासमं ज्ञेयमधीतस्य च नाशनम्\thinspace{\devanagarifontsmall ॥} }}

{\devanagarifont रेतोत्सेकः स्वयोन्यासु कुमारीष्वन्त्यजासु च \thinspace{\dandab} \dontdisplaylinenum }%
     \var{{\devanagarifontvar\numemph\va स्वयोन्यासु\lem \mssALL,\hskip.2em plus .9em सुतोन्यासु \msCb}}% 

%Verse 4:80

{\devanagarifont सख्युः पुत्रस्य च स्त्रीषु गुरुतल्पसमः स्मृतः {॥ ४:\hspace{.11em}८०॥} \veg\dontdisplaylinenum }%
     \var{{\devanagarifontvar\numnoemph\vc सख्युः\lem \eme,\hskip.2em plus .9em सख्य \mssCaCbCc\msNa\msKOb\Ed,\hskip.5em plus .9em \lk\lk\ \msNb,\hskip.5em plus .9em स\uncl{ख्यु} \msNc\oo 
 पुत्रस्य च स्त्रीषु\lem \mssALL,\hskip.2em plus .9em 
\lk\lk\lk\lk\lk\lk\ \msNb,\hskip.5em plus .9em पुत्रीषु चास्त्रीषु \Ed}}% 
    \var{{\devanagarifontvar\numnoemph\vd ॰तल्पसमः\lem \mssCaCbCc\msNa\msNc,\hskip.2em plus .9em ॰\uncl{तल्पसमः} \msNb,\hskip.5em plus .9em 
॰तल्पः समः \msKOb,\hskip.5em plus .9em ॰तल्पसम \Ed}}% 
    \paral{{\devanagarifontsmall \vo \similar\ {\englishfont \MANU\ 11.59:}
                                 रेतःसेकः स्वयोनीषु कुमारीष्वन्त्यजासु च\thinspace{\devanagarifontsmall ।}
                                 सख्युः पुत्रस्य च स्त्रीषु गुरुतल्पसमं विदुः\thinspace{\devanagarifontsmall ॥} }}

{\devanagarifont निक्षेपस्यापहरणं नराश्वरजतस्य च \thinspace{\dandab} \dontdisplaylinenum }%
     \var{{\devanagarifontvar\numemph\va निक्षेप॰\lem \mssALL,\hskip.2em plus .9em निखेप॰ \msCb,\hskip.5em plus .9em \uncl{निक्षेप}॰ \msNb}}% 
    \var{{\devanagarifontvar\numnoemph\vb नराश्वरजतस्य\lem \mssALL,\hskip.2em plus .9em नराणां स्वजनस्य \msCb,\hskip.5em plus .9em 
\uncl{नराश्वरजतस्य} \msNb}}% 

%Verse 4:81

{\devanagarifont भूमिवज्रमणीनां च रुक्मस्तेयसमः स्मृतः {॥ ४:\hspace{.11em}८१॥} \veg\dontdisplaylinenum }%
     \var{{\devanagarifontvar\numnoemph\vd रुक्मस्तेय॰\lem \eme,\hskip.2em plus .9em \uncl{रूग्य}\lk य॰ \msCa,\hskip.5em plus .9em 
रुग्मस्तेय॰ \msCb\msCc\msNa\msNc\msKOb,\hskip.5em plus .9em \lk\lk\lk\lk\ \msNb,\hskip.5em plus .9em हृतस्तेय॰ \Ed\oo 
 ॰समः\lem \mssALL,\hskip.2em plus .9em सः \msCbacorr,\hskip.5em plus .9em ॰सम \Ed}}% 
    \paral{{\devanagarifontsmall \vo {\englishfont = \MANU\ 11.58 } }}

{\devanagarifont चत्वार एते सम्भूय यत्पापं कुरुते नरः \thinspace{\dandab} \dontdisplaylinenum }%
     \var{{\devanagarifontvar\numemph\va एते\lem \mssALL,\hskip.2em plus .9em \uncl{एते} \msNb,\hskip.5em plus .9em एव \Ed\oo 
 सम्भूय\lem \mssALL,\hskip.2em plus .9em संभूयो \msCc,\hskip.5em plus .9em \uncl{संभूयो} \msNb}}% 

{\devanagarifont महापातक पञ्चैतत् तेन सर्वं प्रकाशितम्  \danda\dontdisplaylinenum }%
     \var{{\devanagarifontvar\numnoemph\vc ॰पञ्चैतत्\lem \msKOb,\hskip.2em plus .9em ॰पञ्चैतन् \mssCaCbCc\Ed,\hskip.5em plus .9em ॰पञ्चैते \msNa,\hskip.5em plus .9em 
॰पञ्चैतम् \msNb,\hskip.5em plus .9em ॰पञ्चेतन् \msNc}}% 
    \var{{\devanagarifontvar\numnoemph\vd प्रकाशितम्\lem \mssALL,\hskip.2em plus .9em प्रकीर्तितम् \msKOb}}% 

%Verse 4:82

{\devanagarifont पञ्चप्रमादमेतानि वर्जनीयं द्विजोत्तम {॥ ४:\hspace{.11em}८२॥} \veg\dontdisplaylinenum }%
     \var{{\devanagarifontvar\numnoemph\ve ॰मादमे॰\lem \mssALL,\hskip.2em plus .9em ॰माद ए॰ \Ed}}% 
    \var{{\devanagarifontvar\numnoemph\vf वर्जनीयं\lem \mssALL,\hskip.2em plus .9em वर्जनीयो \msCc}}% 


\alalfejezet{यमेषु माधुर्यम् (९)}
{\devanagarifont कायवाङ्मनमाधुर्यश्चक्षुर्बुद्धिश्च पञ्चमः \thinspace{\dandab} \dontdisplaylinenum }%
     \var{{\devanagarifontvar\numemph\vab मनमाधुर्यश्च॰\lem \eme,\hskip.2em plus .9em ॰मनसा धूर्यश्च॰ \msCa\msCc\msNa\msNc\msKOb,\hskip.5em plus .9em 
॰मन\uncl{मा}धूर्यश्च॰ \msCb,\hskip.5em plus .9em 
॰मन\lk धूर्य\lk ॰ \msNb,\hskip.5em plus .9em ॰मनसा भूयश्च॰ \Ed}}% 
    \var{{\devanagarifontvar\numnoemph\vb ॰क्षुर्बुद्धि॰\lem \msCa\msCb\msNc\Ed,\hskip.2em plus .9em ॰क्षु बुद्धि॰ \msCc\msNa\msKOb,\hskip.5em plus .9em \lk\lk \lk\  \msNb}}% 

%Verse 4:83

{\devanagarifont सौम्यदृष्टिप्रदानं च क्रूरबुद्धिं च वर्जयेत् {॥ ४:\hspace{.11em}८३॥} \veg\dontdisplaylinenum }%
     \var{{\devanagarifontvar\numnoemph\vc ॰दानं च\lem \mssALL,\hskip.2em plus .9em \lk\lk\ \msNb,\hskip.5em plus .9em ॰दानश्च \Ed}}% 
    \var{{\devanagarifontvar\numnoemph\vd ॰बुद्धिं च\lem \msCa\msNa\msNc\msKOb,\hskip.2em plus .9em बुद्धिश्च \msCb,\hskip.5em plus .9em ॰दृष्टिं च \msCc\Ed,\hskip.5em plus .9em \lk\lk \lk\ \msNb}}% 

{\devanagarifont प्रसन्नमनसा ध्यायेत्प्रियवाक्यमुदीरयेत् \thinspace{\dandab} \dontdisplaylinenum }%
     \var{{\devanagarifontvar\numemph\va प्रसन्न॰\lem \mssALL,\hskip.2em plus .9em \uncl{प्रसन्न}॰ \msNb,\hskip.5em plus .9em प्रसंन॰ \msNc}}% 

%Verse 4:84

{\devanagarifont यथाशक्तिप्रदानं च स्वाश्रमाभ्यागतो गुरुः {॥ ४:\hspace{.11em}८४॥} \veg\dontdisplaylinenum }%
     \var{{\devanagarifontvar\numnoemph\vc यथा॰\lem \mssALL,\hskip.2em plus .9em यस्य \Ed\oo 
 ॰दानं\lem \mssALL,\hskip.2em plus .9em ॰दातश् \Ed}}% 
    \var{{\devanagarifontvar\numnoemph\vd स्वाश्रमा॰\lem \mssALL,\hskip.2em plus .9em स्वासमा॰ \msCc\oo 
 ॰गतो\lem \mssALL,\hskip.2em plus .9em ॰सतो \msNc}}% 

{\devanagarifont इन्धनोदकदानं च जातवेदमथापि वा \thinspace{\dandab} \dontdisplaylinenum }%
     \var{{\devanagarifontvar\numemph\vb इन्धनो॰\lem \mssALL,\hskip.2em plus .9em इत्वनो॰ \msNc\oo 
 जात॰\lem \mssALL,\hskip.2em plus .9em जा॰ \msCb}}% 

{\devanagarifont सुलभानि न दत्तानि इन्धनाग्न्युदकानि च  \danda\dontdisplaylinenum }%
     \var{{\devanagarifontvar\numnoemph\vc सुलभानि न\lem \mssALL,\hskip.2em plus .9em सुरभानि च \Ed}}% 
    \var{{\devanagarifontvar\numnoemph\vd ॰दकानि\lem \mssALL,\hskip.2em plus .9em ॰\uncl{त}कानि \msNb}}% 

%Verse 4:85

{\devanagarifont क्षुते जीवेति वा नोक्तं तस्य किं परतः फलम् {॥ ४:\hspace{.11em}८५॥} \veg\dontdisplaylinenum }%
     \var{{\devanagarifontvar\numnoemph\ve क्षुते\lem \conj,\hskip.2em plus .9em क्षुतं \mssCaCbCc\msNa\msNb\msNc\msKOb,\hskip.5em plus .9em शतं \Ed}}% 


\alalfejezet{यमेष्वार्जवम् (१०)}
{\devanagarifont पञ्चार्जवाः प्रशंसन्ति मुनयस्तत्त्वदर्शिनः \thinspace{\dandab} \dontdisplaylinenum }%
     \var{{\devanagarifontvar\numemph\va पञ्चार्जवाः\lem \msCa\msCb\msNa\msNc\msKOb,\hskip.2em plus .9em पञ्चार्जवः \msCc,\hskip.5em plus .9em \lk\lk \lk\lk\ \msNb,\hskip.5em plus .9em पञ्चार्जवा \Ed\oo 
 प्रशंसन्ति\lem \mssCaCbCc\msNc\msKOb,\hskip.2em plus .9em प्रशसन्ति \msNa\Ed,\hskip.5em plus .9em \uncl{प्रससन्ति} \msNb}}% 

{\devanagarifont कर्मवृत्त्याभिवृद्धिं च पारितोषिकमेव च  \danda\dontdisplaylinenum }%
     \var{{\devanagarifontvar\numnoemph\vc कर्म॰\lem \mssALL,\hskip.2em plus .9em \lk र्म्म॰ \msCa,\hskip.5em plus .9em \uncl{कम्मा}॰ \msNb\oo 
 ॰वृत्त्याभिवृद्धिं च\lem \mssCaCbCc\msNa\msNc,\hskip.2em plus .9em 
॰वृत्तिभिवृद्धिञ्च \msNb,\hskip.5em plus .9em ॰वृत्त्यभिवृद्धिञ्च \msKOb,\hskip.5em plus .9em ॰वृत्याभिवृद्धिश्च \Ed}}% 
    \var{{\devanagarifontvar\numnoemph\vd पारितोषिक॰\lem \eme,\hskip.2em plus .9em पारतोषिक॰ \mssCaCbCc\msNa\msNb\msNc\msKOb\Ed}}% 

%Verse 4:86

{\devanagarifont स्त्रीधनोत्कोचवित्तं च आर्जवो नाभिनन्दति {॥ ४:\hspace{.11em}८६॥} \veg\dontdisplaylinenum }%
     \var{{\devanagarifontvar\numnoemph\ve स्त्रीधनोत्कोच॰\lem \mssALL,\hskip.2em plus .9em स्त्रीधनोत्काच॰ \msKOb,\hskip.5em plus .9em स्त्रीधनङ्गो च \Ed\oo 
 ॰वित्तं च\lem \mssALL,\hskip.2em plus .9em ॰वित्तिञ्च \msNb}}% 
    \var{{\devanagarifontvar\numnoemph\vf आर्जवो ना॰\lem \mssALL,\hskip.2em plus .9em आर्जवञ्च \msCc,\hskip.5em plus .9em आर्ज्जवेना॰ \Ed}}% 

{\devanagarifont आर्जवो न वृथा यज्ञ आर्जवो न वृथा तपः \thinspace{\dandab} \dontdisplaylinenum }%
     \var{{\devanagarifontvar\numemph\vab आर्जवो न वृथा यज्ञ आर्जवो न वृथा तपः\lem \mssCaCbCc\msNb\msNc,\hskip.2em plus .9em \om\ \msNaacorr,\hskip.5em plus .9em 
आर्जवो न वृथा यज्ञ आर्जवो न वृथा तप \msNapcorr,\hskip.5em plus .9em 
आर्जवो न वृथा आर्जवो न वृथा तपः \msKOb,\hskip.5em plus .9em 
आर्जवो न वृथा यज्ञश्चार्र्जवो न वृथा तपः \Ed}}% 

%Verse 4:87

{\devanagarifont आर्जवो न वृथा दानमार्जवो न वृथाग्नयः {॥ ४:\hspace{.11em}८७॥} \veg\dontdisplaylinenum }%
     \lacuna{\devanagarifontsmall \vcd {\englishfont \om\ \Ed} }%
  
{\devanagarifont आर्जवस्येन्द्रियग्रामः सुप्रसन्नो ऽपि तिष्ठति \thinspace{\dandab} \dontdisplaylinenum }%
     \lacuna{\devanagarifontsmall \vab {\englishfont \om\ \Ed} }%
      \var{{\devanagarifontvar\numemph\va ॰ग्रामः\lem \msCa\msCb\msNc\msKOb,\hskip.2em plus .9em ॰ग्रामात् \msCc\msNb,\hskip.5em plus .9em ॰ग्रामाः \msNa}}% 
    \var{{\devanagarifontvar\numnoemph\vb ॰प्रसन्नो\lem \mssALL,\hskip.2em plus .9em ॰प्रन्नो \msKObacorr}}% 

%Verse 4:88

{\devanagarifont आर्जवस्य सदा देवाः काये तस्य चरन्ति ते {॥ ४:\hspace{.11em}८८॥} \veg\dontdisplaylinenum }%
     \var{{\devanagarifontvar\numnoemph\vd तस्य चरन्ति\lem \mssALL,\hskip.2em plus .9em 
त\lk\lacwithnum{2}  न्ति \msCa,\hskip.5em plus .9em तस्य रमन्ति \Ed}}% 

\pend
\endnumbering
\vfill\pagebreak\beginnumbering\pstart
\vers

\ujvers\nemsloka {
{\devanagarifont इति यमप्रविभागः कीर्तितो ऽयं द्विजेन्द्र }%
  \dontdisplaylinenum}    \var{{\devanagarifontvar\numemph\va यमप्रविभागः\lem \msCa\msCb\msNb\msNc\msKOb,\hskip.2em plus .9em यमविभागः \msCc,\hskip.5em plus .9em 
यमप्ररिभागः \msNa,\hskip.5em plus .9em नियमपरिभागः \Ed\oo 
 द्विजेन्द्र\lem \mssALL,\hskip.2em plus .9em नरेन्द्र \Ed}}% 


\nemslokab

{\devanagarifont इह परत सुखार्थं कारयेत्तं मनुष्यः  \danda\dontdisplaylinenum }%
     \var{{\devanagarifontvar\numnoemph\vb ॰येत्तं मनुष्यः\lem \corr,\hskip.2em plus .9em ॰येत्तन्मनुष्यः \msCa\msNa\msNb\msNc\Ed,\hskip.5em plus .9em ॰येत्त मनुष्यः \msCb,\hskip.5em plus .9em 
॰येत्तत्मनुष्यः \msCc\msKOb}}% 

\nemslokac

{\devanagarifont दुरितमलपहारी शङ्करस्याज्ञयास्ते }%
  \dontdisplaylinenum    \var{{\devanagarifontvar\numnoemph\vc दुरित॰\lem \mssALL,\hskip.2em plus .9em इरित॰ \Ed\oo 
 ॰पहारी\lem \mssALL,\hskip.2em plus .9em ॰पलपहारी \msCc\oo 
 ॰ज्ञयास्ते\lem \mssALL,\hskip.2em plus .9em ॰ज्ञयाते \msNa}}% 

%Verse 4:89


\nemslokad

{\devanagarifont भवति पृथिविभर्ता ह्येकछत्रप्रवर्ता {॥ ४:\hspace{.11em}८९॥} \veg\dontdisplaylinenum }%
     \var{{\devanagarifontvar\numnoemph\vd ॰वर्ता\lem \conj,\hskip.2em plus .9em ॰वृत्ता \mssCaCbCc\msNb\msNc,\hskip.5em plus .9em ॰वृत्ताः \msNa\Ed,\hskip.5em plus .9em ॰वत्ता \msKOb}}% 

\vers


{\devanagarifont 
\jump
\begin{center}
\ketdanda~इति वृषसारसंग्रहे यमविभागो नामाध्यायश्चतुर्थः~\ketdanda
\end{center}
\dontdisplaylinenum\vers  }%
     \var{{\devanagarifontvar\numnoemph{\englishfont \Colo:} वृषसार॰\lem \mssALL,\hskip.2em plus .9em वृषार॰ \msKOb\oo 
 नामाध्यायश्चतुर्थः\lem \mssALL,\hskip.2em plus .9em 
नामश्चतुर्थो ऽध्यायः \Ed}}% 
\bekveg\szamveg
\vfill
\phpspagebreak

\versno=0\fejno=5
\thispagestyle{empty}

\centerline{\Large\devanagarifontbold [   पञ्चमो ऽध्यायः  ]}{\vrule depth10pt width0pt} \fancyhead[CE]{{\footnotesize\devanagarifont वृषसारसंग्रहे  }}
\fancyhead[CO]{{\footnotesize\devanagarifont पञ्चमो ऽध्यायः  }}
\fancyhead[LE]{}
\fancyhead[RE]{}
\fancyhead[LO]{}
\fancyhead[RO]{}
\szam\bek



\alalfejezet{नियमाः}
\vers


{\devanagarifont विगतराग उवाच {\dandab}\dontdisplaylinenum  }%
     \var{{\devanagarifontvar\numemph\vo विगतराग उवाच\lem \mssALL,\hskip.2em plus .9em 
विगत\uncl{राग उवा}च \msCa}}% 
    \lacuna{\devanagarifontsmall {\englishfont Witnesses used for this chapter: \msCa\ ff.\thinspace 201v--202r, 
                                                  \msCb\ ff.\thinspace 208v--209r, 
                                                  \msCc\ ff.\allowbreak\thinspace 277r--278r,
                                                  \msNa\ ff.\thinspace 9r--9v, 
                                                  \msNb\ exp.\thinspace 50 (upper) and 51 (lower),
                                                  \msNc\ ff.\thinspace 217r--218r,
                                                  \msM\ ff.\thinspace 9r--10r,
                                                  \msKOb\ ff.\thinspace 217v--218r,
                                                  \Ed\ pp.\thinspace 597--599;  
                                                  \mssCaCbCc\ = \msCa + \msCb + \msCc} }%
  
\nemsloka 
{\devanagarifont कथय नियमतत्त्वं साम्प्रतं त्वं विशेषाद् }%
  \dontdisplaylinenum    \var{{\devanagarifontvar\numnoemph\va कथय नि॰\lem \mssALL,\hskip.2em plus .9em कथयति \Ed\oo 
 ॰तत्त्वं\lem \mssALL,\hskip.2em plus .9em तं \msCb\oo 
 साम्प्रतं त्वं विशेषाद्\lem \msCa\msNa\msNc\msKOb\Ed,\hskip.2em plus .9em त्वां वशेषात् \msCb,\hskip.5em plus .9em 
सांप्रत त्वं विसेषात् \msCc\msNb,\hskip.5em plus .9em साम्प्रतं त्वं विशेषा \msM}}% 


\nemslokab

{\devanagarifont अमृतवचनतुल्यं श्रोतुकामो गतो ऽस्मि  \danda\dontdisplaylinenum }%
     \var{{\devanagarifontvar\numnoemph\vb ॰वचनतुल्यं श्रो॰\lem \msM,\hskip.2em plus .9em वदनतुल्यं श्रो॰ \msCa\msCc\msNapcorr\msNb\msNc\msKOb\Ed,\hskip.5em plus .9em 
वदनतुल्यां श्रो॰ \msCb,\hskip.5em plus .9em 
वदन\uncl{तुल्यं श्रो} तुल्यं स्रो॰ \msNaacorr\oo 
 ॰कामो\lem \mssALL,\hskip.2em plus .9em ॰कामा \msM\Ed}}% 

\nemslokac

{\devanagarifont प्रकृतिदहनदग्धं ज्ञानतोयैर्निषिक्तम् }%
  \dontdisplaylinenum    \var{{\devanagarifontvar\numnoemph\vc ॰दहन॰\lem \mssALL,\hskip.2em plus .9em ॰वदन॰ \Ed\oo 
 ॰दग्धं\lem \mssALL,\hskip.2em plus .9em ॰दग्ध \msM\oo 
 ॰र्निषिक्तम्\lem \mssALL,\hskip.2em plus .9em ॰र्विमुक्तम् \msCb,\hskip.5em plus .9em ॰र्निशिक्तः \msM}}% 

%Verse 5:1


\nemslokad

{\devanagarifont अपर वदमतज्ज्ञं नास्ति धर्मेषु तृप्तिः {॥ ५:\hspace{.11em}१॥} \veg\dontdisplaylinenum }%
     \var{{\devanagarifontvar\numnoemph\vd अपर॰\lem \mssALL,\hskip.2em plus .9em अपरं \msNa\ \unmetr,\hskip.5em plus .9em अर॰ \msMacorr\oo 
 मतज्ज्ञं नास्ति\lem \conj,\hskip.2em plus .9em मतज्ञा नास्ति \msCapcorr\msCb\msNa\msNc\msM\msKOb,\hskip.5em plus .9em 
तज्ञा नास्ति \msCaacorr,\hskip.5em plus .9em 
मतज्ञा\uncl{न्ना}स्ति \msCc,\hskip.5em plus .9em 
\uncl{मे} \lk\lk \lk\lk\ \msNb,\hskip.5em plus .9em 
॰न तज्ज्ञान्नास्ति \Ed\oo 
 धर्मेषु तृप्तिः\lem \mssALL,\hskip.2em plus .9em मे धर्मतृप्तिः \msM}}% 

\vers


{\devanagarifont अनर्थयज्ञ उवाच {\dandab}\dontdisplaylinenum  }%
     \var{{\devanagarifontvar\numemph\vo अनर्थ॰\lem \mssALL,\hskip.2em plus .9em अर्थ॰ \msM}}% 

\nemsloka 
{\devanagarifont श्रवणसुखमतो ऽन्यत्कीर्तयिष्ये द्विजेन्द्र }%
  \dontdisplaylinenum    \var{{\devanagarifontvar\numnoemph\va ॰सुख॰\lem \mssALL,\hskip.2em plus .9em ॰मुख॰ \msNaacorr\oo 
 ॰मतो ऽन्यत्\lem \mssCaCbCc\msNa\msNc\msKOb,\hskip.2em plus .9em ॰मतो ऽन्य \msNb,\hskip.5em plus .9em 
॰मतो न्यः \msM,\hskip.5em plus .9em ॰मनो ऽन्यत् \Ed\oo 
 कीर्त॰\lem \mssALL,\hskip.2em plus .9em कीर्ति॰ \msNa\msNb}}% 


\nemslokab

{\devanagarifont नियमकलविशेषः पञ्च पञ्च प्रकारः  \danda\dontdisplaylinenum }%
     \var{{\devanagarifontvar\numnoemph\vb ॰विशेषः\lem \mssALL,\hskip.2em plus .9em विशे\lk\ \msCa,\hskip.5em plus .9em ॰विशेष \msCb\oo 
 प्रकारः\lem \mssALL,\hskip.2em plus .9em पकारः \msNc}}% 

\nemslokac

{\devanagarifont हरिहरमुनिभीष्टं धर्मसारं द्विजेन्द्र }%
  \dontdisplaylinenum
%Verse 5:2


\nemslokad

{\devanagarifont कलिकलुषविनाशं प्रायमोक्षप्रसिद्धम् {॥ ५:\hspace{.11em}२॥} \veg\dontdisplaylinenum }%
     \var{{\devanagarifontvar\numnoemph\vd ॰विनाशं\lem \mssALL,\hskip.2em plus .9em ॰विनाश॰ \msCc\Ed\oo 
 ॰मोक्ष॰\lem \mssALL,\hskip.2em plus .9em ॰मोक्षं \msKOb}}% 

\vers


{\devanagarifont शौचमिज्या तपो दानं स्वाध्यायोपस्थनिग्रहः \thinspace{\dandab} \dontdisplaylinenum }%
     \var{{\devanagarifontvar\numemph\va इज्या\lem \msCa\msCb\msNa\msNc\msKOb\Ed,\hskip.2em plus .9em ईज्या \msCc\msNb\msM\oo 
 दानं\lem \mssALL,\hskip.2em plus .9em दान॰ \msNb}}% 

%Verse 5:3

{\devanagarifont व्रतोपवासमौनं च स्नानं च नियमा दश {॥ ५:\hspace{.11em}३॥} \veg\dontdisplaylinenum }%
     \var{{\devanagarifontvar\numnoemph\vc ॰पवास॰\lem \mssALL,\hskip.2em plus .9em ॰प्रवाष॰ \msM,\hskip.5em plus .9em ॰पवासनियम॰ \msKObacorr}}% 
    \paral{{\devanagarifontsmall \vo {\englishfont  = \LINPU\ 1.8.29cd--30ab = \VDHU\ 3.233.202} }}


\alalfejezet{नियमेषु शौचम् (१)}
{\devanagarifont तत्र शौचादिनिर्देशं वक्ष्यामीह द्विजोत्तम \thinspace{\dandab} \dontdisplaylinenum }%
     \var{{\devanagarifontvar\numemph\va ॰निर्देशं\lem \mssALL,\hskip.2em plus .9em ॰नियमं \msNa,\hskip.5em plus .9em ॰र्द्देशं \msNb}}% 

%Verse 5:4

{\devanagarifont शारीरशौचमाहारो मात्रा भावश्च पञ्चमः {॥ ५:\hspace{.11em}४॥} \veg\dontdisplaylinenum }%
     \var{{\devanagarifontvar\numnoemph\vc शारीर॰\lem \mssALL,\hskip.2em plus .9em शरीर॰ \msNb\oo 
 ॰शौचमाहारो\lem \mssALL,\hskip.2em plus .9em ॰शौच\lk हारो \msCa,\hskip.5em plus .9em 
॰स्रोतमाहार \msM}}% 
    \var{{\devanagarifontvar\numnoemph\vd मात्रा भावश्च\lem \mssALL,\hskip.2em plus .9em मात्रा भावं च \msCa,\hskip.5em plus .9em 
\uncl{सात्राभा}वश्च \msNb}}% 


\alalalfejezet{शरीरशौचम्}

{\devanagarifont ताडयेन्न च बन्धेत न च प्राणैर्वियोजयेत् \thinspace{\dandab} \dontdisplaylinenum }%
     \var{{\devanagarifontvar\numemph\va ताडयेन्न\lem \mssALL,\hskip.2em plus .9em ताडये न \msNc\msM\oo 
 बन्धेत\lem \mssALL,\hskip.2em plus .9em बन्धेन \msM}}% 

%Verse 5:5

{\devanagarifont परस्त्रीपरद्रव्येषु शौचं कायिकमुच्यते {॥ ५:\hspace{.11em}५॥} \veg\dontdisplaylinenum }%
     \var{{\devanagarifontvar\numnoemph\vc ॰द्रव्येषु\lem \mssALL,\hskip.2em plus .9em ॰द्रवेषु \msM}}% 
    \var{{\devanagarifontvar\numnoemph\vd शौचं\lem \mssALL,\hskip.2em plus .9em शौच \msNc\oo 
 कायिकमुच्यते\lem \mssALL,\hskip.2em plus .9em कायिकमुमुच्येते \msNc}}% 

{\devanagarifont श्रोत्रशौचं द्विजश्रेष्ठ गुदोपस्थमुखादयः \thinspace{\dandab} \dontdisplaylinenum }%
     \var{{\devanagarifontvar\numemph\va श्रोत्र॰\lem \msM\msKOb,\hskip.2em plus .9em श्रोत॰ \mssCaCbCc\msNa\msNb\msNc\Ed}}% 
    \var{{\devanagarifontvar\numnoemph\vb गुदोपस्थ॰\lem \mssCaCbCc\msNa\msNb\msM,\hskip.2em plus .9em गुदोप्रस्थ॰ \msNc,\hskip.5em plus .9em गुहोपस्थ॰ \msKOb,\hskip.5em plus .9em गुदापस्थ॰ \Ed}}% 

%Verse 5:6

{\devanagarifont मुखस्याचमनं शौचमाहारवचनेषु च {॥ ५:\hspace{.11em}६॥} \veg\dontdisplaylinenum }%
     \var{{\devanagarifontvar\numnoemph\vc मुखस्या॰\lem \mssALL,\hskip.2em plus .9em मुखस्था॰ \msCb}}% 
    \var{{\devanagarifontvar\numnoemph\vcd शौचमा॰\lem \msCa\msCc\msNa\msNc\msKOb\Ed,\hskip.2em plus .9em शौचंमा॰ \msCb\msNb,\hskip.5em plus .9em शौच आ॰ \msM}}% 
    \var{{\devanagarifontvar\numnoemph\vd ॰वचनेषु\lem \mssALL,\hskip.2em plus .9em ॰वषनेषु \msM}}% 

{\devanagarifont मूत्रविष्टासमुत्सर्गे देवताराधनेषु च \thinspace{\dandab} \dontdisplaylinenum }%
     \var{{\devanagarifontvar\numemph\va ॰विष्टा॰\lem \mssALL,\hskip.2em plus .9em ॰विष्ट॰ \msNb\msM}}% 

%Verse 5:7

{\devanagarifont मृत्तोयैस्तु गुदोपस्थं शौचयीत विचक्षणः {॥ ५:\hspace{.11em}७॥} \veg\dontdisplaylinenum }%
     \var{{\devanagarifontvar\numnoemph\vc मृत्तोयैस्तु\lem \msCc\msNa\msNb\msKOb\Ed,\hskip.2em plus .9em \uncl{मृ}\lk\lk \lk\ \msCa,\hskip.5em plus .9em 
मृतोयैस्तु \msCb\msM,\hskip.5em plus .9em मृत्तोयेस्तु \msNc\oo 
 ॰पस्थं\lem \msCa\msCb\msNa\msNb\msNc\msKOb,\hskip.2em plus .9em ॰पस्थ \msCc\Ed,\hskip.5em plus .9em ॰पस्थः \msM}}% 
    \var{{\devanagarifontvar\numnoemph\vd शौचयीत\lem \mssALL,\hskip.2em plus .9em शौचये च \msM}}% 

\pend
\endnumbering
\vfill\pagebreak\beginnumbering\pstart
\vers

{\devanagarifont एकोपस्थे गुदे पञ्च तथैकत्र करे दश \thinspace{\dandab} \dontdisplaylinenum }%
     \var{{\devanagarifontvar\numemph\va ॰पस्थे\lem \msCa\msCb\msNa\msNc\msKOb\Ed,\hskip.2em plus .9em ॰पस्थ॰ \msCc\msNb\msM\oo 
 गुदे\lem \msCa\msCb\msNa\msNc\msKOb\Ed,\hskip.2em plus .9em गुदो \msCc\msNb,\hskip.5em plus .9em गुद \msM}}% 
    \var{{\devanagarifontvar\numnoemph\vb तथैकत्र\lem \msCa\msCc\msNa\msNb\msNc\msKOb,\hskip.2em plus .9em तथैक\uncl{त्र} \msCb,\hskip.5em plus .9em 
तथैकत्रे \msM,\hskip.5em plus .9em तथैकश्च \Ed\oo 
 दश\lem \mssALL,\hskip.2em plus .9em दशः \msCc}}% 
    \paral{{\devanagarifontsmall \vo {\englishfont \similar\ \MANU\ 5.136:} एका लिङ्गे गुदे तिस्रस्तथैकत्र करे दश\thinspace{\devanagarifontsmall ।}
                                               उभयोः सप्त दातव्या मृदः शुद्धिमभीप्सता\thinspace{\devanagarifontsmall ॥} }}

%Verse 5:8

{\devanagarifont उभयोः सप्त दातव्या मृदः शुद्धिं समीहता {॥ ५:\hspace{.11em}८॥} \veg\dontdisplaylinenum }%
     \var{{\devanagarifontvar\numnoemph\vc उभयोः\lem \mssALL,\hskip.2em plus .9em उभय \msM\oo 
 दातव्या\lem \msCa\msCb\msNa\msNb\msNc\msKOb,\hskip.2em plus .9em दातव्यो \msCc\Ed,\hskip.5em plus .9em दातव्य \msM}}% 
    \var{{\devanagarifontvar\numnoemph\vd मृदः\lem \mssCaCbCc\msNc\msKOb\Ed,\hskip.2em plus .9em मृतः \msNa\msM,\hskip.5em plus .9em मृदा \msNb\oo 
 शुद्धिं समीहता\lem \msCa\msCb\msNa\msKOb,\hskip.2em plus .9em शुद्धिसमीहया \msCc,\hskip.5em plus .9em शु\uncl{द्धि} समीहता \msNb,\hskip.5em plus .9em 
शुद्धिः समीहता \msNc,\hskip.5em plus .9em शुद्धि समीहता \msM,\hskip.5em plus .9em शुद्धिं समाहिता \Ed}}% 

{\devanagarifont एतच्छौचं गृहस्थानां द्विगुणं ब्रह्मचारिणाम् \thinspace{\dandab} \dontdisplaylinenum }%
     \var{{\devanagarifontvar\numemph\va एतच्छौचं\lem \msCa\msCb\msNa\msNc\msM\msKOb,\hskip.2em plus .9em चेतछौच \msCc\Ed,\hskip.5em plus .9em एत\lk\lk\ \msNb}}% 
    \var{{\devanagarifontvar\numnoemph\vb ॰गुणं\lem \mssALL,\hskip.2em plus .9em ॰गुण \msCc}}% 
    \paral{{\devanagarifontsmall \vab {\englishfont \similar\ \MANU\ 5.137:}
                 एतच्छौचं गृहस्थानां द्विगुणं ब्रह्मचारिणाम्\thinspace{\devanagarifontsmall ।}
                 त्रिगुणं स्याद्वनस्थानां यतीनां तु चतुर्गुणम्\thinspace{\devanagarifontsmall ॥} }}

%Verse 5:9

{\devanagarifont वानप्रस्थस्य त्रिगुणं यतीनां तु चतुर्गुणम् {॥ ५:\hspace{.11em}९॥} \veg\dontdisplaylinenum }%
     \var{{\devanagarifontvar\numnoemph\vc वानप्रस्थस्य\lem \mssALL,\hskip.2em plus .9em वानप्रस्थे तु \msM\oo 
 त्रि॰\lem \mssALL,\hskip.2em plus .9em द्वि॰ \msCc}}% 


\alalalfejezet{आहारशौचम्}

{\devanagarifont आहारशौचं वक्ष्यामि शृणुष्वावहितो भव \thinspace{\dandab} \dontdisplaylinenum }%
     \var{{\devanagarifontvar\numemph\va ॰शौचं\lem \mssALL,\hskip.2em plus .9em ॰शौच \msM}}% 
    \var{{\devanagarifontvar\numnoemph\vb शृणुष्वावहितो\lem \mssALL,\hskip.2em plus .9em शृणु\uncl{ष्वाव}\lk\lk\ \msCa,\hskip.5em plus .9em 
शृणुष्ववहितो \msNb}}% 

{\devanagarifont भागद्वयं तु भुञ्जीत भागमेकं जलं पिबेत्  \danda\dontdisplaylinenum }%
     \var{{\devanagarifontvar\numnoemph\vd ॰कं जलं\lem \mssALL,\hskip.2em plus .9em ॰कोदकं \msM\oo 
 पिबेत्\lem \mssALL,\hskip.2em plus .9em पिबे \msCb}}% 

%Verse 5:10

{\devanagarifont वायुसंचारदानार्थं चतुर्थमवशेषयेत् {॥ ५:\hspace{.11em}१०॥} \veg\dontdisplaylinenum }%
     \var{{\devanagarifontvar\numnoemph\ve ॰चारदानार्थं\lem \mssALL,\hskip.2em plus .9em ॰चरदानार्थं \msM,\hskip.5em plus .9em ॰चारणार्थाय \Ed}}% 
    \paral{{\devanagarifontsmall \vo {\englishfont \similar\ Śaṅkara's commentary ad \BHG\ 6.16:}
                                 उक्तं हि\thinspace{\devanagarifontsmall ।} 
                                 अर्धं सव्यञ्जनान्नस्य तृतीयमुदकस्य च\thinspace{\devanagarifontsmall ।} 
                                 वायोः संचरणार्थं तु चतुर्थमवशेषयेत्\thinspace{\devanagarifontsmall ॥};
                    {\englishfont \compare\ \ASTANGHR\ 8.46cd--47ab:}
                                              अन्नेन कुक्षेर्द्वावंशौ पानेनैकं प्रपूरयेत्\thinspace{\devanagarifontsmall ॥} 
                                              आश्रयं पवनादीनां चतुर्थमवशेषयेत्\thinspace{\devanagarifontsmall ।};
                    {\englishfont \compare\ \SANNYASUP\ 59:}
                                              आहारस्य च भागौ द्वौ तृतीयमुदकस्य च\thinspace{\devanagarifontsmall ।} 
                                              वायोः संचरणार्थाय चतुर्थमवशेषयेत्\thinspace{\devanagarifontsmall ॥} }}

\pend
\endnumbering
\vfill\pagebreak\beginnumbering\pstart
\vers

{\devanagarifont स्निग्धस्वादुरसैः षड्भिराहारषड्रसैर्बुधः \thinspace{\dandab} \dontdisplaylinenum }%
     \var{{\devanagarifontvar\numemph\va ॰स्वादुरसैः\lem \mssCaCbCc\msNa\msNc\msKOb,\hskip.2em plus .9em ॰स्वा\lk रसैः \msNb,\hskip.5em plus .9em ॰स्वादुरसं \msM,\hskip.5em plus .9em ॰स्वादरसैः \Ed}}% 
    \var{{\devanagarifontvar\numnoemph\vb ॰हारषड्रसैर्बु॰\lem \msCb\Ed,\hskip.2em plus .9em ॰हारसद्रवैर्बु॰ \msCa\msNa\msNc\msKOb,\hskip.5em plus .9em 
॰हारसद्रवै बु॰ \msCc,\hskip.5em plus .9em ॰हारषड्रसै बु॰ \msNb,\hskip.5em plus .9em ॰हारे सद्रवद्बु॰ \msM}}% 

%Verse 5:11

{\devanagarifont धातुवैषम्यनाशो ऽस्ति न च रोगाः सुदारुणाः {॥ ५:\hspace{.11em}११॥} \veg\dontdisplaylinenum }%
     \var{{\devanagarifontvar\numnoemph\vc ॰वैषम्यनाशो ऽस्ति\lem \msCa\msCc\msNa\msNb\msNc\msKOb,\hskip.2em plus .9em 
॰\uncl{दै}षम्य$\-$नाशास्ति \msCb,\hskip.5em plus .9em ॰वैशम्य नस्यास्ति \msM,\hskip.5em plus .9em ॰वैषम्य नश्यन्ति \Ed}}% 
    \var{{\devanagarifontvar\numnoemph\vd रोगाः\lem \mssALL,\hskip.2em plus .9em रोग \msM\oo 
 सुदारुणाः\lem \mssALL,\hskip.2em plus .9em स्वदारुणाः \msM,\hskip.5em plus .9em सुदारुणः \Ed}}% 

{\devanagarifont अभक्ष्यं च न भक्षेत अपेयं न च पाययेत् \thinspace{\dandab} \dontdisplaylinenum }%
     \var{{\devanagarifontvar\numemph\va अभक्ष्यं\lem \mssCaCbCc\msNa\msNc\msKOb,\hskip.2em plus .9em \lk\lk \lk\ \msNb,\hskip.5em plus .9em अभक्षं \msM\Ed\oo 
 च न भक्षेत\lem \mssALL,\hskip.2em plus .9em न च भक्षेतः \msM,\hskip.5em plus .9em च न भक्ष्येत \msKOb}}% 
    \var{{\devanagarifontvar\numnoemph\vb न च\lem \mssCaCbCc\msNa\msNb\msM,\hskip.2em plus .9em च न \msNc\msKOb\Ed}}% 

%Verse 5:12

{\devanagarifont अगम्यं न च गम्येत अवाच्यं न च भाषयेत् {॥ ५:\hspace{.11em}१२॥} \veg\dontdisplaylinenum }%
     \var{{\devanagarifontvar\numnoemph\vc गम्येत\lem \mssALL,\hskip.2em plus .9em गम्येतः \msM}}% 
    \var{{\devanagarifontvar\numnoemph\vd अवाच्यं\lem \mssALL,\hskip.2em plus .9em अवाचं \msCc}}% 

{\devanagarifont लशुनं च पलाण्डुं च गृञ्जनं कवकानि च \thinspace{\dandab} \dontdisplaylinenum }%
     \var{{\devanagarifontvar\numemph\va पलाण्डुं\lem \Ed,\hskip.2em plus .9em पलण्डुं \mssCaCbCc\msNb\msNc\msM\msKOb,\hskip.5em plus .9em पलडुं \msNa}}% 
    \var{{\devanagarifontvar\numnoemph\vb कवकानि\lem \mssALL,\hskip.2em plus .9em कवचानि च \msKOb,\hskip.5em plus .9em च कचानि \Ed}}% 
    \paral{{\devanagarifontsmall \vab {\englishfont \similar\ \MANU\ 5.5ab: } लशुनं गृञ्जनं चैव पलाण्डुं कवकानि च }}

%Verse 5:13

{\devanagarifont गोरश्वसूकरं मांसं वर्जयेच्च विधानतः {॥ ५:\hspace{.11em}१३॥} \veg\dontdisplaylinenum }%
     \var{{\devanagarifontvar\numnoemph\vc गोरश्व॰\lem \msCa\msNb\msKOb,\hskip.2em plus .9em गोरश्च \msCb\msCc\msNa\msNc\msM,\hskip.5em plus .9em गौरश्च \Ed\oo 
 मांसं\lem \mssALL,\hskip.2em plus .9em मांसः \msM,\hskip.5em plus .9em मासं \Ed}}% 
    \var{{\devanagarifontvar\numnoemph\vd विधानतः\lem \mssALL,\hskip.2em plus .9em विधानत् \msM}}% 

{\devanagarifont छत्त्राकं विड्वराहं च गोमांसं च न भक्षयेत् \thinspace{\dandab} \dontdisplaylinenum }%
     \var{{\devanagarifontvar\numemph\va छत्त्राकं\lem \mssALL,\hskip.2em plus .9em छत्त्राक \msCc,\hskip.5em plus .9em छत्राङ्कं \msKOb\oo 
 विड्व॰\lem \mssCaCbCc\msNb\msM\Ed,\hskip.2em plus .9em विद्व॰ \msNa\msNc\msKOb}}% 
    \var{{\devanagarifontvar\numnoemph\vb गोमांसं\lem \mssALL,\hskip.2em plus .9em गोमाञ् \msCbacorr}}% 
    \paral{{\devanagarifontsmall \vab {\englishfont \compare\ \MANU\ 5.19ab:} छत्राकं विड्वराहं च लशुनं ग्रामकुक्कुटम्  }}

%Verse 5:14

{\devanagarifont चटकं च कपोतं च जालपादांश्च वर्जयेत् {॥ ५:\hspace{.11em}१४॥} \veg\dontdisplaylinenum }%
     \var{{\devanagarifontvar\numnoemph\vc चटकं\lem \mssALL,\hskip.2em plus .9em चटकाम् \msCc\msNb}}% 
    \var{{\devanagarifontvar\numnoemph\vd ॰पादांश्च\lem \mssALL,\hskip.2em plus .9em जालपादञ्च \msM}}% 

{\devanagarifont हंससारसचक्राह्वकुक्कुटान् शुकश्येनकान् \thinspace{\dandab} \dontdisplaylinenum }%
     \var{{\devanagarifontvar\numemph\va ॰चक्राह्व॰\lem \mssALL,\hskip.2em plus .9em ॰चक्राह्वा॰ \msM}}% 
    \var{{\devanagarifontvar\numnoemph\vb ॰कुक्कुटान् शु॰\lem \mssCaCbCc\msNc\msKOb\Ed,\hskip.2em plus .9em ॰कुक्कुटा शु॰ \msNa,\hskip.5em plus .9em ॰कुक्कुटां शु॰ \msNb,\hskip.5em plus .9em ॰कुर्कुटा शु॰ \msM\oo 
 ॰श्येनकान्\lem \msCa\msCc\msNc\Ed,\hskip.2em plus .9em ॰शोनकान् \msCb,\hskip.5em plus .9em ॰श्येनका \msNa,\hskip.5em plus .9em ॰श्येनकां \msNb\msKOb,\hskip.5em plus .9em ॰श्येनकम् \msM}}% 

%Verse 5:15

{\devanagarifont काकोलूकं बलाकं च मत्स्यादींश्चापि वर्जयेत् {॥ ५:\hspace{.11em}१५॥} \veg\dontdisplaylinenum }%
     \var{{\devanagarifontvar\numnoemph\vc काकोलूकं बलाकं च\lem \msCb\msNc\msKOb,\hskip.2em plus .9em काकोलूक\uncl{स्व}\lk\lk ञ्च \msCa,\hskip.5em plus .9em 
काकोलूकबलाकं च \msCc\msNa\msM\Ed,\hskip.5em plus .9em 
\uncl{काकोलूकं बलाकं च} \msNb}}% 
    \var{{\devanagarifontvar\numnoemph\vd मत्स्यादींश्चापि वर्जयेत्\lem \mssALL,\hskip.2em plus .9em मत्स्यादीनि च वर्जये \msM}}% 

{\devanagarifont अमेध्यांश्चापवित्रांश्च सर्वानेव विवर्जयेत् \thinspace{\dandab} \dontdisplaylinenum }%
     \var{{\devanagarifontvar\numemph\va अमेध्यांश्चापवित्रांश्च\lem \mssCaCbCc\msNa\msNc\msKOb,\hskip.2em plus .9em 
\uncl{अमेध्याश्चापवित्रांश्च} \msNb,\hskip.5em plus .9em 
अमेध्याश्च पवित्राश्च \msM,\hskip.5em plus .9em अमेध्यश्चापवित्रांश्च \Ed}}% 
    \var{{\devanagarifontvar\numnoemph\vb सर्वानेव विवर्जयेत्\lem \mssALL,\hskip.2em plus .9em सर्वान्येतानि वर्जयेत् \msM}}% 

%Verse 5:16

{\devanagarifont शाकमूलफलानां च अभक्ष्यं परिवर्जयेत् {॥ ५:\hspace{.11em}१६॥} \veg\dontdisplaylinenum }%
 
{\devanagarifont मानवेषु पुराणेषु शैवभारतसंहिते \thinspace{\dandab} \dontdisplaylinenum }%
 
{\devanagarifont कीर्तितानि विशेषेण शौचाचारमशेषतः  \danda\dontdisplaylinenum }%
     \var{{\devanagarifontvar\numemph\vc विशेषेण\lem \mssALL,\hskip.2em plus .9em मशेषेण \msM}}% 

%Verse 5:17

{\devanagarifont त्वया जिज्ञासितो ऽस्म्यद्य संक्षिप्तः कथितो मया {॥ ५:\hspace{.11em}१७॥} \veg\dontdisplaylinenum }%
     \var{{\devanagarifontvar\numnoemph\ve जिज्ञासितो\lem \mssCaCbCc\msNa\msNb\msM\msKObpcorr,\hskip.2em plus .9em जिज्ञासनो \msNc,\hskip.5em plus .9em जिज्ञासि \msKObacorr,\hskip.5em plus .9em जिज्ञासतो \Ed}}% 
    \var{{\devanagarifontvar\numnoemph\vf ॰क्षिप्तः\lem \msCa\msCc\msNa\msNc\msKOb\Ed,\hskip.2em plus .9em ॰क्षिप्य \msCb,\hskip.5em plus .9em ॰क्षिप्त \msNb\msM\oo 
 कथितो\lem \mssALL,\hskip.2em plus .9em कथितं \Ed}}% 

{\devanagarifont सत्यवादी शुचिर्नित्यं ध्यानयोगरतः शुचिः \thinspace{\dandab} \dontdisplaylinenum }%
     \var{{\devanagarifontvar\numemph\va ॰वादी\lem \mssALL,\hskip.2em plus .9em ॰वादि \msM\oo 
 शुचिर्नित्यं\lem \mssALL,\hskip.2em plus .9em शुचिन्नित्यं \msKOb}}% 
    \var{{\devanagarifontvar\numnoemph\vb ॰रतः शुचिः\lem \msCa\msCb\msKOb\Ed,\hskip.2em plus .9em ॰रतः शुचि \msCc\msNc,\hskip.5em plus .9em ॰रतः शुचिन् \msNa\msNb,\hskip.5em plus .9em ॰रत शुचि \msM}}% 

%Verse 5:18

{\devanagarifont अहिंसकः शुचिर्दान्तो दयाभूतक्षमा शुचिः {॥ ५:\hspace{.11em}१८॥} \veg\dontdisplaylinenum }%
     \var{{\devanagarifontvar\numnoemph\vc अहिंसकः\lem \mssALL,\hskip.2em plus .9em अहिंसक \msCb\msM\oo 
 शुचिर्दान्तो\lem \msCa\msCb\msNa\msNb\msKOb,\hskip.2em plus .9em शुचि दान्तो \msCc\msNc\msM,\hskip.5em plus .9em शुचिर्दान्तौ \Ed}}% 
    \var{{\devanagarifontvar\numnoemph\vd ॰भूत॰\lem \mssALL,\hskip.2em plus .9em ॰भुत॰ \msM\oo 
 शुचिः\lem \mssALL,\hskip.2em plus .9em शुचि \msM}}% 

{\devanagarifont सर्वेषामेव शौचानामर्थशौचं परं स्मृतम् \thinspace{\dandab} \dontdisplaylinenum }%
     \var{{\devanagarifontvar\numemph\vb ॰शौचं परं स्मृतम्\lem \msCa\msNa\msNb\msNc\msKOb,\hskip.2em plus .9em ॰शौचं पर स्मृतम् \msCb\msCc,\hskip.5em plus .9em 
॰शौच पर स्मृतः \msM,\hskip.5em plus .9em 
॰शौचयनं स्मृतः \Ed}}% 
    \paral{{\devanagarifontsmall \vab {\englishfont \similar\ \MANU\ 5.106:}
                         सर्वेषामेव शौचानामर्थशौचं परं स्मृतम्\thinspace{\devanagarifontsmall ।}
                         यो ऽर्थे शुचिर्हि स शुचिर्न मृद्वारिशुचिः शुचिः\thinspace{\devanagarifontsmall ॥} }}

{\devanagarifont यो ऽर्थे हि शुचिः स शुचिर्न मृद्वारिशुचिः शुचिः  \danda\dontdisplaylinenum }%
     \var{{\devanagarifontvar\numnoemph\vcd यो ऽर्थे हि शुचिः स शुचिर्न\lem \mssCaCbCc\msNc\msKOb\ \unmetr,\hskip.2em plus .9em 
यो ऽर्थे हि शुचिः स शुचि न \msNa\msNb,\hskip.5em plus .9em 
यो र्थे शुचि हि स शुद्धि \msM,\hskip.5em plus .9em 
यो ऽर्थे हि सुशुचिर्विप्र न \Ed}}% 
    \var{{\devanagarifontvar\numnoemph\vd ॰शुचिः शुचिः\lem \mssCaCbCc\msNa\msNc\msKOb,\hskip.2em plus .9em शुचि शुचिः \msNb,\hskip.5em plus .9em ॰शुचि शुचि \msM,\hskip.5em plus .9em ॰शुचिः शुचि \Ed}}% 

%Verse 5:19

{\devanagarifont कायवाङ्मनसां शौचं स शुचिः सर्ववस्तुषु {॥ ५:\hspace{.11em}१९॥} \veg\dontdisplaylinenum }%
     \var{{\devanagarifontvar\numnoemph\ve वाङ्मनसां शौचं\lem \mssALL,\hskip.2em plus .9em वाङ्मणसा शुद्धि \msM}}% 
    \var{{\devanagarifontvar\numnoemph\vf शुचिः\lem \mssALL,\hskip.2em plus .9em शुचि \msCc\msM\oo 
 वस्तुषु\lem \mssALL,\hskip.2em plus .9em वस्तुषुः \msNc,\hskip.5em plus .9em वस्तुशु \msM}}% 
    \lacuna{\devanagarifontsmall \vcd {\englishfont \Ed\ adds here, after pādas cd:} शौचाशौचविधिर्ज्ञात्वा मुच्यते सर्वकिल्बिषात् 
                {\englishfont (None of the palm-leaf MSS, nor \msPaperA\ or \msPaperC, add anything.)} }%
  
\nemslokalong


\ujvers\nemsloka {
{\devanagarifont शौचाशौचविधिज्ञमानव यदि कालक्षये निश्चयः }%
  \dontdisplaylinenum}    \var{{\devanagarifontvar\numemph\va शौचाशौच॰\lem \mssALL,\hskip.2em plus .9em शौचाशुच \msCb\oo 
 यदि\lem \mssALL,\hskip.2em plus .9em यदिः \msM\oo 
 कालक्षये निश्चयः\lem \msNaacorr\msNc,\hskip.2em plus .9em 
कालक्षयैर्निश्चयः \msCa\msCb\msNapcorr\msKOb,\hskip.5em plus .9em 
कालक्षयेन्निश्चयः \msCc\msNb,\hskip.5em plus .9em 
कालक्षयानिश्चयः \msM,\hskip.5em plus .9em 
कालक्षयेतिश्च यः \Ed}}% 


\nemslokab

{\devanagarifont सौभाग्यत्वमवाप्नुवन्ति सततं कीर्तिर्यशोऽलङ्कृतः  \danda\dontdisplaylinenum }%
     \var{{\devanagarifontvar\numnoemph\vb कीर्तिर्यशो॰\lem \msCb\msNa\msNb\msNc\msKOb\Ed,\hskip.2em plus .9em कीर्तियशो॰ \msCa\msCc \unmetr,\hskip.5em plus .9em कीर्तिर्यषा॰ \msM\oo 
 ॰लङ्कृतः\lem \mssALL,\hskip.2em plus .9em ॰लकृतः \msCb,\hskip.5em plus .9em ॰लंकृतम् \msM}}% 
    \paral{{\devanagarifontsmall \vb {\englishfont \similar\ 4.67b above:}
                         लोके ऽनिन्दनमाप्नुवन्ति सततं कीर्तिर्यशोऽलंकृतम् }}

\nemslokac

{\devanagarifont प्राप्तं तेन इहैव पुण्यसकलं सद्धर्मशास्त्रेरितं }%
  \dontdisplaylinenum    \var{{\devanagarifontvar\numnoemph\vc सद्धर्म॰\lem \mssALL,\hskip.2em plus .9em य धर्म॰ \msM\oo 
 ॰एरितम्\lem \mssALL,\hskip.2em plus .9em ॰ओदितः \Ed}}% 

%Verse 5:20


\nemslokad

{\devanagarifont जीवान्ते च परत्रमीहितगतिं प्राप्नोति निःसंशयम् {॥ ५:\hspace{.11em}२०॥} \veg\dontdisplaylinenum }%
     \var{{\devanagarifontvar\numnoemph\vd परत्रमीहित॰\lem \mssALL,\hskip.2em plus .9em 
परत्रमीहत॰ \msM,\hskip.5em plus .9em पवित्रमीहित॰ \Ed\oo 
 ॰गतिं\lem \eme,\hskip.2em plus .9em ॰गतिः \mssCaCbCc\msNa\msNb\msNc\msM\msKOb\Ed\oo 
 निःसंशयम्\lem \msCa\msNb\msNc\msKOb,\hskip.2em plus .9em निःसंशयः \msCb\msCc\msNa\msM\Ed}}% 

\vers


{\devanagarifont 
\jump
\begin{center}
\ketdanda~इति वृषसारसंग्रहे शौचाचारविधिर्नामाध्यायः पञ्चमः~\ketdanda
\end{center}
\dontdisplaylinenum\vers  }%
     \var{{\devanagarifontvar\numnoemph{\englishfont \Colo:} वृषसार॰\lem \mssALL,\hskip.2em plus .9em वृषार॰ \msKOb\oo 
 ॰विधिर्नमा॰\lem \msCa\msKOb,\hskip.2em plus .9em ॰विधिनामा॰ \msCb\msCc\msNa\msNc\msM,\hskip.5em plus .9em 
\uncl{विंधि}नामा॰ \msNb,\hskip.5em plus .9em 
॰विधिर्नाम \Ed\oo 
 ॰ध्ययः पञ्चमः\lem \mssALL,\hskip.2em plus .9em ॰ध्यायः पञ्चमः श्लोक २५ \msM,\hskip.5em plus .9em 
पञ्चमो ऽध्यायः \Ed}}% 
\bekveg\szamveg
\vfill
\phpspagebreak

\versno=0\fejno=6
\thispagestyle{empty}

\centerline{\Large\devanagarifontbold [   षष्ठो ऽध्यायः  ]}{\vrule depth10pt width0pt} \fancyhead[CE]{{\footnotesize\devanagarifont वृषसारसंग्रहे  }}
\fancyhead[CO]{{\footnotesize\devanagarifont षष्ठो ऽध्यायः  }}
\fancyhead[LE]{}
\fancyhead[RE]{}
\fancyhead[LO]{}
\fancyhead[RO]{}
\szam\bek


\nemslokanormal



\alalfejezet{नियमेष्विज्या (२)}
\vers


{\devanagarifont अथ पञ्चविधामिज्यां प्रवक्ष्यामि द्विजोत्तम \thinspace{\dandab} \dontdisplaylinenum }%
     \var{{\devanagarifontvar\numemph\va ॰मिज्यां\lem \msCb\msKOb,\hskip.2em plus .9em ॰मीज्यां \msCa\msCc\msNa\msNb\msNc\Ed}}% 
    \var{{\devanagarifontvar\numnoemph\vb ॰त्तम\lem \mssALL,\hskip.2em plus .9em ॰त्तमः \msNb\msNc}}% 
    \lacuna{\devanagarifontsmall {\englishfont Witnesses used for this chapter: \msCa\ ff.\thinspace 202r--203r, 
                                              \msCb\ ff.\thinspace 209r--209v, 
                                              \msCc\ ff.\allowbreak\thinspace 278r--279r,
                                              \msNa\ ff.\thinspace 9v--10v, 
                                              \msNb\ exp.\thinspace 51 (lower--upper) -- 52 (lower),
                                              \msNc\ ff.\thinspace 218r--\allowbreak 218v,
                                              \msKOb\ ff.\thinspace 218r--209r,
                                              \Ed\ pp.\thinspace 599--601;  
                                              \mssCaCbCc\ = \msCa + \msCb + \msCc} }%
  
%Verse 6:1

{\devanagarifont धर्ममोक्षप्रसिद्ध्यर्थं शृणुष्वावहितो द्विज {॥ ६:\hspace{.11em}१॥} \veg\dontdisplaylinenum }%
     \var{{\devanagarifontvar\numnoemph\vc ॰मोक्षप्रसिद्ध्यर्थं\lem \mssCaCbCc\msNc\msKOb,\hskip.2em plus .9em ॰मोक्षप्रसिद्ध्यर्थ \msNa\msNb,\hskip.5em plus .9em 
॰मोक्षेशसिद्ध्यर्थं \Ed}}% 
    \var{{\devanagarifontvar\numnoemph\vd द्विज\lem \mssALL,\hskip.2em plus .9em भव \Ed}}% 

{\devanagarifont अर्थयज्ञः क्रियायज्ञो जपयज्ञस्तथैव च \thinspace{\dandab} \dontdisplaylinenum }%
     \var{{\devanagarifontvar\numemph\va अर्थयज्ञः\lem \msCa\msCc\msNa\msKOb,\hskip.2em plus .9em अनर्थयज्ञः \msCb,\hskip.5em plus .9em 
अर्थयज्ञ \msNb\msNc,\hskip.5em plus .9em अर्थयज्ञ॰ \Ed}}% 

%Verse 6:2

{\devanagarifont ज्ञानं ध्यानं च पञ्चैतत्प्रवक्ष्यामि पृथक्पृथक् {॥ ६:\hspace{.11em}२॥} \veg\dontdisplaylinenum }%
     \var{{\devanagarifontvar\numnoemph\vc ज्ञानं\lem \mssALL,\hskip.2em plus .9em ज्ञान \msCc\msNc}}% 


\alalalfejezet{अर्थयज्ञः}

{\devanagarifont अग्न्युपासनकर्मादि अग्निहोत्रक्रतुक्रिया \thinspace{\dandab} \dontdisplaylinenum }%
     \var{{\devanagarifontvar\numemph\vb अग्नि॰\lem \mssALL,\hskip.2em plus .9em \uncl{अ}\lacwithnum{1}\  \msCa,\hskip.5em plus .9em \lk\lk\ \msNb\oo 
 ॰क्रिया\lem \mssALL,\hskip.2em plus .9em ॰क्रियाः \msCb\msCc}}% 

%Verse 6:3

{\devanagarifont अष्टका पार्वणी श्राद्धं द्रव्ययज्ञः स उच्यते {॥ ६:\hspace{.11em}३॥} \veg\dontdisplaylinenum }%
     \var{{\devanagarifontvar\numnoemph\vc पार्वणी\lem \mssALL,\hskip.2em plus .9em पर्वणी \msCb,\hskip.5em plus .9em \uncl{पर्वणी} \msNb}}% 
    \var{{\devanagarifontvar\numnoemph\vd ॰यज्ञः\lem \mssALL,\hskip.2em plus .9em ॰यज्ञ \msCc,\hskip.5em plus .9em \lk\lk\ \msNb}}% 


\alalalfejezet{क्रियायज्ञः}

{\devanagarifont आरामोद्यानवापीषु देवतायतनेषु च \thinspace{\dandab} \dontdisplaylinenum }%
     \var{{\devanagarifontvar\numemph\vb ॰यतनेषु\lem \msCb\msCc\Ed,\hskip.2em plus .9em ॰लयनेषु \msCa\msNa\msNc\msKOb,\hskip.5em plus .9em ॰यत\lk\lk\ \msNb}}% 

%Verse 6:4

{\devanagarifont स्वहस्तकृतसंस्कारः क्रियायज्ञ स उच्यते {॥ ६:\hspace{.11em}४॥} \veg\dontdisplaylinenum }%
     \var{{\devanagarifontvar\numnoemph\vc ॰हस्त॰\lem \mssALL,\hskip.2em plus .9em \lk\lk\ \msNb,\hskip.5em plus .9em ॰हस्तैः \Ed}}% 


\alalalfejezet{जपयज्ञः}

{\devanagarifont जपयज्ञं ततो वक्ष्ये स्वर्गमोक्षफलप्रदम् \thinspace{\dandab} \dontdisplaylinenum }%
     \var{{\devanagarifontvar\numemph\va ॰यज्ञं ततो\lem \mssALL,\hskip.2em plus .9em ॰यज्ञं तपो \msCb ॰यज्ञस्ततो \msCc}}% 

{\devanagarifont वेदाध्ययन कर्तव्यं शिवसंहितमेव च  \danda\dontdisplaylinenum }%
     \var{{\devanagarifontvar\numnoemph\vc वेदा॰\lem \mssALL,\hskip.2em plus .9em अदा॰ \msNb}}% 

%Verse 6:5

{\devanagarifont इतिहासपुराणं च जपयज्ञः स उच्यते {॥ ६:\hspace{.11em}५॥} \veg\dontdisplaylinenum }%
     \var{{\devanagarifontvar\numnoemph\ve ॰पुराणं च\lem \mssALL,\hskip.2em plus .9em ॰पुराणश्च \Ed}}% 
    \var{{\devanagarifontvar\numnoemph\vf ॰यज्ञः\lem \mssALL,\hskip.2em plus .9em ॰यज्ञ \msCc}}% 


\alalalfejezet{ज्ञानयज्ञः}

{\devanagarifont इदं कर्म अकर्मेदमूहापोहविशारदः \thinspace{\dandab} \dontdisplaylinenum }%
     \var{{\devanagarifontvar\numemph\va कर्म\lem \mssALL,\hskip.2em plus .9em क्रमम् \Ed}}% 

%Verse 6:6

{\devanagarifont शास्त्रचक्षुः समालोक्य ज्ञानयज्ञः स उच्यते {॥ ६:\hspace{.11em}६॥} \veg\dontdisplaylinenum }%
     \var{{\devanagarifontvar\numnoemph\vc ॰चक्षुः\lem \mssALL,\hskip.2em plus .9em ॰चक्षु \msCc}}% 
    \var{{\devanagarifontvar\numnoemph\vd ॰यज्ञः\lem \mssALL,\hskip.2em plus .9em ॰यज्ञ \msCc,\hskip.5em plus .9em ॰\uncl{यज्ञस्} \msNb}}% 


\alalalfejezet{ध्यानयज्ञः}

{\devanagarifont ध्यानयज्ञं समासेन कथयिष्यामि ते शृणु \thinspace{\dandab} \dontdisplaylinenum }%
     \var{{\devanagarifontvar\numemph\va ॰यज्ञं\lem \mssALL,\hskip.2em plus .9em ॰यज्ञ \msCc\msNb}}% 

{\devanagarifont ध्यानं पञ्चविधं चैव कीर्तितं हरिणा पुरा  \danda\dontdisplaylinenum }%
     \var{{\devanagarifontvar\numnoemph\vc ध्यानं\lem \mssALL,\hskip.2em plus .9em ध्यान \msNa\msNc}}% 

%Verse 6:7

{\devanagarifont सूर्यः सोमो ऽग्नि स्फटिकः सूक्ष्मं तत्त्वं च पञ्चमम् {॥ ६:\hspace{.11em}७॥} \veg\dontdisplaylinenum }%
     \var{{\devanagarifontvar\numnoemph\ve सोमो\lem \msCa\msCc\msNa\msNc\msKOb,\hskip.2em plus .9em सोमा॰ \msCb\msNb\Ed}}% 
    \var{{\devanagarifontvar\numnoemph\vf सूक्ष्मं तत्त्वं च पञ्चमम्\lem \msCb,\hskip.2em plus .9em 
सूक्ष्मं त\uncl{त्व}\lacwithnum{2}  ञ्चमम् \msCa,\hskip.5em plus .9em 
सूक्ष्मतत्त्वं च पञ्चमः \msCc\msNa\msNb,\hskip.5em plus .9em 
सूक्ष्मं तत्त्वञ्च पञ्चमः \msNc,\hskip.5em plus .9em 
सूक्ष्मं तत्त्वञ्च पञ्च\uncl{मः} \msKOb,\hskip.5em plus .9em 
सूक्ष्मां तत्त्वश्च पञ्चमम् \Ed}}% 

{\devanagarifont सूर्यमण्डलमादौ तु तत्त्वं प्रकृतिरुच्यते \thinspace{\dandab} \dontdisplaylinenum }%
 
%Verse 6:8

{\devanagarifont तस्य मध्ये शशिं ध्यायेत्तत्त्वं पुरुष उच्यते {॥ ६:\hspace{.11em}८॥} \veg\dontdisplaylinenum }%
     \var{{\devanagarifontvar\numemph\vc शशिं\lem \mssALL,\hskip.2em plus .9em शशि \msNb,\hskip.5em plus .9em शशिंन् \msNc}}% 
    \var{{\devanagarifontvar\numnoemph\vcd ध्यायेत्त॰\lem \mssALL,\hskip.2em plus .9em ध्याये त॰ \msCc}}% 

{\devanagarifont चन्द्रमण्डलमध्ये तु ज्वालामग्निं विचिन्तयेत् \thinspace{\dandab} \dontdisplaylinenum }%
     \var{{\devanagarifontvar\numemph\vb ज्वालामग्निं\lem \mssALL,\hskip.2em plus .9em जालामग्नि \msNc}}% 

%Verse 6:9

{\devanagarifont प्रभुतत्त्वः स विज्ञेयो जन्ममृत्युविनाशनः {॥ ६:\hspace{.11em}९॥} \veg\dontdisplaylinenum }%
     \var{{\devanagarifontvar\numnoemph\vc ॰तत्त्वः\lem \mssCaCbCc\msNc\msKOb,\hskip.2em plus .9em ॰तत्व \msNa,\hskip.5em plus .9em ॰तत्वं \msNb\Ed}}% 
    \var{{\devanagarifontvar\numnoemph\vd ॰नाशनः\lem \mssALL,\hskip.2em plus .9em ॰नाशनम् \msCc\Ed}}% 

{\devanagarifont अग्निमण्डलमध्ये तु ध्यायेत्स्फटिक निर्मलम् \thinspace{\dandab} \dontdisplaylinenum }%
     \var{{\devanagarifontvar\numemph\vb ध्यायेत्स्फटिक\lem \msCapcorr\msCb\msNa\msNb\msNc\msKOb,\hskip.2em plus .9em ध्यायेत्स्फटि \msCaacorr,\hskip.5em plus .9em 
ध्याये स्फटिक \msCc\Ed\oo 
 ॰मलम्\lem \mssALL,\hskip.2em plus .9em ॰मलः \msNa,\hskip.5em plus .9em ॰\uncl{मलः} \msNc}}% 

%Verse 6:10

{\devanagarifont विद्यातत्त्वः स विज्ञेयः कारणमजमव्ययम् {॥ ६:\hspace{.11em}१०॥} \veg\dontdisplaylinenum }%
     \var{{\devanagarifontvar\numnoemph\vc तत्त्वः स\lem \msCb\msNa\msNb\msNc\msKOb,\hskip.2em plus .9em त\uncl{त्वन्}\lacwithnum{1}\  \msCa,\hskip.5em plus .9em तत्व स \msCc,\hskip.5em plus .9em तत्वं स \Ed}}% 
    \var{{\devanagarifontvar\numnoemph\vd ॰जमव्ययम्\lem \mssALL,\hskip.2em plus .9em ॰मव्ययं \msCc}}% 

{\devanagarifont विद्यामण्डलमध्ये तु ध्यायेत्तत्त्वमनुत्तमम् \thinspace{\dandab} \dontdisplaylinenum }%
     \var{{\devanagarifontvar\numemph\vab ध्यायेत्त॰\lem \mssALL,\hskip.2em plus .9em ध्याये त॰ \msCc}}% 

{\devanagarifont अकीर्तितमनौपम्यं शिवमक्षयमव्ययम्  \danda\dontdisplaylinenum }%
     \paral{{\devanagarifontsmall \vcd {\englishfont \DHARMP\ 4.14ab: } अकीर्तितमनौपम्यं पञ्चमं शिवमण्डलम् }}

%Verse 6:11

{\devanagarifont पञ्चमं ध्यानयज्ञस्य तत्त्वमुक्तं समासतः {॥ ६:\hspace{.11em}११॥} \veg\dontdisplaylinenum }%
     \var{{\devanagarifontvar\numnoemph\ve ॰यज्ञस्य\lem \mssALL,\hskip.2em plus .9em ॰यज्ञञ्च \msCc\Ed}}% 
    \var{{\devanagarifontvar\numnoemph\vf समासतः\lem \mssALL,\hskip.2em plus .9em सनातनः \Ed}}% 

{\devanagarifont विगतराग उवाच {\dandab}\dontdisplaylinenum  }%
 
{\devanagarifont एकैकस्य तु तत्त्वस्य फलं कीर्तय कीदृशम् \thinspace{\danda} \dontdisplaylinenum }%
     \var{{\devanagarifontvar\numemph\va तु\lem \conj,\hskip.2em plus .9em त्रि॰ \mssCaCbCc\msNa\msNb\msNc\msKOb,\hskip.5em plus .9em हि \Ed}}% 

%Verse 6:12

{\devanagarifont कानि लोकाः प्रपद्यन्ते कालं वास्य तपोधन {॥ ६:\hspace{.11em}१२॥} \veg\dontdisplaylinenum }%
     \var{{\devanagarifontvar\numnoemph\vc लोकाः\lem \msCa\msNa\msNc,\hskip.2em plus .9em लोका \msCb\msCc\msNb\msKOb\Ed\oo 
 प्रपद्यन्ते\lem \mssALL,\hskip.2em plus .9em प्र\lk\lk\lk\ \msCa}}% 
    \var{{\devanagarifontvar\numnoemph\vd वास्य\lem \mssALL,\hskip.2em plus .9em चास्य \msNa\msKOb\oo 
 ॰धन\lem \mssALL,\hskip.2em plus .9em ॰धनः \msCb\msNc}}% 

{\devanagarifont अनर्थयज्ञ उवाच {\dandab}\dontdisplaylinenum  }%
 
{\devanagarifont ब्रह्मलोकं तु प्रथमं तत्त्वप्रकृतिचिन्तया \thinspace{\danda} \dontdisplaylinenum }%
     \var{{\devanagarifontvar\numemph\vab प्रथमं तत्त्व॰\lem \mssCaCbCc\msNapcorr\msNb\msNc,\hskip.2em plus .9em 
\om\ \msNaacorr,\hskip.5em plus .9em प्रथमं तत्त्वं \msKOb\Ed\oo 
 प्रकृतिचिन्तया\lem \mssALL,\hskip.2em plus .9em च कृतिचिन्तय \Ed}}% 

%Verse 6:13

{\devanagarifont कल्पकोटिसहस्राणि शिववन्मोदते सुखी {॥ ६:\hspace{.11em}१३॥} \veg\dontdisplaylinenum }%
     \var{{\devanagarifontvar\numnoemph\vd सुखी\lem \mssALL,\hskip.2em plus .9em सुखम् \Ed}}% 

{\devanagarifont द्वितीयं तत्त्व पुरुषं ध्यायमानो मृतो यदि \thinspace{\dandab} \dontdisplaylinenum }%
 
%Verse 6:14

{\devanagarifont विष्णुलोकमितो याति कल्पकोट्ययुतं सुखी {॥ ६:\hspace{.11em}१४॥} \veg\dontdisplaylinenum }%
     \var{{\devanagarifontvar\numemph\vc याति\lem \mssALL,\hskip.2em plus .9em यान्ति \Ed}}% 

{\devanagarifont प्रभुतत्त्वं तृतीयं तु ध्यायमानो मरिष्यति \thinspace{\dandab} \dontdisplaylinenum }%
     \var{{\devanagarifontvar\numemph\va ॰तत्त्वं\lem \mssALL,\hskip.2em plus .9em ॰तत्व \msCc\oo 
 तृतीयं\lem \mssALL,\hskip.2em plus .9em तृतीयस् \Ed}}% 
    \var{{\devanagarifontvar\numnoemph\vb ध्यायमानो मरिष्यति\lem \mssALL,\hskip.2em plus .9em ध्याय\lk\lk \lk रिष्यति \msCa,\hskip.5em plus .9em 
धयायामानो मरिष्यति \Ed}}% 

%Verse 6:15

{\devanagarifont शिवलोके वसेन्नित्यं कल्पकोट्ययुतं शतम् {॥ ६:\hspace{.11em}१५॥} \veg\dontdisplaylinenum }%
     \var{{\devanagarifontvar\numnoemph\vc शिवलोके\lem \mssALL,\hskip.2em plus .9em शिवलोक \msCb,\hskip.5em plus .9em रुद्रलोके \Ed\oo 
 वसेन्नि॰\lem \mssALL,\hskip.2em plus .9em वसे नि॰ \msCc}}% 
    \var{{\devanagarifontvar\numnoemph\vd ॰युतं\lem \mssALL,\hskip.2em plus .9em ॰युत \msNb}}% 

{\devanagarifont विद्यातत्त्वामृतं ध्यायेत्सदाशिवमनामयम् \thinspace{\dandab} \dontdisplaylinenum }%
     \var{{\devanagarifontvar\numemph\va ॰तत्त्वामृतं\lem \mssALL,\hskip.2em plus .9em ॰तत्वमृतन् \msCc,\hskip.5em plus .9em ॰तत्त्वामतं \Ed}}% 

%Verse 6:16

{\devanagarifont अक्षयं लोकमाप्नोति कल्पानान्तपरं तथा {॥ ६:\hspace{.11em}१६॥} \veg\dontdisplaylinenum  }%
     \var{{\devanagarifontvar\numnoemph\vc अक्षयं\lem \mssALL,\hskip.2em plus .9em अक्षय॰ \Ed}}% 

{\devanagarifont पञ्चमं शिवतत्त्वं तु सूक्ष्मं चात्मनि संस्थितम् \thinspace{\dandab} \dontdisplaylinenum }%
 
%Verse 6:17

{\devanagarifont न कालसंख्या तत्रास्ति शिवेन सह मोदते {॥ ६:\hspace{.11em}१७॥} \veg\dontdisplaylinenum }%
 
\nemslokalong


\ujvers\nemsloka {
{\devanagarifont पञ्चध्यानाभियुक्तो भवति च न पुनर्जन्मसंस्कारबन्धः }%
  \dontdisplaylinenum}    \var{{\devanagarifontvar\numemph\va ॰युक्तो\lem \mssALL,\hskip.2em plus .9em ॰यु\lk\ \msCa\ \toplost,\hskip.5em plus .9em ॰युक्तौ \Ed\oo 
 च\lem \mssALL,\hskip.2em plus .9em \om\ \msCb\Ed\oo 
 पुनर्जन्म॰\lem \mssALL,\hskip.2em plus .9em 
पुन\uncl{ज}न्म॰ \msCa\ \toplost,\hskip.5em plus .9em पुनजन्म॰ \msCc}}% 


\nemslokab

{\devanagarifont जिज्ञास्यन्तां द्विजेन्द्र भवदहनकरः प्रार्थनाकल्पवृक्षो  \danda\dontdisplaylinenum }%
     \var{{\devanagarifontvar\numnoemph\vb जिज्ञास्यन्तां\lem \msCa\msNb\msNc\msKOb\Ed,\hskip.2em plus .9em जिज्ञास्यतां \msCb\msNa\ \unmetr,\hskip.5em plus .9em जिज्ञास्यन्ता \msCc}}% 

\nemslokac

{\devanagarifont जन्मेनैकेन मुक्तिर्भवति किमु न वा मानवाः साधयन्तु }%
  \dontdisplaylinenum    \var{{\devanagarifontvar\numnoemph\vc जन्मेनैकेन\lem \msCb\msNb\msNc\Ed,\hskip.2em plus .9em जन्मनैकेन \msCa\msCc\msNa\msKOb\ \unmetr\oo 
 मुक्तिर्भ॰\lem \mssALL,\hskip.2em plus .9em मुक्ति भ॰ \msCc\oo 
 न वा\lem \mssALL,\hskip.2em plus .9em भवा \msNa\oo 
 मानवाः\lem \msCa\msNa\msNb\msNc\msKOb,\hskip.2em plus .9em मानमानवाः \msCb,\hskip.5em plus .9em मानवा \msCc,\hskip.5em plus .9em मानव \Ed}}% 

%Verse 6:18


\nemslokad

{\devanagarifont प्रत्यक्षान्नानुमानं सकलमलहरं स्वात्मसंवेदनीयम् {॥ ६:\hspace{.11em}१८॥} \veg\dontdisplaylinenum }%
     \var{{\devanagarifontvar\numnoemph\vd प्रत्यक्षा॰\lem \mssALL,\hskip.2em plus .9em प्रत्यक्ष॰ \msNa\oo 
 ॰वेदनीयम्\lem \msCb\msNa\msNb\msKOb,\hskip.2em plus .9em ॰वेदनीयः \msCa\msCc\msNc,\hskip.5em plus .9em ॰वेदनीय \Ed}}% 

\nemslokanormal


\vers



\alalfejezet{नियमेषु तपः (३)}
{\devanagarifont मानसं तप आदौ तु द्वितीयं वाचिकं तपः \thinspace{\dandab} \dontdisplaylinenum }%
     \var{{\devanagarifontvar\numemph\va तप\lem \mssALL,\hskip.2em plus .9em ॰तपम् \Ed}}% 

{\devanagarifont कायिकं च तृतीयं तु मनोवाक्कर्म तत्परम्  \danda\dontdisplaylinenum }%
     \var{{\devanagarifontvar\numnoemph\vc कायिकं च तृतीयं तु\lem \mssALL,\hskip.2em plus .9em 
मानसं तप आदौ तु \msNb\ {\englishfont (eyeskip)}}}% 
    \var{{\devanagarifontvar\numnoemph\vd मनोवाक्कर्म\lem \msCa\msNc\msKOb\Ed,\hskip.2em plus .9em मनोक्कर्म \msCb,\hskip.5em plus .9em म्मनोवाकर्म॰ \msCc,\hskip.5em plus .9em मनोवाक्काय॰ \msNa\msNb\oo 
 ॰परम्\lem \msCc\msKOb,\hskip.2em plus .9em ॰परः \msCa\msCb\msNa\msNb\msNc\Ed}}% 

%Verse 6:19

{\devanagarifont कायिकं वाचिकं चैव तपो मिश्रक पञ्चमम् {॥ ६:\hspace{.11em}१९॥} \veg\dontdisplaylinenum }%
     \var{{\devanagarifontvar\numnoemph\ve कायिकं\lem \mssALL,\hskip.2em plus .9em कायिक \msNa\msKOb}}% 

{\devanagarifont मनःसौम्यं प्रसादश्च आत्मनिग्रहमेव च \thinspace{\dandab} \dontdisplaylinenum }%
     \var{{\devanagarifontvar\numemph\va ॰सौम्यं\lem \msNc,\hskip.2em plus .9em ॰सौम्य॰ \msCa\msCb\msNa\msNb\msKOb\Ed,\hskip.5em plus .9em ॰सौम्\uncl{य}॰ \msCc\ \toplost\oo 
 प्रसादश्च\lem \msCa\msCc\msNa\msNc\msKOb,\hskip.2em plus .9em प्रसादं च \msCb\Ed,\hskip.5em plus .9em प्रदानश्च \msNb}}% 

%Verse 6:20

{\devanagarifont मौनं भावविशुद्धिश्च पञ्चैतत्तप मानसम् {॥ ६:\hspace{.11em}२०॥} \veg\dontdisplaylinenum }%
     \var{{\devanagarifontvar\numnoemph\vc मौनं\lem \mssALL,\hskip.2em plus .9em मौन\lk  \Ed\oo 
 ॰शुद्धिश्च\lem \msCa\msCb\msNa\msNb\msNc,\hskip.2em plus .9em ॰शुद्धिं च \msCc\msKOb\Ed}}% 
    \var{{\devanagarifontvar\numnoemph\vd पञ्चैतत्\lem \msCa\msNb\msNc\msKOb,\hskip.2em plus .9em पञ्चैते \msCb\msNa,\hskip.5em plus .9em पञ्चेतत् \msCc,\hskip.5em plus .9em पञ्चैतन् \Ed}}% 
    \paral{{\devanagarifontsmall \vo {\englishfont \similar\ \MBH\ 6.39.16 (\BHG\ 17.16):}
                 मनःप्रसादः सौम्यत्वं मौनमात्मविनिग्रहः\thinspace{\devanagarifontsmall ।}
                 भावसंशुद्धिरित्येतत्तपो मानसमुच्यते\thinspace{\devanagarifontsmall ॥} }}

{\devanagarifont अनुद्वेगकरा वाणी प्रियं सत्यं हितं च यत् \thinspace{\dandab} \dontdisplaylinenum }%
 
%Verse 6:21

{\devanagarifont स्वाध्यायाभ्यसनं चैव वाचिकं तप उच्यते {॥ ६:\hspace{.11em}२१॥} \veg\dontdisplaylinenum }%
     \var{{\devanagarifontvar\numemph\vc ॰भ्यसनं चैव\lem \msCb\msCc\msNa\msNc\Ed,\hskip.2em plus .9em ॰भ्यसन\lk\lk\ \msCa,\hskip.5em plus .9em ॰भ्यस\uncl{नं} चैव \msNb,\hskip.5em plus .9em 
॰भ्यसन \msKOb}}% 
    \paral{{\devanagarifontsmall \vcd {\englishfont \similar\ \MBH\ 6.39.15cd (\BHG\ 17.15):}
                                  अनुद्वेगकरं वाक्यं सत्यं प्रियहितं च यत्\thinspace{\devanagarifontsmall ।}
                                  स्वाध्यायाभ्यसनं चैव वाङ्मयं तप उच्यते\thinspace{\devanagarifontsmall ॥} }}

{\devanagarifont आर्जवं च अहिंसा च ब्रह्मचर्यं सुरार्चनम् \thinspace{\dandab} \dontdisplaylinenum }%
     \var{{\devanagarifontvar\numemph\va आर्जवं च अहिंसा च\lem \mssALL,\hskip.2em plus .9em आर्जवत्वमहिंसाश्च \Ed}}% 
    \var{{\devanagarifontvar\numnoemph\vb ॰चर्यं\lem \mssALL,\hskip.2em plus .9em ॰चर्य \msCc\Ed}}% 

%Verse 6:22

{\devanagarifont शौचं पञ्चममित्येतत्कायिकं तप उच्यते {॥ ६:\hspace{.11em}२२॥} \veg\dontdisplaylinenum }%
     \var{{\devanagarifontvar\numnoemph\vc शौचं\lem \mssALL,\hskip.2em plus .9em शौच \Ed}}% 
    \paral{{\devanagarifontsmall \vo {\englishfont \compare\ \MBH\ 6.39.14 (\BHG\ 17.14):}
                          देवद्विजगुरुप्राज्ञपूजनं शौचमार्जवम्\thinspace{\devanagarifontsmall ।}
                          ब्रह्मचर्यमहिंसा च शारीरं तप उच्यते\thinspace{\devanagarifontsmall ॥} }}

{\devanagarifont इष्टं कल्याणभावं च धन्यं पथ्यं हितं वदेत् \thinspace{\dandab} \dontdisplaylinenum }%
     \var{{\devanagarifontvar\numemph\va इष्टं\lem \mssALL,\hskip.2em plus .9em इष्ट \msCc\msNb\oo 
 ॰भावं\lem \mssALL,\hskip.2em plus .9em ॰भावश् \Ed}}% 
    \var{{\devanagarifontvar\numnoemph\vb पथ्यं\lem \mssALL,\hskip.2em plus .9em सत्यं \Ed}}% 

%Verse 6:23

{\devanagarifont मनोमिश्रक पञ्चैतत्तप उक्तं महर्षिभिः {॥ ६:\hspace{.11em}२३॥} \veg\dontdisplaylinenum }%
     \var{{\devanagarifontvar\numnoemph\vc मनो॰\lem \mssALL,\hskip.2em plus .9em मन॰ \Ed\oo 
 पञ्चैतत्\lem \mssALL,\hskip.2em plus .9em पञ्चेतत् \msNc,\hskip.5em plus .9em पञ्चैतान् \Ed}}% 
    \var{{\devanagarifontvar\numnoemph\vd तप उक्तं महर्षिभिः\lem \mssALL,\hskip.2em plus .9em तपमुक्तं महिर्षिभिः \Ed}}% 

{\devanagarifont स्वस्ति मङ्गलमाशीर्भिरतिथिगुरुपूजनम् \thinspace{\dandab} \dontdisplaylinenum }%
     \var{{\devanagarifontvar\numemph\va ॰शीर्भि॰\lem \msCa\Ed,\hskip.2em plus .9em ॰शीभि॰ \msCb\msCc\msNa\msNb\msNc\msKOb}}% 
    \var{{\devanagarifontvar\numnoemph\vb ॰तिथि॰\lem \mssALL,\hskip.2em plus .9em ॰तिथिं \Ed}}% 
    \paral{{\devanagarifontsmall \vab {\englishfont \compare\ \SDHS\ 11.79:}
                 नमस्काराभिवादेषु स्वस्तिमङ्गलवाचकैः\thinspace{\devanagarifontsmall ।}
                 शिवं भवतु सर्वत्र प्रब्रूयात्सर्वकर्मसु\thinspace{\devanagarifontsmall ॥} }}

%Verse 6:24

{\devanagarifont कायमिश्रक पञ्चैतत्तप उक्तं महात्मभिः {॥ ६:\hspace{.11em}२४॥} \veg\dontdisplaylinenum }%
     \var{{\devanagarifontvar\numnoemph\vc ॰मिश्रक\lem \mssALL,\hskip.2em plus .9em ॰\lk\lk क \msCa,\hskip.5em plus .9em ॰मित्यश्रक \msCb\oo 
 पञ्चैतत्\lem \mssALL,\hskip.2em plus .9em पञ्चैतन् \Ed}}% 
    \var{{\devanagarifontvar\numnoemph\vd तप उक्तं\lem \mssALL,\hskip.2em plus .9em तपमुक्तं \Ed}}% 

{\devanagarifont मण्डूकयोगी हेमन्ते ग्रीष्मे पञ्चतपास्तथा \thinspace{\dandab} \dontdisplaylinenum }%
     \var{{\devanagarifontvar\numemph\vb ग्रीष्मे\lem \mssALL,\hskip.2em plus .9em गृष्मे \Ed}}% 
    \paral{{\devanagarifontsmall \vab {\englishfont \similar\ \MBH\ Suppl. 15.801:}
                                 मण्डूकशायी हेमन्ते ग्रीष्मे पञ्चतपा भवेत्
                     {\englishfont \similar\ \UMS\ 6.26ab:}मण्डूकयोगो हेमन्ते ग्रीष्मे पञ्चतपास्तथा;
                     {\englishfont \compare\ \SDHSAMGR\ 9.32ab:}
                         अभ्रावकाश्यं शीतोष्णे पञ्चाग्निर्जलशायिता }}

%Verse 6:25

{\devanagarifont अभ्रावकाशो वर्षासु तपःसाधनमुच्यते {॥ ६:\hspace{.11em}२५॥} \veg\dontdisplaylinenum }%
     \var{{\devanagarifontvar\numnoemph\vc ॰वकाशो\lem \eme,\hskip.2em plus .9em ॰वकाशे \mssCaCbCc\msNa\msNb\msNc\msKOb\Ed}}% 
    \var{{\devanagarifontvar\numnoemph\vd तप॰\lem \mssALL,\hskip.2em plus .9em तप \msCc\oo 
 ॰साधनमु॰\lem \msCa\msNa\msNc\msKOb\Ed,\hskip.2em plus .9em ॰साधन उ॰ \msCb\msCc\msNb}}% 

{\devanagarifont स्वमांसोद्धृत्य दानं च हस्तपादशिरस्तथा \thinspace{\dandab} \dontdisplaylinenum }%
     \var{{\devanagarifontvar\numemph\va दानं\lem \mssALL,\hskip.2em plus .9em \uncl{दान} \msNb\ \toplost,\hskip.5em plus .9em दानश् \Ed}}% 

%Verse 6:26

{\devanagarifont पुष्पमुत्पाद्य दानं च सर्वे ते तपसाधनाः {॥ ६:\hspace{.11em}२६॥} \veg\dontdisplaylinenum }%
     \var{{\devanagarifontvar\numnoemph\vc दानं\lem \mssALL,\hskip.2em plus .9em दानश् \Ed}}% 
    \var{{\devanagarifontvar\numnoemph\vd तप॰\lem \Ed,\hskip.2em plus .9em तपः \mssCaCbCc\msNa\msNb\msNc\msKOb\ \unmetr}}% 

{\devanagarifont कृच्छ्रातिकृच्छ्रं नक्तं च तप्तकृच्छ्रमयाचितम् \thinspace{\dandab} \dontdisplaylinenum }%
     \var{{\devanagarifontvar\numemph\va कृच्छ्रातिकृच्छ्रं\lem \msCa\msCb\msNa\msKOb\Ed,\hskip.2em plus .9em 
कृच्छ्रादिकृच्छ्र \msCc,\hskip.5em plus .9em कृच्छ्रातिकृच्छ्र \msNb,\hskip.5em plus .9em कृच्छातिकृच्छं \msNc}}% 
    \var{{\devanagarifontvar\numnoemph\vb तप्त॰\lem \mssALL,\hskip.2em plus .9em तप॰ \msKOb\oo 
 ॰याचितम्\lem \mssALL,\hskip.2em plus .9em ॰याचितः \Ed}}% 

%Verse 6:27

{\devanagarifont चान्द्रायणं पराकं च तपः सांतपनादयः {॥ ६:\hspace{.11em}२७॥} \veg\dontdisplaylinenum }%
     \var{{\devanagarifontvar\numnoemph\vc चान्द्रायणं पराकं\lem \msCa\msCc\msNb\msNc\msKOb,\hskip.2em plus .9em चान्द्रायनं पराकं \msCb,\hskip.5em plus .9em 
चन्द्रायणं पराकं \msNa,\hskip.5em plus .9em चान्द्रायणवराकश् \Ed}}% 
    \var{{\devanagarifontvar\numnoemph\vd तपः सांतपनादयः\lem \mssALL,\hskip.2em plus .9em तपसान्तपनादयः \msCc\Ed}}% 

\nemslokalong


\ujvers\nemsloka {
{\devanagarifont येनेदं तप तप्यते सुमनसा संसारदुःखच्छिदम् }%
  \dontdisplaylinenum}    \var{{\devanagarifontvar\numemph\va तप त॰\lem \Ed,\hskip.2em plus .9em तपस्त॰ \mssCaCbCc\msNa\msNb\msNc\msKOb\ \unmetr\oo 
 ॰मनसा\lem \eme,\hskip.2em plus .9em ॰मनसः \mssCaCbCc\msNa\msNb\msNc\msKOb\Ed\oo 
 ॰दुःख॰\lem \mssALL,\hskip.2em plus .9em ॰दुःखं \msKOb}}% 


\nemslokab

{\devanagarifont आशापाश विमुच्य निर्मलमतिस्त्यक्त्वा जघन्यं फलम्  \danda\dontdisplaylinenum }%
     \var{{\devanagarifontvar\numnoemph\vb निर्मलमति॰\lem \mssALL,\hskip.2em plus .9em निर्मलर्मति॰ \msCb\oo 
 जघन्यं\lem \mssALL,\hskip.2em plus .9em जगत्यं \Ed}}% 

\nemslokac

{\devanagarifont स्वर्गाकाङ्क्ष्यनृपत्वभोगविषयं सर्वान्तिकं तत्फलं }%
  \dontdisplaylinenum    \var{{\devanagarifontvar\numnoemph\vc ॰काङ्क्ष्य॰\lem \mssALL,\hskip.2em plus .9em ॰कांक्ष॰ \Ed\oo 
 सर्वान्तिकं\lem \mssALL,\hskip.2em plus .9em सर्वार्त्तिकं \msCb}}% 

%Verse 6:28


\nemslokad

{\devanagarifont जन्तुः शाश्वतजन्ममृत्युभवने तन्निष्ठसाध्यं वहेत् {॥ ६:\hspace{.11em}२८॥} \veg\dontdisplaylinenum }%
     \var{{\devanagarifontvar\numnoemph\vd ॰भवने\lem \mssALL,\hskip.2em plus .9em ॰भवेने \msNc\oo 
 ॰साध्यं वहेत्\lem \msCc\msNa\msNb\msNc\msKOb,\hskip.2em plus .9em ॰\uncl{साध्यम्}\lk\lk\ \msCa,\hskip.5em plus .9em 
॰साध्य वहेत् \msCb,\hskip.5em plus .9em ॰साध्यं वदेत् \Ed}}% 

\vers


{\devanagarifont 
\jump
\begin{center}
\ketdanda~इति वृषसारसंग्रहे षष्ठो ऽध्यायः~\ketdanda
\end{center}
\dontdisplaylinenum\vers  }%
 
\nemslokanormal

\bekveg\szamveg
\vfill
\phpspagebreak

\versno=0\fejno=7
\thispagestyle{empty}


\vers

\centerline{\Large\devanagarifontbold [   सप्तमो ऽध्यायः  ]}{\vrule depth10pt width0pt} \fancyhead[CE]{{\footnotesize\devanagarifont वृषसारसंग्रहे  }}
\fancyhead[CO]{{\footnotesize\devanagarifont सप्तमो ऽध्यायः  }}
\fancyhead[LE]{}
\fancyhead[RE]{}
\fancyhead[LO]{}
\fancyhead[RO]{}
\szam\bek



\alalfejezet{नियमेषु दानम् (४)}
{\devanagarifont दानानि च तथेत्याहुः पञ्चधा मुनिभिः पुरा \thinspace{\dandab} \dontdisplaylinenum }%
     \var{{\devanagarifontvar\numemph\va तथेत्याहुः\lem \mssALL,\hskip.2em plus .9em तथैत्याहुः \msCb\msNa}}% 
    \lacuna{\devanagarifontsmall {\englishfont Witnesses used for this chapter: \msCa\ ff.\thinspace 203r--204r, 
                                              \msCb\ ff.\thinspace 209v--210v, 
                                              \msCc\ ff.\allowbreak\thinspace 279r--280v,
                                              \msNa\ ff.\thinspace 10v--11v, 
                                              \msNb\ exp.\thinspace 52 (lower--upper) -- 53 (lower),
                                              \msNc\ ff.\allowbreak\thinspace 218v--\allowbreak 219v,
                                              \Ed\ pp.\thinspace 601--603; 
                                              \mssCaCbCc\ = \msCa + \msCb + \msCc} }%
  
%Verse 7:1

{\devanagarifont अन्नं वस्त्रं हिरण्यं च भूमि गोदान पञ्चमम् {॥ ७:\hspace{.11em}१॥} \veg\dontdisplaylinenum }%
     \var{{\devanagarifontvar\numnoemph\vc वस्त्रं\lem \mssALL,\hskip.2em plus .9em वस्त्र \msCc\msNb}}% 


\alalalfejezet{अन्नदानम्}

{\devanagarifont अन्नात्तेजः स्मृतिः प्राणः अन्नात्पुष्टिर्वपुः सुखम् \thinspace{\dandab} \dontdisplaylinenum }%
     \var{{\devanagarifontvar\numemph\va अन्नात्तेजः स्मृतिः प्राणः\lem \mssCaCbCc\msNapcorr\msNb,\hskip.2em plus .9em अन्नात्तेजः स्मृतिः प्राण \msNaacorr,\hskip.5em plus .9em 
अन्नात्तेजः स्मृति प्राणः \msNc,\hskip.5em plus .9em 
अन्नाद्भवन्ति भूतानि \Ed}}% 

%Verse 7:2

{\devanagarifont अन्नाच्छ्रीः कान्ति वीर्यं च अन्नात्सत्त्वं च जायते {॥ ७:\hspace{.11em}२॥} \veg\dontdisplaylinenum }%
     \var{{\devanagarifontvar\numnoemph\vc अन्नाच्छ्रीः\lem \mssALL,\hskip.2em plus .9em अन्नाच्छ्री \msNb\Ed\oo 
 कान्ति वीर्यं च\lem \msCb\msCc\msNa\msNb,\hskip.2em plus .9em कान्तिर्वीर्यञ्च \msCa\msNc\ \unmetr,\hskip.5em plus .9em 
कान्तिवीर्श्यञ्च \Ed}}% 
    \var{{\devanagarifontvar\numnoemph\vd अन्नात्सत्त्वं च\lem \mssALL,\hskip.2em plus .9em 
अन्ना सत्वञ्च \msCc,\hskip.5em plus .9em अन्नात्सत्त्वश्च \Ed\oo 
 जायते\lem \mssALL,\hskip.2em plus .9em जाय\lk\  \msCa}}% 

{\devanagarifont अन्नाज्जीवन्ति भूतानि अन्नं तुष्टिकरं सदा \thinspace{\dandab} \dontdisplaylinenum }%
     \var{{\devanagarifontvar\numemph\va अन्नाज्जी॰\lem \msCa\msNa\msNb\Ed,\hskip.2em plus .9em अन्ना जी॰ \msCb\msCc\msNc}}% 
    \var{{\devanagarifontvar\numnoemph\vb अन्नं\lem \mssALL,\hskip.2em plus .9em अन्नां \msCc,\hskip.5em plus .9em अन्ना \msNb\oo 
 ॰करं\lem \mssALL,\hskip.2em plus .9em ॰करः \msCc\Ed}}% 

%Verse 7:3

{\devanagarifont आन्नात्कामो मदो दर्पः अन्नाच्छौर्यं च जायते {॥ ७:\hspace{.11em}३॥} \veg\dontdisplaylinenum }%
     \var{{\devanagarifontvar\numnoemph\vc दर्पः\lem \msCa\msCc\msNa\msNb,\hskip.2em plus .9em दर्प्प \msCb\msNc,\hskip.5em plus .9em दर्प्पो \Ed}}% 
    \var{{\devanagarifontvar\numnoemph\vd अन्नाच्छौर्यं च\lem \msCa\msCc\msNc,\hskip.2em plus .9em अन्नात्सौर्यञ्च \msCb\msNa\msNb,\hskip.5em plus .9em 
अन्नाच्छौर्यश्च \Ed}}% 

{\devanagarifont अन्नं क्षुधातृषाव्याधीन्सद्य एव विनाशयेत् \thinspace{\dandab} \dontdisplaylinenum }%
     \var{{\devanagarifontvar\numemph\va अन्नं क्षु॰\lem \msCa\msCb\msNapcorr\msNc,\hskip.2em plus .9em अन्ना क्षु॰ \msCc\msNaacorr,\hskip.5em plus .9em अन्नात्क्षु॰ \msNb\Ed}}% 
    \var{{\devanagarifontvar\numnoemph\vab ॰व्याधीन्स॰\lem \msCb\msNc,\hskip.2em plus .9em ॰व्याधान्स॰ \msCa\msCc\msNb,\hskip.5em plus .9em ॰वाधान्स॰ \msNa,\hskip.5em plus .9em 
॰व्याधा स॰ \Ed}}% 
    \var{{\devanagarifontvar\numnoemph\vb विनाशयेत्\lem \mssALL,\hskip.2em plus .9em विशयेत् \msCb}}% 

%Verse 7:4

{\devanagarifont अन्नदानाच्च सौभाग्यं ख्यातिः कीर्तिश्च जायते {॥ ७:\hspace{.11em}४॥} \veg\dontdisplaylinenum }%
 
{\devanagarifont अन्नदः प्राणदश्चैव प्राणदश्चापि सर्वदः \thinspace{\dandab} \dontdisplaylinenum }%
     \var{{\devanagarifontvar\numemph\va अन्नदः\lem \mssALL,\hskip.2em plus .9em अन्नद \Ed}}% 
    \var{{\devanagarifontvar\numnoemph\vb प्राणदश्चापि\lem \mssALL,\hskip.2em plus .9em प्राणश्चापि \msNb\oo 
 सर्वदः\lem \mssALL,\hskip.2em plus .9em सर्वदाः \msCc}}% 
    \paral{{\devanagarifontsmall \vo {\englishfont \similar\ \SDHU\ 1.27:}
                 अन्नदः प्राणदः प्रोक्तः प्राणदश्चापि सर्वदः\thinspace{\devanagarifontsmall ।}
                 तस्मादन्नप्रदानेन सर्वदानफलं लभेत्\thinspace{\devanagarifontsmall ॥}
                 \similar\ {\englishfont \MBH\ suppl 14.4.2285--86:}
                 अन्नदः प्राणदो लोके प्राणदः सर्वदो भवेत्\thinspace{\devanagarifontsmall ।}
                 तस्मादन्नं विशेषेण दातव्यं भूतिमिच्छता\thinspace{\devanagarifontsmall ॥}
                   \similar\ {\englishfont \NARADAP\ 1.13.71:}
                 अन्नदः प्राणदः प्रोक्तः प्राणदश्चापि सर्वदः\thinspace{\devanagarifontsmall ।}
                 सर्वदानफलं यस्मादन्नदस्य नृपोत्तम\thinspace{\devanagarifontsmall ॥} }}

%Verse 7:5

{\devanagarifont तस्मादन्नसमं दानं न भूतं न भविष्यति {॥ ७:\hspace{.11em}५॥} \veg\dontdisplaylinenum }%
     \var{{\devanagarifontvar\numnoemph\vd भूतं\lem \msCc\msNa\msNb\msNc,\hskip.2em plus .9em \lacwithnum{1}  तन् \msCa,\hskip.5em plus .9em भूते \msCb,\hskip.5em plus .9em भूतो \Ed}}% 
    \paral{{\devanagarifontsmall \vcd {\englishfont  = \SDHU\ 7.31cd \similar\ \MBH\ 13.62.6ab: 
                                         }अन्नेन सदृशं दानं न भूतं न भविष्यति }}


\alalalfejezet{वस्त्रदानम्}

{\devanagarifont वस्त्राभावान्मनुष्यस्य श्रियादपि परित्यजेत् \thinspace{\dandab} \dontdisplaylinenum }%
     \var{{\devanagarifontvar\numemph\va ॰भावान्म॰\lem \mssALL,\hskip.2em plus .9em ॰भावात्म॰ \msNa\msNc}}% 
    \var{{\devanagarifontvar\numnoemph\vb श्रियादपि\lem \mssALL,\hskip.2em plus .9em प्रियादपि \msCb,\hskip.5em plus .9em श्रिया वापि \msNc}}% 

%Verse 7:6

{\devanagarifont वस्त्रहीनो न पूज्येत भार्यापुत्रसखादिभिः {॥ ७:\hspace{.11em}६॥} \veg\dontdisplaylinenum }%
 
{\devanagarifont विद्यावान्सुकुलीनो ऽपि ज्ञानवान्गुणवानपि \thinspace{\dandab} \dontdisplaylinenum }%
 
%Verse 7:7

{\devanagarifont वस्त्रहीनः पराधीनः परिभूतः पदे पदे {॥ ७:\hspace{.11em}७॥} \veg\dontdisplaylinenum }%
 
{\devanagarifont अपमानमवज्ञां च वस्त्रहीनो ह्यवाप्नुयात् \thinspace{\dandab} \dontdisplaylinenum }%
     \var{{\devanagarifontvar\numemph\va ॰वज्ञां\lem \mssALL,\hskip.2em plus .9em ॰वज्ञं \Ed}}% 
    \var{{\devanagarifontvar\numnoemph\vb ॰हीनो\lem \mssALL,\hskip.2em plus .9em ॰ही \msCb}}% 

%Verse 7:8

{\devanagarifont जुगुप्सति महात्मापि सभास्त्रीजनसंसदि {॥ ७:\hspace{.11em}८॥} \veg\dontdisplaylinenum }%
 
{\devanagarifont तस्माद्वस्त्रप्रदानानि प्रशंसन्ति मनीषिणः \thinspace{\dandab} \dontdisplaylinenum }%
 
%Verse 7:9

{\devanagarifont न जीर्णं स्फुटितं दद्याद्वस्त्रं कुत्सितमेव वा {॥ ७:\hspace{.11em}९॥} \veg\dontdisplaylinenum }%
     \var{{\devanagarifontvar\numemph\vc जीर्णं स्फुटितं\lem \mssALL,\hskip.2em plus .9em जीर्णस्फटितं \msNb\Ed}}% 
    \var{{\devanagarifontvar\numnoemph\vd कुत्सितमेव वा\lem \mssALL,\hskip.2em plus .9em कुत्सितमेव च \msCc,\hskip.5em plus .9em कुत्सित्मेव वा \msNc}}% 

{\devanagarifont नवं पुराणरहितं मृदु सूक्ष्मं सुशोभनम् \thinspace{\dandab} \dontdisplaylinenum }%
     \var{{\devanagarifontvar\numemph\vb सूक्ष्मं\lem \mssALL,\hskip.2em plus .9em सूक्ष्म \msCc,\hskip.5em plus .9em शुक्लं \Ed}}% 

%Verse 7:10

{\devanagarifont सुसंस्कृत्य प्रदातव्यं श्रद्धाभक्तिसमन्वितम् {॥ ७:\hspace{.11em}१०॥} \veg\dontdisplaylinenum }%
     \var{{\devanagarifontvar\numnoemph\vc ॰दातव्यं\lem \mssALL,\hskip.2em plus .9em ॰दातव्य \msCc}}% 
    \var{{\devanagarifontvar\numnoemph\vd ॰समन्वितम्\lem \mssALL,\hskip.2em plus .9em ॰तं \msNaacorr}}% 

\pend
\endnumbering
\vfill\pagebreak\beginnumbering\pstart
\vers

{\devanagarifont श्रद्धासत्त्वविशेषेण देशकालविधेन च \thinspace{\dandab} \dontdisplaylinenum }%
     \var{{\devanagarifontvar\numemph\va ॰सत्त्व॰\lem \mssALL,\hskip.2em plus .9em ॰स च॰ \Ed}}% 

%Verse 7:11

{\devanagarifont पात्रद्रव्यविशेषेण फलमाहुः पृथक्पृथक् {॥ ७:\hspace{.11em}११॥} \veg\dontdisplaylinenum }%
     \paral{{\devanagarifontsmall \vo {\englishfont \compare\ \MANU\ 7.86--87 (the latter usually labelled as an additional verse):}
                         पात्रस्य हि विशेषेण श्रद्दधानतयाइव च\thinspace{\devanagarifontsmall ।} 
                         अल्पं वा बहु वा प्रेत्य दानस्य फलमश्नुते\thinspace{\devanagarifontsmall ॥}
                         देशकालविधानेन द्रव्यं श्रद्धासमन्वितम्\thinspace{\devanagarifontsmall ।}
                         पात्रे प्रदीयते यत्तु तद्धर्मस्य प्रसाधनम्\thinspace{\devanagarifontsmall ॥} }}

{\devanagarifont यादृशं दीयते वस्त्रं तादृशं प्राप्यते फलम् \thinspace{\dandab} \dontdisplaylinenum }%
 
{\devanagarifont जीर्णवस्त्रप्रदानेन जीर्णवस्त्रमवाप्नुयात्  \danda\dontdisplaylinenum }%
 
%Verse 7:12

{\devanagarifont शोभनं दीयते वस्त्रं शोभनं वस्त्रमाप्नुयात् {॥ ७:\hspace{.11em}१२॥} \veg\dontdisplaylinenum }%
     \var{{\devanagarifontvar\numemph\vef शोभनं दीयते वस्त्रं शोभनं वस्त्रमाप्नुयात्\lem \mssALL,\hskip.2em plus .9em 
\om\ \msNb}}% 

\nemslokalong


\ujvers\nemsloka {
{\devanagarifont दद्याद्वस्त्र सुशोभनं द्विजवरे काले शुभे सादरं }%
  \dontdisplaylinenum}    \var{{\devanagarifontvar\numemph\va द्विजवरे काले शुभे\lem \mssALL,\hskip.2em plus .9em द्विजयिने एकाशुभं \Ed}}% 


\nemslokab

{\devanagarifont सौभाग्यमतुलं लभेत स नरो रूपं तथा शोभनम्  \danda\dontdisplaylinenum }%
     \var{{\devanagarifontvar\numnoemph\vb सौभाग्यम॰\lem \mssALL,\hskip.2em plus .9em सौभाग्यत्वम॰ \msCcacorr\oo 
 नरो\lem \mssALL,\hskip.2em plus .9em दरो \msCb}}% 

\nemslokac

{\devanagarifont तस्मिन्याति सुवस्त्रकोटि शतशः प्राप्नोति निःसंशयं }%
  \dontdisplaylinenum    \var{{\devanagarifontvar\numnoemph\vc तस्मिन्याति\lem \mssALL,\hskip.2em plus .9em त\uncl{स्मा}न्याति \msNa\oo 
 सुवस्त्र॰\lem \mssALL,\hskip.2em plus .9em स वस्त्र॰ \Ed\oo 
 ॰संशयम्\lem \msCa\msCb\msNc,\hskip.2em plus .9em ॰संशयः \msCc\msNa\msNb\Ed}}% 

%Verse 7:13


\nemslokad

{\devanagarifont तस्मात्त्वं कुरु वस्त्रदानमसकृत्पारत्रिकोत्कर्षणम् {॥ ७:\hspace{.11em}१३॥} \veg\dontdisplaylinenum }%
     \var{{\devanagarifontvar\numnoemph\vd दानमसकृत्पा॰\lem \mssALL,\hskip.2em plus .9em दानसत्पा॰ \msNb}}% 

\nemslokanormal



\alalalfejezet{सुवर्णदानम्}

\vers


{\devanagarifont सुवर्णदानं विप्रेन्द्र संक्षिप्य कथयाम्यहम् \thinspace{\dandab} \dontdisplaylinenum }%
     \var{{\devanagarifontvar\numemph\va ॰दानं\lem \mssALL,\hskip.2em plus .9em ॰दान \msNb\Ed}}% 

%Verse 7:14

{\devanagarifont पवित्रं मङ्गलं पुण्यं सर्वपातकनाशनम् {॥ ७:\hspace{.11em}१४॥} \veg\dontdisplaylinenum }%
     \var{{\devanagarifontvar\numnoemph\vd ॰पातक॰\lem  \mssALL,\hskip.2em plus .9em ॰पापक॰ \msCa}}% 

{\devanagarifont धारयेत्सततं विप्र सुवर्णकटकाङ्गुलिम् \thinspace{\dandab} \dontdisplaylinenum }%
     \var{{\devanagarifontvar\numemph\vb ॰कटकाङ्गुलिम्\lem \mssALL,\hskip.2em plus .9em ॰क\lk\lk गुलिम् \msCa,\hskip.5em plus .9em ॰कटकाङ्गुलम् \msNb}}% 

%Verse 7:15

{\devanagarifont मुच्यते सर्वपापेभ्यो राहुणा चन्द्रमा यथा {॥ ७:\hspace{.11em}१५॥} \veg\dontdisplaylinenum }%
     \paral{{\devanagarifontsmall \vcd {\englishfont = 22.38 below = a line inserted after \MBH\ 1.56.18 in some manuscripts as indicated in 
                     the critical edition} }}

\pend
\endnumbering
\vfill\pagebreak\beginnumbering\pstart
\vers

{\devanagarifont दत्त्वा सुवर्णं विप्रेभ्यो देवेभ्यश्च द्विजर्षभ \thinspace{\dandab} \dontdisplaylinenum }%
     \var{{\devanagarifontvar\numemph\va सुवर्णं\lem \mssALL,\hskip.2em plus .9em सुवर्ण \msNb}}% 
    \var{{\devanagarifontvar\numnoemph\vb ॰र्षभ\lem \mssALL,\hskip.2em plus .9em ॰र्षभः \msCc\msNb}}% 

%Verse 7:16

{\devanagarifont तुटिमात्रे ऽपि यो दद्यात्सर्वपापैः प्रमुच्यते {॥ ७:\hspace{.11em}१६॥} \veg\dontdisplaylinenum }%
     \var{{\devanagarifontvar\numnoemph\vc तुटि॰\lem \mssALL,\hskip.2em plus .9em त्रुटि॰ \Ed\oo 
 ॰मात्रे\lem \mssALL,\hskip.2em plus .9em ॰मात्रो \msNa\Ed}}% 
    \var{{\devanagarifontvar\numnoemph\vd सर्वपापैः प्रमुच्यते\lem \mssALL,\hskip.2em plus .9em 
सर्वपापैः स मुच्यते \msCa,\hskip.5em plus .9em सर्वपापै प्रमुच्यते \Ed}}% 

{\devanagarifont रक्तिमाषककर्षं वा पलार्धं पलमेव वा \thinspace{\dandab} \dontdisplaylinenum }%
     \var{{\devanagarifontvar\numemph\va रक्तिमाषक॰\lem \msNcacorr,\hskip.2em plus .9em रन्तिमाषक॰ \msCa,\hskip.5em plus .9em रत्तिमाषक॰ \msCb\msNa\msNcpcorr,\hskip.5em plus .9em 
रन्तिम्मान्सक॰ \msCc,\hskip.5em plus .9em रत्तिमान्सक॰ \msNb,\hskip.5em plus .9em रत्तिमाषक॰ \Ed}}% 
    \var{{\devanagarifontvar\numnoemph\vb ॰र्धं\lem \msCa\msCb\msNc\Ed,\hskip.2em plus .9em ॰द्ध \msCc\msNa\msNb}}% 

%Verse 7:17

{\devanagarifont एवमेव फलंवृद्धिर्ज्ञेया दानविशेषतः {॥ ७:\hspace{.11em}१७॥} \veg\dontdisplaylinenum }%
     \var{{\devanagarifontvar\numnoemph\vcd ॰वृद्धिर्ज्ञेया\lem \msCa\Ed,\hskip.2em plus .9em ॰वृद्धि ज्ञेया \msCb\msCc\msNa\msNb,\hskip.5em plus .9em ॰वृर्द्धि ज्ञेया \msNc}}% 


\alalalfejezet{भूमिदानम्}

{\devanagarifont सर्वाधारं महीदानं प्रशंसन्ति मनीषिणः \thinspace{\dandab} \dontdisplaylinenum }%
     \var{{\devanagarifontvar\numemph\va ॰धारं\lem \msCb,\hskip.2em plus .9em ॰धार॰ \msCa\msCc\msNa\msNb\msNc\Ed}}% 
    \var{{\devanagarifontvar\numnoemph\vab ॰दानं प्रशंसन्ति\lem \mssALL,\hskip.2em plus .9em 
दा\lk \uncl{नम्प्र}\lacwithnum{1}  सन्ति \msCa}}% 

%Verse 7:18

{\devanagarifont अन्नवस्त्रहिरण्यादि सर्वं वै भूमिसम्भवम् {॥ ७:\hspace{.11em}१८॥} \veg\dontdisplaylinenum }%
     \var{{\devanagarifontvar\numnoemph\vd सर्वं वै\lem \mssALL,\hskip.2em plus .9em सर्वं \uncl{वे} \msCa\ \toplost}}% 

{\devanagarifont भूमिदानेन विप्रेन्द्र सर्वदानफलं लभेत् \thinspace{\dandab} \dontdisplaylinenum }%
     \var{{\devanagarifontvar\numemph\vb ॰फलं लभेत्\lem \mssALL,\hskip.2em plus .9em 
॰ललं भवेत् \msNbacorr,\hskip.5em plus .9em ॰लं भवेत् \msNc}}% 

%Verse 7:19

{\devanagarifont भूमिदानसमं विप्र यद्यस्ति वद तत्त्वतः {॥ ७:\hspace{.11em}१९॥} \veg\dontdisplaylinenum }%
 
{\devanagarifont मातृकुक्षिविमुक्तस्तु धरणीशरणो भवेत् \thinspace{\dandab} \dontdisplaylinenum }%
     \var{{\devanagarifontvar\numemph\va ॰मुक्तस्तु\lem \mssALL,\hskip.2em plus .9em ॰मुक्तिस्तु \Ed}}% 
    \var{{\devanagarifontvar\numnoemph\vb ॰शरणो\lem \mssALL,\hskip.2em plus .9em ॰शरण \msNc,\hskip.5em plus .9em ॰शरणां \Ed}}% 

%Verse 7:20

{\devanagarifont चराचराणां सर्वेषां भूमिः साधारणा स्मृता {॥ ७:\hspace{.11em}२०॥} \veg\dontdisplaylinenum }%
 
{\devanagarifont एकहस्तं द्विहस्तं वा पञ्चाशच्छतमेव वा \thinspace{\dandab} \dontdisplaylinenum }%
     \var{{\devanagarifontvar\numemph\va एकहस्तं\lem \msCb\msNa\msNb\msNc,\hskip.2em plus .9em एकहस्त॰ \msCa\msCc\Ed}}% 

%Verse 7:21

{\devanagarifont सहस्रायुतलक्षं वा भूमिदानं प्रशस्यते {॥ ७:\hspace{.11em}२१॥} \veg\dontdisplaylinenum }%
     \var{{\devanagarifontvar\numnoemph\vd भूमिदानं प्रशस्यते\lem \mssALL,\hskip.2em plus .9em भूमिदान प्रशस्यते \msCb,\hskip.5em plus .9em 
पञ्चाशच्छतमेव वा\thinspace{\devanagarifont ।} सहायुतलक्षम्वा भूमिदं प्रशस्यते \msNb\ {\englishfont (eyeskip)}}}% 

{\devanagarifont एकहस्तां च यो भूमिं दद्याद्द्विजवराय तु \thinspace{\dandab} \dontdisplaylinenum }%
     \var{{\devanagarifontvar\numemph\va ॰हस्तां च\lem \mssALL,\hskip.2em plus .9em ॰हस्तञ्च \msCb\msNb}}% 
    \var{{\devanagarifontvar\numnoemph\vb दद्याद्द्वि॰\lem \mssALL,\hskip.2em plus .9em दद्या द्वि॰ \Ed}}% 

%Verse 7:22

{\devanagarifont वर्षकोटिशतं दिव्यं स्वर्गलोके महीयते {॥ ७:\hspace{.11em}२२॥} \veg\dontdisplaylinenum }%
 
{\devanagarifont एवं बहुषु हस्तेषु गुणागुणि फलं स्मृतम् \thinspace{\dandab} \dontdisplaylinenum }%
     \var{{\devanagarifontvar\numemph\vb गुणागुणि॰\lem \mssALL,\hskip.2em plus .9em गुणागणि॰ \Ed}}% 

%Verse 7:23

{\devanagarifont श्रद्धाधिकं फलं दानं कथितं ते द्विजोत्तम {॥ ७:\hspace{.11em}२३॥} \veg\dontdisplaylinenum }%
     \var{{\devanagarifontvar\numnoemph\vc ॰धिकं\lem \msCb\msCc\msNa\msNb,\hskip.2em plus .9em ॰धिक॰ \msCa\msNc\Ed}}% 
    \var{{\devanagarifontvar\numnoemph\vd ॰त्तम\lem \mssALL,\hskip.2em plus .9em ॰त्तमः \msNc}}% 

{\devanagarifont जामदग्न्येन रामेण भूमिं दत्त्वा द्विजाय वै \thinspace{\dandab} \dontdisplaylinenum }%
     \var{{\devanagarifontvar\numemph\va जामदग्न्येन\lem \msCb\msNa\msNc,\hskip.2em plus .9em जामदग्न्ये\lk\ \msCa,\hskip.5em plus .9em जामदग्नेन \msCc\msNb\Ed\oo 
 रामेण\lem \msCb\msNc\Ed,\hskip.2em plus .9em \lk\lk ण \msCa,\hskip.5em plus .9em रामेन \msCc\msNa\msNb}}% 
    \var{{\devanagarifontvar\numnoemph\vb दत्त्वा द्वि॰\lem \mssALL,\hskip.2em plus .9em दद्याद्द्वि॰ \msCb}}% 

%Verse 7:24

{\devanagarifont आयुरक्षयमाप्तं तु इहैव च द्विजोत्तम {॥ ७:\hspace{.11em}२४॥} \veg\dontdisplaylinenum }%
     \var{{\devanagarifontvar\numnoemph\vd च\lem \mssALL,\hskip.2em plus .9em हि \Ed}}% 


\alalalfejezet{गोदानम्}

{\devanagarifont हेमशृङ्गां रौप्यक्षुरां चैलघण्टां द्विजोत्तम \thinspace{\dandab} \dontdisplaylinenum }%
     \var{{\devanagarifontvar\numemph\va ॰शृङ्गां\lem \mssALL,\hskip.2em plus .9em ॰शृङ्गं \msNa,\hskip.5em plus .9em \om\ \msNb\oo 
 रौप्य॰\lem \mssALL,\hskip.2em plus .9em रोप्यं \msNc\oo 
 ॰क्षुरां\lem \mssALL,\hskip.2em plus .9em ॰खुरां \msCc\Ed}}% 
    \lacuna{\devanagarifontsmall \vab {\englishfont Omitted in \msNb} }%
      \paral{{\devanagarifontsmall \vab {\englishfont \similar\ \VAGMATI\ 17.33ab:}
                         हेमशृङ्गां रौप्यखुरां चैलघण्टावलम्बिनीम्\thinspace{\devanagarifontsmall ।} }}

%Verse 7:25

{\devanagarifont विप्राय वेदविदुषे दत्त्वानन्तफलं स्मृतम् {॥ ७:\hspace{.11em}२५॥} \veg\dontdisplaylinenum }%
     \var{{\devanagarifontvar\numnoemph\vd दत्त्वानन्त॰\lem \mssALL,\hskip.2em plus .9em दत्त्वान्त॰ \Ed}}% 
    \paral{{\devanagarifontsmall \vo {\englishfont \compare, e.g., \MBH\ 7.58.18:}
                 तथा गाः कपिला दोग्ध्रीः सर्षभाः पाण्डुनन्दनः\thinspace{\devanagarifontsmall ।}
                 हेमशृङ्गी रूप्यखुरा दत्त्वा चक्रे प्रदक्षिणम्\thinspace{\devanagarifontsmall ॥}
                       {\englishfont and \BHAVP\ Uttara 12.25:}
                 हेमशृंगीं रौप्यखुरां सघंटां कांस्यदोहनाम्\thinspace{\devanagarifontsmall ।} 
                 महादेवाय गां दद्याद्दीक्षिताय द्विजाय वै\thinspace{\devanagarifontsmall ॥} }}


\alalalfejezet{दानप्रशंसा}

\nemslokalong


\ujvers\nemsloka {
{\devanagarifont दानाभ्यासरतः प्रवर्तनभवां शक्यानुरूपं सदा }%
  \dontdisplaylinenum}    \var{{\devanagarifontvar\numemph\va ॰रूपं\lem \mssALL,\hskip.2em plus .9em ॰रूप \msNb}}% 


\nemslokab

{\devanagarifont अन्नं वस्त्रहिरण्यरौप्यमुदकं गावस्तिलान्मेदिनीम्  \danda\dontdisplaylinenum }%
     \var{{\devanagarifontvar\numnoemph\vb ॰रौप्य॰\lem \mssALL,\hskip.2em plus .9em ॰रोप्य॰ \msCb,\hskip.5em plus .9em ॰\uncl{रौप्य}॰ \msNc\oo 
 गावस्तिलान्मे॰\lem \eme,\hskip.2em plus .9em गावस्तिलाम्मे॰ \msCa\msCc\msNc,\hskip.5em plus .9em गावस्तिला मे॰ \msCb\msNa,\hskip.5em plus .9em 
गावन्तिला मे॰ \msNb,\hskip.5em plus .9em गावस्तिलं मे॰ \Ed}}% 

\nemslokac

{\devanagarifont दद्यात्पादुकछत्त्रपीठकलशं पात्राद्यमन्यच्च वा }%
  \dontdisplaylinenum    \var{{\devanagarifontvar\numnoemph\vc दद्यात्पा॰\lem \mssALL,\hskip.2em plus .9em दद्या पा॰ \msNb\oo 
 पात्राद्यमन्यच्च वा\lem \mssALL,\hskip.2em plus .9em 
पत्राद्यमन्यच्च वा \msCb,\hskip.5em plus .9em पात्रेषु लब्धेषु वै \Ed}}% 

%Verse 7:26


\nemslokad

{\devanagarifont श्रद्धादानमभिन्नरागवदनं कृत्वा मनो निर्मलम् {॥ ७:\hspace{.11em}२६॥} \veg\dontdisplaylinenum }%
     \var{{\devanagarifontvar\numnoemph\vd श्रद्धादान॰\lem \mssALL,\hskip.2em plus .9em दत्त्वादान॰ \Ed}}% 

\pend
\endnumbering
\vfill\pagebreak\beginnumbering\pstart
\vers

\nemslokalong


\ujvers\nemsloka {
{\devanagarifont दानादेव यशः श्रियः सुखकराः ख्यातिमतुल्यां लभेद् }%
  \dontdisplaylinenum}    \var{{\devanagarifontvar\numemph\va यशः\lem \msCb\msNc\Ed,\hskip.2em plus .9em यश \msCa\msCc\msNa\msNb\oo 
 सुखकराः\lem \mssALL,\hskip.2em plus .9em सुखकर \msNcpcorr\oo 
 ख्यातिमतुल्यां\lem \eme,\hskip.2em plus .9em ख्यातिश्च तुल्यं \mssCaCbCc\msNa\msNb\msNc\Ed\oo 
 लभेद्\lem \mssALL,\hskip.2em plus .9em भवेत् \msNc\Ed}}% 


\nemslokab

{\devanagarifont दानादेव निगर्हणं रिपुगणे आनन्ददं सौख्यदम्  \danda\dontdisplaylinenum }%
     \var{{\devanagarifontvar\numnoemph\vb निगर्हणं\lem \msCapcorr\msCc\msNa\Ed,\hskip.2em plus .9em निर्हणं \msCaacorr,\hskip.5em plus .9em निवर्हणं \msCb\msNc,\hskip.5em plus .9em 
निगर्हन \msNb\oo 
 ॰गणे आनन्ददं सौख्यदम्\lem \mssALL,\hskip.2em plus .9em 
॰गणै आनन्ददं सौख्यदम् \msCc,\hskip.5em plus .9em 
॰गणैश्चानन्दसौख्यप्रदम्  \Ed}}% 

\nemslokac

{\devanagarifont दानादूर्जयता प्रसादमतुलं सौभाग्य दानाल्लभेद् }%
  \dontdisplaylinenum    \var{{\devanagarifontvar\numnoemph\vc दानादूर्जयता\lem \mssALL,\hskip.2em plus .9em दानादूर्जयतां \msNa,\hskip.5em plus .9em दानाद्दु॰ \Ed\oo 
 प्रसाद॰\lem \mssALL,\hskip.2em plus .9em प्रासाद॰ \msNa\oo 
 सौभाग्य\lem \mssALL,\hskip.2em plus .9em सौगाग्य \msCb,\hskip.5em plus .9em सौभाग्यं \Ed\ \unmetr\oo 
 दानाल्लभेद्\lem \msCb\Ed,\hskip.2em plus .9em दानं लभेत् \msCa\msCc\msNa\msNb\msNc}}% 

%Verse 7:27


\nemslokad

{\devanagarifont दानादेव अनन्तभोग नियतं स्वर्गं च तस्माद्भवेत् {॥ ७:\hspace{.11em}२७॥} \veg\dontdisplaylinenum }%
     \var{{\devanagarifontvar\numnoemph\vd दानादेव\lem \mssALL,\hskip.2em plus .9em दानादोव \msCc\oo 
 ॰नियतं\lem \mssALL,\hskip.2em plus .9em ॰नियत \msCc}}% 

\ujvers\nemsloka {
{\devanagarifont दानादेव च शक्रलोकसकलं दानाज्जनानन्दनं }%
  \dontdisplaylinenum}    \var{{\devanagarifontvar\numemph\va शक्रलोकसकलं\lem \mssALL,\hskip.2em plus .9em शत्रुलोकसकलं \msNa,\hskip.5em plus .9em शक्रलोकमतुलं \Ed\oo 
 दानाज्ज॰\lem \mssALL,\hskip.2em plus .9em दाना ज॰ \msCa,\hskip.5em plus .9em दानार्ज॰ \msCb}}% 


\nemslokab

{\devanagarifont दानादेव महीं समस्त बुभुजे सम्राड्महीमण्डले  \danda\dontdisplaylinenum }%
     \var{{\devanagarifontvar\numnoemph\vb दानादेव\lem \mssALL,\hskip.2em plus .9em दानेदेव \msCb\oo 
 महीं समस्त\lem \conj,\hskip.2em plus .9em महीसमासु \msCb\msCc,\hskip.5em plus .9em महीं समांसु \msCa\msNa\msNc,\hskip.5em plus .9em 
मही समस्त \msNb,\hskip.5em plus .9em महीयसां स \Ed\oo 
 सम्राड्म॰\lem \mssALL,\hskip.2em plus .9em संम्राड्म॰ \msCb}}% 

\nemslokac

{\devanagarifont दानादेव सुरूपयोनिसुभगश्चन्द्राननो वीक्ष्यते }%
  \dontdisplaylinenum    \var{{\devanagarifontvar\numnoemph\vc सुरूप॰\lem \mssALL,\hskip.2em plus .9em स्वरूप॰ \msNb\oo 
 ॰योनिसु॰\lem \msNb\Ed,\hskip.2em plus .9em ॰योनिस्सु॰ \msCa ॰योनिः सु॰ \msCb\msCc\msNa\msNc\oo 
 ॰भगश्च॰\lem \msCa\msCc\msNb\msNc,\hskip.2em plus .9em ॰भग च॰ \msCb\msNa\Ed\oo 
 ॰न्द्राननो\lem \msCa\msCb\msNa\Ed,\hskip.2em plus .9em ॰न्द्रानने \msCc\msNb,\hskip.5em plus .9em ॰न्द्राननौ \msNc\oo 
 वीक्ष्यते\lem \msCb\msCc,\hskip.2em plus .9em वीक्षते \msCa\msNa\msNb\msNc,\hskip.5em plus .9em विक्षते \Ed}}% 

%Verse 7:28


\nemslokad

{\devanagarifont दानादेव अनेकसम्भवसुखं प्राप्नोति निःसंशयम् {॥ ७:\hspace{.11em}२८॥} \veg\dontdisplaylinenum }%
     \var{{\devanagarifontvar\numnoemph\vd निःसंशयम्\lem \msCa\msCb\msNc,\hskip.2em plus .9em निसंशयः \msCc,\hskip.5em plus .9em निःसंशयः \msNa\Ed,\hskip.5em plus .9em निस्सयः \msNb}}% 

\vers


{\devanagarifont 
\jump
\begin{center}
\ketdanda~इति वृषसारसंग्रहे दानप्रशंसाध्यायः सप्तमः~\ketdanda
\end{center}
\dontdisplaylinenum\vers  }%
     \var{{\devanagarifontvar\numnoemph{\englishfont \Colo:} ॰प्रशंसाध्यायः सप्तमः\lem \mssALL,\hskip.2em plus .9em 
॰प्रशंसाध्यायः समाप्तः \msCb,\hskip.5em plus .9em 
॰प्रशंसा सप्तमो ऽध्यायः \Ed}}% 

\nemslokanormal

\bekveg\szamveg
\vfill
\phpspagebreak

\versno=0\fejno=8
\thispagestyle{empty}

\centerline{\Large\devanagarifontbold [   अष्टमो ऽध्यायः  ]}{\vrule depth10pt width0pt} \fancyhead[CE]{{\footnotesize\devanagarifont वृषसारसंग्रहे  }}
\fancyhead[CO]{{\footnotesize\devanagarifont अष्टमो ऽध्यायः  }}
\fancyhead[LE]{}
\fancyhead[RE]{}
\fancyhead[LO]{}
\fancyhead[RO]{}
\szam\bek



\alalfejezet{नियमेषु स्वाध्यायः (५)}
\vers


{\devanagarifont पञ्चस्वाध्यायनं कार्यमिहामुत्र सुखार्थिना \thinspace{\dandab} \dontdisplaylinenum }%
     \var{{\devanagarifontvar\numemph\va ॰स्वाध्यायनं\lem \mssALL,\hskip.2em plus .9em 
॰स्वाध्ययनं \msNc}}% 
    \var{{\devanagarifontvar\numnoemph\vb कार्यमिहामुत्र\lem \mssCaCbCc\msNa\msNb\msNc\msParis\msKOb,\hskip.2em plus .9em 
इहामुत्र \msKOa,\hskip.5em plus .9em कार्यमिहामूत्र \msPaperA\Ed\oo 
 ॰र्थिना\lem \mssALL,\hskip.2em plus .9em ॰र्थिनां \msNb}}% 
    \lacuna{\devanagarifontsmall {\englishfont Witnesses used for this chapter: \msCa\ ff.\thinspace 204r--205v, 
                                                  \msCb\ ff.\thinspace 210v--211v, 
                                                  \msCc\ ff.\allowbreak\thinspace 280v--282r,
                                                  \msNa\ ff.\thinspace 11v--13r, 
                                                  \msNb\ exp.\thinspace 53 (lower) -- 54 (lower),
                                                  \msNc\ ff.\thinspace 219v--221r,
                                                  \msParis\ exp.\thinspace 426--428,
                                                  \msKOb\ ff.\thinspace 219v--221a,
                                                  \msPaperA\ ff.\thinspace 213r--214v,
                                                  \Ed\ pp.\thinspace 603--606; 
                                                  \mssCaCbCc\ = \msCa + \msCb + \msCc} }%
  
%Verse 8:1

{\devanagarifont शैवं सांख्यं पुराणं च स्मार्तं भारतसंहिताम् {॥ ८:\hspace{.11em}१॥} \veg\dontdisplaylinenum }%
     \var{{\devanagarifontvar\numnoemph\vc शैवं\lem \mssALL,\hskip.2em plus .9em 
\uncl{शै}लं \msCc,\hskip.5em plus .9em सैव॰ \msKOa\oo 
 सांख्यं\lem \msCa\msCb\msNc\msParis\msKOb\msPaperA\Ed,\hskip.2em plus .9em 
शांख्य \msCc,\hskip.5em plus .9em साख्यं \msNa\msNb,\hskip.5em plus .9em सङ्ख्या \msKOa}}% 
    \var{{\devanagarifontvar\numnoemph\vd स्मार्तं\lem \msCa\msCb\msNa\msNc\msParis\msKOb\msPaperA\Ed,\hskip.2em plus .9em स्मार्त॰ \msCc\msNb\msKOa\oo 
 भारतसंहिताम्\lem \mssCaCbCc\msNb\msParis\msPaperA\Ed,\hskip.2em plus .9em 
भारतसंहिताः \msNa,\hskip.5em plus .9em भारत्तसंहितां \msNc,\hskip.5em plus .9em 
भारतसंहिता \msKOa\msKOb}}% 

{\devanagarifont शैवे तत्त्वं विचिन्तेत शैवपाशुपतद्वये \thinspace{\dandab} \dontdisplaylinenum }%
     \var{{\devanagarifontvar\numemph\va शैवे \lem \msCa\msCc\msNa\msNb\msNc\msKOb,\hskip.2em plus .9em शैवै \msCb\msParis,\hskip.5em plus .9em सैव॰ \msKOa,\hskip.5em plus .9em शैवं \msPaperA\Ed\oo 
 तत्त्वं\lem \mssALL,\hskip.2em plus .9em 
॰तत्त्व \msParis\msKOb\oo 
 विचिन्तेत\lem \mssALL,\hskip.2em plus .9em विचिन्तत \msKOa}}% 
    \var{{\devanagarifontvar\numnoemph\vb शैव॰\lem \msParis\msKOa\msKObacorr,\hskip.2em plus .9em शैवः \msCa\msCb\msNb\msNc\msKObpcorr,\hskip.5em plus .9em शैवाः \msCc\msPaperA\Ed,\hskip.5em plus .9em शैवा \msNa\oo 
 ॰द्वये\lem \mssALL,\hskip.2em plus .9em ॰ये \msCb}}% 

%Verse 8:2

{\devanagarifont अत्र विस्तरतः प्रोक्तं तत्त्वसारसमुच्चयम् {॥ ८:\hspace{.11em}२॥} \veg\dontdisplaylinenum }%
     \var{{\devanagarifontvar\numnoemph\vc विस्तरतः प्रोक्तं\lem \mssALL,\hskip.2em plus .9em 
विस्तरत प्रोक्त \msKOa}}% 
    \var{{\devanagarifontvar\numnoemph\vd ॰सारसमुच्चयम्\lem \mssCaCbCc\msNc\msParis\msKOb\msPaperA\Ed,\hskip.2em plus .9em 
॰सारं समुच्चयम् \msNa,\hskip.5em plus .9em ॰सारं समुद्ययं \msNb,\hskip.5em plus .9em ॰सारसमुच्चये \msKOa}}% 

{\devanagarifont संख्यातत्त्वं तु सांख्येषु बोद्धव्यं तत्त्वचिन्तकैः \thinspace{\dandab} \dontdisplaylinenum }%
     \var{{\devanagarifontvar\numemph\va संख्यातत्त्वं तु\lem \msNa\msNc\msParis\msKOa\msKOb\msPaperA,\hskip.2em plus .9em 
सं\uncl{ख्या}\lk\lk\lk\ \msCa,\hskip.5em plus .9em संख्यातत्त्वं \msCb,\hskip.5em plus .9em 
शाङ्ख्यातत्वं तु \msCc,\hskip.5em plus .9em सख्यतत्वन्तु \msNb,\hskip.5em plus .9em संख्यातत्त्व तु \Ed\oo 
 सांख्येषु\lem \mssCaCbCc\msNa\msNc\msParis\msPaperA\Ed,\hskip.2em plus .9em 
सख्येषु \msNb\msKOa\msKOb}}% 
    \var{{\devanagarifontvar\numnoemph\vb बोद्धव्यं\lem \mssALL,\hskip.2em plus .9em बोधव्य \msKOa}}% 

%Verse 8:3

{\devanagarifont पञ्चतत्त्वविभागेन कीर्तितानि महर्षिभिः {॥ ८:\hspace{.11em}३॥} \veg\dontdisplaylinenum }%
     \var{{\devanagarifontvar\numnoemph\vc ॰तत्त्व॰\lem \mssALL,\hskip.2em plus .9em 
॰तत्वा॰ \msCb,\hskip.5em plus .9em \om\ \msNb}}% 
    \var{{\devanagarifontvar\numnoemph\vd महर्षिभिः\lem \mssALL,\hskip.2em plus .9em 
मर्हर्षिभिः \msKOb}}% 

\pend
\endnumbering
\vfill\pagebreak\beginnumbering\pstart
\vers

{\devanagarifont पुराणेषु महीकोषो विस्तरेण प्रकीर्तितः \thinspace{\dandab} \dontdisplaylinenum }%
     \var{{\devanagarifontvar\numemph\va ॰कोषो\lem \mssALL,\hskip.2em plus .9em ॰कोष \msKOa}}% 
    \var{{\devanagarifontvar\numnoemph\vb ॰कीर्तितः\lem \mssALL,\hskip.2em plus .9em ॰कीर्तित \msKOa}}% 

%Verse 8:4

{\devanagarifont अधोर्ध्वमध्यतिर्यं च यत्नतः सम्प्रवेशयेत् {॥ ८:\hspace{.11em}४॥} \veg\dontdisplaylinenum }%
     \var{{\devanagarifontvar\numnoemph\vc अधोर्ध्व॰\lem \mssALL,\hskip.2em plus .9em अधोर्ध्वं \msNb,\hskip.5em plus .9em आयोयश्च \msKOa\oo 
 ॰मध्य॰\lem \mssALL,\hskip.2em plus .9em ॰मध॰ \msCc,\hskip.5em plus .9em \om\ \msKOa}}% 
    \var{{\devanagarifontvar\numnoemph\vd यत्नतः\lem \mssALL,\hskip.2em plus .9em यत्नत \msNb\oo 
 सम्प्रवेशयेत्\lem \mssALL,\hskip.2em plus .9em 
समवेशयेत् \msKOa,\hskip.5em plus .9em सम्प्रबोधयेत् \Ed}}% 

{\devanagarifont स्मार्तं वर्णाश्रमाचारं धर्मन्यायप्रवर्तनम् \thinspace{\dandab} \dontdisplaylinenum }%
     \var{{\devanagarifontvar\numemph\va स्मार्तं वर्णाश्रमा॰\lem \msCa,\hskip.2em plus .9em तस्मार्त्तम्वर्ण्णाश्रमा॰ \msCb,\hskip.5em plus .9em 
स्मार्तवर्णाश्रमा॰ \msCc\msNa\msNb\msNc\msPaperA\Ed,\hskip.5em plus .9em 
स्मार्त्तं वर्ण्णश्रमा॰ \msParis\msKOb,\hskip.5em plus .9em 
स्मार्त वर्ण्णसमा॰ \msKOa}}% 
    \var{{\devanagarifontvar\numnoemph\vb धर्मन्याय॰\lem \mssALL,\hskip.2em plus .9em धर्मं न्याय॰ \msCc,\hskip.5em plus .9em 
धर्माण्याय॰ \msKOa\oo 
 ॰प्रवर्तनम्\lem \mssCaCbCc\msNa\msNb\msNc\msKOb\msPaperA,\hskip.2em plus .9em 
॰प्रव\lk नं \msParis,\hskip.5em plus .9em ॰पवर्तकं \msKOa,\hskip.5em plus .9em ॰प्रवर्तन \Ed}}% 

%Verse 8:5

{\devanagarifont शिष्टाचारो ऽविकल्पेन ग्राह्यस्तत्र अशङ्कितः {॥ ८:\hspace{.11em}५॥} \veg\dontdisplaylinenum }%
     \var{{\devanagarifontvar\numnoemph\vc शिष्टा॰\lem \mssALL,\hskip.2em plus .9em शिष्ट॰ \msPaperA\oo 
 ॰चारो\lem \msCa\msCb\msNb\msNc\msKOa\msPaperA,\hskip.2em plus .9em ॰चार॰ \msCc\Ed,\hskip.5em plus .9em 
॰चारा \msNa\msKOb,\hskip.5em plus .9em 
॰चा\uncl{रो}॰ \msParis}}% 
    \var{{\devanagarifontvar\numnoemph\vd ग्राह्यस्तत्र अशङ्कितः\lem \mssALL,\hskip.2em plus .9em 
ग्राह्यस्त\lk\lk\lk ङ्कितः \msCa,\hskip.5em plus .9em 
ग्राह्य तत्व असहितः \msKOa}}% 

{\devanagarifont इतिहासमधीयानः सर्वज्ञः स नरो भवेत् \thinspace{\dandab} \dontdisplaylinenum }%
     \var{{\devanagarifontvar\numemph\vb ॰ज्ञः\lem \mssALL,\hskip.2em plus .9em ॰ज्ञ \msCc}}% 

%Verse 8:6

{\devanagarifont धर्मार्थकाममोक्षेषु संशयस्तेन छिद्यते {॥ ८:\hspace{.11em}६॥} \veg\dontdisplaylinenum }%
 

\alalfejezet{नियमेष्वुपस्थनिग्रहः (६)}
{\devanagarifont शृणुष्वावहितो विप्र पञ्चोपस्थविनिग्रहम् \thinspace{\dandab} \dontdisplaylinenum }%
     \var{{\devanagarifontvar\numemph\vab शृणुष्वावहितो विप्र पञ्चोपस्थविनिग्रहम्\lem \mssALL,\hskip.2em plus .9em 
शृणुष्वावहितो विप्र पञ्चोपस्थविनिग्र\uncl{हः} \msNa,\hskip.5em plus .9em 
पंचोप्रस्थविनिग्रह सृणुयावंहितो द्विज \msKOa}}% 

{\devanagarifont स्त्रियो वा गर्हितोत्सर्गः स्वयंमुक्तिश्च कीर्त्यते  \danda\dontdisplaylinenum }%
     \var{{\devanagarifontvar\numnoemph\vc गर्हितोत्सर्गः\lem \msCa\msCb\msNb\msNc\msParis\msKOb,\hskip.2em plus .9em 
गर्हितस्सर्ग्गः \msCc,\hskip.5em plus .9em गर्हितो विप्र \msNa,\hskip.5em plus .9em 
गर्हितः स्वर्गः \msKOa,\hskip.5em plus .9em 
गर्हितो स्वर्गः \msPaperA\Ed}}% 
    \var{{\devanagarifontvar\numnoemph\vd स्वयं॰\lem \mssALL,\hskip.2em plus .9em स्वय॰ \msCb\oo 
 कीर्त्यते\lem \mssALL,\hskip.2em plus .9em 
की\uncl{र्त्स्य}ते \msCc}}% 

%Verse 8:7

{\devanagarifont स्वप्नोपघातं विप्रेन्द्र दिवास्वप्नं च पञ्चमः {॥ ८:\hspace{.11em}७॥} \veg\dontdisplaylinenum }%
     \var{{\devanagarifontvar\numnoemph\ve ॰घातं\lem \mssALL,\hskip.2em plus .9em 
॰घात \msCc\Ed}}% 

\pend
\endnumbering
\vfill\pagebreak\beginnumbering\pstart
\vers


\alalalfejezet{स्त्रियः}

{\devanagarifont अगम्या स्त्री दिवा पर्वे धर्मपत्न्यपि वा भवेत् \thinspace{\dandab} \dontdisplaylinenum }%
     \var{{\devanagarifontvar\numemph\va अगम्या स्त्री दिवा पर्वे\lem \msCb\msCc\msNa\msNb\msNc\msKOb\msPaperA,\hskip.2em plus .9em 
अगम्या \lk\ दिवा पर्व्वे \msCa,\hskip.5em plus .9em अगम्या \lk\lk\lk पर्वे \msParis,\hskip.5em plus .9em 
अगम्य स्त्री दिवार्स्यसे \msKOa,\hskip.5em plus .9em 
अगम्या स्त्री दिवापूर्वे \Ed}}% 
    \var{{\devanagarifontvar\numnoemph\vb ॰पत्न्यपि\lem \mssALL,\hskip.2em plus .9em 
॰पत्नी पि \msCc,\hskip.5em plus .9em धर्मपत्नी च \msKOa}}% 

%Verse 8:8

{\devanagarifont विरुद्धस्त्रीं न सेवेत वर्णभ्रष्टाधिकासु च {॥ ८:\hspace{.11em}८॥} \veg\dontdisplaylinenum }%
     \var{{\devanagarifontvar\numnoemph\vc विरुद्धस्त्रीं न सेवेत\lem \msPaperA,\hskip.2em plus .9em 
विरुद्धस्त्री न सेवेत \mssCaCbCc\msNb\msNc,\hskip.5em plus .9em 
विरुद्धस्त्री निसेवेत \msNa\msParis\msKOb,\hskip.5em plus .9em 
विरुद्धस्त्री न भवेत \msKOa,\hskip.5em plus .9em द्विरुद्धास्त्रीन्न सेवेत\Ed}}% 
    \var{{\devanagarifontvar\numnoemph\vd वर्णभ्रष्टाधिकासु च\lem \msCa\msCb\msNa\msParis\msKOb\msPaperA,\hskip.2em plus .9em 
वर्णभ्रष्टाधिकासु त \msCc,\hskip.5em plus .9em वर्णभ्रष्टादिकाषु च \msNb,\hskip.5em plus .9em 
वर्णभ्रष्टाविकाषु च \msNc,\hskip.5em plus .9em वर्ण्णवर्ण्णभ्रष्टाधिकाम च \msKOa,\hskip.5em plus .9em 
वर्णभ्रष्टापिकासु च \Ed}}% 


\alalalfejezet{गर्हितोत्सर्गः}

{\devanagarifont अजमेषगवादीनां वडवामहिषीषु च \thinspace{\dandab} \dontdisplaylinenum }%
     \var{{\devanagarifontvar\numemph\va ॰मेष॰\lem \mssALL,\hskip.2em plus .9em ॰मेय॰ \msCb}}% 

%Verse 8:9

{\devanagarifont गर्हितोत्सर्गमित्येतद्यत्नेन परिवर्जयेत् {॥ ८:\hspace{.11em}९॥} \veg\dontdisplaylinenum }%
 

\alalalfejezet{स्वयंमुक्तिः}

{\devanagarifont अयोनिकषणा वापि अपानकषणापि वा \thinspace{\dandab} \dontdisplaylinenum }%
     \var{{\devanagarifontvar\numemph\va अयोनि॰\lem \conj,\hskip.2em plus .9em अन्योन्य॰ \mssCaCbCc\msNa\msNb\msNc\msParis\msKOb\msPaperA\Ed\oo 
 ॰कषणा\lem \msCa\msNa,\hskip.2em plus .9em ॰कर्षणा \msCb\msCc\msNb\msNc\msParis\msKOb\msPaperA\Ed}}% 
    \var{{\devanagarifontvar\numnoemph\vb ॰कषणापि\lem \mssCaCbCc\msNa,\hskip.2em plus .9em ॰कर्षणापि \msNb\msNc\msParis\msKOb\msPaperA\Ed}}% 

%Verse 8:10

{\devanagarifont स्वयंमुक्तिरियं ज्ञेया तस्मात्तां परिवर्जयेत् {॥ ८:\hspace{.11em}१०॥} \veg\dontdisplaylinenum }%
     \var{{\devanagarifontvar\numnoemph\vc स्वयंमुक्ति॰\lem \mssALL,\hskip.2em plus .9em 
स्वयमुक्ति॰ \msCb\oo 
 ज्ञेया\lem \mssALL,\hskip.2em plus .9em 
ज्ञेयां \msNb}}% 
    \var{{\devanagarifontvar\numnoemph\vd तस्मात्तां\lem \msCa\msCb\msNa\msNc\msParis\msKOb\msPaperA,\hskip.2em plus .9em 
तस्मात्तं \msCc,\hskip.5em plus .9em तस्मार्त्ता \msNb,\hskip.5em plus .9em तस्मात्स्त्री \Ed}}% 


\alalalfejezet{स्वप्नघातम्}

{\devanagarifont स्वप्नघातं द्विजश्रेष्ठ अनिष्टं पण्डितैः सदा \thinspace{\dandab} \dontdisplaylinenum }%
     \var{{\devanagarifontvar\numemph\va स्वप्नघा॰\lem \mssALL,\hskip.2em plus .9em 
स्वप्नजा॰ \msParisacorr}}% 
    \var{{\devanagarifontvar\numnoemph\vb पण्डितैः\lem \mssALL,\hskip.2em plus .9em 
पण्डितै \msCc,\hskip.5em plus .9em पण्डितेः \msNc}}% 

%Verse 8:11

{\devanagarifont स्वप्ने स्त्रीषु रमन्ते च रेतः प्रक्षरते ततः {॥ ८:\hspace{.11em}११॥} \veg\dontdisplaylinenum }%
     \var{{\devanagarifontvar\numnoemph\vc रमन्ते\lem \mssALL,\hskip.2em plus .9em रमक्षन्ते \msPaperA}}% 
    \var{{\devanagarifontvar\numnoemph\vd प्रक्षरते\lem \mssALL,\hskip.2em plus .9em प्रस्खलतस् \Ed\oo 
 ततः\lem \mssALL,\hskip.2em plus .9em तत \msCc}}% 

\pend
\endnumbering
\vfill\pagebreak\beginnumbering\pstart
\vers


\alalalfejezet{दिवास्वप्नम्}

{\devanagarifont दिवाशयं न कर्तव्यं नित्यं धर्मपरेण तु \thinspace{\dandab} \dontdisplaylinenum  }%
     \var{{\devanagarifontvar\numemph\va दिवाशयं न\lem \mssCaCbCc\msParis\msPaperA\Ed,\hskip.2em plus .9em 
दिवाशयेन्न \msNa,\hskip.5em plus .9em दिवासयानं \msNb,\hskip.5em plus .9em 
दिवाशायं \msNc,\hskip.5em plus .9em शिवाशयं \msKOb}}% 
    \var{{\devanagarifontvar\numnoemph\vb नित्यं\lem \mssALL,\hskip.2em plus .9em नित्य \msNb\oo 
 ॰परेण तु\lem \mssALL,\hskip.2em plus .9em 
॰परेन तु \msCa,\hskip.5em plus .9em ॰परेण च \msCc}}% 

%Verse 8:12

{\devanagarifont स्वर्गमार्गार्गला ह्येताः स्त्रियो नाम प्रकीर्तिताः {॥ ८:\hspace{.11em}१२॥} \veg\dontdisplaylinenum }%
     \var{{\devanagarifontvar\numnoemph\vc ह्येताः\lem \msNc,\hskip.2em plus .9em ह्येता \mssCaCbCc\msNa\msNb\msParis\msKOb\msPaperA\Ed}}% 
    \var{{\devanagarifontvar\numnoemph\vd स्त्रियो\lem \mssALL,\hskip.2em plus .9em स्त्रीयो \Ed\oo 
 ॰कीर्तिताः\lem \mssALL,\hskip.2em plus .9em ॰कीर्तिता \msNc}}% 
    \paral{{\devanagarifontsmall \vcd {\englishfont \compare\ \PADMAP\ 1.13.395cd:} परित्यजध्वं दाराणि स्वर्गमार्गार्गलानि च }}


\alalfejezet{नियमेषु व्रतपञ्चकम् (७)}
{\devanagarifont मार्जारकबकश्वानगोमहीव्रतपञ्चकम् \thinspace{\dandab} \dontdisplaylinenum }%
     \var{{\devanagarifontvar\numemph\vab मार्जारकबकश्वानगोमहीव्रत॰\lem \mssCaCbCc\msNa\msNc\msParis\msKOb,\hskip.2em plus .9em 
मार्जारबकबश्वानगोमहीव्रत॰ \msNb,\hskip.5em plus .9em 
मार्जारकवकश्वानगोमहीवेक॰ \msPaperA,\hskip.5em plus .9em 
मार्जारकश्च श्वानाश्च गोमहीवक \Ed}}% 


\alalalfejezet{मार्जारकव्रतम्}

{\devanagarifont स्वविष्ठमूत्रं भूमीषु छादयेद्द्विजसत्तम  \danda\dontdisplaylinenum }%
     \var{{\devanagarifontvar\numnoemph\vc ॰विष्ठ॰\lem \mssALL,\hskip.2em plus .9em ॰विष्टा॰ \Ed\oo 
 ॰मूत्रं\lem \mssALL,\hskip.2em plus .9em ॰मूत्र॰ \msCb\msNb}}% 

%Verse 8:13

{\devanagarifont सूर्यसोमानुमोदन्ति मार्जारव्रतिकेषु च {॥ ८:\hspace{.11em}१३॥} \veg\dontdisplaylinenum }%
     \var{{\devanagarifontvar\numnoemph\ve ॰मोदन्ति\lem \mssALL,\hskip.2em plus .9em ॰षादन्ति \Ed}}% 


\alalalfejezet{बकव्रतम्}

{\devanagarifont बकवच्चेन्द्रियग्रामं सुनियम्य तपोधन \thinspace{\dandab} \dontdisplaylinenum }%
     \var{{\devanagarifontvar\numemph\va तपोधन\lem \mssCaCbCc\msNa\msNb\msParis\msKOb,\hskip.2em plus .9em 
तपोधनः \msNc,\hskip.5em plus .9em तपोधनम् \msPaperA\Ed}}% 

%Verse 8:14

{\devanagarifont साधयेच्च मनस्तुष्टिं मोक्षसाधनतत्परः {॥ ८:\hspace{.11em}१४॥} \veg\dontdisplaylinenum }%
     \var{{\devanagarifontvar\numnoemph\vc साधयेच्च\lem \mssALL,\hskip.2em plus .9em 
साधये च \msCb\oo 
 मनस्तुष्टिं\lem \mssALL,\hskip.2em plus .9em 
मनस्तुष्टि॰ \msCb\msCc}}% 
    \var{{\devanagarifontvar\numnoemph\vd ॰साधन॰\lem \mssALL,\hskip.2em plus .9em 
॰सान॰ \msNc}}% 


\alalalfejezet{श्वानव्रतम्}

{\devanagarifont मूत्रविष्ठे न भूमीषु कुरुते धुनदं सदा \thinspace{\dandab} \dontdisplaylinenum }%
     \var{{\devanagarifontvar\numemph\va मूत्रविष्ठे न\lem \mssALL,\hskip.2em plus .9em 
मूत्रविष्टे च \Ed}}% 
    \var{{\devanagarifontvar\numnoemph\vb धुनदं\lem \mssCaCbCc\msNb\msNc\msParis\msPaperA,\hskip.2em plus .9em 
श्वानदः \msNa,\hskip.5em plus .9em धुनंद \msKOb,\hskip.5em plus .9em छादनं \Ed}}% 

%Verse 8:15

{\devanagarifont तुष्यते भगवान्शर्वः श्वानव्रतचरो यदि {॥ ८:\hspace{.11em}१५॥} \veg\dontdisplaylinenum }%
     \var{{\devanagarifontvar\numnoemph\vc शर्वः\lem \msCa\msNa\msNc\msParis\msKOb\msPaperA\Ed,\hskip.2em plus .9em 
सर्वः \msCb\msNb,\hskip.5em plus .9em सव्वः \msCc}}% 

\pend
\endnumbering
\vfill\pagebreak\beginnumbering\pstart
\vers


\alalalfejezet{गोव्रतम्}

{\devanagarifont मूत्रवर्चो न रुध्येत सदा गोव्रतिको नरः \thinspace{\dandab} \dontdisplaylinenum }%
     \var{{\devanagarifontvar\numemph\va ॰वर्चो\lem \msCa\msCc\msNb\msNc\msParis\msKOb\msPaperA,\hskip.2em plus .9em 
॰वच्चो \msCb\msNa,\hskip.5em plus .9em ॰वर्चा \Ed}}% 
    \var{{\devanagarifontvar\numnoemph\vb गोव्रतिको\lem \mssALL,\hskip.2em plus .9em 
\lk\lk तिको \msCa}}% 

%Verse 8:16

{\devanagarifont भीमस्तुष्टिकरश्चैव पुराणेषु निगद्यते {॥ ८:\hspace{.11em}१६॥} \veg\dontdisplaylinenum }%
     \var{{\devanagarifontvar\numnoemph\vc भीमस्तु॰\lem \msCc\msNb\Ed,\hskip.2em plus .9em 
भीमतु॰ \msCa\msCb\msNa\msNc\msParis\msKOb,\hskip.5em plus .9em 
भिमस्तु॰ \msPaperA}}% 


\alalalfejezet{महीव्रतम्}

{\devanagarifont कुद्दालैर्दारयन्तो ऽपि कीलकोटिशतैश्चितः \thinspace{\dandab} \dontdisplaylinenum }%
     \var{{\devanagarifontvar\numemph\va कुद्दालैर्दारयन्तो\lem \msNa\msParis\msKOb\Ed,\hskip.2em plus .9em 
कुद्दालैर्दारयन्नो \msCa,\hskip.5em plus .9em कुद्दारै दारयन्तो \msCb,\hskip.5em plus .9em 
कुदारै दारयन्ता \msCc,\hskip.5em plus .9em कुद्दालै द्दारयामास \msNb,\hskip.5em plus .9em 
कुद्दालै दारयन्तो \msNc,\hskip.5em plus .9em कुद्दालै \uncl{द्धार}यन्तो \msPaperA}}% 
    \var{{\devanagarifontvar\numnoemph\vb कीलकोटिशतैश्चितः\lem \msCa\msCb\msNa\msNb\msNc\msParis\msKOb,\hskip.2em plus .9em 
कीटकोटीशतैरपि \msCc\msPaperA\Ed}}% 

%Verse 8:17

{\devanagarifont क्षमते पृथिवी देवी एवमेव महीव्रतः {॥ ८:\hspace{.11em}१७॥} \veg\dontdisplaylinenum }%
     \var{{\devanagarifontvar\numnoemph\vd ॰व्रतः\lem \mssALL,\hskip.2em plus .9em ॰व्रत \msNc}}% 

{\devanagarifont व्रतपञ्चकमित्येतद्यश्चरेत जितेन्द्रियः \thinspace{\dandab} \dontdisplaylinenum }%
     \var{{\devanagarifontvar\numemph\vb जितेन्द्रियः\lem \mssALL,\hskip.2em plus .9em द्विजेन्द्रियः \msNb}}% 

%Verse 8:18

{\devanagarifont स चोत्तममिदं लोकं प्राप्नोति न च संशयः {॥ ८:\hspace{.11em}१८॥} \veg\dontdisplaylinenum }%
 

\alalfejezet{नियमेष्वुपवासः (८)}
{\devanagarifont शेषान्नमन्तरान्नं च नक्तायाचितमेव च \thinspace{\dandab} \dontdisplaylinenum }%
     \var{{\devanagarifontvar\numemph\va शेषान्नमन्तरान्नं च\lem \msCa\msCb\msNb\msNc\msParispcorr\msKObpcorr,\hskip.2em plus .9em 
शेषाणामन्तराणाञ्च \msCc\Ed,\hskip.5em plus .9em 
शेषान्नमन्नरान्नं च \msNa,\hskip.5em plus .9em 
शेषान्नमरान्नं च \msParisacorr\msKObacorr,\hskip.5em plus .9em 
शेषाणमन्तराणाञ्च \msPaperA}}% 
    \var{{\devanagarifontvar\numnoemph\vb नक्तायाचित॰\lem \mssALL,\hskip.2em plus .9em 
नक्त\uncl{या}चित॰ \msNc\oo 
 च\lem \mssALL,\hskip.2em plus .9em वा \Ed}}% 

%Verse 8:19

{\devanagarifont उपवासं च पञ्चैतत्कथयिष्यामि तच्छृणु {॥ ८:\hspace{.11em}१९॥} \veg\dontdisplaylinenum }%
     \var{{\devanagarifontvar\numnoemph\vcd पञ्चैतत्क॰\lem \mssALL,\hskip.2em plus .9em 
पञ्चैते क॰ \msCc}}% 


\alalalfejezet{शेषान्नम्}

{\devanagarifont वैश्वदेवातिथिशेषं पितृशेषं च यद्भवेत् \thinspace{\dandab} \dontdisplaylinenum }%
     \var{{\devanagarifontvar\numemph\va ॰शेषं\lem \mssALL,\hskip.2em plus .9em ॰शेषां \msCb}}% 

%Verse 8:20

{\devanagarifont भृत्यपुत्रकलत्रेभ्यः शेषाशी विघसाशनः {॥ ८:\hspace{.11em}२०॥} \veg\dontdisplaylinenum }%
     \var{{\devanagarifontvar\numnoemph\vd विघसाशनः\lem \msCa\msNa\msNb\msKObpcorr,\hskip.2em plus .9em विघसासनम् \msCb,\hskip.5em plus .9em विघसाषिनः \msCc,\hskip.5em plus .9em 
विघशासनः \msNc,\hskip.5em plus .9em 
विघसाश\uncl{नः} \msParispcorr,\hskip.5em plus .9em 
घसाशन \msParisacorr\msKObacorr,\hskip.5em plus .9em 
विघसासनः \msPaperA,\hskip.5em plus .9em विषसासनः \Ed}}% 

\pend
\endnumbering
\vfill\pagebreak\beginnumbering\pstart
\vers


\alalalfejezet{अन्तरान्नम्}

{\devanagarifont अन्तरा प्रातराशी च सायमाशी तथैव च \thinspace{\dandab} \dontdisplaylinenum }%
     \var{{\devanagarifontvar\numemph\va अन्तरा प्रातराशी\lem \eme,\hskip.2em plus .9em अन्तरा प्रान्तराशी \mssCaCbCc\msNa\msNc,\hskip.5em plus .9em 
अन्तरा \uncl{क्रन्त}राशी \msNb,\hskip.5em plus .9em 
अन्तारा प्रा\uncl{त्त}राशी \msParis\msKOb,\hskip.5em plus .9em 
अन्तमा प्रान्तराशी च \msPaperA,\hskip.5em plus .9em  अन्तसम्प्रान्तराशी \Ed}}% 
    \var{{\devanagarifontvar\numnoemph\vb सायमाशी\lem \msCb\msCc\msNa\msNb\msNc\msParis\msKOb,\hskip.2em plus .9em सायमाशीन् \msCa,\hskip.5em plus .9em 
नायमाशी \msPaperA,\hskip.5em plus .9em नियमाशी \Ed}}% 

%Verse 8:21

{\devanagarifont सदोपवासी भवति यो न भुङ्क्ते कदाचन {॥ ८:\hspace{.11em}२१॥} \veg\dontdisplaylinenum }%
     \var{{\devanagarifontvar\numnoemph\vc ॰वासी भवति\lem \mssALL,\hskip.2em plus .9em 
॰वासी च भवति \msCc}}% 
    \var{{\devanagarifontvar\numnoemph\vd कदाचन\lem \mssALL,\hskip.2em plus .9em कदाचनः \msCc}}% 
    \paral{{\devanagarifontsmall \vcd \similar\ {\englishfont \MBH\ 12.214.9:} 
                                 अन्तरा प्रातराशं च सायमाशं तथैव च\thinspace{\devanagarifontsmall ।}
                                 सदोपवासी च भवेद्यो न भुङ्क्ते कथंचन\thinspace{\devanagarifontsmall ॥} 
                     \similar\ {\englishfont \MBH\ 13.93.10:}
                                 अन्तरा सायमाशं च प्रातराशं तथैव च\thinspace{\devanagarifontsmall ।}
                                 सदोपवासी भवति यो न भुङ्क्ते ऽन्तरा पुनः\thinspace{\devanagarifontsmall ॥} }}


\alalalfejezet{नक्तान्नम्}

{\devanagarifont न दिवा भोजनं कार्यं रात्रौ नैव च भोजयेत् \thinspace{\dandab} \dontdisplaylinenum }%
     \var{{\devanagarifontvar\numemph\va भोजनं\lem \mssALL,\hskip.2em plus .9em नोजनं \msNc}}% 
    \var{{\devanagarifontvar\numnoemph\vb च\lem \mssALL,\hskip.2em plus .9em तु \msCb,\hskip.5em plus .9em \om\ \msNa\oo 
 भोजयेत्\lem \mssALL,\hskip.2em plus .9em कारयेत् \msNb}}% 

%Verse 8:22

{\devanagarifont नक्तवेले च भोक्तव्यं नक्तधर्मं समीहता {॥ ८:\hspace{.11em}२२॥} \veg\dontdisplaylinenum }%
     \var{{\devanagarifontvar\numnoemph\vc ॰वेले च\lem \msCa\msCc\msNa\msNb\msParis\msKOb\msPaperA,\hskip.2em plus .9em 
॰वेला च \msCb,\hskip.5em plus .9em ॰वेलो च \msNc,\hskip.5em plus .9em ॰वेले व \Ed}}% 
    \var{{\devanagarifontvar\numnoemph\vd ॰धर्मं समीहता\lem \msCa\msCb\msNa\msNc\msParis\msKOb,\hskip.2em plus .9em 
॰धर्मसमीहता \msCc\msNb,\hskip.5em plus .9em ॰धर्म्मसमीहिता \msPaperA,\hskip.5em plus .9em 
॰धर्म्मः समीहितः \Ed}}% 


\alalalfejezet{अयाचितान्नम्}

{\devanagarifont अनारभ्य य आहारं कुर्यान्नित्यमयाचितम् \thinspace{\dandab} \dontdisplaylinenum }%
     \var{{\devanagarifontvar\numemph\va अनारभ्य य\lem \conj,\hskip.2em plus .9em अनारम्भस्य \mssCaCbCc\msNa\msNb\msNc\msParis\msKOb\msPaperA\Ed}}% 
    \var{{\devanagarifontvar\numnoemph\vb कुर्यान्नि॰\lem \mssALL,\hskip.2em plus .9em कुर्या नि॰ \msNc}}% 

%Verse 8:23

{\devanagarifont परैर्दत्तं तु यो भुङ्क्ते तमयाचितमुच्यते {॥ ८:\hspace{.11em}२३॥} \veg\dontdisplaylinenum }%
     \var{{\devanagarifontvar\numnoemph\vc परैर्दत्तं तु\lem \msCa\msCb\msNa\msParis\msKOb\msPaperA,\hskip.2em plus .9em 
परै दत्तञ्च \msCc,\hskip.5em plus .9em परै दत्तन्तु \msNb,\hskip.5em plus .9em 
परैर्दन्तन्तु \msNc\Ed}}% 
    \var{{\devanagarifontvar\numnoemph\vd तमयाचि॰\lem \mssCaCbCc\msNa\msNb\msNc\msKOb\Ed,\hskip.2em plus .9em नमयाचि॰ \msParisacorr\msPaperA,\hskip.5em plus .9em 
\uncl{तम}याचि॰ \msParispcorr}}% 


\alalalfejezet{उपवासः}

{\devanagarifont भक्ष्यं भोज्यं च लेह्यं च चोष्यं पेयं च पञ्चमम् \thinspace{\dandab} \dontdisplaylinenum }%
     \var{{\devanagarifontvar\numemph\va भक्ष्यं\lem \mssALL,\hskip.2em plus .9em भक्ष्य \msNa}}% 

%Verse 8:24

{\devanagarifont न काङ्क्षेन्नोपयुञ्जीत उपवासः स उच्यते {॥ ८:\hspace{.11em}२४॥} \veg\dontdisplaylinenum }%
     \var{{\devanagarifontvar\numnoemph\vc काङ्क्षेन्नो॰\lem \mssALL,\hskip.2em plus .9em 
काङ्क्षे नो॰ \msCc\oo 
 ॰युञ्जीत\lem \msCc\msNa\msNb\msPaperA,\hskip.2em plus .9em ॰\lk\lk त \msCa,\hskip.5em plus .9em 
॰यञ्जीत \msCb,\hskip.5em plus .9em ॰भुजीत \msNc,\hskip.5em plus .9em ॰भुञ्जीत \msParis\msKOb\Ed}}% 
    \var{{\devanagarifontvar\numnoemph\vd ॰वासः स\lem \mssCaCbCc\msNa\msParis\msKOb\Ed,\hskip.2em plus .9em ॰वास स \msNb,\hskip.5em plus .9em ॰वासस्य \msNc,\hskip.5em plus .9em 
॰वासंः स \msPaperA}}% 

\pend
\endnumbering
\vfill\pagebreak\beginnumbering\pstart
\vers


\alalfejezet{नियमेषु मौनव्रतम् (९)}
{\devanagarifont मिथ्यापिशुनपारुष्यतीक्ष्णवागप्रलापनम् \thinspace{\dandab} \dontdisplaylinenum }%
     \var{{\devanagarifontvar\numemph\va ॰पारुष्य॰\lem \msCa\msCb\msNa\msNb\msNc\msParis\msKOb,\hskip.2em plus .9em ॰संभिन्ना \msCc,\hskip.5em plus .9em 
संभिन्नां \msPaperA,\hskip.5em plus .9em ॰याभिन्ना \Ed}}% 
    \var{{\devanagarifontvar\numnoemph\vb ॰तीक्ष्णवाग॰\lem \conj,\hskip.2em plus .9em ॰स्पृष्टवाग॰ \msCa\msCb\msNa\msNb\msNc\msParis\msKOb,\hskip.5em plus .9em 
पृष्टवाक॰ \msCc\msPaperA,\hskip.5em plus .9em 
पृष्तेवाक॰ \Ed\oo 
 ॰प्रलापनम्\lem \mssALL,\hskip.2em plus .9em ॰प्रलापिनं \msKOb}}% 

%Verse 8:25

{\devanagarifont मौनपञ्चकमित्येतद्धारयेन्नियतव्रतः {॥ ८:\hspace{.11em}२५॥} \veg\dontdisplaylinenum }%
     \var{{\devanagarifontvar\numnoemph\vc मौनपञ्चक॰\lem \msCa\msCb\msNb,\hskip.2em plus .9em मौनं पञ्चक॰ \msCc\msNa\msNc\msPaperA\Ed,\hskip.5em plus .9em 
मौनम्पञ्च॰ \msParis\msKOb\oo 
 ॰त्येत॰\lem \mssALL,\hskip.2em plus .9em ॰त्ये॰ \msParisacorr}}% 
    \var{{\devanagarifontvar\numnoemph\vd ॰रयेन्नि॰\lem \mssALL,\hskip.2em plus .9em ॰रयन्नि॰ \Ed}}% 


\alalalfejezet{मिथ्यावचनम्}

{\devanagarifont असम्भूतमदृष्टं च धर्माच्चापि बहिष्कृतम् \thinspace{\dandab} \dontdisplaylinenum }%
     \var{{\devanagarifontvar\numemph\va ॰दृष्टं च\lem \mssALL,\hskip.2em plus .9em दृष्ट\uncl{ञ्च} \msCc}}% 
    \var{{\devanagarifontvar\numnoemph\vb धर्माच्चापि\lem \msCa\msCb\msNa\msNb\msNc\msParis\msKOb,\hskip.2em plus .9em 
धर्मश्चापि \msCc\msPaperA,\hskip.5em plus .9em धर्मं चापि \Ed\oo 
 बहिष्कृतम्\lem \msCa\msCb\msNa\msNc\msParis\msKOb,\hskip.2em plus .9em बहिष्कृतः \msCc\Ed,\hskip.5em plus .9em नहिष्कृतं \msNb,\hskip.5em plus .9em 
बहिस्कृतंः \msPaperA}}% 

%Verse 8:26

{\devanagarifont अनर्थाप्रियवाक्यं यत् तन्मिथ्यावचनं स्मृतम् {॥ ८:\hspace{.11em}२६॥} \veg\dontdisplaylinenum }%
     \var{{\devanagarifontvar\numnoemph\vc अनर्था॰\lem \msCa\msCb\msNa\msNb\msNc\msParis\msKOb,\hskip.2em plus .9em अनर्थ॰ \msCc\msPaperA\Ed}}% 
    \var{{\devanagarifontvar\numnoemph\vcd ॰वाक्यं यत्तन्मि॰\lem \msCa\msCb\msNa\msParis\msKOb\msPaperA,\hskip.2em plus .9em 
वक्तार तं मि॰ \msCc,\hskip.5em plus .9em 
वाक्य यत्तन्मि॰ \msNb,\hskip.5em plus .9em 
वाक्यं यन्तन्मि॰ \msNc\Ed}}% 
    \var{{\devanagarifontvar\numnoemph\vd स्मृतम्\lem \mssALL,\hskip.2em plus .9em स्मृतः \msCb}}% 


\alalalfejezet{पिशुनः}

{\devanagarifont परश्रीं नाभिनन्दन्ति परस्यैश्वर्यमेव च \thinspace{\dandab} \dontdisplaylinenum }%
     \var{{\devanagarifontvar\numemph\va परश्रीं ना॰\lem \msCa\msCb\msNa\msNc\msParis\msKOb,\hskip.2em plus .9em परस्त्री ना॰ \msCc\msPaperApcorr\Ed,\hskip.5em plus .9em 
परस्त्रीन्ना॰ \msNb,\hskip.5em plus .9em परस्त्री श्री ना॰ \msPaperAacorr\oo 
 ॰भिनन्दन्ति\lem \mssALL,\hskip.2em plus .9em 
॰भिनन्ति \msCb,\hskip.5em plus .9em ॰भिन्नन्दन्ति \msCc}}% 
    \var{{\devanagarifontvar\numnoemph\vb परस्यैश्वर्य॰\lem \mssALL,\hskip.2em plus .9em 
परसैश्वर्य॰ \msCb}}% 

%Verse 8:27

{\devanagarifont अनिष्टदर्शनाकाङ्क्षी पिशुनः समुदाहृतः {॥ ८:\hspace{.11em}२७॥} \veg\dontdisplaylinenum }%
     \var{{\devanagarifontvar\numnoemph\vc ॰दर्शना॰\lem \msCa\msCb\msNa\msNc\msParis\msKOb\Ed,\hskip.2em plus .9em ॰द\uncl{ब्भ}ना॰ \msCc,\hskip.5em plus .9em ॰दर्शनां \msNb,\hskip.5em plus .9em 
॰दशना॰ \msPaperA}}% 
    \var{{\devanagarifontvar\numnoemph\vd पिशुनः\lem \mssALL,\hskip.2em plus .9em पिशुन \msCc}}% 


\alalalfejezet{पारुष्यम्}

{\devanagarifont मृत माता पिता चैव हानिस्थानं कथं भवेत् \thinspace{\dandab} \dontdisplaylinenum }%
     \var{{\devanagarifontvar\numemph\va मृत\lem \mssALL,\hskip.2em plus .9em मृता \msParispcorr}}% 
    \var{{\devanagarifontvar\numnoemph\vb ॰स्थानं\lem \mssALL,\hskip.2em plus .9em ॰स्थान \msCb\msCc}}% 

%Verse 8:28

{\devanagarifont भुङ्क्ष्व कामममृष्टानां पारुष्यं समुदाहृतम् {॥ ८:\hspace{.11em}२८॥} \veg\dontdisplaylinenum }%
     \var{{\devanagarifontvar\numnoemph\vc भुङ्क्ष्व\lem\msNc\msParis\msKOb,\hskip.2em plus .9em भुक्त्व \msCa,\hskip.5em plus .9em भुक्त्वा \msCb\msCc,\hskip.5em plus .9em 
भुं\uncl{क्ष} \msNa,\hskip.5em plus .9em भुक्ष \msNb,\hskip.5em plus .9em 
भु\uncl{क्त} \msPaperA,\hskip.5em plus .9em भुक्ता \Ed\oo 
 कामममृष्टानां\lem \msCa\msNa\msNc\msParis\msKOb\Ed,\hskip.2em plus .9em कममसृष्टानां \msCb,\hskip.5em plus .9em 
कामसुसमृष्तानां \msCc,\hskip.5em plus .9em 
काममुमृष्ताना \msNb,\hskip.5em plus .9em 
पारुष्यमृष्टना \msPaperA}}% 


\alalalfejezet{तीक्ष्णवाक्}

{\devanagarifont हृदि न स्फुटसे मूढ शिरो वा न विदार्यसे \thinspace{\dandab} \dontdisplaylinenum }%
     \var{{\devanagarifontvar\numemph\va स्फुटसे\lem \mssALL,\hskip.2em plus .9em स्फुटय \msNb}}% 

%Verse 8:29

{\devanagarifont एवमादीन्यनेकानि तीक्ष्णवादी स उच्यते {॥ ८:\hspace{.11em}२९॥} \veg\dontdisplaylinenum }%
 

\alalalfejezet{असत्प्रलापः}

{\devanagarifont द्यूतभोजनयुद्धं च मद्यस्त्रीकथमेव च \thinspace{\dandab} \dontdisplaylinenum }%
     \var{{\devanagarifontvar\numemph\va ॰युद्धं\lem \mssALL,\hskip.2em plus .9em ॰युद्धश् \Ed}}% 
    \var{{\devanagarifontvar\numnoemph\vb ॰कथ॰\lem \msNb\msNc\msKOb,\hskip.2em plus .9em ॰कष॰ \mssCaCbCc\msNa\msParis,\hskip.5em plus .9em ॰कर्ष॰ \msPaperA\Ed}}% 

%Verse 8:30

{\devanagarifont असत्प्रलापः पञ्चैतत्कीर्तितं मे द्विजोत्तम {॥ ८:\hspace{.11em}३०॥} \veg\dontdisplaylinenum }%
     \var{{\devanagarifontvar\numnoemph\vcd पञ्चैतत्की॰\lem \mssALL,\hskip.2em plus .9em 
पञ्चैते की॰ \msNb,\hskip.5em plus .9em पञ्चेतत्की॰ \msNc}}% 
    \var{{\devanagarifontvar\numnoemph\vd मे\lem \mssALL,\hskip.2em plus .9em ते \Ed}}% 

{\devanagarifont मौनमेव सदा कार्यं वाक्यसौभाग्यमिच्छता \thinspace{\dandab} \dontdisplaylinenum }%
     \var{{\devanagarifontvar\numemph\va कार्यं\lem \mssALL,\hskip.2em plus .9em कार्या \msNb}}% 
    \var{{\devanagarifontvar\numnoemph\vb वाक्य॰\lem \msCa\msCb\msNa\msNc\msParis\msKOb\Ed,\hskip.2em plus .9em वाक्यं \msCc\msNb\msPaperA\oo 
 ॰सौभाग्य॰\lem \mssALL,\hskip.2em plus .9em ॰सौभार्य॰ \msCb}}% 

%Verse 8:31

{\devanagarifont अपारुष्यमसम्भिन्नं वाक्यं सत्यमुदीरयेत् {॥ ८:\hspace{.11em}३१॥} \veg\dontdisplaylinenum }%
     \var{{\devanagarifontvar\numnoemph\vc ॰भिन्नं\lem \mssALL,\hskip.2em plus .9em 
॰भिन्न \msCc,\hskip.5em plus .9em ॰दिग्धं \Ed}}% 

{\devanagarifont यस्तु मौनस्य नो कर्ता दूषितः स कुलाधमः \thinspace{\dandab} \dontdisplaylinenum }%
     \var{{\devanagarifontvar\numemph\vb दूषितः\lem \mssALL,\hskip.2em plus .9em दूषित \msCc,\hskip.5em plus .9em भूषितः \Ed}}% 

%Verse 8:32

{\devanagarifont जन्मे जन्मे च दुर्गन्धो मूकश्चैवोपजायते {॥ ८:\hspace{.11em}३२॥} \veg\dontdisplaylinenum }%
     \var{{\devanagarifontvar\numnoemph\vc जन्मे जन्मे\lem \msCb\msCc\msNa\msPaperA\Ed,\hskip.2em plus .9em जन्म जन्म \msCa\msNb\msNc\msParis\msKOb\oo 
 दुर्गन्धो\lem \msCa\msNb\msNc\msParis\msKOb\msPaperA,\hskip.2em plus .9em 
दुग्गन्धो \msCb,\hskip.5em plus .9em दुर्गन्धा \msCc,\hskip.5em plus .9em दुगन्धो \msNa,\hskip.5em plus .9em दृगन्धो \Ed}}% 

\pend
\endnumbering
\vfill\pagebreak\beginnumbering\pstart
\vers

\nemslokalong


\ujvers\nemsloka {
{\devanagarifont तस्मान्मौनव्रतं सदैव सुदृढं कुर्वीत यो निश्चितं }%
  \dontdisplaylinenum}    \var{{\devanagarifontvar\numemph\va तस्मान्मौ॰\lem \msCc\msNb\msNc\msParis\msKOb\msPaperA\Ed,\hskip.2em plus .9em 
\lk\lk त्मौ॰ \msCa,\hskip.5em plus .9em तस्मात्मौ॰ \msCb\msNa\oo 
 सदैव\lem \msCa\msCb\msNa\msParis\msKObpcorr\Ed,\hskip.2em plus .9em 
सदेव \msCc\msNc\msKObacorr\msPaperA,\hskip.5em plus .9em सुदैत्य \msNb\oo 
 कुर्वीत यो निश्चितम्\lem \msCa\msCb\msNc\msParis\msKOb\msPaperA\Ed,\hskip.2em plus .9em 
कुर्वन्ति येन्निश्चितम् \msCc\msNa,\hskip.5em plus .9em 
कुर्वन्ति योन्निश्चित \msNb}}% 


\nemslokab

{\devanagarifont वाचा तस्य अलङ्घ्यता च भवति सर्वां सभां नन्दति  \danda\dontdisplaylinenum }%
     \var{{\devanagarifontvar\numnoemph\vb अलङ्घ्यता च\lem \msCa\msCb\msNa\msNb\msParis\msKOb,\hskip.2em plus .9em अलंघ्यताञ्च \msCc\msNc\msPaperA\Ed\oo 
 सर्वां सभां\lem \msCa\msNa\msParis\msKOb\msPaperA\Ed,\hskip.2em plus .9em सर्वा सभा \msCb\msNc,\hskip.5em plus .9em 
सर्वः सभान् \msCc,\hskip.5em plus .9em सर्वा सुभा \msNb}}% 

\nemslokac

{\devanagarifont वक्त्राच्चोत्पलगन्धमस्य सततं वायन्ति गन्धोत्कटाः }%
  \dontdisplaylinenum    \var{{\devanagarifontvar\numnoemph\vc वक्त्राच्चोत्पलगन्धमस्य\lem \msCa\msCb\msNc\msParisacorr\msKOb\msPaperA,\hskip.2em plus .9em 
वक्त्रं चोत्पलमस्य \msCc,\hskip.5em plus .9em वक्त्रं चोत्पलगन्धमस्य \msNa,\hskip.5em plus .9em 
वक्त्रं चोत्पल\uncl{ग}न्धमस्य \msNb,\hskip.5em plus .9em 
वक्त्राश्चोत्पलगन्धमस्य \msParispcorr,\hskip.5em plus .9em 
वक्त्राच्चोतरगन्धमस्य \Ed}}% 

%Verse 8:33


\nemslokad

{\devanagarifont शास्त्रानेकसहस्रशो गिरि नरः प्रोच्चार्यते निर्मलम् {॥ ८:\hspace{.11em}३३॥} \veg\dontdisplaylinenum }%
     \var{{\devanagarifontvar\numnoemph\vd ॰सहस्रशो\lem \mssALL,\hskip.2em plus .9em ॰सहस्राशो \msCb\oo 
 ॰मलम्\lem \msCa\msNa\msNb\msNc\msParis\msKOb,\hskip.2em plus .9em ॰मलः \msCb\msCc\msPaperA\Ed}}% 

\nemslokanormal


\vers



\alalfejezet{नियमेषु स्नानम् (१०)}
{\devanagarifont स्नानं पञ्चविधं चैव प्रवक्ष्यामि यथातथम् \thinspace{\dandab} \dontdisplaylinenum }%
     \var{{\devanagarifontvar\numemph\va पञ्चविधं\lem \mssALL,\hskip.2em plus .9em पञ्चवि \msCb}}% 
    \var{{\devanagarifontvar\numnoemph\vb यथातथम्\lem \mssALL,\hskip.2em plus .9em \lk\lk तथम् \msCa}}% 

%Verse 8:34

{\devanagarifont आग्नेयं वारुणं ब्राह्म्यं वायव्यं दिव्यमेव च {॥ ८:\hspace{.11em}३४॥} \veg\dontdisplaylinenum }%
     \var{{\devanagarifontvar\numnoemph\vc आग्नेयं\lem \mssALL,\hskip.2em plus .9em आग्नेये \msNb\oo 
 वारुणं\lem \mssALL,\hskip.2em plus .9em ब्राह्मणं \msPaperA\Ed\oo 
 ब्राह्म्यं\lem \mssALL,\hskip.2em plus .9em ब्रह्म्यं \msNc}}% 


\alalalfejezet{आग्नेयं स्नानम्}

{\devanagarifont आग्नेयं भस्मना स्नानं तोयाच्छतगुणं फलम् \thinspace{\dandab} \dontdisplaylinenum }%
     \var{{\devanagarifontvar\numemph\va स्नानं\lem \mssALL,\hskip.2em plus .9em स्नाना \msNaacorr}}% 
    \var{{\devanagarifontvar\numnoemph\vb ॰गुणं\lem \mssALL,\hskip.2em plus .9em ॰गुण॰ \msNc}}% 

%Verse 8:35

{\devanagarifont भस्मपूतं पवित्रं च भस्म पापप्रणाशनम् {॥ ८:\hspace{.11em}३५॥} \veg\dontdisplaylinenum }%
     \var{{\devanagarifontvar\numnoemph\vd ॰प्रणाशनम्\lem \mssALL,\hskip.2em plus .9em ॰प्रशनाशनं \msKOb}}% 

{\devanagarifont तस्माद्भस्म प्रयुञ्जीत देहिनां तु मलापहम् \thinspace{\dandab} \dontdisplaylinenum }%
     \var{{\devanagarifontvar\numemph\va तस्माद्भस्म प्रयुञ्जीत\lem \mssALL,\hskip.2em plus .9em 
\lk\lk\lk\lk\lk\lk\lk त \msNb}}% 
    \var{{\devanagarifontvar\numnoemph\vb मला॰\lem \mssALL,\hskip.2em plus .9em पला॰ \msPaperA}}% 

%Verse 8:36

{\devanagarifont सर्वशान्तिकरं भस्म भस्म रक्षकमुत्तमम् {॥ ८:\hspace{.11em}३६॥} \veg\dontdisplaylinenum }%
     \var{{\devanagarifontvar\numnoemph\vc सर्व॰\lem \mssALL,\hskip.2em plus .9em \uncl{ए}ना॰ \msPaperA}}% 
    \var{{\devanagarifontvar\numnoemph\vcd भस्म भस्म\lem \mssALL,\hskip.2em plus .9em 
भस्म \msKObacorr}}% 

\pend
\endnumbering
\vfill\pagebreak\beginnumbering\pstart
\vers

{\devanagarifont भस्मना त्र्यायुषं कृत्वा ब्रह्मचर्यव्रते स्थितम् \thinspace{\dandab} \dontdisplaylinenum }%
     \var{{\devanagarifontvar\numemph\va त्र्यायुषं कृत्वा\lem \msCb\msCc\msNa\msNb\msNc\msPaperA\Ed,\hskip.2em plus .9em 
त्र्यायु\lk\lk\lk\ \msCa,\hskip.5em plus .9em 
त्र्यायुष्यं कृत्वा \msParis,\hskip.5em plus .9em त्र्यायुष्मं क्र्ट्वा \msKOb}}% 
    \var{{\devanagarifontvar\numnoemph\vb ॰व्रते\lem \mssALL,\hskip.2em plus .9em ॰व्रत॰ \msPaperA\Ed}}% 

%Verse 8:37

{\devanagarifont भस्मना ऋषयः सर्वे पवित्रीकृतमात्मनः {॥ ८:\hspace{.11em}३७॥} \veg\dontdisplaylinenum }%
     \var{{\devanagarifontvar\numnoemph\vc ऋषयः सर्वे\lem \mssALL,\hskip.2em plus .9em 
ऋषिभिर्सर्वैः \Ed}}% 

{\devanagarifont भस्मना विबुधा मुक्ता वीरभद्रभयार्दिताः \thinspace{\dandab} \dontdisplaylinenum }%
     \var{{\devanagarifontvar\numemph\va मुक्ता\lem \mssALL,\hskip.2em plus .9em मुक्ताः \Ed}}% 
    \var{{\devanagarifontvar\numnoemph\vb ॰र्दिताः\lem \mssALL,\hskip.2em plus .9em ॰र्त्तिताः \msCb}}% 

%Verse 8:38

{\devanagarifont भस्मानुशंसं दृष्ट्वैव ब्रह्मणानुमतिः कृता {॥ ८:\hspace{.11em}३८॥} \veg\dontdisplaylinenum }%
     \var{{\devanagarifontvar\numnoemph\vc भस्मानुशंसं दृष्ट्वैव\lem \corrTorzsok,\hskip.2em plus .9em 
भस्मानुसंसं दृष्ट्यैव \msCa,\hskip.5em plus .9em भस्मानुशंसां दृष्ट्वव \msCb,\hskip.5em plus .9em 
भस्मानुसंसदृष्टैव \msCc\msNb,\hskip.5em plus .9em भस्मानुसंसन्दृष्ट्वैव \msNa,\hskip.5em plus .9em 
भस्मानुशंसंदृष्ट्यैवं \msNc,\hskip.5em plus .9em भस्मानुशंसं दृष्टैव \msParis,\hskip.5em plus .9em 
भस्मानुशसं दृष्त्वा \msKOb,\hskip.5em plus .9em 
भस्मानुशंसं \uncl{दृष्टै}व \msPaperA,\hskip.5em plus .9em 
भस्मना शं प्रदृश्यैवं \Ed}}% 
    \var{{\devanagarifontvar\numnoemph\vd ब्रह्मणानुमतिः\lem \eme,\hskip.2em plus .9em ब्रह्मणानुमता \mssCaCbCc\msNa\msNb\msNc\msParis\msKOb,\hskip.5em plus .9em 
ब्राह्मणानुमतो \msPaperA\Ed\oo 
 कृता\lem \eme,\hskip.2em plus .9em कृतः \msCa\msCb\msNb\msNc\msParis\msPaperA\Ed,\hskip.5em plus .9em कृतिः \msCc,\hskip.5em plus .9em कृताः \msNa\msKOb}}% 

{\devanagarifont चतुराश्रमतो ऽधिक्यं व्रतं पाशुपतं कृतम् \thinspace{\dandab} \dontdisplaylinenum }%
     \var{{\devanagarifontvar\numemph\va चतुराश्रमतो\lem \msCb\msCc\msNb\msParis\msKOb\Ed,\hskip.2em plus .9em 
चातुराश्रमतो \msCa\msNc\msPaperA,\hskip.5em plus .9em चतुराश्रतो \msNaacorr,\hskip.5em plus .9em 
चातुराश्रमतो \msNapcorr}}% 
    \var{{\devanagarifontvar\numnoemph\vab ऽधिक्यं व्रतं पाशुपतं कृतम्\lem \mssALL,\hskip.2em plus .9em 
\uncl{धिक्यव्रतपाशुपत}\lk\lk\lk\ \msNb\ \toplost}}% 

%Verse 8:39

{\devanagarifont तस्मात्पाशुपतं श्रेष्ठं भस्मधारणहेतुतः {॥ ८:\hspace{.11em}३९॥} \veg\dontdisplaylinenum }%
     \var{{\devanagarifontvar\numnoemph\vc तस्मात्पाशुपतं श्रेष्ठं\lem \mssALL,\hskip.2em plus .9em 
\om \msNb}}% 
    \var{{\devanagarifontvar\numnoemph\vd ॰हेतुतः\lem \emeTorzsok,\hskip.2em plus .9em ॰हेतवः \msCa\msCb\msNa\msNc\msParis\msKOb\msPaperA\Ed,\hskip.5em plus .9em 
॰हेतुना \msCc,\hskip.5em plus .9em ॰हेतुनुतः \msNb}}% 


\alalalfejezet{वारुणं स्नानम्}

{\devanagarifont वारुणं सलिलं स्नानं कर्तव्यं विविधं नरैः \thinspace{\dandab} \dontdisplaylinenum }%
     \var{{\devanagarifontvar\numemph\va वारुणं\lem \msCb\msCc\msNa\msNb\msParis\msKOb\Ed,\hskip.2em plus .9em 
वा\lk\lk\  \msCa,\hskip.5em plus .9em वारुणा \msNcacorr,\hskip.5em plus .9em वारुण \msNcpcorr,\hskip.5em plus .9em 
वरुणं \msPaperA\oo 
 सलिलं\lem \mssCaCbCc\msNa\msNb\msParis\msKOb,\hskip.2em plus .9em सलिल॰ \msNc\msPaperA\Ed}}% 
    \var{{\devanagarifontvar\numnoemph\vb विविधं नरैः\lem \mssCaCbCc\msNa\msKOb\msPaperA,\hskip.2em plus .9em विविन्नरैः \msNb,\hskip.5em plus .9em 
विधिवन्नरैः \msNc\msParis\Ed}}% 

%Verse 8:40

{\devanagarifont नदीतोयतडागेषु प्रस्रवेषु ह्रदेषु च {॥ ८:\hspace{.11em}४०॥} \veg\dontdisplaylinenum }%
     \var{{\devanagarifontvar\numnoemph\vc ॰तडागेषु\lem \mssALL,\hskip.2em plus .9em 
॰तडागेवा \msNb}}% 
    \var{{\devanagarifontvar\numnoemph\vd प्रस्रवेषु\lem \mssALL,\hskip.2em plus .9em 
प्रयेवेषु \msNb,\hskip.5em plus .9em प्रभवेषु \msNc}}% 


\alalalfejezet{ब्राह्म्यं स्नानम्}

{\devanagarifont ब्रह्मस्नानं च विप्रेन्द्र आपोहिष्ठं विदुर्बुधाः \thinspace{\dandab} \dontdisplaylinenum }%
     \var{{\devanagarifontvar\numemph\va विप्रेन्द्र\lem \mssALL,\hskip.2em plus .9em विपेन्द्र \msNc\msParis}}% 
    \var{{\devanagarifontvar\numnoemph\vb विदुर्बु॰\lem \mssALL,\hskip.2em plus .9em विर्दुर्बु॰ \msNc}}% 

%Verse 8:41

{\devanagarifont त्रिसंध्यमेव कर्तव्यं ब्रह्मस्नानं तदुच्यते {॥ ८:\hspace{.11em}४१॥} \veg\dontdisplaylinenum }%
 

\alalalfejezet{वायव्यं स्नानम्}

{\devanagarifont गोषु संचारमार्गेषु यत्र गोधूलिसम्भवः \thinspace{\dandab} \dontdisplaylinenum }%
 
%Verse 8:42

{\devanagarifont तत्र गत्वावसीदेत स्नानमुक्तं मनीषिभिः {॥ ८:\hspace{.11em}४२॥} \veg\dontdisplaylinenum }%
     \var{{\devanagarifontvar\numemph\vd ॰क्तं\lem \mssALL,\hskip.2em plus .9em ॰क्त \msNb}}% 


\alalalfejezet{दिव्यं स्नानम्}

{\devanagarifont वर्षतोयाम्बुधाराभिः प्लावयित्वा स्वकां तनुम् \thinspace{\dandab} \dontdisplaylinenum }%
     \var{{\devanagarifontvar\numemph\vb तनुम्\lem \mssALL,\hskip.2em plus .9em तनं \msNc}}% 

%Verse 8:43

{\devanagarifont स्नानं दिव्यं वदत्येव जगदादिमहेश्वरः {॥ ८:\hspace{.11em}४३॥} \veg\dontdisplaylinenum }%
     \var{{\devanagarifontvar\numnoemph\vc दिव्यं\lem \mssALL,\hskip.2em plus .9em दिव्य \msNb\msPaperA}}% 
    \var{{\devanagarifontvar\numnoemph\vd जगदादि॰\lem \mssALL,\hskip.2em plus .9em गजदादि॰ \msCb}}% 

\ujvers\nemsloka {
{\devanagarifont इति नियमविभागः पञ्चभेदेन विप्र }%
  \dontdisplaylinenum}    \var{{\devanagarifontvar\numemph\va ॰भागः\lem \mssALL,\hskip.2em plus .9em ॰भागं \msNc}}% 


\nemslokab

{\devanagarifont निगदित तव पृष्टः सर्वलोकानुकम्प्य  \danda\dontdisplaylinenum }%
     \var{{\devanagarifontvar\numnoemph\vb निगदित तव\lem \Ed,\hskip.2em plus .9em 
निगदितस्तव \mssCaCbCc\msNa\msNb\msNc\msParis\msKOb\msPaperA\ \unmetr\oo 
 ॰कम्प्य\lem \msCa,\hskip.2em plus .9em ॰कम्प \msCb\msCc\msNa\msNc\msParis\msKOb,\hskip.5em plus .9em 
॰कम्पः \msNb,\hskip.5em plus .9em ॰कम्प्यः \msPaperA\Ed}}% 

\nemslokac

{\devanagarifont सकलमलपहारी धर्मपञ्चाशदेतन् }%
  \dontdisplaylinenum    \var{{\devanagarifontvar\numnoemph\vc ॰पहारी\lem \msCb\msCc\msNb,\hskip.2em plus .9em ॰पहारि \msCa\msNc\unmetr,\hskip.5em plus .9em 
॰प्रहारि \msNa\msParis\msKOb\msPaperA,\hskip.5em plus .9em ॰पहारे \Ed\oo 
 ॰पञ्चाशदेतन्\lem \msCa\msCb\msNa\msNbpcorr\msNc\msParis\msKOb,\hskip.2em plus .9em 
॰पञ्चाशमेतन् \msCc\msPaperA\Ed,\hskip.5em plus .9em 
॰पञ्चादेतन् \msNbacorr}}% 

%Verse 8:44


\nemslokad

{\devanagarifont न भवति पुनजन्म कल्पकोट्यायुते ऽपि {॥ ८:\hspace{.11em}४४॥} \veg\dontdisplaylinenum }%
     \var{{\devanagarifontvar\numnoemph\vd पुनजन्म\lem \msCc\msNb\msKOb,\hskip.2em plus .9em पुनर्जन्म \msCa\msNa\msNc\msParis\msPaperA\Ed,\hskip.5em plus .9em 
पुन\uncl{र्जर्म} \msCb}}% 

\vers


{\devanagarifont 
\jump
\begin{center}
\ketdanda~इति वृषसारसंग्रहे नियमप्रशंसा नामाध्यायो ऽष्टमः~\ketdanda
\end{center}
\dontdisplaylinenum\vers  }%
     \lacuna{\devanagarifontsmall {\englishfont \msM\ resumes with} अष्टमः {\englishfont in the colophon.} }%
      \var{{\devanagarifontvar\numnoemph{\englishfont \Colo:} इति वृषसारसंग्रहे नियमप्रशंसा नामाध्यायो ऽष्टमः\lem \msParis\msKOb,\hskip.2em plus .9em 
इति वृषसारसंग्रहे नियमप्रशंसा नामाध्याय अष्टमः \msCa\msNa\msPaperA,\hskip.5em plus .9em 
\om \msCb,\hskip.5em plus .9em 
इति वृषसारसंग्रहे नियमप्रशंसा नामाध्यायाष्टमः \msCc\msNb,\hskip.5em plus .9em 
इति वृषसारसंग्रहे नियमप्रशंसा नामाध्यायाऽष्टमः \msNc,\hskip.5em plus .9em 
अष्टमः\thinspace{\devanagarifont ।} श्लोक ४६ \msM,\hskip.5em plus .9em 
इति वृषसारसंग्रहे नियमप्रशंसा नाम अष्टमो ऽध्यायः \Ed}}% 
\bekveg\szamveg
\vfill
\phpspagebreak

\versno=0\fejno=9
\thispagestyle{empty}

\centerline{\Large\devanagarifontbold [   नवमो ऽध्यायः  ]}{\vrule depth10pt width0pt} \fancyhead[CE]{{\footnotesize\devanagarifont वृषसारसंग्रहे  }}
\fancyhead[CO]{{\footnotesize\devanagarifont नवमो ऽध्यायः  }}
\fancyhead[LE]{}
\fancyhead[RE]{}
\fancyhead[LO]{}
\fancyhead[RO]{}
\szam\bek



\alalfejezet{त्रैगुण्यम्}
\vers


{\devanagarifont [अनर्थयज्ञ उवाच {\dandab}\dontdisplaylinenum  ] }%
 
{\devanagarifont त्रिकालगुणभेदेन भिन्नं सर्वचराचरम् \thinspace{\danda} \dontdisplaylinenum }%
     \var{{\devanagarifontvar\numemph\va त्रिकाल॰\lem \mssALL,\hskip.2em plus .9em त्रिष्काल॰ \msCc\oo 
 ॰भेदेन\lem \mssALL,\hskip.2em plus .9em ॰भेन \msNbacorr}}% 
    \var{{\devanagarifontvar\numnoemph\vb भिन्नं\lem \mssALL,\hskip.2em plus .9em भिन्न \msNb}}% 
    \lacuna{\devanagarifontsmall {\englishfont Witnesses used for this chapter: \msCa\ ff.\thinspace 205v--207r, 
                                              \msCb\ ff.\thinspace 211v--212v, 
                                              \msCc\ ff.\allowbreak\thinspace 282r--283v,
                                              \msNa\ ff.\thinspace 13r--14v, 
                                              \msNb\ exp.\thinspace 54 (lower) -- 55 (lower),
                                              \msNc\ ff.\thinspace 221r--222v,
                                              \Ed\ pp.\thinspace 606--609; 
                                              \mssCaCbCc~= \msCa + \msCb + \msCc} }%
  
%Verse 9:1

{\devanagarifont तस्मात्त्रिगुणबन्धेन वेष्टितं निखिलं जगत् {॥ ९:\hspace{.11em}१॥} \veg\dontdisplaylinenum }%
     \var{{\devanagarifontvar\numnoemph\vc तस्मात्त्रि॰\lem \mssALL,\hskip.2em plus .9em तस्मा त्रि॰ \msCc\msNc}}% 

{\devanagarifont विगतराग उवाच {\dandab}\dontdisplaylinenum  }%
 
{\devanagarifont त्रैकाल्यमिति किं ज्ञेयं त्रैधातुकशरीरिणः \thinspace{\danda} \dontdisplaylinenum }%
     \var{{\devanagarifontvar\numemph\va ॰काल्यम्\lem \mssALL,\hskip.2em plus .9em ॰कालम् \msCa\msNc}}% 
    \var{{\devanagarifontvar\numnoemph\vab किं ज्ञेयं त्रै॰\lem \msCa\msNc,\hskip.2em plus .9em 
विज्ञेयं त्रै॰ \msCb\msNa\msNb\Ed,\hskip.5em plus .9em कि ज्ञेयम्त्रै॰ \msCc}}% 
    \var{{\devanagarifontvar\numnoemph\vb ॰धातुक॰\lem \mssALL,\hskip.2em plus .9em ॰धायुक्त॰ \Ed}}% 

%Verse 9:2

{\devanagarifont किंचिद्विस्तरमेवेह कथयस्व तपोधन {॥ ९:\hspace{.11em}२॥} \veg\dontdisplaylinenum }%
     \var{{\devanagarifontvar\numnoemph\vc किंचि॰\lem \mssALL,\hskip.2em plus .9em 
सात्त्विको भगव् विष्णु राजसः कमलोद्भवः\thinspace{\devanagarifont ।} 
तामसो भगवानीशः सकलं विक किञ्चि॰ \msCbacorr\ 
{\englishfont (eyeskip to 9.5)}\oo 
 ॰वेह\lem \mssALL,\hskip.2em plus .9em ॰तद्धि \Ed}}% 
    \var{{\devanagarifontvar\numnoemph\vd कथयस्व\lem \mssALL,\hskip.2em plus .9em क\lk\lk \lk\ \msCa}}% 

{\devanagarifont अनर्थयज्ञ उवाच {\dandab}\dontdisplaylinenum  }%
 
{\devanagarifont त्रैकाल्यं त्रिगुणं ज्ञेयं व्यापी प्रकृतिसम्भवः \thinspace{\danda} \dontdisplaylinenum }%
     \var{{\devanagarifontvar\numemph\va ॰काल्यं\lem \mssALL,\hskip.2em plus .9em ॰काल्य \msCc\oo 
 ॰गुणं\lem \mssALL,\hskip.2em plus .9em ॰गुण \msCc}}% 

%Verse 9:3

{\devanagarifont अन्योन्यमुपजीवन्ति अन्योन्यमनुवर्तिनः {॥ ९:\hspace{.11em}३॥} \veg\dontdisplaylinenum }%
     \paral{{\devanagarifontsmall \vcd {\englishfont \similar\ \BRAHMANDAPUR\ 1.4.9--10:}
                         एत एव त्रयो लोका एत एव त्रयो गुणाः\thinspace{\devanagarifontsmall ।}  
                         एत एव त्रयो वेदा एत एव त्रजो ऽग्नयः\thinspace{\devanagarifontsmall ॥}
                         परस्परान्वया ह्येते परस्परमनुव्रताः\thinspace{\devanagarifontsmall ।}
                         परस्परेण वर्तन्ते प्रेरयन्ति परस्परम्\thinspace{\devanagarifontsmall ॥}
                      {\englishfont \similar\ \VAYUP\ 1.5.16--17ab \similar\ \LINPU\ 1.70.78--79} }}

\pend
\endnumbering
\vfill\pagebreak\beginnumbering\pstart
\vers

{\devanagarifont सत्त्वं रजस्तमश्चैव रजः सत्त्वं तमस्तथा \thinspace{\dandab} \dontdisplaylinenum }%
     \var{{\devanagarifontvar\numemph\va सत्त्वं\lem \mssALL,\hskip.2em plus .9em सत्व \msNb\oo 
 रजस्त॰\lem \mssALL,\hskip.2em plus .9em रजत॰ \Ed}}% 
    \var{{\devanagarifontvar\numnoemph\vb रजः\lem \msCa\msCb\msNa\msNc,\hskip.2em plus .9em रज॰ \msCc\msNb\Ed\oo 
 सत्त्वं तमस्तथा\lem \msCa\msNa\msNc,\hskip.2em plus .9em सत्त्वं तमन्तथा \msCb,\hskip.5em plus .9em 
सत्वस्तमस्तथा \msCc\msNb,\hskip.5em plus .9em सत्त्वतमस्तथा \Ed}}% 

%Verse 9:4

{\devanagarifont तमः सत्त्वं रजश्चैव अन्योन्यमिथुनाः स्मृताः {॥ ९:\hspace{.11em}४॥} \veg\dontdisplaylinenum }%
     \var{{\devanagarifontvar\numnoemph\vc तमः सत्त्वं\lem \msCa\msCb\msNa\msNc,\hskip.2em plus .9em तमसत्व॰ \msCc,\hskip.5em plus .9em तमः सत्व॰ \msNb\Ed\oo 
 रजश्चैव\lem \mssALL,\hskip.2em plus .9em रजःश्चैव \msCb}}% 
    \var{{\devanagarifontvar\numnoemph\vd स्मृताः\lem \mssALL,\hskip.2em plus .9em \om\ \msCc}}% 
    \paral{{\devanagarifontsmall \vd {\englishfont \similar\ \BRAHMANDAPUR\ 1.4.11ab:}
                         अन्योन्यं मिथुनं ह्येते अन्योन्यमुपजीविनः
                     {\englishfont \similar\ \VAYUP\ 1.5.17cd \similar\ \LINPU\ 1.70.80ab} }}

{\devanagarifont सात्त्विको भगवान्विष्णू राजसः कमलोद्भवः \thinspace{\dandab} \dontdisplaylinenum }%
     \var{{\devanagarifontvar\numemph\va ॰ष्णू\lem \corr,\hskip.2em plus .9em ॰ष्णु \mssCaCbCc\msNa\msNb\msNc\Ed}}% 
    \var{{\devanagarifontvar\numnoemph\vb राजसः कमलोद्भवः\lem \mssALL,\hskip.2em plus .9em 
\uncl{राज}\lk\lk\lk\lk\lk\lk\ \msCa}}% 
    \paral{{\devanagarifontsmall \vo {\englishfont \compare\ \BRAHMANDAPUR\ 1.4.6cd:}
                 सत्त्वं विष्णू रजो ब्रह्मा तमो रुद्रः प्रजापतिः }}

%Verse 9:5

{\devanagarifont तामसो भगवानीशः सकलंविकलेश्वरः {॥ ९:\hspace{.11em}५॥} \veg\dontdisplaylinenum }%
     \var{{\devanagarifontvar\numnoemph\vcd तामसो भगवानीशः सकलं\lem \mssALL,\hskip.2em plus .9em 
\lk\lk \lk\lk \lk\lk \lk\lk \uncl{सकलम्} \msCa}}% 

{\devanagarifont सत्त्वं कुन्देन्दुवर्णाभं पद्मरागनिभं रजः \thinspace{\dandab} \dontdisplaylinenum }%
     \var{{\devanagarifontvar\numemph\va सत्त्वं\lem \mssALL,\hskip.2em plus .9em सत्व \msCc\msNc\oo 
 ॰वर्णाभं\lem \mssALL,\hskip.2em plus .9em ॰वर्ण्णाभ \msCc,\hskip.5em plus .9em ॰वण्णाभं \msNa}}% 

%Verse 9:6

{\devanagarifont तमश्चाञ्जनशैलाभं कीर्तितानि मनीषिभिः {॥ ९:\hspace{.11em}६॥} \veg\dontdisplaylinenum }%
     \var{{\devanagarifontvar\numnoemph\vc ॰भं\lem \mssALL,\hskip.2em plus .9em ॰भा \Ed}}% 

{\devanagarifont सत्त्वं जलं रजो ऽङ्गारं तमो धूमसमाकुलम् \thinspace{\dandab} \dontdisplaylinenum }%
     \var{{\devanagarifontvar\numemph\va जलं\lem \mssALL,\hskip.2em plus .9em रजं \msCc,\hskip.5em plus .9em ज्वाल \msNb\oo 
 रजो ऽङ्गारं\lem \mssALL,\hskip.2em plus .9em 
र\uncl{ङ्गो}ङ्गारन् \msCc,\hskip.5em plus .9em रजोङ्गरन् \Ed}}% 

%Verse 9:7

{\devanagarifont एतद्गुणमयैर्बद्धाः पच्यन्ते सर्वदेहिनः {॥ ९:\hspace{.11em}७॥} \veg\dontdisplaylinenum }%
     \var{{\devanagarifontvar\numnoemph\vd ॰देहिनः\lem \mssALL,\hskip.2em plus .9em ॰देहिना \msCb}}% 

{\devanagarifont विगतराग उवाच {\dandab}\dontdisplaylinenum  }%
 
{\devanagarifont केन केन प्रकारेण गुणपाशेन बध्यते \thinspace{\danda} \dontdisplaylinenum }%
     \var{{\devanagarifontvar\numemph\vb गुण॰\lem \mssALL,\hskip.2em plus .9em \om\ \msCa}}% 

%Verse 9:8

{\devanagarifont चिह्नमेषां पृथक्त्वेन कथयस्व तपोधन {॥ ९:\hspace{.11em}८॥} \veg\dontdisplaylinenum }%
     \var{{\devanagarifontvar\numnoemph\vc ॰षां पृथक्त्वेन\lem \mssALL,\hskip.2em plus .9em ॰षा पृथकेन \msNc}}% 

{\devanagarifont अनर्थयज्ञ उवाच {\dandab}\dontdisplaylinenum  }%
 
{\devanagarifont अनेकाकारभावेन बध्यन्ते गुणबन्धनैः \thinspace{\danda} \dontdisplaylinenum }%
 
%Verse 9:9

{\devanagarifont मोहिता नाभिजानन्ति जानन्ति शिवयोगिनः {॥ ९:\hspace{.11em}९॥} \veg\dontdisplaylinenum }%
     \var{{\devanagarifontvar\numemph\vc ॰भिजानन्ति\lem \mssALL,\hskip.2em plus .9em ॰भिजानान्ति \msCc}}% 
    \var{{\devanagarifontvar\numnoemph\vd जानन्ति\lem \mssALL,\hskip.2em plus .9em \om\ \msCbacorr}}% 

{\devanagarifont ऊर्ध्वंगो नित्यसत्त्वस्थो मध्यगो रजसावृतः \thinspace{\dandab} \dontdisplaylinenum }%
     \var{{\devanagarifontvar\numemph\va ऊर्ध्वंगो नित्य\lem \conj,\hskip.2em plus .9em 
ऊर्ध्वाङ्गो नित्य॰ \mssCaCbCc\msNapcorr\Ed,\hskip.5em plus .9em 
ऊर्ध्वाङ्गा नत्य॰ \msNaacorr,\hskip.5em plus .9em 
ऊर्ध्वगो सित्य॰ \msNbacorr,\hskip.5em plus .9em 
ऊर्ध्वगो सत्य॰ \msNbpcorr,\hskip.5em plus .9em 
उर्ध्वाङ्गो नित्य॰ \msNc\oo 
 ॰सत्त्व॰\lem \msCa\msCb\msNa\msNc,\hskip.2em plus .9em ॰सत्य॰ \msCc\Ed,\hskip.5em plus .9em ॰नित्य॰ \msNb}}% 
    \var{{\devanagarifontvar\numnoemph\vb मध्यगो\lem \mssALL,\hskip.2em plus .9em मध्यमो \Ed\oo 
 ॰वृतः\lem \mssALL,\hskip.2em plus .9em ॰वृतम् \Ed}}% 

%Verse 9:10

{\devanagarifont अधोगतिस्तमोऽवस्था भवन्ति पुरुषाधमाः {॥ ९:\hspace{.11em}१०॥} \veg\dontdisplaylinenum }%
     \var{{\devanagarifontvar\numnoemph\vc ॰गतिस्तमो॰\lem \mssALL,\hskip.2em plus .9em ॰गतितमो॰ \msCb\msCc}}% 

{\devanagarifont स्वर्गे ऽपि हि त्रयो वैते भावनीयास्तपोधन \thinspace{\dandab} \dontdisplaylinenum }%
 
%Verse 9:11

{\devanagarifont मानुषेषु च तिर्येषु गुणभेदास्त्रयस्त्रयः {॥ ९:\hspace{.11em}११॥} \veg\dontdisplaylinenum }%
     \var{{\devanagarifontvar\numemph\vc मानुषेषु\lem \mssALL,\hskip.2em plus .9em मनुष्येषु \msCb,\hskip.5em plus .9em मानुष्येषु \msNc\oo 
 तिर्येषु\lem \mssALL,\hskip.2em plus .9em तीर्येषु \Ed}}% 
    \var{{\devanagarifontvar\numnoemph\vd ॰स्त्रयः\lem \mssALL,\hskip.2em plus .9em ॰स्त्रः \msCbacorr}}% 


\alalalfejezet{सात्त्विकोत्तमाः}

{\devanagarifont ब्रह्मा विष्णुश्च रुद्रश्च धर्म इन्द्रः प्रजापतिः \thinspace{\dandab} \dontdisplaylinenum }%
     \var{{\devanagarifontvar\numemph\vb धर्म इन्द्रः\lem \mssALL,\hskip.2em plus .9em इर्म इन्द्र \msCb,\hskip.5em plus .9em धर्मरिन्द्र॰ \Ed}}% 

%Verse 9:12

{\devanagarifont सोमो ऽग्निर्वरुणः सूर्यो दश सत्त्वोत्तमाः स्मृताः {॥ ९:\hspace{.11em}१२॥} \veg\dontdisplaylinenum }%
     \var{{\devanagarifontvar\numnoemph\vc ग्निर्वरुणः\lem \msCa\msNa\msNc,\hskip.2em plus .9em ग्नि वरुण \msCb\msCc\msNb\Ed}}% 
    \var{{\devanagarifontvar\numnoemph\vd दश\lem \mssALL,\hskip.2em plus .9em दशः \Ed\oo 
 सत्त्वोत्तमाः\lem \mssALL,\hskip.2em plus .9em सत्वत्तमाः \msCb,\hskip.5em plus .9em सत्तोतमाः \msNc}}% 


\alalalfejezet{सात्त्विकमध्यमाः}

{\devanagarifont रुद्रादित्या वसुसाध्या विश्वेशमरुतो ध्रुवः \thinspace{\dandab} \dontdisplaylinenum }%
     \var{{\devanagarifontvar\numemph\vab ॰दित्या वसुसाध्या\lem \msCb\msNa\msNb\msNc,\hskip.2em plus .9em ॰दित्या वसुसा\lk\ \msCa,\hskip.5em plus .9em ॰दित्य वसुसाध्या \msCc,\hskip.5em plus .9em 
॰दित्य वसुसाध्याः वि॰ \Ed}}% 
    \var{{\devanagarifontvar\numnoemph\vb विश्वेश॰\lem \mssALL,\hskip.2em plus .9em \lk श्वेश \msCa,\hskip.5em plus .9em विश्वेशि॰ \msCc}}% 

%Verse 9:13

{\devanagarifont ऋषयः पितरश्चैव दशैते सत्त्वमध्यमाः {॥ ९:\hspace{.11em}१३॥} \veg\dontdisplaylinenum }%
     \var{{\devanagarifontvar\numnoemph\vd दशैते\lem \mssALL,\hskip.2em plus .9em दशैतेते \msCbacorr}}% 


\alalalfejezet{सात्त्विकाधमाः}

{\devanagarifont तारा ग्रहाः सुरा यक्षा गन्धर्वाः किंनरोरगाः \thinspace{\dandab} \dontdisplaylinenum }%
     \var{{\devanagarifontvar\numemph\va ग्रहाः सुरा\lem \mssALL,\hskip.2em plus .9em ग्रहास्वराः \msCc,\hskip.5em plus .9em ग्रहाऽसुरा \Ed}}% 
    \var{{\devanagarifontvar\numnoemph\vb गन्धर्वाः\lem \msCa\msNb\msNc\Ed,\hskip.2em plus .9em गन्धर्वा \msCb\msNa,\hskip.5em plus .9em गन्धर्व्वाः गन्धर्व्वा \msCc}}% 

%Verse 9:14

{\devanagarifont रक्षोभूतपिशाचाश्च दशैते सात्त्विकाधमाः {॥ ९:\hspace{.11em}१४॥} \veg\dontdisplaylinenum }%
     \var{{\devanagarifontvar\numnoemph\vc ॰पिशाचाश्च\lem \mssALL,\hskip.2em plus .9em ॰पिशाश्चाश्च \msNc}}% 
    \var{{\devanagarifontvar\numnoemph\vd दशैते\lem \mssALL,\hskip.2em plus .9em दशेते \msCb\oo 
 सात्त्विका॰\lem \mssALL,\hskip.2em plus .9em सत्वका॰ \msCb}}% 

\pend
\endnumbering
\vfill\pagebreak\beginnumbering\pstart
\vers


\alalalfejezet{राजसोत्तमाः}

{\devanagarifont ऋत्विक्पुरोहिताचार्ययज्वानो ऽतिथि विज्ञनी \thinspace{\dandab} \dontdisplaylinenum }%
     \var{{\devanagarifontvar\numemph\vb ॰विज्ञनी\lem \mssALL,\hskip.2em plus .9em ॰विज्ञकौ \Ed}}% 

%Verse 9:15

{\devanagarifont राजा मन्त्री व्रती वेदी दशैते राजसोत्तमाः {॥ ९:\hspace{.11em}१५॥} \veg\dontdisplaylinenum }%
     \var{{\devanagarifontvar\numnoemph\vc राजा\lem \eme,\hskip.2em plus .9em राज॰ \mssCaCbCc\msNa\msNb\msNc\Ed\oo 
 ॰मन्त्री व्रती\lem \mssALL,\hskip.2em plus .9em ॰मन्त्रि व्रतो \Ed}}% 
    \var{{\devanagarifontvar\numnoemph\vd राजसो॰\lem \mssALL,\hskip.2em plus .9em रामसो \msCb}}% 


\alalalfejezet{राजसमध्यमाः}

{\devanagarifont सूतो ऽम्बष्ठवणिश्चोग्रः शिल्पिकारुकमागधाः \thinspace{\dandab} \dontdisplaylinenum }%
     \var{{\devanagarifontvar\numemph\va सूतो ऽम्बष्ठ॰\lem \corr,\hskip.2em plus .9em सूतो \lk ष्ट॰ \msCa,\hskip.5em plus .9em सूत\uncl{म्बष्ट}॰ \msCb,\hskip.5em plus .9em 
सूतोन्वष्ठ॰ \msCc,\hskip.5em plus .9em 
सूतोत्वष्टा॰ \msNa,\hskip.5em plus .9em सूतोत्वष्ट॰ \msNb\msNc,\hskip.5em plus .9em 
सूतो ऽम्बष्ट॰ \Ed\oo 
 ॰वणिश्चो॰\lem \mssALL,\hskip.2em plus .9em ॰वणिश्वो॰ \Ed}}% 
    \var{{\devanagarifontvar\numnoemph\vb शिल्पि॰\lem \msNb,\hskip.2em plus .9em शिल्प॰ \mssCaCbCc\msNa\msNc\Ed\oo 
 मागधाः\lem \mssALL,\hskip.2em plus .9em मागधा \msCc}}% 

%Verse 9:16

{\devanagarifont वेणवैदेहकामात्या दशैते रजमध्यमाः {॥ ९:\hspace{.11em}१६॥} \veg\dontdisplaylinenum }%
     \var{{\devanagarifontvar\numnoemph\vc वेणवैदेहकामात्या\lem \msCa\msCc\msNa\msNb,\hskip.2em plus .9em वैणवेदेहकामात्या \msCb,\hskip.5em plus .9em 
वेनवैदेहकामात्या \msNc,\hskip.5em plus .9em वेणवैदेचकौ मात्या \Ed}}% 


\alalalfejezet{राजसाधमाः}

{\devanagarifont चर्मकृत्कुम्भकृत्कोली लोहकृत्त्रपुनीलिकाः \thinspace{\dandab} \dontdisplaylinenum }%
     \var{{\devanagarifontvar\numemph\va ॰कृत्कोली\lem \mssALL,\hskip.2em plus .9em ॰ककोली \msNa,\hskip.5em plus .9em ॰कृत्काली \Ed}}% 
    \var{{\devanagarifontvar\numnoemph\vb ॰नीलिकाः\lem \mssALL,\hskip.2em plus .9em ॰तीलिका \Ed}}% 

%Verse 9:17

{\devanagarifont नटमुष्टिकचण्डाला दशैते रजसाधमाः {॥ ९:\hspace{.11em}१७॥} \veg\dontdisplaylinenum }%
     \var{{\devanagarifontvar\numnoemph\vc ॰मुष्टिक॰\lem \mssALL,\hskip.2em plus .9em ॰मौष्टिक॰ \msCc\oo 
 ॰चण्डाला\lem \mssALL,\hskip.2em plus .9em ॰चाण्डालः \Ed}}% 
    \var{{\devanagarifontvar\numnoemph\vd दशैते\lem \mssALL,\hskip.2em plus .9em दशेते \msCb}}% 
    \paral{{\devanagarifontsmall \vc {\englishfont = \UMS\ 2.10a, 2.20a = \UUMS\ 2.31c} }}


\alalalfejezet{तामसोत्तमाः}

{\devanagarifont गोगजगवया अश्वमृगचामरकिंनराः \thinspace{\dandab} \dontdisplaylinenum }%
     \var{{\devanagarifontvar\numemph\va ॰गवया\lem \mssALL,\hskip.2em plus .9em ॰गवय \msNb,\hskip.5em plus .9em ॰गवयो \Ed}}% 
    \var{{\devanagarifontvar\numnoemph\vb ॰चामर॰\lem \msCa\msCb\msNa\msNc,\hskip.2em plus .9em ॰वानर॰ \msCc\Ed,\hskip.5em plus .9em ॰\uncl{वा}नर॰ \msNb}}% 

%Verse 9:18

{\devanagarifont सिंहव्याघ्रवराहाश्च दशैते तामसोत्तमाः {॥ ९:\hspace{.11em}१८॥} \veg\dontdisplaylinenum }%
     \var{{\devanagarifontvar\numnoemph\vc ॰वराहा॰\lem \mssALL,\hskip.2em plus .9em ॰वराह॰ \msNb\Ed}}% 
    \var{{\devanagarifontvar\numnoemph\vd तामसोत्तमाः\lem \mssALL,\hskip.2em plus .9em तामशोत्तमः \msCb,\hskip.5em plus .9em 
तमसोत्तमाः \Ed}}% 


\alalalfejezet{तामसमध्यमाः}

{\devanagarifont अजमेषमहिष्याश्च मूषिकानकुलादयः \thinspace{\dandab} \dontdisplaylinenum }%
     \var{{\devanagarifontvar\numemph\va ॰महिष्याश्च\lem \mssALL,\hskip.2em plus .9em ॰महिंष्या च \msNb}}% 

%Verse 9:19

{\devanagarifont उष्ट्ररङ्कुशशगण्डा दशैते तममध्यमाः {॥ ९:\hspace{.11em}१९॥} \veg\dontdisplaylinenum }%
     \var{{\devanagarifontvar\numnoemph\vc उष्ट्र॰\lem \mssALL,\hskip.2em plus .9em उष्ट॰ \msCc,\hskip.5em plus .9em दंष्ट्रि॰ \Ed\oo 
 ॰शशगण्डा\lem \mssALL,\hskip.2em plus .9em ॰शगण्डाश्च \Ed}}% 
    \var{{\devanagarifontvar\numnoemph\vd तममध्यमाः\lem \mssALL,\hskip.2em plus .9em तमध्यमाः \msCa}}% 


\alalalfejezet{तामसाधमाः}

{\devanagarifont ऋक्षगोधामृगशृङ्गिबकवानरगर्दभाः \thinspace{\dandab} \dontdisplaylinenum }%
     \var{{\devanagarifontvar\numemph\vb ॰गर्दभाः\lem \mssALL,\hskip.2em plus .9em ॰गर्दभः \Ed}}% 

%Verse 9:20

{\devanagarifont सूकरश्वानगोमायुर्दशैते तामसाधमाः {॥ ९:\hspace{.11em}२०॥} \veg\dontdisplaylinenum }%
     \var{{\devanagarifontvar\numnoemph\vc सूकर॰\lem \mssALL,\hskip.2em plus .9em सुखर॰ \msCb}}% 
    \var{{\devanagarifontvar\numnoemph\vcd ॰गोमायुर्द॰\lem \mssALL,\hskip.2em plus .9em ॰गोमायु द॰ \msNa\msNb}}% 
    \var{{\devanagarifontvar\numnoemph\vd ॰शैते\lem \mssALL,\hskip.2em plus .9em ॰शेते \msCb\oo 
 तामसा॰\lem \msCb,\hskip.2em plus .9em तमसा॰ \msCa\msCc\msNa\msNb\msNc\Ed}}% 


\alalalfejezet{तमसात्त्विकाः}

{\devanagarifont क्रौञ्चहंसशुकश्येनभासबारुण्डसारसाः \thinspace{\dandab} \dontdisplaylinenum }%
     \var{{\devanagarifontvar\numemph\va क्रौञ्च॰\lem \Ed,\hskip.2em plus .9em क्रोञ्च॰ \mssCaCbCc\msNa\msNb\msNc}}% 
    \var{{\devanagarifontvar\numnoemph\vb ॰सारसाः\lem \mssALL,\hskip.2em plus .9em ॰सारसा \msNc}}% 

%Verse 9:21

{\devanagarifont चक्राह्वशुकमायूरा दशैते तमसात्त्विकाः {॥ ९:\hspace{.11em}२१॥} \veg\dontdisplaylinenum }%
     \var{{\devanagarifontvar\numnoemph\vc ॰ह्वशुकमायूरा\lem \mssALL,\hskip.2em plus .9em 
॰\uncl{ङ्ग}\lk\lk \lk यूरा \msCa,\hskip.5em plus .9em ॰ङ्गशुकमायूरा \Ed}}% 
    \var{{\devanagarifontvar\numnoemph\vd दशैते\lem \mssALL,\hskip.2em plus .9em दशेते \msCb\oo 
 तमसात्त्विकाः\lem \msCc\msNc\Ed,\hskip.2em plus .9em तमस्सात्त्विकाः \msCa\msNb\ \unmetr,\hskip.5em plus .9em 
नमः सात्विकाः \msCb\ \unmetr,\hskip.5em plus .9em 
तमःसात्विकाः \msNa\ \unmetr}}% 


\alalalfejezet{तमराजसाः}

{\devanagarifont बलाकाः कुक्कुटाः काकाश्चिल्ललावकतित्तिराः \thinspace{\dandab} \dontdisplaylinenum }%
     \var{{\devanagarifontvar\numemph\va बलाकाः\lem \corr,\hskip.2em plus .9em वलाका \msCa\msNa\msNc,\hskip.5em plus .9em वलाक॰ \msCb\msCc\msNb\Ed}}% 
    \var{{\devanagarifontvar\numnoemph\vab कुक्कुटाः काकाश्चि॰\lem \corr,\hskip.2em plus .9em कुक्कुटकाकाश्चि॰ \msCa\msCb\ \unmetr,\hskip.5em plus .9em 
कुर्कुटा काकाश्चि॰ \msCc\msNc,\hskip.5em plus .9em 
कुर्कुटकाकाश्चि \msNa\msNb,\hskip.5em plus .9em कुक्कुटो काका चि॰ \Ed}}% 
    \var{{\devanagarifontvar\numnoemph\vb ॰तित्तिराः\lem \mssALL,\hskip.2em plus .9em ॰तित्तराः \msNc,\hskip.5em plus .9em ॰तित्तिरिः \Ed}}% 

%Verse 9:22

{\devanagarifont गृध्रकङ्कबकश्येन दशैते तमराजसाः {॥ ९:\hspace{.11em}२२॥} \veg\dontdisplaylinenum }%
     \var{{\devanagarifontvar\numnoemph\vc गृध्र॰\lem \mssALL,\hskip.2em plus .9em गृध॰ \msNc}}% 


\alalalfejezet{तामसाधमादि}

{\devanagarifont कोकिलोलूककञ्जल्यकपोताः पञ्च एव च \thinspace{\dandab} \dontdisplaylinenum }%
     \var{{\devanagarifontvar\numemph\va कोकिलो॰\lem \mssALL,\hskip.2em plus .9em कौकिलो॰ \msCb\oo 
 ॰कञ्जल्य॰\lem \eme,\hskip.2em plus .9em ॰किञ्जल्य॰ \msCa\msCc\msNa,\hskip.5em plus .9em ॰किञ्जल्क॰ \msCb\msNb\msNc\Ed}}% 
    \var{{\devanagarifontvar\numnoemph\vb च\lem \mssALL,\hskip.2em plus .9em चः \msNc}}% 

%Verse 9:23

{\devanagarifont शारिकाश्च कुलिङ्गाश्च दशैते तमसाधमाः {॥ ९:\hspace{.11em}२३॥} \veg\dontdisplaylinenum }%
     \var{{\devanagarifontvar\numnoemph\vc शारिकाश्च\lem \corr,\hskip.2em plus .9em शारिका च \mssCaCbCc\msNa\msNb\msNc,\hskip.5em plus .9em शालिका च \Ed\oo 
 कुलिङ्गाश्च\lem \corr,\hskip.2em plus .9em कुलिङ्गा च \msCa\msNb\Ed,\hskip.5em plus .9em कुलिङ्का च \msCb\msCc\msNc,\hskip.5em plus .9em 
कुलिकां च \msNa}}% 

\pend
\endnumbering
\vfill\pagebreak\beginnumbering\pstart
\vers

{\devanagarifont मकरगोहनक्राश्च ऋक्षाश्च तमसात्त्विकाः \thinspace{\dandab} \dontdisplaylinenum }%
     \var{{\devanagarifontvar\numemph\va ॰गोहनक्राश्च\lem \mssALL,\hskip.2em plus .9em 
॰गोहनक्रा च \msCc,\hskip.5em plus .9em ॰ग्रोहनक्राश्च \msNb}}% 
    \var{{\devanagarifontvar\numnoemph\vb ऋक्षाश्च\lem \conj,\hskip.2em plus .9em ऋषा च \mssCaCbCc\msNa\msNb\msNc\Ed\oo 
 तमसात्त्विकाः\lem \Ed,\hskip.2em plus .9em तम\uncl{स्सा}\lk\lk\ \msCa,\hskip.5em plus .9em 
तमःसात्विकाः \msCb\msCc\msNa\msNb\ \unmetr,\hskip.5em plus .9em तसमात्विकाः \msNc}}% 

{\devanagarifont कच्छपशिशुकुम्भीरमण्डूकास्तमराजसाः  \danda\dontdisplaylinenum }%
     \var{{\devanagarifontvar\numnoemph\vc ॰शिशु॰\lem \eme,\hskip.2em plus .9em ॰शुशु॰ \mssCaCbCc\msNa\msNb\msNc\Ed\oo 
 ॰कुम्भीर॰\lem \mssALL,\hskip.2em plus .9em ॰कम्भीरा \msCc\Ed}}% 
    \var{{\devanagarifontvar\numnoemph\vd ॰मण्डूका॰\lem \mssALL,\hskip.2em plus .9em ॰मण्डूक॰ \msNb,\hskip.5em plus .9em ॰मण्डुका॰ \Ed}}% 

%Verse 9:24

{\devanagarifont शङ्खशुक्तिकशम्बूकाः कवय्यस्तमतामसाः {॥ ९:\hspace{.11em}२४॥} \veg\dontdisplaylinenum }%
     \var{{\devanagarifontvar\numnoemph\ve शम्बूकाः\lem \corr,\hskip.2em plus .9em ॰शम्बूका \mssCaCbCc\msNa\msNb\Ed,\hskip.5em plus .9em ॰\uncl{स}म्बूकाः \msNc}}% 
    \var{{\devanagarifontvar\numnoemph\vf ॰कवय्य॰\lem \conj,\hskip.2em plus .9em ॰कबन्ध्या॰ \mssCaCbCc\msNa\msNbpcorr\msNc\Ed,\hskip.5em plus .9em ॰कबन॰ \msNbacorr\oo 
 ॰मतामसाः\lem \msCb\Ed,\hskip.2em plus .9em ॰मस्तामसाः \msCa\msCc\msNc\ \unmetr,\hskip.5em plus .9em ॰मःतामसाः \msNa\msNb\ \unmetr}}% 

{\devanagarifont चन्दनागरुपद्मं च प्लक्षोदुम्बरपिप्पलाः \thinspace{\dandab} \dontdisplaylinenum }%
     \var{{\devanagarifontvar\numemph\va ॰गरु॰\lem \mssALL,\hskip.2em plus .9em ॰गुरु॰ \Ed}}% 

%Verse 9:25

{\devanagarifont वटदारुशमीबिल्वा दशैते तमसात्त्विकाः {॥ ९:\hspace{.11em}२५॥} \veg\dontdisplaylinenum }%
     \var{{\devanagarifontvar\numnoemph\vc ॰बिल्वा\lem \msCa\msCb\msNa\Ed,\hskip.2em plus .9em ॰बिल्व \msCc\msNb\msNc}}% 
    \var{{\devanagarifontvar\numnoemph\vd दशैते\lem \mssALL,\hskip.2em plus .9em दशै \msCc\oo 
 तमसात्त्विकाः\lem \Ed,\hskip.2em plus .9em तमस्सात्विकाः \msCa\ \unmetr,\hskip.5em plus .9em 
तमःसात्विकाः \msCb\msCc\msNa\msNb\msNc\ \unmetr}}% 

{\devanagarifont जाम्बीरलकुचाम्रातदाडिमाकोलवेतसाः \thinspace{\dandab} \dontdisplaylinenum }%
     \var{{\devanagarifontvar\numemph\va जाम्बीर॰\lem \mssALL,\hskip.2em plus .9em जम्बीर॰ \msCc}}% 
    \var{{\devanagarifontvar\numnoemph\vb ॰दाडिमा॰\lem \mssALL,\hskip.2em plus .9em ॰द्राडिमा॰ \msCc,\hskip.5em plus .9em 
॰द्राडि\uncl{हा}॰ \msNa}}% 

%Verse 9:26

{\devanagarifont निम्बनीपो †ध्रवावश्च† दशैते तमराजसाः {॥ ९:\hspace{.11em}२६॥} \veg\dontdisplaylinenum }%
     \var{{\devanagarifontvar\numnoemph\vc ॰नीपो\lem \mssALL,\hskip.2em plus .9em ॰नीपौ \msNc\oo 
 ध्रवावश्च\lem \mssALL,\hskip.2em plus .9em 
धवावश्च \msCapcorr,\hskip.5em plus .9em धुवावश्च \Ed}}% 
    \var{{\devanagarifontvar\numnoemph\vd दशैते\lem \mssALL,\hskip.2em plus .9em \lk\lk\lk\ \msCa}}% 

{\devanagarifont वृक्षवल्लीलतावेणुत्वक्सारतृणभूरुहाः \thinspace{\dandab} \dontdisplaylinenum }%
     \var{{\devanagarifontvar\numemph\va वृक्षवल्ली॰\lem \mssALL,\hskip.2em plus .9em \uncl{वृक्षवल्ली} \msNb}}% 
    \var{{\devanagarifontvar\numnoemph\vb ॰त्वक्सारतृण॰\lem \msCa\msCb\msNa\msNb,\hskip.2em plus .9em 
॰त्वक्सारस्तृण॰ \msCc\Ed,\hskip.5em plus .9em ॰त्वकसारतृण॰ \msNc\ \unmetr}}% 

%Verse 9:27

{\devanagarifont मीरजाश्च शिलाशस्या दशैते तमसात्त्विकाः {॥ ९:\hspace{.11em}२७॥} \veg\dontdisplaylinenum }%
     \var{{\devanagarifontvar\numnoemph\vc मीरजाश्च\lem \corr,\hskip.2em plus .9em मीरजा च \msCa\msCc\msNa\msNb\msNc\Ed,\hskip.5em plus .9em मीनजा च \msCb}}% 
    \var{{\devanagarifontvar\numnoemph\vd तमसात्त्विकाः\lem \msNc\Ed,\hskip.2em plus .9em तमस्सात्विकाः \msCa,\hskip.5em plus .9em 
तमःसात्विकाः \msCb\msCc\msNa\ \unmetr,\hskip.5em plus .9em तमःसाधिकाः \msNb\ \unmetr}}% 

{\devanagarifont भ्रमरालि पतङ्गाश्च क्रिमिकीटजलौकसः \thinspace{\dandab} \dontdisplaylinenum }%
     \var{{\devanagarifontvar\numemph\va ॰आलि\lem \eme,\hskip.2em plus .9em \mssCaCbCc\msNa\msNb\msNc\Ed\oo 
 पतङ्गाश्च\lem \mssALL,\hskip.2em plus .9em पतङ्गानां \Ed}}% 
    \var{{\devanagarifontvar\numnoemph\vb क्रिमिकीटजलौकसः\lem \mssCaCbCc\msNa,\hskip.2em plus .9em क्रिमिकीटजलोकसः \msNb,\hskip.5em plus .9em 
क्रिमिकीटजलौक\uncl{साः} \msNc,\hskip.5em plus .9em किमिकीटजलौकसां \Ed}}% 

%Verse 9:28

{\devanagarifont यूकोद्दंशमशानां च विष्ठाजास्तमसात्त्विकाः {॥ ९:\hspace{.11em}२८॥} \veg\dontdisplaylinenum }%
     \var{{\devanagarifontvar\numnoemph\vc यूकोद्दंशमशानां च\lem \msCa,\hskip.2em plus .9em 
यूकोदंशमशानाञ्च \msCb\msNa,\hskip.5em plus .9em 
यूकोदंशमसकानाञ्च \msCc\ \unmetr,\hskip.5em plus .9em 
यूकोदंशमसानान्तु \msNb,\hskip.5em plus .9em 
\uncl{यूकोद्दं}\lk\lk \lk\lk \lk\  \msNc,\hskip.5em plus .9em 
युक्तोदंशमशानाश्च \Ed}}% 
    \var{{\devanagarifontvar\numnoemph\vd विष्ठाजास्तमसात्त्विकाः\lem \corr,\hskip.2em plus .9em 
विष्टजास्तमस्सात्विकाः \msCa\ \unmetr,\hskip.5em plus .9em 
विष्टजास्तमःसात्विकाः \msCb\msCc\msNa\ \unmetr,\hskip.5em plus .9em 
विष्टजास्तमःसाधिकाः \msNb\ \unmetr,\hskip.5em plus .9em 
\lk\lk \uncl{जा}तमस्साधिकाः \msNc\ \unmetr,\hskip.5em plus .9em 
विष्टजा तमसात्त्विकाः \Ed}}% 

{\devanagarifont दया सत्यं दमः शौचं ज्ञानं मौनं तपः क्षमा \thinspace{\dandab} \dontdisplaylinenum }%
     \var{{\devanagarifontvar\numemph\vb ज्ञानं\lem \msCa\msCc\msNb\Ed,\hskip.2em plus .9em ज्ञान \msCb\msNc,\hskip.5em plus .9em ज्ञा\uncl{नं} \msNa\oo 
 मौनं\lem \mssALL,\hskip.2em plus .9em मौन \msNa\oo 
 क्षमा\lem \mssALL,\hskip.2em plus .9em क्षमाः \msCb\msNb}}% 

%Verse 9:29

{\devanagarifont शीलं च नाभिमानं च सात्त्विकाश्चोत्तमा जनाः {॥ ९:\hspace{.11em}२९॥} \veg\dontdisplaylinenum }%
     \var{{\devanagarifontvar\numnoemph\vc शीलं च\lem \mssALL,\hskip.2em plus .9em नीलञ्च \msNb,\hskip.5em plus .9em शिलं च \Ed\oo 
 नाभिमानं\lem \mssALL,\hskip.2em plus .9em नाभिमानां \Ed}}% 

{\devanagarifont कामतृष्णारतिद्यूतमानो युद्धं मदः स्पृहा \thinspace{\dandab} \dontdisplaylinenum }%
     \var{{\devanagarifontvar\numemph\va ॰मानो\lem \mssALL,\hskip.2em plus .9em ॰मनो \msCc}}% 
    \var{{\devanagarifontvar\numnoemph\vb युद्धं\lem \mssALL,\hskip.2em plus .9em युद्ध॰ \Ed\oo 
 स्पृहा\lem \mssALL,\hskip.2em plus .9em स्मृत \msNb}}% 

%Verse 9:30

{\devanagarifont निर्घृणाः कलिकर्तारो राजसेषूत्तमा जनाः {॥ ९:\hspace{.11em}३०॥} \veg\dontdisplaylinenum }%
     \var{{\devanagarifontvar\numnoemph\vc निर्घृणाः\lem \mssCaCbCc,\hskip.2em plus .9em निर्घृणा \msNa\Ed,\hskip.5em plus .9em निघृणाः \msNb\msNc}}% 
    \var{{\devanagarifontvar\numnoemph\vd राजसेषूत्तमा\lem \mssALL,\hskip.2em plus .9em 
राजसेसूतमा \msCc,\hskip.5em plus .9em राजसे ह्युत्तमो \Ed}}% 

{\devanagarifont हिंसासूयाघृणामूढनिद्रातन्द्रीभयालसाः \thinspace{\dandab} \dontdisplaylinenum }%
     \var{{\devanagarifontvar\numemph\va ॰सूया॰\lem \mssALL,\hskip.2em plus .9em ॰स\uncl{यू}॰ \msNb\oo 
 ॰मूढ॰\lem \mssALL,\hskip.2em plus .9em ॰मूढा॰ \msCb\msNb}}% 
    \var{{\devanagarifontvar\numnoemph\vb ॰तन्द्री॰\lem \mssALL,\hskip.2em plus .9em ॰तन्त्री॰ \Ed}}% 

%Verse 9:31

{\devanagarifont क्रोधो मत्सरमायी च तामसेषूत्तमा जनाः {॥ ९:\hspace{.11em}३१॥} \veg\dontdisplaylinenum }%
     \var{{\devanagarifontvar\numnoemph\vc क्रोधो\lem \mssALL,\hskip.2em plus .9em क्रोध॰ \Ed}}% 
    \var{{\devanagarifontvar\numnoemph\vd तामसेषूत्तमा\lem \mssALL,\hskip.2em plus .9em 
तामसेसूतमा \msCc,\hskip.5em plus .9em तामसे ह्युत्तमो \Ed}}% 

{\devanagarifont लघुप्रीतिप्रकाशी च ध्यानयोगे सदोत्सुकः \thinspace{\dandab} \dontdisplaylinenum }%
     \var{{\devanagarifontvar\numemph\vb ॰योगे\lem \mssALL,\hskip.2em plus .9em ॰\uncl{योगे} \msCa}}% 

%Verse 9:32

{\devanagarifont प्रज्ञाबुद्धिविरागी च सात्त्विकं गुणलक्षणम् {॥ ९:\hspace{.11em}३२॥} \veg\dontdisplaylinenum }%
     \var{{\devanagarifontvar\numnoemph\vc ॰विरागी च\lem \mssALL,\hskip.2em plus .9em ॰विरागी \msNa,\hskip.5em plus .9em ॰विराङ्क्री च \msNc}}% 

{\devanagarifont बालको निपुणो रागी मानो दर्पश्च लोभकः \thinspace{\dandab} \dontdisplaylinenum }%
     \var{{\devanagarifontvar\numemph\va बालको\lem \mssALL,\hskip.2em plus .9em चालको \msNc\oo 
 निपुणो\lem \Ed,\hskip.2em plus .9em निपुनो \mssCaCbCc\msNa\msNb,\hskip.5em plus .9em निपुणे \msNc}}% 

%Verse 9:33

{\devanagarifont स्पृहा ईर्षा प्रलापी च राजसं गुणलक्षणम् {॥ ९:\hspace{.11em}३३॥} \veg\dontdisplaylinenum }%
     \var{{\devanagarifontvar\numnoemph\vc ईर्षा\lem \mssALL,\hskip.2em plus .9em ईर्ष्या \msCb\Ed\oo 
 प्रलापी\lem \mssALL,\hskip.2em plus .9em च लापी \msCc}}% 
    \var{{\devanagarifontvar\numnoemph\vd राजसं\lem \mssALL,\hskip.2em plus .9em तामसं \Ed}}% 

{\devanagarifont उद्वेग आलसो मोहः क्रूरस्तस्करनिर्दयः \thinspace{\dandab} \dontdisplaylinenum }%
     \var{{\devanagarifontvar\numemph\va आलसो\lem \mssALL,\hskip.2em plus .9em अलसो \msCb}}% 
    \var{{\devanagarifontvar\numnoemph\vb क्रूरस्त॰\lem \msCa\msCb\msNa,\hskip.2em plus .9em क्रूरत॰ \msCc\msNc\Ed,\hskip.5em plus .9em कूरस्त॰ \msNb\oo 
 ॰निर्दयः\lem \mssALL,\hskip.2em plus .9em ॰निर्दयाः \msNc}}% 

%Verse 9:34

{\devanagarifont क्रोधः पिशुन निद्रा च तामसं गुणलक्षणम् {॥ ९:\hspace{.11em}३४॥} \veg\dontdisplaylinenum }%
     \var{{\devanagarifontvar\numnoemph\vc क्रोधः\lem \mssALL,\hskip.2em plus .9em क्रोध॰ \msCb\oo 
 पिशुन\lem \Ed,\hskip.2em plus .9em पिशुनो \mssCaCbCc\msNa\msNb\msNc\oo 
 च\lem \mssALL,\hskip.2em plus .9em \om\ \msNb}}% 
    \var{{\devanagarifontvar\numnoemph\vd गुण॰\lem \mssALL,\hskip.2em plus .9em गु॰ \msCbacorr}}% 


\alalalfejezet{आहारस्त्रैगुण्ये}

{\devanagarifont विगतराग उवाच {\dandab}\dontdisplaylinenum  }%
 
{\devanagarifont केन चिह्नेन विज्ञेय आहारः सर्वदेहिनाम् \thinspace{\danda} \dontdisplaylinenum }%
     \var{{\devanagarifontvar\numemph\vab केन चिह्नेन विज्ञेय आहारः सर्वदेहिनाम्\lem \mssALL,\hskip.2em plus .9em 
\lk\lk\lk\lk\lk\lk\lk\lk\lk\lk\lk\lk\lk  देहिनाम् \msCa,\hskip.5em plus .9em 
केन चिह्नेन विज्ञेय आहार सर्वदेहिनाम् \msNb}}% 

%Verse 9:35

{\devanagarifont त्रैगुण्यस्य पृथक्त्वेन कथयस्व तपोधन {॥ ९:\hspace{.11em}३५॥} \veg\dontdisplaylinenum }%
     \var{{\devanagarifontvar\numnoemph\vc पृथक्त्वेन\lem \mssALL,\hskip.2em plus .9em पृथक्केण \msNc}}% 
    \var{{\devanagarifontvar\numnoemph\vd ॰धन\lem \mssALL,\hskip.2em plus .9em ॰धनः \msNc}}% 

{\devanagarifont अनर्थयज्ञ उवाच {\dandab}\dontdisplaylinenum  }%
 
{\devanagarifont आयुः कीर्तिः सुखं प्रीतिर्बलारोग्यविवर्धनम् \thinspace{\danda} \dontdisplaylinenum }%
     \var{{\devanagarifontvar\numemph\va कीर्तिः\lem \mssALL,\hskip.2em plus .9em किर्तिः \Ed\oo 
 सुखं प्रीतिर्ब॰\lem \msNc,\hskip.2em plus .9em सुखं प्रीतिब॰ \msCa\msCb\msNa\msNb,\hskip.5em plus .9em 
सुखप्रीति ब॰ \msCc,\hskip.5em plus .9em सुखं प्रितिव॰ \Ed}}% 
    \var{{\devanagarifontvar\numnoemph\vb ॰रोग्य॰\lem \mssALL,\hskip.2em plus .9em ॰रोग्यं \msCb}}% 

%Verse 9:36

{\devanagarifont हृद्यस्वादुरसं स्निग्ध आहारः सात्त्विकप्रियः {॥ ९:\hspace{.11em}३६॥} \veg\dontdisplaylinenum }%
     \var{{\devanagarifontvar\numnoemph\vc हृद्य॰\lem \mssALL,\hskip.2em plus .9em हृद॰ \Ed\oo 
 ॰रसं\lem \msCa\msCb\msNa,\hskip.2em plus .9em ॰रस \msCc,\hskip.5em plus .9em ॰\uncl{रस} \msNb,\hskip.5em plus .9em ॰रसां \msNc,\hskip.5em plus .9em ॰रसा \Ed\oo 
 स्निग्ध\lem \mssALL,\hskip.2em plus .9em स्निग्धं \msNa,\hskip.5em plus .9em \uncl{सन्दिग्ध} \msNb}}% 
    \var{{\devanagarifontvar\numnoemph\vd आहारः\lem \msCapcorr\msNb\msNc\Ed,\hskip.2em plus .9em आहार \msCaacorr\msCb\msCc\msNa\oo 
 सात्त्विकप्रियः\lem \msCa\msCb\msNa\msNc,\hskip.2em plus .9em 
सात्विकप्रिया \msCc,\hskip.5em plus .9em सात्विकप्रिय \msNb,\hskip.5em plus .9em सात्विकः कियाः \Ed}}% 

{\devanagarifont अत्युष्णमाम्ललवणं रूक्षं तीक्ष्णं विदाहि च \thinspace{\dandab} \dontdisplaylinenum }%
     \var{{\devanagarifontvar\numemph\va ॰म्ल॰\lem \mssALL,\hskip.2em plus .9em ॰ल्ल॰ \Ed\oo 
 ॰लवणं\lem \mssALL,\hskip.2em plus .9em ॰लक्षणं \msCb}}% 
    \var{{\devanagarifontvar\numnoemph\vb तीक्ष्णं\lem \mssALL,\hskip.2em plus .9em ती\uncl{क्ष्ण} \msCa,\hskip.5em plus .9em स्तीक्षं \Ed\oo 
 विदाहि च\lem \msCb\msNa\msNb\msNc,\hskip.2em plus .9em \lk \uncl{दाहि च} \msCa,\hskip.5em plus .9em 
विदाहिक \msCcpcorr,\hskip.5em plus .9em विदाहिकः \msCcacorr\Ed}}% 

%Verse 9:37

{\devanagarifont राजसश्रेष्ठ-आहारो दुःखशोकामयप्रदः {॥ ९:\hspace{.11em}३७॥} \veg\dontdisplaylinenum }%
     \var{{\devanagarifontvar\numnoemph\vcd राजसश्रेष्ठ आहारो दुःखशोकामयप्रदः\lem \msCb\msNa\msNc,\hskip.2em plus .9em 
\lk\lk \lk\lk \lk\lk \lk\lk \lk\lk \lk\lk \lk\lk \lk\lk\  \msCa,\hskip.5em plus .9em 
राजसश्रेष्ठ आहारो दुःखशोकामयः प्रदः \msCc,\hskip.5em plus .9em 
राजसः श्रेष्ठ आहारो दुःखशोकामयप्रदः \msNb,\hskip.5em plus .9em 
राजसे श्रेष्ठमाहारो दुःखशोकाभयप्रदः \Ed}}% 

\pend
\endnumbering
\vfill\pagebreak\beginnumbering\pstart
\vers

{\devanagarifont अभक्ष्यामेध्यपूती च पूति पर्युषितं च यत् \thinspace{\dandab} \dontdisplaylinenum }%
     \var{{\devanagarifontvar\numemph\va अभक्ष्यामेध्यपूती च\lem \eme,\hskip.2em plus .9em अभक्ष्यमेध्यपूती च \mssCaCbCc\msNa,\hskip.5em plus .9em 
अभक्षमेध्यपूती च \msNb,\hskip.5em plus .9em अभक्षामेध्यपूती च \msNc,\hskip.5em plus .9em अभक्षमद्यपूती वै \Ed}}% 

%Verse 9:38

{\devanagarifont आमयारसविस्वाद आहारस्तामसप्रियः {॥ ९:\hspace{.11em}३८॥} \veg\dontdisplaylinenum }%
     \var{{\devanagarifontvar\numnoemph\vc आमया॰\lem \conj,\hskip.2em plus .9em आयाम॰ \mssCaCbCc\msNa\msNb\msNc,\hskip.5em plus .9em आयास॰ \Ed}}% 
    \var{{\devanagarifontvar\numnoemph\vd ॰मस॰\lem \mssALL,\hskip.2em plus .9em ॰मसः \msCc\Ed\oo 
 ॰प्रियः\lem \mssALL,\hskip.2em plus .9em ॰प्रियाः \msCc}}% 


\alalalfejezet{गुणातीतम्}

{\devanagarifont विगतराग उवाच {\dandab}\dontdisplaylinenum  }%
 
{\devanagarifont गुणातीतं कथं ज्ञेयं संसारपरपारगम् \thinspace{\danda} \dontdisplaylinenum }%
     \var{{\devanagarifontvar\numemph\va ॰तीतं\lem \mssALL,\hskip.2em plus .9em ॰तीत \msCc\msNb}}% 
    \var{{\devanagarifontvar\numnoemph\vb ॰गम्\lem \mssALL,\hskip.2em plus .9em ॰गः \msCc}}% 

%Verse 9:39

{\devanagarifont गुणपाशनिबद्धानां मोक्षं कथय तत्त्वतः {॥ ९:\hspace{.11em}३९॥} \veg\dontdisplaylinenum }%
     \var{{\devanagarifontvar\numnoemph\vc ॰बद्धानां\lem \mssALL,\hskip.2em plus .9em ॰वर्द्धानां \msCb,\hskip.5em plus .9em ॰बद्धामो \Ed}}% 

{\devanagarifont अनर्थयज्ञ उवाच {\dandab}\dontdisplaylinenum  }%
 
{\devanagarifont आत्मवत्सर्वभूतानि सम्यक्पश्येत भो द्विज \thinspace{\danda} \dontdisplaylinenum }%
     \var{{\devanagarifontvar\numemph\va ॰भूतानि\lem \mssALL,\hskip.2em plus .9em ॰भूतां \msNa}}% 
    \var{{\devanagarifontvar\numnoemph\vb सम्यक्प॰\lem \mssALL,\hskip.2em plus .9em सम्यत्प॰ \msNa}}% 
    \paral{{\devanagarifontsmall \vab {\englishfont \similar\ \PADMAP\ 1.19.337ab:} 
                         आत्मवत्सर्वभूतानि यः पश्यति स पश्यति }}

%Verse 9:40

{\devanagarifont गुणातीतः स विज्ञेयः संसारपरपारगः {॥ ९:\hspace{.11em}४०॥} \veg\dontdisplaylinenum }%
     \var{{\devanagarifontvar\numnoemph\vc ॰तीतः\lem \msCa\msCb\msNa\msNb,\hskip.2em plus .9em ॰तीत \msCc\msNc,\hskip.5em plus .9em ॰तीतं \Ed}}% 
    \paral{{\devanagarifontsmall \vo {\englishfont \compare\ \BHG\ 6.32:}
                 आत्मौपम्येन सर्वत्र समं पश्यति यो ऽर्जुन\thinspace{\devanagarifontsmall ।}
                 सुखं वा यदि वा दुःखं स योगी परमो मतः\thinspace{\devanagarifontsmall ॥} }}

{\devanagarifont ईर्षाद्वेषसमो यस्तु सुखदुःखसमाश्च ये \thinspace{\dandab} \dontdisplaylinenum }%
     \var{{\devanagarifontvar\numemph\va ईर्षा॰\lem \mssALL,\hskip.2em plus .9em ईर्ष्या॰ \msNc\Ed}}% 
    \var{{\devanagarifontvar\numnoemph\vb ॰समाश्च ये\lem \mssALL,\hskip.2em plus .9em ॰समाश्रये \msNb}}% 
    \paral{{\devanagarifontsmall \vab {\englishfont \compare\ \VSS\ 11.51ab:}
                     न्यसेद्धर्ममधर्मं च ईर्ष्याद्वेषं परित्यजेत्
                     {\englishfont and \BHG\ 14.25:}
                         मानापमानयोस्तुल्यस्तुल्यो मित्रारिपक्षयोः\thinspace{\devanagarifontsmall ।}
                         सर्वारम्भपरित्यागी गुणातीतः स उच्यते\thinspace{\devanagarifontsmall ॥}
                    {\englishfont and also \BHG\ 12.13:}
                 अद्वेष्टा सर्वभूतानां मैत्रः करुण एव च\thinspace{\devanagarifontsmall ।}
                 निर्ममो निरहंकारः समदुःखसुखः क्षमी\thinspace{\devanagarifontsmall ॥} }}

%Verse 9:41

{\devanagarifont स्तुतिनिन्दासमा ये च गुणातीतः स उच्यते {॥ ९:\hspace{.11em}४१॥} \veg\dontdisplaylinenum }%
     \var{{\devanagarifontvar\numnoemph\vd ॰तीतः\lem \mssALL,\hskip.2em plus .9em ॰तीत \msNb}}% 

{\devanagarifont तुल्यप्रियाप्रियो यश्च अरिमित्रसमस्तथा \thinspace{\dandab} \dontdisplaylinenum }%
     \var{{\devanagarifontvar\numemph\va तुल्य॰\lem \Ed,\hskip.2em plus .9em तुल्यः \mssCaCbCc\msNa\msNb\msNc}}% 
    \var{{\devanagarifontvar\numnoemph\vb ॰सम॰\lem \mssALL,\hskip.2em plus .9em ॰समा॰ \msCc}}% 

%Verse 9:42

{\devanagarifont मानापमानयोस्तुल्यो गुणातीतः स उच्यते {॥ ९:\hspace{.11em}४२॥} \veg\dontdisplaylinenum  }%
     \paral{{\devanagarifontsmall \vo {\englishfont \compare\ \BHG\ 14.24cd--25:}
                         तुल्यप्रियाप्रियो धीरस्तुल्यनिन्दात्मसंस्तुतिः\thinspace{\devanagarifontsmall ॥}
                         मानावमानयोस्तुल्यस्तुल्यो मित्रारिपक्षयोः\thinspace{\devanagarifontsmall ।}
                         सर्वारम्भपरित्यागी गुणातीतः स उच्यते\thinspace{\devanagarifontsmall ॥} }}

{\devanagarifont एष ते कथितो विप्र गुणसद्भावनिर्णयः \thinspace{\dandab} \dontdisplaylinenum }%
     \var{{\devanagarifontvar\numemph\va ते\lem \mssALL,\hskip.2em plus .9em तो \msNb}}% 
    \var{{\devanagarifontvar\numnoemph\vb ॰सद्भाव॰\lem \mssALL,\hskip.2em plus .9em ॰मद्भाव॰ \Ed}}% 

%Verse 9:43

{\devanagarifont गुणयुक्तस्तु संसारी गुणातीतः पराङ्गतिः {॥ ९:\hspace{.11em}४३॥} \veg\dontdisplaylinenum }%
     \var{{\devanagarifontvar\numnoemph\vd गुणातीतः\lem \msCa\msCc\msNa,\hskip.2em plus .9em गुणातीत \msCb\msNb\msNc\Ed\oo 
 पराङ्गतिः\lem \Ed,\hskip.2em plus .9em पराङ्गतिम् \mssCaCbCc\msNa\msNb\msNc}}% 

{\devanagarifont 
\jump
\begin{center}
\ketdanda~इति वृषसारसंग्रहे त्रैगुण्यविशेषणीयो नामाध्यायो नवमः~\ketdanda
\end{center}
\dontdisplaylinenum\vers  }%
     \var{{\devanagarifontvar\numnoemph{\englishfont \Colo:} ॰विशेषणीयो\lem \corr,\hskip.2em plus .9em ॰विशेषनीयो \mssCaCbCc\msNa\msNb\msNc\Ed\oo 
 नामाध्यायो नवमः\lem \mssALL,\hskip.2em plus .9em नाम नवमो ऽध्यायः \Ed}}% 
\bekveg\szamveg
\vfill
\phpspagebreak

\versno=0\fejno=10
\thispagestyle{empty}

\centerline{\Large\devanagarifontbold [   दशमो ऽध्यायः  ]}{\vrule depth10pt width0pt} \fancyhead[CE]{{\footnotesize\devanagarifont वृषसारसंग्रहे  }}
\fancyhead[CO]{{\footnotesize\devanagarifont दशमो ऽध्यायः  }}
\fancyhead[LE]{}
\fancyhead[RE]{}
\fancyhead[LO]{}
\fancyhead[RO]{}
\szam\bek



\alalfejezet{कायतीर्थोपवर्णनम्}
\vers


{\devanagarifont विगतराग उवाच {\dandab}\dontdisplaylinenum  }%
 
{\devanagarifont कतमं सर्वतीर्थानां श्रेष्ठमाहुर्मनीषिनः \thinspace{\danda} \dontdisplaylinenum }%
     \var{{\devanagarifontvar\numemph\va कतमं सर्व॰\lem \mssALL,\hskip.2em plus .9em 
कतमसर्व॰ \msNb,\hskip.5em plus .9em कथमन्सर्व॰ \msNc}}% 
    \var{{\devanagarifontvar\numnoemph\vab ॰तीर्थानां श्रेष्ठ॰\lem \mssALL,\hskip.2em plus .9em ॰तीर्था\lk\lk ष्ठ॰ \msCa}}% 
    \var{{\devanagarifontvar\numnoemph\vb मनीषिनः\lem \mssALL,\hskip.2em plus .9em मनीषिभिः \Ed}}% 
    \lacuna{\devanagarifontsmall {\englishfont Witnesses used for this chapter: \msCa\ ff.\thinspace 207r--208v, 
                                              \msCb\ ff.\thinspace 212v--214r, 
                                              \msCc\ ff.\allowbreak\thinspace 283v--285v,
                                              \msNa\ ff.\thinspace 14v--15v, 
                                              \msNb\ exp.\thinspace 55 (lower) -- 56 (lower),
                                              \msNc\ ff.\thinspace 222v--223v,
                                              \Ed\ pp.\thinspace 610--613; 
                                              \mssCaCbCc\ = \msCa + \msCb + \msCc} }%
  
%Verse 10:1

{\devanagarifont कथयस्व मुनिश्रेष्ठ यद्यस्ति भुवि कामदम् {॥ १०:\hspace{.11em}१॥} \veg\dontdisplaylinenum }%
     \var{{\devanagarifontvar\numnoemph\vd भुवि\lem \mssALL,\hskip.2em plus .9em भूरि \Ed\oo 
 ॰दम्\lem \mssALL,\hskip.2em plus .9em ॰दः \msNa}}% 

{\devanagarifont अनर्थयज्ञ उवाच {\dandab}\dontdisplaylinenum  }%
 
{\devanagarifont अतिगुह्यमिदं प्रश्नं पृष्टः स्नेहाद्द्विजोत्तम \thinspace{\danda} \dontdisplaylinenum }%
     \var{{\devanagarifontvar\numemph\vb स्नेहाद्द्वि॰\lem \mssALL,\hskip.2em plus .9em स्नेहा द्वि॰ \msCc}}% 

%Verse 10:2

{\devanagarifont ब्रवीमि वः पुरावृत्तं नन्दिना कथितो ऽस्म्यहम् {॥ १०:\hspace{.11em}२॥} \veg\dontdisplaylinenum }%
     \var{{\devanagarifontvar\numnoemph\vd ऽस्म्यहम्\lem \mssALL,\hskip.2em plus .9em स्मृहम् \msCc}}% 

{\devanagarifont नन्दिकेश्वर उवाच {\dandab}\dontdisplaylinenum  }%
     \var{{\devanagarifontvar\numemph\vo नन्दि॰\lem \mssALL,\hskip.2em plus .9em नन्दी॰ \msCb}}% 

{\devanagarifont कैलासशिखरे रम्ये सिद्धचारणसेविते \thinspace{\danda} \dontdisplaylinenum }%
     \var{{\devanagarifontvar\numnoemph\va कैलास॰\lem \mssALL,\hskip.2em plus .9em कैलाशे \Ed}}% 
    \paral{{\devanagarifontsmall \vab {\englishfont  \compare\ MBh 12.327.18cd:} मेरौ गिरिवरे रम्ये सिद्धचारणसेविते  }}

%Verse 10:3

{\devanagarifont तत्रासीनं शिवं साक्षाद्देवी वचनमब्रवीत् {॥ १०:\hspace{.11em}३॥} \veg\dontdisplaylinenum }%
 
{\devanagarifont देव्युवाच {\dandab}\dontdisplaylinenum  }%
 
{\devanagarifont भगवन्देवदेवेश सर्वभूतजगत्पते \thinspace{\danda} \dontdisplaylinenum }%
     \var{{\devanagarifontvar\numemph\va ॰देवेश\lem \mssALL,\hskip.2em plus .9em ॰देश \msCb}}% 
    \var{{\devanagarifontvar\numnoemph\vb ॰पते\lem \mssALL,\hskip.2em plus .9em ॰पतिम् \msNaacorr}}% 

%Verse 10:4

{\devanagarifont प्रष्टुमिच्छाम्यहं त्वेकं धर्मगुह्यं सनातनम् {॥ १०:\hspace{.11em}४॥} \veg\dontdisplaylinenum }%
     \var{{\devanagarifontvar\numnoemph\vc धर्म॰\lem \mssALL,\hskip.2em plus .9em ध\uncl{र्मं} \msNa}}% 

\pend
\endnumbering
\vfill\pagebreak\beginnumbering\pstart
\vers

{\devanagarifont अतितीर्थं परं गुह्यं संसाराद्येन मुच्यते \thinspace{\dandab} \dontdisplaylinenum }%
     \var{{\devanagarifontvar\numemph\va ॰तीर्थं\lem \mssALL,\hskip.2em plus .9em ॰तीर्थ \msNb\Ed}}% 
    \var{{\devanagarifontvar\numnoemph\vab गुह्यं संसाराद्येन मुच्यते\lem \mssALL,\hskip.2em plus .9em 
\uncl{ग}\lacwithnum{1}  \uncl{सं}साराद्येन मुच्यते \msNb}}% 

%Verse 10:5

{\devanagarifont मनुष्याणां हितार्थाय ब्रूहि तत्त्वं महेश्वर {॥ १०:\hspace{.11em}५॥} \veg\dontdisplaylinenum }%
     \var{{\devanagarifontvar\numnoemph\vd ॰श्वर\lem \mssALL,\hskip.2em plus .9em ॰श्वरः \msCc}}% 

{\devanagarifont महेश्वर उवाच {\dandab}\dontdisplaylinenum  }%
 
{\devanagarifont को मां पृच्छति तं प्रश्नं मुक्त्वा त्वामेव सुन्दरि \thinspace{\danda} \dontdisplaylinenum }%
     \var{{\devanagarifontvar\numemph\va तं प्रश्नं\lem \msNa\msNb,\hskip.2em plus .9em तत्प्रश्न \msCa\msCb,\hskip.5em plus .9em तत्प्रश्नं \msCc\Ed,\hskip.5em plus .9em 
तं प्रश्न \msNc}}% 
    \var{{\devanagarifontvar\numnoemph\vb मुक्त्वा\lem \mssALL,\hskip.2em plus .9em मुक्ता \Ed}}% 

%Verse 10:6

{\devanagarifont शृणु वक्ष्यामि तं प्रश्नं देवैरपि सुदुर्लभम् {॥ १०:\hspace{.11em}६॥} \veg\dontdisplaylinenum }%
     \var{{\devanagarifontvar\numnoemph\vc तं प्रश्नं\lem \msNc,\hskip.2em plus .9em तत्प्रश्नं \mssCaCbCc\msNa\msNb\Ed}}% 

{\devanagarifont कुरुक्षेत्रं प्रयागं च वाराणसीमतः परम् \thinspace{\dandab} \dontdisplaylinenum }%
 
%Verse 10:7

{\devanagarifont गङ्गाग्निं सोमतीर्थं च सूर्यपुष्करमानसम् {॥ १०:\hspace{.11em}७॥} \veg\dontdisplaylinenum }%
     \var{{\devanagarifontvar\numemph\vc गङ्गाग्निं\lem \msCa\msCb,\hskip.2em plus .9em गङ्गाग्नि \msCc\msNa\msNb\msNc,\hskip.5em plus .9em गङ्गाऽग्नि॰ \Ed}}% 

{\devanagarifont नैमिषं बिन्दुसारं च सेतुबन्धं सुरद्रहम् \thinspace{\dandab} \dontdisplaylinenum }%
     \var{{\devanagarifontvar\numemph\va नैमिषं\lem \mssALL,\hskip.2em plus .9em नेमिस \msNc}}% 
    \var{{\devanagarifontvar\numnoemph\vb ॰बन्धं\lem \mssALL,\hskip.2em plus .9em ॰बन्ध॰ \Ed\oo 
 ॰द्रहम् \lem \mssALL,\hskip.2em plus .9em ॰ह्रदं \Ed}}% 

%Verse 10:8

{\devanagarifont घण्टिकेश्वरवागीशं ज्ञात्वा निश्चयपापहा {॥ १०:\hspace{.11em}८॥} \veg\dontdisplaylinenum }%
     \var{{\devanagarifontvar\numnoemph\vc ॰वागीशं\lem \mssALL,\hskip.2em plus .9em \lacwithnum{1} \uncl{गीश} \msNb}}% 
    \var{{\devanagarifontvar\numnoemph\vd निश्चयपापहा\lem \mssALL,\hskip.2em plus .9em 
निश्च\uncl{य}\lk\lk\lk\  \msCa}}% 

{\devanagarifont उमोवाच {\dandab}\dontdisplaylinenum  }%
 
{\devanagarifont एवमादि महादेव पूर्ववत्कथितास्म्यहम् \thinspace{\danda} \dontdisplaylinenum }%
     \var{{\devanagarifontvar\numemph\vb कथिता॰\lem \msCa\msCc\msNa\msNc,\hskip.2em plus .9em कथितो \msCb\msNb\Ed}}% 

%Verse 10:9

{\devanagarifont स्वर्गभोगप्रदं तीर्थमेतेषां सुरनायक {॥ १०:\hspace{.11em}९॥} \veg\dontdisplaylinenum }%
     \var{{\devanagarifontvar\numnoemph\vcd तीर्थमे॰\lem \mssALL,\hskip.2em plus .9em तीर्थंमे॰ \msCc}}% 
    \var{{\devanagarifontvar\numnoemph\vd सुरनायक\lem \msCapcorr\msNa\msNc,\hskip.2em plus .9em सुरनाक \msCaacorr,\hskip.5em plus .9em सुरनायकम् \msCb\msCc\msNb\Ed}}% 

{\devanagarifont कथं मुच्येत संसाराज्ज्ञानमात्रेण ईश्वर \thinspace{\dandab} \dontdisplaylinenum }%
     \var{{\devanagarifontvar\numemph\va कथं\lem \mssALL,\hskip.2em plus .9em कथ \msCb}}% 
    \var{{\devanagarifontvar\numnoemph\vb ज्ञान॰\lem \mssALL,\hskip.2em plus .9em ज्ञात॰ \msCb\oo 
 ईश्वर\lem \mssALL,\hskip.2em plus .9em चेश्वर \msNa}}% 

%Verse 10:10

{\devanagarifont कौतूहलं महज्जातं छिन्धि संशयकारकम् {॥ १०:\hspace{.11em}१०॥} \veg\dontdisplaylinenum }%
     \var{{\devanagarifontvar\numnoemph\vc कौतूहलं महज्जातं\lem \mssCaCbCc\Ed,\hskip.2em plus .9em कौतूहलम्म\uncl{हो}ज्जातं \msNa,\hskip.5em plus .9em 
कौहलम्महज्जातं \msNbacorr,\hskip.5em plus .9em 
कौ\uncl{तू}हलम्महज्जातं \msNbpcorr,\hskip.5em plus .9em 
कोतूहलं महज्जातं \msNc}}% 
    \var{{\devanagarifontvar\numnoemph\vd ॰कारकम्\lem \Ed,\hskip.2em plus .9em ॰कारक \mssCaCbCc\msNb\msNc,\hskip.5em plus .9em ॰कारकः \msNa}}% 

\pend
\endnumbering
\vfill\pagebreak\beginnumbering\pstart
\vers

{\devanagarifont रुद्र उवाच {\dandab}\dontdisplaylinenum  }%
 
{\devanagarifont किं न जानामि तत्तीर्थं सुलभं दुर्लभं च यत् \thinspace{\danda} \dontdisplaylinenum }%
     \var{{\devanagarifontvar\numemph\va जानामि\lem \mssCaCbCc\msNb,\hskip.2em plus .9em जाना\uncl{मि} \msNaacorr,\hskip.5em plus .9em जाना\uncl{सि} \msNapcorr,\hskip.5em plus .9em 
जानासि \msNc\Ed}}% 
    \var{{\devanagarifontvar\numnoemph\vb दुर्लभं च\lem \msCa\msNa\msNb\Ed,\hskip.2em plus .9em दुलभञ्च \msCb\msNc,\hskip.5em plus .9em दुल्लभञ्च \msCc}}% 

%Verse 10:11

{\devanagarifont सुलभं गुरुसेवीनां दुर्लभं तद्विवर्जयेत् {॥ १०:\hspace{.11em}११॥} \veg\dontdisplaylinenum }%
     \var{{\devanagarifontvar\numnoemph\vc सुलभं गुरुसेवीनां\lem \mssALL,\hskip.2em plus .9em 
\lk\lk \lk\lk \lk\lk वीनां \msCa}}% 
    \var{{\devanagarifontvar\numnoemph\vd ॰वर्जयेत्\lem \mssALL,\hskip.2em plus .9em ॰वर्जये \msNa,\hskip.5em plus .9em ॰वर्जनात् \Ed}}% 


\alalalfejezet{कुरुक्षेत्रम्}

{\devanagarifont कुरुः पुरुष विज्ञेयः शरीरं क्षेत्र उच्यते \thinspace{\dandab} \dontdisplaylinenum }%
     \var{{\devanagarifontvar\numemph\va कुरुः\lem \mssALL,\hskip.2em plus .9em गुरुः \msNb\oo 
 पुरुष\lem \Ed,\hskip.2em plus .9em पुरुषः \mssCaCbCc\msNa\msNb\ \unmetr,\hskip.5em plus .9em पुरुषो \msNc\ \unmetr}}% 
    \var{{\devanagarifontvar\numnoemph\vb शरीरं\lem \mssALL,\hskip.2em plus .9em शरी\uncl{र} \msCa\oo 
 क्षेत्र उच्यते\lem \mssALL,\hskip.2em plus .9em क्षेत्रमुच्यते \msNa}}% 
    \paral{{\devanagarifontsmall \vb {\englishfont \compare\ \BHG\ 13.1:}
                         इदं शरीरं कौन्तेय क्षेत्रमित्यभिधीयते\thinspace{\devanagarifontsmall ।}
                         एतद्यो वेत्ति तं प्राहुः क्षेत्रज्ञ इति तद्विदः\thinspace{\devanagarifontsmall ॥} }}

%Verse 10:12

{\devanagarifont शरीरस्थं कुरुक्षेत्रं सर्वतीर्थफलप्रदम् {॥ १०:\hspace{.11em}१२॥} \veg\dontdisplaylinenum }%
     \var{{\devanagarifontvar\numnoemph\vc ॰स्थं\lem \mssALL,\hskip.2em plus .9em ॰स्थ \msNc\oo 
 ॰क्षेत्रं\lem \mssALL,\hskip.2em plus .9em ॰क्षेत्र \msNc}}% 

{\devanagarifont सर्वयज्ञफलावाप्तिः सर्वदानफलानि च \thinspace{\dandab} \dontdisplaylinenum }%
     \paral{{\devanagarifontsmall \vab {\englishfont \similar\ \UMS\ 21.48cd:}
                                 सर्वयज्ञफलावाप्तिः सर्वदानफलं लभेत् 
                     {\englishfont \similar\ \VSS\ 11.2ab} }}

%Verse 10:13

{\devanagarifont सर्वव्रततपश्चीर्णं तत्फलं सकलं भवेत् {॥ १०:\hspace{.11em}१३॥} \veg\dontdisplaylinenum }%
     \var{{\devanagarifontvar\numemph\vd तत्फलं\lem \mssALL,\hskip.2em plus .9em तत्फल \msNc}}% 

{\devanagarifont एवमेव फलं तेषां तीर्थपञ्चदशेषु च \thinspace{\dandab} \dontdisplaylinenum }%
     \var{{\devanagarifontvar\numemph\vb तीर्थपञ्चदशेषु\lem \mssALL,\hskip.2em plus .9em तीर्थम्पंचदशैषु \msCb}}% 

%Verse 10:14

{\devanagarifont अनघानं महापुण्यं महातीर्थं महासुखम् {॥ १०:\hspace{.11em}१४॥} \veg\dontdisplaylinenum }%
     \var{{\devanagarifontvar\numnoemph\vc अनघानं महापुण्यं\lem \msCb\msNc,\hskip.2em plus .9em \lk\lk \lk\lk \lk\lk पुण्य \msCa,\hskip.5em plus .9em 
अनप्याम्महापुण्यं \msCc\ \hypermetr,\hskip.5em plus .9em 
अनध्यानं महापुण्यं \msNa,\hskip.5em plus .9em अध्वानन्तु महापुण्यं \msNb,\hskip.5em plus .9em 
स्नानध्यानं महापुण्यं \Ed}}% 

{\devanagarifont देव्युवाच {\dandab}\dontdisplaylinenum  }%
 
{\devanagarifont अतीव रोमहर्षो मे जातो ऽस्ति त्रिदशेश्वर \thinspace{\danda} \dontdisplaylinenum }%
     \var{{\devanagarifontvar\numemph\va अतीव\lem \mssALL,\hskip.2em plus .9em अवीव \msCb}}% 
    \var{{\devanagarifontvar\numnoemph\vb ऽस्ति\lem \mssALL,\hskip.2em plus .9em स्मि \msNb\oo 
 त्रिदशेश्वर\lem \mssALL,\hskip.2em plus .9em त्रिदशेश्वरः \msCc,\hskip.5em plus .9em त्रि\lacwithnum{1}  शेश्वर \msNb}}% 

%Verse 10:15

{\devanagarifont सुलभं सुकरं सूक्ष्मं श्रुत्वा तुष्टिश्च मे गता {॥ १०:\hspace{.11em}१५॥} \veg\dontdisplaylinenum }%
     \var{{\devanagarifontvar\numnoemph\vd तुष्टिश्च\lem \mssALL,\hskip.2em plus .9em तुष्टिञ्च \msCc\oo 
 गता\lem \mssALL,\hskip.2em plus .9em गताः \msCb}}% 

\pend
\endnumbering
\vfill\pagebreak\beginnumbering\pstart
\vers

{\devanagarifont चतुर्दश परो भूयः कथयस्व मनोहरम् \thinspace{\dandab} \dontdisplaylinenum }%
 
%Verse 10:16

{\devanagarifont प्रयागादि पृथक्त्वेन तत्त्वतस्तु सुरेश्वर {॥ १०:\hspace{.11em}१६॥} \veg\dontdisplaylinenum }%
     \var{{\devanagarifontvar\numemph\vd तत्त्वतस्तु\lem \mssALL,\hskip.2em plus .9em तत्वत \msNaacorr}}% 


\alalalfejezet{प्रयागो वाराणसी च}

{\devanagarifont रुद्र उवाच {\dandab}\dontdisplaylinenum  }%
 
{\devanagarifont सुषुम्ना भगवती गङ्गा इडा च यमुना नदी \thinspace{\danda} \dontdisplaylinenum }%
     \var{{\devanagarifontvar\numemph\va सुषुम्ना\lem \mssALL,\hskip.2em plus .9em सुषुम्णा \Ed\oo 
 भगवती गङ्गा\lem \mssALL,\hskip.2em plus .9em 
भगवती ग\lk\ \msCa,\hskip.5em plus .9em भवती गङ्गा \Ed}}% 

%Verse 10:17

{\devanagarifont एताः स्रोतोवहा नद्यः प्रयागः स विधीयते {॥ १०:\hspace{.11em}१७॥} \veg\dontdisplaylinenum }%
     \var{{\devanagarifontvar\numnoemph\vc एताः स्रोतोवहा\lem \eme,\hskip.2em plus .9em एता श्रोतवहा \msCa\msNc\Ed,\hskip.5em plus .9em 
एते श्रोतावहा \msCb\msCc,\hskip.5em plus .9em एता श्रोत्रवहा \msNa\msNb}}% 

{\devanagarifont दक्षिणा वारुणी नासा वामनासा असि स्मृता \thinspace{\dandab} \dontdisplaylinenum }%
     \var{{\devanagarifontvar\numemph\va दक्षिणा\lem \mssALL,\hskip.2em plus .9em दक्षि\uncl{णं} \msCa,\hskip.5em plus .9em दक्षिणं \msCc\oo 
 वारुणी\lem \msNapcorr\msNc\Ed,\hskip.2em plus .9em वरुणी \msCa\msCc\msNaacorr\msNb,\hskip.5em plus .9em वरुणा \msCb}}% 
    \var{{\devanagarifontvar\numnoemph\vb ॰नासा\lem \mssALL,\hskip.2em plus .9em ॰ना \msCb\msNb}}% 

%Verse 10:18

{\devanagarifont वारुणा-असिमध्येन तेन वाराणसी स्मृता {॥ १०:\hspace{.11em}१८॥} \veg\dontdisplaylinenum }%
     \var{{\devanagarifontvar\numnoemph\vc वारुणा-असिमध्येन\lem \Ed,\hskip.2em plus .9em वरुणा असिमध्येन \msCa\msCb\msNa\msNc,\hskip.5em plus .9em 
वारुणन्नासमध्येत \msCc,\hskip.5em plus .9em 
वरुण असिमध्येन \msNb}}% 


\alalalfejezet{गङ्गा}

{\devanagarifont आकाशगङ्गा विख्याता तस्याः स्रवति चामृतम् \thinspace{\dandab} \dontdisplaylinenum }%
     \var{{\devanagarifontvar\numemph\vb तस्याः\lem \mssALL,\hskip.2em plus .9em तस्मा \msCc,\hskip.5em plus .9em तस्या \msNb}}% 

%Verse 10:19

{\devanagarifont अहोरात्रमविच्छिन्नं गङ्गा सा तेन उच्यते {॥ १०:\hspace{.11em}१९॥} \veg\dontdisplaylinenum }%
     \var{{\devanagarifontvar\numnoemph\vd तेन\lem \mssALL,\hskip.2em plus .9em ते \msCc}}% 


\alalalfejezet{सोमतीर्थम्}

{\devanagarifont सोमतीर्थमिडा नाडी किङ्किणीरवचिह्निता \thinspace{\dandab} \dontdisplaylinenum }%
     \var{{\devanagarifontvar\numemph\va ॰तीर्थमिडा\lem \mssALL,\hskip.2em plus .9em ॰तीर्थ इडा \msCb}}% 
    \var{{\devanagarifontvar\numnoemph\vb किङ्किणी॰\lem \mssALL,\hskip.2em plus .9em चिञ्चिनी॰ \msCc\oo 
 ॰रव॰\lem \mssALL,\hskip.2em plus .9em ॰रवि॰ \msCbacorr,\hskip.5em plus .9em ॰राव॰ \Ed\oo 
 ॰चिह्निता\lem \mssALL,\hskip.2em plus .9em ॰चिह्निका \msCc,\hskip.5em plus .9em ॰चिह्नता \msNb}}% 

%Verse 10:20

{\devanagarifont तं तु श्रुत्वा न संदेहः सर्वपापक्षयो भवेत् {॥ १०:\hspace{.11em}२०॥} \veg\dontdisplaylinenum }%
     \var{{\devanagarifontvar\numnoemph\vc तं तु\lem \corr,\hskip.2em plus .9em \uncl{तन्तु} \msCa,\hskip.5em plus .9em तन्तु \msCb\msCc\msNa\msNc\Ed,\hskip.5em plus .9em 
त\uncl{त्तु} \msNb\oo 
 न संदेहः\lem \mssALL,\hskip.2em plus .9em वरारोहेः \msCc}}% 

\pend
\endnumbering
\vfill\pagebreak\beginnumbering\pstart
\vers


\alalalfejezet{सूर्यतीर्थम्}

{\devanagarifont सूर्यतीर्थं सुषुम्ना च नीरवारवसंयुता \thinspace{\dandab} \dontdisplaylinenum }%
     \var{{\devanagarifontvar\numemph\va ॰तीर्थं\lem \mssALL,\hskip.2em plus .9em ॰तीर्थ \msNb\oo 
 सुषुम्ना\lem \mssALL,\hskip.2em plus .9em सुषुम्णा \Ed}}% 
    \var{{\devanagarifontvar\numnoemph\vb नीरवा॰\lem \Ed,\hskip.2em plus .9em वीरवा॰ \msCa\msCc,\hskip.5em plus .9em चीरवा॰ \msCb\msNa\msNb\msNc\oo 
 ॰युता\lem \msCa\msNa\msNc\Ed,\hskip.2em plus .9em ॰युतम् \msCb\msCc,\hskip.5em plus .9em ॰युतां \msNb}}% 

%Verse 10:21

{\devanagarifont श्रुतिमात्राद्विमुच्येत पापराशिर्महानपि {॥ १०:\hspace{.11em}२१॥} \veg\dontdisplaylinenum }%
     \var{{\devanagarifontvar\numnoemph\vc ॰मात्रा॰\lem \mssALL,\hskip.2em plus .9em ॰माता॰ \msCc}}% 


\alalalfejezet{अग्नितीर्थम्}

{\devanagarifont अग्नितीर्थार्जुना नाडी ब्रह्मघोषमनोरमा \thinspace{\dandab} \dontdisplaylinenum }%
     \var{{\devanagarifontvar\numemph\va ॰र्जुना\lem \mssALL,\hskip.2em plus .9em ॰जुना \msCc,\hskip.5em plus .9em ॰र्जुनं \Ed}}% 
    \var{{\devanagarifontvar\numnoemph\vb ॰रमा\lem \mssALL,\hskip.2em plus .9em ॰रमाः \msNc\Ed}}% 

%Verse 10:22

{\devanagarifont तत्तदक्षरमाकर्ण्य अमृतत्वाय कल्पते {॥ १०:\hspace{.11em}२२॥} \veg\dontdisplaylinenum }%
     \var{{\devanagarifontvar\numnoemph\vc ॰कर्ण्य\lem \mssALL,\hskip.2em plus .9em ॰र्ण्य \msCb}}% 
    \var{{\devanagarifontvar\numnoemph\vd कल्पते\lem \msCb\msNc\Ed,\hskip.2em plus .9em क\lk \lacwithnum{1}\  \msCa,\hskip.5em plus .9em कल्प्यते \msCc\msNa\msNb}}% 
    \paral{{\devanagarifontsmall \vcd {\englishfont = 22.31} }}


\alalalfejezet{पुष्करम्}

{\devanagarifont पुष्करं हृदि मध्यस्थमष्टपत्त्रं सकर्णिकम् \thinspace{\dandab} \dontdisplaylinenum }%
     \var{{\devanagarifontvar\numemph\vb ॰पत्त्रं\lem \msCb\msNa\msNc\Ed,\hskip.2em plus .9em \lk\lk\ \msCa,\hskip.5em plus .9em ॰पत्र \msCc\msNb\oo 
 ॰कर्णिकम्\lem \mssALL,\hskip.2em plus .9em \lk\lk\lk\  \msCa,\hskip.5em plus .9em ॰कर्णिकाम् \Ed}}% 

%Verse 10:23

{\devanagarifont चिन्तयेत्सूक्ष्म तन्मध्ये जन्ममृत्युविनाशनम् {॥ १०:\hspace{.11em}२३॥} \veg\dontdisplaylinenum }%
     \var{{\devanagarifontvar\numnoemph\vc सूक्ष्म\lem \mssALL,\hskip.2em plus .9em \uncl{सूक्ष्म} \msCa,\hskip.5em plus .9em सूक्ष्मं \Ed}}% 


\alalalfejezet{मानसम्}

{\devanagarifont मानससरमध्यस्थं स हंसः कमलोपरि \thinspace{\dandab} \dontdisplaylinenum }%
     \var{{\devanagarifontvar\numemph\va मानस॰\lem \msCb\msNa,\hskip.2em plus .9em \uncl{मानस} \msCa,\hskip.5em plus .9em मानसं \msCc\msNb\msNc\Ed}}% 
    \var{{\devanagarifontvar\numnoemph\vb स हंसः\lem \conj,\hskip.2em plus .9em सहंस॰ \msCa\msCc\msNa\msNb\msNc\Ed,\hskip.5em plus .9em सहसं \msCb}}% 

%Verse 10:24

{\devanagarifont सलीलो लीलयाचारी परतः परपारगः {॥ १०:\hspace{.11em}२४॥} \veg\dontdisplaylinenum }%
     \var{{\devanagarifontvar\numnoemph\vc सलीलो\lem \mssALL,\hskip.2em plus .9em सलीला \Ed}}% 
    \var{{\devanagarifontvar\numnoemph\vd परतः\lem \mssALL,\hskip.2em plus .9em परत \msNb}}% 


\alalalfejezet{नैमिषम्}

{\devanagarifont नैमिषं शृणु देवेशि निमिषा प्रत्ययो भवेत् \thinspace{\dandab} \dontdisplaylinenum }%
     \var{{\devanagarifontvar\numemph\vb निमिषा प्रत्ययो भवेत्\lem \mssALL,\hskip.2em plus .9em निमि प्रत्ययो भवेत् \msCb,\hskip.5em plus .9em 
नि\lacwithnum{1} \uncl{षो} प्रत्ययो \uncl{भवेत्} \msNb}}% 

%Verse 10:25

{\devanagarifont सम्यग्छायां निरीक्षेत आत्मानो वा परस्य वा {॥ १०:\hspace{.11em}२५॥} \veg\dontdisplaylinenum }%
     \var{{\devanagarifontvar\numnoemph\vd आत्मनो\lem \mssALL,\hskip.2em plus .9em \lk न्मनो \msCa,\hskip.5em plus .9em स्वात्मानो \Ed\oo 
 परस्य वा\lem \mssALL,\hskip.2em plus .9em परस्य च \Ed}}% 

\pend
\endnumbering
\vfill\pagebreak\beginnumbering\pstart
\vers

{\devanagarifont आयतमङ्गुलीमात्रं निमिषाक्षिः स पश्यति \thinspace{\dandab} \dontdisplaylinenum }%
     \var{{\devanagarifontvar\numemph\va आयतमङ्गुली॰\lem \conj,\hskip.2em plus .9em आयतप्यङ्गुली॰ \mssCaCbCc\msNa\msNb,\hskip.5em plus .9em 
आयातप्यङ्गुली॰ \msNc\Ed\oo 
 ॰मात्रं\lem \mssALL,\hskip.2em plus .9em ॰मात्र \msNc,\hskip.5em plus .9em ॰मध्ये \Ed}}% 
    \var{{\devanagarifontvar\numnoemph\vb ॰क्षिः\lem \eme,\hskip.2em plus .9em ॰क्षि \mssCaCbCc\msNa\msNb\msNc\Ed}}% 

%Verse 10:26

{\devanagarifont दृष्ट्वा प्रत्ययमेवं हि नैमिषज्ञः स उच्यते {॥ १०:\hspace{.11em}२६॥} \veg\dontdisplaylinenum }%
     \var{{\devanagarifontvar\numnoemph\vd नैमिषज्ञः\lem \mssALL,\hskip.2em plus .9em नैमिसंज्ञः \msCb,\hskip.5em plus .9em नैमिषज्ञ \msCc}}% 


\alalalfejezet{बिन्दुसरः}

{\devanagarifont तीर्थं बिन्दुसरं नाम शृणु वक्ष्यामि सुन्दरि \thinspace{\dandab} \dontdisplaylinenum }%
     \var{{\devanagarifontvar\numemph\va तीर्थं बिन्दु॰\lem \mssALL,\hskip.2em plus .9em तीर्थमिन्दु॰ \Ed}}% 

%Verse 10:27

{\devanagarifont देहमध्ये हृदि ज्ञेयं हृदिमध्ये तु पङ्कजम् {॥ १०:\hspace{.11em}२७॥} \veg\dontdisplaylinenum }%
     \var{{\devanagarifontvar\numnoemph\vc हृदि ज्ञेयं\lem \mssALL,\hskip.2em plus .9em \om\ \msCb}}% 
    \paral{{\devanagarifontsmall \vcd {\englishfont \similar\ 22.24ab
                 \vo \compare\ \NISVK\ 5.55:}
                 एतेषां नादमध्ये तु शिवं तत्र व्यवस्थितः\thinspace{\devanagarifontsmall ।}
                 हृदयं देहमध्ये तु तत्र पद्मं व्यवस्थितम्\thinspace{\devanagarifontsmall ॥} }}

{\devanagarifont कर्णिका पद्ममध्ये तु बिन्दुः कर्णिकमध्यतः \thinspace{\dandab} \dontdisplaylinenum }%
     \var{{\devanagarifontvar\numemph\va ॰मध्ये\lem \mssALL,\hskip.2em plus .9em ॰ध्ये \msCa,\hskip.5em plus .9em ॰पध्ये \msNa}}% 

%Verse 10:28

{\devanagarifont बिन्दुमध्ये स्थितो नादः स नादः केन भिद्यते {॥ १०:\hspace{.11em}२८॥} \veg\dontdisplaylinenum }%
     \var{{\devanagarifontvar\numnoemph\vc बिन्दुमध्ये\lem \mssALL,\hskip.2em plus .9em \uncl{बिन्दु}\lk\lk\ \msCa}}% 
    \var{{\devanagarifontvar\numnoemph\vd भिद्यते\lem \mssALL,\hskip.2em plus .9em \uncl{वि}द्यते \msCa,\hskip.5em plus .9em विद्यते \msCc}}% 
    \paral{{\devanagarifontsmall \vo {\englishfont \compare\ \NISVK\ 5.56:}
                 कर्णिका पद्ममध्ये तु अकारं तस्य मध्यतः\thinspace{\devanagarifontsmall ।}
                 तस्य मध्ये विनिष्क्रान्तं नादं परमदुर्लभम्\thinspace{\devanagarifontsmall ॥} }}

{\devanagarifont उकारं च मकारं च भित्त्वा नादो विनिर्गतः \thinspace{\dandab} \dontdisplaylinenum }%
     \var{{\devanagarifontvar\numemph\va उकारं च मकारं\lem \mssALL,\hskip.2em plus .9em उकारश्च मकारश् \Ed}}% 
    \paral{{\devanagarifontsmall \vab {\englishfont = \NISVK\ 5.57ab} }}

%Verse 10:29

{\devanagarifont तं विदित्वा विशालाक्षि सो ऽमृतत्वं लभेत च {॥ १०:\hspace{.11em}२९॥} \veg\dontdisplaylinenum }%
     \var{{\devanagarifontvar\numnoemph\vd सो ऽमृतत्वं\lem \mssALL,\hskip.2em plus .9em सोम्यतत्वं \msCc,\hskip.5em plus .9em सोमतत्वं \Ed\oo 
 च\lem \mssALL,\hskip.2em plus .9em वा \Ed}}% 


\alalalfejezet{सेतुबन्धम्}

\nemslokanormal


\ujvers\nemsloka {
{\devanagarifont वक्ष्ये ते सेतुबन्धं दुरितमलहरं नादतोयप्रवाहं }%
  \dontdisplaylinenum}    \var{{\devanagarifontvar\numemph\va ते\lem \mssALL,\hskip.2em plus .9em \om\ \msCaacorr,\hskip.5em plus .9em हं \msCc\oo 
 ॰बन्धं\lem \mssALL,\hskip.2em plus .9em ॰बन्धूं \msCb\oo 
 ॰तोय॰\lem \mssALL,\hskip.2em plus .9em ॰तोयं \msNb}}% 


\nemslokab

{\devanagarifont जिह्वाकण्ठोरकूला स्वरगणपुलिनावर्तघोषा तरङ्गा  \danda\dontdisplaylinenum }%
     \var{{\devanagarifontvar\numnoemph\vb ॰कण्ठोर॰\lem \conj,\hskip.2em plus .9em ॰कण्ठोरु॰ \mssCaCbCc\msNa\msNb\msNc\Ed\oo 
 स्वर॰\lem \mssALL,\hskip.2em plus .9em सुर॰ \msCc\Ed}}% 

\pend
\endnumbering
\vfill\pagebreak\beginnumbering\pstart
\vers

\nemslokac

{\devanagarifont कुम्भीराघोषमीना दशगणमकरा भीमनक्रा विसर्गा }%
  \dontdisplaylinenum    \var{{\devanagarifontvar\numnoemph\vc ॰मीना\lem \mssALL,\hskip.2em plus .9em ॰माना \Ed\oo 
 दश॰\lem \mssALL,\hskip.2em plus .9em \lk\lk\ \msCa\oo 
 विसर्गा\lem \mssCaCbCc,\hskip.2em plus .9em विसर्गाः \msNa\msNb\msNc\Ed}}% 

%Verse 10:30


\nemslokad

{\devanagarifont सानुस्वारे गभीरे मदसुखरसनं सेतुबन्धं व्रजस्व {॥ १०:\hspace{.11em}३०॥} \veg\dontdisplaylinenum }%
     \var{{\devanagarifontvar\numnoemph\vd ॰स्वारे\lem \msCa\msCb\msNc\Ed,\hskip.2em plus .9em ॰सारे \msCc,\hskip.5em plus .9em 
॰स्वारो \msNa,\hskip.5em plus .9em ॰स्वा\uncl{रेण} \msNb\ \unmetr\oo 
 गभीरे\lem \msCa\msCb\msNc,\hskip.2em plus .9em गम्भीरे \msCc\msNb\Ed,\hskip.5em plus .9em \uncl{गं}भीरे \msNa\oo 
 ॰रसनं\lem \mssALL,\hskip.2em plus .9em ॰रमणं \Ed\oo 
 ॰बन्धं\lem \mssALL,\hskip.2em plus .9em ॰बन्ध \msCb\oo 
 व्रजस्व\lem \mssALL,\hskip.2em plus .9em रमस्व \Ed}}% 


\alalalfejezet{सुरद्रहः}

\nemslokalong


\ujvers\nemsloka {
{\devanagarifont सप्तद्वीपान्तमध्ये शृणु शशिवदने सर्वदुःखान्तलाभम् }%
  \dontdisplaylinenum}    \var{{\devanagarifontvar\numemph\va ॰द्वीपा॰\lem \mssALL,\hskip.2em plus .9em ॰दीपा॰ \msNc}}% 


\nemslokab

{\devanagarifont ईशानेनाभिजुष्टं हृदि ह्रद विमलं नादशीताम्बुपूर्णम्  \danda\dontdisplaylinenum }%
     \var{{\devanagarifontvar\numnoemph\vb ईशानेनाभिजुष्टं\lem \msCc\msNa\msNc\Ed,\hskip.2em plus .9em ईशानेनाभिदुष्टं \msCa\msNb,\hskip.5em plus .9em 
ईशानेभिदुष्टं \msCbacorr,\hskip.5em plus .9em ईशानेभि\lacwithnum{1}  दुष्टं \msCbpcorr\oo 
 विमलं नादशीता॰\lem \mssALL,\hskip.2em plus .9em 
विमलान्नादशीता॰ \msNb,\hskip.5em plus .9em विमलं नामशिता॰ \Ed}}% 

\nemslokac

{\devanagarifont तत्रैकं जातपद्मं प्रकृतिदलयुतं केशरं शक्तिभिन्नं }%
  \dontdisplaylinenum    \var{{\devanagarifontvar\numnoemph\vc केशरं\lem \msCb\Ed,\hskip.2em plus .9em केशर॰ \msCa\msCc\msNa\msNc\ \unmetr,\hskip.5em plus .9em केश्वर॰ \msNb\ \unmetr}}% 

%Verse 10:31


\nemslokad

{\devanagarifont पञ्चव्योमप्रशस्तं गतिपरमपदं प्राप्तुकामेन सेव्यम् {॥ १०:\hspace{.11em}३१॥} \veg\dontdisplaylinenum }%
     \var{{\devanagarifontvar\numnoemph\vd ॰व्योम॰\lem \mssALL,\hskip.2em plus .9em ॰व्यो\uncl{मं} \msNa\oo 
 ॰शस्तं ग॰\lem \mssALL,\hskip.2em plus .9em ॰शस्वङ्ग॰ \msCc\oo 
 ॰परम॰\lem \mssALL,\hskip.2em plus .9em ॰परमं \msNa\ \unmetr\oo 
 सेव्यम्\lem \mssALL,\hskip.2em plus .9em सर्वम् \Ed}}% 


\alalalfejezet{घण्टिकेश्वरम्}

\nemslokalong


\ujvers\nemsloka {
{\devanagarifont †नाड्यैकासङ्गतानि† निपतितममृतं घण्टिकापारकेण }%
  \dontdisplaylinenum}    \var{{\devanagarifontvar\numemph\va निपतितममृतं\lem \mssALL,\hskip.2em plus .9em निपतितममृत॰ \msNa\ \unmetr,\hskip.5em plus .9em 
नि\lacwithnum{2}  तममृतं \msNb\oo 
 ॰पारकेण\lem \msCa\msCb\msNa\msNc,\hskip.2em plus .9em ॰याङ्करेण \msCc\Ed,\hskip.5em plus .9em ॰\uncl{पारकेन} \msNb}}% 


\nemslokab

{\devanagarifont तृप्यन्ते तेन नित्यं हृदि कमलपुटं स्थाणुभूतान्तरात्मा  \danda\dontdisplaylinenum }%
     \var{{\devanagarifontvar\numnoemph\vb ॰पुटं\lem \mssALL,\hskip.2em plus .9em ॰पुट \msCb\oo 
 स्थाणु॰\lem \conj,\hskip.2em plus .9em स्थानु॰ \mssCaCbCc\msNa\msNc,\hskip.5em plus .9em 
\uncl{स्थान}॰ \msNb,\hskip.5em plus .9em स्थान॰ \Ed}}% 

\nemslokac

{\devanagarifont यं पश्यन्तीशभक्ताः कलिकलुषहरं व्यापिनं निष्प्रपञ्चं }%
  \dontdisplaylinenum    \var{{\devanagarifontvar\numnoemph\vc यं पश्यन्तीशभक्ताः\lem \msNa,\hskip.2em plus .9em यं पश्यन्तीशभक्ता \msCa\msNb,\hskip.5em plus .9em 
यं पश्यन्तीशभर्त्ताः \msCb,\hskip.5em plus .9em यं पस्यन्तीसभक्त्या \msCc,\hskip.5em plus .9em 
यत्पश्यन्तीशभक्त्या \msNc,\hskip.5em plus .9em यं पश्यन्नीशमक्षा \Ed\oo 
 ॰प्रपञ्चम्\lem \msCa\msNa\msNb\msNc,\hskip.2em plus .9em ॰प्रपञ्च \msCb\msCc\Ed}}% 

%Verse 10:32


\nemslokad

{\devanagarifont देवेशं घण्टिकेशामरभवमभवं तीर्थमाकाशबिन्दुम् {॥ १०:\hspace{.11em}३२॥} \veg\dontdisplaylinenum }%
     \var{{\devanagarifontvar\numnoemph\vd देवेशं\lem \msCb\msNb\Ed,\hskip.2em plus .9em देव्येशं \msCa\msCc\msNa,\hskip.5em plus .9em देव्येश \msNc\oo 
 घण्टिकेशामर॰\lem \msCc,\hskip.2em plus .9em घण्टिकेशमर॰ \msCa\msCb\msNb\msNc,\hskip.5em plus .9em 
घण्टिकेशं मर॰ \msNa,\hskip.5em plus .9em घाण्टकेशामर॰ \Ed\oo 
 ॰भवं तीर्थम्\lem \eme,\hskip.2em plus .9em ॰भवन्तीर्थम् \msCb\msCc\msNa\msNb\msNc\Ed,\hskip.5em plus .9em भव\lk\lk र्थम् \msCa\oo 
 ॰बिन्दुम्\lem \mssALL,\hskip.2em plus .9em ॰बिन्दु \msCc}}% 


\alalalfejezet{वागीश्वरतीर्थम्}

\nemslokalong


\ujvers\nemsloka {
{\devanagarifont मीमांसारत्नकूला क्रमपदपुलिना शैवशास्त्रार्थतोया }%
  \dontdisplaylinenum}    \var{{\devanagarifontvar\numemph\va शैव॰\lem \mssALL,\hskip.2em plus .9em शर्व॰ \Ed}}% 


\nemslokab

{\devanagarifont मीनौघा पञ्चरात्रं श्रुतिकुटिलगतिः स्मार्तवेगा तरङ्गा  \danda\dontdisplaylinenum }%
     \var{{\devanagarifontvar\numnoemph\vb मीनौघा॰\lem \msNa\msNb\Ed,\hskip.2em plus .9em  मीनोघा॰ \mssCaCbCc\msNc\oo 
 पञ्चरात्रं\lem \mssALL,\hskip.2em plus .9em पञ्चशत्रं \Ed\oo 
 ॰गतिः\lem \corr,\hskip.2em plus .9em ॰गति \mssCaCbCc\msNa\msNb\msNc\Ed\oo 
 ॰स्मार्तवेगा तरङ्गा\lem \mssALL,\hskip.2em plus .9em ॰स्मा\lacwithnum{1}  \uncl{वेगा तरङ्गा} \msNb,\hskip.5em plus .9em 
॰स्मार्तवेगास्तरङ्गा \Ed}}% 

\nemslokac

{\devanagarifont योगावर्तातिशोभा उपनिषदिवहा भारतावर्तफेना }%
  \dontdisplaylinenum    \var{{\devanagarifontvar\numnoemph\vc ॰वहा भारता॰\lem \mssALL,\hskip.2em plus .9em महाभारता॰ \msNb}}% 

%Verse 10:33


\nemslokad

{\devanagarifont पञ्चाशद्व्योमरूपी रसभवननदी तीर्थ वागीश्वरीयम् {॥ १०:\hspace{.11em}३३॥} \veg\dontdisplaylinenum }%
     \var{{\devanagarifontvar\numnoemph\vd ॰शद्व्योम॰\lem \mssALL,\hskip.2em plus .9em ॰शव्योम॰ \msNa,\hskip.5em plus .9em ॰सद्व्योम॰ \Ed}}% 

\nemslokalong


\ujvers\nemsloka {
{\devanagarifont यस्तं वेत्ति स वेत्ति वेदनिखिलं संसारदुःखच्छिदं }%
  \dontdisplaylinenum}    \var{{\devanagarifontvar\numemph\va यस्तं\lem \mssALL,\hskip.2em plus .9em यस्त॰ \msCa\msCb\oo 
 स वेत्ति\lem \mssALL,\hskip.2em plus .9em \uncl{न} वेत्ति \msNc}}% 


\nemslokab

{\devanagarifont जन्मव्याधिवियोगतापमरणं क्लेशार्णवं दुःसहम्  \danda\dontdisplaylinenum }%
     \var{{\devanagarifontvar\numnoemph\vb ॰मरणं\lem \mssALL,\hskip.2em plus .9em ॰मरण \msNc\oo 
 ॰र्णवं\lem \mssALL,\hskip.2em plus .9em ॰ण्णवं \msNa,\hskip.5em plus .9em ॰र्णव \Ed}}% 

\nemslokac

{\devanagarifont गर्भावासमतीव सह्यविषयं दुस्तीर्यदुःखालयं }%
  \dontdisplaylinenum    \var{{\devanagarifontvar\numnoemph\vc गर्भावासम्\lem \mssALL,\hskip.2em plus .9em गर्भोवासम् \Ed\oo 
 ॰विषयं\lem \msCa\msCb\msNb,\hskip.2em plus .9em ॰विषमं \msCc\msNa\msNc\Ed\oo 
 ॰लयम्\lem \mssALL,\hskip.2em plus .9em ॰ल\uncl{यः} \msNa\oo 
 दुस्तीर्य॰\lem \mssALL,\hskip.2em plus .9em दुस्तीर्यः \msNc}}% 

%Verse 10:34


\nemslokad

{\devanagarifont प्राप्तं तेन न संशयः शिवपदं दुष्प्राप्य देवैरपि {॥ १०:\hspace{.11em}३४॥} \veg\dontdisplaylinenum }%
     \var{{\devanagarifontvar\numnoemph\vd प्राप्तं तेन न संशयः शिवपदं दुष्प्राप्य देवैरपि\lem \msCa\msCbpcorr\msNa\msNc,\hskip.2em plus .9em 
प्राप्तं तेन न संशयः   शिवदं दुष्प्राप्य देवैरपि \msCbacorr,\hskip.5em plus .9em 
प्राप्तं तेन न संशयं शिवपदं दुष्प्राप्य देवैरपि \msCc\Ed,\hskip.5em plus .9em 
प्रा\lacwithnum{6}  \uncl{यः शिव} \lk\lk \lk\lk  \uncl{य देवैरपि} \msNb}}% 

\vers


{\devanagarifont 
\jump
\begin{center}
\ketdanda~इति वृषसारसंग्रहे कायतीर्थोपवर्णनो नामाध्यायो दशमः~\ketdanda
\end{center}
\dontdisplaylinenum\vers  }%
     \var{{\devanagarifontvar\numnoemph{\englishfont \Colo:} कायतीर्थोपवर्णनो\lem \mssALL,\hskip.2em plus .9em 
कायती\lk\lk \lk र्ण्णनो \msCa\oo 
 नामाध्यायो दशमः\lem \mssALL,\hskip.2em plus .9em नाम दशमो ऽध्यायः \Ed}}% 

\nemslokanormal

\bekveg\szamveg
\vfill
\phpspagebreak

\versno=0\fejno=11
\thispagestyle{empty}

\centerline{\Large\devanagarifontbold [   एकादशमो ऽध्यायः  ]}{\vrule depth10pt width0pt} \fancyhead[CE]{{\footnotesize\devanagarifont वृषसारसंग्रहे  }}
\fancyhead[CO]{{\footnotesize\devanagarifont एकादशमो ऽध्यायः  }}
\fancyhead[LE]{}
\fancyhead[RE]{}
\fancyhead[LO]{}
\fancyhead[RO]{}
\szam\bek


\vers



\alalfejezet{चतुराश्रमधर्मविधानः}
{\devanagarifont देव्युवाच {\dandab}\dontdisplaylinenum  }%
 
{\devanagarifont सर्वयज्ञः परश्रेष्ठ अस्ति अन्यः सुरोत्तम \thinspace{\danda} \dontdisplaylinenum  }%
     \var{{\devanagarifontvar\numemph\vb अन्यः\lem \msCb\msNa\msNc,\hskip.2em plus .9em अन्य \msCa\msCc\msNb,\hskip.5em plus .9em चान्या \Ed\oo 
 ॰त्तम\lem \mssALL,\hskip.2em plus .9em ॰त्तमः \msNc}}% 
    \paral{{\devanagarifontsmall {\englishfont Witnesses used for this chapter:    \msCa\ ff.\thinspace 208v--210r,
                                                     \msCb\ ff.\thinspace 214r--215v,
                                                     \msCc\ ff.\allowbreak\thinspace 285v--287v,
                                                     \msNa\ ff.\thinspace 15v--17v,
                                                     \msNb\ ff.\thinspace 221v--223v 
                                                         (exp.\thinspace 56 lower -- 58 lower),
                                                     \msNc\ ff.\thinspace 223v--225v;
                                                     \Ed\ pp.\thinspace 613--617; 
                                                     \mssCaCbCc~= \msCa + \msCb + \msCc } }}

%Verse 11:1

{\devanagarifont अल्पक्लेशमनायास अर्थप्रायं विनेश्वर {॥ ११:\hspace{.11em}१॥} \veg\dontdisplaylinenum }%
     \var{{\devanagarifontvar\numnoemph\vc ॰नायास\lem \mssALL,\hskip.2em plus .9em 
॰नाया\uncl{सं} \msNa,\hskip.5em plus .9em ॰\uncl{नाया}सं \msNb}}% 
    \var{{\devanagarifontvar\numnoemph\vd ॰र्थप्रायं\lem \msNapcorr\msNc,\hskip.2em plus .9em ॰र्थप्राय \mssCaCbCc,\hskip.5em plus .9em 
॰र्थप्रार्थप्रायं \msNaacorr,\hskip.5em plus .9em ॰\uncl{र्थप्राय} \msNb,\hskip.5em plus .9em 
॰थाम्नाय \Ed\oo 
 विनेश्वर\lem \mssALL,\hskip.2em plus .9em \uncl{विनेश्वर} \msNb,\hskip.5em plus .9em सुरेश्वर \Ed}}% 

{\devanagarifont सर्वयज्ञफलावाप्ति दैवतैश्चापि पूजितम् \thinspace{\dandab} \dontdisplaylinenum }%
     \var{{\devanagarifontvar\numemph\va दैवतै॰\lem \msCa\msCb\msNa\Ed,\hskip.2em plus .9em देवतै॰ \msCc\msNc,\hskip.5em plus .9em \uncl{देवतै} \msNb}}% 

%Verse 11:2

{\devanagarifont कथयस्व सुरश्रेष्ठ मानुषाणां हिताय वै {॥ ११:\hspace{.11em}२॥} \veg\dontdisplaylinenum }%
     \var{{\devanagarifontvar\numnoemph\vcd ॰श्रेष्ठ मानुषाणां हिताय वै\lem \mssALL,\hskip.2em plus .9em 
॰श्रे\lacwithnum{10}  \msNb}}% 

{\devanagarifont महेश्वर उवाच {\dandab}\dontdisplaylinenum  }%
     \var{{\devanagarifontvar\numemph\vo महे॰\lem \mssALL,\hskip.2em plus .9em मेहे॰ \msNc}}% 

{\devanagarifont न तुल्यं तव पश्यामि दया भूतेषु भामिनि \thinspace{\danda} \dontdisplaylinenum }%
     \var{{\devanagarifontvar\numnoemph\va तुल्यं तव\lem \mssALL,\hskip.2em plus .9em \lacwithnum{4}  \msCa}}% 
    \var{{\devanagarifontvar\numnoemph\vb भामिनि\lem \mssALL,\hskip.2em plus .9em भामि \msCc}}% 

%Verse 11:3

{\devanagarifont किमन्यत्कथयिष्यामि दया यत्र न विद्यते {॥ ११:\hspace{.11em}३॥} \veg\dontdisplaylinenum }%
     \var{{\devanagarifontvar\numnoemph\vc किमन्य॰\lem \mssALL,\hskip.2em plus .9em किम्यन्य॰ \msNb}}% 

{\devanagarifont सदाशिवमुखात्पूर्वं श्रुतं मे वरसुन्दरि \thinspace{\dandab} \dontdisplaylinenum }%
 
%Verse 11:4

{\devanagarifont शृणु देवि प्रवक्ष्यामि धर्मसारमनुत्तमम् {॥ ११:\hspace{.11em}४॥} \veg\dontdisplaylinenum }%
     \var{{\devanagarifontvar\numemph\vc देवि प्रवक्ष्यामि\lem \msCb\msCc\msNa\msNb,\hskip.2em plus .9em ते देवि वक्ष्यामि \msCa\msNc\Ed}}% 
    \var{{\devanagarifontvar\numnoemph\vd ॰सारमनुत्तमम्\lem \mssALL,\hskip.2em plus .9em ॰सारसमुच्चयम् \msCc}}% 

\pend
\endnumbering
\vfill\pagebreak\beginnumbering\pstart
\vers


\alalfejezet{गृहस्थः}
{\devanagarifont विनार्थेन तु यो यज्ञः स यज्ञः सार्वकामिकः \thinspace{\dandab} \dontdisplaylinenum }%
     \var{{\devanagarifontvar\numemph\vb यज्ञः\lem \mssALL,\hskip.2em plus .9em यज्ञ \Ed\oo 
 सार्वकामिकः\lem \msCb\Ed,\hskip.2em plus .9em सर्वकालिकः \msCa\msNc,\hskip.5em plus .9em 
सर्वकामिक \msCc,\hskip.5em plus .9em सार्वकालिकः \msNa,\hskip.5em plus .9em सार्वकामिकाः \msNb}}% 
    \paral{{\devanagarifontsmall \vab {\englishfont See a sequence or list of the four āśramas in 4.75 above:}
                 गृहस्थो ब्रह्मचारी च वानप्रस्थो ऽथ भैक्षुकः;
                 {\englishfont see also 5.9:} 
                 एतच्छौचं गृहस्थानां द्विगुणं ब्रह्मचारिणाम्\thinspace{\devanagarifontsmall ।}
                 वानप्रस्थस्य त्रिगुणं यतीनां तु चतुर्गुणम्\thinspace{\devanagarifontsmall ॥}. }}

%Verse 11:5

{\devanagarifont अक्षयश्चाव्ययश्चैव सर्वपातकनाशनः {॥ ११:\hspace{.11em}५॥} \veg\dontdisplaylinenum }%
     \var{{\devanagarifontvar\numnoemph\vc अक्षयश्चाव्ययश्\lem \msCb\msNb\msNc\Ed,\hskip.2em plus .9em अक्षयं चाव्ययं \msCa\msCc\msNa}}% 
    \var{{\devanagarifontvar\numnoemph\vd ॰नाशनः\lem \msCa\msNa\msNb\msNc,\hskip.2em plus .9em ॰नाशनम् \msCb\Ed,\hskip.5em plus .9em ॰नाशन \msCc}}% 

{\devanagarifont बहुविघ्नकरो ह्यर्थो बह्वायासकरस्तथा \thinspace{\dandab} \dontdisplaylinenum }%
     \var{{\devanagarifontvar\numemph\va ॰करो\lem \mssALL,\hskip.2em plus .9em ॰करा \msCc\Ed\oo 
 ह्यर्थो\lem \mssALL,\hskip.2em plus .9em ह्येर्थो \Ed}}% 
    \var{{\devanagarifontvar\numnoemph\vb करस्तथा\lem \mssALL,\hskip.2em plus .9em करतस्था \Ed}}% 

%Verse 11:6

{\devanagarifont ब्रह्महत्या इवेन्द्रस्य प्रविभागफला स्मृता {॥ ११:\hspace{.11em}६॥} \veg\dontdisplaylinenum }%
     \var{{\devanagarifontvar\numnoemph\vd प्रविभाग॰\lem \msCb,\hskip.2em plus .9em प्रविभोग॰ \msCa\msCc(?)\msNa\msNc\Ed,\hskip.5em plus .9em प्रतिभोग॰ \msNb\oo 
 ॰फला स्मृता\lem \msCc,\hskip.2em plus .9em ॰फलः स्मृतः \msCapcorr\msCb\msNa\msNb\msNc,\hskip.5em plus .9em 
॰फल स्मृतः \msCaacorr,\hskip.5em plus .9em ॰प्रदः स्मृतः \Ed}}% 

{\devanagarifont पञ्चशोध्येन शोध्येत अर्थयज्ञो वरानने \thinspace{\dandab} \dontdisplaylinenum }%
     \var{{\devanagarifontvar\numemph\vb ॰यज्ञो\lem \mssALL,\hskip.2em plus .9em ॰यज्ञ \msCc}}% 

%Verse 11:7

{\devanagarifont शोधिते तु फलं शुद्धमशुद्धे निष्फलं भवेत् {॥ ११:\hspace{.11em}७॥} \veg\dontdisplaylinenum }%
     \var{{\devanagarifontvar\numnoemph\vcd शुद्धमशुद्धे\lem \mssALL,\hskip.2em plus .9em 
शुद्धंमशुद्धे \msNa,\hskip.5em plus .9em शुद्धमशुद्धं \Ed}}% 

{\devanagarifont देव्युवाच {\dandab}\dontdisplaylinenum  }%
     \var{{\devanagarifontvar\numemph\vo देव्युवाच\lem \mssALL,\hskip.2em plus .9em \om\ \msNbacorr}}% 

{\devanagarifont पञ्चशोध्ये सुरश्रेष्ठ संशयो ऽत्र भवेन्मम \thinspace{\danda} \dontdisplaylinenum }%
     \var{{\devanagarifontvar\numnoemph\va ॰शोध्ये\lem \mssCaCbCc\msNa,\hskip.2em plus .9em ॰शोध्य \msNb\msNc,\hskip.5em plus .9em ॰शोध्यः \Ed\oo 
 ॰श्रेष्ठ\lem \mssALL,\hskip.2em plus .9em ॰स्रे\uncl{म्न} \msCc}}% 
    \var{{\devanagarifontvar\numnoemph\vb ऽत्र भवे॰\lem \mssALL,\hskip.2em plus .9em ऽत्रा भव॰ \Ed}}% 

%Verse 11:8

{\devanagarifont कथयस्व विभागेन श्रोतुमिच्छामि तत्त्वतः {॥ ११:\hspace{.11em}८॥} \veg\dontdisplaylinenum }%
 
{\devanagarifont रुद्र उवाच {\dandab}\dontdisplaylinenum  }%
 
{\devanagarifont मनःशुद्धिस्तु प्रथमं द्रव्यशुद्धिरतः परम् \thinspace{\danda} \dontdisplaylinenum }%
     \var{{\devanagarifontvar\numemph\vb ॰शुद्धिरतः\lem \mssALL,\hskip.2em plus .9em ॰शुद्धिगतः \msNb}}% 

{\devanagarifont मन्त्रशुद्धिस्तृतीया तु कर्मशुद्धिरतः परम्  \danda\dontdisplaylinenum }%
     \var{{\devanagarifontvar\numnoemph\vc मन्त्रशुद्धिस्तृतीया\lem \mssALL,\hskip.2em plus .9em मन्त्रद्धि तृतीया \msNc}}% 
    \var{{\devanagarifontvar\numnoemph\vd कर्मशुद्धि॰\lem \mssALL,\hskip.2em plus .9em कर्मसिद्धि \msNc}}% 

\pend
\endnumbering
\vfill\pagebreak\beginnumbering\pstart
\vers

%Verse 11:9

{\devanagarifont पञ्चमी सत्त्वशुद्धिस्तु क्रतुशुद्धिश्च पञ्चधा {॥ ११:\hspace{.11em}९॥} \veg\dontdisplaylinenum }%
     \var{{\devanagarifontvar\numnoemph\ve पञ्चमी\lem \mssALL,\hskip.2em plus .9em पञ्चमं \Ed\oo 
 ॰शुद्धिस्तु\lem \mssALL,\hskip.2em plus .9em ॰शुद्धिश्च \msNa\Ed}}% 
    \var{{\devanagarifontvar\numnoemph\vf ॰शुद्धिश्च पञ्चधा\lem \mssALL,\hskip.2em plus .9em ॰शुद्धिस्तु पञ्चधा \msCc,\hskip.5em plus .9em 
॰शुद्धिरतः परम् \msNa}}% 

{\devanagarifont मनःशुद्धिर्नाम अविपरीतभावनया \thinspace{\dandab} \dontdisplaylinenum  }%
     \var{{\devanagarifontvar\numemph\vab ॰शुद्धिर्ना॰\lem \mssALL,\hskip.2em plus .9em ॰शुद्धि ना॰ \msCc\oo 
 ॰भावनया\lem \mssALL,\hskip.2em plus .9em ॰भावनवा \msNa,\hskip.5em plus .9em ॰भावनतया \msNb}}% 

%Verse 11:10

{\devanagarifont द्रव्यशुद्धिर्नाम अनन्यायोपार्जितद्रव्येन {॥ ११:\hspace{.11em}१०॥} \veg\dontdisplaylinenum  }%
     \var{{\devanagarifontvar\numnoemph\vcd ॰शुद्धिर्ना॰\lem \mssALL,\hskip.2em plus .9em ॰शुद्धि ना॰ \msCc\msNc\oo 
 अनन्यायो॰\lem \msCb\msNa\msNb\msNc,\hskip.2em plus .9em अन\lacwithnum{1}  यो॰ \msCa,\hskip.5em plus .9em अन्यायो॰ \msCc,\hskip.5em plus .9em स्वल्पोन्यायो॰ \Ed\oo 
 ॰द्रव्येन\lem \mssALL,\hskip.2em plus .9em ॰व्येन \msNb}}% 

{\devanagarifont मन्त्रशुद्धिर्नाम स्वरव्यञ्जनयुक्ततया \thinspace{\dandab} \dontdisplaylinenum  }%
     \var{{\devanagarifontvar\numemph\vab मन्त्रशुद्धिर्ना॰\lem \msCa\msCb\msNb\Ed,\hskip.2em plus .9em मन्त्रशुद्धि ना॰ \msCc\msNc,\hskip.5em plus .9em 
मन्त्रस्तुद्दिना॰ \msNa\oo 
 ॰युक्ततया\lem \mssALL,\hskip.2em plus .9em ॰युक्तया \msCb}}% 

{\devanagarifont क्रियाशुद्धिर्नाम यथाक्रमाविपरीततया  \danda\dontdisplaylinenum  }%
     \var{{\devanagarifontvar\numnoemph\vcd ॰शुद्धिर्ना॰\lem \mssALL,\hskip.2em plus .9em ॰शुद्धि ना॰ \msCc\msNb\oo 
 ॰क्रमा॰\lem \mssALL,\hskip.2em plus .9em ॰क्रम॰ \msCc\oo 
 ॰रीततया\lem \mssALL,\hskip.2em plus .9em ॰रीतया \msCb,\hskip.5em plus .9em \lacwithnum{2}  तया \msNc}}% 

%Verse 11:11

{\devanagarifont सत्त्वशुद्धिर्नाम रजस्तम-अप्रधानतया {॥ ११:\hspace{.11em}११॥} \veg\dontdisplaylinenum  }%
     \var{{\devanagarifontvar\numnoemph\vef ॰शुद्धिर्ना॰\lem \mssALL,\hskip.2em plus .9em ॰शुद्धि ना॰ \msCa\msCc\oo 
 ॰धानतया\lem \mssALL,\hskip.2em plus .9em ॰धानत \msNc}}% 

\vers


{\devanagarifont विधिमेवं यदा शुध्येद्यदि यज्ञं करोति हि \thinspace{\dandab} \dontdisplaylinenum }%
     \var{{\devanagarifontvar\numemph\va ॰धिमेवं यदा\lem \msCb\Ed,\hskip.2em plus .9em ॰धिमेव यदा \msCa\msCc\msNa,\hskip.5em plus .9em ॰धिमेव य \msNb,\hskip.5em plus .9em 
॰धिमेवं यथा \msNc}}% 
    \var{{\devanagarifontvar\numnoemph\vab शुध्येद्यदि\lem \conj,\hskip.2em plus .9em सूयेद्यदि \msCa\msNa,\hskip.5em plus .9em पूर्य यदि \msCb,\hskip.5em plus .9em 
सूर्येद्यदि \msCc,\hskip.5em plus .9em सूयेद्यति \msNb,\hskip.5em plus .9em पूयेद्यदि \msNc,\hskip.5em plus .9em शूद्ध्य यदि \Ed}}% 
    \var{{\devanagarifontvar\numnoemph\vb यज्ञं\lem \msCa\msCb\msNa\Ed,\hskip.2em plus .9em यज्ञ \msCc\msNc,\hskip.5em plus .9em संज्ञ \msNb\oo 
 हि\lem \mssALL,\hskip.2em plus .9em \om\ \msNb}}% 

%Verse 11:12

{\devanagarifont तस्य यज्ञफलावाप्तिर्जन्ममृत्युश्च नो भवेत् {॥ ११:\hspace{.11em}१२॥} \veg\dontdisplaylinenum }%
     \var{{\devanagarifontvar\numnoemph\vcd ॰वाप्तिर्ज॰\lem \msCa\msCb\Ed,\hskip.2em plus .9em ॰वाप्ति ज \msCc\msNb\msNc,\hskip.5em plus .9em ॰वापि ज॰ \msNa}}% 

{\devanagarifont विनार्थेन तु यो यज्ञं करोति वरसुन्दरि \thinspace{\dandab} \dontdisplaylinenum }%
     \var{{\devanagarifontvar\numemph\vb ॰सुन्दरि\lem \mssALL,\hskip.2em plus .9em ॰सुन्दरी \Ed}}% 

%Verse 11:13

{\devanagarifont न तस्य तत्फलावाप्तिः सर्वयज्ञेष्वशेषतः {॥ ११:\hspace{.11em}१३॥} \veg\dontdisplaylinenum }%
     \var{{\devanagarifontvar\numnoemph\vd ॰यज्ञेष्वशेषतः\lem \mssALL,\hskip.2em plus .9em ॰यज्ञेषु शेषतः \Ed}}% 

{\devanagarifont यज्ञवाट कुरुक्षेत्रं सत्त्वावासकृतालयः \thinspace{\dandab} \dontdisplaylinenum }%
     \var{{\devanagarifontvar\numemph\va ॰वाट कुरु॰\lem \mssALL,\hskip.2em plus .9em ॰वाटङ्कुरु॰ \msCb,\hskip.5em plus .9em ॰वाटकृत॰ \Ed\oo 
 ॰क्षेत्रं\lem \mssALL,\hskip.2em plus .9em ॰क्षेत्र \msNc}}% 
    \var{{\devanagarifontvar\numnoemph\vb सत्त्वा॰\lem \mssALL,\hskip.2em plus .9em सत्वासत्वा॰ \msCbacorr\oo 
 ॰लयः\lem \mssALL,\hskip.2em plus .9em ॰लयम् \msCc}}% 

%Verse 11:14

{\devanagarifont प्रत्याहार महावेदि कुशप्रस्तर संयमः {॥ ११:\hspace{.11em}१४॥} \veg\dontdisplaylinenum }%
     \var{{\devanagarifontvar\numnoemph\vc ॰वेदि\lem \mssALL,\hskip.2em plus .9em ॰देवि \Ed}}% 

{\devanagarifont विधि नियमविस्तारो ध्यानवह्निः प्रदीपितः \thinspace{\dandab} \dontdisplaylinenum }%
     \var{{\devanagarifontvar\numemph\va विधि नि॰\lem \mssALL,\hskip.2em plus .9em विधिर्नि॰ \Ed\oo 
 ॰विस्तारो\lem \mssALL,\hskip.2em plus .9em ॰विस्तारौ \msCb}}% 
    \var{{\devanagarifontvar\numnoemph\vb ध्यानवह्निः प्रदीपितः\lem \msNc,\hskip.2em plus .9em ध्यानवह्निप्रदीपितः \msCa\msNa,\hskip.5em plus .9em 
ध्यानं वह्निप्रदीपितः \msCb,\hskip.5em plus .9em ध्यानमग्निप्रदीपितः \msCc,\hskip.5em plus .9em 
ध्यान अग्निप्रदीपनः \msNb,\hskip.5em plus .9em ध्यानवृद्धिर्प्रदीपिनः \Ed}}% 

%Verse 11:15

{\devanagarifont योगेन्धनसमिज्ज्वालतपोधूमसमाकुलः {॥ ११:\hspace{.11em}१५॥} \veg\dontdisplaylinenum }%
     \var{{\devanagarifontvar\numnoemph\vcd ॰न्धनसमिज्ज्वालतपोधूम॰\lem \msNb\msNc,\hskip.2em plus .9em ॰न्धनसमिज्ज्वालतपोधूप॰ \msCa,\hskip.5em plus .9em 
॰\uncl{न्ध}$\-$सत्वमिज्ज्वालतपोधूम॰ \msCb,\hskip.5em plus .9em ॰न्धनसमिज्वालतपोधूम॰ \msCc,\hskip.5em plus .9em 
॰न्धनशमि\uncl{त}ज्वाल$\-$तयोधूय॰ \msNa,\hskip.5em plus .9em ॰न्धनसमिज्ज्वाला तपोधूम॰ \Ed}}% 

{\devanagarifont पात्रन्यास शिवज्ञानं स्थालीपाक शिवात्मकः \thinspace{\dandab} \dontdisplaylinenum }%
     \var{{\devanagarifontvar\numemph\va पात्र॰\lem \mssALL,\hskip.2em plus .9em पात्रा॰ \msNc}}% 

%Verse 11:16

{\devanagarifont आज्याहुतिमविच्छिन्नं लम्बकस्रुवपातितः {॥ ११:\hspace{.11em}१६॥} \veg\dontdisplaylinenum }%
     \var{{\devanagarifontvar\numnoemph\vc ॰च्छिन्नं\lem \mssALL,\hskip.2em plus .9em ॰च्छिन्न \msNc}}% 
    \var{{\devanagarifontvar\numnoemph\vd लम्बक॰\lem \mssALL,\hskip.2em plus .9em \uncl{ल}म्बक॰ \msCc,\hskip.5em plus .9em त्र्यम्बक॰ \Ed\oo 
 ॰पातितः\lem \mssALL,\hskip.2em plus .9em ॰पातितम् \Ed}}% 

{\devanagarifont धारणाध्वर्युवत्कृत्वा प्राणायामश्च ऋत्विजः \thinspace{\dandab} \dontdisplaylinenum }%
     \var{{\devanagarifontvar\numemph\va ॰ध्वर्युव॰\lem \msNb,\hskip.2em plus .9em ॰ध्वर्यव॰ \mssCaCbCc,\hskip.5em plus .9em ॰\uncl{ध्व}र्यव॰ \msNa,\hskip.5em plus .9em 
ध्व\lk\lk\ \msNc,\hskip.5em plus .9em धर्मव॰ \Ed}}% 

%Verse 11:17

{\devanagarifont तर्कयुक्तः सविस्तारः समाधिर्वयतापनः {॥ ११:\hspace{.11em}१७॥} \veg\dontdisplaylinenum }%
     \var{{\devanagarifontvar\numnoemph\vc ॰युक्तः\lem \mssALL,\hskip.2em plus .9em ॰युक्त \msCc,\hskip.5em plus .9em ॰युक्तिः \msNa\oo 
 ॰विस्तारः\lem \mssALL,\hskip.2em plus .9em ॰विस्तारो \msCc}}% 

{\devanagarifont ब्रह्मविद्यामयो यूपः पशुबन्धो मनोन्मनः \thinspace{\dandab} \dontdisplaylinenum }%
     \var{{\devanagarifontvar\numemph\vb ॰न्मनः\lem \msCa\msNa\msNb\Ed,\hskip.2em plus .9em ॰त्मनः \msCb\msCc\msNc}}% 

%Verse 11:18

{\devanagarifont श्रद्धा पत्नी विशालाक्षि संकल्प पद शाश्वतम् {॥ ११:\hspace{.11em}१८॥} \veg\dontdisplaylinenum }%
     \var{{\devanagarifontvar\numnoemph\vc पत्नी\lem \mssALL,\hskip.2em plus .9em \uncl{पत्नी} \msCa\oo 
 विशालाक्षि\lem \mssALL,\hskip.2em plus .9em विशालाक्षी \msNc\Ed}}% 
    \var{{\devanagarifontvar\numnoemph\vd पद शाश्वतम्\lem \mssALL,\hskip.2em plus .9em प\uncl{द}\lacwithnum{1}  श्वतम् \msCa}}% 

{\devanagarifont पञ्चेन्द्रियजयोत्पन्नः पुरोडाशो ऽमृताशनः \thinspace{\dandab} \dontdisplaylinenum }%
     \var{{\devanagarifontvar\numemph\vb ॰डाशो\lem \mssCaCbCc\msNb\msNc,\hskip.2em plus .9em ॰भा \msNaacorr,\hskip.5em plus .9em ॰भासे \msNapcorr,\hskip.5em plus .9em ॰भागे \Ed\oo 
 मृता॰\lem \mssALL,\hskip.2em plus .9em मृगा॰ \msCc}}% 

%Verse 11:19

{\devanagarifont ब्रह्मनादो महामन्त्रः प्रायश्चित्तानिलो जयः {॥ ११:\hspace{.11em}१९॥} \veg\dontdisplaylinenum }%
     \var{{\devanagarifontvar\numnoemph\vd ॰त्तानिलो\lem \mssALL,\hskip.2em plus .9em ॰त्तनिलो \msCc\msNb\oo 
 जयः\lem \mssALL,\hskip.2em plus .9em जलाः \Ed}}% 

{\devanagarifont सोमपान परिज्ञानमुपाकर्म चतुर्यमः \thinspace{\dandab} \dontdisplaylinenum }%
     \var{{\devanagarifontvar\numemph\va परि॰\lem \mssALL,\hskip.2em plus .9em पर॰ \msCc}}% 

%Verse 11:20

{\devanagarifont इतिहास जलस्नानं पुराणकृतमम्बरः {॥ ११:\hspace{.11em}२०॥} \veg\dontdisplaylinenum }%
     \var{{\devanagarifontvar\numnoemph\vc ॰स्नानं\lem \mssALL,\hskip.2em plus .9em ॰स्नान \msCb}}% 
    \var{{\devanagarifontvar\numnoemph\vd पुराण॰\lem \mssALL,\hskip.2em plus .9em पुराणं \Ed\oo 
 ॰कृतमम्बरः\lem \mssALL,\hskip.2em plus .9em ॰कृतम्बरम् \msCb\ \unmetr}}% 

{\devanagarifont इडासुषुम्नासंवेद्ये स्नानमाचमनं सकृत् \thinspace{\dandab} \dontdisplaylinenum }%
     \var{{\devanagarifontvar\numemph\va ॰सुषुम्ना॰\lem \mssALL,\hskip.2em plus .9em ॰सुषुम्न॰ \msCc\oo 
 ॰वेद्ये\lem  \msCa\Ed,\hskip.2em plus .9em ॰वेद्य \msCb\msNb,\hskip.5em plus .9em ॰वेद्येः \msCc,\hskip.5em plus .9em ॰वैद्य \msNa,\hskip.5em plus .9em ॰भेदो \msNc}}% 
    \var{{\devanagarifontvar\numnoemph\vb सकृत्\lem \mssALL,\hskip.2em plus .9em विदुः \msCc}}% 

%Verse 11:21

{\devanagarifont संतोषातिथिमादृत्य दयाभूतद्विजार्चितः {॥ ११:\hspace{.11em}२१॥} \veg\dontdisplaylinenum }%
     \var{{\devanagarifontvar\numnoemph\vc ॰तोषातिथिमादृत्य\lem \mssALL,\hskip.2em plus .9em ॰तोषतिथिमावृत्य \msNb}}% 
    \var{{\devanagarifontvar\numnoemph\vd ॰द्विजा॰\lem \mssALL,\hskip.2em plus .9em ॰दया॰ \msCb}}% 

{\devanagarifont ब्रह्मकूर्च गुणातीत हविर्गन्ध निरञ्जनः \thinspace{\dandab} \dontdisplaylinenum }%
     \var{{\devanagarifontvar\numemph\vb ॰हविर्ग॰\lem \mssALL,\hskip.2em plus .9em ॰हवि\uncl{र्ग}॰ \msCb,\hskip.5em plus .9em ॰हविग \msNa}}% 

%Verse 11:22

{\devanagarifont ब्रह्मसूत्रं त्रयस्तत्त्वं बोधना मुण्डितं शिरः {॥ ११:\hspace{.11em}२२॥} \veg\dontdisplaylinenum }%
     \var{{\devanagarifontvar\numnoemph\vc ॰सूत्रं त्रयस्\lem \msCb\msNb\msNc\Ed,\hskip.2em plus .9em ॰सूत्रन्त्रयस्तयस् \msCa,\hskip.5em plus .9em 
॰सूत्रं त्रय \msCc,\hskip.5em plus .9em ॰सूत्रत्रयं \msNa}}% 
    \var{{\devanagarifontvar\numnoemph\vd मुण्डितं\lem \mssALL,\hskip.2em plus .9em मुण्डित॰ \msCb\msNc\unmetr}}% 

{\devanagarifont निवृत्त्यादि चतुर्वेदश्चतुःप्रकरणासनः \thinspace{\dandab} \dontdisplaylinenum }%
     \var{{\devanagarifontvar\numemph\va निवृत्त्या॰\lem \eme,\hskip.2em plus .9em निवृत्या॰ \mssCaCbCc\msNa\msNb\msNc,\hskip.5em plus .9em निर्वृत्या॰ \Ed}}% 
    \var{{\devanagarifontvar\numnoemph\vb ॰प्रकरणासनः\lem \mssALL,\hskip.2em plus .9em 
प्रकरनाशनः \msCc,\hskip.5em plus .9em प्रकरशासनः \Ed}}% 

%Verse 11:23

{\devanagarifont दक्षिणामभयं भूते दत्त्वा यज्ञं यजेत्सदा {॥ ११:\hspace{.11em}२३॥} \veg\dontdisplaylinenum }%
     \var{{\devanagarifontvar\numnoemph\vc ॰भयं भूते\lem \mssALL,\hskip.2em plus .9em ॰भक्षयम्भूतै \msCb}}% 
    \var{{\devanagarifontvar\numnoemph\vd यज्ञं यजेत्\lem \mssALL,\hskip.2em plus .9em यज्ञ ददत् \Ed}}% 
    \paral{{\devanagarifontsmall \vc {\englishfont \compare\ \VSS\ 22.14ab:} दक्षिणाभय भूतेभ्यः पशुबन्धः स्वयंकृतः }}

{\devanagarifont विनार्थं यज्ञसम्प्राप्तिः कथिता ते वरानने \thinspace{\dandab} \dontdisplaylinenum }%
     \var{{\devanagarifontvar\numemph\va विनार्थं\lem \mssALL,\hskip.2em plus .9em विनार्थ \msCc}}% 
    \var{{\devanagarifontvar\numnoemph\vb कथिता ते\lem \mssALL,\hskip.2em plus .9em 
कथि\uncl{तो} स्मि \msCc,\hskip.5em plus .9em कथितस्ते \Ed\oo 
 वरानने\lem \mssALL,\hskip.2em plus .9em व\uncl{रा}नने \msCc}}% 

%Verse 11:24

{\devanagarifont आसहस्रस्य यज्ञानां फलं प्राप्नोति नित्यशः {॥ ११:\hspace{.11em}२४॥} \veg\dontdisplaylinenum }%
     \var{{\devanagarifontvar\numnoemph\vd प्राप्नोति\lem \mssALL,\hskip.2em plus .9em प्रा\lacwithnum{1}  ति \msCa\oo 
 नित्यशः\lem \mssALL,\hskip.2em plus .9em मानवः \msNb}}% 

{\devanagarifont आश्रमः प्रथमस्तुभ्यं कथितो ऽस्ति वरानने \thinspace{\dandab} \dontdisplaylinenum }%
     \var{{\devanagarifontvar\numemph\va आश्रमः\lem \mssALL,\hskip.2em plus .9em आश्रम \msCb\msCc\oo 
 ॰स्तुभ्यं\lem \mssALL,\hskip.2em plus .9em ॰स्येष \msCc,\hskip.5em plus .9em ॰स्यैवं \Ed}}% 
    \var{{\devanagarifontvar\numnoemph\vb ऽस्ति\lem \msCa\msCb\msNa\msNc,\hskip.2em plus .9em स्मि \msCc\msNb\Ed}}% 

%Verse 11:25

{\devanagarifont सदाशिवेन सद्धर्मं दैवतैरपि पूजितम् {॥ ११:\hspace{.11em}२५॥} \veg\dontdisplaylinenum }%
     \var{{\devanagarifontvar\numnoemph\vc ॰धर्मं\lem \mssALL,\hskip.2em plus .9em ॰ध\uncl{र्मं} \msCb,\hskip.5em plus .9em ॰धर्मे \Ed}}% 
    \var{{\devanagarifontvar\numnoemph\vd दैव॰\lem \mssALL,\hskip.2em plus .9em देव॰ \msNb\Ed\oo 
 पूजितम्\lem \mssALL,\hskip.2em plus .9em पूपूजितम् \msCb}}% 


\alalfejezet{ब्रह्मचारी}
{\devanagarifont ब्रह्मचर्यं निबोधेदं शृणुष्वावहिता शुभे \thinspace{\dandab} \dontdisplaylinenum }%
     \var{{\devanagarifontvar\numemph\va ॰चर्यं\lem \mssALL,\hskip.2em plus .9em ॰चर्य \msNa}}% 
    \var{{\devanagarifontvar\numnoemph\vb ॰वहिता शुभे\lem \mssALL,\hskip.2em plus .9em 
॰वहितो भव \msCc,\hskip.5em plus .9em ॰वहितो शुभे \msNb}}% 

%Verse 11:26

{\devanagarifont द्वितीयमाश्रमं देवि सर्वपापविनाशनम् {॥ ११:\hspace{.11em}२६॥} \veg\dontdisplaylinenum }%
     \var{{\devanagarifontvar\numnoemph\vd ॰विनाशनम्\lem \mssALL,\hskip.2em plus .9em ॰प्रनाशनम् \msNb}}% 
    \paral{{\devanagarifontsmall \vcd {\englishfont  \compare\ \MBH\ 12.184.10A:} गार्हस्थ्यं खलु द्वितीयमाश्रमं वदन्ति }}

{\devanagarifont व्रतं ब्रह्मपरं ध्यानं सावित्री प्रकृतिर्लयम् \thinspace{\dandab} \dontdisplaylinenum }%
     \var{{\devanagarifontvar\numemph\va ॰परं ध्यानं\lem \mssALL,\hskip.2em plus .9em ॰परिज्ञानं \Ed}}% 
    \var{{\devanagarifontvar\numnoemph\vb ॰कृतिर्लयम्\lem \msCa\msNa\msNc\Ed,\hskip.2em plus .9em ॰कृतालयम् \msCb,\hskip.5em plus .9em ॰कृतीलयम् \msCc,\hskip.5em plus .9em ॰कृतिलः \msNb}}% 
    \paral{{\devanagarifontsmall \vab {\englishfont \compare\ \VSS\ 16.8cd} }}

%Verse 11:27

{\devanagarifont ब्रह्मसूत्राक्षरं सूक्ष्मं त्रिगुणालय मेखलम् {॥ ११:\hspace{.11em}२७॥} \veg\dontdisplaylinenum }%
     \var{{\devanagarifontvar\numnoemph\vd ॰लय\lem \mssALL,\hskip.2em plus .9em ॰ल\lacwithnum{1}\  \msCa\oo 
 मेखलम्\lem \mssALL,\hskip.2em plus .9em यत्फलम् \Ed}}% 

{\devanagarifont दम दण्ड दया पात्रं भिक्षा संसारमोचनम् \thinspace{\dandab} \dontdisplaylinenum }%
     \var{{\devanagarifontvar\numemph\va दण्ड दया\lem \mssALL,\hskip.2em plus .9em दण्डादया \msNa,\hskip.5em plus .9em दण्डादयो \Ed\oo 
 पात्रं\lem \mssALL,\hskip.2em plus .9em पात्र \msNb}}% 

%Verse 11:28

{\devanagarifont त्र्यायुषं द्व्यक्षरातीतं ज्ञानभस्म-अलङ्कृतम् {॥ ११:\hspace{.11em}२८॥} \veg\dontdisplaylinenum }%
     \var{{\devanagarifontvar\numnoemph\vc ॰युषं\lem \mssALL,\hskip.2em plus .9em ॰युष \msNa}}% 
    \var{{\devanagarifontvar\numnoemph\vd भस्म\lem \mssALL,\hskip.2em plus .9em भष्मम् \Ed}}% 

{\devanagarifont स्नानव्रतं सदासत्यं शीलशौचसमन्वितम् \thinspace{\dandab} \dontdisplaylinenum }%
     \var{{\devanagarifontvar\numemph\va ॰व्रतं\lem \msCa\msCc\msNa\msNb,\hskip.2em plus .9em ॰व्रत \msCb\msNc\Ed}}% 

%Verse 11:29

{\devanagarifont अग्निहोत्र त्रयस्तत्त्वं जप ब्रह्मबिलस्वरः {॥ ११:\hspace{.11em}२९॥} \veg\dontdisplaylinenum }%
     \var{{\devanagarifontvar\numnoemph\vc ॰होत्र त्रयस्तत्त्वं\lem \msNa\msNc\Ed,\hskip.2em plus .9em ॰होत्रन्त्रयस्तत्वं \msCa,\hskip.5em plus .9em 
॰होत्र$\-$\uncl{त}यस्तत्वं \msCb,\hskip.5em plus .9em ॰होत्रत्रयं तत्वा \msCc,\hskip.5em plus .9em 
॰होत्रं त्रयंस्तत्वं \msNb}}% 
    \var{{\devanagarifontvar\numnoemph\vd ॰बिलस्वरः\lem \corr,\hskip.2em plus .9em ॰बिलश्वरः \mssCaCbCc\msNa\msNb,\hskip.5em plus .9em ॰बिलेश्वर \msNc\Ed}}% 

{\devanagarifont द्वितीय आश्रमो देवि यथाह भगवान्शिवः \thinspace{\dandab} \dontdisplaylinenum }%
     \var{{\devanagarifontvar\numemph\va द्वितीय आश्रमो\lem \mssALL,\hskip.2em plus .9em द्वितीयमाश्रमो \msCc,\hskip.5em plus .9em 
द्वितीयमाश्रमं \Ed}}% 
    \var{{\devanagarifontvar\numnoemph\vb यथाह\lem \msCa\msCb\msNa\msNc,\hskip.2em plus .9em यथाहं \msCc\msNb,\hskip.5em plus .9em यदाह \Ed}}% 

%Verse 11:30

{\devanagarifont ममापि कथितं तुभ्यं जन्ममृत्युविनाशनम् {॥ ११:\hspace{.11em}३०॥} \veg\dontdisplaylinenum }%
     \var{{\devanagarifontvar\numnoemph\vc ममापि कथितं तु॰\lem \mssALL,\hskip.2em plus .9em 
ममापि कथितस्तु॰ \msNc,\hskip.5em plus .9em मयापि कथितो तु॰ \Ed}}% 
    \var{{\devanagarifontvar\numnoemph\vd ॰मृत्यु॰\lem \mssALL,\hskip.2em plus .9em ॰मृ\lacwithnum{1}\  \msCa\oo 
 ॰नाशनं\lem \mssALL,\hskip.2em plus .9em ॰नाशनः \msNc}}% 


\alalfejezet{वानप्रस्थः}
{\devanagarifont वानप्रस्थविधिं वक्ष्ये शृणुष्वायतलोचने \thinspace{\dandab} \dontdisplaylinenum }%
     \var{{\devanagarifontvar\numemph\va ॰विधिं\lem \mssALL,\hskip.2em plus .9em ॰विधि \msCb}}% 

%Verse 11:31

{\devanagarifont यथाश्रुतं यथातथ्यमृषिदैवतपूजितम् {॥ ११:\hspace{.11em}३१॥} \veg\dontdisplaylinenum }%
     \var{{\devanagarifontvar\numnoemph\vd ॰दैवत॰\lem \mssALL,\hskip.2em plus .9em ॰देवत॰ \msCc}}% 

{\devanagarifont वैराग्यवनमाश्रित्य नियमाश्रममाहरेत् \thinspace{\dandab} \dontdisplaylinenum }%
     \var{{\devanagarifontvar\numemph\va वैराग्य॰\lem \mssALL,\hskip.2em plus .9em वैराग्या \Ed}}% 
    \var{{\devanagarifontvar\numnoemph\vb नियमा॰\lem \mssALL,\hskip.2em plus .9em मा॰ \msNaacorr\oo 
 ॰श्रममा॰\lem \mssALL,\hskip.2em plus .9em ॰श्रमनो हरेत् \msCa}}% 

%Verse 11:32

{\devanagarifont शीलशैलदृढद्वारे प्राकारे विजितेन्द्रियः {॥ ११:\hspace{.11em}३२॥} \veg\dontdisplaylinenum }%
     \var{{\devanagarifontvar\numnoemph\vc ॰दृढ॰\lem \mssALL,\hskip.2em plus .9em ॰दृष॰ \Ed}}% 
    \var{{\devanagarifontvar\numnoemph\vd ॰कारे\lem \mssALL,\hskip.2em plus .9em ॰कार॰ \msCc}}% 

{\devanagarifont अधिभूतः स्मृतो माता अध्यात्मश्च पिता तथा \thinspace{\dandab} \dontdisplaylinenum }%
     \var{{\devanagarifontvar\numemph\va स्मृतो\lem \mssALL,\hskip.2em plus .9em \lacwithnum{2}  \msCb,\hskip.5em plus .9em स्मृतौ \Ed}}% 
    \paral{{\devanagarifontsmall \vab {\englishfont \compare\ \VSS\ 22.10ab:} अध्यात्मनगरस्फीतः अधिभूतजनाकुलः }}

%Verse 11:33

{\devanagarifont अधिदैविकमाचार्यो व्यवसायाश्च भ्रातरः {॥ ११:\hspace{.11em}३३॥} \veg\dontdisplaylinenum }%
     \var{{\devanagarifontvar\numnoemph\vc अधिदैविक॰\lem \emeGoodall,\hskip.2em plus .9em 
\uncl{अ}\lacwithnum{1} \uncl{भौ}\lacwithnum{1}  क॰ \msCa,\hskip.5em plus .9em 
अधिभौतिक॰ \msCb\msCc\msNa\msNc\Ed,\hskip.5em plus .9em 
अधिभौक्तिक॰ \msNb}}% 
    \var{{\devanagarifontvar\numnoemph\vd व्यवसायाश्च\lem \mssALL,\hskip.2em plus .9em व्यवसायश्च \Ed}}% 

{\devanagarifont श्रुतिः स्मृतिः स्मृता भार्या प्रज्ञा पुत्रः क्षमानुजः \thinspace{\dandab} \dontdisplaylinenum }%
     \var{{\devanagarifontvar\numemph\va स्मृता\lem \mssALL,\hskip.2em plus .9em स्मृतो \msCb}}% 

{\devanagarifont मैत्री बन्धुर्जटा चापं करुणा सुपवित्रकम्  \danda\dontdisplaylinenum }%
     \var{{\devanagarifontvar\numnoemph\vc बन्धुर्ज॰\lem \mssALL,\hskip.2em plus .9em बन्धु ज॰ \msCc\msNb}}% 

%Verse 11:34

{\devanagarifont मुदिता मौन चत्वारः सर्वकार्यमुपेक्षका {॥ ११:\hspace{.11em}३४॥} \veg\dontdisplaylinenum }%
     \var{{\devanagarifontvar\numnoemph\ve मौन चत्वारः\lem \mssALL,\hskip.2em plus .9em 
मौनश्चत्वारः \msCb,\hskip.5em plus .9em मौन चत्वार \msCc}}% 
    \var{{\devanagarifontvar\numnoemph\vf ॰कार्यमु॰\lem \mssALL,\hskip.2em plus .9em ॰कार्यामु॰ \msNa\oo 
 ॰पेक्षका\lem \mssALL,\hskip.2em plus .9em ॰पेक्षया \Ed}}% 

{\devanagarifont यमवल्कलसंवीतस्तपःकृष्णाजिनाधरः \thinspace{\dandab} \dontdisplaylinenum }%
     \var{{\devanagarifontvar\numemph\va ॰संवीत॰\lem \mssALL,\hskip.2em plus .9em ॰सान्वीत॰ \Ed}}% 
    \var{{\devanagarifontvar\numnoemph\vb ॰कृष्णा॰\lem \mssALL,\hskip.2em plus .9em ॰कृष्णां \msCc\oo 
 ॰जिनाधरः\lem \msNc,\hskip.2em plus .9em ॰जिनधरः \mssCaCbCc\msNa\msNb\ \unmetr,\hskip.5em plus .9em ॰जिनं पुरः \Ed}}% 

%Verse 11:35
 
{\devanagarifont उत्तरासङ्गमासीनो योगपट्टदृढव्रतः {॥ ११:\hspace{.11em}३५॥} \veg\dontdisplaylinenum }%
     \var{{\devanagarifontvar\numnoemph\vd ॰दृढ॰\lem \mssALL,\hskip.2em plus .9em ॰दृष्ट॰ \msNb\oo 
 ॰व्रतः\lem \mssALL,\hskip.2em plus .9em \lacwithnum{2}  \msCa}}% 

{\devanagarifont वेदघोषेण घोषेण प्राणायामो ऽग्निहावनम् \thinspace{\dandab} \dontdisplaylinenum }%
     \var{{\devanagarifontvar\numemph\va वेद॰\lem \mssALL,\hskip.2em plus .9em \lacwithnum{1}  द॰ \msCa\oo 
 ॰ण घोषेण\lem \mssALL,\hskip.2em plus .9em ॰ण घोषीण \msCc}}% 
    \var{{\devanagarifontvar\numnoemph\vb ॰हावनम्\lem \mssALL,\hskip.2em plus .9em ॰\uncl{हावनम्} \msCb,\hskip.5em plus .9em ॰हावन \msCc}}% 

%Verse 11:36

{\devanagarifont जितप्राण मृगाकूलो धृति यज्ञः क्रिया जपः {॥ ११:\hspace{.11em}३६॥} \veg\dontdisplaylinenum }%
     \var{{\devanagarifontvar\numnoemph\vd ॰जपः\lem \mssALL,\hskip.2em plus .9em ॰जिणः \msCc}}% 

{\devanagarifont अर्थसंग्रह शास्त्रेषु सखा दमदयादयः \thinspace{\dandab} \dontdisplaylinenum }%
     \var{{\devanagarifontvar\numemph\vb सखा\lem \mssALL,\hskip.2em plus .9em सखो \msNb\oo 
 दमद॰\lem \mssALL,\hskip.2em plus .9em 
दम॰ \msCaacorr,\hskip.5em plus .9em दयद॰ \msCc}}% 

%Verse 11:37

{\devanagarifont शिवयज्ञं प्रयुञ्जीत साधनाष्टकपूजनम् {॥ ११:\hspace{.11em}३७॥} \veg\dontdisplaylinenum }%
     \var{{\devanagarifontvar\numnoemph\vc ॰यज्ञं\lem \mssALL,\hskip.2em plus .9em ॰यज्ञ \msCc\msNc}}% 
    \var{{\devanagarifontvar\numnoemph\vd ॰पूजनम्\lem \mssALL,\hskip.2em plus .9em ॰पूजिकं \msCc}}% 
    \paral{{\devanagarifontsmall \vd {\englishfont \compare\ \DHARMP\ 2.1:} 
                 अष्टभिः साधनैरेभिश्चित्तं कायञ्च यत्नतः\thinspace{\devanagarifontsmall ।}
                 शोधयित्वा ततो योगी योगाभ्यासं समाचरेत्\thinspace{\devanagarifontsmall ॥} }}

{\devanagarifont पञ्चब्रह्मजलैः पूतः सत्यतीर्थशिवह्रदे \thinspace{\dandab} \dontdisplaylinenum }%
     \var{{\devanagarifontvar\numemph\va ॰ब्रह्मजलैः पूतः\lem \mssALL,\hskip.2em plus .9em ब्र\lacwithnum{5}  \msNb}}% 
    \var{{\devanagarifontvar\numnoemph\vb ॰तीर्थ॰\lem \mssALL,\hskip.2em plus .9em ॰तीर्थं \Ed}}% 

%Verse 11:38

{\devanagarifont स्नानमाचमनं कृत्वा संध्यात्रयमुपासयेत् {॥ ११:\hspace{.11em}३८॥} \veg\dontdisplaylinenum }%
     \var{{\devanagarifontvar\numnoemph\vc ॰चमनं\lem \mssALL,\hskip.2em plus .9em ॰चनं \msCb}}% 
    \var{{\devanagarifontvar\numnoemph\vd ॰सयेत्\lem \eme,\hskip.2em plus .9em ॰श्रयेत् \mssCaCbCc\msNa\msNb\msNc\Ed}}% 
    \paral{{\devanagarifontsmall \vd {\englishfont \compare\ \VSS\ 11.59cd:} शिवस्य हृदयं संध्या तस्मात्संध्यामुपासयेत् }}

{\devanagarifont अक्षमाला पुराणार्थं जप शान्तं दिवानिशम् \thinspace{\dandab} \dontdisplaylinenum }%
     \var{{\devanagarifontvar\numemph\va अक्षमाला\lem \mssALL,\hskip.2em plus .9em \uncl{अक्ष}\lacwithnum{1}  ला \msCa\oo 
 पुराणार्थं\lem \mssALL,\hskip.2em plus .9em पुराणाञ्च \msNb,\hskip.5em plus .9em 
पुराणा\uncl{र्था} \msNc}}% 
    \var{{\devanagarifontvar\numnoemph\vb ॰शान्तं\lem \mssALL,\hskip.2em plus .9em ॰शन्ति \msCaacorr\msNa}}% 

%Verse 11:39

{\devanagarifont ज्ञानसलिलसम्पूर्णमितिहासकमण्डलुः {॥ ११:\hspace{.11em}३९॥} \veg\dontdisplaylinenum }%
     \var{{\devanagarifontvar\numnoemph\vc ॰सलिल॰\lem \mssALL,\hskip.2em plus .9em ॰सलील॰ \Ed}}% 
    \var{{\devanagarifontvar\numnoemph\vd ॰कमण्डलुः\lem \mssALL,\hskip.2em plus .9em ॰कमण्डलु \Ed}}% 

{\devanagarifont पञ्चकर्मक्रियोत्क्रान्ति जप पञ्चविधः सुखम् \thinspace{\dandab} \dontdisplaylinenum }%
     \var{{\devanagarifontvar\numemph\vab ॰त्क्रान्ति ज॰\lem \msCa\msCb\msNb,\hskip.2em plus .9em ॰क्रान्तिज॰ \msCc,\hskip.5em plus .9em ॰त्क्रान्तिर्ज॰ \msNa,\hskip.5em plus .9em 
॰त्कान्तिज॰ \msNc,\hskip.5em plus .9em ऽक्रान्ति ज॰ \Ed}}% 

%Verse 11:40

{\devanagarifont साधनं शिवसंकल्पो योगसिद्धिफलप्रदः {॥ ११:\hspace{.11em}४०॥} \veg\dontdisplaylinenum }%
     \var{{\devanagarifontvar\numnoemph\vd ॰दः\lem \mssALL,\hskip.2em plus .9em ॰दम् \Ed}}% 

{\devanagarifont संतोषफलमाहारः कामक्रोधपराजितः \thinspace{\dandab} \dontdisplaylinenum }%
 
{\devanagarifont आशापाशजयाभ्यासो ध्यानयोगरतिप्रियः  \danda\dontdisplaylinenum }%
     \var{{\devanagarifontvar\numemph\vc ॰भ्यासो\lem \mssALL,\hskip.2em plus .9em ॰भ्यास \Ed}}% 
    \var{{\devanagarifontvar\numnoemph\vd ॰रति॰\lem \msCc\msNa\msNb\msNc,\hskip.2em plus .9em \lacwithnum{2}  \msCa,\hskip.5em plus .9em ॰रिति॰ \msCb,\hskip.5em plus .9em ॰रतिः \Ed}}% 

%Verse 11:41

{\devanagarifont अतिथिभ्यो ऽभयं दत्त्वा वानप्रस्थश्चरेद्व्रतम् {॥ ११:\hspace{.11em}४१॥} \veg\dontdisplaylinenum }%
     \var{{\devanagarifontvar\numnoemph\ve अतिथिभ्यो ऽभयं\lem \mssALL,\hskip.2em plus .9em 
आर्तिभ्यश्चाभयं \Ed\oo 
 दत्त्वा\lem \mssALL,\hskip.2em plus .9em दारा \msCc}}% 
    \var{{\devanagarifontvar\numnoemph\vf ॰प्रस्थश्च॰\lem \mssALL,\hskip.2em plus .9em ॰प्रस्थ च॰ \msCc\msNb}}% 

\ujvers\nemsloka {
{\devanagarifont वानप्रस्थमयं धर्मं गदित यत्पूर्वमवधारितं }%
  \dontdisplaylinenum}    \var{{\devanagarifontvar\numemph\va गदित यत्पूर्वमवधारितम्\lem \conj,\hskip.2em plus .9em गदितं पूर्वधारितम् \msCa\msCb,\hskip.5em plus .9em 
यत्पूर्वमवधारितं \msCc\Ed,\hskip.5em plus .9em 
गदितं यत्पूर्वधारितं \msNaacorr,\hskip.5em plus .9em 
गदितं यत्पूर्व\uncl{मवधा}रितं \msNapcorr,\hskip.5em plus .9em 
गदित पूर्वधारितं \msNb,\hskip.5em plus .9em 
गदितं यत्पूर्वमेधारितं \msNc}}% 


\nemslokab

{\devanagarifont संसारोद्धरणमनित्यहरणमज्ञाननिर्मूलनम्  \danda\dontdisplaylinenum }%
     \var{{\devanagarifontvar\numnoemph\vb ॰हरणमनित्यहरणमज्ञा॰\lem \msCa\msCb\msNaacorr\msNb\msNc,\hskip.2em plus .9em 
॰हरणंमनित्यहरणमज्ञा॰ \msCc\Ed,\hskip.5em plus .9em 
॰हरणंम् अनित्यहरणन्तज्ञा॰ \msNapcorr}}% 

\pend
\endnumbering
\vfill\pagebreak\beginnumbering\pstart
\vers

\nemslokac

{\devanagarifont प्रज्ञावृद्धिकरममोघकरणं क्लेशार्णवोत्तारणं }%
  \dontdisplaylinenum    \lacuna{\devanagarifontsmall \vc {\englishfont \om\ \msNb} }%
      \var{{\devanagarifontvar\numnoemph\vc ॰करममोघ॰\lem \mssALL,\hskip.2em plus .9em ॰कममोघ॰ \msNc,\hskip.5em plus .9em ॰करं प्रबोध॰ \Ed\oo 
 क्लेशार्णवो॰\lem \mssALL,\hskip.2em plus .9em क्लेशाण्णवो॰ \msNa,\hskip.5em plus .9em शोकार्णवो॰ \Ed}}% 

%Verse 11:42


\nemslokad

{\devanagarifont जन्मव्याधिहरमकर्मदहनं सेवेत्स धर्मोत्तमम् {॥ ११:\hspace{.11em}४२॥} \veg\dontdisplaylinenum }%
     \var{{\devanagarifontvar\numnoemph\vd सेवेत्स\lem \mssALL,\hskip.2em plus .9em 
सेवे स \msCc,\hskip.5em plus .9em सेवेत्त \msNb}}% 
    \lacuna{\devanagarifontsmall \vd {\englishfont \Ed\ (and paper MS \msPaperA) add here a Śārdūlavikrīḍita line:}
                 श्रद्धापूर्वकमेव यः सनियमं साक्षाच्च जीवन्शिवः
                 (शुद्धापूर्व्वकमेव यः सनियतं साक्षाच्च जीवने शिवः {\englishfont \msPaperA}) }%
  
\nemslokanormal



\alalfejezet{परिव्राजकः}
\vers


{\devanagarifont परिव्राजकधर्मो ऽयं कीर्तयिष्यामि तच्छृणु \thinspace{\dandab} \dontdisplaylinenum }%
     \var{{\devanagarifontvar\numemph\vb कीर्तयिष्यामि\lem \mssALL,\hskip.2em plus .9em 
कीर्तयि\lacwithnum{1}  मि \msCa}}% 

%Verse 11:43

{\devanagarifont सुखदुःखं समं कृत्वा लोभमोहविवर्जितः {॥ ११:\hspace{.11em}४३॥} \veg\dontdisplaylinenum }%
     \var{{\devanagarifontvar\numnoemph\vc ॰दुःखं\lem \msCb,\hskip.2em plus .9em ॰दुःख \msCa\msCc\msNa\msNb\msNc\Ed}}% 
    \var{{\devanagarifontvar\numnoemph\vd लोभमोह॰\lem \msCb,\hskip.2em plus .9em लाभालोभ॰ \msCa\msNa\msNb\msNc,\hskip.5em plus .9em 
लाभलोभ॰ \msCc,\hskip.5em plus .9em लाभालाभ॰ \Ed\oo 
 ॰वर्जितः\lem \mssALL,\hskip.2em plus .9em ॰वर्जिताः \msNb}}% 
    \paral{{\devanagarifontsmall \vd {\englishfont \compare\ \VSS\ 4.71:}  
                      कामः क्रोधश्च लोभश्च मोहश्चैव चतुर्विधः\thinspace{\devanagarifontsmall ।}
                      चतुःशत्रुर्निहन्तव्यः सर्वथा वीतकल्मषः\thinspace{\devanagarifontsmall ॥} }}

{\devanagarifont वर्जयेन्मधु मांसानि परदारांश्च वर्जयेत् \thinspace{\dandab} \dontdisplaylinenum }%
     \var{{\devanagarifontvar\numemph\va वर्जयेन्\lem \msCa\msNb,\hskip.2em plus .9em वर्जयेत् \msCb\msCc\msNa\msNc\Ed}}% 
    \paral{{\devanagarifontsmall \vab {\englishfont \compare\ \MANU\ 2.177:}
                 वर्जयेन्मधु मांसं च गन्धं माल्यं रसान्स्त्रियः\thinspace{\devanagarifontsmall ।}
                 शुक्तानि यानि सर्वाणि प्राणिनां चैव हिंसनम्\thinspace{\devanagarifontsmall ॥} }}

%Verse 11:44

{\devanagarifont वर्जयेच्चिरवासं च परवासं च वर्जयेत् {॥ ११:\hspace{.11em}४४॥} \veg\dontdisplaylinenum }%
     \var{{\devanagarifontvar\numnoemph\vc ॰वासं\lem \mssALL,\hskip.2em plus .9em ॰वासश् \Ed}}% 
    \var{{\devanagarifontvar\numnoemph\vd ॰वासं\lem \mssALL,\hskip.2em plus .9em ॰वासश् \Ed}}% 

{\devanagarifont वर्जयेत्सृष्टभोज्यानि भिक्षामेकां च वर्जयेत् \thinspace{\dandab} \dontdisplaylinenum }%
     \var{{\devanagarifontvar\numemph\va वर्जयेत्सृष्ट॰\lem \msCc(?)\msNa\msNc,\hskip.2em plus .9em वर्जयेत्मृष्ट॰ \msCa,\hskip.5em plus .9em 
वर्ज्जन्मृष्ट॰ \msNb,\hskip.5em plus .9em वर्जयेन्मृष्ट॰ \Ed\oo 
 ॰भोज्यानि\lem \mssALL,\hskip.2em plus .9em ॰भोजालि(?) \msNc}}% 
    \var{{\devanagarifontvar\numnoemph\vb ॰क्षामेकां\lem \msCa\msNb,\hskip.2em plus .9em ॰क्षामेकं \msCc\msNa,\hskip.5em plus .9em 
॰क्षमेकञ् \msNc,\hskip.5em plus .9em ॰क्षामेकश् \Ed}}% 
    \lacuna{\devanagarifontsmall \vab {\englishfont Omitted in \msCb} }%
      \paral{{\devanagarifontsmall \vb {\englishfont \compare\ \MANU\ 2.188ab:}
                          भैक्षेण वर्तयेन्नित्यं नैकान्नादी भवेद्व्रती }}

%Verse 11:45

{\devanagarifont वर्जयेत्संग्रहं नित्यमभिमानं च वर्जयेत् {॥ ११:\hspace{.11em}४५॥} \veg\dontdisplaylinenum }%
 
{\devanagarifont सुसूक्ष्मं मनसा ध्यात्वा दृशौ पादं विनिक्षिपेत् \thinspace{\dandab} \dontdisplaylinenum }%
     \var{{\devanagarifontvar\numemph\vb दृशौ\lem \conj,\hskip.2em plus .9em शुचौ \mssCaCbCc\msNa\msNb\msNc\Ed\oo 
 पादं\lem \msCb\msCc\msNa\msNc,\hskip.2em plus .9em पा\uncl{दं} \msCa,\hskip.5em plus .9em पाद \msNb\Ed\oo 
 विनिक्षि॰\lem \mssALL,\hskip.2em plus .9em \lacwithnum{1}  निक्षि॰ \msCa,\hskip.5em plus .9em 
विनिक्ष॰ \msNc}}% 

%Verse 11:46

{\devanagarifont न कुप्येत अनालाभे लाभे वापि न हर्षयेत् {॥ ११:\hspace{.11em}४६॥} \veg\dontdisplaylinenum }%
     \var{{\devanagarifontvar\numnoemph\vc कुप्येत\lem \mssALL,\hskip.2em plus .9em कुपेत \msCc\oo 
 अनालाभे\lem \msNa,\hskip.2em plus .9em मनोलाभे \msCa\msCb\msNb\msNc,\hskip.5em plus .9em 
मनोलाभो \msCc,\hskip.5em plus .9em मनालाभे \Ed}}% 
    \paral{{\devanagarifontsmall \vcd {\englishfont \similar\ \MANU\ 6.57:}
                         अलाभे न विषदी स्याल्लाभे चैव न हर्षयेत् = 
                     {\englishfont \VASISTHADHS\ 10.22} }}

{\devanagarifont अर्थतृष्णास्वनुद्विग्नो रोषे वापि सुदारुणे \thinspace{\dandab} \dontdisplaylinenum }%
     \var{{\devanagarifontvar\numemph\va अर्थ॰\lem \msCb\msCc\msNc,\hskip.2em plus .9em अर्था॰ \msCa\msNa\msNb,\hskip.5em plus .9em अथ \Ed\oo 
 ॰नुद्विग्नो\lem \mssALL,\hskip.2em plus .9em ॰नुदिग्नो \msCc}}% 

%Verse 11:47

{\devanagarifont स्तुतिनिन्दा समं कृत्वा प्रियं वाप्रियमेव वा {॥ ११:\hspace{.11em}४७॥} \veg\dontdisplaylinenum }%
 
{\devanagarifont नियमास्तु परीधानं संयमावृतमेखलः \thinspace{\dandab} \dontdisplaylinenum }%
     \var{{\devanagarifontvar\numemph\va ॰धानं\lem \mssALL,\hskip.2em plus .9em 
॰धाना \msCc,\hskip.5em plus .9em ॰\uncl{धानं} \msNc}}% 
    \var{{\devanagarifontvar\numnoemph\vb ॰वृत॰\lem \mssALL,\hskip.2em plus .9em ॰मृत॰ \msNb,\hskip.5em plus .9em ॰नृत॰ \Ed\oo 
 ॰मेखलः\lem \mssALL,\hskip.2em plus .9em 
॰मेखलाः \msCc,\hskip.5em plus .9em ॰मेखला \msNb}}% 

%Verse 11:48

{\devanagarifont निरालम्बं मनः कृत्वा बुद्धिं कृत्वा निरञ्जनाम् {॥ ११:\hspace{.11em}४८॥} \veg\dontdisplaylinenum }%
     \var{{\devanagarifontvar\numnoemph\vc ॰बं मनः कृत्वा\lem \msNc,\hskip.2em plus .9em ॰बमसत्कृत्वा \msCa\msNa,\hskip.5em plus .9em 
॰बमसंकृत्वा \msCb,\hskip.5em plus .9em ॰बमनंकृत्वा \msCc,\hskip.5em plus .9em 
॰ब मनस्कृत्वा \msNb,\hskip.5em plus .9em ॰बमनङ्कृत्वा \Ed}}% 
    \var{{\devanagarifontvar\numnoemph\vd बुद्धिं\lem \mssALL,\hskip.2em plus .9em बुद्धि \msCb\Ed\oo 
 निरञ्जनाम्\lem \eme,\hskip.2em plus .9em निरञ्जनम् \mssCaCbCc\msNb\msNc\Ed,\hskip.5em plus .9em निरञ्जनः \msNa}}% 

{\devanagarifont आत्मानं पृथिवीं कृत्वा खं च कृत्वा मनोन्मनम् \thinspace{\dandab} \dontdisplaylinenum }%
     \var{{\devanagarifontvar\numemph\vab कृत्वा खं च\lem \mssALL,\hskip.2em plus .9em 
कृ\uncl{त्वा}\lacwithnum{1}  ञ्च \msCa}}% 
    \var{{\devanagarifontvar\numnoemph\vb मनोन्मनम्\lem \mssALL,\hskip.2em plus .9em मनोन्मनः \msNc,\hskip.5em plus .9em मनोन्मनैः \Ed}}% 

%Verse 11:49

{\devanagarifont त्रिदण्डं त्रिगुणं कृत्वा पात्रं कृत्वाक्षरो ऽव्ययः {॥ ११:\hspace{.11em}४९॥} \veg\dontdisplaylinenum }%
     \var{{\devanagarifontvar\numnoemph\vd ॰क्षरो\lem \mssALL,\hskip.2em plus .9em ॰करो \msNb\oo 
 व्ययः\lem \msCa\msCb\msNa\msNb,\hskip.2em plus .9em व्ययं \msCc,\hskip.5em plus .9em व्यय \msNc,\hskip.5em plus .9em द्वयम् \Ed}}% 

{\devanagarifont न्यसेद्धर्ममधर्मं च ईर्ष्याद्वेषं परित्यजेत् \thinspace{\dandab} \dontdisplaylinenum }%
     \var{{\devanagarifontvar\numemph\va ॰धर्मं च\lem \mssALL,\hskip.2em plus .9em ॰धर्मं वा \msNa}}% 
    \var{{\devanagarifontvar\numnoemph\vb ईर्ष्या॰\lem \msNa\msNc\Ed,\hskip.2em plus .9em ईर्षा॰ \mssCaCbCc\msNb\oo 
 ॰द्वेषं\lem \mssALL,\hskip.2em plus .9em ॰द्वेष \msCc}}% 

%Verse 11:50

{\devanagarifont निर्द्वन्द्वो नित्यसत्यस्थो निर्ममो निरहंकृतः {॥ ११:\hspace{.11em}५०॥} \veg\dontdisplaylinenum }%
     \var{{\devanagarifontvar\numnoemph\vc निर्द्वन्द्वो\lem \mssALL,\hskip.2em plus .9em निवंद्वो \msCc\oo 
 ॰सत्य॰\lem \mssALL,\hskip.2em plus .9em ॰संत्य॰ \msCc}}% 
    \var{{\devanagarifontvar\numnoemph\vd निर्ममो\lem \msNc\Ed,\hskip.2em plus .9em निर्मांसो \mssCaCbCc\msNa,\hskip.5em plus .9em निर्मंसो \msNb\oo 
 ॰कृतः\lem \mssALL,\hskip.2em plus .9em ॰कृतं \msNa,\hskip.5em plus .9em ॰कृतिः \Ed}}% 
    \paral{{\devanagarifontsmall \vcd {\englishfont \compare\ \BHG\ 2.45cd: 
                         }निर्द्वन्द्वो नित्यसत्त्वस्थो निर्योगक्षेम आत्मवान् }}

{\devanagarifont दिवसस्याष्टमे भागे भिक्षां सप्तगृहं चरेत् \thinspace{\dandab} \dontdisplaylinenum }%
     \var{{\devanagarifontvar\numemph\va दिवसस्या॰\lem \mssALL,\hskip.2em plus .9em दिवसत्या॰ \msCb}}% 
    \var{{\devanagarifontvar\numnoemph\vb भिक्षां\lem \mssALL,\hskip.2em plus .9em भिक्षा \msNb}}% 
    \paral{{\devanagarifontsmall \vb {\englishfont \compare\ \GAUTDHS\ 23.18:}
                 तस्याजिनमूर्ध्वबालं परिधाय लोहितपत्रः सप्त गृहान्भक्षं चरेत् }}

%Verse 11:51

{\devanagarifont न चासीत न तिष्ठेत न च देहीति वा वदेत् {॥ ११:\hspace{.11em}५१॥} \veg\dontdisplaylinenum }%
 
{\devanagarifont यथालाभेन वर्तेत अष्टौ पिण्डान्दिने दिने \thinspace{\dandab} \dontdisplaylinenum }%
     \var{{\devanagarifontvar\numemph\va यथालाभेन\lem \mssALL,\hskip.2em plus .9em यथाला\lacwithnum{2}  \msCa}}% 
    \var{{\devanagarifontvar\numnoemph\vb अष्टौ\lem \mssALL,\hskip.2em plus .9em अष्ट \Ed}}% 

%Verse 11:52

{\devanagarifont वस्त्रभोजनशय्यासु न प्रसज्येत विस्तरम् {॥ ११:\hspace{.11em}५२॥} \veg\dontdisplaylinenum }%
     \var{{\devanagarifontvar\numnoemph\vc ॰शय्यासु\lem \mssALL,\hskip.2em plus .9em ॰शय्याञ्च \msNb,\hskip.5em plus .9em ॰शैय्यासु \Ed}}% 
    \var{{\devanagarifontvar\numnoemph\vd ॰सज्येत\lem \msCa\msCc\msNa\msNb,\hskip.2em plus .9em ॰युज्ये \msCb,\hskip.5em plus .9em ॰सहेत \msNc,\hskip.5em plus .9em ॰सह्येत \Ed\oo 
 विस्तरम्\lem \mssALL,\hskip.2em plus .9em विस्तरः \Ed}}% 

{\devanagarifont नाभिनन्देत मरणं नाभिनन्देत जीवितम् \thinspace{\dandab} \dontdisplaylinenum }%
     \paral{{\devanagarifontsmall \vab {\englishfont = \MBH\ 12.237.15ab = \MANU\ 6.45ab = \NARADAPARIVR\ 3.61cd} }}

%Verse 11:53

{\devanagarifont इन्द्रियाणि वशंकृत्वा कामं हत्वा यतव्रतः {॥ ११:\hspace{.11em}५३॥} \veg\dontdisplaylinenum }%
     \var{{\devanagarifontvar\numemph\vc वशंकृ॰\lem \mssALL,\hskip.2em plus .9em वसंत्कृ॰ \msCc}}% 
    \var{{\devanagarifontvar\numnoemph\vd हत्वा यतव्रतः\lem \mssALL,\hskip.2em plus .9em 
कृत्वा यतः व्रतः \msNb}}% 

{\devanagarifont अतीतं च भविष्यं च न भिक्षुश्चिन्तयेत्सदा \thinspace{\dandab} \dontdisplaylinenum }%
     \var{{\devanagarifontvar\numemph\vb भिक्षुश्चि॰\lem \mssALL,\hskip.2em plus .9em 
भिक्षुंश्चि॰ \msNa,\hskip.5em plus .9em भिक्षु चि॰ \Ed\oo 
 सदा\lem \mssALL,\hskip.2em plus .9em \om\ \msCb}}% 

%Verse 11:54

{\devanagarifont क्रोधमानमददर्पान्परिव्राड्वर्जयेत्सदा {॥ ११:\hspace{.11em}५४॥} \veg\dontdisplaylinenum }%
     \var{{\devanagarifontvar\numnoemph\vcd ॰दर्पान्प॰\lem \mssALL,\hskip.2em plus .9em ॰दर्पात्प॰ \msCb}}% 

{\devanagarifont विरागं तु धनुः कृत्वा प्राणायामगुणैर्युतम् \thinspace{\dandab} \dontdisplaylinenum }%
     \var{{\devanagarifontvar\numemph\va धनुः\lem \mssALL,\hskip.2em plus .9em धनुष् \Ed}}% 
    \var{{\devanagarifontvar\numnoemph\vb प्राणायामगु॰\lem \mssALL,\hskip.2em plus .9em प्राणायामङ्गु॰ \msCa\oo 
 युतम्\lem \mssALL,\hskip.2em plus .9em युतः \msNa,\hskip.5em plus .9em वृतं \Ed}}% 

%Verse 11:55

{\devanagarifont धारणाशरतीक्ष्णेन मृगं हत्वा मनेन्द्रियम् {॥ ११:\hspace{.11em}५५॥} \veg\dontdisplaylinenum }%
     \var{{\devanagarifontvar\numnoemph\vc ॰तीक्ष्णेन\lem \msNb\Ed,\hskip.2em plus .9em ॰तीक्ष्णेण \mssCaCbCc\msNc,\hskip.5em plus .9em ॰तीक्षेण \msNa}}% 

{\devanagarifont मैत्रीखड्गसुतीक्ष्णेन संसारारिं निकृन्तयेत् \thinspace{\dandab} \dontdisplaylinenum }%
     \var{{\devanagarifontvar\numemph\va सुतीक्ष्णेन\lem \msCa\msNb\msNc\Ed,\hskip.2em plus .9em सुतीक्ष्णेण \msCb\msCc\msNapcorr,\hskip.5em plus .9em ण \msNaacorr}}% 
    \var{{\devanagarifontvar\numnoemph\vb ॰सारारिं\lem \mssALL,\hskip.2em plus .9em ॰सारारि \msCc\msNc}}% 

{\devanagarifont करुणावर्तचक्रेण क्रोधमत्तगजं जयेत्  \danda\dontdisplaylinenum }%
 
%Verse 11:56

{\devanagarifont मुदितावर्मबद्धाङ्गस्तूणं पूर्णमुपेक्षया {॥ ११:\hspace{.11em}५६॥} \veg\dontdisplaylinenum }%
     \var{{\devanagarifontvar\numnoemph\vf तूणं पूर्णमु॰\lem \emeGoodall,\hskip.2em plus .9em तूण्णापूर्ण्णमु॰ \msCa,\hskip.5em plus .9em 
तूणापूर्ण्णमु॰ \msCb,\hskip.5em plus .9em तू$\-$\uncl{न}पूर्ण्णमु॰ \msCc,\hskip.5em plus .9em 
तूण्णापूण्णामु॰ \msNa,\hskip.5em plus .9em तूर्ण्णापूर्ण्णमु॰ \msNb\msNc,\hskip.5em plus .9em तूणीपूर्णमु॰ \Ed}}% 

{\devanagarifont अनक्षरं परं ब्रह्म चिन्तयेत्सततं द्विज \thinspace{\dandab} \dontdisplaylinenum }%
     \var{{\devanagarifontvar\numemph\va अनक्षरं\lem \msCb,\hskip.2em plus .9em अनाक्षरं \msCa\msNa,\hskip.5em plus .9em 
अनाक्षर॰ \msCc\msNc\Ed,\hskip.5em plus .9em अनक्षर॰ \msNb\oo 
 परं\lem \mssALL,\hskip.2em plus .9em पर \msCb\msNc}}% 

{\devanagarifont ब्रह्मणो हृदयं विष्णुर्विष्णोश्च हृदयं शिवः  \danda\dontdisplaylinenum }%
     \var{{\devanagarifontvar\numnoemph\vc हृदयं\lem \mssALL,\hskip.2em plus .9em 
\lacwithnum{1}  दयं \msCa,\hskip.5em plus .9em हृदये \msNc}}% 
    \var{{\devanagarifontvar\numnoemph\vcd विष्णुर्वि॰\lem \msCa\msNa\Ed,\hskip.2em plus .9em विष्णुम्वि॰ \msCb,\hskip.5em plus .9em 
विष्णु वि॰ \msCc\msNb\msNc}}% 
    \var{{\devanagarifontvar\numnoemph\vd शिवः\lem \Ed,\hskip.2em plus .9em शिवं \mssCaCbCc\msNa\msNb\msNc}}% 

%Verse 11:57

{\devanagarifont शिवस्य हृदयं संध्या तस्मात्संध्यामुपासयेत् {॥ ११:\hspace{.11em}५७॥} \veg\dontdisplaylinenum }%
     \var{{\devanagarifontvar\numnoemph\vf ॰सयेत्\lem \msCa\msCc\msNb,\hskip.2em plus .9em ॰शयेत् \msCb\msNa,\hskip.5em plus .9em ॰श्रयेत् \msNc\Ed}}% 
    \paral{{\devanagarifontsmall \vo {\englishfont \similar\ Saubhāgyabhāskara of Bhāskararāya ad Lalitāsahasranāmastotra 302:}
                 ब्रह्मणो हृदयं विष्णुर्विष्णोरपि शिवः स्मृतः\thinspace{\devanagarifontsmall ।}
                 शिवस्य हृदयं सन्ध्या तेनोपास्या द्विजातिभिः\thinspace{\devanagarifontsmall ॥}
                 इति कश्यपादिवचनैः कौर्मपाद्मस्कान्दादिनिखिलपुराणेषु च तत्र 
                 तत्र देवीकालिकाब्रह्माण्डमार्कण्डेयादिपुराणेषु बहुशः 
                 शक्तिरहस्य-देवीभागवत-तृतीयस्कन्धादिषु
                 च इदंपर्येण सर्वत्र ज्ञानार्णवकुलार्णवादितन्त्रेषु त्वपरिमितत्या वर्णितम् }}

\nemslokalong


\ujvers\nemsloka {
{\devanagarifont संसारार्णवतारणं शुभगतिः स ब्रह्म संध्याक्षरं }%
  \dontdisplaylinenum}    \var{{\devanagarifontvar\numemph\va ॰गतिः\lem \msCc\Ed,\hskip.2em plus .9em ॰गति \msCa\msCb\msNa\msNb\ \unmetr,\hskip.5em plus .9em ॰गतिं \msNc\oo 
 ॰क्षरं\lem \mssALL,\hskip.2em plus .9em ॰क्षर \msCb}}% 


\nemslokab

{\devanagarifont ध्यायेन्नित्यमतन्द्रितो ह्यनुपमं व्यक्तात्मवेद्यं शिवम्  \danda\dontdisplaylinenum }%
     \var{{\devanagarifontvar\numnoemph\vb ॰तन्द्रितो\lem \msCa\msNa\msNc\Ed,\hskip.2em plus .9em ॰नन्द्रितो \msCb,\hskip.5em plus .9em ॰तन्द्रिय \msCc,\hskip.5em plus .9em ॰तन्द्रियं \msNb\oo 
 ॰वेद्यं\lem \mssALL,\hskip.2em plus .9em ॰वेद्य \msNb\ \unmetr}}% 

\nemslokac

{\devanagarifont रूपैर्वर्णगुणादिभिश्च विहितं दुर्लक्ष्यलक्ष्योत्तमं }%
  \dontdisplaylinenum    \var{{\devanagarifontvar\numnoemph\vc रूपैर्व॰\lem \msCa\msNa\msNc\Ed,\hskip.2em plus .9em रूपै व॰ \msCb\msCc\msNb\oo 
 विहितं\lem \mssALL,\hskip.2em plus .9em रहितं \msNapcorr(?)\Ed\oo 
 दुर्लक्ष्यलक्ष्योत्तमम्\lem \msCa\msNb,\hskip.2em plus .9em 
दुर्लक्ष्यलक्षोत्तमम् \msCb\msCc\msNc\Ed,\hskip.5em plus .9em 
दुलक्ष्यलक्ष्योत्तमम् \msNa}}% 

%Verse 11:58


\nemslokad

{\devanagarifont यत्नोद्धृत्य समाश्रयेत्सुरगुरुं सर्वार्तिहर्ता हरम् {॥ ११:\hspace{.11em}५८॥} \veg\dontdisplaylinenum }%
     \var{{\devanagarifontvar\numnoemph\vd यत्नोद्धृत्य\lem \mssALL,\hskip.2em plus .9em यत्नाद्धृत्य \Ed\oo 
 समाश्रये॰\lem \mssALL,\hskip.2em plus .9em मणाश्रये॰ \msNb\oo 
 सर्वार्तिहर्ता हरम्\lem \mssCaCbCc\msNb,\hskip.2em plus .9em सर्वार्त्तिह\uncl{र्त्ता} हरं \msNa,\hskip.5em plus .9em 
सर्वात्तिहर्त्ता हरं \msNc,\hskip.5em plus .9em 
सर्वार्तिहन् शङ्करम् \Ed}}% 

\nemslokanormal


\vers


{\devanagarifont 
\jump
\begin{center}
\ketdanda~इति वृषसारसंग्रहे चतुराश्रमधर्मविधानो नामाध्याय एकादशमः~\ketdanda
\end{center}
\dontdisplaylinenum\vers  }%
     \var{{\devanagarifontvar\numnoemph{\englishfont \Colo:} नामाध्याय एकादशमः\lem \mssALL,\hskip.2em plus .9em नामाध्याय एकादश \msNc,\hskip.5em plus .9em 
नाम एकादशो ऽध्यायः \Ed}}% 
\bekveg\szamveg
\vfill
\phpspagebreak

\versno=0\fejno=12
\thispagestyle{empty}

\centerline{\Large\devanagarifontbold [   द्वादशमो ऽध्यायः  ]}{\vrule depth10pt width0pt} \fancyhead[CE]{{\footnotesize\devanagarifont वृषसारसंग्रहे  }}
\fancyhead[CO]{{\footnotesize\devanagarifont द्वादशमो ऽध्यायः  }}
\fancyhead[LE]{}
\fancyhead[RE]{}
\fancyhead[LO]{}
\fancyhead[RO]{}
\szam\bek



\alalfejezet{आतिथ्यधर्मः}
\vers


{\devanagarifont देव्युवाच {\dandab}\dontdisplaylinenum  }%
 
{\devanagarifont अहिंसा परमो धर्मः सततं परिकीर्त्यते \thinspace{\danda} \dontdisplaylinenum }%
     \var{{\devanagarifontvar\numemph\vab धर्मः स॰\lem \mssALL,\hskip.2em plus .9em धर्मोस्स॰ \msCc}}% 
    \lacuna{\devanagarifontsmall {\englishfont Witnesses used for this chapter: \msCa\ ff.\thinspace 210r--215r, 
                                              \msCb\ ff.\thinspace 215v--219v, 
                                              \msCc\ ff.\allowbreak\thinspace 287v--283v 
                                                        (f.\thinspace 291 is missing),
                                              \msNa\ ff.\thinspace 17v--22r, 
                                              \msNb\ exp.\thinspace 58 (lower) -- 62 (lower),
                                              \msNc\ ff.\thinspace 225v--230r,
                                              \Ed\ pp.\thinspace 617--628; 
                                              \mssCaCbCc\ = \msCa + \msCb + \msCc} }%
  
%Verse 12:1

{\devanagarifont आतिथ्यकानां धर्मं च कथयस्व यदुत्तमम् {॥ १२:\hspace{.11em}१॥} \veg\dontdisplaylinenum }%
     \var{{\devanagarifontvar\numnoemph\vc आतिथ्य॰\lem \mssALL,\hskip.2em plus .9em अतिथ्य॰ \msCb\msNb\oo 
 धर्मं च\lem \mssALL,\hskip.2em plus .9em 
धर्मश्च \msCc,\hskip.5em plus .9em धर्मानां \msNb}}% 

{\devanagarifont महेश्वर उवाच {\dandab}\dontdisplaylinenum  }%
     \var{{\devanagarifontvar\numemph\vo महेश्वर\lem \mssALL,\hskip.2em plus .9em भगवान् \msNa}}% 

{\devanagarifont अहिंसातिथ्यकानां च शृणु धर्मं यदुत्तमम् \thinspace{\danda} \dontdisplaylinenum }%
     \var{{\devanagarifontvar\numnoemph\vb शृणु\lem \mssALL,\hskip.2em plus .9em \lacwithnum{1}  णु \msCa\oo 
 धर्मं\lem \mssALL,\hskip.2em plus .9em धर्म \msCc\Ed\oo 
 ॰त्तमम्\lem \mssALL,\hskip.2em plus .9em ॰त्तमां \Ed}}% 

%Verse 12:2

{\devanagarifont त्रैलोक्यमखिलं देवि रत्नपूर्णं सुलोचने {॥ १२:\hspace{.11em}२॥} \veg\dontdisplaylinenum }%
     \var{{\devanagarifontvar\numnoemph\vd ॰पूर्णं\lem \mssALL,\hskip.2em plus .9em पूर्ण्ण \msCc,\hskip.5em plus .9em ॰पूर्णां \Ed\oo 
 ॰लोचने\lem \mssALL,\hskip.2em plus .9em ॰लोचनं \msCb}}% 

{\devanagarifont चतुर्वेदविदे दानं न तत्तुल्यमहिंसकः \thinspace{\dandab} \dontdisplaylinenum }%
     \var{{\devanagarifontvar\numemph\va दानं\lem \mssALL,\hskip.2em plus .9em नानं \msCb}}% 

%Verse 12:3

{\devanagarifont शृणु धर्ममतिथ्यानां कीर्तयिष्यामि सुन्दरि {॥ १२:\hspace{.11em}३॥} \veg\dontdisplaylinenum }%
 

\alalfejezet{विपुलोपाख्यानम्}
{\devanagarifont आसीद्वृत्तं पुराख्यानं नगरे कुसुमाह्वये \thinspace{\dandab} \dontdisplaylinenum }%
     \var{{\devanagarifontvar\numemph\va आसीद्वृत्तं\lem \msCa\msNa\Ed,\hskip.2em plus .9em आशीदत्तं \msCb,\hskip.5em plus .9em आसीद्वृतम् \msCc,\hskip.5em plus .9em आसी वृत्तं \msNb,\hskip.5em plus .9em आसीद्वृत्त \msNc\oo 
 ॰ख्यानं\lem \mssALL,\hskip.2em plus .9em ॰ख्यातं \Ed}}% 
    \var{{\devanagarifontvar\numnoemph\vb नगरे कुसुमाह्वये\lem \mssALL,\hskip.2em plus .9em 
नगरं कुसुमाह्वयम् \msCc\msNb}}% 

%Verse 12:4

{\devanagarifont कपिलस्य सुतो विद्वान्विपुलो नाम विश्रुतः {॥ १२:\hspace{.11em}४॥} \veg\dontdisplaylinenum }%
 
{\devanagarifont धर्मनित्यो जितक्रोधः सत्यवादी जितेन्द्रियः \thinspace{\dandab} \dontdisplaylinenum }%
     \paral{{\devanagarifontsmall \vb {\englishfont  = \MBH\ 12.218.13b} }}

%Verse 12:5

{\devanagarifont ब्रह्मण्यश्च कृतज्ञश्च मद्भक्तः कृतनिश्चयः {॥ १२:\hspace{.11em}५॥} \veg\dontdisplaylinenum }%
     \var{{\devanagarifontvar\numemph\vc ब्रह्मण्य॰\lem \msCb\msNa\msNb\Ed,\hskip.2em plus .9em ब्राह्मण्य॰ \msCa\msCc\msNc\oo 
 ॰ज्ञश्च\lem \mssALL,\hskip.2em plus .9em 
॰ज्ञ \msCb,\hskip.5em plus .9em ॰ज्ञश्च \msNb}}% 
    \var{{\devanagarifontvar\numnoemph\vd ॰भक्तः\lem \mssALL,\hskip.2em plus .9em ॰भक्त॰ \Ed}}% 

{\devanagarifont धनाढ्यो ऽतिथिपूज्यश्च दाता दान्तो दयालुकः \thinspace{\dandab} \dontdisplaylinenum }%
     \var{{\devanagarifontvar\numemph\va ॰पूज्यश्च\lem \msCa\msCc\msNapcorr\msNc\Ed,\hskip.2em plus .9em 
॰पूज्य \msCb\msNaacorr,\hskip.5em plus .9em ॰पूजश्च \msNb}}% 
    \var{{\devanagarifontvar\numnoemph\vb दान्तो\lem \msCbacorr\msNc\Ed,\hskip.2em plus .9em 
दान्त \msCa\msCc\msNa,\hskip.5em plus .9em दान्तोम{\englishfont (?)} \msCbpcorr,\hskip.5em plus .9em 
दान्त \msNb}}% 

%Verse 12:6

{\devanagarifont न्यायार्जितधनो नित्यमन्यायपरिवर्जितः {॥ १२:\hspace{.11em}६॥} \veg\dontdisplaylinenum }%
     \var{{\devanagarifontvar\numnoemph\vc न्याया॰\lem \msCc\msNa\msNc\Ed,\hskip.2em plus .9em न्यायो॰ \msCa\msCb\msNb}}% 
    \var{{\devanagarifontvar\numnoemph\vcd नित्यम॰\lem \mssALL,\hskip.2em plus .9em नित्यंम॰ \msNb}}% 
    \var{{\devanagarifontvar\numnoemph\vd ॰वर्जितः\lem \mssALL,\hskip.2em plus .9em ॰वर्जयेत् \msNb}}% 

{\devanagarifont भार्या च रूपिणी तस्य चन्द्रबिम्बशुभानना \thinspace{\dandab} \dontdisplaylinenum }%
     \var{{\devanagarifontvar\numemph\vb ॰बिम्ब॰\lem \mssALL,\hskip.2em plus .9em ॰बिं\uncl{बा} \msNa\oo 
 ॰शुभानना\lem \mssALL,\hskip.2em plus .9em ॰निभानना \msNb}}% 

{\devanagarifont पीनोत्तुङ्गस्तनी कान्ता सकलानन्दकारिणी  \danda\dontdisplaylinenum }%
     \var{{\devanagarifontvar\numnoemph\vd सकला॰\lem \mssALL,\hskip.2em plus .9em \lacwithnum{3}  \msCa}}% 

%Verse 12:7

{\devanagarifont पतिव्रता पतिरता पतिशुश्रूषणे रता {॥ १२:\hspace{.11em}७॥} \veg\dontdisplaylinenum }%
     \var{{\devanagarifontvar\numnoemph\ve पतिव्रता\lem \mssALL,\hskip.2em plus .9em प्रतिव्रता \msCb\oo 
 पतिरता\lem \mssALL,\hskip.2em plus .9em प्रतिरता \msCb\msNb}}% 
    \var{{\devanagarifontvar\numnoemph\vf पतिशुश्रूषणे\lem \mssALL,\hskip.2em plus .9em प्रतिशुश्रूषणे \msNb}}% 
    \paral{{\devanagarifontsmall \vef {\englishfont \compare\ \BrahmaVP\ 4.27.174cd:}
                          पतिव्रते पतिरते पतिं देहि नमो ऽस्तु ते }}

{\devanagarifont अथ केनापि कालेन सूर्यरागमभूत्ततः \thinspace{\dandab} \dontdisplaylinenum }%
     \var{{\devanagarifontvar\numemph\vb ॰भूत्ततः\lem \mssALL,\hskip.2em plus .9em ॰भूततः \msCc}}% 

%Verse 12:8

{\devanagarifont ग्रस्तभागत्रयस्त्वासीत्कृष्णमाधवमासिके {॥ १२:\hspace{.11em}८॥} \veg\dontdisplaylinenum }%
 
{\devanagarifont स्नातुकामावतीर्यन्ते सर्वे पौरनृपादयः \thinspace{\dandab} \dontdisplaylinenum }%
     \var{{\devanagarifontvar\numemph\va ॰वतीर्यन्ते\lem \mssALL,\hskip.2em plus .9em च तीर्थन्ते \Ed}}% 

%Verse 12:9

{\devanagarifont देवाश्च पितरश्चैव तर्प्यन्ते विधिवत्तथा {॥ १२:\hspace{.11em}९॥} \veg\dontdisplaylinenum }%
     \var{{\devanagarifontvar\numnoemph\vc देवाश्च\lem \mssALL,\hskip.2em plus .9em देवश्च \msCc}}% 
    \var{{\devanagarifontvar\numnoemph\vd तर्प्यन्ते\lem \mssALL,\hskip.2em plus .9em तप्यन्ते \msCb\msNb}}% 

{\devanagarifont केचिज्जुह्वति तत्राग्निं केचिद्विप्रांश्च तर्पयेत् \thinspace{\dandab} \dontdisplaylinenum }%
     \var{{\devanagarifontvar\numemph\va ॰चिज्जुह्वति\lem \mssALL,\hskip.2em plus .9em 
॰चिज्जुति \msCb,\hskip.5em plus .9em ॰चि\uncl{ज्व}ह्वति \msCc}}% 
    \var{{\devanagarifontvar\numnoemph\vb विप्रांश्च\lem \mssALL,\hskip.2em plus .9em विप्राश्च \msCb}}% 

%Verse 12:10

{\devanagarifont केचिद्दानोपतिष्ठन्ति केचित्स्तुवन्ति देवताम् {॥ १२:\hspace{.11em}१०॥} \veg\dontdisplaylinenum }%
     \var{{\devanagarifontvar\numnoemph\vc दानो॰\lem \mssALL,\hskip.2em plus .9em ध्यानो॰ \Ed}}% 
    \var{{\devanagarifontvar\numnoemph\vd केचित्स्तुवन्ति\lem \msCa\msCb\msNc,\hskip.2em plus .9em केचिद्वन्ति \msCc,\hskip.5em plus .9em 
केचि स्तुवन्ति \msNa\msNb,\hskip.5em plus .9em 
केचित्स्तुन्वन्ति \Ed\oo 
 देवताम्\lem \mssALL,\hskip.2em plus .9em देवता \msCb\msNc}}% 

{\devanagarifont ध्यानयोगरताः केचित्केचित्पञ्चतपे रताः \thinspace{\dandab} \dontdisplaylinenum }%
     \var{{\devanagarifontvar\numemph\va ॰रताः\lem \mssALL,\hskip.2em plus .9em ॰रता \msNb}}% 

%Verse 12:11

{\devanagarifont एवं प्रवर्तमानेषु राजनादिषु सर्वशः {॥ १२:\hspace{.11em}११॥} \veg\dontdisplaylinenum }%
     \var{{\devanagarifontvar\numnoemph\vd राजना॰\lem \mssALL,\hskip.2em plus .9em राजाना॰ \Ed}}% 

{\devanagarifont विपुलो ऽपि हि तत्रैव गङ्गागण्डकिसंगमे \thinspace{\dandab} \dontdisplaylinenum }%
     \var{{\devanagarifontvar\numemph\va ऽपि हि\lem \msCa\msCc\msNapcorr\msNb\msNc,\hskip.2em plus .9em 
पि \msCb,\hskip.5em plus .9em हि न \msNaacorr,\hskip.5em plus .9em पि च \Ed}}% 

%Verse 12:12

{\devanagarifont भार्यया सह तत्रैव स्नात्वा क्षोमविभूषणः {॥ १२:\hspace{.11em}१२॥} \veg\dontdisplaylinenum }%
     \var{{\devanagarifontvar\numnoemph\vc भार्यया\lem \msCapcorr\msCb\msNa\msNb\msNc,\hskip.2em plus .9em भार्याया \msCaacorr\msCc\Ed}}% 
    \var{{\devanagarifontvar\numnoemph\vd ॰भूषणः\lem \mssALL,\hskip.2em plus .9em 
॰भूष\uncl{णैः} \msCc,\hskip.5em plus .9em ॰भूषितः \msNa}}% 

{\devanagarifont देवतागुरुविप्राणामन्येषां तर्पणे रतः \thinspace{\dandab} \dontdisplaylinenum }%
     \var{{\devanagarifontvar\numemph\vab देवतागुरुविप्राणामन्येषां तर्पणे रतः\lem \msCb\msNapcorr\msNb\msNc,\hskip.2em plus .9em 
देवतागुरुवि\lacwithnum{1}  णामन्येषां तर्पणे रतः \msCa,\hskip.5em plus .9em 
देवतागुरुविप्राणामन्येषां तर्पणे रताः \msCc,\hskip.5em plus .9em 
\om\ \msNaacorr,\hskip.5em plus .9em 
देवतागुरुविप्राणामन्येषां तर्पणा रतः \Ed}}% 

%Verse 12:13

{\devanagarifont तत्रावसरसम्प्राप्तो ब्राह्मणो ऽतिथिरागतः {॥ १२:\hspace{.11em}१३॥} \veg\dontdisplaylinenum }%
 
{\devanagarifont भार्या तस्यातिरूपेण मोहिता ब्रह्मणस्तदा \thinspace{\dandab} \dontdisplaylinenum }%
     \var{{\devanagarifontvar\numemph\vb मोहिता\lem \mssALL,\hskip.2em plus .9em मोहितो \msCb\oo 
 ब्रह्मणस्तदा\lem \msCa\msCb\msNc,\hskip.2em plus .9em ब्राह्मणास्तथा \msCc,\hskip.5em plus .9em 
ब्राह्मणस्तदा \msNa\msNb,\hskip.5em plus .9em ब्राह्मणस्य च \Ed}}% 

%Verse 12:14

{\devanagarifont ब्राह्मणो ऽपि तथैवेह रूपेणाप्रतिमो भवेत् {॥ १२:\hspace{.11em}१४॥} \veg\dontdisplaylinenum }%
     \var{{\devanagarifontvar\numnoemph\vc ब्राह्मणो\lem \mssALL,\hskip.2em plus .9em ब्रह्मणो \msCb\oo 
 तथैवेह\lem \msCb\msNa\msNb\Ed,\hskip.2em plus .9em 
त\uncl{थे}वेह \msCa,\hskip.5em plus .9em तथेवेह \msCc\msNc}}% 
    \var{{\devanagarifontvar\numnoemph\vd रूपेणा॰\lem \msCa\msNa\msNb\msNc,\hskip.2em plus .9em रूपेना॰ \msCb,\hskip.5em plus .9em रूपेण \msCc,\hskip.5em plus .9em रूपिणा॰ \Ed}}% 

{\devanagarifont अन्योन्यदृष्टिसंसक्तौ जातौ तौ तु परस्परम् \thinspace{\dandab} \dontdisplaylinenum }%
     \var{{\devanagarifontvar\numemph\va ॰संसक्तौ\lem \Ed,\hskip.2em plus .9em ॰संशक्तौ \msCa\msNa\msNc,\hskip.5em plus .9em 
॰शक्तौ \msCb,\hskip.5em plus .9em ॰संसक्तो \msCc\msNb}}% 
    \var{{\devanagarifontvar\numnoemph\vb जातौ तौ\lem \mssALL,\hskip.2em plus .9em 
जातो तौ तौ \msCc,\hskip.5em plus .9em जातौ \uncl{ता} \msNc}}% 

%Verse 12:15

{\devanagarifont विपुलेनाञ्जलिं कृत्वा ब्राह्मण संशितव्रत {॥ १२:\hspace{.11em}१५॥} \veg\dontdisplaylinenum }%
     \var{{\devanagarifontvar\numnoemph\vd ब्राह्मण\lem \msCb\msCc,\hskip.2em plus .9em ब्राह्मणः \msCa\msNa\msNb\msNc\Ed\oo 
 ॰शित॰\lem \eme,\hskip.2em plus .9em ॰श्रित॰ \mssCaCbCc\msNa\msNb\msNc\Ed\oo 
 ॰व्रत\lem \conj,\hskip.2em plus .9em ॰व्र\lk\ \msCa,\hskip.5em plus .9em ॰व्रतः \msCb\msCc\msNa\msNb\msNc\Ed}}% 
    \paral{{\devanagarifontsmall \vd {\englishfont  = MBh 12.213.18d and 12.347.1d } }}

{\devanagarifont आज्ञापय द्विजश्रेष्ठ अद्य मे ऽनुग्रहं कुरु \thinspace{\dandab} \dontdisplaylinenum }%
     \var{{\devanagarifontvar\numemph\vb ॰ग्रहं\lem \mssALL,\hskip.2em plus .9em ॰ग्रह \msCb}}% 

%Verse 12:16

{\devanagarifont भार्याभृत्यपशुग्राम रत्नानि विविधानि च {॥ १२:\hspace{.11em}१६॥} \veg\dontdisplaylinenum }%
     \var{{\devanagarifontvar\numnoemph\vc ॰भृत्य॰\lem \mssALL,\hskip.2em plus .9em ॰भृत्या॰ \msCc}}% 

{\devanagarifont विपुलेनैवमुक्तस्तु गृहीतो ब्राह्मणो ऽब्रवीत् \thinspace{\dandab} \dontdisplaylinenum }%
     \var{{\devanagarifontvar\numemph\vb ब्राह्मणो ऽब्रवीत्\lem \mssALL,\hskip.2em plus .9em 
भ्राह्मणस्तथा \msCc}}% 

%Verse 12:17

{\devanagarifont यदि सत्यं प्रदातासि सुप्रसन्नं मनस्तव {॥ १२:\hspace{.11em}१७॥} \veg\dontdisplaylinenum }%
     \var{{\devanagarifontvar\numnoemph\vc यदि सत्यं प्रदातासि\lem \mssALL,\hskip.2em plus .9em \om\ \msCc}}% 
    \var{{\devanagarifontvar\numnoemph\vd सुप्रसन्नं मनस्तव\lem \msCa\msCb\msNa\msNc,\hskip.2em plus .9em \om\ \msCc,\hskip.5em plus .9em 
सुप्रसन्नमनस्तव \msNb\Ed}}% 

{\devanagarifont विपुल उवाच {\dandab}\dontdisplaylinenum  }%
 
{\devanagarifont सुप्रसन्नं मनो मे ऽद्य सुप्रसन्नं तपःफलम् \thinspace{\danda} \dontdisplaylinenum }%
     \var{{\devanagarifontvar\numemph\va ॰प्रसन्नं मनो\lem \mssALL,\hskip.2em plus .9em 
॰प्रसन्नमनो \msCc\msNb}}% 
    \var{{\devanagarifontvar\numnoemph\vb सुप्रसन्नं तपः॰\lem \mssALL,\hskip.2em plus .9em 
सुप्रसन्नतपः॰ \msNb}}% 

{\devanagarifont शीघ्रमाज्ञापय विप्र यच्चाभिलषितं तव  \danda\dontdisplaylinenum }%
     \var{{\devanagarifontvar\numnoemph\vc शीघ्र॰\lem \mssALL,\hskip.2em plus .9em श्रीघ्र॰ \msNb}}% 

%Verse 12:18

{\devanagarifont अदेयं नास्ति विप्रस्य स्वशिरःप्रभृति द्विज {॥ १२:\hspace{.11em}१८॥} \veg\dontdisplaylinenum }%
     \var{{\devanagarifontvar\numnoemph\ve अदेयं\lem \mssALL,\hskip.2em plus .9em अदेय \msNb}}% 
    \var{{\devanagarifontvar\numnoemph\vf स्वशिरः॰\lem \mssALL,\hskip.2em plus .9em शरीर॰ \msNa\oo 
 ॰भृति\lem \mssALL,\hskip.2em plus .9em ॰भृतिर् \Ed}}% 

{\devanagarifont ब्राह्मण उवाच {\dandab}\dontdisplaylinenum  }%
     \var{{\devanagarifontvar\numemph\vo ब्राह्मण\lem \mssALL,\hskip.2em plus .9em 
ब्राह्मणा \msCaacorr,\hskip.5em plus .9em ब्रह्म \msNb}}% 

{\devanagarifont यद्येवं वदसे भद्र भार्यां मे देहि रूपिणीम् \thinspace{\danda} \dontdisplaylinenum }%
     \var{{\devanagarifontvar\numnoemph\vb भार्यां\lem \mssALL,\hskip.2em plus .9em भार्या \msNb\msNc}}% 

%Verse 12:19

{\devanagarifont स्वस्ति भवतु भद्रं वः कल्याणं भव शाश्वतम् {॥ १२:\hspace{.11em}१९॥} \veg\dontdisplaylinenum }%
     \var{{\devanagarifontvar\numnoemph\vc स्वस्ति\lem \mssALL,\hskip.2em plus .9em स्वस्तिं \msNb,\hskip.5em plus .9em स्वस्तिर् \Ed}}% 
    \var{{\devanagarifontvar\numnoemph\vd कल्याणं\lem \mssALL,\hskip.2em plus .9em कल्या\uncl{ण} \msCc\oo 
 भव\lem \mssALL,\hskip.2em plus .9em तव \Ed}}% 

{\devanagarifont विपुल उवाच {\dandab}\dontdisplaylinenum  }%
     \var{{\devanagarifontvar\numemph\vo विपुल\lem \mssALL,\hskip.2em plus .9em विप्र \Ed}}% 

{\devanagarifont प्रतीच्छ भार्यां सुश्रोणीं रूपयौवनशालिनीम् \thinspace{\danda} \dontdisplaylinenum }%
     \var{{\devanagarifontvar\numnoemph\va भार्यां\lem \mssALL,\hskip.2em plus .9em भार्या \msNb\oo 
 ॰श्रोणीं\lem \msCa\msCb\msNapcorr\msNc\Ed,\hskip.2em plus .9em ॰श्रोणि \msCc\msNaacorr\msNb}}% 
    \var{{\devanagarifontvar\numnoemph\vb ॰शालिनीम्\lem \mssALL,\hskip.2em plus .9em ॰शालिनी \msNb,\hskip.5em plus .9em ॰शीलिनीं \msNc}}% 

%Verse 12:20

{\devanagarifont अकुत्सितां विशालाक्षीं पूर्णचन्द्रनिभाननाम् {॥ १२:\hspace{.11em}२०॥} \veg\dontdisplaylinenum }%
     \var{{\devanagarifontvar\numnoemph\vc अकुत्सितां विशालाक्षीं\lem \mssALL,\hskip.2em plus .9em 
अकुत्सि\uncl{ता} विशालाक्षि \msCc,\hskip.5em plus .9em 
अकुत्सिता विशालाक्सी \msNb}}% 
    \var{{\devanagarifontvar\numnoemph\vd ॰निभाननाम्\lem \mssALL,\hskip.2em plus .9em ॰निभानना \msNb}}% 

{\devanagarifont भार्योवाच {\dandab}\dontdisplaylinenum  }%
 
{\devanagarifont परित्याज्या कथं नाथ अपापां त्यजसे कथम् \thinspace{\danda} \dontdisplaylinenum }%
     \var{{\devanagarifontvar\numemph\va ॰त्याज्या\lem \msCa\msNa\msNc\Ed,\hskip.2em plus .9em 
॰त्याज्य \msCb\msNb,\hskip.5em plus .9em ॰त्या\uncl{ज्य} \msCc}}% 

%Verse 12:21

{\devanagarifont अतीव हि प्रियां भार्यां निर्दोषां च कथं त्यजेः {॥ १२:\hspace{.11em}२१॥} \veg\dontdisplaylinenum }%
     \var{{\devanagarifontvar\numnoemph\vc प्रियां\lem \mssALL,\hskip.2em plus .9em प्रियं \msCc\msNb}}% 
    \var{{\devanagarifontvar\numnoemph\vd निर्दोषां\lem \mssALL,\hskip.2em plus .9em निर्दोष \msCc\oo 
 त्यजेः\lem \msCa\msNa\msNc,\hskip.2em plus .9em त्यज्येत् \msCb\msCc,\hskip.5em plus .9em त्यजेत् \msNb\Ed\oo 
 च\lem \conj,\hskip.2em plus .9em स \mssCaCbCc\msNa\msNb\msNc\Ed}}% 

{\devanagarifont सखा भार्या मनुष्याणामिह लोके परत्र च \thinspace{\dandab} \dontdisplaylinenum }%
     \var{{\devanagarifontvar\numemph\vab मनुष्याणामिह\lem \mssALL,\hskip.2em plus .9em 
मनुष्याणांमिह \msCc}}% 

%Verse 12:22

{\devanagarifont दानं वा सुमहद्दत्त्वा यज्ञो वा सुबहुः कृतः {॥ १२:\hspace{.11em}२२॥} \veg\dontdisplaylinenum }%
     \var{{\devanagarifontvar\numnoemph\vd ॰बहुः\lem \eme,\hskip.2em plus .9em ॰बहु \mssCaCbCc\msNa\msNc\ \unmetr,\hskip.5em plus .9em 
॰बहुं \msNb,\hskip.5em plus .9em ॰बहून् \Ed\oo 
 कृतः\lem \mssALL,\hskip.2em plus .9em कृतम् \msCc}}% 

{\devanagarifont अपुत्रो नाप्नुयात्स्वर्गं तपोभिर्वा सुदुष्करैः \thinspace{\dandab} \dontdisplaylinenum }%
     \var{{\devanagarifontvar\numemph\vab स्वर्गं तपोभिर्वा\lem \mssALL,\hskip.2em plus .9em 
स्व\uncl{र्ग्गन्} \lacwithnum{3}  र्व्वा \msCa}}% 

%Verse 12:23

{\devanagarifont श्रुतो मे पितृभिः प्रोक्तो ब्राह्मणैश्च ममान्तिके {॥ १२:\hspace{.11em}२३॥} \veg\dontdisplaylinenum }%
     \var{{\devanagarifontvar\numnoemph\vd ॰न्तिके\lem \mssALL,\hskip.2em plus .9em ॰न्तिकैः \msCb}}% 

{\devanagarifont अपुत्रो नाप्नुयात्स्वर्गं श्रुतं मे बहुशः पुरा \thinspace{\dandab} \dontdisplaylinenum }%
     \var{{\devanagarifontvar\numemph\va स्वर्गं\lem \msCa\msNa\msNc\Ed,\hskip.2em plus .9em स्वर्ग \msCb\msCc\msNb}}% 

%Verse 12:24

{\devanagarifont मन्दपालो द्विजश्रेष्ठो गतः स्वर्गं तपोबलात् {॥ १२:\hspace{.11em}२४॥} \veg\dontdisplaylinenum }%
     \var{{\devanagarifontvar\numnoemph\vc ॰पालो\lem \msNc\Ed,\hskip.2em plus .9em ॰पाल \mssCaCbCc\msNa\msNb}}% 

{\devanagarifont दानानि च बहून्दत्त्वा यज्ञांश्च विविधांस्तथा \thinspace{\dandab} \dontdisplaylinenum }%
     \var{{\devanagarifontvar\numemph\va बहून्द॰\lem \mssALL,\hskip.2em plus .9em बहू द॰ \msNc}}% 
    \var{{\devanagarifontvar\numnoemph\vb यज्ञांश्च विविधांस्तथा\lem \msCa\msCc\msNa\msNb,\hskip.2em plus .9em 
यत्वा यज्ञांश्च विविधां तथा \msCb,\hskip.5em plus .9em 
यज्ञांश्च विविधाम्तथा \msNc,\hskip.5em plus .9em 
स्यज्ञाश्च विविधास्तथा \Ed}}% 

%Verse 12:25

{\devanagarifont वेदांश्च जपयज्ञांश्च कृत्वा स द्विजसत्तमः {॥ १२:\hspace{.11em}२५॥} \veg\dontdisplaylinenum }%
     \var{{\devanagarifontvar\numnoemph\vc वेदांश्च जपयज्ञांश्च\lem \msCa\msCc\msNa\msNc,\hskip.2em plus .9em 
वेदाश्च जपयज्ञांश्च \msCb,\hskip.5em plus .9em वेदांश्च जपयज्ञाश्च \msNb,\hskip.5em plus .9em 
वेदाश्च जपयज्ञाश्च \Ed}}% 
    \var{{\devanagarifontvar\numnoemph\vd स द्वि॰\lem \conj,\hskip.2em plus .9em तद्द्वि॰ \mssCaCbCc\msNa\Ed,\hskip.5em plus .9em तद्द्वि॰ \msNb,\hskip.5em plus .9em सद्द्वि॰ \msNc\oo 
 ॰सत्तमः\lem \mssALL,\hskip.2em plus .9em ॰सत्तम \msNa}}% 

{\devanagarifont प्राप्तद्वारो ऽपि यस्यापि देवदूतैर्निवारितः \thinspace{\dandab} \dontdisplaylinenum }%
     \var{{\devanagarifontvar\numemph\va ॰द्वारो\lem \mssALL,\hskip.2em plus .9em   ॰द्वारे \msNb}}% 
    \var{{\devanagarifontvar\numnoemph\vab यस्यापि दे॰\lem \mssALL,\hskip.2em plus .9em यस्यापि द्दे॰ \msNb,\hskip.5em plus .9em 
यस्याहि दे॰ \Ed}}% 
    \var{{\devanagarifontvar\numnoemph\vb ॰दूतैर्नि॰\lem \mssALL,\hskip.2em plus .9em ॰दूतै न्नि॰ \msNb,\hskip.5em plus .9em 
॰दूतै नि॰ \msNc}}% 

%Verse 12:26

{\devanagarifont अपुत्रो नाप्नुयात्स्वर्गं यदि यज्ञशतैरपि {॥ १२:\hspace{.11em}२६॥} \veg\dontdisplaylinenum }%
     \var{{\devanagarifontvar\numnoemph\vc ॰यात्स्वर्गं\lem \mssALL,\hskip.2em plus .9em 
॰यात्स्वर्ग्ग \msCc}}% 
    \var{{\devanagarifontvar\numnoemph\vd ॰शतैरपि\lem \mssALL,\hskip.2em plus .9em करोति यः \msCc}}% 

{\devanagarifont इत्युक्तस्तु च्युतः स्वर्गान्मन्दपालो महानृषिः \thinspace{\dandab} \dontdisplaylinenum }%
     \var{{\devanagarifontvar\numemph\va ॰क्तस्तु च्युतः\lem \mssALL,\hskip.2em plus .9em 
॰क्तस्तु\uncl{म्च्यु}तः \msCc}}% 

%Verse 12:27

{\devanagarifont पुत्रानुत्पादयामास शारङ्गांश्चतुरो द्विजः {॥ १२:\hspace{.11em}२७॥} \veg\dontdisplaylinenum }%
     \var{{\devanagarifontvar\numnoemph\vc पुत्रानु॰\lem \mssALL,\hskip.2em plus .9em पुत्रमु॰ \msCc}}% 
    \var{{\devanagarifontvar\numnoemph\vd शारङ्गांश्च\lem \msNa\msNc,\hskip.2em plus .9em शारङ्गाश्च \msCa,\hskip.5em plus .9em शारङ्गंश्च \msCb,\hskip.5em plus .9em 
शारङ्गश्च \msCc\msNb,\hskip.5em plus .9em शारङ्गाच्च \Ed\oo 
 द्विजः\lem \mssALL,\hskip.2em plus .9em द्विज \msCc}}% 

{\devanagarifont तेन पुण्यप्रभावेण स्वर्गं प्राप्तो ह्यवारितः \thinspace{\dandab} \dontdisplaylinenum }%
     \var{{\devanagarifontvar\numemph\vb स्वर्गं\lem \mssALL,\hskip.2em plus .9em स्वर्ग्ग \msCc\oo 
 ॰वारितः\lem \mssALL,\hskip.2em plus .9em ॰वरितः \msNb}}% 

%Verse 12:28

{\devanagarifont कुलत्राणात्कलत्रास्मि भरणाद्भार्य एव च {॥ १२:\hspace{.11em}२८॥} \veg\dontdisplaylinenum }%
     \var{{\devanagarifontvar\numnoemph\vc कुल॰\lem \msCb,\hskip.2em plus .9em कल॰ \msCa\msCc\msNa\msNb\msNc\Ed\oo 
 ॰त्राणात्क॰\lem \msNb,\hskip.2em plus .9em ॰त्राणां क॰ \mssCaCbCc\msNa\Ed,\hskip.5em plus .9em ॰त्राणा क॰ \msNc\oo 
 ॰स्मि\lem \mssALL,\hskip.2em plus .9em ॰स्मिं \msNb}}% 
    \var{{\devanagarifontvar\numnoemph\vd ॰आद्भार्य एव\lem \msCa\msNa\msNc\Ed,\hskip.2em plus .9em 
॰आद्भार्यमेव \msCb,\hskip.5em plus .9em ॰आ भार्य एव \msCc\msNb}}% 

{\devanagarifont दारसंग्रह पुत्रार्थे क्रियते शास्त्रदर्शनात् \thinspace{\dandab} \dontdisplaylinenum }%
     \var{{\devanagarifontvar\numemph\va ॰ग्रह\lem \msCc\msNb\msNc\Ed,\hskip.2em plus .9em ॰ग्रहः \msCa\msCb\msNa\oo 
 पुत्रा॰\lem \mssALL,\hskip.2em plus .9em पात्रा॰ \Ed}}% 
    \var{{\devanagarifontvar\numnoemph\vb क्रियते\lem \mssALL,\hskip.2em plus .9em क्रियाते \msCb}}% 

%Verse 12:29

{\devanagarifont यानि सन्ति गृहे द्रव्यं ग्रामघोषगृहाणि च {॥ १२:\hspace{.11em}२९॥} \veg\dontdisplaylinenum }%
 
{\devanagarifont दातुमर्हसि विप्राय न मां दातुमिहार्हसि \thinspace{\dandab} \dontdisplaylinenum }%
 
%Verse 12:30

{\devanagarifont भार्याया वचनं श्रुत्वा विपुलः पुनरब्रवीत् {॥ १२:\hspace{.11em}३०॥} \veg\dontdisplaylinenum }%
     \var{{\devanagarifontvar\numemph\vc वचनं\lem \mssALL,\hskip.2em plus .9em वचन \msNc}}% 
    \var{{\devanagarifontvar\numnoemph\vd ॰ब्रवीत्\lem \mssALL,\hskip.2em plus .9em 
॰ब्रवीत्\thinspace{\devanagarifont ।} विपुल उवाच\thinspace{\devanagarifont ।} \msCcpcorr\Ed}}% 

{\devanagarifont साधु भामिनि जानामि साधु साधु पतिव्रते \thinspace{\dandab} \dontdisplaylinenum }%
     \var{{\devanagarifontvar\numemph\va जानामि\lem \msCb\msCc\msNa\Ed,\hskip.2em plus .9em जानासि \msCa\msNb\msNc}}% 
    \var{{\devanagarifontvar\numnoemph\vb पति॰\lem \mssALL,\hskip.2em plus .9em प्रति॰ \msNb}}% 

%Verse 12:31

{\devanagarifont जितो ऽस्म्यनेन वाक्येन अनेनास्मि हि तोषितः {॥ १२:\hspace{.11em}३१॥} \veg\dontdisplaylinenum }%
     \var{{\devanagarifontvar\numnoemph\vd तोषितः\lem \mssALL,\hskip.2em plus .9em तोर्षिनः \msNc}}% 

{\devanagarifont अद्य ग्रहणकाले च द्विज आगत्य याचते \thinspace{\dandab} \dontdisplaylinenum }%
 
%Verse 12:32

{\devanagarifont ददामीति प्रतिज्ञाय अदत्त्वा नरकं व्रजे {॥ १२:\hspace{.11em}३२॥} \veg\dontdisplaylinenum }%
     \var{{\devanagarifontvar\numemph\vd व्रजे\lem \msCa\msNapcorr\msNc,\hskip.2em plus .9em व्रजेत् \msCb\msCc\msNb\Ed,\hskip.5em plus .9em 
व्रजे\lk\ \msNaacorr}}% 

{\devanagarifont नरकं यदि गच्छामि कुलेन सह सुन्दरि \thinspace{\dandab} \dontdisplaylinenum }%
     \var{{\devanagarifontvar\numemph\va यदि\lem \mssALL,\hskip.2em plus .9em ययदि \msNc}}% 

{\devanagarifont कल्पकोटिसहस्रे ऽपि नरकस्थो यशस्विनि  \danda\dontdisplaylinenum }%
     \var{{\devanagarifontvar\numnoemph\vc ॰सहस्रे ऽपि\lem \mssALL,\hskip.2em plus .9em ॰सहस्राणि \msCc\Ed}}% 
    \var{{\devanagarifontvar\numnoemph\vd ॰स्थो य॰\lem \msNc\Ed,\hskip.2em plus .9em ॰स्थाद्य॰ \msCa\msCc\msNa\msNb,\hskip.5em plus .9em स्था य॰ \msCb}}% 

%Verse 12:33

{\devanagarifont मुक्तिमेव न पश्यामि जन्मकोटिशतैरपि {॥ १२:\hspace{.11em}३३॥} \veg\dontdisplaylinenum }%
     \var{{\devanagarifontvar\numnoemph\ve मुक्तिमेव\lem \mssALL,\hskip.2em plus .9em मुक्तिमेवन् \Ed}}% 

{\devanagarifont अदानाच्चाशुभं देवि पश्यामि वरवर्णिनि \thinspace{\dandab} \dontdisplaylinenum }%
     \var{{\devanagarifontvar\numemph\va अदानाच्चा॰\lem \mssALL,\hskip.2em plus .9em अदाना चा॰ \msCc}}% 

%Verse 12:34

{\devanagarifont दानेन तु शुभं पश्ये स्वर्गलोके यदक्षयम् {॥ १२:\hspace{.11em}३४॥} \veg\dontdisplaylinenum }%
     \var{{\devanagarifontvar\numnoemph\vd ॰लोके\lem \mssALL,\hskip.2em plus .9em 
\om\ \msNaacorr,\hskip.5em plus .9em ॰लोकं \Ed}}% 

{\devanagarifont नोक्तं मयानृतं पूर्वं नित्यं सत्यव्रते स्थितः \thinspace{\dandab} \dontdisplaylinenum }%
     \var{{\devanagarifontvar\numemph\va नोक्तं\lem \mssALL,\hskip.2em plus .9em नोक्ता \msNcacorr}}% 
    \var{{\devanagarifontvar\numnoemph\vb ॰व्रते\lem \mssALL,\hskip.2em plus .9em ॰व्रत॰ \Ed}}% 

%Verse 12:35

{\devanagarifont सत्यधर्ममतिक्रम्य नान्यधर्मं समाचरे {॥ १२:\hspace{.11em}३५॥} \veg\dontdisplaylinenum }%
     \var{{\devanagarifontvar\numnoemph\vd ॰चरे\lem \mssALL,\hskip.2em plus .9em ॰चरेत् \msNb\Ed}}% 

{\devanagarifont भार्या धर्मसखेत्येवं त्वया पूर्वमुदाहृतम् \thinspace{\dandab} \dontdisplaylinenum }%
     \var{{\devanagarifontvar\numemph\va धर्म॰\lem \mssALL,\hskip.2em plus .9em धर्मं \msNa}}% 
    \var{{\devanagarifontvar\numnoemph\vb त्वया\lem \eme,\hskip.2em plus .9em त्वयि \mssCaCbCc\msNa\msNb\msNc\Ed}}% 

%Verse 12:36

{\devanagarifont यदि धर्मसखायासि सो ऽद्य काल इहागतः {॥ १२:\hspace{.11em}३६॥} \veg\dontdisplaylinenum }%
     \var{{\devanagarifontvar\numnoemph\vc ॰सखाया॰\lem \mssALL,\hskip.2em plus .9em ॰सखा॰ \msCb}}% 

{\devanagarifont द्विजरूपधरो धर्मः स्वयमेव इहागतः \thinspace{\dandab} \dontdisplaylinenum }%
     \var{{\devanagarifontvar\numemph\va ॰धरो\lem \mssALL,\hskip.2em plus .9em ॰परो \msCb}}% 

%Verse 12:37

{\devanagarifont जिज्ञासार्थमहं भद्रे न विघ्नं कर्तुमर्हसि {॥ १२:\hspace{.11em}३७॥} \veg\dontdisplaylinenum }%
     \var{{\devanagarifontvar\numnoemph\vc ॰र्थमहं\lem \mssALL,\hskip.2em plus .9em 
॰र्थम्महं \msNb,\hskip.5em plus .9em ॰र्थमह \msNc}}% 

{\devanagarifont माताव्यक्तः पिता ब्रह्मा बुद्धिर्भार्या दमः सखा \thinspace{\dandab} \dontdisplaylinenum }%
     \var{{\devanagarifontvar\numemph\va ॰व्यक्तः\lem \mssALL,\hskip.2em plus .9em 
॰व्यक्त \msCc,\hskip.5em plus .9em ॰व्यक्त\uncl{ऽ} \msNc}}% 
    \var{{\devanagarifontvar\numnoemph\vb बुद्धिर्भा॰\lem \msCa\msCb\msNb,\hskip.2em plus .9em बुद्धि भा॰ \msCc\msNa\msNc\Ed\oo 
 दमः\lem \mssALL,\hskip.2em plus .9em दम \msNb\ \unmetr\oo 
 सखा\lem \mssALL,\hskip.2em plus .9em समा \msCa}}% 

%Verse 12:38

{\devanagarifont पुत्रो धर्मः क्रियाचार्य इत्येते मम बान्धवाः {॥ १२:\hspace{.11em}३८॥} \veg\dontdisplaylinenum }%
 
{\devanagarifont कालश्रेष्ठो ग्रहः सूर्यो गङ्गा श्रेष्ठा नदीषु च \thinspace{\dandab} \dontdisplaylinenum }%
     \var{{\devanagarifontvar\numemph\va ॰श्रेष्थो\lem \msCb\msNa\msNcpcorr,\hskip.2em plus .9em ॰श्रेष्ठ॰ \msCa\msCc\msNb,\hskip.5em plus .9em ॰श्रेष्ठा \msNcacorr,\hskip.5em plus .9em ॰श्रेष्ठः \Ed}}% 
    \var{{\devanagarifontvar\numnoemph\vb श्रेष्ठा\lem \mssALL,\hskip.2em plus .9em श्रेष्ठो \msNa,\hskip.5em plus .9em श्रेष्ठ \msNb}}% 
    \paral{{\devanagarifontsmall \vb {\englishfont \similar\ 15.18b:} श्रेष्ठा गङ्गा नदीषु च }}

%Verse 12:39

{\devanagarifont चन्द्रक्षये दिनं श्रेष्ठं नरश्रेष्ठो द्विजोत्तमः {॥ १२:\hspace{.11em}३९॥} \veg\dontdisplaylinenum }%
     \var{{\devanagarifontvar\numnoemph\vc दिनं\lem \msCa\msCb\msNa\msNc,\hskip.2em plus .9em दिन॰ \msCc\msNb\Ed}}% 
    \var{{\devanagarifontvar\numnoemph\vd ॰त्तमः\lem \mssALL,\hskip.2em plus .9em ॰त्तम \msCc}}% 

{\devanagarifont शुश्रूषणार्थं विप्रस्य मया दत्तासि सुन्दरि \thinspace{\dandab} \dontdisplaylinenum }%
     \var{{\devanagarifontvar\numemph\va ॰र्थं\lem \mssALL,\hskip.2em plus .9em ॰र्थ \msCb}}% 

%Verse 12:40

{\devanagarifont सर्वस्वं ब्राह्मणे दत्त्वा वनमेवाश्रयाम्यहम् {॥ १२:\hspace{.11em}४०॥} \veg\dontdisplaylinenum }%
 
{\devanagarifont शङ्कर उवाच {\dandab}\dontdisplaylinenum  }%
     \var{{\devanagarifontvar\numemph\vo शङ्कर\lem \mssALL,\hskip.2em plus .9em महेश्वर \Ed}}% 

{\devanagarifont तूष्णीम्भूता ततो भार्या अश्रुपूर्णाकुलेक्षणा \thinspace{\danda} \dontdisplaylinenum }%
     \var{{\devanagarifontvar\numnoemph\va तूष्णीम्भूता\lem \msCa,\hskip.2em plus .9em तूष्णीभूत्वा \msCb,\hskip.5em plus .9em तुष्णीभूत \msCc,\hskip.5em plus .9em तूष्णीभूता \msNa\msNb,\hskip.5em plus .9em 
तुष्णीम्भूती \msNc,\hskip.5em plus .9em तूष्णीभूतां \Ed\oo 
 भार्या\lem \mssALL,\hskip.2em plus .9em भार्यां \Ed}}% 
    \var{{\devanagarifontvar\numnoemph\vb ॰क्षणा\lem \msCa\msCb\msNa\msNc,\hskip.2em plus .9em ॰क्षणः \msCc,\hskip.5em plus .9em ॰क्षणाः \msNb,\hskip.5em plus .9em ॰क्षणाम् \Ed}}% 

%Verse 12:41

{\devanagarifont करे गृह्य विशालाक्षी ब्राह्मणाय निवेदिता {॥ १२:\hspace{.11em}४१॥} \veg\dontdisplaylinenum }%
     \var{{\devanagarifontvar\numnoemph\vc ॰क्षी\lem \mssALL,\hskip.2em plus .9em ॰क्षीं \Ed}}% 
    \var{{\devanagarifontvar\numnoemph\vd ब्राह्मणाय निवेदिता\lem \mssALL,\hskip.2em plus .9em 
ब्राह्मय दिवेदिता \msCb}}% 

{\devanagarifont यानि सन्ति गृहे द्रव्यं हिरण्यं पशवस्तथा \thinspace{\dandab} \dontdisplaylinenum }%
     \var{{\devanagarifontvar\numemph\vb हिरण्यं\lem \mssALL,\hskip.2em plus .9em हिरण्य॰ \msNa\Ed}}% 

%Verse 12:42

{\devanagarifont ददामि ते द्विजश्रेष्ठ ग्रामघोषगृहादिकम् {॥ १२:\hspace{.11em}४२॥} \veg\dontdisplaylinenum }%
     \var{{\devanagarifontvar\numnoemph\vc ददामि\lem \mssALL,\hskip.2em plus .9em ददानि \msCb\oo 
 ते द्विज॰\lem \mssALL,\hskip.2em plus .9em \lacwithnum{2}  ज॰ \msCa,\hskip.5em plus .9em त द्विज॰ \msNc}}% 

{\devanagarifont मुक्तावैडूर्यवासांसि दिव्याण्याभरणानि च \thinspace{\dandab} \dontdisplaylinenum }%
     \var{{\devanagarifontvar\numemph\va ॰वैडूर्य॰\lem \msCa\msCb\msNb\msNc,\hskip.2em plus .9em ॰वैभार्य॰ \msCc,\hskip.5em plus .9em ॰वैर्य॰ \msNaacorr,\hskip.5em plus .9em 
॰वैदूर्य॰ \msNapcorr\Ed\oo 
 ॰वासांसि\lem \mssALL,\hskip.2em plus .9em ॰वासासि \msNc}}% 

%Verse 12:43

{\devanagarifont सर्वान्गृहाण विप्रेन्द्र श्रद्धया दत्तसत्कृतान् {॥ १२:\hspace{.11em}४३॥} \veg\dontdisplaylinenum }%
     \var{{\devanagarifontvar\numnoemph\vc सर्वान्गृहाण\lem \msCa\msCb\msNa\Ed,\hskip.2em plus .9em सर्वान्तान्गृह्ण \msCc,\hskip.5em plus .9em 
सर्वान्गृहान् \msNb,\hskip.5em plus .9em 
सर्वां गृहाण \msNc}}% 
    \var{{\devanagarifontvar\numnoemph\vd ॰सत्कृतान्\lem \eme,\hskip.2em plus .9em ॰सत्कृताम् \mssCaCbCc\msNa\msNc\Ed,\hskip.5em plus .9em ॰सत्कृतम् \msNb}}% 

{\devanagarifont प्रीयतां भगवान्धर्मः प्रीयतां च महेश्वरः \thinspace{\dandab} \dontdisplaylinenum }%
     \var{{\devanagarifontvar\numemph\vb प्रीय॰\lem \mssALL,\hskip.2em plus .9em प्रीन॰ \msNcacorr}}% 

%Verse 12:44

{\devanagarifont प्रीयन्तां पितरः सर्वे यद्यस्ति सुकृतं फलम् {॥ १२:\hspace{.11em}४४॥} \veg\dontdisplaylinenum }%
     \var{{\devanagarifontvar\numnoemph\vc प्रीयन्तां\lem \msCa,\hskip.2em plus .9em प्रीयतां \msCb\msCc\msNa\msNc\Ed,\hskip.5em plus .9em प्रीयता \msNb\oo 
 पितरः\lem \mssALL,\hskip.2em plus .9em पितर \msNa}}% 
    \var{{\devanagarifontvar\numnoemph\vd अस्ति\lem \mssALL,\hskip.2em plus .9em असि \msCa}}% 

{\devanagarifont रुद्र उवाच {\dandab}\dontdisplaylinenum  }%
     \var{{\devanagarifontvar\numemph\vo रुद्र\lem \mssALL,\hskip.2em plus .9em महेश्वर \Ed}}% 

{\devanagarifont विपुलस्य वचः श्रुत्वा ब्राह्मणेन तपस्विना \thinspace{\danda} \dontdisplaylinenum }%
     \var{{\devanagarifontvar\numnoemph\va वचः श्रुत्वा\lem \mssALL,\hskip.2em plus .9em 
वच\uncl{श्श्रु}\lacwithnum{1}\  \msCa}}% 
    \var{{\devanagarifontvar\numnoemph\vb तपस्विना\lem \mssALL,\hskip.2em plus .9em तपस्विनाम् \msNb}}% 

%Verse 12:45

{\devanagarifont आशीः सुविपुलं दत्त्वा विपुलाय महात्मने {॥ १२:\hspace{.11em}४५॥} \veg\dontdisplaylinenum }%
 
{\devanagarifont वसेत्तत्र गृहे रम्ये भार्यामादाय तस्य च \thinspace{\dandab} \dontdisplaylinenum }%
     \var{{\devanagarifontvar\numemph\va वसेत्तत्र गृहे\lem \msCb\msNa,\hskip.2em plus .9em वस तत्र गृहे \msCa\msCc\msNb,\hskip.5em plus .9em 
वस्\uncl{एन्त}त्र गृहे \msNc,\hskip.5em plus .9em 
वसते च गृहं \Ed}}% 

%Verse 12:46

{\devanagarifont विपुलस्तु नमस्कृत्वा कृत्वा चापि प्रदक्षिणम् {॥ १२:\hspace{.11em}४६॥} \veg\dontdisplaylinenum }%
     \var{{\devanagarifontvar\numnoemph\vc विपुलस्तु\lem \mssALL,\hskip.2em plus .9em विपुलस्य \msNb}}% 
    \var{{\devanagarifontvar\numnoemph\vd कृत्वा चापि\lem \mssALL,\hskip.2em plus .9em \lk\lk \lk\lk\ \msNc,\hskip.5em plus .9em 
कृत्वा च वि॰ \Ed}}% 

{\devanagarifont ब्राह्मणमभिवाद्यैवं गतः शीघ्रं वनान्तरम् \thinspace{\dandab} \dontdisplaylinenum }%
     \var{{\devanagarifontvar\numemph\va ब्राह्मण॰\lem \mssALL,\hskip.2em plus .9em ब्राह्मणा॰ \msNb\oo 
 ॰द्यैवं\lem \eme,\hskip.2em plus .9em ॰द्येवं \msCa\msCc\msNa\msNb\Ed,\hskip.5em plus .9em ॰द्येनं \msCb,\hskip.5em plus .9em 
॰द्यवं \msNc}}% 
    \var{{\devanagarifontvar\numnoemph\vb शीघ्रं\lem \mssALL,\hskip.2em plus .9em श्रीघ्रं \msNb}}% 

%Verse 12:47

{\devanagarifont वने मूलफलाहारो विचरेत महीतले {॥ १२:\hspace{.11em}४७॥} \veg\dontdisplaylinenum }%
     \var{{\devanagarifontvar\numnoemph\vc ॰फलाहारो\lem \mssALL,\hskip.2em plus .9em 
॰फाहारो \msNcacorr}}% 

{\devanagarifont एकाकी विजने शून्ये चिन्तया च परिप्लुतः \thinspace{\dandab} \dontdisplaylinenum }%
     \var{{\devanagarifontvar\numemph\va एकाकी\lem \mssALL,\hskip.2em plus .9em 
ए\uncl{का}\lacwithnum{1}\  \msCa}}% 
    \var{{\devanagarifontvar\numnoemph\vb परि॰\lem \mssALL,\hskip.2em plus .9em पलि॰ \msNc}}% 

%Verse 12:48

{\devanagarifont क्व गच्छामि क्व भोक्ष्यामि कुत्र वा किं करोम्यहम् {॥ १२:\hspace{.11em}४८॥} \veg\dontdisplaylinenum }%
     \var{{\devanagarifontvar\numnoemph\vc क्व गच्छामि\lem \mssALL,\hskip.2em plus .9em क्ष गच्छामि \msNc\oo 
 क्व भोक्ष्यामि\lem \msCa,\hskip.2em plus .9em क्व भोज्यामि \msCb\msNa\msNb,\hskip.5em plus .9em क्व भोक्ष्यानि \msCc,\hskip.5em plus .9em 
क्व भोक्षामि \msNc,\hskip.5em plus .9em किं भोक्ष्यामि \Ed\ \unmetr}}% 

{\devanagarifont न पथं विषयं वेद्मि ग्रामं वा नगराणि वा \thinspace{\dandab} \dontdisplaylinenum }%
     \var{{\devanagarifontvar\numemph\va विषयं वेद्मि\lem \msCa\msNa\msNb\Ed,\hskip.2em plus .9em विषमं वेद्मि \msCb\msCc,\hskip.5em plus .9em वियषं वे\uncl{श्मि} \msNc}}% 
    \var{{\devanagarifontvar\numnoemph\vb वा\lem \mssALL,\hskip.2em plus .9em च \msCb\msNa}}% 

%Verse 12:49

{\devanagarifont खेटखर्वटदेशं वा जानामीह न कंचन {॥ १२:\hspace{.11em}४९॥} \veg\dontdisplaylinenum }%
     \var{{\devanagarifontvar\numnoemph\vc खेट॰\lem \mssALL,\hskip.2em plus .9em क्षेत्र॰ \msCc\oo 
 ॰खर्वट॰\lem \Ed,\hskip.2em plus .9em ॰कर्पट॰ \mssCaCbCc\msNa\msNb\msNc}}% 
    \var{{\devanagarifontvar\numnoemph\vd कंचन\lem \eme,\hskip.2em plus .9em कश्चन \mssCaCbCc\msNa\msNb\msNc\Ed}}% 

{\devanagarifont अमुं सुशैलं पश्यामि विपुलोदरकन्दरम् \thinspace{\dandab} \dontdisplaylinenum }%
     \var{{\devanagarifontvar\numemph\va सुशैलं\lem \mssALL,\hskip.2em plus .9em सुशेलं \msNc}}% 
    \var{{\devanagarifontvar\numnoemph\vb विपुलो॰\lem \mssALL,\hskip.2em plus .9em विलो॰ \msNb}}% 

%Verse 12:50

{\devanagarifont तमारुह्य निरीक्ष्यामि ग्रामं नगरपत्तनम् {॥ १२:\hspace{.11em}५०॥} \veg\dontdisplaylinenum }%
     \var{{\devanagarifontvar\numnoemph\vc निरीक्ष्यामि\lem \mssALL,\hskip.2em plus .9em निरीक्षामि \msNc}}% 

{\devanagarifont एवमुक्त्वा तु विपुलः शनैः पर्वतमारुहत् \thinspace{\dandab} \dontdisplaylinenum }%
     \var{{\devanagarifontvar\numemph\va एवमु॰\lem \mssALL,\hskip.2em plus .9em एकं उ॰ \msCb}}% 
    \var{{\devanagarifontvar\numnoemph\vb ॰रुहत्\lem \Ed,\hskip.2em plus .9em ॰रुहेत् \mssCaCbCc\msNa\msNb\msNc}}% 

%Verse 12:51

{\devanagarifont वृक्षच्छायां समालोक्य निषसाद श्रमान्वितः {॥ १२:\hspace{.11em}५१॥} \veg\dontdisplaylinenum }%
     \var{{\devanagarifontvar\numnoemph\vc ॰च्छायां\lem \mssALL,\hskip.2em plus .9em ॰च्छाया \msNc}}% 

{\devanagarifont एतस्मिन्नेव काले तु वृक्षशाखावतार्य च \thinspace{\dandab} \dontdisplaylinenum }%
     \var{{\devanagarifontvar\numemph\va एतस्मिन्नेव\lem \mssALL,\hskip.2em plus .9em एतस्मिंनैव \msCc,\hskip.5em plus .9em एतस्मिन्नैव \msNc\oo 
 काले तु\lem \msCa\msCb\msNa\msNb,\hskip.2em plus .9em कालेन \msCc\Ed,\hskip.5em plus .9em कालेनु \msNc}}% 
    \var{{\devanagarifontvar\numnoemph\vb वृक्ष॰\lem \mssALL,\hskip.2em plus .9em वृक्षा॰ \msNa\msNcacorr}}% 

%Verse 12:52

{\devanagarifont अपूर्वं च सुरूपं च सुगन्धत्वं च शोभनम् {॥ १२:\hspace{.11em}५२॥} \veg\dontdisplaylinenum }%
     \var{{\devanagarifontvar\numnoemph\vc सुरूपं\lem \mssALL,\hskip.2em plus .9em स्वरूपं \msCb\msNa}}% 

{\devanagarifont फलं गृह्य विचित्रं च हृदयानन्दनं शुभम् \thinspace{\dandab} \dontdisplaylinenum }%
 
%Verse 12:53

{\devanagarifont विपुलस्याग्रतः कृत्वा पुनर्वृक्षं समारुहत् {॥ १२:\hspace{.11em}५३॥} \veg\dontdisplaylinenum }%
     \var{{\devanagarifontvar\numemph\vd पुनर्वृक्षं समारुहत्\lem \mssALL,\hskip.2em plus .9em 
पुन वृक्ष समारुहम् \msCc,\hskip.5em plus .9em 
पुनर्वृक्ष समारुहं \msNb}}% 

{\devanagarifont विपुलश्चित्रवद्दृष्ट्वा विस्मयं परमं गतः \thinspace{\dandab} \dontdisplaylinenum }%
     \var{{\devanagarifontvar\numemph\va ॰त्रवद्दृष्ट्वा\lem \mssALL,\hskip.2em plus .9em ॰त्रव दृष्ट्वा \msCc}}% 

%Verse 12:54

{\devanagarifont अहो वा स्वप्नभूतो ऽस्मि अहो वा तपसः फलम् {॥ १२:\hspace{.11em}५४॥} \veg\dontdisplaylinenum }%
     \var{{\devanagarifontvar\numnoemph\vcd ॰भूतो ऽस्मि अहो\lem \mssALL,\hskip.2em plus .9em ॰संभूतो \uncl{स्म्य}हो \msNa}}% 

{\devanagarifont न पश्यामि न जिघ्रामि न च स्वादं च वेद्म्यहम् \thinspace{\dandab} \dontdisplaylinenum }%
     \var{{\devanagarifontvar\numemph\va जिघ्रामि\lem \mssALL,\hskip.2em plus .9em च घ्रामि \msCb}}% 

%Verse 12:55

{\devanagarifont वार्त्तापि न च मे श्रोता प्रतिजानामि कंचन {॥ १२:\hspace{.11em}५५॥} \veg\dontdisplaylinenum }%
     \var{{\devanagarifontvar\numnoemph\vc श्रोता\lem \mssALL,\hskip.2em plus .9em श्रोत्रा \msCa}}% 
    \var{{\devanagarifontvar\numnoemph\vd कंचन\lem \eme,\hskip.2em plus .9em कश्चन \mssCaCbCc\msNa\msNb\msNc\Ed}}% 

{\devanagarifont एवमुक्त्वा ह्यनेकानि फलं गृह्य मनोरमम् \thinspace{\dandab} \dontdisplaylinenum }%
     \var{{\devanagarifontvar\numemph\va ॰मुक्त्वा\lem \mssALL,\hskip.2em plus .9em ॰मुक्ता \msCc}}% 
    \var{{\devanagarifontvar\numnoemph\vb गृह्य\lem \mssALL,\hskip.2em plus .9em गृह \msNc}}% 

%Verse 12:56

{\devanagarifont सुनिरीक्ष्य पुनर्जिघ्रन् पुनर्जिघ्रन्निरीक्ष्य च {॥ १२:\hspace{.11em}५६॥} \veg\dontdisplaylinenum }%
     \var{{\devanagarifontvar\numnoemph\vc ॰निरीक्ष्य\lem \mssALL,\hskip.2em plus .9em ॰निरीक्ष \msNc}}% 
    \var{{\devanagarifontvar\numnoemph\vcd पुनर्जिघ्रन्पुनर्जिघ्रन्\lem \msCa\msCb\msNa\Ed,\hskip.2em plus .9em 
मुन जिघ्रं पुन जिघ्रं \msCc,\hskip.5em plus .9em 
पुनर्जिघ्र पुनर्जिघ्रं \msNb,\hskip.5em plus .9em 
पुनर्जिघ्र पुनर्जिघ्र \msNc}}% 
    \var{{\devanagarifontvar\numnoemph\vd निरीक्ष्य\lem \mssALL,\hskip.2em plus .9em निरीक्ष \msNc}}% 

{\devanagarifont फलं चात्र निरूप्यन्तो देशं वाप्यवलोकयन् \thinspace{\dandab} \dontdisplaylinenum }%
     \var{{\devanagarifontvar\numemph\va चात्र\lem \mssALL,\hskip.2em plus .9em 
चा \msCaacorr,\hskip.5em plus .9em चा\uncl{त्र} \msCapcorr\oo 
 निरूप्यन्तो\lem \Ed,\hskip.2em plus .9em निरूप्यान्ति \msCa,\hskip.5em plus .9em निरूप्यां चा \msCb,\hskip.5em plus .9em 
निरूप्यन्ति \msCc\msNa\msNb\msNc}}% 
    \var{{\devanagarifontvar\numnoemph\vb ॰लोकयन्\lem \mssALL,\hskip.2em plus .9em ॰लोकयत् \msCb}}% 

%Verse 12:57

{\devanagarifont पाथेयरहितश्चास्मि देवदत्तं फलं मम {॥ १२:\hspace{.11em}५७॥} \veg\dontdisplaylinenum }%
     \var{{\devanagarifontvar\numnoemph\vc पाथेय॰\lem \mssALL,\hskip.2em plus .9em पथेय॰ \msNb\oo 
 ॰रहितश्चा॰\lem \mssALL,\hskip.2em plus .9em ॰रहिते चा॰ \msCc}}% 
    \var{{\devanagarifontvar\numnoemph\vd ॰दत्तं\lem \msCa\msNa\msNc,\hskip.2em plus .9em ॰दत्त॰ \msCb\msCc\msNb\Ed\oo 
 फलं\lem \mssALL,\hskip.2em plus .9em \om\ \msNc}}% 

{\devanagarifont तत्फलं प्रतिगृह्यैव नगरं प्रविशाम्यहम् \thinspace{\dandab} \dontdisplaylinenum }%
     \var{{\devanagarifontvar\numemph\va ॰गृह्यैव\lem \msCb\msNb\Ed,\hskip.2em plus .9em ॰गृह्येव \msCa\msNc,\hskip.5em plus .9em गृहे च \msCc,\hskip.5em plus .9em ॰गृह्यैवं \msNa}}% 

%Verse 12:58

{\devanagarifont प्रार्थयित्वा तु यत्किंचिज्जीवनार्थं चराम्यहम् {॥ १२:\hspace{.11em}५८॥} \veg\dontdisplaylinenum }%
     \var{{\devanagarifontvar\numnoemph\vc तु\lem \mssALL,\hskip.2em plus .9em च \Ed}}% 
    \var{{\devanagarifontvar\numnoemph\vcd यत्किंचिज्जी॰\lem \mssALL,\hskip.2em plus .9em 
यत्किंजि जी॰ \msCc}}% 

{\devanagarifont ततः शैलमतिक्रम्य नगरं प्रविवेश ह \thinspace{\dandab} \dontdisplaylinenum }%
 
%Verse 12:59

{\devanagarifont पथि कश्चिज्जनः पृष्ठः किंनाम नगरं त्विदम् {॥ १२:\hspace{.11em}५९॥} \veg\dontdisplaylinenum }%
     \var{{\devanagarifontvar\numemph\vd नगरं त्विदम्\lem \msCa\msNa\msNc\Ed,\hskip.2em plus .9em 
नगर त्विदम् \msCb\msCc,\hskip.5em plus .9em नगरं त्विह \msNb}}% 

{\devanagarifont स होवाच पथीकेन किमपूर्वमिहागतः \thinspace{\dandab} \dontdisplaylinenum }%
     \var{{\devanagarifontvar\numemph\va स हो॰\lem \mssALL,\hskip.2em plus .9em अहो॰ \msCb\msNb\oo 
 पथीकेन\lem \mssALL,\hskip.2em plus .9em पथीको न \msNc}}% 
    \var{{\devanagarifontvar\numnoemph\vb ॰गतः\lem \mssALL,\hskip.2em plus .9em ॰तवः \msNb}}% 

%Verse 12:60

{\devanagarifont दक्षिणापथदेशो ऽयं नरवीरपुरं त्वदः {॥ १२:\hspace{.11em}६०॥} \veg\dontdisplaylinenum }%
     \var{{\devanagarifontvar\numnoemph\vc ॰पथ॰\lem \mssALL,\hskip.2em plus .9em ॰पथे \msCb}}% 
    \var{{\devanagarifontvar\numnoemph\vd ॰पुरं त्वदः\lem \msCb,\hskip.2em plus .9em 
॰पुरं त्वयः \msCa,\hskip.5em plus .9em 
॰पुरं त्वयं \msCc\msNa\msNb,\hskip.5em plus .9em 
पुरन्दरः \msNc,\hskip.5em plus .9em ॰पुरं स्वयम् \Ed}}% 

{\devanagarifont राजा सिंहजटो नाम राज्ञी तस्य च केकयी \thinspace{\dandab} \dontdisplaylinenum }%
     \var{{\devanagarifontvar\numemph\va राजा\lem \mssALL,\hskip.2em plus .9em राजा हि \msNc,\hskip.5em plus .9em राज \Ed\oo 
 ॰जटो\lem \mssALL,\hskip.2em plus .9em ॰यतो \Ed}}% 
    \var{{\devanagarifontvar\numnoemph\vb केकयी\lem \mssALL,\hskip.2em plus .9em कैकयी \msCa}}% 

%Verse 12:61

{\devanagarifont अतिवृद्धो जराग्रस्तः केकयी च तथैव च {॥ १२:\hspace{.11em}६१॥} \veg\dontdisplaylinenum }%
     \var{{\devanagarifontvar\numnoemph\vd केकयी\lem \mssALL,\hskip.2em plus .9em कैकयी \msCa\oo 
 तथैव च\lem \mssALL,\hskip.2em plus .9em तथैव र \msNc}}% 

{\devanagarifont दाता सर्वकलाज्ञश्च युद्धे वीर्यबलान्वितः \thinspace{\dandab} \dontdisplaylinenum }%
     \var{{\devanagarifontvar\numemph\va दाता\lem \mssALL,\hskip.2em plus .9em \lacwithnum{1}  ता \msCa\oo 
 ॰कला॰\lem \Ed,\hskip.2em plus .9em ॰कल॰ \mssCaCbCc\msNa\msNb\msNc}}% 
    \var{{\devanagarifontvar\numnoemph\vb युद्धे\lem \mssALL,\hskip.2em plus .9em युद्धो \msNb}}% 

%Verse 12:62

{\devanagarifont ब्रह्मण्यो वत्सलो लोके सर्वशास्त्रविशारदः {॥ १२:\hspace{.11em}६२॥} \veg\dontdisplaylinenum }%
 
{\devanagarifont विपुल उवाच {\dandab}\dontdisplaylinenum  }%
 
{\devanagarifont अत्र श्रेष्ठिमुपास्यामि नाम वा तस्य किं वद \thinspace{\danda} \dontdisplaylinenum }%
     \var{{\devanagarifontvar\numemph\va ॰पास्यामि\lem \mssALL,\hskip.2em plus .9em ॰पस्यामि \msCc}}% 
    \var{{\devanagarifontvar\numnoemph\vb नाम\lem \msCa\msCb\msNc,\hskip.2em plus .9em नामं \msCc\msNa\msNb\Ed\oo 
 वद\lem \mssALL,\hskip.2em plus .9em वदः \msCb}}% 

%Verse 12:63

{\devanagarifont कतमो देश तद्वासः कथयस्व न संशयः {॥ १२:\hspace{.11em}६३॥} \veg\dontdisplaylinenum }%
     \var{{\devanagarifontvar\numnoemph\vc देश त॰\lem \msCc\msNb,\hskip.2em plus .9em देशस्त॰ \msCa\msCb\msNa\msNc\Ed\ \unmetr}}% 
    \var{{\devanagarifontvar\numnoemph\vd कथयस्व\lem \mssALL,\hskip.2em plus .9em कथयस्य \msCb}}% 

{\devanagarifont विपुलेनैवमुक्तस्तु पथिकोवाच तं पुनः \thinspace{\dandab} \dontdisplaylinenum }%
     \var{{\devanagarifontvar\numemph\va विपुलेनैव॰\lem \mssALL,\hskip.2em plus .9em विपुलेनेव॰ \msNc}}% 

%Verse 12:64

{\devanagarifont मम भीमबलो नाम श्रेष्ठिकस्य गृहागतः {॥ १२:\hspace{.11em}६४॥} \veg\dontdisplaylinenum }%
     \var{{\devanagarifontvar\numnoemph\vc मम भीमबलो नाम\lem \mssALL,\hskip.2em plus .9em 
मम भी\lacwithnum{1}  बलो नाम \msCa,\hskip.5em plus .9em \om\ \Ed}}% 
    \var{{\devanagarifontvar\numnoemph\vd श्रेष्ठिकस्य गृहागतः\lem \mssALL,\hskip.2em plus .9em 
श्रेष्ठिकस्य गृहागतः\thinspace{\devanagarifont ॥} पथिको ऽहमिदानिञ्च\thinspace{\devanagarifont ।} 
को भवान् तस्य विषये किं वा ज्ञातुं चिकीर्षसि\thinspace{\devanagarifont ॥} \Ed}}% 

{\devanagarifont श्रेष्ठिकः पुण्डको नाम ख्यातः श्रेष्ठिक उच्यते \thinspace{\dandab} \dontdisplaylinenum }%
 
%Verse 12:65

{\devanagarifont कौतुकं तव यद्यस्ति तदागच्छ मया सह {॥ १२:\hspace{.11em}६५॥} \veg\dontdisplaylinenum }%
 
{\devanagarifont एवमस्त्विति तेनोक्तो विपुलेन महात्मना \thinspace{\dandab} \dontdisplaylinenum }%
     \var{{\devanagarifontvar\numemph\va ॰स्त्विति\lem \mssALL,\hskip.2em plus .9em ॰स्तिति \msCb\msCc\oo 
 तेनोक्तो\lem \mssALL,\hskip.2em plus .9em तोनोक्तो \msNc,\hskip.5em plus .9em तेनोक्तौ \Ed}}% 
    \var{{\devanagarifontvar\numnoemph\vb ॰त्मना\lem \mssALL,\hskip.2em plus .9em ॰त्मनाः \msNc}}% 

%Verse 12:66

{\devanagarifont तेनैव सह निर्यातः श्रेष्ठिकस्य गृहं प्रति {॥ १२:\hspace{.11em}६६॥} \veg\dontdisplaylinenum }%
     \var{{\devanagarifontvar\numnoemph\vc तेनैव\lem \mssALL,\hskip.2em plus .9em तेनेव \msNc}}% 
    \var{{\devanagarifontvar\numnoemph\vd प्रति\lem \mssALL,\hskip.2em plus .9em प्रतिः \msCc\Ed}}% 

{\devanagarifont श्रेष्ठिकः स्वगृहासीनो दृष्टः स विपुलेन तु \thinspace{\dandab} \dontdisplaylinenum }%
     \var{{\devanagarifontvar\numemph\va श्रेष्ठिकः\lem \mssALL,\hskip.2em plus .9em 
श्रेष्ठितः \msCa,\hskip.5em plus .9em श्रेष्ठिक \msNa}}% 
    \var{{\devanagarifontvar\numnoemph\vb दृष्टः स\lem \msCb\msNa\msNc\Ed,\hskip.2em plus .9em 
\uncl{दृ}\lacwithnum{2}  \msCa,\hskip.5em plus .9em दृष्ट स \msCc,\hskip.5em plus .9em दृष्टस्य \msNb}}% 

%Verse 12:67

{\devanagarifont तस्यान्तिकमुपागम्य तत्फलं स निवेदितः {॥ १२:\hspace{.11em}६७॥} \veg\dontdisplaylinenum }%
     \var{{\devanagarifontvar\numnoemph\vc ॰गम्य\lem \mssALL,\hskip.2em plus .9em ॰गत्य \msNc}}% 
    \var{{\devanagarifontvar\numnoemph\vd स निवेदितः\lem \mssALL,\hskip.2em plus .9em 
सन्निवेदितः \msNa,\hskip.5em plus .9em संनिवेदितः \msNc}}% 

{\devanagarifont अहो फलमिदं श्रेष्ठमहो फलमिहानितम् \thinspace{\dandab} \dontdisplaylinenum }%
     \var{{\devanagarifontvar\numemph\vab श्रेष्ठमहो\lem \mssALL,\hskip.2em plus .9em श्रेष्ठ अहो \msCc}}% 

%Verse 12:68

{\devanagarifont अहो रूपमहो गन्धमहो फलं सुशोभनम् {॥ १२:\hspace{.11em}६८॥} \veg\dontdisplaylinenum }%
     \var{{\devanagarifontvar\numnoemph\vcd गन्धमहो फलं\lem \corr,\hskip.2em plus .9em 
गन्धमहो फल \msCa\msCbpcorr\msCc\msNa\Ed,\hskip.5em plus .9em 
गन्धमहो गन्धमहो फल \msCbacorr,\hskip.5em plus .9em 
गन्ध अहो फल \msNb,\hskip.5em plus .9em गन्धो फलं अहो \msNc}}% 

{\devanagarifont तत्फलं न महीजातं न मेरौ न च मन्दरे \thinspace{\dandab} \dontdisplaylinenum }%
     \var{{\devanagarifontvar\numemph\va तत्फ॰\lem \mssALL,\hskip.2em plus .9em यत्फ॰ \Ed}}% 
    \var{{\devanagarifontvar\numnoemph\vb मेरौ\lem \msCa\msCb\msNa\msNcpcorr\Ed,\hskip.2em plus .9em मेरो \msCc\msNb\msNcacorr\oo 
 मन्दरे\lem \conj,\hskip.2em plus .9em कन्दरे \mssCaCbCc\msNa\msNb\msNc\Ed}}% 

%Verse 12:69

{\devanagarifont देवलोकिक सुव्यक्तं न मर्त्यमुपजायते {॥ १२:\hspace{.11em}६९॥} \veg\dontdisplaylinenum }%
     \var{{\devanagarifontvar\numnoemph\vc देवलोकिक\lem \mssALL,\hskip.2em plus .9em 
देवलोकि \msNbacorr}}% 
    \var{{\devanagarifontvar\numnoemph\vd मर्त्यमुपजायते\lem \msCc\msNa\msNb\msNc,\hskip.2em plus .9em 
मर्त्य\uncl{मुपजा}\lacwithnum{2}  \msCa,\hskip.5em plus .9em 
मर्त्य सुपजायते \msCb,\hskip.5em plus .9em मह्यामुपजायते \Ed}}% 

{\devanagarifont अहो ऽस्मि स फलं भोक्ता राजार्हं च न संशयः \thinspace{\dandab} \dontdisplaylinenum }%
     \var{{\devanagarifontvar\numemph\va अहो\lem \mssALL,\hskip.2em plus .9em \lacwithnum{1}  हो \msCa,\hskip.5em plus .9em अद्यो \Ed\oo 
 स फलं\lem \mssALL,\hskip.2em plus .9em 
\uncl{स}फलम् \msCa,\hskip.5em plus .9em तत्फलं \Ed\oo 
 भोक्ता\lem \mssALL,\hskip.2em plus .9em भोक्तं \msNc}}% 
    \var{{\devanagarifontvar\numnoemph\vb राजार्हं च\lem \msCc\msNb,\hskip.2em plus .9em राजार्हश्च \msCa\msCb\msNc\Ed,\hskip.5em plus .9em 
राजार्ह\uncl{श्च} \msNa}}% 

%Verse 12:70

{\devanagarifont ढौकयित्वा फलं दिव्यं राजानं तोषयाम्यहम् {॥ १२:\hspace{.11em}७०॥} \veg\dontdisplaylinenum }%
     \var{{\devanagarifontvar\numnoemph\vc ढौकयित्वा\lem \mssALL,\hskip.2em plus .9em ढोकयित्वा \msNb}}% 

{\devanagarifont ततस्त्वरित गत्वैव फलं गृह्य मनोहरम् \thinspace{\dandab} \dontdisplaylinenum }%
     \var{{\devanagarifontvar\numemph\va त्वरित\lem \msNa\msNc\Ed,\hskip.2em plus .9em त्वरितं \mssCaCbCc\msNb\ \unmetr}}% 
    \var{{\devanagarifontvar\numnoemph\vb गृह्य\lem \mssALL,\hskip.2em plus .9em गृह \msCb\oo 
 ॰हरम्\lem \mssALL,\hskip.2em plus .9em ॰रमम् \msNb\Ed}}% 

%Verse 12:71

{\devanagarifont आदरेणोपसृत्यैव राजानं स फलं ददौ {॥ १२:\hspace{.11em}७१॥} \veg\dontdisplaylinenum }%
     \var{{\devanagarifontvar\numnoemph\vc ॰सृत्यैव\lem \msCa\msCb\Ed,\hskip.2em plus .9em ॰सृत्येव \msCc\msNb\msNc,\hskip.5em plus .9em ॰संगत्य \msNa}}% 
    \var{{\devanagarifontvar\numnoemph\vd स फलं\lem \mssALL,\hskip.2em plus .9em तत्फलं \Ed}}% 

{\devanagarifont राजा च स फलं दृष्ट्वा विस्मयं परमं गतः \thinspace{\dandab} \dontdisplaylinenum }%
     \var{{\devanagarifontvar\numemph\va स फलं\lem \mssALL,\hskip.2em plus .9em तत्फलं \Ed}}% 
    \var{{\devanagarifontvar\numnoemph\vb विस्मयं\lem \mssALL,\hskip.2em plus .9em विस्मय \msNb}}% 

%Verse 12:72

{\devanagarifont कुतः श्रेष्ठि त्वया नीतं फलं पूर्वं मनोहरम् {॥ १२:\hspace{.11em}७२॥} \veg\dontdisplaylinenum }%
     \var{{\devanagarifontvar\numnoemph\vc श्रेष्ठि\lem \mssALL,\hskip.2em plus .9em श्रेष्ठ \Ed}}% 
    \var{{\devanagarifontvar\numnoemph\vd फलं पूर्वं मनोहरम्\lem \corr,\hskip.2em plus .9em फल\lacwithnum{4}  हरम् \msCa,\hskip.5em plus .9em 
फल\uncl{म्य}र्वमनोहरम् \msCb,\hskip.5em plus .9em 
फलं पूर्व मनोहरम् \msCc\msNa\msNb\msNc,\hskip.5em plus .9em 
फलं सर्वमनोहरम् \Ed}}% 

{\devanagarifont स्वादुमूलं फलं कन्दं दृष्टं पूर्वं न तादृशम् \thinspace{\dandab} \dontdisplaylinenum }%
     \var{{\devanagarifontvar\numemph\va ॰मूलं फलं\lem \msNc,\hskip.2em plus .9em ॰मूलफल॰ \mssCaCbCc\msNa\msNb\Ed}}% 
    \var{{\devanagarifontvar\numnoemph\vab कन्दं दृष्टं पू॰\lem \eme,\hskip.2em plus .9em ॰कन्दं दृष्ट्वा पू॰ \msCa\msNa\msNb,\hskip.5em plus .9em 
॰स्कन्द दृष्ट्वा पू॰ \msCb,\hskip.5em plus .9em 
॰स्कन्द दृष्ट पू॰ \msCc,\hskip.5em plus .9em कन्द दृष्ट\uncl{न्पू}॰ \msNc,\hskip.5em plus .9em 
॰स्कन्द दृष्टा पू॰ \Ed}}% 
    \var{{\devanagarifontvar\numnoemph\vb तादृशम्\lem \mssALL,\hskip.2em plus .9em 
तादृ\uncl{शं} \msCc,\hskip.5em plus .9em यादृशम् \Ed}}% 

%Verse 12:73

{\devanagarifont रूपगन्धगुणोपेतं हृदयानन्दकारकम् {॥ १२:\hspace{.11em}७३॥} \veg\dontdisplaylinenum }%
     \var{{\devanagarifontvar\numnoemph\vd ॰कारकम्\lem \mssALL,\hskip.2em plus .9em ॰कारकः \msNa}}% 

{\devanagarifont सद्य एवोपयुञ्जामि त्वया दत्तमिदं फलम् \thinspace{\dandab} \dontdisplaylinenum }%
     \var{{\devanagarifontvar\numemph\va सद्य एवोपयुञ्जामि\lem \mssALL,\hskip.2em plus .9em 
सत्य एव प्रभुञ्जामि \Ed}}% 

%Verse 12:74

{\devanagarifont कीदृशं स्वाद विज्ञानमिच्छामि कुरु माचिरम् {॥ १२:\hspace{.11em}७४॥} \veg\dontdisplaylinenum }%
     \var{{\devanagarifontvar\numnoemph\vc स्वाद विज्ञानम्\lem \mssALL,\hskip.2em plus .9em स्वादु विज्ञातुम् \Ed}}% 

{\devanagarifont ततः स भक्षयामास फलं चामृतसंनिभम् \thinspace{\dandab} \dontdisplaylinenum }%
     \var{{\devanagarifontvar\numemph\va ततः\lem \mssALL,\hskip.2em plus .9em तत \msCb}}% 

%Verse 12:75

{\devanagarifont अमृतोपमसुस्वादं सर्वं च बुभुजे नृपः {॥ १२:\hspace{.11em}७५॥} \veg\dontdisplaylinenum }%
     \var{{\devanagarifontvar\numnoemph\vcd स्वादं सर्वं च\lem \mssALL,\hskip.2em plus .9em 
स्वा\lacwithnum{4}  \msCa}}% 

{\devanagarifont सद्यः षोडशवर्षस्य यौवनं समपद्यत \thinspace{\dandab} \dontdisplaylinenum }%
     \var{{\devanagarifontvar\numemph\va सद्यः\lem \corr,\hskip.2em plus .9em \mssCaCbCc\msNa\msNb\msNc\Ed}}% 
    \var{{\devanagarifontvar\numnoemph\vb ॰पद्यत\lem \msCa\msCb,\hskip.2em plus .9em ॰पद्यते \msCc\msNa\msNb\Ed,\hskip.5em plus .9em ॰द्यत \msNc}}% 

%Verse 12:76

{\devanagarifont न वलीपलितं सद्यो न जरा न च दुर्बलः {॥ १२:\hspace{.11em}७६॥} \veg\dontdisplaylinenum }%
     \var{{\devanagarifontvar\numnoemph\vc वली॰\lem \mssALL,\hskip.2em plus .9em वलि॰ \Ed}}% 

{\devanagarifont केशदन्तनखस्निग्धो दृढदन्तो दृढेन्द्रियः \thinspace{\dandab} \dontdisplaylinenum }%
     \var{{\devanagarifontvar\numemph\vb ॰दन्तो\lem \mssALL,\hskip.2em plus .9em ॰देहो \Ed\oo 
 दृढेन्द्रियः\lem \mssALL,\hskip.2em plus .9em दृढेन्द्रिः \msNb}}% 

%Verse 12:77

{\devanagarifont तेजश्चक्षुर्बलप्राणान्सद्यः सर्वानवाप्तवान् {॥ १२:\hspace{.11em}७७॥} \veg\dontdisplaylinenum }%
     \var{{\devanagarifontvar\numnoemph\vc ॰चक्षुर्बलप्राणा॰\lem \msCa\msCb\msNa\msNb,\hskip.2em plus .9em ॰चक्षुवलप्राणा॰ \msCc,\hskip.5em plus .9em 
॰चक्षुर्बलं प्राणा॰ \msNc,\hskip.5em plus .9em ॰चक्षुवलप्राण॰ \Ed}}% 
    \var{{\devanagarifontvar\numnoemph\vd ॰न्सद्यः\lem \corr,\hskip.2em plus .9em ॰न्सद्य \mssCaCbCc\msNa\msNb\msNc\Ed\oo 
 सर्वान॰\lem \mssALL,\hskip.2em plus .9em सर्व्वान्न॰ \msCc\oo 
 ॰प्तवान्\lem \mssALL,\hskip.2em plus .9em ॰प्तुयात् \msNa}}% 

{\devanagarifont मन्त्री पुरोहितो ऽमात्यः सर्वे भृत्यजनास्तथा \thinspace{\dandab} \dontdisplaylinenum }%
     \var{{\devanagarifontvar\numemph\va पुरोहितो ऽमात्यः\lem \msCa\msCc\msNb,\hskip.2em plus .9em 
पुरोहितो मात्य \msCb\msNa\msNc,\hskip.5em plus .9em पुरोहितामात्य \Ed}}% 
    \var{{\devanagarifontvar\numnoemph\vb सर्वे भृत्यजनास्तथा\lem \mssALL,\hskip.2em plus .9em 
जनास्तथास्तथा \msCb}}% 

%Verse 12:78

{\devanagarifont पौरस्त्री बालवृद्धाश्च सर्वे ते विस्मयं गताः {॥ १२:\hspace{.11em}७८॥} \veg\dontdisplaylinenum }%
     \var{{\devanagarifontvar\numnoemph\vc ॰स्त्री\lem \mssALL,\hskip.2em plus .9em ॰स्त्रि \Ed}}% 
    \var{{\devanagarifontvar\numnoemph\vd सर्वे\lem \mssALL,\hskip.2em plus .9em \lacwithnum{2}  \msCa\oo 
 गताः\lem \mssALL,\hskip.2em plus .9em गतः \msCc}}% 

{\devanagarifont राजा सिंहजटो नाम तुष्टिमेव परां गतः \thinspace{\dandab} \dontdisplaylinenum }%
     \var{{\devanagarifontvar\numemph\vb परां\lem \mssALL,\hskip.2em plus .9em परं \msNb}}% 

%Verse 12:79

{\devanagarifont प्रहर्षमतुलं चैव प्राप्तवान्स नरेश्वरः {॥ १२:\hspace{.11em}७९॥} \veg\dontdisplaylinenum }%
 
{\devanagarifont उवाच राजा तं श्रेष्ठिं स्वार्थतत्परनिर्दयः \thinspace{\dandab} \dontdisplaylinenum }%
     \var{{\devanagarifontvar\numemph\va राजा तं\lem \mssALL,\hskip.2em plus .9em राजनं \msNb\oo 
 श्रेष्ठिं\lem \mssALL,\hskip.2em plus .9em श्रेष्ठं \Ed}}% 
    \var{{\devanagarifontvar\numnoemph\vb ॰दयः\lem \mssALL,\hskip.2em plus .9em ॰दय \Ed}}% 

%Verse 12:80

{\devanagarifont कुरु भीमबलस्त्वेवं फलमानय अद्य वै {॥ १२:\hspace{.11em}८०॥} \veg\dontdisplaylinenum }%
     \var{{\devanagarifontvar\numnoemph\vc कुरु\lem \mssALL,\hskip.2em plus .9em शृणु \Ed\oo 
 भीमबलस्त्वेवं\lem \msCb\msCc\msNa,\hskip.2em plus .9em भीमवस्त्वेवं \msCa\Ed,\hskip.5em plus .9em 
भीमबलस्त्वेव \msNb,\hskip.5em plus .9em 
भीमबल\uncl{म्त्वे}वं \msNc}}% 

{\devanagarifont पुनर्मे यौवनप्राप्तिस्त्वत्प्रसादान्नरोत्तम \thinspace{\dandab} \dontdisplaylinenum }%
     \var{{\devanagarifontvar\numemph\vb ॰त्तम\lem \mssALL,\hskip.2em plus .9em ॰त्तमः \Ed}}% 

%Verse 12:81

{\devanagarifont केकयीं दुर्बलां वृद्धां पुनः प्रापय यौवनम् {॥ १२:\hspace{.11em}८१॥} \veg\dontdisplaylinenum }%
     \var{{\devanagarifontvar\numnoemph\vc केकयीं दुर्बलां\lem \msNa,\hskip.2em plus .9em कैकयीन्दुर्बलान् \msCa,\hskip.5em plus .9em केकयीं \msCb,\hskip.5em plus .9em 
केकयी दुर्बला \msCc\msNb\Ed,\hskip.5em plus .9em कैकयी दुर्बलां \msNc}}% 
    \var{{\devanagarifontvar\numnoemph\vcd वृद्धां पुनः\lem \msCb\msNa\msNb\msNc,\hskip.2em plus .9em वृ\uncl{द्धा}\lacwithnum{2}  \msCa,\hskip.5em plus .9em 
वृद्धा पुनः \msCc\Ed}}% 
    \var{{\devanagarifontvar\numnoemph\vd प्रापय\lem \mssALL,\hskip.2em plus .9em प्राप \msCc}}% 

{\devanagarifont स राज्ञा एवमुक्तस्तु श्रेष्ठी भीमबलस्तथा \thinspace{\dandab} \dontdisplaylinenum }%
     \var{{\devanagarifontvar\numemph\vb श्रेष्ठी\lem \msCc\Ed,\hskip.2em plus .9em श्रेष्ठि \msCa\msCb\msNa\msNc,\hskip.5em plus .9em श्रिष्ठि \msNb\oo 
 ॰बलस्तथा\lem \mssALL,\hskip.2em plus .9em ॰बलस्तदा \msNb\msNc}}% 

%Verse 12:82

{\devanagarifont प्रत्युवाच ह राजानं प्राञ्जलिः प्रणतः स्थितः {॥ १२:\hspace{.11em}८२॥} \veg\dontdisplaylinenum }%
     \var{{\devanagarifontvar\numnoemph\vc ॰वाच ह\lem \mssALL,\hskip.2em plus .9em ॰वाचाह \Ed\oo 
 राजानं\lem \mssALL,\hskip.2em plus .9em राजान \msNa}}% 

{\devanagarifont न वनेन वने राजन्न वाणिज्यकृषेण वा \thinspace{\dandab} \dontdisplaylinenum }%
     \var{{\devanagarifontvar\numemph\va न वनेन\lem \mssALL,\hskip.2em plus .9em न फलेदं \Ed}}% 
    \var{{\devanagarifontvar\numnoemph\vab राजन्न\lem \mssALL,\hskip.2em plus .9em राजान्न \msCb\msNb}}% 

%Verse 12:83

{\devanagarifont केनापि कुलपुत्रेण तव दर्शनकांक्षया {॥ १२:\hspace{.11em}८३॥} \veg\dontdisplaylinenum }%
     \var{{\devanagarifontvar\numnoemph\vc कुल॰\lem \mssALL,\hskip.2em plus .9em कु॰ \msNc}}% 

{\devanagarifont दत्तो ऽस्मि तेन राजेन्द्र मया दत्तो ऽसि भूपते \thinspace{\dandab} \dontdisplaylinenum }%
     \var{{\devanagarifontvar\numemph\va ऽस्मि तेन\lem \mssALL,\hskip.2em plus .9em 
स्मिन्तेन \msNb,\hskip.5em plus .9em ऽस्मि तव \Ed}}% 
    \var{{\devanagarifontvar\numnoemph\vb दत्तो ऽसि\lem \msCa\msCb\msNb\msNc,\hskip.2em plus .9em 
दत्तासि \msCc,\hskip.5em plus .9em दत्तो स्मि \msNa,\hskip.5em plus .9em प्राप्तोषि \Ed}}% 

%Verse 12:84

{\devanagarifont न ते शक्नोम्यहं राजन्वक्तुं वैदेशिनं नरम् {॥ १२:\hspace{.11em}८४॥} \veg\dontdisplaylinenum }%
     \var{{\devanagarifontvar\numnoemph\vc ते\lem \mssALL,\hskip.2em plus .9em च \Ed}}% 
    \var{{\devanagarifontvar\numnoemph\vcd राजन्वक्तुं\lem \mssALL,\hskip.2em plus .9em 
रा\lacwithnum{2}  क्तुम् \msCa,\hskip.5em plus .9em राजान्वक्तुम् \msCc}}% 
    \var{{\devanagarifontvar\numnoemph\vd वैदेशिनं नरम्\lem \msCb\msCc\msNa\msNc,\hskip.2em plus .9em 
\uncl{वै}देशिनन्नरम् \msCa,\hskip.5em plus .9em 
वैदेशिनं नरः \msNb,\hskip.5em plus .9em च देहि तन्नरः \Ed}}% 

{\devanagarifont श्रुत्वा भीमबलवाक्यं प्रत्युवाच ततः पुनः \thinspace{\dandab} \dontdisplaylinenum }%
     \var{{\devanagarifontvar\numemph\va ॰बल॰\lem \msCa\msCb,\hskip.2em plus .9em ॰बलं \msCc\msNa\msNb\msNc\Ed}}% 

%Verse 12:85

{\devanagarifont अमात्यकुलपुत्रस्त्वं ब्रूहि मद्वचनं पुनः {॥ १२:\hspace{.11em}८५॥} \veg\dontdisplaylinenum }%
     \var{{\devanagarifontvar\numnoemph\vc अमात्य॰\lem \mssALL,\hskip.2em plus .9em अमत्य॰ \msNb\oo 
 ॰पुत्रस्त्वं\lem \mssALL,\hskip.2em plus .9em ॰पुत्रं त्वं \msNc}}% 

{\devanagarifont यदि नास्ति किं मे दत्तं मया वा मार्गितो भवान् \thinspace{\dandab} \dontdisplaylinenum }%
     \var{{\devanagarifontvar\numemph\va किं मे दत्तं\lem \msNc,\hskip.2em plus .9em किमे दत्तं \mssCaCbCc\msNa\msNb,\hskip.5em plus .9em 
किमेतत्तं \Ed}}% 
    \var{{\devanagarifontvar\numnoemph\vb मार्गितो\lem \mssALL,\hskip.2em plus .9em प्रार्थितो \Ed\oo 
 भवान्\lem \mssALL,\hskip.2em plus .9em भगवन् \msNc}}% 

%Verse 12:86

{\devanagarifont यत्र ह्येको बहवो ऽत्र जायन्ते नात्र संशयः {॥ १२:\hspace{.11em}८६॥} \veg\dontdisplaylinenum }%
     \var{{\devanagarifontvar\numnoemph\vc यत्र ह्येको बहवो ऽत्र\lem \msCa\msNa\msNb\msNc,\hskip.2em plus .9em 
यत्रैको बहवो ऽत्रैव \msCb,\hskip.5em plus .9em 
यतश्चैक बहून्तत्र \msCc,\hskip.5em plus .9em 
यत्रश्चैको बहून्तत्र \Ed}}% 
    \var{{\devanagarifontvar\numnoemph\vd जायन्ते\lem \mssALL,\hskip.2em plus .9em जायते \msCc}}% 

{\devanagarifont आगमोपायमार्गं च तेनैव स तु गम्यताम् \thinspace{\dandab} \dontdisplaylinenum }%
     \var{{\devanagarifontvar\numemph\vb तेनैव\lem \mssALL,\hskip.2em plus .9em तैनैव \msCc}}% 

%Verse 12:87

{\devanagarifont अवश्यं तेन गन्तव्यं तेन मार्गेण मार्गय {॥ १२:\hspace{.11em}८७॥} \veg\dontdisplaylinenum }%
     \var{{\devanagarifontvar\numnoemph\vc अवश्यं तेन\lem \mssALL,\hskip.2em plus .9em 
अव\uncl{स्य}\lacwithnum{1}\  न \msCa\oo 
 गन्तव्यं\lem \mssALL,\hskip.2em plus .9em 
\uncl{बुद्ध}व्यं \msCb}}% 
    \var{{\devanagarifontvar\numnoemph\vd मार्गय\lem \mssALL,\hskip.2em plus .9em मार्गयः \Ed}}% 
    \lacuna{\devanagarifontsmall \vd {\englishfont \msCc\ breaks off here missing one folio (f. 291);
                 it resumes at 12.113d on f.~292.} }%
  
{\devanagarifont अदत्त्वा फलमन्यच्च शिरश्छेद्यामि दुर्मते \thinspace{\dandab} \dontdisplaylinenum }%
     \var{{\devanagarifontvar\numemph\va अदत्त्वा\lem \mssALL,\hskip.2em plus .9em 
अदत्ता \msNb,\hskip.5em plus .9em अदत्वाफत्वा \msNcacorr}}% 

%Verse 12:88

{\devanagarifont छेद्यश्चण्डविचण्डाभ्यां रक्ष भीमबलाधमः {॥ १२:\hspace{.11em}८८॥} \veg\dontdisplaylinenum }%
     \var{{\devanagarifontvar\numnoemph\vc छेद्यश्च॰\lem \msNa,\hskip.2em plus .9em छेद्ये च॰ \msCa\msNb,\hskip.5em plus .9em 
छेदे च॰ \msCb\msNc,\hskip.5em plus .9em छेद्य च॰ \Ed}}% 
    \var{{\devanagarifontvar\numnoemph\vd ॰धमः\lem \mssALL,\hskip.2em plus .9em ॰धम \msCb}}% 

{\devanagarifont ततो भीमबलः क्रुद्धः खड्गं गृह्य शशिप्रभम् \thinspace{\dandab} \dontdisplaylinenum }%
     \var{{\devanagarifontvar\numemph\va ॰बलः\lem \mssALL,\hskip.2em plus .9em ॰बल \msNa}}% 
    \var{{\devanagarifontvar\numnoemph\vb शशिप्रभम्\lem \mssALL,\hskip.2em plus .9em शशी प्रदम् \Ed}}% 

%Verse 12:89

{\devanagarifont अलङ्घ्य वचनं राज्ञः कुलपुत्र व्रज त्वरम् {॥ १२:\hspace{.11em}८९॥} \veg\dontdisplaylinenum }%
     \var{{\devanagarifontvar\numnoemph\vc अलङ्घ्य\lem \mssALL,\hskip.2em plus .9em \lk लङ्घ्य \msNb,\hskip.5em plus .9em उवाच \Ed\oo 
 राज्ञः\lem \mssALL,\hskip.2em plus .9em राजा \msNb}}% 
    \var{{\devanagarifontvar\numnoemph\vd कुलपुत्र व्रज त्वरम्\lem \msNb\Ed,\hskip.2em plus .9em कुलपुत्रं व्रजत्यरम् \msCa\msCb,\hskip.5em plus .9em 
कुलपुत्र व्रजन्परं \msNa,\hskip.5em plus .9em 
कुलपुत्रं व्रजन्परं \msNc}}% 

{\devanagarifont मा रुष कुलपुत्र त्वं मया वध्यो भविष्यसि \thinspace{\dandab} \dontdisplaylinenum }%
     \var{{\devanagarifontvar\numemph\va ॰पुत्र त्वं\lem \mssALL,\hskip.2em plus .9em ॰पुत्रस्त्वं \Ed}}% 
    \var{{\devanagarifontvar\numnoemph\vb वध्यो\lem \mssALL,\hskip.2em plus .9em वद्ध्यौ \msNb\oo 
 भविष्यसि\lem \mssALL,\hskip.2em plus .9em भविष्यति \msNb}}% 

%Verse 12:90

{\devanagarifont सद्यो ऽस्ति फलमन्यद्वा देहि राजानमद्य वै {॥ १२:\hspace{.11em}९०॥} \veg\dontdisplaylinenum }%
     \var{{\devanagarifontvar\numnoemph\vc सद्यो ऽस्ति\lem \mssALL,\hskip.2em plus .9em 
\lacwithnum{1}  द्योस्ति \msCa,\hskip.5em plus .9em यद्यस्ति \Ed}}% 

{\devanagarifont यत्र प्राप्तं फलं दिव्यं तत्र वादेशय त्वरम् \thinspace{\dandab} \dontdisplaylinenum }%
     \var{{\devanagarifontvar\numemph\va प्राप्तं\lem \mssALL,\hskip.2em plus .9em प्राप्त॰ \msCb,\hskip.5em plus .9em प्राप्ति \Ed}}% 
    \var{{\devanagarifontvar\numnoemph\vb ॰देशय\lem \mssALL,\hskip.2em plus .9em ॰देशयत् \msNb,\hskip.5em plus .9em ॰देशयन् \Ed\oo 
 त्वरम्\lem \conj,\hskip.2em plus .9em तव \msCa\msCb\msNa\msNb\msNc\Ed}}% 

%Verse 12:91

{\devanagarifont तत्फलेन विना भद्र दुर्लभं तव जीवितम् {॥ १२:\hspace{.11em}९१॥} \veg\dontdisplaylinenum }%
 
{\devanagarifont विपुल उवाच {\dandab}\dontdisplaylinenum  }%
 
{\devanagarifont जीविताशामहं प्राप्तो वैदेशी भवनं तव \thinspace{\danda} \dontdisplaylinenum }%
     \var{{\devanagarifontvar\numemph\vb वैदेशी\lem \eme,\hskip.2em plus .9em वैदेशि \mssCaCbCc\msNa\msNb\msNc\Ed}}% 

%Verse 12:92

{\devanagarifont कृतकर्ता कथं वध्यः प्राप्नुयामहमद्य वै {॥ १२:\hspace{.11em}९२॥} \veg\dontdisplaylinenum }%
     \var{{\devanagarifontvar\numnoemph\vd प्राप्नुयाम॰\lem \mssALL,\hskip.2em plus .9em प्राप्तुयाम॰ \msNa,\hskip.5em plus .9em 
प्राप्तो ऽयम॰ \Ed\oo 
 ॰हमद्य वै\lem \mssALL,\hskip.2em plus .9em ॰हपद्य वै \msNb,\hskip.5em plus .9em 
॰हमद्य वैः \msNc}}% 

{\devanagarifont फलं वा न पुनस्त्वन्यद्दातुं शक्यं न केनचित् \thinspace{\dandab} \dontdisplaylinenum }%
     \var{{\devanagarifontvar\numemph\va वा न\lem \mssALL,\hskip.2em plus .9em वा \msCb}}% 
    \var{{\devanagarifontvar\numnoemph\vab ॰न्यद्दातुं\lem \mssALL,\hskip.2em plus .9em ॰न्य दातुं \msNc}}% 
    \var{{\devanagarifontvar\numnoemph\vb शक्यं न केनचित्\lem \mssALL,\hskip.2em plus .9em 
शक्य\lacwithnum{2}  नचित् \msCa,\hskip.5em plus .9em शक्यं न तेनचिद् \msNc}}% 

%Verse 12:93

{\devanagarifont सह्यपर्वतशैलाग्रे आसीनः श्रान्तमानसः {॥ १२:\hspace{.11em}९३॥} \veg\dontdisplaylinenum }%
     \var{{\devanagarifontvar\numnoemph\vd आसीनः\lem \mssALL,\hskip.2em plus .9em आशीतः \msCb\oo 
 श्रान्त॰\lem \mssALL,\hskip.2em plus .9em श्रोत्त॰ \msCb,\hskip.5em plus .9em सान्त॰ \msNb}}% 

{\devanagarifont वानरस्तत्फलं गृह्य मम दत्त्वा पुनर्गतः \thinspace{\dandab} \dontdisplaylinenum }%
     \var{{\devanagarifontvar\numemph\vb मम\lem \mssALL,\hskip.2em plus .9em मह्यं \Ed}}% 

%Verse 12:94

{\devanagarifont मया दत्तमिदं तुभ्यं त्वयापि च नराधिपे {॥ १२:\hspace{.11em}९४॥} \veg\dontdisplaylinenum }%
     \var{{\devanagarifontvar\numnoemph\vc तुभ्यं\lem \mssALL,\hskip.2em plus .9em तुभ्य \msNb}}% 
    \var{{\devanagarifontvar\numnoemph\vd ॰धिपे\lem \mssALL,\hskip.2em plus .9em ॰धिप \msNb}}% 

{\devanagarifont तत्र गच्छाव भो श्रेष्ठि दृश्यते यदि वानरः \thinspace{\dandab} \dontdisplaylinenum }%
 
%Verse 12:95

{\devanagarifont त्वया मया च गत्वैव याचावः प्लवगाधिपम् {॥ १२:\hspace{.11em}९५॥} \veg\dontdisplaylinenum }%
     \var{{\devanagarifontvar\numemph\vd च गत्वैव\lem \mssALL,\hskip.2em plus .9em 
\uncl{त}गवत्वैव \msNc\oo 
 याचावः प्लवगाधिपम्\lem \msCb,\hskip.2em plus .9em 
यो वासः प्लवगाधिपः \msCa\msNa\msNb\msNc\Ed}}% 

{\devanagarifont श्रेष्ठिना च तथेत्याह गच्छामः सहिता वयम् \thinspace{\dandab} \dontdisplaylinenum }%
     \var{{\devanagarifontvar\numemph\va तथेत्याह\lem \msCa\msNb\Ed,\hskip.2em plus .9em तथैत्याह \msCb\msNa\msNc}}% 
    \var{{\devanagarifontvar\numnoemph\vb गच्छामः\lem \mssALL,\hskip.2em plus .9em 
ग\lacwithnum{1}  मस् \msCa,\hskip.5em plus .9em गच्छाम \msNc}}% 

%Verse 12:96

{\devanagarifont यत्र प्राप्तं फलं तुभ्यं मोक्षयामो न संशयः {॥ १२:\hspace{.11em}९६॥} \veg\dontdisplaylinenum }%
     \var{{\devanagarifontvar\numnoemph\vc प्राप्तं\lem \mssALL,\hskip.2em plus .9em प्राप्त \Ed}}% 
    \var{{\devanagarifontvar\numnoemph\vd तुभ्यं\lem \mssALL,\hskip.2em plus .9em तुभ्य \msNb}}% 

{\devanagarifont रुद्र उवाच {\dandab}\dontdisplaylinenum  }%
 
{\devanagarifont तमारुह्य गिरिं सह्यं मार्गमाणः समन्ततः \thinspace{\danda} \dontdisplaylinenum }%
     \var{{\devanagarifontvar\numemph\va गिरिं\lem \mssALL,\hskip.2em plus .9em गिरि \msCb}}% 
    \var{{\devanagarifontvar\numnoemph\vb ॰मानः\lem \mssALL,\hskip.2em plus .9em ॰मानाः \Ed}}% 

%Verse 12:97

{\devanagarifont विपुलेन ततो दृष्टो वानरः प्लवगाधिपः {॥ १२:\hspace{.11em}९७॥} \veg\dontdisplaylinenum }%
     \var{{\devanagarifontvar\numnoemph\vd वानरः\lem \mssALL,\hskip.2em plus .9em वानर \msCb\oo 
 प्लवगा॰\lem \mssALL,\hskip.2em plus .9em प्लगा॰ \msCa}}% 

{\devanagarifont अयं स वानरश्रेष्ठो वृक्षच्छायां समाश्रितः \thinspace{\dandab} \dontdisplaylinenum }%
     \var{{\devanagarifontvar\numemph\va वानरश्रेष्ठो\lem \mssALL,\hskip.2em plus .9em वानरः श्रे\uncl{ष्ठे} \msNc,\hskip.5em plus .9em 
वानरः श्रेष्ठो \Ed}}% 
    \var{{\devanagarifontvar\numnoemph\vb वृक्षच्छायां\lem \msNc,\hskip.2em plus .9em वृक्षच्छांया॰ \msCa,\hskip.5em plus .9em वृक्षच्छाया॰ \msCb\msNb\Ed,\hskip.5em plus .9em वृच्छायां \msNa}}% 

%Verse 12:98

{\devanagarifont मम पुण्यबलेनैव दृश्यते ऽद्यापि वानरः {॥ १२:\hspace{.11em}९८॥} \veg\dontdisplaylinenum }%
 
{\devanagarifont वानर कुरु मित्रार्थं सद्यो मृत्युर्भवेन्मम \thinspace{\dandab} \dontdisplaylinenum }%
     \var{{\devanagarifontvar\numemph\va वानर\lem \mssALL,\hskip.2em plus .9em वानरं \msNb\oo 
 ॰र्थं\lem \mssALL,\hskip.2em plus .9em ॰र्थ \msCb\msNb}}% 
    \var{{\devanagarifontvar\numnoemph\vb मृत्युर्भ॰\lem \mssALL,\hskip.2em plus .9em मृत्यु भ॰ \msNa\msNb}}% 

%Verse 12:99

{\devanagarifont पूर्वदत्तं फलमन्यद्देहि वानर जीवय {॥ १२:\hspace{.11em}९९॥} \veg\dontdisplaylinenum }%
     \var{{\devanagarifontvar\numnoemph\vc ॰दत्तं\lem \msCa\msNc\Ed,\hskip.2em plus .9em ॰दत्त॰ \msCb\msNa\msNb\oo 
 फलमन्य॰\lem \mssALL,\hskip.2em plus .9em फलंमन्य॰ \msNa}}% 
    \var{{\devanagarifontvar\numnoemph\vd ॰हि वानर जीवय\lem \msCa,\hskip.2em plus .9em ॰वि वानर जीवयः \msCb,\hskip.5em plus .9em 
॰हि वानर जीवयः \msNa\msNb,\hskip.5em plus .9em 
॰हि वान जीवय \msNc,\hskip.5em plus .9em ॰हि वा न च जीवये \Ed}}% 

{\devanagarifont वानर उवाच {\dandab}\dontdisplaylinenum  }%
 
{\devanagarifont गन्धर्वेण तु मे दत्तं फलं दत्तं तु ते मया \thinspace{\danda} \dontdisplaylinenum }%
     \var{{\devanagarifontvar\numemph\va तु मे दत्तं\lem \mssALL,\hskip.2em plus .9em 
तु मे दत्त॰ \msNb,\hskip.5em plus .9em मम दत्तं \Ed}}% 

%Verse 12:100

{\devanagarifont पुनरन्यत्कथं दास्ये तत्र गच्छ यदीच्छसि {॥ १२:\hspace{.11em}१००॥} \veg\dontdisplaylinenum }%
 
{\devanagarifont विपुल उवाच {\dandab}\dontdisplaylinenum  }%
 
{\devanagarifont अदत्त्वा तत्फलं तुभ्यं जीवितुं संशयो भवेत् \thinspace{\danda} \dontdisplaylinenum }%
     \var{{\devanagarifontvar\numemph\va अदत्त्वा\lem \mssALL,\hskip.2em plus .9em अदत्ता \msNc}}% 
    \var{{\devanagarifontvar\numnoemph\vb जीवितुं\lem \mssALL,\hskip.2em plus .9em जीवितु \msNa,\hskip.5em plus .9em जीवितं \msNb\oo 
 भवेत्\lem \mssALL,\hskip.2em plus .9em \uncl{भवेत्} \msNa}}% 

%Verse 12:101

{\devanagarifont अथवा तत्र गच्छामो यत्र चित्ररथः स्वयम् {॥ १२:\hspace{.11em}१०१॥} \veg\dontdisplaylinenum }%
     \var{{\devanagarifontvar\numnoemph\vc अथवा तत्र\lem \mssALL,\hskip.2em plus .9em अ\lacwithnum{3}  त्र \msCa}}% 
    \var{{\devanagarifontvar\numnoemph\vd चित्ररथः\lem \mssALL,\hskip.2em plus .9em 
चिरथः \msCbacorr,\hskip.5em plus .9em चित्ररथ \msNa}}% 

{\devanagarifont वानरः पुनरेवाह एवं कुर्वामहे वयम् \thinspace{\dandab} \dontdisplaylinenum }%
     \var{{\devanagarifontvar\numemph\vb एवं\lem \mssALL,\hskip.2em plus .9em एव \msCb}}% 

%Verse 12:102

{\devanagarifont ततश्चित्ररथावासमुपगम्येदमब्रवीत् {॥ १२:\hspace{.11em}१०२॥} \veg\dontdisplaylinenum }%
     \var{{\devanagarifontvar\numnoemph\vc ततश्चि॰\lem \msCa\msCb\msNa,\hskip.2em plus .9em तत्रश्चि॰ \msNb,\hskip.5em plus .9em तत्र चि॰ \msNc\Ed}}% 
    \var{{\devanagarifontvar\numnoemph\vd ॰ब्रवीत्\lem \msCa\msCb\msNc\Ed,\hskip.2em plus .9em ॰वीत् \msNaacorr,\hskip.5em plus .9em 
॰वीवीत् \msNapcorr,\hskip.5em plus .9em ॰ब्रवी \msNb}}% 

{\devanagarifont गन्धर्वराज कार्यार्थी त्वामहं पुनरागतः \thinspace{\dandab} \dontdisplaylinenum }%
     \var{{\devanagarifontvar\numemph\vb त्वामहं पु॰\lem \conj,\hskip.2em plus .9em त्वन्ह्ययम्पु॰ \msCa\msNc,\hskip.5em plus .9em 
त्वात् ह्यहम्पु॰ \msCb,\hskip.5em plus .9em 
त्वत् ह्ययं पु॰ \msNa,\hskip.5em plus .9em त्वत् ह्यहं पु॰ \msNb\Ed}}% 

%Verse 12:103

{\devanagarifont पूर्वदत्तफलं त्वन्यद्देहि मां यदि शक्यते {॥ १२:\hspace{.11em}१०३॥} \veg\dontdisplaylinenum  }%
 
{\devanagarifont गन्धर्वराज उवाच {\dandab}\dontdisplaylinenum  }%
     \var{{\devanagarifontvar\numemph\vo गन्धर्वराज उवाच\lem \msCb,\hskip.2em plus .9em गन्धर्वराजोवाच \msCa\msNb\Ed,\hskip.5em plus .9em 
गन्धर्वराजौवाच \msNa,\hskip.5em plus .9em 
गन्धराज उवाच \msNc}}% 

{\devanagarifont सूर्यलोकगतश्चास्मि तेन दत्तं फलोत्तमम् \thinspace{\danda} \dontdisplaylinenum }%
     \var{{\devanagarifontvar\numnoemph\va गतश्चास्मि\lem \mssALL,\hskip.2em plus .9em 
गत\uncl{श्चा}\lacwithnum{1}\  \msCa,\hskip.5em plus .9em गतश्चास्मिं \msNb}}% 
    \var{{\devanagarifontvar\numnoemph\vb तेन दत्तं\lem \mssALL,\hskip.2em plus .9em \lacwithnum{3}  त्तम् \msCa}}% 

%Verse 12:104

{\devanagarifont मया दत्तं फलं तुभ्यमत्यन्तसुहृदो ऽसि मे {॥ १२:\hspace{.11em}१०४॥} \veg\dontdisplaylinenum }%
     \var{{\devanagarifontvar\numnoemph\vc दत्तं\lem \corr,\hskip.2em plus .9em दत्त॰ \msCa\msCb\msNa\msNb\msNc\Ed}}% 
    \var{{\devanagarifontvar\numnoemph\vd ॰सुहृदो\lem \mssALL,\hskip.2em plus .9em ॰सुह्यदो \msCb}}% 

{\devanagarifont कुतो ऽन्यत्फलमादास्ये मम नास्ति प्लवङ्गम \thinspace{\dandab} \dontdisplaylinenum }%
     \var{{\devanagarifontvar\numemph\va ऽन्यत्फलमादास्ये\lem \mssALL,\hskip.2em plus .9em 
ऽन्यफल दास्यामि \Ed}}% 
    \var{{\devanagarifontvar\numnoemph\vb मम नास्ति प्लवङ्गम\lem \mssALL,\hskip.2em plus .9em 
मम नास्ति प्लवङ्गमः \msNa,\hskip.5em plus .9em 
मत्तो ऽस्ति प्लवङ्गमः \Ed}}% 

%Verse 12:105

{\devanagarifont सूर्यलोकं गमिष्यामस्तत्र याचस्व भास्करम् {॥ १२:\hspace{.11em}१०५॥} \veg\dontdisplaylinenum }%
     \var{{\devanagarifontvar\numnoemph\vcd गमिष्यामस्तत्र\lem \mssALL,\hskip.2em plus .9em 
गमिष्यामस्तत \msNc,\hskip.5em plus .9em गमिष्यामि तत्र \Ed}}% 

{\devanagarifont गन्धर्वेनैवमुक्तस्तु तथेत्याह प्लवङ्गमः \thinspace{\dandab} \dontdisplaylinenum }%
     \var{{\devanagarifontvar\numemph\vb तथेत्याह\lem \mssALL,\hskip.2em plus .9em तथैत्याह \msCb}}% 

%Verse 12:106

{\devanagarifont सूर्यलोकं ततः प्राप्ता गन्धर्वादय सर्वशः {॥ १२:\hspace{.11em}१०६॥} \veg\dontdisplaylinenum }%
     \var{{\devanagarifontvar\numnoemph\vc प्राप्ता\lem \mssALL,\hskip.2em plus .9em प्राप्ताः \msNc}}% 
    \var{{\devanagarifontvar\numnoemph\vd ॰दय सर्वशः\lem \conj,\hskip.2em plus .9em ॰दयस्सर्वशः \msCa\ \unmetr,\hskip.5em plus .9em 
॰दयः सर्वशः \msCb\msNa\msNc\Ed\ \unmetr,\hskip.5em plus .9em दय सर्वश \msNb}}% 

{\devanagarifont गन्धर्व उवाच {\dandab}\dontdisplaylinenum  }%
     \var{{\devanagarifontvar\numemph\vo गन्धर्व उवाच\lem \mssALL,\hskip.2em plus .9em 
गन्धर्व \uncl{उवा}\lacwithnum{1}\  \msCa,\hskip.5em plus .9em 
गन्धर्वराजोवाच \Ed}}% 

{\devanagarifont कार्यार्थेन पुनः प्राप्तस्त्वत्सकाशं खगेश्वर \thinspace{\danda} \dontdisplaylinenum }%
     \var{{\devanagarifontvar\numnoemph\vab प्राप्तस्त्व॰\lem \mssALL,\hskip.2em plus .9em प्राप्त त्व॰ \msNa}}% 
    \var{{\devanagarifontvar\numnoemph\vb ॰काशं\lem \mssALL,\hskip.2em plus .9em ॰काशां \msNb\oo 
 ॰श्वर\lem \mssALL,\hskip.2em plus .9em ॰श्वरः \msNb\msNc}}% 

%Verse 12:107

{\devanagarifont पूर्वदत्तफलं त्वन्यद्देहि जीवमनाशय {॥ १२:\hspace{.11em}१०७॥} \veg\dontdisplaylinenum }%
     \var{{\devanagarifontvar\numnoemph\vc फलं त्वन्य॰\lem \msCa\msNa\msNc,\hskip.2em plus .9em 
फलं त्व॰ \msCb,\hskip.5em plus .9em फलंस्त्वन्य॰ \msNb\Ed}}% 
    \var{{\devanagarifontvar\numnoemph\vd ॰नाशय\lem \mssALL,\hskip.2em plus .9em अनामयः \msNb,\hskip.5em plus .9em ॰नाशयः \Ed}}% 

{\devanagarifont सूर्य उवाच {\dandab}\dontdisplaylinenum  }%
 
{\devanagarifont सोमलोकगतश्चास्मि तेन दत्तं फलोत्तमम् \thinspace{\danda} \dontdisplaylinenum }%
     \var{{\devanagarifontvar\numemph\vab ॰स्मि तेन\lem \mssALL,\hskip.2em plus .9em ॰स्मिन्तेन \msNb}}% 
    \var{{\devanagarifontvar\numnoemph\vb दत्तं\lem \mssALL,\hskip.2em plus .9em दत्त॰ \msNb}}% 

%Verse 12:108

{\devanagarifont स फलं दत्तमेवासि सुहृदत्वान्मया तव {॥ १२:\hspace{.11em}१०८॥} \veg\dontdisplaylinenum }%
     \var{{\devanagarifontvar\numnoemph\vc ॰वासि\lem \msCa\msCb\msNc,\hskip.2em plus .9em ॰वा\uncl{भि} \msNa,\hskip.5em plus .9em 
॰एवाति \msNb,\hskip.5em plus .9em ॰वाभिः \Ed}}% 
    \var{{\devanagarifontvar\numnoemph\vd सुहृदत्वान्मया\lem \mssALL,\hskip.2em plus .9em 
सुहृदत्वात्मया \msNa,\hskip.5em plus .9em स च दत्वा मया \Ed}}% 

{\devanagarifont अन्यद्दातुं न शक्नोमि गच्छ सोमपुराद्य वै \thinspace{\dandab} \dontdisplaylinenum }%
     \var{{\devanagarifontvar\numemph\va अन्यद्दातुं\lem \msNa\msNc\Ed,\hskip.2em plus .9em अन्य दातुं \msCa\msCb,\hskip.5em plus .9em अन्य दातु \msNb}}% 
    \var{{\devanagarifontvar\numnoemph\vb ॰पुराद्य\lem \mssALL,\hskip.2em plus .9em ॰पराद्य \Ed}}% 

%Verse 12:109

{\devanagarifont तं प्रार्थयाविकल्पेन अत्रिपुत्रं ग्रहेश्वरम् {॥ १२:\hspace{.11em}१०९॥} \veg\dontdisplaylinenum }%
     \var{{\devanagarifontvar\numnoemph\vc तं\lem \mssALL,\hskip.2em plus .9em त \msNb\oo 
 ॰विकल्पेन\lem \mssALL,\hskip.2em plus .9em 
॰\uncl{विक}\lacwithnum{2}  \msCa}}% 
    \var{{\devanagarifontvar\numnoemph\vd ॰पुत्रं\lem \mssALL,\hskip.2em plus .9em ॰पुत्र॰ \msCa\msNb}}% 

{\devanagarifont रुद्र उवाच {\dandab}\dontdisplaylinenum  }%
     \var{{\devanagarifontvar\numemph\vo रुद्र\lem \mssALL,\hskip.2em plus .9em महेश्वर \Ed}}% 

{\devanagarifont गताः सूर्याग्रतः कृत्वा सोमलोकं तथैव हि \thinspace{\danda} \dontdisplaylinenum }%
     \var{{\devanagarifontvar\numnoemph\va गताः\lem \msCb,\hskip.2em plus .9em गत \msCa\msNa\msNb,\hskip.5em plus .9em गतः \msNc\Ed}}% 
    \var{{\devanagarifontvar\numnoemph\vb हि\lem \mssALL,\hskip.2em plus .9em \om\ \msNb}}% 

%Verse 12:110

{\devanagarifont उवाच सूर्यः सोमाय कारणापेक्षया शशिम् {॥ १२:\hspace{.11em}११०॥} \veg\dontdisplaylinenum }%
     \var{{\devanagarifontvar\numnoemph\vc सूर्यः\lem \mssALL,\hskip.2em plus .9em सूर्य \msNb}}% 
    \var{{\devanagarifontvar\numnoemph\vd कारणा॰\lem \mssALL,\hskip.2em plus .9em करुणा॰ \msCb\oo 
 ॰पेक्षया\lem \mssALL,\hskip.2em plus .9em ॰पेक्षणा \msNb\oo 
 शशिम्\lem \msCa\msCb\msNa,\hskip.2em plus .9em शशि \msNb\Ed,\hskip.5em plus .9em शशि\uncl{न्} \msNc}}% 

{\devanagarifont सोम उवाच {\dandab}\dontdisplaylinenum  }%
 
{\devanagarifont किमर्थमागतो भूयः कर्तव्यं तत्र भास्कर \thinspace{\danda} \dontdisplaylinenum }%
     \var{{\devanagarifontvar\numemph\va ॰गतो\lem \mssALL,\hskip.2em plus .9em ॰गता \msNb}}% 
    \var{{\devanagarifontvar\numnoemph\vb तत्र\lem \mssALL,\hskip.2em plus .9em तव \Ed\oo 
 ॰कर\lem \mssALL,\hskip.2em plus .9em ॰करः \Ed}}% 

%Verse 12:111

{\devanagarifont फलं दातुं पुनस्त्वन्यन्मुक्त्वा त्वन्यत्करोम्यहम् {॥ १२:\hspace{.11em}१११॥} \veg\dontdisplaylinenum }%
     \var{{\devanagarifontvar\numnoemph\vcd पुनस्त्वन्यन्मुक्त्वा त्वन्यत्क॰\lem \corr,\hskip.2em plus .9em 
पुनस्त्वन्य मुक्त्वा त्वन्यङ्क॰ \msCa,\hskip.5em plus .9em 
पुनस्त्वन्यन्मुक्त्वास्त्वन्यं क॰ \msCb,\hskip.5em plus .9em 
पुनः त्वन्य मुक्त्वा त्वन्यत्क॰ \msNa,\hskip.5em plus .9em 
पुनस्त्वन्य मुक्त्वा त्वन्यत्क॰ \msNb,\hskip.5em plus .9em 
पुनस्त्वन्यत्मुक्ता त्वन्यङ्क॰ \msNc\Ed}}% 

{\devanagarifont सूर्य उवाच {\dandab}\dontdisplaylinenum  }%
 
{\devanagarifont यदि शक्यं फलं देहि अन्यन्न प्रार्थयाम्यहम् \thinspace{\danda} \dontdisplaylinenum }%
     \var{{\devanagarifontvar\numemph\va शक्यं फलं देहि\lem \msCa\msNa\msNc\Ed,\hskip.2em plus .9em काफलन्देहि \msCbacorr,\hskip.5em plus .9em 
काफल\lk न्देहि \msCbpcorr,\hskip.5em plus .9em शक्य फलं देहि \msNb}}% 
    \var{{\devanagarifontvar\numnoemph\vb अन्यन्न\lem \mssALL,\hskip.2em plus .9em अन्यत्वं \msNc,\hskip.5em plus .9em अन्यान्न \Ed}}% 

%Verse 12:112

{\devanagarifont न दत्तासि फलमन्यन्मया वध्यो भविष्यसि {॥ १२:\hspace{.11em}११२॥} \veg\dontdisplaylinenum }%
     \var{{\devanagarifontvar\numnoemph\vcd फलमन्यन्म॰\lem \mssALL,\hskip.2em plus .9em 
फलंमन्यन्म॰ \msNa,\hskip.5em plus .9em 
फलं मन्ये म॰ \Ed}}% 
    \var{{\devanagarifontvar\numnoemph\vd वध्यो\lem \msNc,\hskip.2em plus .9em वद्ध्यो \msCa\msCb\msNa\msNb,\hskip.5em plus .9em वद्धो \Ed\oo 
 भविष्यसि\lem \mssALL,\hskip.2em plus .9em भविष्यति \msCb}}% 

{\devanagarifont सोम उवाच {\dandab}\dontdisplaylinenum  }%
 
{\devanagarifont आगमं तस्य वक्ष्यामि शृणुष्वावहितो भव \thinspace{\danda} \dontdisplaylinenum }%
     \var{{\devanagarifontvar\numemph\va वक्ष्यामि\lem \mssALL,\hskip.2em plus .9em वक्ष्या\uncl{मि} \msNa}}% 

%Verse 12:113

{\devanagarifont इन्द्रेणास्मि फलं दत्तं स फलं दत्त मे भवान् {॥ १२:\hspace{.11em}११३॥} \veg\dontdisplaylinenum }%
     \var{{\devanagarifontvar\numnoemph\vd दत्त मे\lem \mssALL,\hskip.2em plus .9em वत्त मे \msNa}}% 
    \lacuna{\devanagarifontsmall \vd {\englishfont \msCc\ resumes here with } दत्त मे भवान् }%
  
{\devanagarifont गत्वैवेन्द्रसदस्त्वन्यत्प्रार्थयामः सहैव तु \thinspace{\dandab} \dontdisplaylinenum }%
     \var{{\devanagarifontvar\numemph\va गत्वैवेन्द्र॰\lem \msCa,\hskip.2em plus .9em गत्वेवेन्द्र॰ \msCb\msNb\msNc,\hskip.5em plus .9em \lk\lk \lk\lk\ \msCc,\hskip.5em plus .9em 
गत्वावेन्द्र॰ \msNa,\hskip.5em plus .9em गन्धर्वेन्द्र॰ \Ed}}% 
    \var{{\devanagarifontvar\numnoemph\vb ॰र्थयामः\lem \mssALL,\hskip.2em plus .9em ॰र्थयामा \msNa\oo 
 सहैव तु\lem \mssALL,\hskip.2em plus .9em 
सदैव तु \msCc,\hskip.5em plus .9em सहैव तुः \msNc}}% 

%Verse 12:114

{\devanagarifont एवं कुर्म इति प्राह गत्वेन्द्रसदनं प्रति {॥ १२:\hspace{.11em}११४॥} \veg\dontdisplaylinenum }%
     \var{{\devanagarifontvar\numnoemph\vc कुर्म\lem \mssALL,\hskip.2em plus .9em कर्म \msNb,\hskip.5em plus .9em सोम \Ed}}% 

{\devanagarifont सोमेनेन्द्रमुवाचेदं फलकामा इहागताः \thinspace{\dandab} \dontdisplaylinenum }%
     \var{{\devanagarifontvar\numemph\va सोमेनेन्द्र॰\lem \mssALL,\hskip.2em plus .9em  सोमेवेन्द्र॰ \msNb,\hskip.5em plus .9em सोम इन्द्र॰ \msNc\oo 
 ॰चेदं\lem \mssALL,\hskip.2em plus .9em ॰चेन्द्रं \msCc}}% 

%Verse 12:115

{\devanagarifont पूर्वदत्तफलमन्यद्देहि शक्र ममाद्य वै {॥ १२:\hspace{.11em}११५॥} \veg\dontdisplaylinenum }%
     \var{{\devanagarifontvar\numnoemph\vc पूर्व॰\lem \mssALL,\hskip.2em plus .9em पूर्वं \msNb}}% 
    \var{{\devanagarifontvar\numnoemph\vcd ॰न्यद्देहि\lem \mssALL,\hskip.2em plus .9em ॰न्य देहि \msCc}}% 
    \var{{\devanagarifontvar\numnoemph\vd शक्र\lem \mssALL,\hskip.2em plus .9em शक \Ed\oo 
 वै\lem \mssALL,\hskip.2em plus .9em वैः \msCb}}% 

{\devanagarifont इन्द्र उवाच {\dandab}\dontdisplaylinenum  }%
 
{\devanagarifont यदर्थमिह सम्प्राप्तः स च नास्ति निशाकर \thinspace{\danda} \dontdisplaylinenum }%
     \var{{\devanagarifontvar\numemph\vb ॰कर\lem \mssALL,\hskip.2em plus .9em ॰करः \msCb\Ed}}% 

%Verse 12:116

{\devanagarifont विष्णुहस्तान्मया प्राप्तमेकमेव फलं शुभम् {॥ १२:\hspace{.11em}११६॥} \veg\dontdisplaylinenum }%
     \var{{\devanagarifontvar\numnoemph\vc विष्णुहस्तान्मया\lem \mssALL,\hskip.2em plus .9em 
विष्णुहस्ता मया \msNb}}% 
    \var{{\devanagarifontvar\numnoemph\vd फलं\lem \mssALL,\hskip.2em plus .9em फल \msCb}}% 

{\devanagarifont सर्व एव हि गच्छामो विष्णुलोकं ग्रहेश्वर \thinspace{\dandab} \dontdisplaylinenum }%
     \var{{\devanagarifontvar\numemph\vb ॰लोकं\lem \mssALL,\hskip.2em plus .9em ॰लोक \msCc\oo 
 ॰श्वर\lem \mssALL,\hskip.2em plus .9em ॰श्वरं \msCb,\hskip.5em plus .9em ॰श्व\lk\ \msNb}}% 

%Verse 12:117

{\devanagarifont सर्व एवोपजग्मुस्ते फलार्थं मधुसूदनम् {॥ १२:\hspace{.11em}११७॥} \veg\dontdisplaylinenum }%
     \var{{\devanagarifontvar\numnoemph\vc सर्व एवोपजग्मुस्ते\lem \mssALL,\hskip.2em plus .9em 
सर्व एवोपञ्जग्मुस्ते \msCa\ \unmetr,\hskip.5em plus .9em 
\lk\lk \lk\lk \lk\lk \lk\lk\ \msNb}}% 
    \var{{\devanagarifontvar\numnoemph\vd फलार्थं मधुसूदनम्\lem \mssALL,\hskip.2em plus .9em 
\lk\lk \lk\lk \lk\lk \lk\lk\ \msNb,\hskip.5em plus .9em 
फफालार्थं मधुसूदनम् \msNc}}% 
    \lacuna{\devanagarifontsmall \vcd {\englishfont This folio side in \msNb\ (verses 12.117--138) is faded and 
                     most of it is difficult to read, thus its readings
                     reported are less reliable than usual} }%
  
{\devanagarifont एवमुक्त्वा गताः सर्वे देवराजपुरस्कृताः \thinspace{\dandab} \dontdisplaylinenum }%
     \var{{\devanagarifontvar\numemph\va एवमुक्त्वा गताः सर्वे\lem \mssCaCbCc\msNa,\hskip.2em plus .9em 
\lk\lk \lk\lk \lk\lk \lk\lk\ \msNb,\hskip.5em plus .9em एवमुक्त्वा गता सर्वे \msNc,\hskip.5em plus .9em 
एवमुक्ता गताः सर्वे \Ed}}% 

%Verse 12:118

{\devanagarifont मुहूर्तेनैव सम्प्राप्ता विष्णुलोकं यशस्विनि {॥ १२:\hspace{.11em}११८॥} \veg\dontdisplaylinenum }%
     \var{{\devanagarifontvar\numnoemph\vd विष्णुलोकं\lem \mssALL,\hskip.2em plus .9em 
विष्णुलोक \msCc,\hskip.5em plus .9em \lk\lk \lk\lk\ \msNb}}% 

{\devanagarifont उपसृत्य तत इन्द्रः प्रणिपत्य जनार्दनम् \thinspace{\dandab} \dontdisplaylinenum }%
 
%Verse 12:119

{\devanagarifont सर्वेषामुपरोधेन प्रार्थयामि यशोधर {॥ १२:\hspace{.11em}११९॥} \veg\dontdisplaylinenum }%
     \var{{\devanagarifontvar\numemph\vd ॰धर\lem \mssALL,\hskip.2em plus .9em ॰धरम् \Ed}}% 

{\devanagarifont विष्णुरुवाच {\dandab}\dontdisplaylinenum  }%
     \var{{\devanagarifontvar\numemph\vo विष्णुरुवाच\lem \msCapcorr\msCb\msCc\msNapcorr\msNb\msNc,\hskip.2em plus .9em 
विष्णुरुच \msCaacorr,\hskip.5em plus .9em 
\om\ \msNaacorr,\hskip.5em plus .9em विष्णु उवाच \Ed}}% 

{\devanagarifont पूर्वदत्तफलस्यार्थे तच्च सर्वमिहागताः \thinspace{\danda} \dontdisplaylinenum }%
     \var{{\devanagarifontvar\numnoemph\va ॰दत्त॰\lem \mssALL,\hskip.2em plus .9em ॰दत्तं \Ed\oo 
 ॰र्थे\lem \mssALL,\hskip.2em plus .9em ॰र्थि \Ed}}% 

%Verse 12:120

{\devanagarifont न शक्नोमि फलं दातुं किं वा त्वन्यत्करोम्यहम् {॥ १२:\hspace{.11em}१२०॥} \veg\dontdisplaylinenum }%
     \var{{\devanagarifontvar\numnoemph\vc शक्नोमि\lem \mssALL,\hskip.2em plus .9em शक्नोति \msCb\oo 
 फलं दातुं\lem \mssALL,\hskip.2em plus .9em 
फल\uncl{न्दातु} \msCc}}% 
    \var{{\devanagarifontvar\numnoemph\vd त्वन्यत्करोम्यहम्\lem \msNc,\hskip.2em plus .9em 
त्वन्यं करोम्यहम् \mssCaCbCc\msNa\Ed,\hskip.5em plus .9em 
\lk\lk \lk\lk \lk\lk म्यहम् \msNb}}% 

{\devanagarifont इन्द्र उवाच {\dandab}\dontdisplaylinenum  }%
 
{\devanagarifont ब्रह्माण्डमपि भेत्तुं त्वं शक्नोषि गरुडध्वज \thinspace{\danda} \dontdisplaylinenum }%
     \var{{\devanagarifontvar\numemph\va ब्रह्माण्ड॰\lem \mssALL,\hskip.2em plus .9em ब्रह्मण्ड॰ \msNc\oo 
 भेत्तुं त्वं\lem \mssALL,\hskip.2em plus .9em 
भेत्तु त्वं \msCb,\hskip.5em plus .9em भर्तुंत्वं \Ed}}% 
    \var{{\devanagarifontvar\numnoemph\vb शक्नोषि\lem \mssALL,\hskip.2em plus .9em शक्नोति \msCb}}% 

%Verse 12:121

{\devanagarifont अशक्यं तव नास्तीति जानामि पुरुषोत्तम {॥ १२:\hspace{.11em}१२१॥} \veg\dontdisplaylinenum }%
     \var{{\devanagarifontvar\numnoemph\vc अशक्यं\lem \mssALL,\hskip.2em plus .9em \uncl{अशक्य} \msCb}}% 
    \var{{\devanagarifontvar\numnoemph\vd ॰त्तम\lem \mssALL,\hskip.2em plus .9em ॰त्तमम् \Ed}}% 

{\devanagarifont एवमुक्तः पुनर्विष्णुः प्रत्युवाच पुरन्दरम् \thinspace{\dandab} \dontdisplaylinenum }%
     \var{{\devanagarifontvar\numemph\va एवमुक्तः पुनर्विष्णुः\lem \msCb,\hskip.2em plus .9em 
एवमुक्त्वा पुनर्विष्णुः \msCa\msCc\msNa\msNc\Ed,\hskip.5em plus .9em 
\lk\lk \lk\lk  पुनर्विष्णुः \msNb}}% 
    \var{{\devanagarifontvar\numnoemph\vb पुरन्दरम्\lem \mssALL,\hskip.2em plus .9em पुरदरं \msNc\ \unmetr}}% 

%Verse 12:122

{\devanagarifont फलमेकं परित्यज्य सर्वं शक्नोमि कौशिक {॥ १२:\hspace{.11em}१२२॥} \veg\dontdisplaylinenum }%
     \var{{\devanagarifontvar\numnoemph\vd सर्वं शक्नोमि\lem \mssALL,\hskip.2em plus .9em सर्वं शक्नोसि \msCc,\hskip.5em plus .9em 
\lk\lk  शक्नोमि \msNb}}% 

{\devanagarifont उपायो ऽत्र प्रवक्ष्यामि आगमं शृणु गोपते \thinspace{\dandab} \dontdisplaylinenum }%
 
%Verse 12:123

{\devanagarifont ब्रह्मणा च मम दत्तं तत्फलैकं पुरन्दर {॥ १२:\hspace{.11em}१२३॥} \veg\dontdisplaylinenum }%
     \var{{\devanagarifontvar\numemph\vc मम\lem \mssALL,\hskip.2em plus .9em ममा॰ \Ed}}% 
    \var{{\devanagarifontvar\numnoemph\vd तत्फलैकं\lem \mssALL,\hskip.2em plus .9em तत्फलंकं \msNaacorr,\hskip.5em plus .9em 
तत्फलेकं \msNapcorr\oo 
 पुरन्दर\lem \mssALL,\hskip.2em plus .9em 
पुरन्द\uncl{रं} \msNc}}% 

{\devanagarifont मया दत्तं फलं त्वेकं किमन्यद्दातुमिच्छसि \thinspace{\dandab} \dontdisplaylinenum }%
     \var{{\devanagarifontvar\numemph\va दत्तं\lem \msCc\msNb,\hskip.2em plus .9em दत्त॰ \msCa\msCb\msNa\msNc\Ed\oo 
 त्वेकं\lem \mssALL,\hskip.2em plus .9em त्वैकं \msNc}}% 
    \var{{\devanagarifontvar\numnoemph\vb ॰च्छसि\lem \mssALL,\hskip.2em plus .9em ॰च्छति \msCa}}% 

%Verse 12:124

{\devanagarifont प्रार्थयामो ऽत्र गत्वैकं परमेष्ठिप्रजापतिम् {॥ १२:\hspace{.11em}१२४॥} \veg\dontdisplaylinenum  }%
     \var{{\devanagarifontvar\numnoemph\vc प्रार्थयामो ऽत्र गत्वैकं\lem \mssALL,\hskip.2em plus .9em 
प्रार्थया च गत्वैवं \Ed}}% 
    \var{{\devanagarifontvar\numnoemph\vd ॰ष्ठिप्रजा॰\lem \msCa\msNa\msNb\msNc,\hskip.2em plus .9em ॰ष्ठिं प्रजा॰ \msCb\Ed,\hskip.5em plus .9em 
॰ष्ठि\uncl{प्रजा}॰ \msCc}}% 

{\devanagarifont तवोपरोधाद्देवेन्द्र प्रार्थयामि पितामहम् \thinspace{\dandab} \dontdisplaylinenum }%
     \var{{\devanagarifontvar\numemph\va तवो॰\lem \mssALL,\hskip.2em plus .9em ततो॰ \Ed\oo 
 ॰रोधाद्देवे॰\lem \msCa\msCb\msNa\msNc,\hskip.2em plus .9em ॰रोधा देवे॰ \msCc\msNb,\hskip.5em plus .9em 
॰राधाद्देवे॰ \Ed}}% 
    \var{{\devanagarifontvar\numnoemph\vb ॰महम्\lem \mssALL,\hskip.2em plus .9em ॰मह \msNb}}% 

%Verse 12:125

{\devanagarifont एवमुक्त्वा गताः सर्वे पुरस्कृत्य जनार्दनम् {॥ १२:\hspace{.11em}१२५॥} \veg\dontdisplaylinenum }%
     \var{{\devanagarifontvar\numnoemph\vc गताः\lem \mssALL,\hskip.2em plus .9em गता \msCc\Ed}}% 
    \var{{\devanagarifontvar\numnoemph\vd पुरस्कृत्य\lem \mssALL,\hskip.2em plus .9em पुनस्कृत्य \msNc\oo 
 जनार्दनम्\lem \mssALL,\hskip.2em plus .9em जनार्द्दन \msCc}}% 

{\devanagarifont इन्द्रः सूर्यः शशी चैव गन्धर्वो वानरस्तथा \thinspace{\dandab} \dontdisplaylinenum }%
     \var{{\devanagarifontvar\numemph\va इन्द्रः\lem \mssALL,\hskip.2em plus .9em इन्द्र \msCc\oo 
 सूर्यः शशी चैव\lem \msCa\msCb\msNa\msNc,\hskip.2em plus .9em सूर्य शशी चैव \msCc\msNb,\hskip.5em plus .9em 
सोमश्च सूर्यश्च \Ed}}% 

%Verse 12:126

{\devanagarifont विपुलः श्रेष्ठिकश्चैव राजदूतद्वयं तथा {॥ १२:\hspace{.11em}१२६॥} \veg\dontdisplaylinenum }%
     \var{{\devanagarifontvar\numnoemph\vc विपुलः\lem \mssALL,\hskip.2em plus .9em विपुल \msNa\msNb}}% 
    \var{{\devanagarifontvar\numnoemph\vd ॰द्वयं तथा\lem \Ed,\hskip.2em plus .9em ॰द्वयस्तथा \mssCaCbCc\msNa\msNb\msNc}}% 

{\devanagarifont ब्रह्मलोकं मुहूर्तेन प्राप्तवान्सुरसुन्दरि \thinspace{\dandab} \dontdisplaylinenum }%
     \var{{\devanagarifontvar\numemph\va ॰लोकं\lem \mssALL,\hskip.2em plus .9em ॰लोक \msNb}}% 

%Verse 12:127

{\devanagarifont दृष्ट्वा ब्रह्मसदो रम्यं सर्वकामपरिच्छदम् {॥ १२:\hspace{.11em}१२७॥} \veg\dontdisplaylinenum }%
     \var{{\devanagarifontvar\numnoemph\vc ॰सदो\lem \mssALL,\hskip.2em plus .9em ॰सदं \Ed\oo 
 रम्यं\lem \mssALL,\hskip.2em plus .9em रम्यां \msNb}}% 

{\devanagarifont अनेकानि विचित्राणि रत्नानि विविधानि च \thinspace{\dandab} \dontdisplaylinenum }%
 
%Verse 12:128

{\devanagarifont मन्दारतल शोभानि वैडूर्यमणिकुट्टिमान् {॥ १२:\hspace{.11em}१२८॥} \veg\dontdisplaylinenum }%
     \var{{\devanagarifontvar\numemph\vc ॰तल\lem \mssALL,\hskip.2em plus .9em ॰तरु॰ \Ed}}% 
    \var{{\devanagarifontvar\numnoemph\vd वैडूर्य॰\lem \mssALL,\hskip.2em plus .9em वैदूर्य॰ \Ed\oo 
 ॰कुट्टिमान्\lem \corr,\hskip.2em plus .9em ॰कुटिमाम् \msCa,\hskip.5em plus .9em 
॰कुट्टिमां \msCb\msCc\msNa\msNb\msNc,\hskip.5em plus .9em ॰कुट्टिमम् \Ed}}% 

{\devanagarifont प्रवालमणिस्तम्भानि वज्रकाञ्चनवेदिकाम् \thinspace{\dandab} \dontdisplaylinenum }%
     \var{{\devanagarifontvar\numemph\vb वज्रकाञ्चनवेदिकाम्\lem \msCa\msCb\msNa,\hskip.2em plus .9em 
वज्रकाञ्चनवेदिका \msCc\msNc\Ed,\hskip.5em plus .9em \lk\lk \lk\lk \lk\lk \lk का \msNb}}% 

%Verse 12:129

{\devanagarifont प्रवालस्फाटिको जाल इन्द्रनीलगवाक्षकः {॥ १२:\hspace{.11em}१२९॥} \veg\dontdisplaylinenum }%
     \var{{\devanagarifontvar\numnoemph\vc प्रवालस्फाटिको जाल\lem \mssCaCbCc\msNc,\hskip.2em plus .9em प्रवालस्फणिको जाल \msNa,\hskip.5em plus .9em 
प्र\uncl{ता}लस्फाटिको जाल \msNb,\hskip.5em plus .9em 
प्रवालस्फटिको जाला \Ed}}% 
    \var{{\devanagarifontvar\numnoemph\vd ॰क्षकः\lem \mssALL,\hskip.2em plus .9em ॰क्षकं \msNa\msNb}}% 

{\devanagarifont पश्यते विपुलस्तत्र नानावृक्ष मनोरमाः \thinspace{\dandab} \dontdisplaylinenum }%
     \var{{\devanagarifontvar\numemph\va पश्यते\lem \mssALL,\hskip.2em plus .9em दृश्यन्ते \Ed\oo 
 विपुल॰\lem \mssALL,\hskip.2em plus .9em विपुला॰ \Ed}}% 

%Verse 12:130

{\devanagarifont पुष्पानामितवृक्षाग्राः फलानामितका भवेत् {॥ १२:\hspace{.11em}१३०॥} \veg\dontdisplaylinenum }%
     \var{{\devanagarifontvar\numnoemph\vc पुष्पा॰\lem \mssALL,\hskip.2em plus .9em पुष्प॰ \msNc\Ed\oo 
 ॰ग्राः\lem \eme,\hskip.2em plus .9em ॰ग्रा \mssCaCbCc\msNa\msNc,\hskip.5em plus .9em ॰ग्रा \msNb,\hskip.5em plus .9em ॰या \Ed}}% 
    \var{{\devanagarifontvar\numnoemph\vd फलानामितका\lem \mssALL,\hskip.2em plus .9em फलनामितकां \Ed}}% 

{\devanagarifont सर्वरत्नमया वृक्षाः सर्वरत्नमयं जलम् \thinspace{\dandab} \dontdisplaylinenum }%
     \var{{\devanagarifontvar\numemph\va सर्व॰\lem \msCb\msNa\msNb\Ed,\hskip.2em plus .9em सर्वे \msCa\msCc\msNc\oo 
 वृक्षाः\lem \mssALL,\hskip.2em plus .9em वृक्षा \msCc\oo 
 ॰मया\lem \mssALL,\hskip.2em plus .9em ॰मयो \msNb}}% 
    \var{{\devanagarifontvar\numnoemph\vb सर्व॰\lem \mssALL,\hskip.2em plus .9em सर्वे \Ed}}% 

%Verse 12:131

{\devanagarifont वृक्षगुल्मलतावल्ली कन्दमूलफलानि च {॥ १२:\hspace{.11em}१३१॥} \veg\dontdisplaylinenum }%
     \var{{\devanagarifontvar\numnoemph\vc ॰गुल्म॰\lem \mssALL,\hskip.2em plus .9em \om\ \msNaacorr\oo 
 ॰वल्ली\lem \mssALL,\hskip.2em plus .9em ॰वली \msCc}}% 

{\devanagarifont सर्वे रत्नमया दृष्टा विपुलो विपुलेक्षणः \thinspace{\dandab} \dontdisplaylinenum }%
     \var{{\devanagarifontvar\numemph\va सर्वे\lem \mssALL,\hskip.2em plus .9em सर्वै \msCa,\hskip.5em plus .9em सर्व्व॰ \msCc\oo 
 दृष्टा\lem \mssALL,\hskip.2em plus .9em 
दृष्ट्वा \msCb,\hskip.5em plus .9em दृ \msNcacorr}}% 
    \var{{\devanagarifontvar\numnoemph\vb ॰क्षणः\lem \mssALL,\hskip.2em plus .9em ॰क्षण \msCc}}% 

%Verse 12:132

{\devanagarifont अनेकभौमं प्रासादं मुक्तादामविभूषितम् {॥ १२:\hspace{.11em}१३२॥} \veg\dontdisplaylinenum }%
     \var{{\devanagarifontvar\numnoemph\vc ॰भौमं\lem \mssALL,\hskip.2em plus .9em ॰भौम॰ \msNc}}% 

{\devanagarifont अप्सरोगणकोटीभिः सर्वाभरणभूषितम् \thinspace{\dandab} \dontdisplaylinenum }%
     \var{{\devanagarifontvar\numemph\vab अप्सरोगणकोटीभिः सर्वाभरणभूषितम्\lem \mssALL,\hskip.2em plus .9em 
\lk\lk \lk\lk \lk\lk \lk\lk \lk\lk \lk\lk \lk\lk \lk\  \msNb}}% 

%Verse 12:133

{\devanagarifont विमानकोटिकोटीनां सर्वकामसमन्वितम् {॥ १२:\hspace{.11em}१३३॥} \veg\dontdisplaylinenum }%
     \var{{\devanagarifontvar\numnoemph\vcd विमानकोटिकोटीनां सर्वकामसमन्वितम्\lem \msCb\msCc\msNa\msNc,\hskip.2em plus .9em 
विमानकोटिकोटीशं सर्वकामसमन्वितम् \msCa,\hskip.5em plus .9em 
\lk\lk \lk\lk \lk\lk \lk\lk \lk\lk \lk\lk \lk\lk \lk\lk\ \msNb,\hskip.5em plus .9em \om\ \Ed}}% 
    \paral{{\devanagarifontsmall \vo {\englishfont \compare\ \SDHS\ 10.41 (on the results of an observance):}
                 सूर्यकोटिप्रतीकाशैर्विमानैः सार्वकामिकैः\thinspace{\devanagarifontsmall ।}
                 रुद्रकन्यासमाकीर्णैर्महावृषभसंयुतैः\thinspace{\devanagarifontsmall ॥} }}

{\devanagarifont ब्रह्मलोकसभा रम्या सूर्यकोटिसमप्रभा \thinspace{\dandab} \dontdisplaylinenum }%
     \var{{\devanagarifontvar\numemph\vb ॰कोटि॰\lem \mssALL,\hskip.2em plus .9em ॰\uncl{कौटि}॰ \msNc}}% 

%Verse 12:134

{\devanagarifont तत्र ब्रह्मा सुखासीनो नानारत्नोपशोभिते {॥ १२:\hspace{.11em}१३४॥} \veg\dontdisplaylinenum }%
     \var{{\devanagarifontvar\numnoemph\vd ॰शोभिते\lem \mssALL,\hskip.2em plus .9em ॰शोभिता \msNb}}% 

{\devanagarifont चतुर्मूर्तिश्चतुर्वक्त्रश्चतुर्बाहुश्चतुर्भुजः \thinspace{\dandab} \dontdisplaylinenum }%
     \var{{\devanagarifontvar\numemph\va ॰मूर्तिश्च॰\lem \mssALL,\hskip.2em plus .9em ॰मूर्ति च॰ \msCc,\hskip.5em plus .9em 
॰मूर्\uncl{त्तिंश्च} \msNb}}% 
    \var{{\devanagarifontvar\numnoemph\vab ॰वक्त्रश्चतुर्बाहुश्चतुर्भुजः\lem \mssALL,\hskip.2em plus .9em 
॰वक्त्राश्चतुर्बाहुश्चतुर्भुजः \msCc,\hskip.5em plus .9em ॰वक्त्र\lk\lk \lk\lk \lk\lk \lk\lk\  \msNb}}% 

%Verse 12:135

{\devanagarifont चतुर्वेदधरो देवश्चतुराश्रमनायकः {॥ १२:\hspace{.11em}१३५॥} \veg\dontdisplaylinenum }%
     \var{{\devanagarifontvar\numnoemph\vc चतुर्वेद॰\lem \mssALL,\hskip.2em plus .9em चतुवेद॰ \msNc}}% 
    \var{{\devanagarifontvar\numnoemph\vcd देवश्च॰\lem \mssALL,\hskip.2em plus .9em देव च॰ \msCc}}% 

{\devanagarifont चतुर्वेदावृतस्तत्र मूर्तिमन्तमुपासते \thinspace{\dandab} \dontdisplaylinenum }%
     \var{{\devanagarifontvar\numemph\vab ॰वेदा वृतस्तत्र मूर्तिमन्तमुपासते\lem \msCa\msCb\msNc\Ed,\hskip.2em plus .9em 
॰वेदवृतस्तत्र मूर्तिमन्तमुपासते \msCc,\hskip.5em plus .9em 
॰\uncl{वेदा}वृतस्तत्र मूर्तिमन्तमुपासते \msNa,\hskip.5em plus .9em 
वे\lk\lk \lk\lk \lk\lk \lk\lk \lk\lk \lk\lk \lk\  \msNb}}% 

%Verse 12:136

{\devanagarifont गायत्री वेदमाता च सावित्री च सुरूपिणी {॥ १२:\hspace{.11em}१३६॥} \veg\dontdisplaylinenum }%
     \var{{\devanagarifontvar\numnoemph\vc गायत्री वेदमाता च\lem \mssALL,\hskip.2em plus .9em 
\lk\lk \lk\lk \lk\lk \lk\lk\  \msNb}}% 

{\devanagarifont व्याहृतिः प्रणवश्चैव मूर्तिमान्समुपासते \thinspace{\dandab} \dontdisplaylinenum }%
     \var{{\devanagarifontvar\numemph\va व्याहृतिः\lem \msCa\msNc\Ed,\hskip.2em plus .9em व्याहृदिः \msCb,\hskip.5em plus .9em 
व्याकृतिः \msCc,\hskip.5em plus .9em व्याहृति \msNa,\hskip.5em plus .9em \lk\lk \lk\  \msNb\oo 
 प्रणवश्चैव\lem \msCb\msNa\msNc\Ed,\hskip.2em plus .9em प्रण\uncl{व}\lacwithnum{1}  व \msCa,\hskip.5em plus .9em 
प्रकृतिश्चैव \msCc,\hskip.5em plus .9em \lk\lk \lk\lk \lk\ \msNb}}% 
    \var{{\devanagarifontvar\numnoemph\vb मूर्तिमान्समुपासते\lem \mssALL,\hskip.2em plus .9em 
\lk\lk \lk\lk \lk\lk \lk\lk\ \msNb}}% 

%Verse 12:137

{\devanagarifont वौषट्कारो वषट्कारो नमस्कारः स मूर्तिमान् {॥ १२:\hspace{.11em}१३७॥} \veg\dontdisplaylinenum }%
     \var{{\devanagarifontvar\numnoemph\vc वौषट्कारो वषट्कारो\lem \msCa\msCc\msNa\Ed,\hskip.2em plus .9em 
\om\ \msCb,\hskip.5em plus .9em \lk\lk \lk\lk \lk\lk \lk\lk\ \msNb,\hskip.5em plus .9em 
वौषट्कारो च \uncl{स}त्कारो \msNc}}% 
    \var{{\devanagarifontvar\numnoemph\vd ॰कारः\lem \mssALL,\hskip.2em plus .9em ॰कार \msCc}}% 

{\devanagarifont श्रुतिः स्मृतिश्च नीतिश्च धर्मशास्त्रं समूर्तिमत् \thinspace{\dandab} \dontdisplaylinenum }%
     \var{{\devanagarifontvar\numemph\vb ॰शास्त्रं समूर्तिमत्\lem \mssALL,\hskip.2em plus .9em 
॰शास्त्रसमूर्तिमान् \msCc\Ed}}% 

%Verse 12:138

{\devanagarifont इतिहासः पुराणं च सांख्ययोगः पतञ्जलम् {॥ १२:\hspace{.11em}१३८॥} \veg\dontdisplaylinenum }%
     \var{{\devanagarifontvar\numnoemph\vc इतिहासः पुराणं च\lem \msCa\msCc\msNa\msNc,\hskip.2em plus .9em पुराणश्च \msCb\Ed,\hskip.5em plus .9em 
\lk\lk \lk\lk \lk\lk \lk\lk\ \msNb}}% 
    \var{{\devanagarifontvar\numnoemph\vd सांख्ययोगः\lem \mssALL,\hskip.2em plus .9em 
सांख्ययोग \msCc,\hskip.5em plus .9em \lk\lk \lk\lk\ \msNb\oo 
 पतञ्जलम्\lem \mssALL,\hskip.2em plus .9em 
\lk\lk \lk\lk\ \msNb,\hskip.5em plus .9em पतञ्जलि \Ed}}% 

{\devanagarifont आयुर्वेदो धनुर्वेदो वेदो गान्धर्वमेव च \thinspace{\dandab} \dontdisplaylinenum }%
     \var{{\devanagarifontvar\numemph\va आयुर्वेदो धनुर्वेदो\lem \mssALL,\hskip.2em plus .9em 
॰वेद धनुर्वेद \msCc,\hskip.5em plus .9em \lk\lk \lk\lk \lk\lk \lk\lk\ \msNb}}% 
    \var{{\devanagarifontvar\numnoemph\vb वेदो गान्धर्वमेव\lem \msCa\msNa,\hskip.2em plus .9em वेदो गन्धर्वमेव \msCb,\hskip.5em plus .9em 
वेद गान्धर्वमेव \msCc,\hskip.5em plus .9em \lk\lk \lk\lk \lk\lk \lk\lk\ \msNb,\hskip.5em plus .9em 
वेदो गार्न्धवमेव \msNc,\hskip.5em plus .9em वेदो गान्धर्वरेव \Ed}}% 

%Verse 12:139

{\devanagarifont अर्थवेदो ऽन्यवेदाश्च मूर्तिमान् समुपासते {॥ १२:\hspace{.11em}१३९॥} \veg\dontdisplaylinenum }%
     \var{{\devanagarifontvar\numnoemph\vc अर्थवेदो ऽन्यवेदाश्च\lem \Ed,\hskip.2em plus .9em अर्थवेदान्यवेदाञ्च \msCa,\hskip.5em plus .9em 
अथर्ववेदान्यवेदञ्च \msCb\ \unmetr,\hskip.5em plus .9em अथर्व्वेदान्यवेदाञ्च \msCc,\hskip.5em plus .9em 
अर्थवेदान्यवेदां च \msNa,\hskip.5em plus .9em \lk\lk \lk\lk \lk\lk \lk\lk\ \msNb,\hskip.5em plus .9em 
अर्थवेदान्यवेदञ्च \msNc}}% 
    \var{{\devanagarifontvar\numnoemph\vd मूर्तिमान् समुपासते\lem \mssALL,\hskip.2em plus .9em 
\lk\lk \lk\lk \lk\lk \lk\lk\ \msNb}}% 

{\devanagarifont ततो ब्रह्मा समुत्थाय अभिगम्य जनार्दनम् \thinspace{\dandab} \dontdisplaylinenum }%
     \var{{\devanagarifontvar\numemph\vab ततो ब्रह्मा समुत्थाय अभिगम्य जनार्दनम्\lem \mssALL,\hskip.2em plus .9em 
\lk\lk \lk\lk \lk\lk \lk\lk \lk\lk \lk\lk \lk\lk \lk\lk\ \msNb}}% 

%Verse 12:140

{\devanagarifont गां च अर्घं च दत्त्वैवमास्यतामिति चाब्रवीत् {॥ १२:\hspace{.11em}१४०॥} \veg\dontdisplaylinenum }%
     \var{{\devanagarifontvar\numnoemph\vc अर्घं च\lem \mssALL,\hskip.2em plus .9em 
अ\uncl{घ}ञ्च \msCb,\hskip.5em plus .9em अर्घ्यञ्च \Ed}}% 

{\devanagarifont मणिरत्नमये दिव्ये आसने गरुडध्वजः \thinspace{\dandab} \dontdisplaylinenum }%
 
%Verse 12:141

{\devanagarifont देवराजो रविः सोमो गन्धर्वः प्लवगेश्वरः {॥ १२:\hspace{.11em}१४१॥} \veg\dontdisplaylinenum }%
     \var{{\devanagarifontvar\numemph\vc रविः सोमो\lem \mssALL,\hskip.2em plus .9em 
र\uncl{वि} सोमो \msNb,\hskip.5em plus .9em शशी सूर्यो \Ed}}% 
    \var{{\devanagarifontvar\numnoemph\vd गन्धर्वः\lem \mssALL,\hskip.2em plus .9em गन्धर्व \msNa,\hskip.5em plus .9em \lk\lk \lk\ \msNb\oo 
 प्लवगेश्वरः\lem \msCa\msCbpcorr\msCc\msNa\Ed,\hskip.2em plus .9em प्लगेश्वरः \msCbacorr,\hskip.5em plus .9em 
\lk\lk \lk\lk \lk\ \msNb,\hskip.5em plus .9em प्लवमेश्वरः \msNc}}% 

{\devanagarifont विपुलश्च महासत्त्व आस्यतां रत्न-आसने \thinspace{\dandab} \dontdisplaylinenum }%
     \var{{\devanagarifontvar\numemph\va विपुलश्च महासत्त्व\lem \mssALL,\hskip.2em plus .9em 
विपुलश्च समासत्व \msCb,\hskip.5em plus .9em 
\lk\lk \lk\lk \lk\lk सत्व \msNb}}% 
    \var{{\devanagarifontvar\numnoemph\vb आस्यतां\lem \mssALL,\hskip.2em plus .9em आस्यता \msCb\oo 
 ॰आसने\lem \mssCaCbCc\msNa,\hskip.2em plus .9em ॰शाशने \msNb\Ed,\hskip.5em plus .9em ॰आसनेः \msNc}}% 

%Verse 12:142

{\devanagarifont साधु भो विपुल श्रेष्ठ साधु भो विपुलं तपः {॥ १२:\hspace{.11em}१४२॥} \veg\dontdisplaylinenum }%
     \var{{\devanagarifontvar\numnoemph\vc साधु भो\lem \mssALL,\hskip.2em plus .9em 
साधु हो \msCb,\hskip.5em plus .9em \lk\lk \lk\ \msNb}}% 
    \var{{\devanagarifontvar\numnoemph\vd विपुलं तपः\lem \msNa\msNb\Ed,\hskip.2em plus .9em \uncl{वि}\lacwithnum{3}  पः \msCa,\hskip.5em plus .9em 
विपुलतपः \msCb\msCc\msNc}}% 

{\devanagarifont साधु भो विपुलप्राज्ञ साधु भो विपुलश्रिय \thinspace{\dandab} \dontdisplaylinenum }%
     \var{{\devanagarifontvar\numemph\vb ॰श्रिय\lem \msCa\msNb\msNc,\hskip.2em plus .9em ॰प्रियः \msCb,\hskip.5em plus .9em ॰श्रियः \msCc\msNa\Ed}}% 

%Verse 12:143

{\devanagarifont तोषिताः स्म वयं सर्वे ब्रह्मविष्णुमहेश्वराः {॥ १२:\hspace{.11em}१४३॥} \veg\dontdisplaylinenum }%
     \var{{\devanagarifontvar\numnoemph\vc तोषिताः\lem \mssALL,\hskip.2em plus .9em तोषिता \msNa\Ed}}% 

{\devanagarifont आदित्या वसवो रुद्राः साध्याश्विनौ मरुत्तथा \thinspace{\dandab} \dontdisplaylinenum }%
     \var{{\devanagarifontvar\numemph\va रुद्राः\lem \mssCaCbCc\msNa,\hskip.2em plus .9em रुद्रा \msNb\msNc\Ed}}% 
    \var{{\devanagarifontvar\numnoemph\vb साध्याश्विनौ\lem \msNb,\hskip.2em plus .9em साध्याश्विन्यौ \msCa\msCb\msNa,\hskip.5em plus .9em 
साध्याश्विन्यो \msCc\msNc,\hskip.5em plus .9em साध्या यक्षो \Ed\oo 
 मरुत्तथा\lem \mssALL,\hskip.2em plus .9em 
मरुतस्तथा \msCc}}% 

%Verse 12:144

{\devanagarifont भुङ्क्ष्व भोगान्यथोत्साहं मम लोके यथासुखम् {॥ १२:\hspace{.11em}१४४॥} \veg\dontdisplaylinenum }%
     \var{{\devanagarifontvar\numnoemph\vc भुङ्क्ष्व\lem \mssALL,\hskip.2em plus .9em भुक्त्वा \msNb,\hskip.5em plus .9em भुंक्ष \Ed\oo 
 भोगान्यथोत्साहं\lem \mssALL,\hskip.2em plus .9em 
भोगा यथेत्साह \msCc 
भोगा यथोत्साहं \msNb}}% 
    \var{{\devanagarifontvar\numnoemph\vd लोके\lem \mssALL,\hskip.2em plus .9em लोक \msNb}}% 

{\devanagarifont इयं विमानकोटीनां तवार्थायोपकल्पिता \thinspace{\dandab} \dontdisplaylinenum }%
     \var{{\devanagarifontvar\numemph\va ॰कोटीनां\lem \mssALL,\hskip.2em plus .9em ॰कोटीनि \msCc,\hskip.5em plus .9em ॰कोटीना \msNb}}% 
    \var{{\devanagarifontvar\numnoemph\vb तवार्थायोप॰\lem \msCa\msNa\msNc\Ed,\hskip.2em plus .9em तवायोपि॰ \msCb,\hskip.5em plus .9em 
त्वयार्थं याव॰ \msCc,\hskip.5em plus .9em तवार्थायोप्र॰ \msNb\oo 
 ॰कल्पिता\lem \msCa\msCb\msNa,\hskip.2em plus .9em ॰कल्पितं \msCc,\hskip.5em plus .9em 
॰कल्पि\lk\  \msNb\msNc,\hskip.5em plus .9em ॰कल्पितान् \Ed}}% 

{\devanagarifont सहस्राणां सहस्राणि अप्सरा कामरूपिणी  \danda\dontdisplaylinenum }%
     \var{{\devanagarifontvar\numnoemph\vc सहस्राणां\lem \mssALL,\hskip.2em plus .9em सहस्राणा \msCb}}% 
    \var{{\devanagarifontvar\numnoemph\vd अप्सरा\lem \mssALL,\hskip.2em plus .9em अप्सरो \msCc\oo 
 ॰रूपिणी\lem \mssALL,\hskip.2em plus .9em ॰रूपिणि \Ed}}% 

%Verse 12:145

{\devanagarifont तवार्थीयोपसर्पन्ति सर्वालंकारभूषिताः {॥ १२:\hspace{.11em}१४५॥} \veg\dontdisplaylinenum }%
     \var{{\devanagarifontvar\numnoemph\ve तवार्थीयो॰\lem \msCa,\hskip.2em plus .9em तवार्थायो॰ \msCb\msNa\msNb\msNc,\hskip.5em plus .9em तंवार्थीयो॰ \msCc,\hskip.5em plus .9em 
तवार्थेयो॰ \Ed}}% 
    \var{{\devanagarifontvar\numnoemph\vf ॰सर्पन्ति\lem \mssALL,\hskip.2em plus .9em ॰षप्यन्ति \msNc\oo 
 ॰भूषिताः\lem \mssALL,\hskip.2em plus .9em ॰भूषितः \msNa}}% 

{\devanagarifont यावत्कल्पसहस्राणि परार्धानि तपोधन \thinspace{\dandab} \dontdisplaylinenum }%
     \var{{\devanagarifontvar\numemph\va परार्धानि\lem \mssALL,\hskip.2em plus .9em 
पराणि \msCbacorr\oo 
 ॰धन\lem \mssALL,\hskip.2em plus .9em ॰धनाः \Ed}}% 

%Verse 12:146

{\devanagarifont यत्र यत्र प्रयासित्वं तत्र तत्रोपभुज्यताम् {॥ १२:\hspace{.11em}१४६॥} \veg\dontdisplaylinenum }%
     \var{{\devanagarifontvar\numnoemph\vd ॰पभुज्यताम्\lem \mssALL,\hskip.2em plus .9em ॰प्रभुज्यताम् \msNb}}% 

{\devanagarifont महेश्वर उवाच {\dandab}\dontdisplaylinenum  }%
 
{\devanagarifont इति श्रुत्वा वचस्तस्य विपुलो विपुलेक्षणः \thinspace{\danda} \dontdisplaylinenum }%
     \var{{\devanagarifontvar\numemph\vb विपुलो\lem \mssALL,\hskip.2em plus .9em \om\ \msCb,\hskip.5em plus .9em विपुले \msCc}}% 

%Verse 12:147

{\devanagarifont वेपमानो भयत्रस्त अश्रुपूर्णाकुलेक्षणः {॥ १२:\hspace{.11em}१४७॥} \veg\dontdisplaylinenum }%
     \var{{\devanagarifontvar\numnoemph\vc भयत्रस्त\lem \Ed,\hskip.2em plus .9em भयस्तत्र \mssCaCbCc\msNa\msNb,\hskip.5em plus .9em 
भयस्त्रत्र \msNc}}% 
    \var{{\devanagarifontvar\numnoemph\vd अश्रु॰\lem \mssALL,\hskip.2em plus .9em अश्व॰ \msNc\oo 
 ॰पूर्णा॰\lem \mssALL,\hskip.2em plus .9em ॰पूर्ण्ण॰ \msNb}}% 

{\devanagarifont प्रणम्य शिरसा भूमौ प्रणिपत्य पुनः पुनः \thinspace{\dandab} \dontdisplaylinenum }%
     \var{{\devanagarifontvar\numemph\va शिरसा\lem \mssALL,\hskip.2em plus .9em शिर \msNbacorr}}% 

%Verse 12:148

{\devanagarifont उवाच मधुरं वाक्यं ब्रह्मलोकपितामहम् {॥ १२:\hspace{.11em}१४८॥} \veg\dontdisplaylinenum }%
     \var{{\devanagarifontvar\numnoemph\vc मधुरं\lem \mssALL,\hskip.2em plus .9em मधुर॰ \msCb}}% 
    \var{{\devanagarifontvar\numnoemph\vd ॰लोक॰\lem \mssALL,\hskip.2em plus .9em लोके \Ed}}% 

{\devanagarifont विपुल उवाच {\dandab}\dontdisplaylinenum  }%
 
{\devanagarifont भगवन्सर्वलोकेश सर्वलोकपितामह \thinspace{\danda} \dontdisplaylinenum }%
 
{\devanagarifont स्वप्नभूतमिवाश्चर्यं पश्यामि त्रिदशेश्वर  \danda\dontdisplaylinenum }%
     \var{{\devanagarifontvar\numemph\vc स्वप्नभूतमिवा॰\lem \mssALL,\hskip.2em plus .9em 
स्वप्नमितमिवा॰ \msCc}}% 

%Verse 12:149

{\devanagarifont स्मृतिभ्रंशश्च मे जातो बुद्धिर्जातान्धचेतना {॥ १२:\hspace{.11em}१४९॥} \veg\dontdisplaylinenum }%
     \var{{\devanagarifontvar\numnoemph\vf बुद्धिर्जातान्धचेतना\lem \mssCaCbCc,\hskip.2em plus .9em बुद्धिर्जान्धचेतना \msNaacorr,\hskip.5em plus .9em 
बुद्धिर्जातन्धचेतना \msNapcorr,\hskip.5em plus .9em बुद्धि जातन्धचेना \msNb,\hskip.5em plus .9em 
बुद्धि जातात्वचेतना \msNc,\hskip.5em plus .9em 
बुद्धिर्जातो ऽन्धचेतनः\thinspace{\devanagarifont ।} मूढो ऽहं त्वां कथं स्तौमि ज्ञानातीतं परात्परम्\thinspace{\devanagarifont ॥} \Ed}}% 

\nemslokalong


\ujvers\nemsloka {
{\devanagarifont तुभ्यं त्रैलोक्यबन्धो भव मम शरणं त्राहि संसारघोराद् }%
  \dontdisplaylinenum}    \var{{\devanagarifontvar\numemph\va तुभ्यं\lem \mssALL,\hskip.2em plus .9em तुभ्यंस् \msNb,\hskip.5em plus .9em नमस् \Ed\oo 
 त्रैलोक्य॰\lem \mssALL,\hskip.2em plus .9em त्रेलोक्य॰ \msCb\oo 
 ॰बन्धो\lem \mssALL,\hskip.2em plus .9em ॰\uncl{वन्तो} \msNa\oo 
 ॰घोराद्\lem \corr,\hskip.2em plus .9em ॰घोरम् \msCa\msCc\msNb\Ed,\hskip.5em plus .9em ॰घोरात् \msCb,\hskip.5em plus .9em ॰घोरः \msNa,\hskip.5em plus .9em 
॰\uncl{घोरात}त् \msNc}}% 


\nemslokab

{\devanagarifont भीतो ऽहं गर्भवासाज्जरमरणभयात्त्राहि मां मोहबन्धात्  \danda\dontdisplaylinenum }%
     \var{{\devanagarifontvar\numnoemph\vb ॰साज्जर॰\lem \mssALL,\hskip.2em plus .9em 
॰सा जर॰ \msCc,\hskip.5em plus .9em ॰साज्जनु॰ \Ed\oo 
 ॰मरण॰\lem \mssALL,\hskip.2em plus .9em ॰ण॰ \msNbacorr\oo 
 ॰भयात्\lem \Ed,\hskip.2em plus .9em भयं \mssCaCbCc\msNa\msNb\msNc}}% 

\nemslokac

{\devanagarifont नित्यं रोगाधिवासमनियतवपुषं त्राहि मां कालपाशात् }%
  \dontdisplaylinenum    \var{{\devanagarifontvar\numnoemph\vc नित्यं\lem \mssALL,\hskip.2em plus .9em नित्य॰ \msCb\ \unmetr\oo 
 रोगा॰\lem \mssALL,\hskip.2em plus .9em ॰रागा॰ \Ed\oo 
 ॰वासमनियत॰\lem \mssALL,\hskip.2em plus .9em ॰वासमतियत॰ \msCb,\hskip.5em plus .9em 
॰वासंमनियत॰ \msNa\oo 
 ॰वपुषं त्राहि मां\lem \mssALL,\hskip.2em plus .9em 
॰\uncl{वपुष त्राहि मा} \msCb\oo 
 कालपाशात्\lem \mssALL,\hskip.2em plus .9em 
कापाशात् \msNaacorr,\hskip.5em plus .9em कालपाशान् \msNb}}% 

%Verse 12:150


\nemslokad

{\devanagarifont तिर्यं चान्योन्यभक्षं बहुयुगशतशस्त्राहि मोहान्धकारात् {॥ १२:\hspace{.11em}१५०॥} \veg\dontdisplaylinenum }%
     \var{{\devanagarifontvar\numnoemph\vd तिर्यं चान्योन्यभक्षं\lem \mssALL,\hskip.2em plus .9em 
तिर्यं चान्यान्यभक्षं \msNb,\hskip.5em plus .9em 
तिर्यश्चान्योन्यभक्षं \Ed\oo 
 ॰शतशस्त्राहि\lem \mssALL,\hskip.2em plus .9em 
॰सतस त्राहि \msCc}}% 

\ujvers\nemsloka {
{\devanagarifont श्रुत्वैवोवाच ब्रह्मा विपुलमति पुनर्मानयित्वा यथावद् }%
  \dontdisplaylinenum}    \var{{\devanagarifontvar\numemph\va श्रुत्वैवोवाच\lem \mssALL,\hskip.2em plus .9em श्रुत्वैव वाच \Ed\oo 
 ॰मति\lem \msCc\Ed,\hskip.2em plus .9em ॰मतिः \msCa\msCb\msNa\msNb\msNc\ \unmetr\oo 
 मानयित्वा\lem \mssALL,\hskip.2em plus .9em माणयित्वा \msNc,\hskip.5em plus .9em मानयंवा \Ed\oo 
 यथावद्\lem \corr,\hskip.2em plus .9em यथावत् \mssCaCbCc\msNapcorr\msNb\msNc\Ed,\hskip.5em plus .9em 
वत् \msNaacorr}}% 


\nemslokab

{\devanagarifont आहूतसम्प्लवान्ते भविष्यसि तव मे जन्मलोभो न भूयः  \danda\dontdisplaylinenum }%
     \var{{\devanagarifontvar\numnoemph\vb आहूत\lem \mssALL,\hskip.2em plus .9em आभूत \Ed\oo 
 सम्प्लवान्ते\lem \msCc,\hskip.2em plus .9em सम्प्लवन्ते \msCa\msCb\msNa\msNb\Ed,\hskip.5em plus .9em 
संप्लवंन्ते \msNc\oo 
 भविष्यसि\lem \mssALL,\hskip.2em plus .9em 
भविष्य \msCc,\hskip.5em plus .9em अविपलि \Ed\oo 
 मे जन्मलोभो न\lem \mssCaCbCc\msNa,\hskip.2em plus .9em 
मे जन्मलाभो न \msNb\msNc,\hskip.5em plus .9em यजन्मलाभानु \Ed\oo 
 भूयः\lem \mssALL,\hskip.2em plus .9em भूय \msNc}}% 

\nemslokac

{\devanagarifont गर्भावासं न च त्वन्न च पुनमरणं क्लेशमायासपूर्णं }%
  \dontdisplaylinenum    \var{{\devanagarifontvar\numnoemph\vc ॰वासं न च त्वन्न\lem \msCa\msNa\msNb\msNc,\hskip.2em plus .9em ॰वासन्न \msCb,\hskip.5em plus .9em 
॰वासा न च त्वन्न \msCc,\hskip.5em plus .9em ॰वासानुबन्धं न \Ed\oo 
 पुनमरणं\lem \msCc\Ed,\hskip.2em plus .9em पुनर्मरणं \msCa\msNa\msNb\msNc\ \unmetr,\hskip.5em plus .9em 
पुनर्मण \msCb\oo 
 ॰पूर्णम्\lem \mssALL,\hskip.2em plus .9em ॰पूर्ण्ण \msCc}}% 

%Verse 12:151


\nemslokad

{\devanagarifont छित्त्वा मोहान्धशत्रुं व्रजसि च परमं ब्रह्मभूयत्वमेषि {॥ १२:\hspace{.11em}१५१॥} \veg\dontdisplaylinenum }%
     \var{{\devanagarifontvar\numnoemph\vd ॰शत्रुं\lem \mssALL,\hskip.2em plus .9em ॰शत्रु \msCb\msCc\oo 
 परमं\lem \mssALL,\hskip.2em plus .9em परम \msNb}}% 
    \paral{{\devanagarifontsmall \vd {\englishfont \compare\ Manu 1.98cd:} स हि धर्मार्थमुत्पन्नो ब्रह्मभूयाय कल्पते
                 {\englishfont and Manu 12.102cd:} इहैव लोके तिष्ठन्स ब्रह्मभूयाय कल्पते }}

\nemslokanormal


\vers


{\devanagarifont महेश्वर उवाच {\dandab}\dontdisplaylinenum  }%
 
{\devanagarifont ब्रह्मणा एवमुक्तस्तु विष्णुना प्रभविष्णुना \thinspace{\danda} \dontdisplaylinenum }%
     \var{{\devanagarifontvar\numemph\vb विष्णुना\lem \mssALL,\hskip.2em plus .9em \om\ \msCb,\hskip.5em plus .9em विष्णुनात् \msCc}}% 

%Verse 12:152

{\devanagarifont एवं भवतु भद्रं वो यथोवाच पितामहः {॥ १२:\hspace{.11em}१५२॥} \veg\dontdisplaylinenum }%
     \var{{\devanagarifontvar\numnoemph\vd ॰महः\lem \msCa\msNc\Ed,\hskip.2em plus .9em ॰मह \msCb\msCc\msNa\msNb}}% 

{\devanagarifont इन्द्रेण रविणा चैव सोमेन च पुनः पुनः \thinspace{\dandab} \dontdisplaylinenum }%
     \var{{\devanagarifontvar\numemph\va रविणा\lem \mssALL,\hskip.2em plus .9em रविना \msCc,\hskip.5em plus .9em शशिना \Ed}}% 
    \var{{\devanagarifontvar\numnoemph\vb सोमेन\lem \mssALL,\hskip.2em plus .9em सूर्येण \Ed\oo 
 पुनः पुनः\lem \mssALL,\hskip.2em plus .9em पुन पुनः \msCb\ \unmetr,\hskip.5em plus .9em 
पुन च पुनः पुनः \msCc}}% 

%Verse 12:153

{\devanagarifont साध्यादित्यैर्मरुद्रुद्रैर्विश्वेभिर्वसवैस्तथा {॥ १२:\hspace{.11em}१५३॥} \veg\dontdisplaylinenum }%
     \var{{\devanagarifontvar\numnoemph\vc ॰दित्यैर्म॰\lem \mssALL,\hskip.2em plus .9em ॰दित्यै म॰ \msCc}}% 
    \var{{\devanagarifontvar\numnoemph\vcd ॰रुद्रुद्रैर्विश्वेभिर्\lem \Ed,\hskip.2em plus .9em ॰रुद्रुद्रैर्विश्वेश्वि \msCa\msNa,\hskip.5em plus .9em 
॰रुद्रुद्रै विश्वाश्वि \msCb,\hskip.5em plus .9em ॰रुद्रुद्रै विश्वेश्वि \msCc,\hskip.5em plus .9em 
॰रुद्रै विश्वे\lk\ \msNb,\hskip.5em plus .9em ॰रुद्रैर्विश्वेश्वि \msNc}}% 

{\devanagarifont अहो तपःफलं दिव्यं विपुलस्य महात्मनः \thinspace{\dandab} \dontdisplaylinenum }%
 
%Verse 12:154

{\devanagarifont स्वशरीरो दिवं प्राप्तः श्रद्धयातिथिपूजया {॥ १२:\hspace{.11em}१५४॥} \veg\dontdisplaylinenum }%
     \var{{\devanagarifontvar\numemph\vc स्वशरीरो\lem \eme,\hskip.2em plus .9em स्वशरीरं \msCa\msNa\msNb\msNc,\hskip.5em plus .9em शशरीरो \msCb,\hskip.5em plus .9em 
स्वशरीर \msCc,\hskip.5em plus .9em सशरीरं \Ed\oo 
 प्राप्तः\lem \msCb\msCc,\hskip.2em plus .9em प्राप्तं \msCa\msNa\msNb\msNc\Ed}}% 
    \var{{\devanagarifontvar\numnoemph\vd ॰पूजया\lem \mssALL,\hskip.2em plus .9em ॰पूजनात् \Ed}}% 

{\devanagarifont एवमादीन्यनेकानि विपुले परिकीर्तितम् \thinspace{\dandab} \dontdisplaylinenum }%
     \var{{\devanagarifontvar\numemph\vb ॰नेकानि\lem \mssALL,\hskip.2em plus .9em ॰नेनेकानि \msNb}}% 

%Verse 12:155

{\devanagarifont ब्रह्माणं पुनरेवाह विष्णुर्विश्वजगत्प्रभुः {॥ १२:\hspace{.11em}१५५॥} \veg\dontdisplaylinenum }%
     \var{{\devanagarifontvar\numnoemph\vc ब्रह्माणं\lem \mssALL,\hskip.2em plus .9em ब्राह्मणः \msCb,\hskip.5em plus .9em 
ब्रह्मणं \msCc}}% 
    \var{{\devanagarifontvar\numnoemph\vd विष्णुर्वि॰\lem \mssALL,\hskip.2em plus .9em विष्णु वि॰ \msCc\oo 
 ॰जगत्प्रभुः\lem \mssALL,\hskip.2em plus .9em ॰जगत्प्रभु \msCc}}% 

{\devanagarifont 
\jump
\begin{center}
\ketdanda~इति वृषसारसंग्रहे विपुलोपाख्यानो नामाध्यायो द्वादशमः~\ketdanda
\end{center}
\dontdisplaylinenum\vers  }%
     \var{{\devanagarifontvar\numnoemph{\englishfont \Colo:} वृषसार॰\lem \mssALL,\hskip.2em plus .9em वृष॰ \msNb\oo 
 ॰ख्यानो नामाध्यायो द्वादशमः\lem \mssALL,\hskip.2em plus .9em 
॰ख्या\uncl{न ना}माध्यायो द्वादश \msNc,\hskip.5em plus .9em 
॰ख्यानो नाम द्वादशो ऽध्यायः \Ed}}% 
