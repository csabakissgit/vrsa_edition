\fejno=0\versno=0
\centerline{\Huge\devanagarifontbold वृषसारसंग्रहः  }

 
{\vrule depth10pt width0pt}
\versno=0\fejno=4
\thispagestyle{empty}

\centerline{\Large\devanagarifontbold [   चतुर्थो ऽध्यायः  ]}{\vrule depth10pt width0pt} \fancyhead[CO]{{\footnotesize\devanagarifont वृषसारसंग्रहे  }}
\fancyhead[CE]{{\footnotesize\devanagarifont चतुर्थो ऽध्यायः  }}
\fancyhead[LE]{}
\fancyhead[RE]{}
\fancyhead[LO]{}
\fancyhead[RO]{}
\szam\bek



\alalfejezet{यमेषु सत्यम् (२)}
\vers


{\devanagarifont अनर्थयज्ञ उवाच {\dandab}\dontdisplaylinenum  }%
     \lacuna{\devanagarifontsmall {\englishfont Witnesses used for this chapter: \msCa\ ff.\thinspace 198v--201v, 
                                              \msCb\ ff.\thinspace 206r--208v, 
                                              \msCc\ ff.\thinspace 273v--277r,
                                              \msNa\ ff.\thinspace 6r--9r, 
                                              \msNb\ exp.\thinspace 48--50 (lower--upper),
                                              \msNc\ ff.\thinspace 214v--217r,
                                              \Ed\ pp.\thinspace 591--597;
                                        \mssCaCbCc\ = \msCa + \msCb + \msCc} }%
  
{\devanagarifont सद्भावः सत्यमित्याहुर्दृष्टप्रत्ययमेव वा \thinspace{\danda} \dontdisplaylinenum }%
     \var{{\devanagarifont \numemph\va\textbf{सद्भावः}\lem \mssCaCbCc\msNa\msNc, सद्भाव॰ \msNb\Ed}}% 
    \var{{\devanagarifont \numnoemph\vab\textbf{सत्यमित्याहुर्दृ॰}\lem \msCb\msNa\msNc\Ed, सत्य\uncl{मि}त्याहु दृ॰ \msCa, 
सत्यमित्याहु दृ॰ \msCc, सत्यामित्याहुर्दृ॰ \msNb}}% 
    \var{{\devanagarifont \numnoemph\vb\textbf{॰प्रत्यय॰}\lem \msCa\msCb\msNa\msNb, ॰प्रत्य॰ \msCc, ॰प्रत्येय॰ \msNc, प्रत्यक्ष॰ \Ed}}% 
    \paral{{\devanagarifontsmall \va {\englishfont \similar\ \MBH\ 12.288.45d:} सद्भावः सत्यमुच्यते {\englishfont  \oo \compare\  also \BRAHMANDAPUR\ 3.3.86ab:}
                         असद्भावो ऽनृतं ज्ञेयं सद्भावः सत्यमुच्यते  }}

%Verse 4:1

{\devanagarifont यथाभूतार्थकथनं तत्सत्यकथनं स्मृतम् {॥४:१॥} \veg\dontdisplaylinenum }%
     \var{{\devanagarifont \numnoemph\vc\textbf{यथाभूतार्थकथनं}\lem \msCa\msCb\msNa\msNb\msNc\Ed, 
यथाभूतार्थ \msCcacorr, यथाभूतार्थनं क्त \msCcpcorr}}% 
    \var{{\devanagarifont \numnoemph\vd\textbf{तत्सत्यकथनं}\lem \msCa\msNa\msNb\msNc\Ed, 
तत्सत्यकथकं \msCb, कथनं स्मृतं \msCcacorr, \uncl{सत्यक ज}कथनं स्मृतं \msCcpcorr}}% 
    \paral{{\devanagarifontsmall \vcd {\englishfont \compare\ \SDHS\ 11.105:} 
                 स्वानुभूतं स्वदृष्टं च यः पृष्टार्थं न गूहति\thinspace{\devanagarifontsmall ।}
                 यथाभूतार्थकथनमित्येतत्सत्यलक्षणम्\thinspace{\devanagarifontsmall ॥} }}

{\devanagarifont आक्रोशताडनादीनि यः सहेत सुदुःसहम् \thinspace{\dandab} \dontdisplaylinenum }%
     \var{{\devanagarifont \numemph\va\textbf{॰ताडना॰}\lem \msCa\msCc\msNa\msNb\msNc\Ed, ॰नाडना॰ \msCb}}% 
    \var{{\devanagarifont \numnoemph\vb\textbf{सुदुःसहम्}\lem \msCa\msCb\msNa\msNb\msNc\Ed, सुदुसहं \msCc}}% 

%Verse 4:2

{\devanagarifont क्षमते यो जितात्मा तु स च सत्यमुदाहृतम् {॥४:२॥} \veg\dontdisplaylinenum }%
     \var{{\devanagarifont \numnoemph\vd\textbf{सत्यमुदाहृतम्}\lem \msCb\msCc\msNa\msNb\msNc\Ed, 
\uncl{सत्य}मु\uncl{दा}हृतम् \msCa}}% 
    \paral{{\devanagarifontsmall \vo {\englishfont \compare\ \SDHS\ 11.82:}
                 आक्रुष्टस्ताडितो वापि यो नाक्रोशेन्न ताडयेत्\thinspace{\devanagarifontsmall ।}
                 वागाद्यविकृतः स्वस्थं क्षान्तिरेषा सुनिर्मला\thinspace{\devanagarifontsmall ॥} }}

{\devanagarifont वधार्थमुद्यतः शस्त्रं यदि पृच्छेत कर्हिचित् \thinspace{\dandab} \dontdisplaylinenum }%
     \var{{\devanagarifont \numemph\va\textbf{॰द्यतः}\lem \mssCaCbCc\msNb\msNc\Ed, ॰द्यत \msNa\oo 
\textbf{शस्त्रं}\lem \msCa\msNa\msNb\msNc, सत्य \msCb\Ed, शस्त्र \msCc}}% 
    \var{{\devanagarifont \numnoemph\vb\textbf{कर्हिचित्}\lem \mssCaCbCc\Ed, कर्हचित् \msNa\msNb\msNc}}% 

%Verse 4:3

{\devanagarifont न तत्र सत्यं वक्तव्यमनृतं सत्यमुच्यते {॥४:३॥} \veg\dontdisplaylinenum }%
     \var{{\devanagarifont \numnoemph\vc\textbf{सत्यं}\lem \msCa\msCc\msNa\msNb\msNc, सत्य \msCb\Ed}}% 

{\devanagarifont वधार्हः पुरुषः कश्चिद्व्रजेत्पथि भयातुरः \thinspace{\dandab} \dontdisplaylinenum }%
     \var{{\devanagarifont \numemph\vb\textbf{॰तुरः}\lem \msCa\msCc\msNa\msNb\msNc\Ed, ॰तुर \msCb}}% 

%Verse 4:4

{\devanagarifont पृच्छतो ऽपि न वक्तव्यं सत्यं तद्वापि उच्यते {॥४:४॥} \veg\dontdisplaylinenum }%
     \var{{\devanagarifont \numnoemph\vc\textbf{पृच्छतो}\lem \mssCaCbCc\msNa\msNb\msNc, पृच्छते \Ed}}% 
    \var{{\devanagarifont \numnoemph\vd\textbf{तद्वापि}\lem \mssCaCbCc\msNa\msNc\Ed, तदपि \msNb}}% 

\ujvers\nemsloka {
{\devanagarifont न नर्मयुक्तमनृतं हिनस्ति }%
  \dontdisplaylinenum}    \var{{\devanagarifont \numemph\va\textbf{हिनस्ति}\lem \msCa\msCb\msNb\msNc, हि नास्ति \msCc\msNa\Ed}}% 


\nemslokab

{\devanagarifont न स्त्रीषु राजन्न विवाहकाले  \danda\dontdisplaylinenum }%
     \var{{\devanagarifont \numnoemph\vb\textbf{राजन्न}\lem \msCa\msCb\msNb\msNc\Ed, राज न \msCc, राज्यं न \msNa}}% 

\nemslokac

{\devanagarifont प्राणात्यये सर्वधनापहारे }%
  \dontdisplaylinenum    \var{{\devanagarifont \numnoemph\vc\textbf{॰त्यये}\lem \mssCaCbCc\msNa\msNc\Ed, ॰त्यजे \msNb\oo 
\textbf{॰पहारे}\lem \msCa\msCb\msNa\msNc\Ed, ॰प्रहारे \msCc\msNb}}% 

%Verse 4:5


\nemslokad

{\devanagarifont पञ्चानृतं सत्यमुदाहरन्ति {॥४:५॥} \veg\dontdisplaylinenum }%
     \paral{{\devanagarifontsmall \vo {\englishfont \similar\ \MBH\ 1.77.16:} न नर्मयुक्तं वचनं हिनस्ति न स्त्रीषु राजन्न विवाहकाले\thinspace{\devanagarifontsmall ।}
                                                प्राणात्यये सर्वधनापहारे पञ्चानृतान्याहुरपातकानि\thinspace{\devanagarifontsmall ॥};
                            {\englishfont \MBH\ 12.159.28:} न नर्मयुक्तं वचनं हिनस्ति न स्त्रीषु राजन्न विवाहकाले\thinspace{\devanagarifontsmall ।}
                                                न गुर्वर्थे नात्मनो जीवितार्थे पञ्चानृतान्याहुरपातकानि\thinspace{\devanagarifontsmall ॥};
                              {\englishfont \MATSP\ 31.16:} न नर्मयुक्तं वचनं हिनस्ति न स्त्रीषु राजन्न विवाहकाले\thinspace{\devanagarifontsmall ।}
         {\englishfont Abhidharmakośabhāṣya 24114--24117 (introduced by } मोहजो मृषावादो यथाह{\englishfont ):}
                                                न नर्मयुक्तमनृतं हि नास्ति न स्त्रीषु राजन्न विवाहकाले\thinspace{\devanagarifontsmall ।}
                                                प्राणात्यये सर्वधनापहारे पञ्चानृतान्याहुरपातकानि\thinspace{\devanagarifontsmall ॥} {\englishfont etc.} }}

\vers


{\devanagarifont देवमानुषतिर्येषु सत्यं धर्मः परो यतः \thinspace{\dandab} \dontdisplaylinenum }%
     \var{{\devanagarifont \numemph\vb\textbf{॰मानुष॰}\lem \mssCaCbCc\msNa\msNb\Ed, ॰मानुष्य॰ \msNc\oo 
\textbf{सत्यं धर्मः परो यतः}\lem \msCb\msCc, सत्यं धर्मः पयतः \msCa, 
सत्यं धर्म परो यतः \msNa\msNc, सत्यधर्म परो यतः \msNb, सत्यधर्मपरायणः \Ed}}% 

%Verse 4:6

{\devanagarifont सत्यं श्रेष्ठं वरिष्ठं च सत्यं धर्मः सनातनः {॥४:६॥} \veg\dontdisplaylinenum }%
     \var{{\devanagarifont \numnoemph\vc\textbf{श्रेष्ठं}\lem \mssCaCbCc\msNa\msNc, श्रेष्ठ \msNb\Ed\oo 
\textbf{वरिष्ठं च}\lem \msCa\msCbpcorr\msCc\msNa\msNb\msNc\Ed, वरिष्ठम्वरिष्ठम्वञ्च \msCbacorr}}% 
    \var{{\devanagarifont \numnoemph\vd\textbf{सत्यं}\lem \msCa\msCc\msNa\msNc\Ed, सत्य॰ \msCb\msNb\oo 
\textbf{धर्मः}\lem \msCa\msCb\msNa\msNb\msNc, धर्म \msCc\Ed}}% 

{\devanagarifont सत्यं सागरमव्यक्तं सत्यमक्षयभोगदम् \thinspace{\dandab} \dontdisplaylinenum }%
     \var{{\devanagarifont \numemph\va\textbf{सत्यं}\lem \msCa\msCb\msNa\msNb\msNc\Ed, सत्य \msCc}}% 
    \var{{\devanagarifont \numnoemph\vb\textbf{सत्यमक्षयभोगदम्}\lem \msCa\msNa\msNb\msNc, सत्यंमक्षयभोगदम् \msCb\msCc, 
सत्यमक्षयते नरं \Ed}}% 

%Verse 4:7

{\devanagarifont सत्यं पोतः परत्रार्थं सत्यं पन्थान विस्तरम् {॥४:७॥} \veg\dontdisplaylinenum }%
     \var{{\devanagarifont \numnoemph\vc\textbf{पोतः}\lem \mssCaCbCc\msNb\msNc, पोत \msNa, प्रोक्तः \Ed}}% 
    \var{{\devanagarifont \numnoemph\vd\textbf{पन्थान विस्तरम्}\lem \mssCaCbCc\msNa\msNb\msNc, यज्ज्ञानविस्तरम् \Ed}}% 

{\devanagarifont सत्यमिष्टगतिः प्रोक्तं सत्यं यज्ञमनुत्तमम् \thinspace{\dandab} \dontdisplaylinenum }%
     \var{{\devanagarifont \numemph\va\textbf{॰ष्टगतिः}\lem \mssCaCbCc\msNa\msNc\Ed, ॰\uncl{ष्टा}गतिः \msNb}}% 

%Verse 4:8

{\devanagarifont सत्यं तीर्थं परं तीर्थं सत्यं दानमनन्तकम् {॥४:८॥} \veg\dontdisplaylinenum }%
     \var{{\devanagarifont \numnoemph\vc\textbf{तीर्थं}\lem \mssCaCbCc\msNa, तीर्थ \msNb\msNc, तीर्थात् \Ed}}% 

{\devanagarifont सत्यं शीलं तपो ज्ञानं सत्यं शौचं दमः शमः \thinspace{\dandab} \dontdisplaylinenum }%
     \var{{\devanagarifont \numemph\va\textbf{सत्यं}\lem \msCa\msCc\msNa\msNb\msNc\Ed, सत्य \msCb}}% 
    \var{{\devanagarifont \numnoemph\vb\textbf{शमः}\lem \mssCaCbCc\msNa\msNc\Ed, शमम् \msNb}}% 

%Verse 4:9

{\devanagarifont सत्यं सोपानमूर्ध्वस्य सत्यं कीर्तिर्यशः सुखम् {॥४:९॥} \veg\dontdisplaylinenum }%
     \var{{\devanagarifont \numnoemph\vc\textbf{सत्यं}\lem \msCa\msCc\msNa\msNb\Ed, संत्यं \msCb, सत्य \msNc}}% 
    \var{{\devanagarifont \numnoemph\vd\textbf{सुखम्}\lem \mssCaCbCc\msNa\msNb\msNc, सुखः \Ed}}% 
    \paral{{\devanagarifontsmall \vc {\englishfont \similar\ \VARP\ 193.36cd:} सत्यं स्वर्गस्य सोपानं पारावारस्य नौरिव }}

{\devanagarifont अश्वमेधसहस्रं च सत्यं च तुलया धृतम् \thinspace{\dandab} \dontdisplaylinenum }%
     \var{{\devanagarifont \numemph\va\textbf{॰सहस्रं च}\lem \msCa\msCb\msNa\msNb\msNc\Ed, ॰सहस्रस्य \msCc}}% 
    \var{{\devanagarifont \numnoemph\vb\textbf{तुलया}\lem \msCa\msCb\msNa\msNb\msNc\Ed, तुल्यया \msCc}}% 

%Verse 4:10

{\devanagarifont अश्वमेधसहस्राद्धि सत्यमेव विशिष्यते {॥४:१०॥} \veg\dontdisplaylinenum }%
     \var{{\devanagarifont \numnoemph\vc\textbf{॰सहस्राद्धि}\lem \msCa\msCb\msNa\msNb\msNc\Ed, ॰सहस्रा हि \msCc}}% 
    \var{{\devanagarifont \numnoemph\vd\textbf{एव}\lem \msCa\msCb\msNa\msNb\msNc, एवं \msCc\Ed}}% 
    \paral{{\devanagarifontsmall \vo {\englishfont  = \MBH\ 1.69.22 = \MBH\ Indices 13.20.330 = \MARKP\ 8.42 = \VDHU\ 3.265.7
                        \similar\ \MBH\ 12.156.26 (pāda d reads } सत्यमेवातिरिच्यते{\englishfont ) \similar\ \VDH\ 55.6 
                            (pāda d reads} सत्यमेतद्विशिष्यते{\englishfont )};
                    {\englishfont \compare\ \SDHS\ 11.107:}
                         अश्वमेधायुतं पूर्णं सत्यञ्च तुलितं पुरा\thinspace{\devanagarifontsmall ।}
                         अश्वमेधायुतात्सत्यमधिकं बहुभिर्गुणैः\thinspace{\devanagarifontsmall ॥} }}

{\devanagarifont सत्येन तपते सूर्यः सत्येन पृथिवी स्थिता \thinspace{\dandab} \dontdisplaylinenum }%
     \var{{\devanagarifont \numemph\vab\textbf{सूर्यः सत्येन पृथिवी स्थिता}\lem \msNa\msNc, सू\uncl{र्यः स}त्येन पृथि स्थिताः \msCa, 
सूर्यः सत्यैन पृथिवी स्थिता \msCb, सूर्य  सत्येन पृथिवी स्थिताः \msCc, 
सूर्य \uncl{सत्ये} \lac\  वी स्थिता \msNb, सूर्यः सत्येन पृथिवी स्थिताः \Ed}}% 

%Verse 4:11

{\devanagarifont सत्येन वायवो वान्ति सत्ये तोयं च शीतलम् {॥४:११॥} \veg\dontdisplaylinenum }%
     \var{{\devanagarifont \numnoemph\vc\textbf{वायवो}\lem \mssCaCbCc\msNa\msNc\Ed, वात्यवो \msNb}}% 
    \var{{\devanagarifont \numnoemph\vd\textbf{सत्ये}\lem \mssCaCbCc\msNa\msNb\msNc, सत्यात् \Ed}}% 
    \paral{{\devanagarifontsmall \vo {\englishfont \similar\ \VARP\ 193.37:} 
                         सूर्यस्तपति सत्येन वातः सत्येन वाति च\thinspace{\devanagarifontsmall ।}  
                         अग्निर्दहति सत्येन सत्येन पृथिवी स्थिता\thinspace{\devanagarifontsmall ॥} 
                    {\englishfont \similar\ \VDHU\ 3.265.4cd--5ab:}
                         सत्येन वायुरभ्येति सत्येनाभासते रविः\thinspace{\devanagarifontsmall ॥} 
                         सत्येन चाग्निर्दहति स्वर्गं सत्येन गच्छति\thinspace{\devanagarifontsmall ।}  }}

{\devanagarifont तिष्ठन्ति सागराः सत्ये समयेन प्रियव्रतः \thinspace{\dandab} \dontdisplaylinenum }%
     \var{{\devanagarifont \numemph\va\textbf{सागराः}\lem \msCa\msCb\msNa\msNb\msNc\Ed, सागरा \msCc}}% 
    \var{{\devanagarifont \numnoemph\vb\textbf{समयेन}\lem \mssCaCbCc\msNa\msNb\msNc, सत्येन च \Ed}}% 

%Verse 4:12

{\devanagarifont सत्ये तिष्ठति गोविन्दो बलिबन्धनकारणात् {॥४:१२॥} \veg\dontdisplaylinenum }%
 
{\devanagarifont अग्निर्दहति सत्येन सत्येन शशिनश्चरः \thinspace{\dandab} \dontdisplaylinenum }%
     \var{{\devanagarifont \numemph\vab\textbf{सत्येन सत्येन}\lem \mssCaCbCc\msNapcorr\msNb\Ed, सत्येन \msNaacorr\msNc}}% 
    \var{{\devanagarifont \numnoemph\vb\textbf{शशिनश्चरः}\lem \conj, सशि\uncl{भाचरः} \msCa, 
श\uncl{सि}\lk चरः \msCb, 
स शिरा वरः \msCc, 
शशिराचरः \msNa\msNb\msNc, 
शशिभाष्करः \Ed}}% 
    \paral{{\devanagarifontsmall \vc {\englishfont \similar\ \VARP\ 193.37cd:} 
                 अग्निर्दहति सत्येन सत्येन पृथिवी स्थिता }}
    \paral{{\devanagarifontsmall \vd {\englishfont \compare\ \VARP\ 155.30cd:}
                         सत्येन सूर्यस्तपति सोमः सत्येन राजते;
                  {\englishfont \compare\ \LAKSMINARS\  1.345.50ab:}
                         सत्येन सूर्यस्तपति चन्द्रः सत्येन वर्धते\thinspace{\devanagarifontsmall ।}
                 {\englishfont \compare\ \MBH\ Indeces 13.587:}
                         मुचुकुन्देन मान्धात्रा हरिश्चन्द्रेण चाभिभो\thinspace{\devanagarifontsmall ।}
                         सत्यं वदत मासत्यं सत्यं धर्मः सनातनः\thinspace{\devanagarifontsmall ।}
                         हरिश्चन्द्रश्चरति वै दिवि सत्येन चन्द्रवत्\thinspace{\devanagarifontsmall ॥} }}

%Verse 4:13

{\devanagarifont सत्येन विन्ध्यास्तिष्ठन्ति वर्धमानो न वर्धते {॥४:१३॥} \veg\dontdisplaylinenum }%
     \var{{\devanagarifont \numnoemph\vc\textbf{विन्ध्यास्तिष्ठन्ति}\lem \msCa\msNa\msNc, 
विन्ध्यस्तिष्ठन्ति \msCb\msNb, विन्ध्या तिष्ठन्ति \msCc, तिष्ठते विन्ध्यो \Ed}}% 

{\devanagarifont लोकालोकः स्थितः सत्ये मेरुः सत्ये प्रतिष्ठितः \thinspace{\dandab} \dontdisplaylinenum }%
     \var{{\devanagarifont \numemph\va\textbf{॰लोकः}\lem \Ed, ॰लोक \mssCaCbCc\msNa\msNb\msNc\oo 
\textbf{स्थितः}\lem \mssCaCbCc\msNa\msNb\Ed, स्थिः \msNc\oo 
\textbf{सत्ये}\lem \mssCaCbCc\msNa\msNb\msNc, सत्यं \Ed}}% 
    \var{{\devanagarifont \numnoemph\vb\textbf{मेरुः}\lem \msCa\msCb\msNa\msNb\msNc, मेरु \msCc\Ed}}% 

%Verse 4:14

{\devanagarifont वेदास्तिष्ठन्ति सत्येषु धर्मः सत्ये प्रतिष्ठति {॥४:१४॥} \veg\dontdisplaylinenum }%
     \var{{\devanagarifont \numnoemph\vc\textbf{वेदास्ति॰}\lem \msCa\msCc\msNa\msNb\msNc, देवास्ति॰ \msCb, वेदा ति॰ \Ed}}% 
    \var{{\devanagarifont \numnoemph\vd\textbf{सत्ये}\lem \msCa\msCb\msNa\msNb\msNc\Ed, धर्मे \msCc\oo 
\textbf{प्रतिष्ठति}\lem \mssCaCbCc\msNa\msNb\Ed, प्रतिष्ठिति \msNcacorr, प्रतिष्ठितः \msNcpcorr}}% 

{\devanagarifont सत्यं गौः क्षरते क्षीरं सत्यं क्षीरे घृतं स्थितम् \thinspace{\dandab} \dontdisplaylinenum }%
     \var{{\devanagarifont \numemph\va\textbf{गौः}\lem \msCa\msCb\msNa\msNc\Ed, गौ \msCc\msNb}}% 
    \var{{\devanagarifont \numnoemph\vab\textbf{क्षीरं सत्यं}\lem \msCa\msCc\msNa\msNb\msNc\Ed, क्षीत्यं \msCbacorr, क्सी\lk  नित्यं \msCbpcorr}}% 
    \var{{\devanagarifont \numnoemph\vb\textbf{क्षीरे घृतं स्थितम्}\lem \msCa\msCb\msNa\msNc, क्षीरं घृतं स्थितम् \msCc, क्षीरे घृत स्थितम् \msNb, 
क्षीरं स्थितं घृतम् \Ed}}% 

%Verse 4:15

{\devanagarifont सत्ये जीवः स्थितो देहे सत्यं जीवः सनातनः {॥४:१५॥} \veg\dontdisplaylinenum }%
     \var{{\devanagarifont \numnoemph\vc\textbf{सत्ये जीवः}\lem \mssCaCbCc\msNa\msNb, सत्ये जीव \msNc, सत्यं जीव \Ed}}% 
    \var{{\devanagarifont \numnoemph\vd\textbf{जीवः}\lem \msCa\msCb\msNa\msNb\msNc\Ed, जीव \msCc}}% 

{\devanagarifont सत्यमेकेन सम्प्राप्तो धर्मसाधननिश्चयः \thinspace{\dandab} \dontdisplaylinenum }%
     \var{{\devanagarifont \numemph\va\textbf{सत्यमेकेन}\lem \msCa\msCc\msNa\msNc\Ed, सत्यमेकैन \msCb, सत्येमेकेन \msNb}}% 
    \var{{\devanagarifont \numnoemph\vb\textbf{धर्म॰}\lem \Ed, धर्मः \mssCaCbCc\msNa\msNb\msNc\oo 
\textbf{॰निश्चयः}\lem \msCb\msCc\msNa\msNb\msNc\Ed, ॰निश्चः \msCa}}% 

%Verse 4:16

{\devanagarifont रामराघववीर्येण सत्यमेकं सुरक्षितम् {॥४:१६॥} \veg\dontdisplaylinenum }%
     \var{{\devanagarifont \numnoemph\vd\textbf{सत्यमेकं}\lem \mssCaCbCc\msNa\msNc\Ed, सत्येमेकं \msNb\oo 
\textbf{सुरक्षितम्}\lem \msCa\msCc\msNb\msNc\Ed, सुरिक्षितम् \msCb, सुरक्षितः \msNa}}% 

{\devanagarifont एवं सत्यविधानस्य कीर्तितं तव सुव्रत \thinspace{\dandab} \dontdisplaylinenum }%
     \var{{\devanagarifont \numemph\va\textbf{एवं सत्य॰}\lem \msCb, एतत्सत्य॰ \msCa\msCc\msNa\msNb\msNc\Ed}}% 
    \var{{\devanagarifont \numnoemph\vb\textbf{सुव्रत}\lem \msCa\msNa\msNc, सुव्रते \msCb\msNb, सुव्र\uncl{तः} \msCc, सुव्रतं \Ed}}% 

%Verse 4:17

{\devanagarifont सर्वलोकहितार्थाय किमन्यच्छ्रोतुमिच्छसि {॥४:१७॥} \veg\dontdisplaylinenum }%
 

\alalfejezet{यमेष्वस्तेयम् (३)}
{\devanagarifont विगतराग उवाच {\dandab}\dontdisplaylinenum  }%
 
{\devanagarifont न हि तृप्तिं विजानामि श्रुत्वा धर्मं तवाप्यहम् \thinspace{\danda} \dontdisplaylinenum }%
     \var{{\devanagarifont \numemph\va\textbf{तृप्तिं}\lem \msCa\msCb\msNa\msNb\msNc\Ed, तृप्ति \msCc\oo 
\textbf{विजानामि}\lem \mssCaCbCc\msNa\msNc\Ed, विनामि \msNb}}% 
    \var{{\devanagarifont \numnoemph\vb\textbf{श्रुत्वा धर्मं तवाप्यहम्}\lem \msCb\msCc\msNa\msNb\msNc, श्रु धर्मन्तवाप्यहम् \msCa, 
धर्मं श्रुत्वा तथाप्यहम् \Ed}}% 

%Verse 4:18

{\devanagarifont उपरिष्टादतो भूयः कथयस्व तपोधन {॥४:१८॥} \veg\dontdisplaylinenum }%
     \var{{\devanagarifont \numnoemph\vd\textbf{॰धन}\lem \msCc\msNa\msNb\Ed, ॰धून \msCa, ॰धनः \msCb\msNc}}% 

{\devanagarifont अनर्थयज्ञ उवाच {\dandab}\dontdisplaylinenum  }%
 
{\devanagarifont स्तेयं शृण्वथ विप्रेन्द्र पञ्चधा परिकीर्तितम् \thinspace{\danda} \dontdisplaylinenum }%
     \var{{\devanagarifont \numemph\vb\textbf{॰कीर्तितम्}\lem \msCa\msCc\msNa\msNb\msNc\Ed, ॰कीर्त्तिताम् \msCb}}% 

{\devanagarifont अदत्तादानमादौ तु उत्कोचं च ततः परम्  \danda\dontdisplaylinenum }%
     \var{{\devanagarifont \numnoemph\vd\textbf{उत्कोचं च ततः}\lem \msCa\msCc\msNa\msNb\msNc, त्कोच ततः \msCb, उत्कोचं चानृतः \Ed}}% 

%Verse 4:19

{\devanagarifont प्रस्थव्याजस्तुलाव्याजः प्रसह्यस्तेय पञ्चमम् {॥४:१९॥} \veg\dontdisplaylinenum }%
     \var{{\devanagarifont \numnoemph\ve\textbf{तुलाव्याजः}\lem \msCb\msNc\Ed, तुलाव्याज \msCa\msCc\msNa\msNb}}% 
    \var{{\devanagarifont \numnoemph\vf\textbf{॰सह्य॰}\lem \mssCaCbCc\msNa\msNc\Ed, ॰सह्ये \msNb\oo 
\textbf{॰स्तेय}\lem \msCb\msCc\msNa\msNb\Ed, ॰स्तेन \msCa\msNc\oo 
\textbf{पञ्चमम्}\lem \msCa\msCb\msNa\msNb\msNc, पञ्चमः \msCc\Ed}}% 

{\devanagarifont धृष्टदुष्टप्रभावेन परद्रव्यापकर्षणम् \thinspace{\dandab} \dontdisplaylinenum }%
     \var{{\devanagarifont \numemph\va\textbf{धृष्टदुष्ट॰}\lem \msCa\msNa\msNc\Ed, धृष्टदुम्न॰ \msCb, धृतदुष्ट॰ \msCc, दृष्तदुष्ट॰ \msNb}}% 
    \var{{\devanagarifont \numnoemph\vb\textbf{॰कर्षणम्}\lem \mssCaCbCc\msNb\msNc\Ed, ॰कर्षण \msNa}}% 

%Verse 4:20

{\devanagarifont वार्यमाणापि दुर्बुद्धिरदत्तादानमुच्यते {॥४:२०॥} \veg\dontdisplaylinenum }%
     \var{{\devanagarifont \numnoemph\vc\textbf{वार्यमाणापि}\lem \eme, वार्यमाणो ऽपि \msCa\msCc\msNa\msNb\msNc\Ed, वार्यमानो वि॰ \msCb}}% 

{\devanagarifont उत्कोचं शृणु विप्रेन्द्र धर्मसंकरकारकम् \thinspace{\dandab} \dontdisplaylinenum }%
     \var{{\devanagarifont \numemph\va\textbf{उत्कोचं}\lem \msCb\msCc\msNa\msNb\msNc\Ed, उत्कोच \msCa\oo 
\textbf{विप्रेन्द्र}\lem \mssCaCbCc\msNa\msNc\Ed, विद्रेन्द्र \msNb}}% 
    \var{{\devanagarifont \numnoemph\vb\textbf{॰संकर॰}\lem \msCc\msNa, ॰शङ्कर॰ \msCa\msCb\msNb, ॰सकर॰ \msNc, ॰संहार॰ \Ed\oo 
\textbf{॰कारकम्}\lem \mssCaCbCc\msNb\msNc\Ed, ॰कारकः \msNa}}% 

{\devanagarifont मूल्यं कार्यविनाशार्थमुत्कोचः परिगृह्यते  \danda\dontdisplaylinenum }%
     \var{{\devanagarifont \numnoemph\vc\textbf{मूल्यं}\lem \conj, मूल \mssCaCbCc\msNa\msNb\msNc\Ed\oo 
\textbf{॰विनाशार्थ॰}\lem \mssCaCbCc\msNapcorr\msNb\msNc\Ed, ॰विनार्थ॰ \msNaacorr}}% 
    \var{{\devanagarifont \numnoemph\vd\textbf{॰त्कोचः}\lem \mssCaCbCc\msNa\msNc, ॰त्कोचं \msNb, ॰त्कोच \Ed}}% 

%Verse 4:21

{\devanagarifont तेन चासौ विजानीयाद्द्रव्यलोभबलात्कृतम् {॥४:२१॥} \veg\dontdisplaylinenum }%
     \var{{\devanagarifont \numnoemph\vef\textbf{विजानीयाद्द्र॰}\lem \msCa\msCb\msNa\msNb\msNc\Ed, विजानीया द्र॰ \msCc}}% 

{\devanagarifont प्रस्थव्याज-उपायेन कुटुम्बं त्रातुमिच्छति \thinspace{\dandab} \dontdisplaylinenum }%
 
%Verse 4:22

{\devanagarifont तं च स्तेनं विजानीयात्परद्रव्यापहारकम् {॥४:२२॥} \veg\dontdisplaylinenum }%
     \var{{\devanagarifont \numemph\vc\textbf{तं च स्तेनं}\lem \msCa, तञ्च स्तेन \msCb, 
सो ऽपि तेन \msCc\Ed, तं च स्तेयं \msNa, तञ्च तेय \msNb, तञ्च तेन \msNc}}% 
    \var{{\devanagarifont \numnoemph\vd\textbf{॰हारकम्}\lem \msCa\msCb\msNapcorr\msNc\Ed, ॰हारकः \msCc, ॰हारका \msNaacorr ॰हारकाः \msNb}}% 

{\devanagarifont तुलाव्याज-उपायेन परस्वार्थं हरेद्यदि \thinspace{\dandab} \dontdisplaylinenum }%
     \var{{\devanagarifont \numemph\va\textbf{परस्वार्थं}\lem \msCa\msCc\msNa\msNc, परस्वार्थ \msCb\msNb, परस्यार्थं \Ed\oo 
\textbf{हरेद्यदि}\lem \msCa\msCc\msNa\msNb\msNc\Ed, हरेद्यति \msCb}}% 

%Verse 4:23

{\devanagarifont चौरलक्षणकाश्चान्ये कूटकापटिका नराः {॥४:२३॥} \veg\dontdisplaylinenum }%
     \var{{\devanagarifont \numnoemph\vd\textbf{कूटकापटिका}\lem \msNb, \uncl{कु}टका यटिका \msCa, कूटकायटिका \msCb\msCc\msNaacorr\msNc, 
कूटकार्यटिका \msNapcorr\Ed}}% 
    \paral{{\devanagarifontsmall \vcd {\englishfont \compare\ \UMS\ 8.3cd:} कूटकापटिकाश्चैव सत्यार्जवविवर्जिताः }}

{\devanagarifont दुर्बलार्जवबालेषु च्छद्मना वा बलेन वा \thinspace{\dandab} \dontdisplaylinenum }%
     \var{{\devanagarifont \numemph\va\textbf{॰र्जव॰}\lem \mssCaCbCc\msNa\msNc\Ed, ॰जव॰ \msNb}}% 
    \var{{\devanagarifont \numnoemph\vb\textbf{च्छद्मना}\lem \Ed, च्छन्मना \mssCaCbCc\msNa\msNb, च्छत्माना \msNc}}% 

%Verse 4:24

{\devanagarifont अपहृत्य धनं मूढः स चौरश्चोर उच्यते {॥४:२४॥} \veg\dontdisplaylinenum }%
     \var{{\devanagarifont \numnoemph\vcd\textbf{मूढः स}\lem \mssCaCbCc\msNa\msNc\Ed, मूढास्स \msNb}}% 
    \var{{\devanagarifont \numnoemph\vd\textbf{चौरश्चोर}\lem \msNc, चोरश्चोर \msCa\msCc\msNb\Ed, चौर चोर \msCb, चौरश्चौर \msNa}}% 

{\devanagarifont नास्ति स्तेयसमं पापं नास्त्यधर्मश्च तत्समः \thinspace{\dandab} \dontdisplaylinenum }%
     \var{{\devanagarifont \numemph\vab\textbf{(नास्ति{\englishfont ...} तत्समः)}\lem \mssCaCbCc\msNa\msNb\msNc, \om\ \Ed}}% 
    \var{{\devanagarifont \numnoemph\va\textbf{स्तेय॰}\lem \msNa\msNc, तेन \msCa, स्तेन॰ \msCb\msCc\msNb, \om\ \Ed}}% 
    \var{{\devanagarifont \numnoemph\vb\textbf{॰समः}\lem \msCa\msCb\msNa\msNb\msNc, ॰समं \msCc, \om\ \Ed}}% 

%Verse 4:25

{\devanagarifont नास्ति स्तेनसमाकीर्तिर्नास्ति स्तेनसमो ऽनयः {॥४:२५॥} \veg\dontdisplaylinenum }%
     \var{{\devanagarifont \numnoemph\vcd\textbf{(नास्ति{\englishfont ...} ऽनयः)}\lem \mssCaCbCc\msNa\msNb\msNc, \om\ \Ed}}% 
    \var{{\devanagarifont \numnoemph\vc\textbf{स्तेन॰}\lem \msCa\msCb\msNa\msNb, तेन \msCc, स्तेय॰ \msNc, \om\ \Ed\oo 
\textbf{॰समा॰}\lem \msCb\msCc\msNb, ॰समो \msCa\msNa\msNc, \om\ \Ed}}% 
    \var{{\devanagarifont \numnoemph\vd\textbf{स्तेन॰}\lem \mssCaCbCc\msNb\Ed, स्तेय॰ \msNa\msNc}}% 

{\devanagarifont नास्ति स्तेयसमाविद्या नास्ति स्तेनसमः खलः \thinspace{\dandab} \dontdisplaylinenum }%
     \var{{\devanagarifont \numemph\va\textbf{स्तेय॰}\lem \msNa\msNc\Ed, स्तेन॰ \mssCaCbCc\msNb\oo 
\textbf{॰समा}\lem \msCc\msNb, ॰समो \msCa\msCb\msNa\msNc\Ed}}% 
    \var{{\devanagarifont \numnoemph\vb\textbf{स्तेन॰}\lem \mssCaCbCc\msNb, स्तेय॰ \msNa\msNc, तेन \Ed}}% 

%Verse 4:26

{\devanagarifont नास्ति स्तेनसम अज्ञो नास्ति स्तेनसमो ऽलसः {॥४:२६॥} \veg\dontdisplaylinenum }%
     \var{{\devanagarifont \numnoemph\vc\textbf{स्तेन॰}\lem \msCa\msCb\msNb\msNc, स्तेय॰ \msCc\msNa\Ed\oo 
\textbf{॰सम}\lem \mssCaCbCc\msNa\msNc\Ed\ \unmetr, ॰समं \msNb\oo 
\textbf{अज्ञो}\lem \msCb, अज्ञ\lk\ \msCa, अज्ञ \msCc\msNa\msNb\msNc, अज्ञः \Ed}}% 
    \var{{\devanagarifont \numnoemph\vd\textbf{स्तेन॰}\lem \msCa\msCb\msNb, स्तेय॰ \msCc\msNa\msNc, तेन \Ed}}% 

{\devanagarifont नास्ति स्तेनसमो द्वेष्यो नास्ति स्तेनसमो ऽप्रियः \thinspace{\dandab} \dontdisplaylinenum }%
     \var{{\devanagarifont \numemph\va\textbf{स्तेन॰}\lem \msCa\msCb\msNb, स्तेय॰ \msCc\msNa\msNc, तेन \Ed}}% 
    \var{{\devanagarifont \numnoemph\vb\textbf{स्तेन॰}\lem \msNb, स्तेय॰ \mssCaCbCc\msNa\msNc\Ed}}% 

%Verse 4:27

{\devanagarifont नास्ति स्तेयसमं दुःखं नास्ति स्तेयसमो ऽयशः {॥४:२७॥} \veg\dontdisplaylinenum }%
     \var{{\devanagarifont \numnoemph\vc\textbf{स्तेय॰}\lem \msCc, स्तेन॰ \msCa\msCb\msNa\msNb, स्तेन्य॰ \msNc, तेन \Ed}}% 
    \var{{\devanagarifont \numnoemph\vd\textbf{स्तेय॰}\lem \msCc\msNc, स्तेन॰ \msCa\msCb\msNa\msNb, तेन \Ed}}% 

\ujvers\nemsloka {
{\devanagarifont प्रच्छन्नो ह्रियते ऽर्थमन्यपुरुषः प्रत्यक्षमन्यो हरेत् }%
  \dontdisplaylinenum}    \var{{\devanagarifont \numemph\va\textbf{प्रच्छन्नो}\lem \msCa\msCc\msNa\msNb\msNc\Ed, प्रस्थन्नो \msCb\oo 
\textbf{ऽर्थमन्यपुरुषः}\lem \msCb\msNc, 
वित्तम् \msCa\msNaacorr\msNb, 
चित्त \msCc, च वित्तमथवा \msNapcorr\Ed\oo 
\textbf{प्रत्यक्षमन्यो}\lem \msCa\msCc\msNa\msNb\msNc, प्रत्यक्षमनो \msCb, प्रत्यक्ष्यमन्ये \Ed}}% 


\nemslokab

{\devanagarifont निक्षेपाद्धनहारिणो ऽन्यमधमो व्याजेन चान्यो हरेत्  \danda\dontdisplaylinenum }%
     \var{{\devanagarifont \numnoemph\vb\textbf{निक्षेपाद्धन॰}\lem \msCa\msCb\msNa, निक्षेपा धन॰ \msCc\msNb\msNc, निक्षेपात्रय॰ \Ed\oo 
\textbf{॰हारिणो}\lem \msCa\msCc\msNa\msNc\Ed, ॰हारिण्यो \msCb, ॰हारिणा \msNb\oo 
\textbf{ऽन्यमधमो}\lem \msCa\msCb\msNa\msNb\msNc, ऽन्यमधनो \msCc, ऽन्यविधयो \Ed\oo 
\textbf{चान्यो}\lem \mssCaCbCc\msNa\msNb\msNc, चान्या \Ed\oo 
\textbf{हरेत्}\lem \mssCaCbCc\msNb\msNc\Ed, हरे \msNa}}% 

\nemslokac

{\devanagarifont अन्ये लेख्यविकल्पनाहृतधना †अन्यो हृताद्वै हृता† }%
  \dontdisplaylinenum    \var{{\devanagarifont \numnoemph\vc\textbf{अन्ये लेख्य॰}\lem \corr, अन्या लेख॰ \msCb\msCc, 
अन्यो ले\uncl{ख्य}॰ \msCa, अन्यो लेख्य॰ \msNa\msNb\msNc, अन्योल्लेख्य \Ed\oo 
\textbf{॰धना अन्यो}\lem \msCa\msCc\msNa\msNb\msNc\Ed, ॰धन्यो \msCb\oo 
\textbf{हृताद्वै}\lem \mssCaCbCc\msNc\Ed, हृतद्वै \msNa, हृताद्वे \msNb}}% 

%Verse 4:28


\nemslokad

{\devanagarifont अन्यः क्रीतधनो ऽपरो धयहृत एते जघन्याः स्मृताः {॥४:२८॥} \veg\dontdisplaylinenum }%
     \var{{\devanagarifont \numnoemph\vd\textbf{अन्यः क्रीतधनो}\lem \mssCaCbCc\msNa\msNb, अन्य क्रीतधनो \msNc, अनाश्रीतधनं \Ed\oo 
\textbf{ऽपरो धयहृत}\lem \msCa\msCc\msNb, परो धयह्यत \msCb, परो धन\uncl{हृत} \msNa, 
परोधप्रहृत \msNc, मदा ह्यपहृतं \Ed\oo 
\textbf{जघन्याः}\lem \mssCaCbCc\msNa\msNb\msNc, जघन्यः \Ed}}% 

\ujvers\nemsloka {
{\devanagarifont स्तेनतुल्य न मूढमस्ति पुरुषो धर्मार्थहीनो ऽधमः }%
  \dontdisplaylinenum}    \var{{\devanagarifont \numemph\va\textbf{स्तेनतुल्य}\lem \msCa\msCb\msNc\ \unmetr, स्तेयस्तुल्य \msCc, 
स्तेयतुल्य \msNa\ \unmetr, तेन तुल्य \msNb\ \unmetr, स्तेनस्तुल्य \Ed}}% 


\nemslokab

{\devanagarifont यावज्जीवति शङ्कया नरपतेः संत्रस्यमानो रटन्  \danda\dontdisplaylinenum }%
     \var{{\devanagarifont \numnoemph\vb\textbf{यावज्जीवति}\lem \mssCaCbCc\msNa\msNb\msNc, यावत्तज्जीवति \Ed\oo 
\textbf{॰पतेः}\lem \msCb\msNb\msNc, ॰पतिः \msCa\msCc\msNa\Ed\oo 
\textbf{संत्रस्यमानो रटन्}\lem \mssCaCbCc\msNa\msNb\msNc, संत्रास्यमानो शठः \Ed}}% 
    \lacuna{\devanagarifontsmall \vo {\englishfont The lower folio side in exposure 49 in \msNb\ is rather blurred and seems to be partly erased,
                        therefore all the readings in this MS for verses 4.29--46 are rather uncertain,
                        even if not indicated explicitly.} }%
  
\nemslokac

{\devanagarifont प्राप्तःशासन तीव्रसह्यविषमं प्राप्नोति कर्मेरितः }%
  \dontdisplaylinenum    \var{{\devanagarifont \numnoemph\vc\textbf{प्राप्तः॰}\lem \mssCaCbCc\msNb\msNc\Ed, प्राप्त॰ \msNa\oo 
\textbf{॰सह्य॰}\lem \mssCaCbCc\msNa\msNc, \lac\  \msNb, ॰सद्य॰ \Ed\oo 
\textbf{॰विषमं}\lem \eme, ॰विषमः \mssCaCbCc\msNa\msNc\Ed, \lac\  \msNb\oo 
\textbf{कर्मेरितः}\lem \msCb\msCc\msNa\msNc\Ed, कर्मे\uncl{रित} \msCa, \lac \uncl{रितः} \msNb}}% 

%Verse 4:29


\nemslokad

{\devanagarifont कालेन म्रियते स याति निरयमाक्रन्दमानो भृशम् {॥४:२९॥} \veg\dontdisplaylinenum }%
     \var{{\devanagarifont \numnoemph\vd\textbf{निरयमाक्रन्दमानो}\lem \mssCaCbCc\msNa, \uncl{निर}यमाक्रन्दमा\uncl{नो} \msNb, 
निरयं स क्रन्दमानो \msNc, नियममाक्रन्द्रमानो \Ed}}% 

\ujvers\nemsloka {
{\devanagarifont नीत्वा दुर्गतिकोटिकल्प निरयात्तिर्यत्वमायान्ति ते }%
  \dontdisplaylinenum}    \var{{\devanagarifont \numemph\va\textbf{निरयात्तिर्यत्व॰}\lem \msCb\msNa, निरयान्तिर्यत्व॰ \msCa, निरया तिर्यत्व॰ \msCc, 
नि\uncl{रयात्तिर्यत्व}॰ \msNb, निरयान्तिर्यक्ष॰ \msNc, निरयान्तिर्यक्त्व॰ \Ed}}% 


\nemslokab

{\devanagarifont तिर्यत्वे च तथैवमेकशतिकं प्रभ्रम्य वर्षार्बुदम्  \danda\dontdisplaylinenum }%
     \var{{\devanagarifont \numnoemph\vb\textbf{तिर्यत्वे}\lem \mssCaCbCc\msNa\msNc, \uncl{तिर्यत्वे} \msNb, तिर्यक्त्वं \Ed\oo 
\textbf{तथैवमेकशतिकं}\lem \msCb, तथैकमेकशतिकं \msCa\msNa\msNc, 
तथैकमेकशतिक \msCc, \uncl{तथै}कमेकशतिकं \msNb, तथैकमेकसकिकं \Ed\oo 
\textbf{॰भ्रम्य॰}\lem \mssCaCbCc\msNc\Ed, ॰भ्राम्य \msNa, \lac {ā}म्य \msNb\oo 
\textbf{वर्षार्बुदम्}\lem \msNcpcorr, वर्षाम्बुदम् \msCa\msCb\msNa\msNb\msNcacorr, वर्षाम्बुदः \msCc\Ed}}% 

\nemslokac

{\devanagarifont मानुष्यं तदवाप्नुवन्ति विपुले दारिद्र्यरोगाकुलं }%
  \dontdisplaylinenum    \var{{\devanagarifont \numnoemph\vc\textbf{मानुष्यं}\lem \msCa\msCc\msNa\msNc\Ed, मानुष्य \msCb\ \unmetr, \uncl{मानुष्य} \msNb\ \toplost\oo 
\textbf{विपुले}\lem \mssCaCbCc\msNa\msNc, विपु\uncl{ल} \msNb\ \toplost, विपुलं \Ed\oo 
\textbf{दारिद्र्य॰}\lem \mssCaCbCc\msNa\msNc, \lk रि\lk\ \msNb, दारिध्र॰ \Ed}}% 

%Verse 4:30


\nemslokad

{\devanagarifont तस्माद्दुर्गतिहेतु कर्म सकलं त्यक्त्वा शिवं चाश्रयेत् {॥४:३०॥} \veg\dontdisplaylinenum }%
     \var{{\devanagarifont \numnoemph\vd\textbf{तस्माद्दु॰}\lem \msCa\msCb\msNa\msNc\Ed, तस्मा दु॰ \msCc, \uncl{तस्मा दु}॰ \msNb\oo 
\textbf{चाश्रयेत्}\lem \mssCaCbCc\msNb\msNc\Ed, चाश्रत् \msNa}}% 


\alalfejezet{यमेष्वानृशंस्यम् (४)}
\vers


{\devanagarifont अष्टमूर्तिशिवद्वेष्टा पितुर्मातुश्च यो द्विषेत् \thinspace{\dandab} \dontdisplaylinenum }%
     \var{{\devanagarifont \numemph\va\textbf{॰शिव॰}\lem \mssCaCbCc\msNa\msNb\Ed, ॰शिवं \msNc}}% 

%Verse 4:31

{\devanagarifont गवां वा अतिथेर्द्वेष्टा नृशंसाः पञ्च एव ते {॥४:३१॥} \veg\dontdisplaylinenum }%
     \var{{\devanagarifont \numnoemph\vc\textbf{गवां वा}\lem \msCa\msCc\msNa\msNc\Ed, अवाम्वा \msCb, \lk\lk \uncl{म्वा} \msNb\oo 
\textbf{अतिथेर्द्वे॰}\lem \msCa\msCb\msNb\msNc\Ed, अतिथिद्वे॰ \msCc, अतिथे द्वे॰ \msNa}}% 
    \var{{\devanagarifont \numnoemph\vd\textbf{नृशंसाः}\lem \msCa\msCc\msNa\msNb, नृशंसा \msCb\msNc\Ed}}% 

{\devanagarifont अष्टमूर्तिः शिवः साक्षात्पञ्चव्योमसमन्वितः \thinspace{\dandab} \dontdisplaylinenum }%
     \var{{\devanagarifont \numemph\va\textbf{॰मूर्तिः}\lem \mssCaCbCc\msNa\msNb\msNc, ॰मूर्ति॰ \Ed}}% 
    \var{{\devanagarifont \numnoemph\vb\textbf{॰न्वितः}\lem \msCa\msCb\msNa\msNc\Ed, ॰न्विताः \msCc\msNb}}% 

%Verse 4:32

{\devanagarifont सूर्यः सोमश्च दीक्षश्च दूषकः स नृशंसकः {॥४:३२॥} \veg\dontdisplaylinenum }%
     \var{{\devanagarifont \numnoemph\vc\textbf{सूर्यः}\lem \mssCaCbCc\msNa, \uncl{सूर्य}॰ \msNb\msNc, सूर्य॰ \Ed\oo 
\textbf{दीक्ष॰}\lem \mssCaCbCc\msNa\msNc, \uncl{दी}\lk\ \msNb, दीक्षु॰ \Ed}}% 

{\devanagarifont पिताकाशसमो ज्ञेयो जन्मोत्पत्तिकरः पिता \thinspace{\dandab} \dontdisplaylinenum }%
     \var{{\devanagarifont \numemph\vb\textbf{॰करः पिता}\lem \msCa\msCb\msNa\msNc\Ed, ॰करपिताः \msCc, ॰\uncl{करः पिता} \msNb}}% 

%Verse 4:33

{\devanagarifont पितृदैवत†मादिश्चमानृशंस तमन्वितः† {॥४:३३॥} \veg\dontdisplaylinenum }%
     \var{{\devanagarifont \numnoemph\vc\textbf{॰दैवत॰}\lem \msCa\msCc\msNa\msNc\Ed, ॰देवत॰ \msCb, \lk वत॰ \msNb}}% 
    \var{{\devanagarifont \numnoemph\vcd\textbf{॰दिश्चमानृशंस तमन्वितः}\lem \msCa\msCb, 
॰दित्यमनृशंस तमन्वितः \msCc\msNb, 
॰दिश्च अनृशंस तमान्वितः \msNa, 
॰दिश्चमनृशंस तमान्वितः \msNc, 
॰दित्यम्मानृशंस ततो ऽन्वितः \Ed}}% 

{\devanagarifont पृथ्व्या गुरुतरी माता को न वन्देत मातरम् \thinspace{\dandab} \dontdisplaylinenum }%
     \var{{\devanagarifont \numemph\va\textbf{पृथ्व्या}\lem \msCa\msCb\msNc, \uncl{पृथ्व्या} \msCc\msNa, पृथ्वी \msNb, 
पृथ्व्यां \Ed}}% 
    \var{{\devanagarifont \numnoemph\vb\textbf{वन्देत}\lem \msCa\msNa\msNb\msNc\Ed, वन्देन वन्देत \msCb, वन्द्येत \msCc}}% 

%Verse 4:34

{\devanagarifont यज्ञदानतपोवेदास्तेन सर्वं कृतं भवेत् {॥४:३४॥} \veg\dontdisplaylinenum }%
     \var{{\devanagarifont \numnoemph\vd\textbf{सर्वं}\lem \eme, सर्व \mssCaCbCc\msNa\msNb\msNc\Ed}}% 

{\devanagarifont गावः पवित्रं मङ्गल्यं देवतानां च देवताः \thinspace{\dandab} \dontdisplaylinenum }%
     \var{{\devanagarifont \numemph\va\textbf{पवित्रं}\lem \mssCaCbCc\msNa\msNc\Ed, \uncl{पवित्र} \msNb\oo 
\textbf{मङ्गल्यं}\lem \msCa\msCb\msNa, माङ्गल्यं \msCc\msNc\Ed, \uncl{मङ्गल्यं} \msNb\oo 
\textbf{देवताः}\lem \mssCaCbCc\msNc, दैवताः \msNa, \uncl{देवताः} \msNb, देवता \Ed}}% 
    \paral{{\devanagarifontsmall \va {\englishfont \similar\ \VISNUS\ 23.57c:} गावः पवित्रमङ्गल्यं (गोषु लोकाः प्रतिष्ठिता)\oo
                 {\englishfont \compare\ also \MBH\ Indices 13.15.33:} गावः पवित्रं परमं गोषु लोकाः प्रतिष्ठिताः 
                 {\englishfont and \AGNIP\ 291.1cd:} गावः पवित्रा माङ्गल्या गोषु लोकाः प्रतिष्ठिताः }}

%Verse 4:35

{\devanagarifont सर्वदेवमया गावस्तस्मादेव न हिंसयेत् {॥४:३५॥} \veg\dontdisplaylinenum }%
     \var{{\devanagarifont \numnoemph\vd\textbf{॰स्मादेव}\lem \msCa\msCc\msNa\msNb\msNc, ॰स्मादुव \msCb, ॰स्माद्गावं \Ed}}% 
    \paral{{\devanagarifontsmall \vc {\englishfont = \VDHU\ 3.291.25c} }}

{\devanagarifont जातमात्रस्य लोकस्य गावस्त्राता न संशयः \thinspace{\dandab} \dontdisplaylinenum }%
     \var{{\devanagarifont \numemph\va\textbf{जातमात्रस्य लोकस्य}\lem \msCa\msCc\msNa\msNc\Ed, सतसातस्य \msCbacorr, सतसातस्य नोकस्य \msCbpcorr, 
जातमात्र\uncl{स्य लोकस्य} \msNb}}% 

%Verse 4:36

{\devanagarifont घृतं क्षीरं दधि मूत्रं शकृत्कर्षणमेव च {॥४:३६॥} \veg\dontdisplaylinenum }%
     \var{{\devanagarifont \numnoemph\vd\textbf{शकृत्क॰}\lem \msCa\msCc\msNa\msNc\Ed, क्षत्क॰ \msCb, \uncl{शकृत्क}॰ \msNb}}% 
    \paral{{\devanagarifontsmall \vo {\englishfont \compare\ \SDHU\ 12.92ff} }}

\ujvers\nemsloka {
{\devanagarifont पञ्चामृतं पञ्चपवित्रपूतं }%
  \dontdisplaylinenum}    \var{{\devanagarifont \numemph\va\textbf{॰पवित्रपूतम्}\lem \msCc\msNa\Ed, ॰पवित्रपूतन \msCa\ \unmetr, 
॰पवित्रं \msCb\ \unmetr, ॰पवित्रपूत \msNb, 
॰पवित्रपूतंनं \msNc\ \unmetr}}% 


\nemslokab

{\devanagarifont ये पञ्चगव्यं पुरुषाः पिबन्ति  \danda\dontdisplaylinenum }%
     \var{{\devanagarifont \numnoemph\vb\textbf{॰गव्यं}\lem \msCa\msCb\msNa\msNc\Ed, ॰गव्या \msCc, ॰\uncl{गव्यां} \msNb\oo 
\textbf{पुरुषाः}\lem \msCa\msCb\msNa\msNb\msNc, पुरुषा \msCc, पुरुषः \Ed\oo 
\textbf{पिबन्ति}\lem \msCa\msCb\msNa\msNb\msNc\Ed, विवन्ति \msCc}}% 

\nemslokac

{\devanagarifont ते वाजिमेधस्य फलं लभन्ति }%
  \dontdisplaylinenum    \var{{\devanagarifont \numnoemph\vc\textbf{लभन्ति}\lem \msCa\msCb\msNa\msNb\msNc\Ed, भवन्ति \msCc}}% 

%Verse 4:37


\nemslokad

{\devanagarifont तदक्षयं स्वर्गमवाप्नुवन्ति {॥४:३७॥} \veg\dontdisplaylinenum }%
     \var{{\devanagarifont \numnoemph\vd\textbf{स्वर्ग॰}\lem \msCa\msCc\msNa\msNb\msNc\Ed, स्व॰ \msCb}}% 

\ujvers\nemsloka {
{\devanagarifont गोभिर्न तुल्यं धनमस्ति किंचिद् }%
  \dontdisplaylinenum}    \var{{\devanagarifont \numemph\va\textbf{गोभिर्न तु॰}\lem \msNc, न गोभिस्तु॰ \mssCaCbCc\msNa\msNb\ \unmetr, न गावतु॰ \Ed}}% 
    \paral{{\devanagarifontsmall \va {\englishfont = \SDHU\ 12.102d, 103d, 104d; } 
                    {\englishfont \compare\ \MBH\ 13.51.26cd:} गोभिस्तुल्यं न पश्यामि धनं किंचिदिहाच्युत }}


\nemslokab

{\devanagarifont दुह्यन्ति वाह्यन्ति बहिश्चरन्ति  \danda\dontdisplaylinenum }%
 
\nemslokac

{\devanagarifont तृणानि भुक्त्वा अमृतं स्रवन्ति }%
  \dontdisplaylinenum
%Verse 4:38


\nemslokad

{\devanagarifont विप्रेषु दत्ताः कुलमुद्धरन्ति {॥४:३८॥} \veg\dontdisplaylinenum }%
     \var{{\devanagarifont \numnoemph\vd\textbf{दत्ताः}\lem \msCa\msCb\msNa\msNb\msNc, \uncl{दत्ता} \msCc, दत्ता \Ed}}% 
    \paral{{\devanagarifontsmall \vo {\englishfont \compare\ \SDHU\ 12.92:}
                         तृणानि खादन्ति वसन्त्यरण्ये पिबन्ति तोयान्यपरिग्रहाणि\thinspace{\devanagarifontsmall ।}
                         दुह्यन्ति बाह्यन्ति पुनन्ति पापं गवां रसैर्जीवति जीवलोकः\thinspace{\devanagarifontsmall ॥} }}

\ujvers\nemsloka {
{\devanagarifont गवाह्निकं यश्च करोति नित्यं }%
  \dontdisplaylinenum}    \var{{\devanagarifont \numemph\va\textbf{गवाह्निकं}\lem \msCb\msCc\msNa\msNb\msNc\Ed, गवांह्निकं \msCa\oo 
\textbf{यश्च करोति}\lem \mssCaCbCc\msNa\msNb\msNc, यः प्रकरोति \Ed}}% 


\nemslokab

{\devanagarifont शुश्रूषणं यः कुरुते गवां तु  \danda\dontdisplaylinenum }%
     \var{{\devanagarifont \numnoemph\vb\textbf{गवां तु}\lem \msCb\msNc, गवान्तु \msCa\msCc\msNa\msNb, गवानाम् \Ed}}% 

\nemslokac

{\devanagarifont अशेषयज्ञतपदानपुण्यं }%
  \dontdisplaylinenum    \var{{\devanagarifont \numnoemph\vc\textbf{॰तप॰}\lem \mssCaCbCc\msNa\msNc, ॰\uncl{तप}॰ \msNb, ॰जप॰ \Ed}}% 

%Verse 4:39


\nemslokad

{\devanagarifont लभत्यसौ तामनृशंसकर्ता {॥४:३९॥} \veg\dontdisplaylinenum }%
     \var{{\devanagarifont \numnoemph\vd\textbf{लभत्यसौ तामनृशंसकर्ता}\lem \eme, 
लभत्यसौ तमनृशंसकर्ता \msCb\msNa\msNb\msNc, 
लभत्यसौ भमनृशंसकर्त्ता \msCa, 
लभत्यसौ तमनृतं स कर्त्ता \msCc, 
भवत्यसौ धर्ममशेषकर्ता \Ed}}% 

\vers


{\devanagarifont अतिथिं यो ऽनुगच्छेत अतिथिं यो ऽनुमन्यते \thinspace{\dandab} \dontdisplaylinenum }%
 
%Verse 4:40

{\devanagarifont अतिथिं यो ऽनुपूज्येत अतिथिं यः प्रशंसते {॥४:४०॥} \veg\dontdisplaylinenum }%
     \var{{\devanagarifont \numemph\vd\textbf{प्रशंसते}\lem \msCa\msCb\msNa\msNb\msNc\Ed, प्रशंस्यते \msCc}}% 

{\devanagarifont अतिथिं यो न पीड्येत अतिथिं यो न दुष्यति \thinspace{\dandab} \dontdisplaylinenum }%
     \var{{\devanagarifont \numemph\va\textbf{न पीड्येत}\lem \msCa\msCb\msNa\Ed, न गच्छेत ({\englishfont eyeskip to 4.40c}) \msCc, 
\uncl{न पी}\lk\lk\ \msNb, निपीड्येत \msNc}}% 
    \var{{\devanagarifont \numnoemph\vb\textbf{अतिथिं}\lem \msCa\msCb\msNa\msNc\Ed, अतिं \msCc, \lk\lk \lk\ \msNb\oo 
\textbf{न दुष्यति}\lem \msCa\msCc\msNa\msNc\Ed, नुदुष्यति \msCb, \lk\ दुष्यति \msNb}}% 

{\devanagarifont अतिथिप्रियकर्ता यः अतिथेः परिचारकः  \danda\dontdisplaylinenum }%
     \var{{\devanagarifont \numnoemph\vc\textbf{अतिथि॰}\lem \msCa\msNa, अतिथिं \msCb\msCc\msNc\Ed, अति\uncl{थिं} \msNb\oo 
\textbf{॰प्रिय॰}\lem \msCa\msCb\msNa\msNb\msNc\Ed, प्रियः \msCc\oo 
\textbf{यः}\lem \msCb\msCc\msNb\msNc\Ed, यर् \msCa, य \msNa}}% 

%Verse 4:41

{\devanagarifont अतिथेः कृतसंतोषस्तस्य पुण्यमनन्तकम् {॥४:४१॥} \veg\dontdisplaylinenum }%
     \var{{\devanagarifont \numnoemph\ve\textbf{अतिथेः}\lem \msCb\msCc\msNc, अतिथि॰ \msCa\msNa\msNb, अतिथिं \Ed}}% 
    \var{{\devanagarifont \numnoemph\vef\textbf{॰संतोषस्तस्य}\lem \msCa\msCc\msNa\msNb\msNc\Ed, ॰संता यस्य \msCb}}% 
    \var{{\devanagarifont \numnoemph\vf\textbf{पुण्य॰}\lem \mssCaCbCc\msNa\msNb\Ed, पून॰ \msNc}}% 

{\devanagarifont आसनेनार्घपात्रेण पादशौचजलेन च \thinspace{\dandab} \dontdisplaylinenum }%
     \var{{\devanagarifont \numemph\va\textbf{॰आर्घ॰}\lem \mssCaCbCc\msNa\msNb\msNc, ॰आर्ध्य॰ \Ed\oo 
\textbf{॰पात्रेण}\lem \conj, ॰पाद्येन \mssCaCbCc\msNa\msNb\msNc\Ed}}% 

%Verse 4:42

{\devanagarifont अन्नवस्त्रप्रदानैर्वा सर्वं वापि निवेदयेत् {॥४:४२॥} \veg\dontdisplaylinenum }%
     \var{{\devanagarifont \numnoemph\vc\textbf{अन्नव॰}\lem \msCa\msCb\msNa\msNc\Ed, अन्नम्व॰ \msCc, \uncl{अन्न}व॰ \msNb}}% 
    \var{{\devanagarifont \numnoemph\vd\textbf{निवेदयेत्}\lem \mssCaCbCc\msNa\msNb\msNc, प्रदापयेत् \Ed}}% 

{\devanagarifont पुत्रदारात्मना वापि यो ऽतिथिमनुपूजयेत् \thinspace{\dandab} \dontdisplaylinenum }%
     \var{{\devanagarifont \numemph\va\textbf{॰दारात्मना}\lem \eme, ॰दारात्मनो \msCb\msCc\msNa\msNb\msNc, 
॰\uncl{दारा}त्मनो \msCa, ॰दारात्मको \Ed}}% 
    \var{{\devanagarifont \numnoemph\vb\textbf{॰पूजयेत्}\lem \msCa\msNa\Ed, ॰पूज्यते \msCb\msCc\msNb, ॰पूजते \msNc}}% 

%Verse 4:43

{\devanagarifont श्रद्धया चाविकल्पेन अक्लीबमानसेन च {॥४:४३॥} \veg\dontdisplaylinenum }%
     \var{{\devanagarifont \numnoemph\vc\textbf{श्रद्धया}\lem \msCa\msCb\msNa\msNb\msNc\Ed, श्रद्धाया \msCc\oo 
\textbf{चाविकल्पेन}\lem \msCb\msCc\msNa\msNb\msNc\Ed, चापि कल्पेन \msCa}}% 

{\devanagarifont न पृच्छेद्गोत्रचरणं स्वाध्यायं देशजन्मनी \thinspace{\dandab} \dontdisplaylinenum }%
     \var{{\devanagarifont \numemph\va\textbf{॰चरणं}\lem \mssCaCbCc\msNa\msNb\msNc, ॰प्रवरं \Ed}}% 
    \var{{\devanagarifont \numnoemph\vb\textbf{देशजन्मनी}\lem \msCb\msCc\msNa\msNb\msNc\Ed, देशजन्मना \msCa}}% 
    \paral{{\devanagarifontsmall  {\englishfont \vab = \UUMS\ 10.7ab = \UMS\ 6.11ab \similar\ \MBH\ 13.62.18ab:
                 }न पृच्छेद्गोत्रचरणं स्वाध्यायं देशमेव वा }}

%Verse 4:44

{\devanagarifont चिन्तयेन्मनसा भक्त्या धर्मः स्वयमिहागतः {॥४:४४॥} \veg\dontdisplaylinenum }%
     \var{{\devanagarifont \numnoemph\vc\textbf{चिन्तयेन्म॰}\lem \msCa\msCc\msNa\msNb\Ed, चित्तयेत्म॰ \msCb, चिन्तयेत्म॰ \msNc}}% 
    \var{{\devanagarifont \numnoemph\vd\textbf{॰गतः}\lem \msCa\msCb\msNa\msNc\Ed, ॰गताः \msCc, ग\uncl{तम्} \msNb}}% 
    \paral{{\devanagarifontsmall \vcd {\englishfont \compare\ 12.37cd: }द्विजरूपधरो धर्मः स्वयमेव इहागतः }}

{\devanagarifont अश्वमेधसहस्राणि राजसूयशतानि च \thinspace{\dandab} \dontdisplaylinenum }%
     \var{{\devanagarifont \numemph\vb\textbf{॰सूय॰}\lem \msCa\msNa\msNc\Ed, ॰सूर्य॰ \msCb\msCc, ॰सू\uncl{र्य}॰ \msNb}}% 

%Verse 4:45

{\devanagarifont पुण्डरीकसहस्रं च सर्वतीर्थतपःफलम् {॥४:४५॥} \veg\dontdisplaylinenum }%
     \var{{\devanagarifont \numnoemph\vd\textbf{॰तपः॰}\lem \mssCaCbCc\msNa\msNb\Ed, ॰तप॰ \msNc\ \unmetr}}% 

{\devanagarifont अतिथिर्यस्य तुष्येत नृशंसमतमुत्सृजेत् \thinspace{\dandab} \dontdisplaylinenum }%
     \var{{\devanagarifont \numemph\vb\textbf{नृशंसमतमुत्सृजेत्}\lem \msCa\msNa\msNc, नृशंसमत उत्सृजेत् \msCb, 
नृशंसकमममुत्सृजेत् \msCc, नृससमतमुत्सृजेत् \msNb, न संशय समश्नुते \Ed}}% 

%Verse 4:46

{\devanagarifont स तस्य सकलं पुण्यं प्राप्नुयान्नात्र संशयः {॥४:४६॥} \veg\dontdisplaylinenum }%
 
{\devanagarifont †न गतिमतिथिज्ञस्य† गतिमाप्नोति कर्हिचित् \thinspace{\dandab} \dontdisplaylinenum }%
     \var{{\devanagarifont \numemph\va\textbf{न गतिम॰}\lem \msCa\msCb\msNb\msNc, न तिथिम॰ \msCc\Ed, न गति ना॰ \msNa}}% 
    \var{{\devanagarifont \numnoemph\vb\textbf{कर्हिचित्}\lem \msCa\Ed, कर्हचित् \msCb\msCc\msNa\msNb\msNc}}% 

%Verse 4:47

{\devanagarifont तस्मादतिथिमायान्तमभिगच्छेत्कृताञ्जलिः {॥४:४७॥} \veg\dontdisplaylinenum }%
     \var{{\devanagarifont \numnoemph\vc\textbf{॰यान्त॰}\lem \msCa\msCb\msNa\msNb\msNc\Ed, ॰यान्ति॰ \msCc}}% 
    \paral{{\devanagarifontsmall \vcd {\englishfont = \VAYUP\ 2.17.8 = \BRAHMANDAPUR\ 2.15.8 
                         \similar\ \SDHU\ 4.44ab:}
                         तस्मादतिथिमायान्तमनुगच्छेत्कृताञ्जलिः }}

{\devanagarifont सक्तुप्रस्थेन चैकेन यज्ञ आसीन्महाद्भुतः \thinspace{\dandab} \dontdisplaylinenum }%
     \var{{\devanagarifont \numemph\va\textbf{सक्तु॰}\lem \eme, शन्कु॰ \msCa\msCb, शंक्तु॰ \msCc, शक्तु॰ \msNa\msNc, शक्थु॰ \msNb, शक्ति॰ \Ed\oo 
\textbf{चैकेन}\lem \mssCaCbCc\msNa\msNb\Ed, चेकेन \msNc}}% 
    \var{{\devanagarifont \numnoemph\vb\textbf{आसीन्महाद्भुतः}\lem \corr, आसीन्महद्भुतः \msCa\msCb\msNa\msNb, आसी महद्भुतः \msCc, 
आसीत्महाद्भुतः \msNc, आसीन्महद्भुतम् \Ed}}% 

%Verse 4:48

{\devanagarifont अतिथिप्राप्तदानेन स्वशरीरं दिवं गतम् {॥४:४८॥} \veg\dontdisplaylinenum }%
     \var{{\devanagarifont \numnoemph\vc\textbf{॰दानेन}\lem \msCa\msCb\msNa\msNb\msNc\Ed, ॰प्रादानेन \msCc}}% 
    \var{{\devanagarifont \numnoemph\vd\textbf{स्व॰}\lem \mssCaCbCc\msNa\msNb, \uncl{स}॰ \msNc, स॰ \Ed\oo 
\textbf{॰गतम्}\lem \msCa\msCb\msNa\msNb\msNc\Ed, ॰गतः \msCc}}% 

{\devanagarifont नकुलेन पुराधीतं विस्तरेण द्विजोत्तम \thinspace{\dandab} \dontdisplaylinenum }%
     \var{{\devanagarifont \numemph\vb\textbf{॰त्तम}\lem \msCa\msCb\msNa\msNb\msNc, ॰त्तमम् \msCc, ॰त्तमः \Ed}}% 

%Verse 4:49

{\devanagarifont विदितं च त्वया पूर्वं प्रस्थवार्त्ता च कीर्तिता {॥४:४९॥} \veg\dontdisplaylinenum }%
     \var{{\devanagarifont \numnoemph\vd\textbf{कीर्तिता}\lem \msCa\msCb\msNa\msNb\msNc, कीर्तितम् \msCc, कीर्तिताः \Ed}}% 


\alalfejezet{यमेषु दमः (५)}
{\devanagarifont दम एव मनुष्याणां धर्मसारसमुच्चयः \thinspace{\dandab} \dontdisplaylinenum }%
     \var{{\devanagarifont \numemph\vb\textbf{धर्मसार॰}\lem \eme, धर्मः सार॰ \mssCaCbCc\msNa\msNb\msNc, धर्मभार॰ \Ed}}% 
    \paral{{\devanagarifontsmall \vb {\englishfont \compare\ e.g. \MBH\ Indices 14.4.2477: }श्रोतुमिच्छामि कार्त्स्न्येन धर्मसारसमुच्चयम् }}

%Verse 4:50

{\devanagarifont दमो धर्मो दमः स्वर्गो दमः कीर्तिर्दमः सुखम् {॥४:५०॥} \veg\dontdisplaylinenum }%
     \var{{\devanagarifont \numnoemph\vc\textbf{स्वर्गो}\lem \msCa\msCb\msNa\msNb\msNc\Ed, स्वर्ग \msCc}}% 
    \var{{\devanagarifont \numnoemph\vd\textbf{कीर्तिर्द॰}\lem \msCa\msCb\msNb\Ed, कीर्ति द॰ \msCc\msNa\msNc}}% 

{\devanagarifont दमो यज्ञो दमस्तीर्थं दमः पुण्यं दमस्तपः \thinspace{\dandab} \dontdisplaylinenum }%
     \var{{\devanagarifont \numemph\va\textbf{दमस्ती॰}\lem \msCa\msCc\msNa\msNb\msNc\Ed, दम ती॰ \msCb}}% 

%Verse 4:51

{\devanagarifont दमहीनमधर्मश्च दमः कामकुलप्रदः {॥४:५१॥} \veg\dontdisplaylinenum }%
     \var{{\devanagarifont \numnoemph\vd\textbf{दमः}\lem \msCa\msCb\msNa\msNb\msNc, दम \msCc, दमं \Ed\oo 
\textbf{काम॰}\lem \mssCaCbCc\msNa\msNb\Ed, कामं \msNc}}% 

{\devanagarifont निर्दमः करि मीनश्च पतङ्गभ्रमरमृगाः \thinspace{\dandab} \dontdisplaylinenum }%
     \var{{\devanagarifont \numemph\va\textbf{॰दमः}\lem \msCa\msCb\msNa\msNb\msNc\Ed, ॰दम \msCc}}% 
    \var{{\devanagarifont \numnoemph\vb\textbf{॰भ्रमर॰}\lem \mssCaCbCc\msNa\msNb\Ed\ \unmetr, ॰भ्रम\uncl{रा}॰ \msNc}}% 

%Verse 4:52

{\devanagarifont त्वग्जिह्वा च तथा घ्राणा चक्षुः श्रवणमिन्द्रियाः {॥४:५२॥} \veg\dontdisplaylinenum }%
     \var{{\devanagarifont \numnoemph\vc\textbf{घ्राणा}\lem \msCa\msNa\msNb\msNc\Ed, घ्राणं \msCb, घ्राण \msCc}}% 
    \var{{\devanagarifont \numnoemph\vd\textbf{॰न्द्रियाः}\lem \mssCaCbCc\msNa\msNb\msNc, ॰न्द्रियः \Ed}}% 

{\devanagarifont दुर्जयेन्द्रियमेकैकं सर्वे प्राणहराः स्मृताः \thinspace{\dandab} \dontdisplaylinenum }%
     \var{{\devanagarifont \numemph\vb\textbf{सर्वे}\lem \msCa\msCc\msNa\msNb\msNc\Ed, सर्व॰ \msCb\oo 
\textbf{॰हराः}\lem \mssCaCbCc\msNa\msNb\msNc, ॰हरा \Ed}}% 

%Verse 4:53

{\devanagarifont दमं यो जयते ऽसम्यग्निर्दमो निधनं व्रजेत् {॥४:५३॥} \veg\dontdisplaylinenum }%
     \var{{\devanagarifont \numnoemph\vd\textbf{व्रजेत्}\lem \msCb\msCc\msNa\msNb\msNc\Ed, व्रजे\lac\  \msCa}}% 

{\devanagarifont मृगे श्रोत्रवशान्मृत्युः पतङ्गाश्चक्षुषोर्मृताः \thinspace{\dandab} \dontdisplaylinenum }%
     \var{{\devanagarifont \numemph\va\textbf{मृगे}\lem \mssCaCbCc\msNa\msNc, मृगो \msNb\Ed\oo 
\textbf{श्रोत्र॰}\lem \msCa\msCb\msNa\msNb\Ed, शोत्र॰ \msCc, श्रोत॰ \msNc\oo 
\textbf{॰वशा॰}\lem \msCa\msCc\msNa\msNb\msNc\Ed, ॰वचशा॰ \msCb}}% 
    \var{{\devanagarifont \numnoemph\vb\textbf{पतङ्गाश्च॰}\lem \mssCaCbCc\msNa\msNb\msNc, पतङ्गा च॰ \Ed\oo 
\textbf{॰षोर्मृताः}\lem \msCa\msCb\msNa\msNb\Ed, ॰सो मृताः \msCc, ॰षो मृताः \msNc}}% 

%Verse 4:54

{\devanagarifont घ्राणया भ्रमरो नष्टो नष्टो मीनश्च जिह्वया {॥४:५४॥} \veg\dontdisplaylinenum }%
     \var{{\devanagarifont \numnoemph\vc\textbf{घ्राणया}\lem \msCa\msCc\msNa\msNb\msNc\Ed, घ्रातया \msCb}}% 
    \var{{\devanagarifont \numnoemph\vcd\textbf{नष्टो नष्टो}\lem \msCa\msCc\msNa\msNb\msNc\Ed, नष्टो \msCb}}% 
    \paral{{\devanagarifontsmall \vo {\englishfont \compare\ \BUDDHACARITA\ 11.35:} 
                गीतैर्ह्रियन्ते हि मृगा वधाय रूपार्थमग्नौ शलभाः पतन्ति\thinspace{\devanagarifontsmall ।} 
                मत्स्यो गिरत्यायसमामिषार्थी तस्मादनर्थं विषयाः फलन्ति\thinspace{\devanagarifontsmall ॥} }}

{\devanagarifont स्पर्शेन च करी नष्टो बन्धनावासदुःसहः \thinspace{\dandab} \dontdisplaylinenum }%
     \var{{\devanagarifont \numemph\vb\textbf{॰सदुःसहः}\lem \msCa\msCc\msNa\msNc\Ed, ॰सदुःसह \msCb, ॰सुदुस्सहः \msNb}}% 

%Verse 4:55

{\devanagarifont किं पुनः पञ्चभुक्तानां मृत्युस्तेभ्यः किमद्भुतम् {॥४:५५॥} \veg\dontdisplaylinenum }%
     \var{{\devanagarifont \numnoemph\vc\textbf{पुनः}\lem \msCapcorr\msCb\msCc\msNa\msNb\msNc\Ed, पुन \msCaacorr}}% 
    \var{{\devanagarifont \numnoemph\vd\textbf{तेभ्यः}\lem \mssCaCbCc\msNa\msNb\msNc, तेभ्य \Ed}}% 

{\devanagarifont पुरूरवो ऽतिलोभेन अतिकामेन दण्डकः \thinspace{\dandab} \dontdisplaylinenum }%
     \var{{\devanagarifont \numemph\va\textbf{पुरूरवो}\lem \msCa\msCb\msNa\msNb\msNc, पुरोरवे \msCc, पुरुरवा॰ \Ed\oo 
\textbf{तिलोभेन अतिकामेन}\lem \mssCaCbCc\msNa\msNb\msNc, तिकामेन अतिलोभेन \Ed}}% 
    \var{{\devanagarifont \numnoemph\vb\textbf{दण्डकः}\lem \mssCaCbCc\msNa\msNb\msNc, पुण्डकः \Ed}}% 

%Verse 4:56

{\devanagarifont सागराश्चातिदर्पेण अतिमानेन रावणः {॥४:५६॥} \veg\dontdisplaylinenum }%
     \var{{\devanagarifont \numnoemph\vc\textbf{सागरा॰}\lem \eme, सगर॰ \msCa\msCb\msNa\msNb\msNc\Ed, सागर॰ \msCc}}% 
    \paral{{\devanagarifontsmall \vd {\englishfont \compare\ \MAHASUBHS\ 563cd:}
                         विनष्टो रावणो लौल्यादति सर्वत्र वर्जयेत् }}

{\devanagarifont अतिक्रोधेन सौदास अतिपानेन यादवाः \thinspace{\dandab} \dontdisplaylinenum }%
     \var{{\devanagarifont \numemph\vb\textbf{अतिपानेन}\lem \mssCaCbCc\msNa\msNb\msNc, अतिपापेन \Ed}}% 

%Verse 4:57

{\devanagarifont अतितृष्णाच्च मान्धाता नहुषो द्विजवज्ञया {॥४:५७॥} \veg\dontdisplaylinenum }%
     \var{{\devanagarifont \numnoemph\vc\textbf{अतितृष्णाच्च मान्धाता}\lem \conj, 
अतितृष्णा च मान्दातो \msCa, 
अतितृष्णा च मान्धातो \msCb\msCc\msNa\msNc, 
अतितृष्णा च मन्धातो \msNb, 
अतितृष्णा च मानाच्च च \Ed}}% 
    \var{{\devanagarifont \numnoemph\vd\textbf{नहुषो}\lem \mssCaCbCc\msNa\msNc\Ed, नघुषो \msNb}}% 

{\devanagarifont अतिदानाद्बलिर्नष्ट अतिशौर्येण अर्जुनः \thinspace{\dandab} \dontdisplaylinenum }%
     \var{{\devanagarifont \numemph\va\textbf{॰र्नष्ट}\lem \msCa\msNa\msNb\msNc\Ed, ॰र्नष्टो \msCb, नष्टो \msCc}}% 
    \paral{{\devanagarifontsmall \va {\englishfont \compare\ \MAHASUBHS\ 563ab:}
                         अतिदानाद्बलिर्बद्धो नष्टो मानात्सुयोधनः }}

%Verse 4:58

{\devanagarifont अतिद्यूतान्नलो राजा नृगो गोहरणेन तु {॥४:५८॥} \veg\dontdisplaylinenum }%
     \var{{\devanagarifont \numnoemph\vc\textbf{अतिद्यूतान्नलो}\lem \msCa\msCc\msNb\msNc, अतिद्यूतान्नरो \msCb\msNa, अतिख्यातान्नलो \Ed}}% 
    \var{{\devanagarifont \numnoemph\vd\textbf{नृगो गो॰}\lem \Ed, नृगङ्गो॰ \msCa\msCc\msNb\msNc, नृगं गो॰ \msCb\msNa}}% 
    \lacuna{\devanagarifontsmall \vo {\englishfont After this verse, \Ed\ adds:} 
                        तस्माद्दम सदा स रक्षेत् अति सर्वत्र वर्जयेत्   
                {\englishfont (understand:} तस्माद्दमं सदा रक्षेत् अति सर्वत्र वर्जयेत् {\englishfont )};
                {\englishfont \compare\ \MAHASUBHS\ 563cd:}
                        विनष्टो रावणो लौल्यादति सर्वत्र वर्जयेत्  }%
  
\ujvers\nemsloka {
{\devanagarifont दमेन हीनः पुरुषो द्विजेन्द्र }%
  \dontdisplaylinenum}    \var{{\devanagarifont \numemph\va\textbf{हीनः पुरुषो द्विजेन्द्र}\lem \mssCaCbCc\msNa\msNc, 
हीन पुरुषो द्विजेन्द्र \msNb, हीनं पुरुषं द्विजेन्द्रः \Ed}}% 


\nemslokab

{\devanagarifont स्वर्गं च मोक्षं च सुखं च नास्ति  \danda\dontdisplaylinenum }%
 
\nemslokac

{\devanagarifont विज्ञानधर्मकुलकीर्तिनाश }%
  \dontdisplaylinenum    \var{{\devanagarifont \numnoemph\vc\textbf{॰नाश}\lem \msCb, ॰नाशो \Ed ॰नाम \msCa\msCc\msNa, ॰नश्च \msNb, ॰नागा \msNc}}% 

%Verse 4:59


\nemslokad

{\devanagarifont भवन्ति विप्र दमया विहीनाः {॥४:५९॥} \veg\dontdisplaylinenum }%
     \var{{\devanagarifont \numnoemph\vd\textbf{विप्र}\lem \mssCaCbCc\msNaacorr\msNb\Ed, विप्रा \msNapcorr\msNc\oo 
\textbf{दमया}\lem \msCa\msCbpcorr\msCc\msNa\msNb\msNc\Ed, दया \msCbacorr}}% 


\alalfejezet{यमेषु घृणा (६)}
\vers


{\devanagarifont निर्घृणो न परत्रास्ति निर्घृणो न इहास्ति वै \thinspace{\dandab} \dontdisplaylinenum }%
     \var{{\devanagarifont \numemph\va\textbf{निर्घृणो}\lem \msCa\msCb\msNb, निघृणो \msCc\msNc, निर्घृण \msNaacorr, 
निर्घृ\uncl{णे} \msNapcorr, निर्घृणे \Ed}}% 
    \var{{\devanagarifont \numnoemph\vb\textbf{निर्घृणो}\lem \msCa\msCb\msNaacorr\msNb, निघृणो \msCc\msNc, निर्घृणे \msNapcorr\Ed}}% 

%Verse 4:60

{\devanagarifont निर्घृणे न च धर्मो ऽस्ति निर्घृणे न तपो ऽस्ति वै {॥४:६०॥} \veg\dontdisplaylinenum }%
     \var{{\devanagarifont \numnoemph\vc\textbf{निर्घृणे}\lem \msCa\msCb\msNb\Ed, निघृणे \msCc\msNa\msNc}}% 
    \var{{\devanagarifont \numnoemph\vd\textbf{निर्घृणे}\lem \msCa\msCb\msNa\msNb\Ed, निघृणे \msCc\msNc}}% 

{\devanagarifont परस्त्रीषु परार्थेषु परजीवापकर्षणे \thinspace{\dandab} \dontdisplaylinenum }%
     \var{{\devanagarifont \numemph\vb\textbf{॰जीवापकर्षणे}\lem \msCa\msCc\msNa\msNb\msNc, ॰जीवापर्कणे \msCb, ॰जीवोपकर्षणे \Ed}}% 

%Verse 4:61

{\devanagarifont परनिन्दापरान्नेषु घृणां पञ्चसु कारयेत् {॥४:६१॥} \veg\dontdisplaylinenum }%
     \var{{\devanagarifont \numnoemph\vc\textbf{परनिन्दा॰}\lem \msCb\msCc\msNa\msNb\msNc\Ed, परनिनद्\lk ॰ \msCa\oo 
\textbf{॰परान्नेषु}\lem \mssCaCbCc\msNa\msNc\Ed, ॰परांनेषु \msNb}}% 
    \var{{\devanagarifont \numnoemph\vd\textbf{घृणां}\lem \msCa\msCb\msNa\msNc, घृणा \msCc\msNb\Ed}}% 

{\devanagarifont परस्त्री शृणु विप्रेन्द्र घृणीकार्या सदा बुधैः \thinspace{\dandab} \dontdisplaylinenum }%
     \var{{\devanagarifont \numemph\va\textbf{घृणी॰}\lem \msCa\msCc\msNa\msNb\msNc\Ed, घृणा \msCb}}% 

%Verse 4:62

{\devanagarifont राज्ञी विप्री परिव्राजा स्वयोनिपरयोनिषु {॥४:६२॥} \veg\dontdisplaylinenum }%
     \var{{\devanagarifont \numnoemph\vc\textbf{॰व्राजा}\lem \mssCaCbCc\msNc, ॰व्राजी \msNa\msNb, ॰व्राज्या \Ed}}% 
    \var{{\devanagarifont \numnoemph\vd\textbf{॰पर॰}\lem \mssCaCbCc\msNa\msNc\Ed, ॰पशु॰ \msNb}}% 

{\devanagarifont परार्थे शृणु भूयो ऽन्य अन्यायार्थमुपार्जनम् \thinspace{\dandab} \dontdisplaylinenum }%
     \var{{\devanagarifont \numemph\vb\textbf{अन्याया॰}\lem \mssCaCbCc\msNa\msNc\Ed, अन्यया॰ \msNb\oo 
\textbf{॰र्जनम्}\lem \mssCaCbCc\msNa\msNc\Ed, ॰र्ज्जवम् \msNb}}% 
    \paral{{\devanagarifontsmall \vb {\englishfont \compare\ \BHG\ 16.12:}
                 आशापाशशतैर्बद्धाः कामक्रोधपरायणाः\thinspace{\devanagarifontsmall ।}
                 ईहन्ते कामभोगार्थमन्यायेनार्थसंचयान्\thinspace{\devanagarifontsmall ॥} }}

%Verse 4:63

{\devanagarifont आढप्रस्थतुलाव्याजैः परार्थं यो ऽपकर्षति {॥४:६३॥} \veg\dontdisplaylinenum }%
     \var{{\devanagarifont \numnoemph\vc\textbf{॰तुला॰}\lem \mssCaCbCc\msNa\msNc\Ed, ॰तुल॰ \msNb}}% 
    \var{{\devanagarifont \numnoemph\vd\textbf{॰र्थं}\lem \msCa\msCb\msNa\Ed, ॰र्थ \msCc, ॰\uncl{र्थ} \msNb, ॰र्थे \msNc}}% 

{\devanagarifont जीवापकर्षणे विप्र घृणीकुर्वीत पण्डितः \thinspace{\dandab} \dontdisplaylinenum }%
     \var{{\devanagarifont \numemph\va\textbf{विप्र}\lem \msCb\msNa\msNb\msNc\Ed, वि\uncl{प्र} \msCa, विप्रे \msCc}}% 
    \var{{\devanagarifont \numnoemph\vb\textbf{घृणी॰}\lem \mssCaCbCc\msNa\msNb\msNc, घृणां \Ed}}% 

%Verse 4:64

{\devanagarifont वनजावनजा जीवा विलगाश्चरणाचराः {॥४:६४॥} \veg\dontdisplaylinenum }%
     \var{{\devanagarifont \numnoemph\vc\textbf{वनजावनजा}\lem \msCa\msCc\msNa\msNb\Ed, 
वनजाव\lk जा \msCbacorr, वनजा व\uncl{नि}जा \msCbpcorr, वनज विनजा \msNc}}% 
    \var{{\devanagarifont \numnoemph\vd\textbf{विलगाश्चरणाचराः}\lem \corr, 
विलगाचरणाचराः \msCa\msCb\msNc, विलगोचरगोचरः \msCc\Ed, विलगोचरगोचराः \msNa, 
\uncl{विलगाचर}णाचराः \msNb}}% 

{\devanagarifont परनिन्दा च का विप्र शृणु वक्ष्ये समासतः \thinspace{\dandab} \dontdisplaylinenum }%
     \var{{\devanagarifont \numemph\vb\textbf{वक्ष्ये}\lem \mssCaCbCc\msNa\msNb\msNc, वक्ष्या \Ed}}% 

%Verse 4:65

{\devanagarifont देवानां ब्राह्मणानां च गुरुमातातिथिद्विषः {॥४:६५॥} \veg\dontdisplaylinenum }%
     \lacuna{\devanagarifontsmall \vcd {\englishfont These two pādas are illegible in \msNb} }%
  
{\devanagarifont परान्नेषु घृणा कार्या अभोज्येषु च भोजनम् \thinspace{\dandab} \dontdisplaylinenum }%
     \var{{\devanagarifont \numemph\vb\textbf{अभोज्येषु}\lem \msCa\msCc\msNa\msNb\msNc\Ed, अभोज्ये \msCb}}% 

%Verse 4:66

{\devanagarifont सूतके मृतके शौण्डे वर्णभ्रष्टकुले नटे {॥४:६६॥} \veg\dontdisplaylinenum }%
     \var{{\devanagarifont \numnoemph\vc\textbf{शौण्डे}\lem \msNa, सौण्ड्ये \msCa\msCc\msNc, शोण्ड्ये \msCb, \uncl{सौण्डे} \msNb, सौण्डो \Ed}}% 
    \lacuna{\devanagarifontsmall \vo {\englishfont This verse is mostly illegible in \msNb} }%
  
\ujvers\nemsloka {
{\devanagarifont एते पञ्चघृणासु सक्तपुरुषाः स्वर्गार्थमोक्षार्थिनो }%
  \dontdisplaylinenum}    \var{{\devanagarifont \numemph\va\textbf{॰पुरुषाः}\lem \msNc, ॰पुरुषः \mssCaCbCc\msNa\msNb\Ed\oo 
\textbf{॰र्थिनो}\lem \eme, ॰र्थिनः \msNcpcorr, ॰र्थिनां \mssCaCbCc\msNa\msNb\Ed, ॰र्थिना \msNcacorr}}% 


\nemslokab

{\devanagarifont लोके ऽनिन्दनमाप्नुवन्ति सततं कीर्तिर्यशोऽलंकृताः  \danda\dontdisplaylinenum }%
     \var{{\devanagarifont \numnoemph\vb\textbf{ऽनिन्दनमाप्नुवन्ति}\lem \msCa\msCb\msNa\msNb\msNc, 
ऽनिन्दनवाप्नुवन्ति \msCc, नन्दनवायुवान्ति \Ed\oo 
\textbf{॰कृताः}\lem \eme, ॰कृतम् \mssCaCbCc\msNa\msNb\msNc\Ed}}% 

\nemslokac

{\devanagarifont प्रज्ञाबोधश्रुतिं स्मृतिं च लभते मानं च नित्यं लभेद् }%
  \dontdisplaylinenum    \var{{\devanagarifont \numnoemph\vc\textbf{॰श्रुतिं}\lem \msNc, ॰श्रुति॰ \mssCaCbCc\msNa\msNb\Ed\oo 
\textbf{नित्यं}\lem \msCa\msCc\msNa\msNb\msNc\Ed, नित्य \msCb}}% 

%Verse 4:67


\nemslokad

{\devanagarifont दाक्षिण्यं सभवेत्स आयुष परं प्राप्नोति निःसंशयः {॥४:६७॥} \veg\dontdisplaylinenum }%
     \var{{\devanagarifont \numnoemph\vd\textbf{स आयुष}\lem \eme, समायुष \mssCaCbCc\msNc, समायुषः \msNa\ \unmetr, 
\uncl{समायुष} \msNb, स मानुष \Ed\oo 
\textbf{निःसंशयः}\lem \mssCaCbCc\msNb\msNc\Ed, निसंशयः \msNa}}% 


\alalfejezet{यमेषु पञ्चविधो धन्यः (७)}
\vers


{\devanagarifont चतुर्मौनं चतुःशत्रुश्चतुरायतनं तथा \thinspace{\dandab} \dontdisplaylinenum }%
     \var{{\devanagarifont \numemph\va\textbf{चतुर्मौनं च॰}\lem \corr, चतुर्मौनश्च॰ \msCa\msCb\msNa\msNc\Ed, चतुर्मोणश्च॰ \msCc, 
\uncl{चतुर्मौनश्च}॰ \msNb}}% 
    \var{{\devanagarifont \numnoemph\vab\textbf{॰तुःशत्रुश्च॰}\lem \msCa\msCb\msNa\msNb\msNc, ॰तुशत्रु च॰ \msCc, ॰तुःशत्रु च॰ \Ed}}% 
    \var{{\devanagarifont \numnoemph\vb\textbf{॰तुरायतनं}\lem \msCb\msCc\msNa\msNc\Ed, ॰\uncl{तु}रायतनं \msCa, 
॰\uncl{तुरायतनम्} \msNb}}% 

%Verse 4:68

{\devanagarifont चतुर्ध्यानं चतुष्पादं पञ्चधन्यविधोच्यते {॥४:६८॥} \veg\dontdisplaylinenum }%
     \var{{\devanagarifont \numnoemph\vc\textbf{॰पादं}\lem \mssCaCbCc\msNc\Ed, ॰पादः \msNa, \lk\lk\ \msNb}}% 
    \var{{\devanagarifont \numnoemph\vd\textbf{पञ्चधन्य॰}\lem \mssCaCbCc\msNa\msNb\msNc, धन्यपञ्च॰ \Ed}}% 

{\devanagarifont चतुर्मौनस्य वक्ष्यामि शृणुष्वावहितो भव \thinspace{\dandab} \dontdisplaylinenum }%
     \var{{\devanagarifont \numemph\va\textbf{॰मौनस्य}\lem \msCa\msCc\msNa\msNb\msNc\Ed, ॰मोनस्य \msCb}}% 

%Verse 4:69

{\devanagarifont पारुष्यपिशुनामिथ्यासम्भिन्नानि च वर्जयेत् {॥४:६९॥} \veg\dontdisplaylinenum }%
     \var{{\devanagarifont \numnoemph\vc\textbf{पारुष्य॰}\lem \mssCaCbCc\msNb\msNc\Ed, पारुष्यं \msNa\oo 
\textbf{॰पिशुना॰}\lem \mssCaCbCc\msNa\msNb\msNc, ॰पिण्डाना॰ \Ed}}% 
    \paral{{\devanagarifontsmall \vcd {\englishfont \compare\ \DIVYAV\ 186.21:}
                     आर्य, किमेभिः कर्म कृतम्येनैवंविधानि दुःखानि प्रत्यनुभवन्तीति? 
                     स कथयति\thinspace{\devanagarifontsmall ।} एते प्राणातिपातिका अदत्तादायिकाः काममिथ्याचारिका मृषावादिकाः पैशुनिकाः पारुषिकाः 
                     संभिन्नप्रलापिका अभिध्यालवो व्यापन्नचित्ता मिथ्यादृष्टिकाः\thinspace{\devanagarifontsmall ।};
                     {\englishfont \compare\ \DHARMP\ 1.31cd--32ab:}
                         मिथ्या पिशुनसम्भिन्नपारुष्यवचनानि च\thinspace{\devanagarifontsmall ॥}
                         जल्पतः सम्भवन्त्येते तस्मान्मौनं प्रशस्यते\thinspace{\devanagarifontsmall ।} }}

{\devanagarifont कामः क्रोधश्च लोभश्च मोहश्चैव चतुर्विधः \thinspace{\dandab} \dontdisplaylinenum  }%
 
%Verse 4:70

{\devanagarifont चतुःशत्रुर्निहन्तव्यः सो ऽरिहा वीतकल्मषः {॥४:७०॥} \veg\dontdisplaylinenum }%
     \var{{\devanagarifont \numemph\vc\textbf{चतुःशत्रुर्नि॰}\lem \msCa\msCb\Ed, चतुशत्रु नि॰ \msCc\msNa\msNb\msNc}}% 
    \var{{\devanagarifont \numnoemph\vd\textbf{सो ऽरिहा}\lem \msCa\msCc\msNa\msNb\msNc, स्रोरिहा \msCb, सर्वथा \Ed}}% 

{\devanagarifont चतुरायतनं विप्र कथयिष्यामि तच्छृणु \thinspace{\dandab} \dontdisplaylinenum }%
 
%Verse 4:71

{\devanagarifont करुणा मुदितोपेक्षा मैत्री चायतनं स्मृतम् {॥४:७१॥} \veg\dontdisplaylinenum }%
     \var{{\devanagarifont \numemph\vc\textbf{मुदितो॰}\lem \mssCaCbCc\msNa\msNb\msNc, मुदितौ॰ \Ed}}% 
    \var{{\devanagarifont \numnoemph\vd\textbf{चायतनं}\lem \msCc\msNa\msNb\msNc\Ed, चायतन \msCa, चायत\uncl{न} \msCb}}% 

{\devanagarifont चतुर्ध्यानाधुना वक्ष्ये संसारार्णवतारणम् \thinspace{\dandab} \dontdisplaylinenum }%
 
%Verse 4:72

{\devanagarifont आत्मविद्याभवः सूक्ष्मं ध्यानमुक्तं चतुर्विधम् {॥४:७२॥} \veg\dontdisplaylinenum }%
     \var{{\devanagarifont \numemph\vc\textbf{॰भवः}\lem \msCb\msCcpcorr\msNa\msNb\msNc, ॰भव \msCa\msCcacorr, ॰भवं \Ed}}% 
    \var{{\devanagarifont \numnoemph\vcd\textbf{सूक्ष्मं ध्या॰}\lem \msCa\msNa\msNc\Ed, 
सूक्ष्मा\uncl{न्या}॰ \msCb, 
सू\uncl{क्ष्म}ध्या॰ \msCc, सूक्ष्मध्यान॰ \msNb}}% 
    \var{{\devanagarifont \numnoemph\vd\textbf{॰नमुक्तं चतुर्विधम्}\lem \msCc\msNb, ॰नमुक्तश्चतुर्विधम् \msCa, 
॰नमुक्तश्चतुर्विधः \msCb\msNa, 
॰नमुक्तं चतुर्विधिं \msNc, ॰नयज्ञश्च \Ed}}% 

{\devanagarifont आत्मतत्त्वः स्मृतो धर्मो विद्या पञ्चसु पञ्चधा \thinspace{\dandab} \dontdisplaylinenum }%
     \var{{\devanagarifont \numemph\va\textbf{स्मृतो}\lem \msCa\msCb\msNa\msNb\msNc, स्मृता \msCc\Ed\oo 
\textbf{धर्मो}\lem \mssCaCbCc\msNa\msNb\msNc, धन्या \Ed}}% 

%Verse 4:73

{\devanagarifont षट्त्रिंशाक्षरमित्याहुः सूक्ष्मतत्त्वमलक्षणम् {॥४:७३॥} \veg\dontdisplaylinenum }%
     \var{{\devanagarifont \numnoemph\vcd\textbf{आहुः सू॰}\lem \msCb\msCc\msNa\msNb\msNc\Ed, आ\lk\lk\ \msCa}}% 

{\devanagarifont चतुष्पादः स्मृतो धर्मश्चतुराश्रममाश्रितः \thinspace{\dandab} \dontdisplaylinenum }%
     \var{{\devanagarifont \numemph\vab\textbf{धर्मश्च॰}\lem \msCa\msCb\msNa\msNc\Ed, धर्म च॰ \msCc\msNb}}% 
    \var{{\devanagarifont \numnoemph\vb\textbf{॰श्रितः}\lem \mssCaCbCc\msNa\msNb\Ed, ॰श्रिताः \msNc}}% 

%Verse 4:74

{\devanagarifont गृहस्थो ब्रह्मचारी च वानप्रस्थो ऽथ भैक्षुकः {॥४:७४॥} \veg\dontdisplaylinenum }%
     \var{{\devanagarifont \numnoemph\vd\textbf{भैक्षुकः}\lem \mssCaCbCc\msNa\msNb\msNc, भक्षकः \Ed}}% 
    \paral{{\devanagarifontsmall \vcd {\englishfont  = \MBH\ 12.234.13ab \similar\ \MBH\ 14.4513ab etc. }
                 \vo {\englishfont \compare\ 3.4 above:}
                 श्रुतिस्मृतिद्वयोर्मूर्तिश्चतुष्पादवृषः स्थितः\thinspace{\devanagarifontsmall ।}
                 चतुराश्रम यो धर्मः कीर्तितानि मनीषिभिः\thinspace{\devanagarifontsmall ॥} }}

{\devanagarifont धन्यास्ते यैरिदं वेत्ति निखिलेन द्विजोत्तम \thinspace{\dandab} \dontdisplaylinenum }%
     \var{{\devanagarifont \numemph\va\textbf{यैरिदं}\lem \msCa\msNa\msNb\msNc\Ed, येरिदं \msCb\msCc\oo 
\textbf{वेत्ति}\lem \msCa\msCb\msNa\msNb\msNc\Ed, वेति \msCc}}% 

%Verse 4:75

{\devanagarifont पावनं सर्वपापानां पुण्यानां च प्रवर्धनम् {॥४:७५॥} \veg\dontdisplaylinenum }%
     \var{{\devanagarifont \numnoemph\vd\textbf{प्रवर्धनम्}\lem \mssCaCbCc\msNa\msNb\msNc, प्रवर्धनः \Ed}}% 

{\devanagarifont आयुः कीर्तिर्यशः सौख्यं धन्यादेव प्रवर्धते \thinspace{\dandab} \dontdisplaylinenum }%
     \var{{\devanagarifont \numemph\vb\textbf{धन्यादेव}\lem \mssCaCbCc\msNa\msNb\msNc, धर्मादेव \Ed}}% 

%Verse 4:76

{\devanagarifont शान्तिः पुष्टिः स्मृतिर्मेधा जायते धन्यमानवे {॥४:७६॥} \veg\dontdisplaylinenum }%
     \var{{\devanagarifont \numnoemph\vc\textbf{पुष्टिः}\lem \msCb\msCc\msNa\msNb\msNc\Ed, \lk ष्टिः \msCa\oo 
\textbf{स्मृतिर्मेधा}\lem \msCa\msCb\msNb\msNc\Ed, स्मृति मेधा \msCc\msNa}}% 
    \var{{\devanagarifont \numnoemph\vd\textbf{॰मानवे}\lem \eme, ॰मानवः \mssCaCbCc\msNa\msNb\msNc\Ed}}% 


\alalfejezet{यमेष्वप्रमादः (८)}
{\devanagarifont प्रमादस्थान पञ्चैव कीर्तयिष्यामि तच्छृणु \thinspace{\dandab} \dontdisplaylinenum }%
     \var{{\devanagarifont \numemph\va\textbf{॰स्थान}\lem \msCa\msCc\msNa\msNb, ॰स्थानं \msCb\msNc\Ed\ \unmetr\oo 
\textbf{पञ्चैव}\lem \mssCaCbCc\msNa\msNb\msNc, पञ्चैवं \Ed}}% 
    \var{{\devanagarifont \numnoemph\vb\textbf{कीर्तयिष्यामि}\lem \mssCaCbCc\msNa\msNc\Ed, कीर्तियिष्यामि \msNb}}% 

{\devanagarifont ब्रह्महत्या सुरापानं स्तेयो गुर्वङ्गनागमम्  \danda\dontdisplaylinenum }%
     \paral{{\devanagarifontsmall \vcdef {\englishfont \similar\ \MBH\ Indices 12.30:}
                     ब्रह्महत्यां सुरापानं स्तेयं गुर्वङ्गनागमम्\thinspace{\devanagarifontsmall ।}
                     महान्ति पातकान्याहुः संयोगं चैव तैः सह\thinspace{\devanagarifontsmall ॥}
                     {\englishfont  \similar\ \MANU\ 11.55 (in Olivelle's edition):}
                     ब्रह्महत्या सुरापानं स्तेयं गुर्वङ्गनागमः\thinspace{\devanagarifontsmall ।}
                     महान्ति पातकान्याहुः संसर्गश्चापि तैः सह\thinspace{\devanagarifontsmall ॥}
                 {\englishfont \compare\ also \YAJNS\ 3.228:}
                         ब्रह्महा मद्यपः स्तेनस्तथैव गुरुतल्पगः\thinspace{\devanagarifontsmall ।}
                         एते महापातकिनो यश्च तैः सह संवसेत्\thinspace{\devanagarifontsmall ॥}  }}

%Verse 4:77

{\devanagarifont महापातकमित्याहुस्तत्संयोगी च पञ्चमः {॥४:७७॥} \veg\dontdisplaylinenum }%
 
{\devanagarifont अनृतं च समुत्कर्षे राजगामी च पैशुनः \thinspace{\dandab} \dontdisplaylinenum }%
     \var{{\devanagarifont \numemph\va\textbf{समुत्कर्षे}\lem \eme, समुत्कर्षं \msCa\msNa, 
समुत्क\uncl{र्ष} \msCb, 
समुत्कर्ष \msCc\msNb\msNc\Ed}}% 
    \var{{\devanagarifont \numnoemph\vb\textbf{राज॰}\lem \mssCaCbCc\msNa\msNb\msNc, राज्ञी॰ \Ed}}% 

%Verse 4:78

{\devanagarifont गुरोश्चालीकनिर्बन्धः समानि ब्रह्महत्यया {॥४:७८॥} \veg\dontdisplaylinenum }%
     \var{{\devanagarifont \numnoemph\vc\textbf{॰निर्बन्धः}\lem \eme, ॰निर्बद्धः \msCb\msNc, निबद्धस् \msCa\msCc\msNa\msNb, निर्वद्धस् \Ed}}% 
    \var{{\devanagarifont \numnoemph\vd\textbf{ब्रह्महत्यया}\lem \msCb\msCc\msNa\msNb\msNc\Ed, ब्र\lk\lk \lk या \msCa}}% 
    \paral{{\devanagarifontsmall \vo \similar\ {\englishfont \MBH\ 5.40.3 and \MANU\ 11.56:}
                  अनृतं च समुत्कर्षे राजगामि च पैशुनम्\thinspace{\devanagarifontsmall ।}
                  गुरोश्चालीकनिर्बन्धः समानि ब्रह्महत्यया\thinspace{\devanagarifontsmall ॥}
                 {\englishfont \similar\ \VISNUS\ 37.1--4 \similar\ \AGNIP\ 168.25} }}

{\devanagarifont ब्रह्मोज्झं वेदनिन्दा च कूटसाक्षी सुहृद्वधः \thinspace{\dandab} \dontdisplaylinenum }%
     \var{{\devanagarifont \numemph\va\textbf{ब्रह्मोज्झं}\lem \eme, ब्रह्मो ऋग्॰ \mssCaCbCc\msNa\msNb\msNc, ब्रह्म ऋग्॰ \Ed}}% 
    \var{{\devanagarifont \numnoemph\vb\textbf{सुहृद्वधः}\lem \mssCaCbCc\msNa\msNb\msNc, सकृद्बुधः \Ed}}% 

%Verse 4:79

{\devanagarifont गर्हितानाद्ययोर्जग्धिः सुरापानसमानि षट् {॥४:७९॥} \veg\dontdisplaylinenum }%
     \var{{\devanagarifont \numnoemph\vc\textbf{॰नाद्ययोर्जग्धिः}\lem \eme, ॰न्नञ्च यो जग्धिस् \msCa, ॰न्नञ्च यो जग्धि \msCb, 
॰न्नञ्च योद्विग्नः \msCc, ॰न्नं च यो जग्धिः \msNa, ॰न्नं च यो जग्धिः \msNb, 
॰न्नञ्च यो जवे \msNc, ॰न्नश्च यो विप्रः \Ed}}% 
    \paral{{\devanagarifontsmall \vo \similar\ {\englishfont \MANU\ 11.57:}
                 ब्रह्मोज्झता वेदनिन्दा कौटसाक्ष्यं सुहृद्वधः\thinspace{\devanagarifontsmall ।}
                 गर्हितानाद्ययोर्जग्धिः सुरापानसमानि षट्\thinspace{\devanagarifontsmall ॥}
                 {\englishfont \compare\ \YAJNS\ 3.229:}
                         गुरूणामध्यधिक्षेपो वेदनिन्दा सुहृद्वधः\thinspace{\devanagarifontsmall ।}
                         ब्रह्महत्यासमं ज्ञेयमधीतस्य च नाशनम्\thinspace{\devanagarifontsmall ॥} }}

{\devanagarifont रेतोत्सेकः स्वयोन्यासु कुमारीष्वन्त्यजासु च \thinspace{\dandab} \dontdisplaylinenum }%
     \var{{\devanagarifont \numemph\va\textbf{स्वयोन्यासु}\lem \msCa\msCc\msNa\msNb\msNc\Ed, सुतोन्यासु \msCb}}% 

%Verse 4:80

{\devanagarifont सख्युः पुत्रस्य च स्त्रीषु गुरुतल्पसमः स्मृतः {॥४:८०॥} \veg\dontdisplaylinenum }%
     \var{{\devanagarifont \numnoemph\vc\textbf{सख्युः}\lem \eme, सख्य \mssCaCbCc\msNa\Ed, \lk\lk\ \msNb, स\uncl{ख्यु} \msNc\oo 
\textbf{पुत्रस्य च स्त्रीषु}\lem \mssCaCbCc\msNa\msNc, \lk\lk \lk\lk \lk\lk\ \msNb, पुत्रीषु चास्त्रीषु \Ed}}% 
    \var{{\devanagarifont \numnoemph\vd\textbf{॰समः}\lem \mssCaCbCc\msNa\msNc, \lk\lk\ \msNb, ॰सम \Ed}}% 
    \paral{{\devanagarifontsmall \vo \similar\ {\englishfont \MANU\ 11.59:}
                                 रेतःसेकः स्वयोनीषु कुमारीष्वन्त्यजासु च\thinspace{\devanagarifontsmall ।}
                                 सख्युः पुत्रस्य च स्त्रीषु गुरुतल्पसमं विदुः\thinspace{\devanagarifontsmall ॥} }}

{\devanagarifont निक्षेपस्यापहरणं नराश्वरजतस्य च \thinspace{\dandab} \dontdisplaylinenum }%
     \var{{\devanagarifont \numemph\va\textbf{निक्षेप॰}\lem \msCa\msCc\msNa\msNc\Ed, निखेप॰ \msCb, \uncl{निक्षेप}॰ \msNb}}% 
    \var{{\devanagarifont \numnoemph\vb\textbf{नराश्वरजतस्य}\lem \msCa\msCc\msNa\msNc\Ed, नराणां स्वजनस्य \msCb, 
\uncl{नराश्वरजतस्य} \msNb}}% 

%Verse 4:81

{\devanagarifont भूमिवज्रमणीनां च रुक्मस्तेयसमः स्मृतः {॥४:८१॥} \veg\dontdisplaylinenum }%
     \var{{\devanagarifont \numnoemph\vd\textbf{रुक्मस्तेय॰}\lem \eme, \uncl{रूग्य}\lk य॰ \msCa, 
रुग्मस्तेय॰ \msCb\msCc\msNa\msNc, \lk\lk \lk\lk\ \msNb, हृतस्तेय॰ \Ed\oo 
\textbf{॰समः}\lem \msCa\msCbpcorr\msCc\msNa\msNb\msNc, सः \msCbacorr, ॰सम \Ed}}% 
    \paral{{\devanagarifontsmall \vo {\englishfont = \MANU\ 11.58 } }}

{\devanagarifont चत्वार एते सम्भूय यत्पापं कुरुते नरः \thinspace{\dandab} \dontdisplaylinenum }%
     \var{{\devanagarifont \numemph\va\textbf{एते}\lem \mssCaCbCc\msNa\msNc, \uncl{एते} \msNb, एव \Ed\oo 
\textbf{सम्भूय}\lem \msCa\msCb\msNa\msNc\Ed, संभूयो \msCc, \uncl{संभूयो} \msNb}}% 

{\devanagarifont महापातकपञ्चैतत् तेन सर्वं प्रकाशितम्  \danda\dontdisplaylinenum }%
     \var{{\devanagarifont \numnoemph\vc\textbf{॰पञ्चैतत्}\lem \corr, ॰पञ्चैतन् \mssCaCbCc\Ed, ॰पञ्चैते \msNa, ॰पञ्चैतम् \msNb, ॰पञ्चेतन् \msNc}}% 

%Verse 4:82

{\devanagarifont पञ्चप्रमादमेतानि वर्जनीयं द्विजोत्तम {॥४:८२॥} \veg\dontdisplaylinenum }%
     \var{{\devanagarifont \numnoemph\ve\textbf{॰मादम्}\lem \mssCaCbCc\msNa\msNb\msNc, ॰माद \Ed}}% 
    \var{{\devanagarifont \numnoemph\vf\textbf{वर्जनीयं}\lem \msCa\msCb\msNa\msNb\msNc\Ed, वर्जनीयो \msCc}}% 


\alalfejezet{यमेषु माधुर्यम् (९)}
{\devanagarifont कायवाङ्मनमाधुर्यश्चक्षुर्बुद्धिश्च पञ्चमः \thinspace{\dandab} \dontdisplaylinenum }%
     \var{{\devanagarifont \numemph\vab\textbf{मनमाधुर्यश्च॰}\lem \eme, ॰मनसा धूर्यश्च॰ \msCa\msCc\msNa\msNc, 
॰मन\uncl{मा}धूर्यश्च॰ \msCb, 
॰मन\lk धूर्य\lk ॰ \msNb, ॰मनसा भूयश्च॰ \Ed}}% 
    \var{{\devanagarifont \numnoemph\vb\textbf{॰क्षुर्बुद्धि॰}\lem \msCa\msCb\msNc\Ed, ॰क्षु बुद्धि॰ \msCc\msNa, \lk\lk \lk\  \msNb}}% 

%Verse 4:83

{\devanagarifont सौम्यदृष्टिप्रदानं च क्रूरबुद्धिं च वर्जयेत् {॥४:८३॥} \veg\dontdisplaylinenum }%
     \var{{\devanagarifont \numnoemph\vc\textbf{॰दानं च}\lem \mssCaCbCc\msNa\msNc, \lk\lk\ \msNb, ॰दानश्च \Ed}}% 
    \var{{\devanagarifont \numnoemph\vd\textbf{॰बुद्धिं च}\lem \msCa\msNa\msNc, बुद्धिश्च \msCb, ॰दृष्टिं च \msCc\Ed, \lk\lk \lk\ \msNb}}% 

{\devanagarifont प्रसन्नमनसा ध्यायेत्प्रियवाक्यमुदीरयेत् \thinspace{\dandab} \dontdisplaylinenum }%
     \var{{\devanagarifont \numemph\va\textbf{प्रसन्न॰}\lem \mssCaCbCc\msNa\Ed, \uncl{प्रसन्न}॰ \msNb, प्रसंन॰ \msNc}}% 

%Verse 4:84

{\devanagarifont यथाशक्तिप्रदानं च स्वाश्रमाभ्यागतो गुरुः {॥४:८४॥} \veg\dontdisplaylinenum }%
     \var{{\devanagarifont \numnoemph\vc\textbf{यथा॰}\lem \mssCaCbCc\msNa\msNb\msNc, यस्य \Ed\oo 
\textbf{॰दानं}\lem \mssCaCbCc\msNa\msNb\msNc, ॰दातश् \Ed}}% 
    \var{{\devanagarifont \numnoemph\vd\textbf{स्वाश्रमा॰}\lem \msCa\msCb\msNa\msNb\msNc\Ed, स्वासमा॰ \msCc\oo 
\textbf{॰गतो}\lem \mssCaCbCc\msNa\msNb\Ed, ॰सतो \msNc}}% 

{\devanagarifont इन्धनोदकदानं च जातवेदमथापि वा \thinspace{\dandab} \dontdisplaylinenum }%
     \var{{\devanagarifont \numemph\vb\textbf{इन्धनो॰}\lem \mssCaCbCc\msNa\msNb\Ed, इत्वनो॰ \msNc\oo 
\textbf{जात॰}\lem \msCa\msCc\msNa\msNb\msNc\Ed, जा॰ \msCb}}% 

{\devanagarifont सुलभानि न दत्तानि इन्धनाग्न्युदकानि च  \danda\dontdisplaylinenum }%
     \var{{\devanagarifont \numnoemph\vc\textbf{सुलभानि न}\lem \mssCaCbCc\msNa\msNb\msNc, सुरभानि च \Ed}}% 
    \var{{\devanagarifont \numnoemph\vd\textbf{॰दकानि}\lem \mssCaCbCc\msNa\msNc\Ed, ॰\uncl{त}कानि \msNb}}% 

%Verse 4:85

{\devanagarifont क्षुते जीवेति वा नोक्तं तस्य किं परतः फलम् {॥४:८५॥} \veg\dontdisplaylinenum }%
     \var{{\devanagarifont \numnoemph\ve\textbf{क्षुते}\lem \conj, क्षुतं \mssCaCbCc\msNa\msNb\msNc, शतं \Ed}}% 


\alalfejezet{यमेष्वार्जवम् (१०)}
{\devanagarifont पञ्चार्जवाः प्रशंसन्ति मुनयस्तत्त्वदर्शिनः \thinspace{\dandab} \dontdisplaylinenum }%
     \var{{\devanagarifont \numemph\va\textbf{पञ्चार्जवाः}\lem \msCa\msCb\msNa\msNc, पञ्चार्जवः \msCc, \lk\lk \lk\lk\ \msNb, पञ्चार्जवा \Ed\oo 
\textbf{प्रशंसन्ति}\lem \mssCaCbCc\msNc, प्रशसन्ति \msNa\Ed, \uncl{प्रससन्ति} \msNb}}% 

{\devanagarifont कर्मवृत्त्याभिवृद्धिं च पारतोषिकमेव च  \danda\dontdisplaylinenum }%
     \var{{\devanagarifont \numnoemph\vc\textbf{कर्म॰}\lem \msCb\msCc\msNa\msNc\Ed, \lk र्म्म॰ \msCa, \uncl{कम्मा}॰ \msNb\oo 
\textbf{॰वृत्त्याभिवृद्धिं च}\lem \mssCaCbCc\msNa\msNc, 
॰वृत्तिभिवृद्धिञ्च \msNb, ॰वृत्याभिवृद्धिश्च \Ed}}% 
    \var{{\devanagarifont \numnoemph\vd\textbf{पारितोषिक॰}\lem \eme, पारतोषिक॰ \mssCaCbCc\msNa\msNb\msNc\Ed}}% 

%Verse 4:86

{\devanagarifont स्त्रीधनोत्कोचवित्तं च आर्जवो नाभिनन्दति {॥४:८६॥} \veg\dontdisplaylinenum }%
     \var{{\devanagarifont \numnoemph\ve\textbf{स्त्रीधनोत्कोच॰}\lem \mssCaCbCc\msNa\msNb\msNc, स्त्रीधनङ्गो च \Ed\oo 
\textbf{॰वित्तं च}\lem \mssCaCbCc\msNa\msNc\Ed, ॰वित्तिञ्च \msNb}}% 
    \var{{\devanagarifont \numnoemph\vf\textbf{आर्जवो ना॰}\lem \msCa\msCb\msNa\msNb\msNc, आर्जवञ्च \msCc, आर्ज्जवेना॰ \Ed}}% 

{\devanagarifont आर्जवो न वृथा यज्ञ आर्जवो न वृथा तपः \thinspace{\dandab} \dontdisplaylinenum }%
     \var{{\devanagarifont \numemph\vab\textbf{आर्जवो न वृथा यज्ञ आर्जवो न वृथा तपः}\lem \mssCaCbCc\msNb\msNc, \om\ \msNaacorr, 
आर्जवो न वृथा यज्ञ आर्जवो न वृथा तप \msNapcorr, 
आर्जवो न वृथा यज्ञश्चार्र्जवो न वृथा तपः \Ed}}% 

%Verse 4:87

{\devanagarifont आर्जवो न वृथा दानमार्जवो न वृथाग्नयः {॥४:८७॥} \veg\dontdisplaylinenum }%
     \var{{\devanagarifont \numnoemph\vcd\textbf{(आर्जवो{\englishfont ...} वृथाग्नयः)}\lem \mssCaCbCc\msNa\msNb\msNc, \om\ \Ed}}% 

{\devanagarifont आर्जवस्येन्द्रियग्रामः सुप्रसन्नो ऽपि तिष्ठति \thinspace{\dandab} \dontdisplaylinenum }%
     \var{{\devanagarifont \numemph\vab\textbf{(आर्जव॰{\englishfont ...} तिष्ठति)}\lem \mssCaCbCc\msNa\msNb\msNc, \om\ \Ed}}% 
    \var{{\devanagarifont \numnoemph\va\textbf{॰ग्रामः}\lem \msCa\msCb\msNc\Ed, ॰ग्रामात् \msCc\msNb, ॰ग्रामाः \msNa}}% 

%Verse 4:88

{\devanagarifont आर्जवस्य सदा देवाः काये तस्य चरन्ति ते {॥४:८८॥} \veg\dontdisplaylinenum }%
     \var{{\devanagarifont \numnoemph\vd\textbf{तस्य चरन्ति}\lem \msCb\msCc\msNa\msNb\msNc, त\lk\lac  न्ति \msCa, तस्य रमन्ति \Ed}}% 

\ujvers\nemsloka {
{\devanagarifont इति यमप्रविभागः कीर्तितो ऽयं द्विजेन्द्र }%
  \dontdisplaylinenum}    \var{{\devanagarifont \numemph\va\textbf{यमप्रविभागः}\lem \msCa\msCb\msNb\msNc, यमविभागः \msCc, 
यमप्ररिभागः \msNa, नियमपरिभागः \Ed\oo 
\textbf{द्विजेन्द्र}\lem \mssCaCbCc\msNa\msNb\msNc, नरेन्द्र \Ed}}% 


\nemslokab

{\devanagarifont इह परत सुखार्थं कारयेत्तं मनुष्यः  \danda\dontdisplaylinenum }%
     \var{{\devanagarifont \numnoemph\vb\textbf{॰येत्तं मनुष्यः}\lem \corr, ॰येत्तन्मनुष्यः \msCa\msNa\msNb\msNc\Ed, ॰येत्त मनुष्यः \msCb, 
॰येत्तत्मनुष्यः \msCc}}% 

\nemslokac

{\devanagarifont दुरितमलपहारी शङ्करस्याज्ञयास्ते }%
  \dontdisplaylinenum    \var{{\devanagarifont \numnoemph\vc\textbf{दुरित॰}\lem \mssCaCbCc\msNa\msNb\msNc, इरित॰ \Ed\oo 
\textbf{॰पहारी}\lem \msCa\msCb\msNa\msNb\msNc\Ed, ॰पलपहारी \msCc\oo 
\textbf{॰ज्ञयास्ते}\lem \mssCaCbCc\msNb\msNc\Ed, ॰ज्ञयाते \msNa}}% 

%Verse 4:89


\nemslokad

{\devanagarifont भवति पृथिविभर्ता ह्येकछत्रप्रवर्ता {॥४:८९॥} \veg\dontdisplaylinenum }%
     \var{{\devanagarifont \numnoemph\vd\textbf{॰वर्ता}\lem \conj, ॰वृत्ता \mssCaCbCc\msNb\msNc, ॰वृत्ताः \msNa\Ed}}% 

\vers


{\devanagarifont 
\jump
\begin{center}
\ketdanda~इति वृषसारसंग्रहे यमविभागो नामाध्यायश्चतुर्थः~\ketdanda
\end{center}
\dontdisplaylinenum\vers  }%
     \var{{\devanagarifont \numnoemph{\englishfont \Colo:}\textbf{नामाध्यायश्चतुर्थः}\lem \mssCaCbCc\msNa\msNb\msNc, 
नामश्चतुर्थो ऽध्यायः \Ed}}% 
