\fejno=0\versno=0
\centerline{\Huge\devanagarifontbold वृषसारसंग्रहः  }

 
{\vrule depth10pt width0pt}
\versno=0\fejno=1
\thispagestyle{empty}

\centerline{\Large\devanagarifontbold [   प्रथमो ऽध्यायः  ]}{\vrule depth10pt width0pt} \fancyhead[CO]{{\footnotesize\devanagarifont वृषसारसंग्रहे  }}
\fancyhead[CE]{{\footnotesize\devanagarifont प्रथमो ऽध्यायः  }}
\fancyhead[LE]{}
\fancyhead[RE]{}
\fancyhead[LO]{}
\fancyhead[RO]{}
\szam\bek


\vers



\alalfejezet{स्तुतिः}
\ujvers\nemsloka {
{\devanagarifont अनादिमध्यान्तमनन्तपारं }%
  \dontdisplaylinenum}    \var{{\devanagarifont \numemph\va\textbf{॰न्तमनन्त॰}\lem \mssALL, 
॰न्तमन्त॰ \msCbacorr\oo 
\textbf{॰पारं}\lem \mssCaCbCc\msNc\msM\msPaperA\msPaperC\Ed, ॰पारगं \msNa\msNb\msNd\msKOa}}% 
    \paral{{\devanagarifontsmall \va {\englishfont \compare\ \SDHU\ 10.6:}
                 आदिमध्यान्तनिर्मुक्तः स्वभावविमलः प्रभुः\thinspace{\devanagarifontsmall ।}
                 सर्वज्ञः परिपूर्णश्च शिवो ज्ञेयः शिवागमे\thinspace{\devanagarifontsmall ॥} }}


\nemslokab

{\devanagarifont सुसूक्ष्ममव्यक्तजगत्सुसारम्  \danda\dontdisplaylinenum }%
     \var{{\devanagarifont \numnoemph\vb\textbf{सुसूक्ष्म॰}\lem \mssALL, 
शुसुक्ष्म॰ \msCc\oo 
\textbf{॰व्यक्त॰}\lem \mssALL, ॰व्य॰ \msKOa\oo 
\textbf{॰जगत्सुसारम्}\lem \msCa\msCb\msNa\msNc\msM\msKOa\msPaperA\msPaperC\Ed, ॰जगशुसारं \msCc, 
॰जगत्सुरासुरं \msNb, 
॰जगतसुसारम् \msNd}}% 

\nemslokac

{\devanagarifont हरीन्द्रब्रह्मादिभिरासमग्रं }%
  \dontdisplaylinenum    \var{{\devanagarifont \numnoemph\vc\textbf{हरी॰}\lem \mssALL, हरीं \msKOa\oo 
\textbf{॰भिरासमग्रं}\lem \mssALL, 
॰भिर्यत्समग्रं \msM\ \unmetr, 
॰भिरोसमग्रं \msPaperC}}% 

%Verse 1:1


\nemslokad

{\devanagarifont प्रणम्य वक्ष्ये वृषसारसंग्रहम् {॥ १:१॥} \veg\dontdisplaylinenum }%
     \var{{\devanagarifont \numnoemph\vd\textbf{वृष॰}\lem \mssALL, 
॰वृषो \msCaacorr}}% 
    \lacuna{\devanagarifontsmall {\englishfont Witnesses used for this chapter:       \msCa\ ff.\thinspace 193v--195v,
                                                        \msCb\ ff.\thinspace 201v--203v,
                                                        \msCc\ ff.\thinspace 267r--270r,
                                                        \msNa\ ff.\thinspace 1v--3v,
                                                        \msNb\ exp.\thinspace 44, 43 lower and then upper leaf
                                                                              (1.62cd--2.22 are missing),
                                                        \msNc\  ff.\thinspace 209v--211v,
                                                        \msNd\  ff.\thinspace 227v--229v 
                                                                                (collated only up to 1.15ab),
                                                        \msM\   ff.\thinspace 1r--3v,
                                                        \msKOa\ ff.\thinspace 1v--4r 
                                                                                (collated only up to 1.16),
                                                        \msPaperA\ ff.\thinspace 204r--206r,
                                                        \msPaperC\ ff.\thinspace 206r--209r 
                                                                                (collated only up to 1.15),
                                                        \Ed\ pp.\thinspace 580--585;
                                                        \mssCaCbCc\ = \msCa + \msCb + \msCc} }%
  

\alalfejezet{जनमेजयवैशम्पायनसंवादः}
\vers


{\devanagarifont शतसाहस्रिकं ग्रन्थं सहस्राध्यायमुत्तमम् \thinspace{\dandab} \dontdisplaylinenum }%
     \var{{\devanagarifont \numemph\va\textbf{॰स्रिकं}\lem \mssALL, 
॰स्रकं \msPaperA\oo 
\textbf{ग्रन्थं}\lem \mssALL, 
ग्रंथ \msKOa}}% 
    \var{{\devanagarifont \numnoemph\vb\textbf{सहस्राध्यायमु॰}\lem \mssALL, 
सहश्रध्यायमु॰ \msCc, 
सहस्राध्यायरु॰ \Ed}}% 

%Verse 1:2

{\devanagarifont पर्व चास्य शतं पूर्णं श्रुत्वा भारतसंहिताम् {॥ १:२॥} \veg\dontdisplaylinenum }%
     \var{{\devanagarifont \numnoemph\vc\textbf{पर्व चास्य}\lem \msCa\msNa\msNb\msNc\msMpcorr, पर्वञ्चास्य \msCb, 
पर्वमस्य \msCc\msNd\msMacorr\msPaperA\msPaperC\Ed, पूर्व चास्य \msKOa\oo 
\textbf{शतं पूर्णं}\lem \mssALL, 
त \msCc, शतं पूर्ण्ण \msKOa}}% 
    \var{{\devanagarifont \numnoemph\vd\textbf{श्रुत्वा}\lem \mssALL, 
श्रद्धा \msCb\oo 
\textbf{भारतसंहिताम्}\lem \msCa\msCb\msNa\msNb\msNc\msM\msKOa, 
भारसंहिता \msCc, भारतसंहितं \msNd, 
नारदसंहिताम् \msPaperA\msPaperC\Ed}}% 
    \paral{{\devanagarifontsmall \vc {\englishfont \compare\ \MBH\ 1.2.70ab:} एतत्पर्वशतं पूर्णं व्यासेनोक्तं महात्मना }}

\vers


{\devanagarifont अतृप्तः पुन पप्रच्छ वैशम्पायनमेव हि \thinspace{\dandab} \dontdisplaylinenum }%
     \var{{\devanagarifont \numemph\va \lem \eme, 
अ\uncl{तृप्तः पु}\lk\lk प्रच्छ \msCa, 
अतृप्तः पुनः पप्रच्छ \msCb\msNa\msNb\msNc, 
अतृप्तः पुनरप्रच्छे \msCc, 
अतृप्तः पुन पःप्रच्छ \msNd, 
अतृप्तः पुनः पपृच्छ \msM, 
पप्रच्छ पुनरतृप्तो \msKOa, 
अतृप्ताः पुनः पप्रेच्छ \msPaperA, 
अतृप्त पुनः पप्रच्छ \msPaperC, 
अतृप्ता पुनः पप्रच्छ \Ed}}% 
    \var{{\devanagarifont \numnoemph\vb\textbf{वैशम्पायन॰}\lem \mssALL, 
वेसम्पायन॰ \msCc}}% 

%Verse 1:3

{\devanagarifont जनमेजयेन यत्पूर्वं तच्छृणु त्वमतन्द्रितम् {॥ १:३॥} \veg\dontdisplaylinenum }%
     \var{{\devanagarifont \numnoemph\vc \lem \msCapcorr\msCb\msNc\msNd\msPaperA\msPaperC\Ed, 
जनमेजये यत्पूर्वं \msCaacorr, 
जन्मेजयेन यम्पूर्वं \msCc, 
जनमेजयेन यत्पूर्व \msNa, 
जनमेजयेन यत्पू\uncl{र्व} \msNb, 
जन्मेजयेण यत्पूर्वं \msM, 
जन्मेजयेन य\lac\ \msKOa}}% 
    \var{{\devanagarifont \numnoemph\vd\textbf{तच्छृणु त्वम॰}\lem \msCa\msCb\msNa\msNc\msM\msPaperA\msPaperC\Ed, 
तच्छृण त्वम॰ \msCc, \lac\  \msNb, तच्छृणु स्वम॰ \msNd, 
त शृणु त्वम॰ \msKOa\oo 
\textbf{॰तन्द्रितम्}\lem \msCa\msCb\msNc\msNd\msM\msKOa\msPaperA\msPaperC\Ed, ॰तन्द्रितः \msCc\msNa, 
\lac\  \msNb}}% 

{\devanagarifont जनमेजय उवाच {\dandab}\dontdisplaylinenum  }%
     \var{{\devanagarifont \numemph\vo\textbf{जनमेजय}\lem \mssALL, 
जन्मेजय \msCc}}% 

{\devanagarifont भगवन्सर्वधर्मज्ञ सर्वशास्त्रविशारद \thinspace{\danda} \dontdisplaylinenum }%
     \var{{\devanagarifont \numnoemph\va\textbf{भगवन्स॰}\lem \msCa\msCb\msNa\msNb\msNc\msKOa\msPaperA\msPaperC\Ed, 
भचावं स॰ \msCc, भगव स॰ \msNd, 
भगवं स॰ \msM\oo 
\textbf{॰धर्मज्ञ}\lem \mssALL, 
॰ज्ञ \msNa, ॰धर्मज्ञः \msNd}}% 
    \var{{\devanagarifont \numnoemph\vb\textbf{॰विशारद}\lem \msCa\msNb\msNc\msNd\msPaperA, 
॰विसारदः \msCb\msCc\msNa\msKOa\msPaperC\Ed, ॰विशारदम् \msM}}% 
    \paral{{\devanagarifontsmall \vab {\englishfont = \MBH\ 13.112.9ab} }}

%Verse 1:4

{\devanagarifont अस्ति धर्मं परं गुह्यं संसारार्णवतारणम् {॥ १:४॥} \veg\dontdisplaylinenum }%
     \var{{\devanagarifont \numnoemph\vc\textbf{अस्ति धर्मं}\lem \msCa\msNa\msNb\msNc\msPaperA\msPaperC\Ed, अस्ति धर्मः \msCb, 
अस्ति धर्म \msCc\msM\msKOa, अधर्म \msNd\oo 
\textbf{परं गुह्यं}\lem \msCa\msNb\msNd\msM\msKOa\msPaperA\msPaperC\Ed, 
परो गुह्य \msCb, परं गुह्य \msCc\msNa, 
परगुह्यं \msNc}}% 
    \var{{\devanagarifont \numnoemph\vd\textbf{॰तारणम्}\lem \mssALL, 
॰तारणा \msKOa}}% 

{\devanagarifont द्वैपायनमुखोद्गीर्णं धर्मं वा यद्द्विजोत्तम \thinspace{\dandab} \dontdisplaylinenum }%
     \var{{\devanagarifont \numemph\va\textbf{द्वैपायन॰}\lem \mssALL, 
द्वेपायन॰ \msCc, 
वैसांपायन॰ \msKOa\oo 
\textbf{॰मुखोद्गीर्णं}\lem \msCa\msCb\msNa\msNb\msNc\msPaperA\msPaperC, 
॰मुखोद्गीर्ण \msCc\msKOa, 
॰मुद्गीर्ण्ण \msNd, 
मुखं गीर्ण्णं \msMacorr, 
मु\uncl{खां} गीर्ण्णं \msMpcorr, 
मुखाद्गीर्णं \Ed}}% 
    \var{{\devanagarifont \numnoemph\vb\textbf{धर्मं वा यद्द्वि॰}\lem \msCa\msNa\msNb\msNc\msPaperA\msPaperC\Ed, 
धर्मं यत्तद्द्वि॰ \msCb, 
धर्मवत्य द्वि॰ \msCc\msKOa, धर्म वा यद्द्वि॰ \msNd, 
धर्मवाक्यं द्वि॰ \msM\oo 
\textbf{॰त्तम}\lem \mssALL, 
॰त्तमः \msCc, ॰तमः \msM}}% 

%Verse 1:5

{\devanagarifont कथयस्व हि मे तृप्तिं कुरु यत्नात्तपोधन {॥ १:५॥} \veg\dontdisplaylinenum }%
     \var{{\devanagarifont \numnoemph\vc\textbf{हि मे तृप्तिं}\lem \mssCaCbCc\msNa\msNb\msNc\msPaperA\msPaperC\Ed, 
हि मे तृप्ति \msNd\msKOa, 
प्रसादेन \msM}}% 
    \var{{\devanagarifont \numnoemph\vd\textbf{यत्नात्तपोधन}\lem \msCb\msNa\msNb\msNc\msPaperA\msPaperC\Ed, 
यन्नात्त\lk\lk न \msCa, 
यत्ना तपोधनः \msCc, यत्ना तपोधन \msNd, 
यत्नन्तपोधन \msM, यंनात्त॰ \msKOa}}% 

{\devanagarifont वैशम्पायन उवाच {\dandab}\dontdisplaylinenum  }%
     \var{{\devanagarifont \numemph\vo\textbf{वैशम्पायन उवाच}\lem \mssALL, 
\om\ \msMacorr, वै\thinspace{\devanagarifont ॥} वैशम्पायन \msPaperC}}% 

{\devanagarifont शृणु राजन्नवहितो धर्माख्यानमनुत्तमम् \thinspace{\danda} \dontdisplaylinenum }%
     \var{{\devanagarifont \numnoemph\va\textbf{राजन्न॰}\lem \mssALL, 
राजंन॰ \msNd, राजन॰ \msM\oo 
\textbf{॰हितो}\lem \mssALL, 
॰हितं \msPaperA}}% 
    \var{{\devanagarifont \numnoemph\vb\textbf{॰ख्यानमनुत्तमम्}\lem \msCa\msNa\msNb\msNc\msM\Ed, ॰ख्यानमुत्तमम् \msCb, 
॰ख्यानमुतमम् \msCc, ॰धर्मव्याख्यानमुत्तमं \msNd\ \hypermetr, 
॰ख\lac मनुत्तमं \msKOa, 
॰ख्यानमनुत्तमः \msPaperA, 
॰ख्यानमुत्तमः \msPaperC}}% 

%Verse 1:6

{\devanagarifont व्यासानुग्रहसम्प्राप्तं गुह्यधर्मं शृणोतु मे {॥ १:६॥} \veg\dontdisplaylinenum }%
     \var{{\devanagarifont \numnoemph\vc\textbf{॰प्राप्तं}\lem \mssALL, 
॰प्राप्त \msCc}}% 
    \var{{\devanagarifont \numnoemph\vd\textbf{॰धर्मं}\lem \mssALL, 
॰र्मं \msCc, ॰धर्म \msKOa\oo 
\textbf{शृणोतु}\lem \mssALL, 
शृणोत \msCc\oo 
\textbf{मे}\lem \mssALL, 
मै \msCb}}% 

{\devanagarifont अनर्थयज्ञकर्तारं तपोव्रतपरायणम् \thinspace{\dandab} \dontdisplaylinenum }%
     \var{{\devanagarifont \numemph\va\textbf{॰कर्तारं}\lem \mssALL, 
॰कर्त्तन्तं \msNb}}% 
    \var{{\devanagarifont \numnoemph\vb\textbf{॰व्रत॰}\lem \mssALL, 
॰प्रत॰ \msM\oo 
\textbf{॰यणम्}\lem \msCa\msCb\msNb\msM\msKOa\msPaperA\msPaperC\Ed, 
॰यन \msCc, ॰यणः \msNa, 
॰यनं \msNc, ॰\uncl{यणं} \msNd}}% 

%Verse 1:7

{\devanagarifont शीलशौचसमाचारं सर्वभूतदयापरम् {॥ १:७॥} \veg\dontdisplaylinenum }%
     \var{{\devanagarifont \numnoemph\vc\textbf{॰चारं}\lem \mssALL, 
॰चार \msKOa}}% 
    \var{{\devanagarifont \numnoemph\vd\textbf{॰परम्}\lem \msCa\msCb\msNa\msNc\msM\msPaperA\msPaperC\Ed, ॰न्वितम् \msCc\msNd\msKOa, 
॰\uncl{प}रं \msNb}}% 

{\devanagarifont जिज्ञासनार्थं प्रश्नैकं विष्णुना प्रभविष्णुना \thinspace{\dandab} \dontdisplaylinenum }%
     \var{{\devanagarifont \numemph\va\textbf{॰र्थं प्रश्नैकं}\lem \msCb\msNa\msNb\msNc, ॰र्थं प्रश्नेकं \msCa\msNd, 
॰र्थप्रश्नेकम् \msCc\msPaperA\msPaperC\Ed, ॰र्थप्रश्चैकं \msM, 
॰थप्रश्नैक \msKOa}}% 
    \var{{\devanagarifont \numnoemph\vb\textbf{प्रभ॰}\lem \mssALL, 
प्रभु॰ \msCc, प्राभ॰ \msNc}}% 

%Verse 1:8

{\devanagarifont द्विजरूपधरो भूत्वा पप्रच्छ विनयान्वितः {॥ १:८॥} \veg\dontdisplaylinenum }%
     \var{{\devanagarifont \numnoemph\vc\textbf{॰धरो}\lem \mssALL, 
॰\lk रो \msCa, ॰धरा \msNb}}% 
    \var{{\devanagarifont \numnoemph\vd\textbf{॰न्वितः}\lem \msCa\msCb\msNa\msNb\msNc\msKOa\msPaperA\msPaperC\Ed, 
॰न्वितं \msCc\msNd\msM}}% 


\alalfejezet{ब्रह्मविद्या}
{\devanagarifont [विगतराग उवाच {\dandab}\dontdisplaylinenum  ] }%
 
{\devanagarifont ब्रह्मविद्या कथं ज्ञेया रूपवर्णविवर्जिता \thinspace{\danda} \dontdisplaylinenum }%
     \var{{\devanagarifont \numemph\va\textbf{कथं}\lem \mssALL, कथ \msKOa\oo 
\textbf{ज्ञेया}\lem \msCa\msNa\msNb\msNc\msM\msKOa\msPaperA\msPaperC, 
ज्ञेयं \msCb\msCc, ज्ञेय \msNd, भूयो \Ed}}% 
    \var{{\devanagarifont \numnoemph\vb\textbf{॰वर्ण॰}\lem \mssALL, 
॰वर्णा॰ \Ed\oo 
\textbf{॰वर्जिता}\lem \msCa\msCb\msNa\msNb\msNd\msM\msPaperA\msPaperC\Ed, 
॰वर्जितं \msCc, ॰वर्जिताः \msNc, \lac ता \msKOa}}% 

%Verse 1:9

{\devanagarifont स्वरव्यञ्जननिर्मुक्तमक्षरं किमु तत्परम् {॥ १:९॥} \veg\dontdisplaylinenum }%
     \var{{\devanagarifont \numnoemph\vc\textbf{॰व्यञ्जन॰}\lem \mssALL, 
॰व्यज्जन॰ \Ed}}% 
    \var{{\devanagarifont \numnoemph\vcd\textbf{॰मुक्तमक्ष॰}\lem \msCa\msCc\msNa\msNb\msNc\msPaperC\Ed, ॰मुक्त अक्ष॰ \msCb\msKOa, 
॰मुक्तं अख॰ \msNd, ॰मुक्तं अक्ष॰ \msM, ॰म्मुक्तंमक्ष॰ \msPaperA}}% 
    \var{{\devanagarifont \numnoemph\vd\textbf{किमु तत्परम्}\lem \msCa\msNa\msNc\msKOa\msPaperA\msPaperC\Ed, 
किमतः परम् \msCb\msCc, 
किमतत्परं \msNb\msNd\msM}}% 

{\devanagarifont अनर्थयज्ञ उवाच {\dandab}\dontdisplaylinenum  }%
 
{\devanagarifont अनुच्चार्यमसन्दिग्धमविच्छिन्नमनाकुलम् \thinspace{\danda} \dontdisplaylinenum }%
     \var{{\devanagarifont \numemph\va\textbf{अनुच्चार्य॰}\lem \msCa\msCb\msNa\msNb\msM\msPaperA\msPaperC\Ed, 
अनुचार्य॰ \msCc\msNc\msNd, 
अन्त्रचाय॰ \msKOa}}% 
    \var{{\devanagarifont \numnoemph\vab\textbf{॰सन्दिग्धमविच्छिन्नमनाकुलम्}\lem \msCa\msCb\msNa\msNc\msNd\msM\msPaperA\msPaperC\Ed, 
॰विच्छिन्नसन्दिग्धमनाकुन \msCc, ॰सन्दिग्धमनच्छिन्नमनाकुलम् \msNb, 
॰सन्दिग्धमविच्छिनमनाकुलं \msKOa}}% 

%Verse 1:10

{\devanagarifont निर्मलं सर्वगं सूक्ष्ममक्षरं किमतः परम् {॥ १:१०॥} \veg\dontdisplaylinenum }%
     \var{{\devanagarifont \numnoemph\vc\textbf{॰गं}\lem \mssALL, ॰ग \msKOa}}% 
    \var{{\devanagarifont \numnoemph\vc\textbf{॰क्षरं किमतः परम्}\lem \msCb\msM, ॰क्षरं किमु तत्परम् \msCa\msNa\msNb\msNc\Ed, 
॰क्षरं किमतत्परं \msCc\msNd\msPaperC, 
॰क्षर किमतः परं \msKOa, 
॰क्षराङ्कमतत्परं \msPaperA}}% 


\alalfejezet{कालपाशः}
{\devanagarifont विगतराग उवाच {\dandab}\dontdisplaylinenum  }%
     \var{{\devanagarifont \numemph\vo\textbf{॰राग उवाच}\lem \mssALL, ॰रागोवाच \msNd}}% 

{\devanagarifont देही देहे क्षयं याते भूजलाग्निशिवादिभिः \thinspace{\danda} \dontdisplaylinenum }%
     \var{{\devanagarifont \numnoemph\va\textbf{देहे क्ष॰}\lem \msCa\msCc\msNc, देहात्क्ष॰ \msCb, 
देहक्ष॰ \msNa\msNb\msNd\msM\msKOa\msPaperA\msPaperC\Ed\oo 
\textbf{याते}\lem \mssALL, यान्ते \msNd}}% 
    \var{{\devanagarifont \numnoemph\vb\textbf{॰जलाग्निशिवादिभिः}\lem \msCa\msCb\msNa\msNb\msNc\msM\msPaperA\msPaperC\Ed, 
॰जलाग्निशिवादिभि \msCc, 
॰जलाग्निं शि\lk दिभि \msNd, ॰जालादिशिवादिभिः \msKOa}}% 
    \paral{{\devanagarifontsmall \vb {\englishfont \compare\ \KURMP\ 2.23.74:} 
                 अथ कश्चित्प्रमादेन म्रियते ऽग्निविषादिभिः\thinspace{\devanagarifontsmall ।} 
                 तस्याशौचं विधातव्यं कार्यं चैवोदकादिकम्\thinspace{\devanagarifontsmall ॥} }}

%Verse 1:11

{\devanagarifont यमदूतैः कथं नीतो निरालम्बो निरञ्जनः {॥ १:११॥} \veg\dontdisplaylinenum }%
     \var{{\devanagarifont \numnoemph\vc\textbf{॰दूतैः}\lem \mssALL, 
॰दूते \msCc\msNd\oo 
\textbf{कथं}\lem \mssALL, 
कथ \msKOa\oo 
\textbf{नीतो}\lem \msCa\msCb\msNa\msNb\msNc\msNd, नीत्वा \msCc, नीतः \msM, नीते \msKOa, 
नीता \msPaperA\msPaperC\Ed}}% 
    \var{{\devanagarifont \numnoemph\vd\textbf{निरालम्बो}\lem \mssALL, 
निरोलया \msPaperA, निरोरैन्वो \msPaperC\oo 
\textbf{निरञ्जनः}\lem \mssALL, 
निरञ्जन \msCc, 
निरञ्ज\lk\ \msKOa}}% 

{\devanagarifont कालपाशैः कथं बद्धो निर्देहश्च कथं व्रजेत् \thinspace{\dandab} \dontdisplaylinenum }%
     \var{{\devanagarifont \numemph\va\textbf{॰पाशैः}\lem \mssALL, 
॰पाशे \msCc, ॰पाशै \msNd\oo 
\textbf{बद्धो}\lem \mssALL, 
ब\uncl{द्धो} \msCb, बद्ध \msNd}}% 
    \var{{\devanagarifont \numnoemph\vb\textbf{निर्देहश्च}\lem \msCa\msCb\msNa\msNb\msNc\msMpcorr\msPaperA\msPaperC\Ed, 
निर्दहः स \msCc, निर्देहस्य \msNd, 
निर्देहन्म \msMacorr, निदेहश्च \msKOa\oo 
\textbf{व्रजेत्}\lem \mssALL, भवेत् \msNb}}% 

{\devanagarifont स्वर्गं वा स कथं याति निर्देहो बहुधर्मकृत्  \danda\dontdisplaylinenum }%
     \var{{\devanagarifont \numnoemph\vc\textbf{स्वर्गं}\lem \msCa\msCb\msNa\msNb\msNc\msPaperA\msPaperC\Ed, 
स्वर्ग \msCc\msNd\msM, स्वागं \msKOa\oo 
\textbf{स}\lem \mssALL, 
सं \msNb\msM\oo 
\textbf{याति}\lem \msNa\msNb\msNc\msNd\msM\msKOa\msPaperA\msPaperC, 
यान्ति \mssCaCbCc\Ed}}% 
    \var{{\devanagarifont \numnoemph\vd\textbf{निर्देहो}\lem \mssALL, 
निदेहो \msKOa}}% 

%Verse 1:12

{\devanagarifont एतन्मे संशयं ब्रूहि ज्ञातुमिच्छामि तत्त्वतः {॥ १:१२॥} \veg\dontdisplaylinenum }%
     \var{{\devanagarifont \numnoemph\ve\textbf{एतन्मे संशयं}\lem \mssCaCbCc\msNc\msM\msPaperA\msPaperC\Ed, 
एतन्मे संशये \msNa, एतन्मे संशयो \msNb\msNd, 
एवं विस्मयसंसय \msKOa}}% 
    \var{{\devanagarifont \numnoemph\vf\textbf{॰तुमि$\-$च्छामि}\lem \mssALL, 
॰तुमि \msCb}}% 

{\devanagarifont अनर्थयज्ञ उवाच {\dandab}\dontdisplaylinenum  }%
     \var{{\devanagarifont \numemph\vo\textbf{अनर्थयज्ञ उवाच}\lem \mssALL, 
\om\ \msNaacorr}}% 

{\devanagarifont अतिसंशयकष्टं ते पृष्टो ऽहं द्विजसत्तम \thinspace{\danda} \dontdisplaylinenum }%
     \var{{\devanagarifont \numnoemph\va \lem \msCb\msNa\msNb\msNc\msMpcorr\msPaperC, 
अतिशंस$\-$\uncl{य}कष्टन्ते \msCa, 
अतिशंसयक$\-$ष्टम्मे \msCc\msMacorr\Ed, 
अतिसंशयकष्टो मो \msNd, 
अतिसंसयकष्टञ्च \msKOa, 
अतिसंसयकष्ट\lk न्ते पा \msPaperA}}% 
    \var{{\devanagarifont \numnoemph\vb\textbf{द्विजसत्तम}\lem \msCa\msCb\msNa\msNb\msNc\msM\msPaperA\msPaperC\Ed, 
च द्विजोत्तमः \msCc\msKOa, द्विजसत्तमः \msNd}}% 

%Verse 1:13

{\devanagarifont दुर्विज्ञेयं मनुष्यैस्तु देवदानवपन्नगैः {॥ १:१३॥} \veg\dontdisplaylinenum }%
     \var{{\devanagarifont \numnoemph\vc\textbf{॰ज्ञेयं}\lem \msCa\msCb\msNa\msNc, ॰ज्ञेय \msCc\msNb\msNd\msM\msKOa\msPaperA\msPaperC\Ed\oo 
\textbf{मनुष्यैस्तु}\lem \msCa\msNa\msNb\msNc\msM\msKOa\msPaperA\msPaperC\Ed, 
मनुषैश्च \msCb, मणुक्षे\uncl{प्तु} \msCc, 
मनुष्येस्तु \msNd}}% 

{\devanagarifont कर्महेतु शरीरस्य उत्पत्ति निधनं च यत् \thinspace{\dandab} \dontdisplaylinenum }%
     \var{{\devanagarifont \numemph\va\textbf{कर्म॰}\lem \msCa\msCb\msNa\msNb\msNc\msNd\msM\msKOa, 
अनर्थयज्ञ उवाच\thinspace{\devanagarifont ॥} कर्म॰ \msCc\msPaperA\msPaperC\Ed\oo 
\textbf{॰हेतु}\lem \mssALL, 
॰हेतुः \msCb, ॰हेंतु \msCc\oo 
\textbf{शरीरस्य}\lem \mssALL, 
शरीरस्यं \msCc, 
स\lac \uncl{स्य} \msKOa}}% 
    \var{{\devanagarifont \numnoemph\vb\textbf{उत्पत्ति नि॰}\lem \msCa\msCb\msNa\msNb\msNc\msKOa\msPaperA\msPaperC\Ed, 
उत्पतिनि॰ \msCc\msNd, उत्पत्तिर्नि॰ \msM\oo 
\textbf{च यत्}\lem \mssALL, 
च यः \msNb, यत् \msNd}}% 

%Verse 1:14

{\devanagarifont सुकृतं दुष्कृतं चैव पाशद्वयमुदाहृतम् {॥ १:१४॥} \veg\dontdisplaylinenum }%
     \var{{\devanagarifont \numnoemph\vc\textbf{सुकृतं}\lem \mssALL, 
सुकृतकृतन् \msCc, सुकृत \msNd\oo 
\textbf{चैव}\lem \mssALL, वापि \msNd\msKOa}}% 
    \var{{\devanagarifont \numnoemph\vd\textbf{पाश॰}\lem \mssALL, पासा॰ \msKOa\oo 
\textbf{॰हृतम्}\lem \mssALL, 
॰हृतः \msCc}}% 

{\devanagarifont तेनैव सह संयाति नरकं स्वर्गमेव वा \thinspace{\dandab} \dontdisplaylinenum }%
     \var{{\devanagarifont \numemph\va\textbf{तेनैव}\lem \mssALL, 
तेनेव \msCc\msNd\oo 
\textbf{सह संयाति}\lem \msCa\msCb\msNa\msNb\msNc\msPaperC\Ed, 
सह सा यान्ति \msCc\msNd, सह सा याति \msM, 
सह संयान्ति \msKOa, सहं स याति \msPaperA}}% 
    \var{{\devanagarifont \numnoemph\vb\textbf{नरकं स्वर्ग॰}\lem \mssALL, 
नरकदुर्ग्ग॰ \msKOa\oo 
\textbf{वा}\lem \mssCaCbCc\msNb\msNc\msM\msPaperA\msPaperC\Ed, च \msNa\msNd\msKOa}}% 

%Verse 1:15

{\devanagarifont सुखदुःखं शरीरेण भोक्तव्यं कर्मसम्भवम् {॥ १:१५॥} \veg\dontdisplaylinenum }%
     \var{{\devanagarifont \numnoemph\vc\textbf{सुख॰}\lem \mssALL, सुखं \msM\oo 
\textbf{॰दुःखं}\lem \msCa\msCb\msNa\msNc\msM, ॰दुःख \msCc\msNb\msKOa\msPaperA\msPaperC\Ed}}% 
    \var{{\devanagarifont \numnoemph\vd\textbf{भोक्तव्यं}\lem \mssALL, 
भोक्तव्य \msKOa\oo 
\textbf{॰सम्भवम्}\lem \msCa\msCb\msNa\msNb\msNc\msM, 
॰सम्भवः \msCc\msPaperA\msPaperC\Ed, ॰संभावात् \msKOa}}% 

{\devanagarifont हेतुनानेन विप्रेन्द्र देहः सम्भवते नृणाम् \thinspace{\dandab} \dontdisplaylinenum }%
     \var{{\devanagarifont \numemph\va\textbf{हेतुनानेन}\lem \mssALL, 
हेतुना तेन \msKOa, हेतुनाने \msPaperCacorr\oo 
\textbf{॰न्द्र}\lem \mssALL, ॰न्द्रः \msNb}}% 
    \var{{\devanagarifont \numnoemph\vb\textbf{देहः}\lem \msCa\msCb\msNa\msNc\Ed, देहे \msCc, देह \msNb\msM\msKOa\msPaperA, 
देहं \msPaperC\oo 
\textbf{नृणाम्}\lem \mssALL, नृणा \msCb\msCc}}% 

%Verse 1:16

{\devanagarifont यं कालपाशमित्याहुः शृणु वक्ष्यामि सुव्रत {॥ १:१६॥} \veg\dontdisplaylinenum }%
     \var{{\devanagarifont \numnoemph\vc \lem \eme, यं कालपाशमित्याह \msCa\msCb\msNa, 
कालपासेति सत्वाह \msCc, यं कालपाशमित्याहु \msNb\msNc\msPaperA\Ed, 
कालपाषेति \uncl{पस्त्वे}ह \msM, 
यां कालपासमित्याहु \msKOa}}% 
    \var{{\devanagarifont \numnoemph\vd\textbf{॰व्रत}\lem \msCa\msNa\msNb\msNc\msM\msPaperA\Ed, ॰व्रतः \msCb\msCc\msKOa}}% 

{\devanagarifont न त्वया विदितं किञ्चिज्जिज्ञास्यसि कथं द्विज \thinspace{\dandab} \dontdisplaylinenum }%
     \var{{\devanagarifont \numemph\va\textbf{विदितं}\lem \mssALL, विदित \msCc}}% 
    \var{{\devanagarifont \numnoemph\vab\textbf{किञ्चिज्जि॰}\lem \msCb\msM, किञ्चिद्वि॰ \msCapcorr\msNa\msNb\msNc\msPaperA\Ed, 
किद्वि॰ \msCaacorr, 
किञ्चि जि॰ \msCc}}% 
    \var{{\devanagarifont \numnoemph\vb\textbf{कथं द्विज}\lem \mssALL, 
\lk\lk\lk\lk\lk\lk\lk\lk\lk  \uncl{म त्वया विदितं किञ्चिद्विज्ञास्यसि} 
\cancelled\ कथं द्विज \msCc}}% 

%Verse 1:17

{\devanagarifont कालपाशं च विप्रेन्द्र सकलं वेत्तुमर्हसि {॥ १:१७॥} \veg\dontdisplaylinenum }%
     \var{{\devanagarifont \numnoemph\vc\textbf{कालपाशं च}\lem \mssALL, कालपाषेति \msM}}% 
    \var{{\devanagarifont \numnoemph\vd\textbf{वेत्तुमर्हसि}\lem \mssCaCbCc\msNa\msNb, 
वेत्तुमूहसि \msNc, वक्तुमर्हसि \msM\msPaperA\Ed}}% 

{\devanagarifont कलाकलितकालं च कालतत्त्वकलां शृणु \thinspace{\dandab} \dontdisplaylinenum }%
     \var{{\devanagarifont \numemph\va\textbf{कला॰}\lem \mssALL, काला॰ \msCc\msNaacorr\oo 
\textbf{॰कलित॰}\lem \mssALL, ॰\uncl{कन्मित}॰ \msPaperA\oo 
\textbf{॰कालं च}\lem \mssALL, ॰कालश्च \msM\Ed}}% 
    \var{{\devanagarifont \numnoemph\vb\textbf{॰कलां}\lem \msCa\msCc\msNb\msPaperA\Ed, ॰कला \msCb\msNc, ॰विधिं \msNa, ॰कलाः \msM}}% 

%Verse 1:18

{\devanagarifont त्रुटिद्वयं निमेषस्तु निमेषद्विगुणा कला {॥ १:१८॥} \veg\dontdisplaylinenum }%
     \var{{\devanagarifont \numnoemph\vc\textbf{त्रुटिद्वयं}\lem \msCa\msCc\msNc\Ed, तुटिद्वय \msCb\msNb, तुटिद्वयं \msNa\msM, 
त्रुविद्वयं \msPaperA\oo 
\textbf{॰मेषस्तु}\lem \mssALL, 
॰मेवस्तु \msCa, ॰मेषद्वि॰ \msNa}}% 
    \var{{\devanagarifont \numnoemph\vd\textbf{निमेषद्वि॰}\lem \mssALL, निमेषाद्वि॰ \msM}}% 

{\devanagarifont कलाद्विगुणिता काष्ठा काष्ठा वै त्रिंशतिः कला \thinspace{\dandab} \dontdisplaylinenum }%
     \var{{\devanagarifont \numemph\va\textbf{॰गुणिता काष्ठा}\lem \mssALL, ॰गुणितं काष्ठा \msM, 
॰गुणितं काष्ठी \msPaperA}}% 
    \var{{\devanagarifont \numnoemph\vb\textbf{काष्ठा वै त्रिंशतिः}\lem \msCa\msNa\msNb\msNc\msPaperA\Ed, वै त्रिंशता \msCb, 
काष्ठा वै त्रिंशति \msCc, काष्ठान्वै त्रिंशति \msM}}% 

%Verse 1:19

{\devanagarifont त्रिंशत्कला मुहूर्तश्च मानुषेन द्विजोत्तम {॥ १:१९॥} \veg\dontdisplaylinenum }%
     \var{{\devanagarifont \numnoemph\vc\textbf{मुहूर्तश्च}\lem \mssALL, 
मुहूर्त्त \msCb, मुहूर्तञ्च \Ed}}% 
    \var{{\devanagarifont \numnoemph\vd\textbf{मानुषेन}\lem \mssALL, मानु\uncl{षश्च} \msCc\oo 
\textbf{॰त्तम}\lem \mssCaCbCc\msNa\msNcpcorr\msPaperA\Ed, ॰तमः \msNb\msM, ॰त्तमः \msNcacorr}}% 

{\devanagarifont मुहूर्तत्रिंशकेनैव अहोरात्रं विदुर्बुधाः \thinspace{\dandab} \dontdisplaylinenum }%
     \var{{\devanagarifont \numemph\va\textbf{मुहूर्त॰}\lem \mssALL, मुहूर्त्ता \msM, मुहूर्तं \Ed}}% 
    \var{{\devanagarifont \numnoemph\vb\textbf{॰धाः}\lem \mssALL, ॰धा \msPaperA}}% 

%Verse 1:20

{\devanagarifont अहोरात्रं पुनस्त्रिंशन्मासमाहुर्मनीषिणः {॥ १:२०॥} \veg\dontdisplaylinenum }%
     \var{{\devanagarifont \numnoemph\vc\textbf{॰रात्रं}\lem \mssALL, ॰रात्र \msM}}% 
    \var{{\devanagarifont \numnoemph\vd\textbf{॰नीषिणः}\lem \mssALL, ॰नीषिन \msM}}% 

{\devanagarifont समा द्वादश मासाश्च कालतत्त्वविदो जनाः \thinspace{\dandab} \dontdisplaylinenum }%
     \var{{\devanagarifont \numemph\va\textbf{समा}\lem \mssALL, मास \msCc, समा समाया \msPaperA\oo 
\textbf{॰मासाश्च}\lem \msCa\msCb\msNa\msNb\msNc\msPaperA, ॰मासश्च \msCc\Ed, मासाहुः \msM}}% 
    \var{{\devanagarifont \numnoemph\vb\textbf{काल॰}\lem \mssALL, कला॰ \msNc}}% 

{\devanagarifont शतं वर्षसहस्राणि त्रीणि मानुषसंख्यया  \danda\dontdisplaylinenum }%
     \var{{\devanagarifont \numnoemph\vc\textbf{शतं}\lem   \mssALL,        शत॰ \msPaperA\Ed}}% 
    \var{{\devanagarifont \numnoemph\vd\textbf{मानुष॰}\lem \mssALL, माणुष्य॰ \msCb\msCc\ \unmetr}}% 

%Verse 1:21

{\devanagarifont षष्टिं चैव सहस्राणि कालः कलियुगः स्मृतः {॥ १:२१॥} \veg\dontdisplaylinenum }%
     \var{{\devanagarifont \numnoemph\ve\textbf{षष्टिं चैव}\lem \mssCaCbCc\msNc\msM, षष्टिं वर्ष॰ \msNa\msPaperA, षष्टिश्चैव \Ed}}% 
    \var{{\devanagarifont \numnoemph\vf\textbf{॰युगः}\lem       \mssALL, ॰युग \msM\Ed}}% 
    \lacuna{\devanagarifontsmall \vo {\englishfont \msNb\ omits verses 21ef--24ab} }%
  
{\devanagarifont द्विगुणः कलिसंख्यातो द्वापरो युग संज्ञितः \thinspace{\dandab} \dontdisplaylinenum }%
     \var{{\devanagarifont \numemph\va \lem \mssCaCbCc\msNa\msNc, कलिसंख्यास्तु द्विगुणो \msM, 
द्विगुर्णः कलिसंख्यातो \msPaperA, 
द्विगुणा कलिसंख्यातो \Ed}}% 
    \var{{\devanagarifont \numnoemph\vb \lem \mssALL, 
द्वापरः युगः संज्ञिकम् \msM, 
द्वापरे युग संज्ञितः \Ed}}% 

%Verse 1:22

{\devanagarifont त्रेता तु त्रिगुणा ज्ञेया चतुः कृतयुगः स्मृतः {॥ १:२२॥} \veg\dontdisplaylinenum }%
     \var{{\devanagarifont \numnoemph\vc\textbf{त्रेता}\lem   \msCa\msCb\msNa\msPaperA\Ed,              तेत्रा \msCc\msM, त्रेत्रा \msNc\oo 
\textbf{त्रिगुणा}\lem \mssALL,  तृगुणो \msM\oo 
\textbf{ज्ञेया}\lem   \mssALL,  ज्ञेयः \msM}}% 
    \var{{\devanagarifont \numnoemph\vd\textbf{॰युगः}\lem  \mssALL, ॰युग \Ed}}% 

{\devanagarifont एषा चतुर्युगासंख्या कृत्वा वै ह्येकसप्ततिः \thinspace{\dandab} \dontdisplaylinenum }%
     \var{{\devanagarifont \numemph\vb\textbf{ह्ये॰}\lem   \mssALL,   हे॰ \msNc\oo 
\textbf{॰सप्ततिः}\lem \mssALL, ॰सप्तति \msM}}% 

%Verse 1:23

{\devanagarifont मन्वन्तरस्य चैकस्य ज्ञानमुक्तं समासतः {॥ १:२३॥} \veg\dontdisplaylinenum }%
     \var{{\devanagarifont \numnoemph\vc\textbf{चैकस्य}\lem \mssALL, \om\ \msNaacorr\msMacorr}}% 
    \var{{\devanagarifont \numnoemph\vd\textbf{॰क्तं}\lem    \mssALL,                ॰क्त \msM}}% 

{\devanagarifont कल्पो मन्वन्तराणां तु चतुर्दश तु संख्यया \thinspace{\dandab} \dontdisplaylinenum }%
     \var{{\devanagarifont \numemph\va\textbf{कल्पो}\lem \msCb, कल्प \msCa\msCc\msNa\msNc\msM\msPaperA\Ed\oo 
\textbf{मन्वन्त॰}\lem \mssALL, 
न्वन्त॰ \msMacorr, मंन्वन्त॰ \msMpcorr}}% 
    \var{{\devanagarifont \numnoemph\vb\textbf{॰दश}\lem     \mssALL, ॰दशं \msCb\oo 
\textbf{संख्यया}\lem \mssALL,      शंक्षया \msM}}% 

{\devanagarifont दश कल्पसहस्राणि ब्रह्माहः परिकल्पितम्  \danda\dontdisplaylinenum }%
     \var{{\devanagarifont \numnoemph\vd\textbf{॰आहः}\lem \mssALL, ॰आह \msCa\oo 
\textbf{परिकल्पितम्}\lem \msCa\msNc, करिकल्पितम् \msCb, परिकल्पितः \msCc\msNb\msM\msPaperA\Ed, 
परिकीर्तिताः \msNa}}% 

%Verse 1:24

{\devanagarifont रात्रिरेतावती प्रोक्ता मुनिभिस्तत्त्वदर्शिभिः {॥ १:२४॥} \veg\dontdisplaylinenum }%
     \var{{\devanagarifont \numnoemph\vf\textbf{॰दर्शिभिः}\lem \mssALL, ॰दर्शिभि \msM}}% 

{\devanagarifont रात्र्यागमे प्रलीयन्ते जगत्सर्वं चराचरम् \thinspace{\dandab} \dontdisplaylinenum }%
     \var{{\devanagarifont \numemph\va\textbf{॰गमे}\lem      \mssALL,         ॰गम \msPaperA\oo 
\textbf{प्रलीयन्ते}\lem \mssALL, प्रलीयते \msCb}}% 
    \var{{\devanagarifont \numnoemph\vb\textbf{सर्वं च॰}\lem \mssALL,     सर्वश्च॰ \msM}}% 

%Verse 1:25

{\devanagarifont अहागमे तथैवेह उत्पद्यन्ते चराचरम् {॥ १:२५॥} \veg\dontdisplaylinenum }%
     \var{{\devanagarifont \numnoemph\vc\textbf{अहागमे}\lem \mssCaCbCc\msNa\msNc, अहाग\lac\  \msNb, 
अहरागमे \msM\ \unmetr, अहागम \msPaperA, अह्नागमे \Ed}}% 
    \var{{\devanagarifont \numnoemph\vd\textbf{॰पद्यन्ते}\lem \mssALL, ॰पद्यंति \msM}}% 

{\devanagarifont परार्धपरकल्पानि अतीतानि द्विजोत्तम \thinspace{\dandab} \dontdisplaylinenum }%
     \var{{\devanagarifont \numemph\va\textbf{॰र्ध॰}\lem \mssALL, ॰र्धं \msNb, ॰ध॰ \msPaperA}}% 

%Verse 1:26

{\devanagarifont अनागतं तथैवाहुर्भृगुरादिमहर्षयः {॥ १:२६॥} \veg\dontdisplaylinenum }%
     \var{{\devanagarifont \numnoemph\vcd\textbf{॰वाहुर्भृ॰}\lem \msCa\msCb\msNa\msNc\msPaperA\Ed, 
॰वाहु भृ॰ \msCc\msNb\msM}}% 
    \var{{\devanagarifont \numnoemph\vd\textbf{॰महर्षयः}\lem    \mssCaCbCc\msNapcorr\msNb\msPaperA\Ed, 
॰महयः \msNaacorr, ॰मर्हषयः \msNc, 
॰महर्षिभिः \msM}}% 

{\devanagarifont यथार्कग्रहतारेन्दु भ्रमतो दृश्यते त्विह \thinspace{\dandab} \dontdisplaylinenum }%
     \var{{\devanagarifont \numemph\va\textbf{॰आर्क॰}\lem   \mssALL, ॰आर्का॰ \msMacorr\oo 
\textbf{॰तारेन्दु}\lem \mssALL,          ॰तारैन्दु \msM}}% 
    \var{{\devanagarifont \numnoemph\vb\textbf{भ्रमतो}\lem \mssALL,                भुमनो \msPaperA\oo 
\textbf{दृश्यते त्विह}\lem \msCa\msNa\msNb\msNc\msPaperA\Ed, 
दृश्यन्दिह \msCb, दृस्यते त्विहः \msCc, 
दृश्यते त्विहः \msM}}% 

%Verse 1:27

{\devanagarifont कालचक्रं भ्रमित्वैव विश्रमं न च विद्महे {॥ १:२७॥} \veg\dontdisplaylinenum }%
     \var{{\devanagarifont \numnoemph\vc\textbf{भ्रमित्वैव}\lem \corr, भ्रमत्वैव \msCa\msNa\msNc\Ed, 
भ्रमत्वेव  \msCb\msNb\msM, भ्रमत्वेह \msCc, 
भ्रमत्यैव \msPaperA}}% 
    \var{{\devanagarifont \numnoemph\vd\textbf{॰श्रमं}\lem \mssCaCbCc\msNapcorr\msNc\msPaperA\Ed, 
॰श्रमो \msNaacorr, ॰श्रामन् \msNb, ॰श्रामो \msM\oo 
\textbf{विद्महे}\lem \mssALL, विग्रहे \msCb, विद्यते \msM}}% 

{\devanagarifont कालः सृजति भूतानि कालः संहरते पुनः \thinspace{\dandab} \dontdisplaylinenum }%
     \var{{\devanagarifont \numemph\vb\textbf{कालः}\lem \mssALL, काल \Ed}}% 
    \paral{{\devanagarifontsmall \vab {\englishfont \similar\ \UMS\ 12.34cd:}
                         कालः पचति भूतानि कालः संहरते प्रजाः }}

%Verse 1:28

{\devanagarifont कालस्य वशगाः सर्वे न कालवशकृत्क्वचित् {॥ १:२८॥} \veg\dontdisplaylinenum }%
     \var{{\devanagarifont \numnoemph\vc\textbf{कालस्य}\lem     \mssALL, कालःस्य \msMacorr\oo 
\textbf{वशगाः}\lem     \mssALL,         वशगा \Ed}}% 
    \var{{\devanagarifont \numnoemph\vd\textbf{कालवशकृ॰}\lem \mssALL,          कालो वशकृ॰ \msM}}% 
    \paral{{\devanagarifontsmall \vo \similar\ {\englishfont \KURMP\ 1.11.32:}
                 कालः सृजति भूतानि कालः संहरते प्रजाः\thinspace{\devanagarifontsmall ।}
                 सर्वे कालस्य वशगा न कालः कस्यचिद्वशे\thinspace{\devanagarifontsmall ॥} }}

{\devanagarifont चतुर्दश परार्धानि देवराजा द्विजोत्तम \thinspace{\dandab} \dontdisplaylinenum }%
     \var{{\devanagarifont \numemph\vb\textbf{देवराजा}\lem \mssALL,     देवराज \msM\Ed\oo 
\textbf{॰त्तम}\lem   \mssALL, ॰त्तमः \msM}}% 

%Verse 1:29

{\devanagarifont कालेन समतीतानि कालो हि दुरतिक्रमः {॥ १:२९॥} \veg\dontdisplaylinenum }%
     \paral{{\devanagarifontsmall \vd {\englishfont = \MBH\ 12.220.41d = \GARPUR\ 1.108.7d} }}

{\devanagarifont एष कालो महायोगी ब्रह्मा विष्णुः परः शिवः \thinspace{\dandab} \dontdisplaylinenum }%
     \var{{\devanagarifont \numemph\va\textbf{कालो}\lem \msCa\msCb\msNa,      काल \msCc\msNb\msNc\msM\msPaperA\Ed}}% 
    \var{{\devanagarifont \numnoemph\vb\textbf{ब्रह्मा विष्णुः परः}\lem \msCb, ब्रह्मविष्णुपरः \msCa\msNc\msM\msPaperA, 
ब्रह्मा विष्णु परः \msCc\msNa\msNb, 
ब्रह्मविष्णुपर \Ed\ \unmetr}}% 

%Verse 1:30

{\devanagarifont अनादिनिधनो धाता स महात्मा नमस्कुरु {॥ १:३०॥} \veg\dontdisplaylinenum }%
 

\alalfejezet{परार्धादि}
{\devanagarifont विगतराग उवाच {\dandab}\dontdisplaylinenum  }%
 
{\devanagarifont श्रुतं वै कालचक्रं तु मुखपद्मविनिःसृतम् \thinspace{\danda} \dontdisplaylinenum }%
     \var{{\devanagarifont \numemph\va\textbf{श्रुतं वै}\lem \mssALL, श्रुतो वः \msM\oo 
\textbf{॰चक्रं तु}\lem \mssALL, ॰चक्रस्य \msCc, ॰चक्रत्तु \msM}}% 
    \var{{\devanagarifont \numnoemph\vb\textbf{विनिःसृतम्}\lem \corr, विनिसृतम् \mssCaCbCc\msNa\msNb\msNc\msM\msPaperA\Ed\ \unmetr}}% 

%Verse 1:31

{\devanagarifont परार्धं च परं चैव श्रोतुं वः प्रतिदीपितम् {॥ १:३१॥} \veg\dontdisplaylinenum }%
     \var{{\devanagarifont \numnoemph\vc\textbf{परार्धं च}\lem \msCb\msCc\msNa\msNb\msNc\msPaperA\Ed, 
\uncl{प}रार्द्धं च \msCa, 
पराधञ्च \msMacorr, 
परार्धंञ्चे \msMpcorr\oo 
\textbf{परं चैव}\lem \mssALL,                पराञ्चैव \msM\msPaperA}}% 
    \var{{\devanagarifont \numnoemph\vd\textbf{वः}\lem         \mssALL, नः \msMpcorr, यः \Ed\oo 
\textbf{॰दीपितम्}\lem    \mssALL,      ॰दीयतां \msM}}% 

{\devanagarifont अनर्थयज्ञ उवाच {\dandab}\dontdisplaylinenum  }%
     \var{{\devanagarifont \numemph\vo\textbf{अनर्थयज्ञ उवाच}\lem \mssALL, \om\ \msNaacorr}}% 

{\devanagarifont एकं दशं शतं चैव सहस्रमयुतं तथा \thinspace{\danda} \dontdisplaylinenum }%
     \var{{\devanagarifont \numnoemph\vb\textbf{सहस्र॰}\lem \mssALL, साहस्र॰ \msM\oo 
\textbf{॰युतं}\lem   \mssALL,  ॰तन् \msNb}}% 

%Verse 1:32

{\devanagarifont प्रयुतं नियुतं कोटिमर्बुदं वृन्दमेव च {॥ १:३२॥} \veg\dontdisplaylinenum }%
     \var{{\devanagarifont \numnoemph\vc\textbf{प्र॰}\lem        \mssALL,      प॰ \msPaperA}}% 
    \var{{\devanagarifont \numnoemph\vcd\textbf{कोटिम॰}\lem \mssALL,  कोटिर॰ \msNc}}% 
    \var{{\devanagarifont \numnoemph\vd\textbf{॰र्बुदं}\lem     \mssALL, ॰बुदं \msNc}}% 

{\devanagarifont खर्वं चैव निखर्वं च शङ्कु पद्मं तथैव च \thinspace{\dandab} \dontdisplaylinenum }%
     \var{{\devanagarifont \numemph\va\textbf{निखर्वं च}\lem \mssALL,       निखर्वं तु \msNb, निसर्वञ्च \msM}}% 
    \var{{\devanagarifont \numnoemph\vb\textbf{शङ्कु}\lem        \mssALL, शंख \Ed\oo 
\textbf{पद्मं}\lem       \mssALL,  पद्म \msM}}% 
    \lacuna{\devanagarifontsmall \vab {\englishfont After these two pādas, \msPaperA\ inserts this:}
                                वृन्दञ्चैव महावृन्द द्विपरो नन्तनेव च }%
      \paral{{\devanagarifontsmall \vab {\englishfont  = \BRAHMANDAPUR\ 3.2.101 }  }}

%Verse 1:33

{\devanagarifont समुद्रो मध्यमन्तं च परार्धं च परं तथा {॥ १:३३॥} \veg\dontdisplaylinenum }%
     \var{{\devanagarifont \numnoemph\vc\textbf{समुद्रो}\lem            \mssALL, समुद्र॰ \msM\oo 
\textbf{मध्यमन्तं च}\lem \mssCaCbCc\msNaacorr\msM\msPaperA,         मध्यमान्तं च \msNapcorr, 
मध्य\uncl{मन्तञ्च} \msNb, 
मध्यमन्तश्च \msNc}}% 
    \var{{\devanagarifont \numnoemph\vd \lem \mssALL,  परार्द्धपरद्वेगुणाम् \msM}}% 
    \lacuna{\devanagarifontsmall \vcd {\englishfont \Ed\ omits 34cd--35 and then inserts this:}
                                वृन्दञ्चैव महावृन्द द्विपरानन्तमेव च }%
  
{\devanagarifont सर्वे दशगुणा ज्ञेयाः परार्धं यावदेव हि \thinspace{\dandab} \dontdisplaylinenum }%
     \var{{\devanagarifont \numemph\va\textbf{सर्वे}\lem     \mssALL, सर्वं \msPaperA}}% 
    \var{{\devanagarifont \numnoemph\vb\textbf{परार्धं}\lem \msNc,                                परा\uncl{र्ध} \msCa, 
परार्ध \msCb\msCc\msNa\msNb\msM\msPaperA\oo 
\textbf{यावदेव}\lem \mssALL, दशदव \msPaperA}}% 

%Verse 1:34

{\devanagarifont परार्धद्विगुणेनैव परसंख्या विधीयते {॥ १:३४॥} \veg\dontdisplaylinenum }%
     \var{{\devanagarifont \numnoemph\vc\textbf{परार्ध॰}\lem \mssALL,   परार्धं \msNc}}% 
    \var{{\devanagarifont \numnoemph\vd\textbf{॰संख्या}\lem  \mssALL, ॰सख्या \msM}}% 

{\devanagarifont परात्परतरं नास्ति इति मे निश्चिता मतिः \thinspace{\dandab} \dontdisplaylinenum }%
     \var{{\devanagarifont \numemph\vab \lem \mssCaCbCc\msNb\msNcpcorr\msPaperA\Ed, 
परात्परतरं नास्ति इति मे निश्चिता मति \msNa\msNcacorr, 
परापरतरन्नास्ति इति मे निश्चिता मति \msM}}% 

%Verse 1:35

{\devanagarifont पुराणवेदपठिता मयाख्याता द्विजोत्तम {॥ १:३५॥} \veg\dontdisplaylinenum }%
     \var{{\devanagarifont \numnoemph\vc\textbf{॰वेद॰}\lem \msCa\Ed, ॰वेदे \msCb\msCc\msNb\msNc\msPaperA, 
॰वेदा \msNa, ॰वेदैः \msM}}% 
    \var{{\devanagarifont \numnoemph\vd\textbf{॰ख्याता}\lem \msCa\msCb\msNa, ॰ख्यातं \msCc\msNb\msNc\msM\msPaperA\Ed\oo 
\textbf{॰त्तम}\lem \mssALL, ॰तम \msM}}% 


\alalfejezet{ब्रह्माण्डम्}
{\devanagarifont विगतराग उवाच {\dandab}\dontdisplaylinenum  }%
 
{\devanagarifont ब्रह्माण्डं कति विज्ञेयं प्रमाणं ज्ञापितं क्वचित् \thinspace{\danda} \dontdisplaylinenum }%
     \var{{\devanagarifont \numemph\va\textbf{ब्रह्माण्डं}\lem \mssALL, ब्रह्माण्ड \msCc}}% 
    \var{{\devanagarifont \numnoemph\vb \lem \conj, प्रमाणं चापितं क्वचित् \mssCaCbCc\msNa\msNb\msPaperA\Ed, 
प्रमाञ्चापितत् क्वचित् \msNc, प्रमाणञ्चापितां कति \msM}}% 

%Verse 1:36

{\devanagarifont कति चाङ्गुलिमूर्ध्वेषु सूर्यस्तपति वै महीम् {॥ १:३६॥} \veg\dontdisplaylinenum }%
     \var{{\devanagarifont \numnoemph\vc\textbf{॰र्ध्वेषु}\lem \eme, ॰र्धेषु \mssCaCbCc\msNa\msNb\msNc\msM\msPaperA\Ed}}% 
    \var{{\devanagarifont \numnoemph\vd\textbf{सूर्यस्त॰}\lem \mssALL, र्यो \msMacorr, शूर्यो \msMpcorr\oo 
\textbf{महीम्}\lem \msCb\msCc\msNa\msM\msPaperA, मही\uncl{म् } \msCa, मही \msNb\msNc\Ed}}% 

{\devanagarifont अनर्थयज्ञ उवाच {\dandab}\dontdisplaylinenum  }%
 
{\devanagarifont ब्रह्माण्डानां प्रसंख्यातुं मया शक्यं कथं द्विज \thinspace{\danda} \dontdisplaylinenum }%
     \var{{\devanagarifont \numemph\va\textbf{ब्रह्मा॰}\lem \mssALL, ब्रह्म॰ \msM\oo 
\textbf{प्रसंख्यातुं}\lem \mssALL, प्रसंसा तु \msNb, च संख्यातुं \Ed}}% 
    \var{{\devanagarifont \numnoemph\vb\textbf{शक्यं क॰}\lem \msNa\msNb\msPaperApcorr\Ed, शक्या क॰ \mssCaCbCc\msNc, सक्याङ्क॰ \msM, 
ह्यक्यं क॰ \msPaperAacorr}}% 

%Verse 1:37

{\devanagarifont देवास्ते ऽपि न जानन्ति मानुषाणां च का कथा {॥ १:३७॥} \veg\dontdisplaylinenum }%
     \var{{\devanagarifont \numnoemph\vc\textbf{देवास्ते}\lem   \mssALL, देवतापि \msM}}% 
    \var{{\devanagarifont \numnoemph\vd\textbf{मानुषाणां च}\lem \mssALL, मानुषार्नञ्च \msMacorr, 
मानुषानाञ्च \msMpcorr}}% 

{\devanagarifont पर्यायेण तु वक्ष्यामि यथाशक्यं द्विजोत्तम \thinspace{\dandab} \dontdisplaylinenum }%
 
%Verse 1:38

{\devanagarifont ब्रह्मणा यत्पुराख्यातो मातरिश्वा यथा तथा {॥ १:३८॥} \veg\dontdisplaylinenum }%
     \var{{\devanagarifont \numemph\vc\textbf{यत्पुराख्यातो}\lem \mssCaCbCc\msNa\msNb\msNc, यत्पुराख्यातं \msM, 
यत्प्रयात्परायाख्यातो \msPaperA, 
यत्ममाख्यातो \Ed}}% 
    \paral{{\devanagarifontsmall \vcd {\englishfont cf. \BRAHMANDAPUR\ 3.4.58cd:} 
                         ब्रह्मा ददौ शास्त्रमिदं पुराणं मातरिश्वने }}

{\devanagarifont शिवाण्डाभ्यन्तरेणैव सर्वेषामिव भूभृताम् \thinspace{\dandab} \dontdisplaylinenum }%
     \var{{\devanagarifont \numemph\va\textbf{शिवाण्डा॰}\lem \mssALL, शिवाण्ड॰ \msMacorr, शिवाण्डे॰ \msMpcorr}}% 
    \var{{\devanagarifont \numnoemph\vb \lem \conj, सर्वेषामिव भूरिताः \msCa\msCb\msNc, 
सर्वेषामेव भूरिताः \msCc, 
सर्वेषामिव भूरिता \msNa, सर्वेषामेव भूरिणाम् \msNb, 
स\uncl{र्षपा} इव भाविता \msM, 
सर्वेषामेव भूरिनाः \msPaperA, 
सर्वेषामेव भूरिमां \Ed}}% 

%Verse 1:39

{\devanagarifont दश नाम दिशाष्टानां ब्रह्माण्डे कीर्तितं शृणु {॥ १:३९॥} \veg\dontdisplaylinenum }%
     \var{{\devanagarifont \numnoemph\vc\textbf{दिशा॰}\lem         \mssALL,  शिवा॰ \msNb}}% 
    \var{{\devanagarifont \numnoemph\vd\textbf{ब्रह्माण्डे}\lem     \mssALL, ब्रह्मण्डा \msM\oo 
\textbf{कीर्तितं शृणु}\lem \mssALL, य च कीर्तितम् \msCb, 
कीर्त्तिता शृणु \msM}}% 


\alalfejezet{भूभृतां नामानि}

\alalalfejezet{पूर्वतः}

{\devanagarifont सहासहः सहः सह्यो विसहः संहतो ऽसभा \thinspace{\dandab} \dontdisplaylinenum }%
     \var{{\devanagarifont \numemph\va\textbf{सहासहः}\lem   \msNc,                     साहासह \mssCaCbCc\msNa\msNb\msM\msPaperA\Ed\oo 
\textbf{सहः सह्यो}\lem \msCa\msCc\msNa\msNb\msNc, सहः सज्ञा \msCb, 
सहो सह्यः \msM, 
सहः सज्ञो \msPaperA\Ed}}% 
    \var{{\devanagarifont \numnoemph\vb\textbf{विसहः}\lem     \msCa\msCb\msNa\msNb\msNc\Ed, विसह \msCc\msM, विंसहः \msPaperA\oo 
\textbf{ऽसभा}\lem      \msCa\msCc\msNa\msNb\msNc,    सभाः \msCb, सहा \msM, सता \msPaperA\Ed}}% 

%Verse 1:40

{\devanagarifont प्रसहो ऽप्रसहः सानुः पूर्वतो दश नायकाः {॥ १:४०॥} \veg\dontdisplaylinenum }%
     \var{{\devanagarifont \numnoemph\vc\textbf{प्रसहो}\lem  \mssALL,  प्रसहेः \Ed\oo 
\textbf{प्रसहः}\lem \mssALL,  प्रस\uncl{वः} \msCc, सप्रहः \Ed\oo 
\textbf{सानुः}\lem    \mssCaCbCc\msNa\msNb\msPaperA,                  सानु \msNc\msM\Ed}}% 
    \var{{\devanagarifont \numnoemph\vd\textbf{पूर्वतो}\lem  \mssALL,  पर्वतो \Ed}}% 


\alalalfejezet{आग्नेये}

{\devanagarifont प्रभासो भासनो भानुः प्रद्योतो द्युतिमो द्युतिः \thinspace{\dandab} \dontdisplaylinenum }%
     \var{{\devanagarifont \numemph\va\textbf{भासनो}\lem \msCa\msCb\msNa\msNb\msNc\msM,                भास\lac\  \msCc, 
भांसतो \msPaperA, 
भासतो \Ed\oo 
\textbf{भानुः}\lem  \mssALL, भानु \msCb\msM}}% 
    \var{{\devanagarifont \numnoemph\vb\textbf{द्युतिमो}\lem \mssCaCbCc\msNa\msNb\msM,                     द्युतिनो \msNc\msPaperA\Ed}}% 

{\devanagarifont दीप्ततेजाश्च तेजाश्च तेजा तेजवहो दश  \danda\dontdisplaylinenum }%
     \var{{\devanagarifont \numnoemph\vc \lem \msCa\msCc\msNa\msNb\msNc\msPaperA, दीप्ततेजाश्च तेजश्च \msCb, 
दीप्ततेजस् तेजश्च \msM\ \unmetr, 
दीप्ततेजश्च तेजाश्च \Ed}}% 
    \var{{\devanagarifont \numnoemph\vd\textbf{तेजा तेजवहो}\lem \mssALL,      तेजतेजयह \msM}}% 

%Verse 1:41

{\devanagarifont आग्नेये त्वेतदाख्यातं याम्ये शृण्वथ भो द्विज {॥ १:४१॥} \veg\dontdisplaylinenum }%
     \var{{\devanagarifont \numnoemph\ve\textbf{आग्नेये}\lem         \mssCaCbCc\msNa\msNb\Ed,                      आग्नेय \msNc\msPaperA, 
आग्नेर्ये \msM\oo 
\textbf{त्वेतदा॰}\lem \mssALL, त्वेचमा \msM}}% 
    \var{{\devanagarifont \numnoemph\vf\textbf{शृण्वथ}\lem    \mssALL, शृणुथ \msM\oo 
\textbf{द्विज}\lem          \mssALL,  द्विजः \msNb}}% 


\alalalfejezet{याम्ये}

{\devanagarifont यमो ऽथ यमुना यामः संयमो यमुनो ऽयमः \thinspace{\dandab} \dontdisplaylinenum }%
     \var{{\devanagarifont \numemph\va\textbf{यमो}\lem    \mssALL, यमा \msPaperA}}% 
    \var{{\devanagarifont \numnoemph\vb\textbf{संयमो}\lem \mssALL,     संयम \msM, 
संयमा \msPaperA\oo 
\textbf{यमुनो}\lem  \msCa\msCb\msNb\msPaperA,                यमनो \msCc\msNc, 
युमुना \msNa, 
यमतो \msM, यमुना॰ \Ed\oo 
\textbf{यमः}\lem   \mssALL,     यन \msM, 
यामः \msPaperA\ \unmetr}}% 

%Verse 1:42

{\devanagarifont संयनो यमनोयानो यनियुग्मा यनोयनः {॥ १:४२॥} \veg\dontdisplaylinenum }%
     \var{{\devanagarifont \numnoemph\vc \lem \msNa, संयमो यमनोयानो \msCa\msCc\Ed, 
संयमो यमुनोयानो \msCb\msNb, 
संयमा यमनो यामो \msNc, 
यमियुग्मा यनो यानः \msM, 
संयमा यमनो यानो \msPaperA}}% 
    \var{{\devanagarifont \numnoemph\vd \lem \msNb, यनियुग्मा नयो यनः \msCa\msCc\msNa, 
यनियुग्मा नयो नयः \msCb\msPaperA, 
यनियुग्मा नयो यमः \msNc, 
दशमा याम्यमाशृता \msM, 
यनियुग्मा नयोनय \Ed}}% 


\alalalfejezet{नै\char"0930\char"094D\char"090Bते}

{\devanagarifont नगजो नगना नन्दो नगरो नग नन्दनः \thinspace{\dandab} \dontdisplaylinenum }%
     \var{{\devanagarifont \numemph\va\textbf{नगना नन्दो}\lem        \msCa\msCc\msNa\msNb\msNc, 
नगजा नन्दो \msCb, 
नगनागेन्द्र \msM, 
नगनो नदो \msPaperA\Ed}}% 
    \var{{\devanagarifont \numnoemph\vb \lem \msNb\msMacorr\msPaperA, नगरोरगनन्दनः \msCa\msNc, 
नगरो\uncl{नगनन्द}नः \msCb, 
नग\uncl{रो}\lac  नन्दनः \msCc, 
नगरोगरनन्दनः \msNa, 
नगरो नननन्दनः \msMpcorr, 
नगरोन्नगनन्दनः \Ed}}% 

%Verse 1:43

{\devanagarifont नगर्भो गहनो गुह्यो गूढजो दश तत्परः {॥ १:४३॥} \veg\dontdisplaylinenum }%
     \var{{\devanagarifont \numnoemph\vc\textbf{नगर्भो}\lem     \mssALL,      नृगभो \msNb, नगर्भ \msM\oo 
\textbf{गहनो गुह्यो}\lem \mssALL,    गुहनो गुह्य \msM, गहनो गुह्ये \Ed}}% 
    \var{{\devanagarifont \numnoemph\vd\textbf{गूढजो}\lem      \mssALL, गुडजो \msM\oo 
\textbf{तत्परः}\lem     \mssALL, तत्परम् \msM}}% 


\alalalfejezet{वारुणे}

{\devanagarifont वारुणेन प्रवक्ष्यामि शृणु विप्र निबोध मे \thinspace{\dandab} \dontdisplaylinenum }%
     \var{{\devanagarifont \numemph\va\textbf{वारुणेन}\lem \mssALL, वारुणे च \Ed}}% 
    \var{{\devanagarifont \numnoemph\vb\textbf{शृणु}\lem      \msNb\msM,                                    शृङ्गे \msCa\msCb\msNa\msNc, 
शृ\uncl{ङ्गे} \msCc, मृद्धे \uncl{पाप्त} \cancelled\ \msPaperA, मृद्धे \Ed}}% 

{\devanagarifont बभ्रः सेतुर्भवोद्भद्रः प्रभवोद्भवभाजनः  \danda\dontdisplaylinenum }%
     \var{{\devanagarifont \numnoemph\vc\textbf{बभ्रः सेतुर्भ॰}\lem \corr, बभ्रं सेतुर्भ॰ \msCa\msCb, 
बभ्रं सेतु भ॰ \msCc, बभ्रः सेतु भ॰ \msNa, 
बभ्रं सोतुर्भ॰ \msNb, बभ्र सेतुर्भ॰ \msNc, 
बभ्रू सेतु भ॰ \msM, बभ्रून्सेतुर्भ॰ \msPaperA, 
बभ्रून्सतुर्भ॰ \Ed}}% 
    \var{{\devanagarifont \numnoemph\vd\textbf{प्रभवोद्भव॰}\lem \mssALL,   प्रभवोभव॰ \msM\oo 
\textbf{॰भाजनः}\lem       \mssALL, ॰भाजन \Ed}}% 

%Verse 1:44

{\devanagarifont भरणो भुवनो भर्ता दशैते वरुणालयाः {॥ १:४४॥} \veg\dontdisplaylinenum }%
     \var{{\devanagarifont \numnoemph\ve\textbf{भरणो}\lem \msCb\msNc, भरण \msCa\msNa, भरणां \msCc\msPaperA\Ed, 
भरणा \msNb, भरणः \msM}}% 
    \var{{\devanagarifont \numnoemph\vf\textbf{दशैते}\lem \mssALL,    दशेते \msNc, दशैता \msM\oo 
\textbf{॰लयाः}\lem  \mssALL, ॰लया \msM\Ed}}% 


\alalalfejezet{वायव्ये}

{\devanagarifont नृगर्भो ऽसुरगर्भश्च देवगर्भो महीधरः \thinspace{\dandab} \dontdisplaylinenum }%
     \var{{\devanagarifont \numemph\va\textbf{नृगर्भो}\lem      \mssALL, नृगभा \msM\oo 
\textbf{॰गर्भश्च}\lem \msCa\msCb\msNb\msNc\msPaperA,               ॰गर्भाश्च \msCc\msNa\msM\Ed}}% 
    \var{{\devanagarifont \numnoemph\vb\textbf{देवगर्भो}\lem    \mssALL, देवगर्भ \msM}}% 

%Verse 1:45

{\devanagarifont वृषभो वृषगर्भश्च वृषाङ्को वृषभध्वजः {॥ १:४५॥} \veg\dontdisplaylinenum }%
     \var{{\devanagarifont \numnoemph\vc\textbf{॰गर्भश्च}\lem \mssCaCbCc\msNb\msNc\Ed, ॰गर्भाश्च \msNa, ॰गर्भोश्च \msM, 
॰शभश्च \msPaperA}}% 
    \var{{\devanagarifont \numnoemph\vd\textbf{वृषाङ्को}\lem  \mssALL,     वृषांगो \msM\oo 
\textbf{वृषभ॰}\lem \mssALL, वृष\lk ॰ \msCc}}% 

{\devanagarifont ज्ञातव्यश्च तथा सम्यग् वृषजो वृषनन्दनः \thinspace{\dandab} \dontdisplaylinenum }%
     \var{{\devanagarifont \numemph\va \lem \mssCaCbCc\msNa\msNb\msNc, 
वृषञ्जवृषनन्दश्च \msM, 
ज्ञानवाञ्च तथा सम्य \msPaperA, 
ज्ञानवाञ्च तथा सत्य॰ \Ed}}% 
    \var{{\devanagarifont \numnoemph\vb \lem \mssALL, वृषनन्दनः \msNa, 
दशनायक वायवे \msM}}% 

%Verse 1:46

{\devanagarifont नायका दश वायव्ये कीर्तिता ये मया द्विज {॥ १:४६॥} \veg\dontdisplaylinenum }%
     \var{{\devanagarifont \numnoemph\vcd \lem \msCa\msCb\msNa\msPaperA\Ed, 
नायका दश वायव्ये कीर्तिता ये मया द्विजः \msCc\msNb, 
नायका दश वायव्ये कीर्तिता य मया द्विज \msNc, 
कीर्तितो यं मया द्विप्र यथा तथ्येन सुव्रतः \msM}}% 


\alalalfejezet{उत्तरे}

{\devanagarifont सुलभः सुमनः सौम्यः सुप्रजः सुतनुः शिवः \thinspace{\dandab} \dontdisplaylinenum }%
     \var{{\devanagarifont \numemph\va\textbf{सुलभः}\lem \mssALL, सुरभः \msPaperA\Ed\oo 
\textbf{सुमनः}\lem  \mssCaCbCc\msNa\msNb\Ed, सुमनाः \msNc, सुमनो \msM, सुमन \msPaperA\oo 
\textbf{सौम्यः}\lem  \mssALL, सोम्य \msM}}% 

%Verse 1:47

{\devanagarifont सतः सत्य लयः शम्भुर्दश नायकमुत्तरे {॥ १:४७॥} \veg\dontdisplaylinenum }%
     \var{{\devanagarifont \numnoemph\vc\textbf{सतः सत्य}\lem \corr, सत सत्य \mssCaCbCc\msNc\msPaperA, सत्यसत्य \msNa, सुत सत्य \msNb, 
सुतः सत्य \msM, सत सत्या॰ \Ed\oo 
\textbf{लयः}\lem \mssALL, लयं \msNc}}% 
    \var{{\devanagarifont \numnoemph\vcd\textbf{शम्भुर्द॰}\lem \msCa\msCb\msNb\msPaperA\Ed, शम्भु द॰ \msCc\msNa\msNc, 
शम्\uncl{भुं} द॰ \msM}}% 
    \var{{\devanagarifont \numnoemph\vd\textbf{॰नायकमु॰}\lem \mssALL, ॰नायक उ॰ \Ed}}% 


\alalalfejezet{ईशाने}

{\devanagarifont इन्दु बिन्दु भुवो वज्र वरदो वर वर्षणः \thinspace{\dandab} \dontdisplaylinenum }%
     \var{{\devanagarifont \numemph\va\textbf{वज्र}\lem \mssALL, व्रजः \msM}}% 
    \var{{\devanagarifont \numnoemph\vb\textbf{॰वर्षणः}\lem \mssCaCbCc\msNa\msNb\msM, ॰\lk \uncl{र्शणम्} \msNc, 
॰दर्प्पणः \msPaperA, 
॰दर्य्य च \Ed}}% 

%Verse 1:48

{\devanagarifont इलनो वलिनो ब्रह्मा दशेशानेषु नायकाः {॥ १:४८॥} \veg\dontdisplaylinenum }%
     \var{{\devanagarifont \numnoemph\vc \lem \mssALL, इलिनो वलिनो ब्रह्मः \msM}}% 
    \var{{\devanagarifont \numnoemph\vd\textbf{दशे॰}\lem               \msCa\msNa\msNc\msPaperA\Ed,                  दशै॰ \msCb\msCc\msNb, दिशै॰ \msM\oo 
\textbf{नायकाः}\lem             \mssALL, नायका \msM}}% 


\alalalfejezet{मध्यमे}

{\devanagarifont अपरो विमलो मोहो निर्मलो मन मोहनः \thinspace{\dandab} \dontdisplaylinenum }%
     \var{{\devanagarifont \numemph\va \lem \mssALL, अपरः विमला मोहा \msM}}% 
    \var{{\devanagarifont \numnoemph\vb\textbf{निर्मलो म॰}\lem \eme, निमलो म॰ \msCa, निर्मलोन्म॰ \msCb\msNc\msPaperA, 
निर्मलोत्म॰ \msCc\Ed, निमलोर्म॰ \msNa\msNb, निर्मलोन्म॰ \msM}}% 

%Verse 1:49

{\devanagarifont अक्षयश्चाव्ययो विष्णुर्वरदो मध्यमे दश {॥ १:४९॥} \veg\dontdisplaylinenum }%
     \var{{\devanagarifont \numnoemph\vc\textbf{अक्षयश्चाव्ययो}\lem \msCa\msCb\msNa\msNb\msNc\msPaperA, अक्षयाश्चाव्ययो \msCc, 
अक्षयश्चाव्ययं \msM, अक्षयञ्चाव्ययो \Ed}}% 
    \var{{\devanagarifont \numnoemph\vcd\textbf{विष्णुर्व॰}\lem \msCa\msCb\msNc\msPaperA\Ed, विष्णु व॰ \msCc\msNa\msM, र्विष्णुर्व \msNb}}% 
    \var{{\devanagarifont \numnoemph\vd\textbf{मध्यमे दश}\lem \msCa\msCb\msNc\msPaperA, मध्यमो दश \msCc\msNa, 
वरवर्षणः \msNb, मध्यमो दशः \msM, मध्यमे दशः \Ed}}% 


\alalalfejezet{परिवाराः}

{\devanagarifont सर्वेषां दशमीशानां परिवारशतं शतम् \thinspace{\dandab} \dontdisplaylinenum }%
     \var{{\devanagarifont \numemph\va\textbf{सर्वेषां}\lem       \mssALL,   सर्वेषा \msNc\oo 
\textbf{दशमीशानां}\lem \mssALL, दशरीशानां \Ed}}% 
    \var{{\devanagarifont \numnoemph\vb\textbf{परिवार॰}\lem      \mssALL,   परि॰ \msCb, परिवारं \msNa}}% 

%Verse 1:50

{\devanagarifont शतानां पृथगेकैकं सहस्रैः परिवारितम् {॥ १:५०॥} \veg\dontdisplaylinenum }%
     \var{{\devanagarifont \numnoemph\vd\textbf{सहस्रैः}\lem \mssALL,  सहस्रै \msM\oo 
\textbf{॰वारितम्}\lem  \msCa\msCb\msCcpcorr\msNa\msNb\msNc\msPaperA, ॰वारिता \msCcacorr, 
॰वारितः \msM, ॰वारिताः \Ed}}% 

{\devanagarifont सहस्रेषु च एकैकमयुतैः परिवारितम् \thinspace{\dandab} \dontdisplaylinenum }%
     \var{{\devanagarifont \numemph\vab\textbf{एकैकम॰}\lem \msCa\msCb\msNb\msNc\msPaperA\Ed,          एकैकं म॰ \msCc\msNa\msM}}% 
    \var{{\devanagarifont \numnoemph\vb\textbf{परिवारितम्}\lem  \mssALL, परिवारितः \msM, परिवारितमाः \Ed}}% 

%Verse 1:51

{\devanagarifont अयुतं प्रयुतैर्वृन्दैः प्रयुतं नियुतैर्वृतम् {॥ १:५१॥} \veg\dontdisplaylinenum }%
     \var{{\devanagarifont \numnoemph\vc\textbf{अयुतं}\lem \Ed, अयुतैः \mssCaCbCc\msNa\msNc\msM\msPaperA, अयुतै \msNb\oo 
\textbf{प्रयुतैर्वृन्दैः}\lem \mssALL, प्रयुतै वृन्दैः \msNc, 
प्रयुतैर्भृत्य \msM}}% 
    \var{{\devanagarifont \numnoemph\vd \lem \corr, प्रयुतैर्नियुतैर्वृतः \msCa\msCb\msNa\msNc, 
प्रयुतेर्नियुतैर्वृतः \msCc, प्रयुतै नियुतै वृतः \msNb, 
प्रयुतः नियुतैः वृतः \msM, प्रयुते नियुतैर्वृतः \msPaperA, प्रयुतं नियुतैर्वृतः \Ed}}% 

{\devanagarifont एकैकस्य परीवारो नियुतः पृथगेव च \thinspace{\dandab} \dontdisplaylinenum }%
     \var{{\devanagarifont \numemph\va\textbf{परीवारो}\lem \mssALL,             परिवार \msM\ \unmetr, 
परिवारो \Ed\ \unmetr}}% 
    \var{{\devanagarifont \numnoemph\vb\textbf{नियुतः}\lem  \mssALL,      नियुत \msCc\oo 
\textbf{च}\lem       \mssALL, चः \msNcacorr}}% 

%Verse 1:52

{\devanagarifont कोटिभिर्दशकोट्येन एकैकः परिवारितः {॥ १:५२॥} \veg\dontdisplaylinenum }%
     \var{{\devanagarifont \numnoemph\vc \lem \msCa\msCc\msPaperA\Ed, कोटिभि दशकोट्येन \msCb, 
कोटिभिर्दशकोट्योन \msNa\msNc, कोटिभिर्दशकोट्येनः \msNb, 
कोटिभिः परिवाराणि कोटिभि दशकोटिकम् \msM}}% 
    \var{{\devanagarifont \numnoemph\vd \lem \msCb\msNa\Ed, एकैकः परिवरि\uncl{तः} \msCa, 
एकैकपरिवारितः \msCc\msNb\msNc, एकैकपरिवाराणां \msM, 
एकैकः परिवारितं \msPaperA}}% 

{\devanagarifont दशकोटिषु एकैकं वृन्दवृन्दभृतैर्वृतम् \thinspace{\dandab} \dontdisplaylinenum }%
     \var{{\devanagarifont \numemph\va \lem \msCb\msCc\msNb\msPaperA\Ed, दशकोटीषु एकैकं \msCa\msNa\msNc, 
दशकोट्येषु एककं \msM}}% 
    \var{{\devanagarifont \numnoemph\vb \lem \mssCaCbCc\msNb, वृन्दवृन्दवृतैर्वृतं \msNa, 
वृन्दवृन्दभृतै वृतं \msNc, 
वृन्द्रवृन्देषु एकैकं \msM, 
वृन्दवृन्दवृतैर्वृत \msPaperA, 
वृन्दवृन्दं वृतैर्वृतः \Ed}}% 

%Verse 1:53

{\devanagarifont वृन्दवर्गेषु एकैकं खर्वभिः परिवारितम् {॥ १:५३॥} \veg\dontdisplaylinenum }%
     \var{{\devanagarifont \numnoemph\vc\textbf{वृन्दवर्गेषु}\lem \mssALL, वृन्दवर्गेभिः तै वृतम् \msM}}% 
    \var{{\devanagarifont \numnoemph\vd \lem \mssCaCbCc\msNa\msNb, खर्वर्भिः परिवारितम् \msNc, 
खर्वाभिः परिवाराणि \msM, 
खर्वर्भिः परिवारित \msPaperA, 
खर्वर्भिः परिवारितः \Ed}}% 

{\devanagarifont खर्ववर्गेषु एकैकं दशखर्वगणैर्वृतम् \thinspace{\dandab} \dontdisplaylinenum }%
     \var{{\devanagarifont \numemph\va \lem    \mssALL, खर्ववर्गेव एककम् \msM}}% 
    \var{{\devanagarifont \numnoemph\vb \lem \msCa\msCc\msNa\msNb\msPaperA, दशखर्वगणै वृतम् \msCb, 
दशखर्वगणे वृत्तं \msNc, 
दशखर्वेषु एकैकं दशखर्वगणैर्वृतम् \msM, 
दशखर्वगणैर्वृतः \Ed}}% 

%Verse 1:54

{\devanagarifont दशखर्वेषु एकैकं शङ्कुभिः परिवारितम् {॥ १:५४॥} \veg\dontdisplaylinenum }%
     \var{{\devanagarifont \numnoemph\vc\textbf{॰खर्वेषु}\lem   \mssALL, ॰गर्वेषु \msNc}}% 
    \var{{\devanagarifont \numnoemph\vd\textbf{परिवारितम्}\lem \mssALL, परिवारित \msPaperA, परिवारितः \Ed}}% 

{\devanagarifont शङ्कुभिः पृथगेकैकं पद्मेन परिवारितम् \thinspace{\dandab} \dontdisplaylinenum }%
     \var{{\devanagarifont \numemph\va\textbf{पृथगेकैकं}\lem \eme, पृथगेनैव \msCa\msCc\msNa\msNb\msNc\msM\msPaperA\Ed, 
पृथगैनैव \msCb}}% 
    \var{{\devanagarifont \numnoemph\vb\textbf{॰वारितम्}\lem \msNapcorr\msM, ॰वारितः \mssCaCbCc\msNb\msNc\msPaperA\Ed, ॰तं \msNaacorr}}% 

%Verse 1:55

{\devanagarifont पद्मवर्गेषु एकैकं समुद्रैः परिवारितम् {॥ १:५५॥} \veg\dontdisplaylinenum }%
     \var{{\devanagarifont \numnoemph\vd\textbf{समुद्रैः}\lem \mssALL,    समुदैः \msCa, दमु\uncl{दैः} \msCb\oo 
\textbf{॰वारितम्}\lem  \mssALL, ॰वारितः \Ed}}% 

{\devanagarifont समुद्रेषु तथैकैकं मध्यसंख्यैस्तु तैर्वृतम् \thinspace{\dandab} \dontdisplaylinenum }%
     \var{{\devanagarifont \numemph\va\textbf{तथै॰}\lem \mssALL, तथे॰ \msCc}}% 
    \var{{\devanagarifont \numnoemph\vb \lem \mssCaCbCc\msNa\msM\msPaperA, 
मध्यसख्यैस्तु तै वृतम् \msNb, 
मध्यसख्यैस्तु तेर्वृतं \msNc, 
मध्ये शङ्ख्यायुतैर्वृतः \Ed}}% 

%Verse 1:56

{\devanagarifont मध्यसंख्येषु एकैकमनन्तैः परिवारितम् {॥ १:५६॥} \veg\dontdisplaylinenum }%
     \var{{\devanagarifont \numnoemph\vc\textbf{मध्यसंख्येषु}\lem     \mssALL, 
मध्यसांखो च \msM, मध्ये शंखेषु \Ed}}% 
    \var{{\devanagarifont \numnoemph\vcd\textbf{एकैकमनन्तैः}\lem \mssALL, 
एकैकं मनतैः \msNc, एकैकं अनन्तै \msM}}% 
    \var{{\devanagarifont \numnoemph\vd\textbf{॰वारितम्}\lem            \mssALL, ॰वारितः \Ed}}% 

{\devanagarifont अनन्तेषु च एकैकं परार्धपरिवारितम् \thinspace{\dandab} \dontdisplaylinenum }%
     \var{{\devanagarifont \numemph\vb \lem \msCa\msCb\msNa\msNb\msNc\msPaperA, परार्ध\lac  रितम् \msCc, 
परार्धै परिवारितम्\thinspace{\devanagarifont ।} अनन्तेषु च एकैकं परार्धपरिवारितं \msM, परार्धैः परिवारितः \Ed}}% 

{\devanagarifont परार्धेषु च एकैकं परेण परिवारितम्  \danda\dontdisplaylinenum }%
     \var{{\devanagarifont \numnoemph\vd\textbf{॰वारितम्}\lem \mssALL, ॰वारिवारितं \msNb, ॰वारितः \Ed}}% 

%Verse 1:57

{\devanagarifont एष वै कथितो विप्र शक्यं सांख्यमुदीरितम् {॥ १:५७॥} \veg\dontdisplaylinenum }%
     \var{{\devanagarifont \numnoemph\ve\textbf{कथितो}\lem \mssALL, \uncl{कथितो} \msNb, कथिता \Ed}}% 
    \var{{\devanagarifont \numnoemph\vf\textbf{शक्यं}\lem  \mssALL,  शक्य \msCc, संख्यां शक्यं \msPaperA\oo 
\textbf{सांख्यमु॰}\lem \msCa\msCc\msNb\msM, साख्यमु॰ \msCb, स्यख्यमु॰ \msNa, 
संख्यमु \msNc, संख्यामु॰ \msPaperA\Ed}}% 


\alalfejezet{प्रमाणम्}
{\devanagarifont प्रमाणं शृणु मे विप्र संक्षेपाद्ब्रुवतो मम \thinspace{\dandab} \dontdisplaylinenum }%
     \var{{\devanagarifont \numemph\va\textbf{प्रमाणं}\lem \msCc\msNa\msNc\msM\msPaperA\Ed, प्रणामं \msCa\msCb, प्रमाण \msNb}}% 
    \var{{\devanagarifont \numnoemph\vb\textbf{संक्षेपाद्ब्रुवतो}\lem \msCa\msCc\msNa\msNb\msPaperA\Ed, संक्षेपाद्वदतो \msCb, 
संख्येपाद्ब्रुवतो \msNc, 
संक्षेप ब्रुवतो \msM}}% 

%Verse 1:58

{\devanagarifont चन्द्रोदये पूर्णमास्यां वपुरण्डस्य तादृशम् {॥ १:५८॥} \veg\dontdisplaylinenum }%
 
{\devanagarifont कोटिकोटिसहस्रं तु योजनानां समन्ततः \thinspace{\dandab} \dontdisplaylinenum }%
     \var{{\devanagarifont \numemph\va\textbf{कोटिकोटि॰}\lem \mssALL, कोटीकोटि॰ \msM}}% 
    \var{{\devanagarifont \numnoemph\vb\textbf{योज॰}\lem     \mssALL,      याज॰ \msPaperA}}% 

%Verse 1:59

{\devanagarifont अण्डानां च परीमाणं ब्रह्मणा परिकीर्तितम् {॥ १:५९॥} \veg\dontdisplaylinenum }%
     \var{{\devanagarifont \numnoemph\vc\textbf{च परीमाणं}\lem \mssALL, 
च परिमाणं \msCb\ \unmetr, 
परिमाणञ्च \msM}}% 
    \var{{\devanagarifont \numnoemph\vd\textbf{ब्रह्मणा}\lem \mssALL, \lac\  \msCc\oo 
\textbf{॰कीर्तितम्}\lem \msCa\msCb\msNb\msNc\msPaperA\Ed, ॰कीर्ति\uncl{ताः} \msCc, ॰कीर्तितः \msNa\msM}}% 

{\devanagarifont सप्तकोटिसहस्राणि सप्तकोटिशतानि च \thinspace{\dandab} \dontdisplaylinenum }%
     \var{{\devanagarifont \numemph\va\textbf{॰स्राणि}\lem \mssALL, ॰स्रणि \msPaperA}}% 

%Verse 1:60

{\devanagarifont विंशकोटिष्वङ्गुलीषु ऊर्ध्वतस्तपते रविः {॥ १:६०॥} \veg\dontdisplaylinenum }%
     \var{{\devanagarifont \numnoemph\vc \lem \conj, विंशकोटिषु गुल्मेषु \mssCaCbCc\msNa\msNb\msNc\msPaperA\Ed, 
विंशकोटि विना गुल्मे \msM}}% 
    \var{{\devanagarifont \numnoemph\vd\textbf{ऊर्ध्वतस्त॰}\lem \mssCaCbCc\msNa\msNc\Ed, ऊर्ध्व\lac\  \msNb, ऊर्द्ध्वतो त॰ \msM, 
उद्धतस्त॰ \msPaperA\oo 
\textbf{रविः}\lem \mssALL, रवि \Ed}}% 
    \lacuna{\devanagarifontsmall \vcd {\englishfont The folio in \msNb\ ends with } ऊर्ध्व॰, {\englishfont and the folios 
                                that may have contained verses 1.60d--2.22 are missing.} }%
  
{\devanagarifont प्रमाणं नाम संख्या च कीर्तितानि समासतः \thinspace{\dandab} \dontdisplaylinenum }%
     \var{{\devanagarifont \numemph\va \lem \mssALL, प्रणामं नाम संख्या च \msCb, 
प्रमाणेनाणञ्चम संख्या\lk त च \msPaperA}}% 
    \var{{\devanagarifont \numnoemph\vb\textbf{कीर्तितानि}\lem \mssALL, कीर्त्तियानानि \msPaperA}}% 

%Verse 1:61

{\devanagarifont ब्रह्माण्डं चाप्रमेयाणां लक्षणं परिकीर्तितम् {॥ १:६१॥} \veg\dontdisplaylinenum }%
     \var{{\devanagarifont \numnoemph\vc\textbf{ब्रह्माण्डं चा॰}\lem \msNa, ब्रह्माण्डश्च \msCa\msCb\msNc\msM\msPaperA, \uncl{ब्रह्माण्डाश्चा}॰ \msCc, 
ब्रह्माण्डाश्चा \Ed\oo 
\textbf{॰मेयाणां}\lem \msCa\msNa\msM\msPaperA\Ed, ॰मेयाणा \msCb\msCc\msNc}}% 
    \var{{\devanagarifont \numnoemph\vd\textbf{॰कीर्तितम्}\lem \mssALL, ॰कीर्तिताः \msCc, ॰कीर्त्तितः \msM}}% 


\alalfejezet{पुराणम्}
{\devanagarifont पुराणाशीसहस्राणि शतानि द्विजसत्तम \thinspace{\dandab} \dontdisplaylinenum }%
     \var{{\devanagarifont \numemph\vb\textbf{॰सत्तम}\lem \mssALL, \lac  मः  \msCc}}% 

%Verse 1:62

{\devanagarifont ब्रह्मणा कथितं पूर्णं मातरिश्वा यथातथम् {॥ १:६२॥} \veg\dontdisplaylinenum }%
     \var{{\devanagarifont \numnoemph\vc\textbf{पूर्णं}\lem    \msCa\msCc\msNa\msPaperA\Ed,              पूर्वे \msCb, पूर्ण्ण \msNc, पूर्वं \msM}}% 
    \var{{\devanagarifont \numnoemph\vd\textbf{मातरिश्वा}\lem \mssALL,  मातरिश्व \msM\oo 
\textbf{॰तथम्}\lem   \mssALL, ॰तथा \msCc\msM}}% 

{\devanagarifont वायुना पाद संक्षिप्य प्राप्तं चोशनसं पुरा \thinspace{\dandab} \dontdisplaylinenum }%
     \var{{\devanagarifont \numemph\va\textbf{संक्षिप्य}\lem \mssALL, संक्षिप्यः \msM}}% 
    \var{{\devanagarifont \numnoemph\vb\textbf{प्राप्तं चोशनसं}\lem \msCb\msNa\msNc, प्राप्तं चौसनसं \msCa\msPaperA, प्राप्त\lk औसनसं \msCc, 
प्राप्ताश्चोशनसम \msM\ \unmetr, प्राप्तश्चोशनसं \Ed}}% 

%Verse 1:63

{\devanagarifont तेनापि पाद संक्षिप्य प्राप्तवांश्च बृहस्पतिः {॥ १:६३॥} \veg\dontdisplaylinenum }%
     \var{{\devanagarifont \numnoemph\vc\textbf{संक्षिप्य}\lem \mssALL, संक्षिप्यः \msM}}% 
    \var{{\devanagarifont \numnoemph\vd \lem \mssALL, प्राप्तधञ्च वृहस्पति \msM}}% 

{\devanagarifont बृहस्पतिस्तु प्रोवाच सूर्यं त्रिंशत्सहस्रिकम् \thinspace{\dandab} \dontdisplaylinenum }%
     \var{{\devanagarifont \numemph\vb\textbf{सूर्यं}\lem \msCc\Ed, सूर्यस् \msCa\msNa\msNc\msPaperA, सूर्य \msCb\msM\oo 
\textbf{त्रिंशत्स॰}\lem \mssALL, त्रिंशस॰ \msCc\msM}}% 

%Verse 1:64

{\devanagarifont पञ्चविंशत्सहस्राणि मृत्युं प्राह दिवाकरः {॥ १:६४॥} \veg\dontdisplaylinenum }%
     \var{{\devanagarifont \numnoemph\vc\textbf{॰विंशत्सहस्राणि}\lem \corr, ॰विंशहस्राणि \msCa, 
॰विंशसहस्राणि \msCb\msCc\msNa\msNc\msM\msPaperA, ॰विशत्सहस्राणि \Ed}}% 
    \var{{\devanagarifont \numnoemph\vd\textbf{मृत्युं प्राह}\lem \mssALL, मृत्यु प्राहः \msM}}% 

{\devanagarifont एकविंशत्सहस्राणि मृत्युनेन्द्राय कीर्तितम् \thinspace{\dandab} \dontdisplaylinenum  }%
     \var{{\devanagarifont \numemph\va\textbf{॰विंशत्॰}\lem \Ed, ॰विंश॰ \mssCaCbCc\msNa\msNc\msM\msPaperA}}% 
    \var{{\devanagarifont \numnoemph\vb\textbf{कीर्तितम्}\lem \Ed, कीर्तितः \msCa\msCb\msNa\msNcpcorr\msM, कीर्तिताः \msCc, 
कीर्त्तित \msNcacorr, कीर्तितंः \msPaperA}}% 

%Verse 1:65

{\devanagarifont इन्द्रेणाह वसिष्ठाय विंशत्श्लोकसहस्रिकम् {॥ १:६५॥} \veg\dontdisplaylinenum }%
     \var{{\devanagarifont \numnoemph\vc\textbf{इन्द्रे॰}\lem \mssALL, इन्दे॰ \msPaperA}}% 
    \var{{\devanagarifont \numnoemph\vc\textbf{वसिष्ठाय}\lem \mssALL, विशिष्ठाय \msCb, वहिष्ठाय \msNc}}% 
    \var{{\devanagarifont \numnoemph\vd\textbf{विंशत्श्लो॰}\lem \corr, विंशश्लो॰ \msCa\msCc\msNa\msNc\msPaperA\Ed, विशश्लो॰ \msCb, त्रिंशश्लो॰ \msM}}% 

{\devanagarifont अष्टादशसहस्राणि तेन सारस्वताय तु \thinspace{\dandab} \dontdisplaylinenum }%
     \var{{\devanagarifont \numemph\va \lem \mssALL, 
आष्टादशसहस्राणि \msNc, वसिष्ठेदशसहस्रं \msM}}% 

%Verse 1:66

{\devanagarifont सारस्वतस्त्रिधामाय सहस्रदश सप्त च {॥ १:६६॥} \veg\dontdisplaylinenum }%
     \var{{\devanagarifont \numnoemph\vc\textbf{सारस्वतस्त्रि॰}\lem \eme, सारस्वता त्रि॰ \msCa\msCc\msNa\msNc\msPaperA\Ed, सारस्वतास्त्रि॰ \msCb, 
सारस्वत तृ॰ \msM\oo 
\textbf{॰धामाय}\lem \mssALL, \om\ \msNaacorr}}% 
    \var{{\devanagarifont \numnoemph\vd\textbf{सहस्रदश}\lem \mssALL, सहस्रादश \msM}}% 

{\devanagarifont षोडशानां सहस्राणि भरद्वाजाय वै ततः \thinspace{\dandab} \dontdisplaylinenum }%
     \var{{\devanagarifont \numemph\vb\textbf{भर॰}\lem \mssALL, भार॰ \msCc, सन॰ \msM}}% 

%Verse 1:67

{\devanagarifont दश पञ्चसहस्राणि त्रिवृषाय अभाषत {॥ १:६७॥} \veg\dontdisplaylinenum }%
     \var{{\devanagarifont \numnoemph\vd\textbf{अभाषत}\lem \msCa\msCb\msNa\msPaperA, अ\uncl{भाषत} \msCc, अभाषतः \msNc\Ed, मभासतः \msM}}% 

{\devanagarifont चतुर्दशसहस्राणि अन्तरीक्षाय वै ततः \thinspace{\dandab} \dontdisplaylinenum }%
     \var{{\devanagarifont \numemph\vb\textbf{अन्तरी॰}\lem \mssALL, अन्तरि॰ \msM}}% 

%Verse 1:68

{\devanagarifont त्रय्यारुणिं सहस्राणि त्रयोदश अभाषत {॥ १:६८॥} \veg\dontdisplaylinenum }%
     \var{{\devanagarifont \numnoemph\vc\textbf{त्रय्यारुणिं}\lem \corr, त्र्यैयारुणि \msCa\msCb\msNa\msM\msPaperA, त्रैयारुणि \msCc\Ed, 
त्र्यैयारूपिनि \msNc}}% 
    \var{{\devanagarifont \numnoemph\vd\textbf{अभाषत}\lem \msCa\msCc\msNc\msPaperA, अभाषतः \msCb, स्वभावत \msNa, मभासतः \msM, 
ह्यभाषत \Ed}}% 

{\devanagarifont त्रय्यारुणिस्तु विप्रेन्द्रो धनंजयमभाषत \thinspace{\dandab} \dontdisplaylinenum }%
     \var{{\devanagarifont \numemph\va\textbf{त्रय्यारुणि॰}\lem \corr, त्र्यैयारुणि॰ \mssCaCbCc\msNc\msPaperA, त्रैयारुणि॰ \msNa\Ed, त्र्यैर्यारुणि॰ \msM\oo 
\textbf{विप्रेन्द्रो}\lem \mssALL, विप्रेन्द \msCc\msM}}% 
    \var{{\devanagarifont \numnoemph\vb\textbf{धनंजय॰}\lem \mssALL, धन॰ \msNaacorr\oo 
\textbf{॰भाषत}\lem \msCa\msCc\msNa\msNc\msPaperA, ॰भाषतः \msCb\msM\Ed}}% 

%Verse 1:69

{\devanagarifont द्वादशानि सहस्राणि संक्षिप्य पुनरब्रवीत् {॥ १:६९॥} \veg\dontdisplaylinenum }%
 
{\devanagarifont कृतंजयाय सम्प्राप्तो धनंजयमहामुनिः \thinspace{\dandab} \dontdisplaylinenum }%
     \var{{\devanagarifont \numemph\vb\textbf{॰मुनिः}\lem \mssALL, ॰मुणि \msM}}% 

%Verse 1:70

{\devanagarifont कृतंजयाद्द्विजश्रेष्ठ ऋणंजयमहात्मने {॥ १:७०॥} \veg\dontdisplaylinenum }%
     \var{{\devanagarifont \numnoemph\vc\textbf{कृतंजयाद्द्वि॰}\lem \msCa\msNa\msPaperA\Ed, कृतंजया द्वि॰ \msCb\msCc\msNc, धनञ्जय द्वि॰ \msM\oo 
\textbf{॰श्रेष्ठ}\lem \mssALL, ॰श्रेष्ठो \Ed}}% 
    \var{{\devanagarifont \numnoemph\vd\textbf{ऋणंजय॰}\lem \mssALL, ऋणंजाय॰ \msCb\oo 
\textbf{॰महात्मने}\lem \mssALL, ॰मभाशतः \msM}}% 

{\devanagarifont ऋणञ्जयात्पुनः प्राप्तो गौतमाय महर्षिणे \thinspace{\dandab} \dontdisplaylinenum }%
     \var{{\devanagarifont \numemph\va\textbf{प्राप्तो}\lem \mssALL, प्राप्तः \msM, प्राप्तौ \Ed}}% 
    \var{{\devanagarifont \numnoemph\vb\textbf{महर्षिणे}\lem \mssALL, महर्षिणः \msM}}% 

%Verse 1:71

{\devanagarifont गौतमाच्च भरद्वाजस्तस्माद्धर्यद्वताय तु {॥ १:७१॥} \veg\dontdisplaylinenum }%
     \var{{\devanagarifont \numnoemph\vc\textbf{गौतमाच्च}\lem \mssCaCbCc\msNa\Ed, गौतमाश्च \msNc\msPaperA, गौतमेन \msM}}% 
    \var{{\devanagarifont \numnoemph\vcd\textbf{भरद्वाजस्तस्माद्धर्यद्वताय}\lem \msCa\msCc\msNa\msNc, 
भरद्वारस्तस्माद्धर्यद्वताय \msCb, 
भरद्वाज तस्मा हर्यद्वताय \msM, 
भरद्वाजस्तस्माद्धर्यद्वनाय \msPaperA, 
भरद्वाजस्तस्माद्दम्याद्दमाय \Ed}}% 

{\devanagarifont राजश्रवास्ततः प्राप्तः सोमशुष्माय वै ततः \thinspace{\dandab} \dontdisplaylinenum }%
     \var{{\devanagarifont \numemph\va\textbf{राजश्रवास्त॰}\lem \eme, राजश्रव त॰ \mssCaCbCc\msNa\msPaperA\Ed, राजश्रवे त॰ \msNc, 
राजर्षव त॰ \msM}}% 
    \var{{\devanagarifont \numnoemph\vab\textbf{प्राप्तः सोम॰}\lem \mssALL, प्राप्त साम॰ \msPaperA}}% 

%Verse 1:72

{\devanagarifont सोमशुष्मात्ततः प्राप्तस्तृणबिन्दुस्तु भो द्विज {॥ १:७२॥} \veg\dontdisplaylinenum }%
     \var{{\devanagarifont \numnoemph\vc\textbf{॰शुष्मात्त॰}\lem \mssALL, ॰शुष्मा त॰ \msNa}}% 
    \var{{\devanagarifont \numnoemph\vcd\textbf{प्राप्तस्तृणबिन्दुस्तु}\lem \mssALL, 
प्रा\uncl{प्त तृ}णबिन्दुस्तु \msCc, 
प्राप्तस्तृणविन्दुन्तु \msPaperA}}% 
    \var{{\devanagarifont \numnoemph\vd\textbf{भो}\lem \mssALL, \om\ \msCb}}% 

{\devanagarifont तृणबिन्दुस्तु वृक्षाय वृक्षः शक्तिमभाषत \thinspace{\dandab} \dontdisplaylinenum }%
     \var{{\devanagarifont \numemph\vb\textbf{वृक्षः}\lem \mssALL, वृक्ष \msM\oo 
\textbf{॰भाषत}\lem \msCa\msCb\msNa\msNc\msPaperA, ॰भाषतः \msCc\msM\Ed}}% 

%Verse 1:73

{\devanagarifont शक्तिः पराशरं प्राह जतुकर्णाय वै ततः {॥ १:७३॥} \veg\dontdisplaylinenum }%
     \var{{\devanagarifont \numnoemph\vc\textbf{शक्तिः पराशरं}\lem \mssALL, 
शपरासर \msMacorr, शक्ति परासर \msMpcorr}}% 
    \var{{\devanagarifont \numnoemph\vd\textbf{जतु॰}\lem \mssALL, तु॰ \msCb, जंतु॰ \msM}}% 

{\devanagarifont द्वैपायनं तु प्रोवाच जतुकर्णो महर्षिणम् \thinspace{\dandab} \dontdisplaylinenum }%
     \var{{\devanagarifont \numemph\va\textbf{द्वैपायनं तु}\lem \eme, द्वैपायनस्तु \mssCaCbCc\msNa\msNc\msM\msPaperA, 
द्वैपायनाय \Ed\ \unmetr}}% 
    \var{{\devanagarifont \numnoemph\vb \lem \msCa\msCb\msNapcorr\msNc, जतुकर्णा महर्षिणः \msCc, 
जकर्णो महर्षिणं \msNaacorr, जंतुकर्ण्णमहर्षिणा \msM, जतुकर्णा महषिण \msPaperA, 
जतुकर्णमहर्षिणा \Ed}}% 

%Verse 1:74

{\devanagarifont रोमहर्षाय सम्प्राप्तो द्वैपायनमहामुनिः {॥ १:७४॥} \veg\dontdisplaylinenum }%
     \var{{\devanagarifont \numnoemph\vd\textbf{॰मुनिः}\lem \mssALL, ॰मुनि \msM\Ed}}% 

{\devanagarifont रोमहर्षेण प्रोवाच पुत्रायामितबुद्धये \thinspace{\dandab} \dontdisplaylinenum }%
     \var{{\devanagarifont \numemph\va\textbf{॰हर्षेण}\lem \msM, ॰हर्षाय \mssCaCbCc\msNa\msNc\msPaperA, ॰हर्षणाय \Ed}}% 
    \var{{\devanagarifont \numnoemph\vb\textbf{॰बुद्धये}\lem \mssALL, ॰बुद्धयः \msM}}% 
    \paral{{\devanagarifontsmall \vab \similar\ {\englishfont \BRAHMANDAPUR\ 3.4.67ab:}
                 मया चैतत्पुनः प्रोक्तं पुत्रायामितबुद्धये }}

{\devanagarifont दश द्वे च सहस्राणि पुराणं सम्प्रकाशितम्  \danda\dontdisplaylinenum }%
     \var{{\devanagarifont \numnoemph\vd \lem \mssALL, 
पुराण सम्प्रकाशितां \msCc}}% 

%Verse 1:75

{\devanagarifont मानुषाणां हितार्थाय किं भूयः श्रोतुमिच्छसि {॥ १:७५॥} \veg\dontdisplaylinenum }%
     \var{{\devanagarifont \numnoemph\ve\textbf{मानुषाणां}\lem \mssALL, मनुषाणां \msCb, मानुषाना \msM\oo 
\textbf{हितार्थाय}\lem \mssALL, हित्यथाय \msM, हिताथयि \msPaperA}}% 
    \var{{\devanagarifont \numnoemph\vf\textbf{भूयः}\lem \mssALL, भूय \msM\Ed}}% 

{\devanagarifont 
\jump
\begin{center}
\ketdanda~इति वृषसारसंग्रहे ब्रह्माण्डसंख्या नामाध्यायः प्रथमः~\ketdanda
\end{center}
\dontdisplaylinenum\vers  }%
     \var{{\devanagarifont \numnoemph{\englishfont \Colo:}\textbf{नामाध्यायः प्रथमः}\lem \mssALL, 
नामाध्यायः प्रथमः श्लोक ७७ \msM, 
नाम प्रथमो ऽध्याय \Ed}}% 
\bekveg\szamveg
\vfill
\phpspagebreak

\versno=0\fejno=2
\thispagestyle{empty}

\centerline{\Large\devanagarifontbold [   द्वितीयो ऽध्यायः  ]}{\vrule depth10pt width0pt} \fancyhead[CO]{{\footnotesize\devanagarifont वृषसारसंग्रहे  }}
\fancyhead[CE]{{\footnotesize\devanagarifont द्वितीयो ऽध्यायः  }}
\fancyhead[LE]{}
\fancyhead[RE]{}
\fancyhead[LO]{}
\fancyhead[RO]{}
\szam\bek


\vers


{\devanagarifont विगतराग उवाच {\dandab}\dontdisplaylinenum  }%
 
{\devanagarifont श्रुतं मया जनाग्रेण ब्रह्माण्डस्य तु निर्णयम् \thinspace{\danda} \dontdisplaylinenum }%
     \var{{\devanagarifont \numemph\va\textbf{जनाग्रेण}\lem \mssALL, जना\lac\  \msCa}}% 

%Verse 2:1

{\devanagarifont प्रमाणं वर्णरूपं च संख्या तस्य समासतः {॥ २:१॥} \veg\dontdisplaylinenum }%
     \lacuna{\devanagarifontsmall {\englishfont Witnesses used for this chapter: \msCa\ ff.\thinspace 195v--197r, 
                                             \msCb\ ff.\thinspace 203v--204v,
                                             \msCc\ ff.\thinspace 270r--270v (it breaks off at 2.21 and resumes at 3.30b),
                                             \msNa\ ff.\thinspace 3v--4v, 
                                             \msNb\ exp.\thinspace 43 and 42 (sic!; it broke off at 1.60d and resumes at 2.23),
                                             \msNc\ ff.\thinspace 211v--213r,
                                             \Ed\ pp.\thinspace 585--588;
                                             \mssCaCbCc\ = \msCa + \msCb + \msCc } }%
  
{\devanagarifont शिवाण्डेति त्वया प्रोक्तो ब्रह्माण्डालयकीर्तितः \thinspace{\dandab} \dontdisplaylinenum }%
     \var{{\devanagarifont \numemph\vb\textbf{ब्रह्माण्डा॰}\lem \mssALL, ब्रह्माण्ड \Ed}}% 

%Verse 2:2

{\devanagarifont कीदृशं लक्षणं ज्ञेयं प्रमाणं तस्य वा कति {॥ २:२॥} \veg\dontdisplaylinenum }%
     \var{{\devanagarifont \numnoemph\vc\textbf{ज्ञेयं}\lem \mssALL, ज्ञेया \msCc}}% 
    \var{{\devanagarifont \numnoemph\vd\textbf{कति}\lem \mssALL, कतिः \msCc}}% 

{\devanagarifont कस्य वा लयनं ज्ञेयं प्रमाणं वात्र वासिनः \thinspace{\dandab} \dontdisplaylinenum }%
     \var{{\devanagarifont \numemph\va\textbf{लयनं ज्ञेयं}\lem \mssALL, लयनं \msCb, लक्षणं ज्ञेयं \Ed}}% 
    \var{{\devanagarifont \numnoemph\vb\textbf{वासिनः}\lem \mssALL, वासिरानः \msCb}}% 

%Verse 2:3

{\devanagarifont का वा तत्र प्रजा ज्ञेया को वा तत्र प्रजापतिः {॥ २:३॥} \veg\dontdisplaylinenum }%
     \var{{\devanagarifont \numnoemph\vc\textbf{का}\lem \eme, को \mssCaCbCc\msNa\msNc, किं \Ed\oo 
\textbf{प्रजा ज्ञेया}\lem \mssALL, प्र\uncl{जा}\lac  या \msCa}}% 


\alalfejezet{शिवाण्डसंख्या}
{\devanagarifont अनर्थयज्ञ उवाच {\dandab}\dontdisplaylinenum  }%
 
{\devanagarifont शिवाण्डलक्षणं विप्र न त्वं प्रष्टुमिहार्हसि \thinspace{\danda} \dontdisplaylinenum }%
     \var{{\devanagarifont \numemph\vb\textbf{न त्वं}\lem \mssALL, तत्वं \Ed\oo 
\textbf{॰र्हसि}\lem \mssALL, ॰हसि \msNc}}% 

%Verse 2:4

{\devanagarifont दैवतैरपि का शक्तिर्ज्ञातुं द्रष्टुं च तत्त्वतः {॥ २:४॥} \veg\dontdisplaylinenum }%
     \var{{\devanagarifont \numnoemph\vc\textbf{दैवतै॰}\lem \msCa\msCb\msNa, देवतै॰ \msCc\msNc\Ed\oo 
\textbf{शक्तिर्}\lem \msCa, शक्ति \msCb\msCc\msNa\msNc\Ed}}% 

{\devanagarifont अगम्यगमनं गुह्यं गुह्यादपि समुद्धितम् \thinspace{\dandab} \dontdisplaylinenum }%
     \var{{\devanagarifont \numemph\va\textbf{अगम्यगमनं}\lem \mssALL, अगम्यगगहनं \msCc, अगम्यगगमनं \msNc}}% 
    \var{{\devanagarifont \numnoemph\vb\textbf{गुह्या॰}\lem \msNc\Ed, गुहा॰ \mssCaCbCc\msNa\oo 
\textbf{समुद्धितं}\lem \mssALL, सम्रद्धितं \msNc, समृद्धिदम् \Ed}}% 
    \paral{{\devanagarifontsmall \vab {\englishfont \compare\ \LINPU\ 1.21.71ab:} नमो गुण्याय गुह्याय अगम्यगमनाय च }}

%Verse 2:5

{\devanagarifont न प्रभुर्नेतरस्तत्र न दण्ड्यो न च दण्डकः {॥ २:५॥} \veg\dontdisplaylinenum }%
     \var{{\devanagarifont \numnoemph\vc\textbf{प्रभुर्ने॰}\lem \mssALL, प्रने॰ \msCc}}% 
    \var{{\devanagarifont \numnoemph\vd\textbf{दण्ड्यो}\lem \msCc\msNa\msNc, दण्डो \msCa\msCb, दण्ड्या \Ed\oo 
\textbf{दण्डकः}\lem \mssALL, ण्डकः \msCbacorr, पण्डकः \msCbpcorr}}% 

{\devanagarifont न सत्यो नानृतस्तत्र सुशीलो नो दुःशीलवान् \thinspace{\dandab} \dontdisplaylinenum }%
     \var{{\devanagarifont \numemph\va\textbf{सत्यो}\lem \mssALL, सत्यौ \Ed\oo 
\textbf{तत्र}\lem \mssALL, तत्रा \Ed}}% 
    \var{{\devanagarifont \numnoemph\vb\textbf{नो}\lem \mssALL, \lac\  \msCa}}% 

%Verse 2:6

{\devanagarifont नानृजुर्न च दम्भित्वं न तृष्णा न च ईर्ष्यता {॥ २:६॥} \veg\dontdisplaylinenum }%
     \var{{\devanagarifont \numnoemph\vc\textbf{नानृजुर्न}\lem \eme, नाऋजुर्न्न \msCa\Ed, नाऋजुर्न \msCb\msNc, 
\uncl{नाऋजु न} \msCc, नाऋजुन्न \msNa}}% 
    \var{{\devanagarifont \numnoemph\vd\textbf{न तृष्णा न च}\lem \mssALL,  न च तृष्णा न \msNa\oo 
\textbf{ईर्ष्यता}\lem \mssALL, ईर्ष्यताः \msCc, इर्ष्यता \Ed}}% 

{\devanagarifont न क्रोधो न च लोभो ऽस्ति न मानो ऽस्ति न सूयकः \thinspace{\dandab} \dontdisplaylinenum }%
     \var{{\devanagarifont \numemph\va\textbf{क्रोधो}\lem \mssALL, क्रोधौ \msCc}}% 
    \var{{\devanagarifont \numnoemph\vb\textbf{सूयकः}\lem \mssALL, सूचकः \msCb, स्तेयकः \Ed\ \unmetr}}% 

%Verse 2:7

{\devanagarifont ईर्ष्या द्वेषो न तत्रास्ति न शठो न च मत्सरः {॥ २:७॥} \veg\dontdisplaylinenum }%
     \var{{\devanagarifont \numnoemph\vd\textbf{शठो}\lem \mssALL, षठो \msCc, शठे \Ed\oo 
\textbf{मत्सरः}\lem \mssALL, मत्सराः \Ed}}% 

{\devanagarifont न व्याधिर्न जरा तत्र न शोको ऽस्ति न विक्लवः \thinspace{\dandab} \dontdisplaylinenum }%
     \var{{\devanagarifont \numemph\va\textbf{व्याधिर्न}\lem \mssALL, व्याधि न \msCc\msNc\oo 
\textbf{जरा तत्र}\lem \msCb\msNc, जरास्तत्र \msCa\msCc\msNa\Ed}}% 
    \var{{\devanagarifont \numnoemph\vb\textbf{विक्लवः}\lem \mssALL, विक्लव \Ed}}% 

%Verse 2:8

{\devanagarifont नाधमः पुरुषस्तत्र नोत्तमो न च मध्यमः {॥ २:८॥} \veg\dontdisplaylinenum }%
 
{\devanagarifont नोत्कृष्टो मानवस्तस्मिन्स्त्रियश्चैव शिवालये \thinspace{\dandab} \dontdisplaylinenum }%
     \var{{\devanagarifont \numemph\va\textbf{मानव॰}\lem \mssALL, मा\lac  व॰ \msCa}}% 

%Verse 2:9

{\devanagarifont न निन्दा न प्रशंसास्ति मत्सरी पिशुनो न च {॥ २:९॥} \veg\dontdisplaylinenum }%
     \var{{\devanagarifont \numnoemph\vc\textbf{प्रशंसास्ति}\lem \mssALL, प्रशंसाश्च \Ed}}% 

{\devanagarifont गर्वदर्पं न तत्रास्ति क्रूरमायादिकं तथा \thinspace{\dandab} \dontdisplaylinenum }%
 
%Verse 2:10

{\devanagarifont याचमानो न तत्रास्ति दाता चैव न विद्यते {॥ २:१०॥} \veg\dontdisplaylinenum }%
     \var{{\devanagarifont \numemph\vc\textbf{तत्रास्ति}\lem \mssALL, तत्रा \msNaacorr}}% 

{\devanagarifont अनर्थी व्रज तत्रस्थः कल्पवृक्षसमाश्रितः \thinspace{\dandab} \dontdisplaylinenum }%
     \var{{\devanagarifont \numemph\va\textbf{व्रज त॰}\lem \mssALL, व्रजस्त॰ \msNc}}% 

%Verse 2:11

{\devanagarifont न कर्म नाप्रियस्तत्र न कलिः कलहो न च {॥ २:११॥} \veg\dontdisplaylinenum }%
     \var{{\devanagarifont \numnoemph\vc\textbf{कर्म ना॰}\lem \eme, कर्म न \mssCaCbCc\msNa\msNc, कर्मणा \Ed}}% 
    \var{{\devanagarifont \numnoemph\vd\textbf{कलिः}\lem \mssALL, कलि \msNcacorr\Ed}}% 

{\devanagarifont द्वापरो न च न त्रेता कृतं चापि न विद्यते \thinspace{\dandab} \dontdisplaylinenum }%
     \var{{\devanagarifont \numemph\va\textbf{च न त्रेता}\lem \mssALL, च न त्रेत्रा \msCa, च त्रेता न \msCb}}% 
    \var{{\devanagarifont \numnoemph\vb\textbf{कृतं चा॰}\lem \msCc\msNa, कृतश्चा॰ \msCa\msCb\msNc\Ed}}% 

%Verse 2:12

{\devanagarifont मन्वन्तरं न तत्रास्ति कल्पश्चैव न विद्यते {॥ २:१२॥} \veg\dontdisplaylinenum }%
     \var{{\devanagarifont \numnoemph\vc \lem \mssALL, मन्वन्तत्रास्ति \msCc, 
मन्वन्तरनन्त तत्रास्ति \msNc}}% 
    \var{{\devanagarifont \numnoemph\vd\textbf{कल्पश्चैव}\lem \mssALL, कल्पं चैव \msNa}}% 

{\devanagarifont आहूतसम्प्लवं नास्ति ब्रह्मरात्रिदिनं तथा \thinspace{\dandab} \dontdisplaylinenum }%
     \var{{\devanagarifont \numemph\va\textbf{आहूत॰}\lem \mssALL, आभूत॰ \Ed}}% 
    \var{{\devanagarifont \numnoemph\vb\textbf{ब्रह्मरात्रिदिनं}\lem \mssALL, ब्रह्मरात्रिदिवस् \Ed}}% 

%Verse 2:13

{\devanagarifont न जन्ममरणं तत्र आपदं नाप्नुयात्क्वचित् {॥ २:१३॥} \veg\dontdisplaylinenum }%
     \var{{\devanagarifont \numnoemph\vc\textbf{जन्ममरणं तत्र}\lem \msCc\msNa\Ed, जन्मरणं तत्र \msCa\msCb, 
जन्ममरणन्त्रत \msNc}}% 
    \var{{\devanagarifont \numnoemph\vd\textbf{आपदं}\lem \mssALL, अपदं \Ed}}% 

{\devanagarifont न चाशापाशबद्धो ऽस्ति रागमोहं न विद्यते \thinspace{\dandab} \dontdisplaylinenum }%
     \var{{\devanagarifont \numemph\va\textbf{चाशापाश॰}\lem \msCb\msNcpcorr, च सायाश॰ \msCa\msCc\msNa\msNcacorr\Ed\oo 
\textbf{॰बद्धो}\lem \mssALL, ॰द्धो \msCc, ॰वृद्धो \Ed}}% 
    \var{{\devanagarifont \numnoemph\vb\textbf{॰मोहं}\lem \mssALL, ॰मोहो \msCa}}% 

%Verse 2:14

{\devanagarifont न देवा नासुरास्तत्र न यक्षोरगराक्षसाः {॥ २:१४॥} \veg\dontdisplaylinenum }%
     \var{{\devanagarifont \numnoemph\vc\textbf{देवा}\lem \mssALL, देवो \msCb}}% 

{\devanagarifont न भूता न पिशाचाश्च गन्धर्वा ऋषयस्तथा \thinspace{\dandab} \dontdisplaylinenum }%
     \var{{\devanagarifont \numemph\vb\textbf{गन्धर्वा}\lem \mssALL,  गन्धर्वो \Ed}}% 

%Verse 2:15

{\devanagarifont ताराग्रहं न तत्रास्ति नागकिंनरगारुडम् {॥ २:१५॥} \veg\dontdisplaylinenum }%
 
{\devanagarifont न जपो नाह्निकस्तत्र नाग्निहोत्री न यज्ञकृत् \thinspace{\dandab} \dontdisplaylinenum }%
     \var{{\devanagarifont \numemph\va\textbf{जपो}\lem \mssALL, जयो \msCa\oo 
\textbf{नाह्निकस्त॰}\lem \mssALL, नाह्निक त॰ \msCb}}% 

%Verse 2:16

{\devanagarifont न व्रतं न तपश्चैव न तिर्यङ्नरकं तथा {॥ २:१६॥} \veg\dontdisplaylinenum }%
     \var{{\devanagarifont \numnoemph\vd\textbf{न तिर्यङ्नरकं}\lem \eme, नातिर्यन्नरकस् \msCa\msCc\msNa, 
नातिर्यनरकन् \msCb, नात्रिर्यं नरकस् \msNc, न तीर्थन्नरकन् \Ed}}% 
    \paral{{\devanagarifontsmall \vd {\englishfont \compare\ 19.49cd:} विसृष्टे त्विन्द्रियग्रामे तिर्यङ्नरकसाधनम् }}

{\devanagarifont तस्येशानस्य देवस्य ऐश्वर्यगुणविस्तरम् \thinspace{\dandab} \dontdisplaylinenum }%
     \paral{{\devanagarifontsmall \vc {\englishfont \compare\ \MBH\ Suppl. 14.4.2743:} ऐश्वर्यगुणसंपन्नाः क्रीडन्ति च यथासुखम्, 
                               {\englishfont and \BRAHMANDAPUR\ 1.26.1:} महादेवस्य महात्म्यं प्रभुत्वं च महात्मनः\thinspace{\devanagarifontsmall ।}  
                                                             श्रोतुमिच्छामहे सम्यगैश्वर्यगुणविस्तरम्\thinspace{\devanagarifontsmall ॥} }}

%Verse 2:17

{\devanagarifont अपि वर्षशतेनापि शक्यं वक्तुं न केनचित् {॥ २:१७॥} \veg\dontdisplaylinenum }%
 
{\devanagarifont हरेच्छाप्रभवाः सर्वे पर्यायेण ब्रवीमि ते \thinspace{\dandab} \dontdisplaylinenum }%
     \var{{\devanagarifont \numemph\va\textbf{हरेच्छाप्रभवाः}\lem \msNc, हरेच्छप्रभवाः \mssCaCbCc\msNa, हरेच्छाप्रभवा \Ed}}% 

%Verse 2:18

{\devanagarifont देवमानुषवर्ज्यानि वृक्षगुल्मलतादयः {॥ २:१८॥} \veg\dontdisplaylinenum }%
     \var{{\devanagarifont \numnoemph\vc\textbf{वर्ज्यानि}\lem \mssALL, वज्ज्ञानि \Ed}}% 

{\devanagarifont परार्धद्विगुणोत्सेधो विस्तारश्च तथाविधः \thinspace{\dandab} \dontdisplaylinenum }%
     \var{{\devanagarifont \numemph\va\textbf{॰गुणोत्सेधो}\lem \conj, ॰गुणोच्छेधा \msCa\msCb\msNa\msNc, ॰गुणेच्छेधा \msCc, ॰गुणाच्छ्रेधा \Ed}}% 
    \var{{\devanagarifont \numnoemph\vb\textbf{विस्तारश्च}\lem \msNc, विस्तारं च \mssCaCbCc\msNa\Ed\oo 
\textbf{॰विधः}\lem \msNc, ॰विधा \mssCaCbCc\msNa\Ed}}% 

%Verse 2:19

{\devanagarifont अनेकाकारपुष्पाणि फलानि च मनोहरम् {॥ २:१९॥} \veg\dontdisplaylinenum }%
     \var{{\devanagarifont \numnoemph\vc\textbf{अनेकाकार॰}\lem \mssALL, अनेकार॰ \msCa}}% 

{\devanagarifont अन्ये काञ्चनवृक्षाणि मणिवृक्षाण्यथापरे \thinspace{\dandab} \dontdisplaylinenum }%
     \var{{\devanagarifont \numemph\va\textbf{अन्ये}\lem \mssALL, बहु॰ \Ed}}% 

%Verse 2:20

{\devanagarifont प्रवालमणिषण्डाश्च पद्मरागरुहाणि च {॥ २:२०॥} \veg\dontdisplaylinenum }%
     \var{{\devanagarifont \numnoemph\vc\textbf{षण्डाश्च}\lem \mssALL, घण्टाश्च \Ed}}% 
    \var{{\devanagarifont \numnoemph\vd\textbf{॰रुहाणि}\lem \msCc, ॰रुहानि \msCa\msCb\msNa\msNc, ॰सहानि \Ed}}% 

{\devanagarifont स्वादुमूलफलाः स्कन्धलताविटपपादपाः \thinspace{\dandab} \dontdisplaylinenum }%
     \var{{\devanagarifont \numemph\va\textbf{स्वादु॰}\lem \mssALL, स्वाधु॰ \msCa\oo 
\textbf{॰मूल॰}\lem \mssALL, ॰मूला \msNa\oo 
\textbf{॰फलाः}\lem \conj, ॰फला \mssCaCbCc\msNa\msNc\Ed}}% 
    \var{{\devanagarifont \numnoemph\vb\textbf{स्कन्ध॰}\lem \conj, स्कन्द॰ \mssCaCbCc\msNa\msNc\Ed}}% 

%Verse 2:21

{\devanagarifont कामरूपाश्च ते सर्वे कामदाः कामभाषिणः {॥ २:२१॥} \veg\dontdisplaylinenum }%
     \lacuna{\devanagarifontsmall \vc {\englishfont After }कामरू॰, {\englishfont \msCc\ has two folios missing (ff.\ 271--272) and resumes only at 3.30b} }%
  
{\devanagarifont तत्र विप्र प्रजाः सर्वे अनन्तगुणसागराः \thinspace{\dandab} \dontdisplaylinenum }%
 
%Verse 2:22

{\devanagarifont तुल्यरूपबलाः सर्वे सूर्यायुतसमप्रभाः {॥ २:२२॥} \veg\dontdisplaylinenum }%
     \var{{\devanagarifont \numemph\vc\textbf{॰बलाः}\lem \mssALL, ॰वराः \Ed}}% 

{\devanagarifont परार्धद्वयविस्तारं परार्धद्वयमायतम् \thinspace{\dandab} \dontdisplaylinenum }%
 
%Verse 2:23

{\devanagarifont परार्धद्वयविक्षेपं योजनानां द्विजोत्तम {॥ २:२३॥} \veg\dontdisplaylinenum }%
     \var{{\devanagarifont \numemph\vc\textbf{॰द्वय॰}\lem \mssALL, ॰द्व॰ \msNaacorr\oo 
\textbf{विक्षेपं}\lem \eme, विक्षेपा \msCa\msCb\msNa\msNb\msNc, विज्ञेया \Ed}}% 
    \var{{\devanagarifont \numnoemph\vd\textbf{॰त्तम}\lem \mssALL, ॰त्तमः \msNa}}% 

{\devanagarifont ऐश्वर्यत्वं न संख्यास्ति बलशक्तिश्च भो द्विज \thinspace{\dandab} \dontdisplaylinenum }%
     \var{{\devanagarifont \numemph\vb \lem \mssALL, 
\om\ \msNaacorr, तव शक्तिश्च भो द्विज \Ed}}% 

%Verse 2:24

{\devanagarifont अधोर्ध्वो न च संख्यास्ति न तिर्यञ्चैति कश्चन {॥ २:२४॥} \veg\dontdisplaylinenum }%
     \var{{\devanagarifont \numnoemph\vc \lem \mssALL, \om\ \msNaacorr}}% 
    \var{{\devanagarifont \numnoemph\vd \lem \msNapcorr\msNc, 
न तिर्यञ्चेति कश्चन \msCa\msCb\msNb\Ed, 
न तिर्यं चेति कश्चन \msNaacorr}}% 

{\devanagarifont शिवाण्डस्य च विस्तारमायामं च न वेद्म्यहम् \thinspace{\dandab} \dontdisplaylinenum }%
 
%Verse 2:25

{\devanagarifont भोगमक्षय तत्रैव जन्ममृत्युर्न विद्यते {॥ २:२५॥} \veg\dontdisplaylinenum }%
     \var{{\devanagarifont \numemph\vc\textbf{भोगमक्षय त॰}\lem \eme, भोगमक्षयस्त॰ \msCa\msCb\msNa\msNb\msNc\ \unmetr, 
भोगमयास्तु त॰ \Ed}}% 
    \var{{\devanagarifont \numnoemph\vd\textbf{॰मृत्युर्न}\lem \mssALL, ॰मृत्यु न \msNb}}% 

{\devanagarifont शिवाण्डमध्यमाश्रित्य गोक्षीरसदृशप्रभाः \thinspace{\dandab} \dontdisplaylinenum }%
     \var{{\devanagarifont \numemph\vb\textbf{प्रभाः}\lem \mssALL, प्रभा \Ed}}% 

%Verse 2:26

{\devanagarifont परार्धपरकोटीनामीशानानां स्मृतालयः {॥ २:२६॥} \veg\dontdisplaylinenum }%
     \var{{\devanagarifont \numnoemph\vd\textbf{॰शानानां}\lem \mssALL, ॰शानाना \msNb, ॰गानानां \msNc\oo 
\textbf{स्मृतालयः}\lem \msCa\msNb\msNc, स्मृतालय \msCb, स्मृतालयं \msNa, स्मृतालया \Ed}}% 

{\devanagarifont बालसूर्यप्रभाः सर्वे ज्ञेयास्तत्पुरुषालये \thinspace{\dandab} \dontdisplaylinenum }%
     \var{{\devanagarifont \numemph\va\textbf{॰भाः}\lem \mssALL, ॰भा \Ed}}% 
    \var{{\devanagarifont \numnoemph\vb\textbf{ज्ञेयास्त॰}\lem \mssALL, ज्ञेया त॰ \msNa\Ed\oo 
\textbf{॰आलये}\lem \mssALL, ॰आलयं \Ed}}% 

%Verse 2:27

{\devanagarifont परार्धपरकोटीनां पूर्वस्यां दिशमाश्रिताः {॥ २:२७॥} \veg\dontdisplaylinenum }%
     \var{{\devanagarifont \numnoemph\vd\textbf{दिश॰}\lem \mssALL, दिशि॰ \msNb}}% 

{\devanagarifont भिन्नाञ्जनप्रभाः सर्वे दक्षिणां दिशमाश्रिताः \thinspace{\dandab} \dontdisplaylinenum }%
     \var{{\devanagarifont \numemph\va\textbf{॰प्रभाः}\lem \mssALL, ॰प्रभा \Ed}}% 
    \var{{\devanagarifont \numnoemph\vb\textbf{दक्षिणां}\lem \mssALL, दक्षिण॰ \Ed\oo 
\textbf{दिशम्}\lem \mssALL, दिशिम् \msCb\Ed}}% 

%Verse 2:28

{\devanagarifont परार्धपरकोटीनामघोरालयमाश्रिताः {॥ २:२८॥} \veg\dontdisplaylinenum }%
     \var{{\devanagarifont \numnoemph\vd\textbf{॰घोरा॰}\lem \mssALL, ॰धोरा॰ \Ed\oo 
\textbf{॰श्रिताः}\lem \mssALL, ॰श्रिता \Ed}}% 

{\devanagarifont कुन्देन्दुहिमशैलाभाः पश्चिमां दिशमाश्रिताः \thinspace{\dandab} \dontdisplaylinenum }%
     \var{{\devanagarifont \numemph\vb\textbf{पश्चिमां}\lem \mssALL, पश्चिमा \msCb\oo 
\textbf{दिश॰}\lem \mssALL, दिशि॰ \msNc\oo 
\textbf{॰श्रिताः}\lem \mssALL, ॰श्रिता \Ed}}% 

%Verse 2:29

{\devanagarifont परार्धपरकोटीनां सद्यमिष्टालयः स्मृतः {॥ २:२९॥} \veg\dontdisplaylinenum }%
     \var{{\devanagarifont \numnoemph\vd\textbf{सद्यमिष्टा॰}\lem \mssALL, सद्यमिष्ट्वा॰ \msNa\oo 
\textbf{स्मृतः}\lem \mssALL, स्मृताः \msCb}}% 

{\devanagarifont कुङ्कुमोदकसंकाशा उत्तरां दिशमाश्रिताः \thinspace{\dandab} \dontdisplaylinenum }%
     \var{{\devanagarifont \numemph\vb\textbf{उत्तरां}\lem \mssALL, उत्तरा \msCb\oo 
\textbf{दिशम्}\lem \mssALL, दिशिम् \msCa}}% 

%Verse 2:30

{\devanagarifont परार्धपरकोतीनां वामदेवालयः स्मृतः {॥ २:३०॥} \veg\dontdisplaylinenum }%
     \var{{\devanagarifont \numnoemph\vd\textbf{॰लयः}\lem \mssALL, ॰लय \msNc}}% 

{\devanagarifont ईशानस्य कलाः पञ्च वक्त्रस्यापि चतुष्कलाः \thinspace{\dandab} \dontdisplaylinenum }%
     \var{{\devanagarifont \numemph\va\textbf{कलाः}\lem \mssALL, कला \Ed}}% 
    \var{{\devanagarifont \numnoemph\vb\textbf{चतुष्कलाः}\lem \mssALL, चतुष्तके \Ed}}% 

%Verse 2:31

{\devanagarifont अघोरस्य कला अष्टौ वामदेवास्त्रयोदश {॥ २:३१॥} \veg\dontdisplaylinenum }%
     \var{{\devanagarifont \numnoemph\vd\textbf{वामदेवा॰}\lem \mssALL, वामदेव॰ \msNb}}% 

{\devanagarifont सद्यश्चाष्टौ कला ज्ञेयाः संसारार्णवतारकाः \thinspace{\dandab} \dontdisplaylinenum }%
     \var{{\devanagarifont \numemph\va\textbf{ज्ञेयाः}\lem \mssALL, ज्ञेया \Ed}}% 
    \var{{\devanagarifont \numnoemph\vb\textbf{संसारा॰}\lem \mssALL, संसा॰ \msCbacorr}}% 

%Verse 2:32

{\devanagarifont अष्टत्रिंशत्कला ह्येताः कीर्तिता द्विजसत्तम {॥ २:३२॥} \veg\dontdisplaylinenum }%
     \var{{\devanagarifont \numnoemph\vc\textbf{॰त्रिंशत्क॰}\lem \corr, ॰त्रिंशक॰ \msCa\msCb\msNa\msNb\msNc\Ed\oo 
\textbf{ह्येताः}\lem \mssALL, ज्ञेयाः \Ed}}% 
    \var{{\devanagarifont \numnoemph\vd\textbf{॰सत्तम}\lem \mssALL, ॰सत्तमः \msNb\Ed}}% 

{\devanagarifont संख्या वर्णा दिशश्चैव एकैकस्य पृथक्पृथक् \thinspace{\dandab} \dontdisplaylinenum }%
     \var{{\devanagarifont \numemph\va\textbf{संख्या वर्णा}\lem \msCb\msNc, संख्या वर्ण्णो \msCa\msNb, संख्या वण्णा \msNa, संध्या वर्णा \Ed}}% 
    \var{{\devanagarifont \numnoemph\vb\textbf{एकैकस्य}\lem \mssALL, ऐकैकस्य \msCb\msNa}}% 

%Verse 2:33

{\devanagarifont पूर्वोक्तेन विधानेन बोधव्यास्तत्त्वचिन्तकैः {॥ २:३३॥} \veg\dontdisplaylinenum }%
     \var{{\devanagarifont \numnoemph\vd\textbf{बोधव्यास्त॰}\lem \eme, बोधव्या त॰ \msCa\msCb\msNa\msNb\msNc\Ed}}% 

{\devanagarifont शिवाण्डगमनाकृष्ट्या शिवयोगं सदाभ्यसेत् \thinspace{\dandab} \dontdisplaylinenum }%
     \var{{\devanagarifont \numemph\va\textbf{॰कृष्ट्या}\lem \mssALL, कृष्टा \msNa\msNc}}% 
    \var{{\devanagarifont \numnoemph\vb\textbf{योगं सदाभ्यसेत्}\lem \mssALL, योग समभ्यसेत् \msNb}}% 

%Verse 2:34

{\devanagarifont शिवयोगं विना विप्र तत्र गन्तुं न शक्यते {॥ २:३४॥} \veg\dontdisplaylinenum }%
     \var{{\devanagarifont \numnoemph\vc\textbf{॰योगं}\lem \mssALL, ॰योग \Ed}}% 

{\devanagarifont अश्वमेधादियज्ञानां कोट्यायुतशतानि च \thinspace{\dandab} \dontdisplaylinenum }%
 
{\devanagarifont कृच्छ्रादितप सर्वाणि कृत्वा कल्पशतानि च  \danda\dontdisplaylinenum }%
     \var{{\devanagarifont \numemph\vc\textbf{॰तप}\lem \Ed, ॰तपः \msCa\msCb\msNa\msNb\msNc\ \unmetr}}% 

%Verse 2:35

{\devanagarifont तत्र गन्तुं न शक्येत देवैरपि तपोधन {॥ २:३५॥} \veg\dontdisplaylinenum }%
     \var{{\devanagarifont \numnoemph\ve\textbf{शक्येत}\lem \mssALL, शक्यैत \msCb, शक्येते \Ed}}% 
    \var{{\devanagarifont \numnoemph\vf\textbf{देवै॰}\lem \mssALL, देवे॰ \msNc\oo 
\textbf{॰धन}\lem \mssALL, ॰धनम् \msCb}}% 

{\devanagarifont गङ्गादिसर्वतीर्थेषु स्नात्वा तप्त्वा च वै पुनः \thinspace{\dandab} \dontdisplaylinenum }%
 
%Verse 2:36

{\devanagarifont तत्र गन्तुं न शक्येत ऋषिभिर्वा महात्मभिः {॥ २:३६॥} \veg\dontdisplaylinenum }%
     \var{{\devanagarifont \numemph\vc\textbf{गन्तुं}\lem \mssALL, गन्तु \msNb\msNc\oo 
\textbf{शक्येत}\lem \mssALL, शक्यन्ते \Ed}}% 

{\devanagarifont सप्तद्वीपसमुद्राणि रत्नपूर्णानि भो द्विज \thinspace{\dandab} \dontdisplaylinenum }%
     \var{{\devanagarifont \numemph\va\textbf{॰द्वीप॰}\lem \mssALL, ॰दीप॰ \msNc\oo 
\textbf{॰समुद्राणि}\lem \mssALL, ॰समुद्राय \msNb}}% 
    \paral{{\devanagarifontsmall \vab {\englishfont Cf. \SDHU\ 2.104:} त्रिः प्रदत्वा महीं पूर्णां{\englishfont ...} }}

{\devanagarifont दत्त्वा वा वेदविदुषे श्रद्धाभक्तिसमन्वितः  \danda\dontdisplaylinenum }%
 
%Verse 2:37

{\devanagarifont तत्र गन्तुं न शक्येत विना ध्यानेन निश्चयः {॥ २:३७॥} \veg\dontdisplaylinenum }%
     \var{{\devanagarifont \numnoemph\ve\textbf{गन्तुं}\lem \mssALL, गन्तु \msNb, गंन्तु \msNc\oo 
\textbf{शक्येत}\lem \mssALL, शक्यन्ते \Ed}}% 

{\devanagarifont स्वदेहान्मांसमुद्धृत्य दत्त्वार्थिभ्यश्च निश्चयात् \thinspace{\dandab} \dontdisplaylinenum }%
     \var{{\devanagarifont \numemph\va\textbf{स्वदेहान्मांस॰}\lem \mssALL, स्वदेहात्मांस॰ \msNc, स्वदेहात्मां स॰ \Ed}}% 

{\devanagarifont स्वदारपुत्रसर्वस्वं शिरो ऽर्थिभ्यश्च यो ददेत्  \danda\dontdisplaylinenum }%
     \var{{\devanagarifont \numnoemph\vc\textbf{॰स्वं}\lem \mssALL, ॰स्व \msNb}}% 

%Verse 2:38

{\devanagarifont न तत्र गन्तुं शक्येत अन्यैर्वापि सुदुष्करैः {॥ २:३८॥} \veg\dontdisplaylinenum }%
     \var{{\devanagarifont \numnoemph\ve\textbf{न तत्र गन्तुं}\lem \mssALL, न तत्र गन्तुं न \msCb}}% 
    \var{{\devanagarifont \numnoemph\vf\textbf{॰दुष्करैः}\lem \mssALL, ॰दुष्कृतः \msNb}}% 

{\devanagarifont यज्ञतीर्थतपोदानवेदाध्ययनपारगः \thinspace{\dandab} \dontdisplaylinenum }%
     \var{{\devanagarifont \numemph\va\textbf{॰दान॰}\lem \mssALL, ॰दानं \msNa, ॰दानै \msNb}}% 
    \var{{\devanagarifont \numnoemph\vb\textbf{॰पारगः}\lem \mssALL, ॰पारगाः \msCa\msNb}}% 

%Verse 2:39

{\devanagarifont ब्रह्माण्डान्तस्य भोगांस्तु भुङ्क्ते कालवशानुगः {॥ २:३९॥} \veg\dontdisplaylinenum }%
     \var{{\devanagarifont \numnoemph\vc \lem \mssALL, 
ब्रह्माण्डान्तस्य भोगास्तु \msNb, 
ब्रह्माण्डात्तस्य भोगास्तु \Ed}}% 
    \var{{\devanagarifont \numnoemph\vd\textbf{भुङ्क्ते}\lem \mssALL, \uncl{भुङ्क्ते} \msNc, भुक्त्वा \Ed\oo 
\textbf{॰गः}\lem \mssALL, ॰गाः \msNaacorr}}% 

{\devanagarifont कालेन समप्रेष्येण धर्मो याति परिक्षयम् \thinspace{\dandab} \dontdisplaylinenum }%
     \var{{\devanagarifont \numemph\vb\textbf{धर्मो}\lem \mssALL, धर्मे \msNc}}% 

{\devanagarifont अलातचक्रवत्सर्वं कालो याति परिभ्रमन्  \danda\dontdisplaylinenum }%
 
%Verse 2:40

{\devanagarifont त्रैकाल्यकलनात्कालस्तेन कालः प्रकीर्तितः {॥ २:४०॥} \veg\dontdisplaylinenum }%
     \var{{\devanagarifont \numnoemph\ve\textbf{॰कलनात्काल॰}\lem \mssALL, ॰कलना काल॰ \msNb}}% 

{\devanagarifont 
\jump
\begin{center}
\ketdanda~इति वृषसारसंग्रहे शिवाण्डसंख्या नामाध्यायो द्वितीयः~\ketdanda
\end{center}
\dontdisplaylinenum\vers  }%
     \var{{\devanagarifont \numnoemph{\englishfont \Colo:}\textbf{नामाध्यायो द्वितीयः}\lem \mssALL, 
नामाध्याय द्वितीयः \msNb, 
नाम द्वितीयो ऽध्यायः \Ed}}% 
\bekveg\szamveg
\vfill
\phpspagebreak

\versno=0\fejno=3
\thispagestyle{empty}

\centerline{\Large\devanagarifontbold [   तृतीयो ऽध्यायः  ]}{\vrule depth10pt width0pt} \fancyhead[CO]{{\footnotesize\devanagarifont वृषसारसंग्रहे  }}
\fancyhead[CE]{{\footnotesize\devanagarifont तृतीयो ऽध्यायः  }}
\fancyhead[LE]{}
\fancyhead[RE]{}
\fancyhead[LO]{}
\fancyhead[RO]{}
\szam\bek



\alalfejezet{धर्मप्रवचनम्}
\vers


{\devanagarifont विगतराग उवाच {\dandab}\dontdisplaylinenum  }%
 
{\devanagarifont किमर्थं धर्ममित्याहुः कतिमूर्तिश्च कीर्त्यते \thinspace{\danda} \dontdisplaylinenum }%
     \var{{\devanagarifont \numemph\va\textbf{आहुः}\lem \mssALL, आहु \Ed}}% 
    \lacuna{\devanagarifontsmall {\englishfont Witnesses used for this chapter: \msParis\ exp.\thinspace 215r--215v (breaks off after 3.14d and resumes at 4.8a),
                                             \msCa\ ff.\thinspace 197r--198v, 
                                             \msCb\ ff.\thinspace 204v--206r, 
                                             \msCc\ ff.\thinspace 273r--273v (broke off at 2.21 and resumes at 3.30b),
                                             \msNa\ ff.\thinspace 4v--6r, 
                                             \msNb\ exp.\thinspace 42, 47 (upper), 48 (lower),
                                             \msNc\ ff.\thinspace 213r--214v,
                                             \Ed\ pp.\thinspace 588--591;
                                        \mssCaCbCc\ = \msCa + \msCb + \msCc } }%
  
%Verse 3:1

{\devanagarifont कतिपादवृषो ज्ञेयो गतिस्तस्य कति स्मृताः {॥ ३:१॥} \veg\dontdisplaylinenum }%
     \var{{\devanagarifont \numnoemph\vd\textbf{स्मृताः}\lem \mssALL, स्मृता \msCb, स्मृतः \Ed}}% 

{\devanagarifont कौतूहलं ममोत्पन्नं संशयं छिन्धि तत्त्वतः \thinspace{\dandab} \dontdisplaylinenum }%
     \var{{\devanagarifont \numemph\va\textbf{कौतूहलं}\lem \mssALL, कौतुहल \Ed\oo 
\textbf{ममोत्पन्नं}\lem \mssALL, समोत्पन्नं \msNc}}% 
    \var{{\devanagarifont \numnoemph\vb\textbf{संशयं}\lem \mssALL, सशयं \msCa}}% 

%Verse 3:2

{\devanagarifont कस्य पुत्रो मुनिश्रेष्ठ प्रजास्तस्य कति स्मृताः {॥ ३:२॥} \veg\dontdisplaylinenum }%
 
{\devanagarifont अनर्थयज्ञ उवाच {\dandab}\dontdisplaylinenum  }%
 
{\devanagarifont धृतिरित्येष धातुर्वै पर्यायः परिकीर्तितः \thinspace{\danda} \dontdisplaylinenum }%
 
%Verse 3:3

{\devanagarifont आधारणान्महत्त्वाच्च धर्म इत्यभिधीयते {॥ ३:३॥} \veg\dontdisplaylinenum  }%
     \var{{\devanagarifont \numemph\vc\textbf{आधारणान्म॰}\lem \msParis\msCa\msNb, आधारणात्प॰ \msCb, आधारणात्म॰ \msNa\msNc, आधारेण म॰ \Ed}}% 
    \var{{\devanagarifont \numnoemph\vd\textbf{इत्यभिधीयते}\lem \msCa\msNa\msNc\Ed, इ\uncl{त्यभिधीयते} \msParis, 
इत्यविधीयते \msCb\msNb}}% 
    \paral{{\devanagarifontsmall \vcd {\englishfont \compare\ \LINPU\ 1.10.12cd--13ab:}
                         धारणार्थे महान्ह्येष धर्मशब्दः प्रकीर्तितः\thinspace{\devanagarifontsmall ॥}
                         अधारणे ऽमहत्त्वे च अधर्म इति चोच्यते\thinspace{\devanagarifontsmall ।}
                \vo\ {\englishfont \compare\ \BRAHMANDAPUR\ 1.32.29:}
                         धारणार्थो धृतिश्चैव धातुः शब्दे प्रकीर्तितः\thinspace{\devanagarifontsmall ।}
                         अधारणामहत्त्वे च अधर्म इति चोच्यते\thinspace{\devanagarifontsmall ॥}
                     {\englishfont \compare\ \VAYUP\ 1.59.28:}
                         धारणा धृतिरित्यर्थाद्धातोर्धर्मः प्रकीर्तितः\thinspace{\devanagarifontsmall ।}
                         अधारणे ऽमहत्त्वे च अधर्म इति चोच्यते\thinspace{\devanagarifontsmall ॥}
                     {\englishfont \compare\ \MATSP\ 145.27:}  
                         धर्मेति धारणे धातुर्महत्वे चैव उच्यते\thinspace{\devanagarifontsmall ।}
                         आधारणे महत्त्वे वा धर्मः स तु निरुच्यते\thinspace{\devanagarifontsmall ।} }}

{\devanagarifont श्रुतिस्मृतिद्वयोर्मूर्तिश्चतुष्पादवृषः स्थितः \thinspace{\dandab} \dontdisplaylinenum }%
     \var{{\devanagarifont \numemph\vab\textbf{॰स्मृतिद्वयोर्मूर्तिश्च॰}\lem \msCa, ॰स्मृतिद्वयो मूर्त्तिश्च॰ \msParis\msCb\msNb, 
॰स्मृतिद्वयो मूर्त्ति च॰ \msNa\msNc, 
॰स्मृतिर्द्वयो मूर्तिश्च \Ed}}% 
    \var{{\devanagarifont \numnoemph\vb\textbf{॰वृषः}\lem \mssALL, ॰वृष \msNc}}% 

%Verse 3:4

{\devanagarifont चतुराश्रम यो धर्मः कीर्तितानि मनीषिभिः {॥ ३:४॥} \veg\dontdisplaylinenum }%
     \var{{\devanagarifont \numnoemph\vc\textbf{चतुरा॰}\lem \mssALL, चातुरा॰ \msCa\msNc}}% 
    \paral{{\devanagarifontsmall \vo {\englishfont \compare\ 4.74 below:}
                 चतुष्पादः स्मृतो धर्मश्चतुराश्रममाश्रितः\thinspace{\devanagarifontsmall ।}
                 गृहस्थो ब्रह्मचारी च वानप्रस्थो ऽथ भैक्षुकः\thinspace{\devanagarifontsmall ॥} }}

{\devanagarifont गतिश्च पञ्च विज्ञेयाः शृणु धर्मस्य भो द्विज \thinspace{\dandab} \dontdisplaylinenum }%
     \var{{\devanagarifont \numemph\va\textbf{विज्ञेयाः}\lem \eme, विज्ञेयः \msParis\msCa\msNa\msNb\msNc\Ed, \om\ \msCb}}% 
    \lacuna{\devanagarifontsmall \vab {\englishfont \msCb\ reads here } गतिश्च पौत्राश्च अनेकाश्च बभूव ह,
                        {\englishfont skipping to 3.7cd, omitting 3.5--7ab.} }%
  
%Verse 3:5

{\devanagarifont देवमानुषतिर्यं च नरकस्थावरादयः {॥ ३:५॥} \veg\dontdisplaylinenum }%
     \var{{\devanagarifont \numnoemph\vc\textbf{॰मानुष॰}\lem \mssALL, ॰मानुषि॰ \msParis}}% 

{\devanagarifont ब्रह्मणो हृदयं भित्त्वा जातो धर्मः सनातनः \thinspace{\dandab} \dontdisplaylinenum }%
     \var{{\devanagarifont \numemph\va\textbf{ब्रह्मणो}\lem \mssALL, \om\ \msCb, ब्राह्मणो \Ed\oo 
\textbf{भित्त्वा}\lem \mssALL, वित्त्वा \msNb}}% 
    \var{{\devanagarifont \numnoemph\vb\textbf{धर्मः}\lem \mssALL, धर्म \msNb}}% 
    \paral{{\devanagarifontsmall \vab {\englishfont \compare\ \DEVIP\ 4.59cd:} ब्रह्मणो हृदयाज्जातः पुत्रो धर्म इति स्मृतः \oo 
                     {\englishfont \compare\ also \MBH\ 1.60.40ab:} ब्रह्मणो हृदयं भित्त्वा निःसृतो भगवान्भृगुः }}

%Verse 3:6

{\devanagarifont तस्य पत्नी महाभागा त्रयोदश सुमध्यमाः {॥ ३:६॥} \veg\dontdisplaylinenum }%
     \var{{\devanagarifont \numnoemph\vd\textbf{॰मध्यमाः}\lem \mssALL, \om\ \msCb}}% 

{\devanagarifont दक्षकन्या विशालाक्षी श्रद्धाद्या सुमनोहराः \thinspace{\dandab} \dontdisplaylinenum }%
     \var{{\devanagarifont \numemph\va\textbf{॰आक्षी}\lem \mssALL, \om\ \msCb, ॰आक्षि \Ed}}% 
    \var{{\devanagarifont \numnoemph\vb\textbf{॰आद्या}\lem ॰आद्या \msParis\msNb\msNc\Ed, ॰आढ्या \msCa, \om\ \msCb, ॰आढ्याः \msNa\oo 
\textbf{॰हराः}\lem \msNb\Ed, ॰हरा \msParis\msCa\msNc,  \om\ \msCb, ॰\lk \uncl{माः} \msNa}}% 

{\devanagarifont तस्य पुत्राश्च पौत्राश्च अनेकाश्च बभूव ह  \danda\dontdisplaylinenum }%
     \var{{\devanagarifont \numnoemph\vcd \lem \msParis\msCa\msNb, 
गतिश्च पौत्राश्च अनेकाश्च बभूव ह {\englishfont (eyeskip to 3.5a)} \msCb, 
तस्य पुत्राश्च योत्राश्च अनेकाश्च बभूव ह \msNa\msNc, 
तस्य पुत्रा अनेकाश्च तथा पौत्रा बभूवहः \Ed}}% 

%Verse 3:7

{\devanagarifont एष धर्मनिसर्गो ऽयं किं भूयः श्रोतुमिच्छसि {॥ ३:७॥} \veg\dontdisplaylinenum }%
 
{\devanagarifont विगतराग उवाच {\dandab}\dontdisplaylinenum  }%
     \var{{\devanagarifont \numemph\vo\textbf{विगतराग उवाच}\lem \msCb\msNapcorr\msNc\Ed, विगतराग उ \msParis\msCa\msNb, \om\ \msNaacorr}}% 

{\devanagarifont धर्मपत्नी विशेषेण पुत्रस्तेभ्यः पृथक्पृथक् \thinspace{\danda} \dontdisplaylinenum }%
 
%Verse 3:8

{\devanagarifont श्रोतुमिच्छामि तत्त्वेन कथयस्व तपोधन {॥ ३:८॥} \veg\dontdisplaylinenum }%
 
{\devanagarifont अनर्थयज्ञ उवाच {\dandab}\dontdisplaylinenum  }%
 
{\devanagarifont श्रद्धा लक्ष्मीर्धृतिस्तुष्टिः पुष्टिर्मेधा क्रिया लज्जा \thinspace{\danda} \dontdisplaylinenum }%
     \var{{\devanagarifont \numemph\va\textbf{लक्ष्मीर्धृतिस्तुष्टिः}\lem \msCa, 
लक्ष्मी धृतिस्तुष्टिः \msParis\msNc, 
लक्ष्मीर्धृतिस्तुष् \msCb, 
लक्ष्मी द्धृतिर्द्धृतिस्तुष्टिः \msNaacorr, 
लक्ष्मीर्द्धृतिस्तुष्टिः \msNapcorr, 
लक्ष्मीं धृति तुष्टिः \msNb, 
लक्ष्मी धृतिस्तुष्टी \Ed}}% 
    \var{{\devanagarifont \numnoemph\vb\textbf{पुष्टिर्मे॰}\lem \mssALL, पुष्टि मे॰ \Ed\oo 
\textbf{लज्जा}\lem \mssALL, लजा \msNa}}% 

%Verse 3:9

{\devanagarifont बुद्धिः शान्तिर्वपुः कीर्तिः सिद्धिः प्रसूतिसम्भवाः {॥ ३:९॥} \veg\dontdisplaylinenum }%
     \var{{\devanagarifont \numnoemph\vc\textbf{बुद्धिः}\lem \mssALL, बुद्धि \msCa}}% 
    \var{{\devanagarifont \numnoemph\vd \lem \conj, सिद्धिश्चाभूतिसम्भवाः \msParis, 
सिद्धिश्चाभूतिसम्भवा \msCa\msNa\msNb\msNc, 
सिद्धिश्चातिसम्भवा \msCb, सिद्धिश्च भूतिसम्भवा \Ed}}% 

{\devanagarifont श्रद्धा कामः सुतो जातो दर्पो लक्ष्मीसुतः स्मृतः \thinspace{\dandab} \dontdisplaylinenum }%
     \var{{\devanagarifont \numemph\va\textbf{कामः}\lem \msNa, काम॰ \msParis\msCa\msCb\msNb\msNc, धर्म॰ \Ed}}% 

%Verse 3:10

{\devanagarifont धृत्यास्तु नियमः पुत्रः संतोषस्तुष्टिजः स्मृतः {॥ ३:१०॥} \veg\dontdisplaylinenum }%
     \paral{{\devanagarifontsmall \vo {\englishfont See a passage similar to \VSS\ 3.10--13,
         e.g., in \KURMP\ 1.8.20 ff.:}
         श्रद्धाया आत्मजः कामो दर्पो लक्ष्मीसुतः स्मृतः\thinspace{\devanagarifontsmall ।}
         धृत्यास्तु नियमः पुत्रस्तुष्ट्याः संतोष उच्यते\thinspace{\devanagarifontsmall ॥} 
         पुष्ट्या लाभः सुतश्चापि मेधापुत्रः श्रुतस्तथा\thinspace{\devanagarifontsmall ।} 
         क्रियायाश्चाभवत्पुत्रो दण्डः समय एव च\thinspace{\devanagarifontsmall ॥}  
         बुद्ध्या बोधः सुतस्तद्वदप्रमादो व्यजायत\thinspace{\devanagarifontsmall ।} 
         लज्जाया विनयः पुत्रो वपुषो व्यवसायकः\thinspace{\devanagarifontsmall ॥}  
         क्षेमः शान्तिसुतश्चापि सुखं सिद्धिरजायत\thinspace{\devanagarifontsmall ।}
         यशः कीर्तिसुतस्तद्वदित्येते धर्मसूनवः\thinspace{\devanagarifontsmall ॥}   
         कामस्य हर्षः पुत्रो ऽभूद्देवानन्दो व्यजायत\thinspace{\devanagarifontsmall ।}  
         इत्येष वै सुखोदर्कः सर्गो धर्मस्य कीर्तितः\thinspace{\devanagarifontsmall ॥} }}

{\devanagarifont पुष्ट्या लाभः सुतो जातो मेधापुत्रः श्रुतस्तथा \thinspace{\dandab} \dontdisplaylinenum }%
     \var{{\devanagarifont \numemph\va\textbf{लाभः}\lem \mssALL, लाभ॰ \msNa\Ed\oo 
\textbf{जातो}\lem \mssALL, \om\ \msParis}}% 
    \var{{\devanagarifont \numnoemph\vb\textbf{॰पुत्रः}\lem \eme, ॰पुत्र \msParis\msCa\msCb\msNa\msNb\msNc\Ed\oo 
\textbf{श्रुत॰}\lem \mssALL, श्रुति॰ \msParis, श्रत॰ \msCb}}% 

%Verse 3:11

{\devanagarifont क्रियायास्त्वभवत्पुत्रो दण्डः समय एव च {॥ ३:११॥} \veg\dontdisplaylinenum }%
     \var{{\devanagarifont \numnoemph\vc\textbf{त्वभवत्पुत्रो}\lem \eme, त्वभयः पुत्रो \msParis\msCa\msCb\msNa\msNb\msNc, तूभयः पुत्रौ \Ed}}% 
    \var{{\devanagarifont \numnoemph\vd\textbf{दण्डः}\lem \corr, दण्डे \msCa\msNaacorr दण्ड॰ \msParis\msNapcorr\msNb\msNc\Ed, दण्डो \msCb\oo 
\textbf{च}\lem \mssALL, तु \Ed}}% 
    \paral{{\devanagarifontsmall \vcd {\englishfont \similar\ \LINPU\ 1.70.295ab:}क्रियायामभवत्पुत्रो दण्डः समय एव च;
                     {\englishfont \similar\ \KURMP\ 1.8.22cd:   }क्रियायाश्चाभवत्पुत्रो दण्डः समय एव च;
                     {\englishfont \compare\ \LINPU\ 1,5.37:     }धर्मस्य वै क्रियायां तु दण्डः समय एव च }}

{\devanagarifont लज्जाया विनयः पुत्रो बुद्ध्या बोधःसुतः स्मृतः \thinspace{\dandab} \dontdisplaylinenum }%
     \var{{\devanagarifont \numemph\va\textbf{लज्जाया विनयः}\lem \mssALL, लज्जायाः विनय॰ \Ed}}% 
    \var{{\devanagarifont \numnoemph\vb\textbf{सुतः स्मृतः}\lem \mssALL, सुतः \lk\lk\ \msCa, सुतःस्तथा \msCb}}% 

%Verse 3:12

{\devanagarifont लज्जायाः सुधियः पुत्र अप्रमादश्च तावुभौ {॥ ३:१२॥} \veg\dontdisplaylinenum }%
     \var{{\devanagarifont \numnoemph\vc\textbf{सुधियः}\lem \Ed, सुधिय \msParis\msCa\msCb\msNa\msNb\msNc\oo 
\textbf{पुत्र}\lem \mssALL, पुत्रः \Ed}}% 
    \var{{\devanagarifont \numnoemph\vd\textbf{अप्रमाद॰}\lem \mssALL, अप्रमादा॰ \msNa}}% 

{\devanagarifont क्षेमः शान्तिसुतो विन्द्याद्व्यवसायो वपोः सुतः \thinspace{\dandab} \dontdisplaylinenum }%
     \var{{\devanagarifont \numemph\vb\textbf{वपोः}\lem \mssALL, वपो \msNa}}% 

{\devanagarifont यशः कीर्तिसुतो ज्ञेयः सुखं सिद्धेर्व्यजायत  \danda\dontdisplaylinenum }%
     \var{{\devanagarifont \numnoemph\vd\textbf{सिद्धे॰}\lem \msParis\msCb\msNa\msNb, सिद्धि \msCa\msNc\Ed\oo 
\textbf{व्यजायत}\lem \msParis\msCa\msCb\msNa, व्यजायते \msNb\Ed, व्यजायतः \msNc}}% 

%Verse 3:13

{\devanagarifont स्वायम्भुवे ऽन्तरे त्वासन्कीर्तिता धर्मसूनवः {॥ ३:१३॥} \veg\dontdisplaylinenum }%
     \var{{\devanagarifont \numnoemph\ve\textbf{स्वायम्भुवे}\lem \msParis\msCa\msNa\msNc, स्वायम्भुवो \msCb, स्वयम्भुवे \msNb\Ed\oo 
\textbf{ऽन्तरे त्वासन्}\lem \conj, ऽन्तरे त्वासि \msParis\msCa\msCb\msNa, 
ऽन्तरे त्वासीत् \msNb, ऽन्तरे त्वासं \msNc, ऽन्तरेवासि \Ed}}% 

{\devanagarifont विगतराग उवाच {\dandab}\dontdisplaylinenum  }%
 
{\devanagarifont मूर्तिद्वयं कथं धर्मं कथयस्व तपोधन \thinspace{\danda} \dontdisplaylinenum }%
     \var{{\devanagarifont \numemph\va\textbf{धर्मं}\lem \mssALL, द्धर्म \msNc, धर्मः \Ed}}% 

%Verse 3:14

{\devanagarifont कौतूहलमतीवं मे कर्तय ज्ञानसंशयम् {॥ ३:१४॥} \veg\dontdisplaylinenum }%
     \var{{\devanagarifont \numnoemph\vc\textbf{कौतूहल॰}\lem \mssALL, कोतूहल॰ \msCb\oo 
\textbf{॰तीवं मे}\lem \mssALL, ॰तीव मे \msCb}}% 
    \var{{\devanagarifont \numnoemph\vd\textbf{कर्तय}\lem \eme, कीर्तय \msCa\msCb\msNa\msNb\msNc\Ed\oo 
\textbf{॰संशयम्}\lem \mssALL, ॰संशयः \msCb\msNb}}% 
    \lacuna{\devanagarifontsmall \vc {\englishfont In \msParis, folio 215v ends with} कौतूहलमती {\englishfont and the next available 
                      folio side (217r) starts with} त्यमिष्टगतिः प्रोक्तं {\englishfont  in 4.8a. Thus one folio (f. 216), 
                      containing 3.14d--4.7, is missing.} }%
  
{\devanagarifont अनर्थयज्ञ उवाच {\dandab}\dontdisplaylinenum  }%
 
{\devanagarifont श्रुतिस्मृतिद्वयोर्मूर्तिर्धर्मस्य परिकीर्तिता \thinspace{\danda} \dontdisplaylinenum }%
     \var{{\devanagarifont \numemph\va\textbf{श्रुति॰}\lem \mssALL, श्रुतिः \msCb\Ed}}% 
    \var{{\devanagarifont \numnoemph\vab\textbf{॰द्वयोर्मूर्तिर्ध॰}\lem \msCa, ॰द्वयो मूर्ति ध॰ \msCb\msNa\msNb, ॰द्वयी मूर्ति ध॰ \msNc, 
॰द्वयोर्मूर्ति ध॰ \Ed}}% 
    \var{{\devanagarifont \numnoemph\vb\textbf{॰कीर्तिता}\lem \mssALL, ॰कीर्त्तितः \msNb, कीर्त्तिताः \msNc}}% 

{\devanagarifont दाराग्निहोत्रसम्बन्ध इज्या श्रौतस्य लक्षणम्  \danda\dontdisplaylinenum }%
     \var{{\devanagarifont \numnoemph\vcd\textbf{॰बन्ध इ॰}\lem \eme, ॰बद्ध इ॰ \msCa\msCb\msNa\msNc, ॰बन्ध इ॰ \msNb\Ed}}% 
    \var{{\devanagarifont \numnoemph\vd\textbf{श्रौतस्य}\lem \eme, श्रोतस्य \msCa\msCb\msNc, श्रौत्रस्य \msNa, स्रोत्रस्य \msNb, श्रुतस्य \Ed}}% 
    \paral{{\devanagarifontsmall \vcd {\englishfont \compare\ \MANU\ 3.171ab:}दाराग्निहोत्रसंयोगं कुरुते यो ऽग्रजे स्थिते; 
                         {\englishfont and also \MATSP\ 142.41:} 
                         दाराग्निहोत्रसम्बन्धमृग्यजुःसामसंहिताः\thinspace{\devanagarifontsmall ।}
                         इत्यादिबहुलं श्रौतं धर्मं सप्तर्षयो ऽब्रुवन्\thinspace{\devanagarifontsmall ॥} }}

%Verse 3:15

{\devanagarifont स्मार्तो वर्णाश्रमाचारो यमैश्च नियमैर्युतः {॥ ३:१५॥} \veg\dontdisplaylinenum }%
     \var{{\devanagarifont \numnoemph\ve\textbf{स्मार्तो}\lem \eme, स्मार्त \msCa\msCb\msNa\msNb\msNc\Ed}}% 
    \paral{{\devanagarifontsmall \vcdef\ {\englishfont  \similar\ \MBH\ Suppl. 1.36.10: 
                                 }दानाग्निहोत्रमिज्या च श्रौतस्यैतद्धि लक्षणम्\thinspace{\devanagarifontsmall ।}
                                 स्मार्तो वर्णाश्रमाचारो यमैश्च नियमैर्युतः\thinspace{\devanagarifontsmall ॥}
                          \similar\ {\englishfont \MATSP\ 145.30cd--31ab:
                                 }दाराग्निहोत्रसम्बन्धमिज्या श्रौतस्य लक्षणम्\thinspace{\devanagarifontsmall ।}
                                 स्मार्तो वर्णाश्रमाचारो यमैश्च नियमैर्युतः\thinspace{\devanagarifontsmall ॥}
                          \similar\ {\englishfont \BRAHMANDAPUR\ 1.32.33cd--34ab:}
                                 दाराग्निहोत्रसम्बन्धाद् द्विधा श्रौतस्य लक्षणम्\thinspace{\devanagarifontsmall ।}
                                 स्मार्तो वर्णाश्रमाचारैर्यमैः स नियमैः स्मृतः\thinspace{\devanagarifontsmall ॥} }}


\alalfejezet{यमनियमभेदः}
{\devanagarifont यमश्च नियमश्चैव द्वयोर्भेदमतः शृणु \thinspace{\dandab} \dontdisplaylinenum }%
     \var{{\devanagarifont \numemph\va\textbf{नियम॰}\lem \mssALL, नियमै॰ \msNa}}% 

{\devanagarifont अहिंसा सत्यमस्तेयमानृशंस्यं दमो घृणा  \danda\dontdisplaylinenum }%
     \var{{\devanagarifont \numnoemph\vd\textbf{॰मानृशंस्यं}\lem \eme, ॰मनृशंस्यो \msCa\msCb\msNa\msNb\Ed, ॰मानृशंस्या \msNc}}% 
    \paral{{\devanagarifontsmall \vcd {\englishfont \similar\ \MBH\ 12.8.17ab:} अहिंसा सत्यवचनमानृशंस्यं दमो घृणा
                 \vo {\englishfont \similar\ \VDHU\ 3.233.203: 
                         }आनृशंस्यं क्षमा सत्यमहिंसा च दमः स्पृहा\thinspace{\devanagarifontsmall ।}
                         ध्यानं प्रसादो माधुर्यं चार्जवं च यमा दश\thinspace{\devanagarifontsmall ॥} }}

%Verse 3:16

{\devanagarifont धन्याप्रमादो माधुर्यमार्जवं च यमा दश {॥ ३:१६॥} \veg\dontdisplaylinenum }%
     \var{{\devanagarifont \numnoemph\ve\textbf{धन्या॰}\lem \Ed, धन्यः \msCa\msCb\msNb\msNc, ध्यन्यं \msNa\oo 
\textbf{माधुर्य॰}\lem \Ed, माधूर्य॰ \msCa\msCb\msNa\msNb\msNc}}% 
    \var{{\devanagarifont \numnoemph\vf\textbf{आर्जवं च}\lem \mssALL, आर्जवश्च \Ed}}% 

{\devanagarifont एकैकस्य पुनः पञ्चभेदमाहुर्मनीषिणः \thinspace{\dandab} \dontdisplaylinenum }%
     \var{{\devanagarifont \numemph\vb\textbf{॰माहुर्म॰}\lem \mssALL, ॰माहु म॰ \msNc}}% 

%Verse 3:17

{\devanagarifont अहिंसादि प्रवक्ष्यामि शृणुष्वावहितो द्विज {॥ ३:१७॥} \veg\dontdisplaylinenum }%
     \var{{\devanagarifont \numnoemph\vd\textbf{शृणुष्वा॰}\lem \mssALL, शृणुष्व॰ \msNa\msNb}}% 


\alalfejezet{यमेष्वहिंसा (१)}

\alalalfejezet{पञ्चविधा हिंसा}

{\devanagarifont त्रासनं ताडनं बन्धो मारणं वृत्तिनाशनम् \thinspace{\dandab} \dontdisplaylinenum }%
     \var{{\devanagarifont \numemph\va\textbf{बन्धो}\lem \mssALL, बद्धो \msNb, बन्ध \Ed}}% 

%Verse 3:18

{\devanagarifont हिंसां पञ्चविधामाहुर्मुनयस्तत्त्वदर्शिनः {॥ ३:१८॥} \veg\dontdisplaylinenum }%
     \var{{\devanagarifont \numnoemph\vc\textbf{हिंसां}\lem \msCa\msNa\msNc, हिंसा \msCb\msNb\Ed\oo 
\textbf{॰विधामाहु॰}\lem \msCb\msNa\msNc, ॰विधमाहु॰ \msCa, 
॰विधान्याहु॰ \msNb, ॰विध प्राहु॰ \Ed}}% 

{\devanagarifont काष्ठलोष्टकशाद्यैस्तु ताडयन्तीह निर्दयाः \thinspace{\dandab} \dontdisplaylinenum }%
     \var{{\devanagarifont \numemph\va\textbf{काष्ठलोष्ट॰}\lem \mssALL, का\uncl{ष्ठ}\lac\  \msNb}}% 
    \var{{\devanagarifont \numnoemph\vb\textbf{निर्दयाः}\lem \mssALL, निर्दया \Ed}}% 

%Verse 3:19

{\devanagarifont तत्प्रहारविभिन्नाङ्गो मृतवध्यमवाप्नुयात् {॥ ३:१९॥} \veg\dontdisplaylinenum }%
     \var{{\devanagarifont \numnoemph\vc\textbf{॰भिन्नाङ्गो}\lem \mssALL, ॰भिन्नाङ्गा \Ed}}% 
    \var{{\devanagarifont \numnoemph\vd\textbf{॰वध्यमवा॰}\lem \mssALL, ॰वध्यववा॰ \msCa}}% 

{\devanagarifont बद्ध्वा पादौ भुजोरश्च शिरोरुक्कण्ठपाशिताः \thinspace{\dandab} \dontdisplaylinenum }%
     \var{{\devanagarifont \numemph\va\textbf{भुजोरश्च}\lem \mssALL, भुजौरश्च \msNa\Ed}}% 
    \var{{\devanagarifont \numnoemph\vb\textbf{शिरोरुक्कण्ठ॰}\lem \eme, शिरोरुकण्ठ॰ \msCa\msCb\msNa\msNb\msNc, शिरोरुः कण्ठ॰ \Ed}}% 

%Verse 3:20

{\devanagarifont अनाहता म्रियन्त्येवं वधो बन्धनजः स्मृतः {॥ ३:२०॥} \veg\dontdisplaylinenum }%
     \var{{\devanagarifont \numnoemph\vc \lem \mssALL, अनाहत म्रियंत्येष \msNb}}% 
    \var{{\devanagarifont \numnoemph\vd\textbf{॰नजः स्मृतः}\lem \conj, ॰नजाः स्मृताः \msCa\msCb\msNa\msNb, 
॰नजाः स्मृता \msNc, ॰नज स्मृतः \Ed}}% 

{\devanagarifont शत्रुचौरभयैर्घोरैः सिंहव्याघ्रगजोरगैः \thinspace{\dandab} \dontdisplaylinenum }%
     \var{{\devanagarifont \numemph\va\textbf{॰चौरभयैर्घोरैः}\lem \mssALL, ॰चोरभयै घोरै \msNb}}% 

%Verse 3:21

{\devanagarifont त्रासनाद्वधमाप्नोति अन्यैर्वापि सुदुःसहैः {॥ ३:२१॥} \veg\dontdisplaylinenum }%
     \var{{\devanagarifont \numnoemph\vd\textbf{अन्यैर्वापि}\lem \mssALL, अन्ये चापि \msNc}}% 

{\devanagarifont यस्य यस्य हरेद्वित्तं तस्य तस्य वधः स्मृतः \thinspace{\dandab} \dontdisplaylinenum }%
     \var{{\devanagarifont \numemph\va\textbf{हरेद्वि॰}\lem \mssALL, हरे वि॰ \msNb}}% 
    \var{{\devanagarifont \numnoemph\vb\textbf{वधः}\lem \mssALL, वध \Ed}}% 

%Verse 3:22

{\devanagarifont वृत्तिजीवाभिभूतानां तद्द्वारा निहतः स्मृतः {॥ ३:२२॥} \veg\dontdisplaylinenum }%
     \var{{\devanagarifont \numnoemph\vc\textbf{॰भिभूतानां}\lem \mssALL, ॰विभूतानां \msNb}}% 
    \var{{\devanagarifont \numnoemph\vd\textbf{तद्द्वारा नि॰}\lem \conj, तद्वारान्नि॰ \msCa\msCb\msNa\msNb\msNc, तद्द्वारान्नि॰ \Ed}}% 

{\devanagarifont विषवह्निशरशस्त्रैर्मायायोगबलेन वा \thinspace{\dandab} \dontdisplaylinenum }%
     \var{{\devanagarifont \numemph\vab\textbf{॰शस्त्रैर्माया॰}\lem \mssALL, ॰शस्त्रै मा॰ \msNc, ॰शस्त्रैर्म्मया॰ \Ed}}% 

%Verse 3:23

{\devanagarifont हिंसकान्याहु विप्रेन्द्र मुनयस्तत्त्वदर्शिनः {॥ ३:२३॥} \veg\dontdisplaylinenum }%
     \var{{\devanagarifont \numnoemph\vc\textbf{हिंसकान्याहु वि॰}\lem \msCb\msNb\msNc, 
हिंसकान्याहुर्वि॰ \msCa\msNa\ \unmetr, हिंसकेत्याहु वि॰ \Ed}}% 


\alalalfejezet{अहिंसाप्रशंसा}

{\devanagarifont अहिंसा परमं धर्मं यस्त्यजेत्स दुरात्मवान् \thinspace{\dandab} \dontdisplaylinenum }%
     \var{{\devanagarifont \numemph\va\textbf{परमं धर्मं}\lem \mssALL, परमं धर्म \msNb, परमो धर्मं \msNc}}% 
    \var{{\devanagarifont \numnoemph\vb\textbf{त्यजेत्स दुरात्मवान्}\lem \msCb\msNc\Ed, त्यजेच्छ दुरात्म\lk\ \msCa, त्यजेत्सुदुरात्मवान् \msNa, 
त्यजेत्स दुरात्मनम् \msNb}}% 

%Verse 3:24

{\devanagarifont क्लेशायासविनिर्मुक्तं सर्वधर्मफलप्रदम् {॥ ३:२४॥} \veg\dontdisplaylinenum }%
 
{\devanagarifont नातः परतरो मूर्खो नातः परतरं तमः \thinspace{\dandab} \dontdisplaylinenum }%
     \var{{\devanagarifont \numemph\vb\textbf{॰तरं}\lem \mssALL, ॰तन् \msCbacorr\Ed}}% 

%Verse 3:25

{\devanagarifont नातः परतरं दुःखं नातः परतरो ऽयशः {॥ ३:२५॥} \veg\dontdisplaylinenum }%
 
{\devanagarifont नातः परतरं पापं नातः परतरं विषम् \thinspace{\dandab} \dontdisplaylinenum }%
 
%Verse 3:26

{\devanagarifont नातः परतराविद्या नातः परं तपोधन {॥ ३:२६॥} \veg\dontdisplaylinenum }%
     \var{{\devanagarifont \numemph\vd\textbf{परं तपोधन}\lem \mssALL, पर तपोद्यमाः \Ed}}% 

{\devanagarifont यो हिनस्ति न भूतानि उद्भिज्जादि चतुर्विधम् \thinspace{\dandab} \dontdisplaylinenum }%
     \var{{\devanagarifont \numemph\va\textbf{यो हिनस्ति न}\lem \mssALL, यो न हिन्सन्ति \msNb, यो हि नास्ति न \Ed}}% 
    \var{{\devanagarifont \numnoemph\vb\textbf{उद्भिज्जादि}\lem \eme, उद्भिजादि \msCa\msCb\msNb\msNc\Ed, उद्भिजानि \msNa\oo 
\textbf{॰विधम्}\lem \mssALL, ॰विधिं \msNc}}% 

%Verse 3:27

{\devanagarifont स भवेत्पुरुषः श्रेष्ठः सर्वभूतदयान्वितः {॥ ३:२७॥} \veg\dontdisplaylinenum }%
     \var{{\devanagarifont \numnoemph\vc\textbf{पुरुषः}\lem \mssALL, पुरुष॰ \Ed}}% 

{\devanagarifont सर्वभूतदयां नित्यं यः करोति स पण्डितः \thinspace{\dandab} \dontdisplaylinenum }%
     \var{{\devanagarifont \numemph\va\textbf{॰दयां नित्यं}\lem \msCa\msNa\Ed, ॰दया नित्यं \msCb\msNb, ॰दया नित्य \msNc}}% 

%Verse 3:28

{\devanagarifont स यज्वा स तपस्वी च स दाता स दृढव्रतः {॥ ३:२८॥} \veg\dontdisplaylinenum }%
     \var{{\devanagarifont \numnoemph\vc\textbf{यज्वा}\lem \mssALL, यज्या \msNb}}% 

{\devanagarifont अहिंसा परमं तीर्थमहिंसा परमं तपः \thinspace{\dandab} \dontdisplaylinenum }%
     \var{{\devanagarifont \numemph\va\textbf{परमं ती॰}\lem \mssALL, परन्ती॰ \msCb}}% 

%Verse 3:29

{\devanagarifont अहिंसा परमं दानमहिंसा परमं सुखम् {॥ ३:२९॥} \veg\dontdisplaylinenum }%
     \paral{{\devanagarifontsmall \vo {\englishfont This and the following verses are similar to MBh 13.117.37--38} }}
    \lacuna{\devanagarifontsmall \vd {\englishfont \msCc\ resumes here in exp.\ 189, f. 273r (sic!) with }रमं सुखम् }%
  
{\devanagarifont अहिंसा परमो यज्ञः अहिंसा परमं व्रतम् \thinspace{\dandab} \dontdisplaylinenum }%
     \var{{\devanagarifont \numemph\va\textbf{यज्ञः}\lem \msCb\msCc\msNb\Ed, यज्ञर् \msCa, यज्ञ \msNa\msNc}}% 

%Verse 3:30

{\devanagarifont अहिंसा परमं ज्ञानमहिंसा परमा क्रिया {॥ ३:३०॥} \veg\dontdisplaylinenum }%
     \var{{\devanagarifont \numnoemph\vc\textbf{परमं}\lem \mssALL, परमो \Ed}}% 
    \var{{\devanagarifont \numnoemph\vd\textbf{परमा}\lem \mssALL, परमां \msNb}}% 

{\devanagarifont अहिंसा परमं शौचमहिंसा परमो दमः \thinspace{\dandab} \dontdisplaylinenum }%
     \var{{\devanagarifont \numemph\vab\textbf{(अहिंसा{\englishfont ...} दमः)}\lem \mssALL, \om\ \Ed}}% 

%Verse 3:31

{\devanagarifont अहिंसा परमो लाभः अहिंसा परमं यशः {॥ ३:३१॥} \veg\dontdisplaylinenum }%
     \var{{\devanagarifont \numnoemph\vc\textbf{लाभः}\lem \msNc, लाभ \msCa\msCb\msNa\msNb\Ed, लाभो \msCc}}% 
    \var{{\devanagarifont \numnoemph\vd\textbf{परमं}\lem \mssALL, परमा \msNa}}% 
    \lacuna{\devanagarifontsmall \vcd {\englishfont After pādas cd, \Ed\ inserts this: }अहिंसा परमा कीर्ति अहिंसा परमो दमः,
                 {\englishfont which is not to be found in \mssCaCbCc\msNa\msNb\msNc\ (or in 
                 paper MS \msPaperA)} }%
  
{\devanagarifont अहिंसा परमो धर्मः अहिंसा परमा गतिः \thinspace{\dandab} \dontdisplaylinenum }%
     \var{{\devanagarifont \numemph\va\textbf{धर्मः}\lem \msNa\msNc, धर्म \msCa\msCb\Ed, धर्मो \msCc, ध\lac\  \msNb}}% 
    \var{{\devanagarifont \numnoemph\vb \lem \mssALL, \lac\  \msNb, अहिंसा परमो गतिः \Ed}}% 

%Verse 3:32

{\devanagarifont अहिंसा परमं ब्रह्म अहिंसा परमः शिवः {॥ ३:३२॥} \veg\dontdisplaylinenum }%
     \var{{\devanagarifont \numnoemph\vc \lem \mssALL, 
\uncl{अहिंसा परमं ब्रह्म} \msNb, अहिंसा परंमं ब्रह्म \msNc}}% 


\alalalfejezet{मांसाहारः}

{\devanagarifont मांसाशनान्निवर्तेत मनसापि न काङ्क्षयेत् \thinspace{\dandab} \dontdisplaylinenum }%
     \var{{\devanagarifont \numemph\va\textbf{मांसाशनान्नि॰}\lem \msCa\msCb\Ed, मान्साशन नि॰ \msCc, 
मांसाशनन्नि॰ \msNa, मन्सासनन्नि॰ \msNb, \uncl{मांसशानान्नि}॰ \msNc}}% 

%Verse 3:33

{\devanagarifont स महत्फलमाप्नोति यस्तु मांसं विवर्जयेत् {॥ ३:३३॥} \veg\dontdisplaylinenum }%
     \var{{\devanagarifont \numnoemph\vd\textbf{मांसं}\lem \mssCaCbCc\msNa, मांस \msNb\Ed, मासं \msNc}}% 

{\devanagarifont स्वमांसं परमांसेन यो वर्धयितुमिच्छति \thinspace{\dandab} \dontdisplaylinenum }%
     \var{{\devanagarifont \numemph\va\textbf{॰मांसेन}\lem \mssALL, ॰मासेन \msNc}}% 
    \var{{\devanagarifont \numnoemph\vb\textbf{वर्धयितु॰}\lem \mssALL, वर्द्धयति \msNb}}% 
    \paral{{\devanagarifontsmall \vab {\englishfont  = \MBH\ 13.116.14ab and 13.116.34ab \similar\ \UUMS\ 2.48cd:
                          }स्वमांसं परमांसेन यो देहे वृद्धिमिच्छति }}

%Verse 3:34

{\devanagarifont अनभ्यर्च्य पितॄन्देवान्न ततो ऽन्यो ऽस्ति पापकृत् {॥ ३:३४॥} \veg\dontdisplaylinenum }%
     \var{{\devanagarifont \numnoemph\vc\textbf{पितॄन्}\lem \msCa\msCb\msNa\msNc, पितृन् \msCc\Ed, \uncl{पितॄन्} \msNb}}% 
    \var{{\devanagarifont \numnoemph\vd\textbf{ततो ऽन्यो}\lem \mssALL, तदन्यो \Ed}}% 
    \paral{{\devanagarifontsmall \vo {\englishfont \similar\ \MANU\ 5.52 (Olivelle's edition):} 
                 स्वमांसं परमांसेन यो वर्धयितुमिच्छति\thinspace{\devanagarifontsmall ।}
                 अनभ्यर्च्य पितॄन्देवान्न ततो ऽन्यो स्त्यपुण्यकृत्\thinspace{\devanagarifontsmall ॥} }}

{\devanagarifont मधुपर्के च यज्ञे च पितृदैवतकर्मणि \thinspace{\dandab} \dontdisplaylinenum }%
     \var{{\devanagarifont \numemph\vb\textbf{॰दैवत॰}\lem \mssALL, ॰देवत॰ \msCc\msNb}}% 

%Verse 3:35

{\devanagarifont अत्रैव पशवो हिंस्या नान्यत्र मनुरब्रवीत् {॥ ३:३५॥} \veg\dontdisplaylinenum }%
     \var{{\devanagarifont \numnoemph\vc \lem \msCa\msCc\msNc\Ed, 
अत्रैव पशवो हिंसा \msCb, अत्रैव पशवो हिंस्यान् \msNa, 
\lac\  \msNb}}% 
    \var{{\devanagarifont \numnoemph\vd \lem \mssALL, 
\lac \uncl{त्र मनुरब्रवीत्} \msNb}}% 
    \paral{{\devanagarifontsmall \vo {\englishfont \similar\ \MANU\ 5.41 (Olivelle's edition):}
                         मधुपर्के च यज्ञे च पितृदैवतकर्मणि\thinspace{\devanagarifontsmall ।}
                         अत्रैव पशवो हिंस्या नान्यत्रेत्यब्रवीन्मनुः\thinspace{\devanagarifontsmall ॥} }}

{\devanagarifont क्रीत्वा स्वयं वाप्युत्पाद्य परोपहृतमेव वा \thinspace{\dandab} \dontdisplaylinenum }%
     \var{{\devanagarifont \numemph\va\textbf{क्रीत्वा}\lem \mssALL, कृत्वा \Ed\oo 
\textbf{॰प्युत्पाद्य}\lem \mssALL, ॰प्युत्पाद्या॰ \Ed}}% 
    \var{{\devanagarifont \numnoemph\vb\textbf{॰हृत॰}\lem \mssALL, ॰हित॰ \Ed\oo 
\textbf{वा}\lem \mssALL, च \Ed}}% 

%Verse 3:36

{\devanagarifont देवान्पितॄंश्चार्चयित्वा खादन्मांसं न दोषभाक् {॥ ३:३६॥} \veg\dontdisplaylinenum }%
     \var{{\devanagarifont \numnoemph\vc\textbf{पितॄंश्चार्चयित्वा}\lem \mssALL, पितॄश्चार्चयित्वा \msNb, पितृश्चार्पयित्वा \Ed}}% 
    \var{{\devanagarifont \numnoemph\vd\textbf{मांसं}\lem \mssALL, मासं \msNc}}% 
    \paral{{\devanagarifontsmall \vo {\englishfont = \MANU\ 5.32 (in Olivelle's critical edition; other editions read}
                          परोपकृत॰ {\englishfont in pāda b)} }}

{\devanagarifont वेदयज्ञतपस्तीर्थदानशीलक्रियाव्रतैः \thinspace{\dandab} \dontdisplaylinenum }%
     \var{{\devanagarifont \numemph\vb\textbf{॰शील॰}\lem \mssALL, ॰शल॰ \msCc\oo 
\textbf{॰व्रतैः}\lem \mssALL, ॰व्र\uncl{तः} \msCb}}% 

%Verse 3:37

{\devanagarifont मांसाहारनिवृत्तानां षोडशांशं न पूर्यते {॥ ३:३७॥} \veg\dontdisplaylinenum }%
     \var{{\devanagarifont \numnoemph\vc\textbf{॰वृत्तानां}\lem \mssALL, ॰वृत्ताना \msNb, ॰वृत्तीनां \Ed}}% 
    \var{{\devanagarifont \numnoemph\vd\textbf{न}\lem \mssALL, त \msCb}}% 

{\devanagarifont मृगाः पर्णतृणाहारादजमेषगवादिभिः \thinspace{\dandab} \dontdisplaylinenum }%
     \var{{\devanagarifont \numemph\va\textbf{पर्ण॰}\lem \mssALL, पण्ण॰ \msNa, पर्णा॰ \Ed}}% 
    \var{{\devanagarifont \numnoemph\vab\textbf{॰हाराद॰}\lem \msCa\msCc\msNbpcorr\msNc\Ed, ॰हारा अ॰ \msCb\msNa, ॰हाद॰ \msNbacorr}}% 

%Verse 3:38

{\devanagarifont सुखिनो बलवन्तश्च विचरन्ति महीतले {॥ ३:३८॥} \veg\dontdisplaylinenum }%
 
{\devanagarifont वानराः फलमाहारा राक्षसा रुधिरप्रियाः \thinspace{\dandab} \dontdisplaylinenum }%
     \var{{\devanagarifont \numemph\vab\textbf{॰हारा रा॰}\lem \msCb\msNa\msNb, ॰हाराद्रा॰ \msCa\msCc\msNc\Ed}}% 

%Verse 3:39

{\devanagarifont निहता राक्षसाः सर्वे वानरैः फलभोजिभिः {॥ ३:३९॥} \veg\dontdisplaylinenum }%
     \var{{\devanagarifont \numnoemph\vd\textbf{॰भोजिभिः}\lem \mssALL, ॰भोगिभिः \Ed}}% 

{\devanagarifont तस्मान्मांसं न हीहेत बलकामेन भो द्विज \thinspace{\dandab} \dontdisplaylinenum }%
     \var{{\devanagarifont \numemph\va\textbf{मांसं}\lem \mssALL, मासं \msNc}}% 
    \var{{\devanagarifont \numnoemph\vb\textbf{हीहेत}\lem \mssALL, हीयेत \msNa\msNb}}% 

%Verse 3:40

{\devanagarifont बलेन च गुणाकर्षात्परतो भयभीरुणा {॥ ३:४०॥} \veg\dontdisplaylinenum }%
     \var{{\devanagarifont \numnoemph\vc\textbf{गुणाकर्षा॰}\lem \conjTorzsok, गुणाकाशा॰ \mssCaCbCc\msNa\msNb\msNc, गुणा कुर्या॰ \Ed}}% 

{\devanagarifont अहिंसकसमो नास्ति दानयज्ञसमीहया \thinspace{\dandab} \dontdisplaylinenum }%
     \var{{\devanagarifont \numemph\vb\textbf{॰यज्ञसमीहया}\lem \msCa\msCb\msNa\msNb, ॰धर्मसमीहया \msCc, 
॰यज्ञसमीहयाः \msNc, ॰धर्मसमीहय \Ed}}% 

%Verse 3:41

{\devanagarifont इह लोके यशः कीर्तिः परत्र च परा गतिः {॥ ३:४१॥} \veg\dontdisplaylinenum }%
     \var{{\devanagarifont \numnoemph\vc\textbf{यशः}\lem \mssALL, य\uncl{शं} \msCc}}% 
    \var{{\devanagarifont \numnoemph\vd\textbf{परा गतिः}\lem \msCc\msNa\msNc, \uncl{परा गतिः} \msCa, 
पराङ्गतिम् \msCb\msNb, परां गतिः \Ed}}% 

\ujvers\nemsloka {
{\devanagarifont त्रैलोक्यं मणिरत्नपूर्णमखिलं दत्त्वोत्तमे ब्राह्मणे }%
  \dontdisplaylinenum}    \var{{\devanagarifont \numemph\va\textbf{त्रैलोक्यं}\lem \mssALL, त्रैलोक्य \msNb\oo 
\textbf{अखिलं दत्त्वोत्तमे ब्राह्मणे}\lem \mssALL, 
अ\uncl{खिलं}\lk\lk \lk\lk \lk\lk \lk\ \msCa, अखिलं दत्तोत्तमे ब्राह्मणे \msNa}}% 
    \paral{{\devanagarifontsmall \va {\englishfont \SDHS\ 11.91:}
                    त्रैलोक्यमपि यो दद्यादखिलं रत्नपूरितम्\thinspace{\devanagarifontsmall ।}
                    चरेत्तपांसि सर्वाणि न तत्तुल्यमहिंसया\thinspace{\devanagarifontsmall ॥} }}


\nemslokab

{\devanagarifont कोटीयज्ञसहस्रपद्ममयुतं दत्त्वा महीं दक्षिणाम्  \danda\dontdisplaylinenum }%
     \var{{\devanagarifont \numnoemph\vb\textbf{कोटीयज्ञसहस्रपद्मम्}\lem \mssALL, \lk\lk \lk\lk \lk\lk \lk\lk \lk\  \msCa\oo 
\textbf{महीं}\lem \mssALL, मही \msCc}}% 

\nemslokac

{\devanagarifont तीर्थानां च सहस्रकोटिनियुतं स्नात्वा सकृन्मानव }%
  \dontdisplaylinenum    \var{{\devanagarifont \numnoemph\vc\textbf{॰कोटि॰}\lem \mssALL, ॰कोटी॰ \Ed\ \unmetr\oo 
\textbf{स्नात्वा}\lem \mssALL, स्ना ऽ \msCb}}% 

%Verse 3:42


\nemslokad

{\devanagarifont एतत्पुण्यफलमहिंसकजनः प्राप्नोति निःसंशयः {॥ ३:४२॥} \veg\dontdisplaylinenum }%
     \var{{\devanagarifont \numnoemph\vd\textbf{॰फलमहिंस॰}\lem \mssALL, ॰फलं त्वहिंस॰ \msNc\oo 
\textbf{निःसंशयः}\lem \msCc\msNa\msNb\msNc, \lk\lk \lk\lk\ \msCa, निःसंशय\lk\ \msCb, निःसंशयं \Ed}}% 

\vers


{\devanagarifont 
\jump
\begin{center}
\ketdanda~इति वृषसारसंग्रहे अहिंसाप्रशंसा नामाध्यायस्तृतीयः~\ketdanda
\end{center}
\dontdisplaylinenum\vers  }%
     \var{{\devanagarifont \numnoemph{\englishfont \Colo:}\textbf{नामाध्यायस्तृतीयः}\lem \mssALL, नामाध्यायस्तृतीय \msNc, 
नामस्तृतीयो ऽध्यायः \Ed}}% 
\bekveg\szamveg
\vfill
\phpspagebreak

\versno=0\fejno=4
\thispagestyle{empty}

\centerline{\Large\devanagarifontbold [   चतुर्थो ऽध्यायः  ]}{\vrule depth10pt width0pt} \fancyhead[CO]{{\footnotesize\devanagarifont वृषसारसंग्रहे  }}
\fancyhead[CE]{{\footnotesize\devanagarifont चतुर्थो ऽध्यायः  }}
\fancyhead[LE]{}
\fancyhead[RE]{}
\fancyhead[LO]{}
\fancyhead[RO]{}
\szam\bek



\alalfejezet{यमेषु सत्यम् (२)}
\vers


{\devanagarifont अनर्थयज्ञ उवाच {\dandab}\dontdisplaylinenum  }%
     \lacuna{\devanagarifontsmall {\englishfont Witnesses used for this chapter: \msCa\ ff.\thinspace 198v--201v, 
                                              \msCb\ ff.\thinspace 206r--208v, 
                                              \msCc\ ff.\thinspace 273v--277r,
                                              \msNa\ ff.\thinspace 6r--9r, 
                                              \msNb\ exp.\thinspace 48--50 (lower--upper),
                                              \msNc\ ff.\thinspace 214v--217r,
                                              \Ed\ pp.\thinspace 591--597;
                                        \mssCaCbCc\ = \msCa + \msCb + \msCc} }%
  
{\devanagarifont सद्भावः सत्यमित्याहुर्दृष्टप्रत्ययमेव वा \thinspace{\danda} \dontdisplaylinenum }%
     \var{{\devanagarifont \numemph\va\textbf{सद्भावः}\lem \mssALL, सद्भाव॰ \msNb\Ed}}% 
    \var{{\devanagarifont \numnoemph\vab\textbf{सत्यमित्याहुर्दृ॰}\lem \msCb\msNa\msNc\Ed, सत्य\uncl{मि}$\-$त्याहु दृ॰ \msCa, 
सत्यमित्याहु दृ॰ \msCc, सत्यामित्याहुर्दृ॰ \msNb}}% 
    \var{{\devanagarifont \numnoemph\vb\textbf{॰प्रत्यय॰}\lem \msCa\msCb\msNa\msNb, ॰प्रत्य॰ \msCc, ॰प्रत्येय॰ \msNc, प्रत्यक्ष॰ \Ed}}% 
    \paral{{\devanagarifontsmall \va {\englishfont \similar\ \MBH\ 12.288.45d:} 
                         सद्भावः सत्यमुच्यते 
                    {\englishfont \compare\  also \BRAHMANDAPUR\ 3.3.86ab:}
                         असद्भावो ऽनृतं ज्ञेयं सद्भावः सत्यमुच्यते  }}

%Verse 4:1

{\devanagarifont यथाभूतार्थकथनं तत्सत्यकथनं स्मृतम् {॥ ४:१॥} \veg\dontdisplaylinenum }%
     \var{{\devanagarifont \numnoemph\vc \lem \mssALL, 
यथाभूतार्थ \msCcacorr, 
यथाभूतार्थनं क्त \msCcpcorr}}% 
    \var{{\devanagarifont \numnoemph\vd\textbf{तत्सत्यकथनं}\lem \msCa\msNa\msNb\msNc\Ed, 
तत्सत्यकथकं \msCb, 
कथनं स्मृतं \msCcacorr, 
\uncl{सत्यक ज}कथनं स्मृतं \msCcpcorr}}% 
    \paral{{\devanagarifontsmall \vcd {\englishfont \compare\ \SDHS\ 11.105:} 
                 स्वानुभूतं स्वदृष्टं च यः पृष्टार्थं न गूहति\thinspace{\devanagarifontsmall ।}
                 यथाभूतार्थकथनमित्येतत्सत्यलक्षणम्\thinspace{\devanagarifontsmall ॥} }}

{\devanagarifont आक्रोशताडनादीनि यः सहेत सुदुःसहम् \thinspace{\dandab} \dontdisplaylinenum }%
     \var{{\devanagarifont \numemph\va\textbf{॰ताडना॰}\lem \mssALL, ॰नाडना॰ \msCb}}% 
    \var{{\devanagarifont \numnoemph\vb\textbf{सुदुःसहम्}\lem \mssALL, सुदुसहं \msCc}}% 

%Verse 4:2

{\devanagarifont क्षमते यो जितात्मा तु स च सत्यमुदाहृतम् {॥ ४:२॥} \veg\dontdisplaylinenum }%
     \var{{\devanagarifont \numnoemph\vd\textbf{सत्यमुदाहृतम्}\lem \mssALL, 
\uncl{सत्य}मु\uncl{दा}हृतम् \msCa}}% 
    \paral{{\devanagarifontsmall \vo {\englishfont \compare\ \SDHS\ 11.82:}
                 आक्रुष्टस्ताडितो वापि यो नाक्रोशेन्न ताडयेत्\thinspace{\devanagarifontsmall ।}
                 वागाद्यविकृतः स्वस्थं क्षान्तिरेषा सुनिर्मला\thinspace{\devanagarifontsmall ॥} }}

{\devanagarifont वधार्थमुद्यतः शस्त्रं यदि पृच्छेत कर्हिचित् \thinspace{\dandab} \dontdisplaylinenum }%
     \var{{\devanagarifont \numemph\va\textbf{॰द्यतः}\lem \mssALL, ॰द्यत \msNa\oo 
\textbf{शस्त्रं}\lem \msCa\msNa\msNb\msNc, सत्य \msCb\Ed, शस्त्र \msCc}}% 
    \var{{\devanagarifont \numnoemph\vb\textbf{कर्हिचित्}\lem \mssCaCbCc\Ed, कर्हचित् \msNa\msNb\msNc}}% 

%Verse 4:3

{\devanagarifont न तत्र सत्यं वक्तव्यमनृतं सत्यमुच्यते {॥ ४:३॥} \veg\dontdisplaylinenum }%
     \var{{\devanagarifont \numnoemph\vc\textbf{सत्यं}\lem \mssALL, सत्य \msCb\Ed}}% 

{\devanagarifont वधार्हः पुरुषः कश्चिद्व्रजेत्पथि भयातुरः \thinspace{\dandab} \dontdisplaylinenum }%
     \var{{\devanagarifont \numemph\vb\textbf{॰तुरः}\lem \mssALL, ॰तुर \msCb}}% 

%Verse 4:4

{\devanagarifont पृच्छतो ऽपि न वक्तव्यं सत्यं तद्वापि उच्यते {॥ ४:४॥} \veg\dontdisplaylinenum }%
     \var{{\devanagarifont \numnoemph\vc\textbf{पृच्छतो}\lem \mssALL, पृच्छते \Ed}}% 
    \var{{\devanagarifont \numnoemph\vd\textbf{तद्वापि}\lem \mssALL, तदपि \msNb}}% 

\ujvers\nemsloka {
{\devanagarifont न नर्मयुक्तमनृतं हिनस्ति }%
  \dontdisplaylinenum}    \var{{\devanagarifont \numemph\va\textbf{हिनस्ति}\lem \msCa\msCb\msNb\msNc, हि नास्ति \msCc\msNa\Ed}}% 


\nemslokab

{\devanagarifont न स्त्रीषु राजन्न विवाहकाले  \danda\dontdisplaylinenum }%
     \var{{\devanagarifont \numnoemph\vb\textbf{राजन्न}\lem \mssALL, राज न \msCc, राज्यं न \msNa}}% 

\nemslokac

{\devanagarifont प्राणात्यये सर्वधनापहारे }%
  \dontdisplaylinenum    \var{{\devanagarifont \numnoemph\vc\textbf{॰त्यये}\lem \mssALL, ॰त्यजे \msNb\oo 
\textbf{॰पहारे}\lem \mssALL, ॰प्रहारे \msCc\msNb}}% 

%Verse 4:5


\nemslokad

{\devanagarifont पञ्चानृतं सत्यमुदाहरन्ति {॥ ४:५॥} \veg\dontdisplaylinenum }%
     \paral{{\devanagarifontsmall \vo {\englishfont \similar\ \MBH\ 1.77.16:} न नर्मयुक्तं वचनं हिनस्ति न स्त्रीषु राजन्न विवाहकाले\thinspace{\devanagarifontsmall ।}
                                                प्राणात्यये सर्वधनापहारे पञ्चानृतान्याहुरपातकानि\thinspace{\devanagarifontsmall ॥};
                            {\englishfont \MBH\ 12.159.28:} न नर्मयुक्तं वचनं हिनस्ति न स्त्रीषु राजन्न विवाहकाले\thinspace{\devanagarifontsmall ।}
                                                न गुर्वर्थे नात्मनो जीवितार्थे पञ्चानृतान्याहुरपातकानि\thinspace{\devanagarifontsmall ॥};
                              {\englishfont \MATSP\ 31.16:} न नर्मयुक्तं वचनं हिनस्ति न स्त्रीषु राजन्न विवाहकाले\thinspace{\devanagarifontsmall ।}
         {\englishfont Abhidharmakośabhāṣya 24114--24117 (introduced by } मोहजो मृषावादो यथाह{\englishfont ):}
                                                न नर्मयुक्तमनृतं हि नास्ति न स्त्रीषु राजन्न विवाहकाले\thinspace{\devanagarifontsmall ।}
                                                प्राणात्यये सर्वधनापहारे पञ्चानृतान्याहुरपातकानि\thinspace{\devanagarifontsmall ॥} {\englishfont etc.} }}

\vers


{\devanagarifont देवमानुषतिर्येषु सत्यं धर्मः परो यतः \thinspace{\dandab} \dontdisplaylinenum }%
     \var{{\devanagarifont \numemph\vb\textbf{॰मानुष॰}\lem \mssALL, ॰मानुष्य॰ \msNc\oo 
\lem  \msCb\msCc, सत्यं धर्मः पयतः \msCa, 
सत्यं धर्म परो यतः \msNa\msNc, सत्यधर्म परो यतः \msNb, सत्यधर्मपरायणः \Ed}}% 

%Verse 4:6

{\devanagarifont सत्यं श्रेष्ठं वरिष्ठं च सत्यं धर्मः सनातनः {॥ ४:६॥} \veg\dontdisplaylinenum }%
     \var{{\devanagarifont \numnoemph\vc\textbf{श्रेष्ठं}\lem \mssALL, श्रेष्ठ \msNb\Ed\oo 
\textbf{वरिष्ठं च}\lem \mssALL, वरिष्ठम्वरिष्ठम्वञ्च \msCbacorr}}% 
    \var{{\devanagarifont \numnoemph\vd\textbf{सत्यं}\lem \mssALL, सत्य॰ \msCb\msNb\oo 
\textbf{धर्मः}\lem \mssALL, धर्म \msCc\Ed}}% 

{\devanagarifont सत्यं सागरमव्यक्तं सत्यमक्षयभोगदम् \thinspace{\dandab} \dontdisplaylinenum }%
     \var{{\devanagarifont \numemph\va\textbf{सत्यं}\lem \mssALL, सत्य \msCc}}% 
    \var{{\devanagarifont \numnoemph\vb \lem \msCa\msNa\msNb\msNc, सत्यंमक्षयभोगदम् \msCb\msCc, 
सत्यमक्षयते नरं \Ed}}% 

%Verse 4:7

{\devanagarifont सत्यं पोतः परत्रार्थं सत्यं पन्थान विस्तरम् {॥ ४:७॥} \veg\dontdisplaylinenum }%
     \var{{\devanagarifont \numnoemph\vc\textbf{पोतः}\lem \mssALL, पोत \msNa, प्रोक्तः \Ed}}% 
    \var{{\devanagarifont \numnoemph\vd\textbf{पन्थान विस्तरम्}\lem \mssALL, यज्ज्ञानविस्तरम् \Ed}}% 

{\devanagarifont सत्यमिष्टगतिः प्रोक्तं सत्यं यज्ञमनुत्तमम् \thinspace{\dandab} \dontdisplaylinenum }%
     \var{{\devanagarifont \numemph\va\textbf{॰ष्टगतिः}\lem \mssALL, ॰\uncl{ष्टा}गतिः \msNb}}% 

%Verse 4:8

{\devanagarifont सत्यं तीर्थं परं तीर्थं सत्यं दानमनन्तकम् {॥ ४:८॥} \veg\dontdisplaylinenum }%
     \var{{\devanagarifont \numnoemph\vc\textbf{तीर्थं}\lem \mssCaCbCc\msNa, तीर्थ \msNb\msNc, तीर्थात् \Ed}}% 

{\devanagarifont सत्यं शीलं तपो ज्ञानं सत्यं शौचं दमः शमः \thinspace{\dandab} \dontdisplaylinenum }%
     \var{{\devanagarifont \numemph\va\textbf{सत्यं}\lem \mssALL, सत्य \msCb}}% 
    \var{{\devanagarifont \numnoemph\vb\textbf{शमः}\lem \mssALL, शमम् \msNb}}% 

%Verse 4:9

{\devanagarifont सत्यं सोपानमूर्ध्वस्य सत्यं कीर्तिर्यशः सुखम् {॥ ४:९॥} \veg\dontdisplaylinenum }%
     \var{{\devanagarifont \numnoemph\vc\textbf{सत्यं}\lem \mssALL, संत्यं \msCb, सत्य \msNc}}% 
    \var{{\devanagarifont \numnoemph\vd\textbf{सुखम्}\lem \mssALL, सुखः \Ed}}% 
    \paral{{\devanagarifontsmall \vc {\englishfont \similar\ \VARP\ 193.36cd:} सत्यं स्वर्गस्य सोपानं पारावारस्य नौरिव }}

{\devanagarifont अश्वमेधसहस्रं च सत्यं च तुलया धृतम् \thinspace{\dandab} \dontdisplaylinenum }%
     \var{{\devanagarifont \numemph\va\textbf{॰सहस्रं च}\lem \mssALL, ॰सहस्रस्य \msCc}}% 
    \var{{\devanagarifont \numnoemph\vb\textbf{तुलया}\lem \mssALL, तुल्यया \msCc}}% 

%Verse 4:10

{\devanagarifont अश्वमेधसहस्राद्धि सत्यमेव विशिष्यते {॥ ४:१०॥} \veg\dontdisplaylinenum }%
     \var{{\devanagarifont \numnoemph\vc\textbf{॰सहस्राद्धि}\lem \mssALL, ॰सहस्रा हि \msCc}}% 
    \var{{\devanagarifont \numnoemph\vd\textbf{एव}\lem \mssALL, एवं \msCc\Ed}}% 
    \paral{{\devanagarifontsmall \vo {\englishfont  = \MBH\ 1.69.22 = \MBH\ Suppl. 13.20.330 = \MARKP\ 8.42 = \VDHU\ 3.265.7
                        \similar\ \MBH\ 12.156.26 (pāda d reads } सत्यमेवातिरिच्यते{\englishfont ) \similar\ \VDH\ 55.6 
                            (pāda d reads} सत्यमेतद्विशिष्यते{\englishfont )};
                    {\englishfont \compare\ \SDHS\ 11.107:}
                         अश्वमेधायुतं पूर्णं सत्यञ्च तुलितं पुरा\thinspace{\devanagarifontsmall ।}
                         अश्वमेधायुतात्सत्यमधिकं बहुभिर्गुणैः\thinspace{\devanagarifontsmall ॥} }}

{\devanagarifont सत्येन तपते सूर्यः सत्येन पृथिवी स्थिता \thinspace{\dandab} \dontdisplaylinenum }%
     \var{{\devanagarifont \numemph\vab\textbf{सूर्यः सत्येन पृथिवी स्थिता}\lem \msNa\msNc, सू\uncl{र्यः स}त्येन पृथि स्थिताः \msCa, 
सूर्यः सत्यैन पृथिवी स्थिता \msCb, सूर्य सत्येन पृथिवी स्थिताः \msCc, 
सूर्य \uncl{सत्ये} \lac\  वी स्थिता \msNb, सूर्यः सत्येन पृथिवी स्थिताः \Ed}}% 

%Verse 4:11

{\devanagarifont सत्येन वायवो वान्ति सत्ये तोयं च शीतलम् {॥ ४:११॥} \veg\dontdisplaylinenum }%
     \var{{\devanagarifont \numnoemph\vc\textbf{वायवो}\lem \mssALL, वात्यवो \msNb}}% 
    \var{{\devanagarifont \numnoemph\vd\textbf{सत्ये}\lem \mssALL, सत्यात् \Ed}}% 
    \paral{{\devanagarifontsmall \vo {\englishfont \similar\ \VARP\ 193.37:} 
                         सूर्यस्तपति सत्येन वातः सत्येन वाति च\thinspace{\devanagarifontsmall ।}  
                         अग्निर्दहति सत्येन सत्येन पृथिवी स्थिता\thinspace{\devanagarifontsmall ॥} 
                    {\englishfont \similar\ \VDHU\ 3.265.4cd--5ab:}
                         सत्येन वायुरभ्येति सत्येनाभासते रविः\thinspace{\devanagarifontsmall ॥} 
                         सत्येन चाग्निर्दहति स्वर्गं सत्येन गच्छति\thinspace{\devanagarifontsmall ।}  }}

{\devanagarifont तिष्ठन्ति सागराः सत्ये समयेन प्रियव्रतः \thinspace{\dandab} \dontdisplaylinenum }%
     \var{{\devanagarifont \numemph\va\textbf{सागराः}\lem \mssALL, सागरा \msCc}}% 
    \var{{\devanagarifont \numnoemph\vb\textbf{समयेन}\lem \mssALL, सत्येन च \Ed}}% 

%Verse 4:12

{\devanagarifont सत्ये तिष्ठति गोविन्दो बलिबन्धनकारणात् {॥ ४:१२॥} \veg\dontdisplaylinenum }%
 
{\devanagarifont अग्निर्दहति सत्येन सत्येन शशिनश्चरः \thinspace{\dandab} \dontdisplaylinenum }%
     \var{{\devanagarifont \numemph\vab\textbf{सत्येन सत्येन}\lem \mssALL, सत्येन \msNaacorr\msNc}}% 
    \var{{\devanagarifont \numnoemph\vb\textbf{शशिनश्चरः}\lem \conj, सशि\uncl{भाचरः} \msCa, 
श\uncl{सि}\lk चरः \msCb, 
स शिरा वरः \msCc, 
शशिराचरः \msNa\msNb\msNc, 
शशिभाष्करः \Ed}}% 
    \paral{{\devanagarifontsmall \vc {\englishfont \similar\ \VARP\ 193.37cd:} 
                 अग्निर्दहति सत्येन सत्येन पृथिवी स्थिता }}
    \paral{{\devanagarifontsmall \vd {\englishfont \compare\ \VARP\ 155.30cd:}
                         सत्येन सूर्यस्तपति सोमः सत्येन राजते;
                  {\englishfont \compare\ \LAKSMINARS\  1.345.50ab:}
                         सत्येन सूर्यस्तपति चन्द्रः सत्येन वर्धते\thinspace{\devanagarifontsmall ।}
                 {\englishfont \compare\ \MBH\ Suppl. 13.587:}
                         मुचुकुन्देन मान्धात्रा हरिश्चन्द्रेण चाभिभो\thinspace{\devanagarifontsmall ।}
                         सत्यं वदत मासत्यं सत्यं धर्मः सनातनः\thinspace{\devanagarifontsmall ।}
                         हरिश्चन्द्रश्चरति वै दिवि सत्येन चन्द्रवत्\thinspace{\devanagarifontsmall ॥} }}

%Verse 4:13

{\devanagarifont सत्येन विन्ध्यास्तिष्ठन्ति वर्धमानो न वर्धते {॥ ४:१३॥} \veg\dontdisplaylinenum }%
     \var{{\devanagarifont \numnoemph\vc\textbf{विन्ध्यास्तिष्ठन्ति}\lem \msCa\msNa\msNc, 
विन्ध्यस्तिष्ठन्ति \msCb\msNb, विन्ध्या तिष्ठन्ति \msCc, तिष्ठते विन्ध्यो \Ed}}% 

{\devanagarifont लोकालोकः स्थितः सत्ये मेरुः सत्ये प्रतिष्ठितः \thinspace{\dandab} \dontdisplaylinenum }%
     \var{{\devanagarifont \numemph\va\textbf{॰लोकः}\lem \Ed, ॰लोक \mssCaCbCc\msNa\msNb\msNc\oo 
\textbf{स्थितः}\lem \mssALL, स्थिः \msNc\oo 
\textbf{सत्ये}\lem \mssALL, सत्यं \Ed}}% 
    \var{{\devanagarifont \numnoemph\vb\textbf{मेरुः}\lem \mssALL, मेरु \msCc\Ed}}% 

%Verse 4:14

{\devanagarifont वेदास्तिष्ठन्ति सत्येषु धर्मः सत्ये प्रतिष्ठति {॥ ४:१४॥} \veg\dontdisplaylinenum }%
     \var{{\devanagarifont \numnoemph\vc\textbf{वेदास्ति॰}\lem \mssALL, देवास्ति॰ \msCb, वेदा ति॰ \Ed}}% 
    \var{{\devanagarifont \numnoemph\vd\textbf{सत्ये}\lem \mssALL, धर्मे \msCc\oo 
\textbf{प्रतिष्ठति}\lem \mssALL, प्रतिष्ठिति \msNcacorr, प्रतिष्ठितः \msNcpcorr}}% 

{\devanagarifont सत्यं गौः क्षरते क्षीरं सत्यं क्षीरे घृतं स्थितम् \thinspace{\dandab} \dontdisplaylinenum }%
     \var{{\devanagarifont \numemph\va\textbf{गौः}\lem \mssALL, गौ \msCc\msNb}}% 
    \var{{\devanagarifont \numnoemph\vab\textbf{क्षीरं सत्यं}\lem \mssALL, क्षीत्यं \msCbacorr, क्सी\lk  नित्यं \msCbpcorr}}% 
    \var{{\devanagarifont \numnoemph\vb\textbf{क्षीरे घृतं स्थितम्}\lem \msCa\msCb\msNa\msNc, क्षीरं घृतं स्थितम् \msCc, क्षीरे घृत स्थितम् \msNb, 
क्षीरं स्थितं घृतम् \Ed}}% 

%Verse 4:15

{\devanagarifont सत्ये जीवः स्थितो देहे सत्यं जीवः सनातनः {॥ ४:१५॥} \veg\dontdisplaylinenum }%
     \var{{\devanagarifont \numnoemph\vc\textbf{सत्ये जीवः}\lem \mssALL, सत्ये जीव \msNc, सत्यं जीव \Ed}}% 
    \var{{\devanagarifont \numnoemph\vd\textbf{जीवः}\lem \mssALL, जीव \msCc}}% 

{\devanagarifont सत्यमेकेन सम्प्राप्तो धर्मसाधननिश्चयः \thinspace{\dandab} \dontdisplaylinenum }%
     \var{{\devanagarifont \numemph\va\textbf{सत्यमेकेन}\lem \mssALL, सत्यमेकैन \msCb, सत्येमेकेन \msNb}}% 
    \var{{\devanagarifont \numnoemph\vb\textbf{धर्म॰}\lem \Ed, धर्मः \mssCaCbCc\msNa\msNb\msNc\oo 
\textbf{॰निश्चयः}\lem \mssALL, ॰निश्चः \msCa}}% 

%Verse 4:16

{\devanagarifont रामराघववीर्येण सत्यमेकं सुरक्षितम् {॥ ४:१६॥} \veg\dontdisplaylinenum }%
     \var{{\devanagarifont \numnoemph\vd\textbf{सत्यमेकं}\lem \mssALL, सत्येमेकं \msNb\oo 
\textbf{सुरक्षितम्}\lem \mssALL, सुरिक्षितम् \msCb, सुरक्षितः \msNa}}% 

{\devanagarifont एवं सत्यविधानस्य कीर्तितं तव सुव्रत \thinspace{\dandab} \dontdisplaylinenum }%
     \var{{\devanagarifont \numemph\va\textbf{एवं सत्य॰}\lem \msCb, एतत्सत्य॰ \msCa\msCc\msNa\msNb\msNc\Ed}}% 
    \var{{\devanagarifont \numnoemph\vb\textbf{सुव्रत}\lem \msCa\msNa\msNc, सुव्रते \msCb\msNb, सुव्र\uncl{तः} \msCc, सुव्रतं \Ed}}% 

%Verse 4:17

{\devanagarifont सर्वलोकहितार्थाय किमन्यच्छ्रोतुमिच्छसि {॥ ४:१७॥} \veg\dontdisplaylinenum }%
 

\alalfejezet{यमेष्वस्तेयम् (३)}
{\devanagarifont विगतराग उवाच {\dandab}\dontdisplaylinenum  }%
 
{\devanagarifont न हि तृप्तिं विजानामि श्रुत्वा धर्मं तवाप्यहम् \thinspace{\danda} \dontdisplaylinenum }%
     \var{{\devanagarifont \numemph\va\textbf{तृप्तिं}\lem \mssALL, तृप्ति \msCc\oo 
\textbf{विजानामि}\lem \mssALL, विनामि \msNb}}% 
    \var{{\devanagarifont \numnoemph\vb \lem \mssALL, 
श्रु धर्मन्तवाप्यहम् \msCa, 
धर्मं श्रुत्वा तथाप्यहम् \Ed}}% 

%Verse 4:18

{\devanagarifont उपरिष्टादतो भूयः कथयस्व तपोधन {॥ ४:१८॥} \veg\dontdisplaylinenum }%
     \var{{\devanagarifont \numnoemph\vd\textbf{॰धन}\lem \msCc\msNa\msNb\Ed, ॰धून \msCa, ॰धनः \msCb\msNc}}% 

{\devanagarifont अनर्थयज्ञ उवाच {\dandab}\dontdisplaylinenum  }%
 
{\devanagarifont स्तेयं शृण्वथ विप्रेन्द्र पञ्चधा परिकीर्तितम् \thinspace{\danda} \dontdisplaylinenum }%
     \var{{\devanagarifont \numemph\vb\textbf{॰कीर्तितम्}\lem \mssALL, ॰कीर्त्तिताम् \msCb}}% 

{\devanagarifont अदत्तादानमादौ तु उत्कोचं च ततः परम्  \danda\dontdisplaylinenum }%
     \var{{\devanagarifont \numnoemph\vd\textbf{उत्कोचं च ततः}\lem \mssALL, त्कोच ततः \msCb, उत्कोचं चानृतः \Ed}}% 

%Verse 4:19

{\devanagarifont प्रस्थव्याजस्तुलाव्याजः प्रसह्यस्तेय पञ्चमम् {॥ ४:१९॥} \veg\dontdisplaylinenum }%
     \var{{\devanagarifont \numnoemph\ve\textbf{तुलाव्याजः}\lem \msCb\msNc\Ed, तुलाव्याज \msCa\msCc\msNa\msNb}}% 
    \var{{\devanagarifont \numnoemph\vf\textbf{॰सह्य॰}\lem \mssALL, ॰सह्ये \msNb\oo 
\textbf{॰स्तेय}\lem \mssALL, ॰स्तेन \msCa\msNc\oo 
\textbf{पञ्चमम्}\lem \mssALL, पञ्चमः \msCc\Ed}}% 

{\devanagarifont धृष्टदुष्टप्रभावेन परद्रव्यापकर्षणम् \thinspace{\dandab} \dontdisplaylinenum }%
     \var{{\devanagarifont \numemph\va\textbf{धृष्टदुष्ट॰}\lem \msCa\msNa\msNc\Ed, धृष्टदुम्न॰ \msCb, धृतदुष्ट॰ \msCc, दृष्टदुष्ट॰ \msNb}}% 
    \var{{\devanagarifont \numnoemph\vb\textbf{॰कर्षणम्}\lem \mssALL, ॰कर्षण \msNa}}% 

%Verse 4:20

{\devanagarifont वार्यमाणो ऽपि दुर्बुद्धिरदत्तादानमुच्यते {॥ ४:२०॥} \veg\dontdisplaylinenum }%
     \var{{\devanagarifont \numnoemph\vc\textbf{वार्यमाणो ऽपि}\lem \mssALL, वार्यमानो वि॰ \msCb}}% 

{\devanagarifont उत्कोचं शृणु विप्रेन्द्र धर्मसंकरकारकम् \thinspace{\dandab} \dontdisplaylinenum }%
     \var{{\devanagarifont \numemph\va\textbf{उत्कोचं}\lem \mssALL, उत्कोच \msCa\oo 
\textbf{विप्रेन्द्र}\lem \mssALL, विद्रेन्द्र \msNb}}% 
    \var{{\devanagarifont \numnoemph\vb\textbf{॰संकर॰}\lem \msCc\msNa, ॰शङ्कर॰ \msCa\msCb\msNb, ॰सकर॰ \msNc, ॰संहार॰ \Ed\oo 
\textbf{॰कारकम्}\lem \mssALL, ॰कारकः \msNa}}% 

{\devanagarifont मूल्यं कार्यविनाशार्थमुत्कोचः परिगृह्यते  \danda\dontdisplaylinenum }%
     \var{{\devanagarifont \numnoemph\vc\textbf{मूल्यं}\lem \conj, मूल \mssCaCbCc\msNa\msNb\msNc\Ed\oo 
\textbf{॰विनाशार्थ॰}\lem \mssALL, ॰विनार्थ॰ \msNaacorr}}% 
    \var{{\devanagarifont \numnoemph\vd\textbf{॰त्कोचः}\lem \mssALL, ॰त्कोचं \msNb, ॰त्कोच \Ed}}% 

%Verse 4:21

{\devanagarifont तेन चासौ विजानीयाद्द्रव्यलोभबलात्कृतम् {॥ ४:२१॥} \veg\dontdisplaylinenum }%
     \var{{\devanagarifont \numnoemph\vef\textbf{विजानीयाद्द्र॰}\lem \mssALL, विजानीया द्र॰ \msCc}}% 

{\devanagarifont प्रस्थव्याज-उपायेन कुटुम्बं त्रातुमिच्छति \thinspace{\dandab} \dontdisplaylinenum }%
 
%Verse 4:22

{\devanagarifont तं च स्तेनं विजानीयात्परद्रव्यापहारकम् {॥ ४:२२॥} \veg\dontdisplaylinenum }%
     \var{{\devanagarifont \numemph\vc\textbf{तं च स्तेनं}\lem \msCa, तञ्च स्तेन \msCb, 
सो ऽपि तेन \msCc\Ed, तं च स्तेयं \msNa, तञ्च तेय \msNb, तञ्च तेन \msNc}}% 
    \var{{\devanagarifont \numnoemph\vd\textbf{॰हारकम्}\lem \msCa\msCb\msNapcorr\msNc\Ed, ॰हारकः \msCc, ॰हारका \msNaacorr ॰हारकाः \msNb}}% 

{\devanagarifont तुलाव्याज-उपायेन परस्वार्थं हरेद्यदि \thinspace{\dandab} \dontdisplaylinenum }%
     \var{{\devanagarifont \numemph\va\textbf{परस्वार्थं}\lem \msCa\msCc\msNa\msNc, परस्वार्थ \msCb\msNb, परस्यार्थं \Ed\oo 
\textbf{हरेद्यदि}\lem \mssALL, हरेद्यति \msCb}}% 

%Verse 4:23

{\devanagarifont चौरलक्षणकाश्चान्ये कूटकापटिका नराः {॥ ४:२३॥} \veg\dontdisplaylinenum }%
     \var{{\devanagarifont \numnoemph\vd\textbf{कूटकापटिका}\lem \msNb, \uncl{कु}टका यटिका \msCa, कूटकायटिका \msCb\msCc\msNaacorr\msNc, 
कूटकार्यटिका \msNapcorr\Ed}}% 
    \paral{{\devanagarifontsmall \vcd {\englishfont \compare\ \UMS\ 8.3cd:} कूटकापटिकाश्चैव सत्यार्जवविवर्जिताः }}

{\devanagarifont दुर्बलार्जवबालेषु च्छद्मना वा बलेन वा \thinspace{\dandab} \dontdisplaylinenum }%
     \var{{\devanagarifont \numemph\va\textbf{॰र्जव॰}\lem \mssALL, ॰जव॰ \msNb}}% 
    \var{{\devanagarifont \numnoemph\vb\textbf{च्छद्मना}\lem \Ed, च्छन्मना \mssCaCbCc\msNa\msNb, च्छत्माना \msNc}}% 

%Verse 4:24

{\devanagarifont अपहृत्य धनं मूढः स चौरश्चोर उच्यते {॥ ४:२४॥} \veg\dontdisplaylinenum }%
     \var{{\devanagarifont \numnoemph\vcd\textbf{मूढः स}\lem \mssALL, मूढास्स \msNb}}% 
    \var{{\devanagarifont \numnoemph\vd\textbf{चौरश्चोर}\lem \msNc, चोरश्चोर \msCa\msCc\msNb\Ed, चौर चोर \msCb, चौरश्चौर \msNa}}% 

{\devanagarifont नास्ति स्तेयसमं पापं नास्त्यधर्मश्च तत्समः \thinspace{\dandab} \dontdisplaylinenum }%
     \var{{\devanagarifont \numemph\va\textbf{स्तेय॰}\lem \msNa\msNc, तेन \msCa, स्तेन॰ \msCb\msCc\msNb}}% 
    \var{{\devanagarifont \numnoemph\vb\textbf{॰समः}\lem \mssALL, ॰समं \msCc}}% 
    \lacuna{\devanagarifontsmall \vo {\englishfont This verse is missing in \Ed.} }%
  
%Verse 4:25

{\devanagarifont नास्ति स्तेनसमाकीर्तिर्नास्ति स्तेनसमो ऽनयः {॥ ४:२५॥} \veg\dontdisplaylinenum }%
     \var{{\devanagarifont \numnoemph\vc\textbf{स्तेन॰}\lem \mssALL, तेन \msCc, स्तेय॰ \msNc\oo 
\textbf{॰समा॰}\lem \msCb\msCc\msNb, ॰समो \msCa\msNa\msNc}}% 
    \var{{\devanagarifont \numnoemph\vd\textbf{स्तेन॰}\lem \mssALL, स्तेय॰ \msNa\msNc}}% 

{\devanagarifont नास्ति स्तेयसमाविद्या नास्ति स्तेनसमः खलः \thinspace{\dandab} \dontdisplaylinenum }%
     \var{{\devanagarifont \numemph\va\textbf{स्तेय॰}\lem \msNa\msNc\Ed, स्तेन॰ \mssCaCbCc\msNb\oo 
\textbf{॰समा}\lem \msCc\msNb, ॰समो \msCa\msCb\msNa\msNc\Ed}}% 
    \var{{\devanagarifont \numnoemph\vb\textbf{स्तेन॰}\lem \mssCaCbCc\msNb, स्तेय॰ \msNa\msNc, तेन \Ed}}% 

%Verse 4:26

{\devanagarifont नास्ति स्तेनसम अज्ञो नास्ति स्तेनसमो ऽलसः {॥ ४:२६॥} \veg\dontdisplaylinenum }%
     \var{{\devanagarifont \numnoemph\vc\textbf{स्तेन॰}\lem \msCa\msCb\msNb\msNc, स्तेय॰ \msCc\msNa\Ed\oo 
\textbf{॰सम}\lem \mssALL, ॰समं \msNb\oo 
\textbf{अज्ञो}\lem \msCb, अज्ञ\lk\ \msCa, अज्ञ \msCc\msNa\msNb\msNc, अज्ञः \Ed}}% 
    \var{{\devanagarifont \numnoemph\vd\textbf{स्तेन॰}\lem \msCa\msCb\msNb, स्तेय॰ \msCc\msNa\msNc, तेन \Ed}}% 

{\devanagarifont नास्ति स्तेनसमो द्वेष्यो नास्ति स्तेनसमो ऽप्रियः \thinspace{\dandab} \dontdisplaylinenum }%
     \var{{\devanagarifont \numemph\va\textbf{स्तेन॰}\lem \msCa\msCb\msNb, स्तेय॰ \msCc\msNa\msNc, तेन \Ed}}% 
    \var{{\devanagarifont \numnoemph\vb\textbf{स्तेन॰}\lem \msNb, स्तेय॰ \mssCaCbCc\msNa\msNc\Ed}}% 

%Verse 4:27

{\devanagarifont नास्ति स्तेयसमं दुःखं नास्ति स्तेयसमो ऽयशः {॥ ४:२७॥} \veg\dontdisplaylinenum }%
     \var{{\devanagarifont \numnoemph\vc\textbf{स्तेय॰}\lem \msCc, स्तेन॰ \msCa\msCb\msNa\msNb, स्तेन्य॰ \msNc, तेन \Ed}}% 
    \var{{\devanagarifont \numnoemph\vd\textbf{स्तेय॰}\lem \msCc\msNc, स्तेन॰ \msCa\msCb\msNa\msNb, तेन \Ed}}% 

\nemslokalong


\ujvers\nemsloka {
{\devanagarifont प्रच्छन्नो ह्रियते ऽर्थमन्यपुरुषः प्रत्यक्षमन्यो हरेत् }%
  \dontdisplaylinenum}    \var{{\devanagarifont \numemph\va\textbf{प्रच्छन्नो}\lem \mssALL, प्रस्थन्नो \msCb\oo 
\textbf{ऽर्थमन्यपुरुषः}\lem \msCb\msNc, 
वित्तम् \msCa\msNaacorr\msNb, 
चित्त \msCc, च वित्तमथवा \msNapcorr\Ed\oo 
\textbf{प्रत्यक्षमन्यो}\lem \mssALL, प्रत्यक्षमनो \msCb, 
प्रत्यक्ष्यमन्ये \Ed}}% 


\nemslokab

{\devanagarifont निक्षेपाद्धनहारिणो ऽन्यमधमो व्याजेन चान्यो हरेत्  \danda\dontdisplaylinenum }%
     \var{{\devanagarifont \numnoemph\vb\textbf{निक्षेपाद्धन॰}\lem \msCa\msCb\msNa, निक्षेपा धन॰ \msCc\msNb\msNc, निक्षेपात्रय॰ \Ed\oo 
\textbf{॰हारिणो}\lem \mssALL, ॰हारिण्यो \msCb, ॰हारिणा \msNb\oo 
\textbf{ऽन्यमधमो}\lem \mssALL, ऽन्यमधनो \msCc, ऽन्यविधयो \Ed\oo 
\textbf{चान्यो}\lem \mssALL, चान्या \Ed\oo 
\textbf{हरेत्}\lem \mssALL, हरे \msNa}}% 

\nemslokac

{\devanagarifont अन्ये लेख्यविकल्पनाहृतधना †अन्यो हृताद्वै हृता† }%
  \dontdisplaylinenum    \var{{\devanagarifont \numnoemph\vc\textbf{अन्ये लेख्य॰}\lem \corr, अन्या लेख॰ \msCb\msCc, 
अन्यो ले\uncl{ख्य}॰ \msCa, अन्यो लेख्य॰ \msNa\msNb\msNc, अन्योल्लेख्य \Ed\oo 
\textbf{॰धना अन्यो}\lem \mssALL, ॰धन्यो \msCb\oo 
\textbf{हृताद्वै}\lem \mssALL, हृतद्वै \msNa, हृताद्वे \msNb}}% 

%Verse 4:28


\nemslokad

{\devanagarifont अन्यः क्रीतधनो ऽपरो धयहृत एते जघन्याः स्मृताः {॥ ४:२८॥} \veg\dontdisplaylinenum }%
     \var{{\devanagarifont \numnoemph\vd\textbf{अन्यः क्रीतधनो}\lem \mssALL, अन्य क्रीतधनो \msNc, अनाश्रीतधनं \Ed\oo 
\textbf{ऽपरो धयहृत}\lem \msCa\msCc\msNb, परो धयह्यत \msCb, परो धन\uncl{हृत} \msNa, 
परोधप्रहृत \msNc, मदा ह्यपहृतं \Ed\oo 
\textbf{जघन्याः}\lem \mssALL, जघन्यः \Ed}}% 

\ujvers\nemsloka {
{\devanagarifont स्तेनतुल्य न मूढमस्ति पुरुषो धर्मार्थहीनो ऽधमः }%
  \dontdisplaylinenum}    \var{{\devanagarifont \numemph\va\textbf{स्तेनतुल्य}\lem \msCa\msCb\msNc\ \unmetr, स्तेयस्तुल्य \msCc, 
स्तेयतुल्य \msNa\ \unmetr, तेन तुल्य \msNb\ \unmetr, स्तेनस्तुल्य \Ed}}% 


\nemslokab

{\devanagarifont यावज्जीवति शङ्कया नरपतेः संत्रस्यमानो रटन्  \danda\dontdisplaylinenum }%
     \var{{\devanagarifont \numnoemph\vb\textbf{यावज्जीवति}\lem \mssALL, यावत्तज्जीवति \Ed\oo 
\textbf{॰पतेः}\lem \msCb\msNb\msNc, ॰पतिः \msCa\msCc\msNa\Ed\oo 
\textbf{संत्रस्यमानो रटन्}\lem \mssALL, संत्रास्यमानो शठः \Ed}}% 
    \lacuna{\devanagarifontsmall \vo {\englishfont The lower folio side in exposure 49 in \msNb\ is rather blurred and seems to be partly erased,
                        therefore all the readings in this MS for verses 4.29--46 are rather uncertain,
                        even if not indicated explicitly.} }%
  
\nemslokac

{\devanagarifont प्राप्तःशासन तीव्रसह्यविषमं प्राप्नोति कर्मेरितः }%
  \dontdisplaylinenum    \var{{\devanagarifont \numnoemph\vc\textbf{प्राप्तः॰}\lem \mssALL, प्राप्त॰ \msNa\oo 
\textbf{॰सह्य॰}\lem \mssALL, \lac\  \msNb, ॰सद्य॰ \Ed\oo 
\textbf{॰विषमं}\lem \eme, ॰विषमः \mssCaCbCc\msNa\msNc\Ed, \lac\  \msNb\oo 
\textbf{कर्मेरितः}\lem \mssALL, कर्मे\uncl{रित} \msCa, 
\lac \uncl{रितः} \msNb}}% 

%Verse 4:29


\nemslokad

{\devanagarifont कालेन म्रियते स याति निरयमाक्रन्दमानो भृशम् {॥ ४:२९॥} \veg\dontdisplaylinenum }%
     \var{{\devanagarifont \numnoemph\vd\textbf{निरयमाक्रन्दमानो}\lem \mssCaCbCc\msNa, \uncl{निर}यमाक्रन्दमा\uncl{नो} \msNb, 
निरयं स क्रन्दमानो \msNc, नियममाक्रन्द्रमानो \Ed}}% 

\nemslokalong


\ujvers\nemsloka {
{\devanagarifont नीत्वा दुर्गतिकोटिकल्प निरयात्तिर्यत्वमायान्ति ते }%
  \dontdisplaylinenum}    \var{{\devanagarifont \numemph\va\textbf{निरयात्तिर्यत्व॰}\lem \msCb\msNa, निरयान्तिर्यत्व॰ \msCa, निरया तिर्यत्व॰ \msCc, 
नि\uncl{रया$\-$त्तिर्यत्व}॰ \msNb, निरयान्तिर्यक्ष॰ \msNc, निरयान्तिर्यक्त्व॰ \Ed}}% 


\nemslokab

{\devanagarifont तिर्यत्वे च तथैवमेकशतिकं प्रभ्रम्य वर्षार्बुदम्  \danda\dontdisplaylinenum }%
     \var{{\devanagarifont \numnoemph\vb\textbf{तिर्यत्वे}\lem \mssALL, \uncl{तिर्यत्वे} \msNb, तिर्यक्त्वं \Ed\oo 
\textbf{तथैवमेकशतिकं}\lem \msCb, तथैकमेकशतिकं \msCa\msNa\msNc, 
तथैकमेकशतिक \msCc, \uncl{तथै}कमेकशतिकं \msNb, तथैकमेकसकिकं \Ed\oo 
\textbf{॰भ्रम्य॰}\lem \mssALL, ॰भ्राम्य \msNa, \lac  म्य \msNb\oo 
\textbf{वर्षार्बुदम्}\lem \msNcpcorr, वर्षाम्बुदम् \msCa\msCb\msNa\msNb\msNcacorr, वर्षाम्बुदः \msCc\Ed}}% 

\nemslokac

{\devanagarifont मानुष्यं तदवाप्नुवन्ति विपुले दारिद्र्यरोगाकुलं }%
  \dontdisplaylinenum    \var{{\devanagarifont \numnoemph\vc\textbf{मानुष्यं}\lem \mssALL, 
मानुष्य \msCb\ \unmetr, \uncl{मानुष्य} \msNb\ \toplost\oo 
\textbf{विपुले}\lem \mssALL, विपु\uncl{ल} \msNb\ \toplost, विपुलं \Ed\oo 
\textbf{दारिद्र्य॰}\lem \mssALL, \lk रि\lk\ \msNb, दारिध्र॰ \Ed}}% 

%Verse 4:30


\nemslokad

{\devanagarifont तस्माद्दुर्गतिहेतु कर्म सकलं त्यक्त्वा शिवं चाश्रयेत् {॥ ४:३०॥} \veg\dontdisplaylinenum }%
     \var{{\devanagarifont \numnoemph\vd\textbf{तस्माद्दु॰}\lem \mssALL, तस्मा दु॰ \msCc, \uncl{तस्मा दु}॰ \msNb\oo 
\textbf{चाश्रयेत्}\lem \mssALL, चाश्रत् \msNa}}% 

\nemslokanormal



\alalfejezet{यमेष्वानृशंस्यम् (४)}
\vers


{\devanagarifont अष्टमूर्तिशिवद्वेष्टा पितुर्मातुश्च यो द्विषेत् \thinspace{\dandab} \dontdisplaylinenum }%
     \var{{\devanagarifont \numemph\va\textbf{॰शिव॰}\lem \mssALL, ॰शिवं \msNc}}% 

%Verse 4:31

{\devanagarifont गवां वा अतिथेर्द्वेष्टा नृशंसाः पञ्च एव ते {॥ ४:३१॥} \veg\dontdisplaylinenum }%
     \var{{\devanagarifont \numnoemph\vc\textbf{गवां वा}\lem \mssALL, अवाम्वा \msCb, \lk\lk \uncl{म्वा} \msNb\oo 
\textbf{अतिथेर्द्वे॰}\lem \mssALL, अतिथिद्वे॰ \msCc, अतिथे द्वे॰ \msNa}}% 
    \var{{\devanagarifont \numnoemph\vd\textbf{नृशंसाः}\lem \msCa\msCc\msNa\msNb, नृशंसा \msCb\msNc\Ed}}% 

{\devanagarifont अष्टमूर्तिः शिवः साक्षात्पञ्चव्योमसमन्वितः \thinspace{\dandab} \dontdisplaylinenum }%
     \var{{\devanagarifont \numemph\va\textbf{॰मूर्तिः}\lem \mssALL, ॰मूर्ति॰ \Ed}}% 
    \var{{\devanagarifont \numnoemph\vb\textbf{॰न्वितः}\lem \mssALL, ॰न्विताः \msCc\msNb}}% 

%Verse 4:32

{\devanagarifont सूर्यः सोमश्च दीक्षश्च दूषकः स नृशंसकः {॥ ४:३२॥} \veg\dontdisplaylinenum }%
     \var{{\devanagarifont \numnoemph\vc\textbf{सूर्यः}\lem \mssCaCbCc\msNa, \uncl{सूर्य}॰ \msNb\msNc, सूर्य॰ \Ed\oo 
\textbf{दीक्ष॰}\lem \mssALL, \uncl{दी}\lk\ \msNb, दीक्षु॰ \Ed}}% 
    \paral{{\devanagarifontsmall \vo {\englishfont \compare\ \SDHS\ 12.17:}
                 मूर्तयो याः शिवस्याष्टौ तासु निन्दां विवर्जयेत्\thinspace{\devanagarifontsmall ।}
                 गुरोश्च शिवभक्तानां नृपसाधुतपस्विनां\thinspace{\devanagarifontsmall ॥} }}

{\devanagarifont पिताकाशसमो ज्ञेयो जन्मोत्पत्तिकरः पिता \thinspace{\dandab} \dontdisplaylinenum }%
     \var{{\devanagarifont \numemph\vb\textbf{॰करः पिता}\lem \mssALL, ॰करपिताः \msCc, ॰\uncl{करः पिता} \msNb}}% 

%Verse 4:33

{\devanagarifont पितृदैवत†मादिश्चमानृशंस तमन्वितः† {॥ ४:३३॥} \veg\dontdisplaylinenum }%
     \var{{\devanagarifont \numnoemph\vc\textbf{॰दैवत॰}\lem \mssALL, ॰देवत॰ \msCb, \lk वत॰ \msNb}}% 
    \var{{\devanagarifont \numnoemph\vcd\textbf{॰दिश्चमानृशंस तमन्वितः}\lem \msCa\msCb, 
॰दित्यमनृशंस तमन्वितः \msCc\msNb, 
॰दिश्च अनृशंस तमान्वितः \msNa, 
॰दिश्चमनृशंस तमान्वितः \msNc, 
॰दित्यम्मानृशंस ततो ऽन्वितः \Ed}}% 

{\devanagarifont पृथ्व्या गुरुतरी माता को न वन्देत मातरम् \thinspace{\dandab} \dontdisplaylinenum }%
     \var{{\devanagarifont \numemph\va\textbf{पृथ्व्या}\lem \msCa\msCb\msNc, \uncl{पृथ्व्या} \msCc\msNa, पृथ्वी \msNb, 
पृथ्व्यां \Ed}}% 
    \var{{\devanagarifont \numnoemph\vb\textbf{वन्देत}\lem \mssALL, वन्देन वन्देत \msCb, वन्द्येत \msCc}}% 

%Verse 4:34

{\devanagarifont यज्ञदानतपोवेदास्तेन सर्वं कृतं भवेत् {॥ ४:३४॥} \veg\dontdisplaylinenum }%
     \var{{\devanagarifont \numnoemph\vd\textbf{सर्वं}\lem \eme, सर्व \mssCaCbCc\msNa\msNb\msNc\Ed}}% 

{\devanagarifont गावः पवित्रं मङ्गल्यं देवतानां च देवताः \thinspace{\dandab} \dontdisplaylinenum }%
     \var{{\devanagarifont \numemph\va\textbf{पवित्रं}\lem \mssALL, \uncl{पवित्र} \msNb\oo 
\textbf{मङ्गल्यं}\lem \msCa\msCb\msNa, माङ्गल्यं \msCc\msNc\Ed, \uncl{मङ्गल्यं} \msNb\oo 
\textbf{देवताः}\lem \mssCaCbCc\msNc, दैवताः \msNa, \uncl{देवताः} \msNb, देवता \Ed}}% 
    \paral{{\devanagarifontsmall \va {\englishfont \similar\ \VISNUS\ 23.57c:} गावः पवित्रमङ्गल्यं (गोषु लोकाः प्रतिष्ठिता)\oo
                 {\englishfont \compare\ also \MBH\ Suppl. 13.15.33:} गावः पवित्रं परमं गोषु लोकाः प्रतिष्ठिताः 
                 {\englishfont and \AGNIP\ 291.1cd:} गावः पवित्रा माङ्गल्या गोषु लोकाः प्रतिष्ठिताः }}

%Verse 4:35

{\devanagarifont सर्वदेवमया गावस्तस्मादेव न हिंसयेत् {॥ ४:३५॥} \veg\dontdisplaylinenum }%
     \var{{\devanagarifont \numnoemph\vd\textbf{॰स्मादेव}\lem \mssALL, ॰स्मादुव \msCb, ॰स्माद्गावं \Ed}}% 
    \paral{{\devanagarifontsmall \vc {\englishfont = \VDHU\ 3.291.25c} }}

{\devanagarifont जातमात्रस्य लोकस्य गावस्त्राता न संशयः \thinspace{\dandab} \dontdisplaylinenum }%
     \var{{\devanagarifont \numemph\va \lem \msCa\msCc\msNa\msNc\Ed, सतसातस्य \msCbacorr, 
सतसातस्य नोकस्य \msCbpcorr, 
जातमात्र\uncl{स्य लोकस्य} \msNb}}% 

%Verse 4:36

{\devanagarifont घृतं क्षीरं दधि मूत्रं शकृत्कर्षणमेव च {॥ ४:३६॥} \veg\dontdisplaylinenum }%
     \var{{\devanagarifont \numnoemph\vd\textbf{शकृत्क॰}\lem \mssALL, क्षत्क॰ \msCb, \uncl{शकृत्क}॰ \msNb}}% 
    \paral{{\devanagarifontsmall \vo {\englishfont \compare\ \SDHU\ 12.92ff} }}

\ujvers\nemsloka {
{\devanagarifont पञ्चामृतं पञ्चपवित्रपूतं }%
  \dontdisplaylinenum}    \var{{\devanagarifont \numemph\va\textbf{॰पवित्रपूतम्}\lem \msCc\msNa\Ed, ॰पवित्रपूतन \msCa\ \unmetr, 
॰पवित्रं \msCb\ \unmetr, ॰पवित्रपूत \msNb, 
॰पवित्रपूतंनं \msNc\ \unmetr}}% 


\nemslokab

{\devanagarifont ये पञ्चगव्यं पुरुषाः पिबन्ति  \danda\dontdisplaylinenum }%
     \var{{\devanagarifont \numnoemph\vb\textbf{॰गव्यं}\lem \mssALL, ॰गव्या \msCc, ॰\uncl{गव्यां} \msNb\oo 
\textbf{पुरुषाः}\lem  \mssALL, पुरुषा \msCc, पुरुषः \Ed\oo 
\textbf{पिबन्ति}\lem  \mssALL, विवन्ति \msCc}}% 

\nemslokac

{\devanagarifont ते वाजिमेधस्य फलं लभन्ति }%
  \dontdisplaylinenum    \var{{\devanagarifont \numnoemph\vc\textbf{लभन्ति}\lem \mssALL, भवन्ति \msCc}}% 

%Verse 4:37


\nemslokad

{\devanagarifont तदक्षयं स्वर्गमवाप्नुवन्ति {॥ ४:३७॥} \veg\dontdisplaylinenum }%
     \var{{\devanagarifont \numnoemph\vd\textbf{स्वर्ग॰}\lem \mssALL, स्व॰ \msCb}}% 

\ujvers\nemsloka {
{\devanagarifont गोभिर्न तुल्यं धनमस्ति किंचिद् }%
  \dontdisplaylinenum}    \var{{\devanagarifont \numemph\va\textbf{गोभिर्न तु॰}\lem \msNc, न गोभिस्तु॰ \mssCaCbCc\msNa\msNb\ \unmetr, न गावतु॰ \Ed}}% 
    \paral{{\devanagarifontsmall \va {\englishfont = \SDHU\ 12.102d, 103d, 104d; } 
                    {\englishfont \compare\ \MBH\ 13.51.26cd:} गोभिस्तुल्यं न पश्यामि धनं किंचिदिहाच्युत }}


\nemslokab

{\devanagarifont दुह्यन्ति वाह्यन्ति बहिश्चरन्ति  \danda\dontdisplaylinenum }%
 
\nemslokac

{\devanagarifont तृणानि भुक्त्वा अमृतं स्रवन्ति }%
  \dontdisplaylinenum
%Verse 4:38


\nemslokad

{\devanagarifont विप्रेषु दत्ताः कुलमुद्धरन्ति {॥ ४:३८॥} \veg\dontdisplaylinenum }%
     \var{{\devanagarifont \numnoemph\vd\textbf{दत्ताः}\lem \mssALL, \uncl{दत्ता} \msCc, दत्ता \Ed}}% 
    \paral{{\devanagarifontsmall \vo {\englishfont \compare\ \SDHU\ 12.92:}
                         तृणानि खादन्ति वसन्त्यरण्ये पिबन्ति तोयान्यपरिग्रहाणि\thinspace{\devanagarifontsmall ।}
                         दुह्यन्ति वाह्यन्ति पुनन्ति पापं गवां रसैर्जीवति जीवलोकः\thinspace{\devanagarifontsmall ॥} }}

\ujvers\nemsloka {
{\devanagarifont गवाह्निकं यश्च करोति नित्यं }%
  \dontdisplaylinenum}    \var{{\devanagarifont \numemph\va\textbf{गवाह्निकं}\lem \mssALL, गवांह्निकं \msCa\oo 
\textbf{यश्च करोति}\lem \mssALL, यः प्रकरोति \Ed}}% 


\nemslokab

{\devanagarifont शुश्रूषणं यः कुरुते गवां तु  \danda\dontdisplaylinenum }%
     \var{{\devanagarifont \numnoemph\vb\textbf{गवां तु}\lem \msCb\msNc, गवान्तु \msCa\msCc\msNa\msNb, गवानाम् \Ed}}% 

\nemslokac

{\devanagarifont अशेषयज्ञतपदानपुण्यं }%
  \dontdisplaylinenum    \var{{\devanagarifont \numnoemph\vc\textbf{॰तप॰}\lem \mssALL, ॰\uncl{तप}॰ \msNb, ॰जप॰ \Ed}}% 

%Verse 4:39


\nemslokad

{\devanagarifont लभत्यसौ तामनृशंसकर्ता {॥ ४:३९॥} \veg\dontdisplaylinenum }%
     \var{{\devanagarifont \numnoemph\vd \lem \eme, 
लभत्यसौ तमनृशंसकर्ता \msCb\msNa\msNb\msNc, 
लभत्यसौ भमनृशंसकर्त्ता \msCa, 
लभत्यसौ तमनृतं स कर्त्ता \msCc, 
भवत्यसौ धर्ममशेषकर्ता \Ed}}% 

\vers


{\devanagarifont अतिथिं यो ऽनुगच्छेत अतिथिं यो ऽनुमन्यते \thinspace{\dandab} \dontdisplaylinenum }%
 
%Verse 4:40

{\devanagarifont अतिथिं यो ऽनुपूज्येत अतिथिं यः प्रशंसते {॥ ४:४०॥} \veg\dontdisplaylinenum }%
     \var{{\devanagarifont \numemph\vd\textbf{प्रशंसते}\lem \mssALL, प्रशंस्यते \msCc}}% 

{\devanagarifont अतिथिं यो न पीड्येत अतिथिं यो न दुष्यति \thinspace{\dandab} \dontdisplaylinenum }%
     \var{{\devanagarifont \numemph\va\textbf{न पीड्येत}\lem \msCa\msCb\msNa\Ed, न गच्छेत ({\englishfont eyeskip to 4.40c}) \msCc, 
\uncl{न पी}\lk\lk\ \msNb, निपीड्येत \msNc}}% 
    \var{{\devanagarifont \numnoemph\vb\textbf{अतिथिं}\lem \mssALL, अतिं \msCc, \lk\lk\lk\ \msNb\oo 
\textbf{न दुष्यति}\lem \mssALL, नुदुष्यति \msCb, \lk\ दुष्यति \msNb}}% 

{\devanagarifont अतिथिप्रियकर्ता यः अतिथेः परिचारकः  \danda\dontdisplaylinenum }%
     \var{{\devanagarifont \numnoemph\vc\textbf{अतिथि॰}\lem \msCa\msNa, अतिथिं \msCb\msCc\msNc\Ed, अति\uncl{थिं} \msNb\oo 
\textbf{॰प्रिय॰}\lem \mssALL, प्रियः \msCc\oo 
\textbf{यः}\lem \mssALL, यर् \msCa, य \msNa}}% 

%Verse 4:41

{\devanagarifont अतिथेः कृतसंतोषस्तस्य पुण्यमनन्तकम् {॥ ४:४१॥} \veg\dontdisplaylinenum }%
     \var{{\devanagarifont \numnoemph\ve\textbf{अतिथेः}\lem \msCb\msCc\msNc, अतिथि॰ \msCa\msNa\msNb, अतिथिं \Ed}}% 
    \var{{\devanagarifont \numnoemph\vef\textbf{॰संतोषस्तस्य}\lem \mssALL, ॰संता यस्य \msCb}}% 
    \var{{\devanagarifont \numnoemph\vf\textbf{पुण्य॰}\lem \mssALL, पून॰ \msNc}}% 

{\devanagarifont आसनेनार्घपात्रेण पादशौचजलेन च \thinspace{\dandab} \dontdisplaylinenum }%
     \var{{\devanagarifont \numemph\va\textbf{॰आर्घ॰}\lem \mssALL, ॰आर्ध्य॰ \Ed\oo 
\textbf{॰पात्रेण}\lem \conj, ॰पाद्येन \mssCaCbCc\msNa\msNb\msNc\Ed}}% 

%Verse 4:42

{\devanagarifont अन्नवस्त्रप्रदानैर्वा सर्वं वापि निवेदयेत् {॥ ४:४२॥} \veg\dontdisplaylinenum }%
     \var{{\devanagarifont \numnoemph\vc\textbf{अन्नव॰}\lem \mssALL, अन्नम्व॰ \msCc, \uncl{अन्न}व॰ \msNb}}% 
    \var{{\devanagarifont \numnoemph\vd\textbf{निवेदयेत्}\lem \mssALL, प्रदापयेत् \Ed}}% 

{\devanagarifont पुत्रदारात्मनो वापि यो ऽतिथिमनुपूजयेत् \thinspace{\dandab} \dontdisplaylinenum }%
     \var{{\devanagarifont \numemph\va\textbf{॰दारात्मनो}\lem \mssALL, ॰\uncl{दारा}त्मनो \msCa, ॰दारात्मको \Ed}}% 
    \var{{\devanagarifont \numnoemph\vb\textbf{॰पूजयेत्}\lem \msCa\msNa\Ed, ॰पूज्यते \msCb\msCc\msNb, ॰पूजते \msNc}}% 

%Verse 4:43

{\devanagarifont श्रद्धया चाविकल्पेन अक्लीबमानसेन च {॥ ४:४३॥} \veg\dontdisplaylinenum }%
     \var{{\devanagarifont \numnoemph\vc\textbf{श्रद्धया}\lem \mssALL, श्रद्धाया \msCc\oo 
\textbf{चाविकल्पेन}\lem \mssALL, चापि कल्पेन \msCa}}% 

{\devanagarifont न पृच्छेद्गोत्रचरणं स्वाध्यायं देशजन्मनी \thinspace{\dandab} \dontdisplaylinenum }%
     \var{{\devanagarifont \numemph\va\textbf{॰चरणं}\lem \mssALL, ॰प्रवरं \Ed}}% 
    \var{{\devanagarifont \numnoemph\vb\textbf{देशजन्मनी}\lem \mssALL, देशजन्मना \msCa}}% 
    \paral{{\devanagarifontsmall  {\englishfont \vab = \UUMS\ 10.7ab = \UMS\ 6.11ab \similar\ \MBH\ 13.62.18ab:
                 }न पृच्छेद्गोत्रचरणं स्वाध्यायं देशमेव वा }}

%Verse 4:44

{\devanagarifont चिन्तयेन्मनसा भक्त्या धर्मः स्वयमिहागतः {॥ ४:४४॥} \veg\dontdisplaylinenum }%
     \var{{\devanagarifont \numnoemph\vc\textbf{चिन्तयेन्म॰}\lem \mssALL, चित्तयेत्म॰ \msCb, चिन्तयेत्म॰ \msNc}}% 
    \var{{\devanagarifont \numnoemph\vd\textbf{॰गतः}\lem \mssALL, ॰गताः \msCc, ग\uncl{तम्} \msNb}}% 
    \paral{{\devanagarifontsmall \vcd {\englishfont \compare\ 12.37cd: }द्विजरूपधरो धर्मः स्वयमेव इहागतः }}

{\devanagarifont अश्वमेधसहस्राणि राजसूयशतानि च \thinspace{\dandab} \dontdisplaylinenum }%
     \var{{\devanagarifont \numemph\vb\textbf{॰सूय॰}\lem \msCa\msNa\msNc\Ed, ॰सूर्य॰ \msCb\msCc, ॰सू\uncl{र्य}॰ \msNb}}% 

%Verse 4:45

{\devanagarifont पुण्डरीकसहस्रं च सर्वतीर्थतपःफलम् {॥ ४:४५॥} \veg\dontdisplaylinenum }%
     \var{{\devanagarifont \numnoemph\vd\textbf{॰तपः॰}\lem \mssALL, ॰तप॰ \msNc\ \unmetr}}% 

{\devanagarifont अतिथिर्यस्य तुष्येत नृशंसमतमुत्सृजेत् \thinspace{\dandab} \dontdisplaylinenum }%
     \var{{\devanagarifont \numemph\vb \lem \msCa\msNa\msNc, नृशंसमत उत्सृजेत् \msCb, 
नृशंसकमममुत्सृजेत् \msCc, नृससमतमुत्सृजेत् \msNb, न संशय समश्नुते \Ed}}% 

%Verse 4:46

{\devanagarifont स तस्य सकलं पुण्यं प्राप्नुयान्नात्र संशयः {॥ ४:४६॥} \veg\dontdisplaylinenum }%
 
{\devanagarifont †न गतिमतिथिज्ञस्य† गतिमाप्नोति कर्हचित् \thinspace{\dandab} \dontdisplaylinenum }%
     \var{{\devanagarifont \numemph\va\textbf{न गतिम॰}\lem \msCa\msCb\msNb\msNc, न तिथिम॰ \msCc\Ed, न गति ना॰ \msNa}}% 
    \var{{\devanagarifont \numnoemph\vb\textbf{कर्हचित्}\lem \mssALL, कर्हिचित् \msCa\Ed}}% 

%Verse 4:47

{\devanagarifont तस्मादतिथिमायान्तमभिगच्छेत्कृताञ्जलिः {॥ ४:४७॥} \veg\dontdisplaylinenum }%
     \var{{\devanagarifont \numnoemph\vc\textbf{॰यान्त॰}\lem \mssALL, ॰यान्ति॰ \msCc}}% 
    \paral{{\devanagarifontsmall \vcd {\englishfont = \VAYUP\ 2.17.8 = \BRAHMANDAPUR\ 2.15.8 
                         \similar\ \SDHU\ 4.44ab:}
                         तस्मादतिथिमायान्तमनुगच्छेत्कृताञ्जलिः }}

{\devanagarifont सक्तुप्रस्थेन चैकेन यज्ञ आसीन्महाद्भुतः \thinspace{\dandab} \dontdisplaylinenum }%
     \var{{\devanagarifont \numemph\va\textbf{सक्तु॰}\lem \eme, शन्कु॰ \msCa\msCb, शंक्तु॰ \msCc, शक्तु॰ \msNa\msNc, शक्थु॰ \msNb, शक्ति॰ \Ed\oo 
\textbf{चैकेन}\lem \mssALL, चेकेन \msNc}}% 
    \var{{\devanagarifont \numnoemph\vb\textbf{आसीन्महाद्भुतः}\lem \corr, आसीन्महद्भुतः \msCa\msCb\msNa\msNb, आसी महद्भुतः \msCc, 
आसीत्महाद्भुतः \msNc, आसीन्महद्भुतम् \Ed}}% 

%Verse 4:48

{\devanagarifont अतिथिप्राप्तदानेन स्वशरीरं दिवं गतम् {॥ ४:४८॥} \veg\dontdisplaylinenum }%
     \var{{\devanagarifont \numnoemph\vc\textbf{॰दानेन}\lem \mssALL, ॰प्रादानेन \msCc}}% 
    \var{{\devanagarifont \numnoemph\vd\textbf{स्व॰}\lem \mssALL, \uncl{स}॰ \msNc, स॰ \Ed\oo 
\textbf{॰गतम्}\lem \mssALL, ॰गतः \msCc}}% 

{\devanagarifont नकुलेन पुराधीतं विस्तरेण द्विजोत्तम \thinspace{\dandab} \dontdisplaylinenum }%
     \var{{\devanagarifont \numemph\vb\textbf{॰त्तम}\lem \mssALL, ॰त्तमम् \msCc, ॰त्तमः \Ed}}% 

%Verse 4:49

{\devanagarifont विदितं च त्वया पूर्वं प्रस्थवार्त्ता च कीर्तिता {॥ ४:४९॥} \veg\dontdisplaylinenum }%
     \var{{\devanagarifont \numnoemph\vd\textbf{कीर्तिता}\lem \mssALL, कीर्तितम् \msCc, कीर्तिताः \Ed}}% 


\alalfejezet{यमेषु दमः (५)}
{\devanagarifont दम एव मनुष्याणां धर्मसारसमुच्चयः \thinspace{\dandab} \dontdisplaylinenum }%
     \var{{\devanagarifont \numemph\vb\textbf{धर्मसार॰}\lem \eme, धर्मः सार॰ \mssCaCbCc\msNa\msNb\msNc, धर्मभार॰ \Ed}}% 
    \paral{{\devanagarifontsmall \vb {\englishfont \compare, e.g., \MBH\ Suppl. 14.4.2477: }श्रोतुमिच्छामि कार्त्स्न्येन धर्मसारसमुच्चयम् }}

%Verse 4:50

{\devanagarifont दमो धर्मो दमः स्वर्गो दमः कीर्तिर्दमः सुखम् {॥ ४:५०॥} \veg\dontdisplaylinenum }%
     \var{{\devanagarifont \numnoemph\vc\textbf{स्वर्गो}\lem \mssALL, स्वर्ग \msCc}}% 
    \var{{\devanagarifont \numnoemph\vd\textbf{कीर्तिर्द॰}\lem \msCa\msCb\msNb\Ed, कीर्ति द॰ \msCc\msNa\msNc}}% 

{\devanagarifont दमो यज्ञो दमस्तीर्थं दमः पुण्यं दमस्तपः \thinspace{\dandab} \dontdisplaylinenum }%
     \var{{\devanagarifont \numemph\va\textbf{दमस्ती॰}\lem \mssALL, दम ती॰ \msCb}}% 

%Verse 4:51

{\devanagarifont दमहीनमधर्मश्च दमः कामकुलप्रदः {॥ ४:५१॥} \veg\dontdisplaylinenum }%
     \var{{\devanagarifont \numnoemph\vd\textbf{दमः}\lem \mssALL, दम \msCc, दमं \Ed\oo 
\textbf{काम॰}\lem \mssALL, कामं \msNc}}% 

{\devanagarifont निर्दमः करि मीनश्च पतङ्गभ्रमरमृगाः \thinspace{\dandab} \dontdisplaylinenum }%
     \var{{\devanagarifont \numemph\va\textbf{॰दमः}\lem \mssALL, ॰दम \msCc}}% 
    \var{{\devanagarifont \numnoemph\vb\textbf{॰भ्रमर॰}\lem \mssALL, ॰भ्रम\uncl{रा}॰ \msNc}}% 

%Verse 4:52

{\devanagarifont त्वग्जिह्वा च तथा घ्राणा चक्षुः श्रवणमिन्द्रियाः {॥ ४:५२॥} \veg\dontdisplaylinenum }%
     \var{{\devanagarifont \numnoemph\vc\textbf{घ्राणा}\lem \mssALL, घ्राणं \msCb, घ्राण \msCc}}% 
    \var{{\devanagarifont \numnoemph\vd\textbf{॰न्द्रियाः}\lem \mssALL, ॰न्द्रियः \Ed}}% 

{\devanagarifont दुर्जयेन्द्रियमेकैकं सर्वे प्राणहराः स्मृताः \thinspace{\dandab} \dontdisplaylinenum }%
     \var{{\devanagarifont \numemph\vb\textbf{सर्वे}\lem \mssALL, सर्व॰ \msCb\oo 
\textbf{॰हराः}\lem \mssALL, ॰हरा \Ed}}% 

%Verse 4:53

{\devanagarifont दमं यो जयते ऽसम्यग्निर्दमो निधनं व्रजेत् {॥ ४:५३॥} \veg\dontdisplaylinenum }%
     \var{{\devanagarifont \numnoemph\vd\textbf{व्रजेत्}\lem \mssALL, व्रजे\lac\  \msCa}}% 

{\devanagarifont मृगे श्रोत्रवशान्मृत्युः पतङ्गाश्चक्षुषोर्मृताः \thinspace{\dandab} \dontdisplaylinenum }%
     \var{{\devanagarifont \numemph\va\textbf{मृगे}\lem \mssALL, मृगो \msNb\Ed\oo 
\textbf{श्रोत्र॰}\lem \mssALL, शोत्र॰ \msCc, श्रोत॰ \msNc\oo 
\textbf{॰वशा॰}\lem \mssALL, ॰वचशा॰ \msCb}}% 
    \var{{\devanagarifont \numnoemph\vb\textbf{पतङ्गाश्च॰}\lem \mssALL, पतङ्गा च॰ \Ed\oo 
\textbf{॰षोर्मृताः}\lem \mssALL, ॰सो मृताः \msCc, ॰षो मृताः \msNc}}% 

%Verse 4:54

{\devanagarifont घ्राणया भ्रमरो नष्टो नष्टो मीनश्च जिह्वया {॥ ४:५४॥} \veg\dontdisplaylinenum }%
     \var{{\devanagarifont \numnoemph\vc\textbf{घ्राणया}\lem \mssALL, घ्रातया \msCb}}% 
    \var{{\devanagarifont \numnoemph\vcd\textbf{नष्टो नष्टो}\lem \mssALL, नष्टो \msCb}}% 
    \paral{{\devanagarifontsmall \vo {\englishfont \compare\ \BUDDHACARITA\ 11.35:} 
                गीतैर्ह्रियन्ते हि मृगा वधाय रूपार्थमग्नौ शलभाः पतन्ति\thinspace{\devanagarifontsmall ।} 
                मत्स्यो गिरत्यायसमामिषार्थी तस्मादनर्थं विषयाः फलन्ति\thinspace{\devanagarifontsmall ॥} }}

{\devanagarifont स्पर्शेन च करी नष्टो बन्धनावासदुःसहः \thinspace{\dandab} \dontdisplaylinenum }%
     \var{{\devanagarifont \numemph\vb\textbf{॰सदुःसहः}\lem \mssALL, ॰सदुःसह \msCb, ॰सुदुस्सहः \msNb}}% 

%Verse 4:55

{\devanagarifont किं पुनः पञ्चभुक्तानां मृत्युस्तेभ्यः किमद्भुतम् {॥ ४:५५॥} \veg\dontdisplaylinenum }%
     \var{{\devanagarifont \numnoemph\vc\textbf{पुनः}\lem \mssALL, पुन \msCaacorr}}% 
    \var{{\devanagarifont \numnoemph\vd\textbf{तेभ्यः}\lem \mssALL, तेभ्य \Ed}}% 

{\devanagarifont पुरूरवो ऽतिलोभेन अतिकामेन दण्डकः \thinspace{\dandab} \dontdisplaylinenum }%
     \var{{\devanagarifont \numemph\va\textbf{पुरूरवो}\lem \mssALL, पुरोरवे \msCc, पुरुरवा॰ \Ed}}% 
    \var{{\devanagarifont \numnoemph\vab\textbf{तिलोभेन अतिकामेन}\lem \mssALL, तिकामेन अतिलोभेन \Ed}}% 
    \var{{\devanagarifont \numnoemph\vb\textbf{दण्डकः}\lem \mssALL, पुण्डकः \Ed}}% 

%Verse 4:56

{\devanagarifont सागराश्चातिदर्पेण अतिमानेन रावणः {॥ ४:५६॥} \veg\dontdisplaylinenum }%
     \var{{\devanagarifont \numnoemph\vc\textbf{सागरा॰}\lem \eme, सगर॰ \msCa\msCb\msNa\msNb\msNc\Ed, सागर॰ \msCc}}% 
    \paral{{\devanagarifontsmall \vd {\englishfont \compare\ \MAHASUBHS\ 563cd:}
                         विनष्टो रावणो लौल्यादति सर्वत्र वर्जयेत् }}

{\devanagarifont अतिक्रोधेन सौदास अतिपानेन यादवाः \thinspace{\dandab} \dontdisplaylinenum }%
     \var{{\devanagarifont \numemph\vb\textbf{अतिपानेन}\lem \mssALL, अतिपापेन \Ed}}% 

%Verse 4:57

{\devanagarifont अतितृष्णाच्च मान्धाता नहुषो द्विजवज्ञया {॥ ४:५७॥} \veg\dontdisplaylinenum }%
     \var{{\devanagarifont \numnoemph\vc \lem \conj, 
अतितृष्णा च मान्दातो \msCa, 
अतितृष्णा च मान्धातो \msCb\msCc\msNa\msNc, 
अतितृष्णा च मन्धातो \msNb, 
अतितृष्णा च मानाच्च च \Ed}}% 
    \var{{\devanagarifont \numnoemph\vd\textbf{नहुषो}\lem \mssALL, नघुषो \msNb}}% 

{\devanagarifont अतिदानाद्बलिर्नष्ट अतिशौर्येण अर्जुनः \thinspace{\dandab} \dontdisplaylinenum }%
     \var{{\devanagarifont \numemph\va\textbf{॰र्नष्ट}\lem \mssALL, ॰र्नष्टो \msCb, नष्टो \msCc}}% 
    \paral{{\devanagarifontsmall \va {\englishfont \compare\ \MAHASUBHS\ 563ab:}
                         अतिदानाद्बलिर्बद्धो नष्टो मानात्सुयोधनः }}

%Verse 4:58

{\devanagarifont अतिद्यूतान्नलो राजा नृगो गोहरणेन तु {॥ ४:५८॥} \veg\dontdisplaylinenum }%
     \var{{\devanagarifont \numnoemph\vc\textbf{अतिद्यूतान्नलो}\lem \msCa\msCc\msNb\msNc, अतिद्यूतान्नरो \msCb\msNa, अतिख्यातान्नलो \Ed}}% 
    \var{{\devanagarifont \numnoemph\vd\textbf{नृगो गो॰}\lem \Ed, नृगङ्गो॰ \msCa\msCc\msNb\msNc, नृगं गो॰ \msCb\msNa}}% 
    \lacuna{\devanagarifontsmall \vo {\englishfont After this verse, \Ed\ adds:} 
                        तस्माद्दम सदा स रक्षेत् अति सर्वत्र वर्जयेत्   
                {\englishfont (understand:} तस्माद्दमं सदा रक्षेत् अति सर्वत्र वर्जयेत् {\englishfont )};
                {\englishfont \compare\ \MAHASUBHS\ 563cd:}
                        विनष्टो रावणो लौल्यादति सर्वत्र वर्जयेत्  }%
  
\ujvers\nemsloka {
{\devanagarifont दमेन हीनः पुरुषो द्विजेन्द्र }%
  \dontdisplaylinenum}    \var{{\devanagarifont \numemph\va \lem \mssALL, 
हीन पुरुषो द्विजेन्द्र \msNb, हीनं पुरुषं द्विजेन्द्रः \Ed}}% 


\nemslokab

{\devanagarifont स्वर्गं च मोक्षं च सुखं च नास्ति  \danda\dontdisplaylinenum }%
 
\nemslokac

{\devanagarifont विज्ञानधर्मकुलकीर्तिनाश }%
  \dontdisplaylinenum    \var{{\devanagarifont \numnoemph\vc\textbf{॰नाश}\lem \msCb, ॰नाशो \Ed ॰नाम \msCa\msCc\msNa, ॰नश्च \msNb, ॰नागा \msNc}}% 

%Verse 4:59


\nemslokad

{\devanagarifont भवन्ति विप्र दमया विहीनाः {॥ ४:५९॥} \veg\dontdisplaylinenum }%
     \var{{\devanagarifont \numnoemph\vd\textbf{विप्र}\lem \mssALL, विप्रा \msNapcorr\msNc\oo 
\textbf{दमया}\lem \mssALL, दया \msCbacorr}}% 


\alalfejezet{यमेषु घृणा (६)}
\vers


{\devanagarifont निर्घृणो न परत्रास्ति निर्घृणो न इहास्ति वै \thinspace{\dandab} \dontdisplaylinenum }%
     \var{{\devanagarifont \numemph\va\textbf{निर्घृणो}\lem \msCa\msCb\msNb, निघृणो \msCc\msNc, निर्घृण \msNaacorr, 
निर्घृ\uncl{णे} \msNapcorr, निर्घृणे \Ed}}% 
    \var{{\devanagarifont \numnoemph\vb\textbf{निर्घृणो}\lem \msCa\msCb\msNaacorr\msNb, निघृणो \msCc\msNc, निर्घृणे \msNapcorr\Ed}}% 

%Verse 4:60

{\devanagarifont निर्घृणे न च धर्मो ऽस्ति निर्घृणे न तपो ऽस्ति वै {॥ ४:६०॥} \veg\dontdisplaylinenum }%
     \var{{\devanagarifont \numnoemph\vc\textbf{निर्घृणे}\lem \msCa\msCb\msNb\Ed, निघृणे \msCc\msNa\msNc}}% 
    \var{{\devanagarifont \numnoemph\vd\textbf{निर्घृणे}\lem \mssALL, निघृणे \msCc\msNc}}% 

{\devanagarifont परस्त्रीषु परार्थेषु परजीवापकर्षणे \thinspace{\dandab} \dontdisplaylinenum }%
     \var{{\devanagarifont \numemph\vb\textbf{॰जीवापकर्षणे}\lem \mssALL, ॰जीवापर्कणे \msCb, ॰जीवोपकर्षणे \Ed}}% 

%Verse 4:61

{\devanagarifont परनिन्दापरान्नेषु घृणां पञ्चसु कारयेत् {॥ ४:६१॥} \veg\dontdisplaylinenum }%
     \var{{\devanagarifont \numnoemph\vc\textbf{परनिन्दा॰}\lem \mssALL, परनि$\-$न्द\lk\ \msCa\oo 
\textbf{॰परान्नेषु}\lem \mssALL, ॰परांनेषु \msNb}}% 
    \var{{\devanagarifont \numnoemph\vd\textbf{घृणां}\lem \msCa\msCb\msNa\msNc, घृणा \msCc\msNb\Ed}}% 

{\devanagarifont परस्त्री शृणु विप्रेन्द्र घृणीकार्या सदा बुधैः \thinspace{\dandab} \dontdisplaylinenum }%
     \var{{\devanagarifont \numemph\va\textbf{घृणी॰}\lem \mssALL, घृणा \msCb}}% 

%Verse 4:62

{\devanagarifont राज्ञी विप्री परिव्राजा स्वयोनिपरयोनिषु {॥ ४:६२॥} \veg\dontdisplaylinenum }%
     \var{{\devanagarifont \numnoemph\vc\textbf{॰व्राजा}\lem \mssCaCbCc\msNc, ॰व्राजी \msNa\msNb, ॰व्राज्या \Ed}}% 
    \var{{\devanagarifont \numnoemph\vd\textbf{॰पर॰}\lem \mssALL, ॰पशु॰ \msNb}}% 

{\devanagarifont परार्थे शृणु भूयो ऽन्य अन्यायार्थमुपार्जनम् \thinspace{\dandab} \dontdisplaylinenum }%
     \var{{\devanagarifont \numemph\vb\textbf{अन्याया॰}\lem \mssALL, अन्यया॰ \msNb\oo 
\textbf{॰र्जनम्}\lem \mssALL, ॰र्ज्जवम् \msNb}}% 
    \paral{{\devanagarifontsmall \vb {\englishfont \compare\ \BHG\ 16.12:}
                 आशापाशशतैर्बद्धाः कामक्रोधपरायणाः\thinspace{\devanagarifontsmall ।}
                 ईहन्ते कामभोगार्थमन्यायेनार्थसंचयान्\thinspace{\devanagarifontsmall ॥} }}

%Verse 4:63

{\devanagarifont आढप्रस्थतुलाव्याजैः परार्थं यो ऽपकर्षति {॥ ४:६३॥} \veg\dontdisplaylinenum }%
     \var{{\devanagarifont \numnoemph\vc\textbf{॰तुला॰}\lem \mssALL, ॰तुल॰ \msNb}}% 
    \var{{\devanagarifont \numnoemph\vd\textbf{॰र्थं}\lem \msCa\msCb\msNa\Ed, ॰र्थ \msCc, ॰\uncl{र्थ} \msNb, ॰र्थे \msNc}}% 

{\devanagarifont जीवापकर्षणे विप्र घृणीकुर्वीत पण्डितः \thinspace{\dandab} \dontdisplaylinenum }%
     \var{{\devanagarifont \numemph\va\textbf{विप्र}\lem \mssALL, वि\uncl{प्र} \msCa, विप्रे \msCc}}% 
    \var{{\devanagarifont \numnoemph\vb\textbf{घृणी॰}\lem \mssALL, घृणां \Ed}}% 

%Verse 4:64

{\devanagarifont वनजावनजा जीवा विलगाश्चरणाचराः {॥ ४:६४॥} \veg\dontdisplaylinenum }%
     \var{{\devanagarifont \numnoemph\vc\textbf{वनजावनजा}\lem \msCa\msCc\msNa\msNb\Ed, 
वनजाव\lk जा \msCbacorr, वनजा व\uncl{नि}जा \msCbpcorr, वनज विनजा \msNc}}% 
    \var{{\devanagarifont \numnoemph\vd \lem \corr, 
विलगाचरणाचराः \msCa\msCb\msNc, विलगोचरगोचरः \msCc\Ed, विलगोचरगोचराः \msNa, 
\uncl{विलगाचर}णाचराः \msNb}}% 

{\devanagarifont परनिन्दा च का विप्र शृणु वक्ष्ये समासतः \thinspace{\dandab} \dontdisplaylinenum }%
     \var{{\devanagarifont \numemph\vb\textbf{वक्ष्ये}\lem \mssALL, वक्ष्या \Ed}}% 

%Verse 4:65

{\devanagarifont देवानां ब्राह्मणानां च गुरुमातातिथिद्विषः {॥ ४:६५॥} \veg\dontdisplaylinenum }%
     \lacuna{\devanagarifontsmall \vcd {\englishfont These two pādas are illegible in \msNb} }%
  
{\devanagarifont परान्नेषु घृणा कार्या अभोज्येषु च भोजनम् \thinspace{\dandab} \dontdisplaylinenum }%
     \var{{\devanagarifont \numemph\vb\textbf{अभोज्येषु}\lem \mssALL, अभोज्ये \msCb}}% 

%Verse 4:66

{\devanagarifont सूतके मृतके शौण्डे वर्णभ्रष्टकुले नटे {॥ ४:६६॥} \veg\dontdisplaylinenum }%
     \var{{\devanagarifont \numnoemph\vc\textbf{शौण्डे}\lem \msNa, सौण्ड्ये \msCa\msCc\msNc, शोण्ड्ये \msCb, \uncl{सौण्डे} \msNb, सौण्डो \Ed}}% 
    \lacuna{\devanagarifontsmall \vo {\englishfont This verse is mostly illegible in \msNb} }%
  
\nemslokalong


\ujvers\nemsloka {
{\devanagarifont एते पञ्चघृणासु सक्तपुरुषाः स्वर्गार्थमोक्षार्थिनो }%
  \dontdisplaylinenum}    \var{{\devanagarifont \numemph\va\textbf{॰पुरुषाः}\lem \msNc, ॰पुरुषः \mssCaCbCc\msNa\msNb\Ed\oo 
\textbf{॰र्थिनो}\lem \eme, ॰र्थिनः \msNcpcorr, ॰र्थिनां \mssCaCbCc\msNa\msNb\Ed, ॰र्थिना \msNcacorr}}% 


\nemslokab

{\devanagarifont लोके ऽनिन्दनमाप्नुवन्ति सततं कीर्तिर्यशोऽलंकृतम्  \danda\dontdisplaylinenum }%
     \var{{\devanagarifont \numnoemph\vb\textbf{ऽनिन्दनमाप्नुवन्ति}\lem \mssALL, 
ऽनिन्दनवाप्नुवन्ति \msCc, नन्दनवायुवान्ति \Ed}}% 

\nemslokac

{\devanagarifont प्रज्ञाबोधश्रुतिं स्मृतिं च लभते मानं च नित्यं लभेद् }%
  \dontdisplaylinenum    \var{{\devanagarifont \numnoemph\vc\textbf{॰श्रुतिं}\lem \msNc, ॰श्रुति॰ \mssCaCbCc\msNa\msNb\Ed\oo 
\textbf{नित्यं}\lem \mssALL, नित्य \msCb}}% 

%Verse 4:67


\nemslokad

{\devanagarifont दाक्षिण्यं सभवेत्स आयुष परं प्राप्नोति निःसंशयः {॥ ४:६७॥} \veg\dontdisplaylinenum }%
     \var{{\devanagarifont \numnoemph\vd\textbf{स आयुष}\lem \eme, समायुष \mssCaCbCc\msNc, समायुषः \msNa\ \unmetr, 
\uncl{समायुष} \msNb, स मानुष \Ed\oo 
\textbf{निःसंशयः}\lem \mssALL, निसंशयः \msNa}}% 

\nemslokanormal



\alalfejezet{यमेषु पञ्चविधो धन्यः (७)}
\vers


{\devanagarifont चतुर्मौनं चतुःशत्रुश्चतुरायतनं तथा \thinspace{\dandab} \dontdisplaylinenum }%
     \var{{\devanagarifont \numemph\va\textbf{चतुर्मौनं च॰}\lem \corr, चतुर्मौनश्च॰ \msCa\msCb\msNa\msNc\Ed, चतुर्मोणश्च॰ \msCc, 
\uncl{चतुर्मौनश्च}॰ \msNb}}% 
    \var{{\devanagarifont \numnoemph\vab\textbf{॰तुःशत्रुश्च॰}\lem \mssALL, 
॰तुशत्रु च॰ \msCc, ॰तुःशत्रु च॰ \Ed}}% 
    \var{{\devanagarifont \numnoemph\vb\textbf{॰तुरायतनं}\lem \mssALL, ॰\uncl{तु}रायतनं \msCa, 
॰\uncl{तुरायतनम्} \msNb}}% 

%Verse 4:68

{\devanagarifont चतुर्ध्यानं चतुष्पादं पञ्चधन्यविधोच्यते {॥ ४:६८॥} \veg\dontdisplaylinenum }%
     \var{{\devanagarifont \numnoemph\vc\textbf{॰पादं}\lem \mssALL, ॰पादः \msNa, \lk\lk\ \msNb}}% 
    \var{{\devanagarifont \numnoemph\vd\textbf{पञ्चधन्य॰}\lem \mssALL, धन्यपञ्च॰ \Ed}}% 

{\devanagarifont चतुर्मौनस्य वक्ष्यामि शृणुष्वावहितो भव \thinspace{\dandab} \dontdisplaylinenum }%
     \var{{\devanagarifont \numemph\va\textbf{॰मौनस्य}\lem \mssALL, ॰मोनस्य \msCb}}% 

%Verse 4:69

{\devanagarifont पारुष्यपिशुनामिथ्या सम्भिन्नानि च वर्जयेत् {॥ ४:६९॥} \veg\dontdisplaylinenum }%
     \var{{\devanagarifont \numnoemph\vc\textbf{पारुष्य॰}\lem \mssALL, पारुष्यं \msNa\oo 
\textbf{॰पिशुना॰}\lem \mssALL, ॰पिण्डाना॰ \Ed}}% 
    \paral{{\devanagarifontsmall \vcd {\englishfont \compare\ \DIVYAV\ 186.21:}
                     आर्य, किमेभिः कर्म कृतम्येनैवंविधानि दुःखानि प्रत्यनुभवन्तीति? 
                     स कथयति\thinspace{\devanagarifontsmall ।} एते प्राणातिपातिका अदत्तादायिकाः काममिथ्याचारिका मृषावादिकाः पैशुनिकाः पारुषिकाः 
                     संभिन्नप्रलापिका अभिध्यालवो व्यापन्नचित्ता मिथ्यादृष्टिकाः\thinspace{\devanagarifontsmall ।};
                     {\englishfont \compare\ \DHARMP\ 1.31cd--32ab:}
                         मिथ्या पिशुनसम्भिन्नपारुष्यवचनानि च\thinspace{\devanagarifontsmall ॥}
                         जल्पतः सम्भवन्त्येते तस्मान्मौनं प्रशस्यते\thinspace{\devanagarifontsmall ।} }}

{\devanagarifont कामः क्रोधश्च लोभश्च मोहश्चैव चतुर्विधः \thinspace{\dandab} \dontdisplaylinenum  }%
 
%Verse 4:70

{\devanagarifont चतुःशत्रुर्निहन्तव्यः सो ऽरिहा वीतकल्मषः {॥ ४:७०॥} \veg\dontdisplaylinenum }%
     \var{{\devanagarifont \numemph\vc\textbf{चतुःशत्रुर्नि॰}\lem \msCa\msCb\Ed, चतुशत्रु नि॰ \msCc\msNa\msNb\msNc}}% 
    \var{{\devanagarifont \numnoemph\vd\textbf{सो ऽरिहा}\lem \mssALL, स्रोरिहा \msCb, सर्वथा \Ed}}% 

{\devanagarifont चतुरायतनं विप्र कथयिष्यामि तच्छृणु \thinspace{\dandab} \dontdisplaylinenum }%
 
%Verse 4:71

{\devanagarifont करुणा मुदितोपेक्षा मैत्री चायतनं स्मृतम् {॥ ४:७१॥} \veg\dontdisplaylinenum }%
     \var{{\devanagarifont \numemph\vc\textbf{मुदितो॰}\lem \mssALL, मुदितौ॰ \Ed}}% 
    \var{{\devanagarifont \numnoemph\vd\textbf{चायतनं}\lem \mssALL, चायतन \msCa, चायत\uncl{न} \msCb}}% 

{\devanagarifont चतुर्ध्यानाधुना वक्ष्ये संसारार्णवतारणम् \thinspace{\dandab} \dontdisplaylinenum }%
 
%Verse 4:72

{\devanagarifont आत्मविद्याभवः सूक्ष्मं ध्यानमुक्तं चतुर्विधम् {॥ ४:७२॥} \veg\dontdisplaylinenum }%
     \var{{\devanagarifont \numemph\vc\textbf{॰भवः}\lem \msCb\msCcpcorr\msNa\msNb\msNc, ॰भव \msCa\msCcacorr, ॰भवं \Ed}}% 
    \var{{\devanagarifont \numnoemph\vcd\textbf{सूक्ष्मं ध्या॰}\lem \msCa\msNa\msNc\Ed, 
सूक्ष्मा\uncl{न्या}॰ \msCb, 
सू\uncl{क्ष्म}ध्या॰ \msCc, सूक्ष्मध्यान॰ \msNb}}% 
    \var{{\devanagarifont \numnoemph\vd\textbf{॰नमुक्तं चतुर्विधम्}\lem \msCc\msNb, ॰नमुक्तश्चतुर्विधम् \msCa, 
॰नमुक्तश्चतुर्विधः \msCb\msNa, 
॰नमुक्तं चतुर्विधिं \msNc, ॰नयज्ञश्च \Ed}}% 

{\devanagarifont आत्मतत्त्वः स्मृतो धर्मो विद्या पञ्चसु पञ्चधा \thinspace{\dandab} \dontdisplaylinenum }%
     \var{{\devanagarifont \numemph\va\textbf{स्मृतो}\lem \mssALL, स्मृता \msCc\Ed\oo 
\textbf{धर्मो}\lem \mssALL, धन्या \Ed}}% 

%Verse 4:73

{\devanagarifont षट्त्रिंशाक्षरमित्याहुः सूक्ष्मतत्त्वमलक्षणम् {॥ ४:७३॥} \veg\dontdisplaylinenum }%
     \var{{\devanagarifont \numnoemph\vcd\textbf{आहुः सू॰}\lem \mssALL, आ\lk\lk\ \msCa}}% 

{\devanagarifont चतुष्पादः स्मृतो धर्मश्चतुराश्रममाश्रितः \thinspace{\dandab} \dontdisplaylinenum }%
     \var{{\devanagarifont \numemph\vab\textbf{धर्मश्च॰}\lem \mssALL, धर्म च॰ \msCc\msNb}}% 
    \var{{\devanagarifont \numnoemph\vb\textbf{॰श्रितः}\lem \mssALL, ॰श्रिताः \msNc}}% 

%Verse 4:74

{\devanagarifont गृहस्थो ब्रह्मचारी च वानप्रस्थो ऽथ भैक्षुकः {॥ ४:७४॥} \veg\dontdisplaylinenum }%
     \var{{\devanagarifont \numnoemph\vd\textbf{भैक्षुकः}\lem \mssALL, भक्षकः \Ed}}% 
    \paral{{\devanagarifontsmall \vcd {\englishfont  = \MBH\ 12.234.13ab \similar\ \MBH\ 14.4513ab etc. }
                 \vo {\englishfont \compare\ 3.4 above:}
                 श्रुतिस्मृतिद्वयोर्मूर्तिश्चतुष्पादवृषः स्थितः\thinspace{\devanagarifontsmall ।}
                 चतुराश्रम यो धर्मः कीर्तितानि मनीषिभिः\thinspace{\devanagarifontsmall ॥} }}

{\devanagarifont धन्यास्ते यैरिदं वेत्ति निखिलेन द्विजोत्तम \thinspace{\dandab} \dontdisplaylinenum }%
     \var{{\devanagarifont \numemph\va\textbf{यैरिदं}\lem \mssALL, येरिदं \msCb\msCc\oo 
\textbf{वेत्ति}\lem \mssALL, वेति \msCc}}% 

%Verse 4:75

{\devanagarifont पावनं सर्वपापानां पुण्यानां च प्रवर्धनम् {॥ ४:७५॥} \veg\dontdisplaylinenum }%
     \var{{\devanagarifont \numnoemph\vd\textbf{प्रवर्धनम्}\lem \mssALL, प्रवर्धनः \Ed}}% 

{\devanagarifont आयुः कीर्तिर्यशः सौख्यं धन्यादेव प्रवर्धते \thinspace{\dandab} \dontdisplaylinenum }%
     \var{{\devanagarifont \numemph\vb\textbf{धन्यादेव}\lem \mssALL, धर्मादेव \Ed}}% 

%Verse 4:76

{\devanagarifont शान्तिः पुष्टिः स्मृतिर्मेधा जायते धन्यमानवे {॥ ४:७६॥} \veg\dontdisplaylinenum }%
     \var{{\devanagarifont \numnoemph\vc\textbf{पुष्टिः}\lem \mssALL, \lk ष्टिः \msCa\oo 
\textbf{स्मृतिर्मेधा}\lem \\mssALL, स्मृति मेधा \msCc\msNa}}% 
    \var{{\devanagarifont \numnoemph\vd\textbf{॰मानवे}\lem \eme, ॰मानवः \mssCaCbCc\msNa\msNb\msNc\Ed}}% 


\alalfejezet{यमेष्वप्रमादः (८)}
{\devanagarifont प्रमादस्थान पञ्चैव कीर्तयिष्यामि तच्छृणु \thinspace{\dandab} \dontdisplaylinenum }%
     \var{{\devanagarifont \numemph\va\textbf{॰स्थान}\lem \msCa\msCc\msNa\msNb, ॰स्थानं \msCb\msNc\Ed\ \unmetr\oo 
\textbf{पञ्चैव}\lem \mssALL, पञ्चैवं \Ed}}% 
    \var{{\devanagarifont \numnoemph\vb\textbf{कीर्तयिष्यामि}\lem \mssALL, कीर्तियिष्यामि \msNb}}% 

{\devanagarifont ब्रह्महत्या सुरापानं स्तेयो गुर्वङ्गनागमम्  \danda\dontdisplaylinenum }%
     \paral{{\devanagarifontsmall \vcdef {\englishfont \similar\ \MBH\ Suppl. 12.30:}
                     ब्रह्महत्यां सुरापानं स्तेयं गुर्वङ्गनागमम्\thinspace{\devanagarifontsmall ।}
                     महान्ति पातकान्याहुः संयोगं चैव तैः सह\thinspace{\devanagarifontsmall ॥}
                     {\englishfont  \similar\ \MANU\ 11.55 (in Olivelle's edition):}
                     ब्रह्महत्या सुरापानं स्तेयं गुर्वङ्गनागमः\thinspace{\devanagarifontsmall ।}
                     महान्ति पातकान्याहुः संसर्गश्चापि तैः सह\thinspace{\devanagarifontsmall ॥}
                 {\englishfont \compare\ also \YAJNS\ 3.228:}
                         ब्रह्महा मद्यपः स्तेनस्तथैव गुरुतल्पगः\thinspace{\devanagarifontsmall ।}
                         एते महापातकिनो यश्च तैः सह संवसेत्\thinspace{\devanagarifontsmall ॥}  }}

%Verse 4:77

{\devanagarifont महापातकमित्याहुस्तत्संयोगी च पञ्चमः {॥ ४:७७॥} \veg\dontdisplaylinenum }%
 
{\devanagarifont अनृतं च समुत्कर्षे राजगामी च पैशुनः \thinspace{\dandab} \dontdisplaylinenum }%
     \var{{\devanagarifont \numemph\va\textbf{समुत्कर्षे}\lem \eme, समुत्कर्षं \msCa\msNa, 
समुत्क\uncl{र्ष} \msCb, 
समुत्कर्ष \msCc\msNb\msNc\Ed}}% 
    \var{{\devanagarifont \numnoemph\vb\textbf{राज॰}\lem \mssALL, राज्ञी॰ \Ed}}% 

%Verse 4:78

{\devanagarifont गुरोश्चालीकनिर्बन्धः समानि ब्रह्महत्यया {॥ ४:७८॥} \veg\dontdisplaylinenum }%
     \var{{\devanagarifont \numnoemph\vc\textbf{॰निर्बन्धः}\lem \eme, ॰निर्बद्धः \msCb\msNc, निबद्धस् \msCa\msCc\msNa\msNb, निर्वद्धस् \Ed}}% 
    \var{{\devanagarifont \numnoemph\vd\textbf{ब्रह्महत्यया}\lem \mssALL, ब्र\lk\lk \lk या \msCa}}% 
    \paral{{\devanagarifontsmall \vo \similar\ {\englishfont \MBH\ 5.40.3 and \MANU\ 11.56:}
                  अनृतं च समुत्कर्षे राजगामि च पैशुनम्\thinspace{\devanagarifontsmall ।}
                  गुरोश्चालीकनिर्बन्धः समानि ब्रह्महत्यया\thinspace{\devanagarifontsmall ॥}
                 {\englishfont \similar\ \VISNUS\ 37.1--4 \similar\ \AGNIP\ 168.25} }}

{\devanagarifont ब्रह्मोज्झं वेदनिन्दा च कूटसाक्षी सुहृद्वधः \thinspace{\dandab} \dontdisplaylinenum }%
     \var{{\devanagarifont \numemph\va\textbf{ब्रह्मोज्झं}\lem \eme, ब्रह्मो ऋग्॰ \mssCaCbCc\msNa\msNb\msNc, ब्रह्म ऋग्॰ \Ed}}% 
    \var{{\devanagarifont \numnoemph\vb\textbf{सुहृद्वधः}\lem \mssALL, सकृद्बुधः \Ed}}% 

%Verse 4:79

{\devanagarifont गर्हितानाद्ययोर्जग्धिः सुरापानसमानि षट् {॥ ४:७९॥} \veg\dontdisplaylinenum }%
     \var{{\devanagarifont \numnoemph\vc\textbf{॰नाद्ययोर्जग्धिः}\lem \eme, ॰न्नञ्च यो जग्धिस् \msCa, ॰न्नञ्च यो जग्धि \msCb, 
॰न्नञ्च योद्विग्नः \msCc, ॰न्नं च यो जग्धिः \msNa, ॰न्नं च यो जग्धिः \msNb, 
॰न्नञ्च यो जवे \msNc, ॰न्नश्च यो विप्रः \Ed}}% 
    \paral{{\devanagarifontsmall \vo \similar\ {\englishfont \MANU\ 11.57:}
                         ब्रह्मोज्झता वेदनिन्दा कौटसाक्ष्यं सुहृद्वधः\thinspace{\devanagarifontsmall ।}
                         गर्हितानाद्ययोर्जग्धिः सुरापानसमानि षट्\thinspace{\devanagarifontsmall ॥}
                 {\englishfont \compare\ \YAJNS\ 3.229:}
                         गुरूणामध्यधिक्षेपो वेदनिन्दा सुहृद्वधः\thinspace{\devanagarifontsmall ।}
                         ब्रह्महत्यासमं ज्ञेयमधीतस्य च नाशनम्\thinspace{\devanagarifontsmall ॥} }}

{\devanagarifont रेतोत्सेकः स्वयोन्यासु कुमारीष्वन्त्यजासु च \thinspace{\dandab} \dontdisplaylinenum }%
     \var{{\devanagarifont \numemph\va\textbf{स्वयोन्यासु}\lem \mssALL, सुतोन्यासु \msCb}}% 

%Verse 4:80

{\devanagarifont सख्युः पुत्रस्य च स्त्रीषु गुरुतल्पसमः स्मृतः {॥ ४:८०॥} \veg\dontdisplaylinenum }%
     \var{{\devanagarifont \numnoemph\vc\textbf{सख्युः}\lem \eme, सख्य \mssCaCbCc\msNa\Ed, \lk\lk\ \msNb, स\uncl{ख्यु} \msNc\oo 
\textbf{पुत्रस्य च स्त्रीषु}\lem \mssALL, 
\lk\lk\lk\lk\lk\lk\ \msNb, पुत्रीषु चास्त्रीषु \Ed}}% 
    \var{{\devanagarifont \numnoemph\vd\textbf{॰समः}\lem \mssALL, \lk\lk\ \msNb, ॰सम \Ed}}% 
    \paral{{\devanagarifontsmall \vo \similar\ {\englishfont \MANU\ 11.59:}
                                 रेतःसेकः स्वयोनीषु कुमारीष्वन्त्यजासु च\thinspace{\devanagarifontsmall ।}
                                 सख्युः पुत्रस्य च स्त्रीषु गुरुतल्पसमं विदुः\thinspace{\devanagarifontsmall ॥} }}

{\devanagarifont निक्षेपस्यापहरणं नराश्वरजतस्य च \thinspace{\dandab} \dontdisplaylinenum }%
     \var{{\devanagarifont \numemph\va\textbf{निक्षेप॰}\lem \mssALL, निखेप॰ \msCb, \uncl{निक्षेप}॰ \msNb}}% 
    \var{{\devanagarifont \numnoemph\vb\textbf{नराश्वरजतस्य}\lem \mssALL, नराणां स्वजनस्य \msCb, 
\uncl{नराश्वरजतस्य} \msNb}}% 

%Verse 4:81

{\devanagarifont भूमिवज्रमणीनां च रुक्मस्तेयसमः स्मृतः {॥ ४:८१॥} \veg\dontdisplaylinenum }%
     \var{{\devanagarifont \numnoemph\vd\textbf{रुक्मस्तेय॰}\lem \eme, \uncl{रूग्य}\lk य॰ \msCa, 
रुग्मस्तेय॰ \msCb\msCc\msNa\msNc, \lk\lk \lk\lk\ \msNb, हृतस्तेय॰ \Ed\oo 
\textbf{॰समः}\lem \mssALL, सः \msCbacorr, ॰सम \Ed}}% 
    \paral{{\devanagarifontsmall \vo {\englishfont = \MANU\ 11.58 } }}

{\devanagarifont चत्वार एते सम्भूय यत्पापं कुरुते नरः \thinspace{\dandab} \dontdisplaylinenum }%
     \var{{\devanagarifont \numemph\va\textbf{एते}\lem \mssALL, \uncl{एते} \msNb, एव \Ed\oo 
\textbf{सम्भूय}\lem \mssALL, संभूयो \msCc, \uncl{संभूयो} \msNb}}% 

{\devanagarifont महापातक पञ्चैतत् तेन सर्वं प्रकाशितम्  \danda\dontdisplaylinenum }%
     \var{{\devanagarifont \numnoemph\vc\textbf{॰पञ्चैतत्}\lem \corr, ॰पञ्चैतन् \mssCaCbCc\Ed, ॰पञ्चैते \msNa, 
॰पञ्चैतम् \msNb, ॰पञ्चेतन् \msNc}}% 

%Verse 4:82

{\devanagarifont पञ्चप्रमादमेतानि वर्जनीयं द्विजोत्तम {॥ ४:८२॥} \veg\dontdisplaylinenum }%
     \var{{\devanagarifont \numnoemph\ve\textbf{॰मादम्}\lem \mssALL, ॰माद \Ed}}% 
    \var{{\devanagarifont \numnoemph\vf\textbf{वर्जनीयं}\lem \mssALL, वर्जनीयो \msCc}}% 


\alalfejezet{यमेषु माधुर्यम् (९)}
{\devanagarifont कायवाङ्मनमाधुर्यश्चक्षुर्बुद्धिश्च पञ्चमः \thinspace{\dandab} \dontdisplaylinenum }%
     \var{{\devanagarifont \numemph\vab\textbf{मनमाधुर्यश्च॰}\lem \eme, ॰मनसा धूर्यश्च॰ \msCa\msCc\msNa\msNc, 
॰मन\uncl{मा}धूर्यश्च॰ \msCb, 
॰मन\lk धूर्य\lk ॰ \msNb, ॰मनसा भूयश्च॰ \Ed}}% 
    \var{{\devanagarifont \numnoemph\vb\textbf{॰क्षुर्बुद्धि॰}\lem \msCa\msCb\msNc\Ed, ॰क्षु बुद्धि॰ \msCc\msNa, \lk\lk \lk\  \msNb}}% 

%Verse 4:83

{\devanagarifont सौम्यदृष्टिप्रदानं च क्रूरबुद्धिं च वर्जयेत् {॥ ४:८३॥} \veg\dontdisplaylinenum }%
     \var{{\devanagarifont \numnoemph\vc\textbf{॰दानं च}\lem \mssALL, \lk\lk\ \msNb, ॰दानश्च \Ed}}% 
    \var{{\devanagarifont \numnoemph\vd\textbf{॰बुद्धिं च}\lem \msCa\msNa\msNc, बुद्धिश्च \msCb, ॰दृष्टिं च \msCc\Ed, \lk\lk \lk\ \msNb}}% 

{\devanagarifont प्रसन्नमनसा ध्यायेत्प्रियवाक्यमुदीरयेत् \thinspace{\dandab} \dontdisplaylinenum }%
     \var{{\devanagarifont \numemph\va\textbf{प्रसन्न॰}\lem \mssALL, \uncl{प्रसन्न}॰ \msNb, प्रसंन॰ \msNc}}% 

%Verse 4:84

{\devanagarifont यथाशक्तिप्रदानं च स्वाश्रमाभ्यागतो गुरुः {॥ ४:८४॥} \veg\dontdisplaylinenum }%
     \var{{\devanagarifont \numnoemph\vc\textbf{यथा॰}\lem \mssALL, यस्य \Ed\oo 
\textbf{॰दानं}\lem \mssALL, ॰दातश् \Ed}}% 
    \var{{\devanagarifont \numnoemph\vd\textbf{स्वाश्रमा॰}\lem \mssALL, स्वासमा॰ \msCc\oo 
\textbf{॰गतो}\lem \mssALL, ॰सतो \msNc}}% 

{\devanagarifont इन्धनोदकदानं च जातवेदमथापि वा \thinspace{\dandab} \dontdisplaylinenum }%
     \var{{\devanagarifont \numemph\vb\textbf{इन्धनो॰}\lem \mssALL, इत्वनो॰ \msNc\oo 
\textbf{जात॰}\lem \mssALL, जा॰ \msCb}}% 

{\devanagarifont सुलभानि न दत्तानि इन्धनाग्न्युदकानि च  \danda\dontdisplaylinenum }%
     \var{{\devanagarifont \numnoemph\vc\textbf{सुलभानि न}\lem \mssALL, सुरभानि च \Ed}}% 
    \var{{\devanagarifont \numnoemph\vd\textbf{॰दकानि}\lem \mssALL, ॰\uncl{त}कानि \msNb}}% 

%Verse 4:85

{\devanagarifont क्षुते जीवेति वा नोक्तं तस्य किं परतः फलम् {॥ ४:८५॥} \veg\dontdisplaylinenum }%
     \var{{\devanagarifont \numnoemph\ve\textbf{क्षुते}\lem \conj, क्षुतं \mssCaCbCc\msNa\msNb\msNc, शतं \Ed}}% 


\alalfejezet{यमेष्वार्जवम् (१०)}
{\devanagarifont पञ्चार्जवाः प्रशंसन्ति मुनयस्तत्त्वदर्शिनः \thinspace{\dandab} \dontdisplaylinenum }%
     \var{{\devanagarifont \numemph\va\textbf{पञ्चार्जवाः}\lem \msCa\msCb\msNa\msNc, पञ्चार्जवः \msCc, \lk\lk \lk\lk\ \msNb, पञ्चार्जवा \Ed\oo 
\textbf{प्रशंसन्ति}\lem \mssCaCbCc\msNc, प्रशसन्ति \msNa\Ed, \uncl{प्रससन्ति} \msNb}}% 

{\devanagarifont कर्मवृत्त्याभिवृद्धिं च पारितोषिकमेव च  \danda\dontdisplaylinenum }%
     \var{{\devanagarifont \numnoemph\vc\textbf{कर्म॰}\lem \mssALL, \lk र्म्म॰ \msCa, \uncl{कम्मा}॰ \msNb\oo 
\textbf{॰वृत्त्याभिवृद्धिं च}\lem \mssALL, 
॰वृत्तिभिवृद्धिञ्च \msNb, ॰वृत्याभिवृद्धिश्च \Ed}}% 
    \var{{\devanagarifont \numnoemph\vd\textbf{पारितोषिक॰}\lem \eme, पारतोषिक॰ \mssCaCbCc\msNa\msNb\msNc\Ed}}% 

%Verse 4:86

{\devanagarifont स्त्रीधनोत्कोचवित्तं च आर्जवो नाभिनन्दति {॥ ४:८६॥} \veg\dontdisplaylinenum }%
     \var{{\devanagarifont \numnoemph\ve\textbf{स्त्रीधनोत्कोच॰}\lem \mssALL, स्त्रीधनङ्गो च \Ed\oo 
\textbf{॰वित्तं च}\lem \mssALL, ॰वित्तिञ्च \msNb}}% 
    \var{{\devanagarifont \numnoemph\vf\textbf{आर्जवो ना॰}\lem \mssALL, आर्जवञ्च \msCc, आर्ज्जवेना॰ \Ed}}% 

{\devanagarifont आर्जवो न वृथा यज्ञ आर्जवो न वृथा तपः \thinspace{\dandab} \dontdisplaylinenum }%
     \var{{\devanagarifont \numemph\vab \lem \mssCaCbCc\msNb\msNc, \om\ \msNaacorr, 
आर्जवो न वृथा यज्ञ आर्जवो न वृथा तप \msNapcorr, 
आर्जवो न वृथा यज्ञश्चार्र्जवो न वृथा तपः \Ed}}% 

%Verse 4:87

{\devanagarifont आर्जवो न वृथा दानमार्जवो न वृथाग्नयः {॥ ४:८७॥} \veg\dontdisplaylinenum }%
     \var{{\devanagarifont \numnoemph\vcd\textbf{(आर्जवो{\englishfont ...} वृथाग्नयः)}\lem \mssALL, \om\ \Ed}}% 

{\devanagarifont आर्जवस्येन्द्रियग्रामः सुप्रसन्नो ऽपि तिष्ठति \thinspace{\dandab} \dontdisplaylinenum }%
     \var{{\devanagarifont \numemph\vab\textbf{(आर्जव॰{\englishfont ...} तिष्ठति)}\lem \mssALL, \om\ \Ed}}% 
    \var{{\devanagarifont \numnoemph\va\textbf{॰ग्रामः}\lem \msCa\msCb\msNc\Ed, ॰ग्रामात् \msCc\msNb, ॰ग्रामाः \msNa}}% 

%Verse 4:88

{\devanagarifont आर्जवस्य सदा देवाः काये तस्य चरन्ति ते {॥ ४:८८॥} \veg\dontdisplaylinenum }%
     \var{{\devanagarifont \numnoemph\vd\textbf{तस्य चरन्ति}\lem \mssALL, त\lk\lac  न्ति \msCa, तस्य रमन्ति \Ed}}% 

\ujvers\nemsloka {
{\devanagarifont इति यमप्रविभागः कीर्तितो ऽयं द्विजेन्द्र }%
  \dontdisplaylinenum}    \var{{\devanagarifont \numemph\va\textbf{यमप्रविभागः}\lem \msCa\msCb\msNb\msNc, यमविभागः \msCc, 
यमप्ररिभागः \msNa, नियमपरिभागः \Ed\oo 
\textbf{द्विजेन्द्र}\lem \mssALL, नरेन्द्र \Ed}}% 


\nemslokab

{\devanagarifont इह परत सुखार्थं कारयेत्तं मनुष्यः  \danda\dontdisplaylinenum }%
     \var{{\devanagarifont \numnoemph\vb\textbf{॰येत्तं मनुष्यः}\lem \corr, ॰येत्तन्मनुष्यः \msCa\msNa\msNb\msNc\Ed, ॰येत्त मनुष्यः \msCb, 
॰येत्तत्मनुष्यः \msCc}}% 

\nemslokac

{\devanagarifont दुरितमलपहारी शङ्करस्याज्ञयास्ते }%
  \dontdisplaylinenum    \var{{\devanagarifont \numnoemph\vc\textbf{दुरित॰}\lem \mssALL, इरित॰ \Ed\oo 
\textbf{॰पहारी}\lem \mssALL, ॰पलपहारी \msCc\oo 
\textbf{॰ज्ञयास्ते}\lem \mssALL, ॰ज्ञयाते \msNa}}% 

%Verse 4:89


\nemslokad

{\devanagarifont भवति पृथिविभर्ता ह्येकछत्रप्रवर्ता {॥ ४:८९॥} \veg\dontdisplaylinenum }%
     \var{{\devanagarifont \numnoemph\vd\textbf{॰वर्ता}\lem \conj, ॰वृत्ता \mssCaCbCc\msNb\msNc, ॰वृत्ताः \msNa\Ed}}% 

\vers


{\devanagarifont 
\jump
\begin{center}
\ketdanda~इति वृषसारसंग्रहे यमविभागो नामाध्यायश्चतुर्थः~\ketdanda
\end{center}
\dontdisplaylinenum\vers  }%
     \var{{\devanagarifont \numnoemph{\englishfont \Colo:}\textbf{नामाध्यायश्चतुर्थः}\lem \mssALL, 
नामश्चतुर्थो ऽध्यायः \Ed}}% 
\bekveg\szamveg
\vfill
\phpspagebreak

\versno=0\fejno=5
\thispagestyle{empty}

\centerline{\Large\devanagarifontbold [   पञ्चमो ऽध्यायः  ]}{\vrule depth10pt width0pt} \fancyhead[CO]{{\footnotesize\devanagarifont वृषसारसंग्रहे  }}
\fancyhead[CE]{{\footnotesize\devanagarifont पञ्चमो ऽध्यायः  }}
\fancyhead[LE]{}
\fancyhead[RE]{}
\fancyhead[LO]{}
\fancyhead[RO]{}
\szam\bek



\alalfejezet{नियमाः}
\vers


{\devanagarifont विगतराग उवाच {\dandab}\dontdisplaylinenum  }%
     \var{{\devanagarifont \numemph\vo\textbf{विगतराग उवाच}\lem \mssALL, 
विगत\uncl{राग उवा}च \msCa}}% 
    \lacuna{\devanagarifontsmall {\englishfont Witnesses used for this chapter: \msCa\ ff.\thinspace 201v--202r, 
                                                  \msCb\ ff.\thinspace 208v--209r, 
                                                  \msCc\ ff.\thinspace 277r--278r,
                                                  \msNa\ ff.\thinspace 9r--9v, 
                                                  \msNb\ exp.\thinspace 50 (upper) and 51 (lower),
                                                  \msNc\ ff.\thinspace 217r--218r,
                                                  \msM\ ff.\thinspace 9r--10r,
                                                  \Ed\ pp.\thinspace 597--599;  
                                                  \mssCaCbCc\ = \msCa + \msCb + \msCc} }%
  
\nemsloka 
{\devanagarifont कथय नियमतत्त्वं साम्प्रतं त्वं विशेषाद् }%
  \dontdisplaylinenum    \var{{\devanagarifont \numnoemph\va\textbf{कथय नि॰}\lem \mssALL, कथयति \Ed\oo 
\textbf{॰तत्त्वं}\lem \mssALL, तं \msCb\oo 
\textbf{साम्प्रतं त्वं विशेषाद्}\lem \msCa\msNa\msNc\Ed, त्वां वशेषात् \msCb, 
सांप्रत त्वं विसेषात् \msCc\msNb, साम्प्रतं त्वं विशेषा \msM}}% 


\nemslokab

{\devanagarifont अमृतवचनतुल्यं श्रोतुकामो गतो ऽस्मि  \danda\dontdisplaylinenum }%
     \var{{\devanagarifont \numnoemph\vb\textbf{॰वचनतुल्यं श्रो॰}\lem \msM, वदनतुल्यं श्रो॰ \msCa\msCc\msNapcorr\msNb\msNc\Ed, 
वदनतुल्यां श्रो॰ \msCb, 
वदन\uncl{तुल्यं श्रो} तुल्यं स्रो॰ \msNaacorr\oo 
\textbf{॰कामो}\lem \mssALL, ॰कामा \msM\Ed}}% 

\nemslokac

{\devanagarifont प्रकृतिदहनदग्धं ज्ञानतोयैर्निषिक्तम् }%
  \dontdisplaylinenum    \var{{\devanagarifont \numnoemph\vc\textbf{॰दहन॰}\lem \mssALL, ॰वदन॰ \Ed\oo 
\textbf{॰दग्धं}\lem \mssALL, ॰दग्ध \msM\oo 
\textbf{॰र्निषिक्तम्}\lem \mssALL, ॰र्विमुक्तम् \msCb, ॰र्निशिक्तः \msM}}% 

%Verse 5:1


\nemslokad

{\devanagarifont अपर वदमतज्ज्ञं नास्ति धर्मेषु तृप्तिः {॥ ५:१॥} \veg\dontdisplaylinenum }%
     \var{{\devanagarifont \numnoemph\vd\textbf{अपर॰}\lem \mssALL, अपरं \msNa\ \unmetr, अर॰ \msMacorr\oo 
\textbf{मतज्ज्ञं नास्ति}\lem \conj, मतज्ञा नास्ति \msCapcorr\msCb\msNa\msNc\msM, 
तज्ञा नास्ति \msCaacorr, 
मतज्ञा\uncl{न्ना}स्ति \msCc, 
\uncl{मे} \lk\lk \lk\lk\ \msNb, 
॰न तज्ज्ञान्नास्ति \Ed\oo 
\textbf{धर्मेषु तृप्तिः}\lem \mssALL, मे धर्मतृप्तिः \msM}}% 

\vers


{\devanagarifont अनर्थयज्ञ उवाच {\dandab}\dontdisplaylinenum  }%
     \var{{\devanagarifont \numemph\vo\textbf{अनर्थ॰}\lem \mssALL, अर्थ॰ \msM}}% 

\nemsloka 
{\devanagarifont श्रवणसुखमतो ऽन्यत्कीर्तयिष्ये द्विजेन्द्र }%
  \dontdisplaylinenum    \var{{\devanagarifont \numnoemph\va\textbf{॰सुख॰}\lem \mssALL, ॰मुख॰ \msNaacorr\oo 
\textbf{॰मतो ऽन्यत्}\lem \mssCaCbCc\msNa\msNc, ॰मतो ऽन्य \msNb, 
॰मतो न्यः \msM, ॰मनो ऽन्यत् \Ed\oo 
\textbf{कीर्त॰}\lem \mssALL, कीर्ति॰ \msNa\msNb}}% 


\nemslokab

{\devanagarifont नियमकलविशेषः पञ्च पञ्च प्रकारः  \danda\dontdisplaylinenum }%
     \var{{\devanagarifont \numnoemph\vb\textbf{॰विशेषः}\lem \mssALL, विशे\lk\ \msCa, ॰विशेष \msCb\oo 
\textbf{प्रकारः}\lem \mssALL, पकारः \msNc}}% 

\nemslokac

{\devanagarifont हरिहरमुनिभीष्टं धर्मसारं द्विजेन्द्र }%
  \dontdisplaylinenum
%Verse 5:2


\nemslokad

{\devanagarifont कलिकलुषविनाशं प्रायमोक्षप्रसिद्धम् {॥ ५:२॥} \veg\dontdisplaylinenum }%
     \var{{\devanagarifont \numnoemph\vd\textbf{॰विनाशं}\lem \mssALL, ॰विनाश॰ \msCc\Ed}}% 

\vers


{\devanagarifont शौचमिज्या तपो दानं स्वाध्यायोपस्थनिग्रहः \thinspace{\dandab} \dontdisplaylinenum }%
     \var{{\devanagarifont \numemph\va\textbf{इज्या}\lem \msCa\msCb\msNa\msNc\Ed, ईज्या \msCc\msNb\msM\oo 
\textbf{दानं}\lem \mssALL, दान॰ \msNb}}% 

%Verse 5:3

{\devanagarifont व्रतोपवासमौनं च स्नानं च नियमा दश {॥ ५:३॥} \veg\dontdisplaylinenum }%
     \var{{\devanagarifont \numnoemph\vc\textbf{॰पवास॰}\lem \mssALL, ॰प्रवाष॰ \msM}}% 
    \paral{{\devanagarifontsmall \vo {\englishfont  = \LINPU\ 1.8.29cd--30ab = \VDHU\ 3.233.202} }}


\alalfejezet{नियमेषु शौचम् (१)}
{\devanagarifont तत्र शौचादिनिर्देशं वक्ष्यामीह द्विजोत्तम \thinspace{\dandab} \dontdisplaylinenum }%
     \var{{\devanagarifont \numemph\va\textbf{॰निर्देशं}\lem \mssALL, ॰नियमं \msNa, ॰र्द्देशं \msNb}}% 

%Verse 5:4

{\devanagarifont शारीरशौचमाहारो मात्रा भावश्च पञ्चमः {॥ ५:४॥} \veg\dontdisplaylinenum }%
     \var{{\devanagarifont \numnoemph\vc\textbf{शारीर॰}\lem \mssALL, शरीर॰ \msNb\oo 
\textbf{॰शौचमाहारो}\lem \mssALL, ॰शौच\lk हारो \msCa, 
॰स्रोतमाहार \msM}}% 
    \var{{\devanagarifont \numnoemph\vd\textbf{मात्रा भावश्च}\lem \mssALL, मात्रा भावं च \msCa, 
\uncl{सात्राभा}वश्च \msNb}}% 


\alalalfejezet{शरीरशौचम्}

{\devanagarifont ताडयेन्न च बन्धेत न च प्राणैर्वियोजयेत् \thinspace{\dandab} \dontdisplaylinenum }%
     \var{{\devanagarifont \numemph\va\textbf{ताडयेन्न}\lem \mssALL, ताडये न \msNc\msM\oo 
\textbf{बन्धेत}\lem \mssALL, बन्धेन \msM}}% 

%Verse 5:5

{\devanagarifont परस्त्रीपरद्रव्येषु शौचं कायिकमुच्यते {॥ ५:५॥} \veg\dontdisplaylinenum }%
     \var{{\devanagarifont \numnoemph\vc\textbf{॰द्रव्येषु}\lem \mssALL, ॰द्रवेषु \msM}}% 
    \var{{\devanagarifont \numnoemph\vd\textbf{शौचं}\lem \mssALL, शौच \msNc\oo 
\textbf{कायिकमुच्यते}\lem \mssALL, कायिकमुमुच्येते \msNc}}% 

{\devanagarifont श्रोत्रशौचं द्विजश्रेष्ठ गुदोपस्थमुखादयः \thinspace{\dandab} \dontdisplaylinenum }%
     \var{{\devanagarifont \numemph\va\textbf{श्रोत्र॰}\lem \msM, श्रोत॰ \mssCaCbCc\msNa\msNb\msNc\Ed}}% 
    \var{{\devanagarifont \numnoemph\vb\textbf{गुदोपस्थ॰}\lem \mssALL, गुदोप्रस्थ॰ \msNc, गुदापस्थ॰ \Ed}}% 

%Verse 5:6

{\devanagarifont मुखस्याचमनं शौचमाहारवचनेषु च {॥ ५:६॥} \veg\dontdisplaylinenum }%
     \var{{\devanagarifont \numnoemph\vc\textbf{मुखस्या॰}\lem \mssALL, मुखस्था॰ \msCb}}% 
    \var{{\devanagarifont \numnoemph\vcd\textbf{शौचमा॰}\lem \msCa\msCc\msNa\msNc\Ed, शौचंमा॰ \msCb\msNb, शौच आ॰ \msM}}% 
    \var{{\devanagarifont \numnoemph\vd\textbf{॰वचनेषु}\lem \mssALL, ॰वषनेषु \msM}}% 

{\devanagarifont मूत्रविष्टासमुत्सर्गे देवताराधनेषु च \thinspace{\dandab} \dontdisplaylinenum }%
     \var{{\devanagarifont \numemph\va\textbf{॰विष्टा॰}\lem \mssALL, ॰विष्ट॰ \msNb\msM}}% 

%Verse 5:7

{\devanagarifont मृत्तोयैस्तु गुदोपस्थं शौचयीत विचक्षणः {॥ ५:७॥} \veg\dontdisplaylinenum }%
     \var{{\devanagarifont \numnoemph\vc\textbf{मृत्तोयैस्तु}\lem \msCc\msNa\msNb\Ed, \uncl{मृ}\lk\lk \lk\ \msCa, 
मृतोयैस्तु \msCb\msM, मृत्तोयेस्तु \msNc\oo 
\textbf{॰पस्थं}\lem \msCa\msCb\msNa\msNb\msNc, ॰पस्थ \msCc\Ed, ॰पस्थः \msM}}% 
    \var{{\devanagarifont \numnoemph\vd\textbf{शौचयीत}\lem \mssALL, शौचये च \msM}}% 

{\devanagarifont एकोपस्थे गुदे पञ्च तथैकत्र करे दश \thinspace{\dandab} \dontdisplaylinenum }%
     \var{{\devanagarifont \numemph\va\textbf{॰पस्थे}\lem \msCa\msCb\msNa\msNc\Ed, ॰पस्थ॰ \msCc\msNb\msM\oo 
\textbf{गुदे}\lem \msCa\msCb\msNa\msNc\Ed, गुदो \msCc\msNb, गुद \msM}}% 
    \var{{\devanagarifont \numnoemph\vb\textbf{तथैकत्र}\lem \msCa\msCc\msNa\msNb\msNc, तथैक\uncl{त्र} \msCb, 
तथैकत्रे \msM, तथैकश्च \Ed\oo 
\textbf{दश}\lem \mssALL, दशः \msCc}}% 
    \paral{{\devanagarifontsmall \vo {\englishfont \similar\ \MANU\ 5.136:} एका लिङ्गे गुदे तिस्रस्तथैकत्र करे दश\thinspace{\devanagarifontsmall ।}
                                               उभयोः सप्त दातव्या मृदः शुद्धिमभीप्सता\thinspace{\devanagarifontsmall ॥} }}

%Verse 5:8

{\devanagarifont उभयोः सप्त दातव्या मृदः शुद्धिं समीहता {॥ ५:८॥} \veg\dontdisplaylinenum }%
     \var{{\devanagarifont \numnoemph\vc\textbf{उभयोः}\lem \mssALL, उभय \msM\oo 
\textbf{दातव्या}\lem \msCa\msCb\msNa\msNb\msNc, दातव्यो \msCc\Ed, दातव्य \msM}}% 
    \var{{\devanagarifont \numnoemph\vd\textbf{मृदः}\lem \mssCaCbCc\msNc\Ed, मृतः \msNa\msM, मृदा \msNb\oo 
\textbf{शुद्धिं समीहता}\lem \msCa\msCb\msNa, शुद्धिसमीहया \msCc, शु\uncl{द्धि} समीहता \msNb, 
शुद्धिः समीहता \msNc, शुद्धि समीहता \msM, शुद्धिं समाहिता \Ed}}% 

{\devanagarifont एतच्छौचं गृहस्थानां द्विगुणं ब्रह्मचारिणाम् \thinspace{\dandab} \dontdisplaylinenum }%
     \var{{\devanagarifont \numemph\va\textbf{एतच्छौचं}\lem \msCa\msCb\msNa\msNc\msM, चेतच्हौच \msCc\Ed, एत\lk\lk\ \msNb}}% 
    \var{{\devanagarifont \numnoemph\vb\textbf{॰गुणं}\lem \mssALL, ॰गुण \msCc}}% 
    \paral{{\devanagarifontsmall \vab {\englishfont \similar\ \MANU\ 5.137:}
                 एतच्छौचं गृहस्थानां द्विगुणं ब्रह्मचारिणाम्\thinspace{\devanagarifontsmall ।}
                 त्रिगुणं स्याद्वनस्थानां यतीनां तु चतुर्गुणम्\thinspace{\devanagarifontsmall ॥} }}

%Verse 5:9

{\devanagarifont वानप्रस्थस्य त्रिगुणं यतीनां तु चतुर्गुणम् {॥ ५:९॥} \veg\dontdisplaylinenum }%
     \var{{\devanagarifont \numnoemph\vc\textbf{वानप्रस्थस्य}\lem \mssALL, वानप्रस्थे तु \msM\oo 
\textbf{त्रि॰}\lem \mssALL, द्वि॰ \msCc}}% 


\alalalfejezet{आहारशौचम्}

{\devanagarifont आहारशौचं वक्ष्यामि शृणुष्वावहितो भव \thinspace{\dandab} \dontdisplaylinenum }%
     \var{{\devanagarifont \numemph\va\textbf{॰शौचं}\lem \mssALL, ॰शौच \msM}}% 
    \var{{\devanagarifont \numnoemph\vb\textbf{शृणुष्वावहितो}\lem \mssALL, शृणु\uncl{ष्वाव}\lk\lk\ \msCa, 
शृणुष्ववहितो \msNb}}% 

{\devanagarifont भागद्वयं तु भुञ्जीत भागमेकं जलं पिबेत्  \danda\dontdisplaylinenum }%
     \var{{\devanagarifont \numnoemph\vd\textbf{॰कं जलं}\lem \mssALL, ॰कोदकं \msM\oo 
\textbf{पिबेत्}\lem \mssALL, पिबे \msCb}}% 

%Verse 5:10

{\devanagarifont वायुसंचारदानार्थं चतुर्थमवशेषयेत् {॥ ५:१०॥} \veg\dontdisplaylinenum }%
     \var{{\devanagarifont \numnoemph\ve\textbf{॰चारदानार्थं}\lem \mssALL, ॰चरदानार्थं \msM, ॰चारणार्थाय \Ed}}% 
    \paral{{\devanagarifontsmall \vo {\englishfont \similar\ Śaṅkara's commentary ad \BHG\ 6.16:}
                                 उक्तं हि\thinspace{\devanagarifontsmall ।} 
                                 अर्धं सव्यञ्जनान्नस्य तृतीयमुदकस्य च\thinspace{\devanagarifontsmall ।} 
                                 वायोः संचरणार्थं तु चतुर्थमवशेषयेत्\thinspace{\devanagarifontsmall ॥};
                    {\englishfont \compare\ \ASTANGHR\ 8.46cd--47ab:}
                                              अन्नेन कुक्षेर्द्वावंशौ पानेनैकं प्रपूरयेत्\thinspace{\devanagarifontsmall ॥} 
                                              आश्रयं पवनादीनां चतुर्थमवशेषयेत्\thinspace{\devanagarifontsmall ।};
                    {\englishfont \compare\ \SANNYASUP\ 59:}
                                              आहारस्य च भागौ द्वौ तृतीयमुदकस्य च\thinspace{\devanagarifontsmall ।} 
                                              वायोः संचरणार्थाय चतुर्थमवशेषयेत्\thinspace{\devanagarifontsmall ॥} }}

{\devanagarifont स्निग्धस्वादुरसैः षड्भिराहारषड्रसैर्बुधः \thinspace{\dandab} \dontdisplaylinenum }%
     \var{{\devanagarifont \numemph\va\textbf{॰स्वादुरसैः}\lem \mssCaCbCc\msNa\msNc, ॰स्वा\lk रसैः \msNb, ॰स्वादुरसं \msM, ॰स्वादरसैः \Ed}}% 
    \var{{\devanagarifont \numnoemph\vb\textbf{॰हारषड्रसैर्बु॰}\lem \msCb\Ed, ॰हारसद्रवैर्बु॰ \msCa\msNa\msNc, 
॰हारसद्रवै बु॰ \msCc, ॰हारषड्रसै बु॰ \msNb, ॰हारे सद्रवद्बु॰ \msM}}% 

%Verse 5:11

{\devanagarifont धातुवैषम्यनाशो ऽस्ति न च रोगाः सुदारुणाः {॥ ५:११॥} \veg\dontdisplaylinenum }%
     \var{{\devanagarifont \numnoemph\vc\textbf{॰वैषम्यनाशो ऽस्ति}\lem \msCa\msCc\msNa\msNb\msNc, 
॰\uncl{दै}षम्य$\-$नाशास्ति \msCb, ॰वैशम्य नस्यास्ति \msM, ॰वैषम्य नश्यन्ति \Ed}}% 
    \var{{\devanagarifont \numnoemph\vd\textbf{रोगाः}\lem \mssALL, रोग \msM\oo 
\textbf{सुदारुणाः}\lem \mssALL, स्वदारुणाः \msM, सुदारुणः \Ed}}% 

{\devanagarifont अभक्ष्यं च न भक्षेत अपेयं न च पाययेत् \thinspace{\dandab} \dontdisplaylinenum }%
     \var{{\devanagarifont \numemph\va\textbf{अभक्ष्यं}\lem \mssCaCbCc\msNa\msNc, \lk\lk \lk\ \msNb, अभक्षं \msM\Ed\oo 
\textbf{च न भक्षेत}\lem \mssALL, न च भक्षेतः \msM}}% 
    \var{{\devanagarifont \numnoemph\vb\textbf{न च}\lem \mssALL, च न \msNc\Ed}}% 

%Verse 5:12

{\devanagarifont अगम्यं न च गम्येत अवाच्यं न च भाषयेत् {॥ ५:१२॥} \veg\dontdisplaylinenum }%
     \var{{\devanagarifont \numnoemph\vc\textbf{गम्येत}\lem \mssALL, गम्येतः \msM}}% 
    \var{{\devanagarifont \numnoemph\vd\textbf{अवाच्यं}\lem \mssALL, अवाचं \msCc}}% 

{\devanagarifont लशुनं च पलाण्डुं च गृञ्जनं कवकानि च \thinspace{\dandab} \dontdisplaylinenum }%
     \var{{\devanagarifont \numemph\va\textbf{पलाण्डुं}\lem \Ed, पलण्डुं \mssCaCbCc\msNb\msNc\msM, पलडुं \msNa}}% 
    \var{{\devanagarifont \numnoemph\vb\textbf{कवकानि}\lem \mssALL, च कचानि \Ed}}% 
    \paral{{\devanagarifontsmall \vab {\englishfont \similar\ \MANU\ 5.5ab: } लशुनं गृञ्जनं चैव पलाण्डुं कवकानि च }}

%Verse 5:13

{\devanagarifont गौरं च सूकरं मांसं वर्जयेच्च विधानतः {॥ ५:१३॥} \veg\dontdisplaylinenum }%
     \var{{\devanagarifont \numnoemph\vc\textbf{गौरं च}\lem \eme, गोरस्व \msCa\msNb, गोरश्च \msCb\msCc\msNa\msNc\msM, गौरश्च \Ed\oo 
\textbf{मांसं}\lem \mssALL, मांसः \msM, मासं \Ed}}% 
    \var{{\devanagarifont \numnoemph\vd\textbf{विधानतः}\lem \mssALL, विधानत् \msM}}% 

{\devanagarifont छत्त्राकं विड्वराहं च गोमांसं च न भक्षयेत् \thinspace{\dandab} \dontdisplaylinenum }%
     \var{{\devanagarifont \numemph\va\textbf{छत्त्राकं}\lem \mssALL, छत्त्राक \msCc\oo 
\textbf{विड्व॰}\lem \mssALL, विद्व॰ \msNa\msNc}}% 
    \var{{\devanagarifont \numnoemph\vb\textbf{गोमांसं}\lem \mssALL, गोमाञ् \msCbacorr}}% 
    \paral{{\devanagarifontsmall \vab {\englishfont \compare\ \MANU\ 5.19ab:} छत्राकं विड्वराहं च लशुनं ग्रामकुक्कुटम्  }}

%Verse 5:14

{\devanagarifont चटकं च कपोतं च जालपादांश्च वर्जयेत् {॥ ५:१४॥} \veg\dontdisplaylinenum }%
     \var{{\devanagarifont \numnoemph\vc\textbf{चटकं}\lem \mssALL, चटकाम् \msCc\msNb}}% 
    \var{{\devanagarifont \numnoemph\vd\textbf{॰पादांश्च}\lem \mssALL, जालपादञ्च \msM}}% 

{\devanagarifont हंससारसचक्राह्वकुक्कुटान्शुकश्येनकान् \thinspace{\dandab} \dontdisplaylinenum }%
     \var{{\devanagarifont \numemph\va\textbf{॰चक्राह्व॰}\lem \mssALL, ॰चक्राह्वा॰ \msM}}% 
    \var{{\devanagarifont \numnoemph\vb\textbf{॰कुक्कुटान्शु॰}\lem \mssCaCbCc\msNc\Ed, ॰कुक्कुटा शु॰ \msNa, ॰कुक्कुटां शु॰ \msNb, ॰कुर्कुटा शु॰ \msM\oo 
\textbf{॰श्येनकान्}\lem \msCa\msCc\msNc\Ed, ॰शोनकान् \msCb, ॰श्येनका \msNa, ॰श्येनकां \msNb, ॰श्येनकम् \msM}}% 

%Verse 5:15

{\devanagarifont काकोलूकं बलाकं च मत्स्यादींश्चापि वर्जयेत् {॥ ५:१५॥} \veg\dontdisplaylinenum }%
     \var{{\devanagarifont \numnoemph\vc \lem \msCb\msNc, काकोलूक\uncl{स्व}\lk\lk ञ्च \msCa, 
काकोलूकबलाकं च \msCc\msNa\msM\Ed, 
\uncl{काकोलूकं बलाकं च} \msNb}}% 
    \var{{\devanagarifont \numnoemph\vd \lem \mssALL, मत्स्यादीनि च वर्जये \msM}}% 

{\devanagarifont अमेध्यांश्चापवित्रांश्च सर्वानेव विवर्जयेत् \thinspace{\dandab} \dontdisplaylinenum }%
     \var{{\devanagarifont \numemph\va \lem \mssCaCbCc\msNa\msNc, 
\uncl{अमेध्याश्चापवित्रांश्च} \msNb, 
अमेध्याश्च पवित्राश्च \msM, अमेध्यश्चापवित्रांश्च \Ed}}% 
    \var{{\devanagarifont \numnoemph\vb \lem \mssALL, सर्वान्येतानि वर्जयेत् \msM}}% 

%Verse 5:16

{\devanagarifont शाकमूलफलानां च अभक्ष्यं परिवर्जयेत् {॥ ५:१६॥} \veg\dontdisplaylinenum }%
 
{\devanagarifont मानवेषु पुराणेषु शैवभारतसंहिते \thinspace{\dandab} \dontdisplaylinenum }%
 
{\devanagarifont कीर्तितानि विशेषेण शौचाचारमशेषतः  \danda\dontdisplaylinenum }%
     \var{{\devanagarifont \numemph\vc\textbf{विशेषेण}\lem \mssALL, मशेषेण \msM}}% 

%Verse 5:17

{\devanagarifont त्वया जिज्ञासितो ऽस्म्यद्य संक्षिप्तः कथितो मया {॥ ५:१७॥} \veg\dontdisplaylinenum }%
     \var{{\devanagarifont \numnoemph\ve\textbf{जिज्ञासितो}\lem \mssALL, जिज्ञासनो \msNc, जिज्ञासतो \Ed}}% 
    \var{{\devanagarifont \numnoemph\vf\textbf{॰क्षिप्तः}\lem \msCa\msCc\msNa\msNc\Ed, ॰क्षिप्य \msCb, ॰क्षिप्त \msNb\msM\oo 
\textbf{कथितो}\lem \mssALL, कथितं \Ed}}% 

{\devanagarifont सत्यवादी शुचिर्नित्यं ध्यानयोगरतः शुचिः \thinspace{\dandab} \dontdisplaylinenum }%
     \var{{\devanagarifont \numemph\va\textbf{॰वादी}\lem \mssALL, ॰वादि \msM\oo 
\textbf{॰रतः शुचिर्}\lem \msCa\msCb\Ed, ॰रतः शुचि \msCc\msNc, रतः शुचिन् \msNa\msNb, ॰रत शुचि \msM}}% 

%Verse 5:18

{\devanagarifont अहिंसकः शुचिर्दान्तो दयाभूतक्षमा शुचिः {॥ ५:१८॥} \veg\dontdisplaylinenum }%
     \var{{\devanagarifont \numnoemph\vc\textbf{अहिंसकः}\lem \mssALL, अहिंसक \msCb\msM\oo 
\textbf{शुचिर्दान्तो}\lem \msCa\msCb\msNa\msNb, शुचि दान्तो \msCc\msNc\msM, शुचिर्दान्तौ \Ed}}% 
    \var{{\devanagarifont \numnoemph\vd\textbf{॰भूत॰}\lem \mssALL, ॰भुत॰ \msM\oo 
\textbf{शुचिः}\lem \mssALL, शुचि \msM}}% 

{\devanagarifont सर्वेषामेव शौचानामर्थशौचं परं स्मृतम् \thinspace{\dandab} \dontdisplaylinenum }%
     \var{{\devanagarifont \numemph\vb\textbf{॰शौचं परं स्मृतम्}\lem \msCa\msNa\msNb\msNc, ॰शौचं पर स्मृतम् \msCb\msCc, 
॰शौच पर स्मृतः \msM, 
॰शौचयनं स्मृतः \Ed}}% 
    \paral{{\devanagarifontsmall \vab {\englishfont \similar\ \MANU\ 5.106:}
                         सर्वेषामेव शौचानामर्थशौचं परं स्मृतम्\thinspace{\devanagarifontsmall ।}
                         यो ऽर्थे शुचिर्हि स शुचिर्न मृद्वारिशुचिः शुचिः\thinspace{\devanagarifontsmall ॥} }}

{\devanagarifont यो ऽर्थे हि शुचिः स शुचिर्न मृद्वारिशुचिः शुचिः  \danda\dontdisplaylinenum }%
     \var{{\devanagarifont \numnoemph\vcd\textbf{यो ऽर्थे हि शुचिः स शुचिर्न}\lem \mssCaCbCc\msNc\ \unmetr, 
यो ऽर्थे हि शुचिः स शुचि न \msNa\msNb, 
यो र्थे शुचि हि स शुद्धि \msM, 
यो ऽर्थे हि सुशुचिर्विप्र न \Ed}}% 
    \var{{\devanagarifont \numnoemph\vd\textbf{॰शुचिः शुचिः}\lem \mssCaCbCc\msNa\msNc, शुचि शुचिः \msNb, ॰शुचि शुचि \msM, ॰शुचिः शुचि \Ed}}% 

%Verse 5:19

{\devanagarifont कायवाङ्मनसां शौचं स शुचिः सर्ववस्तुषु {॥ ५:१९॥} \veg\dontdisplaylinenum }%
     \var{{\devanagarifont \numnoemph\ve\textbf{वाङ्मनसां शौचं}\lem \mssALL, वाङ्मणसा शुद्धि \msM}}% 
    \var{{\devanagarifont \numnoemph\vf\textbf{शुचिः}\lem \mssALL, शुचि \msCc\msM\oo 
\textbf{वस्तुषु}\lem \mssALL, वस्तुषुः \msNc, वस्तुशु \msM}}% 
    \lacuna{\devanagarifontsmall \vcd {\englishfont \Ed\ adds here, after pādas cd:} शौचाशौचविधिर्ज्ञात्वा मुच्यते सर्वकिल्बिषात् }%
  
\nemslokalong


\ujvers\nemsloka {
{\devanagarifont शौचाशौचविधिज्ञमानव यदि कालक्षये निश्चयः }%
  \dontdisplaylinenum}    \var{{\devanagarifont \numemph\va\textbf{शौचाशौच॰}\lem \mssALL, शौचाशुच \msCb\oo 
\textbf{यदि}\lem \mssALL, यदिः \msM\oo 
\textbf{कालक्षये निश्चयः}\lem \msNaacorr\msNc, 
कालक्षयैर्निश्चयः \msCa\msCb\msNapcorr, 
कालक्षयेन्निश्चयः \msCc\msNb, 
कालक्षयानिश्चयः \msM, 
कालक्षयेतिश्च यः \Ed}}% 


\nemslokab

{\devanagarifont सौभाग्यत्वमवाप्नुवन्ति सततं कीर्तिर्यशोऽलङ्कृतम्  \danda\dontdisplaylinenum }%
     \var{{\devanagarifont \numnoemph\vb\textbf{कीर्तिर्यशो॰}\lem \msCb\msNa\msNb\msNc\Ed, कीर्तियशो॰ \msCa\msCc \unmetr, कीर्तिर्यषा॰ \msM\oo 
\textbf{॰लंकृतम्}\lem \msM, ॰लङ्कृतः \msCa\msCc\msNa\msNb\msNc\Ed, ॰लकृतः \msCb}}% 
    \paral{{\devanagarifontsmall \vb {\englishfont \similar\ 4.67b above:}
                         लोके ऽनिन्दनमाप्नुवन्ति सततं कीर्तिर्यशोऽलंकृतम् }}

\nemslokac

{\devanagarifont प्राप्तं तेन इहैव पुण्यसकलं सद्धर्मशास्त्रेरितं }%
  \dontdisplaylinenum    \var{{\devanagarifont \numnoemph\vc\textbf{सद्धर्म॰}\lem \mssALL, य धर्म॰ \msM\oo 
\textbf{॰एरितम्}\lem \mssALL, ॰ओदितः \Ed}}% 

%Verse 5:20


\nemslokad

{\devanagarifont जीवान्ते च परत्रमीहितगतिं प्राप्नोति निःसंशयम् {॥ ५:२०॥} \veg\dontdisplaylinenum }%
     \var{{\devanagarifont \numnoemph\vd\textbf{परत्रमीहित॰}\lem \mssALL, 
परत्रमीहत॰ \msM, पवित्रमीहित॰ \Ed\oo 
\textbf{॰गतिं}\lem \eme, ॰गतिः \mssCaCbCc\msNa\msNb\msNc\msM\Ed\oo 
\textbf{निःसंशयम्}\lem \msCa\msNb\msNc, निःसंशयः \msCb\msCc\msNa\msM\Ed}}% 

\vers


{\devanagarifont 
\jump
\begin{center}
\ketdanda~इति वृषसारसंग्रहे शौचाचारविधिर्नामाध्यायः पञ्चमः~\ketdanda
\end{center}
\dontdisplaylinenum\vers  }%
     \var{{\devanagarifont \numnoemph{\englishfont \Colo:}\textbf{॰विधिर्नमा॰}\lem \msCa, ॰विधिनामा॰ \msCb\msCc\msNa\msNc\msM, \uncl{विंधि}नामा॰ \msNb, ॰विधिर्नाम \Ed\oo 
\textbf{॰ध्ययः पञ्चमः}\lem \mssALL, ॰ध्यायः पञ्चमः श्लोक २५ \msM, 
पञ्चमो ऽध्यायः \Ed}}% 
\bekveg\szamveg
\vfill
\phpspagebreak

\versno=0\fejno=6
\thispagestyle{empty}

\centerline{\Large\devanagarifontbold [   षष्ठो ऽध्यायः  ]}{\vrule depth10pt width0pt} \fancyhead[CO]{{\footnotesize\devanagarifont वृषसारसंग्रहे  }}
\fancyhead[CE]{{\footnotesize\devanagarifont षष्ठो ऽध्यायः  }}
\fancyhead[LE]{}
\fancyhead[RE]{}
\fancyhead[LO]{}
\fancyhead[RO]{}
\szam\bek


\nemslokanormal



\alalfejezet{नियमेष्विज्या (२)}
\vers


{\devanagarifont अथ पञ्चविधामिज्यां प्रवक्ष्यामि द्विजोत्तम \thinspace{\dandab} \dontdisplaylinenum }%
     \var{{\devanagarifont \numemph\va\textbf{॰मिज्यां}\lem \msCb, ॰मीज्यां \msCa\msCc\msNa\msNb\msNc\Ed}}% 
    \var{{\devanagarifont \numnoemph\vb\textbf{॰त्तम}\lem \mssALL, ॰त्तमः \msNb\msNc}}% 
    \lacuna{\devanagarifontsmall {\englishfont Witnesses used for this chapter: \msCa\ ff.\thinspace 202r--203r, 
                                              \msCb\ ff.\thinspace 209r--209v, 
                                              \msCc\ ff.\thinspace 278r--279r,
                                              \msNa\ ff.\thinspace 9v--10v, 
                                              \msNb\ exp.\thinspace 51 (lower--upper) -- 52 (lower),
                                              \msNc\ ff.\thinspace 218r--218v,
                                              \Ed\ pp.\thinspace 599--601;  
                                              \mssCaCbCc\ = \msCa + \msCb + \msCc} }%
  
%Verse 6:1

{\devanagarifont धर्ममोक्षप्रसिद्ध्यर्थं शृणुष्वावहितो द्विज {॥ ६:१॥} \veg\dontdisplaylinenum }%
     \var{{\devanagarifont \numnoemph\vc\textbf{॰मोक्षप्रसिद्ध्यर्थं}\lem \mssCaCbCc\msNc, ॰मोक्षप्रसिद्ध्यर्थ \msNa\msNb, 
॰मोक्षेशसिद्ध्यर्थं \Ed}}% 
    \var{{\devanagarifont \numnoemph\vd\textbf{द्विज}\lem \mssALL, भव \Ed}}% 

{\devanagarifont अर्थयज्ञः क्रियायज्ञो जपयज्ञस्तथैव च \thinspace{\dandab} \dontdisplaylinenum }%
     \var{{\devanagarifont \numemph\va\textbf{अर्थयज्ञः}\lem \msCa\msCc\msNa, अनर्थयज्ञः \msCb, 
अर्थयज्ञ \msNb\msNc, अर्थयज्ञ॰ \Ed}}% 

%Verse 6:2

{\devanagarifont ज्ञानं ध्यानं च पञ्चैतत्प्रवक्ष्यामि पृथक्पृथक् {॥ ६:२॥} \veg\dontdisplaylinenum }%
     \var{{\devanagarifont \numnoemph\vc\textbf{ज्ञानं}\lem \mssALL, ज्ञान \msCc\msNc}}% 


\alalalfejezet{अर्थयज्ञः}

{\devanagarifont अग्न्युपासनकर्मादि अग्निहोत्रक्रतुक्रिया \thinspace{\dandab} \dontdisplaylinenum }%
     \var{{\devanagarifont \numemph\vb\textbf{अग्नि॰}\lem \mssALL, \uncl{अ}\lac\  \msCa, \lk\lk\ \msNb\oo 
\textbf{॰क्रिया}\lem \mssALL, ॰क्रियाः \msCb\msCc}}% 

%Verse 6:3

{\devanagarifont अष्टका पार्वणी श्राद्धं द्रव्ययज्ञः स उच्यते {॥ ६:३॥} \veg\dontdisplaylinenum }%
     \var{{\devanagarifont \numnoemph\vc\textbf{पार्वणी}\lem \mssALL, पर्वणी \msCb, \uncl{पर्वणी} \msNb}}% 
    \var{{\devanagarifont \numnoemph\vd\textbf{॰यज्ञः}\lem \mssALL, ॰यज्ञ \msCc, \lk\lk\ \msNb}}% 


\alalalfejezet{क्रियायज्ञः}

{\devanagarifont आरामोद्यानवापीषु देवतायतनेषु च \thinspace{\dandab} \dontdisplaylinenum }%
     \var{{\devanagarifont \numemph\vb\textbf{॰यतनेषु}\lem \msCb\msCc\Ed, ॰लयनेषु \msCa\msNa\msNc, ॰यत\lk\lk\ \msNb}}% 

%Verse 6:4

{\devanagarifont स्वहस्तकृतसंस्कारः क्रियायज्ञ स उच्यते {॥ ६:४॥} \veg\dontdisplaylinenum }%
     \var{{\devanagarifont \numnoemph\vc\textbf{॰हस्त॰}\lem \mssALL, \lk\lk\ \msNb, ॰हस्तैः \Ed}}% 


\alalalfejezet{जपयज्ञः}

{\devanagarifont जपयज्ञं ततो वक्ष्ये स्वर्गमोक्षफलप्रदम् \thinspace{\dandab} \dontdisplaylinenum }%
     \var{{\devanagarifont \numemph\va\textbf{॰यज्ञं ततो}\lem \mssALL, ॰यज्ञं तपो \msCb ॰यज्ञस्ततो \msCc}}% 

{\devanagarifont वेदाध्ययन कर्तव्यं शिवसंहितमेव च  \danda\dontdisplaylinenum }%
     \var{{\devanagarifont \numnoemph\vc\textbf{वेदा॰}\lem \mssALL, अदा॰ \msNb}}% 

%Verse 6:5

{\devanagarifont इतिहासपुराणं च जपयज्ञः स उच्यते {॥ ६:५॥} \veg\dontdisplaylinenum }%
     \var{{\devanagarifont \numnoemph\ve\textbf{॰पुराणं च}\lem \mssALL, ॰पुराणश्च \Ed}}% 
    \var{{\devanagarifont \numnoemph\vf\textbf{॰यज्ञः}\lem \mssALL, ॰यज्ञ \msCc}}% 


\alalalfejezet{ज्ञानयज्ञः}

{\devanagarifont इदं कर्म अकर्मेदमूहापोहविशारदः \thinspace{\dandab} \dontdisplaylinenum }%
     \var{{\devanagarifont \numemph\va\textbf{कर्म}\lem \mssALL, क्रमम् \Ed}}% 

%Verse 6:6

{\devanagarifont शास्त्रचक्षुः समालोक्य ज्ञानयज्ञः स उच्यते {॥ ६:६॥} \veg\dontdisplaylinenum }%
     \var{{\devanagarifont \numnoemph\vc\textbf{॰चक्षुः}\lem \mssALL, ॰चक्षु \msCc}}% 
    \var{{\devanagarifont \numnoemph\vd\textbf{॰यज्ञः}\lem \mssALL, ॰यज्ञ \msCc, ॰\uncl{यज्ञस्} \msNb}}% 


\alalalfejezet{ध्यानयज्ञः}

{\devanagarifont ध्यानयज्ञं समासेन कथयिष्यामि ते शृणु \thinspace{\dandab} \dontdisplaylinenum }%
     \var{{\devanagarifont \numemph\va\textbf{॰यज्ञं}\lem \mssALL, ॰यज्ञ \msCc\msNb}}% 

{\devanagarifont ध्यानं पञ्चविधं चैव कीर्तितं हरिणा पुरा  \danda\dontdisplaylinenum }%
     \var{{\devanagarifont \numnoemph\vc\textbf{ध्यानं}\lem \mssALL, ध्यान \msNa\msNc}}% 

%Verse 6:7

{\devanagarifont सूर्यः सोमो ऽग्नि स्फटिकः सूक्ष्मं तत्त्वं च पञ्चमम् {॥ ६:७॥} \veg\dontdisplaylinenum }%
     \var{{\devanagarifont \numnoemph\ve\textbf{सोमो}\lem \msCa\msCc\msNa\msNc, सोमा॰ \msCb\msNb\Ed}}% 
    \var{{\devanagarifont \numnoemph\vf \lem \msCb, 
सूक्ष्मं त\uncl{त्व}\lac  ञ्चमम् \msCa, 
सूक्ष्मतत्त्वं च पञ्चमः \msCc\msNa\msNb, 
सूक्ष्मं तत्त्वञ्च पञ्चमः \msNc, 
सूक्ष्मां तत्त्वश्च पञ्चमम् \Ed}}% 

{\devanagarifont सूर्यमण्डलमादौ तु तत्त्वं प्रकृतिरुच्यते \thinspace{\dandab} \dontdisplaylinenum }%
 
%Verse 6:8

{\devanagarifont तस्य मध्ये शशिं ध्यायेत्तत्त्वं पुरुष उच्यते {॥ ६:८॥} \veg\dontdisplaylinenum }%
     \var{{\devanagarifont \numemph\vc\textbf{शशिं}\lem \mssALL, शशि \msNb, शशिंन् \msNc}}% 
    \var{{\devanagarifont \numnoemph\vcd\textbf{ध्यायेत्त॰}\lem \mssALL, ध्याये त॰ \msCc}}% 

{\devanagarifont चन्द्रमण्डलमध्ये तु ज्वालामग्निं विचिन्तयेत् \thinspace{\dandab} \dontdisplaylinenum }%
     \var{{\devanagarifont \numemph\vb\textbf{ज्वालामग्निं}\lem \mssALL, जालामग्नि \msNc}}% 

%Verse 6:9

{\devanagarifont प्रभुतत्त्वः स विज्ञेयो जन्ममृत्युविनाशनः {॥ ६:९॥} \veg\dontdisplaylinenum }%
     \var{{\devanagarifont \numnoemph\vc\textbf{॰तत्त्वः}\lem \mssCaCbCc\msNc, ॰तत्व \msNa, ॰तत्वं \msNb\Ed}}% 
    \var{{\devanagarifont \numnoemph\vd\textbf{॰नाशनः}\lem \mssALL, ॰नाशनम् \msCc\Ed}}% 

{\devanagarifont अग्निमण्डलमध्ये तु ध्यायेत्स्फटिक निर्मलम् \thinspace{\dandab} \dontdisplaylinenum }%
     \var{{\devanagarifont \numemph\vb\textbf{ध्यायेत्स्फटिक}\lem \msCapcorr\msCb\msNa\msNb\msNc, ध्यायेत्स्फटि \msCaacorr, 
ध्याये स्फटिक \msCc\Ed\oo 
\textbf{॰मलम्}\lem \mssALL, ॰मलः \msNa, ॰\uncl{मलः} \msNc}}% 

%Verse 6:10

{\devanagarifont विद्यातत्त्वः स विज्ञेयः कारणमजमव्ययम् {॥ ६:१०॥} \veg\dontdisplaylinenum }%
     \var{{\devanagarifont \numnoemph\vc\textbf{तत्त्वः स}\lem \msCb\msNa\msNb\msNc, त\uncl{त्वन्}\lac\  \msCa, तत्व स \msCc, तत्वं स \Ed}}% 
    \var{{\devanagarifont \numnoemph\vd\textbf{॰जमव्ययम्}\lem \mssALL, ॰मव्ययं \msCc}}% 

{\devanagarifont विद्यामण्डलमध्ये तु ध्यायेत्तत्त्वमनुत्तमम् \thinspace{\dandab} \dontdisplaylinenum }%
     \var{{\devanagarifont \numemph\vab\textbf{ध्यायेत्त॰}\lem \mssALL, ध्याये त॰ \msCc}}% 

{\devanagarifont अकीर्तितमनौपम्यं शिवमक्षयमव्ययम्  \danda\dontdisplaylinenum }%
     \paral{{\devanagarifontsmall \vcd {\englishfont \DHARMP\ 4.14ab: } अकीर्तितमनौपम्यं पञ्चमं शिवमण्डलम् }}

%Verse 6:11

{\devanagarifont पञ्चमं ध्यानयज्ञस्य तत्त्वमुक्तं समासतः {॥ ६:११॥} \veg\dontdisplaylinenum }%
     \var{{\devanagarifont \numnoemph\ve\textbf{॰यज्ञस्य}\lem \mssALL, ॰यज्ञञ्च \msCc\Ed}}% 
    \var{{\devanagarifont \numnoemph\vf\textbf{समासतः}\lem \mssALL, सनातनः \Ed}}% 

{\devanagarifont विगतराग उवाच {\dandab}\dontdisplaylinenum  }%
 
{\devanagarifont एकैकस्य तु तत्त्वस्य फलं कीर्तय कीदृशम् \thinspace{\danda} \dontdisplaylinenum }%
     \var{{\devanagarifont \numemph\va\textbf{तु}\lem \conj, त्रि॰ \mssCaCbCc\msNa\msNb\msNc, हि \Ed}}% 

%Verse 6:12

{\devanagarifont कानि लोकाः प्रपद्यन्ते कालं वास्य तपोधन {॥ ६:१२॥} \veg\dontdisplaylinenum }%
     \var{{\devanagarifont \numnoemph\vc\textbf{लोकाः}\lem \msCa\msNa\msNc, लोका \msCb\msCc\msNb\Ed\oo 
\textbf{प्रपद्यन्ते}\lem \mssALL, प्र\lk\lk\lk\ \msCa}}% 
    \var{{\devanagarifont \numnoemph\vd\textbf{॰धन}\lem \mssALL, ॰धनः \msCb\msNc}}% 

{\devanagarifont अनर्थयज्ञ उवाच {\dandab}\dontdisplaylinenum  }%
 
{\devanagarifont ब्रह्मलोकं तु प्रथमं तत्त्वप्रकृतिचिन्तया \thinspace{\danda} \dontdisplaylinenum }%
     \var{{\devanagarifont \numemph\vab\textbf{प्रथमं तत्त्व॰}\lem \mssALL, 
\om\ \msNaacorr, प्रथमं तत्त्वं \Ed\oo 
\textbf{प्रकृतिचिन्तया}\lem \mssALL, च कृतिचिन्तय \Ed}}% 

%Verse 6:13

{\devanagarifont कल्पकोटिसहस्राणि शिववन्मोदते सुखी {॥ ६:१३॥} \veg\dontdisplaylinenum }%
     \var{{\devanagarifont \numnoemph\vd\textbf{सुखी}\lem \mssALL, सुखम् \Ed}}% 

{\devanagarifont द्वितीयं तत्त्व पुरुषं ध्यायमानो मृतो यदि \thinspace{\dandab} \dontdisplaylinenum }%
 
%Verse 6:14

{\devanagarifont विष्णुलोकमितो याति कल्पकोट्ययुतं सुखी {॥ ६:१४॥} \veg\dontdisplaylinenum }%
     \var{{\devanagarifont \numemph\vc\textbf{याति}\lem \mssALL, यान्ति \Ed}}% 

{\devanagarifont प्रभुतत्त्वं तृतीयं तु ध्यायमानो मरिष्यति \thinspace{\dandab} \dontdisplaylinenum }%
     \var{{\devanagarifont \numemph\va\textbf{॰तत्त्वं}\lem \mssALL, ॰तत्व \msCc\oo 
\textbf{तृतीयं}\lem \mssALL, तृतीयस् \Ed}}% 
    \var{{\devanagarifont \numnoemph\vb \lem \mssALL, ध्याय\lk\lk \lk रिष्यति \msCa, 
धयायामानो मरिष्यति \Ed}}% 

%Verse 6:15

{\devanagarifont शिवलोके वसेन्नित्यं कल्पकोट्ययुतं शतम् {॥ ६:१५॥} \veg\dontdisplaylinenum }%
     \var{{\devanagarifont \numnoemph\vc\textbf{शिवलोके}\lem \mssALL, शिवलोक \msCb, रुद्रलोके \Ed\oo 
\textbf{वसेन्नि॰}\lem \mssALL, वसे नि॰ \msCc}}% 
    \var{{\devanagarifont \numnoemph\vd\textbf{॰युतं}\lem \mssALL, ॰युत \msNb}}% 

{\devanagarifont विद्यातत्त्वामृतं ध्यायेत्सदाशिवमनामयम् \thinspace{\dandab} \dontdisplaylinenum }%
     \var{{\devanagarifont \numemph\va\textbf{॰तत्त्वामृतं}\lem \mssALL, ॰तत्वमृतन् \msCc, ॰तत्त्वामतं \Ed}}% 

%Verse 6:16

{\devanagarifont अक्षयं लोकमाप्नोति कल्पानान्तपरं तथा {॥ ६:१६॥} \veg\dontdisplaylinenum  }%
     \var{{\devanagarifont \numnoemph\vc\textbf{अक्षयं}\lem \mssALL, अक्षय॰ \Ed}}% 

{\devanagarifont पञ्चमं शिवतत्त्वं तु सूक्ष्मं चात्मनि संस्थितम् \thinspace{\dandab} \dontdisplaylinenum }%
 
%Verse 6:17

{\devanagarifont न कालसंख्या तत्रास्ति शिवेन सह मोदते {॥ ६:१७॥} \veg\dontdisplaylinenum }%
 
\nemslokalong


\ujvers\nemsloka {
{\devanagarifont पञ्चध्यानाभियुक्तो भवति च न पुनर्जन्मसंस्कारबन्धः }%
  \dontdisplaylinenum}    \var{{\devanagarifont \numemph\va\textbf{॰युक्तो}\lem \mssALL, ॰यु\lk\ \msCa\ \toplost, ॰युक्तौ \Ed\oo 
\textbf{च}\lem \mssALL, \om\ \msCb\Ed\oo 
\textbf{पुनर्जन्म॰}\lem \mssALL, 
पुन\uncl{ज}न्म॰ \msCa\ \toplost, पुनजन्म॰ \msCc}}% 


\nemslokab

{\devanagarifont जिज्ञास्यन्तां द्विजेन्द्र भवदहनकरः प्रार्थनाकल्पवृक्षः  \danda\dontdisplaylinenum }%
     \var{{\devanagarifont \numnoemph\vb\textbf{जिज्ञास्यन्तां}\lem \msCa\msNb\msNc\Ed, जिज्ञास्यतां \msCb\msNa\ \unmetr, जिज्ञास्यन्ता \msCc}}% 

\nemslokac

{\devanagarifont जन्मेनैकेन मुक्तिर्भवति किमु न वा मानवाः साधयन्तु }%
  \dontdisplaylinenum    \var{{\devanagarifont \numnoemph\vc\textbf{जन्मेनैकेन}\lem \msCb\msNb\msNc\Ed, जन्मनैकेन \msCa\msCc\msNa\ \unmetr\oo 
\textbf{मुक्तिरभ्॰}\lem \mssALL, मुक्ति भ्॰ \msCc\oo 
\textbf{न वा}\lem \mssALL, भवा \msNa\oo 
\textbf{मानवाः}\lem \msCa\msNa\msNb\msNc, मानमानवाः \msCb, मानवा \msCc, मानव \Ed}}% 

%Verse 6:18


\nemslokad

{\devanagarifont प्रत्यक्षान्नानुमानं सकलमलहरं स्वात्मसंवेदनीयम् {॥ ६:१८॥} \veg\dontdisplaylinenum }%
     \var{{\devanagarifont \numnoemph\vd\textbf{प्रत्यक्षा॰}\lem \mssALL, प्रत्यक्ष॰ \msNa\oo 
\textbf{॰वेदनीयम्}\lem \msCb\msNa\msNb, ॰वेदनीयः \msCa\msCc\msNc, ॰वेदनीय \Ed}}% 

\nemslokanormal


\vers



\alalfejezet{नियमेषु तपः (३)}
{\devanagarifont मानसं तप आदौ तु द्वितीयं वाचिकं तपः \thinspace{\dandab} \dontdisplaylinenum }%
     \var{{\devanagarifont \numemph\va\textbf{॰तप}\lem \mssALL, ॰तपम् \Ed}}% 

{\devanagarifont कायिकं च तृतीयं तु मनोवाक्कर्म तत्परम्  \danda\dontdisplaylinenum }%
     \var{{\devanagarifont \numnoemph\vc \lem \mssALL, 
मानसं तप आदौ तु \msNb\ {\englishfont (eyeskip)}}}% 
    \var{{\devanagarifont \numnoemph\vd\textbf{मनोवाक्कर्म}\lem \msCa\msNc\Ed, मनोक्कर्म \msCb, म्मनोवाकर्म॰ \msCc, मनोवाक्काय॰ \msNa\msNb\oo 
\textbf{॰परम्}\lem \msCc, ॰परः \msCa\msCb\msNa\msNb\msNc\Ed}}% 

%Verse 6:19

{\devanagarifont कायिकं वाचिकं चैव तपो मिश्रक पञ्चमम् {॥ ६:१९॥} \veg\dontdisplaylinenum }%
     \var{{\devanagarifont \numnoemph\ve\textbf{कायिकं}\lem \mssALL, कायिक \msNa}}% 

{\devanagarifont मनःसौम्यं प्रसादश्च आत्मनिग्रहमेव च \thinspace{\dandab} \dontdisplaylinenum }%
     \var{{\devanagarifont \numemph\va\textbf{॰सौम्यं}\lem \msNc, ॰सौम्य॰ \msCa\msCb\msNa\msNb\Ed, ॰सौम्\uncl{य}॰ \msCc\ \toplost\oo 
\textbf{प्रसादश्च}\lem \msCa\msCc\msNa\msNc, प्रसादं च \msCb\Ed, प्रदानश्च \msNb}}% 

%Verse 6:20

{\devanagarifont मौनं भावविशुद्धिश्च पञ्चैतत्तप मानसम् {॥ ६:२०॥} \veg\dontdisplaylinenum }%
     \var{{\devanagarifont \numnoemph\vc\textbf{मौनं}\lem \mssALL, मौन\lk  \Ed\oo 
\textbf{॰शुद्धिश्च}\lem \mssALL, ॰शुद्धिं च \msCc\Ed}}% 
    \var{{\devanagarifont \numnoemph\vd\textbf{पञ्चैतत्}\lem \msCa\msNb\msNc, पञ्चैते \msCb\msNa, पञ्चेतत् \msCc, पञ्चैतन् \Ed}}% 
    \paral{{\devanagarifontsmall \vo {\englishfont \similar\ \MBH\ 6.39.16 (\BHG\ 17.16):}
                 मनःप्रसादः सौम्यत्वं मौनमात्मविनिग्रहः\thinspace{\devanagarifontsmall ।}
                 भावसंशुद्धिरित्येतत्तपो मानसमुच्यते\thinspace{\devanagarifontsmall ॥} }}

{\devanagarifont अनुद्वेगकरा वाणी प्रियं सत्यं हितं च यत् \thinspace{\dandab} \dontdisplaylinenum }%
 
%Verse 6:21

{\devanagarifont स्वाध्यायाभ्यसनं चैव वाचिकं तप उच्यते {॥ ६:२१॥} \veg\dontdisplaylinenum }%
     \var{{\devanagarifont \numemph\vc\textbf{॰भ्यसनं चैव}\lem \mssALL, ॰भ्यसन\lk\lk\ \msCa, 
॰भ्यस\uncl{नं} चैव \msNb}}% 
    \paral{{\devanagarifontsmall \vcd {\englishfont \similar\ \MBH\ 6.39.15cd (\BHG\ 17.15):}
                                  अनुद्वेगकरं वाक्यं सत्यं प्रियहितं च यत्\thinspace{\devanagarifontsmall ।}
                                  स्वाध्यायाभ्यसनं चैव वाङ्मयं तप उच्यते\thinspace{\devanagarifontsmall ॥} }}

{\devanagarifont आर्जवं च अहिंसा च ब्रह्मचर्यं सुरार्चनम् \thinspace{\dandab} \dontdisplaylinenum }%
     \var{{\devanagarifont \numemph\va \lem \mssALL, आर्जवत्वमहिंसाश्च \Ed}}% 
    \var{{\devanagarifont \numnoemph\vb\textbf{॰चर्यं}\lem \mssALL, ॰चर्य \msCc\Ed}}% 

%Verse 6:22

{\devanagarifont शौचं पञ्चममित्येतत्कायिकं तप उच्यते {॥ ६:२२॥} \veg\dontdisplaylinenum }%
     \var{{\devanagarifont \numnoemph\vc\textbf{शौचं}\lem \mssALL, शौच \Ed}}% 
    \paral{{\devanagarifontsmall \vo {\englishfont \compare\ \MBH\ 6.39.14 (\BHG\ 17.14):}
                          देवद्विजगुरुप्राज्ञपूजनं शौचमार्जवम्\thinspace{\devanagarifontsmall ।}
                          ब्रह्मचर्यमहिंसा च शारीरं तप उच्यते\thinspace{\devanagarifontsmall ॥} }}

{\devanagarifont इष्टं कल्याणभावं च धन्यं पथ्यं हितं वदेत् \thinspace{\dandab} \dontdisplaylinenum }%
     \var{{\devanagarifont \numemph\va\textbf{इष्टं}\lem \mssALL, इष्ट \msCc\msNb\oo 
\textbf{॰भावं}\lem \mssALL, ॰भावश् \Ed}}% 
    \var{{\devanagarifont \numnoemph\vb\textbf{पथ्यं}\lem \mssALL, सत्यं \Ed}}% 

%Verse 6:23

{\devanagarifont मनोमिश्रक पञ्चैतत्तप उक्तं महर्षिभिः {॥ ६:२३॥} \veg\dontdisplaylinenum }%
     \var{{\devanagarifont \numnoemph\vc\textbf{मनो॰}\lem \mssALL, मन॰ \Ed\oo 
\textbf{पञ्चैतत्}\lem \mssALL, पञ्चेतत् \msNc, पञ्चैतान् \Ed}}% 
    \var{{\devanagarifont \numnoemph\vd \lem \mssALL, तपमुक्तं महिर्षिभिः \Ed}}% 

{\devanagarifont स्वस्ति मङ्गलमाशीर्भिरतिथिगुरुपूजनम् \thinspace{\dandab} \dontdisplaylinenum }%
     \var{{\devanagarifont \numemph\va\textbf{॰शीर्भि॰}\lem \msCa\Ed, ॰शीभि॰ \msCb\msCc\msNa\msNb\msNc}}% 
    \var{{\devanagarifont \numnoemph\vb\textbf{॰तिथि॰}\lem \mssALL, ॰तिथिं \Ed}}% 
    \paral{{\devanagarifontsmall \vab {\englishfont \compare\ \SDHS\ 11.79:}
                 नमस्काराभिवादेषु स्वस्तिमङ्गलवाचकैः\thinspace{\devanagarifontsmall ।}
                 शिवं भवतु सर्वत्र प्रब्रूयात्सर्वकर्मसु\thinspace{\devanagarifontsmall ॥} }}

%Verse 6:24

{\devanagarifont कायमिश्रक पञ्चैतत्तप उक्तं महात्मभिः {॥ ६:२४॥} \veg\dontdisplaylinenum }%
     \var{{\devanagarifont \numnoemph\vc\textbf{॰मिश्रक}\lem \mssALL, ॰\lk\lk क \msCa, ॰मित्यश्रक \msCb\oo 
\textbf{पञ्चैतत्}\lem \mssALL, पञ्चैतन् \Ed}}% 
    \var{{\devanagarifont \numnoemph\vd\textbf{तप उक्तं}\lem \mssALL, तपमुक्तं \Ed}}% 

{\devanagarifont मण्डूकयोगी हेमन्ते ग्रीष्मे पञ्चतपास्तथा \thinspace{\dandab} \dontdisplaylinenum }%
     \var{{\devanagarifont \numemph\vb\textbf{ग्रीष्मे}\lem \mssALL, गृष्मे \Ed}}% 
    \paral{{\devanagarifontsmall \vab {\englishfont \similar\ \MBH\ Suppl. 15.801:}
                                 मण्डूकशायी हेमन्ते ग्रीष्मे पञ्चतपा भवेत
                     {\englishfont \similar\ \UMS\ 6.26ab:}मण्डूकयोगो हेमन्ते ग्रीष्मे पञ्चतपास्तथा;
                     {\englishfont \compare\ \SDHSAMGR\ 9.32ab:}
                         अभ्रावकाश्यं शीतोष्णे पञ्चाग्निर्जलशायिता }}

%Verse 6:25

{\devanagarifont अभ्रावकाशो वर्षासु तपःसाधनमुच्यते {॥ ६:२५॥} \veg\dontdisplaylinenum }%
     \var{{\devanagarifont \numnoemph\vc\textbf{॰वकाशो}\lem \eme, ॰वकाशे \mssCaCbCc\msNa\msNb\msNc\Ed}}% 
    \var{{\devanagarifont \numnoemph\vd\textbf{तप॰}\lem \mssALL, तप \msCc\oo 
\textbf{साधनमु॰}\lem \msCa\msNa\msNc\Ed, साधन उ॰ \msCb\msCc\msNb}}% 

{\devanagarifont स्वमांसोद्धृत्य दानं च हस्तपादशिरस्तथा \thinspace{\dandab} \dontdisplaylinenum }%
     \var{{\devanagarifont \numemph\va\textbf{दानं}\lem \mssALL, \uncl{दान} \msNb\ \toplost, दानश् \Ed}}% 

%Verse 6:26

{\devanagarifont पुष्पमुत्पाद्य दानंच सर्वे ते तपसाधनाः {॥ ६:२६॥} \veg\dontdisplaylinenum }%
     \var{{\devanagarifont \numnoemph\vc\textbf{दानं}\lem \mssALL, दानश् \Ed}}% 
    \var{{\devanagarifont \numnoemph\vd\textbf{तप॰}\lem \Ed, तपः \mssCaCbCc\msNa\msNb\msNc\ \unmetr}}% 

{\devanagarifont कृच्छ्रातिकृच्छ्रं नक्तं च तप्तकृच्छ्रमयाचितम् \thinspace{\dandab} \dontdisplaylinenum }%
     \var{{\devanagarifont \numemph\va\textbf{कृच्छ्रातिकृच्छ्रं}\lem \msCa\msCb\msNa\Ed, 
कृच्छ्रादिकृच्छ्र \msCc, कृच्छ्रातिकृच्छ्र \msNb, कृच्छातिकृच्छं \msNc}}% 
    \var{{\devanagarifont \numnoemph\vb\textbf{॰याचितम्}\lem \mssALL, ॰याचितः \Ed}}% 

%Verse 6:27

{\devanagarifont चान्द्रायणं पराकं च तपः सांतपनादयः {॥ ६:२७॥} \veg\dontdisplaylinenum }%
     \var{{\devanagarifont \numnoemph\vc\textbf{चान्द्रायणं पराकं}\lem \msCa\msCc\msNb\msNc, चान्द्रायनं पराकं \msCb, 
चन्द्रायणं पराकं \msNa, चान्द्रायणवराकश् \Ed}}% 
    \var{{\devanagarifont \numnoemph\vd \lem \mssALL, तपसान्तपनादयः \msCc\Ed}}% 

\nemslokalong


\ujvers\nemsloka {
{\devanagarifont येनेदं तप तप्यते सुमनसा संसारदुःखच्छिदम् }%
  \dontdisplaylinenum}    \var{{\devanagarifont \numemph\va\textbf{तप त॰}\lem \Ed, तपस्त॰ \mssCaCbCc\msNa\msNb\msNc\ \unmetr\oo 
\textbf{॰मनसा}\lem \eme, ॰मनसः \mssCaCbCc\msNa\msNb\msNc\Ed}}% 


\nemslokab

{\devanagarifont आशापाश विमुच्य निर्मलमतिस्त्यक्त्वा जघन्यं फलम्  \danda\dontdisplaylinenum }%
     \var{{\devanagarifont \numnoemph\vb\textbf{निर्मलमति॰}\lem \mssALL, निर्मलर्मति॰ \msCb\oo 
\textbf{जघन्यं}\lem \mssALL, जगत्यं \Ed}}% 

\nemslokac

{\devanagarifont स्वर्गाकाङ्क्ष्यनृपत्वभोगविषयं सर्वान्तिकं तत्फलं }%
  \dontdisplaylinenum    \var{{\devanagarifont \numnoemph\vc\textbf{॰काङ्क्ष्य॰}\lem \mssALL, ॰कांक्ष॰ \Ed\oo 
\textbf{सर्वान्तिकं}\lem \mssALL, सर्वार्त्तिकं \msCb}}% 

%Verse 6:28


\nemslokad

{\devanagarifont जन्तुः शाश्वतजन्ममृत्युभवने तन्निष्ठसाध्यं वहेत् {॥ ६:२८॥} \veg\dontdisplaylinenum }%
     \var{{\devanagarifont \numnoemph\vd\textbf{॰भवने}\lem \mssALL, ॰भवेने \msNc\oo 
\textbf{॰साध्यं वहेत्}\lem \msCc\msNa\msNb\msNc, ॰\uncl{साध्यम्}\lk\lk\ \msCa, 
॰साध्य वहेत् \msCb, ॰साध्यं वदेत् \Ed}}% 

\vers


{\devanagarifont 
\jump
\begin{center}
\ketdanda~इति वृषसारसंग्रहे षष्ठो ऽध्यायः~\ketdanda
\end{center}
\dontdisplaylinenum\vers  }%
 
\nemslokanormal

\bekveg\szamveg
\vfill
\phpspagebreak

\versno=0\fejno=7
\thispagestyle{empty}


\vers

\centerline{\Large\devanagarifontbold [   सप्तमो ऽध्यायः  ]}{\vrule depth10pt width0pt} \fancyhead[CO]{{\footnotesize\devanagarifont वृषसारसंग्रहे  }}
\fancyhead[CE]{{\footnotesize\devanagarifont सप्तमो ऽध्यायः  }}
\fancyhead[LE]{}
\fancyhead[RE]{}
\fancyhead[LO]{}
\fancyhead[RO]{}
\szam\bek



\alalfejezet{नियमेषु दानम् (४)}
{\devanagarifont दानानि च तथेत्याहुः पञ्चधा मुनिभिः पुरा \thinspace{\dandab} \dontdisplaylinenum }%
     \var{{\devanagarifont \numemph\va\textbf{तथेत्याहुः}\lem \mssALL, तथैत्याहुः \msCb\msNa}}% 
    \lacuna{\devanagarifontsmall {\englishfont Witnesses used for this chapter: \msCa\ ff.\thinspace 203r--204r, 
                                              \msCb\ ff.\thinspace 209v--210v, 
                                              \msCc\ ff.\thinspace 279r--280v,
                                              \msNa\ ff.\thinspace 10v--11v, 
                                              \msNb\ exp.\thinspace 52 (lower--upper) -- 53 (lower),
                                              \msNc\ ff.\thinspace 218v--219v,
                                              \Ed\ pp.\thinspace 601--603; 
                                              \mssCaCbCc\ = \msCa + \msCb + \msCc} }%
  
%Verse 7:1

{\devanagarifont अन्नं वस्त्रं हिरण्यं च भूमि गोदान पञ्चमम् {॥ ७:१॥} \veg\dontdisplaylinenum }%
     \var{{\devanagarifont \numnoemph\vc\textbf{वस्त्रं}\lem \mssALL, वस्त्र \msCc\msNb}}% 


\alalalfejezet{अन्नदानम्}

{\devanagarifont अन्नात्तेजः स्मृतिः प्राणः अन्नात्पुष्टिर्वपुः सुखम् \thinspace{\dandab} \dontdisplaylinenum }%
     \var{{\devanagarifont \numemph\va \lem \mssCaCbCc\msNapcorr\msNb, अन्नात्तेजः स्मृतिः प्राण \msNaacorr, 
अन्नात्तेजः स्मृति प्राणः \msNc, 
अन्नाद्भवन्ति भूतानि \Ed}}% 

%Verse 7:2

{\devanagarifont अन्नाच्छ्रीः कान्ति वीर्यं च अन्नात्सत्त्वं च जायते {॥ ७:२॥} \veg\dontdisplaylinenum }%
     \var{{\devanagarifont \numnoemph\vc\textbf{अन्नाच्छ्रीः}\lem \mssALL, अन्नाच्छ्री \msNb\Ed\oo 
\textbf{कान्ति वीर्यं च}\lem \msCb\msCc\msNa\msNb, कान्तिर्वीर्यञ्च \msCa\msNc\ \unmetr, 
कान्तिवीर्श्यञ्च \Ed}}% 
    \var{{\devanagarifont \numnoemph\vd\textbf{अन्नात्सत्त्वं च}\lem \mssALL, 
अन्ना सत्वञ्च \msCc, अन्नात्सत्त्वश्च \Ed\oo 
\textbf{जायते}\lem \mssALL, जाय\lk\  \msCa}}% 

{\devanagarifont अन्नाज्जीवन्ति भूतानि अन्नं तुष्टिकरं सदा \thinspace{\dandab} \dontdisplaylinenum }%
     \var{{\devanagarifont \numemph\va\textbf{अन्नाज्जी॰}\lem \msCa\msNa\msNb\Ed, अन्ना जी॰ \msCb\msCc\msNc}}% 
    \var{{\devanagarifont \numnoemph\vb\textbf{अन्नं}\lem \mssALL, अन्नां \msCc, अन्ना \msNb\oo 
\textbf{॰करं}\lem \mssALL, ॰करः \msCc\Ed}}% 

%Verse 7:3

{\devanagarifont आन्नात्कामो मदो दर्पः अन्नाच्छौर्यं च जायते {॥ ७:३॥} \veg\dontdisplaylinenum }%
     \var{{\devanagarifont \numnoemph\vc\textbf{दर्पः}\lem \msCa\msCc\msNa\msNb, दर्प्प \msCb\msNc, दर्प्पो \Ed}}% 
    \var{{\devanagarifont \numnoemph\vd\textbf{अन्नाच्छौर्यं च}\lem \msCa\msCc\msNc, अन्नात्सौर्यञ्च \msCb\msNa\msNb, 
अन्नाच्छौर्यश्च \Ed}}% 

{\devanagarifont अन्नं क्षुधातृषाव्याधीन्सद्य एव विनाशयेत् \thinspace{\dandab} \dontdisplaylinenum }%
     \var{{\devanagarifont \numemph\va\textbf{अन्नं क्षु॰}\lem \msCa\msCb\msNapcorr\msNc, अन्ना क्षु॰ \msCc\msNaacorr, अन्नात्क्षु॰ \msNb\Ed}}% 
    \var{{\devanagarifont \numnoemph\vab\textbf{॰व्याधीन्स॰}\lem \msCb\msNc, ॰व्याधान्स॰ \msCa\msCc\msNb, ॰वाधान्स॰ \msNa, 
॰व्याधा स॰ \Ed}}% 
    \var{{\devanagarifont \numnoemph\vb\textbf{विनाशयेत्}\lem \mssALL, विशयेत् \msCb}}% 

%Verse 7:4

{\devanagarifont अन्नदानाच्च सौभाग्यं ख्यातिः कीर्तिश्च जायते {॥ ७:४॥} \veg\dontdisplaylinenum }%
 
{\devanagarifont अन्नदः प्राणदश्चैव प्राणदश्चापि सर्वदः \thinspace{\dandab} \dontdisplaylinenum }%
     \var{{\devanagarifont \numemph\va\textbf{अन्नदः}\lem \mssALL, अन्नद \Ed}}% 
    \var{{\devanagarifont \numnoemph\vb\textbf{प्राणदश्चापि}\lem \mssALL, प्राणश्चापि \msNb\oo 
\textbf{सर्वदः}\lem \mssALL, सर्वदाः \msCc}}% 
    \paral{{\devanagarifontsmall \vo {\englishfont \similar\ \SDHU\ 1.27:}
                 अन्नदः प्राणदः प्रोक्तः प्राणदश्चापि सर्वदः\thinspace{\devanagarifontsmall ।}
                 तस्मादन्नप्रदानेन सर्वदानफलं लभेत्\thinspace{\devanagarifontsmall ॥}
                 \similar\ {\englishfont \MBH\ suppl 14.4.2285--86:}
                 अन्नदः प्राणदो लोके प्राणदः सर्वदो भवेत्\thinspace{\devanagarifontsmall ।}
                 तस्मादन्नं विशेषेण दातव्यं भूतिमिच्छता\thinspace{\devanagarifontsmall ॥}
                   \similar\ {\englishfont \NARADAP\ 1.13.71:}
                 अन्नदः प्राणदः प्रोक्तः प्राणदश्चापि सर्वदः\thinspace{\devanagarifontsmall ।}
                 सर्वदानफलं यस्मादन्नदस्य नृपोत्तम\thinspace{\devanagarifontsmall ॥} }}

%Verse 7:5

{\devanagarifont तस्मादन्नसमं दानं न भूतं न भविष्यति {॥ ७:५॥} \veg\dontdisplaylinenum }%
     \var{{\devanagarifont \numnoemph\vd\textbf{भूतं}\lem \msCc\msNa\msNb\msNc, \lac  तन् \msCa, भूते \msCb, भूतो \Ed}}% 
    \paral{{\devanagarifontsmall \vcd {\englishfont  = \SDHU\ 7.31cd \similar\ \MBH\ 13.62.6ab: 
                                         }अन्नेन सदृशं दानं न भूतं न भविष्यति }}


\alalalfejezet{वस्त्रदानम्}

{\devanagarifont वस्त्राभावान्मनुष्यस्य श्रियादपि परित्यजेत् \thinspace{\dandab} \dontdisplaylinenum }%
     \var{{\devanagarifont \numemph\va\textbf{॰भावान्म॰}\lem \mssALL, ॰भावात्म॰ \msNa\msNc}}% 
    \var{{\devanagarifont \numnoemph\vb\textbf{श्रियादपि}\lem \mssALL, प्रियादपि \msCb, श्रिया वापि \msNc}}% 

%Verse 7:6

{\devanagarifont वस्त्रहीनो न पूज्येत भार्यापुत्रसखादिभिः {॥ ७:६॥} \veg\dontdisplaylinenum }%
 
{\devanagarifont विद्यावान्सुकुलीनो ऽपि ज्ञानवान्गुणवानपि \thinspace{\dandab} \dontdisplaylinenum }%
 
%Verse 7:7

{\devanagarifont वस्त्रहीनः पराधीनः परिभूतः पदे पदे {॥ ७:७॥} \veg\dontdisplaylinenum }%
 
{\devanagarifont अपमानमवज्ञां च वस्त्रहीनो ह्यवाप्नुयात् \thinspace{\dandab} \dontdisplaylinenum }%
     \var{{\devanagarifont \numemph\va\textbf{॰वज्ञां}\lem \mssALL, ॰वज्ञं \Ed}}% 
    \var{{\devanagarifont \numnoemph\vb\textbf{॰हीनो}\lem \mssALL, ॰ही \msCb}}% 

%Verse 7:8

{\devanagarifont जुगुप्सति महात्मापि सभास्त्रीजनसंसदि {॥ ७:८॥} \veg\dontdisplaylinenum }%
 
{\devanagarifont तस्माद्वस्त्रप्रदानानि प्रशंसन्ति मनीषिणः \thinspace{\dandab} \dontdisplaylinenum }%
 
%Verse 7:9

{\devanagarifont न जीर्णं स्फुटितं दद्याद्वस्त्रं कुत्सितमेव वा {॥ ७:९॥} \veg\dontdisplaylinenum }%
     \var{{\devanagarifont \numemph\vc\textbf{जीर्णं स्फुटितं}\lem \mssALL, जीर्णस्फटितं \msNb\Ed}}% 
    \var{{\devanagarifont \numnoemph\vd\textbf{कुत्सितमेव वा}\lem \mssALL, कुत्सितमेव च \msCc, कुत्सित्मेव वा \msNc}}% 

{\devanagarifont नवं पुराणरहितं मृदु सूक्ष्मं सुशोभनम् \thinspace{\dandab} \dontdisplaylinenum }%
     \var{{\devanagarifont \numemph\vb\textbf{सूक्ष्मं}\lem \mssALL, सूक्ष्म \msCc, शुक्लं \Ed}}% 

%Verse 7:10

{\devanagarifont सुसंस्कृत्य प्रदातव्यं श्रद्धाभक्तिसमन्वितम् {॥ ७:१०॥} \veg\dontdisplaylinenum }%
     \var{{\devanagarifont \numnoemph\vc\textbf{॰दातव्यं}\lem \mssALL, ॰दातव्य \msCc}}% 
    \var{{\devanagarifont \numnoemph\vd\textbf{॰समन्वितम्}\lem \mssALL, ॰तं \msNaacorr}}% 

{\devanagarifont श्रद्धासत्त्वविशेषेण देशकालविधेन च \thinspace{\dandab} \dontdisplaylinenum }%
     \var{{\devanagarifont \numemph\va\textbf{॰सत्त्व॰}\lem \mssALL, ॰स च॰ \Ed}}% 

%Verse 7:11

{\devanagarifont पात्रद्रव्यविशेषेण फलमाहुः पृथक्पृथक् {॥ ७:११॥} \veg\dontdisplaylinenum }%
     \paral{{\devanagarifontsmall \vo {\englishfont \compare\ \MANU\ 7.86--87 (the latter usually labelled as an additional verse):}
                         पात्रस्य हि विशेषेण श्रद्दधानतयाइव च\thinspace{\devanagarifontsmall ।} 
                         अल्पं वा बहु वा प्रेत्य दानस्य फलमश्नुते\thinspace{\devanagarifontsmall ॥}
                         देशकालविधानेन द्रव्यं श्रद्धासमन्वितम्\thinspace{\devanagarifontsmall ।}
                         पात्रे प्रदीयते यत्तु तद्धर्मस्य प्रसाधनम्\thinspace{\devanagarifontsmall ॥} }}

{\devanagarifont यादृशं दीयते वस्त्रं तादृशं प्राप्यते फलम् \thinspace{\dandab} \dontdisplaylinenum }%
 
{\devanagarifont जीर्णवस्त्रप्रदानेन जीर्णवस्त्रमवाप्नुयात्  \danda\dontdisplaylinenum }%
 
%Verse 7:12

{\devanagarifont शोभनं दीयते वस्त्रं शोभनं वस्त्रमाप्नुयात् {॥ ७:१२॥} \veg\dontdisplaylinenum }%
     \var{{\devanagarifont \numemph\vef \lem \mssALL, 
\om\ \msNb}}% 

\nemslokalong


\ujvers\nemsloka {
{\devanagarifont दद्याद्वस्त्र सुशोभनं द्विजवरे काले शुभे सादरं }%
  \dontdisplaylinenum}    \var{{\devanagarifont \numemph\va\textbf{द्विजवरे काले शुभे}\lem \mssALL, द्विजयिने एकाशुभं \Ed}}% 


\nemslokab

{\devanagarifont सौभाग्यमतुलं लभेत स नरो रूपं तथा शोभनम्  \danda\dontdisplaylinenum }%
     \var{{\devanagarifont \numnoemph\vb\textbf{नरो}\lem \mssALL, दरो \msCb}}% 

\nemslokac

{\devanagarifont तस्मिन्याति सुवस्त्रकोटि शतशः प्राप्नोति निःसंशयं }%
  \dontdisplaylinenum    \var{{\devanagarifont \numnoemph\vc\textbf{तस्मिन्याति}\lem \mssALL, त\uncl{स्मा}न्याति \msNa\oo 
\textbf{सुवस्त्र॰}\lem \mssALL, स वस्त्र॰ \Ed\oo 
\textbf{॰संशयम्}\lem \msCa\msCb\msNc, ॰संशयः \msCc\msNa\msNb\Ed}}% 

%Verse 7:13


\nemslokad

{\devanagarifont तस्मात्त्वं कुरु वस्त्रदानमसकृत्पारत्रिकोत्कर्षणम् {॥ ७:१३॥} \veg\dontdisplaylinenum }%
     \var{{\devanagarifont \numnoemph\vd\textbf{दानमसकृत्पा॰}\lem \mssALL, दानसत्पा॰ \msNb}}% 

\nemslokanormal



\alalalfejezet{सुवर्णदानम्}

\vers


{\devanagarifont सुवर्णदानं विप्रेन्द्र संक्षिप्य कथयाम्यहम् \thinspace{\dandab} \dontdisplaylinenum }%
     \var{{\devanagarifont \numemph\va\textbf{॰दानं}\lem \mssALL, ॰दान \msNb\Ed}}% 

%Verse 7:14

{\devanagarifont पवित्रं मङ्गलं पुण्यं सर्वपातकनाशनम् {॥ ७:१४॥} \veg\dontdisplaylinenum }%
     \var{{\devanagarifont \numnoemph\vd\textbf{॰पातक॰}\lem  \mssALL, ॰पापक॰ \msCa}}% 

{\devanagarifont धारयेत्सततं विप्र सुवर्णकटकाङ्गुलिम् \thinspace{\dandab} \dontdisplaylinenum }%
     \var{{\devanagarifont \numemph\vb\textbf{॰कटकाङ्गुलिम्}\lem \mssALL, ॰क\lk\lk गुलिम् \msCa, ॰कटकाङ्गुलम् \msNb}}% 

%Verse 7:15

{\devanagarifont मुच्यते सर्वपापेभ्यो राहुणा चन्द्रमा यथा {॥ ७:१५॥} \veg\dontdisplaylinenum }%
     \paral{{\devanagarifontsmall \vcd {\englishfont = 22.38 below = a line inserted after \MBH\ 1.56.18 in some manuscripts as indicated in 
                     the critical edition} }}

{\devanagarifont दत्त्वा सुवर्णं विप्रेभ्यो देवेभ्यश्च द्विजर्षभ \thinspace{\dandab} \dontdisplaylinenum }%
     \var{{\devanagarifont \numemph\va\textbf{सुवर्णं}\lem \mssALL, सुवर्ण \msNb}}% 
    \var{{\devanagarifont \numnoemph\vb\textbf{॰र्षभ}\lem \mssALL, ॰र्षभः \msCc\msNb}}% 

%Verse 7:16

{\devanagarifont तुटिमात्रे ऽपि यो दद्यात्सर्वपापैः प्रमुच्यते {॥ ७:१६॥} \veg\dontdisplaylinenum }%
     \var{{\devanagarifont \numnoemph\vc\textbf{तुटि॰}\lem \mssALL, त्रुटि॰ \Ed\oo 
\textbf{॰मात्रे}\lem \mssALL, ॰मात्रो \msNa\Ed}}% 
    \var{{\devanagarifont \numnoemph\vd \lem \mssALL, 
सर्वपापैः स मुच्यते \msCa, सर्वपापै प्रमुच्यते \Ed}}% 

{\devanagarifont रक्तिमाषककर्षं वा पलार्धं पलमेव वा \thinspace{\dandab} \dontdisplaylinenum }%
     \var{{\devanagarifont \numemph\va\textbf{रक्तिमाषक॰}\lem \msNcacorr, रन्तिमाषक॰ \msCa, रत्तिमाषक॰ \msCb\msNa\msNcpcorr, 
रन्तिम्मान्सक॰ \msCc, रत्तिमान्सक॰ \msNb, रत्तिमाषक॰ \Ed}}% 
    \var{{\devanagarifont \numnoemph\vb\textbf{॰र्धं}\lem \msCa\msCb\msNc\Ed, ॰द्ध \msCc\msNa\msNb}}% 

%Verse 7:17

{\devanagarifont एवमेव फलंवृद्धिर्ज्ञेया दानविशेषतः {॥ ७:१७॥} \veg\dontdisplaylinenum }%
     \var{{\devanagarifont \numnoemph\vcd\textbf{॰वृद्धिर्ज्ञेया}\lem \msCa\Ed, ॰वृद्धि ज्ञेया \msCb\msCc\msNa\msNb, ॰वृर्द्धि ज्ञेया \msNc}}% 


\alalalfejezet{भूमिदानम्}

{\devanagarifont सर्वाधारं महीदानं प्रशंसन्ति मनीषिणः \thinspace{\dandab} \dontdisplaylinenum }%
     \var{{\devanagarifont \numemph\va\textbf{॰धारं}\lem \msCb, ॰धार॰ \msCa\msCc\msNa\msNb\msNc\Ed}}% 
    \var{{\devanagarifont \numnoemph\vab\textbf{॰दानं प्रशंसन्ति}\lem \mssALL, 
दा\lk \uncl{नम्प्र}\lac  सन्ति \msCa}}% 

%Verse 7:18

{\devanagarifont अन्नवस्त्रहिरण्यादि सर्वं वै भूमिसम्भवम् {॥ ७:१८॥} \veg\dontdisplaylinenum }%
     \var{{\devanagarifont \numnoemph\vd\textbf{सर्वं वै}\lem \mssALL, सर्वं \uncl{वे} \msCa\ \toplost}}% 

{\devanagarifont भूमिदानेन विप्रेन्द्र सर्वदानफलं लभेत् \thinspace{\dandab} \dontdisplaylinenum }%
     \var{{\devanagarifont \numemph\vb\textbf{॰फलं लभेत्}\lem \mssALL, 
॰ललं भवेत् \msNbacorr, ॰लं भवेत् \msNc}}% 

%Verse 7:19

{\devanagarifont भूमिदानसमं विप्र यद्यस्ति वद तत्त्वतः {॥ ७:१९॥} \veg\dontdisplaylinenum }%
 
{\devanagarifont मातृकुक्षिविमुक्तस्तु धरणीशरणो भवेत् \thinspace{\dandab} \dontdisplaylinenum }%
     \var{{\devanagarifont \numemph\va\textbf{॰मुक्तस्तु}\lem \mssALL, ॰मुक्तिस्तु \Ed}}% 
    \var{{\devanagarifont \numnoemph\vb\textbf{॰शरणो}\lem \mssALL, ॰शरण \msNc, ॰शरणां \Ed}}% 

%Verse 7:20

{\devanagarifont चराचराणां सर्वेषां भूमिः साधारणा स्मृता {॥ ७:२०॥} \veg\dontdisplaylinenum }%
 
{\devanagarifont एकहस्तं द्विहस्तं वा पञ्चाशच्छतमेव वा \thinspace{\dandab} \dontdisplaylinenum }%
     \var{{\devanagarifont \numemph\va\textbf{एकहस्तं}\lem \msCb\msNa\msNb\msNc, एकहस्त॰ \msCa\msCc\Ed}}% 

%Verse 7:21

{\devanagarifont सहस्रायुतलक्षं वा भूमिदानं प्रशस्यते {॥ ७:२१॥} \veg\dontdisplaylinenum }%
     \var{{\devanagarifont \numnoemph\vd \lem \mssALL, भूमिदान प्रशस्यते \msCb, 
पञ्चाशच्छतमेव वा\thinspace{\devanagarifont ।} सहायुतलक्षम्वा भूमिदं प्रशस्यते \msNb\ {\englishfont (eyeskip)}}}% 

{\devanagarifont एकहस्तां च यो भूमिं दद्याद्द्विजवराय तु \thinspace{\dandab} \dontdisplaylinenum }%
     \var{{\devanagarifont \numemph\va\textbf{॰हस्तां च}\lem \mssALL, ॰हस्तञ्च \msCb\msNb}}% 
    \var{{\devanagarifont \numnoemph\vb\textbf{दद्याद्द्वि॰}\lem \mssALL, दद्या द्वि॰ \Ed}}% 

%Verse 7:22

{\devanagarifont वर्षकोटिशतं दिव्यं स्वर्गलोके महीयते {॥ ७:२२॥} \veg\dontdisplaylinenum }%
 
{\devanagarifont एवं बहुषु हस्तेषु गुणागुणि फलं स्मृतम् \thinspace{\dandab} \dontdisplaylinenum }%
     \var{{\devanagarifont \numemph\vb\textbf{गुणागुणि॰}\lem \mssALL, गुणागणि॰ \Ed}}% 

%Verse 7:23

{\devanagarifont श्रद्धाधिकं फलं दानं कथितं ते द्विजोत्तम {॥ ७:२३॥} \veg\dontdisplaylinenum }%
     \var{{\devanagarifont \numnoemph\vc\textbf{॰धिकं}\lem \msCb\msCc\msNa\msNb, ॰धिक॰ \msCa\msNc\Ed}}% 
    \var{{\devanagarifont \numnoemph\vd\textbf{॰त्तम}\lem \mssALL, ॰त्तमः \msNc}}% 

{\devanagarifont जामदग्न्येन रामेण भूमिं दत्त्वा द्विजाय वै \thinspace{\dandab} \dontdisplaylinenum }%
     \var{{\devanagarifont \numemph\va\textbf{जामदग्न्येन}\lem \msCb\msNa\msNc, जामदग्न्ये\lk\ \msCa, जामदग्नेन \msCc\msNb\Ed\oo 
\textbf{रामेण}\lem \msCb\msNc\Ed, \lk\lk ण \msCa, रामेन \msCc\msNa\msNb}}% 
    \var{{\devanagarifont \numnoemph\vb\textbf{दत्त्वा द्वि॰}\lem \mssALL, दद्याद्द्वि॰ \msCb}}% 

%Verse 7:24

{\devanagarifont आयुरक्षयमाप्तं तु इहैव च द्विजोत्तम {॥ ७:२४॥} \veg\dontdisplaylinenum }%
     \var{{\devanagarifont \numnoemph\vd\textbf{च}\lem \mssALL, हि \Ed}}% 


\alalalfejezet{गोदानम्}

{\devanagarifont हेमशृङ्गां रौप्यक्षुरां चैलघण्टां द्विजोत्तम \thinspace{\dandab} \dontdisplaylinenum }%
     \var{{\devanagarifont \numemph\va\textbf{॰शृङ्गां}\lem \mssALL, ॰शृङ्गं \msNa, \om\ \msNb\oo 
\textbf{रौप्य॰}\lem \mssALL, रोप्यं \msNc\oo 
\textbf{॰क्षुरां}\lem \mssALL, ॰खुरां \msCc\Ed}}% 
    \lacuna{\devanagarifontsmall \vab {\englishfont Omitted in \msNb} }%
      \paral{{\devanagarifontsmall \vab {\englishfont \similar\ \VAGMATI\ 17.33ab:}
                         हेमशृङ्गां रौप्यखुरां चैलघण्टावलम्बिनीम्\thinspace{\devanagarifontsmall ।} }}

%Verse 7:25

{\devanagarifont विप्राय वेदविदुषे दत्त्वानन्तफलं स्मृतम् {॥ ७:२५॥} \veg\dontdisplaylinenum }%
     \var{{\devanagarifont \numnoemph\vd\textbf{दत्त्वानन्त॰}\lem \mssALL, दत्त्वान्त॰ \Ed}}% 
    \paral{{\devanagarifontsmall \vo {\englishfont \compare, e.g., \MBH\ 7.58.18:}
                 तथा गाः कपिला दोग्ध्रीः सर्षभाः पाण्डुनन्दनः\thinspace{\devanagarifontsmall ।}
                 हेमशृङ्गी रूप्यखुरा दत्त्वा चक्रे प्रदक्षिणम्\thinspace{\devanagarifontsmall ॥}
                       {\englishfont and \BHAVP\ Uttara 12.25:}
                 हेमशृंगीं रौप्यखुरां सघंटां कांस्यदोहनाम्\thinspace{\devanagarifontsmall ।} 
                 महादेवाय गां दद्याद्दीक्षिताय द्विजाय वै\thinspace{\devanagarifontsmall ॥} }}


\alalalfejezet{दानप्रशंसा}

\nemslokalong


\ujvers\nemsloka {
{\devanagarifont दानाभ्यासरतः प्रवर्तनभवां शक्यानुरूपं सदा }%
  \dontdisplaylinenum}    \var{{\devanagarifont \numemph\va\textbf{॰रूपं}\lem \mssALL, ॰रूप \msNb}}% 


\nemslokab

{\devanagarifont अन्नं वस्त्रहिरण्यरौप्यमुदकं गावस्तिलान्मेदिनीम्  \danda\dontdisplaylinenum }%
     \var{{\devanagarifont \numnoemph\vb\textbf{॰रौप्य॰}\lem \mssALL, ॰रोप्य॰ \msCb, ॰\uncl{रौप्य}॰ \msNc\oo 
\textbf{गावस्तिलान्मे॰}\lem \eme, गावस्तिलाम्मे॰ \msCa\msCc\msNc, गावस्तिला मे॰ \msCb\msNa, 
गावन्तिला मे॰ \msNb, गावस्तिलं मे॰ \Ed}}% 

\nemslokac

{\devanagarifont दद्यात्पादुकछत्त्रपीठकलशं पात्राद्यमन्यच्च वा }%
  \dontdisplaylinenum    \var{{\devanagarifont \numnoemph\vc\textbf{दद्यात्पा॰}\lem \mssALL, दद्या पा॰ \msNb\oo 
\textbf{पात्राद्यमन्यच्च वा}\lem \mssALL, 
पत्राद्यमन्यच्च वा \msCb, पात्रेषु लब्धेषु वै \Ed}}% 

%Verse 7:26


\nemslokad

{\devanagarifont श्रद्धादानमभिन्नरागवदनं कृत्वा मनो निर्मलम् {॥ ७:२६॥} \veg\dontdisplaylinenum }%
     \var{{\devanagarifont \numnoemph\vd\textbf{श्रद्धादान॰}\lem \mssALL, दत्त्वादान॰ \Ed}}% 

\nemslokalong


\ujvers\nemsloka {
{\devanagarifont दानादेव यशः श्रियः सुखकराः ख्यातिमतुल्यां लभेद् }%
  \dontdisplaylinenum}    \var{{\devanagarifont \numemph\va\textbf{यशः}\lem \msCb\msNc\Ed, यश \msCa\msCc\msNa\msNb\oo 
\textbf{सुखकराः}\lem \mssALL, सुखकर \msNcpcorr\oo 
\textbf{ख्यातिमतुल्यां}\lem \eme, ख्यातिश्च तुल्यं \mssCaCbCc\msNa\msNb\msNc\Ed\oo 
\textbf{लभेद्}\lem \mssALL, भवेत् \msNc\Ed}}% 


\nemslokab

{\devanagarifont दानादेव निगर्हणं रिपुगणे आनन्ददं सौख्यदम्  \danda\dontdisplaylinenum }%
     \var{{\devanagarifont \numnoemph\vb\textbf{निगर्हणं}\lem \msCapcorr\msCc\msNa\Ed, निर्हणं \msCaacorr, निवर्हणं \msCb\msNc, 
निगर्हन \msNb\oo 
\textbf{॰गणे आनन्ददं सौख्यदम्}\lem \mssALL, 
॰गणै आनन्ददं सौख्यदम् \msCc, 
॰गणैश्चानन्दसौख्यप्रदम्  \Ed}}% 

\nemslokac

{\devanagarifont दानादूर्जयता प्रसादमतुलं सौभाग्य दानाल्लभेद् }%
  \dontdisplaylinenum    \var{{\devanagarifont \numnoemph\vc\textbf{दानादूर्जयता}\lem \mssALL, दानादूर्जयतां \msNa, दानाद्दु॰ \Ed\oo 
\textbf{प्रसाद॰}\lem \mssALL, प्रासाद॰ \msNa\oo 
\textbf{सौभाग्य}\lem \mssALL, सौगाग्य \msCb, सौभाग्यं \Ed\ \unmetr\oo 
\textbf{दानाल्लभेद्}\lem \msCb\Ed, दानं लभेत् \msCa\msCc\msNa\msNb\msNc}}% 

%Verse 7:27


\nemslokad

{\devanagarifont दानादेव अनन्तभोग नियतं स्वर्गं च तस्माद्भवेत् {॥ ७:२७॥} \veg\dontdisplaylinenum }%
     \var{{\devanagarifont \numnoemph\vd\textbf{दानादेव}\lem \mssALL, दानादोव \msCc\oo 
\textbf{॰नियतं}\lem \mssALL, ॰नियत \msCc}}% 

\ujvers\nemsloka {
{\devanagarifont दानादेव च शक्रलोकसकलं दानाज्जनानन्दनं }%
  \dontdisplaylinenum}    \var{{\devanagarifont \numemph\va\textbf{शक्रलोकसकलं}\lem \mssALL, शत्रुलोकसकलं \msNa, शक्रलोकमतुलं \Ed\oo 
\textbf{दानाज्ज॰}\lem \mssALL, दाना ज॰ \msCa, दानार्ज॰ \msCb}}% 


\nemslokab

{\devanagarifont दानादेव महीं समस्त बुभुजे सम्राड्महीमण्डले  \danda\dontdisplaylinenum }%
     \var{{\devanagarifont \numnoemph\vb\textbf{दानादेव}\lem \mssALL, दानेदेव \msCb\oo 
\textbf{महीं समस्त}\lem \conj, महीसमासु \msCb\msCc, महीं समांसु \msCa\msNa\msNc, 
मही समस्त \msNb, महीयसां स \Ed\oo 
\textbf{सम्राड्म॰}\lem \mssALL, संम्राड्म॰ \msCb}}% 

\nemslokac

{\devanagarifont दानादेव सुरूपयोनिसुभगश्चन्द्राननो वीक्ष्यते }%
  \dontdisplaylinenum    \var{{\devanagarifont \numnoemph\vc\textbf{सुरूप॰}\lem \mssALL, स्वरूप॰ \msNb\oo 
\textbf{॰योनिसु॰}\lem \msNb\Ed, ॰योनिस्सु॰ \msCa ॰योनिः सु॰ \msCb\msCc\msNa\msNc\oo 
\textbf{॰भगश्च॰}\lem \msCa\msCc\msNb\msNc, ॰भग च॰ \msCb\msNa\Ed\oo 
\textbf{॰न्द्राननो}\lem \msCa\msCb\msNa\Ed, ॰न्द्रानने \msCc\msNb, ॰न्द्राननौ \msNc\oo 
\textbf{वीक्ष्यते}\lem \msCb\msCc, वीक्षते \msCa\msNa\msNb\msNc, विक्षते \Ed}}% 

%Verse 7:28


\nemslokad

{\devanagarifont दानादेव अनेकसम्भवसुखं प्राप्नोति निःसंशयम् {॥ ७:२८॥} \veg\dontdisplaylinenum }%
     \var{{\devanagarifont \numnoemph\vd\textbf{निःसंशयम्}\lem \msCa\msCb\msNc, निसंशयः \msCc, निःसंशयः \msNa\Ed, निस्सयः \msNb}}% 

\vers


{\devanagarifont 
\jump
\begin{center}
\ketdanda~इति वृषसारसंग्रहे दानप्रशंसाध्यायः सप्तमः~\ketdanda
\end{center}
\dontdisplaylinenum\vers  }%
     \var{{\devanagarifont \numnoemph{\englishfont \Colo:}\textbf{॰प्रशंसाध्यायः सप्तमः}\lem \mssALL, 
॰प्रशंसाध्यायः समाप्तः \msCb, 
॰प्रशंसा सप्तमो ऽध्यायः \Ed}}% 

\nemslokanormal

\bekveg\szamveg
\vfill
\phpspagebreak

\versno=0\fejno=8
\thispagestyle{empty}

\centerline{\Large\devanagarifontbold [   अष्टमो ऽध्यायः  ]}{\vrule depth10pt width0pt} \fancyhead[CO]{{\footnotesize\devanagarifont वृषसारसंग्रहे  }}
\fancyhead[CE]{{\footnotesize\devanagarifont अष्टमो ऽध्यायः  }}
\fancyhead[LE]{}
\fancyhead[RE]{}
\fancyhead[LO]{}
\fancyhead[RO]{}
\szam\bek



\alalfejezet{नियमेषु स्वाध्यायः (५)}
\vers


{\devanagarifont पञ्चस्वाध्यायनं कार्यमिहामुत्र सुखार्थिना \thinspace{\dandab} \dontdisplaylinenum }%
     \var{{\devanagarifont \numemph\va\textbf{॰स्वाध्यायनं}\lem \mssALL, 
॰स्वाध्ययनं \msNc}}% 
    \var{{\devanagarifont \numnoemph\vb\textbf{॰मुत्र}\lem \mssALL, ॰मूत्र \msPaperA\Ed\oo 
\textbf{॰र्थिना}\lem \mssALL, ॰र्थिनां \msNb}}% 
    \lacuna{\devanagarifontsmall {\englishfont Witnesses used for this chapter: \msCa\ ff.\thinspace 204r--205v, 
                                                  \msCb\ ff.\thinspace 210v--211v, 
                                                  \msCc\ ff.\thinspace 280v--282r,
                                                  \msNa\ ff.\thinspace 11v--13r, 
                                                  \msNb\ exp.\thinspace 53 (lower) -- 54 (lower),
                                                  \msNc\ ff.\thinspace 219v--221r,
                                                  \msParis\ exp.\thinspace 426--428,
                                                  \msPaperA\ ff.\thinspace 213r--214v,
                                                  \Ed\ pp.\thinspace 603--606; 
                                                  \mssCaCbCc\ = \msCa + \msCb + \msCc} }%
  
%Verse 8:1

{\devanagarifont शैवं सांख्यं पुराणं च स्मार्तं भारतसंहिताम् {॥ ८:१॥} \veg\dontdisplaylinenum }%
     \var{{\devanagarifont \numnoemph\vc\textbf{शैवं}\lem \mssALL, 
\uncl{शै}लं \msCc\oo 
\textbf{सांख्यं}\lem \msCa\msCb\msNc\msParis\msPaperA\Ed, 
शांख्य \msCc, साख्यं \msNa\msNb}}% 
    \var{{\devanagarifont \numnoemph\vd\textbf{स्मार्तं}\lem \mssALL, स्मार्त \msCc\msNb\oo 
\textbf{भारतसंहिताम्}\lem \mssALL, 
भारतसंहिताः \msNa, भारत्तसंहितां \msNc}}% 

{\devanagarifont शैवे तत्त्वं विचिन्तेत शैवपाशुपतद्वये \thinspace{\dandab} \dontdisplaylinenum }%
     \var{{\devanagarifont \numemph\va\textbf{शैवे }\lem \msCa\msCc\msNa\msNb\msNc, शैवै \msCb\msParis, शैवं \msPaperA\Ed\oo 
\textbf{तत्त्वं}\lem \mssALL, ॰तत्त्व \msParis}}% 
    \var{{\devanagarifont \numnoemph\vb\textbf{शैव॰}\lem \msParis, शैवः \msCa\msCb\msNb\msNc, शैवाः \msCc\msPaperA\Ed, शैवा \msNa\oo 
\textbf{॰द्वये}\lem \mssALL, ॰ये \msCb}}% 

%Verse 8:2

{\devanagarifont अत्र विस्तरतः प्रोक्तं तत्त्वसारसमुच्चयम् {॥ ८:२॥} \veg\dontdisplaylinenum }%
     \var{{\devanagarifont \numnoemph\vd\textbf{॰सारसमुच्चयम्}\lem \mssALL, 
॰सारं समुच्चयम् \msNa, ॰सारं समुद्ययं \msNb}}% 

{\devanagarifont संख्यातत्त्वं तु सांख्येषु बोद्धव्यं तत्त्वचिन्तकैः \thinspace{\dandab} \dontdisplaylinenum }%
     \var{{\devanagarifont \numemph\va\textbf{संख्यातत्त्वं तु}\lem \msNa\msNc\msParis\msPaperA, 
सं\uncl{ख्या}\lk\lk\lk\ \msCa, संख्यातत्त्वं \msCb, 
शाङ्ख्यातत्वं तु \msCc, सख्यतत्वन्तु \msNb, संख्यातत्त्व तु \Ed\oo 
\textbf{सांख्येषु}\lem \mssALL, सख्येषु \msNb}}% 

%Verse 8:3

{\devanagarifont पञ्चतत्त्वविभागेन कीर्तितानि महर्षिभिः {॥ ८:३॥} \veg\dontdisplaylinenum }%
     \var{{\devanagarifont \numnoemph\vc\textbf{॰तत्त्व॰}\lem \mssALL, 
॰तत्वा॰ \msCb, \om\ \msNb}}% 

{\devanagarifont पुराणेषु महीकोषो विस्तरेण प्रकीर्तितः \thinspace{\dandab} \dontdisplaylinenum }%
 
%Verse 8:4

{\devanagarifont अधोर्ध्वमध्यतिर्यं च यत्नतः सम्प्रवेशयेत् {॥ ८:४॥} \veg\dontdisplaylinenum }%
     \var{{\devanagarifont \numemph\vc\textbf{अधोर्ध्व॰}\lem \mssALL, अधोर्ध्वं \msNb\oo 
\textbf{॰मध्य॰}\lem \mssALL, ॰मध॰ \msCc}}% 
    \var{{\devanagarifont \numnoemph\vd\textbf{यत्नतः}\lem \mssALL, यत्नत \msNb\oo 
\textbf{सम्प्रवेशयेत्}\lem \mssALL, 
सम्प्रबोधयेत् \Ed}}% 

{\devanagarifont स्मार्तं वर्णाश्रमाचारं धर्मन्यायप्रवर्तनम् \thinspace{\dandab} \dontdisplaylinenum }%
     \var{{\devanagarifont \numemph\va\textbf{स्मार्तं वर्णा॰}\lem \msCa, तस्मार्त्तम्वर्ण्णा॰ \msCb, 
स्मार्तवर्णा॰ \msCc\msNa\msNb\msNc\msPaperA\Ed, 
स्मार्त्तं वर्ण्ण॰ \msParis}}% 
    \var{{\devanagarifont \numnoemph\vb\textbf{धर्म॰}\lem \mssALL, धर्मं \msCc\oo 
\textbf{॰वर्तनम्}\lem \mssALL, 
॰व\lk नं \msParis, ॰वर्तन \Ed}}% 

%Verse 8:5

{\devanagarifont शिष्टाचारो ऽविकल्पेन ग्राह्यस्तत्र अशङ्कितः {॥ ८:५॥} \veg\dontdisplaylinenum }%
     \var{{\devanagarifont \numnoemph\vc\textbf{शिष्टा॰}\lem \mssALL, शिष्ट॰ \msPaperA\oo 
\textbf{॰चारो}\lem \msCa\msCb\msNb\msNc\msPaperA, 
॰चार॰ \msCc\Ed, 
॰चारा \msNa, 
॰चा\uncl{रो}॰ \msParis}}% 
    \var{{\devanagarifont \numnoemph\vd \lem \mssALL, 
ग्राह्यस्त\lk\lk\lk ङ्कितः \msCa}}% 

{\devanagarifont इतिहासमधीयानः सर्वज्ञः स नरो भवेत् \thinspace{\dandab} \dontdisplaylinenum }%
     \var{{\devanagarifont \numemph\vb\textbf{॰ज्ञः}\lem \mssALL, ॰ज्ञ \msCc}}% 

%Verse 8:6

{\devanagarifont धर्मार्थकाममोक्षेषु संशयस्तेन छिद्यते {॥ ८:६॥} \veg\dontdisplaylinenum }%
 

\alalfejezet{नियमेष्वुपस्थनिग्रहः (६)}
{\devanagarifont शृणुष्वावहितो विप्र पञ्चोपस्थविनिग्रहम् \thinspace{\dandab} \dontdisplaylinenum }%
     \var{{\devanagarifont \numemph\vb\textbf{॰ग्रहम्}\lem \mssALL, 
॰ग्र\uncl{हः} \msNa}}% 

{\devanagarifont स्त्रियो वा गर्हितोत्सर्गः स्वयंमुक्तिश्च कीर्त्यते  \danda\dontdisplaylinenum }%
     \var{{\devanagarifont \numnoemph\vc\textbf{गर्हितोत्सर्गः}\lem \msCa\msCb\msNb\msNc\msParis, 
गर्हितस्सर्ग्गः \msCc, गर्हितो विप्र \msNa, 
गर्हितो स्वर्गः \msPaperA\Ed}}% 
    \var{{\devanagarifont \numnoemph\vd\textbf{स्वयं॰}\lem \mssALL, स्वय॰ \msCb\oo 
\textbf{कीर्त्यते}\lem \mssALL, 
की\uncl{र्त्स्य}ते \msCc}}% 

%Verse 8:7

{\devanagarifont स्वप्नोपघातं विप्रेन्द्र दिवास्वप्नं च पञ्चमः {॥ ८:७॥} \veg\dontdisplaylinenum }%
     \var{{\devanagarifont \numnoemph\ve\textbf{॰घातं}\lem \mssALL, 
॰घात \msCc\Ed}}% 


\alalalfejezet{स्त्रियः}

{\devanagarifont अगम्या स्त्री दिवा पर्वे धर्मपत्न्यपि वा भवेत् \thinspace{\dandab} \dontdisplaylinenum }%
     \var{{\devanagarifont \numemph\va\textbf{स्त्री दिवा पर्वे}\lem \msCb\msCc\msNa\msNb\msNc\msPaperA, 
\lk  दिवा पर्व्वे \msCa, \lk\lk \lk  पर्वे \msParis, स्त्री दिवापूर्वे \Ed}}% 
    \var{{\devanagarifont \numnoemph\vb\textbf{॰पत्न्यपि}\lem \mssALL, 
॰पत्नी पि \msCc}}% 

%Verse 8:8

{\devanagarifont विरुद्धस्त्रीं न सेवेत वर्णभ्रष्टाधिकासु च {॥ ८:८॥} \veg\dontdisplaylinenum }%
     \var{{\devanagarifont \numnoemph\vc\textbf{विरुद्धस्त्रीं न}\lem \msPaperA, विरुद्धस्त्री न \mssCaCbCc\msNb\msNc, 
विरुद्धस्त्री नि॰ \msNa\msParis, द्विरुद्धास्त्रीन्न \Ed}}% 
    \var{{\devanagarifont \numnoemph\vd\textbf{॰धिकासु च}\lem \msCa\msCb\msNa\msParis\msPaperA, ॰धिकासु त \msCc, ॰दिकाषु च \msNb, 
॰विकाषु च \msNc, ॰पिकासु च \Ed}}% 


\alalalfejezet{गर्हितोत्सर्गः}

{\devanagarifont अजमेषगवादीनां वडवामहिषीषु च \thinspace{\dandab} \dontdisplaylinenum }%
     \var{{\devanagarifont \numemph\va\textbf{॰मेष॰}\lem \mssALL, ॰मेय॰ \msCb}}% 

%Verse 8:9

{\devanagarifont गर्हितोत्सर्गमित्येतद्यत्नेन परिवर्जयेत् {॥ ८:९॥} \veg\dontdisplaylinenum }%
 

\alalalfejezet{स्वयंमुक्तिः}

{\devanagarifont अयोनिकषणा वापि अपानकषणापि वा \thinspace{\dandab} \dontdisplaylinenum }%
     \var{{\devanagarifont \numemph\va\textbf{अयोनि॰}\lem \conj, अन्योन्य॰ \mssCaCbCc\msNa\msNb\msNc\msParis\msPaperA\Ed\oo 
\textbf{॰कषणा}\lem \msCa\msNa, ॰कर्षणा \msCb\msCc\msNb\msNc\msParis\msPaperA\Ed}}% 
    \var{{\devanagarifont \numnoemph\vb\textbf{॰कषणापि}\lem \mssCaCbCc\msNa, ॰कर्षणापि \msNb\msNc\msParis\msPaperA\Ed}}% 

%Verse 8:10

{\devanagarifont स्वयंमुक्तिरियं ज्ञेया तस्मात्तां परिवर्जयेत् {॥ ८:१०॥} \veg\dontdisplaylinenum }%
     \var{{\devanagarifont \numnoemph\vc\textbf{स्वयंमुक्ति॰}\lem \mssALL, 
स्वयमुक्ति॰ \msCb\oo 
\textbf{ज्ञेया}\lem \mssALL, 
ज्ञेयां \msNb}}% 
    \var{{\devanagarifont \numnoemph\vd\textbf{तस्मात्तां}\lem \msCa\msCb\msNa\msNc\msParis\msPaperA, 
तस्मात्तं \msCc, तस्मार्त्ता \msNb, तस्मात्स्त्री \Ed}}% 


\alalalfejezet{स्वप्नघातम्}

{\devanagarifont स्वप्नघातं द्विजश्रेष्ठ अनिष्टं पण्डितैः सदा \thinspace{\dandab} \dontdisplaylinenum }%
     \var{{\devanagarifont \numemph\va\textbf{स्वप्नघा॰}\lem \mssALL, 
स्वप्नजा॰ \msParisacorr}}% 
    \var{{\devanagarifont \numnoemph\vb\textbf{पण्डितैः}\lem \mssALL, 
पण्डितै \msCc, पण्डितेः \msNc}}% 

%Verse 8:11

{\devanagarifont स्वप्ने स्त्रीषु रमन्ते च रेतः प्रक्षरते ततः {॥ ८:११॥} \veg\dontdisplaylinenum }%
     \var{{\devanagarifont \numnoemph\vc\textbf{रमन्ते}\lem \mssALL, रमक्षन्ते \msPaperA}}% 
    \var{{\devanagarifont \numnoemph\vd\textbf{प्रक्षरते}\lem \mssALL, प्रस्खलतस् \Ed\oo 
\textbf{ततः}\lem \mssALL, तत \msCc}}% 


\alalalfejezet{दिवास्वप्नम्}

{\devanagarifont दिवाशयं न कर्तव्यं नित्यं धर्मपरेण तु \thinspace{\dandab} \dontdisplaylinenum  }%
     \var{{\devanagarifont \numemph\va\textbf{दिवाशयं न}\lem \mssCaCbCc\msParis\msPaperA\Ed, 
दिवाशयेन्न \msNa, दिवासयानं \msNb, दिवाशायं \msNc}}% 
    \var{{\devanagarifont \numnoemph\vb\textbf{नित्यं}\lem \mssALL, नित्य \msNb\oo 
\textbf{॰परेण तु}\lem \mssALL, 
॰परेन तु \msCa, ॰परेण च \msCc}}% 

%Verse 8:12

{\devanagarifont स्वर्गमार्गार्गला ह्येताः स्त्रियो नाम प्रकीर्तिताः {॥ ८:१२॥} \veg\dontdisplaylinenum }%
     \var{{\devanagarifont \numnoemph\vc\textbf{ह्येताः}\lem \msNc, ह्येता \mssCaCbCc\msNa\msNb\msParis\msPaperA\Ed}}% 
    \var{{\devanagarifont \numnoemph\vd\textbf{स्त्रियो}\lem \mssALL, स्त्रीयो \Ed\oo 
\textbf{॰कीर्तिताः}\lem \mssALL, ॰कीर्तिता \msNc}}% 
    \paral{{\devanagarifontsmall \vcd {\englishfont \compare\ \PADMAP\ 1.13.395cd:} परित्यजध्वं दाराणि स्वर्गमार्गार्गलानि च }}


\alalfejezet{नियमेषु व्रतपञ्चकम् (७)}
{\devanagarifont मार्जारकबकश्वानगोमहीव्रतपञ्चकम् \thinspace{\dandab} \dontdisplaylinenum }%
     \var{{\devanagarifont \numemph\vab\textbf{मार्जारकबकश्वानगोमहीव्रत॰}\lem \mssCaCbCc\msNa\msNc\msParis, 
मार्जारबकबश्वानगोमहीव्रत॰ \msNb, 
मार्जारकवकश्वानगोमहीवेक॰ \msPaperA, 
मार्जारकश्च श्वानाश्च गोमहीवक \Ed}}% 


\alalalfejezet{मार्जारकव्रतम्}

{\devanagarifont स्वविष्ठमूत्रं भूमीषु छादयेद्द्विजसत्तम  \danda\dontdisplaylinenum }%
     \var{{\devanagarifont \numnoemph\vc\textbf{॰विष्ठ॰}\lem \mssALL, ॰विष्टा॰ \Ed\oo 
\textbf{॰मूत्रं}\lem \mssALL, ॰मूत्र॰ \msCb\msNb}}% 

%Verse 8:13

{\devanagarifont सूर्यसोमानुमोदन्ति मार्जारव्रतिकेषु च {॥ ८:१३॥} \veg\dontdisplaylinenum }%
     \var{{\devanagarifont \numnoemph\ve\textbf{॰मोदन्ति}\lem \mssALL, ॰षादन्ति \Ed}}% 


\alalalfejezet{बकव्रतम्}

{\devanagarifont बकवच्चेन्द्रियग्रामं सुनियम्य तपोधन \thinspace{\dandab} \dontdisplaylinenum }%
     \var{{\devanagarifont \numemph\va\textbf{तपोधन}\lem \mssCaCbCc\msNa\msNb\msParis, 
तपोधनः \msNc, तपोधनम् \msPaperA\Ed}}% 

%Verse 8:14

{\devanagarifont साधयेच्च मनस्तुष्टिं मोक्षसाधनतत्परः {॥ ८:१४॥} \veg\dontdisplaylinenum }%
     \var{{\devanagarifont \numnoemph\vc\textbf{साधयेच्च}\lem \mssALL, 
साधये च \msCb\oo 
\textbf{मनस्तुष्टिं}\lem \mssALL, 
मनस्तुष्टि॰ \msCb\msCc}}% 
    \var{{\devanagarifont \numnoemph\vd\textbf{॰साधन॰}\lem \mssALL, 
॰सान॰ \msNc}}% 


\alalalfejezet{श्वानव्रतम्}

{\devanagarifont मूत्रविष्ठे न भूमीषु कुरुते धुनदं सदा \thinspace{\dandab} \dontdisplaylinenum }%
     \var{{\devanagarifont \numemph\va\textbf{मूत्रविष्ठे न}\lem \mssALL, 
मूत्रविष्टे च \Ed}}% 
    \var{{\devanagarifont \numnoemph\vb\textbf{धुनदं}\lem \mssALL, 
श्वानदः \msNa, छादनं \Ed}}% 

%Verse 8:15

{\devanagarifont तुष्यते भगवान्शर्वः श्वानव्रतचरो यदि {॥ ८:१५॥} \veg\dontdisplaylinenum }%
     \var{{\devanagarifont \numnoemph\vc\textbf{शर्वः}\lem \msCa\msNa\msNc\msParis\msPaperA\Ed, 
सर्वः \msCb\msNb, सव्वः \msCc}}% 


\alalalfejezet{गोव्रतम्}

{\devanagarifont मूत्रवर्चो न रुध्येत सदा गोव्रतिको नरः \thinspace{\dandab} \dontdisplaylinenum }%
     \var{{\devanagarifont \numemph\va\textbf{॰वर्चो}\lem \msCa\msCc\msNb\msNc\msParis\msPaperA, 
॰वच्चो \msCb\msNa, ॰वर्चा \Ed}}% 
    \var{{\devanagarifont \numnoemph\vb\textbf{गोव्रतिको}\lem \mssALL, 
\lk\lk तिको \msCa}}% 

%Verse 8:16

{\devanagarifont भीमस्तुष्टिकरश्चैव पुराणेषु निगद्यते {॥ ८:१६॥} \veg\dontdisplaylinenum }%
     \var{{\devanagarifont \numnoemph\vc\textbf{भीमस्तु॰}\lem \msCc\msNb\Ed, 
भीमतु॰ \msCa\msCb\msNa\msNc\msParis, 
भिमस्तु॰ \msPaperA}}% 


\alalalfejezet{महीव्रतम्}

{\devanagarifont कुद्दालैर्दारयन्तो ऽपि कीलकोटिशतैश्चितः \thinspace{\dandab} \dontdisplaylinenum }%
     \var{{\devanagarifont \numemph\va\textbf{कुद्दालैर्दारयन्तो}\lem \msNa\msParis\Ed, 
कुद्दालैर्दारयन्नो \msCa, कुद्दारै दारयन्तो \msCb, 
कुदारै दारयन्ता \msCc, कुद्दालै द्दारयामास \msNb, 
कुद्दालै दारयन्तो \msNc, कुद्दालै \uncl{द्धार}यन्तो \msPaperA}}% 
    \var{{\devanagarifont \numnoemph\vb \lem \msCa\msCb\msNa\msNb\msNc\msParis, 
कीटकोटीशतैरपि \msCc\msPaperA\Ed}}% 

%Verse 8:17

{\devanagarifont क्षमते पृथिवी देवी एवमेव महीव्रतः {॥ ८:१७॥} \veg\dontdisplaylinenum }%
     \var{{\devanagarifont \numnoemph\vd\textbf{॰व्रतः}\lem \mssALL, ॰व्रत \msNc}}% 

{\devanagarifont व्रतपञ्चकमित्येतद्यश्चरेत जितेन्द्रियः \thinspace{\dandab} \dontdisplaylinenum }%
     \var{{\devanagarifont \numemph\vb\textbf{जितेन्द्रियः}\lem \mssALL, द्विजेन्द्रियः \msNb}}% 

%Verse 8:18

{\devanagarifont स चोत्तममिदं लोकं प्राप्नोति न च संशयः {॥ ८:१८॥} \veg\dontdisplaylinenum }%
 

\alalfejezet{नियमेष्वुपवासः (८)}
{\devanagarifont शेषान्नमन्तरान्नं च नक्तायाचितमेव च \thinspace{\dandab} \dontdisplaylinenum }%
     \var{{\devanagarifont \numemph\va \lem \msCa\msCb\msNb\msNc\msParispcorr, 
शेषाणामन्तराणाञ्च \msCc\Ed, 
शेषान्नमन्नरान्नं च \msNa, 
शेषान्नमरान्नं च \msParisacorr, 
शेषाणमन्तराणाञ्च \msPaperA}}% 
    \var{{\devanagarifont \numnoemph\vb\textbf{नक्तायाचित॰}\lem \mssALL, 
नक्त\uncl{या}चित॰ \msNc\oo 
\textbf{च}\lem \mssALL, वा \Ed}}% 

%Verse 8:19

{\devanagarifont उपवासं च पञ्चैतत्कथयिष्यामि तच्छृणु {॥ ८:१९॥} \veg\dontdisplaylinenum }%
     \var{{\devanagarifont \numnoemph\vcd\textbf{पञ्चैतत्क॰}\lem \mssALL, 
पञ्चैते क॰ \msCc}}% 


\alalalfejezet{शेषान्नम्}

{\devanagarifont वैश्वदेवातिथिशेषं पितृशेषं च यद्भवेत् \thinspace{\dandab} \dontdisplaylinenum }%
     \var{{\devanagarifont \numemph\va\textbf{॰शेषं}\lem \mssALL, ॰शेषां \msCb}}% 

%Verse 8:20

{\devanagarifont भृत्यपुत्रकलत्रेभ्यः शेषाशी विघसाशनः {॥ ८:२०॥} \veg\dontdisplaylinenum }%
     \var{{\devanagarifont \numnoemph\vd\textbf{विघसाशनः}\lem \msCa\msNa\msNb, विघसासनम् \msCb, विघसाषिनः \msCc, 
विघशासनः \msNc, विघसाश\uncl{नः} \msParispcorr, 
घसाशन \msParisacorr, विघसासनः \msPaperA, विषसासनः \Ed}}% 


\alalalfejezet{अन्तरान्नम्}

{\devanagarifont अन्तरा प्रातराशी च सायमाशी तथैव च \thinspace{\dandab} \dontdisplaylinenum }%
     \var{{\devanagarifont \numemph\va\textbf{अन्तरा प्रातराशी}\lem \eme, अन्तरा प्रान्तराशी \mssCaCbCc\msNa\msNc, 
अन्तरा \uncl{क्रन्त}राशी \msNb, 
अन्तारा प्रा\uncl{त्त}राशी \msParis, 
अन्तमा प्रान्तराशी च \msPaperA,  अन्तसम्प्रान्तराशी \Ed}}% 
    \var{{\devanagarifont \numnoemph\vb\textbf{सायमाशी}\lem \msCb\msCc\msNa\msNb\msNc\msParis, सायमाशीन् \msCa, 
नायमाशी \msPaperA, नियमाशी \Ed}}% 

%Verse 8:21

{\devanagarifont सदोपवासी भवति यो न भुङ्क्ते कदाचन {॥ ८:२१॥} \veg\dontdisplaylinenum }%
     \var{{\devanagarifont \numnoemph\vc\textbf{॰वासी भवति}\lem \mssALL, 
॰वासी च भवति \msCc}}% 
    \var{{\devanagarifont \numnoemph\vd\textbf{कदाचन}\lem \mssALL, कदाचनः \msCc}}% 
    \paral{{\devanagarifontsmall \vcd \similar\ {\englishfont \MBH\ 12.214.9:} 
                                 अन्तरा प्रातराशं च सायमाशं तथैव च\thinspace{\devanagarifontsmall ।}
                                 सदोपवासी च भवेद् यो न भुङ्क्ते कथंचन\thinspace{\devanagarifontsmall ॥} 
                     \similar\ {\englishfont \MBH\ 13.93.10:}
                                 अन्तरा सायमाशं च प्रातराशं तथैव च\thinspace{\devanagarifontsmall ।}
                                 सदोपवासी भवति यो न भुङ्क्ते ऽन्तरा पुनः\thinspace{\devanagarifontsmall ॥} }}


\alalalfejezet{नक्तान्नम्}

{\devanagarifont न दिवा भोजनं कार्यं रात्रौ नैव च भोजयेत् \thinspace{\dandab} \dontdisplaylinenum }%
     \var{{\devanagarifont \numemph\va\textbf{भोजनं}\lem \mssALL, नोजनं \msNc}}% 
    \var{{\devanagarifont \numnoemph\vb\textbf{च}\lem \mssALL, तु \msCb, \om\ \msNa\oo 
\textbf{भोजयेत्}\lem \mssALL, कारयेत् \msNb}}% 

%Verse 8:22

{\devanagarifont नक्तवेले च भोक्तव्यं नक्तधर्मं समीहता {॥ ८:२२॥} \veg\dontdisplaylinenum }%
     \var{{\devanagarifont \numnoemph\vc\textbf{॰वेले च}\lem \msCa\msCc\msNa\msNb\msParis\msPaperA, 
॰वेला च \msCb, ॰वेलो च \msNc, ॰वेले व \Ed}}% 
    \var{{\devanagarifont \numnoemph\vd\textbf{॰धर्मं समीहता}\lem \msCa\msCb\msNa\msNc\msParis, 
॰धर्मसमीहता \msCc\msNb, ॰धर्म्मसमीहिता \msPaperA, 
॰धर्म्मः समीहितः \Ed}}% 


\alalalfejezet{अयाचितान्नम्}

{\devanagarifont अनारभ्य य आहारं कुर्यान्नित्यमयाचितम् \thinspace{\dandab} \dontdisplaylinenum }%
     \var{{\devanagarifont \numemph\va\textbf{अनारभ्य य}\lem \conj, अनारम्भस्य \mssCaCbCc\msNa\msNb\msNc\msParis\msPaperA\Ed}}% 
    \var{{\devanagarifont \numnoemph\vb\textbf{कुर्यान्नि॰}\lem \mssALL, कुर्या नि॰ \msNc}}% 

%Verse 8:23

{\devanagarifont परैर्दत्तं तु यो भुङ्क्ते तमयाचितमुच्यते {॥ ८:२३॥} \veg\dontdisplaylinenum }%
     \var{{\devanagarifont \numnoemph\vc\textbf{परैर्दत्तं तु}\lem \msCa\msCb\msNa\msParis\msPaperA, 
परै दत्तञ्च \msCc, परै दत्तन्तु \msNb, 
परैर्दन्तन्तु \msNc\Ed}}% 
    \var{{\devanagarifont \numnoemph\vd\textbf{तमयाचि॰}\lem \mssCaCbCc\msNa\msNb\msNc\Ed, नमयाचि॰ \msParisacorr\msPaperA, 
\uncl{तम}याचि॰ \msParispcorr}}% 


\alalalfejezet{उपवासः}

{\devanagarifont भक्ष्यं भोज्यं च लेह्यं च चोष्यं पेयं च पञ्चमम् \thinspace{\dandab} \dontdisplaylinenum }%
     \var{{\devanagarifont \numemph\va\textbf{भक्ष्यं}\lem \mssALL, भक्ष्य \msNa}}% 

%Verse 8:24

{\devanagarifont न काङ्क्षेन्नोपयुञ्जीत उपवासः स उच्यते {॥ ८:२४॥} \veg\dontdisplaylinenum }%
     \var{{\devanagarifont \numnoemph\vc\textbf{काङ्क्षेन्नो॰}\lem \mssALL, 
काङ्क्षे नो॰ \msCc\oo 
\textbf{॰युञ्जीत}\lem \msCc\msNa\msNb\msPaperA, ॰\lk\lk त \msCa, 
॰यञ्जीत \msCb, ॰भुजीत \msNc, ॰भुञ्जीत \msParis\Ed}}% 
    \var{{\devanagarifont \numnoemph\vd\textbf{॰वासः स}\lem \mssCaCbCc\msNa\msParis\Ed, ॰वास स \msNb, ॰वासस्य \msNc, 
॰वासंः स \msPaperA}}% 


\alalfejezet{नियमेषु मौनव्रतम् (९)}
{\devanagarifont मिथ्यापिशुनपारुष्यतीक्ष्णवागप्रलापनम् \thinspace{\dandab} \dontdisplaylinenum }%
     \var{{\devanagarifont \numemph\va\textbf{॰पारुष्य॰}\lem \msCa\msCb\msNa\msNb\msNc\msParis, ॰संभिन्ना \msCc, 
संभिन्नां \msPaperA, ॰याभिन्ना \Ed}}% 
    \var{{\devanagarifont \numnoemph\vb\textbf{॰तीक्ष्णवाग॰}\lem \conj, ॰स्पृष्टवाग॰ \msCa\msCb\msNa\msNb\msNc\msParis, 
पृष्टवाक॰ \msCc\msPaperA, 
पृष्तेवाक॰ \Ed}}% 

%Verse 8:25

{\devanagarifont मौनपञ्चकमित्येतद्धारयेन्नियतव्रतः {॥ ८:२५॥} \veg\dontdisplaylinenum }%
     \var{{\devanagarifont \numnoemph\vc\textbf{मौनपञ्चक॰}\lem \msCa\msCb\msNb, मौनं पञ्चक॰ \msCc\msNa\msNc\msPaperA\Ed, 
मौनम्पञ्च॰ \msParis\oo 
\textbf{॰त्येत॰}\lem \mssALL, ॰त्ये॰ \msParisacorr}}% 
    \var{{\devanagarifont \numnoemph\vd\textbf{॰रयेन्नि॰}\lem \mssALL, ॰रयन्नि॰ \Ed}}% 


\alalalfejezet{मिथ्यावचनम्}

{\devanagarifont असम्भूतमदृष्टं च धर्माच्चापि बहिष्कृतम् \thinspace{\dandab} \dontdisplaylinenum }%
     \var{{\devanagarifont \numemph\va\textbf{॰दृष्टं च}\lem \mssALL, दृष्ट\uncl{ञ्च} \msCc}}% 
    \var{{\devanagarifont \numnoemph\vb\textbf{धर्माच्चापि}\lem \msCa\msCb\msNa\msNb\msNc\msParis, 
धर्मश्चापि \msCc\msPaperA, धर्मं चापि \Ed\oo 
\textbf{बहिष्कृतम्}\lem \msCa\msCb\msNa\msNc\msParis, बहिष्कृतः \msCc\Ed, नहिष्कृतं \msNb, 
बहिस्कृतंः \msPaperA}}% 

%Verse 8:26

{\devanagarifont अनर्थाप्रियवाक्यं यत् तन्मिथ्यावचनं स्मृतम् {॥ ८:२६॥} \veg\dontdisplaylinenum }%
     \var{{\devanagarifont \numnoemph\vc\textbf{अनर्था॰}\lem \msCa\msCb\msNa\msNb\msNc\msParis, अनर्थ॰ \msCc\msPaperA\Ed}}% 
    \var{{\devanagarifont \numnoemph\vcd\textbf{॰वाक्यं यत्तन्मि॰}\lem \msCa\msCb\msNa\msParis\msPaperA, 
वक्तार तं मि॰ \msCc, 
वाक्य यत्तन्मि॰ \msNb, 
वाक्यं यन्तन्मि॰ \msNc\Ed}}% 
    \var{{\devanagarifont \numnoemph\vd\textbf{स्मृतम्}\lem \mssALL, स्मृतः \msCb}}% 


\alalalfejezet{पिशुनः}

{\devanagarifont परश्रीं नाभिनन्दन्ति परस्यैश्वर्यमेव च \thinspace{\dandab} \dontdisplaylinenum }%
     \var{{\devanagarifont \numemph\va\textbf{परश्रीं ना॰}\lem \msCa\msCb\msNa\msNc\msParis, परस्त्री ना॰ \msCc\msPaperApcorr\Ed, 
परस्त्रीन्ना॰ \msNb, परस्त्री श्री ना॰ \msPaperAacorr\oo 
\textbf{॰भिनन्दन्ति}\lem \mssALL, 
॰भिनन्ति \msCb, ॰भिन्नन्दन्ति \msCc}}% 
    \var{{\devanagarifont \numnoemph\vb\textbf{परस्यैश्वर्य॰}\lem \mssALL, 
परसैश्वर्य॰ \msCb}}% 

%Verse 8:27

{\devanagarifont अनिष्टदर्शनाकाङ्क्षी पिशुनः समुदाहृतः {॥ ८:२७॥} \veg\dontdisplaylinenum }%
     \var{{\devanagarifont \numnoemph\vc\textbf{॰दर्शना॰}\lem \msCa\msCb\msNa\msNc\msParis\Ed, ॰द\uncl{ब्भ}ना॰ \msCc, ॰दर्शनां \msNb, 
॰दशना॰ \msPaperA}}% 
    \var{{\devanagarifont \numnoemph\vd\textbf{पिशुनः}\lem \mssALL, पिशुन \msCc}}% 


\alalalfejezet{पारुष्यम्}

{\devanagarifont मृतमाता पिता चैव हानिस्थानं कथं भवेत् \thinspace{\dandab} \dontdisplaylinenum }%
     \var{{\devanagarifont \numemph\va\textbf{मृत॰}\lem \mssALL, मृता \msParispcorr}}% 
    \var{{\devanagarifont \numnoemph\vb\textbf{॰स्थानं}\lem \mssALL, ॰स्थान \msCb\msCc}}% 

%Verse 8:28

{\devanagarifont भुङ्क्ष्व कामममृष्टानां पारुष्यं समुदाहृतम् {॥ ८:२८॥} \veg\dontdisplaylinenum }%
     \var{{\devanagarifont \numnoemph\vc\textbf{भुङ्क्ष्व}\lem\msNc\msParis, भुक्त्व \msCa, भुक्त्वा \msCb\msCc, 
भुं\uncl{क्ष} \msNa, भुक्ष \msNb, 
भु\uncl{क्त} \msPaperA, भुक्ता \Ed\oo 
\textbf{कामममृष्टानां}\lem \msCa\msNa\msNc\msParis\Ed, कममसृष्टानां \msCb, 
कामसुसमृष्तानां \msCc, 
काममुमृष्ताना \msNb, 
पारुष्यमृष्टना \msPaperA}}% 


\alalalfejezet{तीक्ष्णवाक्}

{\devanagarifont हृदि न स्फुटसे मूढ शिरो वा न विदार्यसे \thinspace{\dandab} \dontdisplaylinenum }%
     \var{{\devanagarifont \numemph\va\textbf{स्फुटसे}\lem \mssALL, स्फुटय \msNb}}% 

%Verse 8:29

{\devanagarifont एवमादीन्यनेकानि तीक्ष्णवादी स उच्यते {॥ ८:२९॥} \veg\dontdisplaylinenum }%
 

\alalalfejezet{असत्प्रलापः}

{\devanagarifont द्यूतभोजनयुद्धं च मद्यस्त्रीकथमेव च \thinspace{\dandab} \dontdisplaylinenum }%
     \var{{\devanagarifont \numemph\va\textbf{॰युद्धं}\lem \mssALL, ॰युद्धश् \Ed}}% 
    \var{{\devanagarifont \numnoemph\vb\textbf{॰कथ॰}\lem \msNb\msNc, ॰कष॰ \mssCaCbCc\msNa\msParis, ॰कर्ष॰ \msPaperA\Ed}}% 

%Verse 8:30

{\devanagarifont असत्प्रलापः पञ्चैतत्कीर्तितं मे द्विजोत्तम {॥ ८:३०॥} \veg\dontdisplaylinenum }%
     \var{{\devanagarifont \numnoemph\vcd\textbf{पञ्चैतत्की॰}\lem \mssALL, 
पञ्चैते की॰ \msNb, पञ्चेतत्की॰ \msNc}}% 
    \var{{\devanagarifont \numnoemph\vd\textbf{मे}\lem \mssALL, ते \Ed}}% 

{\devanagarifont मौनमेव सदा कार्यं वाक्यसौभाग्यमिच्छता \thinspace{\dandab} \dontdisplaylinenum }%
     \var{{\devanagarifont \numemph\va\textbf{कार्यं}\lem \mssALL, कार्या \msNb}}% 
    \var{{\devanagarifont \numnoemph\vb\textbf{वाक्य॰}\lem \msCa\msCb\msNa\msNc\msParis\Ed, वाक्यं \msCc\msNb\msPaperA\oo 
\textbf{॰सौभाग्य॰}\lem \mssALL, ॰सौभार्य॰ \msCb}}% 

%Verse 8:31

{\devanagarifont अपारुष्यमसम्भिन्नं वाक्यं सत्यमुदीरयेत् {॥ ८:३१॥} \veg\dontdisplaylinenum }%
     \var{{\devanagarifont \numnoemph\vc\textbf{॰भिन्नं}\lem \mssALL, ॰भिन्न \msCc, ॰दिग्धं \Ed}}% 

{\devanagarifont यस्तु मौनस्य नो कर्ता दूषितः स कुलाधमः \thinspace{\dandab} \dontdisplaylinenum }%
     \var{{\devanagarifont \numemph\vb\textbf{दूषितः}\lem \mssALL, दूषित \msCc, भूषितः \Ed}}% 

%Verse 8:32

{\devanagarifont जन्मे जन्मे च दुर्गन्धो मूकश्चैवोपजायते {॥ ८:३२॥} \veg\dontdisplaylinenum }%
     \var{{\devanagarifont \numnoemph\vc\textbf{जन्मे जन्मे}\lem \msCb\msCc\msNa\msPaperA\Ed, जन्म जन्म \msCa\msNb\msNc\msParis\oo 
\textbf{दुर्गन्धो}\lem \msCa\msNb\msNc\msParis\msPaperA, 
दुग्गन्धो \msCb, दुर्गन्धा \msCc, दुगन्धो \msNa, दृगन्धो \Ed}}% 

\nemslokalong


\ujvers\nemsloka {
{\devanagarifont तस्मान्मौनव्रतं सदैव सुदृढं कुर्वीत यो निश्चितं }%
  \dontdisplaylinenum}    \var{{\devanagarifont \numemph\va\textbf{तस्मान्मौ॰}\lem \msCc\msNb\msNc\msParis\msPaperA\Ed, 
\lk\lk त्मौ॰ \msCa, तस्मात्मौ॰ \msCb\msNa\oo 
\textbf{सदैव}\lem \msCa\msCb\msNa\msParis\Ed, सदेव \msCc\msNc\msPaperA, सुदैत्य \msNb\oo 
\textbf{कुर्वीत यो निश्चितम्}\lem \msCa\msCb\msNc\msParis\msPaperA\Ed, 
कुर्वन्ति येन्निश्चितम् \msCc\msNa, 
कुर्वन्ति योन्निश्चित \msNb}}% 


\nemslokab

{\devanagarifont वाचा तस्य अलङ्घ्यता च भवति सर्वां सभां नन्दति  \danda\dontdisplaylinenum }%
     \var{{\devanagarifont \numnoemph\vb\textbf{अलङ्घ्यता च}\lem \msCa\msCb\msNa\msNb\msParis, अलंघ्यताञ्च \msCc\msNc\msPaperA\Ed\oo 
\textbf{सर्वां सभां}\lem \msCa\msNa\msParis\msPaperA\Ed, सर्वा सभा \msCb\msNc, 
सर्वः सभान् \msCc, सर्वा सुभा \msNb}}% 

\nemslokac

{\devanagarifont वक्त्राच्चोत्पलगन्धमस्य सततं वायन्ति गन्धोत्कटाः }%
  \dontdisplaylinenum    \var{{\devanagarifont \numnoemph\vc\textbf{वक्त्राच्चोत्पलगन्धमस्य}\lem \msCa\msCb\msNc\msParisacorr\msPaperA, 
वक्त्रं चोत्पलमस्य \msCc, वक्त्रं चोत्पलगन्धमस्य \msNa, 
वक्त्रं चोत्पल\uncl{ग}न्धमस्य \msNb, 
वक्त्राश्चोत्पलगन्धमस्य \msParispcorr, 
वक्त्राच्चोतरगन्धमस्य \Ed}}% 

%Verse 8:33


\nemslokad

{\devanagarifont शास्त्रानेकसहस्रशो गिरि नरः प्रोच्चार्यते निर्मलम् {॥ ८:३३॥} \veg\dontdisplaylinenum }%
     \var{{\devanagarifont \numnoemph\vd\textbf{॰सहस्रशो}\lem \mssALL, ॰सहस्राशो \msCb\oo 
\textbf{॰मलम्}\lem \msCa\msNa\msNb\msNc\msParis, ॰मलः \msCb\msCc\msPaperA\Ed}}% 

\nemslokanormal


\vers



\alalfejezet{नियमेषु स्नानम् (१०)}
{\devanagarifont स्नानं पञ्चविधं चैव प्रवक्ष्यामि यथातथम् \thinspace{\dandab} \dontdisplaylinenum }%
     \var{{\devanagarifont \numemph\va\textbf{पञ्चविधं}\lem \mssALL, पञ्चवि \msCb}}% 
    \var{{\devanagarifont \numnoemph\vb\textbf{यथातथम्}\lem \mssALL, \lk\lk तथम् \msCa}}% 

%Verse 8:34

{\devanagarifont आग्नेयं वारुणं ब्राह्म्यं वायव्यं दिव्यमेव च {॥ ८:३४॥} \veg\dontdisplaylinenum }%
     \var{{\devanagarifont \numnoemph\vc\textbf{आग्नेयं}\lem \mssALL, आग्नेये \msNb\oo 
\textbf{वारुणं}\lem \mssALL, ब्राह्मणं \msPaperA\Ed\oo 
\textbf{ब्राह्म्यं}\lem \mssALL, ब्रह्म्यं \msNc}}% 


\alalalfejezet{आग्नेयं स्नानम्}

{\devanagarifont आग्नेयं भस्मना स्नानं तोयाच्छतगुणं फलम् \thinspace{\dandab} \dontdisplaylinenum }%
     \var{{\devanagarifont \numemph\va\textbf{स्नानं}\lem \mssALL, स्नाना \msNaacorr}}% 
    \var{{\devanagarifont \numnoemph\vb\textbf{॰गुणं}\lem \mssALL, ॰गुण॰ \msNc}}% 

%Verse 8:35

{\devanagarifont भस्मपूतं पवित्रं च भस्म पापप्रणाशनम् {॥ ८:३५॥} \veg\dontdisplaylinenum }%
 
{\devanagarifont तस्माद्भस्म प्रयुञ्जीत देहिनां तु मलापहम् \thinspace{\dandab} \dontdisplaylinenum }%
     \var{{\devanagarifont \numemph\va \lem \mssALL, 
\lk\lk\lk\lk\lk\lk\lk त \msNb}}% 
    \var{{\devanagarifont \numnoemph\vb\textbf{मला॰}\lem \mssALL, पला॰ \msPaperA}}% 

%Verse 8:36

{\devanagarifont सर्वशान्तिकरं भस्म भस्म रक्षकमुत्तमम् {॥ ८:३६॥} \veg\dontdisplaylinenum }%
     \var{{\devanagarifont \numnoemph\vc\textbf{सर्व॰}\lem \mssALL, \uncl{ए}ना॰ \msPaperA}}% 

{\devanagarifont भस्मना त्र्यायुषं कृत्वा ब्रह्मचर्यव्रते स्थितम् \thinspace{\dandab} \dontdisplaylinenum }%
     \var{{\devanagarifont \numemph\va\textbf{त्र्यायुषं कृत्वा}\lem \mssALL, 
त्र्यायु\lk\lk\lk\ \msCa, त्र्यायुष्यं कृत्वा \msParis}}% 
    \var{{\devanagarifont \numnoemph\vb\textbf{॰व्रते}\lem \mssALL, ॰व्रत॰ \msPaperA\Ed}}% 

%Verse 8:37

{\devanagarifont भस्मना ऋषयः सर्वे पवित्रीकृतमात्मनः {॥ ८:३७॥} \veg\dontdisplaylinenum }%
     \var{{\devanagarifont \numnoemph\vc\textbf{ऋषयः सर्वे}\lem \mssALL, 
ऋषिभिर्सर्वैः \Ed}}% 

{\devanagarifont भस्मना विबुधा मुक्ता वीरभद्रभयार्दिताः \thinspace{\dandab} \dontdisplaylinenum }%
     \var{{\devanagarifont \numemph\va\textbf{मुक्ता}\lem \mssALL, मुक्ताः \Ed}}% 
    \var{{\devanagarifont \numnoemph\vb\textbf{॰र्दिताः}\lem \mssALL, ॰र्त्तिताः \msCb}}% 

%Verse 8:38

{\devanagarifont भस्मानुशंसं दृष्ट्वैव ब्रह्मनानुमतिः कृता {॥ ८:३८॥} \veg\dontdisplaylinenum }%
     \var{{\devanagarifont \numnoemph\vc \lem \corrTorzsok, 
भस्मानुसंसं दृष्ट्यैव \msCa, भस्मानुशंसां दृष्ट्वव \msCb, 
भस्मानुसंसदृष्टैव \msCc\msNb, भस्मानुसंसन्दृष्ट्वैव \msNa, 
भस्मानुशंसंदृष्ट्यैवं \msNc, भस्मानुशंसं दृष्टैव \msParis, 
भस्मानुशंसं \uncl{दृष्टै}व \msPaperA, 
भस्मना शं प्रदृश्यैवं \Ed}}% 
    \var{{\devanagarifont \numnoemph\vd\textbf{ब्रह्मणानुमतिः}\lem \eme, ब्रह्मणानुमता \mssCaCbCc\msNa\msNb\msNc\msParis, 
ब्राह्मणानुमतो \msPaperA\Ed\oo 
\textbf{कृता}\lem \eme, कृतः \msCa\msCb\msNb\msNc\msParis\msPaperA\Ed, कृतिः \msCc, कृताः \msNa}}% 

{\devanagarifont चतुराश्रमतो ऽधिक्यं व्रतं पाशुपतं कृतम् \thinspace{\dandab} \dontdisplaylinenum }%
     \var{{\devanagarifont \numemph\va\textbf{चतुराश्रमतो}\lem \msCb\msCc\msNb\msParis\Ed, 
चातुराश्रमतो \msCa\msNc\msPaperA, चतुराश्रतो \msNaacorr, 
चातुराश्रमतो \msNapcorr}}% 
    \var{{\devanagarifont \numnoemph\vab\textbf{ऽधिक्यं व्रतं पाशुपतं कृतम्}\lem \mssALL, 
\uncl{धिक्यव्रतपाशुपत}\lk\lk\lk\ \msNb\ \toplost}}% 

%Verse 8:39

{\devanagarifont तस्मात्पाशुपतं श्रेष्ठं भस्मधारणहेतुतः {॥ ८:३९॥} \veg\dontdisplaylinenum }%
     \var{{\devanagarifont \numnoemph\vc \lem \mssALL, \om \msNb}}% 
    \var{{\devanagarifont \numnoemph\vd\textbf{॰हेतुतः}\lem \emeTorzsok, ॰हेतवः \msCa\msCb\msNa\msNc\msParis\msPaperA\Ed, 
॰हेतुना \msCc, ॰हेतुनुतः \msNb}}% 


\alalalfejezet{वारुणं स्नानम्}

{\devanagarifont वारुणं सलिलं स्नानं कर्तव्यं विविधं नरैः \thinspace{\dandab} \dontdisplaylinenum }%
     \var{{\devanagarifont \numemph\va\textbf{वारुणं}\lem \msCb\msCc\msNa\msNb\msParis\Ed, 
वा\lk\lk\  \msCa, वारुणा \msNcacorr, वारुण \msNcpcorr, 
वरुणं \msPaperA\oo 
\textbf{सलिलं}\lem \mssCaCbCc\msNa\msNb\msParis, सलिल॰ \msNc\msPaperA\Ed}}% 
    \var{{\devanagarifont \numnoemph\vb\textbf{विविधं नरैः}\lem \mssCaCbCc\msNa\msPaperA, विविन्नरैः \msNb, 
विधिवन्नरैः \msNc\msParis\Ed}}% 

%Verse 8:40

{\devanagarifont नदीतोयतडागेषु प्रस्रवेषु ह्रदेषु च {॥ ८:४०॥} \veg\dontdisplaylinenum }%
     \var{{\devanagarifont \numnoemph\vc\textbf{॰तडागेषु}\lem \mssALL, 
॰तडागेवा \msNb}}% 
    \var{{\devanagarifont \numnoemph\vd\textbf{प्रस्रवेषु}\lem \mssALL, 
प्रयेवेषु \msNb, प्रभवेषु \msNc}}% 


\alalalfejezet{ब्राह्म्यं स्नानम्}

{\devanagarifont ब्रह्मस्नानं च विप्रेन्द्र आपोहिष्ठं विदुर्बुधाः \thinspace{\dandab} \dontdisplaylinenum }%
     \var{{\devanagarifont \numemph\va\textbf{विप्रेन्द्र}\lem \mssALL, विपेन्द्र \msNc\msParis}}% 
    \var{{\devanagarifont \numnoemph\vb\textbf{विदुर्बु॰}\lem \mssALL, विर्दुर्बु॰ \msNc}}% 

%Verse 8:41

{\devanagarifont त्रिसंध्यमेव कर्तव्यं ब्रह्मस्नानं तदुच्यते {॥ ८:४१॥} \veg\dontdisplaylinenum }%
 

\alalalfejezet{वायव्यं स्नानम्}

{\devanagarifont गोषु संचारमार्गेषु यत्र गोधूलिसम्भवः \thinspace{\dandab} \dontdisplaylinenum }%
 
%Verse 8:42

{\devanagarifont तत्र गत्वावसीदेत स्नानमुक्तं मनीषिभिः {॥ ८:४२॥} \veg\dontdisplaylinenum }%
     \var{{\devanagarifont \numemph\vd\textbf{॰क्तं}\lem \mssALL, ॰क्त \msNb}}% 


\alalalfejezet{दिव्यं स्नानम्}

{\devanagarifont वर्षतोयाम्बुधाराभिः प्लावयित्वा स्वकां तनुम् \thinspace{\dandab} \dontdisplaylinenum }%
     \var{{\devanagarifont \numemph\vb\textbf{तनुम्}\lem \mssALL, तनं \msNc}}% 

%Verse 8:43

{\devanagarifont स्नानं दिव्यं वदत्येव जगदादिमहेश्वरः {॥ ८:४३॥} \veg\dontdisplaylinenum }%
     \var{{\devanagarifont \numnoemph\vc\textbf{दिव्यं}\lem \mssALL, दिव्य \msNb\msPaperA}}% 
    \var{{\devanagarifont \numnoemph\vd\textbf{जगदादि॰}\lem \mssALL, गजदादि॰ \msCb}}% 

\ujvers\nemsloka {
{\devanagarifont इति नियमविभागः पञ्चभेदेन विप्र }%
  \dontdisplaylinenum}    \var{{\devanagarifont \numemph\va\textbf{॰भागः}\lem \mssALL, ॰भागं \msNc}}% 


\nemslokab

{\devanagarifont निगदित तव पृष्टः सर्वलोकानुकम्प्य  \danda\dontdisplaylinenum }%
     \var{{\devanagarifont \numnoemph\vb\textbf{निगदित तव}\lem \Ed, 
निगदितस्तव \mssCaCbCc\msNa\msNb\msNc\msParis\msPaperA\ \unmetr\oo 
\textbf{॰कम्प्य}\lem \msCa, ॰कम्प \msCb\msCc\msNa\msNc\msParis, 
॰कम्पः \msNb, ॰कम्प्यः \msPaperA\Ed}}% 

\nemslokac

{\devanagarifont सकलमलपहारी धर्मपञ्चाशदेतन् }%
  \dontdisplaylinenum    \var{{\devanagarifont \numnoemph\vc\textbf{॰पहारी}\lem \msCb\msCc\msNb, ॰पहारि \msCa\msNc\unmetr, ॰प्रहारि \msNa\msParis\msPaperA, 
॰पहारे \Ed\oo 
\textbf{॰पञ्चाशदेतन्}\lem \msCa\msCb\msNa\msNbpcorr\msNc\msParis, 
॰पञ्चाशमेतन् \msCc\msPaperA\Ed, 
॰पञ्चादेतन् \msNbacorr}}% 

%Verse 8:44


\nemslokad

{\devanagarifont न भवति पुनजन्म कल्पकोट्यायुते ऽपि {॥ ८:४४॥} \veg\dontdisplaylinenum }%
     \var{{\devanagarifont \numnoemph\vd\textbf{पुनजन्म}\lem \msCc\msNb, पुनर्जन्म \msCa\msNa\msNc\msParis\msPaperA\Ed, 
पुन\uncl{र्जर्म} \msCb}}% 

\vers


{\devanagarifont 
\jump
\begin{center}
\ketdanda~इति वृषसारसंग्रहे नियमप्रशंसा नामाध्यायो ऽष्टमः~\ketdanda
\end{center}
\dontdisplaylinenum\vers  }%
     \var{{\devanagarifont \numnoemph{\englishfont \Colo:}\textbf{इति वृषसारसंग्रहे नियमप्रशंसा नामाध्यायो ऽष्टमः}\lem \msParis, 
इति वृषसारसंग्रहे नियमप्रशंसा नामाध्याय अष्टमः \msCa\msNa\msPaperA, 
\om \msCb, 
इति वृषसारसंग्रहे नियमप्रशंसा नामाध्यायाष्टमः \msCc\msNb, 
इति वृषसारसंग्रहे नियमप्रशंसा नामाध्यायाऽष्टमः \msNc, 
इति वृषसारसंग्रहे नियमप्रशंसा नाम अष्टमो ऽध्यायः \Ed}}% 
\bekveg\szamveg
\vfill
\phpspagebreak

\versno=0\fejno=9
\thispagestyle{empty}

\centerline{\Large\devanagarifontbold [   नवमो ऽध्यायः  ]}{\vrule depth10pt width0pt} \fancyhead[CO]{{\footnotesize\devanagarifont वृषसारसंग्रहे  }}
\fancyhead[CE]{{\footnotesize\devanagarifont नवमो ऽध्यायः  }}
\fancyhead[LE]{}
\fancyhead[RE]{}
\fancyhead[LO]{}
\fancyhead[RO]{}
\szam\bek



\alalfejezet{त्रैगुण्यम्}
\vers


{\devanagarifont [अनर्थयज्ञ उवाच {\dandab}\dontdisplaylinenum  ] }%
 
{\devanagarifont त्रिकालगुणभेदेन भिन्नं सर्वचराचरम् \thinspace{\danda} \dontdisplaylinenum }%
     \var{{\devanagarifont \numemph\va\textbf{त्रिकाल॰}\lem \mssALL, त्रिष्काल॰ \msCc\oo 
\textbf{॰भेदेन}\lem \mssALL, ॰भेन \msNbacorr}}% 
    \var{{\devanagarifont \numnoemph\vb\textbf{भिन्नं}\lem \mssALL, भिन्न \msNb}}% 
    \lacuna{\devanagarifontsmall {\englishfont Witnesses used for this chapter: \msCa\ ff.\thinspace 205v--207r, 
                                              \msCb\ ff.\thinspace 211v--212v, 
                                              \msCc\ ff.\thinspace 282r--283v,
                                              \msNa\ ff.\thinspace 13r--14v, 
                                              \msNb\ exp.\thinspace 54 (lower) -- 55 (lower),
                                              \msNc\ ff.\thinspace 221r--222v,
                                              \Ed\ pp.\thinspace 606--609; 
                                              \mssCaCbCc~= \msCa + \msCb + \msCc} }%
  
%Verse 9:1

{\devanagarifont तस्मात्त्रिगुणबन्धेन वेष्टितं निखिलं जगत् {॥ ९:१॥} \veg\dontdisplaylinenum }%
     \var{{\devanagarifont \numnoemph\vc\textbf{तस्मात्त्रि॰}\lem \mssALL, तस्मा त्रि॰ \msCc\msNc}}% 

{\devanagarifont विगतराग उवाच {\dandab}\dontdisplaylinenum  }%
 
{\devanagarifont त्रैकाल्यमिति किं ज्ञेयं त्रैधातुकशरीरिणः \thinspace{\danda} \dontdisplaylinenum }%
     \var{{\devanagarifont \numemph\va\textbf{॰काल्यम्}\lem \mssALL, ॰कालम् \msCa\msNc}}% 
    \var{{\devanagarifont \numnoemph\vab\textbf{किं ज्ञेयं त्रै॰}\lem \msCa\msNc, 
विज्ञेयं त्रै॰ \msCb\msNa\msNb\Ed, कि ज्ञेयम्त्रै॰ \msCc}}% 
    \var{{\devanagarifont \numnoemph\vb\textbf{॰धातुक॰}\lem \mssALL, ॰धायुक्त॰ \Ed}}% 

%Verse 9:2

{\devanagarifont किंचिद्विस्तरमेवेह कथयस्व तपोधन {॥ ९:२॥} \veg\dontdisplaylinenum }%
     \var{{\devanagarifont \numnoemph\vc\textbf{किंचि॰}\lem \mssALL, 
सात्त्विको भगव् विष्णु राजसः कमलोद्भवः\thinspace{\devanagarifont ।} 
तामसो भगवानीशः सकलं विक किञ्चि॰ \msCbacorr\ 
{\englishfont (eyeskip to 9.5)}\oo 
\textbf{॰वेह}\lem \mssALL, ॰तद्धि \Ed}}% 
    \var{{\devanagarifont \numnoemph\vd\textbf{कथयस्व}\lem \mssALL, क\lk\lk \lk\ \msCa}}% 

{\devanagarifont अनर्थयज्ञ उवाच {\dandab}\dontdisplaylinenum  }%
 
{\devanagarifont त्रैकाल्यं त्रिगुणं ज्ञेयं व्यापी प्रकृतिसम्भवः \thinspace{\danda} \dontdisplaylinenum }%
     \var{{\devanagarifont \numemph\va\textbf{॰काल्यं}\lem \mssALL, ॰काल्य \msCc\oo 
\textbf{॰गुणं}\lem \mssALL, ॰गुण \msCc}}% 

%Verse 9:3

{\devanagarifont अन्योन्यमुपजीवन्ति अन्योन्यमनुवर्तिनः {॥ ९:३॥} \veg\dontdisplaylinenum }%
     \paral{{\devanagarifontsmall \vcd {\englishfont \similar\ \BRAHMANDAPUR\ 1.4.9--10:}
                         एत एव त्रयो लोका एत एव त्रयो गुणाः\thinspace{\devanagarifontsmall ।}  
                         एत एव त्रयो वेदा एत एव त्रजो ऽग्नयः\thinspace{\devanagarifontsmall ॥}
                         परस्परान्वया ह्येते परस्परमनुव्रताः\thinspace{\devanagarifontsmall ।}
                         परस्परेण वर्तन्ते प्रेरयन्ति परस्परम्\thinspace{\devanagarifontsmall ॥}
                      {\englishfont \similar\ \VAYUP\ 1.5.16--17ab \similar\ \LINPU\ 1.70.78--79} }}

{\devanagarifont सत्त्वं रजस्तमश्चैव रजः सत्त्वं तमस्तथा \thinspace{\dandab} \dontdisplaylinenum }%
     \var{{\devanagarifont \numemph\va\textbf{सत्त्वं}\lem \mssALL, सत्व \msNb\oo 
\textbf{रजस्त॰}\lem \mssALL, रजत॰ \Ed}}% 
    \var{{\devanagarifont \numnoemph\vb\textbf{रजः}\lem \msCa\msCb\msNa\msNc, रज॰ \msCc\msNb\Ed\oo 
\textbf{सत्त्वं तमस्तथा}\lem \msCa\msNa\msNc, सत्त्वं तमन्तथा \msCb, 
सत्वस्तमस्तथा \msCc\msNb, सत्त्वतमस्तथा \Ed}}% 

%Verse 9:4

{\devanagarifont तमः सत्त्वं रजश्चैव अन्योन्यमिथुनाः स्मृताः {॥ ९:४॥} \veg\dontdisplaylinenum }%
     \var{{\devanagarifont \numnoemph\vc\textbf{तमः सत्त्वं}\lem \msCa\msCb\msNa\msNc, तमसत्व॰ \msCc, तमः सत्व॰ \msNb\Ed\oo 
\textbf{रजश्चैव}\lem \mssALL, रजःश्चैव \msCb}}% 
    \var{{\devanagarifont \numnoemph\vd\textbf{स्मृताः}\lem \mssALL, \om\ \msCc}}% 
    \paral{{\devanagarifontsmall \vd {\englishfont \similar\ \BRAHMANDAPUR\ 1.4.11ab:}
                         अन्योन्यं मिथुनं ह्येते अन्योन्यमुपजीविनः
                     {\englishfont \similar\ \VAYUP\ 1.5.17cd \similar\ \LINPU\ 1.70.80ab} }}

{\devanagarifont सात्त्विको भगवान्विष्णू राजसः कमलोद्भवः \thinspace{\dandab} \dontdisplaylinenum }%
     \var{{\devanagarifont \numemph\va\textbf{॰ष्णू}\lem \corr, ॰ष्णु \mssCaCbCc\msNa\msNb\msNc\Ed}}% 
    \var{{\devanagarifont \numnoemph\vb \lem \mssALL, 
\uncl{राज}\lk\lk\lk\lk\lk\lk\ \msCa}}% 
    \paral{{\devanagarifontsmall \vo {\englishfont \compare\ \BRAHMANDAPUR\ 1.4.6cd:}
                 सत्त्वं विष्णू रजो ब्रह्मा तमो रुद्रः प्रजापतिः }}

%Verse 9:5

{\devanagarifont तामसो भगवानीशः सकलंविकलेश्वरः {॥ ९:५॥} \veg\dontdisplaylinenum }%
     \var{{\devanagarifont \numnoemph\vcd\textbf{तामसो भगवानीशः सकलं}\lem \mssALL, 
\lk\lk \lk\lk \lk\lk \lk\lk \uncl{सकलम्} \msCa}}% 

{\devanagarifont सत्त्वं कुन्देन्दुवर्णाभं पद्मरागनिभं रजः \thinspace{\dandab} \dontdisplaylinenum }%
     \var{{\devanagarifont \numemph\va\textbf{सत्त्वं}\lem \mssALL, सत्व \msCc\msNc\oo 
\textbf{॰वर्णाभं}\lem \mssALL, ॰वर्ण्णाभ \msCc, ॰वण्णाभं \msNa}}% 

%Verse 9:6

{\devanagarifont तमश्चाञ्जनशैलाभं कीर्तितानि मनीषिभिः {॥ ९:६॥} \veg\dontdisplaylinenum }%
     \var{{\devanagarifont \numnoemph\vc\textbf{॰भं}\lem \mssALL, ॰भा \Ed}}% 

{\devanagarifont सत्त्वं जलं रजो ऽङ्गारं तमो धूमसमाकुलम् \thinspace{\dandab} \dontdisplaylinenum }%
     \var{{\devanagarifont \numemph\va\textbf{जलं}\lem \mssALL, रजं \msCc, ज्वाल \msNb\oo 
\textbf{रजो ऽङ्गारं}\lem \mssALL, 
र\uncl{ङ्गो}ङ्गारन् \msCc, रजोङ्गरन् \Ed}}% 

%Verse 9:7

{\devanagarifont एतद्गुणमयैर्बद्धाः पच्यन्ते सर्वदेहिनः {॥ ९:७॥} \veg\dontdisplaylinenum }%
     \var{{\devanagarifont \numnoemph\vd\textbf{॰देहिनः}\lem \mssALL, ॰देहिना \msCb}}% 

{\devanagarifont विगतराग उवाच {\dandab}\dontdisplaylinenum  }%
 
{\devanagarifont केन केन प्रकारेण गुणपाशेन बध्यते \thinspace{\danda} \dontdisplaylinenum }%
     \var{{\devanagarifont \numemph\vb\textbf{गुण॰}\lem \mssALL, \om\ \msCa}}% 

%Verse 9:8

{\devanagarifont चिह्नमेषां पृथक्त्वेन कथयस्व तपोधन {॥ ९:८॥} \veg\dontdisplaylinenum }%
     \var{{\devanagarifont \numnoemph\vc\textbf{॰षां पृथक्त्वेन}\lem \mssALL, ॰षा पृथकेन \msNc}}% 

{\devanagarifont अनर्थयज्ञ उवाच {\dandab}\dontdisplaylinenum  }%
 
{\devanagarifont अनेकाकारभावेन बध्यन्ते गुणबन्धनैः \thinspace{\danda} \dontdisplaylinenum }%
 
%Verse 9:9

{\devanagarifont मोहिता नाभिजानन्ति जानन्ति शिवयोगिनः {॥ ९:९॥} \veg\dontdisplaylinenum }%
     \var{{\devanagarifont \numemph\vc\textbf{॰भिजानन्ति}\lem \mssALL, ॰भिजानान्ति \msCc}}% 
    \var{{\devanagarifont \numnoemph\vd\textbf{जानन्ति}\lem \mssALL, \om\ \msCbacorr}}% 

{\devanagarifont ऊर्ध्वंगो नित्यसत्त्वस्थो मध्यगो रजसावृतः \thinspace{\dandab} \dontdisplaylinenum }%
     \var{{\devanagarifont \numemph\va\textbf{ऊर्ध्वंगो नित्य}\lem \conj, 
ऊर्ध्वाङ्गो नित्य॰ \mssCaCbCc\msNapcorr\Ed, 
ऊर्ध्वाङ्गा नत्य॰ \msNaacorr, 
ऊर्ध्वगो सित्य॰ \msNbacorr, 
ऊर्ध्वगो सत्य॰ \msNbpcorr, 
उर्ध्वाङ्गो नित्य॰ \msNc\oo 
\textbf{॰सत्त्व॰}\lem \msCa\msCb\msNa\msNc, ॰सत्य॰ \msCc\Ed, ॰नित्य॰ \msNb}}% 
    \var{{\devanagarifont \numnoemph\vb\textbf{मध्यगो}\lem \mssALL, मध्यमो \Ed\oo 
\textbf{॰वृतः}\lem \mssALL, ॰वृतम् \Ed}}% 

%Verse 9:10

{\devanagarifont अधोगतिस्तमोऽवस्था भवन्ति पुरुषाधमाः {॥ ९:१०॥} \veg\dontdisplaylinenum }%
     \var{{\devanagarifont \numnoemph\vc\textbf{॰गतिस्तमो॰}\lem \mssALL, ॰गतितमो॰ \msCb\msCc}}% 

{\devanagarifont स्वर्गे ऽपि हि त्रयो वैते भावनीयास्तपोधन \thinspace{\dandab} \dontdisplaylinenum }%
 
%Verse 9:11

{\devanagarifont मानुषेषु च तिर्येषु गुणभेदास्त्रयस्त्रयः {॥ ९:११॥} \veg\dontdisplaylinenum }%
     \var{{\devanagarifont \numemph\vc\textbf{मानुषेषु}\lem \mssALL, मनुष्येषु \msCb, मानुष्येषु \msNc\oo 
\textbf{तिर्येषु}\lem \mssALL, तीर्येषु \Ed}}% 
    \var{{\devanagarifont \numnoemph\vd\textbf{॰स्त्रयः}\lem \mssALL, ॰स्त्रः \msCbacorr}}% 


\alalalfejezet{सात्त्विकोत्तमाः}

{\devanagarifont ब्रह्मा विष्णुश्च रुद्रश्च धर्म इन्द्रः प्रजापतिः \thinspace{\dandab} \dontdisplaylinenum }%
     \var{{\devanagarifont \numemph\vb\textbf{धर्म इन्द्रः}\lem \mssALL, इर्म इन्द्र \msCb, धर्मरिन्द्र॰ \Ed}}% 

%Verse 9:12

{\devanagarifont सोमो ऽग्निर्वरुणः सूर्यो दश सत्त्वोत्तमाः स्मृताः {॥ ९:१२॥} \veg\dontdisplaylinenum }%
     \var{{\devanagarifont \numnoemph\vc\textbf{ग्निर्वरुणः}\lem \msCa\msNa\msNc, ग्नि वरुण \msCb\msCc\msNb\Ed}}% 
    \var{{\devanagarifont \numnoemph\vd\textbf{दश}\lem \mssALL, दशः \Ed\oo 
\textbf{सत्त्वोत्तमाः}\lem \mssALL, सत्वत्तमाः \msCb, सत्तोतमाः \msNc}}% 


\alalalfejezet{सात्त्विकमध्यमाः}

{\devanagarifont रुद्रादित्या वसुसाध्या विश्वेशमरुतो ध्रुवः \thinspace{\dandab} \dontdisplaylinenum }%
     \var{{\devanagarifont \numemph\vab\textbf{॰दित्या वसुसाध्या}\lem \msCb\msNa\msNb\msNc, ॰दित्या वसुसा\lk\ \msCa, ॰दित्य वसुसाध्या \msCc, 
॰दित्य वसुसाध्याः वि॰ \Ed}}% 
    \var{{\devanagarifont \numnoemph\vb\textbf{विश्वेश॰}\lem \mssALL, \lk श्वेश \msCa, विश्वेशि॰ \msCc}}% 

%Verse 9:13

{\devanagarifont ऋषयः पितरश्चैव दशैते सत्त्वमध्यमाः {॥ ९:१३॥} \veg\dontdisplaylinenum }%
     \var{{\devanagarifont \numnoemph\vd\textbf{दशैते}\lem \mssALL, दशैतेते \msCbacorr}}% 


\alalalfejezet{सात्त्विकाधमाः}

{\devanagarifont तारा ग्रहाः सुरा यक्षा गन्धर्वाः किंनरोरगाः \thinspace{\dandab} \dontdisplaylinenum }%
     \var{{\devanagarifont \numemph\va\textbf{ग्रहाः सुरा}\lem \mssALL, ग्रहास्वराः \msCc, ग्रहाऽसुरा \Ed}}% 
    \var{{\devanagarifont \numnoemph\vb\textbf{गन्धर्वाः}\lem \msCa\msNb\msNc\Ed, गन्धर्वा \msCb\msNa, गन्धर्व्वाः गन्धर्व्वा \msCc}}% 

%Verse 9:14

{\devanagarifont रक्षोभूतपिशाचाश्च दशैते सात्त्विकाधमाः {॥ ९:१४॥} \veg\dontdisplaylinenum }%
     \var{{\devanagarifont \numnoemph\vc\textbf{॰पिशाचाश्च}\lem \mssALL, ॰पिशाश्चाश्च \msNc}}% 
    \var{{\devanagarifont \numnoemph\vd\textbf{दशैते}\lem \mssALL, दशेते \msCb\oo 
\textbf{सात्त्विका॰}\lem \mssALL, सत्वका॰ \msCb}}% 


\alalalfejezet{राजसोत्तमाः}

{\devanagarifont ऋत्विक्पुरोहिताचार्ययज्वानो ऽतिथि विज्ञनी \thinspace{\dandab} \dontdisplaylinenum }%
     \var{{\devanagarifont \numemph\vb\textbf{॰विज्ञनी}\lem \mssALL, ॰विज्ञकौ \Ed}}% 

%Verse 9:15

{\devanagarifont राजा मन्त्री व्रती वेदी दशैते राजसोत्तमाः {॥ ९:१५॥} \veg\dontdisplaylinenum }%
     \var{{\devanagarifont \numnoemph\vc\textbf{राजा}\lem \eme, राज॰ \mssCaCbCc\msNa\msNb\msNc\Ed\oo 
\textbf{॰मन्त्री व्रती}\lem \mssALL, ॰मन्त्रि व्रतो \Ed}}% 
    \var{{\devanagarifont \numnoemph\vd\textbf{राजसो॰}\lem \mssALL, रामसो \msCb}}% 


\alalalfejezet{राजसमध्यमाः}

{\devanagarifont सूतो ऽम्बष्ठवणिश्चोग्रः शिल्पिकारुकमागधाः \thinspace{\dandab} \dontdisplaylinenum }%
     \var{{\devanagarifont \numemph\va\textbf{सूतो ऽम्बष्ठ॰}\lem \corr, सूतो \lk ष्ट॰ \msCa, सूत\uncl{म्बष्ट}॰ \msCb, 
सूतोन्वष्ठ॰ \msCc, 
सूतोत्वष्टा॰ \msNa, सूतोत्वष्ट॰ \msNb\msNc, 
सूतो ऽम्बष्ट॰ \Ed\oo 
\textbf{॰वणिश्चो॰}\lem \mssALL, ॰वणिश्वो॰ \Ed}}% 
    \var{{\devanagarifont \numnoemph\vb\textbf{शिल्पि॰}\lem \msNb, शिल्प॰ \mssCaCbCc\msNa\msNc\Ed\oo 
\textbf{मागधाः}\lem \mssALL, मागधा \msCc}}% 

%Verse 9:16

{\devanagarifont वेणवैदेहकामात्या दशैते रजमध्यमाः {॥ ९:१६॥} \veg\dontdisplaylinenum }%
     \var{{\devanagarifont \numnoemph\vc \lem \msCa\msCc\msNa\msNb, वैणवेदेहकामात्या \msCb, 
वेनवैदेहकामात्या \msNc, वेणवैदेचकौ मात्या \Ed}}% 


\alalalfejezet{राजसाधमाः}

{\devanagarifont चर्मकृत्कुम्भकृत्कोली लोहकृत्त्रपुनीलिकाः \thinspace{\dandab} \dontdisplaylinenum }%
     \var{{\devanagarifont \numemph\va\textbf{॰कृत्कोली}\lem \mssALL, ॰ककोली \msNa, ॰कृत्काली \Ed}}% 
    \var{{\devanagarifont \numnoemph\vb\textbf{॰नीलिकाः}\lem \mssALL, ॰तीलिका \Ed}}% 

%Verse 9:17

{\devanagarifont नटमुष्टिकचण्डाला दशैते रजसाधमाः {॥ ९:१७॥} \veg\dontdisplaylinenum }%
     \var{{\devanagarifont \numnoemph\vc\textbf{॰मुष्टिक॰}\lem \mssALL, ॰मौष्टिक॰ \msCc\oo 
\textbf{॰चण्डाला}\lem \mssALL, ॰चाण्डालः \Ed}}% 
    \var{{\devanagarifont \numnoemph\vd\textbf{दशैते}\lem \mssALL, दशेते \msCb}}% 
    \paral{{\devanagarifontsmall \vc {\englishfont = \UMS\ 2.10a, 2.20a = \UUMS\ 2.31c} }}


\alalalfejezet{तामसोत्तमाः}

{\devanagarifont गोगजगवया अश्वमृगचामरकिंनराः \thinspace{\dandab} \dontdisplaylinenum }%
     \var{{\devanagarifont \numemph\va\textbf{॰गवया}\lem \mssALL, ॰गवय \msNb, ॰गवयो \Ed}}% 
    \var{{\devanagarifont \numnoemph\vb\textbf{॰चामर॰}\lem \msCa\msCb\msNa\msNc, ॰वानर॰ \msCc\Ed, ॰\uncl{वा}नर॰ \msNb}}% 

%Verse 9:18

{\devanagarifont सिंहव्याघ्रवराहाश्च दशैते तामसोत्तमाः {॥ ९:१८॥} \veg\dontdisplaylinenum }%
     \var{{\devanagarifont \numnoemph\vc\textbf{॰वराहा॰}\lem \mssALL, ॰वराह॰ \msNb\Ed}}% 
    \var{{\devanagarifont \numnoemph\vd\textbf{तामसोत्तमाः}\lem \mssALL, तामशोत्तमः \msCb, 
तमसोत्तमाः \Ed}}% 


\alalalfejezet{तामसमध्यमाः}

{\devanagarifont अजमेषमहिष्याश्च मूषिकानकुलादयः \thinspace{\dandab} \dontdisplaylinenum }%
     \var{{\devanagarifont \numemph\va\textbf{॰महिष्याश्च}\lem \mssALL, ॰महिंष्या च \msNb}}% 

%Verse 9:19

{\devanagarifont उष्ट्ररङ्कुशशगण्डा दशैते तममध्यमाः {॥ ९:१९॥} \veg\dontdisplaylinenum }%
     \var{{\devanagarifont \numnoemph\vc\textbf{उष्ट्र॰}\lem \mssALL, उष्ट॰ \msCc, दंष्ट्रि॰ \Ed\oo 
\textbf{॰शशगण्डा}\lem \mssALL, ॰शगण्डाश्च \Ed}}% 
    \var{{\devanagarifont \numnoemph\vd\textbf{तममध्यमाः}\lem \mssALL, तमध्यमाः \msCa}}% 


\alalalfejezet{तामसाधमाः}

{\devanagarifont ऋक्षगोधामृगशृङ्गिबकवानरगर्दभाः \thinspace{\dandab} \dontdisplaylinenum }%
     \var{{\devanagarifont \numemph\vb\textbf{॰गर्दभाः}\lem \mssALL, ॰गर्दभः \Ed}}% 

%Verse 9:20

{\devanagarifont सूकरश्वानगोमायुर्दशैते तामसाधमाः {॥ ९:२०॥} \veg\dontdisplaylinenum }%
     \var{{\devanagarifont \numnoemph\vc\textbf{सूकर॰}\lem \mssALL, सुखर॰ \msCb}}% 
    \var{{\devanagarifont \numnoemph\vcd\textbf{॰गोमायुर्द॰}\lem \mssALL, ॰गोमायु द॰ \msNa\msNb}}% 
    \var{{\devanagarifont \numnoemph\vd\textbf{॰शैते}\lem \mssALL, ॰शेते \msCb}}% 


\alalalfejezet{तमसात्त्विकाः}

{\devanagarifont क्रौञ्चहंसशुकश्येनभासबारुण्डसारसाः \thinspace{\dandab} \dontdisplaylinenum }%
     \var{{\devanagarifont \numemph\va\textbf{क्रौञ्च॰}\lem \Ed, क्रोञ्च॰ \mssCaCbCc\msNa\msNb\msNc}}% 
    \var{{\devanagarifont \numnoemph\vb\textbf{॰सारसाः}\lem \mssALL, ॰सारसा \msNc}}% 

%Verse 9:21

{\devanagarifont चक्राह्वशुकमायूरा दशैते तमसात्त्विकाः {॥ ९:२१॥} \veg\dontdisplaylinenum }%
     \var{{\devanagarifont \numnoemph\vc\textbf{॰ह्वशुकमायूरा}\lem \mssALL, 
॰\uncl{ङ्ग}\lk\lk \lk यूरा \msCa, ॰ङ्गशुकमायूरा \Ed}}% 
    \var{{\devanagarifont \numnoemph\vd\textbf{दशैते}\lem \mssALL, दशेते \msCb\oo 
\textbf{तमसात्त्विकाः}\lem \msCc\msNc\Ed, तमस्सात्त्विकाः \msCa\msNb\ \unmetr, 
नमः सात्विकाः \msCb\ \unmetr, 
तमःसात्विकाः \msNa\ \unmetr}}% 


\alalalfejezet{तमराजसाः}

{\devanagarifont बलाकाः कुक्कुटाः काकाश्चिल्ललावकतित्तिराः \thinspace{\dandab} \dontdisplaylinenum }%
     \var{{\devanagarifont \numemph\va\textbf{बलाकाः}\lem \corr, वलाका \msCa\msNa\msNc, वलाक॰ \msCb\msCc\msNb\Ed}}% 
    \var{{\devanagarifont \numnoemph\vab\textbf{कुक्कुटाः काकाश्चि॰}\lem \corr, कुक्कुटकाकाश्चि॰ \msCa\msCb\ \unmetr, 
कुर्कुटा काकाश्चि॰ \msCc\msNc, 
कुर्कुटकाकाश्चि \msNa\msNb, कुक्कुटो काका चि॰ \Ed}}% 
    \var{{\devanagarifont \numnoemph\vb\textbf{॰तित्तिराः}\lem \mssALL, ॰तित्तराः \msNc, ॰तित्तिरिः \Ed}}% 

%Verse 9:22

{\devanagarifont गृध्रकङ्कबकश्येन दशैते तमराजसाः {॥ ९:२२॥} \veg\dontdisplaylinenum }%
     \var{{\devanagarifont \numnoemph\vc\textbf{गृध्र॰}\lem \mssALL, गृध॰ \msNc}}% 


\alalalfejezet{तामसाधमादि}

{\devanagarifont कोकिलोलूककञ्जल्यकपोताः पञ्च एव च \thinspace{\dandab} \dontdisplaylinenum }%
     \var{{\devanagarifont \numemph\va\textbf{कोकिलो॰}\lem \mssALL, कौकिलो॰ \msCb\oo 
\textbf{॰कञ्जल्य॰}\lem \eme, ॰किञ्जल्य॰ \msCa\msCc\msNa, ॰किञ्जल्क॰ \msCb\msNb\msNc\Ed}}% 
    \var{{\devanagarifont \numnoemph\vb\textbf{च}\lem \mssALL, चः \msNc}}% 

%Verse 9:23

{\devanagarifont शारिकाश्च कुलिङ्गाश्च दशैते तमसाधमाः {॥ ९:२३॥} \veg\dontdisplaylinenum }%
     \var{{\devanagarifont \numnoemph\vc\textbf{शारिकाश्च}\lem \corr, शारिका च \mssCaCbCc\msNa\msNb\msNc, शालिका च \Ed\oo 
\textbf{कुलिङ्गाश्च}\lem \corr, कुलिङ्गा च \msCa\msNb\Ed, कुलिङ्का च \msCb\msCc\msNc, 
कुलिकां च \msNa}}% 

{\devanagarifont मकरगोहनक्राश्च ऋक्षाश्च तमसात्त्विकाः \thinspace{\dandab} \dontdisplaylinenum }%
     \var{{\devanagarifont \numemph\va\textbf{॰गोहनक्राश्च}\lem \mssALL, 
॰गोहनक्रा च \msCc, ॰ग्रोहनक्राश्च \msNb}}% 
    \var{{\devanagarifont \numnoemph\vb\textbf{ऋक्षाश्च}\lem \conj, ऋषा च \mssCaCbCc\msNa\msNb\msNc\Ed\oo 
\textbf{तमसात्त्विकाः}\lem \Ed, तम\uncl{स्सा}\lk\lk\ \msCa, 
तमःसात्विकाः \msCb\msCc\msNa\msNb\ \unmetr, तसमात्विकाः \msNc}}% 

{\devanagarifont कच्छपशिशुकुम्भीरमण्डूकास्तमराजसाः  \danda\dontdisplaylinenum }%
     \var{{\devanagarifont \numnoemph\vc\textbf{॰शिशु॰}\lem \eme, ॰शुशु॰ \mssCaCbCc\msNa\msNb\msNc\Ed\oo 
\textbf{॰कुम्भीर॰}\lem \mssALL, ॰कम्भीरा \msCc\Ed}}% 
    \var{{\devanagarifont \numnoemph\vd\textbf{॰मण्डूका॰}\lem \mssALL, ॰मण्डूक॰ \msNb, ॰मण्डुका॰ \Ed}}% 

%Verse 9:24

{\devanagarifont शङ्खशुक्तिकशम्बूकाः कवय्यस्तमतामसाः {॥ ९:२४॥} \veg\dontdisplaylinenum }%
     \var{{\devanagarifont \numnoemph\ve\textbf{शम्बूकाः}\lem \corr, ॰शम्बूका \mssCaCbCc\msNa\msNb\Ed, ॰\uncl{स}म्बूकाः \msNc}}% 
    \var{{\devanagarifont \numnoemph\vf\textbf{॰कवय्य॰}\lem \conj, ॰कबन्ध्या॰ \mssCaCbCc\msNa\msNbpcorr\msNc\Ed, ॰कबन॰ \msNbacorr\oo 
\textbf{॰मतामसाः}\lem \msCb\Ed, ॰मस्तामसाः \msCa\msCc\msNc\ \unmetr, ॰मःतामसाः \msNa\msNb\ \unmetr}}% 

{\devanagarifont चन्दनागरुपद्मं च प्लक्षोदुम्बरपिप्पलाः \thinspace{\dandab} \dontdisplaylinenum }%
     \var{{\devanagarifont \numemph\va\textbf{॰गरु॰}\lem \mssALL, ॰गुरु॰ \Ed}}% 

%Verse 9:25

{\devanagarifont वटदारुशमीबिल्वा दशैते तमसात्त्विकाः {॥ ९:२५॥} \veg\dontdisplaylinenum }%
     \var{{\devanagarifont \numnoemph\vc\textbf{॰बिल्वा}\lem \msCa\msCb\msNa\Ed, ॰बिल्व \msCc\msNb\msNc}}% 
    \var{{\devanagarifont \numnoemph\vd\textbf{दशैते}\lem \mssALL, दशै \msCc\oo 
\textbf{तमसात्त्विकाः}\lem \Ed, तमस्सात्विकाः \msCa\ \unmetr, 
तमःसात्विकाः \msCb\msCc\msNa\msNb\msNc\ \unmetr}}% 

{\devanagarifont जाम्बीरलकुचाम्रातदाडिमाकोलवेतसाः \thinspace{\dandab} \dontdisplaylinenum }%
     \var{{\devanagarifont \numemph\va\textbf{जाम्बीर॰}\lem \mssALL, जम्बीर॰ \msCc}}% 
    \var{{\devanagarifont \numnoemph\vb\textbf{॰दाडिमा॰}\lem \mssALL, ॰द्राडिमा॰ \msCc, 
॰द्राडि\uncl{हा}॰ \msNa}}% 

%Verse 9:26

{\devanagarifont निम्बनीपो †ध्रवावश्च† दशैते तमराजसाः {॥ ९:२६॥} \veg\dontdisplaylinenum }%
     \var{{\devanagarifont \numnoemph\vc\textbf{॰नीपो}\lem \mssALL, ॰नीपौ \msNc\oo 
\textbf{ध्रवावश्च}\lem \mssALL, 
धवावश्च \msCapcorr, धुवावश्च \Ed}}% 
    \var{{\devanagarifont \numnoemph\vd\textbf{दशैते}\lem \mssALL, \lk\lk\lk\ \msCa}}% 

{\devanagarifont वृक्षवल्लीलतावेणुत्वक्सारतृणभूरुहाः \thinspace{\dandab} \dontdisplaylinenum }%
     \var{{\devanagarifont \numemph\va\textbf{वृक्षवल्ली॰}\lem \mssALL, \uncl{वृक्षवल्ली} \msNb}}% 
    \var{{\devanagarifont \numnoemph\vb\textbf{॰त्वक्सारतृण॰}\lem \msCa\msCb\msNa\msNb, 
॰त्वक्सारस्तृण॰ \msCc\Ed, ॰त्वकसारतृण॰ \msNc\ \unmetr}}% 

%Verse 9:27

{\devanagarifont मीरजाश्च शिलाशस्या दशैते तमसात्त्विकाः {॥ ९:२७॥} \veg\dontdisplaylinenum }%
     \var{{\devanagarifont \numnoemph\vc\textbf{मीरजाश्च}\lem \corr, मीरजा च \msCa\msCc\msNa\msNb\msNc\Ed, मीनजा च \msCb}}% 
    \var{{\devanagarifont \numnoemph\vd\textbf{तमसात्त्विकाः}\lem \msNc\Ed, तमस्सात्विकाः \msCa, 
तमःसात्विकाः \msCb\msCc\msNa\ \unmetr, तमःसाधिकाः \msNb\ \unmetr}}% 

{\devanagarifont भ्रमरालि पतङ्गाश्च क्रिमिकीटजलौकसः \thinspace{\dandab} \dontdisplaylinenum }%
     \var{{\devanagarifont \numemph\va\textbf{॰आलि}\lem \eme, \mssCaCbCc\msNa\msNb\msNc\Ed\oo 
\textbf{पतङ्गाश्च}\lem \mssALL, पतङ्गानां \Ed}}% 
    \var{{\devanagarifont \numnoemph\vb \lem \mssCaCbCc\msNa, क्रिमिकीटजलोकसः \msNb, 
क्रिमिकीटजलौक\uncl{साः} \msNc, किमिकीटजलौकसां \Ed}}% 

%Verse 9:28

{\devanagarifont यूकोद्दंशमशानां च विष्ठाजास्तमसात्त्विकाः {॥ ९:२८॥} \veg\dontdisplaylinenum }%
     \var{{\devanagarifont \numnoemph\vc \lem \msCa, 
यूकोदंशमशानाञ्च \msCb\msNa, 
यूकोदंशमसकानाञ्च \msCc\ \unmetr, 
यूकोदंशमसानान्तु \msNb, 
\uncl{यूकोद्दं}\lk\lk \lk\lk \lk\  \msNc, 
युक्तोदंशमशानाश्च \Ed}}% 
    \var{{\devanagarifont \numnoemph\vd \lem \corr, 
विष्टजास्तमस्सात्विकाः \msCa\ \unmetr, 
विष्टजास्तमःसात्विकाः \msCb\msCc\msNa\ \unmetr, 
विष्टजास्तमःसाधिकाः \msNb\ \unmetr, 
\lk\lk \uncl{जा}तमस्साधिकाः \msNc\ \unmetr, 
विष्टजा तमसात्त्विकाः \Ed}}% 

{\devanagarifont दया सत्यं दमः शौचं ज्ञानं मौनं तपः क्षमा \thinspace{\dandab} \dontdisplaylinenum }%
     \var{{\devanagarifont \numemph\vb\textbf{ज्ञानं}\lem \msCa\msCc\msNb\Ed, ज्ञान \msCb\msNc, ज्ञा\uncl{नं} \msNa\oo 
\textbf{मौनं}\lem \mssALL, मौन \msNa\oo 
\textbf{क्षमा}\lem \mssALL, क्षमाः \msCb\msNb}}% 

%Verse 9:29

{\devanagarifont शीलं च नाभिमानं च सात्त्विकाश्चोत्तमा जनाः {॥ ९:२९॥} \veg\dontdisplaylinenum }%
     \var{{\devanagarifont \numnoemph\vc\textbf{शीलं च}\lem \mssALL, नीलञ्च \msNb, शिलं च \Ed\oo 
\textbf{नाभिमानं}\lem \mssALL, नाभिमानां \Ed}}% 

{\devanagarifont कामतृष्णारतिद्यूतमानो युद्धं मदः स्पृहा \thinspace{\dandab} \dontdisplaylinenum }%
     \var{{\devanagarifont \numemph\va\textbf{॰मानो}\lem \mssALL, ॰मनो \msCc}}% 
    \var{{\devanagarifont \numnoemph\vb\textbf{युद्धं}\lem \mssALL, युद्ध॰ \Ed\oo 
\textbf{स्पृहा}\lem \mssALL, स्मृत \msNb}}% 

%Verse 9:30

{\devanagarifont निर्घृणाः कलिकर्तारो राजसेषूत्तमा जनाः {॥ ९:३०॥} \veg\dontdisplaylinenum }%
     \var{{\devanagarifont \numnoemph\vc\textbf{निर्घृणाः}\lem \mssCaCbCc, निर्घृणा \msNa\Ed, निघृणाः \msNb\msNc}}% 
    \var{{\devanagarifont \numnoemph\vd\textbf{राजसेषूत्तमा}\lem \mssALL, 
राजसेसूतमा \msCc, राजसे ह्युत्तमो \Ed}}% 

{\devanagarifont हिंसासूयाघृणामूढनिद्रातन्द्रीभयालसाः \thinspace{\dandab} \dontdisplaylinenum }%
     \var{{\devanagarifont \numemph\va\textbf{॰सूया॰}\lem \mssALL, ॰स\uncl{यू}॰ \msNb\oo 
\textbf{॰मूढ॰}\lem \mssALL, ॰मूढा॰ \msCb\msNb}}% 
    \var{{\devanagarifont \numnoemph\vb\textbf{॰तन्द्री॰}\lem \mssALL, ॰तन्त्री॰ \Ed}}% 

%Verse 9:31

{\devanagarifont क्रोधो मत्सरमायी च तामसेषूत्तमा जनाः {॥ ९:३१॥} \veg\dontdisplaylinenum }%
     \var{{\devanagarifont \numnoemph\vc\textbf{क्रोधो}\lem \mssALL, क्रोध॰ \Ed}}% 
    \var{{\devanagarifont \numnoemph\vd\textbf{तामसेषूत्तमा}\lem \mssALL, 
तामसेसूतमा \msCc, तामसे ह्युत्तमो \Ed}}% 

{\devanagarifont लघुप्रीतिप्रकाशी च ध्यानयोगे सदोत्सुकः \thinspace{\dandab} \dontdisplaylinenum }%
     \var{{\devanagarifont \numemph\vb\textbf{॰योगे}\lem \mssALL, ॰\uncl{योगे} \msCa}}% 

%Verse 9:32

{\devanagarifont प्रज्ञाबुद्धिविरागी च सात्त्विकं गुणलक्षणम् {॥ ९:३२॥} \veg\dontdisplaylinenum }%
     \var{{\devanagarifont \numnoemph\vc\textbf{॰विरागी च}\lem \mssALL, ॰विरागी \msNa, ॰विराङ्क्री च \msNc}}% 

{\devanagarifont बालको निपुणो रागी मानो दर्पश्च लोभकः \thinspace{\dandab} \dontdisplaylinenum }%
     \var{{\devanagarifont \numemph\va\textbf{बालको}\lem \mssALL, चालको \msNc\oo 
\textbf{निपुणो}\lem \Ed, निपुनो \mssCaCbCc\msNa\msNb, निपुणे \msNc}}% 

%Verse 9:33

{\devanagarifont स्पृहा ईर्षा प्रलापी च राजसं गुणलक्षणम् {॥ ९:३३॥} \veg\dontdisplaylinenum }%
     \var{{\devanagarifont \numnoemph\vc\textbf{ईर्षा}\lem \mssALL, ईर्ष्या \msCb\Ed\oo 
\textbf{प्रलापी}\lem \mssALL, च लापी \msCc}}% 
    \var{{\devanagarifont \numnoemph\vd\textbf{राजसं}\lem \mssALL, तामसं \Ed}}% 

{\devanagarifont उद्वेग आलसो मोहः क्रूरस्तस्करनिर्दयः \thinspace{\dandab} \dontdisplaylinenum }%
     \var{{\devanagarifont \numemph\va\textbf{आलसो}\lem \mssALL, अलसो \msCb}}% 
    \var{{\devanagarifont \numnoemph\vb\textbf{क्रूरस्त॰}\lem \msCa\msCb\msNa, क्रूरत॰ \msCc\msNc\Ed, कूरस्त॰ \msNb\oo 
\textbf{॰निर्दयः}\lem \mssALL, ॰निर्दयाः \msNc}}% 

%Verse 9:34

{\devanagarifont क्रोधः पिशुन निद्रा च तामसं गुणलक्षणम् {॥ ९:३४॥} \veg\dontdisplaylinenum }%
     \var{{\devanagarifont \numnoemph\vc\textbf{क्रोधः}\lem \mssALL, क्रोध॰ \msCb\oo 
\textbf{पिशुन}\lem \Ed, पिशुनो \mssCaCbCc\msNa\msNb\msNc\oo 
\textbf{च}\lem \mssALL, \om\ \msNb}}% 
    \var{{\devanagarifont \numnoemph\vd\textbf{गुण॰}\lem \mssALL, गु॰ \msCbacorr}}% 


\alalalfejezet{आहारस्त्रैगुण्ये}

{\devanagarifont विगतराग उवाच {\dandab}\dontdisplaylinenum  }%
 
{\devanagarifont केन चिह्नेन विज्ञेय आहारः सर्वदेहिनाम् \thinspace{\danda} \dontdisplaylinenum }%
     \var{{\devanagarifont \numemph\vab \lem \mssALL, 
\lk\lk\lk\lk\lk\lk\lk\lk\lk\lk\lk\lk\lk  देहिनाम् \msCa, 
केन चिह्नेन विज्ञेय आहार सर्वदेहिनाम् \msNb}}% 

%Verse 9:35

{\devanagarifont त्रैगुण्यस्य पृथक्त्वेन कथयस्व तपोधन {॥ ९:३५॥} \veg\dontdisplaylinenum }%
     \var{{\devanagarifont \numnoemph\vc\textbf{पृथक्त्वेन}\lem \mssALL, पृथक्केण \msNc}}% 
    \var{{\devanagarifont \numnoemph\vd\textbf{॰धन}\lem \mssALL, ॰धनः \msNc}}% 

{\devanagarifont अनर्थयज्ञ उवाच {\dandab}\dontdisplaylinenum  }%
 
{\devanagarifont आयुः कीर्तिः सुखं प्रीतिर्बलारोग्यविवर्धनम् \thinspace{\danda} \dontdisplaylinenum }%
     \var{{\devanagarifont \numemph\va\textbf{कीर्तिः}\lem \mssALL, किर्तिः \Ed\oo 
\textbf{सुखं प्रीतिर्ब॰}\lem \msNc, सुखं प्रीतिब॰ \msCa\msCb\msNa\msNb, 
सुखप्रीति ब॰ \msCc, सुखं प्रितिव॰ \Ed}}% 
    \var{{\devanagarifont \numnoemph\vb\textbf{॰रोग्य॰}\lem \mssALL, ॰रोग्यं \msCb}}% 

%Verse 9:36

{\devanagarifont हृद्यस्वादुरसं स्निग्ध आहारः सात्त्विकप्रियः {॥ ९:३६॥} \veg\dontdisplaylinenum }%
     \var{{\devanagarifont \numnoemph\vc\textbf{हृद्य॰}\lem \mssALL, हृद॰ \Ed\oo 
\textbf{॰रसं}\lem \msCa\msCb\msNa, ॰रस \msCc, ॰\uncl{रस} \msNb, ॰रसां \msNc, ॰रसा \Ed\oo 
\textbf{स्निग्ध}\lem \mssALL, स्निग्धं \msNa, \uncl{सन्दिग्ध} \msNb}}% 
    \var{{\devanagarifont \numnoemph\vd\textbf{आहारः}\lem \msCapcorr\msNb\msNc\Ed, आहार \msCaacorr\msCb\msCc\msNa\oo 
\textbf{सात्त्विकप्रियः}\lem \msCa\msCb\msNa\msNc, 
सात्विकप्रिया \msCc, सात्विकप्रिय \msNb, सात्विकः कियाः \Ed}}% 

{\devanagarifont अत्युष्णमाम्ललवणं रूक्षं तीक्ष्णं विदाहि च \thinspace{\dandab} \dontdisplaylinenum }%
     \var{{\devanagarifont \numemph\va\textbf{॰म्ल॰}\lem \mssALL, ॰ल्ल॰ \Ed\oo 
\textbf{॰लवणं}\lem \mssALL, ॰लक्षणं \msCb}}% 
    \var{{\devanagarifont \numnoemph\vb\textbf{तीक्ष्णं}\lem \mssALL, ती\uncl{क्ष्ण} \msCa, स्तीक्षं \Ed\oo 
\textbf{विदाहि च}\lem \msCb\msNa\msNb\msNc, \lk \uncl{दाहि च} \msCa, 
विदाहिक \msCcpcorr, विदाहिकः \msCcacorr\Ed}}% 

%Verse 9:37

{\devanagarifont राजसश्रेष्ठ-आहारो दुःखशोकामयप्रदः {॥ ९:३७॥} \veg\dontdisplaylinenum }%
     \var{{\devanagarifont \numnoemph\vcd \lem \msCb\msNa\msNc, 
\lk\lk \lk\lk \lk\lk \lk\lk \lk\lk \lk\lk \lk\lk \lk\lk\  \msCa, 
राजसश्रेष्ठ आहारो दुःखशोकामयः प्रदः \msCc, 
राजसः श्रेष्ठ आहारो दुःखशोकामयप्रदः \msNb, 
राजसे श्रेष्ठमाहारो दुःखशोकाभयप्रदः \Ed}}% 

{\devanagarifont अभक्ष्यामेध्यपूती च पूति पर्युषितं च यत् \thinspace{\dandab} \dontdisplaylinenum }%
     \var{{\devanagarifont \numemph\va \lem \eme, अभक्ष्यमेध्यपूती च \mssCaCbCc\msNa, 
अभक्षमेध्यपूती च \msNb, अभक्षामेध्यपूती च \msNc, अभक्षमद्यपूती वै \Ed}}% 

%Verse 9:38

{\devanagarifont आमयारसविस्वाद आहारस्तामसप्रियः {॥ ९:३८॥} \veg\dontdisplaylinenum }%
     \var{{\devanagarifont \numnoemph\vc\textbf{आमया॰}\lem \conj, आयाम॰ \mssCaCbCc\msNa\msNb\msNc, आयास॰ \Ed}}% 
    \var{{\devanagarifont \numnoemph\vd\textbf{॰मस॰}\lem \mssALL, ॰मसः \msCc\Ed\oo 
\textbf{॰प्रियः}\lem \mssALL, ॰प्रियाः \msCc}}% 


\alalalfejezet{गुणातीतम्}

{\devanagarifont विगतराग उवाच {\dandab}\dontdisplaylinenum  }%
 
{\devanagarifont गुणातीतं कथं ज्ञेयं संसारपरपारगम् \thinspace{\danda} \dontdisplaylinenum }%
     \var{{\devanagarifont \numemph\va\textbf{॰तीतं}\lem \mssALL, ॰तीत \msCc\msNb}}% 
    \var{{\devanagarifont \numnoemph\vb\textbf{॰गम्}\lem \mssALL, ॰गः \msCc}}% 

%Verse 9:39

{\devanagarifont गुणपाशनिबद्धानां मोक्षं कथय तत्त्वतः {॥ ९:३९॥} \veg\dontdisplaylinenum }%
     \var{{\devanagarifont \numnoemph\vc\textbf{॰बद्धानां}\lem \mssALL, ॰वर्द्धानां \msCb, ॰बद्धामो \Ed}}% 

{\devanagarifont अनर्थयज्ञ उवाच {\dandab}\dontdisplaylinenum  }%
 
{\devanagarifont आत्मवत्सर्वभूतानि सम्यक्पश्येत भो द्विज \thinspace{\danda} \dontdisplaylinenum }%
     \var{{\devanagarifont \numemph\va\textbf{॰भूतानि}\lem \mssALL, ॰भूतां \msNa}}% 
    \var{{\devanagarifont \numnoemph\vb\textbf{सम्यक्प॰}\lem \mssALL, सम्यत्प॰ \msNa}}% 
    \paral{{\devanagarifontsmall \vab {\englishfont \similar\ \PADMAP\ 1.19.337ab:} 
                         आत्मवत्सर्वभूतानि यः पश्यति स पश्यति }}

%Verse 9:40

{\devanagarifont गुणातीतः स विज्ञेयः संसारपरपारगः {॥ ९:४०॥} \veg\dontdisplaylinenum }%
     \var{{\devanagarifont \numnoemph\vc\textbf{॰तीतः}\lem \msCa\msCb\msNa\msNb, ॰तीत \msCc\msNc, ॰तीतं \Ed}}% 
    \paral{{\devanagarifontsmall \vo {\englishfont \compare\ \BHG\ 6.32:}
                 आत्मौपम्येन सर्वत्र समं पश्यति यो ऽर्जुन\thinspace{\devanagarifontsmall ।}
                 सुखं वा यदि वा दुःखं स योगी परमो मतः\thinspace{\devanagarifontsmall ॥} }}

{\devanagarifont ईर्षाद्वेषसमो यस्तु सुखदुःखसमाश्च ये \thinspace{\dandab} \dontdisplaylinenum }%
     \var{{\devanagarifont \numemph\va\textbf{ईर्षा॰}\lem \mssALL, ईर्ष्या॰ \msNc\Ed}}% 
    \var{{\devanagarifont \numnoemph\vb\textbf{॰समाश्च ये}\lem \mssALL, ॰समाश्रये \msNb}}% 
    \paral{{\devanagarifontsmall \vab {\englishfont \compare\ \VSS\ 11.51ab:}
                     न्यसेद्धर्ममधर्मं च ईर्ष्याद्वेषं परित्यजेत
                     {\englishfont and \BHG\ 14.25:}
                         मानापमानयोस्तुल्यस्तुल्यो मित्रारिपक्षयोः\thinspace{\devanagarifontsmall ।}
                         सर्वारम्भपरित्यागी गुणातीतः स उच्यते\thinspace{\devanagarifontsmall ॥}
                    {\englishfont and also \BHG\ 12.13:}
                 अद्वेष्टा सर्वभूतानां मैत्रः करुण एव च\thinspace{\devanagarifontsmall ।}
                 निर्ममो निरहंकारः समदुःखसुखः क्षमी\thinspace{\devanagarifontsmall ॥} }}

%Verse 9:41

{\devanagarifont स्तुतिनिन्दासमा ये च गुणातीतः स उच्यते {॥ ९:४१॥} \veg\dontdisplaylinenum }%
     \var{{\devanagarifont \numnoemph\vd\textbf{॰तीतः}\lem \mssALL, ॰तीत \msNb}}% 

{\devanagarifont तुल्यप्रियाप्रियो यश्च अरिमित्रसमस्तथा \thinspace{\dandab} \dontdisplaylinenum }%
     \var{{\devanagarifont \numemph\va\textbf{तुल्य॰}\lem \Ed, तुल्यः \mssCaCbCc\msNa\msNb\msNc}}% 
    \var{{\devanagarifont \numnoemph\vb\textbf{॰सम॰}\lem \mssALL, ॰समा॰ \msCc}}% 

%Verse 9:42

{\devanagarifont मानापमानयोस्तुल्यो गुणातीतः स उच्यते {॥ ९:४२॥} \veg\dontdisplaylinenum  }%
     \paral{{\devanagarifontsmall \vo {\englishfont \compare\ \BHG\ 14.24cd--25:}
                         तुल्यप्रियाप्रियो धीरस्तुल्यनिन्दात्मसंस्तुतिः\thinspace{\devanagarifontsmall ॥}
                         मानावमानयोस्तुल्यस्तुल्यो मित्रारिपक्षयोः\thinspace{\devanagarifontsmall ।}
                         सर्वारम्भपरित्यागी गुणातीतः स उच्यते\thinspace{\devanagarifontsmall ॥} }}

{\devanagarifont एष ते कथितो विप्र गुणसद्भावनिर्णयः \thinspace{\dandab} \dontdisplaylinenum }%
     \var{{\devanagarifont \numemph\va\textbf{ते}\lem \mssALL, तो \msNb}}% 
    \var{{\devanagarifont \numnoemph\vb\textbf{॰सद्भाव॰}\lem \mssALL, ॰मद्भाव॰ \Ed}}% 

%Verse 9:43

{\devanagarifont गुणयुक्तस्तु संसारी गुणातीतः पराङ्गतिः {॥ ९:४३॥} \veg\dontdisplaylinenum }%
     \var{{\devanagarifont \numnoemph\vd\textbf{गुणातीतः}\lem \msCa\msCc\msNa, गुणातीत \msCb\msNb\msNc\Ed\oo 
\textbf{पराङ्गतिः}\lem \Ed, पराङ्गतिम् \mssCaCbCc\msNa\msNb\msNc}}% 

{\devanagarifont 
\jump
\begin{center}
\ketdanda~इति वृषसारसंग्रहे त्रैगुण्यविशेषणीयो नामाध्यायो नवमः~\ketdanda
\end{center}
\dontdisplaylinenum\vers  }%
     \var{{\devanagarifont \numnoemph{\englishfont \Colo:}\textbf{॰विशेषणीयो}\lem \corr, ॰विशेषनीयो \mssCaCbCc\msNa\msNb\msNc\Ed\oo 
\textbf{नामाध्यायो नवमः}\lem \mssALL, नाम नवमो ऽध्यायः \Ed}}% 
\bekveg\szamveg
\vfill
\phpspagebreak

\versno=0\fejno=10
\thispagestyle{empty}

\centerline{\Large\devanagarifontbold [   दशमो ऽध्यायः  ]}{\vrule depth10pt width0pt} \fancyhead[CO]{{\footnotesize\devanagarifont वृषसारसंग्रहे  }}
\fancyhead[CE]{{\footnotesize\devanagarifont दशमो ऽध्यायः  }}
\fancyhead[LE]{}
\fancyhead[RE]{}
\fancyhead[LO]{}
\fancyhead[RO]{}
\szam\bek



\alalfejezet{कायतीर्थोपवर्णनम्}
\vers


{\devanagarifont विगतराग उवाच {\dandab}\dontdisplaylinenum  }%
 
{\devanagarifont कतमं सर्वतीर्थानां श्रेष्ठमाहुर्मनीषिनः \thinspace{\danda} \dontdisplaylinenum }%
     \var{{\devanagarifont \numemph\va\textbf{कतमं सर्व॰}\lem \mssALL, 
कतमसर्व॰ \msNb, कथमन्सर्व॰ \msNc}}% 
    \var{{\devanagarifont \numnoemph\vab\textbf{॰तीर्थानां श्रेष्ठ॰}\lem \mssALL, ॰तीर्था\lk\lk ष्ठ॰ \msCa}}% 
    \var{{\devanagarifont \numnoemph\vb\textbf{मनीषिनः}\lem \mssALL, मनीषिभिः \Ed}}% 
    \lacuna{\devanagarifontsmall {\englishfont Witnesses used for this chapter: \msCa\ ff.\thinspace 207r--208v, 
                                              \msCb\ ff.\thinspace 212v--214r, 
                                              \msCc\ ff.\thinspace 283v--285v,
                                              \msNa\ ff.\thinspace 14v--15v, 
                                              \msNb\ exp.\thinspace 55 (lower) -- 56 (lower),
                                              \msNc\ ff.\thinspace 222v--223v,
                                              \Ed\ pp.\thinspace 610--613; 
                                              \mssCaCbCc\ = \msCa + \msCb + \msCc} }%
  
%Verse 10:1

{\devanagarifont कथयस्व मुनिश्रेष्ठ यद्यस्ति भुवि कामदम् {॥ १०:१॥} \veg\dontdisplaylinenum }%
     \var{{\devanagarifont \numnoemph\vd\textbf{भुवि}\lem \mssALL, भूरि \Ed\oo 
\textbf{॰दम्}\lem \mssALL, ॰दः \msNa}}% 

{\devanagarifont अनर्थयज्ञ उवाच {\dandab}\dontdisplaylinenum  }%
 
{\devanagarifont अतिगुह्यमिदं प्रश्नं पृष्टः स्नेहाद्द्विजोत्तम \thinspace{\danda} \dontdisplaylinenum }%
     \var{{\devanagarifont \numemph\vb\textbf{स्नेहाद्द्वि॰}\lem \mssALL, स्नेहा द्वि॰ \msCc}}% 

%Verse 10:2

{\devanagarifont ब्रवीमि वः पुरावृत्तं नन्दिना कथितो ऽस्म्यहम् {॥ १०:२॥} \veg\dontdisplaylinenum }%
     \var{{\devanagarifont \numnoemph\vd\textbf{ऽस्म्यहम्}\lem \mssALL, स्मृहम् \msCc}}% 

{\devanagarifont नन्दिकेश्वर उवाच {\dandab}\dontdisplaylinenum  }%
     \var{{\devanagarifont \numemph\vo\textbf{नन्दि॰}\lem \mssALL, नन्दी॰ \msCb}}% 

{\devanagarifont कैलासशिखरे रम्ये सिद्धचारणसेविते \thinspace{\danda} \dontdisplaylinenum }%
     \var{{\devanagarifont \numnoemph\va\textbf{कैलास॰}\lem \mssALL, कैलाशे \Ed}}% 
    \paral{{\devanagarifontsmall \vab {\englishfont  \compare\ MBh 12.327.18cd:} मेरौ गिरिवरे रम्ये सिद्धचारणसेविते  }}

%Verse 10:3

{\devanagarifont तत्रासीनं शिवं साक्षाद्देवी वचनमब्रवीत् {॥ १०:३॥} \veg\dontdisplaylinenum }%
 
{\devanagarifont देव्युवाच {\dandab}\dontdisplaylinenum  }%
 
{\devanagarifont भगवन्देवदेवेश सर्वभूतजगत्पते \thinspace{\danda} \dontdisplaylinenum }%
     \var{{\devanagarifont \numemph\va\textbf{॰देवेश}\lem \mssALL, ॰देश \msCb}}% 
    \var{{\devanagarifont \numnoemph\vb\textbf{॰पते}\lem \mssALL, ॰पतिम् \msNaacorr}}% 

%Verse 10:4

{\devanagarifont प्रष्टुमिच्छाम्यहं त्वेकं धर्मगुह्यं सनातनम् {॥ १०:४॥} \veg\dontdisplaylinenum }%
     \var{{\devanagarifont \numnoemph\vc\textbf{धर्म॰}\lem \mssALL, ध\uncl{र्मं} \msNa}}% 

{\devanagarifont अतितीर्थं परं गुह्यं संसाराद्येन मुच्यते \thinspace{\dandab} \dontdisplaylinenum }%
     \var{{\devanagarifont \numemph\va\textbf{॰तीर्थं}\lem \mssALL, ॰तीर्थ \msNb\Ed}}% 
    \var{{\devanagarifont \numnoemph\vab\textbf{गुह्यं संसाराद्येन मुच्यते}\lem \mssALL, 
\uncl{ग}\lac  \uncl{सं}साराद्येन मुच्यते \msNb}}% 

%Verse 10:5

{\devanagarifont मनुष्याणां हितार्थाय ब्रूहि तत्त्वं महेश्वर {॥ १०:५॥} \veg\dontdisplaylinenum }%
     \var{{\devanagarifont \numnoemph\vd\textbf{॰श्वर}\lem \mssALL, ॰श्वरः \msCc}}% 

{\devanagarifont महेश्वर उवाच {\dandab}\dontdisplaylinenum  }%
 
{\devanagarifont को मां पृच्छति तं प्रश्नं मुक्त्वा त्वामेव सुन्दरि \thinspace{\danda} \dontdisplaylinenum }%
     \var{{\devanagarifont \numemph\va\textbf{तं प्रश्नं}\lem \msNa\msNb, तत्प्रश्न \msCa\msCb, तत्प्रश्नं \msCc\Ed, 
तं प्रश्न \msNc}}% 
    \var{{\devanagarifont \numnoemph\vb\textbf{मुक्त्वा}\lem \mssALL, मुक्ता \Ed}}% 

%Verse 10:6

{\devanagarifont शृणु वक्ष्यामि तं प्रश्नं देवैरपि सुदुर्लभम् {॥ १०:६॥} \veg\dontdisplaylinenum }%
     \var{{\devanagarifont \numnoemph\vc\textbf{तं प्रश्नं}\lem \msNc, तत्प्रश्नं \mssCaCbCc\msNa\msNb\Ed}}% 

{\devanagarifont कुरुक्षेत्रं प्रयागं च वाराणसीमतः परम् \thinspace{\dandab} \dontdisplaylinenum }%
 
%Verse 10:7

{\devanagarifont गङ्गाग्निं सोमतीर्थं च सूर्यपुष्करमानसम् {॥ १०:७॥} \veg\dontdisplaylinenum }%
     \var{{\devanagarifont \numemph\vc\textbf{गङ्गाग्निं}\lem \msCa\msCb, गङ्गाग्नि \msCc\msNa\msNb\msNc, गङ्गाऽग्नि॰ \Ed}}% 

{\devanagarifont नैमिषं बिन्दुसारं च सेतुबन्धं सुरद्रहम् \thinspace{\dandab} \dontdisplaylinenum }%
     \var{{\devanagarifont \numemph\va\textbf{नैमिषं}\lem \mssALL, नेमिस \msNc}}% 
    \var{{\devanagarifont \numnoemph\vb\textbf{॰बन्धं}\lem \mssALL, ॰बन्ध॰ \Ed\oo 
\textbf{॰द्रहम् }\lem \mssALL, ॰ह्रदं \Ed}}% 

%Verse 10:8

{\devanagarifont घण्टिकेश्वरवागीशं ज्ञात्वा निश्चयपापहा {॥ १०:८॥} \veg\dontdisplaylinenum }%
     \var{{\devanagarifont \numnoemph\vc\textbf{॰वागीशं}\lem \mssALL, \lac \uncl{गीश} \msNb}}% 
    \var{{\devanagarifont \numnoemph\vd\textbf{निश्चयपापहा}\lem \mssALL, 
निश्च\uncl{य}\lk\lk\lk\  \msCa}}% 

{\devanagarifont उमोवाच {\dandab}\dontdisplaylinenum  }%
 
{\devanagarifont एवमादि महादेव पूर्ववत्कथितास्म्यहम् \thinspace{\danda} \dontdisplaylinenum }%
     \var{{\devanagarifont \numemph\vb\textbf{कथिता॰}\lem \msCa\msCc\msNa\msNc, कथितो \msCb\msNb\Ed}}% 

%Verse 10:9

{\devanagarifont स्वर्गभोगप्रदं तीर्थमेतेषां सुरनायक {॥ १०:९॥} \veg\dontdisplaylinenum }%
     \var{{\devanagarifont \numnoemph\vcd\textbf{तीर्थमे॰}\lem \mssALL, तीर्थंमे॰ \msCc}}% 
    \var{{\devanagarifont \numnoemph\vd\textbf{सुरनायक}\lem \msCapcorr\msNa\msNc, सुरनाक \msCaacorr, सुरनायकम् \msCb\msCc\msNb\Ed}}% 

{\devanagarifont कथं मुच्येत संसाराज्ज्ञानमात्रेण ईश्वर \thinspace{\dandab} \dontdisplaylinenum }%
     \var{{\devanagarifont \numemph\va\textbf{कथं}\lem \mssALL, कथ \msCb}}% 
    \var{{\devanagarifont \numnoemph\vb\textbf{ज्ञान॰}\lem \mssALL, ज्ञात॰ \msCb\oo 
\textbf{ईश्वर}\lem \mssALL, चेश्वर \msNa}}% 

%Verse 10:10

{\devanagarifont कौतूहलं महज्जातं छिन्धि संशयकारकम् {॥ १०:१०॥} \veg\dontdisplaylinenum }%
     \var{{\devanagarifont \numnoemph\vc \lem \mssCaCbCc\Ed, कौतूहलम्म\uncl{हो}ज्जातं \msNa, 
कौहलम्महज्जातं \msNbacorr, 
कौ\uncl{तू}हलम्महज्जातं \msNbpcorr, 
कोतूहलं महज्जातं \msNc}}% 
    \var{{\devanagarifont \numnoemph\vd\textbf{॰कारकम्}\lem \Ed, ॰कारक \mssCaCbCc\msNb\msNc, ॰कारकः \msNa}}% 

{\devanagarifont रुद्र उवाच {\dandab}\dontdisplaylinenum  }%
 
{\devanagarifont किं न जानामि तत्तीर्थं सुलभं दुर्लभं च यत् \thinspace{\danda} \dontdisplaylinenum }%
     \var{{\devanagarifont \numemph\va\textbf{जानामि}\lem \mssCaCbCc\msNb, जाना\uncl{मि} \msNaacorr, जाना\uncl{सि} \msNapcorr, 
जानासि \msNc\Ed}}% 
    \var{{\devanagarifont \numnoemph\vb\textbf{दुर्लभं च}\lem \msCa\msNa\msNb\Ed, दुलभञ्च \msCb\msNc, दुल्लभञ्च \msCc}}% 

%Verse 10:11

{\devanagarifont सुलभं गुरुसेवीनां दुर्लभं तद्विवर्जयेत् {॥ १०:११॥} \veg\dontdisplaylinenum }%
     \var{{\devanagarifont \numnoemph\vc \lem \mssALL, 
\lk\lk \lk\lk \lk\lk वीनां \msCa}}% 
    \var{{\devanagarifont \numnoemph\vd\textbf{॰वर्जयेत्}\lem \mssALL, ॰वर्जये \msNa, ॰वर्जनात् \Ed}}% 


\alalalfejezet{कुरुक्षेत्रम्}

{\devanagarifont कुरुः पुरुष विज्ञेयः शरीरं क्षेत्र उच्यते \thinspace{\dandab} \dontdisplaylinenum }%
     \var{{\devanagarifont \numemph\va\textbf{कुरुः}\lem \mssALL, गुरुः \msNb\oo 
\textbf{पुरुष}\lem \Ed, पुरुषः \mssCaCbCc\msNa\msNb\ \unmetr, पुरुषो \msNc\ \unmetr}}% 
    \var{{\devanagarifont \numnoemph\vb\textbf{शरीरं}\lem \mssALL, शरी\uncl{र} \msCa\oo 
\textbf{क्षेत्र उच्यते}\lem \mssALL, क्षेत्रमुच्यते \msNa}}% 
    \paral{{\devanagarifontsmall \vb {\englishfont \compare\ \BHG\ 13.1:}
                         इदं शरीरं कौन्तेय क्षेत्रमित्यभिधीयते\thinspace{\devanagarifontsmall ।}
                         एतद्यो वेत्ति तं प्राहुः क्षेत्रज्ञ इति तद्विदः\thinspace{\devanagarifontsmall ॥} }}

%Verse 10:12

{\devanagarifont शरीरस्थं कुरुक्षेत्रं सर्वतीर्थफलप्रदम् {॥ १०:१२॥} \veg\dontdisplaylinenum }%
     \var{{\devanagarifont \numnoemph\vc\textbf{॰स्थं}\lem \mssALL, ॰स्थ \msNc\oo 
\textbf{॰क्षेत्रं}\lem \mssALL, ॰क्षेत्र \msNc}}% 

{\devanagarifont सर्वयज्ञफलावाप्तिः सर्वदानफलानि च \thinspace{\dandab} \dontdisplaylinenum }%
     \paral{{\devanagarifontsmall \vab {\englishfont \similar\ \UMS\ 21.48cd:}
                                 सर्वयज्ञफलावाप्तिः सर्वदानफलं लभेत् }}

%Verse 10:13

{\devanagarifont सर्वव्रततपश्चीर्णं तत्फलं सकलं भवेत् {॥ १०:१३॥} \veg\dontdisplaylinenum }%
     \var{{\devanagarifont \numemph\vd\textbf{तत्फलं}\lem \mssALL, तत्फल \msNc}}% 

{\devanagarifont एवमेव फलं तेषां तीर्थपञ्चदशेषु च \thinspace{\dandab} \dontdisplaylinenum }%
     \var{{\devanagarifont \numemph\vb\textbf{तीर्थपञ्चदशेषु}\lem \mssALL, तीर्थम्पंचदशैषु \msCb}}% 

%Verse 10:14

{\devanagarifont अनघानं महापुण्यं महातीर्थं महासुखम् {॥ १०:१४॥} \veg\dontdisplaylinenum }%
     \var{{\devanagarifont \numnoemph\vc \lem \msCb\msNc, \lk\lk \lk\lk \lk\lk पुण्य \msCa, 
अनप्याम्महापुण्यं \msCc\ \hypermetr, 
अनध्यानं महापुण्यं \msNa, अध्वानन्तु महापुण्यं \msNb, 
स्नानध्यानं महापुण्यं \Ed}}% 

{\devanagarifont देव्युवाच {\dandab}\dontdisplaylinenum  }%
 
{\devanagarifont अतीव रोमहर्षो मे जातो ऽस्ति त्रिदशेश्वर \thinspace{\danda} \dontdisplaylinenum }%
     \var{{\devanagarifont \numemph\va\textbf{अतीव}\lem \mssALL, अवीव \msCb}}% 
    \var{{\devanagarifont \numnoemph\vb\textbf{ऽस्ति}\lem \mssALL, स्मि \msNb\oo 
\textbf{त्रिदशेश्वर}\lem \mssALL, त्रिदशेश्वरः \msCc, त्रि\lac  शेश्वर \msNb}}% 

%Verse 10:15

{\devanagarifont सुलभं सुकरं सूक्ष्मं श्रुत्वा तुष्टिश्च मे गता {॥ १०:१५॥} \veg\dontdisplaylinenum }%
     \var{{\devanagarifont \numnoemph\vd\textbf{तुष्टिश्च}\lem \mssALL, तुष्टिञ्च \msCc\oo 
\textbf{गता}\lem \mssALL, गताः \msCb}}% 

{\devanagarifont चतुर्दश परो भूयः कथयस्व मनोहरम् \thinspace{\dandab} \dontdisplaylinenum }%
 
%Verse 10:16

{\devanagarifont प्रयागादि पृथक्त्वेन तत्त्वतस्तु सुरेश्वर {॥ १०:१६॥} \veg\dontdisplaylinenum }%
     \var{{\devanagarifont \numemph\vd\textbf{तत्त्वतस्तु}\lem \mssALL, तत्वत \msNaacorr}}% 


\alalalfejezet{प्रयागो वाराणसी च}

{\devanagarifont रुद्र उवाच {\dandab}\dontdisplaylinenum  }%
 
{\devanagarifont सुषुम्ना भगवती गङ्गा इडा च यमुना नदी \thinspace{\danda} \dontdisplaylinenum }%
     \var{{\devanagarifont \numemph\va\textbf{सुषुम्ना}\lem \mssALL, सुषुम्णा \Ed\oo 
\textbf{भगवती गङ्गा}\lem \mssALL, 
भगवती ग\lk\ \msCa, भवती गङ्गा \Ed}}% 

%Verse 10:17

{\devanagarifont एताः स्रोतोवहा नद्यः प्रयागः स विधीयते {॥ १०:१७॥} \veg\dontdisplaylinenum }%
     \var{{\devanagarifont \numnoemph\vc\textbf{एताः स्रोतोवहा}\lem \eme, एता श्रोतवहा \msCa\msNc\Ed, 
एते श्रोतावहा \msCb\msCc, एता श्रोत्रवहा \msNa\msNb}}% 

{\devanagarifont दक्षिणा वारुणी नासा वामनासा असि स्मृता \thinspace{\dandab} \dontdisplaylinenum }%
     \var{{\devanagarifont \numemph\va\textbf{दक्षिणा}\lem \mssALL, दक्षि\uncl{णं} \msCa, दक्षिणं \msCc\oo 
\textbf{वारुणी}\lem \msNapcorr\msNc\Ed, वरुणी \msCa\msCc\msNaacorr\msNb, वरुणा \msCb}}% 
    \var{{\devanagarifont \numnoemph\vb\textbf{॰नासा}\lem \mssALL, ॰ना \msCb\msNb}}% 

%Verse 10:18

{\devanagarifont वारुणा-असिमध्येन तेन वाराणसी स्मृता {॥ १०:१८॥} \veg\dontdisplaylinenum }%
     \var{{\devanagarifont \numnoemph\vc \lem \Ed, वरुणा असिमध्येन \msCa\msCb\msNa\msNc, 
वारुणन्नासमध्येत \msCc, 
वरुण असिमध्येन \msNb}}% 


\alalalfejezet{गङ्गा}

{\devanagarifont आकाशगङ्गा विख्याता तस्याः स्रवति चामृतम् \thinspace{\dandab} \dontdisplaylinenum }%
     \var{{\devanagarifont \numemph\vb\textbf{तस्याः}\lem \mssALL, तस्मा \msCc, तस्या \msNb}}% 

%Verse 10:19

{\devanagarifont अहोरात्रमविच्छिन्नं गङ्गा सा तेन उच्यते {॥ १०:१९॥} \veg\dontdisplaylinenum }%
     \var{{\devanagarifont \numnoemph\vd\textbf{तेन}\lem \mssALL, ते \msCc}}% 


\alalalfejezet{सोमतीर्थम्}

{\devanagarifont सोमतीर्थमिडा नाडी किङ्किणीरवचिह्निता \thinspace{\dandab} \dontdisplaylinenum }%
     \var{{\devanagarifont \numemph\va\textbf{॰तीर्थमिडा}\lem \mssALL, ॰तीर्थ इडा \msCb}}% 
    \var{{\devanagarifont \numnoemph\vb\textbf{किङ्किणी॰}\lem \mssALL, चिञ्चिनी॰ \msCc\oo 
\textbf{॰रव॰}\lem \mssALL, ॰रवि॰ \msCbacorr, ॰राव॰ \Ed\oo 
\textbf{॰चिह्निता}\lem \mssALL, ॰चिह्निका \msCc, ॰चिह्नता \msNb}}% 

%Verse 10:20

{\devanagarifont तं तु श्रुत्वा न संदेहः सर्वपापक्षयो भवेत् {॥ १०:२०॥} \veg\dontdisplaylinenum }%
     \var{{\devanagarifont \numnoemph\vc\textbf{तं तु}\lem \corr, \uncl{तन्तु} \msCa, तन्तु \msCb\msCc\msNa\msNc\Ed, 
त\uncl{त्तु} \msNb\oo 
\textbf{न संदेहः}\lem \mssALL, वरारोहेः \msCc}}% 


\alalalfejezet{सूर्यतीर्थम्}

{\devanagarifont सूर्यतीर्थं सुषुम्ना च नीरवारवसंयुता \thinspace{\dandab} \dontdisplaylinenum }%
     \var{{\devanagarifont \numemph\va\textbf{॰तीर्थं}\lem \mssALL, ॰तीर्थ \msNb\oo 
\textbf{सुषुम्ना}\lem \mssALL, सुषुम्णा \Ed}}% 
    \var{{\devanagarifont \numnoemph\vb\textbf{नीरवा॰}\lem \Ed, वीरवा॰ \msCa\msCc, चीरवा॰ \msCb\msNa\msNb\msNc\oo 
\textbf{॰युता}\lem \msCa\msNa\msNc\Ed, ॰युतम् \msCb\msCc, ॰युतां \msNb}}% 

%Verse 10:21

{\devanagarifont श्रुतिमात्राद्विमुच्येत पापराशिर्महानपि {॥ १०:२१॥} \veg\dontdisplaylinenum }%
     \var{{\devanagarifont \numnoemph\vc\textbf{॰मात्रा॰}\lem \mssALL, ॰माता॰ \msCc}}% 


\alalalfejezet{अग्नितीर्थम्}

{\devanagarifont अग्नितीर्थार्जुना नाडी ब्रह्मघोषमनोरमा \thinspace{\dandab} \dontdisplaylinenum }%
     \var{{\devanagarifont \numemph\va\textbf{॰र्जुना}\lem \mssALL, ॰जुना \msCc, ॰र्जुनं \Ed}}% 
    \var{{\devanagarifont \numnoemph\vb\textbf{॰रमा}\lem \mssALL, ॰रमाः \msNc\Ed}}% 

%Verse 10:22

{\devanagarifont तत्तदक्षरमाकर्ण्य अमृतत्वाय कल्पते {॥ १०:२२॥} \veg\dontdisplaylinenum }%
     \var{{\devanagarifont \numnoemph\vc\textbf{॰कर्ण्य}\lem \mssALL, ॰र्ण्य \msCb}}% 
    \var{{\devanagarifont \numnoemph\vd\textbf{कल्पते}\lem \msCb\msNc\Ed, क\lk \lac\  \msCa, कल्प्यते \msCc\msNa\msNb}}% 


\alalalfejezet{पुष्करम्}

{\devanagarifont पुष्करं हृदि मध्यस्थमष्टपत्त्रं सकर्णिकम् \thinspace{\dandab} \dontdisplaylinenum }%
     \var{{\devanagarifont \numemph\vb\textbf{॰पत्त्रं}\lem \msCb\msNa\msNc\Ed, \lk\lk\ \msCa, ॰पत्र \msCc\msNb\oo 
\textbf{॰कर्णिकम्}\lem \mssALL, \lk\lk\lk\  \msCa, ॰कर्णिकाम् \Ed}}% 

%Verse 10:23

{\devanagarifont चिन्तयेत्सूक्ष्म तन्मध्ये जन्ममृत्युविनाशनम् {॥ १०:२३॥} \veg\dontdisplaylinenum }%
     \var{{\devanagarifont \numnoemph\vc\textbf{सूक्ष्म}\lem \mssALL, \uncl{सूक्ष्म} \msCa, सूक्ष्मं \Ed}}% 


\alalalfejezet{मानसम्}

{\devanagarifont मानससरमध्यस्थं स हंसः कमलोपरि \thinspace{\dandab} \dontdisplaylinenum }%
     \var{{\devanagarifont \numemph\va\textbf{मानस॰}\lem \msCb\msNa, \uncl{मानस} \msCa, मानसं \msCc\msNb\msNc\Ed}}% 
    \var{{\devanagarifont \numnoemph\vb\textbf{स हंसः}\lem \conj, सहंस॰ \msCa\msCc\msNa\msNb\msNc\Ed, सहसं \msCb}}% 

%Verse 10:24

{\devanagarifont सलीलो लीलयाचारी परतः परपारगः {॥ १०:२४॥} \veg\dontdisplaylinenum }%
     \var{{\devanagarifont \numnoemph\vc\textbf{सलीलो}\lem \mssALL, सलीला \Ed}}% 
    \var{{\devanagarifont \numnoemph\vd\textbf{परतः}\lem \mssALL, परत \msNb}}% 


\alalalfejezet{नैमिषम्}

{\devanagarifont नैमिषं शृणु देवेशि निमिषा प्रत्ययो भवेत् \thinspace{\dandab} \dontdisplaylinenum }%
     \var{{\devanagarifont \numemph\vb \lem \mssALL, निमि प्रत्ययो भवेत् \msCb, 
नि\lac \uncl{षो} प्रत्ययो \uncl{भवेत्} \msNb}}% 

%Verse 10:25

{\devanagarifont सम्यग्छायां निरीक्षेत आत्मानो वा परस्य वा {॥ १०:२५॥} \veg\dontdisplaylinenum }%
     \var{{\devanagarifont \numnoemph\vd\textbf{आत्मनो}\lem \mssALL, \lk न्मनो \msCa, स्वात्मानो \Ed\oo 
\textbf{परस्य वा}\lem \mssALL, परस्य च \Ed}}% 

{\devanagarifont आयतमङ्गुलीमात्रं निमिषाक्षिः स पश्यति \thinspace{\dandab} \dontdisplaylinenum }%
     \var{{\devanagarifont \numemph\va\textbf{आयतमङ्गुली॰}\lem \conj, आयतप्यङ्गुली॰ \mssCaCbCc\msNa\msNb, 
आयातप्यङ्गुली॰ \msNc\Ed\oo 
\textbf{॰मात्रं}\lem \mssALL, ॰मात्र \msNc, ॰मध्ये \Ed}}% 
    \var{{\devanagarifont \numnoemph\vb\textbf{॰क्षिः}\lem \eme, ॰क्षि \mssCaCbCc\msNa\msNb\msNc\Ed}}% 

%Verse 10:26

{\devanagarifont दृष्ट्वा प्रत्ययमेवं हि नैमिषज्ञः स उच्यते {॥ १०:२६॥} \veg\dontdisplaylinenum }%
     \var{{\devanagarifont \numnoemph\vd\textbf{नैमिषज्ञः}\lem \mssALL, नैमिसंज्ञः \msCb, नैमिषज्ञ \msCc}}% 


\alalalfejezet{बिन्दुसरः}

{\devanagarifont तीर्थं बिन्दुसरं नाम शृणु वक्ष्यामि सुन्दरि \thinspace{\dandab} \dontdisplaylinenum }%
     \var{{\devanagarifont \numemph\va\textbf{तीर्थं बिन्दु॰}\lem \mssALL, तीर्थमिन्दु॰ \Ed}}% 

%Verse 10:27

{\devanagarifont देहमध्ये हृदि ज्ञेयं हृदिमध्ये तु पङ्कजम् {॥ १०:२७॥} \veg\dontdisplaylinenum }%
     \var{{\devanagarifont \numnoemph\vc\textbf{हृदि ज्ञेयं}\lem \mssALL, \om\ \msCb}}% 
    \paral{{\devanagarifontsmall \vo {\englishfont \compare\ \NISVK\ 5.55:}
                 एतेषां नादमध्ये तु शिवं तत्र व्यवस्थितः\thinspace{\devanagarifontsmall ।}
                 हृदयं देहमध्ये तु तत्र पद्मं व्यवस्थितम्\thinspace{\devanagarifontsmall ॥} }}

{\devanagarifont कर्णिका पद्ममध्ये तु बिन्दुः कर्णिकमध्यतः \thinspace{\dandab} \dontdisplaylinenum }%
     \var{{\devanagarifont \numemph\va\textbf{॰मध्ये}\lem \mssALL, ॰ध्ये \msCa, ॰पध्ये \msNa}}% 

%Verse 10:28

{\devanagarifont बिन्दुमध्ये स्थितो नादः स नादः केन भिद्यते {॥ १०:२८॥} \veg\dontdisplaylinenum }%
     \var{{\devanagarifont \numnoemph\vc\textbf{बिन्दुमध्ये}\lem \mssALL, \uncl{बिन्दु}\lk\lk\ \msCa}}% 
    \var{{\devanagarifont \numnoemph\vd\textbf{भिद्यते}\lem \mssALL, \uncl{वि}द्यते \msCa, विद्यते \msCc}}% 
    \paral{{\devanagarifontsmall \vo {\englishfont \compare\ \NISVK\ 5.56:}
                 कर्णिका पद्ममध्ये तु अकारं तस्य मध्यतः\thinspace{\devanagarifontsmall ।}
                 तस्य मध्ये विनिष्क्रान्तं नादं परमदुर्लभम्\thinspace{\devanagarifontsmall ॥} }}

{\devanagarifont उकारं च मकारं च भित्त्वा नादो विनिर्गतः \thinspace{\dandab} \dontdisplaylinenum }%
     \var{{\devanagarifont \numemph\va\textbf{उकारं च मकारं}\lem \mssALL, उकारश्च मकारश् \Ed}}% 
    \paral{{\devanagarifontsmall \vab {\englishfont = \NISVK\ 5.57ab} }}

%Verse 10:29

{\devanagarifont तं विदित्वा विशालाक्षि सो ऽमृतत्वं लभेत च {॥ १०:२९॥} \veg\dontdisplaylinenum }%
     \var{{\devanagarifont \numnoemph\vd\textbf{सो ऽमृतत्वं}\lem \mssALL, सोम्यतत्वं \msCc, सोमतत्वं \Ed\oo 
\textbf{च}\lem \mssALL, वा \Ed}}% 


\alalalfejezet{सेतुबन्धम्}

\nemslokalong


\ujvers\nemsloka {
{\devanagarifont वक्ष्ये ते सेतुबन्धं दुरितमलहरं नादतोयप्रवाहं }%
  \dontdisplaylinenum}    \var{{\devanagarifont \numemph\va\textbf{ते}\lem \mssALL, \om\ \msCaacorr, हं \msCc\oo 
\textbf{॰बन्धं}\lem \mssALL, ॰बन्धूं \msCb\oo 
\textbf{॰तोय॰}\lem \mssALL, ॰तोयं \msNb}}% 


\nemslokab

{\devanagarifont जिह्वाकण्ठोरकूला स्वरगणपुलिनावर्तघोषा तरङ्गा  \danda\dontdisplaylinenum }%
     \var{{\devanagarifont \numnoemph\vb\textbf{॰कण्ठोर॰}\lem \conj, ॰कण्ठोरु॰ \mssCaCbCc\msNa\msNb\msNc\Ed\oo 
\textbf{स्वर॰}\lem \mssALL, सुर॰ \msCc\Ed}}% 

\nemslokac

{\devanagarifont कुम्भीराघोषमीना दशगणमकरा भीमनक्रा विसर्गा }%
  \dontdisplaylinenum    \var{{\devanagarifont \numnoemph\vc\textbf{॰मीना}\lem \mssALL, ॰माना \Ed\oo 
\textbf{दश॰}\lem \mssALL, \lk\lk\ \msCa\oo 
\textbf{विसर्गा}\lem \mssCaCbCc, विसर्गाः \msNa\msNb\msNc\Ed}}% 

%Verse 10:30


\nemslokad

{\devanagarifont सानुस्वारे गभीरे मदसुखरसनं सेतुबन्धं व्रजस्व {॥ १०:३०॥} \veg\dontdisplaylinenum }%
     \var{{\devanagarifont \numnoemph\vd\textbf{॰स्वारे}\lem \msCa\msCb\msNc\Ed, ॰सारे \msCc, 
॰स्वारो \msNa, ॰स्वा\uncl{रेण} \msNb\ \unmetr\oo 
\textbf{गभीरे}\lem \msCa\msCb\msNc, गम्भीरे \msCc\msNb\Ed, \uncl{गं}भीरे \msNa\oo 
\textbf{॰रसनं}\lem \mssALL, ॰रमणं \Ed\oo 
\textbf{॰बन्धं}\lem \mssALL, ॰बन्ध \msCb\oo 
\textbf{व्रजस्व}\lem \mssALL, रमस्व \Ed}}% 


\alalalfejezet{सुरद्रहः}

\nemslokalong


\ujvers\nemsloka {
{\devanagarifont सप्तद्वीपान्तमध्ये शृणु शशिवदने सर्वदुःखान्तलाभम् }%
  \dontdisplaylinenum}    \var{{\devanagarifont \numemph\va\textbf{॰द्वीपा॰}\lem \mssALL, ॰दीपा॰ \msNc}}% 


\nemslokab

{\devanagarifont ईशानेनाभिजुष्टं हृदि ह्रद विमलं नादशीताम्बुपूर्णम्  \danda\dontdisplaylinenum }%
     \var{{\devanagarifont \numnoemph\vb\textbf{ईशानेनाभिजुष्टं}\lem \msCc\msNa\msNc\Ed, ईशानेनाभिदुष्टं \msCa\msNb, 
ईशानेभिदुष्टं \msCbacorr, ईशानेभि\lac  दुष्टं \msCbpcorr\oo 
\textbf{विमलं नादशीता॰}\lem \mssALL, 
विमलान्नादशीता॰ \msNb, विमलं नामशिता॰ \Ed}}% 

\nemslokac

{\devanagarifont तत्रैकं जातपद्मं प्रकृतिदलयुतं केशरं शक्तिभिन्नं }%
  \dontdisplaylinenum    \var{{\devanagarifont \numnoemph\vc\textbf{केशरं}\lem \msCb\Ed, केशर॰ \msCa\msCc\msNa\msNc\ \unmetr, केश्वर॰ \msNb\ \unmetr}}% 

%Verse 10:31


\nemslokad

{\devanagarifont पञ्चव्योमप्रशस्तं गतिपरमपदं प्राप्तुकामेन सेव्यम् {॥ १०:३१॥} \veg\dontdisplaylinenum }%
     \var{{\devanagarifont \numnoemph\vd\textbf{॰व्योम॰}\lem \mssALL, ॰व्यो\uncl{मं} \msNa\oo 
\textbf{॰शस्तं ग॰}\lem \mssALL, ॰शस्वङ्ग॰ \msCc\oo 
\textbf{॰परम॰}\lem \mssALL, ॰परमं \msNa\ \unmetr\oo 
\textbf{सेव्यम्}\lem \mssALL, सर्वम् \Ed}}% 


\alalalfejezet{घण्टिकेश्वरम्}

\nemslokalong


\ujvers\nemsloka {
{\devanagarifont †नाड्यैकासङ्गतानि† निपतितममृतं घण्टिकापारकेण }%
  \dontdisplaylinenum}    \var{{\devanagarifont \numemph\va\textbf{निपतितममृतं}\lem \mssALL, निपतितममृत॰ \msNa\ \unmetr, 
नि\lac  तममृतं \msNb\oo 
\textbf{॰पारकेण}\lem \msCa\msCb\msNa\msNc, ॰याङ्करेण \msCc\Ed, ॰\uncl{पारकेन} \msNb}}% 


\nemslokab

{\devanagarifont तृप्यन्ते तेन नित्यं हृदि कमलपुटं स्थाणुभूतान्तरात्मा  \danda\dontdisplaylinenum }%
     \var{{\devanagarifont \numnoemph\vb\textbf{॰पुटं}\lem \mssALL, ॰पुट \msCb\oo 
\textbf{स्थाणु॰}\lem \conj, स्थानु॰ \mssCaCbCc\msNa\msNc, 
\uncl{स्थान}॰ \msNb, स्थान॰ \Ed}}% 

\nemslokac

{\devanagarifont यं पश्यन्तीशभक्ताः कलिकलुषहरं व्यापिनं निष्प्रपञ्चं }%
  \dontdisplaylinenum    \var{{\devanagarifont \numnoemph\vc\textbf{यं पश्यन्तीशभक्ताः}\lem \msNa, यं पश्यन्तीशभक्ता \msCa\msNb, 
यं पश्यन्तीशभर्त्ताः \msCb, यं पस्यन्तीसभक्त्या \msCc, 
यत्पश्यन्तीशभक्त्या \msNc, यं पश्यन्नीशमक्षा \Ed\oo 
\textbf{॰प्रपञ्चम्}\lem \msCa\msNa\msNb\msNc, ॰प्रपञ्च \msCb\msCc\Ed}}% 

%Verse 10:32


\nemslokad

{\devanagarifont देवेशं घण्टिकेशामरभवमभवं तीर्थमाकाशबिन्दुम् {॥ १०:३२॥} \veg\dontdisplaylinenum }%
     \var{{\devanagarifont \numnoemph\vd\textbf{देवेशं}\lem \msCb\msNb\Ed, देव्येशं \msCa\msCc\msNa, देव्येश \msNc\oo 
\textbf{घण्टिकेशामर॰}\lem \msCc, घण्टिकेशमर॰ \msCa\msCb\msNb\msNc, 
घण्टिकेशं मर॰ \msNa, घाण्टकेशामर॰ \Ed\oo 
\textbf{॰भवं तीर्थम्}\lem \eme, ॰भवन्तीर्थम् \msCb\msCc\msNa\msNb\msNc\Ed, भव\lk\lk र्थम् \msCa\oo 
\textbf{॰बिन्दुम्}\lem \mssALL, ॰बिन्दु \msCc}}% 


\alalalfejezet{वागीश्वरतीर्थम्}

\nemslokalong


\ujvers\nemsloka {
{\devanagarifont मीमांसारत्नकूला क्रमपदपुलिना शैवशास्त्रार्थतोया }%
  \dontdisplaylinenum}    \var{{\devanagarifont \numemph\va\textbf{शैव॰}\lem \mssALL, शर्व॰ \Ed}}% 


\nemslokab

{\devanagarifont मीनौघा पञ्चरात्रं श्रुतिकुटिलगतिः स्मार्तवेगा तरङ्गा  \danda\dontdisplaylinenum }%
     \var{{\devanagarifont \numnoemph\vb\textbf{मीनौघा॰}\lem \msNa\msNb\Ed,  मीनोघा॰ \mssCaCbCc\msNc\oo 
\textbf{पञ्चरात्रं}\lem \mssALL, पञ्चशत्रं \Ed\oo 
\textbf{॰गतिः}\lem \corr, ॰गति \mssCaCbCc\msNa\msNb\msNc\Ed\oo 
\textbf{॰स्मार्तवेगा तरङ्गा}\lem \mssALL, ॰स्मा\lac  \uncl{वेगा तरङ्गा} \msNb, 
॰स्मार्तवेगास्तरङ्गा \Ed}}% 

\nemslokac

{\devanagarifont योगावर्तातिशोभा उपनिषदिवहा भारतावर्तफेना }%
  \dontdisplaylinenum    \var{{\devanagarifont \numnoemph\vc\textbf{॰वहा भारता॰}\lem \mssALL, महाभारता॰ \msNb}}% 

%Verse 10:33


\nemslokad

{\devanagarifont पञ्चाशद्व्योमरूपी रसभवननदी तीर्थ वागीश्वरीयम् {॥ १०:३३॥} \veg\dontdisplaylinenum }%
     \var{{\devanagarifont \numnoemph\vd\textbf{॰शद्व्योम॰}\lem \mssALL, ॰शव्योम॰ \msNa, ॰सद्व्योम॰ \Ed}}% 

\nemslokalong


\ujvers\nemsloka {
{\devanagarifont यस्तं वेत्ति स वेत्ति वेदनिखिलं संसारदुःखच्छिदं }%
  \dontdisplaylinenum}    \var{{\devanagarifont \numemph\va\textbf{यस्तं}\lem \mssALL, यस्त॰ \msCa\msCb\oo 
\textbf{स वेत्ति}\lem \mssALL, \uncl{न} वेत्ति \msNc}}% 


\nemslokab

{\devanagarifont जन्मव्याधिवियोगतापमरणं क्लेशार्णवं दुःसहम्  \danda\dontdisplaylinenum }%
     \var{{\devanagarifont \numnoemph\vb\textbf{॰मरणं}\lem \mssALL, ॰मरण \msNc\oo 
\textbf{॰र्णवं}\lem \mssALL, ॰ण्णवं \msNa, ॰र्णव \Ed}}% 

\nemslokac

{\devanagarifont गर्भावासमतीव सह्यविषयं दुस्तीर्यदुःखालयं }%
  \dontdisplaylinenum    \var{{\devanagarifont \numnoemph\vc\textbf{गर्भावासम्}\lem \mssALL, गर्भोवासम् \Ed\oo 
\textbf{॰विषयं}\lem \msCa\msCb\msNb, ॰विषमं \msCc\msNa\msNc\Ed\oo 
\textbf{॰लयम्}\lem \mssALL, ॰लय\uncl{ः} \msNa\oo 
\textbf{दुस्तीर्य॰}\lem \mssALL, दुस्तीर्यः \msNc}}% 

%Verse 10:34


\nemslokad

{\devanagarifont प्राप्तं तेन न संशयः शिवपदं दुष्प्राप्य देवैरपि {॥ १०:३४॥} \veg\dontdisplaylinenum }%
     \var{{\devanagarifont \numnoemph\vd \lem \msCa\msCbpcorr\msNa\msNc, 
प्राप्तं तेन न संशयः   शिवदं दुष्प्राप्य देवैरपि \msCbacorr, 
प्राप्तं तेन न संशयं शिवपदं दुष्प्राप्य देवैरपि \msCc\Ed, 
प्रा\lac  \uncl{यः शिव} \lk\lk \lk\lk  \uncl{य देवैरपि} \msNb}}% 

\vers


{\devanagarifont 
\jump
\begin{center}
\ketdanda~इति वृषसारसंग्रहे कायतीर्थोपवर्णनो नामाध्यायो दशमः~\ketdanda
\end{center}
\dontdisplaylinenum\vers  }%
     \var{{\devanagarifont \numnoemph{\englishfont \Colo:}\textbf{कायतीर्थोपवर्णनो}\lem \mssALL, 
कायती\lk\lk \lk र्ण्णनो \msCa\oo 
\textbf{नामाध्यायो दशमः}\lem \mssALL, नाम दशमो ऽध्यायः \Ed}}% 

\nemslokanormal

\bekveg\szamveg
\vfill
\phpspagebreak

\versno=0\fejno=11
\thispagestyle{empty}

\centerline{\Large\devanagarifontbold [   एकादशमो ऽध्यायः  ]}{\vrule depth10pt width0pt} \fancyhead[CO]{{\footnotesize\devanagarifont वृषसारसंग्रहे  }}
\fancyhead[CE]{{\footnotesize\devanagarifont एकादशमो ऽध्यायः  }}
\fancyhead[LE]{}
\fancyhead[RE]{}
\fancyhead[LO]{}
\fancyhead[RO]{}
\szam\bek


\vers



\alalfejezet{चतुराश्रमधर्मविधानः}
{\devanagarifont देव्युवाच {\dandab}\dontdisplaylinenum  }%
 
{\devanagarifont सर्वयज्ञः परश्रेष्ठ अस्ति अन्यः सुरोत्तम \thinspace{\danda} \dontdisplaylinenum  }%
     \var{{\devanagarifont \numemph\vb\textbf{अन्यः}\lem \msCb\msNa\msNc, अन्य \msCa\msCc\msNb, चान्या \Ed\oo 
\textbf{॰त्तम}\lem \mssALL, ॰त्तमः \msNc}}% 
    \paral{{\devanagarifontsmall {\englishfont Witnesses used for this chapter:    \msCa\ ff.\thinspace 208v--210r,
                                                     \msCb\ ff.\thinspace 214r--215v,
                                                     \msCc\ ff.\thinspace 285v--287v,
                                                     \msNa\ ff.\thinspace 15v--17v,
                                                     \msNb\ ff.\thinspace 221v--223v 
                                                         (exp.\thinspace 56 lower -- 58 lower),
                                                     \msNc\ ff.\thinspace 223v--225v;
                                                     \Ed\ pp.\thinspace 613--617; 
                                                     \mssCaCbCc~= \msCa + \msCb + \msCc } }}

%Verse 11:1

{\devanagarifont अल्पक्लेशमनायास अर्थप्रायं विनेश्वर {॥ ११:१॥} \veg\dontdisplaylinenum }%
     \var{{\devanagarifont \numnoemph\vc\textbf{॰नायास}\lem \mssALL, 
॰नाया\uncl{सं} \msNa, ॰\uncl{नाया}सं \msNb}}% 
    \var{{\devanagarifont \numnoemph\vd\textbf{॰र्थप्रायं}\lem \msNapcorr\msNc, ॰र्थप्राय \mssCaCbCc, 
॰र्थप्रार्थप्रायं \msNaacorr, ॰\uncl{र्थप्राय} \msNb, 
॰थाम्नाय \Ed\oo 
\textbf{विनेश्वर}\lem \mssALL, \uncl{विनेश्वर} \msNb, सुरेश्वर \Ed}}% 

{\devanagarifont सर्वयज्ञफलावाप्ति दैवतैश्चापि पूजितम् \thinspace{\dandab} \dontdisplaylinenum }%
     \var{{\devanagarifont \numemph\va\textbf{दैवतै॰}\lem \msCa\msCb\msNa\Ed, देवतै॰ \msCc\msNc, \uncl{देवतै} \msNb}}% 

%Verse 11:2

{\devanagarifont कथयस्व सुरश्रेष्ठ मानुषाणां हिताय वै {॥ ११:२॥} \veg\dontdisplaylinenum }%
     \var{{\devanagarifont \numnoemph\vcd\textbf{॰श्रेष्ठ मानुषाणां हिताय वै}\lem \mssALL, 
॰श्रे\lac\  \msNb}}% 

{\devanagarifont महेश्वर उवाच {\dandab}\dontdisplaylinenum  }%
     \var{{\devanagarifont \numemph\vo\textbf{महे॰}\lem \mssALL, मेहे॰ \msNc}}% 

{\devanagarifont न तुल्यं तव पश्यामि दया भूतेषु भामिनि \thinspace{\danda} \dontdisplaylinenum }%
     \var{{\devanagarifont \numnoemph\va\textbf{तुल्यं तव}\lem \mssALL, \lac\  \msCa}}% 
    \var{{\devanagarifont \numnoemph\vb\textbf{भामिनि}\lem \mssALL, भामि \msCc}}% 

%Verse 11:3

{\devanagarifont किमन्यत्कथयिष्यामि दया यत्र न विद्यते {॥ ११:३॥} \veg\dontdisplaylinenum }%
     \var{{\devanagarifont \numnoemph\vc\textbf{किमन्य॰}\lem \mssALL, किम्यन्य॰ \msNb}}% 

{\devanagarifont सदाशिवमुखात्पूर्वं श्रुतं मे वरसुन्दरि \thinspace{\dandab} \dontdisplaylinenum }%
 
%Verse 11:4

{\devanagarifont शृणु देवि प्रवक्ष्यामि धर्मसारमनुत्तमम् {॥ ११:४॥} \veg\dontdisplaylinenum }%
     \var{{\devanagarifont \numemph\vc\textbf{देवि प्रवक्ष्यामि}\lem \msCb\msCc\msNa\msNb, ते देवि वक्ष्यामि \msCa\msNc\Ed}}% 
    \var{{\devanagarifont \numnoemph\vd\textbf{॰सारमनुत्तमम्}\lem \mssALL, ॰सारसमुच्चयम् \msCc}}% 


\alalfejezet{गृहस्थः(?)}
{\devanagarifont विनार्थेन तु यो यज्ञः स यज्ञः सार्वकामिकः \thinspace{\dandab} \dontdisplaylinenum }%
     \var{{\devanagarifont \numemph\vb\textbf{यज्ञः}\lem \mssALL, यज्ञ \Ed\oo 
\textbf{सार्वकामिकः}\lem \msCb\Ed, सर्वकालिकः \msCa\msNc, 
सर्वकामिक \msCc, सार्वकालिकः \msNa, सार्वकामिकाः \msNb}}% 
    \paral{{\devanagarifontsmall \vab {\englishfont See a sequence or list of the four āśramas in 4.75 above:}
                 गृहस्थो ब्रह्मचारी च वानप्रस्थो ऽथ भैक्षुकः;
                 {\englishfont see also 5.9:} 
                 एतच्छौचं गृहस्थानां द्विगुणं ब्रह्मचारिणाम्\thinspace{\devanagarifontsmall ।}
                 वानप्रस्थस्य त्रिगुणं यतीनां तु चतुर्गुणम्\thinspace{\devanagarifontsmall ॥} }}

%Verse 11:5

{\devanagarifont अक्षयश्चाव्ययश्चैव सर्वपातकनाशनः {॥ ११:५॥} \veg\dontdisplaylinenum }%
     \var{{\devanagarifont \numnoemph\vc\textbf{अक्षयश्चाव्ययश्}\lem \msCb\msNb\msNc\Ed, अक्षयं चाव्ययं \msCa\msCc\msNa}}% 
    \var{{\devanagarifont \numnoemph\vd\textbf{॰नाशनः}\lem \msCa\msNa\msNb\msNc, ॰नाशनम् \msCb\Ed, ॰नाशन \msCc}}% 

{\devanagarifont बहुविघ्नकरो ह्यर्थो बह्वायासकरस्तथा \thinspace{\dandab} \dontdisplaylinenum }%
     \var{{\devanagarifont \numemph\va\textbf{॰करो}\lem \mssALL, ॰करा \msCc\Ed\oo 
\textbf{ह्यर्थो}\lem \mssALL, ह्येर्थो \Ed}}% 
    \var{{\devanagarifont \numnoemph\vb\textbf{करस्तथा}\lem \mssALL, करतस्था \Ed}}% 

%Verse 11:6

{\devanagarifont ब्रह्महत्या इवेन्द्रस्य प्रविभागफला स्मृता {॥ ११:६॥} \veg\dontdisplaylinenum }%
     \var{{\devanagarifont \numnoemph\vd\textbf{प्रविभाग॰}\lem \msCb, प्रविभोग॰ \msCa\msCc(?)\msNa\msNc\Ed, प्रतिभोग॰ \msNb\oo 
\textbf{॰फला स्मृता}\lem \msCc, ॰फलः स्मृतः \msCapcorr\msCb\msNa\msNb\msNc, 
॰फल स्मृतः \msCaacorr, ॰प्रदः स्मृतः \Ed}}% 

{\devanagarifont पञ्चशोध्येन शोध्येत अर्थयज्ञो वरानने \thinspace{\dandab} \dontdisplaylinenum }%
     \var{{\devanagarifont \numemph\vb\textbf{॰यज्ञो}\lem \mssALL, ॰यज्ञ \msCc}}% 

%Verse 11:7

{\devanagarifont शोधिते तु फलं शुद्धमशुद्धे निष्फलं भवेत् {॥ ११:७॥} \veg\dontdisplaylinenum }%
     \var{{\devanagarifont \numnoemph\vcd\textbf{शुद्धमशुद्धे}\lem \mssALL, 
शुद्धंमशुद्धे \msNa, शुद्धमशुद्धं \Ed}}% 

{\devanagarifont देव्युवाच {\dandab}\dontdisplaylinenum  }%
     \var{{\devanagarifont \numemph\vo\textbf{देव्युवाच}\lem \mssALL, \om\ \msNbacorr}}% 

{\devanagarifont पञ्चशोध्ये सुरश्रेष्ठ संशयो ऽत्र भवेन्मम \thinspace{\danda} \dontdisplaylinenum }%
     \var{{\devanagarifont \numnoemph\va\textbf{॰शोध्ये}\lem \mssCaCbCc\msNa, ॰शोध्य \msNb\msNc, ॰शोध्यः \Ed\oo 
\textbf{॰श्रेष्ठ}\lem \mssALL, ॰स्रे\uncl{म्न} \msCc}}% 
    \var{{\devanagarifont \numnoemph\vb\textbf{ऽत्र भवे॰}\lem \mssALL, ऽत्रा भव॰ \Ed}}% 

%Verse 11:8

{\devanagarifont कथयस्व विभागेन श्रोतुमिच्छामि तत्त्वतः {॥ ११:८॥} \veg\dontdisplaylinenum }%
 
{\devanagarifont रुद्र उवाच {\dandab}\dontdisplaylinenum  }%
 
{\devanagarifont मनःशुद्धिस्तु प्रथमं द्रव्यशुद्धिरतः परम् \thinspace{\danda} \dontdisplaylinenum }%
     \var{{\devanagarifont \numemph\vb\textbf{॰शुद्धिरतः}\lem \mssALL, ॰शुद्धिगतः \msNb}}% 

{\devanagarifont मन्त्रशुद्धिस्तृतीया तु कर्मशुद्धिरतः परम्  \danda\dontdisplaylinenum }%
     \var{{\devanagarifont \numnoemph\vc\textbf{मन्त्रशुद्धिस्तृतीया}\lem \mssALL, मन्त्रद्धि तृतीया \msNc}}% 
    \var{{\devanagarifont \numnoemph\vd\textbf{कर्मशुद्धि॰}\lem \mssALL, कर्मसिद्धि \msNc}}% 

%Verse 11:9

{\devanagarifont पञ्चमी सत्त्वशुद्धिस्तु क्रतुशुद्धिश्च पञ्चधा {॥ ११:९॥} \veg\dontdisplaylinenum }%
     \var{{\devanagarifont \numnoemph\ve\textbf{पञ्चमी}\lem \mssALL, पञ्चमं \Ed\oo 
\textbf{॰शुद्धिस्तु}\lem \mssALL, ॰शुद्धिश्च \msNa\Ed}}% 
    \var{{\devanagarifont \numnoemph\vf\textbf{॰शुद्धिश्च पञ्चधा}\lem \mssALL, ॰शुद्धिस्तु पञ्चधा \msCc, 
॰शुद्धिरतः परम् \msNa}}% 

{\devanagarifont मनःशुद्धिर्नाम अविपरीतभावनया \thinspace{\dandab} \dontdisplaylinenum  }%
     \var{{\devanagarifont \numemph\vab\textbf{॰शुद्धिर्ना॰}\lem \mssALL, ॰शुद्धि ना॰ \msCc\oo 
\textbf{॰भावनया}\lem \mssALL, ॰भावनवा \msNa, ॰भावनतया \msNb}}% 

%Verse 11:10

{\devanagarifont द्रव्यशुद्धिर्नाम अनन्यायोपार्जितद्रव्येन {॥ ११:१०॥} \veg\dontdisplaylinenum  }%
     \var{{\devanagarifont \numnoemph\vcd\textbf{॰शुद्धिर्ना॰}\lem \mssALL, ॰शुद्धि ना॰ \msCc\msNc\oo 
\textbf{अनन्यायो॰}\lem \msCb\msNa\msNb\msNc, अन\lac  यो॰ \msCa, अन्यायो॰ \msCc, स्वल्पोन्यायो॰ \Ed\oo 
\textbf{॰द्रव्येन}\lem \mssALL, ॰व्येन \msNb}}% 

{\devanagarifont मन्त्रशुद्धिर्नाम स्वरव्यञ्जनयुक्ततया \thinspace{\dandab} \dontdisplaylinenum  }%
     \var{{\devanagarifont \numemph\vab\textbf{मन्त्रशुद्धिर्ना॰}\lem \msCa\msCb\msNb\Ed, मन्त्रशुद्धि ना॰ \msCc\msNc, 
मन्त्रस्तुद्दिना॰ \msNa\oo 
\textbf{॰युक्ततया}\lem \mssALL, ॰युक्तया \msCb}}% 

{\devanagarifont क्रियाशुद्धिर्नाम यथाक्रमाविपरीततया  \danda\dontdisplaylinenum  }%
     \var{{\devanagarifont \numnoemph\vcd\textbf{॰शुद्धिर्ना॰}\lem \mssALL, ॰शुद्धि ना॰ \msCc\msNb\oo 
\textbf{॰क्रमा॰}\lem \mssALL, ॰क्रम॰ \msCc\oo 
\textbf{॰रीततया}\lem \mssALL, ॰रीतया \msCb, \lac  तया \msNc}}% 

%Verse 11:11

{\devanagarifont सत्त्वशुद्धिर्नाम रजस्तम-अप्रधानतया {॥ ११:११॥} \veg\dontdisplaylinenum  }%
     \var{{\devanagarifont \numnoemph\vef\textbf{॰शुद्धिर्ना॰}\lem \mssALL, ॰शुद्धि ना॰ \msCa\msCc\oo 
\textbf{॰धानतया}\lem \mssALL, ॰धानत \msNc}}% 

\vers


{\devanagarifont विधिमेवं यदा शुध्येद्यदि यज्ञं करोति हि \thinspace{\dandab} \dontdisplaylinenum }%
     \var{{\devanagarifont \numemph\va\textbf{॰धिमेवं यदा}\lem \msCb\Ed, ॰धिमेव यदा \msCa\msCc\msNa, ॰धिमेव य \msNb, 
॰धिमेवं यथा \msNc}}% 
    \var{{\devanagarifont \numnoemph\vab\textbf{शुध्येद्यदि}\lem \conj, सूयेद्यदि \msCa\msNa, पूर्य यदि \msCb, 
सूर्येद्यदि \msCc, सूयेद्यति \msNb, पूयेद्यदि \msNc, शूद्ध्य यदि \Ed}}% 
    \var{{\devanagarifont \numnoemph\vb\textbf{यज्ञं}\lem \msCa\msCb\msNa\Ed, यज्ञ \msCc\msNc, संज्ञ \msNb\oo 
\textbf{हि}\lem \mssALL, \om\ \msNb}}% 

%Verse 11:12

{\devanagarifont तस्य यज्ञफलावाप्तिर्जन्ममृत्युश्च नो भवेत् {॥ ११:१२॥} \veg\dontdisplaylinenum }%
     \var{{\devanagarifont \numnoemph\vcd\textbf{॰वाप्तिर्ज॰}\lem \msCa\msCb\Ed, ॰वाप्ति ज \msCc\msNb\msNc, ॰वापि ज॰ \msNa}}% 

{\devanagarifont विनार्थेन तु यो यज्ञं करोति वरसुन्दरि \thinspace{\dandab} \dontdisplaylinenum }%
     \var{{\devanagarifont \numemph\vb\textbf{॰सुन्दरि}\lem \mssALL, ॰सुन्दरी \Ed}}% 

%Verse 11:13

{\devanagarifont न तस्य तत्फलावाप्तिः सर्वयज्ञेष्वशेषतः {॥ ११:१३॥} \veg\dontdisplaylinenum }%
     \var{{\devanagarifont \numnoemph\vd\textbf{॰यज्ञेष्वशेषतः}\lem \mssALL, ॰यज्ञेषु शेषतः \Ed}}% 

{\devanagarifont यज्ञवाट कुरुक्षेत्रं सत्त्वावासकृतालयः \thinspace{\dandab} \dontdisplaylinenum }%
     \var{{\devanagarifont \numemph\va\textbf{॰वाट कुरु॰}\lem \mssALL, ॰वाटङ्कुरु॰ \msCb, ॰वाटकृत॰ \Ed\oo 
\textbf{॰क्षेत्रं}\lem \mssALL, ॰क्षेत्र \msNc}}% 
    \var{{\devanagarifont \numnoemph\vb\textbf{सत्त्वा॰}\lem \mssALL, सत्वासत्वा॰ \msCbacorr\oo 
\textbf{॰लयः}\lem \mssALL, ॰लयम् \msCc}}% 

%Verse 11:14

{\devanagarifont प्रत्याहार महावेदि कुशप्रस्तर संयमः {॥ ११:१४॥} \veg\dontdisplaylinenum }%
     \var{{\devanagarifont \numnoemph\vc\textbf{॰वेदि}\lem \mssALL, ॰देवि \Ed}}% 

{\devanagarifont विधि नियमविस्तारो ध्यानवह्निः प्रदीपितः \thinspace{\dandab} \dontdisplaylinenum }%
     \var{{\devanagarifont \numemph\va\textbf{विधि नि॰}\lem \mssALL, विधिर्नि॰ \Ed\oo 
\textbf{॰विस्तारो}\lem \mssALL, ॰विस्तारौ \msCb}}% 
    \var{{\devanagarifont \numnoemph\vb \lem \msNc, ध्यानवह्निप्रदीपितः \msCa\msNa, 
ध्यानं वह्निप्रदीपितः \msCb, ध्यानमग्निप्रदीपितः \msCc, 
ध्यान अग्निप्रदीपनः \msNb, ध्यानवृद्धिर्प्रदीपिनः \Ed}}% 

%Verse 11:15

{\devanagarifont योगेन्धनसमिज्ज्वालतपोधूमसमाकुलः {॥ ११:१५॥} \veg\dontdisplaylinenum }%
     \var{{\devanagarifont \numnoemph\vcd\textbf{॰न्धनसमिज्ज्वालतपोधूम॰}\lem \msNb\msNc, ॰न्धनसमिज्ज्वालतपोधूप॰ \msCa, 
॰\uncl{न्ध}$\-$सत्वमिज्ज्वालतपोधूम॰ \msCb, ॰न्धनसमिज्वालतपोधूम॰ \msCc, 
॰न्धनशमि\uncl{त}ज्वाल$\-$तयोधूय॰ \msNa, ॰न्धनसमिज्ज्वाला तपोधूम॰ \Ed}}% 

{\devanagarifont पात्रन्यास शिवज्ञानं स्थालीपाक शिवात्मकः \thinspace{\dandab} \dontdisplaylinenum }%
     \var{{\devanagarifont \numemph\va\textbf{पात्र॰}\lem \mssALL, पात्रा॰ \msNc}}% 

%Verse 11:16

{\devanagarifont आज्याहुतिमविच्छिन्नं लम्बकस्रुवपातितः {॥ ११:१६॥} \veg\dontdisplaylinenum }%
     \var{{\devanagarifont \numnoemph\vc\textbf{॰च्छिन्नं}\lem \mssALL, ॰च्छिन्न \msNc}}% 
    \var{{\devanagarifont \numnoemph\vd\textbf{लम्बक॰}\lem \mssALL, \uncl{ल}म्बक॰ \msCc, त्र्यम्बक॰ \Ed\oo 
\textbf{॰पातितः}\lem \mssALL, ॰पातितम् \Ed}}% 

{\devanagarifont धारणाध्वर्युवत्कृत्वा प्राणायामश्च ऋत्विजः \thinspace{\dandab} \dontdisplaylinenum }%
     \var{{\devanagarifont \numemph\va\textbf{॰ध्वर्युव॰}\lem \msNb, ॰ध्वर्यव॰ \mssCaCbCc, ॰\uncl{ध्व}र्यव॰ \msNa, 
ध्व\lk\lk\ \msNc, धर्मव॰ \Ed}}% 

%Verse 11:17

{\devanagarifont तर्कयुक्तः सविस्तारः समाधिर्वयतापनः {॥ ११:१७॥} \veg\dontdisplaylinenum }%
     \var{{\devanagarifont \numnoemph\vc\textbf{॰युक्तः}\lem \mssALL, ॰युक्त \msCc, ॰युक्तिः \msNa\oo 
\textbf{॰विस्तारः}\lem \mssALL, ॰विस्तारो \msCc}}% 

{\devanagarifont ब्रह्मविद्यामयो यूपः पशुबन्धो मनोन्मनः \thinspace{\dandab} \dontdisplaylinenum }%
     \var{{\devanagarifont \numemph\vb\textbf{॰न्मनः}\lem \msCa\msNa\msNb\Ed, ॰त्मनः \msCb\msCc\msNc}}% 

%Verse 11:18

{\devanagarifont श्रद्धा पत्नी विशालाक्षि संकल्प पद शाश्वतम् {॥ ११:१८॥} \veg\dontdisplaylinenum }%
     \var{{\devanagarifont \numnoemph\vc\textbf{पत्नी}\lem \mssALL, \uncl{पत्नी} \msCa\oo 
\textbf{विशालाक्षि}\lem \mssALL, विशालाक्षी \msNc\Ed}}% 
    \var{{\devanagarifont \numnoemph\vd\textbf{पद शाश्वतम्}\lem \mssALL, प\uncl{द}\lac  श्वतम् \msCa}}% 

{\devanagarifont पञ्चेन्द्रियजयोत्पन्नः पुरोडाशो ऽमृताशनः \thinspace{\dandab} \dontdisplaylinenum }%
     \var{{\devanagarifont \numemph\vb\textbf{॰डाशो}\lem \mssCaCbCc\msNb\msNc, ॰भा \msNaacorr, ॰भासे \msNapcorr, ॰भागे \Ed\oo 
\textbf{मृता॰}\lem \mssALL, मृगा॰ \msCc}}% 

%Verse 11:19

{\devanagarifont ब्रह्मनादो महामन्त्रः प्रायश्चित्तानिलो जयः {॥ ११:१९॥} \veg\dontdisplaylinenum }%
     \var{{\devanagarifont \numnoemph\vd\textbf{॰त्तानिलो}\lem \mssALL, ॰त्तनिलो \msCc\msNb\oo 
\textbf{जयः}\lem \mssALL, जलाः \Ed}}% 

{\devanagarifont सोमपान परिज्ञानमुपाकर्म चतुर्यमः \thinspace{\dandab} \dontdisplaylinenum }%
     \var{{\devanagarifont \numemph\va\textbf{परि॰}\lem \mssALL, पर॰ \msCc}}% 

%Verse 11:20

{\devanagarifont इतिहास जलस्नानं पुराणकृतमम्बरः {॥ ११:२०॥} \veg\dontdisplaylinenum }%
     \var{{\devanagarifont \numnoemph\vc\textbf{॰स्नानं}\lem \mssALL, ॰स्नान \msCb}}% 
    \var{{\devanagarifont \numnoemph\vd\textbf{पुराण॰}\lem \mssALL, पुराणं \Ed\oo 
\textbf{॰कृतमम्बरः}\lem \mssALL, ॰कृतम्बरम् \msCb\ \unmetr}}% 

{\devanagarifont इडासुषुम्नासंवेद्ये स्नानमाचमनं सकृत् \thinspace{\dandab} \dontdisplaylinenum }%
     \var{{\devanagarifont \numemph\va\textbf{॰सुषुम्ना॰}\lem \mssALL, ॰सुषुम्न॰ \msCc\oo 
\textbf{॰वेद्ये}\lem  \msCa\Ed, ॰वेद्य \msCb\msNb, ॰वेद्येः \msCc, ॰वैद्य \msNa, ॰भेदो \msNc}}% 
    \var{{\devanagarifont \numnoemph\vb\textbf{सकृत्}\lem \mssALL, विदुः \msCc}}% 

%Verse 11:21

{\devanagarifont संतोषातिथिमादृत्य दयाभूतद्विजार्चितः {॥ ११:२१॥} \veg\dontdisplaylinenum }%
     \var{{\devanagarifont \numnoemph\vc\textbf{॰तोषातिथिमादृत्य}\lem \mssALL, ॰तोषतिथिमावृत्य \msNb}}% 
    \var{{\devanagarifont \numnoemph\vd\textbf{॰द्विजा॰}\lem \mssALL, ॰दया॰ \msCb}}% 

{\devanagarifont ब्रह्मकूर्च गुणातीत हविर्गन्ध निरञ्जनः \thinspace{\dandab} \dontdisplaylinenum }%
     \var{{\devanagarifont \numemph\vb\textbf{॰हविर्ग॰}\lem \mssALL, ॰हवि\uncl{र्ग}॰ \msCb, ॰हविग \msNa}}% 

%Verse 11:22

{\devanagarifont ब्रह्मसूत्रं त्रयस्तत्त्वं बोधना मुण्डितं शिरः {॥ ११:२२॥} \veg\dontdisplaylinenum }%
     \var{{\devanagarifont \numnoemph\vc\textbf{॰सूत्रं त्रयस्}\lem \msCb\msNb\msNc\Ed, ॰सूत्रन्त्रयस्तयस् \msCa, 
॰सूत्रं त्रय \msCc, ॰सूत्रत्रयं \msNa}}% 
    \var{{\devanagarifont \numnoemph\vd\textbf{मुण्डितं}\lem \mssALL, मुण्डित॰ \msCb\msNc\unmetr}}% 

{\devanagarifont निवृत्त्यादि चतुर्वेदश्चतुःप्रकरणासनः \thinspace{\dandab} \dontdisplaylinenum }%
     \var{{\devanagarifont \numemph\va\textbf{निवृत्त्या॰}\lem \eme, निवृत्या॰ \mssCaCbCc\msNa\msNb\msNc, निर्वृत्या॰ \Ed}}% 
    \var{{\devanagarifont \numnoemph\vb\textbf{॰प्रकरणासनः}\lem \mssALL, 
प्रकरनाशनः \msCc, प्रकरशासनः \Ed}}% 

%Verse 11:23

{\devanagarifont दक्षिणामभयं भूते दत्त्वा यज्ञं यजेत्सदा {॥ ११:२३॥} \veg\dontdisplaylinenum }%
     \var{{\devanagarifont \numnoemph\vc\textbf{॰भयं भूते}\lem \mssALL, ॰भक्षयम्भूतै \msCb}}% 
    \var{{\devanagarifont \numnoemph\vd\textbf{यज्ञं यजेत्}\lem \mssALL, यज्ञ ददत् \Ed}}% 
    \paral{{\devanagarifontsmall \vc {\englishfont \compare\ \VSS\ 22.14ab:} दक्षिणाभय भूतेभ्यः पशुबन्धः स्वयंकृतः }}

{\devanagarifont विनार्थं यज्ञसम्प्राप्तिः कथिता ते वरानने \thinspace{\dandab} \dontdisplaylinenum }%
     \var{{\devanagarifont \numemph\va\textbf{विनार्थं}\lem \mssALL, विनार्थ \msCc}}% 
    \var{{\devanagarifont \numnoemph\vb\textbf{कथिता ते}\lem \mssALL, 
कथि\uncl{तो} स्मि \msCc, कथितस्ते \Ed\oo 
\textbf{वरानने}\lem \mssALL, व\uncl{रा}नने \msCc}}% 

%Verse 11:24

{\devanagarifont आसहस्रस्य यज्ञानां फलं प्राप्नोति नित्यशः {॥ ११:२४॥} \veg\dontdisplaylinenum }%
     \var{{\devanagarifont \numnoemph\vd\textbf{प्राप्नोति}\lem \mssALL, प्रा\lac  ति \msCa\oo 
\textbf{नित्यशः}\lem \mssALL, मानवः \msNb}}% 

{\devanagarifont आश्रमः प्रथमस्तुभ्यं कथितो ऽस्ति वरानने \thinspace{\dandab} \dontdisplaylinenum }%
     \var{{\devanagarifont \numemph\va\textbf{आश्रमः}\lem \mssALL, आश्रम \msCb\msCc\oo 
\textbf{॰स्तुभ्यं}\lem \mssALL, ॰स्येष \msCc, ॰स्यैवं \Ed}}% 
    \var{{\devanagarifont \numnoemph\vb\textbf{ऽस्ति}\lem \msCa\msCb\msNa\msNc, स्मि \msCc\msNb\Ed}}% 

%Verse 11:25

{\devanagarifont सदाशिवेन सद्धर्मं दैवतैरपि पूजितम् {॥ ११:२५॥} \veg\dontdisplaylinenum }%
     \var{{\devanagarifont \numnoemph\vc\textbf{॰धर्मं}\lem \mssALL, ॰ध\uncl{र्मं} \msCb, ॰धर्मे \Ed}}% 
    \var{{\devanagarifont \numnoemph\vd\textbf{दैव॰}\lem \mssALL, देव॰ \msNb\Ed\oo 
\textbf{पूजितम्}\lem \mssALL, पूपूजितम् \msCb}}% 


\alalfejezet{ब्रह्मचारी}
{\devanagarifont ब्रह्मचर्यं निबोधेदं शृणुष्वावहिता शुभे \thinspace{\dandab} \dontdisplaylinenum }%
     \var{{\devanagarifont \numemph\va\textbf{॰चर्यं}\lem \mssALL, ॰चर्य \msNa}}% 
    \var{{\devanagarifont \numnoemph\vb\textbf{॰वहिता शुभे}\lem \mssALL, 
॰वहितो भव \msCc, ॰वहितो शुभे \msNb}}% 

%Verse 11:26

{\devanagarifont द्वितीयमाश्रमं देवि सर्वपापविनाशनम् {॥ ११:२६॥} \veg\dontdisplaylinenum }%
     \var{{\devanagarifont \numnoemph\vd\textbf{॰विनाशनम्}\lem \mssALL, ॰प्रनाशनम् \msNb}}% 
    \paral{{\devanagarifontsmall \vcd {\englishfont  \compare\ \MBH\ 12.184.10A:} गार्हस्थ्यं खलु द्वितीयमाश्रमं वदन्ति }}

{\devanagarifont व्रतं ब्रह्मपरं ध्यानं सावित्री प्रकृतिर्लयम् \thinspace{\dandab} \dontdisplaylinenum }%
     \var{{\devanagarifont \numemph\va\textbf{॰परं ध्यानं}\lem \mssALL, ॰परिज्ञानं \Ed}}% 
    \var{{\devanagarifont \numnoemph\vb\textbf{॰कृतिर्लयम्}\lem \msCa\msNa\msNc\Ed, ॰कृतालयम् \msCb, ॰कृतीलयम् \msCc, ॰कृतिलः \msNb}}% 
    \paral{{\devanagarifontsmall \vab {\englishfont cf.\ \VSS\ 16.8cd} }}

%Verse 11:27

{\devanagarifont ब्रह्मसूत्राक्षरं सूक्ष्मं त्रिगुणालय मेखलम् {॥ ११:२७॥} \veg\dontdisplaylinenum }%
     \var{{\devanagarifont \numnoemph\vd\textbf{॰लय}\lem \mssALL, ॰ल\lac\  \msCa\oo 
\textbf{मेखलम्}\lem \mssALL, यत्फलम् \Ed}}% 

{\devanagarifont दम दण्ड दया पात्रं भिक्षा संसारमोचनम् \thinspace{\dandab} \dontdisplaylinenum }%
     \var{{\devanagarifont \numemph\va\textbf{दण्ड दया}\lem \mssALL, दण्डादया \msNa, दण्डादयो \Ed\oo 
\textbf{पात्रं}\lem \mssALL, पात्र \msNb}}% 

%Verse 11:28

{\devanagarifont त्र्यायुषं द्व्यक्षरातीतं ज्ञानभस्म-अलङ्कृतम् {॥ ११:२८॥} \veg\dontdisplaylinenum }%
     \var{{\devanagarifont \numnoemph\vc\textbf{॰युषं}\lem \mssALL, ॰युष \msNa}}% 
    \var{{\devanagarifont \numnoemph\vd\textbf{भस्म}\lem \mssALL, भष्मम् \Ed}}% 

{\devanagarifont स्नानव्रतं सदासत्यं शीलशौचसमन्वितम् \thinspace{\dandab} \dontdisplaylinenum }%
     \var{{\devanagarifont \numemph\va\textbf{॰व्रतं}\lem \msCa\msCc\msNa\msNb, ॰व्रत \msCb\msNc\Ed}}% 

%Verse 11:29

{\devanagarifont अग्निहोत्र त्रयस्तत्त्वं जप ब्रह्मबिलस्वरः {॥ ११:२९॥} \veg\dontdisplaylinenum }%
     \var{{\devanagarifont \numnoemph\vc\textbf{॰होत्र त्रयस्तत्त्वं}\lem \msNa\msNc\Ed, ॰होत्रन्त्रयस्तत्वं \msCa, 
॰होत्र$\-$\uncl{त}यस्तत्वं \msCb, ॰होत्रत्रयं तत्वा \msCc, 
॰होत्रं त्रयंस्तत्वं \msNb}}% 
    \var{{\devanagarifont \numnoemph\vd\textbf{॰बिलस्वरः}\lem \corr, ॰बिलश्वरः \mssCaCbCc\msNa\msNb, ॰बिलेश्वर \msNc\Ed}}% 

{\devanagarifont द्वितीय आश्रमो देवि यथाह भगवान्शिवः \thinspace{\dandab} \dontdisplaylinenum }%
     \var{{\devanagarifont \numemph\va\textbf{द्वितीय आश्रमो}\lem \mssALL, द्वितीयमाश्रमो \msCc, 
द्वितीयमाश्रमं \Ed}}% 
    \var{{\devanagarifont \numnoemph\vb\textbf{यथाह}\lem \msCa\msCb\msNa\msNc, यथाहं \msCc\msNb, यदाह \Ed}}% 

%Verse 11:30

{\devanagarifont ममापि कथितं तुभ्यं जन्ममृत्युविनाशनम् {॥ ११:३०॥} \veg\dontdisplaylinenum }%
     \var{{\devanagarifont \numnoemph\vc\textbf{ममापि कथितं तु॰}\lem \mssALL, 
ममापि कथितस्तु॰ \msNc, मयापि कथितो तु॰ \Ed}}% 
    \var{{\devanagarifont \numnoemph\vd\textbf{॰मृत्यु॰}\lem \mssALL, ॰मृ\lac\  \msCa\oo 
\textbf{॰नाशनं}\lem \mssALL, ॰नाशनः \msNc}}% 


\alalfejezet{वानप्रस्थः}
{\devanagarifont वानप्रस्थविधिं वक्ष्ये शृणुष्वायतलोचने \thinspace{\dandab} \dontdisplaylinenum }%
     \var{{\devanagarifont \numemph\va\textbf{॰विधिं}\lem \mssALL, ॰विधि \msCb}}% 

%Verse 11:31

{\devanagarifont यथाश्रुतं यथातथ्यमृषिदैवतपूजितम् {॥ ११:३१॥} \veg\dontdisplaylinenum }%
     \var{{\devanagarifont \numnoemph\vd\textbf{॰दैवत॰}\lem \mssALL, ॰देवत॰ \msCc}}% 

{\devanagarifont वैराग्यवनमाश्रित्य नियमाश्रममाहरेत् \thinspace{\dandab} \dontdisplaylinenum }%
     \var{{\devanagarifont \numemph\va\textbf{वैराग्य॰}\lem \mssALL, वैराग्या \Ed}}% 
    \var{{\devanagarifont \numnoemph\vb\textbf{नियमा॰}\lem \mssALL, मा॰ \msNaacorr\oo 
\textbf{॰श्रममा॰}\lem \mssALL, ॰श्रमनो हरेत् \msCa}}% 

%Verse 11:32

{\devanagarifont शीलशैलदृढद्वारे प्राकारे विजितेन्द्रियः {॥ ११:३२॥} \veg\dontdisplaylinenum }%
     \var{{\devanagarifont \numnoemph\vc\textbf{॰दृढ॰}\lem \mssALL, ॰दृष॰ \Ed}}% 
    \var{{\devanagarifont \numnoemph\vd\textbf{॰कारे}\lem \mssALL, ॰कार॰ \msCc}}% 

{\devanagarifont अधिभूतः स्मृतो माता अध्यात्मश्च पिता तथा \thinspace{\dandab} \dontdisplaylinenum }%
     \var{{\devanagarifont \numemph\va\textbf{स्मृतो}\lem \mssALL, \lac\  \msCb, स्मृतौ \Ed}}% 
    \paral{{\devanagarifontsmall \vab {\englishfont cf.\ \VSS\ 22.10ab:} अध्यात्मनगरस्फीतः अधिभूतजनाकुलः }}

%Verse 11:33

{\devanagarifont अधिदैविकमाचार्यो व्यवसायाश्च भ्रातरः {॥ ११:३३॥} \veg\dontdisplaylinenum }%
     \var{{\devanagarifont \numnoemph\vc\textbf{अधिदैविक॰}\lem \emeGoodall, 
\uncl{अ}\lac \uncl{भौ}\lac  क॰ \msCa, 
अधिभौतिक॰ \msCb\msCc\msNa\msNc\Ed, 
अधिभौक्तिक॰ \msNb}}% 
    \var{{\devanagarifont \numnoemph\vd\textbf{व्यवसायाश्च}\lem \mssALL, व्यवसायश्च \Ed}}% 

{\devanagarifont श्रुतिः स्मृतिः स्मृता भार्या प्रज्ञा पुत्रः क्षमानुजः \thinspace{\dandab} \dontdisplaylinenum }%
     \var{{\devanagarifont \numemph\va\textbf{स्मृता}\lem \mssALL, स्मृतो \msCb}}% 

{\devanagarifont मैत्री बन्धुर्जटा चापं करुणा सुपवित्रकम्  \danda\dontdisplaylinenum }%
     \var{{\devanagarifont \numnoemph\vc\textbf{बन्धुर्ज॰}\lem \mssALL, बन्धु ज॰ \msCc\msNb}}% 

%Verse 11:34

{\devanagarifont मुदिता मौन चत्वारः सर्वकार्यमुपेक्षका {॥ ११:३४॥} \veg\dontdisplaylinenum }%
     \var{{\devanagarifont \numnoemph\ve\textbf{मौन चत्वारः}\lem \mssALL, 
मौनश्चत्वारः \msCb, मौन चत्वार \msCc}}% 
    \var{{\devanagarifont \numnoemph\vf\textbf{॰कार्यमु॰}\lem \mssALL, ॰कार्यामु॰ \msNa\oo 
\textbf{॰पेक्षका}\lem \mssALL, ॰पेक्षया \Ed}}% 

{\devanagarifont यमवल्कलसंवीतस्तपःकृष्णाजिनाधरः \thinspace{\dandab} \dontdisplaylinenum }%
     \var{{\devanagarifont \numemph\va\textbf{॰संवीत॰}\lem \mssALL, ॰सान्वीत॰ \Ed}}% 
    \var{{\devanagarifont \numnoemph\vb\textbf{॰कृष्णा॰}\lem \mssALL, ॰कृष्णां \msCc\oo 
\textbf{॰जिनाधरः}\lem \msNc, ॰जिनधरः \mssCaCbCc\msNa\msNb\ \unmetr, ॰जिनं पुरः \Ed}}% 

%Verse 11:35
 
{\devanagarifont उत्तरासङ्गमासीनो योगपट्टदृढव्रतः {॥ ११:३५॥} \veg\dontdisplaylinenum }%
     \var{{\devanagarifont \numnoemph\vd\textbf{॰दृढ॰}\lem \mssALL, ॰दृष्ट॰ \msNb\oo 
\textbf{॰व्रतः}\lem \mssALL, \lac\  \msCa}}% 

{\devanagarifont वेदघोषेण घोषेण प्राणायामो ऽग्निहावनम् \thinspace{\dandab} \dontdisplaylinenum }%
     \var{{\devanagarifont \numemph\va\textbf{वेद॰}\lem \mssALL, \lac  द॰ \msCa\oo 
\textbf{॰ण घोषेण}\lem \mssALL, ॰ण घोषीण \msCc}}% 
    \var{{\devanagarifont \numnoemph\vb\textbf{॰हावनम्}\lem \mssALL, ॰\uncl{हावनम्} \msCb, ॰हावन \msCc}}% 

%Verse 11:36

{\devanagarifont जितप्राण मृगाकूलो धृति यज्ञः क्रिया जपः {॥ ११:३६॥} \veg\dontdisplaylinenum }%
     \var{{\devanagarifont \numnoemph\vd\textbf{॰जपः}\lem \mssALL, ॰जिणः \msCc}}% 

{\devanagarifont अर्थसंग्रह शास्त्रेषु सखा दमदयादयः \thinspace{\dandab} \dontdisplaylinenum }%
     \var{{\devanagarifont \numemph\vb\textbf{सखा}\lem \mssALL, सखो \msNb\oo 
\textbf{दमद॰}\lem \mssALL, 
दम॰ \msCaacorr, दयद॰ \msCc}}% 

%Verse 11:37

{\devanagarifont शिवयज्ञं प्रयुञ्जीत साधनाष्टकपूजनम् {॥ ११:३७॥} \veg\dontdisplaylinenum }%
     \var{{\devanagarifont \numnoemph\vc\textbf{॰यज्ञं}\lem \mssALL, ॰यज्ञ \msCc\msNc}}% 
    \var{{\devanagarifont \numnoemph\vd\textbf{॰पूजनम्}\lem \mssALL, ॰पूजिकं \msCc}}% 
    \paral{{\devanagarifontsmall \vd {\englishfont \compare\ \DHARMP\ 2.1:} 
                 अष्टभिः साधनैरेभिश्चित्तं कायञ्च यत्नतः\thinspace{\devanagarifontsmall ।}
                 शोधयित्वा ततो योगी योगाभ्यासं समाचरेत्\thinspace{\devanagarifontsmall ॥} }}

{\devanagarifont पञ्चब्रह्मजलैः पूतः सत्यतीर्थशिवह्रदे \thinspace{\dandab} \dontdisplaylinenum }%
     \var{{\devanagarifont \numemph\va\textbf{॰ब्रह्मजलैः पूतः}\lem \mssALL, ब्र\lac\  \msNb}}% 
    \var{{\devanagarifont \numnoemph\vb\textbf{॰तीर्थ॰}\lem \mssALL, ॰तीर्थं \Ed}}% 

%Verse 11:38

{\devanagarifont स्नानमाचमनं कृत्वा संध्यात्रयमुपासयेत् {॥ ११:३८॥} \veg\dontdisplaylinenum }%
     \var{{\devanagarifont \numnoemph\vc\textbf{॰चमनं}\lem \mssALL, ॰चनं \msCb}}% 
    \var{{\devanagarifont \numnoemph\vd\textbf{॰सयेत्}\lem \eme, ॰श्रयेत् \mssCaCbCc\msNa\msNb\msNc\Ed}}% 
    \paral{{\devanagarifontsmall \vd {\englishfont \compare\ \VSS\ 11.59cd:} शिवस्य हृदयं संध्या तस्मात्संध्यामुपासयेत् }}

{\devanagarifont अक्षमाला पुराणार्थं जप शान्तं दिवानिशम् \thinspace{\dandab} \dontdisplaylinenum }%
     \var{{\devanagarifont \numemph\va\textbf{अक्षमाला}\lem \mssALL, \uncl{अक्ष}\lac  ला \msCa\oo 
\textbf{पुराणार्थं}\lem \mssALL, पुराणाञ्च \msNb, 
पुराणा\uncl{र्था} \msNc}}% 
    \var{{\devanagarifont \numnoemph\vb\textbf{॰शान्तं}\lem \mssALL, ॰शन्ति \msCaacorr\msNa}}% 

%Verse 11:39

{\devanagarifont ज्ञानसलिलसम्पूर्णमितिहासकमण्डलुः {॥ ११:३९॥} \veg\dontdisplaylinenum }%
     \var{{\devanagarifont \numnoemph\vc\textbf{॰सलिल॰}\lem \mssALL, ॰सलील॰ \Ed}}% 
    \var{{\devanagarifont \numnoemph\vd\textbf{॰कमण्डलुः}\lem \mssALL, ॰कमण्डलु \Ed}}% 

{\devanagarifont पञ्चकर्मक्रियोत्क्रान्ति जप पञ्चविधः सुखम् \thinspace{\dandab} \dontdisplaylinenum }%
     \var{{\devanagarifont \numemph\vab\textbf{॰त्क्रान्ति ज॰}\lem \msCa\msCb\msNb, ॰क्रान्तिज॰ \msCc, ॰त्क्रान्तिर्ज॰ \msNa, 
॰त्कान्तिज॰ \msNc, ऽक्रान्ति ज॰ \Ed}}% 

%Verse 11:40

{\devanagarifont साधनं शिवसंकल्पो योगसिद्धिफलप्रदः {॥ ११:४०॥} \veg\dontdisplaylinenum }%
     \var{{\devanagarifont \numnoemph\vd\textbf{॰दः}\lem \mssALL, ॰दम् \Ed}}% 

{\devanagarifont संतोषफलमाहारः कामक्रोधपराजितः \thinspace{\dandab} \dontdisplaylinenum }%
 
{\devanagarifont आशापाशजयाभ्यासो ध्यानयोगरतिप्रियः  \danda\dontdisplaylinenum }%
     \var{{\devanagarifont \numemph\vc\textbf{॰भ्यासो}\lem \mssALL, ॰भ्यास \Ed}}% 
    \var{{\devanagarifont \numnoemph\vd\textbf{॰रति॰}\lem \msCc\msNa\msNb\msNc, \lac\  \msCa, ॰रिति॰ \msCb, ॰रतिः \Ed}}% 

%Verse 11:41

{\devanagarifont अतिथिभ्यो ऽभयं दत्त्वा वानप्रस्थश्चरेद्व्रतम् {॥ ११:४१॥} \veg\dontdisplaylinenum }%
     \var{{\devanagarifont \numnoemph\ve\textbf{अतिथिभ्यो ऽभयं}\lem \mssALL, 
आर्तिभ्यश्चाभयं \Ed\oo 
\textbf{दत्त्वा}\lem \mssALL, दारा \msCc}}% 
    \var{{\devanagarifont \numnoemph\vf\textbf{॰प्रस्थश्च॰}\lem \mssALL, ॰प्रस्थ च॰ \msCc\msNb}}% 

\nemslokalong


\ujvers\nemsloka {
{\devanagarifont वानप्रस्थमयं धर्मं गदित यत्पूर्वमवधारितं }%
  \dontdisplaylinenum}    \var{{\devanagarifont \numemph\va\textbf{गदित यत्पूर्वमवधारितम्}\lem \conj, गदितं पूर्वधारितम् \msCa\msCb, 
यत्पूर्वमवधारितं \msCc\Ed, 
गदितं यत्पूर्वधारितं \msNaacorr, 
गदितं यत्पूर्व\uncl{मवधा}रितं \msNapcorr, 
गदित पूर्वधारितं \msNb, 
गदितं यत्पूर्वमेधारितं \msNc}}% 


\nemslokab

{\devanagarifont संसारोद्धरणमनित्यहरणमज्ञाननिर्मूलनम्  \danda\dontdisplaylinenum }%
     \var{{\devanagarifont \numnoemph\vb\textbf{॰हरणमनित्यहरणमज्ञा॰}\lem \msCa\msCb\msNaacorr\msNb\msNc, 
॰हरणंमनित्यहरणमज्ञा॰ \msCc\Ed, 
॰हरणंम् अनित्यहरणन्तज्ञा॰ \msNapcorr}}% 

\nemslokac

{\devanagarifont प्रज्ञावृद्धिकरममोघकरणं क्लेशार्णवोत्तारणं }%
  \dontdisplaylinenum    \var{{\devanagarifont \numnoemph\vc\textbf{(प्रज्ञा॰{\englishfont ...} ॰त्तारणम)}\lem \mssALL, \om\ \msNb\oo 
\textbf{॰करममोघ॰}\lem \mssCaCbCc\msNa\ \unmetr, \om\ \msNb, ॰कममोघ॰ \msNc, 
॰करं प्रबोध॰ \Ed\oo 
\textbf{क्लेशार्णवो॰}\lem \mssCaCbCc\msNc, क्लेशाण्णवो॰ \msNa, 
\om\ \msNb, शोकार्णवो॰ \Ed}}% 

%Verse 11:42


\nemslokad

{\devanagarifont जन्मव्याधिहरमकर्मदहनं सेवेत्स धर्मोत्तमम् {॥ ११:४२॥} \veg\dontdisplaylinenum }%
     \var{{\devanagarifont \numnoemph\vd\textbf{सेवेत्स}\lem \mssALL, 
सेवे स \msCc, सेवेत्त \msNb}}% 
    \lacuna{\devanagarifontsmall \vd {\englishfont \Ed\ (and paper MS \msPaperA) add here a Śārdūlavikrīḍita line:}
                 श्रद्धापूर्वकमेव यः सनियमं साक्षाच्च जीवन्शिवः
                 (शुद्धापूर्व्वकमेव यः सनियतं साक्षाच्च जीवने शिवः {\englishfont \msPaperA}) }%
  
\nemslokanormal



\alalfejezet{परिव्राजकः}
\vers


{\devanagarifont परिव्राजकधर्मो ऽयं कीर्तयिष्यामि तच्छृणु \thinspace{\dandab} \dontdisplaylinenum }%
     \var{{\devanagarifont \numemph\vb\textbf{कीर्तयिष्यामि}\lem \mssALL, 
कीर्तयि\lac  मि \msCa}}% 

%Verse 11:43

{\devanagarifont सुखदुःखं समं कृत्वा लोभमोहविवर्जितः {॥ ११:४३॥} \veg\dontdisplaylinenum }%
     \var{{\devanagarifont \numnoemph\vc\textbf{॰दुःखं}\lem \msCb, ॰दुःख \msCa\msCc\msNa\msNb\msNc\Ed}}% 
    \var{{\devanagarifont \numnoemph\vd\textbf{लोभमोह॰}\lem \msCb, लाभालोभ॰ \msCa\msNa\msNb\msNc, 
लाभलोभ॰ \msCc, लाभालाभ॰ \Ed\oo 
\textbf{॰वर्जितः}\lem \mssALL, ॰वर्जिताः \msNb}}% 
    \paral{{\devanagarifontsmall \vd {\englishfont \compare\ \VSS\ 4.71:}  
                      कामः क्रोधश्च लोभश्च मोहश्चैव चतुर्विधः\thinspace{\devanagarifontsmall ।}
                      चतुःशत्रुर्निहन्तव्यः सर्वथा वीतकल्मषः\thinspace{\devanagarifontsmall ॥} }}

{\devanagarifont वर्जयेन्मधु मांसानि परदारांश्च वर्जयेत् \thinspace{\dandab} \dontdisplaylinenum }%
     \var{{\devanagarifont \numemph\va\textbf{वर्जयेन्}\lem \msCa\msNb, वर्जयेत् \msCb\msCc\msNa\msNc\Ed}}% 
    \paral{{\devanagarifontsmall \vab {\englishfont \compare\ \MANU\ 2.177:}
                 वर्जयेन्मधु मांसं च गन्धं माल्यं रसान्स्त्रियः\thinspace{\devanagarifontsmall ।}
                 शुक्तानि यानि सर्वाणि प्राणिनां चैव हिंसनम्\thinspace{\devanagarifontsmall ॥} }}

%Verse 11:44

{\devanagarifont वर्जयेच्चिरवासं च परवासं च वर्जयेत् {॥ ११:४४॥} \veg\dontdisplaylinenum }%
     \var{{\devanagarifont \numnoemph\vc\textbf{॰वासं}\lem \mssALL, ॰वासश् \Ed}}% 
    \var{{\devanagarifont \numnoemph\vd\textbf{॰वासं}\lem \mssALL, ॰वासश् \Ed}}% 

{\devanagarifont वर्जयेत्सृष्टभोज्यानि भिक्षामेकां च वर्जयेत् \thinspace{\dandab} \dontdisplaylinenum }%
     \var{{\devanagarifont \numemph\va\textbf{वर्जयेत्सृष्ट॰}\lem \msCc(?)\msNa\msNc, वर्जयेत्मृष्ट॰ \msCa, 
वर्ज्जन्मृष्ट॰ \msNb, वर्जयेन्मृष्ट॰ \Ed\oo 
\textbf{॰भोज्यानि}\lem \mssALL, ॰भोजालि(?) \msNc}}% 
    \var{{\devanagarifont \numnoemph\vb\textbf{॰क्षामेकां}\lem \msCa\msNb, ॰क्षामेकं \msCc\msNa, 
॰क्षमेकञ् \msNc, ॰क्षामेकश् \Ed}}% 
    \lacuna{\devanagarifontsmall \vab {\englishfont Omitted in \msCb} }%
      \paral{{\devanagarifontsmall \vb {\englishfont \compare\ \MANU\ 2.188ab:}
                          भैक्षेण वर्तयेन्नित्यं नैकान्नादी भवेद्व्रती }}

%Verse 11:45

{\devanagarifont वर्जयेत्संग्रहं नित्यमभिमानं च वर्जयेत् {॥ ११:४५॥} \veg\dontdisplaylinenum }%
 
{\devanagarifont सुसूक्ष्मं मनसा ध्यात्वा दृशौ पादं विनिक्षिपेत् \thinspace{\dandab} \dontdisplaylinenum }%
     \var{{\devanagarifont \numemph\vb\textbf{दृशौ}\lem \conj, शुचौ \mssCaCbCc\msNa\msNb\msNc\Ed\oo 
\textbf{पादं}\lem \msCb\msCc\msNa\msNc, पा\uncl{दं} \msCa, पाद \msNb\Ed\oo 
\textbf{विनिक्षि॰}\lem \mssALL, \lac  निक्षि॰ \msCa, 
विनिक्ष॰ \msNc}}% 

%Verse 11:46

{\devanagarifont न कुप्येत अनालाभे लाभे वापि न हर्षयेत् {॥ ११:४६॥} \veg\dontdisplaylinenum }%
     \var{{\devanagarifont \numnoemph\vc\textbf{कुप्येत}\lem \mssALL, कुपेत \msCc\oo 
\textbf{अनालाभे}\lem \msNa, मनोलाभे \msCa\msCb\msNb\msNc, 
मनोलाभो \msCc, मनालाभे \Ed}}% 
    \paral{{\devanagarifontsmall \vcd {\englishfont \similar\ \MANU\ 6.57:}
                         अलाभे न विषदी स्याल्लाभे चैव न हर्षयेत् = 
                     {\englishfont \VASISTHADHS\ 10.22} }}

{\devanagarifont अर्थतृष्णास्वनुद्विग्नो रोषे वापि सुदारुणे \thinspace{\dandab} \dontdisplaylinenum }%
     \var{{\devanagarifont \numemph\va\textbf{अर्थ॰}\lem \msCb\msCc\msNc, अर्था॰ \msCa\msNa\msNb, अथ \Ed\oo 
\textbf{॰नुद्विग्नो}\lem \mssALL, ॰नुदिग्नो \msCc}}% 

%Verse 11:47

{\devanagarifont स्तुतिनिन्दा समं कृत्वा प्रियं वाप्रियमेव वा {॥ ११:४७॥} \veg\dontdisplaylinenum }%
 
{\devanagarifont नियमास्तु परीधानं संयमावृतमेखलः \thinspace{\dandab} \dontdisplaylinenum }%
     \var{{\devanagarifont \numemph\va\textbf{॰धानं}\lem \mssALL, 
॰धाना \msCc, ॰\uncl{धानं} \msNc}}% 
    \var{{\devanagarifont \numnoemph\vb\textbf{॰वृत॰}\lem \mssALL, ॰मृत॰ \msNb, ॰नृत॰ \Ed\oo 
\textbf{॰मेखलः}\lem \mssALL, 
॰मेखलाः \msCc, ॰मेखला \msNb}}% 

%Verse 11:48

{\devanagarifont निरालम्बं मनः कृत्वा बुद्धिं कृत्वा निरञ्जनाम् {॥ ११:४८॥} \veg\dontdisplaylinenum }%
     \var{{\devanagarifont \numnoemph\vc\textbf{॰बं मनः कृत्वा}\lem \msNc, ॰बमसत्कृत्वा \msCa\msNa, 
॰बमसंकृत्वा \msCb, ॰बमनंकृत्वा \msCc, 
॰ब मनस्कृत्वा \msNb, ॰बमनङ्कृत्वा \Ed}}% 
    \var{{\devanagarifont \numnoemph\vd\textbf{बुद्धिं}\lem \mssALL, बुद्धि \msCb\Ed\oo 
\textbf{निरञ्जनाम्}\lem \eme, निरञ्जनम् \mssCaCbCc\msNb\msNc\Ed, निरञ्जनः \msNa}}% 

{\devanagarifont आत्मानं पृथिवीं कृत्वा खं च कृत्वा मनोन्मनम् \thinspace{\dandab} \dontdisplaylinenum }%
     \var{{\devanagarifont \numemph\vab\textbf{कृत्वा खं च}\lem \mssALL, 
कृ\uncl{त्वा}\lac  ञ्च \msCa}}% 
    \var{{\devanagarifont \numnoemph\vb\textbf{मनोन्मनम्}\lem \mssALL, मनोन्मनः \msNc, मनोन्मनैः \Ed}}% 

%Verse 11:49

{\devanagarifont त्रिदण्डं त्रिगुणं कृत्वा पात्रं कृत्वाक्षरो ऽव्ययः {॥ ११:४९॥} \veg\dontdisplaylinenum }%
     \var{{\devanagarifont \numnoemph\vd\textbf{॰क्षरो}\lem \mssALL, ॰करो \msNb\oo 
\textbf{व्ययः}\lem \msCa\msCb\msNa\msNb, व्ययं \msCc, व्यय \msNc, द्वयम् \Ed}}% 

{\devanagarifont न्यसेद्धर्ममधर्मं च ईर्ष्याद्वेषं परित्यजेत् \thinspace{\dandab} \dontdisplaylinenum }%
     \var{{\devanagarifont \numemph\va\textbf{॰धर्मं च}\lem \mssALL, ॰धर्मं वा \msNa}}% 
    \var{{\devanagarifont \numnoemph\vb\textbf{ईर्ष्या॰}\lem \msNa\msNc\Ed, ईर्षा॰ \mssCaCbCc\msNb\oo 
\textbf{॰द्वेषं}\lem \mssALL, ॰द्वेष \msCc}}% 

%Verse 11:50

{\devanagarifont निर्द्वन्द्वो नित्यसत्यस्थो निर्ममो निरहंकृतः {॥ ११:५०॥} \veg\dontdisplaylinenum }%
     \var{{\devanagarifont \numnoemph\vc\textbf{निर्द्वन्द्वो}\lem \mssALL, निवंद्वो \msCc\oo 
\textbf{॰सत्य॰}\lem \mssALL, ॰संत्य॰ \msCc}}% 
    \var{{\devanagarifont \numnoemph\vd\textbf{निर्ममो}\lem \msNc\Ed, निर्मांसो \mssCaCbCc\msNa, निर्मंसो \msNb\oo 
\textbf{॰कृतः}\lem \mssALL, ॰कृतं \msNa, ॰कृतिः \Ed}}% 
    \paral{{\devanagarifontsmall \vcd {\englishfont \compare\ \BHG\ 2.45cd: 
                         }निर्द्वन्द्वो नित्यसत्त्वस्थो निर्योगक्षेम आत्मवान् }}

{\devanagarifont दिवसस्याष्टमे भागे भिक्षां सप्तगृहं चरेत् \thinspace{\dandab} \dontdisplaylinenum }%
     \var{{\devanagarifont \numemph\va\textbf{दिवसस्या॰}\lem \mssALL, दिवसत्या॰ \msCb}}% 
    \var{{\devanagarifont \numnoemph\vb\textbf{भिक्षां}\lem \mssALL, भिक्षा \msNb}}% 
    \paral{{\devanagarifontsmall \vb {\englishfont \compare\ \GAUTDHS\ 23.18:}
                 तस्याजिनमूर्ध्वबालं परिधाय लोहितपत्रः सप्त गृहान्भक्षं चरेत् }}

%Verse 11:51

{\devanagarifont न चासीत न तिष्ठेत न च देहीति वा वदेत् {॥ ११:५१॥} \veg\dontdisplaylinenum }%
 
{\devanagarifont यथालाभेन वर्तेत अष्टौ पिण्डान्दिने दिने \thinspace{\dandab} \dontdisplaylinenum }%
     \var{{\devanagarifont \numemph\va\textbf{यथालाभेन}\lem \mssALL, यथाला\lac\  \msCa}}% 
    \var{{\devanagarifont \numnoemph\vb\textbf{अष्टौ}\lem \mssALL, अष्ट \Ed}}% 

%Verse 11:52

{\devanagarifont वस्त्रभोजनशय्यासु न प्रसज्येत विस्तरम् {॥ ११:५२॥} \veg\dontdisplaylinenum }%
     \var{{\devanagarifont \numnoemph\vc\textbf{॰शय्यासु}\lem \mssALL, ॰शय्याञ्च \msNb, ॰शैय्यासु \Ed}}% 
    \var{{\devanagarifont \numnoemph\vd\textbf{॰सज्येत}\lem \msCa\msCc\msNa\msNb, ॰युज्ये \msCb, ॰सहेत \msNc, ॰सह्येत \Ed\oo 
\textbf{विस्तरम्}\lem \mssALL, विस्तरः \Ed}}% 

{\devanagarifont नाभिनन्देत मरणं नाभिनन्देत जीवितम् \thinspace{\dandab} \dontdisplaylinenum }%
     \paral{{\devanagarifontsmall \vab {\englishfont = \MBH\ 12.237.15ab = \MANU\ 6.45ab = \NARADAPARIVR\ 3.61cd} }}

%Verse 11:53

{\devanagarifont इन्द्रियाणि वशंकृत्वा कामं हत्वा यतव्रतः {॥ ११:५३॥} \veg\dontdisplaylinenum }%
     \var{{\devanagarifont \numemph\vc\textbf{वशंकृ॰}\lem \mssALL, वसंत्कृ॰ \msCc}}% 
    \var{{\devanagarifont \numnoemph\vd\textbf{हत्वा यतव्रतः}\lem \mssALL, 
कृत्वा यतः व्रतः \msNb}}% 

{\devanagarifont अतीतं च भविष्यं च न भिक्षुश्चिन्तयेत्सदा \thinspace{\dandab} \dontdisplaylinenum }%
     \var{{\devanagarifont \numemph\vb\textbf{भिक्षुश्चि॰}\lem \mssALL, 
भिक्षुंश्चि॰ \msNa, भिक्षु चि॰ \Ed\oo 
\textbf{सदा}\lem \mssALL, \om\ \msCb}}% 

%Verse 11:54

{\devanagarifont क्रोधमानमददर्पान्परिव्राड्वर्जयेत्सदा {॥ ११:५४॥} \veg\dontdisplaylinenum }%
     \var{{\devanagarifont \numnoemph\vcd\textbf{॰दर्पान्प॰}\lem \mssALL, ॰दर्पात्प॰ \msCb}}% 

{\devanagarifont विरागं तु धनुः कृत्वा प्राणायामगुणैर्युतम् \thinspace{\dandab} \dontdisplaylinenum }%
     \var{{\devanagarifont \numemph\va\textbf{धनुः}\lem \mssALL, धनुष् \Ed}}% 
    \var{{\devanagarifont \numnoemph\vb\textbf{प्राणायामगु॰}\lem \mssALL, प्राणायामङ्गु॰ \msCa\oo 
\textbf{युतम्}\lem \mssALL, युतः \msNa, वृतं \Ed}}% 

%Verse 11:55

{\devanagarifont धारणाशरतीक्ष्णेन मृगं हत्वा मनेन्द्रियम् {॥ ११:५५॥} \veg\dontdisplaylinenum }%
     \var{{\devanagarifont \numnoemph\vc\textbf{॰तीक्ष्णेन}\lem \msNb\Ed, ॰तीक्ष्णेण \mssCaCbCc\msNc, ॰तीक्षेण \msNa}}% 

{\devanagarifont मैत्रीखड्गसुतीक्ष्णेन संसारारिं निकृन्तयेत् \thinspace{\dandab} \dontdisplaylinenum }%
     \var{{\devanagarifont \numemph\va\textbf{सुतीक्ष्णेन}\lem \msCa\msNb\msNc\Ed, सुतीक्ष्णेण \msCb\msCc\msNapcorr, ण \msNaacorr}}% 
    \var{{\devanagarifont \numnoemph\vb\textbf{॰सारारिं}\lem \mssALL, ॰सारारि \msCc\msNc}}% 

{\devanagarifont करुणावर्तचक्रेण क्रोधमत्तगजं जयेत्  \danda\dontdisplaylinenum }%
 
%Verse 11:56

{\devanagarifont मुदितावर्मबद्धाङ्गस्तूणं पूर्णमुपेक्षया {॥ ११:५६॥} \veg\dontdisplaylinenum }%
     \var{{\devanagarifont \numnoemph\vf\textbf{तूणं पूर्णमु॰}\lem \emeGoodall, तूण्णापूर्ण्णमु॰ \msCa, 
तूणापूर्ण्णमु॰ \msCb, तू$\-$\uncl{न}पूर्ण्णमु॰ \msCc, 
तूण्णापूण्णामु॰ \msNa, तूर्ण्णापूर्ण्णमु॰ \msNb\msNc, तूणीपूर्णमु॰ \Ed}}% 

{\devanagarifont अनक्षरं परं ब्रह्म चिन्तयेत्सततं द्विज \thinspace{\dandab} \dontdisplaylinenum }%
     \var{{\devanagarifont \numemph\va\textbf{अनक्षरं}\lem \msCb, अनाक्षरं \msCa\msNa, 
अनाक्षर॰ \msCc\msNc\Ed, अनक्षर॰ \msNb\oo 
\textbf{परं}\lem \mssALL, पर \msCb\msNc}}% 

{\devanagarifont ब्रह्मणो हृदयं विष्णुर्विष्णोश्च हृदयं शिवः  \danda\dontdisplaylinenum }%
     \var{{\devanagarifont \numnoemph\vc\textbf{हृदयं}\lem \mssALL, 
\lac  दयं \msCa, हृदये \msNc}}% 
    \var{{\devanagarifont \numnoemph\vcd\textbf{विष्णुर्वि॰}\lem \msCa\msNa\Ed, विष्णुम्वि॰ \msCb, 
विष्णु वि॰ \msCc\msNb\msNc}}% 
    \var{{\devanagarifont \numnoemph\vd\textbf{शिवः}\lem \Ed, शिवं \mssCaCbCc\msNa\msNb\msNc}}% 

%Verse 11:57

{\devanagarifont शिवस्य हृदयं संध्या तस्मात्संध्यामुपासयेत् {॥ ११:५७॥} \veg\dontdisplaylinenum }%
     \var{{\devanagarifont \numnoemph\vf\textbf{॰सयेत्}\lem \msCa\msCc\msNb, ॰शयेत् \msCb\msNa, ॰श्रयेत् \msNc\Ed}}% 
    \paral{{\devanagarifontsmall \vo {\englishfont \similar\ Saubhāgyabhāskara of Bhāskararāya ad Lalitāsahasranāmastotra 302:}
                 ब्रह्मणो हृदयं विष्णुर्विष्णोरपि शिवः स्मृतः\thinspace{\devanagarifontsmall ।}
                 शिवस्य हृदयं सन्ध्या तेनोपास्या द्विजातिभिः\thinspace{\devanagarifontsmall ॥}
                 इति कश्यपादिवचनैः कौर्मपाद्मस्कान्दादिनिखिलपुराणेषु च तत्र 
                 तत्र देवीकालिकाब्रह्माण्डमार्कण्डेयादिपुराणेषु बहुशः 
                 शक्तिरहस्य-देवीभागवत-तृतीयस्कन्धादिषु
                 च इदंपर्येण सर्वत्र ज्ञानार्णवकुलार्णवादितन्त्रेषु त्वपरिमितत्या वर्णितम् }}

\nemslokalong


\ujvers\nemsloka {
{\devanagarifont संसारार्णवतारणं शुभगतिः स ब्रह्म संध्याक्षरं }%
  \dontdisplaylinenum}    \var{{\devanagarifont \numemph\va\textbf{॰गतिः}\lem \msCc\Ed, ॰गति \msCa\msCb\msNa\msNb\ \unmetr, ॰गतिं \msNc\oo 
\textbf{॰क्षरं}\lem \mssALL, ॰क्षर \msCb}}% 


\nemslokab

{\devanagarifont ध्यायेन्नित्यमतन्द्रितो ह्यनुपमं व्यक्तात्मवेद्यं शिवम्  \danda\dontdisplaylinenum }%
     \var{{\devanagarifont \numnoemph\vb\textbf{॰तन्द्रितो}\lem \msCa\msNa\msNc\Ed, ॰नन्द्रितो \msCb, ॰तन्द्रिय \msCc, ॰तन्द्रियं \msNb\oo 
\textbf{॰वेद्यं}\lem \mssALL, ॰वेद्य \msNb\ \unmetr}}% 

\nemslokac

{\devanagarifont रूपैर्वर्णगुणादिभिश्च विहितं दुर्लक्ष्यलक्ष्योत्तमं }%
  \dontdisplaylinenum    \var{{\devanagarifont \numnoemph\vc\textbf{रूपैर्व॰}\lem \msCa\msNa\msNc\Ed, रूपै व॰ \msCb\msCc\msNb\oo 
\textbf{विहितं}\lem \mssALL, रहितं \msNapcorr(?)\Ed\oo 
\textbf{दुर्लक्ष्यलक्ष्योत्तमम्}\lem \msCa\msNb, 
दुर्लक्ष्यलक्षोत्तमम् \msCb\msCc\msNc\Ed, 
दुलक्ष्यलक्ष्योत्तमम् \msNa}}% 

%Verse 11:58


\nemslokad

{\devanagarifont यत्नोद्धृत्य समाश्रयेत्सुरगुरुं सर्वार्तिहर्ता हरम् {॥ ११:५८॥} \veg\dontdisplaylinenum }%
     \var{{\devanagarifont \numnoemph\vd\textbf{यत्नोद्धृत्य}\lem \mssALL, यत्नाद्धृत्य \Ed\oo 
\textbf{समाश्रये॰}\lem \mssALL, मणाश्रये॰ \msNb\oo 
\textbf{सर्वार्तिहर्ता हरम्}\lem \mssCaCbCc\msNb, सर्वार्त्तिह\uncl{र्त्ता} हरं \msNa, 
सर्वात्तिहर्त्ता हरं \msNc, 
सर्वार्तिहन् शङ्करम् \Ed}}% 

\nemslokanormal


\vers


{\devanagarifont 
\jump
\begin{center}
\ketdanda~इति वृषसारसंग्रहे चतुराश्रमधर्मविधानो नामाध्याय एकादशमः~\ketdanda
\end{center}
\dontdisplaylinenum\vers  }%
     \var{{\devanagarifont \numnoemph{\englishfont \Colo:}\textbf{नामाध्याय एकादशमः}\lem \mssALL, नामाध्याय एकादश \msNc, 
नाम एकादशो ऽध्यायः \Ed}}% 
\bekveg\szamveg
\vfill
\phpspagebreak

\versno=0\fejno=12
\thispagestyle{empty}

\centerline{\Large\devanagarifontbold [   द्वादशमो ऽध्यायः  ]}{\vrule depth10pt width0pt} \fancyhead[CO]{{\footnotesize\devanagarifont वृषसारसंग्रहे  }}
\fancyhead[CE]{{\footnotesize\devanagarifont द्वादशमो ऽध्यायः  }}
\fancyhead[LE]{}
\fancyhead[RE]{}
\fancyhead[LO]{}
\fancyhead[RO]{}
\szam\bek



\alalfejezet{आतिथ्यधर्मः}
\vers


{\devanagarifont देव्युवाच {\dandab}\dontdisplaylinenum  }%
 
{\devanagarifont अहिंसा परमो धर्मः सततं परिकीर्त्यते \thinspace{\danda} \dontdisplaylinenum }%
     \var{{\devanagarifont \numemph\vab\textbf{धर्मः स॰}\lem \mssALL, धर्मोस्स॰ \msCc}}% 
    \lacuna{\devanagarifontsmall {\englishfont Witnesses used for this chapter: \msCa\ ff.\thinspace 210r--215r, 
                                              \msCb\ ff.\thinspace 215v--219v, 
                                              \msCc\ ff.\thinspace 287v--283v 
                                                        (f.\thinspace 291 is missing),
                                              \msNa\ ff.\thinspace 17v--22r, 
                                              \msNb\ exp.\thinspace 58 (lower) -- 62 (lower),
                                              \msNc\ ff.\thinspace 225v--230r,
                                              \Ed\ pp.\thinspace 617--628; 
                                              \mssCaCbCc\ = \msCa + \msCb + \msCc} }%
  
%Verse 12:1

{\devanagarifont आतिथ्यकानां धर्मं च कथयस्व यदुत्तमम् {॥ १२:१॥} \veg\dontdisplaylinenum }%
     \var{{\devanagarifont \numnoemph\vc\textbf{आतिथ्य॰}\lem \mssALL, अतिथ्य॰ \msCb\msNb\oo 
\textbf{धर्मं च}\lem \mssALL, 
धर्मश्च \msCc, धर्मानां \msNb}}% 

{\devanagarifont महेश्वर उवाच {\dandab}\dontdisplaylinenum  }%
     \var{{\devanagarifont \numemph\vo\textbf{महेश्वर}\lem \mssALL, भगवान् \msNa}}% 

{\devanagarifont अहिंसातिथ्यकानां च शृणु धर्मं यदुत्तमम् \thinspace{\danda} \dontdisplaylinenum }%
     \var{{\devanagarifont \numnoemph\vb\textbf{शृणु}\lem \mssALL, \lac  णु \msCa\oo 
\textbf{धर्मं}\lem \mssALL, धर्म \msCc\Ed\oo 
\textbf{॰त्तमम्}\lem \mssALL, ॰त्तमां \Ed}}% 

%Verse 12:2

{\devanagarifont त्रैलोक्यमखिलं देवि रत्नपूर्णं सुलोचने {॥ १२:२॥} \veg\dontdisplaylinenum }%
     \var{{\devanagarifont \numnoemph\vd\textbf{॰पूर्णं}\lem \mssALL, पूर्ण्ण \msCc, ॰पूर्णां \Ed\oo 
\textbf{॰लोचने}\lem \mssALL, ॰लोचनं \msCb}}% 

{\devanagarifont चतुर्वेदविदे दानं न तत्तुल्यमहिंसकः \thinspace{\dandab} \dontdisplaylinenum }%
     \var{{\devanagarifont \numemph\va\textbf{दानं}\lem \mssALL, नानं \msCb}}% 

%Verse 12:3

{\devanagarifont शृणु धर्ममतिथ्यानां कीर्तयिष्यामि सुन्दरि {॥ १२:३॥} \veg\dontdisplaylinenum }%
 

\alalfejezet{विपुलोपाख्यानम्}
{\devanagarifont आसीद्वृत्तं पुराख्यानं नगरे कुसुमाह्वये \thinspace{\dandab} \dontdisplaylinenum }%
     \var{{\devanagarifont \numemph\va\textbf{आसीद्वृत्तं}\lem \msCa\msNa\Ed, आशीदत्तं \msCb, आसीद्वृतम् \msCc, आसी वृत्तं \msNb, आसीद्वृत्त \msNc\oo 
\textbf{॰ख्यानं}\lem \mssALL, ॰ख्यातं \Ed}}% 
    \var{{\devanagarifont \numnoemph\vb \lem \mssALL, 
नगरं कुसुमाह्वयम् \msCc\msNb}}% 

%Verse 12:4

{\devanagarifont कपिलस्य सुतो विद्वान्विपुलो नाम विश्रुतः {॥ १२:४॥} \veg\dontdisplaylinenum }%
 
{\devanagarifont धर्मनित्यो जितक्रोधः सत्यवादी जितेन्द्रियः \thinspace{\dandab} \dontdisplaylinenum }%
     \paral{{\devanagarifontsmall \vb {\englishfont  = \MBH\ 12.218.13b} }}

%Verse 12:5

{\devanagarifont ब्रह्मण्यश्च कृतज्ञश्च मद्भक्तः कृतनिश्चयः {॥ १२:५॥} \veg\dontdisplaylinenum }%
     \var{{\devanagarifont \numemph\vc\textbf{ब्रह्मण्य॰}\lem \msCb\msNa\msNb\Ed, ब्राह्मण्य॰ \msCa\msCc\msNc\oo 
\textbf{॰ज्ञश्च}\lem \mssALL, 
॰ज्ञ \msCb, ॰ज्ञश्च \msNb}}% 
    \var{{\devanagarifont \numnoemph\vd\textbf{॰भक्तः}\lem \mssALL, ॰भक्त॰ \Ed}}% 

{\devanagarifont धनाढ्यो ऽतिथिपूज्यश्च दाता दान्तो दयालुकः \thinspace{\dandab} \dontdisplaylinenum }%
     \var{{\devanagarifont \numemph\va\textbf{॰पूज्यश्च}\lem \msCa\msCc\msNapcorr\msNc\Ed, 
॰पूज्य \msCb\msNaacorr, ॰पूजश्च \msNb}}% 
    \var{{\devanagarifont \numnoemph\vb\textbf{दान्तो}\lem \msCbacorr\msNc\Ed, 
दान्त \msCa\msCc\msNa, दान्तोम{\englishfont (?)} \msCbpcorr, 
दान्त \msNb}}% 

%Verse 12:6

{\devanagarifont न्यायार्जितधनो नित्यमन्यायपरिवर्जितः {॥ १२:६॥} \veg\dontdisplaylinenum }%
     \var{{\devanagarifont \numnoemph\vc\textbf{न्याया॰}\lem \msCc\msNa\msNc\Ed, न्यायो॰ \msCa\msCb\msNb}}% 
    \var{{\devanagarifont \numnoemph\vcd\textbf{नित्यम॰}\lem \mssALL, नित्यंम॰ \msNb}}% 
    \var{{\devanagarifont \numnoemph\vd\textbf{॰वर्जितः}\lem \mssALL, ॰वर्जयेत् \msNb}}% 

{\devanagarifont भार्या च रूपिणी तस्य चन्द्रबिम्बशुभानना \thinspace{\dandab} \dontdisplaylinenum }%
     \var{{\devanagarifont \numemph\vb\textbf{॰बिम्ब॰}\lem \mssALL, ॰बिं\uncl{बा} \msNa\oo 
\textbf{॰शुभानना}\lem \mssALL, ॰निभानना \msNb}}% 

{\devanagarifont पीनोत्तुङ्गस्तनी कान्ता सकलानन्दकारिणी  \danda\dontdisplaylinenum }%
     \var{{\devanagarifont \numnoemph\vd\textbf{सकला॰}\lem \mssALL, \lac\  \msCa}}% 

%Verse 12:7

{\devanagarifont पतिव्रता पतिरता पतिशुश्रूषणे रता {॥ १२:७॥} \veg\dontdisplaylinenum }%
     \var{{\devanagarifont \numnoemph\ve\textbf{पतिव्रता}\lem \mssALL, प्रतिव्रता \msCb\oo 
\textbf{पतिरता}\lem \mssALL, प्रतिरता \msCb\msNb}}% 
    \var{{\devanagarifont \numnoemph\vf\textbf{पतिशुश्रूषणे}\lem \mssALL, प्रतिशुश्रूषणे \msNb}}% 
    \paral{{\devanagarifontsmall \vef {\englishfont \compare\ \BrahmaVP\ 4.27.174cd:}
                          पतिव्रते पतिरते पतिं देहि नमो ऽस्तु ते }}

{\devanagarifont अथ केनापि कालेन सूर्यरागमभूत्ततः \thinspace{\dandab} \dontdisplaylinenum }%
     \var{{\devanagarifont \numemph\vb\textbf{॰भूत्ततः}\lem \mssALL, ॰भूततः \msCc}}% 

%Verse 12:8

{\devanagarifont ग्रस्तभागत्रयस्त्वासीत्कृष्णमाधवमासिके {॥ १२:८॥} \veg\dontdisplaylinenum }%
 
{\devanagarifont स्नातुकामावतीर्यन्ते सर्वे पौरनृपादयः \thinspace{\dandab} \dontdisplaylinenum }%
     \var{{\devanagarifont \numemph\va\textbf{॰वतीर्यन्ते}\lem \mssALL, च तीर्थन्ते \Ed}}% 

%Verse 12:9

{\devanagarifont देवाश्च पितरश्चैव तर्प्यन्ते विधिवत्तथा {॥ १२:९॥} \veg\dontdisplaylinenum }%
     \var{{\devanagarifont \numnoemph\vc\textbf{देवाश्च}\lem \mssALL, देवश्च \msCc}}% 
    \var{{\devanagarifont \numnoemph\vd\textbf{तर्प्यन्ते}\lem \mssALL, तप्यन्ते \msCb\msNb}}% 

{\devanagarifont केचिज्जुह्वति तत्राग्निं केचिद्विप्रांश्च तर्पयेत् \thinspace{\dandab} \dontdisplaylinenum }%
     \var{{\devanagarifont \numemph\va\textbf{॰चिज्जुह्वति}\lem \mssALL, 
॰चिज्जुति \msCb, ॰चि\uncl{ज्व}ह्वति \msCc}}% 
    \var{{\devanagarifont \numnoemph\vb\textbf{विप्रांश्च}\lem \mssALL, विप्राश्च \msCb}}% 

%Verse 12:10

{\devanagarifont केचिद्दानोपतिष्ठन्ति केचित्स्तुवन्ति देवताम् {॥ १२:१०॥} \veg\dontdisplaylinenum }%
     \var{{\devanagarifont \numnoemph\vc\textbf{दानो॰}\lem \mssALL, ध्यानो॰ \Ed}}% 
    \var{{\devanagarifont \numnoemph\vd\textbf{केचित्स्तुवन्ति}\lem \msCa\msCb\msNc, केचिद्वन्ति \msCc, 
केचि स्तुवन्ति \msNa\msNb, 
केचित्स्तुन्वन्ति \Ed\oo 
\textbf{देवताम्}\lem \mssALL, देवता \msCb\msNc}}% 

{\devanagarifont ध्यानयोगरताः केचित्केचित्पञ्चतपे रताः \thinspace{\dandab} \dontdisplaylinenum }%
     \var{{\devanagarifont \numemph\va\textbf{॰रताः}\lem \mssALL, ॰रता \msNb}}% 

%Verse 12:11

{\devanagarifont एवं प्रवर्तमानेषु राजनादिषु सर्वशः {॥ १२:११॥} \veg\dontdisplaylinenum }%
     \var{{\devanagarifont \numnoemph\vd\textbf{राजना॰}\lem \mssALL, राजाना॰ \Ed}}% 

{\devanagarifont विपुलो ऽपि हि तत्रैव गङ्गागण्डकिसंगमे \thinspace{\dandab} \dontdisplaylinenum }%
     \var{{\devanagarifont \numemph\va\textbf{ऽपि हि}\lem \msCa\msCc\msNapcorr\msNb\msNc, 
पि \msCb, हि न \msNaacorr, पि च \Ed}}% 

%Verse 12:12

{\devanagarifont भार्यया सह तत्रैव स्नात्वा क्षोमविभूषणः {॥ १२:१२॥} \veg\dontdisplaylinenum }%
     \var{{\devanagarifont \numnoemph\vc\textbf{भार्यया}\lem \msCapcorr\msCb\msNa\msNb\msNc, भार्याया \msCaacorr\msCc\Ed}}% 
    \var{{\devanagarifont \numnoemph\vd\textbf{॰भूषणः}\lem \mssALL, 
॰भूष\uncl{णैः} \msCc, ॰भूषितः \msNa}}% 

{\devanagarifont देवतागुरुविप्राणामन्येषां तर्पणे रतः \thinspace{\dandab} \dontdisplaylinenum }%
     \var{{\devanagarifont \numemph\vab \lem \msCb\msNapcorr\msNb\msNc, 
देवतागुरुवि\lac  णामन्येषां तर्पणे रतः \msCa, 
देवतागुरुविप्राणामन्येषां तर्पणे रताः \msCc, 
\om\ \msNaacorr, 
देवतागुरुविप्राणामन्येषां तर्पणा रतः \Ed}}% 

%Verse 12:13

{\devanagarifont तत्रावसरसम्प्राप्तो ब्राह्मणो ऽतिथिरागतः {॥ १२:१३॥} \veg\dontdisplaylinenum }%
 
{\devanagarifont भार्या तस्यातिरूपेण मोहिता ब्रह्मणस्तदा \thinspace{\dandab} \dontdisplaylinenum }%
     \var{{\devanagarifont \numemph\vb\textbf{मोहिता}\lem \mssALL, मोहितो \msCb\oo 
\textbf{ब्रह्मणस्तदा}\lem \msCa\msCb\msNc, ब्राह्मणास्तथा \msCc, 
ब्राह्मणस्तदा \msNa\msNb, ब्राह्मणस्य च \Ed}}% 

%Verse 12:14

{\devanagarifont ब्राह्मणो ऽपि तथैवेह रूपेणाप्रतिमो भवेत् {॥ १२:१४॥} \veg\dontdisplaylinenum }%
     \var{{\devanagarifont \numnoemph\vc\textbf{ब्राह्मणो}\lem \mssALL, ब्रह्मणो \msCb\oo 
\textbf{तथैवेह}\lem \msCb\msNa\msNb\Ed, 
त\uncl{थे}वेह \msCa, तथेवेह \msCc\msNc}}% 
    \var{{\devanagarifont \numnoemph\vd\textbf{रूपेणा॰}\lem \msCa\msNa\msNb\msNc, रूपेना॰ \msCb, रूपेण \msCc, रूपिणा॰ \Ed}}% 

{\devanagarifont अन्योन्यदृष्टिसंसक्तौ जातौ तौ तु परस्परम् \thinspace{\dandab} \dontdisplaylinenum }%
     \var{{\devanagarifont \numemph\va\textbf{॰संसक्तौ}\lem \Ed, ॰संशक्तौ \msCa\msNa\msNc, 
॰शक्तौ \msCb, ॰संसक्तो \msCc\msNb}}% 
    \var{{\devanagarifont \numnoemph\vb\textbf{जातौ तौ}\lem \mssALL, 
जातो तौ तौ \msCc, जातौ \uncl{ता} \msNc}}% 

%Verse 12:15

{\devanagarifont विपुलेनाञ्जलिं कृत्वा ब्राह्मण संशितव्रत {॥ १२:१५॥} \veg\dontdisplaylinenum }%
     \var{{\devanagarifont \numnoemph\vd\textbf{ब्राह्मण}\lem \msCb\msCc, ब्राह्मणः \msCa\msNa\msNb\msNc\Ed\oo 
\textbf{॰शित॰}\lem \eme, ॰श्रित॰ \mssCaCbCc\msNa\msNb\msNc\Ed\oo 
\textbf{॰व्रत}\lem \conj, ॰व्र\lk\ \msCa, ॰व्रतः \msCb\msCc\msNa\msNb\msNc\Ed}}% 
    \paral{{\devanagarifontsmall \vd {\englishfont  = MBh 12.213.18d and 12.347.1d } }}

{\devanagarifont आज्ञापय द्विजश्रेष्ठ अद्य मे ऽनुग्रहं कुरु \thinspace{\dandab} \dontdisplaylinenum }%
     \var{{\devanagarifont \numemph\vb\textbf{॰ग्रहं}\lem \mssALL, ॰ग्रह \msCb}}% 

%Verse 12:16

{\devanagarifont भार्याभृत्यपशुग्राम रत्नानि विविधानि च {॥ १२:१६॥} \veg\dontdisplaylinenum }%
     \var{{\devanagarifont \numnoemph\vc\textbf{॰भृत्य॰}\lem \mssALL, ॰भृत्या॰ \msCc}}% 

{\devanagarifont विपुलेनैवमुक्तस्तु गृहीतो ब्राह्मणो ऽब्रवीत् \thinspace{\dandab} \dontdisplaylinenum }%
     \var{{\devanagarifont \numemph\vb\textbf{ब्राह्मणो ऽब्रवीत्}\lem \mssALL, 
भ्राह्मणस्तथा \msCc}}% 

%Verse 12:17

{\devanagarifont यदि सत्यं प्रदातासि सुप्रसन्नं मनस्तव {॥ १२:१७॥} \veg\dontdisplaylinenum }%
     \var{{\devanagarifont \numnoemph\vc \lem \mssALL, \om\ \msCc}}% 
    \var{{\devanagarifont \numnoemph\vd \lem \msCa\msCb\msNa\msNc, \om\ \msCc, 
सुप्रसन्नमनस्तव \msNb\Ed}}% 

{\devanagarifont विपुल उवाच {\dandab}\dontdisplaylinenum  }%
 
{\devanagarifont सुप्रसन्नं मनो मे ऽद्य सुप्रसन्नं तपःफलम् \thinspace{\danda} \dontdisplaylinenum }%
     \var{{\devanagarifont \numemph\va\textbf{॰प्रसन्नं मनो}\lem \mssALL, 
॰प्रसन्नमनो \msCc\msNb}}% 
    \var{{\devanagarifont \numnoemph\vb\textbf{सुप्रसन्नं तपः॰}\lem \mssALL, 
सुप्रसन्नतपः॰ \msNb}}% 

{\devanagarifont शीघ्रमाज्ञापय विप्र यच्चाभिलषितं तव  \danda\dontdisplaylinenum }%
     \var{{\devanagarifont \numnoemph\vc\textbf{शीघ्र॰}\lem \mssALL, श्रीघ्र॰ \msNb}}% 

%Verse 12:18

{\devanagarifont अदेयं नास्ति विप्रस्य स्वशिरःप्रभृति द्विज {॥ १२:१८॥} \veg\dontdisplaylinenum }%
     \var{{\devanagarifont \numnoemph\ve\textbf{अदेयं}\lem \mssALL, अदेय \msNb}}% 
    \var{{\devanagarifont \numnoemph\vf\textbf{स्वशिरः॰}\lem \mssALL, शरीर॰ \msNa\oo 
\textbf{॰भृति}\lem \mssALL, ॰भृतिर् \Ed}}% 

{\devanagarifont ब्राह्मण उवाच {\dandab}\dontdisplaylinenum  }%
     \var{{\devanagarifont \numemph\vo\textbf{ब्राह्मण}\lem \mssALL, 
ब्राह्मणा \msCaacorr, ब्रह्म \msNb}}% 

{\devanagarifont यद्येवं वदसे भद्र भार्यां मे देहि रूपिणीम् \thinspace{\danda} \dontdisplaylinenum }%
     \var{{\devanagarifont \numnoemph\vb\textbf{भार्यां}\lem \mssALL, भार्या \msNb\msNc}}% 

%Verse 12:19

{\devanagarifont स्वस्ति भवतु भद्रं वः कल्याणं भव शाश्वतम् {॥ १२:१९॥} \veg\dontdisplaylinenum }%
     \var{{\devanagarifont \numnoemph\vc\textbf{स्वस्ति}\lem \mssALL, स्वस्तिं \msNb, स्वस्तिर् \Ed}}% 
    \var{{\devanagarifont \numnoemph\vd\textbf{कल्याणं}\lem \mssALL, कल्या\uncl{ण} \msCc\oo 
\textbf{भव}\lem \mssALL, तव \Ed}}% 

{\devanagarifont विपुल उवाच {\dandab}\dontdisplaylinenum  }%
     \var{{\devanagarifont \numemph\vo\textbf{विपुल}\lem \mssALL, विप्र \Ed}}% 

{\devanagarifont प्रतीच्छ भार्यां सुश्रोणीं रूपयौवनशालिनीम् \thinspace{\danda} \dontdisplaylinenum }%
     \var{{\devanagarifont \numnoemph\va\textbf{भार्यां}\lem \mssALL, भार्या \msNb\oo 
\textbf{॰श्रोणीं}\lem \msCa\msCb\msNapcorr\msNc\Ed, ॰श्रोणि \msCc\msNaacorr\msNb}}% 
    \var{{\devanagarifont \numnoemph\vb\textbf{॰शालिनीम्}\lem \mssALL, ॰शालिनी \msNb, ॰शीलिनीं \msNc}}% 

%Verse 12:20

{\devanagarifont अकुत्सितां विशालाक्षीं पूर्णचन्द्रनिभाननाम् {॥ १२:२०॥} \veg\dontdisplaylinenum }%
     \var{{\devanagarifont \numnoemph\vc \lem \mssALL, 
अकुत्सि\uncl{ता} विशालाक्षि \msCc, 
अकुत्सिता विशालाक्सी \msNb}}% 
    \var{{\devanagarifont \numnoemph\vd\textbf{॰निभाननाम्}\lem \mssALL, ॰निभानना \msNb}}% 

{\devanagarifont भार्योवाच {\dandab}\dontdisplaylinenum  }%
 
{\devanagarifont परित्याज्या कथं नाथ अपापां त्यजसे कथम् \thinspace{\danda} \dontdisplaylinenum }%
     \var{{\devanagarifont \numemph\va\textbf{॰त्याज्या}\lem \msCa\msNa\msNc\Ed, 
॰त्याज्य \msCb\msNb, ॰त्या\uncl{ज्य} \msCc}}% 

%Verse 12:21

{\devanagarifont अतीव हि प्रियां भार्यां निर्दोषां च कथं त्यजेः {॥ १२:२१॥} \veg\dontdisplaylinenum }%
     \var{{\devanagarifont \numnoemph\vc\textbf{प्रियां}\lem \mssALL, प्रियं \msCc\msNb}}% 
    \var{{\devanagarifont \numnoemph\vd\textbf{निर्दोषां}\lem \mssALL, निर्दोष \msCc\oo 
\textbf{त्यजेः}\lem \msCa\msNa\msNc, त्यज्येत् \msCb\msCc, त्यजेत् \msNb\Ed\oo 
\textbf{च}\lem \conj, स \mssCaCbCc\msNa\msNb\msNc\Ed}}% 

{\devanagarifont सखा भार्या मनुष्याणामिह लोके परत्र च \thinspace{\dandab} \dontdisplaylinenum }%
     \var{{\devanagarifont \numemph\vab\textbf{मनुष्याणामिह}\lem \mssALL, 
मनुष्याणांमिह \msCc}}% 

%Verse 12:22

{\devanagarifont दानं वा सुमहद्दत्त्वा यज्ञो वा सुबहुः कृतः {॥ १२:२२॥} \veg\dontdisplaylinenum }%
     \var{{\devanagarifont \numnoemph\vd\textbf{॰बहुः}\lem \eme, ॰बहु \mssCaCbCc\msNa\msNc\ \unmetr, 
॰बहुं \msNb, ॰बहून् \Ed\oo 
\textbf{कृतः}\lem \mssALL, कृतम् \msCc}}% 

{\devanagarifont अपुत्रो नाप्नुयात्स्वर्गं तपोभिर्वा सुदुष्करैः \thinspace{\dandab} \dontdisplaylinenum }%
     \var{{\devanagarifont \numemph\vab\textbf{स्वर्गं तपोभिर्वा}\lem \mssALL, 
स्व\uncl{र्ग्गन्} \lac  र्व्वा \msCa}}% 

%Verse 12:23

{\devanagarifont श्रुतो मे पितृभिः प्रोक्तो ब्राह्मणैश्च ममान्तिके {॥ १२:२३॥} \veg\dontdisplaylinenum }%
     \var{{\devanagarifont \numnoemph\vd\textbf{॰न्तिके}\lem \mssALL, ॰न्तिकैः \msCb}}% 

{\devanagarifont अपुत्रो नाप्नुयात्स्वर्गं श्रुतं मे बहुशः पुरा \thinspace{\dandab} \dontdisplaylinenum }%
     \var{{\devanagarifont \numemph\va\textbf{स्वर्गं}\lem \msCa\msNa\msNc\Ed, स्वर्ग \msCb\msCc\msNb}}% 

%Verse 12:24

{\devanagarifont मन्दपालो द्विजश्रेष्ठो गतः स्वर्गं तपोबलात् {॥ १२:२४॥} \veg\dontdisplaylinenum }%
     \var{{\devanagarifont \numnoemph\vc\textbf{॰पालो}\lem \msNc\Ed, ॰पाल \mssCaCbCc\msNa\msNb}}% 

{\devanagarifont दानानि च बहून्दत्त्वा यज्ञांश्च विविधांस्तथा \thinspace{\dandab} \dontdisplaylinenum }%
     \var{{\devanagarifont \numemph\va\textbf{बहून्द॰}\lem \mssALL, बहू द॰ \msNc}}% 
    \var{{\devanagarifont \numnoemph\vb \lem \msCa\msCc\msNa\msNb, 
यत्वा यज्ञांश्च विविधां तथा \msCb, 
यज्ञांश्च विविधाम्तथा \msNc, 
स्यज्ञाश्च विविधास्तथा \Ed}}% 

%Verse 12:25

{\devanagarifont वेदांश्च जपयज्ञांश्च कृत्वा स द्विजसत्तमः {॥ १२:२५॥} \veg\dontdisplaylinenum }%
     \var{{\devanagarifont \numnoemph\vc \lem \msCa\msCc\msNa\msNc, 
वेदाश्च जपयज्ञांश्च \msCb, वेदांश्च जपयज्ञाश्च \msNb, 
वेदाश्च जपयज्ञाश्च \Ed}}% 
    \var{{\devanagarifont \numnoemph\vd\textbf{स द्वि॰}\lem \conj, तद्द्वि॰ \mssCaCbCc\msNa\Ed, तद्द्वि॰ \msNb, सद्द्वि॰ \msNc\oo 
\textbf{॰सत्तमः}\lem \mssALL, ॰सत्तम \msNa}}% 

{\devanagarifont प्राप्तद्वारो ऽपि यस्यापि देवदूतैर्निवारितः \thinspace{\dandab} \dontdisplaylinenum }%
     \var{{\devanagarifont \numemph\va\textbf{॰द्वारो}\lem \mssALL,   ॰द्वारे \msNb}}% 
    \var{{\devanagarifont \numnoemph\vab\textbf{यस्यापि दे॰}\lem \mssALL, यस्यापि द्दे॰ \msNb, 
यस्याहि दे॰ \Ed}}% 
    \var{{\devanagarifont \numnoemph\vb\textbf{॰दूतैर्नि॰}\lem \mssALL, ॰दूतै न्नि॰ \msNb, 
॰दूतै नि॰ \msNc}}% 

%Verse 12:26

{\devanagarifont अपुत्रो नाप्नुयात्स्वर्गं यदि यज्ञशतैरपि {॥ १२:२६॥} \veg\dontdisplaylinenum }%
     \var{{\devanagarifont \numnoemph\vc\textbf{॰यात्स्वर्गं}\lem \mssALL, 
॰यात्स्वर्ग्ग \msCc}}% 
    \var{{\devanagarifont \numnoemph\vd\textbf{॰शतैरपि}\lem \mssALL, करोति यः \msCc}}% 

{\devanagarifont इत्युक्तस्तु च्युतः स्वर्गान्मन्दपालो महानृषिः \thinspace{\dandab} \dontdisplaylinenum }%
     \var{{\devanagarifont \numemph\va\textbf{॰क्तस्तु च्युतः}\lem \mssALL, 
॰क्तस्तु\uncl{म्च्यु}तः \msCc}}% 

%Verse 12:27

{\devanagarifont पुत्रानुत्पादयामास शारङ्गांश्चतुरो द्विजः {॥ १२:२७॥} \veg\dontdisplaylinenum }%
     \var{{\devanagarifont \numnoemph\vc\textbf{पुत्रानु॰}\lem \mssALL, पुत्रमु॰ \msCc}}% 
    \var{{\devanagarifont \numnoemph\vd\textbf{शारङ्गांश्च}\lem \msNa\msNc, शारङ्गाश्च \msCa, शारङ्गंश्च \msCb, 
शारङ्गश्च \msCc\msNb, शारङ्गाच्च \Ed\oo 
\textbf{द्विजः}\lem \mssALL, द्विज \msCc}}% 

{\devanagarifont तेन पुण्यप्रभावेण स्वर्गं प्राप्तो ह्यवारितः \thinspace{\dandab} \dontdisplaylinenum }%
     \var{{\devanagarifont \numemph\vb\textbf{स्वर्गं}\lem \mssALL, स्वर्ग्ग \msCc\oo 
\textbf{॰वारितः}\lem \mssALL, ॰वरितः \msNb}}% 

%Verse 12:28

{\devanagarifont कुलत्राणात्कलत्रास्मि भरणाद्भार्य एव च {॥ १२:२८॥} \veg\dontdisplaylinenum }%
     \var{{\devanagarifont \numnoemph\vc\textbf{कुल॰}\lem \msCb, कल॰ \msCa\msCc\msNa\msNb\msNc\Ed\oo 
\textbf{॰त्राणात्क॰}\lem \msNb, ॰त्राणां क॰ \mssCaCbCc\msNa\Ed, ॰त्राणा क॰ \msNc\oo 
\textbf{॰स्मि}\lem \mssALL, ॰स्मिं \msNb}}% 
    \var{{\devanagarifont \numnoemph\vd\textbf{॰आद्भार्य एव}\lem \msCa\msNa\msNc\Ed, 
॰आद्भार्यमेव \msCb, ॰आ भार्य एव \msCc\msNb}}% 

{\devanagarifont दारसंग्रह पुत्रार्थे क्रियते शास्त्रदर्शनात् \thinspace{\dandab} \dontdisplaylinenum }%
     \var{{\devanagarifont \numemph\va\textbf{॰ग्रह}\lem \msCc\msNb\msNc\Ed, ॰ग्रहः \msCa\msCb\msNa\oo 
\textbf{पुत्रा॰}\lem \mssALL, पात्रा॰ \Ed}}% 
    \var{{\devanagarifont \numnoemph\vb\textbf{क्रियते}\lem \mssALL, क्रियाते \msCb}}% 

%Verse 12:29

{\devanagarifont यानि सन्ति गृहे द्रव्यं ग्रामघोषगृहाणि च {॥ १२:२९॥} \veg\dontdisplaylinenum }%
 
{\devanagarifont दातुमर्हसि विप्राय न मां दातुमिहार्हसि \thinspace{\dandab} \dontdisplaylinenum }%
 
%Verse 12:30

{\devanagarifont भार्याया वचनं श्रुत्वा विपुलः पुनरब्रवीत् {॥ १२:३०॥} \veg\dontdisplaylinenum }%
     \var{{\devanagarifont \numemph\vc\textbf{वचनं}\lem \mssALL, वचन \msNc}}% 
    \var{{\devanagarifont \numnoemph\vd\textbf{॰ब्रवीत्}\lem \mssALL, 
॰ब्रवीत्\thinspace{\devanagarifont ।} विपुल उवाच\thinspace{\devanagarifont ।} \msCcpcorr\Ed}}% 

{\devanagarifont साधु भामिनि जानामि साधु साधु पतिव्रते \thinspace{\dandab} \dontdisplaylinenum }%
     \var{{\devanagarifont \numemph\va\textbf{जानामि}\lem \msCb\msCc\msNa\Ed, जानासि \msCa\msNb\msNc}}% 
    \var{{\devanagarifont \numnoemph\vb\textbf{पति॰}\lem \mssALL, प्रति॰ \msNb}}% 

%Verse 12:31

{\devanagarifont जितो ऽस्म्यनेन वाक्येन अनेनास्मि हि तोषितः {॥ १२:३१॥} \veg\dontdisplaylinenum }%
     \var{{\devanagarifont \numnoemph\vd\textbf{तोषितः}\lem \mssALL, तोर्षिनः \msNc}}% 

{\devanagarifont अद्य ग्रहणकाले च द्विज आगत्य याचते \thinspace{\dandab} \dontdisplaylinenum }%
 
%Verse 12:32

{\devanagarifont ददामीति प्रतिज्ञाय अदत्त्वा नरकं व्रजे {॥ १२:३२॥} \veg\dontdisplaylinenum }%
     \var{{\devanagarifont \numemph\vd\textbf{व्रजे}\lem \msCa\msNapcorr\msNc, व्रजेत् \msCb\msCc\msNb\Ed, 
व्रजे\lk\ \msNaacorr}}% 

{\devanagarifont नरकं यदि गच्छामि कुलेन सह सुन्दरि \thinspace{\dandab} \dontdisplaylinenum }%
     \var{{\devanagarifont \numemph\va\textbf{यदि}\lem \mssALL, ययदि \msNc}}% 

{\devanagarifont कल्पकोटिसहस्रे ऽपि नरकस्थो यशस्विनि  \danda\dontdisplaylinenum }%
     \var{{\devanagarifont \numnoemph\vc\textbf{॰सहस्रे ऽपि}\lem \mssALL, ॰सहस्राणि \msCc\Ed}}% 
    \var{{\devanagarifont \numnoemph\vd\textbf{॰स्थो य॰}\lem \msNc\Ed, ॰स्थाद्य॰ \msCa\msCc\msNa\msNb, स्था य॰ \msCb}}% 

%Verse 12:33

{\devanagarifont मुक्तिमेव न पश्यामि जन्मकोटिशतैरपि {॥ १२:३३॥} \veg\dontdisplaylinenum }%
     \var{{\devanagarifont \numnoemph\ve\textbf{मुक्तिमेव}\lem \mssALL, मुक्तिमेवन् \Ed}}% 

{\devanagarifont अदानाच्चाशुभं देवि पश्यामि वरवर्णिनि \thinspace{\dandab} \dontdisplaylinenum }%
     \var{{\devanagarifont \numemph\va\textbf{अदानाच्चा॰}\lem \mssALL, अदाना चा॰ \msCc}}% 

%Verse 12:34

{\devanagarifont दानेन तु शुभं पश्ये स्वर्गलोके यदक्षयम् {॥ १२:३४॥} \veg\dontdisplaylinenum }%
     \var{{\devanagarifont \numnoemph\vd\textbf{॰लोके}\lem \mssALL, 
\om\ \msNaacorr, ॰लोकं \Ed}}% 

{\devanagarifont नोक्तं मयानृतं पूर्वं नित्यं सत्यव्रते स्थितः \thinspace{\dandab} \dontdisplaylinenum }%
     \var{{\devanagarifont \numemph\va\textbf{नोक्तं}\lem \mssALL, नोक्ता \msNcacorr}}% 
    \var{{\devanagarifont \numnoemph\vb\textbf{॰व्रते}\lem \mssALL, ॰व्रत॰ \Ed}}% 

%Verse 12:35

{\devanagarifont सत्यधर्ममतिक्रम्य नान्यधर्मं समाचरे {॥ १२:३५॥} \veg\dontdisplaylinenum }%
     \var{{\devanagarifont \numnoemph\vd\textbf{॰चरे}\lem \mssALL, ॰चरेत् \msNb\Ed}}% 

{\devanagarifont भार्या धर्मसखेत्येवं त्वया पूर्वमुदाहृतम् \thinspace{\dandab} \dontdisplaylinenum }%
     \var{{\devanagarifont \numemph\va\textbf{धर्म॰}\lem \mssALL, धर्मं \msNa}}% 
    \var{{\devanagarifont \numnoemph\vb\textbf{त्वया}\lem \eme, त्वयि \mssCaCbCc\msNa\msNb\msNc\Ed}}% 

%Verse 12:36

{\devanagarifont यदि धर्मसखायासि सो ऽद्य काल इहागतः {॥ १२:३६॥} \veg\dontdisplaylinenum }%
     \var{{\devanagarifont \numnoemph\vc\textbf{॰सखाया॰}\lem \mssALL, ॰सखा॰ \msCb}}% 

{\devanagarifont द्विजरूपधरो धर्मः स्वयमेव इहागतः \thinspace{\dandab} \dontdisplaylinenum }%
     \var{{\devanagarifont \numemph\va\textbf{॰धरो}\lem \mssALL, ॰परो \msCb}}% 

%Verse 12:37

{\devanagarifont जिज्ञासार्थमहं भद्रे न विघ्नं कर्तुमर्हसि {॥ १२:३७॥} \veg\dontdisplaylinenum }%
     \var{{\devanagarifont \numnoemph\vc\textbf{॰र्थमहं}\lem \mssALL, 
॰र्थम्महं \msNb, ॰र्थमह \msNc}}% 

{\devanagarifont माताव्यक्तः पिता ब्रह्मा बुद्धिर्भार्या दमः सखा \thinspace{\dandab} \dontdisplaylinenum }%
     \var{{\devanagarifont \numemph\va\textbf{॰व्यक्तः}\lem \mssALL, 
॰व्यक्त \msCc, ॰व्यक्त\uncl{ऽ} \msNc}}% 
    \var{{\devanagarifont \numnoemph\vb\textbf{बुद्धिर्भा॰}\lem \msCa\msCb\msNb, बुद्धि भा॰ \msCc\msNa\msNc\Ed\oo 
\textbf{दमः}\lem \mssALL, दम \msNb\ \unmetr\oo 
\textbf{सखा}\lem \mssALL, समा \msCa}}% 

%Verse 12:38

{\devanagarifont पुत्रो धर्मः क्रियाचार्य इत्येते मम बान्धवाः {॥ १२:३८॥} \veg\dontdisplaylinenum }%
 
{\devanagarifont कालश्रेष्ठो ग्रहः सूर्यो गङ्गा श्रेष्ठा नदीषु च \thinspace{\dandab} \dontdisplaylinenum }%
     \var{{\devanagarifont \numemph\va\textbf{॰श्रेष्थो}\lem \msCb\msNa\msNcpcorr, ॰श्रेष्ठ॰ \msCa\msCc\msNb, ॰श्रेष्ठा \msNcacorr, ॰श्रेष्ठः \Ed}}% 
    \var{{\devanagarifont \numnoemph\vb\textbf{श्रेष्ठा}\lem \mssALL, श्रेष्ठो \msNa, श्रेष्ठ \msNb}}% 
    \paral{{\devanagarifontsmall \vb {\englishfont \similar\ 15.18b:} श्रेष्ठा गङ्गा नदीषु च }}

%Verse 12:39

{\devanagarifont चन्द्रक्षये दिनं श्रेष्ठं नरश्रेष्ठो द्विजोत्तमः {॥ १२:३९॥} \veg\dontdisplaylinenum }%
     \var{{\devanagarifont \numnoemph\vc\textbf{दिनं}\lem \msCa\msCb\msNa\msNc, दिन॰ \msCc\msNb\Ed}}% 
    \var{{\devanagarifont \numnoemph\vd\textbf{॰त्तमः}\lem \mssALL, ॰त्तम \msCc}}% 

{\devanagarifont शुश्रूषणार्थं विप्रस्य मया दत्तासि सुन्दरि \thinspace{\dandab} \dontdisplaylinenum }%
     \var{{\devanagarifont \numemph\va\textbf{॰र्थं}\lem \mssALL, ॰र्थ \msCb}}% 

%Verse 12:40

{\devanagarifont सर्वस्वं ब्राह्मणे दत्त्वा वनमेवाश्रयाम्यहम् {॥ १२:४०॥} \veg\dontdisplaylinenum }%
 
{\devanagarifont शङ्कर उवाच {\dandab}\dontdisplaylinenum  }%
     \var{{\devanagarifont \numemph\vo\textbf{शङ्कर}\lem \mssALL, महेश्वर \Ed}}% 

{\devanagarifont तूष्णीम्भूता ततो भार्या अश्रुपूर्णाकुलेक्षणा \thinspace{\danda} \dontdisplaylinenum }%
     \var{{\devanagarifont \numnoemph\va\textbf{तूष्णीम्भूता}\lem \msCa, तूष्णीभूत्वा \msCb, तुष्णीभूत \msCc, तूष्णीभूता \msNa\msNb, 
तुष्णीम्भूती \msNc, तूष्णीभूतां \Ed\oo 
\textbf{भार्या}\lem \mssALL, भार्यां \Ed}}% 
    \var{{\devanagarifont \numnoemph\vb\textbf{॰क्षणा}\lem \msCa\msCb\msNa\msNc, ॰क्षणः \msCc, ॰क्षणाः \msNb, ॰क्षणाम् \Ed}}% 

%Verse 12:41

{\devanagarifont करे गृह्य विशालाक्षी ब्राह्मणाय निवेदिता {॥ १२:४१॥} \veg\dontdisplaylinenum }%
     \var{{\devanagarifont \numnoemph\vc\textbf{॰क्षी}\lem \mssALL, ॰क्षीं \Ed}}% 
    \var{{\devanagarifont \numnoemph\vd \lem \mssALL, 
ब्राह्मय दिवेदिता \msCb}}% 

{\devanagarifont यानि सन्ति गृहे द्रव्यं हिरण्यं पशवस्तथा \thinspace{\dandab} \dontdisplaylinenum }%
     \var{{\devanagarifont \numemph\vb\textbf{हिरण्यं}\lem \mssALL, हिरण्य॰ \msNa\Ed}}% 

%Verse 12:42

{\devanagarifont ददामि ते द्विजश्रेष्ठ ग्रामघोषगृहादिकम् {॥ १२:४२॥} \veg\dontdisplaylinenum }%
     \var{{\devanagarifont \numnoemph\vc\textbf{ददामि}\lem \mssALL, ददानि \msCb\oo 
\textbf{ते द्विज॰}\lem \mssALL, \lac  ज॰ \msCa, त द्विज॰ \msNc}}% 

{\devanagarifont मुक्तावैडूर्यवासांसि दिव्याण्याभरणानि च \thinspace{\dandab} \dontdisplaylinenum }%
     \var{{\devanagarifont \numemph\va\textbf{॰वैडूर्य॰}\lem \msCa\msCb\msNb\msNc, ॰वैभार्य॰ \msCc, ॰वैर्य॰ \msNaacorr, 
॰वैदूर्य॰ \msNapcorr\Ed\oo 
\textbf{॰वासांसि}\lem \mssALL, ॰वासासि \msNc}}% 

%Verse 12:43

{\devanagarifont सर्वान्गृहाण विप्रेन्द्र श्रद्धया दत्तसत्कृतान् {॥ १२:४३॥} \veg\dontdisplaylinenum }%
     \var{{\devanagarifont \numnoemph\vc\textbf{सर्वान्गृहाण}\lem \msCa\msCb\msNa\Ed, सर्वान्तान्गृह्ण \msCc, 
सर्वान्गृहान् \msNb, 
सर्वां गृहाण \msNc}}% 
    \var{{\devanagarifont \numnoemph\vd\textbf{॰सत्कृतान्}\lem \eme, ॰सत्कृताम् \mssCaCbCc\msNa\msNc\Ed, ॰सत्कृतम् \msNb}}% 

{\devanagarifont प्रीयतां भगवान्धर्मः प्रीयतां च महेश्वरः \thinspace{\dandab} \dontdisplaylinenum }%
     \var{{\devanagarifont \numemph\vb\textbf{प्रीय॰}\lem \mssALL, प्रीन॰ \msNcacorr}}% 

%Verse 12:44

{\devanagarifont प्रीयन्तां पितरः सर्वे यद्यस्ति सुकृतं फलम् {॥ १२:४४॥} \veg\dontdisplaylinenum }%
     \var{{\devanagarifont \numnoemph\vc\textbf{प्रीयन्तां}\lem \msCa, प्रीयतां \msCb\msCc\msNa\msNc\Ed, प्रीयता \msNb\oo 
\textbf{पितरः}\lem \mssALL, पितर \msNa}}% 
    \var{{\devanagarifont \numnoemph\vd\textbf{अस्ति}\lem \mssALL, असि \msCa}}% 

{\devanagarifont रुद्र उवाच {\dandab}\dontdisplaylinenum  }%
     \var{{\devanagarifont \numemph\vo\textbf{रुद्र}\lem \mssALL, महेश्वर \Ed}}% 

{\devanagarifont विपुलस्य वचः श्रुत्वा ब्राह्मणेन तपस्विना \thinspace{\danda} \dontdisplaylinenum }%
     \var{{\devanagarifont \numnoemph\va\textbf{वचः श्रुत्वा}\lem \mssALL, 
वच\uncl{श्श्रु}\lac\  \msCa}}% 
    \var{{\devanagarifont \numnoemph\vb\textbf{तपस्विना}\lem \mssALL, तपस्विनाम् \msNb}}% 

%Verse 12:45

{\devanagarifont आशीः सुविपुलं दत्त्वा विपुलाय महात्मने {॥ १२:४५॥} \veg\dontdisplaylinenum }%
 
{\devanagarifont वसेत्तत्र गृहे रम्ये भार्यामादाय तस्य च \thinspace{\dandab} \dontdisplaylinenum }%
     \var{{\devanagarifont \numemph\va\textbf{वसेत्तत्र गृहे}\lem \msCb\msNa, वस तत्र गृहे \msCa\msCc\msNb, 
वस्\uncl{एन्त}त्र गृहे \msNc, 
वसते च गृहं \Ed}}% 

%Verse 12:46

{\devanagarifont विपुलस्तु नमस्कृत्वा कृत्वा चापि प्रदक्षिणम् {॥ १२:४६॥} \veg\dontdisplaylinenum }%
     \var{{\devanagarifont \numnoemph\vc\textbf{विपुलस्तु}\lem \mssALL, विपुलस्य \msNb}}% 
    \var{{\devanagarifont \numnoemph\vd\textbf{कृत्वा चापि}\lem \mssALL, \lk\lk \lk\lk\ \msNc, 
कृत्वा च वि॰ \Ed}}% 

{\devanagarifont ब्राह्मणमभिवाद्यैवं गतः शीघ्रं वनान्तरम् \thinspace{\dandab} \dontdisplaylinenum }%
     \var{{\devanagarifont \numemph\va\textbf{ब्राह्मण॰}\lem \mssALL, ब्राह्मणा॰ \msNb\oo 
\textbf{॰द्यैवं}\lem \eme, ॰द्येवं \msCa\msCc\msNa\msNb\Ed, ॰द्येनं \msCb, 
॰द्यवं \msNc}}% 
    \var{{\devanagarifont \numnoemph\vb\textbf{शीघ्रं}\lem \mssALL, श्रीघ्रं \msNb}}% 

%Verse 12:47

{\devanagarifont वने मूलफलाहारो विचरेत महीतले {॥ १२:४७॥} \veg\dontdisplaylinenum }%
     \var{{\devanagarifont \numnoemph\vc\textbf{॰फलाहारो}\lem \mssALL, 
॰फाहारो \msNcacorr}}% 

{\devanagarifont एकाकी विजने शून्ये चिन्तया च परिप्लुतः \thinspace{\dandab} \dontdisplaylinenum }%
     \var{{\devanagarifont \numemph\va\textbf{एकाकी}\lem \mssALL, 
ए\uncl{का}\lac\  \msCa}}% 
    \var{{\devanagarifont \numnoemph\vb\textbf{परि॰}\lem \mssALL, पलि॰ \msNc}}% 

%Verse 12:48

{\devanagarifont क्व गच्छामि क्व भोक्ष्यामि कुत्र वा किं करोम्यहम् {॥ १२:४८॥} \veg\dontdisplaylinenum }%
     \var{{\devanagarifont \numnoemph\vc\textbf{क्व गच्छामि}\lem \mssALL, क्ष गच्छामि \msNc\oo 
\textbf{क्व भोक्ष्यामि}\lem \msCa, क्व भोज्यामि \msCb\msNa\msNb, क्व भोक्ष्यानि \msCc, 
क्व भोक्षामि \msNc, किं भोक्ष्यामि \Ed\ \unmetr}}% 

{\devanagarifont न पथं विषयं वेद्मि ग्रामं वा नगराणि वा \thinspace{\dandab} \dontdisplaylinenum }%
     \var{{\devanagarifont \numemph\va\textbf{विषयं वेद्मि}\lem \msCa\msNa\msNb\Ed, विषमं वेद्मि \msCb\msCc, वियषं वे\uncl{श्मि} \msNc}}% 
    \var{{\devanagarifont \numnoemph\vb\textbf{वा}\lem \mssALL, च \msCb\msNa}}% 

%Verse 12:49

{\devanagarifont खेटखर्वटदेशं वा जानामीह न कंचन {॥ १२:४९॥} \veg\dontdisplaylinenum }%
     \var{{\devanagarifont \numnoemph\vc\textbf{खेट॰}\lem \mssALL, क्षेत्र॰ \msCc\oo 
\textbf{॰खर्वट॰}\lem \Ed, ॰कर्पट॰ \mssCaCbCc\msNa\msNb\msNc}}% 
    \var{{\devanagarifont \numnoemph\vd\textbf{कंचन}\lem \eme, कश्चन \mssCaCbCc\msNa\msNb\msNc\Ed}}% 

{\devanagarifont अमुं सुशैलं पश्यामि विपुलोदरकन्दरम् \thinspace{\dandab} \dontdisplaylinenum }%
     \var{{\devanagarifont \numemph\va\textbf{सुशैलं}\lem \mssALL, सुशेलं \msNc}}% 
    \var{{\devanagarifont \numnoemph\vb\textbf{विपुलो॰}\lem \mssALL, विलो॰ \msNb}}% 

%Verse 12:50

{\devanagarifont तमारुह्य निरीक्ष्यामि ग्रामं नगरपत्तनम् {॥ १२:५०॥} \veg\dontdisplaylinenum }%
     \var{{\devanagarifont \numnoemph\vc\textbf{निरीक्ष्यामि}\lem \mssALL, निरीक्षामि \msNc}}% 

{\devanagarifont एवमुक्त्वा तु विपुलः शनैः पर्वतमारुहत् \thinspace{\dandab} \dontdisplaylinenum }%
     \var{{\devanagarifont \numemph\va\textbf{एवमु॰}\lem \mssALL, एकं उ॰ \msCb}}% 
    \var{{\devanagarifont \numnoemph\vb\textbf{॰रुहत्}\lem \Ed, ॰रुहेत् \mssCaCbCc\msNa\msNb\msNc}}% 

%Verse 12:51

{\devanagarifont वृक्षच्छायां समालोक्य निषसाद श्रमान्वितः {॥ १२:५१॥} \veg\dontdisplaylinenum }%
     \var{{\devanagarifont \numnoemph\vc\textbf{॰च्छायां}\lem \mssALL, ॰च्छाया \msNc}}% 

{\devanagarifont एतस्मिन्नेव काले तु वृक्षशाखावतार्य च \thinspace{\dandab} \dontdisplaylinenum }%
     \var{{\devanagarifont \numemph\va\textbf{एतस्मिन्नेव}\lem \mssALL, एतस्मिंनैव \msCc, एतस्मिन्नैव \msNc\oo 
\textbf{काले तु}\lem \msCa\msCb\msNa\msNb, कालेन \msCc\Ed, कालेनु \msNc}}% 
    \var{{\devanagarifont \numnoemph\vb\textbf{वृक्ष॰}\lem \mssALL, वृक्षा॰ \msNa\msNcacorr}}% 

%Verse 12:52

{\devanagarifont अपूर्वं च सुरूपं च सुगन्धत्वं च शोभनम् {॥ १२:५२॥} \veg\dontdisplaylinenum }%
     \var{{\devanagarifont \numnoemph\vc\textbf{सुरूपं}\lem \mssALL, स्वरूपं \msCb\msNa}}% 

{\devanagarifont फलं गृह्य विचित्रं च हृदयानन्दनं शुभम् \thinspace{\dandab} \dontdisplaylinenum }%
 
%Verse 12:53

{\devanagarifont विपुलस्याग्रतः कृत्वा पुनर्वृक्षं समारुहत् {॥ १२:५३॥} \veg\dontdisplaylinenum }%
     \var{{\devanagarifont \numemph\vd \lem \mssALL, 
पुन वृक्ष समारुहम् \msCc, 
पुनर्वृक्ष समारुहं \msNb}}% 

{\devanagarifont विपुलश्चित्रवद्दृष्ट्वा विस्मयं परमं गतः \thinspace{\dandab} \dontdisplaylinenum }%
     \var{{\devanagarifont \numemph\va\textbf{॰त्रवद्दृष्ट्वा}\lem \mssALL, ॰त्रव दृष्ट्वा \msCc}}% 

%Verse 12:54

{\devanagarifont अहो वा स्वप्नभूतो ऽस्मि अहो वा तपसः फलम् {॥ १२:५४॥} \veg\dontdisplaylinenum }%
     \var{{\devanagarifont \numnoemph\vcd\textbf{॰भूतो ऽस्मि अहो}\lem \mssALL, ॰संभूतो \uncl{स्म्य}हो \msNa}}% 

{\devanagarifont न पश्यामि न जिघ्रामि न च स्वादं च वेद्म्यहम् \thinspace{\dandab} \dontdisplaylinenum }%
     \var{{\devanagarifont \numemph\va\textbf{जिघ्रामि}\lem \mssALL, च घ्रामि \msCb}}% 

%Verse 12:55

{\devanagarifont वार्त्तापि न च मे श्रोता प्रतिजानामि कंचन {॥ १२:५५॥} \veg\dontdisplaylinenum }%
     \var{{\devanagarifont \numnoemph\vc\textbf{श्रोता}\lem \mssALL, श्रोत्रा \msCa}}% 
    \var{{\devanagarifont \numnoemph\vd\textbf{कंचन}\lem \eme, कश्चन \mssCaCbCc\msNa\msNb\msNc\Ed}}% 

{\devanagarifont एवमुक्त्वा ह्यनेकानि फलं गृह्य मनोरमम् \thinspace{\dandab} \dontdisplaylinenum }%
     \var{{\devanagarifont \numemph\va\textbf{॰मुक्त्वा}\lem \mssALL, ॰मुक्ता \msCc}}% 
    \var{{\devanagarifont \numnoemph\vb\textbf{गृह्य}\lem \mssALL, गृह \msNc}}% 

%Verse 12:56

{\devanagarifont सुनिरीक्ष्य पुनर्जिघ्रन् पुनर्जिघ्रन्निरीक्ष्य च {॥ १२:५६॥} \veg\dontdisplaylinenum }%
     \var{{\devanagarifont \numnoemph\vc\textbf{॰निरीक्ष्य}\lem \mssALL, ॰निरीक्ष \msNc}}% 
    \var{{\devanagarifont \numnoemph\vcd\textbf{पुनर्जिघ्रन्पुनर्जिघ्रन्}\lem \msCa\msCb\msNa\Ed, 
मुन जिघ्रं पुन जिघ्रं \msCc, 
पुनर्जिघ्र पुनर्जिघ्रं \msNb, 
पुनर्जिघ्र पुनर्जिघ्र \msNc}}% 
    \var{{\devanagarifont \numnoemph\vd\textbf{निरीक्ष्य}\lem \mssALL, निरीक्ष \msNc}}% 

{\devanagarifont फलं चात्र निरूप्यन्तो देशं वाप्यवलोकयन् \thinspace{\dandab} \dontdisplaylinenum }%
     \var{{\devanagarifont \numemph\va\textbf{चात्र}\lem \mssALL, 
चा \msCaacorr, चा\uncl{त्र} \msCapcorr\oo 
\textbf{निरूप्यन्तो}\lem \Ed, निरूप्यान्ति \msCa, निरूप्यां चा \msCb, 
निरूप्यन्ति \msCc\msNa\msNb\msNc}}% 
    \var{{\devanagarifont \numnoemph\vb\textbf{॰लोकयन्}\lem \mssALL, ॰लोकयत् \msCb}}% 

%Verse 12:57

{\devanagarifont पाथेयरहितश्चास्मि देवदत्तं फलं मम {॥ १२:५७॥} \veg\dontdisplaylinenum }%
     \var{{\devanagarifont \numnoemph\vc\textbf{पाथेय॰}\lem \mssALL, पथेय॰ \msNb\oo 
\textbf{॰रहितश्चा॰}\lem \mssALL, ॰रहिते चा॰ \msCc}}% 
    \var{{\devanagarifont \numnoemph\vd\textbf{॰दत्तं}\lem \msCa\msNa\msNc, ॰दत्त॰ \msCb\msCc\msNb\Ed\oo 
\textbf{फलं}\lem \mssALL, \om\ \msNc}}% 

{\devanagarifont तत्फलं प्रतिगृह्यैव नगरं प्रविशाम्यहम् \thinspace{\dandab} \dontdisplaylinenum }%
     \var{{\devanagarifont \numemph\va\textbf{॰गृह्यैव}\lem \msCb\msNb\Ed, ॰गृह्येव \msCa\msNc, गृहे च \msCc, ॰गृह्यैवं \msNa}}% 

%Verse 12:58

{\devanagarifont प्रार्थयित्वा तु यत्किंचिज्जीवनार्थं चराम्यहम् {॥ १२:५८॥} \veg\dontdisplaylinenum }%
     \var{{\devanagarifont \numnoemph\vc\textbf{तु}\lem \mssALL, च \Ed}}% 
    \var{{\devanagarifont \numnoemph\vcd\textbf{यत्किंचिज्जी॰}\lem \mssALL, 
यत्किंजि जी॰ \msCc}}% 

{\devanagarifont ततः शैलमतिक्रम्य नगरं प्रविवेश ह \thinspace{\dandab} \dontdisplaylinenum }%
 
%Verse 12:59

{\devanagarifont पथि कश्चिज्जनः पृष्ठः किंनाम नगरं त्विदम् {॥ १२:५९॥} \veg\dontdisplaylinenum }%
     \var{{\devanagarifont \numemph\vd\textbf{नगरं त्विदम्}\lem \msCa\msNa\msNc\Ed, 
नगर त्विदम् \msCb\msCc, नगरं त्विह \msNb}}% 

{\devanagarifont स होवाच पथीकेन किमपूर्वमिहागतः \thinspace{\dandab} \dontdisplaylinenum }%
     \var{{\devanagarifont \numemph\va\textbf{स हो॰}\lem \mssALL, अहो॰ \msCb\msNb\oo 
\textbf{पथीकेन}\lem \mssALL, पथीको न \msNc}}% 
    \var{{\devanagarifont \numnoemph\vb\textbf{॰गतः}\lem \mssALL, ॰तवः \msNb}}% 

%Verse 12:60

{\devanagarifont दक्षिणापथदेशो ऽयं नरवीरपुरं त्वदः {॥ १२:६०॥} \veg\dontdisplaylinenum }%
     \var{{\devanagarifont \numnoemph\vc\textbf{॰पथ॰}\lem \mssALL, ॰पथे \msCb}}% 
    \var{{\devanagarifont \numnoemph\vd\textbf{॰पुरं त्वदः}\lem \msCb, 
॰पुरं त्वयः \msCa, 
॰पुरं त्वयं \msCc\msNa\msNb, 
पुरन्दरः \msNc, ॰पुरं स्वयम् \Ed}}% 

{\devanagarifont राजा सिंहजटो नाम राज्ञी तस्य च केकयी \thinspace{\dandab} \dontdisplaylinenum }%
     \var{{\devanagarifont \numemph\va\textbf{राजा}\lem \mssALL, राजा हि \msNc, राज \Ed\oo 
\textbf{॰जटो}\lem \mssALL, ॰यतो \Ed}}% 
    \var{{\devanagarifont \numnoemph\vb\textbf{केकयी}\lem \mssALL, कैकयी \msCa}}% 

%Verse 12:61

{\devanagarifont अतिवृद्धो जराग्रस्तः केकयी च तथैव च {॥ १२:६१॥} \veg\dontdisplaylinenum }%
     \var{{\devanagarifont \numnoemph\vd\textbf{केकयी}\lem \mssALL, कैकयी \msCa\oo 
\textbf{तथैव च}\lem \mssALL, तथैव र \msNc}}% 

{\devanagarifont दाता सर्वकलाज्ञश्च युद्धे वीर्यबलान्वितः \thinspace{\dandab} \dontdisplaylinenum }%
     \var{{\devanagarifont \numemph\va\textbf{दाता}\lem \mssALL, \lac  ता \msCa\oo 
\textbf{॰कला॰}\lem \Ed, ॰कल॰ \mssCaCbCc\msNa\msNb\msNc}}% 
    \var{{\devanagarifont \numnoemph\vb\textbf{युद्धे}\lem \mssALL, युद्धो \msNb}}% 

%Verse 12:62

{\devanagarifont ब्रह्मण्यो वत्सलो लोके सर्वशास्त्रविशारदः {॥ १२:६२॥} \veg\dontdisplaylinenum }%
 
{\devanagarifont विपुल उवाच {\dandab}\dontdisplaylinenum  }%
 
{\devanagarifont अत्र श्रेष्ठिमुपास्यामि नाम वा तस्य किं वद \thinspace{\danda} \dontdisplaylinenum }%
     \var{{\devanagarifont \numemph\va\textbf{॰पास्यामि}\lem \mssALL, ॰पस्यामि \msCc}}% 
    \var{{\devanagarifont \numnoemph\vb\textbf{नाम}\lem \msCa\msCb\msNc, नामं \msCc\msNa\msNb\Ed\oo 
\textbf{वद}\lem \mssALL, वदः \msCb}}% 

%Verse 12:63

{\devanagarifont कतमो देश तद्वासः कथयस्व न संशयः {॥ १२:६३॥} \veg\dontdisplaylinenum }%
     \var{{\devanagarifont \numnoemph\vc\textbf{देश त॰}\lem \msCc\msNb, देशस्त॰ \msCa\msCb\msNa\msNc\Ed\ \unmetr}}% 
    \var{{\devanagarifont \numnoemph\vd\textbf{कथयस्व}\lem \mssALL, कथयस्य \msCb}}% 

{\devanagarifont विपुलेनैवमुक्तस्तु पथिकोवाच तं पुनः \thinspace{\dandab} \dontdisplaylinenum }%
     \var{{\devanagarifont \numemph\va\textbf{विपुलेनैव॰}\lem \mssALL, विपुलेनेव॰ \msNc}}% 

%Verse 12:64

{\devanagarifont मम भीमबलो नाम श्रेष्ठिकस्य गृहागतः {॥ १२:६४॥} \veg\dontdisplaylinenum }%
     \var{{\devanagarifont \numnoemph\vc \lem \mssALL, 
मम भी\lac  बलो नाम \msCa, \om\ \Ed}}% 
    \var{{\devanagarifont \numnoemph\vd \lem \mssALL, 
श्रेष्ठिकस्य गृहागतः\thinspace{\devanagarifont ॥} पथिको ऽहमिदानिञ्च\thinspace{\devanagarifont ।} 
को भवान् तस्य विषये किं वा ज्ञातुं चिकीर्षसि\thinspace{\devanagarifont ॥} \Ed}}% 

{\devanagarifont श्रेष्ठिकः पुण्डको नाम ख्यातः श्रेष्ठिक उच्यते \thinspace{\dandab} \dontdisplaylinenum }%
 
%Verse 12:65

{\devanagarifont कौतुकं तव यद्यस्ति तदागच्छ मया सह {॥ १२:६५॥} \veg\dontdisplaylinenum }%
 
{\devanagarifont एवमस्त्विति तेनोक्तो विपुलेन महात्मना \thinspace{\dandab} \dontdisplaylinenum }%
     \var{{\devanagarifont \numemph\va\textbf{॰स्त्विति}\lem \mssALL, ॰स्तिति \msCb\msCc\oo 
\textbf{तेनोक्तो}\lem \mssALL, तोनोक्तो \msNc, तेनोक्तौ \Ed}}% 
    \var{{\devanagarifont \numnoemph\vb\textbf{॰त्मना}\lem \mssALL, ॰त्मनाः \msNc}}% 

%Verse 12:66

{\devanagarifont तेनैव सह निर्यातः श्रेष्ठिकस्य गृहं प्रति {॥ १२:६६॥} \veg\dontdisplaylinenum }%
     \var{{\devanagarifont \numnoemph\vc\textbf{तेनैव}\lem \mssALL, तेनेव \msNc}}% 
    \var{{\devanagarifont \numnoemph\vd\textbf{प्रति}\lem \mssALL, प्रतिः \msCc\Ed}}% 

{\devanagarifont श्रेष्ठिकः स्वगृहासीनो दृष्टः स विपुलेन तु \thinspace{\dandab} \dontdisplaylinenum }%
     \var{{\devanagarifont \numemph\va\textbf{श्रेष्ठिकः}\lem \mssALL, 
श्रेष्ठितः \msCa, श्रेष्ठिक \msNa}}% 
    \var{{\devanagarifont \numnoemph\vb\textbf{दृष्टः स}\lem \msCb\msNa\msNc\Ed, 
\uncl{दृ}\lac\  \msCa, दृष्ट स \msCc, दृष्टस्य \msNb}}% 

%Verse 12:67

{\devanagarifont तस्यान्तिकमुपागम्य तत्फलं स निवेदितः {॥ १२:६७॥} \veg\dontdisplaylinenum }%
     \var{{\devanagarifont \numnoemph\vc\textbf{॰गम्य}\lem \mssALL, ॰गत्य \msNc}}% 
    \var{{\devanagarifont \numnoemph\vd\textbf{स निवेदितः}\lem \mssALL, 
सन्निवेदितः \msNa, संनिवेदितः \msNc}}% 

{\devanagarifont अहो फलमिदं श्रेष्ठमहो फलमिहानितम् \thinspace{\dandab} \dontdisplaylinenum }%
     \var{{\devanagarifont \numemph\vab\textbf{श्रेष्ठमहो}\lem \mssALL, श्रेष्ठ अहो \msCc}}% 

%Verse 12:68

{\devanagarifont अहो रूपमहो गन्धमहो फलं सुशोभनम् {॥ १२:६८॥} \veg\dontdisplaylinenum }%
     \var{{\devanagarifont \numnoemph\vcd\textbf{गन्धमहो फलं}\lem \corr, 
गन्धमहो फल \msCa\msCbpcorr\msCc\msNa\Ed, 
गन्धमहो गन्धमहो फल \msCbacorr, 
गन्ध अहो फल \msNb, गन्धो फलं अहो \msNc}}% 

{\devanagarifont तत्फलं न महीजातं न मेरौ न च मन्दरे \thinspace{\dandab} \dontdisplaylinenum }%
     \var{{\devanagarifont \numemph\va\textbf{तत्फ॰}\lem \mssALL, यत्फ॰ \Ed}}% 
    \var{{\devanagarifont \numnoemph\vb\textbf{मेरौ}\lem \msCa\msCb\msNa\msNcpcorr\Ed, मेरो \msCc\msNb\msNcacorr\oo 
\textbf{मन्दरे}\lem \conj, कन्दरे \mssCaCbCc\msNa\msNb\msNc\Ed}}% 

%Verse 12:69

{\devanagarifont देवलोकिक सुव्यक्तं न मर्त्यमुपजायते {॥ १२:६९॥} \veg\dontdisplaylinenum }%
     \var{{\devanagarifont \numnoemph\vc\textbf{देवलोकिक}\lem \mssALL, 
देवलोकि \msNbacorr}}% 
    \var{{\devanagarifont \numnoemph\vd\textbf{मर्त्यमुपजायते}\lem \msCc\msNa\msNb\msNc, 
मर्त्य\uncl{मुपजा}\lac\  \msCa, 
मर्त्य सुपजायते \msCb, मह्यामुपजायते \Ed}}% 

{\devanagarifont अहो ऽस्मि स फलं भोक्ता राजार्हं च न संशयः \thinspace{\dandab} \dontdisplaylinenum }%
     \var{{\devanagarifont \numemph\va\textbf{अहो}\lem \mssALL, \lac  हो \msCa, अद्यो \Ed\oo 
\textbf{स फलं}\lem \mssALL, 
\uncl{स}फलम् \msCa, तत्फलं \Ed\oo 
\textbf{भोक्ता}\lem \mssALL, भोक्तं \msNc}}% 
    \var{{\devanagarifont \numnoemph\vb\textbf{राजार्हं च}\lem \msCc\msNb, राजार्हश्च \msCa\msCb\msNc\Ed, 
राजार्ह\uncl{श्च} \msNa}}% 

%Verse 12:70

{\devanagarifont ढौकयित्वा फलं दिव्यं राजानं तोषयाम्यहम् {॥ १२:७०॥} \veg\dontdisplaylinenum }%
     \var{{\devanagarifont \numnoemph\vc\textbf{ढौकयित्वा}\lem \mssALL, ढोकयित्वा \msNb}}% 

{\devanagarifont ततस्त्वरित गत्वैव फलं गृह्य मनोहरम् \thinspace{\dandab} \dontdisplaylinenum }%
     \var{{\devanagarifont \numemph\va\textbf{त्वरित}\lem \msNa\msNc\Ed, त्वरितं \mssCaCbCc\msNb\ \unmetr}}% 
    \var{{\devanagarifont \numnoemph\vb\textbf{गृह्य}\lem \mssALL, गृह \msCb\oo 
\textbf{॰हरम्}\lem \mssALL, ॰रमम् \msNb\Ed}}% 

%Verse 12:71

{\devanagarifont आदरेणोपसृत्यैव राजानं स फलं ददौ {॥ १२:७१॥} \veg\dontdisplaylinenum }%
     \var{{\devanagarifont \numnoemph\vc\textbf{॰सृत्यैव}\lem \msCa\msCb\Ed, ॰सृत्येव \msCc\msNb\msNc, ॰संगत्य \msNa}}% 
    \var{{\devanagarifont \numnoemph\vd\textbf{स फलं}\lem \mssALL, तत्फलं \Ed}}% 

{\devanagarifont राजा च स फलं दृष्ट्वा विस्मयं परमं गतः \thinspace{\dandab} \dontdisplaylinenum }%
     \var{{\devanagarifont \numemph\va\textbf{स फलं}\lem \mssALL, तत्फलं \Ed}}% 
    \var{{\devanagarifont \numnoemph\vb\textbf{विस्मयं}\lem \mssALL, विस्मय \msNb}}% 

%Verse 12:72

{\devanagarifont कुतः श्रेष्ठि त्वया नीतं फलं पूर्वं मनोहरम् {॥ १२:७२॥} \veg\dontdisplaylinenum }%
     \var{{\devanagarifont \numnoemph\vc\textbf{श्रेष्ठि}\lem \mssALL, श्रेष्ठ \Ed}}% 
    \var{{\devanagarifont \numnoemph\vd \lem \corr, फल\lac  हरम् \msCa, 
फल\uncl{म्य}र्वमनोहरम् \msCb, 
फलं पूर्व मनोहरम् \msCc\msNa\msNb\msNc, 
फलं सर्वमनोहरम् \Ed}}% 

{\devanagarifont स्वादुमूलं फलं कन्दं दृष्टं पूर्वं न तादृशम् \thinspace{\dandab} \dontdisplaylinenum }%
     \var{{\devanagarifont \numemph\va\textbf{॰मूलं फलं}\lem \msNc, ॰मूलफल॰ \mssCaCbCc\msNa\msNb\Ed}}% 
    \var{{\devanagarifont \numnoemph\vab\textbf{कन्दं दृष्टं पू॰}\lem \eme, ॰कन्दं दृष्ट्वा पू॰ \msCa\msNa\msNb, 
॰स्कन्द दृष्ट्वा पू॰ \msCb, 
॰स्कन्द दृष्ट पू॰ \msCc, कन्द दृष्ट\uncl{न्पू}॰ \msNc, 
॰स्कन्द दृष्टा पू॰ \Ed}}% 
    \var{{\devanagarifont \numnoemph\vb\textbf{तादृशम्}\lem \mssALL, 
तादृ\uncl{शं} \msCc, यादृशम् \Ed}}% 

%Verse 12:73

{\devanagarifont रूपगन्धगुणोपेतं हृदयानन्दकारकम् {॥ १२:७३॥} \veg\dontdisplaylinenum }%
     \var{{\devanagarifont \numnoemph\vd\textbf{॰कारकम्}\lem \mssALL, ॰कारकः \msNa}}% 

{\devanagarifont सद्य एवोपयुञ्जामि त्वया दत्तमिदं फलम् \thinspace{\dandab} \dontdisplaylinenum }%
     \var{{\devanagarifont \numemph\va \lem \mssALL, 
सत्य एव प्रभुञ्जामि \Ed}}% 

%Verse 12:74

{\devanagarifont कीदृशं स्वाद विज्ञानमिच्छामि कुरु माचिरम् {॥ १२:७४॥} \veg\dontdisplaylinenum }%
     \var{{\devanagarifont \numnoemph\vc\textbf{स्वाद विज्ञानम्}\lem \mssALL, स्वादु विज्ञातुम् \Ed}}% 

{\devanagarifont ततः स भक्षयामास फलं चामृतसंनिभम् \thinspace{\dandab} \dontdisplaylinenum }%
     \var{{\devanagarifont \numemph\va\textbf{ततः}\lem \mssALL, तत \msCb}}% 

%Verse 12:75

{\devanagarifont अमृतोपमसुस्वादं सर्वं च बुभुजे नृपः {॥ १२:७५॥} \veg\dontdisplaylinenum }%
     \var{{\devanagarifont \numnoemph\vcd\textbf{स्वादं सर्वं च}\lem \mssALL, 
स्वा\lac\  \msCa}}% 

{\devanagarifont सद्यः षोडशवर्षस्य यौवनं समपद्यत \thinspace{\dandab} \dontdisplaylinenum }%
     \var{{\devanagarifont \numemph\va\textbf{सद्यः}\lem \corr, \mssCaCbCc\msNa\msNb\msNc\Ed}}% 
    \var{{\devanagarifont \numnoemph\vb\textbf{॰पद्यत}\lem \msCa\msCb, ॰पद्यते \msCc\msNa\msNb\Ed, ॰द्यत \msNc}}% 

%Verse 12:76

{\devanagarifont न वलीपलितं सद्यो न जरा न च दुर्बलः {॥ १२:७६॥} \veg\dontdisplaylinenum }%
     \var{{\devanagarifont \numnoemph\vc\textbf{वली॰}\lem \mssALL, वलि॰ \Ed}}% 

{\devanagarifont केशदन्तनखस्निग्धो दृढदन्तो दृढेन्द्रियः \thinspace{\dandab} \dontdisplaylinenum }%
     \var{{\devanagarifont \numemph\vb\textbf{॰दन्तो}\lem \mssALL, ॰देहो \Ed\oo 
\textbf{दृढेन्द्रियः}\lem \mssALL, दृढेन्द्रिः \msNb}}% 

%Verse 12:77

{\devanagarifont तेजश्चक्षुर्बलप्राणान्सद्यः सर्वानवाप्तवान् {॥ १२:७७॥} \veg\dontdisplaylinenum }%
     \var{{\devanagarifont \numnoemph\vc\textbf{॰चक्षुर्बलप्राणा॰}\lem \msCa\msCb\msNa\msNb, ॰चक्षुवलप्राणा॰ \msCc, 
॰चक्षुर्बलं प्राणा॰ \msNc, ॰चक्षुवलप्राण॰ \Ed}}% 
    \var{{\devanagarifont \numnoemph\vd\textbf{॰न्सद्यः}\lem \corr, ॰न्सद्य \mssCaCbCc\msNa\msNb\msNc\Ed\oo 
\textbf{सर्वान॰}\lem \mssALL, सर्व्वान्न॰ \msCc\oo 
\textbf{॰प्तवान्}\lem \mssALL, ॰प्तुयात् \msNa}}% 

{\devanagarifont मन्त्री पुरोहितो ऽमात्यः सर्वे भृत्यजनास्तथा \thinspace{\dandab} \dontdisplaylinenum }%
     \var{{\devanagarifont \numemph\va\textbf{पुरोहितो ऽमात्यः}\lem \msCa\msCc\msNb, 
पुरोहितो मात्य \msCb\msNa\msNc, पुरोहितामात्य \Ed}}% 
    \var{{\devanagarifont \numnoemph\vb \lem \mssALL, 
जनास्तथास्तथा \msCb}}% 

%Verse 12:78

{\devanagarifont पौरस्त्री बालवृद्धाश्च सर्वे ते विस्मयं गताः {॥ १२:७८॥} \veg\dontdisplaylinenum }%
     \var{{\devanagarifont \numnoemph\vc\textbf{॰स्त्री}\lem \mssALL, ॰स्त्रि \Ed}}% 
    \var{{\devanagarifont \numnoemph\vd\textbf{सर्वे}\lem \mssALL, \lac\  \msCa\oo 
\textbf{गताः}\lem \mssALL, गतः \msCc}}% 

{\devanagarifont राजा सिंहजटो नाम तुष्टिमेव परां गतः \thinspace{\dandab} \dontdisplaylinenum }%
     \var{{\devanagarifont \numemph\vb\textbf{परां}\lem \mssALL, परं \msNb}}% 

%Verse 12:79

{\devanagarifont प्रहर्षमतुलं चैव प्राप्तवान्स नरेश्वरः {॥ १२:७९॥} \veg\dontdisplaylinenum }%
 
{\devanagarifont उवाच राजा तं श्रेष्ठिं स्वार्थतत्परनिर्दयः \thinspace{\dandab} \dontdisplaylinenum }%
     \var{{\devanagarifont \numemph\va\textbf{राजा तं}\lem \mssALL, राजनं \msNb\oo 
\textbf{श्रेष्ठिं}\lem \mssALL, श्रेष्ठं \Ed}}% 
    \var{{\devanagarifont \numnoemph\vb\textbf{॰दयः}\lem \mssALL, ॰दय \Ed}}% 

%Verse 12:80

{\devanagarifont कुरु भीमबलस्त्वेवं फलमानय अद्य वै {॥ १२:८०॥} \veg\dontdisplaylinenum }%
     \var{{\devanagarifont \numnoemph\vc\textbf{कुरु}\lem \mssALL, शृणु \Ed\oo 
\textbf{भीमबलस्त्वेवं}\lem \msCb\msCc\msNa, भीमवस्त्वेवं \msCa\Ed, 
भीमबलस्त्वेव \msNb, 
भीमबल\uncl{म्त्वे}वं \msNc}}% 

{\devanagarifont पुनर्मे यौवनप्राप्तिस्त्वत्प्रसादान्नरोत्तम \thinspace{\dandab} \dontdisplaylinenum }%
     \var{{\devanagarifont \numemph\vb\textbf{॰त्तम}\lem \mssALL, ॰त्तमः \Ed}}% 

%Verse 12:81

{\devanagarifont केकयीं दुर्बलां वृद्धां पुनः प्रापय यौवनम् {॥ १२:८१॥} \veg\dontdisplaylinenum }%
     \var{{\devanagarifont \numnoemph\vc\textbf{केकयीं दुर्बलां}\lem \msNa, कैकयीन्दुर्बलान् \msCa, केकयीं \msCb, 
केकयी दुर्बला \msCc\msNb\Ed, कैकयी दुर्बलां \msNc}}% 
    \var{{\devanagarifont \numnoemph\vcd\textbf{वृद्धां पुनः}\lem \msCb\msNa\msNb\msNc, वृ\uncl{द्धा}\lac\  \msCa, 
वृद्धा पुनः \msCc\Ed}}% 
    \var{{\devanagarifont \numnoemph\vd\textbf{प्रापय}\lem \mssALL, प्राप \msCc}}% 

{\devanagarifont स राज्ञा एवमुक्तस्तु श्रेष्ठी भीमबलस्तथा \thinspace{\dandab} \dontdisplaylinenum }%
     \var{{\devanagarifont \numemph\vb\textbf{श्रेष्ठी}\lem \msCc\Ed, श्रेष्ठि \msCa\msCb\msNa\msNc, श्रिष्ठि \msNb\oo 
\textbf{॰बलस्तथा}\lem \mssALL, ॰बलस्तदा \msNb\msNc}}% 

%Verse 12:82

{\devanagarifont प्रत्युवाच ह राजानं प्राञ्जलिः प्रणतः स्थितः {॥ १२:८२॥} \veg\dontdisplaylinenum }%
     \var{{\devanagarifont \numnoemph\vc\textbf{॰वाच ह}\lem \mssALL, ॰वाचाह \Ed\oo 
\textbf{राजानं}\lem \mssALL, राजान \msNa}}% 

{\devanagarifont न वनेन वने राजन्न वाणिज्यकृषेण वा \thinspace{\dandab} \dontdisplaylinenum }%
     \var{{\devanagarifont \numemph\va\textbf{न वनेन}\lem \mssALL, न फलेदं \Ed}}% 
    \var{{\devanagarifont \numnoemph\vab\textbf{राजन्न}\lem \mssALL, राजान्न \msCb\msNb}}% 

%Verse 12:83

{\devanagarifont केनापि कुलपुत्रेण तव दर्शनकांक्षया {॥ १२:८३॥} \veg\dontdisplaylinenum }%
     \var{{\devanagarifont \numnoemph\vc\textbf{कुल॰}\lem \mssALL, कु॰ \msNc}}% 

{\devanagarifont दत्तो ऽस्मि तेन राजेन्द्र मया दत्तो ऽसि भूपते \thinspace{\dandab} \dontdisplaylinenum }%
     \var{{\devanagarifont \numemph\va\textbf{ऽस्मि तेन}\lem \mssALL, 
स्मिन्तेन \msNb, ऽस्मि तव \Ed}}% 
    \var{{\devanagarifont \numnoemph\vb\textbf{दत्तो ऽसि}\lem \msCa\msCb\msNb\msNc, 
दत्तासि \msCc, दत्तो स्मि \msNa, प्राप्तोषि \Ed}}% 

%Verse 12:84

{\devanagarifont न ते शक्नोम्यहं राजन्वक्तुं वैदेशिनं नरम् {॥ १२:८४॥} \veg\dontdisplaylinenum }%
     \var{{\devanagarifont \numnoemph\vc\textbf{ते}\lem \mssALL, च \Ed}}% 
    \var{{\devanagarifont \numnoemph\vcd\textbf{राजन्वक्तुं}\lem \mssALL, 
रा\lac  क्तुम् \msCa, राजान्वक्तुम् \msCc}}% 
    \var{{\devanagarifont \numnoemph\vd\textbf{वैदेशिनं नरम्}\lem \msCb\msCc\msNa\msNc, 
\uncl{वै}देशिनन्नरम् \msCa, 
वैदेशिनं नरः \msNb, च देहि तन्नरः \Ed}}% 

{\devanagarifont श्रुत्वा भीमबलवाक्यं प्रत्युवाच ततः पुनः \thinspace{\dandab} \dontdisplaylinenum }%
     \var{{\devanagarifont \numemph\va\textbf{॰बल॰}\lem \msCa\msCb, ॰बलं \msCc\msNa\msNb\msNc\Ed}}% 

%Verse 12:85

{\devanagarifont अमात्यकुलपुत्रस्त्वं ब्रूहि मद्वचनं पुनः {॥ १२:८५॥} \veg\dontdisplaylinenum }%
     \var{{\devanagarifont \numnoemph\vc\textbf{अमात्य॰}\lem \mssALL, अमत्य॰ \msNb\oo 
\textbf{॰पुत्रस्त्वं}\lem \mssALL, ॰पुत्रं त्वं \msNc}}% 

{\devanagarifont यदि नास्ति किं मे दत्तं मया वा मार्गितो भवान् \thinspace{\dandab} \dontdisplaylinenum }%
     \var{{\devanagarifont \numemph\va\textbf{किं मे दत्तं}\lem \msNc, किमे दत्तं \mssCaCbCc\msNa\msNb, 
किमेतत्तं \Ed}}% 
    \var{{\devanagarifont \numnoemph\vb\textbf{मार्गितो}\lem \mssALL, प्रार्थितो \Ed\oo 
\textbf{भवान्}\lem \mssALL, भगवन् \msNc}}% 

%Verse 12:86

{\devanagarifont यत्र ह्येको बहवो ऽत्र जायन्ते नात्र संशयः {॥ १२:८६॥} \veg\dontdisplaylinenum }%
     \var{{\devanagarifont \numnoemph\vc \lem \msCa\msNa\msNb\msNc, 
यत्रैको बहवो ऽत्रैव \msCb, 
यतश्चैक बहून्तत्र \msCc, 
यत्रश्चैको बहून्तत्र \Ed}}% 
    \var{{\devanagarifont \numnoemph\vd\textbf{जायन्ते}\lem \mssALL, जायते \msCc}}% 

{\devanagarifont आगमोपायमार्गं च तेनैव स तु गम्यताम् \thinspace{\dandab} \dontdisplaylinenum }%
     \var{{\devanagarifont \numemph\vb\textbf{तेनैव}\lem \mssALL, तैनैव \msCc}}% 

%Verse 12:87

{\devanagarifont अवश्यं तेन गन्तव्यं तेन मार्गेण मार्गय {॥ १२:८७॥} \veg\dontdisplaylinenum }%
     \var{{\devanagarifont \numnoemph\vc\textbf{अवश्यं तेन}\lem \mssALL, 
अव\uncl{स्य}\lac\  न \msCa\oo 
\textbf{गन्तव्यं}\lem \mssALL, 
\uncl{बुद्ध}व्यं \msCb}}% 
    \var{{\devanagarifont \numnoemph\vd\textbf{मार्गय}\lem \mssALL, मार्गयः \Ed}}% 
    \lacuna{\devanagarifontsmall \vd {\englishfont \msCc\ breaks off here missing one folio (f. 291);
                 it resumes at 12.113d in f. 292.} }%
  
{\devanagarifont अदत्त्वा फलमन्यच्च शिरश्छेद्यामि दुर्मते \thinspace{\dandab} \dontdisplaylinenum }%
     \var{{\devanagarifont \numemph\va\textbf{अदत्त्वा}\lem \mssALL, 
अदत्ता \msNb, अदत्वाफत्वा \msNcacorr}}% 

%Verse 12:88

{\devanagarifont छेद्यश्चण्डविचण्डाभ्यां रक्ष भीमबलाधमः {॥ १२:८८॥} \veg\dontdisplaylinenum }%
     \var{{\devanagarifont \numnoemph\vc\textbf{छेद्यश्च॰}\lem \msNa, छेद्ये च॰ \msCa\msNb, 
छेदे च॰ \msCb\msNc, छेद्य च॰ \Ed}}% 
    \var{{\devanagarifont \numnoemph\vd\textbf{॰धमः}\lem \mssALL, ॰धम \msCb}}% 

{\devanagarifont ततो भीमबलः क्रुद्धः खड्गं गृह्य शशिप्रभम् \thinspace{\dandab} \dontdisplaylinenum }%
     \var{{\devanagarifont \numemph\va\textbf{॰बलः}\lem \mssALL, ॰बल \msNa}}% 
    \var{{\devanagarifont \numnoemph\vb\textbf{शशिप्रभम्}\lem \mssALL, शशी प्रदम् \Ed}}% 

%Verse 12:89

{\devanagarifont अलङ्घ्य वचनं राज्ञः कुलपुत्र व्रज त्वरम् {॥ १२:८९॥} \veg\dontdisplaylinenum }%
     \var{{\devanagarifont \numnoemph\vc\textbf{अलङ्घ्य}\lem \mssALL, \lk लङ्घ्य \msNb, उवाच \Ed\oo 
\textbf{राज्ञः}\lem \mssALL, राजा \msNb}}% 
    \var{{\devanagarifont \numnoemph\vd \lem \msNb\Ed, कुलपुत्रं व्रजत्यरम् \msCa\msCb, 
कुलपुत्र व्रजन्परं \msNa, 
कुलपुत्रं व्रजन्परं \msNc}}% 

{\devanagarifont मा रुष कुलपुत्र त्वं मया वध्यो भविष्यसि \thinspace{\dandab} \dontdisplaylinenum }%
     \var{{\devanagarifont \numemph\va\textbf{॰पुत्र त्वं}\lem \mssALL, ॰पुत्रस्त्वं \Ed}}% 
    \var{{\devanagarifont \numnoemph\vb\textbf{वध्यो}\lem \mssALL, वद्ध्यौ \msNb\oo 
\textbf{भविष्यसि}\lem \mssALL, भविष्यति \msNb}}% 

%Verse 12:90

{\devanagarifont सद्यो ऽस्ति फलमन्यद्वा देहि राजानमद्य वै {॥ १२:९०॥} \veg\dontdisplaylinenum }%
     \var{{\devanagarifont \numnoemph\vc\textbf{सद्यो ऽस्ति}\lem \mssALL, 
\lac  द्योस्ति \msCa, यद्यस्ति \Ed}}% 

{\devanagarifont यत्र प्राप्तं फलं दिव्यं तत्र वादेशय त्वरम् \thinspace{\dandab} \dontdisplaylinenum }%
     \var{{\devanagarifont \numemph\va\textbf{प्राप्तं}\lem \mssALL, प्राप्त॰ \msCb, प्राप्ति \Ed}}% 
    \var{{\devanagarifont \numnoemph\vb\textbf{॰देशय}\lem \mssALL, ॰देशयत् \msNb, ॰देशयन् \Ed\oo 
\textbf{त्वरम्}\lem \conj, तव \msCa\msCb\msNa\msNb\msNc\Ed}}% 

%Verse 12:91

{\devanagarifont तत्फलेन विना भद्र दुर्लभं तव जीवितम् {॥ १२:९१॥} \veg\dontdisplaylinenum }%
 
{\devanagarifont विपुल उवाच {\dandab}\dontdisplaylinenum  }%
 
{\devanagarifont जीविताशामहं प्राप्तो वैदेशी भवनं तव \thinspace{\danda} \dontdisplaylinenum }%
     \var{{\devanagarifont \numemph\vb\textbf{वैदेशी}\lem \eme, वैदेशि \mssCaCbCc\msNa\msNb\msNc\Ed}}% 

%Verse 12:92

{\devanagarifont कृतकर्ता कथं वध्यः प्राप्नुयामहमद्य वै {॥ १२:९२॥} \veg\dontdisplaylinenum }%
     \var{{\devanagarifont \numnoemph\vd\textbf{प्राप्नुयाम॰}\lem \mssALL, प्राप्तुयाम॰ \msNa, 
प्राप्तो ऽयम॰ \Ed\oo 
\textbf{॰हमद्य वै}\lem \mssALL, ॰हपद्य वै \msNb, 
॰हमद्य वैः \msNc}}% 

{\devanagarifont फलं वा न पुनस्त्वन्यद्दातुं शक्यं न केनचित् \thinspace{\dandab} \dontdisplaylinenum }%
     \var{{\devanagarifont \numemph\va\textbf{वा न}\lem \mssALL, वा \msCb}}% 
    \var{{\devanagarifont \numnoemph\vab\textbf{॰न्यद्दातुं}\lem \mssALL, ॰न्य दातुं \msNc}}% 
    \var{{\devanagarifont \numnoemph\vb\textbf{शक्यं न केनचित्}\lem \mssALL, 
शक्य\lac  नचित् \msCa, शक्यं न तेनचिद् \msNc}}% 

%Verse 12:93

{\devanagarifont सह्यपर्वतशैलाग्रे आसीनः श्रान्तमानसः {॥ १२:९३॥} \veg\dontdisplaylinenum }%
     \var{{\devanagarifont \numnoemph\vd\textbf{आसीनः}\lem \mssALL, आशीतः \msCb\oo 
\textbf{श्रान्त॰}\lem \mssALL, श्रोत्त॰ \msCb, सान्त॰ \msNb}}% 

{\devanagarifont वानरस्तत्फलं गृह्य मम दत्त्वा पुनर्गतः \thinspace{\dandab} \dontdisplaylinenum }%
     \var{{\devanagarifont \numemph\vb\textbf{मम}\lem \mssALL, मह्यं \Ed}}% 

%Verse 12:94

{\devanagarifont मया दत्तमिदं तुभ्यं त्वयापि च नराधिपे {॥ १२:९४॥} \veg\dontdisplaylinenum }%
     \var{{\devanagarifont \numnoemph\vc\textbf{तुभ्यं}\lem \mssALL, तुभ्य \msNb}}% 
    \var{{\devanagarifont \numnoemph\vd\textbf{॰धिपे}\lem \mssALL, ॰धिप \msNb}}% 

{\devanagarifont तत्र गच्छाव भो श्रेष्ठि दृश्यते यदि वानरः \thinspace{\dandab} \dontdisplaylinenum }%
 
%Verse 12:95

{\devanagarifont त्वया मया च गत्वैव याचावः प्लवगाधिपम् {॥ १२:९५॥} \veg\dontdisplaylinenum }%
     \var{{\devanagarifont \numemph\vd\textbf{च गत्वैव}\lem \mssALL, 
\uncl{त}गवत्वैव \msNc\oo 
\lem  \msCb, 
यो वासः प्लवगाधिपः \msCa\msNa\msNb\msNc\Ed}}% 

{\devanagarifont श्रेष्ठिना च तथेत्याह गच्छामः सहिता वयम् \thinspace{\dandab} \dontdisplaylinenum }%
     \var{{\devanagarifont \numemph\va\textbf{तथेत्याह}\lem \msCa\msNb\Ed, तथैत्याह \msCb\msNa\msNc}}% 
    \var{{\devanagarifont \numnoemph\vb\textbf{गच्छामः}\lem \mssALL, 
ग\lac  मस् \msCa, गच्छाम \msNc}}% 

%Verse 12:96

{\devanagarifont यत्र प्राप्तं फलं तुभ्यं मोक्षयामो न संशयः {॥ १२:९६॥} \veg\dontdisplaylinenum }%
     \var{{\devanagarifont \numnoemph\vc\textbf{प्राप्तं}\lem \mssALL, प्राप्त \Ed}}% 
    \var{{\devanagarifont \numnoemph\vd\textbf{तुभ्यं}\lem \mssALL, तुभ्य \msNb}}% 

{\devanagarifont रुद्र उवाच {\dandab}\dontdisplaylinenum  }%
 
{\devanagarifont तमारुह्य गिरिं सह्यं मार्गमाणः समन्ततः \thinspace{\danda} \dontdisplaylinenum }%
     \var{{\devanagarifont \numemph\va\textbf{गिरिं}\lem \mssALL, गिरि \msCb}}% 
    \var{{\devanagarifont \numnoemph\vb\textbf{॰मानः}\lem \mssALL, ॰मानाः \Ed}}% 

%Verse 12:97

{\devanagarifont विपुलेन ततो दृष्टो वानरः प्लवगाधिपः {॥ १२:९७॥} \veg\dontdisplaylinenum }%
     \var{{\devanagarifont \numnoemph\vd\textbf{वानरः}\lem \mssALL, वानर \msCb\oo 
\textbf{प्लवगा॰}\lem \mssALL, प्लगा॰ \msCa}}% 

{\devanagarifont अयं स वानरश्रेष्ठो वृक्षच्छायां समाश्रितः \thinspace{\dandab} \dontdisplaylinenum }%
     \var{{\devanagarifont \numemph\va\textbf{वानरश्रेष्ठो}\lem \mssALL, वानरः श्रे\uncl{ष्ठे} \msNc, 
वानरः श्रेष्ठो \Ed}}% 
    \var{{\devanagarifont \numnoemph\vb\textbf{वृक्षच्छायां}\lem \msNc, वृक्षच्छांया॰ \msCa, वृक्षच्छाया॰ \msCb\msNb\Ed, वृच्छायां \msNa}}% 

%Verse 12:98

{\devanagarifont मम पुण्यबलेनैव दृश्यते ऽद्यापि वानरः {॥ १२:९८॥} \veg\dontdisplaylinenum }%
 
{\devanagarifont वानर कुरु मित्रार्थं सद्यो मृत्युर्भवेन्मम \thinspace{\dandab} \dontdisplaylinenum }%
     \var{{\devanagarifont \numemph\va\textbf{वानर}\lem \mssALL, वानरं \msNb\oo 
\textbf{॰र्थं}\lem \mssALL, ॰र्थ \msCb\msNb}}% 
    \var{{\devanagarifont \numnoemph\vb\textbf{मृत्युर्भ॰}\lem \mssALL, मृत्यु भ॰ \msNa\msNb}}% 

%Verse 12:99

{\devanagarifont पूर्वदत्तं फलमन्यद्देहि वानर जीवय {॥ १२:९९॥} \veg\dontdisplaylinenum }%
     \var{{\devanagarifont \numnoemph\vc\textbf{॰दत्तं}\lem \msCa\msNc\Ed, ॰दत्त॰ \msCb\msNa\msNb\oo 
\textbf{फलमन्य॰}\lem \mssALL, फलंमन्य॰ \msNa}}% 
    \var{{\devanagarifont \numnoemph\vd\textbf{॰हि वानर जीवय}\lem \msCa, ॰वि वानर जीवयः \msCb, 
॰हि वानर जीवयः \msNa\msNb, 
॰हि वान जीवय \msNc, ॰हि वा न च जीवये \Ed}}% 

{\devanagarifont वानर उवाच {\dandab}\dontdisplaylinenum  }%
 
{\devanagarifont गन्धर्वेण तु मे दत्तं फलं दत्तं तु ते मया \thinspace{\danda} \dontdisplaylinenum }%
     \var{{\devanagarifont \numemph\va\textbf{तु मे दत्तं}\lem \mssALL, 
तु मे दत्त॰ \msNb, मम दत्तं \Ed}}% 

%Verse 12:100

{\devanagarifont पुनरन्यत्कथं दास्ये तत्र गच्छ यदीच्छसि {॥ १२:१००॥} \veg\dontdisplaylinenum }%
 
{\devanagarifont विपुल उवाच {\dandab}\dontdisplaylinenum  }%
 
{\devanagarifont अदत्त्वा तत्फलं तुभ्यं जीवितुं संशयो भवेत् \thinspace{\danda} \dontdisplaylinenum }%
     \var{{\devanagarifont \numemph\va\textbf{अदत्त्वा}\lem \mssALL, अदत्ता \msNc}}% 
    \var{{\devanagarifont \numnoemph\vb\textbf{जीवितुं}\lem \mssALL, जीवितु \msNa, जीवितं \msNb\oo 
\textbf{भवेत्}\lem \mssALL, \uncl{भवेत्} \msNa}}% 

%Verse 12:101

{\devanagarifont अथवा तत्र गच्छामो यत्र चित्ररथः स्वयम् {॥ १२:१०१॥} \veg\dontdisplaylinenum }%
     \var{{\devanagarifont \numnoemph\vc\textbf{अथवा तत्र}\lem \mssALL, अ\lac  त्र \msCa}}% 
    \var{{\devanagarifont \numnoemph\vd\textbf{चित्ररथः}\lem \mssALL, 
चिरथः \msCbacorr, चित्ररथ \msNa}}% 

{\devanagarifont वानरः पुनरेवाह एवं कुर्वामहे वयम् \thinspace{\dandab} \dontdisplaylinenum }%
     \var{{\devanagarifont \numemph\vb\textbf{एवं}\lem \mssALL, एव \msCb}}% 

%Verse 12:102

{\devanagarifont ततश्चित्ररथावासमुपगम्येदमब्रवीत् {॥ १२:१०२॥} \veg\dontdisplaylinenum }%
     \var{{\devanagarifont \numnoemph\vc\textbf{ततश्चि॰}\lem \msCa\msCb\msNa, तत्रश्चि॰ \msNb, तत्र चि॰ \msNc\Ed}}% 
    \var{{\devanagarifont \numnoemph\vd\textbf{॰ब्रवीत्}\lem \msCa\msCb\msNc\Ed, ॰वीत् \msNaacorr, 
॰वीवीत् \msNapcorr, ॰ब्रवी \msNb}}% 

{\devanagarifont गन्धर्वराज कार्यार्थी त्वामहं पुनरागतः \thinspace{\dandab} \dontdisplaylinenum }%
     \var{{\devanagarifont \numemph\vb\textbf{त्वामहं पु॰}\lem \conj, त्वन्ह्ययम्पु॰ \msCa\msNc, 
त्वात् ह्यहम्पु॰ \msCb, 
त्वत् ह्ययं पु॰ \msNa, त्वत् ह्यहं पु॰ \msNb\Ed}}% 

%Verse 12:103

{\devanagarifont पूर्वदत्तफलं त्वन्यद्देहि मां यदि शक्यते {॥ १२:१०३॥} \veg\dontdisplaylinenum  }%
 
{\devanagarifont गन्धर्वराज उवाच {\dandab}\dontdisplaylinenum  }%
     \var{{\devanagarifont \numemph\vo\textbf{गन्धर्वराज उवाच}\lem \msCb, गन्धर्वराजोवाच \msCa\msNb\Ed, 
गन्धर्वराजौवाच \msNa, 
गन्धराज उवाच \msNc}}% 

{\devanagarifont सूर्यलोकगतश्चास्मि तेन दत्तं फलोत्तमम् \thinspace{\danda} \dontdisplaylinenum }%
     \var{{\devanagarifont \numnoemph\va\textbf{गतश्चास्मि}\lem \mssALL, 
गत\uncl{श्चा}\lac\  \msCa, गतश्चास्मिं \msNb}}% 
    \var{{\devanagarifont \numnoemph\vb\textbf{तेन दत्तं}\lem \mssALL, \lac  त्तम् \msCa}}% 

%Verse 12:104

{\devanagarifont मया दत्तं फलं तुभ्यमत्यन्तसुहृदो ऽसि मे {॥ १२:१०४॥} \veg\dontdisplaylinenum }%
     \var{{\devanagarifont \numnoemph\vc\textbf{दत्तं}\lem \corr, दत्त॰ \msCa\msCb\msNa\msNb\msNc\Ed}}% 
    \var{{\devanagarifont \numnoemph\vd\textbf{॰सुहृदो}\lem \mssALL, ॰सुह्यदो \msCb}}% 

{\devanagarifont कुतो ऽन्यत्फलमादास्ये मम नास्ति प्लवङ्गम \thinspace{\dandab} \dontdisplaylinenum }%
     \var{{\devanagarifont \numemph\va\textbf{ऽन्यत्फलमादास्ये}\lem \mssALL, 
ऽन्यफल दास्यामि \Ed}}% 
    \var{{\devanagarifont \numnoemph\vb \lem \mssALL, 
मम नास्ति प्लवङ्गमः \msNa, 
मत्तो ऽस्ति प्लवङ्गमः \Ed}}% 

%Verse 12:105

{\devanagarifont सूर्यलोकं गमिष्यामस्तत्र याचस्व भास्करम् {॥ १२:१०५॥} \veg\dontdisplaylinenum }%
     \var{{\devanagarifont \numnoemph\vcd\textbf{गमिष्यामस्तत्र}\lem \mssALL, 
गमिष्यामस्तत \msNc, गमिष्यामि तत्र \Ed}}% 

{\devanagarifont गन्धर्वेनैवमुक्तस्तु तथेत्याह प्लवङ्गमः \thinspace{\dandab} \dontdisplaylinenum }%
     \var{{\devanagarifont \numemph\vb\textbf{तथेत्याह}\lem \mssALL, तथैत्याह \msCb}}% 

%Verse 12:106

{\devanagarifont सूर्यलोकं ततः प्राप्ता गन्धर्वादय सर्वशः {॥ १२:१०६॥} \veg\dontdisplaylinenum }%
     \var{{\devanagarifont \numnoemph\vc\textbf{प्राप्ता}\lem \mssALL, प्राप्ताः \msNc}}% 
    \var{{\devanagarifont \numnoemph\vd\textbf{॰दय सर्वशः}\lem \conj, ॰दयस्सर्वशः \msCa\ \unmetr, 
॰दयः सर्वशः \msCb\msNa\msNc\Ed\ \unmetr, दय सर्वश \msNb}}% 

{\devanagarifont गन्धर्व उवाच {\dandab}\dontdisplaylinenum  }%
     \var{{\devanagarifont \numemph\vo\textbf{गन्धर्व उवाच}\lem \mssALL, 
गन्धर्व \uncl{उवा}\lac\  \msCa, 
गन्धर्वराजोवाच \Ed}}% 

{\devanagarifont कार्यार्थेन पुनः प्राप्तस्त्वत्सकाशं खगेश्वर \thinspace{\danda} \dontdisplaylinenum }%
     \var{{\devanagarifont \numnoemph\vab\textbf{प्राप्तस्त्व॰}\lem \mssALL, प्राप्त त्व॰ \msNa}}% 
    \var{{\devanagarifont \numnoemph\vb\textbf{॰काशं}\lem \mssALL, ॰काशां \msNb\oo 
\textbf{॰श्वर}\lem \mssALL, ॰श्वरः \msNb\msNc}}% 

%Verse 12:107

{\devanagarifont पूर्वदत्तफलं त्वन्यद्देहि जीवमनाशय {॥ १२:१०७॥} \veg\dontdisplaylinenum }%
     \var{{\devanagarifont \numnoemph\vc\textbf{फलं त्वन्य॰}\lem \msCa\msNa\msNc, 
फलं त्व॰ \msCb, फलंस्त्वन्य॰ \msNb\Ed}}% 
    \var{{\devanagarifont \numnoemph\vd\textbf{॰नाशय}\lem \mssALL, अनामयः \msNb, ॰नाशयः \Ed}}% 

{\devanagarifont सूर्य उवाच {\dandab}\dontdisplaylinenum  }%
 
{\devanagarifont सोमलोकगतश्चास्मि तेन दत्तं फलोत्तमम् \thinspace{\danda} \dontdisplaylinenum }%
     \var{{\devanagarifont \numemph\vab\textbf{॰स्मि तेन}\lem \mssALL, ॰स्मिन्तेन \msNb}}% 
    \var{{\devanagarifont \numnoemph\vb\textbf{दत्तं}\lem \mssALL, दत्त॰ \msNb}}% 

%Verse 12:108

{\devanagarifont स फलं दत्तमेवासि सुहृदत्वान्मया तव {॥ १२:१०८॥} \veg\dontdisplaylinenum }%
     \var{{\devanagarifont \numnoemph\vc\textbf{॰वासि}\lem \msCa\msCb\msNc, ॰वा\uncl{भि} \msNa, 
॰एवाति \msNb, ॰वाभिः \Ed}}% 
    \var{{\devanagarifont \numnoemph\vd\textbf{सुहृदत्वान्मया}\lem \mssALL, 
सुहृदत्वात्मया \msNa, स च दत्वा मया \Ed}}% 

{\devanagarifont अन्यद्दातुं न शक्नोमि गच्छ सोमपुराद्य वै \thinspace{\dandab} \dontdisplaylinenum }%
     \var{{\devanagarifont \numemph\va\textbf{अन्यद्दातुं}\lem \msNa\msNc\Ed, अन्य दातुं \msCa\msCb, अन्य दातु \msNb}}% 
    \var{{\devanagarifont \numnoemph\vb\textbf{॰पुराद्य}\lem \mssALL, ॰पराद्य \Ed}}% 

%Verse 12:109

{\devanagarifont तं प्रार्थयाविकल्पेन अत्रिपुत्रं ग्रहेश्वरम् {॥ १२:१०९॥} \veg\dontdisplaylinenum }%
     \var{{\devanagarifont \numnoemph\vc\textbf{तं}\lem \mssALL, त \msNb\oo 
\textbf{॰विकल्पेन}\lem \mssALL, 
॰\uncl{विक}\lac\  \msCa}}% 
    \var{{\devanagarifont \numnoemph\vd\textbf{॰पुत्रं}\lem \mssALL, ॰पुत्र॰ \msCa\msNb}}% 

{\devanagarifont रुद्र उवाच {\dandab}\dontdisplaylinenum  }%
     \var{{\devanagarifont \numemph\vo\textbf{रुद्र}\lem \mssALL, महेश्वर \Ed}}% 

{\devanagarifont गताः सूर्याग्रतः कृत्वा सोमलोकं तथैव हि \thinspace{\danda} \dontdisplaylinenum }%
     \var{{\devanagarifont \numnoemph\va\textbf{गताः}\lem \msCb, गत \msCa\msNa\msNb, गतः \msNc\Ed}}% 
    \var{{\devanagarifont \numnoemph\vb\textbf{हि}\lem \mssALL, \om\ \msNb}}% 

%Verse 12:110

{\devanagarifont उवाच सूर्यः सोमाय कारणापेक्षया शशिम् {॥ १२:११०॥} \veg\dontdisplaylinenum }%
     \var{{\devanagarifont \numnoemph\vc\textbf{सूर्यः}\lem \mssALL, सूर्य \msNb}}% 
    \var{{\devanagarifont \numnoemph\vd\textbf{कारणा॰}\lem \mssALL, करुणा॰ \msCb\oo 
\textbf{॰पेक्षया}\lem \mssALL, ॰पेक्षणा \msNb\oo 
\textbf{शशिम्}\lem \msCa\msCb\msNa, शशि \msNb\Ed, शशि\uncl{न्} \msNc}}% 

{\devanagarifont सोम उवाच {\dandab}\dontdisplaylinenum  }%
 
{\devanagarifont किमर्थमागतो भूयः कर्तव्यं तत्र भास्कर \thinspace{\danda} \dontdisplaylinenum }%
     \var{{\devanagarifont \numemph\va\textbf{॰गतो}\lem \mssALL, ॰गता \msNb}}% 
    \var{{\devanagarifont \numnoemph\vb\textbf{तत्र}\lem \mssALL, तव \Ed\oo 
\textbf{॰कर}\lem \mssALL, ॰करः \Ed}}% 

%Verse 12:111

{\devanagarifont फलं दातुं पुनस्त्वन्यन्मुक्त्वा त्वन्यत्करोम्यहम् {॥ १२:१११॥} \veg\dontdisplaylinenum }%
     \var{{\devanagarifont \numnoemph\vcd\textbf{पुनस्त्वन्यन्मुक्त्वा त्वन्यत्क॰}\lem \corr, 
पुनस्त्वन्य मुक्त्वा त्वन्यङ्क॰ \msCa, 
पुनस्त्वन्यन्मुक्त्वास्त्वन्यं क॰ \msCb, 
पुनः त्वन्य मुक्त्वा त्वन्यत्क॰ \msNa, 
पुनस्त्वन्य मुक्त्वा त्वन्यत्क॰ \msNb, 
पुनस्त्वन्यत्मुक्ता त्वन्यङ्क॰ \msNc\Ed}}% 

{\devanagarifont सूर्य उवाच {\dandab}\dontdisplaylinenum  }%
 
{\devanagarifont यदि शक्यं फलं देहि अन्यन्न प्रार्थयाम्यहम् \thinspace{\danda} \dontdisplaylinenum }%
     \var{{\devanagarifont \numemph\va\textbf{शक्यं फलं देहि}\lem \msCa\msNa\msNc\Ed, काफलन्देहि \msCbacorr, 
काफल\lk न्देहि \msCbpcorr, शक्य फलं देहि \msNb}}% 
    \var{{\devanagarifont \numnoemph\vb\textbf{अन्यन्न}\lem \mssALL, अन्यत्वं \msNc, अन्यान्न \Ed}}% 

%Verse 12:112

{\devanagarifont न दत्तासि फलमन्यन्मया वध्यो भविष्यसि {॥ १२:११२॥} \veg\dontdisplaylinenum }%
     \var{{\devanagarifont \numnoemph\vcd\textbf{फलमन्यन्म॰}\lem \mssALL, 
फलंमन्यन्म॰ \msNa, 
फलं मन्ये म॰ \Ed}}% 
    \var{{\devanagarifont \numnoemph\vd\textbf{वध्यो}\lem \msNc, वद्ध्यो \msCa\msCb\msNa\msNb, वद्धो \Ed\oo 
\textbf{भविष्यसि}\lem \mssALL, भविष्यति \msCb}}% 

{\devanagarifont सोम उवाच {\dandab}\dontdisplaylinenum  }%
 
{\devanagarifont आगमं तस्य वक्ष्यामि शृणुष्वावहितो भव \thinspace{\danda} \dontdisplaylinenum }%
     \var{{\devanagarifont \numemph\va\textbf{वक्ष्यामि}\lem \mssALL, वक्ष्या\uncl{मि} \msNa}}% 

%Verse 12:113

{\devanagarifont इन्द्रेणास्मि फलं दत्तं स फलं दत्त मे भवान् {॥ १२:११३॥} \veg\dontdisplaylinenum }%
     \var{{\devanagarifont \numnoemph\vd\textbf{दत्त मे}\lem \mssALL, वत्त मे \msNa}}% 
    \lacuna{\devanagarifontsmall \vd {\englishfont \msCc\ resumes here with } दत्त मे भवान् }%
  
{\devanagarifont गत्वैवेन्द्रसदस्त्वन्यत्प्रार्थयामः सहैव तु \thinspace{\dandab} \dontdisplaylinenum }%
     \var{{\devanagarifont \numemph\va\textbf{गत्वैवेन्द्र॰}\lem \msCa, गत्वेवेन्द्र॰ \msCb\msNb\msNc, \lk\lk \lk\lk\ \msCc, 
गत्वावेन्द्र॰ \msNa, गन्धर्वेन्द्र॰ \Ed}}% 
    \var{{\devanagarifont \numnoemph\vb\textbf{॰र्थयामः}\lem \mssALL, ॰र्थयामा \msNa\oo 
\textbf{सहैव तु}\lem \mssALL, 
सदैव तु \msCc, सहैव तुः \msNc}}% 

%Verse 12:114

{\devanagarifont एवं कुर्म इति प्राह गत्वेन्द्रसदनं प्रति {॥ १२:११४॥} \veg\dontdisplaylinenum }%
     \var{{\devanagarifont \numnoemph\vc\textbf{कुर्म}\lem \mssALL, कर्म \msNb, सोम \Ed}}% 

{\devanagarifont सोमेनेन्द्रमुवाचेदं फलकामा इहागताः \thinspace{\dandab} \dontdisplaylinenum }%
     \var{{\devanagarifont \numemph\va\textbf{सोमेनेन्द्र॰}\lem \mssCaCbCc\msNa\Ed, सोम इन्द्र॰ \msNc, सोमेवेन्द्र॰ \msNb\oo 
\textbf{॰चेदं}\lem \mssALL, ॰चेन्द्रं \msCc}}% 

%Verse 12:115

{\devanagarifont पूर्वदत्तफलमन्यद्देहि शक्र ममाद्य वै {॥ १२:११५॥} \veg\dontdisplaylinenum }%
     \var{{\devanagarifont \numnoemph\vc\textbf{पूर्व॰}\lem \mssALL, पूर्वं \msNb}}% 
    \var{{\devanagarifont \numnoemph\vcd\textbf{॰न्यद्देहि}\lem \mssALL, ॰न्य देहि \msCc}}% 
    \var{{\devanagarifont \numnoemph\vd\textbf{शक्र}\lem \mssALL, शक \Ed\oo 
\textbf{वै}\lem \mssALL, वैः \msCb}}% 

{\devanagarifont इन्द्र उवाच {\dandab}\dontdisplaylinenum  }%
 
{\devanagarifont यदर्थमिह सम्प्राप्तः स च नास्ति निशाकर \thinspace{\danda} \dontdisplaylinenum }%
     \var{{\devanagarifont \numemph\vb\textbf{॰कर}\lem \mssALL, ॰करः \msCb\Ed}}% 

%Verse 12:116

{\devanagarifont विष्णुहस्तान्मया प्राप्तमेकमेव फलं शुभम् {॥ १२:११६॥} \veg\dontdisplaylinenum }%
     \var{{\devanagarifont \numnoemph\vc\textbf{विष्णुहस्तान्मया}\lem \mssALL, 
विष्णुहस्ता मया \msNb}}% 
    \var{{\devanagarifont \numnoemph\vd\textbf{फलं}\lem \mssALL, फल \msCb}}% 

{\devanagarifont सर्व एव हि गच्छामो विष्णुलोकं ग्रहेश्वर \thinspace{\dandab} \dontdisplaylinenum }%
     \var{{\devanagarifont \numemph\vb\textbf{॰लोकं}\lem \mssALL, ॰लोक \msCc\oo 
\textbf{॰श्वर}\lem \mssALL, ॰श्वरं \msCb, ॰श्व\lk\ \msNb}}% 

%Verse 12:117

{\devanagarifont सर्व एवोपजग्मुस्ते फलार्थं मधुसूदनम् {॥ १२:११७॥} \veg\dontdisplaylinenum }%
     \var{{\devanagarifont \numnoemph\vc \lem \mssALL, 
सर्व एवोपञ्जग्मुस्ते \msCa\ \unmetr, 
\lk\lk \lk\lk \lk\lk \lk\lk\ \msNb}}% 
    \var{{\devanagarifont \numnoemph\vd \lem \mssALL, 
\lk\lk \lk\lk \lk\lk \lk\lk\ \msNb, 
फफालार्थं मधुसूदनम् \msNc}}% 
    \lacuna{\devanagarifontsmall \vcd {\englishfont This folio side in \msNb\ (verses 12.117--138) is faded and 
                     most of it is difficult to read, thus its readings
                     reported are less reliable than usual} }%
  
{\devanagarifont एवमुक्त्वा गताः सर्वे देवराजपुरस्कृताः \thinspace{\dandab} \dontdisplaylinenum }%
     \var{{\devanagarifont \numemph\va \lem \mssCaCbCc\msNa, 
\lk\lk \lk\lk \lk\lk \lk\lk\ \msNb, एवमुक्त्वा गता सर्वे \msNc, 
एवमुक्ता गताः सर्वे \Ed}}% 

%Verse 12:118

{\devanagarifont मुहूर्तेनैव सम्प्राप्ता विष्णुलोकं यशस्विनि {॥ १२:११८॥} \veg\dontdisplaylinenum }%
     \var{{\devanagarifont \numnoemph\vd\textbf{विष्णुलोकं}\lem \mssALL, 
विष्णुलोक \msCc, \lk\lk \lk\lk\ \msNb}}% 

{\devanagarifont उपसृत्य तत इन्द्रः प्रणिपत्य जनार्दनम् \thinspace{\dandab} \dontdisplaylinenum }%
 
%Verse 12:119

{\devanagarifont सर्वेषामुपरोधेन प्रार्थयामि यशोधर {॥ १२:११९॥} \veg\dontdisplaylinenum }%
     \var{{\devanagarifont \numemph\vd\textbf{॰धर}\lem \mssALL, ॰धरम् \Ed}}% 

{\devanagarifont विष्णुरुवाच {\dandab}\dontdisplaylinenum  }%
     \var{{\devanagarifont \numemph\vo\textbf{विष्णुरुवाच}\lem \msCapcorr\msCb\msCc\msNapcorr\msNb\msNc, 
विष्णुरुच \msCaacorr, 
\om\ \msNaacorr, विष्णु उवाच \Ed}}% 

{\devanagarifont पूर्वदत्तफलस्यार्थे तच्च सर्वमिहागताः \thinspace{\danda} \dontdisplaylinenum }%
     \var{{\devanagarifont \numnoemph\va\textbf{॰दत्त॰}\lem \mssALL, ॰दत्तं \Ed\oo 
\textbf{॰र्थे}\lem \mssALL, ॰र्थि \Ed}}% 

%Verse 12:120

{\devanagarifont न शक्नोमि फलं दातुं किं वा त्वन्यत्करोम्यहम् {॥ १२:१२०॥} \veg\dontdisplaylinenum }%
     \var{{\devanagarifont \numnoemph\vc\textbf{शक्नोमि}\lem \mssALL, शक्नोति \msCb\oo 
\textbf{फलं दातुं}\lem \mssALL, 
फल\uncl{न्दातु} \msCc}}% 
    \var{{\devanagarifont \numnoemph\vd\textbf{त्वन्यत्करोम्यहम्}\lem \msNc, 
त्वन्यं करोम्यहम् \mssCaCbCc\msNa\Ed, 
\lk\lk \lk\lk \lk\lk म्यहम् \msNb}}% 

{\devanagarifont इन्द्र उवाच {\dandab}\dontdisplaylinenum  }%
 
{\devanagarifont ब्रह्माण्डमपि भेत्तुं त्वं शक्नोषि गरुडध्वज \thinspace{\danda} \dontdisplaylinenum }%
     \var{{\devanagarifont \numemph\va\textbf{ब्रह्माण्ड॰}\lem \mssALL, ब्रह्मण्ड॰ \msNc\oo 
\textbf{भेत्तुं त्वं}\lem \mssALL, 
भेत्तु त्वं \msCb, भर्तुंत्वं \Ed}}% 
    \var{{\devanagarifont \numnoemph\vb\textbf{शक्नोषि}\lem \mssALL, शक्नोति \msCb}}% 

%Verse 12:121

{\devanagarifont अशक्यं तव नास्तीति जानामि पुरुषोत्तम {॥ १२:१२१॥} \veg\dontdisplaylinenum }%
     \var{{\devanagarifont \numnoemph\vc\textbf{अशक्यं}\lem \mssALL, \uncl{अशक्य} \msCb}}% 
    \var{{\devanagarifont \numnoemph\vd\textbf{॰त्तम}\lem \mssALL, ॰त्तमम् \Ed}}% 

{\devanagarifont एवमुक्तः पुनर्विष्णुः प्रत्युवाच पुरन्दरम् \thinspace{\dandab} \dontdisplaylinenum }%
     \var{{\devanagarifont \numemph\va \lem \msCb, 
एवमुक्त्वा पुनर्विष्णुः \msCa\msCc\msNa\msNc\Ed, 
\lk\lk \lk\lk  पुनर्विष्णुः \msNb}}% 
    \var{{\devanagarifont \numnoemph\vb\textbf{पुरन्दरम्}\lem \mssALL, पुरदरं \msNc\ \unmetr}}% 

%Verse 12:122

{\devanagarifont फलमेकं परित्यज्य सर्वं शक्नोमि कौशिक {॥ १२:१२२॥} \veg\dontdisplaylinenum }%
     \var{{\devanagarifont \numnoemph\vd\textbf{सर्वं शक्नोमि}\lem \mssALL, सर्वं शक्नोसि \msCc, 
\lk\lk  शक्नोमि \msNb}}% 

{\devanagarifont उपायो ऽत्र प्रवक्ष्यामि आगमं शृणु गोपते \thinspace{\dandab} \dontdisplaylinenum }%
 
%Verse 12:123

{\devanagarifont ब्रह्मणा च मम दत्तं तत्फलैकं पुरन्दर {॥ १२:१२३॥} \veg\dontdisplaylinenum }%
     \var{{\devanagarifont \numemph\vc\textbf{मम}\lem \mssALL, ममा॰ \Ed}}% 
    \var{{\devanagarifont \numnoemph\vd\textbf{तत्फलैकं}\lem \mssALL, तत्फलंकं \msNaacorr, 
तत्फलेकं \msNapcorr\oo 
\textbf{पुरन्दर}\lem \mssALL, 
पुरन्द\uncl{रं} \msNc}}% 

{\devanagarifont मया दत्तं फलं त्वेकं किमन्यद्दातुमिच्छसि \thinspace{\dandab} \dontdisplaylinenum }%
     \var{{\devanagarifont \numemph\va\textbf{दत्तं}\lem \msCc\msNb, दत्त॰ \msCa\msCb\msNa\msNc\Ed\oo 
\textbf{त्वेकं}\lem \mssALL, त्वैकं \msNc}}% 
    \var{{\devanagarifont \numnoemph\vb\textbf{॰च्छसि}\lem \mssALL, ॰च्छति \msCa}}% 

%Verse 12:124

{\devanagarifont प्रार्थयामो ऽत्र गत्वैकं परमेष्ठिप्रजापतिम् {॥ १२:१२४॥} \veg\dontdisplaylinenum  }%
     \var{{\devanagarifont \numnoemph\vc \lem \mssALL, 
प्रार्थया च गत्वैवं \Ed}}% 
    \var{{\devanagarifont \numnoemph\vd\textbf{॰ष्ठिप्रजा॰}\lem \msCa\msNa\msNb\msNc, ॰ष्ठिं प्रजा॰ \msCb\Ed, 
॰ष्ठि\uncl{प्रजा}॰ \msCc}}% 

{\devanagarifont तवोपरोधाद्देवेन्द्र प्रार्थयामि पितामहम् \thinspace{\dandab} \dontdisplaylinenum }%
     \var{{\devanagarifont \numemph\va\textbf{तवो॰}\lem \mssALL, ततो॰ \Ed\oo 
\textbf{॰रोधाद्देवे॰}\lem \msCa\msCb\msNa\msNc, ॰रोधा देवे॰ \msCc\msNb, 
॰राधाद्देवे॰ \Ed}}% 
    \var{{\devanagarifont \numnoemph\vb\textbf{॰महम्}\lem \mssALL, ॰मह \msNb}}% 

%Verse 12:125

{\devanagarifont एवमुक्त्वा गताः सर्वे पुरस्कृत्य जनार्दनम् {॥ १२:१२५॥} \veg\dontdisplaylinenum }%
     \var{{\devanagarifont \numnoemph\vc\textbf{गताः}\lem \mssALL, गता \msCc\Ed}}% 
    \var{{\devanagarifont \numnoemph\vd\textbf{पुरस्कृत्य}\lem \mssALL, पुनस्कृत्य \msNc\oo 
\textbf{जनार्दनम्}\lem \mssALL, जनार्द्दन \msCc}}% 

{\devanagarifont इन्द्रः सूर्यः शशी चैव गन्धर्वो वानरस्तथा \thinspace{\dandab} \dontdisplaylinenum }%
     \var{{\devanagarifont \numemph\va\textbf{इन्द्रः}\lem \mssALL, इन्द्र \msCc\oo 
\textbf{सूर्यः शशी चैव}\lem \msCa\msCb\msNa\msNc, सूर्य शशी चैव \msCc\msNb, 
सोमश्च सूर्यश्च \Ed}}% 

%Verse 12:126

{\devanagarifont विपुलः श्रेष्ठिकश्चैव राजदूतद्वयं तथा {॥ १२:१२६॥} \veg\dontdisplaylinenum }%
     \var{{\devanagarifont \numnoemph\vc\textbf{विपुलः}\lem \mssALL, विपुल \msNa\msNb}}% 
    \var{{\devanagarifont \numnoemph\vd\textbf{॰द्वयं तथा}\lem \Ed, ॰द्वयस्तथा \mssCaCbCc\msNa\msNb\msNc}}% 

{\devanagarifont ब्रह्मलोकं मुहूर्तेन प्राप्तवान्सुरसुन्दरि \thinspace{\dandab} \dontdisplaylinenum }%
     \var{{\devanagarifont \numemph\va\textbf{॰लोकं}\lem \mssALL, ॰लोक \msNb}}% 

%Verse 12:127

{\devanagarifont दृष्ट्वा ब्रह्मसदो रम्यं सर्वकामपरिच्छदम् {॥ १२:१२७॥} \veg\dontdisplaylinenum }%
     \var{{\devanagarifont \numnoemph\vc\textbf{॰सदो}\lem \mssALL, ॰सदं \Ed\oo 
\textbf{रम्यं}\lem \mssALL, रम्यां \msNb}}% 

{\devanagarifont अनेकानि विचित्राणि रत्नानि विविधानि च \thinspace{\dandab} \dontdisplaylinenum }%
 
%Verse 12:128

{\devanagarifont मन्दारतल शोभानि वैडूर्यमणिकुट्टिमान् {॥ १२:१२८॥} \veg\dontdisplaylinenum }%
     \var{{\devanagarifont \numemph\vc\textbf{॰तल}\lem \mssALL, ॰तरु॰ \Ed}}% 
    \var{{\devanagarifont \numnoemph\vd\textbf{वैडूर्य॰}\lem \mssALL, वैदूर्य॰ \Ed\oo 
\textbf{॰कुट्टिमान्}\lem \corr, ॰कुटिमाम् \msCa, 
॰कुट्टिमां \msCb\msCc\msNa\msNb\msNc, ॰कुट्टिमम् \Ed}}% 

{\devanagarifont प्रवालमणिस्तम्भानि वज्रकाञ्चनवेदिकाम् \thinspace{\dandab} \dontdisplaylinenum }%
     \var{{\devanagarifont \numemph\vb \lem \msCa\msCb\msNa, 
वज्रकाञ्चनवेदिका \msCc\msNc\Ed, \lk\lk \lk\lk \lk\lk \lk का \msNb}}% 

%Verse 12:129

{\devanagarifont प्रवालस्फाटिको जाल इन्द्रनीलगवाक्षकः {॥ १२:१२९॥} \veg\dontdisplaylinenum }%
     \var{{\devanagarifont \numnoemph\vc \lem \mssCaCbCc\msNc, प्रवालस्फणिको जाल \msNa, 
प्र\uncl{ता}लस्फाटिको जाल \msNb, 
प्रवालस्फटिको जाला \Ed}}% 
    \var{{\devanagarifont \numnoemph\vd\textbf{॰क्षकः}\lem \mssALL, ॰क्षकं \msNa\msNb}}% 

{\devanagarifont पश्यते विपुलस्तत्र नानावृक्ष मनोरमाः \thinspace{\dandab} \dontdisplaylinenum }%
     \var{{\devanagarifont \numemph\va\textbf{पश्यते}\lem \mssALL, दृश्यन्ते \Ed\oo 
\textbf{विपुल॰}\lem \mssALL, विपुला॰ \Ed}}% 

%Verse 12:130

{\devanagarifont पुष्पानामितवृक्षाग्राः फलानामितका भवेत् {॥ १२:१३०॥} \veg\dontdisplaylinenum }%
     \var{{\devanagarifont \numnoemph\vc\textbf{पुष्पा॰}\lem \mssALL, पुष्प॰ \msNc\Ed\oo 
\textbf{॰ग्राः}\lem \eme, ॰ग्रा \mssCaCbCc\msNa\msNc, ॰ग्रा \msNb, ॰या \Ed}}% 
    \var{{\devanagarifont \numnoemph\vd\textbf{फलानामितका}\lem \mssALL, फलनामितकां \Ed}}% 

{\devanagarifont सर्वरत्नमया वृक्षाः सर्वरत्नमयं जलम् \thinspace{\dandab} \dontdisplaylinenum }%
     \var{{\devanagarifont \numemph\va\textbf{सर्व॰}\lem \msCb\msNa\msNb\Ed, सर्वे \msCa\msCc\msNc\oo 
\textbf{वृक्षाः}\lem \mssALL, वृक्षा \msCc\oo 
\textbf{॰मया}\lem \mssALL, ॰मयो \msNb}}% 
    \var{{\devanagarifont \numnoemph\vb\textbf{सर्व॰}\lem \mssALL, सर्वे \Ed}}% 

%Verse 12:131

{\devanagarifont वृक्षगुल्मलतावल्ली कन्दमूलफलानि च {॥ १२:१३१॥} \veg\dontdisplaylinenum }%
     \var{{\devanagarifont \numnoemph\vc\textbf{॰गुल्म॰}\lem \mssALL, \om\ \msNaacorr\oo 
\textbf{॰वल्ली}\lem \mssALL, ॰वली \msCc}}% 

{\devanagarifont सर्वे रत्नमया दृष्टा विपुलो विपुलेक्षणः \thinspace{\dandab} \dontdisplaylinenum }%
     \var{{\devanagarifont \numemph\va\textbf{सर्वे}\lem \mssALL, सर्वै \msCa, सर्व्व॰ \msCc\oo 
\textbf{दृष्टा}\lem \mssALL, 
दृष्ट्वा \msCb, दृ \msNcacorr}}% 
    \var{{\devanagarifont \numnoemph\vb\textbf{॰क्षणः}\lem \mssALL, ॰क्षण \msCc}}% 

%Verse 12:132

{\devanagarifont अनेकभौमं प्रासादं मुक्तादामविभूषितम् {॥ १२:१३२॥} \veg\dontdisplaylinenum }%
     \var{{\devanagarifont \numnoemph\vc\textbf{॰भौमं}\lem \mssALL, ॰भौम॰ \msNc}}% 

{\devanagarifont अप्सरोगणकोटीभिः सर्वाभरणभूषितम् \thinspace{\dandab} \dontdisplaylinenum }%
     \var{{\devanagarifont \numemph\vab \lem \mssALL, 
\lk\lk \lk\lk \lk\lk \lk\lk \lk\lk \lk\lk \lk\lk \lk\  \msNb}}% 

%Verse 12:133

{\devanagarifont विमानकोटिकोटीनां सर्वकामसमन्वितम् {॥ १२:१३३॥} \veg\dontdisplaylinenum }%
     \var{{\devanagarifont \numnoemph\vcd \lem \msCb\msCc\msNa\msNc, 
विमानकोटिकोटीशं सर्वकामसमन्वितम् \msCa, 
\lk\lk \lk\lk \lk\lk \lk\lk \lk\lk \lk\lk \lk\lk \lk\lk\ \msNb, \om\ \Ed}}% 
    \paral{{\devanagarifontsmall \vo {\englishfont \compare\ \SDHS\ 10.41 (on the results of an observance):}
                 सूर्यकोटिप्रतीकाशैर्विमानैः सार्वकामिकैः\thinspace{\devanagarifontsmall ।}
                 रुद्रकन्यासमाकीर्णैर्महावृषभसंयुतैः\thinspace{\devanagarifontsmall ॥} }}

{\devanagarifont ब्रह्मलोकसभा रम्या सूर्यकोटिसमप्रभा \thinspace{\dandab} \dontdisplaylinenum }%
     \var{{\devanagarifont \numemph\vb\textbf{॰कोटि॰}\lem \mssALL, ॰\uncl{कौटि}॰ \msNc}}% 

%Verse 12:134

{\devanagarifont तत्र ब्रह्मा सुखासीनो नानारत्नोपशोभिते {॥ १२:१३४॥} \veg\dontdisplaylinenum }%
     \var{{\devanagarifont \numnoemph\vd\textbf{॰शोभिते}\lem \mssALL, ॰शोभिता \msNb}}% 

{\devanagarifont चतुर्मूर्तिश्चतुर्वक्त्रश्चतुर्बाहुश्चतुर्भुजः \thinspace{\dandab} \dontdisplaylinenum }%
     \var{{\devanagarifont \numemph\va\textbf{॰मूर्तिश्च॰}\lem \mssALL, ॰मूर्ति च॰ \msCc, 
॰मूर्\uncl{त्तिंश्च} \msNb}}% 
    \var{{\devanagarifont \numnoemph\vab\textbf{॰वक्त्रश्चतुर्बाहुश्चतुर्भुजः}\lem \mssALL, 
॰वक्त्राश्चतुर्बाहुश्चतुर्भुजः \msCc, ॰वक्त्र\lk\lk \lk\lk \lk\lk \lk\lk\  \msNb}}% 

%Verse 12:135

{\devanagarifont चतुर्वेदधरो देवश्चतुराश्रमनायकः {॥ १२:१३५॥} \veg\dontdisplaylinenum }%
     \var{{\devanagarifont \numnoemph\vc\textbf{चतुर्वेद॰}\lem \mssALL, चतुवेद॰ \msNc}}% 
    \var{{\devanagarifont \numnoemph\vcd\textbf{देवश्च॰}\lem \mssALL, देव च॰ \msCc}}% 

{\devanagarifont चतुर्वेदावृतस्तत्र मूर्तिमन्तमुपासते \thinspace{\dandab} \dontdisplaylinenum }%
     \var{{\devanagarifont \numemph\vab\textbf{॰वेदा वृतस्तत्र मूर्तिमन्तमुपासते}\lem \msCa\msCb\msNc\Ed, 
॰वेदवृतस्तत्र मूर्तिमन्तमुपासते \msCc, 
॰\uncl{वेदा}वृतस्तत्र मूर्तिमन्तमुपासते \msNa, 
वे\lk\lk \lk\lk \lk\lk \lk\lk \lk\lk \lk\lk \lk\  \msNb}}% 

%Verse 12:136

{\devanagarifont गायत्री वेदमाता च सावित्री च सुरूपिणी {॥ १२:१३६॥} \veg\dontdisplaylinenum }%
     \var{{\devanagarifont \numnoemph\vc \lem \mssALL, 
\lk\lk \lk\lk \lk\lk \lk\lk\  \msNb}}% 

{\devanagarifont व्याहृतिः प्रणवश्चैव मूर्तिमान्समुपासते \thinspace{\dandab} \dontdisplaylinenum }%
     \var{{\devanagarifont \numemph\va\textbf{व्याहृतिः}\lem \msCa\msNc\Ed, व्याहृदिः \msCb, 
व्याकृतिः \msCc, व्याहृति \msNa, \lk\lk \lk\  \msNb\oo 
\textbf{प्रणवश्चैव}\lem \msCb\msNa\msNc\Ed, प्रण\uncl{व}\lac  व \msCa, 
प्रकृतिश्चैव \msCc, \lk\lk \lk\lk \lk\ \msNb}}% 
    \var{{\devanagarifont \numnoemph\vb \lem \mssALL, 
\lk\lk \lk\lk \lk\lk \lk\lk\ \msNb}}% 

%Verse 12:137

{\devanagarifont वौषट्कारो वषट्कारो नमस्कारः स मूर्तिमान् {॥ १२:१३७॥} \veg\dontdisplaylinenum }%
     \var{{\devanagarifont \numnoemph\vc \lem \msCa\msCc\msNa\Ed, 
\om\ \msCb, \lk\lk \lk\lk \lk\lk \lk\lk\ \msNb, 
वौषट्कारो च \uncl{स}त्कारो \msNc}}% 
    \var{{\devanagarifont \numnoemph\vd\textbf{॰कारः}\lem \mssALL, ॰कार \msCc}}% 

{\devanagarifont श्रुतिः स्मृतिश्च नीतिश्च धर्मशास्त्रं समूर्तिमत् \thinspace{\dandab} \dontdisplaylinenum }%
     \var{{\devanagarifont \numemph\vb\textbf{॰शास्त्रं समूर्तिमत्}\lem \mssALL, 
॰शास्त्रसमूर्तिमान् \msCc\Ed}}% 

%Verse 12:138

{\devanagarifont इतिहासः पुराणं च सांख्ययोगः पतञ्जलम् {॥ १२:१३८॥} \veg\dontdisplaylinenum }%
     \var{{\devanagarifont \numnoemph\vc \lem \msCa\msCc\msNa\msNc, पुराणश्च \msCb\Ed, 
\lk\lk \lk\lk \lk\lk \lk\lk\ \msNb}}% 
    \var{{\devanagarifont \numnoemph\vd\textbf{सांख्ययोगः}\lem \mssALL, 
सांख्ययोग \msCc, \lk\lk \lk\lk\ \msNb\oo 
\textbf{पतञ्जलम्}\lem \mssALL, 
\lk\lk \lk\lk\ \msNb, पतञ्जलि \Ed}}% 

{\devanagarifont आयुर्वेदो धनुर्वेदो वेदो गान्धर्वमेव च \thinspace{\dandab} \dontdisplaylinenum }%
     \var{{\devanagarifont \numemph\va \lem \mssALL, 
॰वेद धनुर्वेद \msCc, \lk\lk \lk\lk \lk\lk \lk\lk\ \msNb}}% 
    \var{{\devanagarifont \numnoemph\vb\textbf{वेदो गान्धर्वमेव}\lem \msCa\msNa, वेदो गन्धर्वमेव \msCb, 
वेद गान्धर्वमेव \msCc, \lk\lk \lk\lk \lk\lk \lk\lk\ \msNb, 
वेदो गार्न्धवमेव \msNc, वेदो गान्धर्वरेव \Ed}}% 

%Verse 12:139

{\devanagarifont अर्थवेदो ऽन्यवेदाश्च मूर्तिमान् समुपासते {॥ १२:१३९॥} \veg\dontdisplaylinenum }%
     \var{{\devanagarifont \numnoemph\vc \lem \Ed, अर्थवेदान्यवेदाञ्च \msCa, 
अथर्ववेदान्यवेदञ्च \msCb\ \unmetr, अथर्व्वेदान्यवेदाञ्च \msCc, 
अर्थवेदान्यवेदां च \msNa, \lk\lk \lk\lk \lk\lk \lk\lk\ \msNb, 
अर्थवेदान्यवेदञ्च \msNc}}% 
    \var{{\devanagarifont \numnoemph\vd \lem \mssALL, 
\lk\lk \lk\lk \lk\lk \lk\lk\ \msNb}}% 

{\devanagarifont ततो ब्रह्मा समुत्थाय अभिगम्य जनार्दनम् \thinspace{\dandab} \dontdisplaylinenum }%
     \var{{\devanagarifont \numemph\vab \lem \mssALL, 
\lk\lk \lk\lk \lk\lk \lk\lk \lk\lk \lk\lk \lk\lk \lk\lk\ \msNb}}% 

%Verse 12:140

{\devanagarifont गां च अर्घं च दत्त्वैवमास्यतामिति चाब्रवीत् {॥ १२:१४०॥} \veg\dontdisplaylinenum }%
     \var{{\devanagarifont \numnoemph\vc\textbf{अर्घं च}\lem \mssALL, 
अ\uncl{घ}ञ्च \msCb, अर्घ्यञ्च \Ed}}% 

{\devanagarifont मणिरत्नमये दिव्ये आसने गरुडध्वजः \thinspace{\dandab} \dontdisplaylinenum }%
 
%Verse 12:141

{\devanagarifont देवराजो रविः सोमो गन्धर्वः प्लवगेश्वरः {॥ १२:१४१॥} \veg\dontdisplaylinenum }%
     \var{{\devanagarifont \numemph\vc\textbf{रविः सोमो}\lem \mssALL, 
र\uncl{वि} सोमो \msNb, शशी सूर्यो \Ed}}% 
    \var{{\devanagarifont \numnoemph\vd\textbf{गन्धर्वः}\lem \mssALL, गन्धर्व \msNa, \lk\lk \lk\ \msNb\oo 
\textbf{प्लवगेश्वरः}\lem \msCa\msCbpcorr\msCc\msNa\Ed, प्लगेश्वरः \msCbacorr, 
\lk\lk \lk\lk \lk\ \msNb, प्लवमेश्वरः \msNc}}% 

{\devanagarifont विपुलश्च महासत्त्व आस्यतां रत्न-आसने \thinspace{\dandab} \dontdisplaylinenum }%
     \var{{\devanagarifont \numemph\va \lem \mssALL, 
विपुलश्च समासत्व \msCb, 
\lk\lk \lk\lk \lk\lk सत्व \msNb}}% 
    \var{{\devanagarifont \numnoemph\vb\textbf{आस्यतां}\lem \mssALL, आस्यता \msCb\oo 
\textbf{॰आसने}\lem \mssCaCbCc\msNa, ॰शाशने \msNb\Ed, ॰आसनेः \msNc}}% 

%Verse 12:142

{\devanagarifont साधु भो विपुल श्रेष्ठ साधु भो विपुलं तपः {॥ १२:१४२॥} \veg\dontdisplaylinenum }%
     \var{{\devanagarifont \numnoemph\vc\textbf{साधु भो}\lem \mssALL, 
साधु हो \msCb, \lk\lk \lk\ \msNb}}% 
    \var{{\devanagarifont \numnoemph\vd\textbf{विपुलं तपः}\lem \msNa\msNb\Ed, \uncl{वि}\lac  पः \msCa, 
विपुलतपः \msCb\msCc\msNc}}% 

{\devanagarifont साधु भो विपुलप्राज्ञ साधु भो विपुलश्रिय \thinspace{\dandab} \dontdisplaylinenum }%
     \var{{\devanagarifont \numemph\vb\textbf{॰श्रिय}\lem \msCa\msNb\msNc, ॰प्रियः \msCb, ॰श्रियः \msCc\msNa\Ed}}% 

%Verse 12:143

{\devanagarifont तोषिताः स्म वयं सर्वे ब्रह्मविष्णुमहेश्वराः {॥ १२:१४३॥} \veg\dontdisplaylinenum }%
     \var{{\devanagarifont \numnoemph\vc\textbf{तोषिताः}\lem \mssALL, तोषिता \msNa\Ed}}% 

{\devanagarifont आदित्या वसवो रुद्राः साध्याश्विनौ मरुत्तथा \thinspace{\dandab} \dontdisplaylinenum }%
     \var{{\devanagarifont \numemph\va\textbf{रुद्राः}\lem \mssCaCbCc\msNa, रुद्रा \msNb\msNc\Ed}}% 
    \var{{\devanagarifont \numnoemph\vb\textbf{साध्याश्विनौ}\lem \msNb, साध्याश्विन्यौ \msCa\msCb\msNa, 
साध्याश्विन्यो \msCc\msNc, साध्या यक्षो \Ed\oo 
\textbf{मरुत्तथा}\lem \mssALL, 
मरुतस्तथा \msCc}}% 

%Verse 12:144

{\devanagarifont भुङ्क्ष्व भोगान्यथोत्साहं मम लोके यथासुखम् {॥ १२:१४४॥} \veg\dontdisplaylinenum }%
     \var{{\devanagarifont \numnoemph\vc\textbf{भुङ्क्ष्व}\lem \mssALL, भुक्त्वा \msNb, भुंक्ष \Ed\oo 
\textbf{भोगान्यथोत्साहं}\lem \mssALL, 
भोगा यथेत्साह \msCc 
भोगा यथोत्साहं \msNb}}% 
    \var{{\devanagarifont \numnoemph\vd\textbf{लोके}\lem \mssALL, लोक \msNb}}% 

{\devanagarifont इयं विमानकोटीनां तवार्थायोपकल्पिता \thinspace{\dandab} \dontdisplaylinenum }%
     \var{{\devanagarifont \numemph\va\textbf{॰कोटीनां}\lem \mssALL, ॰कोटीनि \msCc, ॰कोटीना \msNb}}% 
    \var{{\devanagarifont \numnoemph\vb\textbf{तवार्थायोप॰}\lem \msCa\msNa\msNc\Ed, तवायोपि॰ \msCb, 
त्वयार्थं याव॰ \msCc, तवार्थायोप्र॰ \msNb\oo 
\textbf{॰कल्पिता}\lem \msCa\msCb\msNa, ॰कल्पितं \msCc, 
॰कल्पि\lk\  \msNb\msNc, ॰कल्पितान् \Ed}}% 

{\devanagarifont सहस्राणां सहस्राणि अप्सरा कामरूपिणी  \danda\dontdisplaylinenum }%
     \var{{\devanagarifont \numnoemph\vc\textbf{सहस्राणां}\lem \mssALL, सहस्राणा \msCb}}% 
    \var{{\devanagarifont \numnoemph\vd\textbf{अप्सरा}\lem \mssALL, अप्सरो \msCc\oo 
\textbf{॰रूपिणी}\lem \mssALL, ॰रूपिणि \Ed}}% 

%Verse 12:145

{\devanagarifont तवार्थीयोपसर्पन्ति सर्वालंकारभूषिताः {॥ १२:१४५॥} \veg\dontdisplaylinenum }%
     \var{{\devanagarifont \numnoemph\ve\textbf{तवार्थीयो॰}\lem \msCa, तवार्थायो॰ \msCb\msNa\msNb\msNc, तंवार्थीयो॰ \msCc, 
तवार्थेयो॰ \Ed}}% 
    \var{{\devanagarifont \numnoemph\vf\textbf{॰सर्पन्ति}\lem \mssALL, ॰षप्यन्ति \msNc\oo 
\textbf{॰भूषिताः}\lem \mssALL, ॰भूषितः \msNa}}% 

{\devanagarifont यावत्कल्पसहस्राणि परार्धानि तपोधन \thinspace{\dandab} \dontdisplaylinenum }%
     \var{{\devanagarifont \numemph\va\textbf{परार्धानि}\lem \mssALL, 
पराणि \msCbacorr\oo 
\textbf{॰धन}\lem \mssALL, ॰धनाः \Ed}}% 

%Verse 12:146

{\devanagarifont यत्र यत्र प्रयासित्वं तत्र तत्रोपभुज्यताम् {॥ १२:१४६॥} \veg\dontdisplaylinenum }%
     \var{{\devanagarifont \numnoemph\vd\textbf{॰पभुज्यताम्}\lem \mssALL, ॰प्रभुज्यताम् \msNb}}% 

{\devanagarifont महेश्वर उवाच {\dandab}\dontdisplaylinenum  }%
 
{\devanagarifont इति श्रुत्वा वचस्तस्य विपुलो विपुलेक्षणः \thinspace{\danda} \dontdisplaylinenum }%
     \var{{\devanagarifont \numemph\vb\textbf{विपुलो}\lem \mssALL, \om\ \msCb, विपुले \msCc}}% 

%Verse 12:147

{\devanagarifont वेपमानो भयत्रस्त अश्रुपूर्णाकुलेक्षणः {॥ १२:१४७॥} \veg\dontdisplaylinenum }%
     \var{{\devanagarifont \numnoemph\vc\textbf{भयत्रस्त}\lem \Ed, भयस्तत्र \mssCaCbCc\msNa\msNb, 
भयस्त्रत्र \msNc}}% 
    \var{{\devanagarifont \numnoemph\vd\textbf{अश्रु॰}\lem \mssALL, अश्व॰ \msNc\oo 
\textbf{॰पूर्णा॰}\lem \mssALL, ॰पूर्ण्ण॰ \msNb}}% 

{\devanagarifont प्रणम्य शिरसा भूमौ प्रणिपत्य पुनः पुनः \thinspace{\dandab} \dontdisplaylinenum }%
     \var{{\devanagarifont \numemph\va\textbf{शिरसा}\lem \mssALL, शिर \msNbacorr}}% 

%Verse 12:148

{\devanagarifont उवाच मधुरं वाक्यं ब्रह्मलोकपितामहम् {॥ १२:१४८॥} \veg\dontdisplaylinenum }%
     \var{{\devanagarifont \numnoemph\vc\textbf{मधुरं}\lem \mssALL, मधुर॰ \msCb}}% 
    \var{{\devanagarifont \numnoemph\vd\textbf{॰लोक॰}\lem \mssALL, लोके \Ed}}% 

{\devanagarifont विपुल उवाच {\dandab}\dontdisplaylinenum  }%
 
{\devanagarifont भगवन्सर्वलोकेश सर्वलोकपितामह \thinspace{\danda} \dontdisplaylinenum }%
 
{\devanagarifont स्वप्नभूतमिवाश्चर्यं पश्यामि त्रिदशेश्वर  \danda\dontdisplaylinenum }%
     \var{{\devanagarifont \numemph\vc\textbf{स्वप्नभूतमिवा॰}\lem \mssALL, 
स्वप्नमितमिवा॰ \msCc}}% 

%Verse 12:149

{\devanagarifont स्मृतिभ्रंशश्च मे जातो बुद्धिर्जातान्धचेतना {॥ १२:१४९॥} \veg\dontdisplaylinenum }%
     \var{{\devanagarifont \numnoemph\vf \lem \mssCaCbCc, बुद्धिर्जान्धचेतना \msNaacorr, 
बुद्धिर्जातन्धचेतना \msNapcorr, बुद्धि जातन्धचेना \msNb, 
बुद्धि जातात्वचेतना \msNc, 
बुद्धिर्जातो ऽन्धचेतनः\thinspace{\devanagarifont ।} मूढो ऽहं त्वां कथं स्तौमि ज्ञानातीतं परात्परम्\thinspace{\devanagarifont ॥} \Ed}}% 

\ujvers\nemsloka {
{\devanagarifont तुभ्यं त्रैलोक्यबन्धो भव मम शरणं त्राहि संसारघोराद् }%
  \dontdisplaylinenum}    \var{{\devanagarifont \numemph\va\textbf{तुभ्यं}\lem \mssALL, तुभ्यंस् \msNb, नमस् \Ed\oo 
\textbf{त्रैलोक्य॰}\lem \mssALL, त्रेलोक्य॰ \msCb\oo 
\textbf{॰बन्धो}\lem \mssALL, ॰\uncl{वन्तो} \msNa\oo 
\textbf{॰घोराद्}\lem \corr, ॰घोरम् \msCa\msCc\msNb\Ed, ॰घोरात् \msCb, ॰घोरः \msNa, 
॰\uncl{घोरात}त् \msNc}}% 


\nemslokab

{\devanagarifont भीतो ऽहं गर्भवासाज्जरमरणभयात्त्राहि मां मोहबन्धात्  \danda\dontdisplaylinenum }%
     \var{{\devanagarifont \numnoemph\vb\textbf{॰साज्जर॰}\lem \mssALL, 
॰सा जर॰ \msCc, ॰साज्जनु॰ \Ed\oo 
\textbf{॰मरण॰}\lem \mssALL, ॰ण॰ \msNbacorr\oo 
\textbf{॰भयात्}\lem \Ed, भयं \mssCaCbCc\msNa\msNb\msNc}}% 

\nemslokac

{\devanagarifont नित्यं रोगाधिवासमनियतवपुषं त्राहि मां कालपाशात् }%
  \dontdisplaylinenum    \var{{\devanagarifont \numnoemph\vc\textbf{नित्यं}\lem \mssALL, नित्य॰ \msCb\ \unmetr\oo 
\textbf{रोगा॰}\lem \mssALL, ॰रागा॰ \Ed\oo 
\textbf{॰वासमनियत॰}\lem \mssALL, ॰वासमतियत॰ \msCb, 
॰वासंमनियत॰ \msNa\oo 
\textbf{॰वपुषं त्राहि मां}\lem \mssALL, 
॰\uncl{वपुष त्राहि मा} \msCb\oo 
\textbf{कालपाशात्}\lem \mssALL, 
कापाशात् \msNaacorr, कालपाशान् \msNb}}% 

%Verse 12:150


\nemslokad

{\devanagarifont तिर्यं चान्योन्यभक्षं बहुयुगशतशस्त्राहि मोहान्धकारात् {॥ १२:१५०॥} \veg\dontdisplaylinenum }%
     \var{{\devanagarifont \numnoemph\vd\textbf{तिर्यं चान्योन्यभक्षं}\lem \mssALL, 
तिर्यं चान्यान्यभक्षं \msNb, 
तिर्यश्चान्योन्यभक्षं \Ed\oo 
\textbf{॰शतशस्त्राहि}\lem \mssALL, 
॰सतस त्राहि \msCc}}% 

\ujvers\nemsloka {
{\devanagarifont श्रुत्वैवोवाच ब्रह्मा विपुलमति पुनर्मानयित्वा यथावद् }%
  \dontdisplaylinenum}    \var{{\devanagarifont \numemph\va\textbf{श्रुत्वैवोवाच}\lem \mssALL, श्रुत्वैव वाच \Ed\oo 
\textbf{॰मति}\lem \msCc\Ed, ॰मतिः \msCa\msCb\msNa\msNb\msNc\ \unmetr\oo 
\textbf{मानयित्वा}\lem \mssALL, माणयित्वा \msNc, मानयंवा \Ed\oo 
\textbf{यथावद्}\lem \corr, यथावत् \mssCaCbCc\msNapcorr\msNb\msNc\Ed, 
वत् \msNaacorr}}% 


\nemslokab

{\devanagarifont आहूतसम्प्लवान्ते भविष्यसि तव मे जन्मलोभो न भूयः  \danda\dontdisplaylinenum }%
     \var{{\devanagarifont \numnoemph\vb\textbf{आहूत}\lem \mssALL, आभूत \Ed\oo 
\textbf{सम्प्लवान्ते}\lem \msCc, सम्प्लवन्ते \msCa\msCb\msNa\msNb\Ed, 
संप्लवंन्ते \msNc\oo 
\textbf{भविष्यसि}\lem \mssALL, 
भविष्य \msCc, अविपलि \Ed\oo 
\textbf{मे जन्मलोभो न}\lem \mssCaCbCc\msNa, 
मे जन्मलाभो न \msNb\msNc, यजन्मलाभानु \Ed\oo 
\textbf{भूयः}\lem \mssALL, भूय \msNc}}% 

\nemslokac

{\devanagarifont गर्भावासं न च त्वन्न च पुनमरणं क्लेशमायासपूर्णं }%
  \dontdisplaylinenum    \var{{\devanagarifont \numnoemph\vc\textbf{॰वासं न च त्वन्न}\lem \msCa\msNa\msNb\msNc, ॰वासन्न \msCb, 
॰वासा न च त्वन्न \msCc, ॰वासानुबन्धं न \Ed\oo 
\textbf{पुनमरणं}\lem \msCc\Ed, पुनर्मरणं \msCa\msNa\msNb\msNc\ \unmetr, 
पुनर्मण \msCb\oo 
\textbf{॰पूर्णम्}\lem \mssALL, ॰पूर्ण्ण \msCc}}% 

%Verse 12:151


\nemslokad

{\devanagarifont छित्त्वा मोहान्धशत्रुं व्रजसि च परमं ब्रह्मभूयत्वमेषि {॥ १२:१५१॥} \veg\dontdisplaylinenum }%
     \var{{\devanagarifont \numnoemph\vd\textbf{॰शत्रुं}\lem \mssALL, ॰शत्रु \msCb\msCc\oo 
\textbf{परमं}\lem \mssALL, परम \msNb}}% 
    \paral{{\devanagarifontsmall \vd {\englishfont cf.\ Manu 1.98cd:} स हि धर्मार्थमुत्पन्नो ब्रह्मभूयाय कल्पते
                 {\englishfont and Manu 12.102cd:} इहैव लोके तिष्ठन्स ब्रह्मभूयाय कल्पते }}

\vers


{\devanagarifont महेश्वर उवाच {\dandab}\dontdisplaylinenum  }%
 
{\devanagarifont ब्रह्मणा एवमुक्तस्तु विष्णुना प्रभविष्णुना \thinspace{\danda} \dontdisplaylinenum }%
     \var{{\devanagarifont \numemph\vb\textbf{विष्णुना}\lem \mssALL, \om\ \msCb, विष्णुनात् \msCc}}% 

%Verse 12:152

{\devanagarifont एवं भवतु भद्रं वो यथोवाच पितामहः {॥ १२:१५२॥} \veg\dontdisplaylinenum }%
     \var{{\devanagarifont \numnoemph\vd\textbf{॰महः}\lem \msCa\msNc\Ed, ॰मह \msCb\msCc\msNa\msNb}}% 

{\devanagarifont इन्द्रेण रविणा चैव सोमेन च पुनः पुनः \thinspace{\dandab} \dontdisplaylinenum }%
     \var{{\devanagarifont \numemph\va\textbf{रविणा}\lem \mssALL, रविना \msCc, शशिना \Ed}}% 
    \var{{\devanagarifont \numnoemph\vb\textbf{सोमेन}\lem \mssALL, सूर्येण \Ed\oo 
\textbf{पुनः पुनः}\lem \mssALL, पुन पुनः \msCb\ \unmetr, 
पुन च पुनः पुनः \msCc}}% 

%Verse 12:153

{\devanagarifont साध्यादित्यैर्मरुद्रुद्रैर्विश्वेभिर्वसवैस्तथा {॥ १२:१५३॥} \veg\dontdisplaylinenum }%
     \var{{\devanagarifont \numnoemph\vc\textbf{॰दित्यैर्म॰}\lem \mssALL, ॰दित्यै म॰ \msCc}}% 
    \var{{\devanagarifont \numnoemph\vcd\textbf{॰रुद्रुद्रैर्विश्वेभिर्}\lem \Ed, ॰रुद्रुद्रैर्विश्वेश्वि \msCa\msNa, 
॰रुद्रुद्रै विश्वाश्वि \msCb, ॰रुद्रुद्रै विश्वेश्वि \msCc, 
॰रुद्रै विश्वे\lk\ \msNb, ॰रुद्रैर्विश्वेश्वि \msNc}}% 

{\devanagarifont अहो तपःफलं दिव्यं विपुलस्य महात्मनः \thinspace{\dandab} \dontdisplaylinenum }%
 
%Verse 12:154

{\devanagarifont स्वशरीरो दिवं प्राप्तः श्रद्धयातिथिपूजया {॥ १२:१५४॥} \veg\dontdisplaylinenum }%
     \var{{\devanagarifont \numemph\vc\textbf{स्वशरीरो}\lem \eme, स्वशरीरं \msCa\msNa\msNb\msNc, शशरीरो \msCb, 
स्वशरीर \msCc, सशरीरं \Ed\oo 
\textbf{प्राप्तः}\lem \msCb\msCc, प्राप्तं \msCa\msNa\msNb\msNc\Ed}}% 
    \var{{\devanagarifont \numnoemph\vd\textbf{॰पूजया}\lem \mssALL, ॰पूजनात् \Ed}}% 

{\devanagarifont एवमादीन्यनेकानि विपुले परिकीर्तितम् \thinspace{\dandab} \dontdisplaylinenum }%
     \var{{\devanagarifont \numemph\vb\textbf{॰नेकानि}\lem \mssALL, ॰नेनेकानि \msNb}}% 

%Verse 12:155

{\devanagarifont ब्रह्माणं पुनरेवाह विष्णुर्विश्वजगत्प्रभुः {॥ १२:१५५॥} \veg\dontdisplaylinenum }%
     \var{{\devanagarifont \numnoemph\vc\textbf{ब्रह्माणं}\lem \mssALL, ब्राह्मणः \msCb, 
ब्रह्मणं \msCc}}% 
    \var{{\devanagarifont \numnoemph\vd\textbf{विष्णुर्वि॰}\lem \mssALL, विष्णु वि॰ \msCc\oo 
\textbf{॰जगत्प्रभुः}\lem \mssALL, ॰जगत्प्रभु \msCc}}% 

{\devanagarifont 
\jump
\begin{center}
\ketdanda~इति वृषसारसंग्रहे विपुलोपाख्यानो नामाध्यायो द्वादशमः~\ketdanda
\end{center}
\dontdisplaylinenum\vers  }%
     \var{{\devanagarifont \numnoemph{\englishfont \Colo:}\textbf{वृषसार॰}\lem \mssALL, वृष॰ \msNb\oo 
\textbf{॰ख्यानो नामाध्यायो द्वादशमः}\lem \mssALL, 
॰ख्या\uncl{न ना}माध्यायो द्वादश \msNc, 
॰ख्यानो नाम द्वादशो ऽध्यायः \Ed}}% 
