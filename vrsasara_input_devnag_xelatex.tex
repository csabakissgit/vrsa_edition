\fejno=0\versno=0
\centerline{\Huge\devanagarifontbold वृषसारसंग्रहः  }

 
{\vrule depth10pt width0pt}
\versno=0\fejno=1
\thispagestyle{empty}

\centerline{\Large\devanagarifontbold [   प्रथमो ऽध्यायः  ]}{\vrule depth10pt width0pt} \fancyhead[CO]{{\footnotesize\devanagarifont वृषसारसंग्रहे  }}
\fancyhead[CE]{{\footnotesize\devanagarifont प्रथमो ऽध्यायः  }}
\fancyhead[LE]{}
\fancyhead[RE]{}
\fancyhead[LO]{}
\fancyhead[RO]{}
\szam\bek


\vers



\alalfejezet{स्तुतिः}
\ujvers\nemsloka {
{\devanagarifont अनादिमध्यान्तमनन्तपारं }%
  \dontdisplaylinenum}    \var{{\devanagarifont \numemph\va\textbf{॰न्तमनन्त॰}\lem \mssALL, 
॰न्तमन्त॰ \msCbacorr\oo 
\textbf{॰पारं}\lem \mssCaCbCc\msNc\msM\msPaperA\msPaperC\Ed, ॰पारगं \msNa\msNb\msNd\msKOa}}% 
    \paral{{\devanagarifontsmall \va {\englishfont \compare\ \SDHU\ 10.6:}
                 आदिमध्यान्तनिर्मुक्तः स्वभावविमलः प्रभुः\thinspace{\devanagarifontsmall ।}
                 सर्वज्ञः परिपूर्णश्च शिवो ज्ञेयः शिवागमे\thinspace{\devanagarifontsmall ॥} }}


\nemslokab

{\devanagarifont सुसूक्ष्ममव्यक्तजगत्सुसारम्  \danda\dontdisplaylinenum }%
     \var{{\devanagarifont \numnoemph\vb\textbf{सुसूक्ष्म॰}\lem \mssALL, 
शुसुक्ष्म॰ \msCc\oo 
\textbf{॰व्यक्त॰}\lem \mssALL, ॰व्य॰ \msKOa\oo 
\textbf{॰जगत्सुसारम्}\lem \msCa\msCb\msNa\msNc\msM\msKOa\msPaperA\msPaperC\Ed, ॰जगशुसारं \msCc, 
॰जगत्सुरासुरं \msNb, 
॰जगतसुसारम् \msNd}}% 

\nemslokac

{\devanagarifont हरीन्द्रब्रह्मादिभिरासमग्रं }%
  \dontdisplaylinenum    \var{{\devanagarifont \numnoemph\vc\textbf{हरी॰}\lem \mssALL, हरीं \msKOa\oo 
\textbf{॰भिरासमग्रं}\lem \mssALL, 
॰भिर्यत्समग्रं \msM\ \unmetr, 
॰भिरोसमग्रं \msPaperC}}% 

%Verse 1:1


\nemslokad

{\devanagarifont प्रणम्य वक्ष्ये वृषसारसंग्रहम् {॥१:१॥} \veg\dontdisplaylinenum }%
     \var{{\devanagarifont \numnoemph\vd\textbf{वृष॰}\lem \mssALL, 
॰वृषो \msCaacorr}}% 
    \lacuna{\devanagarifontsmall {\englishfont Witnesses used for this chapter:       \msCa\ ff.\thinspace 193v--195v,
                                                        \msCb\ ff.\thinspace 201v--203v,
                                                        \msCc\ ff.\thinspace 267r--270r,
                                                        \msNa\ ff.\thinspace 1v--3v,
                                                        \msNb\ exp.\thinspace 44, 43 lower and then upper leaf
                                                                              (1.62cd--2.22 are missing),
                                                        \msNc\  ff.\thinspace 209v--211v,
                                                        \msNd\  ff.\thinspace 227v--229v 
                                                                                (collated only up to 1.15ab),
                                                        \msM\   ff.\thinspace 1r--3v,
                                                        \msKOa\ ff.\thinspace 1v--4r 
                                                                                (collated only up to 1.16),
                                                        \msPaperA\ ff.\thinspace 204r--206r,
                                                        \msPaperC\ ff.\thinspace 206r--209r 
                                                                                (collated only up to 1.15),
                                                        \Ed\ pp.\thinspace 580--585;
                                                        \mssCaCbCc\ = \msCa + \msCb + \msCc} }%
  

\alalfejezet{जनमेजयवैशम्पायनसंवादः}
\vers


{\devanagarifont शतसाहस्रिकं ग्रन्थं सहस्राध्यायमुत्तमम् \thinspace{\dandab} \dontdisplaylinenum }%
     \var{{\devanagarifont \numemph\va\textbf{॰स्रिकं}\lem \mssALL, 
॰स्रकं \msPaperA\oo 
\textbf{ग्रन्थं}\lem \mssALL, 
ग्रंथ \msKOa}}% 
    \var{{\devanagarifont \numnoemph\vb\textbf{सहस्राध्यायमु॰}\lem \mssALL, 
सहश्रध्यायमु॰ \msCc, 
सहस्राध्यायरु॰ \Ed}}% 

%Verse 1:2

{\devanagarifont पर्व चास्य शतं पूर्णं श्रुत्वा भारतसंहिताम् {॥१:२॥} \veg\dontdisplaylinenum }%
     \var{{\devanagarifont \numnoemph\vc\textbf{पर्व चास्य}\lem \msCa\msNa\msNb\msNc\msMpcorr, पर्वञ्चास्य \msCb, 
पर्वमस्य \msCc\msNd\msMacorr\msPaperA\msPaperC\Ed, पूर्व चास्य \msKOa\oo 
\textbf{शतं पूर्णं}\lem \mssALL, 
त \msCc, शतं पूर्ण्ण \msKOa}}% 
    \var{{\devanagarifont \numnoemph\vd\textbf{श्रुत्वा}\lem \mssALL, 
श्रद्धा \msCb\oo 
\textbf{भारतसंहिताम्}\lem \msCa\msCb\msNa\msNb\msNc\msM\msKOa, 
भारसंहिता \msCc, भारतसंहितं \msNd, 
नारदसंहिताम् \msPaperA\msPaperC\Ed}}% 
    \paral{{\devanagarifontsmall \vc {\englishfont \compare\ \MBH\ 1.2.70ab:} एतत्पर्वशतं पूर्णं व्यासेनोक्तं महात्मना }}

\vers


{\devanagarifont अतृप्तः पुन पप्रच्छ वैशम्पायनमेव हि \thinspace{\dandab} \dontdisplaylinenum }%
     \var{{\devanagarifont \numemph\va \lem \eme, 
अ\uncl{तृप्तः पु}\lk\lk प्रच्छ \msCa, 
अतृप्तः पुनः पप्रच्छ \msCb\msNa\msNb\msNc, 
अतृप्तः पुनरप्रच्छे \msCc, 
अतृप्तः पुन पःप्रच्छ \msNd, 
अतृप्तः पुनः पपृच्छ \msM, 
पप्रच्छ पुनरतृप्तो \msKOa, 
अतृप्ताः पुनः पप्रेच्छ \msPaperA, 
अतृप्त पुनः पप्रच्छ \msPaperC, 
अतृप्ता पुनः पप्रच्छ \Ed}}% 
    \var{{\devanagarifont \numnoemph\vb\textbf{वैशम्पायन॰}\lem \mssALL, 
वेसम्पायन॰ \msCc}}% 

%Verse 1:3

{\devanagarifont जनमेजयेन यत्पूर्वं तच्छृणु त्वमतन्द्रितम् {॥१:३॥} \veg\dontdisplaylinenum }%
     \var{{\devanagarifont \numnoemph\vc \lem \msCapcorr\msCb\msNc\msNd\msPaperA\msPaperC\Ed, 
जनमेजये यत्पूर्वं \msCaacorr, 
जन्मेजयेन यम्पूर्वं \msCc, 
जनमेजयेन यत्पूर्व \msNa, 
जनमेजयेन यत्पू\uncl{र्व} \msNb, 
जन्मेजयेण यत्पूर्वं \msM, 
जन्मेजयेन य\lac\ \msKOa}}% 
    \var{{\devanagarifont \numnoemph\vd\textbf{तच्छृणु त्वम॰}\lem \msCa\msCb\msNa\msNc\msM\msPaperA\msPaperC\Ed, 
तच्छृण त्वम॰ \msCc, \lac\  \msNb, तच्छृणु स्वम॰ \msNd, 
त शृणु त्वम॰ \msKOa\oo 
\textbf{॰तन्द्रितम्}\lem \msCa\msCb\msNc\msNd\msM\msKOa\msPaperA\msPaperC\Ed, ॰तन्द्रितः \msCc\msNa, 
\lac\  \msNb}}% 

{\devanagarifont जनमेजय उवाच {\dandab}\dontdisplaylinenum  }%
     \var{{\devanagarifont \numemph\vo\textbf{जनमेजय}\lem \mssALL, 
जन्मेजय \msCc}}% 

{\devanagarifont भगवन्सर्वधर्मज्ञ सर्वशास्त्रविशारद \thinspace{\danda} \dontdisplaylinenum }%
     \var{{\devanagarifont \numnoemph\va\textbf{भगवन्स॰}\lem \msCa\msCb\msNa\msNb\msNc\msKOa\msPaperA\msPaperC\Ed, 
भचावं स॰ \msCc, भगव स॰ \msNd, 
भगवं स॰ \msM\oo 
\textbf{॰धर्मज्ञ}\lem \mssALL, 
॰ज्ञ \msNa, ॰धर्मज्ञः \msNd}}% 
    \var{{\devanagarifont \numnoemph\vb\textbf{॰विशारद}\lem \msCa\msNb\msNc\msNd\msPaperA, 
॰विसारदः \msCb\msCc\msNa\msKOa\msPaperC\Ed, ॰विशारदम् \msM}}% 
    \paral{{\devanagarifontsmall \vab {\englishfont = \MBH\ 13.112.9ab} }}

%Verse 1:4

{\devanagarifont अस्ति धर्मं परं गुह्यं संसारार्णवतारणम् {॥१:४॥} \veg\dontdisplaylinenum }%
     \var{{\devanagarifont \numnoemph\vc\textbf{अस्ति धर्मं}\lem \msCa\msNa\msNb\msNc\msPaperA\msPaperC\Ed, अस्ति धर्मः \msCb, 
अस्ति धर्म \msCc\msM\msKOa, अधर्म \msNd\oo 
\textbf{परं गुह्यं}\lem \msCa\msNb\msNd\msM\msKOa\msPaperA\msPaperC\Ed, 
परो गुह्य \msCb, परं गुह्य \msCc\msNa, 
परगुह्यं \msNc}}% 
    \var{{\devanagarifont \numnoemph\vd\textbf{॰तारणम्}\lem \mssALL, 
॰तारणा \msKOa}}% 

{\devanagarifont द्वैपायनमुखोद्गीर्णं धर्मं वा यद्द्विजोत्तम \thinspace{\dandab} \dontdisplaylinenum }%
     \var{{\devanagarifont \numemph\va\textbf{द्वैपायन॰}\lem \mssALL, 
द्वेपायन॰ \msCc, 
वैसांपायन॰ \msKOa\oo 
\textbf{॰मुखोद्गीर्णं}\lem \msCa\msCb\msNa\msNb\msNc\msPaperA\msPaperC, 
॰मुखोद्गीर्ण \msCc\msKOa, 
॰मुद्गीर्ण्ण \msNd, 
मुखं गीर्ण्णं \msMacorr, 
मु\uncl{खां} गीर्ण्णं \msMpcorr, 
मुखाद्गीर्णं \Ed}}% 
    \var{{\devanagarifont \numnoemph\vb\textbf{धर्मं वा यद्द्वि॰}\lem \msCa\msNa\msNb\msNc\msPaperA\msPaperC\Ed, 
धर्मं यत्तद्द्वि॰ \msCb, 
धर्मवत्य द्वि॰ \msCc\msKOa, धर्म वा यद्द्वि॰ \msNd, 
धर्मवाक्यं द्वि॰ \msM\oo 
\textbf{॰त्तम}\lem \mssALL, 
॰त्तमः \msCc, ॰तमः \msM}}% 

%Verse 1:5

{\devanagarifont कथयस्व हि मे तृप्तिं कुरु यत्नात्तपोधन {॥१:५॥} \veg\dontdisplaylinenum }%
     \var{{\devanagarifont \numnoemph\vc\textbf{हि मे तृप्तिं}\lem \mssCaCbCc\msNa\msNb\msNc\msPaperA\msPaperC\Ed, 
हि मे तृप्ति \msNd\msKOa, 
प्रसादेन \msM}}% 
    \var{{\devanagarifont \numnoemph\vd\textbf{यत्नात्तपोधन}\lem \msCb\msNa\msNb\msNc\msPaperA\msPaperC\Ed, 
यन्नात्त\lk\lk न \msCa, 
यत्ना तपोधनः \msCc, यत्ना तपोधन \msNd, 
यत्नन्तपोधन \msM, यंनात्त॰ \msKOa}}% 

{\devanagarifont वैशम्पायन उवाच {\dandab}\dontdisplaylinenum  }%
     \var{{\devanagarifont \numemph\vo\textbf{वैशम्पायन उवाच}\lem \mssALL, 
\om\ \msMacorr, वै\thinspace{\devanagarifont ॥} वैशम्पायन \msPaperC}}% 

{\devanagarifont शृणु राजन्नवहितो धर्माख्यानमनुत्तमम् \thinspace{\danda} \dontdisplaylinenum }%
     \var{{\devanagarifont \numnoemph\va\textbf{राजन्न॰}\lem \mssALL, 
राजंन॰ \msNd, राजन॰ \msM\oo 
\textbf{॰हितो}\lem \mssALL, 
॰हितं \msPaperA}}% 
    \var{{\devanagarifont \numnoemph\vb\textbf{॰ख्यानमनुत्तमम्}\lem \msCa\msNa\msNb\msNc\msM\Ed, ॰ख्यानमुत्तमम् \msCb, 
॰ख्यानमुतमम् \msCc, ॰धर्मव्याख्यानमुत्तमं \msNd\ \hypermetr, 
॰ख\lac मनुत्तमं \msKOa, 
॰ख्यानमनुत्तमः \msPaperA, 
॰ख्यानमुत्तमः \msPaperC}}% 

%Verse 1:6

{\devanagarifont व्यासानुग्रहसम्प्राप्तं गुह्यधर्मं शृणोतु मे {॥१:६॥} \veg\dontdisplaylinenum }%
     \var{{\devanagarifont \numnoemph\vc\textbf{॰प्राप्तं}\lem \mssALL, 
॰प्राप्त \msCc}}% 
    \var{{\devanagarifont \numnoemph\vd\textbf{॰धर्मं}\lem \mssALL, 
॰र्मं \msCc, ॰धर्म \msKOa\oo 
\textbf{शृणोतु}\lem \mssALL, 
शृणोत \msCc\oo 
\textbf{मे}\lem \mssALL, 
मै \msCb}}% 

{\devanagarifont अनर्थयज्ञकर्तारं तपोव्रतपरायणम् \thinspace{\dandab} \dontdisplaylinenum }%
     \var{{\devanagarifont \numemph\va\textbf{॰कर्तारं}\lem \mssALL, 
॰कर्त्तन्तं \msNb}}% 
    \var{{\devanagarifont \numnoemph\vb\textbf{॰व्रत॰}\lem \mssALL, 
॰प्रत॰ \msM\oo 
\textbf{॰यणम्}\lem \msCa\msCb\msNb\msM\msKOa\msPaperA\msPaperC\Ed, 
॰यन \msCc, ॰यणः \msNa, 
॰यनं \msNc, ॰\uncl{यणं} \msNd}}% 

%Verse 1:7

{\devanagarifont शीलशौचसमाचारं सर्वभूतदयापरम् {॥१:७॥} \veg\dontdisplaylinenum }%
     \var{{\devanagarifont \numnoemph\vc\textbf{॰चारं}\lem \mssALL, 
॰चार \msKOa}}% 
    \var{{\devanagarifont \numnoemph\vd\textbf{॰परम्}\lem \msCa\msCb\msNa\msNc\msM\msPaperA\msPaperC\Ed, ॰न्वितम् \msCc\msNd\msKOa, 
॰\uncl{प}रं \msNb}}% 

{\devanagarifont जिज्ञासनार्थं प्रश्नैकं विष्णुना प्रभविष्णुना \thinspace{\dandab} \dontdisplaylinenum }%
     \var{{\devanagarifont \numemph\va\textbf{॰र्थं प्रश्नैकं}\lem \msCb\msNa\msNb\msNc, ॰र्थं प्रश्नेकं \msCa\msNd, 
॰र्थप्रश्नेकम् \msCc\msPaperA\msPaperC\Ed, ॰र्थप्रश्चैकं \msM, 
॰थप्रश्नैक \msKOa}}% 
    \var{{\devanagarifont \numnoemph\vb\textbf{प्रभ॰}\lem \mssALL, 
प्रभु॰ \msCc, प्राभ॰ \msNc}}% 

%Verse 1:8

{\devanagarifont द्विजरूपधरो भूत्वा पप्रच्छ विनयान्वितः {॥१:८॥} \veg\dontdisplaylinenum }%
     \var{{\devanagarifont \numnoemph\vc\textbf{॰धरो}\lem \mssALL, 
॰\lk रो \msCa, ॰धरा \msNb}}% 
    \var{{\devanagarifont \numnoemph\vd\textbf{॰न्वितः}\lem \msCa\msCb\msNa\msNb\msNc\msKOa\msPaperA\msPaperC\Ed, 
॰न्वितं \msCc\msNd\msM}}% 


\alalfejezet{ब्रह्मविद्या}
{\devanagarifont [विगतराग उवाच {\dandab}\dontdisplaylinenum  ] }%
 
{\devanagarifont ब्रह्मविद्या कथं ज्ञेया रूपवर्णविवर्जिता \thinspace{\danda} \dontdisplaylinenum }%
     \var{{\devanagarifont \numemph\va\textbf{कथं}\lem \mssALL, कथ \msKOa\oo 
\textbf{ज्ञेया}\lem \msCa\msNa\msNb\msNc\msM\msKOa\msPaperA\msPaperC, 
ज्ञेयं \msCb\msCc, ज्ञेय \msNd, भूयो \Ed}}% 
    \var{{\devanagarifont \numnoemph\vb\textbf{॰वर्ण॰}\lem \mssALL, 
॰वर्णा॰ \Ed\oo 
\textbf{॰वर्जिता}\lem \msCa\msCb\msNa\msNb\msNd\msM\msPaperA\msPaperC\Ed, 
॰वर्जितं \msCc, ॰वर्जिताः \msNc, \lac ता \msKOa}}% 

%Verse 1:9

{\devanagarifont स्वरव्यञ्जननिर्मुक्तमक्षरं किमु तत्परम् {॥१:९॥} \veg\dontdisplaylinenum }%
     \var{{\devanagarifont \numnoemph\vc\textbf{॰व्यञ्जन॰}\lem \mssALL, 
॰व्यज्जन॰ \Ed}}% 
    \var{{\devanagarifont \numnoemph\vcd\textbf{॰मुक्तमक्ष॰}\lem \msCa\msCc\msNa\msNb\msNc\msPaperC\Ed, ॰मुक्त अक्ष॰ \msCb\msKOa, 
॰मुक्तं अख॰ \msNd, ॰मुक्तं अक्ष॰ \msM, ॰म्मुक्तंमक्ष॰ \msPaperA}}% 
    \var{{\devanagarifont \numnoemph\vd\textbf{किमु तत्परम्}\lem \msCa\msNa\msNc\msKOa\msPaperA\msPaperC\Ed, 
किमतः परम् \msCb\msCc, 
किमतत्परं \msNb\msNd\msM}}% 

{\devanagarifont अनर्थयज्ञ उवाच {\dandab}\dontdisplaylinenum  }%
 
{\devanagarifont अनुच्चार्यमसन्दिग्धमविच्छिन्नमनाकुलम् \thinspace{\danda} \dontdisplaylinenum }%
     \var{{\devanagarifont \numemph\va\textbf{अनुच्चार्य॰}\lem \msCa\msCb\msNa\msNb\msM\msPaperA\msPaperC\Ed, 
अनुचार्य॰ \msCc\msNc\msNd, 
अन्त्रचाय॰ \msKOa}}% 
    \var{{\devanagarifont \numnoemph\vab\textbf{॰सन्दिग्धमविच्छिन्नमनाकुलम्}\lem \msCa\msCb\msNa\msNc\msNd\msM\msPaperA\msPaperC\Ed, 
॰विच्छिन्नसन्दिग्धमनाकुन \msCc, ॰सन्दिग्धमनच्छिन्नमनाकुलम् \msNb, 
॰सन्दिग्धमविच्छिनमनाकुलं \msKOa}}% 

%Verse 1:10

{\devanagarifont निर्मलं सर्वगं सूक्ष्ममक्षरं किमतः परम् {॥१:१०॥} \veg\dontdisplaylinenum }%
     \var{{\devanagarifont \numnoemph\vc\textbf{॰गं}\lem \mssALL, ॰ग \msKOa}}% 
    \var{{\devanagarifont \numnoemph\vc\textbf{॰क्षरं किमतः परम्}\lem \msCb\msM, ॰क्षरं किमु तत्परम् \msCa\msNa\msNb\msNc\Ed, 
॰क्षरं किमतत्परं \msCc\msNd\msPaperC, 
॰क्षर किमतः परं \msKOa, 
॰क्षराङ्कमतत्परं \msPaperA}}% 


\alalfejezet{कालपाशः}
{\devanagarifont विगतराग उवाच {\dandab}\dontdisplaylinenum  }%
     \var{{\devanagarifont \numemph\vo\textbf{॰राग उवाच}\lem \mssALL, ॰रागोवाच \msNd}}% 

{\devanagarifont देही देहे क्षयं याते भूजलाग्निशिवादिभिः \thinspace{\danda} \dontdisplaylinenum }%
     \var{{\devanagarifont \numnoemph\va\textbf{देहे क्ष॰}\lem \msCa\msCc\msNc, देहात्क्ष॰ \msCb, 
देहक्ष॰ \msNa\msNb\msNd\msM\msKOa\msPaperA\msPaperC\Ed\oo 
\textbf{याते}\lem \mssALL, यान्ते \msNd}}% 
    \var{{\devanagarifont \numnoemph\vb\textbf{॰जलाग्निशिवादिभिः}\lem \msCa\msCb\msNa\msNb\msNc\msM\msPaperA\msPaperC\Ed, 
॰जलाग्निशिवादिभि \msCc, 
॰जलाग्निं शि\lk दिभि \msNd, ॰जालादिशिवादिभिः \msKOa}}% 
    \paral{{\devanagarifontsmall \vb {\englishfont \compare\ \KURMP\ 2.23.74:} 
                 अथ कश्चित्प्रमादेन म्रियते ऽग्निविषादिभिः\thinspace{\devanagarifontsmall ।} 
                 तस्याशौचं विधातव्यं कार्यं चैवोदकादिकम्\thinspace{\devanagarifontsmall ॥} }}

%Verse 1:11

{\devanagarifont यमदूतैः कथं नीतो निरालम्बो निरञ्जनः {॥१:११॥} \veg\dontdisplaylinenum }%
     \var{{\devanagarifont \numnoemph\vc\textbf{॰दूतैः}\lem \mssALL, 
॰दूते \msCc\msNd\oo 
\textbf{कथं}\lem \mssALL, 
कथ \msKOa\oo 
\textbf{नीतो}\lem \msCa\msCb\msNa\msNb\msNc\msNd, नीत्वा \msCc, नीतः \msM, नीते \msKOa, 
नीता \msPaperA\msPaperC\Ed}}% 
    \var{{\devanagarifont \numnoemph\vd\textbf{निरालम्बो}\lem \mssALL, 
निरोलया \msPaperA, निरोरैन्वो \msPaperC}}% 

{\devanagarifont कालपाशैः कथं बद्धो निर्देहश्च कथं व्रजेत् \thinspace{\dandab} \dontdisplaylinenum }%
     \var{{\devanagarifont \numemph\va\textbf{॰पाशैः}\lem \mssALL, 
॰पाशे \msCc, ॰पाशै \msNd\oo 
\textbf{बद्धो}\lem \mssALL, 
ब\uncl{द्धो} \msCb, बद्ध \msNd}}% 
    \var{{\devanagarifont \numnoemph\vb\textbf{निर्देहश्च}\lem \msCa\msCb\msNa\msNb\msNc\msMpcorr\msPaperA\msPaperC\Ed, 
निर्दहः स \msCc, निर्देहस्य \msNd, 
निर्देहन्म \msMacorr, निदेहश्च \msKOa\oo 
\textbf{व्रजेत्}\lem \mssALL, भवेत् \msNb}}% 

{\devanagarifont स्वर्गं वा स कथं याति निर्देहो बहुधर्मकृत्  \danda\dontdisplaylinenum }%
     \var{{\devanagarifont \numnoemph\vc\textbf{स्वर्गं}\lem \msCa\msCb\msNa\msNb\msNc\msPaperA\msPaperC\Ed, 
स्वर्ग \msCc\msNd\msM, स्वागं \msKOa\oo 
\textbf{स}\lem \mssALL, 
सं \msNb\msM\oo 
\textbf{याति}\lem \msNa\msNb\msNc\msNd\msM\msKOa\msPaperA\msPaperC, 
यान्ति \mssCaCbCc\Ed}}% 
    \var{{\devanagarifont \numnoemph\vd\textbf{निर्देहो}\lem \mssALL, 
निदेहो \msKOa}}% 

%Verse 1:12

{\devanagarifont एतन्मे संशयं ब्रूहि ज्ञातुमिच्छामि तत्त्वतः {॥१:१२॥} \veg\dontdisplaylinenum }%
     \var{{\devanagarifont \numnoemph\ve\textbf{एतन्मे संशयं}\lem \mssCaCbCc\msNc\msM\msPaperA\msPaperC\Ed, 
एतन्मे संशये \msNa, एतन्मे संशयो \msNb\msNd, 
एवं विस्मयसंसय \msKOa}}% 
    \var{{\devanagarifont \numnoemph\vf\textbf{॰तुमि$\-$च्छामि}\lem \mssALL, 
॰तुमि \msCb}}% 

{\devanagarifont अनर्थयज्ञ उवाच {\dandab}\dontdisplaylinenum  }%
     \var{{\devanagarifont \numemph\vo\textbf{अनर्थयज्ञ उवाच}\lem \mssALL, 
\om\ \msNaacorr}}% 

{\devanagarifont अतिसंशयकष्टं ते पृष्टो ऽहं द्विजसत्तम \thinspace{\danda} \dontdisplaylinenum }%
     \var{{\devanagarifont \numnoemph\va \lem \msCb\msNa\msNb\msNc\msMpcorr\msPaperC, 
अतिशंस\uncl{य}कष्टन्ते \msCa, 
अतिशंसयक$\-$ष्टम्मे \msCc\msMacorr\Ed, 
अतिसंशयकष्टो मो \msNd, 
अतिसंसयकष्टञ्च \msKOa, 
अतिसंसयकष्ट\lk न्ते पा \msPaperA}}% 
    \var{{\devanagarifont \numnoemph\vb\textbf{द्विजसत्तम}\lem \msCa\msCb\msNa\msNb\msNc\msM\msPaperA\msPaperC\Ed, 
च द्विजोत्तमः \msCc\msKOa, द्विजसत्तमः \msNd}}% 

%Verse 1:13

{\devanagarifont दुर्विज्ञेयं मनुष्यैस्तु देवदानवपन्नगैः {॥१:१३॥} \veg\dontdisplaylinenum }%
     \var{{\devanagarifont \numnoemph\vc\textbf{॰ज्ञेयं}\lem \msCa\msCb\msNa\msNc, ॰ज्ञेय \msCc\msNb\msNd\msM\msKOa\msPaperA\msPaperC\Ed\oo 
\textbf{मनुष्यैस्तु}\lem \msCa\msNa\msNb\msNc\msM\msKOa\msPaperA\msPaperC\Ed, 
मनुषैश्च \msCb, मणुक्षे\uncl{प्तु} \msCc, 
मनुष्येस्तु \msNd}}% 

{\devanagarifont कर्महेतु शरीरस्य उत्पत्ति निधनं च यत् \thinspace{\dandab} \dontdisplaylinenum }%
     \var{{\devanagarifont \numemph\va\textbf{कर्म॰}\lem \msCa\msCb\msNa\msNb\msNc\msNd\msM\msKOa, 
अनर्थयज्ञ उवाच\thinspace{\devanagarifont ॥} कर्म॰ \msCc\msPaperA\msPaperC\Ed\oo 
\textbf{॰हेतु}\lem \mssALL, 
॰हेतुः \msCb, ॰हेंतु \msCc\oo 
\textbf{शरीरस्य}\lem \mssALL, 
शरीरस्यं \msCc, 
स\lac \uncl{स्य} \msKOa}}% 
    \var{{\devanagarifont \numnoemph\vb\textbf{उत्पत्ति नि॰}\lem \msCa\msCb\msNa\msNb\msNc\msKOa\msPaperA\msPaperC\Ed, 
उत्पतिनि॰ \msCc\msNd, उत्पत्तिर्नि॰ \msM\oo 
\textbf{च यत्}\lem \mssALL, 
च यः \msNb, यत् \msNd}}% 

%Verse 1:14

{\devanagarifont सुकृतं दुष्कृतं चैव पाशद्वयमुदाहृतम् {॥१:१४॥} \veg\dontdisplaylinenum }%
     \var{{\devanagarifont \numnoemph\vc\textbf{सुकृतं}\lem \mssALL, 
सुकृतकृतन् \msCc, सुकृत \msNd\oo 
\textbf{चैव}\lem \mssALL, वापि \msNd\msKOa}}% 
    \var{{\devanagarifont \numnoemph\vd\textbf{पाश॰}\lem \mssALL, पासा॰ \msKOa\oo 
\textbf{॰हृतम्}\lem \mssALL, 
॰हृतः \msCc}}% 

{\devanagarifont तेनैव सह संयाति नरकं स्वर्गमेव वा \thinspace{\dandab} \dontdisplaylinenum }%
     \var{{\devanagarifont \numemph\va\textbf{तेनैव}\lem \mssALL, 
तेनेव \msCc\msNd\oo 
\textbf{सह संयाति}\lem \msCa\msCb\msNa\msNb\msNc\msPaperC\Ed, 
सह सा यान्ति \msCc\msNd, सह सा याति \msM, 
सह संयान्ति \msKOa, सहं स याति \msPaperA}}% 
    \var{{\devanagarifont \numnoemph\vb\textbf{नरकं स्वर्ग॰}\lem \mssALL, 
नरकदुर्ग्ग॰ \msKOa\oo 
\textbf{वा}\lem \mssCaCbCc\msNb\msNc\msM\msPaperA\msPaperC\Ed, च \msNa\msNd\msKOa}}% 

%Verse 1:15

{\devanagarifont सुखदुःखं शरीरेण भोक्तव्यं कर्मसम्भवम् {॥१:१५॥} \veg\dontdisplaylinenum }%
     \var{{\devanagarifont \numnoemph\vc\textbf{सुख॰}\lem \mssALL, सुखं \msM\oo 
\textbf{॰दुःखं}\lem \msCa\msCb\msNa\msNc\msM, ॰दुःख \msCc\msNb\msKOa\msPaperA\msPaperC\Ed}}% 
    \var{{\devanagarifont \numnoemph\vd\textbf{भोक्तव्यं}\lem \mssALL, 
भोक्तव्य \msKOa\oo 
\textbf{॰सम्भवम्}\lem \msCa\msCb\msNa\msNb\msNc\msM, 
॰सम्भवः \msCc\msPaperA\msPaperC\Ed, ॰संभावात् \msKOa}}% 

{\devanagarifont हेतुनानेन विप्रेन्द्र देहः सम्भवते नृणाम् \thinspace{\dandab} \dontdisplaylinenum }%
     \var{{\devanagarifont \numemph\va\textbf{हेतुनानेन}\lem \mssALL, 
हेतुना तेन \msKOa, हेतुनाने \msPaperCacorr\oo 
\textbf{॰न्द्र}\lem \mssALL, ॰न्द्रः \msNb}}% 
    \var{{\devanagarifont \numnoemph\vb\textbf{देहः}\lem \msCa\msCb\msNa\msNc\Ed, देहे \msCc, देह \msNb\msM\msKOa\msPaperA, 
देहं \msPaperC\oo 
\textbf{नृणाम्}\lem \mssALL, नृणा \msCb\msCc}}% 

%Verse 1:16

{\devanagarifont यं कालपाशमित्याहुः शृणु वक्ष्यामि सुव्रत {॥१:१६॥} \veg\dontdisplaylinenum }%
     \var{{\devanagarifont \numnoemph\vc\textbf{यं कालपाशमित्याहुः}\lem \eme, यं कालपाशमित्याह \msCa\msCb\msNa, 
कालपासेति सत्वाह \msCc, यं कालपाशमित्याहु \msNb\msNc\msPaperA\Ed, 
कालपाषेति \uncl{पस्त्वे}ह \msM, 
यां कालपासमित्याहु \msKOa}}% 
    \var{{\devanagarifont \numnoemph\vd\textbf{॰व्रत}\lem \msCa\msNa\msNb\msNc\msM\msPaperA\Ed, ॰व्रतः \msCb\msCc\msKOa}}% 

{\devanagarifont न त्वया विदितं किञ्चिज्जिज्ञास्यसि कथं द्विज \thinspace{\dandab} \dontdisplaylinenum }%
     \var{{\devanagarifont \numemph\va\textbf{विदितं}\lem \mssALL, विदित \msCc}}% 
    \var{{\devanagarifont \numnoemph\vab\textbf{किञ्चिज्जि॰}\lem \msCb\msM, किञ्चिद्वि॰ \msCapcorr\msNa\msNb\msNc\msPaperA\Ed, 
किद्वि॰ \msCaacorr, 
किञ्चि जि॰ \msCc}}% 
    \var{{\devanagarifont \numnoemph\vb\textbf{कथं द्विज}\lem \mssALL, 
\lk\lk\lk\lk\lk\lk\lk\lk\lk  \uncl{म त्वया विदितं किञ्चिद्विज्ञास्यसि} 
\cancelled\ कथं द्विज \msCc}}% 

%Verse 1:17

{\devanagarifont कालपाशं च विप्रेन्द्र सकलं वेत्तुमर्हसि {॥१:१७॥} \veg\dontdisplaylinenum }%
     \var{{\devanagarifont \numnoemph\vc\textbf{कालपाशं च}\lem \mssALL, कालपाषेति \msM}}% 
    \var{{\devanagarifont \numnoemph\vd\textbf{वेत्तुमर्हसि}\lem \mssCaCbCc\msNa\msNb, 
वेत्तुमूहसि \msNc, वक्तुमर्हसि \msM\msPaperA\Ed}}% 

{\devanagarifont कलाकलितकालं च कालतत्त्वकलां शृणु \thinspace{\dandab} \dontdisplaylinenum }%
     \var{{\devanagarifont \numemph\va\textbf{कला॰}\lem \mssALL, काला॰ \msCc\msNaacorr\oo 
\textbf{॰कलित॰}\lem \mssALL, ॰\uncl{कन्मित}॰ \msPaperA\oo 
\textbf{॰कालं च}\lem \mssALL, ॰कालश्च \msM\Ed}}% 
    \var{{\devanagarifont \numnoemph\vb\textbf{॰कलां}\lem \msCa\msCc\msNb\msPaperA\Ed, ॰कला \msCb\msNc, ॰विधिं \msNa, ॰कलाः \msM}}% 

%Verse 1:18

{\devanagarifont त्रुटिद्वयं निमेषस्तु निमेषद्विगुणा कला {॥१:१८॥} \veg\dontdisplaylinenum }%
     \var{{\devanagarifont \numnoemph\vc\textbf{त्रुटिद्वयं}\lem \msCa\msCc\msNc\Ed, तुटिद्वय \msCb\msNb, तुटिद्वयं \msNa\msM, 
त्रुविद्वयं \msPaperA\oo 
\textbf{॰मेषस्तु}\lem \mssALL, 
॰मेवस्तु \msCa, ॰मेषद्वि॰ \msNa}}% 
    \var{{\devanagarifont \numnoemph\vd\textbf{निमेषद्वि॰}\lem \mssALL, निमेषाद्वि॰ \msM}}% 

{\devanagarifont कलाद्विगुणिता काष्ठा काष्ठा वै त्रिंशतिः कला \thinspace{\dandab} \dontdisplaylinenum }%
     \var{{\devanagarifont \numemph\va\textbf{॰गुणिता काष्ठा}\lem \mssALL, ॰गुणितं काष्ठा \msM, 
॰गुणितं काष्ठी \msPaperA}}% 
    \var{{\devanagarifont \numnoemph\vb\textbf{काष्ठा वै त्रिंशतिः}\lem \msCa\msNa\msNb\msNc\msPaperA\Ed, वै त्रिंशता \msCb, 
काष्ठा वै त्रिंशति \msCc, काष्ठान्वै त्रिंशति \msM}}% 

%Verse 1:19

{\devanagarifont त्रिंशत्कला मुहूर्तश्च मानुषेन द्विजोत्तम {॥१:१९॥} \veg\dontdisplaylinenum }%
     \var{{\devanagarifont \numnoemph\vc\textbf{मुहूर्तश्च}\lem \mssALL, 
मुहूर्त्त \msCb, मुहूर्तञ्च \Ed}}% 
    \var{{\devanagarifont \numnoemph\vd\textbf{मानुषेन}\lem \mssALL, मानु\uncl{षश्च} \msCc\oo 
\textbf{॰त्तम}\lem \mssCaCbCc\msNa\msNcpcorr\msPaperA\Ed, ॰तमः \msNb\msM, ॰त्तमः \msNcacorr}}% 

{\devanagarifont मुहूर्तत्रिंशकेनैव अहोरात्रं विदुर्बुधाः \thinspace{\dandab} \dontdisplaylinenum }%
     \var{{\devanagarifont \numemph\va\textbf{मुहूर्त॰}\lem \mssALL, मुहूर्त्ता \msM, मुहूर्तं \Ed}}% 
    \var{{\devanagarifont \numnoemph\vb\textbf{॰धाः}\lem \mssALL, ॰धा \msPaperA}}% 

%Verse 1:20

{\devanagarifont अहोरात्रं पुनस्त्रिंशन्मासमाहुर्मनीषिणः {॥१:२०॥} \veg\dontdisplaylinenum }%
     \var{{\devanagarifont \numnoemph\vc\textbf{॰रात्रं}\lem \mssALL, ॰रात्र \msM}}% 
    \var{{\devanagarifont \numnoemph\vd\textbf{॰नीषिणः}\lem \mssALL, ॰नीषिन \msM}}% 

{\devanagarifont समा द्वादश मासाश्च कालतत्त्वविदो जनाः \thinspace{\dandab} \dontdisplaylinenum }%
     \var{{\devanagarifont \numemph\va\textbf{समा}\lem \mssALL, मास \msCc, समा समाया \msPaperA\oo 
\textbf{॰मासाश्च}\lem \msCa\msCb\msNa\msNb\msNc\msPaperA, ॰मासश्च \msCc\Ed, मासाहुः \msM}}% 
    \var{{\devanagarifont \numnoemph\vb\textbf{काल॰}\lem \mssALL, कला॰ \msNc}}% 

%Verse 1:21

{\devanagarifont शतं वर्षसहस्राणि त्रीणि मानुषसंख्यया {॥१:२१॥} \veg\dontdisplaylinenum }%
     \var{{\devanagarifont \numnoemph\vc\textbf{शतं}\lem   \mssALL,        शत॰ \msPaperA\Ed}}% 
    \var{{\devanagarifont \numnoemph\vd\textbf{मानुष॰}\lem \mssALL, माणुष्य॰ \msCb\msCc\ \unmetr}}% 

{\devanagarifont षष्टिं चैव सहस्राणि कालः कलियुगः स्मृतः \thinspace{\dandab} \dontdisplaylinenum }%
     \var{{\devanagarifont \numemph\va\textbf{षष्टिं चैव}\lem \mssCaCbCc\msNc\msM, षष्टिं वर्ष॰ \msNa\msPaperA, षष्टिश्चैव \Ed}}% 
    \var{{\devanagarifont \numnoemph\vb\textbf{॰युगः}\lem       \mssALL, ॰युग \msM\Ed}}% 
    \lacuna{\devanagarifontsmall \vo {\englishfont \msNb\ omits verses 22--24} }%
  
%Verse 1:22

{\devanagarifont द्विगुणः कलिसंख्यातो द्वापरो युग संज्ञितः {॥१:२२॥} \veg\dontdisplaylinenum }%
     \var{{\devanagarifont \numnoemph\vc\textbf{द्विगुणः कलिसंख्यातो}\lem \mssCaCbCc\msNa\msNc, कलिसंख्यास्तु द्विगुणो \msM, 
द्विगुर्णः कलिसंख्यातो \msPaperA, 
द्विगुणा कलिसंख्यातो \Ed}}% 
    \var{{\devanagarifont \numnoemph\vd\textbf{द्वापरो युग संज्ञितः}\lem \mssALL, 
द्वापरः युगः संज्ञिकम् \msM, 
द्वापरे युग संज्ञितः \Ed}}% 

{\devanagarifont त्रेता तु त्रिगुणा ज्ञेया चतुः कृतयुगः स्मृतः \thinspace{\dandab} \dontdisplaylinenum }%
     \var{{\devanagarifont \numemph\va\textbf{त्रेता}\lem   \msCa\msCb\msNa\msPaperA\Ed,              तेत्रा \msCc\msM, त्रेत्रा \msNc\oo 
\textbf{त्रिगुणा}\lem \mssALL,  तृगुणो \msM\oo 
\textbf{ज्ञेया}\lem   \mssALL,  ज्ञेयः \msM}}% 
    \var{{\devanagarifont \numnoemph\vb\textbf{॰युगः}\lem  \mssALL, ॰युग \Ed}}% 

%Verse 1:23

{\devanagarifont एषा चतुर्युगासंख्या कृत्वा वै ह्येकसप्ततिः {॥१:२३॥} \veg\dontdisplaylinenum }%
     \var{{\devanagarifont \numnoemph\vd\textbf{ह्ये॰}\lem   \mssALL,   हे॰ \msNc\oo 
\textbf{॰सप्ततिः}\lem \mssALL, ॰सप्तति \msM}}% 

{\devanagarifont मन्वन्तरस्य चैकस्य ज्ञानमुक्तं समासतः \thinspace{\dandab} \dontdisplaylinenum }%
     \var{{\devanagarifont \numemph\va\textbf{चैकस्य}\lem \mssALL, \om\ \msNaacorr\msMacorr}}% 
    \var{{\devanagarifont \numnoemph\vb\textbf{॰क्तं}\lem    \mssALL,                ॰क्त \msM}}% 

%Verse 1:24

{\devanagarifont कल्पो मन्वन्तराणां तु चतुर्दश तु संख्यया {॥१:२४॥} \veg\dontdisplaylinenum }%
     \var{{\devanagarifont \numnoemph\vc\textbf{कल्पो}\lem \msCb, कल्प \msCa\msCc\msNa\msNc\msM\msPaperA\Ed\oo 
\textbf{मन्वन्त॰}\lem \mssALL, 
न्वन्त॰ \msMacorr, मंन्वन्त॰ \msMpcorr}}% 
    \var{{\devanagarifont \numnoemph\vd\textbf{॰दश}\lem     \mssALL, ॰दशं \msCb\oo 
\textbf{संख्यया}\lem \mssALL,      शंक्षया \msM}}% 

{\devanagarifont दश कल्पसहस्राणि ब्रह्माहः परिकल्पितम् \thinspace{\dandab} \dontdisplaylinenum }%
     \var{{\devanagarifont \numemph\vb\textbf{॰आहः}\lem \mssALL, ॰आह \msCa\oo 
\textbf{परिकल्पितम्}\lem \msCa\msNc, करिकल्पितम् \msCb, परिकल्पितः \msCc\msNb\msM\msPaperA\Ed, 
परिकीर्तिताः \msNa}}% 

%Verse 1:25

{\devanagarifont रात्रिरेतावती प्रोक्ता मुनिभिस्तत्त्वदर्शिभिः {॥१:२५॥} \veg\dontdisplaylinenum }%
     \var{{\devanagarifont \numnoemph\vd\textbf{॰दर्शिभिः}\lem \mssALL, ॰दर्शिभि \msM}}% 

{\devanagarifont रात्र्यागमे प्रलीयन्ते जगत्सर्वं चराचरम् \thinspace{\dandab} \dontdisplaylinenum }%
     \var{{\devanagarifont \numemph\va\textbf{॰गमे}\lem      \mssALL,         ॰गम \msPaperA\oo 
\textbf{प्रलीयन्ते}\lem \mssALL, प्रलीयते \msCb}}% 
    \var{{\devanagarifont \numnoemph\vb\textbf{सर्वं च॰}\lem \mssALL,     सर्वश्च॰ \msM}}% 

%Verse 1:26

{\devanagarifont अहागमे तथैवेह उत्पद्यन्ते चराचरम् {॥१:२६॥} \veg\dontdisplaylinenum }%
     \var{{\devanagarifont \numnoemph\vc\textbf{अहागमे}\lem \mssCaCbCc\msNa\msNc, अहाग\lac\  \msNb, 
अहरागमे \msM\ \unmetr, अहागम \msPaperA, अह्नागमे \Ed}}% 
    \var{{\devanagarifont \numnoemph\vd\textbf{॰पद्यन्ते}\lem \mssALL, ॰पद्यंति \msM}}% 

{\devanagarifont परार्धपरकल्पानि अतीतानि द्विजोत्तम \thinspace{\dandab} \dontdisplaylinenum }%
     \var{{\devanagarifont \numemph\va\textbf{॰र्ध॰}\lem \mssALL, ॰र्धं \msNb, ॰ध॰ \msPaperA}}% 

%Verse 1:27

{\devanagarifont अनागतं तथैवाहुर्भृगुरादिमहर्षयः {॥१:२७॥} \veg\dontdisplaylinenum }%
     \var{{\devanagarifont \numnoemph\vcd\textbf{॰वाहुर्भृ॰}\lem \msCa\msCb\msNa\msNc\msPaperA\Ed, 
॰वाहु भृ॰ \msCc\msNb\msM}}% 
    \var{{\devanagarifont \numnoemph\vd\textbf{॰महर्षयः}\lem    \mssCaCbCc\msNapcorr\msNb\msPaperA\Ed, 
॰महयः \msNaacorr, ॰मर्हषयः \msNc, 
॰महर्षिभिः \msM}}% 

{\devanagarifont यथार्कग्रहतारेन्दु भ्रमतो दृश्यते त्विह \thinspace{\dandab} \dontdisplaylinenum }%
     \var{{\devanagarifont \numemph\va\textbf{॰आर्क॰}\lem   \mssALL, ॰आर्का॰ \msMacorr\oo 
\textbf{॰तारेन्दु}\lem \mssALL,          ॰तारैन्दु \msM}}% 
    \var{{\devanagarifont \numnoemph\vb\textbf{भ्रमतो}\lem \mssALL,                भुमनो \msPaperA\oo 
\textbf{दृश्यते त्विह}\lem \msCa\msNa\msNb\msNc\msPaperA\Ed, 
दृश्यन्दिह \msCb, दृस्यते त्विहः \msCc, 
दृश्यते त्विहः \msM}}% 

%Verse 1:28

{\devanagarifont कालचक्रं भ्रमित्वैव विश्रमं न च विद्महे {॥१:२८॥} \veg\dontdisplaylinenum }%
     \var{{\devanagarifont \numnoemph\vc\textbf{भ्रमित्वैव}\lem \corr, भ्रमत्वैव \msCa\msNa\msNc\Ed, 
भ्रमत्वेव  \msCb\msNb\msM, भ्रमत्वेह \msCc, 
भ्रमत्यैव \msPaperA}}% 
    \var{{\devanagarifont \numnoemph\vd\textbf{॰श्रमं}\lem \mssCaCbCc\msNapcorr\msNc\msPaperA\Ed, 
॰श्रमो \msNaacorr, ॰श्रामन् \msNb, ॰श्रामो \msM\oo 
\textbf{विद्महे}\lem \mssALL, विग्रहे \msCb, विद्यते \msM}}% 

{\devanagarifont कालः सृजति भूतानि कालः संहरते पुनः \thinspace{\dandab} \dontdisplaylinenum }%
     \var{{\devanagarifont \numemph\vb\textbf{कालः}\lem \mssALL, काल \Ed}}% 
    \paral{{\devanagarifontsmall \vab {\englishfont \similar\ \UMS\ 12.34cd:}
                         कालः पचति भूतानि कालः संहरते प्रजाः }}

%Verse 1:29

{\devanagarifont कालस्य वशगाः सर्वे न कालवशकृत्क्वचित् {॥१:२९॥} \veg\dontdisplaylinenum }%
     \var{{\devanagarifont \numnoemph\vc\textbf{कालस्य}\lem     \mssALL, कालःस्य \msMacorr\oo 
\textbf{वशगाः}\lem     \mssALL,         वशगा \Ed}}% 
    \var{{\devanagarifont \numnoemph\vd\textbf{कालवशकृ॰}\lem \mssALL,          कालो वशकृ॰ \msM}}% 
    \paral{{\devanagarifontsmall \vo \similar\ {\englishfont \KURMP\ 1.11.32:}
                 कालः सृजति भूतानि कालः संहरते प्रजाः\thinspace{\devanagarifontsmall ।}
                 सर्वे कालस्य वशगा न कालः कस्यचिद्वशे\thinspace{\devanagarifontsmall ॥} }}

{\devanagarifont चतुर्दश परार्धानि देवराजा द्विजोत्तम \thinspace{\dandab} \dontdisplaylinenum }%
     \var{{\devanagarifont \numemph\vb\textbf{देवराजा}\lem \mssALL,     देवराज \msM\Ed\oo 
\textbf{॰त्तम}\lem   \mssALL, ॰त्तमः \msM}}% 

%Verse 1:30

{\devanagarifont कालेन समतीतानि कालो हि दुरतिक्रमः {॥१:३०॥} \veg\dontdisplaylinenum }%
     \paral{{\devanagarifontsmall \vd {\englishfont = \MBH\ 12.220.41d = \GARPUR\ 1.108.7d} }}

{\devanagarifont एष कालो महायोगी ब्रह्मा विष्णुः परः शिवः \thinspace{\dandab} \dontdisplaylinenum }%
     \var{{\devanagarifont \numemph\va\textbf{कालो}\lem \msCa\msCb\msNa,      काल \msCc\msNb\msNc\msM\msPaperA\Ed}}% 
    \var{{\devanagarifont \numnoemph\vb\textbf{ब्रह्मा विष्णुः परः}\lem \msCb, ब्रह्मविष्णुपरः \msCa\msNc\msM\msPaperA, 
ब्रह्मा विष्णु परः \msCc\msNa\msNb, 
ब्रह्मविष्णुपर \Ed\ \unmetr}}% 

%Verse 1:31

{\devanagarifont अनादिनिधनो धाता स महात्मा नमस्कुरु {॥१:३१॥} \veg\dontdisplaylinenum }%
 

\alalfejezet{परार्धादि}
{\devanagarifont विगतराग उवाच {\dandab}\dontdisplaylinenum  }%
 
{\devanagarifont श्रुतं वै कालचक्रं तु मुखपद्मविनिःसृतम् \thinspace{\danda} \dontdisplaylinenum }%
     \var{{\devanagarifont \numemph\va\textbf{श्रुतं वै}\lem \mssALL, श्रुतो वः \msM\oo 
\textbf{॰चक्रं तु}\lem \mssALL, ॰चक्रस्य \msCc, ॰चक्रत्तु \msM}}% 
    \var{{\devanagarifont \numnoemph\vb\textbf{विनिःसृतम्}\lem \corr, विनिसृतम् \mssCaCbCc\msNa\msNb\msNc\msM\msPaperA\Ed\ \unmetr}}% 

%Verse 1:32

{\devanagarifont परार्धं च परं चैव श्रोतुं वः प्रतिदीपितम् {॥१:३२॥} \veg\dontdisplaylinenum }%
     \var{{\devanagarifont \numnoemph\vc\textbf{परार्धं च}\lem \msCb\msCc\msNa\msNb\msNc\msPaperA\Ed, 
\uncl{प}रार्द्धं च \msCa, 
पराधञ्च \msMacorr, 
परार्धंञ्चे \msMpcorr\oo 
\textbf{परं चैव}\lem \mssALL,                पराञ्चैव \msM\msPaperA}}% 
    \var{{\devanagarifont \numnoemph\vd\textbf{वः}\lem         \mssALL, नः \msMpcorr, यः \Ed\oo 
\textbf{॰दीपितम्}\lem    \mssALL,      ॰दीयतां \msM}}% 

{\devanagarifont अनर्थयज्ञ उवाच {\dandab}\dontdisplaylinenum  }%
     \var{{\devanagarifont \numemph\vo\textbf{अनर्थयज्ञ उवाच}\lem \mssALL, \om\ \msNaacorr}}% 

{\devanagarifont एकं दशं शतं चैव सहस्रमयुतं तथा \thinspace{\danda} \dontdisplaylinenum }%
     \var{{\devanagarifont \numnoemph\vb\textbf{सहस्र॰}\lem \mssALL, साहस्र॰ \msM\oo 
\textbf{॰युतं}\lem   \mssALL,  ॰तन् \msNb}}% 

%Verse 1:33

{\devanagarifont प्रयुतं नियुतं कोटिमर्बुदं वृन्दमेव च {॥१:३३॥} \veg\dontdisplaylinenum }%
     \var{{\devanagarifont \numnoemph\vc\textbf{प्र॰}\lem        \mssALL,      प॰ \msPaperA}}% 
    \var{{\devanagarifont \numnoemph\vcd\textbf{कोटिम॰}\lem \mssALL,  कोटिर॰ \msNc}}% 
    \var{{\devanagarifont \numnoemph\vd\textbf{॰र्बुदं}\lem     \mssALL, ॰बुदं \msNc}}% 

{\devanagarifont खर्वं चैव निखर्वं च शङ्कु पद्मं तथैव च \thinspace{\dandab} \dontdisplaylinenum }%
     \var{{\devanagarifont \numemph\va\textbf{निखर्वं च}\lem \mssALL,       निखर्वं तु \msNb, निसर्वञ्च \msM}}% 
    \var{{\devanagarifont \numnoemph\vb\textbf{शङ्कु}\lem        \mssALL, शंख \Ed\oo 
\textbf{पद्मं}\lem       \mssALL,  पद्म \msM}}% 
    \lacuna{\devanagarifontsmall \vab {\englishfont After these two pādas, \msPaperA\ inserts this:}
                                वृन्दञ्चैव महावृन्द द्विपरो नन्तनेव च }%
      \paral{{\devanagarifontsmall \vab {\englishfont  = \BRAHMANDAPUR\ 3.2.101 }  }}

%Verse 1:34

{\devanagarifont समुद्रो मध्यमन्तं च परार्धं च परं तथा {॥१:३४॥} \veg\dontdisplaylinenum }%
     \var{{\devanagarifont \numnoemph\vc\textbf{समुद्रो}\lem            \mssALL, समुद्र॰ \msM\oo 
\textbf{मध्यमन्तं च}\lem \mssCaCbCc\msNaacorr\msM\msPaperA,         मध्यमान्तं च \msNapcorr, 
मध्य\uncl{मन्तञ्च} \msNb, 
मध्यमन्तश्च \msNc}}% 
    \var{{\devanagarifont \numnoemph\vd\textbf{परार्धं च परं तथा}\lem \mssALL,  परार्द्धपरद्वेगुणाम् \msM}}% 
    \lacuna{\devanagarifontsmall \vcd {\englishfont \Ed\ omits 34cd--35 and then inserts this:}
                                वृन्दञ्चैव महावृन्द द्विपरानन्तमेव च }%
  
{\devanagarifont सर्वे दशगुणा ज्ञेयाः परार्धं यावदेव हि \thinspace{\dandab} \dontdisplaylinenum }%
     \var{{\devanagarifont \numemph\va\textbf{सर्वे}\lem     \mssALL, सर्वं \msPaperA}}% 
    \var{{\devanagarifont \numnoemph\vb\textbf{परार्धं}\lem \msNc,                                परा\uncl{र्ध} \msCa, 
परार्ध \msCb\msCc\msNa\msNb\msM\msPaperA}}% 

%Verse 1:35

{\devanagarifont परार्धद्विगुणेनैव परसंख्या विधीयते {॥१:३५॥} \veg\dontdisplaylinenum }%
     \var{{\devanagarifont \numnoemph\vc\textbf{परार्ध॰}\lem \mssALL,   परार्धं \msNc}}% 
    \var{{\devanagarifont \numnoemph\vd\textbf{॰संख्या}\lem  \mssALL, ॰सख्या \msM}}% 

{\devanagarifont परात्परतरं नास्ति इति मे निश्चिता मतिः \thinspace{\dandab} \dontdisplaylinenum }%
     \var{{\devanagarifont \numemph\vab\textbf{परात्परतरं नास्ति इति मे निश्चिता मतिः}\lem \mssCaCbCc\msNb\msNcpcorr\msPaperA\Ed, 
परात्परतरं नास्ति इति मे निश्चिता मति \msNa\msNcacorr, 
परापरतरन्नास्ति इति मे निश्चिता मति \msM}}% 

%Verse 1:36

{\devanagarifont पुराणवेदपठिता मयाख्याता द्विजोत्तम {॥१:३६॥} \veg\dontdisplaylinenum }%
     \var{{\devanagarifont \numnoemph\vc\textbf{॰वेद॰}\lem \msCa\Ed, ॰वेदे \msCb\msCc\msNb\msNc\msPaperA, 
॰वेदा \msNa, ॰वेदैः \msM}}% 
    \var{{\devanagarifont \numnoemph\vd\textbf{॰ख्याता}\lem \msCa\msCb\msNa, ॰ख्यातं \msCc\msNb\msNc\msM\msPaperA\Ed\oo 
\textbf{॰त्तम}\lem \mssALL, ॰तम \msM}}% 


\alalfejezet{ब्रह्माण्डम्}
{\devanagarifont विगतराग उवाच {\dandab}\dontdisplaylinenum  }%
 
{\devanagarifont ब्रह्माण्डं कति विज्ञेयं प्रमाणं ज्ञापितं क्वचित् \thinspace{\danda} \dontdisplaylinenum }%
     \var{{\devanagarifont \numemph\va\textbf{ब्रह्माण्डं}\lem \mssALL, ब्रह्माण्ड \msCc}}% 
    \var{{\devanagarifont \numnoemph\vb\textbf{प्रमाणं ज्ञापितं क्वचित्}\lem \conj, प्रमाणं चापितं क्वचित् \mssCaCbCc\msNa\msNb\msPaperA\Ed, 
प्रमाञ्चापितत् क्वचित् \msNc, प्रमाणञ्चापितां कति \msM}}% 

%Verse 1:37

{\devanagarifont कति चाङ्गुलिमूर्ध्वेषु सूर्यस्तपति वै महीम् {॥१:३७॥} \veg\dontdisplaylinenum }%
     \var{{\devanagarifont \numnoemph\vc\textbf{॰र्ध्वेषु}\lem \eme, ॰र्धेषु \mssCaCbCc\msNa\msNb\msNc\msM\msPaperA\Ed}}% 
    \var{{\devanagarifont \numnoemph\vd\textbf{सूर्यस्त॰}\lem \mssALL, र्यो \msMacorr, शूर्यो \msMpcorr\oo 
\textbf{महीम्}\lem \msCb\msCc\msNa\msM\msPaperA, मही\uncl{म् } \msCa, मही \msNb\msNc\Ed}}% 

{\devanagarifont अनर्थयज्ञ उवाच {\dandab}\dontdisplaylinenum  }%
 
{\devanagarifont ब्रह्माण्डानां प्रसंख्यातुं मया शक्यं कथं द्विज \thinspace{\danda} \dontdisplaylinenum }%
     \var{{\devanagarifont \numemph\va\textbf{ब्रह्मा॰}\lem \mssALL, ब्रह्म॰ \msM\oo 
\textbf{प्रसंख्यातुं}\lem \mssALL, प्रसंसा तु \msNb, च संख्यातुं \Ed}}% 
    \var{{\devanagarifont \numnoemph\vb\textbf{शक्यं क॰}\lem \msNa\msNb\msPaperApcorr\Ed, शक्या क॰ \mssCaCbCc\msNc, सक्याङ्क॰ \msM, 
ह्यक्यं क॰ \msPaperAacorr}}% 

%Verse 1:38

{\devanagarifont देवास्ते ऽपि न जानन्ति मानुषाणां च का कथा {॥१:३८॥} \veg\dontdisplaylinenum }%
     \var{{\devanagarifont \numnoemph\vc\textbf{देवास्ते}\lem   \mssALL, देवतापि \msM}}% 
    \var{{\devanagarifont \numnoemph\vd\textbf{मानुषाणां च}\lem \mssALL, मानुषार्नञ्च \msMacorr, 
मानुषानाञ्च \msMpcorr}}% 

{\devanagarifont पर्यायेण तु वक्ष्यामि यथाशक्यं द्विजोत्तम \thinspace{\dandab} \dontdisplaylinenum }%
 
%Verse 1:39

{\devanagarifont ब्रह्मणा यत्पुराख्यातो मातरिश्वा यथा तथा {॥१:३९॥} \veg\dontdisplaylinenum }%
     \var{{\devanagarifont \numemph\vc\textbf{यत्पुराख्यातो}\lem \mssCaCbCc\msNa\msNb\msNc, यत्पुराख्यातं \msM, 
यत्प्रयात्परायाख्यातो \msPaperA, 
यत्ममाख्यातो \Ed}}% 
    \paral{{\devanagarifontsmall \vcd {\englishfont cf. \BRAHMANDAPUR\ 3.4.58cd:} 
                         ब्रह्मा ददौ शास्त्रमिदं पुराणं मातरिश्वने }}

{\devanagarifont शिवाण्डाभ्यन्तरेणैव सर्वेषामिव भूभृताम् \thinspace{\dandab} \dontdisplaylinenum }%
     \var{{\devanagarifont \numemph\va\textbf{शिवाण्डा॰}\lem \mssALL, शिवाण्ड॰ \msMacorr, शिवाण्डे॰ \msMpcorr}}% 
    \var{{\devanagarifont \numnoemph\vb\textbf{सर्वेषामिव भूभृताम्}\lem \conj, सर्वेषामिव भूरिताः \msCa\msCb\msNc, 
सर्वेषामेव भूरिताः \msCc, 
सर्वेषामिव भूरिता \msNa, सर्वेषामेव भूरिणाम् \msNb, 
स\uncl{र्षपा} इव भाविता \msM, 
सर्वेषामेव भूरिनाः \msPaperA, 
सर्वेषामेव भूरिमां \Ed}}% 

%Verse 1:40

{\devanagarifont दश नाम दिशाष्टानां ब्रह्माण्डे कीर्तितं शृणु {॥१:४०॥} \veg\dontdisplaylinenum }%
     \var{{\devanagarifont \numnoemph\vc\textbf{दिशा॰}\lem \mssALL, शिवा॰ \msNb}}% 
    \var{{\devanagarifont \numnoemph\vd\textbf{ब्रह्माण्डे}\lem \mssALL, ब्रह्मण्डा \msM\oo 
\textbf{कीर्तितं शृणु}\lem \mssALL, 
य च कीर्तितम् \msCb, कीर्त्तिता शृणु \msM}}% 
