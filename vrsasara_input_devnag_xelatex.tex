\fejno=0\versno=0
\centerline{\Huge\devanagarifontbold वृषसारसंग्रहः  }

 
{\vrule depth10pt width0pt}
\versno=0\fejno=8
\thispagestyle{empty}

\centerline{\Large\devanagarifontbold [   अष्टमो ऽध्यायः  ]}{\vrule depth10pt width0pt} \fancyhead[CO]{{\footnotesize\devanagarifont वृषसारसंग्रहे  }}
\fancyhead[CE]{{\footnotesize\devanagarifont अष्टमो ऽध्यायः  }}
\fancyhead[LE]{}
\fancyhead[RE]{}
\fancyhead[LO]{}
\fancyhead[RO]{}
\szam\bek



\alalfejezet{नियमेषु स्वाध्यायः (५)}
\vers


{\devanagarifont पञ्चस्वाध्यायनं कार्यमिहामुत्र सुखार्थिना \thinspace{\dandab} \dontdisplaylinenum }%
     \var{{\devanagarifont \numemph\va\textbf{॰स्वाध्यायनं}\lem \mssALL, 
॰स्वाध्ययनं \msNc}}% 
    \var{{\devanagarifont \numnoemph\vb\textbf{॰मुत्र}\lem \mssALL, ॰मूत्र \msPaperA\Ed\oo 
\textbf{॰र्थिना}\lem \mssALL, ॰र्थिनां \msNb}}% 
    \lacuna{\devanagarifontsmall {\englishfont Witnesses used for this chapter: \msCa\ ff.\thinspace 204r--205v, 
                                                  \msCb\ ff.\thinspace 210v--211v, 
                                                  \msCc\ ff.\thinspace 280v--282r,
                                                  \msNa\ ff.\thinspace 11v--13r, 
                                                  \msNb\ exp.\thinspace 53 (lower) -- 54 (lower),
                                                  \msNc\ ff.\thinspace 219v--221r,
                                                  \msParis\ exp.\thinspace 426--428,
                                                  \msPaperA\ ff.\thinspace 213r--214v,
                                                  \Ed\ pp.\thinspace 603--606; 
                                                  \mssCaCbCc\ = \msCa + \msCb + \msCc} }%
  
%Verse 8:1

{\devanagarifont शैवं सांख्यं पुराणं च स्मार्तं भारतसंहिताम् {॥ ८:१॥} \veg\dontdisplaylinenum }%
     \var{{\devanagarifont \numnoemph\vc\textbf{शैवं}\lem \mssALL, 
\uncl{शै}लं \msCc\oo 
\textbf{सांख्यं}\lem \msCa\msCb\msNc\msParis\msPaperA\Ed, 
शांख्य \msCc, साख्यं \msNa\msNb}}% 
    \var{{\devanagarifont \numnoemph\vd\textbf{स्मार्तं}\lem \mssALL, स्मार्त \msCc\msNb\oo 
\textbf{भारतसंहिताम्}\lem \mssALL, 
भारतसंहिताः \msNa, भारत्तसंहितां \msNc}}% 

{\devanagarifont शैवे तत्त्वं विचिन्तेत शैवपाशुपतद्वये \thinspace{\dandab} \dontdisplaylinenum }%
     \var{{\devanagarifont \numemph\va\textbf{शैवे }\lem \msCa\msCc\msNa\msNb\msNc, शैवै \msCb\msParis, शैवं \msPaperA\Ed\oo 
\textbf{तत्त्वं}\lem \mssALL, ॰तत्त्व \msParis}}% 
    \var{{\devanagarifont \numnoemph\vb\textbf{शैव॰}\lem \msParis, शैवः \msCa\msCb\msNb\msNc, शैवाः \msCc\msPaperA\Ed, शैवा \msNa\oo 
\textbf{॰द्वये}\lem \mssALL, ॰ये \msCb}}% 

%Verse 8:2

{\devanagarifont अत्र विस्तरतः प्रोक्तं तत्त्वसारसमुच्चयम् {॥ ८:२॥} \veg\dontdisplaylinenum }%
     \var{{\devanagarifont \numnoemph\vd\textbf{॰सारसमुच्चयम्}\lem \mssALL, 
॰सारं समुच्चयम् \msNa, ॰सारं समुद्ययं \msNb}}% 

{\devanagarifont संख्यातत्त्वं तु सांख्येषु बोद्धव्यं तत्त्वचिन्तकैः \thinspace{\dandab} \dontdisplaylinenum }%
     \var{{\devanagarifont \numemph\va\textbf{संख्यातत्त्वं तु}\lem \msNa\msNc\msParis\msPaperA, 
सं\uncl{ख्या}\lk\lk\lk\ \msCa, संख्यातत्त्वं \msCb, 
शाङ्ख्यातत्वं तु \msCc, सख्यतत्वन्तु \msNb, संख्यातत्त्व तु \Ed\oo 
\textbf{सांख्येषु}\lem \mssALL, सख्येषु \msNb}}% 

%Verse 8:3

{\devanagarifont पञ्चतत्त्वविभागेन कीर्तितानि महर्षिभिः {॥ ८:३॥} \veg\dontdisplaylinenum }%
     \var{{\devanagarifont \numnoemph\vc\textbf{॰तत्त्व॰}\lem \mssALL, 
॰तत्वा॰ \msCb, \om\ \msNb}}% 

{\devanagarifont पुराणेषु महीकोषो विस्तरेण प्रकीर्तितः \thinspace{\dandab} \dontdisplaylinenum }%
 
%Verse 8:4

{\devanagarifont अधोर्ध्वमध्यतिर्यं च यत्नतः सम्प्रवेशयेत् {॥ ८:४॥} \veg\dontdisplaylinenum }%
     \var{{\devanagarifont \numemph\vc\textbf{अधोर्ध्व॰}\lem \mssALL, अधोर्ध्वं \msNb\oo 
\textbf{॰मध्य॰}\lem \mssALL, ॰मध॰ \msCc}}% 
    \var{{\devanagarifont \numnoemph\vd\textbf{यत्नतः}\lem \mssALL, यत्नत \msNb\oo 
\textbf{सम्प्रवेशयेत्}\lem \mssALL, 
सम्प्रबोधयेत् \Ed}}% 

{\devanagarifont स्मार्तं वर्णाश्रमाचारं धर्मन्यायप्रवर्तनम् \thinspace{\dandab} \dontdisplaylinenum }%
     \var{{\devanagarifont \numemph\va\textbf{स्मार्तं वर्णा॰}\lem \msCa, तस्मार्त्तम्वर्ण्णा॰ \msCb, 
स्मार्तवर्णा॰ \msCc\msNa\msNb\msNc\msPaperA\Ed, 
स्मार्त्तं वर्ण्ण॰ \msParis}}% 
    \var{{\devanagarifont \numnoemph\vb\textbf{धर्म॰}\lem \mssALL, धर्मं \msCc\oo 
\textbf{॰वर्तनम्}\lem \mssALL, 
॰व\lk नं \msParis, ॰वर्तन \Ed}}% 

%Verse 8:5

{\devanagarifont शिष्टाचारो ऽविकल्पेन ग्राह्यस्तत्र अशङ्कितः {॥ ८:५॥} \veg\dontdisplaylinenum }%
     \var{{\devanagarifont \numnoemph\vc\textbf{शिष्टा॰}\lem \mssALL, शिष्ट॰ \msPaperA\oo 
\textbf{॰चारो}\lem \msCa\msCb\msNb\msNc\msPaperA, 
॰चार॰ \msCc\Ed, 
॰चारा \msNa, 
॰चा\uncl{रो}॰ \msParis}}% 
    \var{{\devanagarifont \numnoemph\vd \lem \mssALL, 
ग्राह्यस्त\lk\lk\lk ङ्कितः \msCa}}% 

{\devanagarifont इतिहासमधीयानः सर्वज्ञः स नरो भवेत् \thinspace{\dandab} \dontdisplaylinenum }%
     \var{{\devanagarifont \numemph\vb\textbf{॰ज्ञः}\lem \mssALL, ॰ज्ञ \msCc}}% 

%Verse 8:6

{\devanagarifont धर्मार्थकाममोक्षेषु संशयस्तेन छिद्यते {॥ ८:६॥} \veg\dontdisplaylinenum }%
 

\alalfejezet{नियमेष्वुपस्थनिग्रहः (६)}
{\devanagarifont शृणुष्वावहितो विप्र पञ्चोपस्थविनिग्रहम् \thinspace{\dandab} \dontdisplaylinenum }%
     \var{{\devanagarifont \numemph\vb\textbf{॰ग्रहम्}\lem \mssALL, 
॰ग्र\uncl{हः} \msNa}}% 

{\devanagarifont स्त्रियो वा गर्हितोत्सर्गः स्वयंमुक्तिश्च कीर्त्यते  \danda\dontdisplaylinenum }%
     \var{{\devanagarifont \numnoemph\vc\textbf{गर्हितोत्सर्गः}\lem \msCa\msCb\msNb\msNc\msParis, 
गर्हितस्सर्ग्गः \msCc, गर्हितो विप्र \msNa, 
गर्हितो स्वर्गः \msPaperA\Ed}}% 
    \var{{\devanagarifont \numnoemph\vd\textbf{स्वयं॰}\lem \mssALL, स्वय॰ \msCb\oo 
\textbf{कीर्त्यते}\lem \mssALL, 
की\uncl{र्त्स्य}ते \msCc}}% 

%Verse 8:7

{\devanagarifont स्वप्नोपघातं विप्रेन्द्र दिवास्वप्नं च पञ्चमः {॥ ८:७॥} \veg\dontdisplaylinenum }%
     \var{{\devanagarifont \numnoemph\ve\textbf{॰घातं}\lem \mssALL, 
॰घात \msCc\Ed}}% 


\alalalfejezet{स्त्रियः}

{\devanagarifont अगम्या स्त्री दिवा पर्वे धर्मपत्न्यपि वा भवेत् \thinspace{\dandab} \dontdisplaylinenum }%
     \var{{\devanagarifont \numemph\va\textbf{स्त्री दिवा पर्वे}\lem \msCb\msCc\msNa\msNb\msNc\msPaperA, 
\lk  दिवा पर्व्वे \msCa, \lk\lk \lk  पर्वे \msParis, स्त्री दिवापूर्वे \Ed}}% 
    \var{{\devanagarifont \numnoemph\vb\textbf{॰पत्न्यपि}\lem \mssALL, 
॰पत्नी पि \msCc}}% 

%Verse 8:8

{\devanagarifont विरुद्धस्त्रीं न सेवेत वर्णभ्रष्टाधिकासु च {॥ ८:८॥} \veg\dontdisplaylinenum }%
     \var{{\devanagarifont \numnoemph\vc\textbf{विरुद्धस्त्रीं न}\lem \msPaperA, विरुद्धस्त्री न \mssCaCbCc\msNb\msNc, 
विरुद्धस्त्री नि॰ \msNa\msParis, द्विरुद्धास्त्रीन्न \Ed}}% 
    \var{{\devanagarifont \numnoemph\vd\textbf{॰धिकासु च}\lem \msCa\msCb\msNa\msParis\msPaperA, ॰धिकासु त \msCc, ॰दिकाषु च \msNb, 
॰विकाषु च \msNc, ॰पिकासु च \Ed}}% 


\alalalfejezet{गर्हितोत्सर्गः}

{\devanagarifont अजमेषगवादीनां वडवामहिषीषु च \thinspace{\dandab} \dontdisplaylinenum }%
     \var{{\devanagarifont \numemph\va\textbf{॰मेष॰}\lem \mssALL, ॰मेय॰ \msCb}}% 

%Verse 8:9

{\devanagarifont गर्हितोत्सर्गमित्येतद्यत्नेन परिवर्जयेत् {॥ ८:९॥} \veg\dontdisplaylinenum }%
 

\alalalfejezet{स्वयंमुक्तिः}

{\devanagarifont अयोनिकषणा वापि अपानकषणापि वा \thinspace{\dandab} \dontdisplaylinenum }%
     \var{{\devanagarifont \numemph\va\textbf{अयोनि॰}\lem \conj, अन्योन्य॰ \mssCaCbCc\msNa\msNb\msNc\msParis\msPaperA\Ed\oo 
\textbf{॰कषणा}\lem \msCa\msNa, ॰कर्षणा \msCb\msCc\msNb\msNc\msParis\msPaperA\Ed}}% 
    \var{{\devanagarifont \numnoemph\vb\textbf{॰कषणापि}\lem \mssCaCbCc\msNa, ॰कर्षणापि \msNb\msNc\msParis\msPaperA\Ed}}% 

%Verse 8:10

{\devanagarifont स्वयंमुक्तिरियं ज्ञेया तस्मात्तां परिवर्जयेत् {॥ ८:१०॥} \veg\dontdisplaylinenum }%
     \var{{\devanagarifont \numnoemph\vc\textbf{स्वयंमुक्ति॰}\lem \mssALL, 
स्वयमुक्ति॰ \msCb\oo 
\textbf{ज्ञेया}\lem \mssALL, 
ज्ञेयां \msNb}}% 
    \var{{\devanagarifont \numnoemph\vd\textbf{तस्मात्तां}\lem \msCa\msCb\msNa\msNc\msParis\msPaperA, 
तस्मात्तं \msCc, तस्मार्त्ता \msNb, तस्मात्स्त्री \Ed}}% 


\alalalfejezet{स्वप्नघातम्}

{\devanagarifont स्वप्नघातं द्विजश्रेष्ठ अनिष्टं पण्डितैः सदा \thinspace{\dandab} \dontdisplaylinenum }%
     \var{{\devanagarifont \numemph\va\textbf{स्वप्नघा॰}\lem \mssALL, 
स्वप्नजा॰ \msParisacorr}}% 
    \var{{\devanagarifont \numnoemph\vb\textbf{पण्डितैः}\lem \mssALL, 
पण्डितै \msCc, पण्डितेः \msNc}}% 

%Verse 8:11

{\devanagarifont स्वप्ने स्त्रीषु रमन्ते च रेतः प्रक्षरते ततः {॥ ८:११॥} \veg\dontdisplaylinenum }%
     \var{{\devanagarifont \numnoemph\vc\textbf{रमन्ते}\lem \mssALL, रमक्षन्ते \msPaperA}}% 
    \var{{\devanagarifont \numnoemph\vd\textbf{प्रक्षरते}\lem \mssALL, प्रस्खलतस् \Ed\oo 
\textbf{ततः}\lem \mssALL, तत \msCc}}% 


\alalalfejezet{दिवास्वप्नम्}

{\devanagarifont दिवाशयं न कर्तव्यं नित्यं धर्मपरेण तु \thinspace{\dandab} \dontdisplaylinenum  }%
     \var{{\devanagarifont \numemph\va\textbf{दिवाशयं न}\lem \mssCaCbCc\msParis\msPaperA\Ed, 
दिवाशयेन्न \msNa, दिवासयानं \msNb, दिवाशायं \msNc}}% 
    \var{{\devanagarifont \numnoemph\vb\textbf{नित्यं}\lem \mssALL, नित्य \msNb\oo 
\textbf{॰परेण तु}\lem \mssALL, 
॰परेन तु \msCa, ॰परेण च \msCc}}% 

%Verse 8:12

{\devanagarifont स्वर्गमार्गार्गला ह्येताः स्त्रियो नाम प्रकीर्तिताः {॥ ८:१२॥} \veg\dontdisplaylinenum }%
     \var{{\devanagarifont \numnoemph\vc\textbf{ह्येताः}\lem \msNc, ह्येता \mssCaCbCc\msNa\msNb\msParis\msPaperA\Ed}}% 
    \var{{\devanagarifont \numnoemph\vd\textbf{स्त्रियो}\lem \mssALL, स्त्रीयो \Ed\oo 
\textbf{॰कीर्तिताः}\lem \mssALL, ॰कीर्तिता \msNc}}% 
    \paral{{\devanagarifontsmall \vcd {\englishfont \compare\ \PADMAP\ 1.13.395cd:} परित्यजध्वं दाराणि स्वर्गमार्गार्गलानि च }}


\alalfejezet{नियमेषु व्रतपञ्चकम् (७)}
{\devanagarifont मार्जारकबकश्वानगोमहीव्रतपञ्चकम् \thinspace{\dandab} \dontdisplaylinenum }%
     \var{{\devanagarifont \numemph\vab\textbf{मार्जारकबकश्वानगोमहीव्रत॰}\lem \mssCaCbCc\msNa\msNc\msParis, 
मार्जारबकबश्वानगोमहीव्रत॰ \msNb, 
मार्जारकवकश्वानगोमहीवेक॰ \msPaperA, 
मार्जारकश्च श्वानाश्च गोमहीवक \Ed}}% 


\alalalfejezet{मार्जारकव्रतम्}

{\devanagarifont स्वविष्ठमूत्रं भूमीषु छादयेद्द्विजसत्तम  \danda\dontdisplaylinenum }%
     \var{{\devanagarifont \numnoemph\vc\textbf{॰विष्ठ॰}\lem \mssALL, ॰विष्टा॰ \Ed\oo 
\textbf{॰मूत्रं}\lem \mssALL, ॰मूत्र॰ \msCb\msNb}}% 

%Verse 8:13

{\devanagarifont सूर्यसोमानुमोदन्ति मार्जारव्रतिकेषु च {॥ ८:१३॥} \veg\dontdisplaylinenum }%
     \var{{\devanagarifont \numnoemph\ve\textbf{॰मोदन्ति}\lem \mssALL, ॰षादन्ति \Ed}}% 


\alalalfejezet{बकव्रतम्}

{\devanagarifont बकवच्चेन्द्रियग्रामं सुनियम्य तपोधन \thinspace{\dandab} \dontdisplaylinenum }%
     \var{{\devanagarifont \numemph\va\textbf{तपोधन}\lem \mssCaCbCc\msNa\msNb\msParis, 
तपोधनः \msNc, तपोधनम् \msPaperA\Ed}}% 

%Verse 8:14

{\devanagarifont साधयेच्च मनस्तुष्टिं मोक्षसाधनतत्परः {॥ ८:१४॥} \veg\dontdisplaylinenum }%
     \var{{\devanagarifont \numnoemph\vc\textbf{साधयेच्च}\lem \mssALL, 
साधये च \msCb\oo 
\textbf{मनस्तुष्टिं}\lem \mssALL, 
मनस्तुष्टि॰ \msCb\msCc}}% 
    \var{{\devanagarifont \numnoemph\vd\textbf{॰साधन॰}\lem \mssALL, 
॰सान॰ \msNc}}% 


\alalalfejezet{श्वानव्रतम्}

{\devanagarifont मूत्रविष्ठे न भूमीषु कुरुते धुनदं सदा \thinspace{\dandab} \dontdisplaylinenum }%
     \var{{\devanagarifont \numemph\va\textbf{मूत्रविष्ठे न}\lem \mssALL, 
मूत्रविष्टे च \Ed}}% 
    \var{{\devanagarifont \numnoemph\vb\textbf{धुनदं}\lem \mssALL, 
श्वानदः \msNa, छादनं \Ed}}% 

%Verse 8:15

{\devanagarifont तुष्यते भगवान्शर्वः श्वानव्रतचरो यदि {॥ ८:१५॥} \veg\dontdisplaylinenum }%
     \var{{\devanagarifont \numnoemph\vc\textbf{शर्वः}\lem \msCa\msNa\msNc\msParis\msPaperA\Ed, 
सर्वः \msCb\msNb, सव्वः \msCc}}% 


\alalalfejezet{गोव्रतम्}

{\devanagarifont मूत्रवर्चो न रुध्येत सदा गोव्रतिको नरः \thinspace{\dandab} \dontdisplaylinenum }%
     \var{{\devanagarifont \numemph\va\textbf{॰वर्चो}\lem \msCa\msCc\msNb\msNc\msParis\msPaperA, 
॰वच्चो \msCb\msNa, ॰वर्चा \Ed}}% 
    \var{{\devanagarifont \numnoemph\vb\textbf{गोव्रतिको}\lem \mssALL, 
\lk\lk तिको \msCa}}% 

%Verse 8:16

{\devanagarifont भीमस्तुष्टिकरश्चैव पुराणेषु निगद्यते {॥ ८:१६॥} \veg\dontdisplaylinenum }%
     \var{{\devanagarifont \numnoemph\vc\textbf{भीमस्तु॰}\lem \msCc\msNb\Ed, 
भीमतु॰ \msCa\msCb\msNa\msNc\msParis, 
भिमस्तु॰ \msPaperA}}% 


\alalalfejezet{महीव्रतम्}

{\devanagarifont कुद्दालैर्दारयन्तो ऽपि कीलकोटिशतैश्चितः \thinspace{\dandab} \dontdisplaylinenum }%
     \var{{\devanagarifont \numemph\va\textbf{कुद्दालैर्दारयन्तो}\lem \msNa\msParis\Ed, 
कुद्दालैर्दारयन्नो \msCa, कुद्दारै दारयन्तो \msCb, 
कुदारै दारयन्ता \msCc, कुद्दालै द्दारयामास \msNb, 
कुद्दालै दारयन्तो \msNc, कुद्दालै \uncl{द्धार}यन्तो \msPaperA}}% 
    \var{{\devanagarifont \numnoemph\vb \lem \msCa\msCb\msNa\msNb\msNc\msParis, 
कीटकोटीशतैरपि \msCc\msPaperA\Ed}}% 

%Verse 8:17

{\devanagarifont क्षमते पृथिवी देवी एवमेव महीव्रतः {॥ ८:१७॥} \veg\dontdisplaylinenum }%
     \var{{\devanagarifont \numnoemph\vd\textbf{॰व्रतः}\lem \mssALL, ॰व्रत \msNc}}% 

{\devanagarifont व्रतपञ्चकमित्येतद्यश्चरेत जितेन्द्रियः \thinspace{\dandab} \dontdisplaylinenum }%
     \var{{\devanagarifont \numemph\vb\textbf{जितेन्द्रियः}\lem \mssALL, द्विजेन्द्रियः \msNb}}% 

%Verse 8:18

{\devanagarifont स चोत्तममिदं लोकं प्राप्नोति न च संशयः {॥ ८:१८॥} \veg\dontdisplaylinenum }%
 

\alalfejezet{नियमेष्वुपवासः (८)}
{\devanagarifont शेषान्नमन्तरान्नं च नक्तायाचितमेव च \thinspace{\dandab} \dontdisplaylinenum }%
     \var{{\devanagarifont \numemph\va \lem \msCa\msCb\msNb\msNc\msParispcorr, 
शेषाणामन्तराणाञ्च \msCc\Ed, 
शेषान्नमन्नरान्नं च \msNa, 
शेषान्नमरान्नं च \msParisacorr, 
शेषाणमन्तराणाञ्च \msPaperA}}% 
    \var{{\devanagarifont \numnoemph\vb\textbf{नक्तायाचित॰}\lem \mssALL, 
नक्त\uncl{या}चित॰ \msNc\oo 
\textbf{च}\lem \mssALL, वा \Ed}}% 

%Verse 8:19

{\devanagarifont उपवासं च पञ्चैतत्कथयिष्यामि तच्छृणु {॥ ८:१९॥} \veg\dontdisplaylinenum }%
     \var{{\devanagarifont \numnoemph\vcd\textbf{पञ्चैतत्क॰}\lem \mssALL, 
पञ्चैते क॰ \msCc}}% 


\alalalfejezet{शेषान्नम्}

{\devanagarifont वैश्वदेवातिथिशेषं पितृशेषं च यद्भवेत् \thinspace{\dandab} \dontdisplaylinenum }%
     \var{{\devanagarifont \numemph\va\textbf{॰शेषं}\lem \mssALL, ॰शेषां \msCb}}% 

%Verse 8:20

{\devanagarifont भृत्यपुत्रकलत्रेभ्यः शेषाशी विघसाशनः {॥ ८:२०॥} \veg\dontdisplaylinenum }%
     \var{{\devanagarifont \numnoemph\vd\textbf{विघसाशनः}\lem \msCa\msNa\msNb, विघसासनम् \msCb, विघसाषिनः \msCc, 
विघशासनः \msNc, विघसाश\uncl{नः} \msParispcorr, 
घसाशन \msParisacorr, विघसासनः \msPaperA, विषसासनः \Ed}}% 


\alalalfejezet{अन्तरान्नम्}

{\devanagarifont अन्तरा प्रातराशी च सायमाशी तथैव च \thinspace{\dandab} \dontdisplaylinenum }%
     \var{{\devanagarifont \numemph\va\textbf{अन्तरा प्रातराशी}\lem \eme, अन्तरा प्रान्तराशी \mssCaCbCc\msNa\msNc, 
अन्तरा \uncl{क्रन्त}राशी \msNb, 
अन्तारा प्रा\uncl{त्त}राशी \msParis, 
अन्तमा प्रान्तराशी च \msPaperA,  अन्तसम्प्रान्तराशी \Ed}}% 
    \var{{\devanagarifont \numnoemph\vb\textbf{सायमाशी}\lem \msCb\msCc\msNa\msNb\msNc\msParis, सायमाशीन् \msCa, 
नायमाशी \msPaperA, नियमाशी \Ed}}% 

%Verse 8:21

{\devanagarifont सदोपवासी भवति यो न भुङ्क्ते कदाचन {॥ ८:२१॥} \veg\dontdisplaylinenum }%
     \var{{\devanagarifont \numnoemph\vc\textbf{॰वासी भवति}\lem \mssALL, 
॰वासी च भवति \msCc}}% 
    \var{{\devanagarifont \numnoemph\vd\textbf{कदाचन}\lem \mssALL, कदाचनः \msCc}}% 
    \paral{{\devanagarifontsmall \vcd \similar\ {\englishfont \MBH\ 12.214.9:} 
                                 अन्तरा प्रातराशं च सायमाशं तथैव च\thinspace{\devanagarifontsmall ।}
                                 सदोपवासी च भवेद् यो न भुङ्क्ते कथंचन\thinspace{\devanagarifontsmall ॥} 
                     \similar\ {\englishfont \MBH\ 13.93.10:}
                                 अन्तरा सायमाशं च प्रातराशं तथैव च\thinspace{\devanagarifontsmall ।}
                                 सदोपवासी भवति यो न भुङ्क्ते ऽन्तरा पुनः\thinspace{\devanagarifontsmall ॥} }}


\alalalfejezet{नक्तान्नम्}

{\devanagarifont न दिवा भोजनं कार्यं रात्रौ नैव च भोजयेत् \thinspace{\dandab} \dontdisplaylinenum }%
     \var{{\devanagarifont \numemph\va\textbf{भोजनं}\lem \mssALL, नोजनं \msNc}}% 
    \var{{\devanagarifont \numnoemph\vb\textbf{च}\lem \mssALL, तु \msCb, \om\ \msNa\oo 
\textbf{भोजयेत्}\lem \mssALL, कारयेत् \msNb}}% 

%Verse 8:22

{\devanagarifont नक्तवेले च भोक्तव्यं नक्तधर्मं समीहता {॥ ८:२२॥} \veg\dontdisplaylinenum }%
     \var{{\devanagarifont \numnoemph\vc\textbf{॰वेले च}\lem \msCa\msCc\msNa\msNb\msParis\msPaperA, 
॰वेला च \msCb, ॰वेलो च \msNc, ॰वेले व \Ed}}% 
    \var{{\devanagarifont \numnoemph\vd\textbf{॰धर्मं समीहता}\lem \msCa\msCb\msNa\msNc\msParis, 
॰धर्मसमीहता \msCc\msNb, ॰धर्म्मसमीहिता \msPaperA, 
॰धर्म्मः समीहितः \Ed}}% 


\alalalfejezet{अयाचितान्नम्}

{\devanagarifont अनारभ्य य आहारं कुर्यान्नित्यमयाचितम् \thinspace{\dandab} \dontdisplaylinenum }%
     \var{{\devanagarifont \numemph\va\textbf{अनारभ्य य}\lem \conj, अनारम्भस्य \mssCaCbCc\msNa\msNb\msNc\msParis\msPaperA\Ed}}% 
    \var{{\devanagarifont \numnoemph\vb\textbf{कुर्यान्नि॰}\lem \mssALL, कुर्या नि॰ \msNc}}% 

%Verse 8:23

{\devanagarifont परैर्दत्तं तु यो भुङ्क्ते तमयाचितमुच्यते {॥ ८:२३॥} \veg\dontdisplaylinenum }%
     \var{{\devanagarifont \numnoemph\vc\textbf{परैर्दत्तं तु}\lem \msCa\msCb\msNa\msParis\msPaperA, 
परै दत्तञ्च \msCc, परै दत्तन्तु \msNb, 
परैर्दन्तन्तु \msNc\Ed}}% 
    \var{{\devanagarifont \numnoemph\vd\textbf{तमयाचि॰}\lem \mssCaCbCc\msNa\msNb\msNc\Ed, नमयाचि॰ \msParisacorr\msPaperA, 
\uncl{तम}याचि॰ \msParispcorr}}% 


\alalalfejezet{उपवासः}

{\devanagarifont भक्ष्यं भोज्यं च लेह्यं च चोष्यं पेयं च पञ्चमम् \thinspace{\dandab} \dontdisplaylinenum }%
     \var{{\devanagarifont \numemph\va\textbf{भक्ष्यं}\lem \mssALL, भक्ष्य \msNa}}% 

%Verse 8:24

{\devanagarifont न काङ्क्षेन्नोपयुञ्जीत उपवासः स उच्यते {॥ ८:२४॥} \veg\dontdisplaylinenum }%
     \var{{\devanagarifont \numnoemph\vc\textbf{काङ्क्षेन्नो॰}\lem \mssALL, 
काङ्क्षे नो॰ \msCc\oo 
\textbf{॰युञ्जीत}\lem \msCc\msNa\msNb\msPaperA, ॰\lk\lk त \msCa, 
॰यञ्जीत \msCb, ॰भुजीत \msNc, ॰भुञ्जीत \msParis\Ed}}% 
    \var{{\devanagarifont \numnoemph\vd\textbf{॰वासः स}\lem \mssCaCbCc\msNa\msParis\Ed, ॰वास स \msNb, ॰वासस्य \msNc, 
॰वासंः स \msPaperA}}% 


\alalfejezet{नियमेषु मौनव्रतम् (९)}
{\devanagarifont मिथ्यापिशुनपारुष्यतीक्ष्णवागप्रलापनम् \thinspace{\dandab} \dontdisplaylinenum }%
     \var{{\devanagarifont \numemph\va\textbf{॰पारुष्य॰}\lem \msCa\msCb\msNa\msNb\msNc\msParis, ॰संभिन्ना \msCc, 
संभिन्नां \msPaperA, ॰याभिन्ना \Ed}}% 
    \var{{\devanagarifont \numnoemph\vb\textbf{॰तीक्ष्णवाग॰}\lem \conj, ॰स्पृष्टवाग॰ \msCa\msCb\msNa\msNb\msNc\msParis, 
पृष्टवाक॰ \msCc\msPaperA, 
पृष्तेवाक॰ \Ed}}% 

%Verse 8:25

{\devanagarifont मौनपञ्चकमित्येतद्धारयेन्नियतव्रतः {॥ ८:२५॥} \veg\dontdisplaylinenum }%
     \var{{\devanagarifont \numnoemph\vc\textbf{मौनपञ्चक॰}\lem \msCa\msCb\msNb, मौनं पञ्चक॰ \msCc\msNa\msNc\msPaperA\Ed, 
मौनम्पञ्च॰ \msParis\oo 
\textbf{॰त्येत॰}\lem \mssALL, ॰त्ये॰ \msParisacorr}}% 
    \var{{\devanagarifont \numnoemph\vd\textbf{॰रयेन्नि॰}\lem \mssALL, ॰रयन्नि॰ \Ed}}% 


\alalalfejezet{मिथ्यावचनम्}

{\devanagarifont असम्भूतमदृष्टं च धर्माच्चापि बहिष्कृतम् \thinspace{\dandab} \dontdisplaylinenum }%
     \var{{\devanagarifont \numemph\va\textbf{॰दृष्टं च}\lem \mssALL, दृष्ट\uncl{ञ्च} \msCc}}% 
    \var{{\devanagarifont \numnoemph\vb\textbf{धर्माच्चापि}\lem \msCa\msCb\msNa\msNb\msNc\msParis, 
धर्मश्चापि \msCc\msPaperA, धर्मं चापि \Ed\oo 
\textbf{बहिष्कृतम्}\lem \msCa\msCb\msNa\msNc\msParis, बहिष्कृतः \msCc\Ed, नहिष्कृतं \msNb, 
बहिस्कृतंः \msPaperA}}% 

%Verse 8:26

{\devanagarifont अनर्थाप्रियवाक्यं यत् तन्मिथ्यावचनं स्मृतम् {॥ ८:२६॥} \veg\dontdisplaylinenum }%
     \var{{\devanagarifont \numnoemph\vc\textbf{अनर्था॰}\lem \msCa\msCb\msNa\msNb\msNc\msParis, अनर्थ॰ \msCc\msPaperA\Ed}}% 
    \var{{\devanagarifont \numnoemph\vcd\textbf{॰वाक्यं यत्तन्मि॰}\lem \msCa\msCb\msNa\msParis\msPaperA, 
वक्तार तं मि॰ \msCc, 
वाक्य यत्तन्मि॰ \msNb, 
वाक्यं यन्तन्मि॰ \msNc\Ed}}% 
    \var{{\devanagarifont \numnoemph\vd\textbf{स्मृतम्}\lem \mssALL, स्मृतः \msCb}}% 


\alalalfejezet{पिशुनः}

{\devanagarifont परश्रीं नाभिनन्दन्ति परस्यैश्वर्यमेव च \thinspace{\dandab} \dontdisplaylinenum }%
     \var{{\devanagarifont \numemph\va\textbf{परश्रीं ना॰}\lem \msCa\msCb\msNa\msNc\msParis, परस्त्री ना॰ \msCc\msPaperApcorr\Ed, 
परस्त्रीन्ना॰ \msNb, परस्त्री श्री ना॰ \msPaperAacorr\oo 
\textbf{॰भिनन्दन्ति}\lem \mssALL, 
॰भिनन्ति \msCb, ॰भिन्नन्दन्ति \msCc}}% 
    \var{{\devanagarifont \numnoemph\vb\textbf{परस्यैश्वर्य॰}\lem \mssALL, 
परसैश्वर्य॰ \msCb}}% 

%Verse 8:27

{\devanagarifont अनिष्टदर्शनाकाङ्क्षी पिशुनः समुदाहृतः {॥ ८:२७॥} \veg\dontdisplaylinenum }%
     \var{{\devanagarifont \numnoemph\vc\textbf{॰दर्शना॰}\lem \msCa\msCb\msNa\msNc\msParis\Ed, ॰द\uncl{ब्भ}ना॰ \msCc, ॰दर्शनां \msNb, 
॰दशना॰ \msPaperA}}% 
    \var{{\devanagarifont \numnoemph\vd\textbf{पिशुनः}\lem \mssALL, पिशुन \msCc}}% 


\alalalfejezet{पारुष्यम्}

{\devanagarifont मृतमाता पिता चैव हानिस्थानं कथं भवेत् \thinspace{\dandab} \dontdisplaylinenum }%
     \var{{\devanagarifont \numemph\va\textbf{मृत॰}\lem \mssALL, मृता \msParispcorr}}% 
    \var{{\devanagarifont \numnoemph\vb\textbf{॰स्थानं}\lem \mssALL, ॰स्थान \msCb\msCc}}% 

%Verse 8:28

{\devanagarifont भुङ्क्ष्व कामममृष्टानां पारुष्यं समुदाहृतम् {॥ ८:२८॥} \veg\dontdisplaylinenum }%
     \var{{\devanagarifont \numnoemph\vc\textbf{भुङ्क्ष्व}\lem\msNc\msParis, भुक्त्व \msCa, भुक्त्वा \msCb\msCc, 
भुं\uncl{क्ष} \msNa, भुक्ष \msNb, 
भु\uncl{क्त} \msPaperA, भुक्ता \Ed\oo 
\textbf{कामममृष्टानां}\lem \msCa\msNa\msNc\msParis\Ed, कममसृष्टानां \msCb, 
कामसुसमृष्तानां \msCc, 
काममुमृष्ताना \msNb, 
पारुष्यमृष्टना \msPaperA}}% 


\alalalfejezet{तीक्ष्णवाक्}

{\devanagarifont हृदि न स्फुटसे मूढ शिरो वा न विदार्यसे \thinspace{\dandab} \dontdisplaylinenum }%
     \var{{\devanagarifont \numemph\va\textbf{स्फुटसे}\lem \mssALL, स्फुटय \msNb}}% 

%Verse 8:29

{\devanagarifont एवमादीन्यनेकानि तीक्ष्णवादी स उच्यते {॥ ८:२९॥} \veg\dontdisplaylinenum }%
 

\alalalfejezet{असत्प्रलापः}

{\devanagarifont द्यूतभोजनयुद्धं च मद्यस्त्रीकथमेव च \thinspace{\dandab} \dontdisplaylinenum }%
     \var{{\devanagarifont \numemph\va\textbf{॰युद्धं}\lem \mssALL, ॰युद्धश् \Ed}}% 
    \var{{\devanagarifont \numnoemph\vb\textbf{॰कथ॰}\lem \msNb\msNc, ॰कष॰ \mssCaCbCc\msNa\msParis, ॰कर्ष॰ \msPaperA\Ed}}% 

%Verse 8:30

{\devanagarifont असत्प्रलापः पञ्चैतत्कीर्तितं मे द्विजोत्तम {॥ ८:३०॥} \veg\dontdisplaylinenum }%
     \var{{\devanagarifont \numnoemph\vcd\textbf{पञ्चैतत्की॰}\lem \mssALL, 
पञ्चैते की॰ \msNb, पञ्चेतत्की॰ \msNc}}% 
    \var{{\devanagarifont \numnoemph\vd\textbf{मे}\lem \mssALL, ते \Ed}}% 

{\devanagarifont मौनमेव सदा कार्यं वाक्यसौभाग्यमिच्छता \thinspace{\dandab} \dontdisplaylinenum }%
     \var{{\devanagarifont \numemph\va\textbf{कार्यं}\lem \mssALL, कार्या \msNb}}% 
    \var{{\devanagarifont \numnoemph\vb\textbf{वाक्य॰}\lem \msCa\msCb\msNa\msNc\msParis\Ed, वाक्यं \msCc\msNb\msPaperA\oo 
\textbf{॰सौभाग्य॰}\lem \mssALL, ॰सौभार्य॰ \msCb}}% 

%Verse 8:31

{\devanagarifont अपारुष्यमसम्भिन्नं वाक्यं सत्यमुदीरयेत् {॥ ८:३१॥} \veg\dontdisplaylinenum }%
     \var{{\devanagarifont \numnoemph\vc\textbf{॰भिन्नं}\lem \mssALL, ॰भिन्न \msCc, ॰दिग्धं \Ed}}% 

{\devanagarifont यस्तु मौनस्य नो कर्ता दूषितः स कुलाधमः \thinspace{\dandab} \dontdisplaylinenum }%
     \var{{\devanagarifont \numemph\vb\textbf{दूषितः}\lem \mssALL, दूषित \msCc, भूषितः \Ed}}% 

%Verse 8:32

{\devanagarifont जन्मे जन्मे च दुर्गन्धो मूकश्चैवोपजायते {॥ ८:३२॥} \veg\dontdisplaylinenum }%
     \var{{\devanagarifont \numnoemph\vc\textbf{जन्मे जन्मे}\lem \msCb\msCc\msNa\msPaperA\Ed, जन्म जन्म \msCa\msNb\msNc\msParis\oo 
\textbf{दुर्गन्धो}\lem \msCa\msNb\msNc\msParis\msPaperA, 
दुग्गन्धो \msCb, दुर्गन्धा \msCc, दुगन्धो \msNa, दृगन्धो \Ed}}% 

\nemslokalong


\ujvers\nemsloka {
{\devanagarifont तस्मान्मौनव्रतं सदैव सुदृढं कुर्वीत यो निश्चितं }%
  \dontdisplaylinenum}    \var{{\devanagarifont \numemph\va\textbf{तस्मान्मौ॰}\lem \msCc\msNb\msNc\msParis\msPaperA\Ed, \lk\lk त्मौ॰ \msCa, तस्मात्मौ॰ \msCb\msNa\oo 
\textbf{सदैव}\lem \msCa\msCb\msNa\msParis\Ed, सदेव \msCc\msNc\msPaperA, सुदैत्य \msNb\oo 
\textbf{कुर्वीत यो निश्चितम्}\lem \msCa\msCb\msNc\msParis\msPaperA\Ed, कुर्वन्ति येन्निश्चितम् \msCc\msNa, 
कुर्वन्ति योन्निश्चित \msNb}}% 


\nemslokab

{\devanagarifont वाचा तस्य अलङ्घ्यता च भवति सर्वां सभां नन्दति  \danda\dontdisplaylinenum }%
     \var{{\devanagarifont \numnoemph\vb\textbf{अलङ्घ्यता च}\lem \msCa\msCb\msNa\msNb\msParis, अलंघ्यताञ्च \msCc\msNc\msPaperA\Ed\oo 
\textbf{सर्वां सभां}\lem \msCa\msNa\msParis\msPaperA\Ed, सर्वा सभा \msCb\msNc, सर्वः सभान् \msCc, 
सर्वा सुभा \msNb}}% 

\nemslokac

{\devanagarifont वक्त्राच्चोत्पलगन्धमस्य सततं वायन्ति गन्धोत्कटाः }%
  \dontdisplaylinenum    \var{{\devanagarifont \numnoemph\vc\textbf{वक्त्राच्चोत्पलगन्धमस्य}\lem \msCa\msCb\msNc\msParisacorr\msPaperA, वक्त्रं चोत्पलमस्य \msCc, 
वक्त्रं चोत्पलगन्धमस्य \msNa, वक्त्रं चोत्पल\uncl{ग}न्धमस्य \msNb, 
वक्त्राश्चोत्पलगन्धमस्य \msParispcorr, 
वक्त्राच्चोतरगन्धमस्य \Ed}}% 

%Verse 8:33


\nemslokad

{\devanagarifont शास्त्रानेकसहस्रशो गिरि नरः प्रोच्चार्यते निर्मलम् {॥ ८:३३॥} \veg\dontdisplaylinenum }%
     \var{{\devanagarifont \numnoemph\vd\textbf{॰सहस्रशो}\lem \mssALL, ॰सहस्राशो \msCb\oo 
\textbf{॰मलम्}\lem \msCa\msNa\msNb\msNc\msParis, ॰मलः \msCb\msCc\msPaperA\Ed}}% 

\nemslokanormal


\vers



\alalfejezet{नियमेषु स्नानम् (१०)}
{\devanagarifont स्नानं पञ्चविधं चैव प्रवक्ष्यामि यथातथम् \thinspace{\dandab} \dontdisplaylinenum }%
     \var{{\devanagarifont \numemph\va\textbf{पञ्चविधं}\lem \mssALL, पञ्चवि \msCb}}% 
    \var{{\devanagarifont \numnoemph\vb\textbf{यथातथम्}\lem \mssALL, \lk\lk तथम् \msCa}}% 

%Verse 8:34

{\devanagarifont आग्नेयं वारुणं ब्राह्म्यं वायव्यं दिव्यमेव च {॥ ८:३४॥} \veg\dontdisplaylinenum }%
     \var{{\devanagarifont \numnoemph\vc\textbf{आग्नेयं}\lem \mssALL, आग्नेये \msNb\oo 
\textbf{वारुणं}\lem \mssALL, ब्राह्मणं \msPaperA\Ed\oo 
\textbf{ब्राह्म्यं}\lem \mssALL, ब्रह्म्यं \msNc}}% 


\alalalfejezet{आग्नेयं स्नानम्}

{\devanagarifont आग्नेयं भस्मना स्नानं तोयाच्छतगुणं फलम् \thinspace{\dandab} \dontdisplaylinenum }%
     \var{{\devanagarifont \numemph\va\textbf{स्नानं}\lem \mssALL, स्नाना \msNaacorr}}% 
    \var{{\devanagarifont \numnoemph\vb\textbf{॰गुणं}\lem \mssALL, ॰गुण॰ \msNc}}% 

%Verse 8:35

{\devanagarifont भस्मपूतं पवित्रं च भस्म पापप्रणाशनम् {॥ ८:३५॥} \veg\dontdisplaylinenum }%
 
{\devanagarifont तस्माद्भस्म प्रयुञ्जीत देहिनां तु मलापहम् \thinspace{\dandab} \dontdisplaylinenum }%
     \var{{\devanagarifont \numemph\va \lem \mssALL, 
\lk\lk\lk\lk\lk\lk\lk त \msNb}}% 
    \var{{\devanagarifont \numnoemph\vb\textbf{मला॰}\lem \mssALL, पला॰ \msPaperA}}% 

%Verse 8:36

{\devanagarifont सर्वशान्तिकरं भस्म भस्म रक्षकमुत्तमम् {॥ ८:३६॥} \veg\dontdisplaylinenum }%
     \var{{\devanagarifont \numnoemph\vc\textbf{सर्व॰}\lem \mssALL, \uncl{ए}ना॰ \msPaperA}}% 

{\devanagarifont भस्मना त्र्यायुषं कृत्वा ब्रह्मचर्यव्रते स्थितम् \thinspace{\dandab} \dontdisplaylinenum }%
     \var{{\devanagarifont \numemph\va\textbf{त्र्यायुषं कृत्वा}\lem \mssALL, 
त्र्यायु\lk\lk\lk\ \msCa, त्र्यायुष्यं कृत्वा \msParis}}% 
    \var{{\devanagarifont \numnoemph\vb\textbf{॰व्रते}\lem \mssALL, ॰व्रत॰ \msPaperA\Ed}}% 

%Verse 8:37

{\devanagarifont भस्मना ऋषयः सर्वे पवित्रीकृतमात्मनः {॥ ८:३७॥} \veg\dontdisplaylinenum }%
     \var{{\devanagarifont \numnoemph\vc\textbf{ऋषयः सर्वे}\lem \mssALL, ऋषिभिर्सर्वैः \Ed}}% 

{\devanagarifont भस्मना विबुधा मुक्ता वीरभद्रभयार्दिताः \thinspace{\dandab} \dontdisplaylinenum }%
     \var{{\devanagarifont \numemph\va\textbf{मुक्ता}\lem \mssALL, मुक्ताः \Ed}}% 
    \var{{\devanagarifont \numnoemph\vb\textbf{॰र्दिताः}\lem \mssALL, ॰र्त्तिताः \msCb}}% 

%Verse 8:38

{\devanagarifont भस्मानुशंसं दृष्ट्वैव ब्रह्मनानुमतिः कृता {॥ ८:३८॥} \veg\dontdisplaylinenum }%
     \var{{\devanagarifont \numnoemph\vc \lem \corrTorzsok, 
भस्मानुसंसं दृष्ट्यैव \msCa, भस्मानुशंसां दृष्ट्वव \msCb, 
भस्मानुसंसदृष्टैव \msCc\msNb, भस्मानुसंसन्दृष्ट्वैव \msNa, 
भस्मानुशंसंदृष्ट्यैवं \msNc, भस्मानुशंसं दृष्टैव \msParis, 
भस्मानुशंसं \uncl{दृष्टै}व \msPaperA, 
भस्मना शं प्रदृश्यैवं \Ed}}% 
    \var{{\devanagarifont \numnoemph\vd\textbf{ब्रह्मणानुमतिः}\lem \eme, ब्रह्मणानुमता \mssCaCbCc\msNa\msNb\msNc\msParis, 
ब्राह्मणानुमतो \msPaperA\Ed\oo 
\textbf{कृता}\lem \eme, कृतः \msCa\msCb\msNb\msNc\msParis\msPaperA\Ed, कृतिः \msCc, कृताः \msNa}}% 

{\devanagarifont चतुराश्रमतो ऽधिक्यं व्रतं पाशुपतं कृतम् \thinspace{\dandab} \dontdisplaylinenum }%
     \var{{\devanagarifont \numemph\va\textbf{चतुराश्रमतो}\lem \msCb\msCc\msNb\msParis\Ed, चातुराश्रमतो \msCa\msNc\msPaperA, चतुराश्रतो \msNaacorr, 
चातुराश्रमतो \msNapcorr}}% 
    \var{{\devanagarifont \numnoemph\vab\textbf{ऽधिक्यं व्रतं पाशुपतं कृतम्}\lem \mssALL, 
\uncl{धिक्यव्रतपाशुपत}\lk\lk\lk\ \msNb\ \toplost}}% 

%Verse 8:39

{\devanagarifont तस्मात्पाशुपतं श्रेष्ठं भस्मधारणहेतुतः {॥ ८:३९॥} \veg\dontdisplaylinenum }%
     \var{{\devanagarifont \numnoemph\vc \lem \mssALL, \om \msNb}}% 
    \var{{\devanagarifont \numnoemph\vd\textbf{॰हेतुतः}\lem \emeTorzsok, ॰हेतवः \msCa\msCb\msNa\msNc\msParis\msPaperA\Ed, 
॰हेतुना \msCc, ॰हेतुनुतः \msNb}}% 


\alalalfejezet{वारुणं स्नानम्}

{\devanagarifont वारुणं सलिलं स्नानं कर्तव्यं विविधं नरैः \thinspace{\dandab} \dontdisplaylinenum }%
     \var{{\devanagarifont \numemph\va\textbf{वारुणं}\lem \msCb\msCc\msNa\msNb\msParis\Ed, 
वा\lk\lk\  \msCa, वारुणा \msNcacorr, वारुण \msNcpcorr, 
वरुणं \msPaperA\oo 
\textbf{सलिलं}\lem \mssCaCbCc\msNa\msNb\msParis, सलिल॰ \msNc\msPaperA\Ed}}% 
    \var{{\devanagarifont \numnoemph\vb\textbf{विविधं नरैः}\lem \mssCaCbCc\msNa\msPaperA, विविन्नरैः \msNb, 
विधिवन्नरैः \msNc\msParis\Ed}}% 

%Verse 8:40

{\devanagarifont नदीतोयतडागेषु प्रस्रवेषु ह्रदेषु च {॥ ८:४०॥} \veg\dontdisplaylinenum }%
     \var{{\devanagarifont \numnoemph\vc\textbf{॰तडागेषु}\lem \mssALL, 
॰तडागेवा \msNb}}% 
    \var{{\devanagarifont \numnoemph\vd\textbf{प्रस्रवेषु}\lem \mssALL, 
प्रयेवेषु \msNb, प्रभवेषु \msNc}}% 


\alalalfejezet{ब्राह्म्यं स्नानम्}

{\devanagarifont ब्रह्मस्नानं च विप्रेन्द्र आपोहिष्ठं विदुर्बुधाः \thinspace{\dandab} \dontdisplaylinenum }%
     \var{{\devanagarifont \numemph\va\textbf{विप्रेन्द्र}\lem \mssALL, विपेन्द्र \msNc\msParis}}% 
    \var{{\devanagarifont \numnoemph\vb\textbf{विदुर्बु॰}\lem \mssALL, विर्दुर्बु॰ \msNc}}% 

%Verse 8:41

{\devanagarifont त्रिसंध्यमेव कर्तव्यं ब्रह्मस्नानं तदुच्यते {॥ ८:४१॥} \veg\dontdisplaylinenum }%
 

\alalalfejezet{वायव्यं स्नानम्}

{\devanagarifont गोषु संचारमार्गेषु यत्र गोधूलिसम्भवः \thinspace{\dandab} \dontdisplaylinenum }%
 
%Verse 8:42

{\devanagarifont तत्र गत्वावसीदेत स्नानमुक्तं मनीषिभिः {॥ ८:४२॥} \veg\dontdisplaylinenum }%
     \var{{\devanagarifont \numemph\vd\textbf{॰क्तं}\lem \mssALL, ॰क्त \msNb}}% 


\alalalfejezet{दिव्यं स्नानम्}

{\devanagarifont वर्षतोयाम्बुधाराभिः प्लावयित्वा स्वकां तनुम् \thinspace{\dandab} \dontdisplaylinenum }%
     \var{{\devanagarifont \numemph\vb\textbf{तनुम्}\lem \mssALL, तनं \msNc}}% 

%Verse 8:43

{\devanagarifont स्नानं दिव्यं वदत्येव जगदादिमहेश्वरः {॥ ८:४३॥} \veg\dontdisplaylinenum }%
     \var{{\devanagarifont \numnoemph\vc\textbf{दिव्यं}\lem \mssALL, दिव्य \msNb\msPaperA}}% 
    \var{{\devanagarifont \numnoemph\vd\textbf{जगदादि॰}\lem \mssALL, गजदादि॰ \msCb}}% 

\ujvers\nemsloka {
{\devanagarifont इति नियमविभागः पञ्चभेदेन विप्र }%
  \dontdisplaylinenum}    \var{{\devanagarifont \numemph\va\textbf{॰भागः}\lem \mssALL, ॰भागं \msNc}}% 


\nemslokab

{\devanagarifont निगदित तव पृष्टः सर्वलोकानुकम्प्य  \danda\dontdisplaylinenum }%
     \var{{\devanagarifont \numnoemph\vb\textbf{निगदित तव}\lem \Ed, 
निगदितस्तव \mssCaCbCc\msNa\msNb\msNc\msParis\msPaperA\ \unmetr\oo 
\textbf{॰कम्प्य}\lem \msCa, ॰कम्प \msCb\msCc\msNa\msNc\msParis, 
॰कम्पः \msNb, ॰कम्प्यः \msPaperA\Ed}}% 

\nemslokac

{\devanagarifont सकलमलपहारी धर्मपञ्चाशदेतन् }%
  \dontdisplaylinenum    \var{{\devanagarifont \numnoemph\vc\textbf{॰पहारी}\lem \msCb\msCc\msNb, ॰पहारि \msCa\msNc\unmetr, ॰प्रहारि \msNa\msParis\msPaperA, 
॰पहारे \Ed\oo 
\textbf{॰पञ्चाशदेतन्}\lem \msCa\msCb\msNa\msNbpcorr\msNc\msParis, 
॰पञ्चाशमेतन् \msCc\msPaperA\Ed, 
॰पञ्चादेतन् \msNbacorr}}% 

%Verse 8:44


\nemslokad

{\devanagarifont न भवति पुनजन्म कल्पकोट्यायुते ऽपि {॥ ८:४४॥} \veg\dontdisplaylinenum }%
     \var{{\devanagarifont \numnoemph\vd\textbf{पुनजन्म}\lem \msCc\msNb, पुनर्जन्म \msCa\msNa\msNc\msParis\msPaperA\Ed, 
पुन\uncl{र्जर्म} \msCb}}% 

\vers


{\devanagarifont 
\jump
\begin{center}
\ketdanda~इति वृषसारसंग्रहे नियमप्रशंसा नामाध्यायो ऽष्टमः~\ketdanda
\end{center}
\dontdisplaylinenum\vers  }%
     \var{{\devanagarifont \numnoemph{\englishfont \Colo:}\textbf{इति वृषसारसंग्रहे नियमप्रशंसा नामाध्यायो ऽष्टमः}\lem \msParis, 
इति वृषसारसंग्रहे नियमप्रशंसा नामाध्याय अष्टमः \msCa\msNa\msPaperA, 
\om \msCb, 
इति वृषसारसंग्रहे नियमप्रशंसा नामाध्यायाष्टमः \msCc\msNb, 
इति वृषसारसंग्रहे नियमप्रशंसा नामाध्यायाऽष्टमः \msNc, 
इति वृषसारसंग्रहे नियमप्रशंसा नाम अष्टमो ऽध्यायः \Ed}}% 
