\fejno=0\versno=0
\centerline{\Huge\devanagarifontbold वृषसारसंग्रहः  }

 
{\vrule depth10pt width0pt}
\versno=0\fejno=1
\thispagestyle{empty}

\centerline{\Large\devanagarifontbold [   प्रथमो ऽध्यायः  ]}{\vrule depth10pt width0pt} \fancyhead[CO]{{\footnotesize\devanagarifont वृषसारसंग्रहे  }}
\fancyhead[CE]{{\footnotesize\devanagarifont प्रथमो ऽध्यायः  }}
\fancyhead[LE]{}
\fancyhead[RE]{}
\fancyhead[LO]{}
\fancyhead[RO]{}
\szam\bek


\vers



\alalfejezet{स्तुतिः}
\ujvers\nemsloka {
{\devanagarifont अनादिमध्यान्तमनन्तपारं }%
  \dontdisplaylinenum}    \var{{\devanagarifont \numemph\va\textbf{॰न्तमनन्त॰}\lem \mssALL, 
॰न्तमन्त॰ \msCbacorr\oo 
\textbf{॰पारं}\lem \mssCaCbCc\msNc\msM\msPaperA\msPaperC\Ed, ॰पारगं \msNa\msNb\msNd\msKOa}}% 
    \paral{{\devanagarifontsmall \va {\englishfont \compare\ \SDHU\ 10.6:}
                 आदिमध्यान्तनिर्मुक्तः स्वभावविमलः प्रभुः\thinspace{\devanagarifontsmall ।}
                 सर्वज्ञः परिपूर्णश्च शिवो ज्ञेयः शिवागमे\thinspace{\devanagarifontsmall ॥} }}


\nemslokab

{\devanagarifont सुसूक्ष्ममव्यक्तजगत्सुसारम्  \danda\dontdisplaylinenum }%
     \var{{\devanagarifont \numnoemph\vb\textbf{सुसूक्ष्म॰}\lem \mssALL, 
शुसुक्ष्म॰ \msCc\oo 
\textbf{॰व्यक्त॰}\lem \mssALL, ॰व्य॰ \msKOa\oo 
\textbf{॰जगत्सुसारम्}\lem \msCa\msCb\msNa\msNc\msM\msKOa\msPaperA\msPaperC\Ed, ॰जगशुसारं \msCc, 
॰जगत्सुरासुरं \msNb, 
॰जगतसुसारम् \msNd}}% 

\nemslokac

{\devanagarifont हरीन्द्रब्रह्मादिभिरासमग्रं }%
  \dontdisplaylinenum    \var{{\devanagarifont \numnoemph\vc\textbf{हरी॰}\lem \mssALL, हरीं \msKOa\oo 
\textbf{॰भिरासमग्रं}\lem \mssALL, 
॰भिर्यत्समग्रं \msM\ \unmetr, 
॰भिरोसमग्रं \msPaperC}}% 

%Verse 1:1


\nemslokad

{\devanagarifont प्रणम्य वक्ष्ये वृषसारसंग्रहम् {॥ १:१॥} \veg\dontdisplaylinenum }%
     \var{{\devanagarifont \numnoemph\vd\textbf{वृष॰}\lem \mssALL, 
॰वृषो \msCaacorr}}% 
    \lacuna{\devanagarifontsmall {\englishfont Witnesses used for this chapter:       \msCa\ ff.\thinspace 193v--195v,
                                                        \msCb\ ff.\thinspace 201v--203v,
                                                        \msCc\ ff.\thinspace 267r--270r,
                                                        \msNa\ ff.\thinspace 1v--3v,
                                                        \msNb\ exp.\thinspace 44, 43 lower and then upper leaf
                                                                              (1.62cd--2.22 are missing),
                                                        \msNc\  ff.\thinspace 209v--211v,
                                                        \msNd\  ff.\thinspace 227v--229v 
                                                                                (collated only up to 1.15ab),
                                                        \msM\   ff.\thinspace 1r--3v,
                                                        \msKOa\ ff.\thinspace 1v--4r 
                                                                                (collated only up to 1.16),
                                                        \msPaperA\ ff.\thinspace 204r--206r,
                                                        \msPaperC\ ff.\thinspace 206r--209r 
                                                                                (collated only up to 1.15),
                                                        \Ed\ pp.\thinspace 580--585;
                                                        \mssCaCbCc\ = \msCa + \msCb + \msCc} }%
  

\alalfejezet{जनमेजयवैशम्पायनसंवादः}
\vers


{\devanagarifont शतसाहस्रिकं ग्रन्थं सहस्राध्यायमुत्तमम् \thinspace{\dandab} \dontdisplaylinenum }%
     \var{{\devanagarifont \numemph\va\textbf{॰स्रिकं}\lem \mssALL, 
॰स्रकं \msPaperA\oo 
\textbf{ग्रन्थं}\lem \mssALL, 
ग्रंथ \msKOa}}% 
    \var{{\devanagarifont \numnoemph\vb\textbf{सहस्राध्यायमु॰}\lem \mssALL, 
सहश्रध्यायमु॰ \msCc, 
सहस्राध्यायरु॰ \Ed}}% 

%Verse 1:2

{\devanagarifont पर्व चास्य शतं पूर्णं श्रुत्वा भारतसंहिताम् {॥ १:२॥} \veg\dontdisplaylinenum }%
     \var{{\devanagarifont \numnoemph\vc\textbf{पर्व चास्य}\lem \msCa\msNa\msNb\msNc\msMpcorr, पर्वञ्चास्य \msCb, 
पर्वमस्य \msCc\msNd\msMacorr\msPaperA\msPaperC\Ed, पूर्व चास्य \msKOa\oo 
\textbf{शतं पूर्णं}\lem \mssALL, 
त \msCc, शतं पूर्ण्ण \msKOa}}% 
    \var{{\devanagarifont \numnoemph\vd\textbf{श्रुत्वा}\lem \mssALL, 
श्रद्धा \msCb\oo 
\textbf{भारतसंहिताम्}\lem \msCa\msCb\msNa\msNb\msNc\msM\msKOa, 
भारसंहिता \msCc, भारतसंहितं \msNd, 
नारदसंहिताम् \msPaperA\msPaperC\Ed}}% 
    \paral{{\devanagarifontsmall \vc {\englishfont \compare\ \MBH\ 1.2.70ab:} एतत्पर्वशतं पूर्णं व्यासेनोक्तं महात्मना }}

\vers


{\devanagarifont अतृप्तः पुन पप्रच्छ वैशम्पायनमेव हि \thinspace{\dandab} \dontdisplaylinenum }%
     \var{{\devanagarifont \numemph\va \lem \eme, 
अ\uncl{तृप्तः पु}\lk\lk प्रच्छ \msCa, 
अतृप्तः पुनः पप्रच्छ \msCb\msNa\msNb\msNc, 
अतृप्तः पुनरप्रच्छे \msCc, 
अतृप्तः पुन पःप्रच्छ \msNd, 
अतृप्तः पुनः पपृच्छ \msM, 
पप्रच्छ पुनरतृप्तो \msKOa, 
अतृप्ताः पुनः पप्रेच्छ \msPaperA, 
अतृप्त पुनः पप्रच्छ \msPaperC, 
अतृप्ता पुनः पप्रच्छ \Ed}}% 
    \var{{\devanagarifont \numnoemph\vb\textbf{वैशम्पायन॰}\lem \mssALL, 
वेसम्पायन॰ \msCc}}% 

%Verse 1:3

{\devanagarifont जनमेजयेन यत्पूर्वं तच्छृणु त्वमतन्द्रितम् {॥ १:३॥} \veg\dontdisplaylinenum }%
     \var{{\devanagarifont \numnoemph\vc \lem \msCapcorr\msCb\msNc\msNd\msPaperA\msPaperC\Ed, 
जनमेजये यत्पूर्वं \msCaacorr, 
जन्मेजयेन यम्पूर्वं \msCc, 
जनमेजयेन यत्पूर्व \msNa, 
जनमेजयेन यत्पू\uncl{र्व} \msNb, 
जन्मेजयेण यत्पूर्वं \msM, 
जन्मेजयेन य\lac\ \msKOa}}% 
    \var{{\devanagarifont \numnoemph\vd\textbf{तच्छृणु त्वम॰}\lem \msCa\msCb\msNa\msNc\msM\msPaperA\msPaperC\Ed, 
तच्छृण त्वम॰ \msCc, \lac\  \msNb, तच्छृणु स्वम॰ \msNd, 
त शृणु त्वम॰ \msKOa\oo 
\textbf{॰तन्द्रितम्}\lem \msCa\msCb\msNc\msNd\msM\msKOa\msPaperA\msPaperC\Ed, ॰तन्द्रितः \msCc\msNa, 
\lac\  \msNb}}% 

{\devanagarifont जनमेजय उवाच {\dandab}\dontdisplaylinenum  }%
     \var{{\devanagarifont \numemph\vo\textbf{जनमेजय}\lem \mssALL, 
जन्मेजय \msCc}}% 

{\devanagarifont भगवन्सर्वधर्मज्ञ सर्वशास्त्रविशारद \thinspace{\danda} \dontdisplaylinenum }%
     \var{{\devanagarifont \numnoemph\va\textbf{भगवन्स॰}\lem \msCa\msCb\msNa\msNb\msNc\msKOa\msPaperA\msPaperC\Ed, 
भचावं स॰ \msCc, भगव स॰ \msNd, 
भगवं स॰ \msM\oo 
\textbf{॰धर्मज्ञ}\lem \mssALL, 
॰ज्ञ \msNa, ॰धर्मज्ञः \msNd}}% 
    \var{{\devanagarifont \numnoemph\vb\textbf{॰विशारद}\lem \msCa\msNb\msNc\msNd\msPaperA, 
॰विसारदः \msCb\msCc\msNa\msKOa\msPaperC\Ed, ॰विशारदम् \msM}}% 
    \paral{{\devanagarifontsmall \vab {\englishfont = \MBH\ 13.112.9ab} }}

%Verse 1:4

{\devanagarifont अस्ति धर्मं परं गुह्यं संसारार्णवतारणम् {॥ १:४॥} \veg\dontdisplaylinenum }%
     \var{{\devanagarifont \numnoemph\vc\textbf{अस्ति धर्मं}\lem \msCa\msNa\msNb\msNc\msPaperA\msPaperC\Ed, अस्ति धर्मः \msCb, 
अस्ति धर्म \msCc\msM\msKOa, अधर्म \msNd\oo 
\textbf{परं गुह्यं}\lem \msCa\msNb\msNd\msM\msKOa\msPaperA\msPaperC\Ed, 
परो गुह्य \msCb, परं गुह्य \msCc\msNa, 
परगुह्यं \msNc}}% 
    \var{{\devanagarifont \numnoemph\vd\textbf{॰तारणम्}\lem \mssALL, 
॰तारणा \msKOa}}% 

{\devanagarifont द्वैपायनमुखोद्गीर्णं धर्मं वा यद्द्विजोत्तम \thinspace{\dandab} \dontdisplaylinenum }%
     \var{{\devanagarifont \numemph\va\textbf{द्वैपायन॰}\lem \mssALL, 
द्वेपायन॰ \msCc, 
वैसांपायन॰ \msKOa\oo 
\textbf{॰मुखोद्गीर्णं}\lem \msCa\msCb\msNa\msNb\msNc\msPaperA\msPaperC, 
॰मुखोद्गीर्ण \msCc\msKOa, 
॰मुद्गीर्ण्ण \msNd, 
मुखं गीर्ण्णं \msMacorr, 
मु\uncl{खां} गीर्ण्णं \msMpcorr, 
मुखाद्गीर्णं \Ed}}% 
    \var{{\devanagarifont \numnoemph\vb\textbf{धर्मं वा यद्द्वि॰}\lem \msCa\msNa\msNb\msNc\msPaperA\msPaperC\Ed, 
धर्मं यत्तद्द्वि॰ \msCb, 
धर्मवत्य द्वि॰ \msCc\msKOa, धर्म वा यद्द्वि॰ \msNd, 
धर्मवाक्यं द्वि॰ \msM\oo 
\textbf{॰त्तम}\lem \mssALL, 
॰त्तमः \msCc, ॰तमः \msM}}% 

%Verse 1:5

{\devanagarifont कथयस्व हि मे तृप्तिं कुरु यत्नात्तपोधन {॥ १:५॥} \veg\dontdisplaylinenum }%
     \var{{\devanagarifont \numnoemph\vc\textbf{हि मे तृप्तिं}\lem \mssCaCbCc\msNa\msNb\msNc\msPaperA\msPaperC\Ed, 
हि मे तृप्ति \msNd\msKOa, 
प्रसादेन \msM}}% 
    \var{{\devanagarifont \numnoemph\vd\textbf{यत्नात्तपोधन}\lem \msCb\msNa\msNb\msNc\msPaperA\msPaperC\Ed, 
यन्नात्त\lk\lk न \msCa, 
यत्ना तपोधनः \msCc, यत्ना तपोधन \msNd, 
यत्नन्तपोधन \msM, यंनात्त॰ \msKOa}}% 

\vfill
\pageparbreak
\vers

{\devanagarifont वैशम्पायन उवाच {\dandab}\dontdisplaylinenum  }%
     \var{{\devanagarifont \numemph\vo\textbf{वैशम्पायन उवाच}\lem \mssALL, 
\om\ \msMacorr, वै\thinspace{\devanagarifont ॥} वैशम्पायन \msPaperC}}% 

{\devanagarifont शृणु राजन्नवहितो धर्माख्यानमनुत्तमम् \thinspace{\danda} \dontdisplaylinenum }%
     \var{{\devanagarifont \numnoemph\va\textbf{राजन्न॰}\lem \mssALL, 
राजंन॰ \msNd, राजन॰ \msM\oo 
\textbf{॰हितो}\lem \mssALL, 
॰हितं \msPaperA}}% 
    \var{{\devanagarifont \numnoemph\vb\textbf{॰ख्यानमनुत्तमम्}\lem \msCa\msNa\msNb\msNc\msM\Ed, ॰ख्यानमुत्तमम् \msCb, 
॰ख्यानमुतमम् \msCc, ॰धर्मव्याख्यानमुत्तमं \msNd\ \hypermetr, 
॰ख\lac मनुत्तमं \msKOa, 
॰ख्यानमनुत्तमः \msPaperA, 
॰ख्यानमुत्तमः \msPaperC}}% 

%Verse 1:6

{\devanagarifont व्यासानुग्रहसम्प्राप्तं गुह्यधर्मं शृणोतु मे {॥ १:६॥} \veg\dontdisplaylinenum }%
     \var{{\devanagarifont \numnoemph\vc\textbf{॰प्राप्तं}\lem \mssALL, 
॰प्राप्त \msCc}}% 
    \var{{\devanagarifont \numnoemph\vd\textbf{॰धर्मं}\lem \mssALL, 
॰र्मं \msCc, ॰धर्म \msKOa\oo 
\textbf{शृणोतु}\lem \mssALL, 
शृणोत \msCc\oo 
\textbf{मे}\lem \mssALL, 
मै \msCb}}% 

{\devanagarifont अनर्थयज्ञकर्तारं तपोव्रतपरायणम् \thinspace{\dandab} \dontdisplaylinenum }%
     \var{{\devanagarifont \numemph\va\textbf{॰कर्तारं}\lem \mssALL, 
॰कर्त्तन्तं \msNb}}% 
    \var{{\devanagarifont \numnoemph\vb\textbf{॰व्रत॰}\lem \mssALL, 
॰प्रत॰ \msM\oo 
\textbf{॰यणम्}\lem \msCa\msCb\msNb\msM\msKOa\msPaperA\msPaperC\Ed, 
॰यन \msCc, ॰यणः \msNa, 
॰यनं \msNc, ॰\uncl{यणं} \msNd}}% 

%Verse 1:7

{\devanagarifont शीलशौचसमाचारं सर्वभूतदयापरम् {॥ १:७॥} \veg\dontdisplaylinenum }%
     \var{{\devanagarifont \numnoemph\vc\textbf{॰चारं}\lem \mssALL, 
॰चार \msKOa}}% 
    \var{{\devanagarifont \numnoemph\vd\textbf{॰परम्}\lem \msCa\msCb\msNa\msNc\msM\msPaperA\msPaperC\Ed, ॰न्वितम् \msCc\msNd\msKOa, 
॰\uncl{प}रं \msNb}}% 

{\devanagarifont जिज्ञासनार्थं प्रश्नैकं विष्णुना प्रभविष्णुना \thinspace{\dandab} \dontdisplaylinenum }%
     \var{{\devanagarifont \numemph\va\textbf{॰र्थं प्रश्नैकं}\lem \msCb\msNa\msNb\msNc, ॰र्थं प्रश्नेकं \msCa\msNd, 
॰र्थप्रश्नेकम् \msCc\msPaperA\msPaperC\Ed, ॰र्थप्रश्चैकं \msM, 
॰थप्रश्नैक \msKOa}}% 
    \var{{\devanagarifont \numnoemph\vb\textbf{प्रभ॰}\lem \mssALL, 
प्रभु॰ \msCc, प्राभ॰ \msNc}}% 

%Verse 1:8

{\devanagarifont द्विजरूपधरो भूत्वा पप्रच्छ विनयान्वितः {॥ १:८॥} \veg\dontdisplaylinenum }%
     \var{{\devanagarifont \numnoemph\vc\textbf{॰धरो}\lem \mssALL, 
॰\lk रो \msCa, ॰धरा \msNb}}% 
    \var{{\devanagarifont \numnoemph\vd\textbf{॰न्वितः}\lem \msCa\msCb\msNa\msNb\msNc\msKOa\msPaperA\msPaperC\Ed, 
॰न्वितं \msCc\msNd\msM}}% 


\alalfejezet{ब्रह्मविद्या}
{\devanagarifont [विगतराग उवाच {\dandab}\dontdisplaylinenum  ] }%
 
{\devanagarifont ब्रह्मविद्या कथं ज्ञेया रूपवर्णविवर्जिता \thinspace{\danda} \dontdisplaylinenum }%
     \var{{\devanagarifont \numemph\va\textbf{कथं}\lem \mssALL, कथ \msKOa\oo 
\textbf{ज्ञेया}\lem \msCa\msNa\msNb\msNc\msM\msKOa\msPaperA\msPaperC, 
ज्ञेयं \msCb\msCc, ज्ञेय \msNd, भूयो \Ed}}% 
    \var{{\devanagarifont \numnoemph\vb\textbf{॰वर्ण॰}\lem \mssALL, 
॰वर्णा॰ \Ed\oo 
\textbf{॰वर्जिता}\lem \msCa\msCb\msNa\msNb\msNd\msM\msPaperA\msPaperC\Ed, 
॰वर्जितं \msCc, ॰वर्जिताः \msNc, \lac ता \msKOa}}% 

%Verse 1:9

{\devanagarifont स्वरव्यञ्जननिर्मुक्तमक्षरं किमु तत्परम् {॥ १:९॥} \veg\dontdisplaylinenum }%
     \var{{\devanagarifont \numnoemph\vc\textbf{॰व्यञ्जन॰}\lem \mssALL, 
॰व्यज्जन॰ \Ed}}% 
    \var{{\devanagarifont \numnoemph\vcd\textbf{॰मुक्तमक्ष॰}\lem \msCa\msCc\msNa\msNb\msNc\msPaperC\Ed, ॰मुक्त अक्ष॰ \msCb\msKOa, 
॰मुक्तं अख॰ \msNd, ॰मुक्तं अक्ष॰ \msM, ॰म्मुक्तंमक्ष॰ \msPaperA}}% 
    \var{{\devanagarifont \numnoemph\vd\textbf{किमु तत्परम्}\lem \msCa\msNa\msNc\msKOa\msPaperA\msPaperC\Ed, 
किमतः परम् \msCb\msCc, 
किमतत्परं \msNb\msNd\msM}}% 

\vfill
\pageparbreak
\vers

{\devanagarifont अनर्थयज्ञ उवाच {\dandab}\dontdisplaylinenum  }%
 
{\devanagarifont अनुच्चार्यमसन्दिग्धमविच्छिन्नमनाकुलम् \thinspace{\danda} \dontdisplaylinenum }%
     \var{{\devanagarifont \numemph\va\textbf{अनुच्चार्य॰}\lem \msCa\msCb\msNa\msNb\msM\msPaperA\msPaperC\Ed, 
अनुचार्य॰ \msCc\msNc\msNd, 
अन्त्रचाय॰ \msKOa}}% 
    \var{{\devanagarifont \numnoemph\vab\textbf{॰सन्दिग्धमविच्छिन्नमनाकुलम्}\lem \msCa\msCb\msNa\msNc\msNd\msM\msPaperA\msPaperC\Ed, 
॰विच्छिन्नसन्दिग्धमनाकुन \msCc, ॰सन्दिग्धमनच्छिन्नमनाकुलम् \msNb, 
॰सन्दिग्धमविच्छिनमनाकुलं \msKOa}}% 

%Verse 1:10

{\devanagarifont निर्मलं सर्वगं सूक्ष्ममक्षरं किमतः परम् {॥ १:१०॥} \veg\dontdisplaylinenum }%
     \var{{\devanagarifont \numnoemph\vc\textbf{॰गं}\lem \mssALL, ॰ग \msKOa}}% 
    \var{{\devanagarifont \numnoemph\vc\textbf{॰क्षरं किमतः परम्}\lem \msCb\msM, ॰क्षरं किमु तत्परम् \msCa\msNa\msNb\msNc\Ed, 
॰क्षरं किमतत्परं \msCc\msNd\msPaperC, 
॰क्षर किमतः परं \msKOa, 
॰क्षराङ्कमतत्परं \msPaperA}}% 


\alalfejezet{कालपाशः}
{\devanagarifont विगतराग उवाच {\dandab}\dontdisplaylinenum  }%
     \var{{\devanagarifont \numemph\vo\textbf{॰राग उवाच}\lem \mssALL, ॰रागोवाच \msNd}}% 

{\devanagarifont देही देहे क्षयं याते भूजलाग्निशिवादिभिः \thinspace{\danda} \dontdisplaylinenum }%
     \var{{\devanagarifont \numnoemph\va\textbf{देहे क्ष॰}\lem \msCa\msCc\msNc, देहात्क्ष॰ \msCb, 
देहक्ष॰ \msNa\msNb\msNd\msM\msKOa\msPaperA\msPaperC\Ed\oo 
\textbf{याते}\lem \mssALL, यान्ते \msNd}}% 
    \var{{\devanagarifont \numnoemph\vb\textbf{॰जलाग्निशिवादिभिः}\lem \msCa\msCb\msNa\msNb\msNc\msM\msPaperA\msPaperC\Ed, 
॰जलाग्निशिवादिभि \msCc, 
॰जलाग्निं शि\lk दिभि \msNd, ॰जालादिशिवादिभिः \msKOa}}% 
    \paral{{\devanagarifontsmall \vb {\englishfont \compare\ \KURMP\ 2.23.74:} 
                 अथ कश्चित्प्रमादेन म्रियते ऽग्निविषादिभिः\thinspace{\devanagarifontsmall ।} 
                 तस्याशौचं विधातव्यं कार्यं चैवोदकादिकम्\thinspace{\devanagarifontsmall ॥} }}

%Verse 1:11

{\devanagarifont यमदूतैः कथं नीतो निरालम्बो निरञ्जनः {॥ १:११॥} \veg\dontdisplaylinenum }%
     \var{{\devanagarifont \numnoemph\vc\textbf{॰दूतैः}\lem \mssALL, 
॰दूते \msCc\msNd\oo 
\textbf{कथं}\lem \mssALL, 
कथ \msKOa\oo 
\textbf{नीतो}\lem \msCa\msCb\msNa\msNb\msNc\msNd, नीत्वा \msCc, नीतः \msM, नीते \msKOa, 
नीता \msPaperA\msPaperC\Ed}}% 
    \var{{\devanagarifont \numnoemph\vd\textbf{निरालम्बो}\lem \mssALL, 
निरोलया \msPaperA, निरोरैन्वो \msPaperC\oo 
\textbf{निरञ्जनः}\lem \mssALL, 
निरञ्जन \msCc, 
निरञ्ज\lk\ \msKOa}}% 

{\devanagarifont कालपाशैः कथं बद्धो निर्देहश्च कथं व्रजेत् \thinspace{\dandab} \dontdisplaylinenum }%
     \var{{\devanagarifont \numemph\va\textbf{॰पाशैः}\lem \mssALL, 
॰पाशे \msCc, ॰पाशै \msNd\oo 
\textbf{बद्धो}\lem \mssALL, 
ब\uncl{द्धो} \msCb, बद्ध \msNd}}% 
    \var{{\devanagarifont \numnoemph\vb\textbf{निर्देहश्च}\lem \msCa\msCb\msNa\msNb\msNc\msMpcorr\msPaperA\msPaperC\Ed, 
निर्दहः स \msCc, निर्देहस्य \msNd, 
निर्देहन्म \msMacorr, निदेहश्च \msKOa\oo 
\textbf{व्रजेत्}\lem \mssALL, भवेत् \msNb}}% 

{\devanagarifont स्वर्गं वा स कथं याति निर्देहो बहुधर्मकृत्  \danda\dontdisplaylinenum }%
     \var{{\devanagarifont \numnoemph\vc\textbf{स्वर्गं}\lem \msCa\msCb\msNa\msNb\msNc\msPaperA\msPaperC\Ed, 
स्वर्ग \msCc\msNd\msM, स्वागं \msKOa\oo 
\textbf{स}\lem \mssALL, 
सं \msNb\msM\oo 
\textbf{याति}\lem \msNa\msNb\msNc\msNd\msM\msKOa\msPaperA\msPaperC, 
यान्ति \mssCaCbCc\Ed}}% 
    \var{{\devanagarifont \numnoemph\vd\textbf{निर्देहो}\lem \mssALL, 
निदेहो \msKOa}}% 

%Verse 1:12

{\devanagarifont एतन्मे संशयं ब्रूहि ज्ञातुमिच्छामि तत्त्वतः {॥ १:१२॥} \veg\dontdisplaylinenum }%
     \var{{\devanagarifont \numnoemph\ve\textbf{एतन्मे संशयं}\lem \mssCaCbCc\msNc\msM\msPaperA\msPaperC\Ed, 
एतन्मे संशये \msNa, एतन्मे संशयो \msNb\msNd, 
एवं विस्मयसंसय \msKOa}}% 
    \var{{\devanagarifont \numnoemph\vf\textbf{॰तुमि$\-$च्छामि}\lem \mssALL, 
॰तुमि \msCb}}% 

\vfill
\pageparbreak
\vers

{\devanagarifont अनर्थयज्ञ उवाच {\dandab}\dontdisplaylinenum  }%
     \var{{\devanagarifont \numemph\vo\textbf{अनर्थयज्ञ उवाच}\lem \mssALL, 
\om\ \msNaacorr}}% 

{\devanagarifont अतिसंशयकष्टं ते पृष्टो ऽहं द्विजसत्तम \thinspace{\danda} \dontdisplaylinenum }%
     \var{{\devanagarifont \numnoemph\va \lem \msCb\msNa\msNb\msNc\msMpcorr\msPaperC, 
अतिशंस$\-$\uncl{य}कष्टन्ते \msCa, 
अतिशंसयक$\-$ष्टम्मे \msCc\msMacorr\Ed, 
अतिसंशयकष्टो मो \msNd, 
अतिसंसयकष्टञ्च \msKOa, 
अतिसंसयकष्ट\lk न्ते पा \msPaperA}}% 
    \var{{\devanagarifont \numnoemph\vb\textbf{द्विजसत्तम}\lem \msCa\msCb\msNa\msNb\msNc\msM\msPaperA\msPaperC\Ed, 
च द्विजोत्तमः \msCc\msKOa, द्विजसत्तमः \msNd}}% 

%Verse 1:13

{\devanagarifont दुर्विज्ञेयं मनुष्यैस्तु देवदानवपन्नगैः {॥ १:१३॥} \veg\dontdisplaylinenum }%
     \var{{\devanagarifont \numnoemph\vc\textbf{॰ज्ञेयं}\lem \msCa\msCb\msNa\msNc, ॰ज्ञेय \msCc\msNb\msNd\msM\msKOa\msPaperA\msPaperC\Ed\oo 
\textbf{मनुष्यैस्तु}\lem \msCa\msNa\msNb\msNc\msM\msKOa\msPaperA\msPaperC\Ed, 
मनुषैश्च \msCb, मणुक्षे\uncl{प्तु} \msCc, 
मनुष्येस्तु \msNd}}% 

{\devanagarifont कर्महेतु शरीरस्य उत्पत्ति निधनं च यत् \thinspace{\dandab} \dontdisplaylinenum }%
     \var{{\devanagarifont \numemph\va\textbf{कर्म॰}\lem \msCa\msCb\msNa\msNb\msNc\msNd\msM\msKOa, 
अनर्थयज्ञ उवाच\thinspace{\devanagarifont ॥} कर्म॰ \msCc\msPaperA\msPaperC\Ed\oo 
\textbf{॰हेतु}\lem \mssALL, 
॰हेतुः \msCb, ॰हेंतु \msCc\oo 
\textbf{शरीरस्य}\lem \mssALL, 
शरीरस्यं \msCc, 
स\lac \uncl{स्य} \msKOa}}% 
    \var{{\devanagarifont \numnoemph\vb\textbf{उत्पत्ति नि॰}\lem \msCa\msCb\msNa\msNb\msNc\msKOa\msPaperA\msPaperC\Ed, 
उत्पतिनि॰ \msCc\msNd, उत्पत्तिर्नि॰ \msM\oo 
\textbf{च यत्}\lem \mssALL, 
च यः \msNb, यत् \msNd}}% 

%Verse 1:14

{\devanagarifont सुकृतं दुष्कृतं चैव पाशद्वयमुदाहृतम् {॥ १:१४॥} \veg\dontdisplaylinenum }%
     \var{{\devanagarifont \numnoemph\vc\textbf{सुकृतं}\lem \mssALL, 
सुकृतकृतन् \msCc, सुकृत \msNd\oo 
\textbf{चैव}\lem \mssALL, वापि \msNd\msKOa}}% 
    \var{{\devanagarifont \numnoemph\vd\textbf{पाश॰}\lem \mssALL, पासा॰ \msKOa\oo 
\textbf{॰हृतम्}\lem \mssALL, 
॰हृतः \msCc}}% 

{\devanagarifont तेनैव सह संयाति नरकं स्वर्गमेव वा \thinspace{\dandab} \dontdisplaylinenum }%
     \var{{\devanagarifont \numemph\va\textbf{तेनैव}\lem \mssALL, 
तेनेव \msCc\msNd\oo 
\textbf{सह संयाति}\lem \msCa\msCb\msNa\msNb\msNc\msPaperC\Ed, 
सह सा यान्ति \msCc\msNd, सह सा याति \msM, 
सह संयान्ति \msKOa, सहं स याति \msPaperA}}% 
    \var{{\devanagarifont \numnoemph\vb\textbf{नरकं स्वर्ग॰}\lem \mssALL, 
नरकदुर्ग्ग॰ \msKOa\oo 
\textbf{वा}\lem \mssCaCbCc\msNb\msNc\msM\msPaperA\msPaperC\Ed, च \msNa\msNd\msKOa}}% 

%Verse 1:15

{\devanagarifont सुखदुःखं शरीरेण भोक्तव्यं कर्मसम्भवम् {॥ १:१५॥} \veg\dontdisplaylinenum }%
     \var{{\devanagarifont \numnoemph\vc\textbf{सुख॰}\lem \mssALL, सुखं \msM\oo 
\textbf{॰दुःखं}\lem \msCa\msCb\msNa\msNc\msM, ॰दुःख \msCc\msNb\msKOa\msPaperA\msPaperC\Ed}}% 
    \var{{\devanagarifont \numnoemph\vd\textbf{भोक्तव्यं}\lem \mssALL, 
भोक्तव्य \msKOa\oo 
\textbf{॰सम्भवम्}\lem \msCa\msCb\msNa\msNb\msNc\msM, 
॰सम्भवः \msCc\msPaperA\msPaperC\Ed, ॰संभावात् \msKOa}}% 

{\devanagarifont हेतुनानेन विप्रेन्द्र देहः सम्भवते नृणाम् \thinspace{\dandab} \dontdisplaylinenum }%
     \var{{\devanagarifont \numemph\va\textbf{हेतुनानेन}\lem \mssALL, 
हेतुना तेन \msKOa, हेतुनाने \msPaperCacorr\oo 
\textbf{॰न्द्र}\lem \mssALL, ॰न्द्रः \msNb}}% 
    \var{{\devanagarifont \numnoemph\vb\textbf{देहः}\lem \msCa\msCb\msNa\msNc\Ed, देहे \msCc, देह \msNb\msM\msKOa\msPaperA, 
देहं \msPaperC\oo 
\textbf{नृणाम्}\lem \mssALL, नृणा \msCb\msCc}}% 

%Verse 1:16

{\devanagarifont यं कालपाशमित्याहुः शृणु वक्ष्यामि सुव्रत {॥ १:१६॥} \veg\dontdisplaylinenum }%
     \var{{\devanagarifont \numnoemph\vc \lem \eme, यं कालपाशमित्याह \msCa\msCb\msNa, 
कालपासेति सत्वाह \msCc, यं कालपाशमित्याहु \msNb\msNc\msPaperA\Ed, 
कालपाषेति \uncl{पस्त्वे}ह \msM, 
यां कालपासमित्याहु \msKOa}}% 
    \var{{\devanagarifont \numnoemph\vd\textbf{॰व्रत}\lem \msCa\msNa\msNb\msNc\msM\msPaperA\Ed, ॰व्रतः \msCb\msCc\msKOa}}% 

{\devanagarifont न त्वया विदितं किञ्चिज्जिज्ञास्यसि कथं द्विज \thinspace{\dandab} \dontdisplaylinenum }%
     \var{{\devanagarifont \numemph\va\textbf{विदितं}\lem \mssALL, विदित \msCc}}% 
    \var{{\devanagarifont \numnoemph\vab\textbf{किञ्चिज्जि॰}\lem \msCb\msM, किञ्चिद्वि॰ \msCapcorr\msNa\msNb\msNc\msPaperA\Ed, 
किद्वि॰ \msCaacorr, 
किञ्चि जि॰ \msCc}}% 
    \var{{\devanagarifont \numnoemph\vb\textbf{कथं द्विज}\lem \mssALL, 
\lk\lk\lk\lk\lk\lk\lk\lk\lk  \uncl{म त्वया विदितं किञ्चिद्विज्ञास्यसि} 
\cancelled\ कथं द्विज \msCc}}% 

%Verse 1:17

{\devanagarifont कालपाशं च विप्रेन्द्र सकलं वेत्तुमर्हसि {॥ १:१७॥} \veg\dontdisplaylinenum }%
     \var{{\devanagarifont \numnoemph\vc\textbf{कालपाशं च}\lem \mssALL, कालपाषेति \msM}}% 
    \var{{\devanagarifont \numnoemph\vd\textbf{वेत्तुमर्हसि}\lem \mssCaCbCc\msNa\msNb, 
वेत्तुमूहसि \msNc, वक्तुमर्हसि \msM\msPaperA\Ed}}% 

{\devanagarifont कलाकलितकालं च कालतत्त्वकलां शृणु \thinspace{\dandab} \dontdisplaylinenum }%
     \var{{\devanagarifont \numemph\va\textbf{कला॰}\lem \mssALL, काला॰ \msCc\msNaacorr\oo 
\textbf{॰कलित॰}\lem \mssALL, ॰\uncl{कन्मित}॰ \msPaperA\oo 
\textbf{॰कालं च}\lem \mssALL, ॰कालश्च \msM\Ed}}% 
    \var{{\devanagarifont \numnoemph\vb\textbf{॰कलां}\lem \msCa\msCc\msNb\msPaperA\Ed, ॰कला \msCb\msNc, ॰विधिं \msNa, ॰कलाः \msM}}% 

%Verse 1:18

{\devanagarifont त्रुटिद्वयं निमेषस्तु निमेषद्विगुणा कला {॥ १:१८॥} \veg\dontdisplaylinenum }%
     \var{{\devanagarifont \numnoemph\vc\textbf{त्रुटिद्वयं}\lem \msCa\msCc\msNc\Ed, तुटिद्वय \msCb\msNb, तुटिद्वयं \msNa\msM, 
त्रुविद्वयं \msPaperA\oo 
\textbf{॰मेषस्तु}\lem \mssALL, 
॰मेवस्तु \msCa, ॰मेषद्वि॰ \msNa}}% 
    \var{{\devanagarifont \numnoemph\vd\textbf{निमेषद्वि॰}\lem \mssALL, निमेषाद्वि॰ \msM}}% 

{\devanagarifont कलाद्विगुणिता काष्ठा काष्ठा वै त्रिंशतिः कला \thinspace{\dandab} \dontdisplaylinenum }%
     \var{{\devanagarifont \numemph\va\textbf{॰गुणिता काष्ठा}\lem \mssALL, ॰गुणितं काष्ठा \msM, 
॰गुणितं काष्ठी \msPaperA}}% 
    \var{{\devanagarifont \numnoemph\vb\textbf{काष्ठा वै त्रिंशतिः}\lem \msCa\msNa\msNb\msNc\msPaperA\Ed, वै त्रिंशता \msCb, 
काष्ठा वै त्रिंशति \msCc, काष्ठान्वै त्रिंशति \msM}}% 

%Verse 1:19

{\devanagarifont त्रिंशत्कला मुहूर्तश्च मानुषेन द्विजोत्तम {॥ १:१९॥} \veg\dontdisplaylinenum }%
     \var{{\devanagarifont \numnoemph\vc\textbf{मुहूर्तश्च}\lem \mssALL, 
मुहूर्त्त \msCb, मुहूर्तञ्च \Ed}}% 
    \var{{\devanagarifont \numnoemph\vd\textbf{मानुषेन}\lem \mssALL, मानु\uncl{षश्च} \msCc\oo 
\textbf{॰त्तम}\lem \mssCaCbCc\msNa\msNcpcorr\msPaperA\Ed, ॰तमः \msNb\msM, ॰त्तमः \msNcacorr}}% 

{\devanagarifont मुहूर्तत्रिंशकेनैव अहोरात्रं विदुर्बुधाः \thinspace{\dandab} \dontdisplaylinenum }%
     \var{{\devanagarifont \numemph\va\textbf{मुहूर्त॰}\lem \mssALL, मुहूर्त्ता \msM, मुहूर्तं \Ed}}% 
    \var{{\devanagarifont \numnoemph\vb\textbf{॰धाः}\lem \mssALL, ॰धा \msPaperA}}% 

%Verse 1:20

{\devanagarifont अहोरात्रं पुनस्त्रिंशन्मासमाहुर्मनीषिणः {॥ १:२०॥} \veg\dontdisplaylinenum }%
     \var{{\devanagarifont \numnoemph\vc\textbf{॰रात्रं}\lem \mssALL, ॰रात्र \msM}}% 
    \var{{\devanagarifont \numnoemph\vd\textbf{॰नीषिणः}\lem \mssALL, ॰नीषिन \msM}}% 

{\devanagarifont समा द्वादश मासाश्च कालतत्त्वविदो जनाः \thinspace{\dandab} \dontdisplaylinenum }%
     \var{{\devanagarifont \numemph\va\textbf{समा}\lem \mssALL, मास \msCc, समा समाया \msPaperA\oo 
\textbf{॰मासाश्च}\lem \msCa\msCb\msNa\msNb\msNc\msPaperA, ॰मासश्च \msCc\Ed, मासाहुः \msM}}% 
    \var{{\devanagarifont \numnoemph\vb\textbf{काल॰}\lem \mssALL, कला॰ \msNc}}% 

{\devanagarifont शतं वर्षसहस्राणि त्रीणि मानुषसंख्यया  \danda\dontdisplaylinenum }%
     \var{{\devanagarifont \numnoemph\vc\textbf{शतं}\lem   \mssALL,        शत॰ \msPaperA\Ed}}% 
    \var{{\devanagarifont \numnoemph\vd\textbf{मानुष॰}\lem \mssALL, माणुष्य॰ \msCb\msCc\ \unmetr}}% 

%Verse 1:21

{\devanagarifont षष्टिं चैव सहस्राणि कालः कलियुगः स्मृतः {॥ १:२१॥} \veg\dontdisplaylinenum }%
     \var{{\devanagarifont \numnoemph\ve\textbf{षष्टिं चैव}\lem \mssCaCbCc\msNc\msM, षष्टिं वर्ष॰ \msNa\msPaperA, षष्टिश्चैव \Ed}}% 
    \var{{\devanagarifont \numnoemph\vf\textbf{॰युगः}\lem       \mssALL, ॰युग \msM\Ed}}% 
    \lacuna{\devanagarifontsmall \vo {\englishfont \msNb\ omits verses 21ef--24ab} }%
  
\vfill
\pageparbreak
\vers

{\devanagarifont द्विगुणः कलिसंख्यातो द्वापरो युग संज्ञितः \thinspace{\dandab} \dontdisplaylinenum }%
     \var{{\devanagarifont \numemph\va \lem \mssCaCbCc\msNa\msNc, कलिसंख्यास्तु द्विगुणो \msM, 
द्विगुर्णः कलिसंख्यातो \msPaperA, 
द्विगुणा कलिसंख्यातो \Ed}}% 
    \var{{\devanagarifont \numnoemph\vb \lem \mssALL, 
द्वापरः युगः संज्ञिकम् \msM, 
द्वापरे युग संज्ञितः \Ed}}% 

%Verse 1:22

{\devanagarifont त्रेता तु त्रिगुणा ज्ञेया चतुः कृतयुगः स्मृतः {॥ १:२२॥} \veg\dontdisplaylinenum }%
     \var{{\devanagarifont \numnoemph\vc\textbf{त्रेता}\lem   \msCa\msCb\msNa\msPaperA\Ed,              तेत्रा \msCc\msM, त्रेत्रा \msNc\oo 
\textbf{त्रिगुणा}\lem \mssALL,  तृगुणो \msM\oo 
\textbf{ज्ञेया}\lem   \mssALL,  ज्ञेयः \msM}}% 
    \var{{\devanagarifont \numnoemph\vd\textbf{॰युगः}\lem  \mssALL, ॰युग \Ed}}% 

{\devanagarifont एषा चतुर्युगासंख्या कृत्वा वै ह्येकसप्ततिः \thinspace{\dandab} \dontdisplaylinenum }%
     \var{{\devanagarifont \numemph\vb\textbf{ह्ये॰}\lem   \mssALL,   हे॰ \msNc\oo 
\textbf{॰सप्ततिः}\lem \mssALL, ॰सप्तति \msM}}% 

%Verse 1:23

{\devanagarifont मन्वन्तरस्य चैकस्य ज्ञानमुक्तं समासतः {॥ १:२३॥} \veg\dontdisplaylinenum }%
     \var{{\devanagarifont \numnoemph\vc\textbf{चैकस्य}\lem \mssALL, \om\ \msNaacorr\msMacorr}}% 
    \var{{\devanagarifont \numnoemph\vd\textbf{॰क्तं}\lem    \mssALL,                ॰क्त \msM}}% 

{\devanagarifont कल्पो मन्वन्तराणां तु चतुर्दश तु संख्यया \thinspace{\dandab} \dontdisplaylinenum }%
     \var{{\devanagarifont \numemph\va\textbf{कल्पो}\lem \msCb, कल्प \msCa\msCc\msNa\msNc\msM\msPaperA\Ed\oo 
\textbf{मन्वन्त॰}\lem \mssALL, 
न्वन्त॰ \msMacorr, मंन्वन्त॰ \msMpcorr}}% 
    \var{{\devanagarifont \numnoemph\vb\textbf{॰दश}\lem     \mssALL, ॰दशं \msCb\oo 
\textbf{संख्यया}\lem \mssALL,      शंक्षया \msM}}% 

{\devanagarifont दश कल्पसहस्राणि ब्रह्माहः परिकल्पितम्  \danda\dontdisplaylinenum }%
     \var{{\devanagarifont \numnoemph\vd\textbf{॰आहः}\lem \mssALL, ॰आह \msCa\oo 
\textbf{परिकल्पितम्}\lem \msCa\msNc, करिकल्पितम् \msCb, परिकल्पितः \msCc\msNb\msM\msPaperA\Ed, 
परिकीर्तिताः \msNa}}% 

%Verse 1:24

{\devanagarifont रात्रिरेतावती प्रोक्ता मुनिभिस्तत्त्वदर्शिभिः {॥ १:२४॥} \veg\dontdisplaylinenum }%
     \var{{\devanagarifont \numnoemph\vf\textbf{॰दर्शिभिः}\lem \mssALL, ॰दर्शिभि \msM}}% 

{\devanagarifont रात्र्यागमे प्रलीयन्ते जगत्सर्वं चराचरम् \thinspace{\dandab} \dontdisplaylinenum }%
     \var{{\devanagarifont \numemph\va\textbf{॰गमे}\lem      \mssALL,         ॰गम \msPaperA\oo 
\textbf{प्रलीयन्ते}\lem \mssALL, प्रलीयते \msCb}}% 
    \var{{\devanagarifont \numnoemph\vb\textbf{सर्वं च॰}\lem \mssALL,     सर्वश्च॰ \msM}}% 

%Verse 1:25

{\devanagarifont अहागमे तथैवेह उत्पद्यन्ते चराचरम् {॥ १:२५॥} \veg\dontdisplaylinenum }%
     \var{{\devanagarifont \numnoemph\vc\textbf{अहागमे}\lem \mssCaCbCc\msNa\msNc, अहाग\lac\  \msNb, 
अहरागमे \msM\ \unmetr, अहागम \msPaperA, अह्नागमे \Ed}}% 
    \var{{\devanagarifont \numnoemph\vd\textbf{॰पद्यन्ते}\lem \mssALL, ॰पद्यंति \msM}}% 

{\devanagarifont परार्धपरकल्पानि अतीतानि द्विजोत्तम \thinspace{\dandab} \dontdisplaylinenum }%
     \var{{\devanagarifont \numemph\va\textbf{॰र्ध॰}\lem \mssALL, ॰र्धं \msNb, ॰ध॰ \msPaperA}}% 

%Verse 1:26

{\devanagarifont अनागतं तथैवाहुर्भृगुरादिमहर्षयः {॥ १:२६॥} \veg\dontdisplaylinenum }%
     \var{{\devanagarifont \numnoemph\vcd\textbf{॰वाहुर्भृ॰}\lem \msCa\msCb\msNa\msNc\msPaperA\Ed, 
॰वाहु भृ॰ \msCc\msNb\msM}}% 
    \var{{\devanagarifont \numnoemph\vd\textbf{॰महर्षयः}\lem    \mssCaCbCc\msNapcorr\msNb\msPaperA\Ed, 
॰महयः \msNaacorr, ॰मर्हषयः \msNc, 
॰महर्षिभिः \msM}}% 

\vfill
\pageparbreak
\vers

{\devanagarifont यथार्कग्रहतारेन्दु भ्रमतो दृश्यते त्विह \thinspace{\dandab} \dontdisplaylinenum }%
     \var{{\devanagarifont \numemph\va\textbf{॰आर्क॰}\lem   \mssALL, ॰आर्का॰ \msMacorr\oo 
\textbf{॰तारेन्दु}\lem \mssALL,          ॰तारैन्दु \msM}}% 
    \var{{\devanagarifont \numnoemph\vb\textbf{भ्रमतो}\lem \mssALL,                भुमनो \msPaperA\oo 
\textbf{दृश्यते त्विह}\lem \msCa\msNa\msNb\msNc\msPaperA\Ed, 
दृश्यन्दिह \msCb, दृस्यते त्विहः \msCc, 
दृश्यते त्विहः \msM}}% 

%Verse 1:27

{\devanagarifont कालचक्रं भ्रमित्वैव विश्रमं न च विद्महे {॥ १:२७॥} \veg\dontdisplaylinenum }%
     \var{{\devanagarifont \numnoemph\vc\textbf{भ्रमित्वैव}\lem \corr, भ्रमत्वैव \msCa\msNa\msNc\Ed, 
भ्रमत्वेव  \msCb\msNb\msM, भ्रमत्वेह \msCc, 
भ्रमत्यैव \msPaperA}}% 
    \var{{\devanagarifont \numnoemph\vd\textbf{॰श्रमं}\lem \mssCaCbCc\msNapcorr\msNc\msPaperA\Ed, 
॰श्रमो \msNaacorr, ॰श्रामन् \msNb, ॰श्रामो \msM\oo 
\textbf{विद्महे}\lem \mssALL, विग्रहे \msCb, विद्यते \msM}}% 

{\devanagarifont कालः सृजति भूतानि कालः संहरते पुनः \thinspace{\dandab} \dontdisplaylinenum }%
     \var{{\devanagarifont \numemph\vb\textbf{कालः}\lem \mssALL, काल \Ed}}% 
    \paral{{\devanagarifontsmall \vab {\englishfont \similar\ \UMS\ 12.34cd:}
                         कालः पचति भूतानि कालः संहरते प्रजाः }}

%Verse 1:28

{\devanagarifont कालस्य वशगाः सर्वे न कालवशकृत्क्वचित् {॥ १:२८॥} \veg\dontdisplaylinenum }%
     \var{{\devanagarifont \numnoemph\vc\textbf{कालस्य}\lem     \mssALL, कालःस्य \msMacorr\oo 
\textbf{वशगाः}\lem     \mssALL,         वशगा \Ed}}% 
    \var{{\devanagarifont \numnoemph\vd\textbf{कालवशकृ॰}\lem \mssALL,          कालो वशकृ॰ \msM}}% 
    \paral{{\devanagarifontsmall \vo \similar\ {\englishfont \KURMP\ 1.11.32:}
                 कालः सृजति भूतानि कालः संहरते प्रजाः\thinspace{\devanagarifontsmall ।}
                 सर्वे कालस्य वशगा न कालः कस्यचिद्वशे\thinspace{\devanagarifontsmall ॥} }}

{\devanagarifont चतुर्दश परार्धानि देवराजा द्विजोत्तम \thinspace{\dandab} \dontdisplaylinenum }%
     \var{{\devanagarifont \numemph\vb\textbf{देवराजा}\lem \mssALL,     देवराज \msM\Ed\oo 
\textbf{॰त्तम}\lem   \mssALL, ॰त्तमः \msM}}% 

%Verse 1:29

{\devanagarifont कालेन समतीतानि कालो हि दुरतिक्रमः {॥ १:२९॥} \veg\dontdisplaylinenum }%
     \paral{{\devanagarifontsmall \vd {\englishfont = \MBH\ 12.220.41d = \GARPUR\ 1.108.7d} }}

{\devanagarifont एष कालो महायोगी ब्रह्मा विष्णुः परः शिवः \thinspace{\dandab} \dontdisplaylinenum }%
     \var{{\devanagarifont \numemph\va\textbf{कालो}\lem \msCa\msCb\msNa,      काल \msCc\msNb\msNc\msM\msPaperA\Ed}}% 
    \var{{\devanagarifont \numnoemph\vb\textbf{ब्रह्मा विष्णुः परः}\lem \msCb, ब्रह्मविष्णुपरः \msCa\msNc\msM\msPaperA, 
ब्रह्मा विष्णु परः \msCc\msNa\msNb, 
ब्रह्मविष्णुपर \Ed\ \unmetr}}% 

%Verse 1:30

{\devanagarifont अनादिनिधनो धाता स महात्मा नमस्कुरु {॥ १:३०॥} \veg\dontdisplaylinenum }%
 

\alalfejezet{परार्धादि}
{\devanagarifont विगतराग उवाच {\dandab}\dontdisplaylinenum  }%
 
{\devanagarifont श्रुतं वै कालचक्रं तु मुखपद्मविनिःसृतम् \thinspace{\danda} \dontdisplaylinenum }%
     \var{{\devanagarifont \numemph\va\textbf{श्रुतं वै}\lem \mssALL, श्रुतो वः \msM\oo 
\textbf{॰चक्रं तु}\lem \mssALL, ॰चक्रस्य \msCc, ॰चक्रत्तु \msM}}% 
    \var{{\devanagarifont \numnoemph\vb\textbf{विनिःसृतम्}\lem \corr, विनिसृतम् \mssCaCbCc\msNa\msNb\msNc\msM\msPaperA\Ed\ \unmetr}}% 

%Verse 1:31

{\devanagarifont परार्धं च परं चैव श्रोतुं वः प्रतिदीपितम् {॥ १:३१॥} \veg\dontdisplaylinenum }%
     \var{{\devanagarifont \numnoemph\vc\textbf{परार्धं च}\lem \msCb\msCc\msNa\msNb\msNc\msPaperA\Ed, 
\uncl{प}रार्द्धं च \msCa, 
पराधञ्च \msMacorr, 
परार्धंञ्चे \msMpcorr\oo 
\textbf{परं चैव}\lem \mssALL,                पराञ्चैव \msM\msPaperA}}% 
    \var{{\devanagarifont \numnoemph\vd\textbf{वः}\lem         \mssALL, नः \msMpcorr, यः \Ed\oo 
\textbf{॰दीपितम्}\lem    \mssALL,      ॰दीयतां \msM}}% 

\vfill
\pageparbreak
\vers

{\devanagarifont अनर्थयज्ञ उवाच {\dandab}\dontdisplaylinenum  }%
     \var{{\devanagarifont \numemph\vo\textbf{अनर्थयज्ञ उवाच}\lem \mssALL, \om\ \msNaacorr}}% 

{\devanagarifont एकं दशं शतं चैव सहस्रमयुतं तथा \thinspace{\danda} \dontdisplaylinenum }%
     \var{{\devanagarifont \numnoemph\vb\textbf{सहस्र॰}\lem \mssALL, साहस्र॰ \msM\oo 
\textbf{॰युतं}\lem   \mssALL,  ॰तन् \msNb}}% 

%Verse 1:32

{\devanagarifont प्रयुतं नियुतं कोटिमर्बुदं वृन्दमेव च {॥ १:३२॥} \veg\dontdisplaylinenum }%
     \var{{\devanagarifont \numnoemph\vc\textbf{प्र॰}\lem        \mssALL,      प॰ \msPaperA}}% 
    \var{{\devanagarifont \numnoemph\vcd\textbf{कोटिम॰}\lem \mssALL,  कोटिर॰ \msNc}}% 
    \var{{\devanagarifont \numnoemph\vd\textbf{॰र्बुदं}\lem     \mssALL, ॰बुदं \msNc}}% 

{\devanagarifont खर्वं चैव निखर्वं च शङ्कु पद्मं तथैव च \thinspace{\dandab} \dontdisplaylinenum }%
     \var{{\devanagarifont \numemph\va\textbf{निखर्वं च}\lem \mssALL,       निखर्वं तु \msNb, निसर्वञ्च \msM}}% 
    \var{{\devanagarifont \numnoemph\vb\textbf{शङ्कु}\lem        \mssALL, शंख \Ed\oo 
\textbf{पद्मं}\lem       \mssALL,  पद्म \msM}}% 
    \lacuna{\devanagarifontsmall \vab {\englishfont After these two pādas, \msPaperA\ inserts this:}
                                वृन्दञ्चैव महावृन्द द्विपरो नन्तनेव च }%
      \paral{{\devanagarifontsmall \vab {\englishfont  = \BRAHMANDAPUR\ 3.2.101 }  }}

%Verse 1:33

{\devanagarifont समुद्रो मध्यमन्तं च परार्धं च परं तथा {॥ १:३३॥} \veg\dontdisplaylinenum }%
     \var{{\devanagarifont \numnoemph\vc\textbf{समुद्रो}\lem            \mssALL, समुद्र॰ \msM\oo 
\textbf{मध्यमन्तं च}\lem \mssCaCbCc\msNaacorr\msM\msPaperA,         मध्यमान्तं च \msNapcorr, 
मध्य\uncl{मन्तञ्च} \msNb, 
मध्यमन्तश्च \msNc}}% 
    \var{{\devanagarifont \numnoemph\vd \lem \mssALL,  परार्द्धपरद्वेगुणाम् \msM}}% 
    \lacuna{\devanagarifontsmall \vcd {\englishfont \Ed\ omits 34cd--35 and then inserts this:}
                                वृन्दञ्चैव महावृन्द द्विपरानन्तमेव च }%
  
{\devanagarifont सर्वे दशगुणा ज्ञेयाः परार्धं यावदेव हि \thinspace{\dandab} \dontdisplaylinenum }%
     \var{{\devanagarifont \numemph\va\textbf{सर्वे}\lem     \mssALL, सर्वं \msPaperA}}% 
    \var{{\devanagarifont \numnoemph\vb\textbf{परार्धं}\lem \msNc,                                परा\uncl{र्ध} \msCa, 
परार्ध \msCb\msCc\msNa\msNb\msM\msPaperA\oo 
\textbf{यावदेव}\lem \mssALL, दशदव \msPaperA}}% 

%Verse 1:34

{\devanagarifont परार्धद्विगुणेनैव परसंख्या विधीयते {॥ १:३४॥} \veg\dontdisplaylinenum }%
     \var{{\devanagarifont \numnoemph\vc\textbf{परार्ध॰}\lem \mssALL,   परार्धं \msNc}}% 
    \var{{\devanagarifont \numnoemph\vd\textbf{॰संख्या}\lem  \mssALL, ॰सख्या \msM}}% 

{\devanagarifont परात्परतरं नास्ति इति मे निश्चिता मतिः \thinspace{\dandab} \dontdisplaylinenum }%
     \var{{\devanagarifont \numemph\vab \lem \mssCaCbCc\msNb\msNcpcorr\msPaperA\Ed, 
परात्परतरं नास्ति इति मे निश्चिता मति \msNa\msNcacorr, 
परापरतरन्नास्ति इति मे निश्चिता मति \msM}}% 

%Verse 1:35

{\devanagarifont पुराणवेदपठिता मयाख्याता द्विजोत्तम {॥ १:३५॥} \veg\dontdisplaylinenum }%
     \var{{\devanagarifont \numnoemph\vc\textbf{॰वेद॰}\lem \msCa\Ed, ॰वेदे \msCb\msCc\msNb\msNc\msPaperA, 
॰वेदा \msNa, ॰वेदैः \msM}}% 
    \var{{\devanagarifont \numnoemph\vd\textbf{॰ख्याता}\lem \msCa\msCb\msNa, ॰ख्यातं \msCc\msNb\msNc\msM\msPaperA\Ed\oo 
\textbf{॰त्तम}\lem \mssALL, ॰तम \msM}}% 

\vfill
\pageparbreak
\vers


\alalfejezet{ब्रह्माण्डम्}
{\devanagarifont विगतराग उवाच {\dandab}\dontdisplaylinenum  }%
 
{\devanagarifont ब्रह्माण्डं कति विज्ञेयं प्रमाणं ज्ञापितं क्वचित् \thinspace{\danda} \dontdisplaylinenum }%
     \var{{\devanagarifont \numemph\va\textbf{ब्रह्माण्डं}\lem \mssALL, ब्रह्माण्ड \msCc}}% 
    \var{{\devanagarifont \numnoemph\vb \lem \conj, प्रमाणं चापितं क्वचित् \mssCaCbCc\msNa\msNb\msPaperA\Ed, 
प्रमाञ्चापितत् क्वचित् \msNc, प्रमाणञ्चापितां कति \msM}}% 

%Verse 1:36

{\devanagarifont कति चाङ्गुलिमूर्ध्वेषु सूर्यस्तपति वै महीम् {॥ १:३६॥} \veg\dontdisplaylinenum }%
     \var{{\devanagarifont \numnoemph\vc\textbf{॰र्ध्वेषु}\lem \eme, ॰र्धेषु \mssCaCbCc\msNa\msNb\msNc\msM\msPaperA\Ed}}% 
    \var{{\devanagarifont \numnoemph\vd\textbf{सूर्यस्त॰}\lem \mssALL, र्यो \msMacorr, शूर्यो \msMpcorr\oo 
\textbf{महीम्}\lem \msCb\msCc\msNa\msM\msPaperA, मही\uncl{म् } \msCa, मही \msNb\msNc\Ed}}% 

{\devanagarifont अनर्थयज्ञ उवाच {\dandab}\dontdisplaylinenum  }%
 
{\devanagarifont ब्रह्माण्डानां प्रसंख्यातुं मया शक्यं कथं द्विज \thinspace{\danda} \dontdisplaylinenum }%
     \var{{\devanagarifont \numemph\va\textbf{ब्रह्मा॰}\lem \mssALL, ब्रह्म॰ \msM\oo 
\textbf{प्रसंख्यातुं}\lem \mssALL, प्रसंसा तु \msNb, च संख्यातुं \Ed}}% 
    \var{{\devanagarifont \numnoemph\vb\textbf{शक्यं क॰}\lem \msNa\msNb\msPaperApcorr\Ed, शक्या क॰ \mssCaCbCc\msNc, सक्याङ्क॰ \msM, 
ह्यक्यं क॰ \msPaperAacorr}}% 

%Verse 1:37

{\devanagarifont देवास्ते ऽपि न जानन्ति मानुषाणां च का कथा {॥ १:३७॥} \veg\dontdisplaylinenum }%
     \var{{\devanagarifont \numnoemph\vc\textbf{देवास्ते}\lem   \mssALL, देवतापि \msM}}% 
    \var{{\devanagarifont \numnoemph\vd\textbf{मानुषाणां च}\lem \mssALL, मानुषार्नञ्च \msMacorr, 
मानुषानाञ्च \msMpcorr}}% 

{\devanagarifont पर्यायेण तु वक्ष्यामि यथाशक्यं द्विजोत्तम \thinspace{\dandab} \dontdisplaylinenum }%
 
%Verse 1:38

{\devanagarifont ब्रह्मणा यत्पुराख्यातो मातरिश्वा यथा तथा {॥ १:३८॥} \veg\dontdisplaylinenum }%
     \var{{\devanagarifont \numemph\vc\textbf{यत्पुराख्यातो}\lem \mssCaCbCc\msNa\msNb\msNc, यत्पुराख्यातं \msM, 
यत्प्रयात्परायाख्यातो \msPaperA, 
यत्ममाख्यातो \Ed}}% 
    \paral{{\devanagarifontsmall \vcd {\englishfont cf. \BRAHMANDAPUR\ 3.4.58cd:} 
                         ब्रह्मा ददौ शास्त्रमिदं पुराणं मातरिश्वने }}

{\devanagarifont शिवाण्डाभ्यन्तरेणैव सर्वेषामिव भूभृताम् \thinspace{\dandab} \dontdisplaylinenum }%
     \var{{\devanagarifont \numemph\va\textbf{शिवाण्डा॰}\lem \mssALL, शिवाण्ड॰ \msMacorr, शिवाण्डे॰ \msMpcorr}}% 
    \var{{\devanagarifont \numnoemph\vb \lem \conj, सर्वेषामिव भूरिताः \msCa\msCb\msNc, 
सर्वेषामेव भूरिताः \msCc, 
सर्वेषामिव भूरिता \msNa, सर्वेषामेव भूरिणाम् \msNb, 
स\uncl{र्षपा} इव भाविता \msM, 
सर्वेषामेव भूरिनाः \msPaperA, 
सर्वेषामेव भूरिमां \Ed}}% 

%Verse 1:39

{\devanagarifont दश नाम दिशाष्टानां ब्रह्माण्डे कीर्तितं शृणु {॥ १:३९॥} \veg\dontdisplaylinenum }%
     \var{{\devanagarifont \numnoemph\vc\textbf{दिशा॰}\lem         \mssALL,  शिवा॰ \msNb}}% 
    \var{{\devanagarifont \numnoemph\vd\textbf{ब्रह्माण्डे}\lem     \mssALL, ब्रह्मण्डा \msM\oo 
\textbf{कीर्तितं शृणु}\lem \mssALL, य च कीर्तितम् \msCb, 
कीर्त्तिता शृणु \msM}}% 

\vfill
\pageparbreak
\vers


\alalfejezet{भूभृतां नामानि}

\alalalfejezet{पूर्वतः}

{\devanagarifont सहासहः सहः सह्यो विसहः संहतो ऽसभा \thinspace{\dandab} \dontdisplaylinenum }%
     \var{{\devanagarifont \numemph\va\textbf{सहासहः}\lem   \msNc,                     साहासह \mssCaCbCc\msNa\msNb\msM\msPaperA\Ed\oo 
\textbf{सहः सह्यो}\lem \msCa\msCc\msNa\msNb\msNc, सहः सज्ञा \msCb, 
सहो सह्यः \msM, 
सहः सज्ञो \msPaperA\Ed}}% 
    \var{{\devanagarifont \numnoemph\vb\textbf{विसहः}\lem     \msCa\msCb\msNa\msNb\msNc\Ed, विसह \msCc\msM, विंसहः \msPaperA\oo 
\textbf{ऽसभा}\lem      \msCa\msCc\msNa\msNb\msNc,    सभाः \msCb, सहा \msM, सता \msPaperA\Ed}}% 

%Verse 1:40

{\devanagarifont प्रसहो ऽप्रसहः सानुः पूर्वतो दश नायकाः {॥ १:४०॥} \veg\dontdisplaylinenum }%
     \var{{\devanagarifont \numnoemph\vc\textbf{प्रसहो}\lem  \mssALL,  प्रसहेः \Ed\oo 
\textbf{प्रसहः}\lem \mssALL,  प्रस\uncl{वः} \msCc, सप्रहः \Ed\oo 
\textbf{सानुः}\lem    \mssCaCbCc\msNa\msNb\msPaperA,                  सानु \msNc\msM\Ed}}% 
    \var{{\devanagarifont \numnoemph\vd\textbf{पूर्वतो}\lem  \mssALL,  पर्वतो \Ed}}% 


\alalalfejezet{आग्नेये}

{\devanagarifont प्रभासो भासनो भानुः प्रद्योतो द्युतिमो द्युतिः \thinspace{\dandab} \dontdisplaylinenum }%
     \var{{\devanagarifont \numemph\va\textbf{भासनो}\lem \msCa\msCb\msNa\msNb\msNc\msM,                भास\lac\  \msCc, 
भांसतो \msPaperA, 
भासतो \Ed\oo 
\textbf{भानुः}\lem  \mssALL, भानु \msCb\msM}}% 
    \var{{\devanagarifont \numnoemph\vb\textbf{द्युतिमो}\lem \mssCaCbCc\msNa\msNb\msM,                     द्युतिनो \msNc\msPaperA\Ed}}% 

{\devanagarifont दीप्ततेजाश्च तेजाश्च तेजा तेजवहो दश  \danda\dontdisplaylinenum }%
     \var{{\devanagarifont \numnoemph\vc \lem \msCa\msCc\msNa\msNb\msNc\msPaperA, दीप्ततेजाश्च तेजश्च \msCb, 
दीप्ततेजस् तेजश्च \msM\ \unmetr, 
दीप्ततेजश्च तेजाश्च \Ed}}% 
    \var{{\devanagarifont \numnoemph\vd\textbf{तेजा तेजवहो}\lem \mssALL,      तेजतेजयह \msM}}% 

%Verse 1:41

{\devanagarifont आग्नेये त्वेतदाख्यातं याम्ये शृण्वथ भो द्विज {॥ १:४१॥} \veg\dontdisplaylinenum }%
     \var{{\devanagarifont \numnoemph\ve\textbf{आग्नेये}\lem         \mssCaCbCc\msNa\msNb\Ed,                      आग्नेय \msNc\msPaperA, 
आग्नेर्ये \msM\oo 
\textbf{त्वेतदा॰}\lem \mssALL, त्वेचमा \msM}}% 
    \var{{\devanagarifont \numnoemph\vf\textbf{शृण्वथ}\lem    \mssALL, शृणुथ \msM\oo 
\textbf{द्विज}\lem          \mssALL,  द्विजः \msNb}}% 


\alalalfejezet{याम्ये}

{\devanagarifont यमो ऽथ यमुना यामः संयमो यमुनो ऽयमः \thinspace{\dandab} \dontdisplaylinenum }%
     \var{{\devanagarifont \numemph\va\textbf{यमो}\lem    \mssALL, यमा \msPaperA}}% 
    \var{{\devanagarifont \numnoemph\vb\textbf{संयमो}\lem \mssALL,     संयम \msM, 
संयमा \msPaperA\oo 
\textbf{यमुनो}\lem  \msCa\msCb\msNb\msPaperA,                यमनो \msCc\msNc, 
युमुना \msNa, 
यमतो \msM, यमुना॰ \Ed\oo 
\textbf{यमः}\lem   \mssALL,     यन \msM, 
यामः \msPaperA\ \unmetr}}% 

%Verse 1:42

{\devanagarifont संयनो यमनोयानो यनियुग्मा यनोयनः {॥ १:४२॥} \veg\dontdisplaylinenum }%
     \var{{\devanagarifont \numnoemph\vc \lem \msNa, संयमो यमनोयानो \msCa\msCc\Ed, 
संयमो यमुनोयानो \msCb\msNb, 
संयमा यमनो यामो \msNc, 
यमियुग्मा यनो यानः \msM, 
संयमा यमनो यानो \msPaperA}}% 
    \var{{\devanagarifont \numnoemph\vd \lem \msNb, यनियुग्मा नयो यनः \msCa\msCc\msNa, 
यनियुग्मा नयो नयः \msCb\msPaperA, 
यनियुग्मा नयो यमः \msNc, 
दशमा याम्यमाशृता \msM, 
यनियुग्मा नयोनय \Ed}}% 

\vfill
\pageparbreak
\vers


\alalalfejezet{नै\char"0930\char"094D\char"090Bते}

{\devanagarifont नगजो नगना नन्दो नगरो नग नन्दनः \thinspace{\dandab} \dontdisplaylinenum }%
     \var{{\devanagarifont \numemph\va\textbf{नगना नन्दो}\lem        \msCa\msCc\msNa\msNb\msNc, 
नगजा नन्दो \msCb, 
नगनागेन्द्र \msM, 
नगनो नदो \msPaperA\Ed}}% 
    \var{{\devanagarifont \numnoemph\vb \lem \msNb\msMacorr\msPaperA, नगरोरगनन्दनः \msCa\msNc, 
नगरो\uncl{नगनन्द}नः \msCb, 
नग\uncl{रो}\lac  नन्दनः \msCc, 
नगरोगरनन्दनः \msNa, 
नगरो नननन्दनः \msMpcorr, 
नगरोन्नगनन्दनः \Ed}}% 

%Verse 1:43

{\devanagarifont नगर्भो गहनो गुह्यो गूढजो दश तत्परः {॥ १:४३॥} \veg\dontdisplaylinenum }%
     \var{{\devanagarifont \numnoemph\vc\textbf{नगर्भो}\lem     \mssALL,      नृगभो \msNb, नगर्भ \msM\oo 
\textbf{गहनो गुह्यो}\lem \mssALL,    गुहनो गुह्य \msM, गहनो गुह्ये \Ed}}% 
    \var{{\devanagarifont \numnoemph\vd\textbf{गूढजो}\lem      \mssALL, गुडजो \msM\oo 
\textbf{तत्परः}\lem     \mssALL, तत्परम् \msM}}% 


\alalalfejezet{वारुणे}

{\devanagarifont वारुणेन प्रवक्ष्यामि शृणु विप्र निबोध मे \thinspace{\dandab} \dontdisplaylinenum }%
     \var{{\devanagarifont \numemph\va\textbf{वारुणेन}\lem \mssALL, वारुणे च \Ed}}% 
    \var{{\devanagarifont \numnoemph\vb\textbf{शृणु}\lem      \msNb\msM,                                    शृङ्गे \msCa\msCb\msNa\msNc, 
शृ\uncl{ङ्गे} \msCc, मृद्धे \uncl{पाप्त} \cancelled\ \msPaperA, मृद्धे \Ed}}% 

{\devanagarifont बभ्रः सेतुर्भवोद्भद्रः प्रभवोद्भवभाजनः  \danda\dontdisplaylinenum }%
     \var{{\devanagarifont \numnoemph\vc\textbf{बभ्रः सेतुर्भ॰}\lem \corr, बभ्रं सेतुर्भ॰ \msCa\msCb, 
बभ्रं सेतु भ॰ \msCc, बभ्रः सेतु भ॰ \msNa, 
बभ्रं सोतुर्भ॰ \msNb, बभ्र सेतुर्भ॰ \msNc, 
बभ्रू सेतु भ॰ \msM, बभ्रून्सेतुर्भ॰ \msPaperA, 
बभ्रून्सतुर्भ॰ \Ed}}% 
    \var{{\devanagarifont \numnoemph\vd\textbf{प्रभवोद्भव॰}\lem \mssALL,   प्रभवोभव॰ \msM\oo 
\textbf{॰भाजनः}\lem       \mssALL, ॰भाजन \Ed}}% 

%Verse 1:44

{\devanagarifont भरणो भुवनो भर्ता दशैते वरुणालयाः {॥ १:४४॥} \veg\dontdisplaylinenum }%
     \var{{\devanagarifont \numnoemph\ve\textbf{भरणो}\lem \msCb\msNc, भरण \msCa\msNa, भरणां \msCc\msPaperA\Ed, 
भरणा \msNb, भरणः \msM}}% 
    \var{{\devanagarifont \numnoemph\vf\textbf{दशैते}\lem \mssALL,    दशेते \msNc, दशैता \msM\oo 
\textbf{॰लयाः}\lem  \mssALL, ॰लया \msM\Ed}}% 


\alalalfejezet{वायव्ये}

{\devanagarifont नृगर्भो ऽसुरगर्भश्च देवगर्भो महीधरः \thinspace{\dandab} \dontdisplaylinenum }%
     \var{{\devanagarifont \numemph\va\textbf{नृगर्भो}\lem      \mssALL, नृगभा \msM\oo 
\textbf{॰गर्भश्च}\lem \msCa\msCb\msNb\msNc\msPaperA,               ॰गर्भाश्च \msCc\msNa\msM\Ed}}% 
    \var{{\devanagarifont \numnoemph\vb\textbf{देवगर्भो}\lem    \mssALL, देवगर्भ \msM}}% 

%Verse 1:45

{\devanagarifont वृषभो वृषगर्भश्च वृषाङ्को वृषभध्वजः {॥ १:४५॥} \veg\dontdisplaylinenum }%
     \var{{\devanagarifont \numnoemph\vc\textbf{॰गर्भश्च}\lem \mssCaCbCc\msNb\msNc\Ed, ॰गर्भाश्च \msNa, ॰गर्भोश्च \msM, 
॰शभश्च \msPaperA}}% 
    \var{{\devanagarifont \numnoemph\vd\textbf{वृषाङ्को}\lem  \mssALL,     वृषांगो \msM\oo 
\textbf{वृषभ॰}\lem \mssALL, वृष\lk ॰ \msCc}}% 

{\devanagarifont ज्ञातव्यश्च तथा सम्यग् वृषजो वृषनन्दनः \thinspace{\dandab} \dontdisplaylinenum }%
     \var{{\devanagarifont \numemph\va \lem \mssCaCbCc\msNa\msNb\msNc, 
वृषञ्जवृषनन्दश्च \msM, 
ज्ञानवाञ्च तथा सम्य \msPaperA, 
ज्ञानवाञ्च तथा सत्य॰ \Ed}}% 
    \var{{\devanagarifont \numnoemph\vb \lem \mssALL, वृषनन्दनः \msNa, 
दशनायक वायवे \msM}}% 

%Verse 1:46

{\devanagarifont नायका दश वायव्ये कीर्तिता ये मया द्विज {॥ १:४६॥} \veg\dontdisplaylinenum }%
     \var{{\devanagarifont \numnoemph\vcd \lem \msCa\msCb\msNa\msPaperA\Ed, 
नायका दश वायव्ये कीर्तिता ये मया द्विजः \msCc\msNb, 
नायका दश वायव्ये कीर्तिता य मया द्विज \msNc, 
कीर्तितो यं मया द्विप्र यथा तथ्येन सुव्रतः \msM}}% 


\alalalfejezet{उत्तरे}

{\devanagarifont सुलभः सुमनः सौम्यः सुप्रजः सुतनुः शिवः \thinspace{\dandab} \dontdisplaylinenum }%
     \var{{\devanagarifont \numemph\va\textbf{सुलभः}\lem \mssALL, सुरभः \msPaperA\Ed\oo 
\textbf{सुमनः}\lem  \mssCaCbCc\msNa\msNb\Ed, सुमनाः \msNc, सुमनो \msM, सुमन \msPaperA\oo 
\textbf{सौम्यः}\lem  \mssALL, सोम्य \msM}}% 

%Verse 1:47

{\devanagarifont सतः सत्य लयः शम्भुर्दश नायकमुत्तरे {॥ १:४७॥} \veg\dontdisplaylinenum }%
     \var{{\devanagarifont \numnoemph\vc\textbf{सतः सत्य}\lem \corr, सत सत्य \mssCaCbCc\msNc\msPaperA, सत्यसत्य \msNa, सुत सत्य \msNb, 
सुतः सत्य \msM, सत सत्या॰ \Ed\oo 
\textbf{लयः}\lem \mssALL, लयं \msNc}}% 
    \var{{\devanagarifont \numnoemph\vcd\textbf{शम्भुर्द॰}\lem \msCa\msCb\msNb\msPaperA\Ed, शम्भु द॰ \msCc\msNa\msNc, 
शम्\uncl{भुं} द॰ \msM}}% 
    \var{{\devanagarifont \numnoemph\vd\textbf{॰नायकमु॰}\lem \mssALL, ॰नायक उ॰ \Ed}}% 


\alalalfejezet{ईशाने}

{\devanagarifont इन्दु बिन्दु भुवो वज्र वरदो वर वर्षणः \thinspace{\dandab} \dontdisplaylinenum }%
     \var{{\devanagarifont \numemph\va\textbf{वज्र}\lem \mssALL, व्रजः \msM}}% 
    \var{{\devanagarifont \numnoemph\vb\textbf{॰वर्षणः}\lem \mssCaCbCc\msNa\msNb\msM, ॰\lk \uncl{र्शणम्} \msNc, 
॰दर्प्पणः \msPaperA, 
॰दर्य्य च \Ed}}% 

%Verse 1:48

{\devanagarifont इलनो वलिनो ब्रह्मा दशेशानेषु नायकाः {॥ १:४८॥} \veg\dontdisplaylinenum }%
     \var{{\devanagarifont \numnoemph\vc \lem \mssALL, इलिनो वलिनो ब्रह्मः \msM}}% 
    \var{{\devanagarifont \numnoemph\vd\textbf{दशे॰}\lem               \msCa\msNa\msNc\msPaperA\Ed,                  दशै॰ \msCb\msCc\msNb, दिशै॰ \msM\oo 
\textbf{नायकाः}\lem             \mssALL, नायका \msM}}% 


\alalalfejezet{मध्यमे}

{\devanagarifont अपरो विमलो मोहो निर्मलो मन मोहनः \thinspace{\dandab} \dontdisplaylinenum }%
     \var{{\devanagarifont \numemph\va \lem \mssALL, अपरः विमला मोहा \msM}}% 
    \var{{\devanagarifont \numnoemph\vb\textbf{निर्मलो म॰}\lem \eme, निमलो म॰ \msCa, निर्मलोन्म॰ \msCb\msNc\msPaperA, 
निर्मलोत्म॰ \msCc\Ed, निमलोर्म॰ \msNa\msNb, निर्मलोन्म॰ \msM}}% 

%Verse 1:49

{\devanagarifont अक्षयश्चाव्ययो विष्णुर्वरदो मध्यमे दश {॥ १:४९॥} \veg\dontdisplaylinenum }%
     \var{{\devanagarifont \numnoemph\vc\textbf{अक्षयश्चाव्ययो}\lem \msCa\msCb\msNa\msNb\msNc\msPaperA, अक्षयाश्चाव्ययो \msCc, 
अक्षयश्चाव्ययं \msM, अक्षयञ्चाव्ययो \Ed}}% 
    \var{{\devanagarifont \numnoemph\vcd\textbf{विष्णुर्व॰}\lem \msCa\msCb\msNc\msPaperA\Ed, विष्णु व॰ \msCc\msNa\msM, र्विष्णुर्व \msNb}}% 
    \var{{\devanagarifont \numnoemph\vd\textbf{मध्यमे दश}\lem \msCa\msCb\msNc\msPaperA, मध्यमो दश \msCc\msNa, 
वरवर्षणः \msNb, मध्यमो दशः \msM, मध्यमे दशः \Ed}}% 


\alalalfejezet{परिवाराः}

{\devanagarifont सर्वेषां दशमीशानां परिवारशतं शतम् \thinspace{\dandab} \dontdisplaylinenum }%
     \var{{\devanagarifont \numemph\va\textbf{सर्वेषां}\lem       \mssALL,   सर्वेषा \msNc\oo 
\textbf{दशमीशानां}\lem \mssALL, दशरीशानां \Ed}}% 
    \var{{\devanagarifont \numnoemph\vb\textbf{परिवार॰}\lem      \mssALL,   परि॰ \msCb, परिवारं \msNa}}% 

%Verse 1:50

{\devanagarifont शतानां पृथगेकैकं सहस्रैः परिवारितम् {॥ १:५०॥} \veg\dontdisplaylinenum }%
     \var{{\devanagarifont \numnoemph\vd\textbf{सहस्रैः}\lem \mssALL,  सहस्रै \msM\oo 
\textbf{॰वारितम्}\lem  \msCa\msCb\msCcpcorr\msNa\msNb\msNc\msPaperA, ॰वारिता \msCcacorr, 
॰वारितः \msM, ॰वारिताः \Ed}}% 

\vfill
\pageparbreak
\vers

{\devanagarifont सहस्रेषु च एकैकमयुतैः परिवारितम् \thinspace{\dandab} \dontdisplaylinenum }%
     \var{{\devanagarifont \numemph\vab\textbf{एकैकम॰}\lem \msCa\msCb\msNb\msNc\msPaperA\Ed,          एकैकं म॰ \msCc\msNa\msM}}% 
    \var{{\devanagarifont \numnoemph\vb\textbf{परिवारितम्}\lem  \mssALL, परिवारितः \msM, परिवारितमाः \Ed}}% 

%Verse 1:51

{\devanagarifont अयुतं प्रयुतैर्वृन्दैः प्रयुतं नियुतैर्वृतम् {॥ १:५१॥} \veg\dontdisplaylinenum }%
     \var{{\devanagarifont \numnoemph\vc\textbf{अयुतं}\lem \Ed, अयुतैः \mssCaCbCc\msNa\msNc\msM\msPaperA, अयुतै \msNb\oo 
\textbf{प्रयुतैर्वृन्दैः}\lem \mssALL, प्रयुतै वृन्दैः \msNc, 
प्रयुतैर्भृत्य \msM}}% 
    \var{{\devanagarifont \numnoemph\vd \lem \corr, प्रयुतैर्नियुतैर्वृतः \msCa\msCb\msNa\msNc, 
प्रयुतेर्नियुतैर्वृतः \msCc, प्रयुतै नियुतै वृतः \msNb, 
प्रयुतः नियुतैः वृतः \msM, प्रयुते नियुतैर्वृतः \msPaperA, प्रयुतं नियुतैर्वृतः \Ed}}% 

{\devanagarifont एकैकस्य परीवारो नियुतः पृथगेव च \thinspace{\dandab} \dontdisplaylinenum }%
     \var{{\devanagarifont \numemph\va\textbf{परीवारो}\lem \mssALL,             परिवार \msM\ \unmetr, 
परिवारो \Ed\ \unmetr}}% 
    \var{{\devanagarifont \numnoemph\vb\textbf{नियुतः}\lem  \mssALL,      नियुत \msCc\oo 
\textbf{च}\lem       \mssALL, चः \msNcacorr}}% 

%Verse 1:52

{\devanagarifont कोटिभिर्दशकोट्येन एकैकः परिवारितः {॥ १:५२॥} \veg\dontdisplaylinenum }%
     \var{{\devanagarifont \numnoemph\vc \lem \msCa\msCc\msPaperA\Ed, कोटिभि दशकोट्येन \msCb, 
कोटिभिर्दशकोट्योन \msNa\msNc, कोटिभिर्दशकोट्येनः \msNb, 
कोटिभिः परिवाराणि कोटिभि दशकोटिकम् \msM}}% 
    \var{{\devanagarifont \numnoemph\vd \lem \msCb\msNa\Ed, एकैकः परिवरि\uncl{तः} \msCa, 
एकैकपरिवारितः \msCc\msNb\msNc, एकैकपरिवाराणां \msM, 
एकैकः परिवारितं \msPaperA}}% 

{\devanagarifont दशकोटिषु एकैकं वृन्दवृन्दभृतैर्वृतम् \thinspace{\dandab} \dontdisplaylinenum }%
     \var{{\devanagarifont \numemph\va \lem \msCb\msCc\msNb\msPaperA\Ed, दशकोटीषु एकैकं \msCa\msNa\msNc, 
दशकोट्येषु एककं \msM}}% 
    \var{{\devanagarifont \numnoemph\vb \lem \mssCaCbCc\msNb, वृन्दवृन्दवृतैर्वृतं \msNa, 
वृन्दवृन्दभृतै वृतं \msNc, 
वृन्द्रवृन्देषु एकैकं \msM, 
वृन्दवृन्दवृतैर्वृत \msPaperA, 
वृन्दवृन्दं वृतैर्वृतः \Ed}}% 

%Verse 1:53

{\devanagarifont वृन्दवर्गेषु एकैकं खर्वभिः परिवारितम् {॥ १:५३॥} \veg\dontdisplaylinenum }%
     \var{{\devanagarifont \numnoemph\vc\textbf{वृन्दवर्गेषु}\lem \mssALL, वृन्दवर्गेभिः तै वृतम् \msM}}% 
    \var{{\devanagarifont \numnoemph\vd \lem \mssCaCbCc\msNa\msNb, खर्वर्भिः परिवारितम् \msNc, 
खर्वाभिः परिवाराणि \msM, 
खर्वर्भिः परिवारित \msPaperA, 
खर्वर्भिः परिवारितः \Ed}}% 

{\devanagarifont खर्ववर्गेषु एकैकं दशखर्वगणैर्वृतम् \thinspace{\dandab} \dontdisplaylinenum }%
     \var{{\devanagarifont \numemph\va \lem    \mssALL, खर्ववर्गेव एककम् \msM}}% 
    \var{{\devanagarifont \numnoemph\vb \lem \msCa\msCc\msNa\msNb\msPaperA, दशखर्वगणै वृतम् \msCb, 
दशखर्वगणे वृत्तं \msNc, 
दशखर्वेषु एकैकं दशखर्वगणैर्वृतम् \msM, 
दशखर्वगणैर्वृतः \Ed}}% 

%Verse 1:54

{\devanagarifont दशखर्वेषु एकैकं शङ्कुभिः परिवारितम् {॥ १:५४॥} \veg\dontdisplaylinenum }%
     \var{{\devanagarifont \numnoemph\vc\textbf{॰खर्वेषु}\lem   \mssALL, ॰गर्वेषु \msNc}}% 
    \var{{\devanagarifont \numnoemph\vd\textbf{परिवारितम्}\lem \mssALL, परिवारित \msPaperA, परिवारितः \Ed}}% 

\vfill
\pageparbreak
\vers

{\devanagarifont शङ्कुभिः पृथगेकैकं पद्मेन परिवारितम् \thinspace{\dandab} \dontdisplaylinenum }%
     \var{{\devanagarifont \numemph\va\textbf{पृथगेकैकं}\lem \eme, पृथगेनैव \msCa\msCc\msNa\msNb\msNc\msM\msPaperA\Ed, 
पृथगैनैव \msCb}}% 
    \var{{\devanagarifont \numnoemph\vb\textbf{॰वारितम्}\lem \msNapcorr\msM, ॰वारितः \mssCaCbCc\msNb\msNc\msPaperA\Ed, ॰तं \msNaacorr}}% 

%Verse 1:55

{\devanagarifont पद्मवर्गेषु एकैकं समुद्रैः परिवारितम् {॥ १:५५॥} \veg\dontdisplaylinenum }%
     \var{{\devanagarifont \numnoemph\vd\textbf{समुद्रैः}\lem \mssALL,    समुदैः \msCa, दमु\uncl{दैः} \msCb\oo 
\textbf{॰वारितम्}\lem  \mssALL, ॰वारितः \Ed}}% 

{\devanagarifont समुद्रेषु तथैकैकं मध्यसंख्यैस्तु तैर्वृतम् \thinspace{\dandab} \dontdisplaylinenum }%
     \var{{\devanagarifont \numemph\va\textbf{तथै॰}\lem \mssALL, तथे॰ \msCc}}% 
    \var{{\devanagarifont \numnoemph\vb \lem \mssCaCbCc\msNa\msM\msPaperA, 
मध्यसख्यैस्तु तै वृतम् \msNb, 
मध्यसख्यैस्तु तेर्वृतं \msNc, 
मध्ये शङ्ख्यायुतैर्वृतः \Ed}}% 

%Verse 1:56

{\devanagarifont मध्यसंख्येषु एकैकमनन्तैः परिवारितम् {॥ १:५६॥} \veg\dontdisplaylinenum }%
     \var{{\devanagarifont \numnoemph\vc\textbf{मध्यसंख्येषु}\lem     \mssALL, 
मध्यसांखो च \msM, मध्ये शंखेषु \Ed}}% 
    \var{{\devanagarifont \numnoemph\vcd\textbf{एकैकमनन्तैः}\lem \mssALL, 
एकैकं मनतैः \msNc, एकैकं अनन्तै \msM}}% 
    \var{{\devanagarifont \numnoemph\vd\textbf{॰वारितम्}\lem            \mssALL, ॰वारितः \Ed}}% 

{\devanagarifont अनन्तेषु च एकैकं परार्धपरिवारितम् \thinspace{\dandab} \dontdisplaylinenum }%
     \var{{\devanagarifont \numemph\vb \lem \msCa\msCb\msNa\msNb\msNc\msPaperA, परार्ध\lac  रितम् \msCc, 
परार्धै परिवारितम्\thinspace{\devanagarifont ।} अनन्तेषु च एकैकं परार्धपरिवारितं \msM, परार्धैः परिवारितः \Ed}}% 

{\devanagarifont परार्धेषु च एकैकं परेण परिवारितम्  \danda\dontdisplaylinenum }%
     \var{{\devanagarifont \numnoemph\vd\textbf{॰वारितम्}\lem \mssALL, ॰वारिवारितं \msNb, ॰वारितः \Ed}}% 

%Verse 1:57

{\devanagarifont एष वै कथितो विप्र शक्यं सांख्यमुदीरितम् {॥ १:५७॥} \veg\dontdisplaylinenum }%
     \var{{\devanagarifont \numnoemph\ve\textbf{कथितो}\lem \mssALL, \uncl{कथितो} \msNb, कथिता \Ed}}% 
    \var{{\devanagarifont \numnoemph\vf\textbf{शक्यं}\lem  \mssALL,  शक्य \msCc, संख्यां शक्यं \msPaperA\oo 
\textbf{सांख्यमु॰}\lem \msCa\msCc\msNb\msM, साख्यमु॰ \msCb, स्यख्यमु॰ \msNa, 
संख्यमु \msNc, संख्यामु॰ \msPaperA\Ed}}% 


\alalfejezet{प्रमाणम्}
{\devanagarifont प्रमाणं शृणु मे विप्र संक्षेपाद्ब्रुवतो मम \thinspace{\dandab} \dontdisplaylinenum }%
     \var{{\devanagarifont \numemph\va\textbf{प्रमाणं}\lem \msCc\msNa\msNc\msM\msPaperA\Ed, प्रणामं \msCa\msCb, प्रमाण \msNb}}% 
    \var{{\devanagarifont \numnoemph\vb\textbf{संक्षेपाद्ब्रुवतो}\lem \msCa\msCc\msNa\msNb\msPaperA\Ed, संक्षेपाद्वदतो \msCb, 
संख्येपाद्ब्रुवतो \msNc, 
संक्षेप ब्रुवतो \msM}}% 

%Verse 1:58

{\devanagarifont चन्द्रोदये पूर्णमास्यां वपुरण्डस्य तादृशम् {॥ १:५८॥} \veg\dontdisplaylinenum }%
 
{\devanagarifont कोटिकोटिसहस्रं तु योजनानां समन्ततः \thinspace{\dandab} \dontdisplaylinenum }%
     \var{{\devanagarifont \numemph\va\textbf{कोटिकोटि॰}\lem \mssALL, कोटीकोटि॰ \msM}}% 
    \var{{\devanagarifont \numnoemph\vb\textbf{योज॰}\lem     \mssALL,      याज॰ \msPaperA}}% 

%Verse 1:59

{\devanagarifont अण्डानां च परीमाणं ब्रह्मणा परिकीर्तितम् {॥ १:५९॥} \veg\dontdisplaylinenum }%
     \var{{\devanagarifont \numnoemph\vc\textbf{च परीमाणं}\lem \mssALL, 
च परिमाणं \msCb\ \unmetr, 
परिमाणञ्च \msM}}% 
    \var{{\devanagarifont \numnoemph\vd\textbf{ब्रह्मणा}\lem \mssALL, \lac\  \msCc\oo 
\textbf{॰कीर्तितम्}\lem \msCa\msCb\msNb\msNc\msPaperA\Ed, ॰कीर्ति\uncl{ताः} \msCc, ॰कीर्तितः \msNa\msM}}% 

\vfill
\pageparbreak
\vers

{\devanagarifont सप्तकोटिसहस्राणि सप्तकोटिशतानि च \thinspace{\dandab} \dontdisplaylinenum }%
     \var{{\devanagarifont \numemph\va\textbf{॰स्राणि}\lem \mssALL, ॰स्रणि \msPaperA}}% 

%Verse 1:60

{\devanagarifont विंशकोटिष्वङ्गुलीषु ऊर्ध्वतस्तपते रविः {॥ १:६०॥} \veg\dontdisplaylinenum }%
     \var{{\devanagarifont \numnoemph\vc \lem \conj, विंशकोटिषु गुल्मेषु \mssCaCbCc\msNa\msNb\msNc\msPaperA\Ed, 
विंशकोटि विना गुल्मे \msM}}% 
    \var{{\devanagarifont \numnoemph\vd\textbf{ऊर्ध्वतस्त॰}\lem \mssCaCbCc\msNa\msNc\Ed, ऊर्ध्व\lac\  \msNb, ऊर्द्ध्वतो त॰ \msM, 
उद्धतस्त॰ \msPaperA\oo 
\textbf{रविः}\lem \mssALL, रवि \Ed}}% 
    \lacuna{\devanagarifontsmall \vcd {\englishfont The folio in \msNb\ ends with } ऊर्ध्व॰, {\englishfont and the folios 
                                that may have contained verses 1.60d--2.22 are missing.} }%
  
{\devanagarifont प्रमाणं नाम संख्या च कीर्तितानि समासतः \thinspace{\dandab} \dontdisplaylinenum }%
     \var{{\devanagarifont \numemph\va \lem \mssALL, प्रणामं नाम संख्या च \msCb, 
प्रमाणेनाणञ्चम संख्या\lk त च \msPaperA}}% 
    \var{{\devanagarifont \numnoemph\vb\textbf{कीर्तितानि}\lem \mssALL, कीर्त्तियानानि \msPaperA}}% 

%Verse 1:61

{\devanagarifont ब्रह्माण्डं चाप्रमेयाणां लक्षणं परिकीर्तितम् {॥ १:६१॥} \veg\dontdisplaylinenum }%
     \var{{\devanagarifont \numnoemph\vc\textbf{ब्रह्माण्डं चा॰}\lem \msNa, ब्रह्माण्डश्च \msCa\msCb\msNc\msM\msPaperA, \uncl{ब्रह्माण्डाश्चा}॰ \msCc, 
ब्रह्माण्डाश्चा \Ed\oo 
\textbf{॰मेयाणां}\lem \msCa\msNa\msM\msPaperA\Ed, ॰मेयाणा \msCb\msCc\msNc}}% 
    \var{{\devanagarifont \numnoemph\vd\textbf{॰कीर्तितम्}\lem \mssALL, ॰कीर्तिताः \msCc, ॰कीर्त्तितः \msM}}% 


\alalfejezet{पुराणम्}
{\devanagarifont पुराणाशीसहस्राणि शतानि द्विजसत्तम \thinspace{\dandab} \dontdisplaylinenum }%
     \var{{\devanagarifont \numemph\vb\textbf{॰सत्तम}\lem \mssALL, \lac  मः  \msCc}}% 

%Verse 1:62

{\devanagarifont ब्रह्मणा कथितं पूर्णं मातरिश्वा यथातथम् {॥ १:६२॥} \veg\dontdisplaylinenum }%
     \var{{\devanagarifont \numnoemph\vc\textbf{पूर्णं}\lem    \msCa\msCc\msNa\msPaperA\Ed,              पूर्वे \msCb, पूर्ण्ण \msNc, पूर्वं \msM}}% 
    \var{{\devanagarifont \numnoemph\vd\textbf{मातरिश्वा}\lem \mssALL,  मातरिश्व \msM\oo 
\textbf{॰तथम्}\lem   \mssALL, ॰तथा \msCc\msM}}% 

{\devanagarifont वायुना पाद संक्षिप्य प्राप्तं चोशनसं पुरा \thinspace{\dandab} \dontdisplaylinenum }%
     \var{{\devanagarifont \numemph\va\textbf{संक्षिप्य}\lem \mssALL, संक्षिप्यः \msM}}% 
    \var{{\devanagarifont \numnoemph\vb\textbf{प्राप्तं चोशनसं}\lem \msCb\msNa\msNc, प्राप्तं चौसनसं \msCa\msPaperA, प्राप्त\lk औसनसं \msCc, 
प्राप्ताश्चोशनसम \msM\ \unmetr, प्राप्तश्चोशनसं \Ed}}% 

%Verse 1:63

{\devanagarifont तेनापि पाद संक्षिप्य प्राप्तवांश्च बृहस्पतिः {॥ १:६३॥} \veg\dontdisplaylinenum }%
     \var{{\devanagarifont \numnoemph\vc\textbf{संक्षिप्य}\lem \mssALL, संक्षिप्यः \msM}}% 
    \var{{\devanagarifont \numnoemph\vd \lem \mssALL, प्राप्तधञ्च वृहस्पति \msM}}% 

{\devanagarifont बृहस्पतिस्तु प्रोवाच सूर्यं त्रिंशत्सहस्रिकम् \thinspace{\dandab} \dontdisplaylinenum }%
     \var{{\devanagarifont \numemph\vb\textbf{सूर्यं}\lem \msCc\Ed, सूर्यस् \msCa\msNa\msNc\msPaperA, सूर्य \msCb\msM\oo 
\textbf{त्रिंशत्स॰}\lem \mssALL, त्रिंशस॰ \msCc\msM}}% 

%Verse 1:64

{\devanagarifont पञ्चविंशत्सहस्राणि मृत्युं प्राह दिवाकरः {॥ १:६४॥} \veg\dontdisplaylinenum }%
     \var{{\devanagarifont \numnoemph\vc\textbf{॰विंशत्सहस्राणि}\lem \corr, ॰विंशहस्राणि \msCa, 
॰विंशसहस्राणि \msCb\msCc\msNa\msNc\msM\msPaperA, ॰विशत्सहस्राणि \Ed}}% 
    \var{{\devanagarifont \numnoemph\vd\textbf{मृत्युं प्राह}\lem \mssALL, मृत्यु प्राहः \msM}}% 

\vfill
\pageparbreak
\vers

{\devanagarifont एकविंशत्सहस्राणि मृत्युनेन्द्राय कीर्तितम् \thinspace{\dandab} \dontdisplaylinenum  }%
     \var{{\devanagarifont \numemph\va\textbf{॰विंशत्॰}\lem \Ed, ॰विंश॰ \mssCaCbCc\msNa\msNc\msM\msPaperA}}% 
    \var{{\devanagarifont \numnoemph\vb\textbf{कीर्तितम्}\lem \Ed, कीर्तितः \msCa\msCb\msNa\msNcpcorr\msM, कीर्तिताः \msCc, 
कीर्त्तित \msNcacorr, कीर्तितंः \msPaperA}}% 

%Verse 1:65

{\devanagarifont इन्द्रेणाह वसिष्ठाय विंशत्श्लोकसहस्रिकम् {॥ १:६५॥} \veg\dontdisplaylinenum }%
     \var{{\devanagarifont \numnoemph\vc\textbf{इन्द्रे॰}\lem \mssALL, इन्दे॰ \msPaperA}}% 
    \var{{\devanagarifont \numnoemph\vc\textbf{वसिष्ठाय}\lem \mssALL, विशिष्ठाय \msCb, वहिष्ठाय \msNc}}% 
    \var{{\devanagarifont \numnoemph\vd\textbf{विंशत्श्लो॰}\lem \corr, विंशश्लो॰ \msCa\msCc\msNa\msNc\msPaperA\Ed, विशश्लो॰ \msCb, त्रिंशश्लो॰ \msM}}% 

{\devanagarifont अष्टादशसहस्राणि तेन सारस्वताय तु \thinspace{\dandab} \dontdisplaylinenum }%
     \var{{\devanagarifont \numemph\va \lem \mssALL, 
आष्टादशसहस्राणि \msNc, वसिष्ठेदशसहस्रं \msM}}% 

%Verse 1:66

{\devanagarifont सारस्वतस्त्रिधामाय सहस्रदश सप्त च {॥ १:६६॥} \veg\dontdisplaylinenum }%
     \var{{\devanagarifont \numnoemph\vc\textbf{सारस्वतस्त्रि॰}\lem \eme, सारस्वता त्रि॰ \msCa\msCc\msNa\msNc\msPaperA\Ed, सारस्वतास्त्रि॰ \msCb, 
सारस्वत तृ॰ \msM\oo 
\textbf{॰धामाय}\lem \mssALL, \om\ \msNaacorr}}% 
    \var{{\devanagarifont \numnoemph\vd\textbf{सहस्रदश}\lem \mssALL, सहस्रादश \msM}}% 

{\devanagarifont षोडशानां सहस्राणि भरद्वाजाय वै ततः \thinspace{\dandab} \dontdisplaylinenum }%
     \var{{\devanagarifont \numemph\vb\textbf{भर॰}\lem \mssALL, भार॰ \msCc, सन॰ \msM}}% 

%Verse 1:67

{\devanagarifont दश पञ्चसहस्राणि त्रिवृषाय अभाषत {॥ १:६७॥} \veg\dontdisplaylinenum }%
     \var{{\devanagarifont \numnoemph\vd\textbf{अभाषत}\lem \msCa\msCb\msNa\msPaperA, अ\uncl{भाषत} \msCc, अभाषतः \msNc\Ed, मभासतः \msM}}% 

{\devanagarifont चतुर्दशसहस्राणि अन्तरीक्षाय वै ततः \thinspace{\dandab} \dontdisplaylinenum }%
     \var{{\devanagarifont \numemph\vb\textbf{अन्तरी॰}\lem \mssALL, अन्तरि॰ \msM}}% 

%Verse 1:68

{\devanagarifont त्रय्यारुणिं सहस्राणि त्रयोदश अभाषत {॥ १:६८॥} \veg\dontdisplaylinenum }%
     \var{{\devanagarifont \numnoemph\vc\textbf{त्रय्यारुणिं}\lem \corr, त्र्यैयारुणि \msCa\msCb\msNa\msM\msPaperA, त्रैयारुणि \msCc\Ed, 
त्र्यैयारूपिनि \msNc}}% 
    \var{{\devanagarifont \numnoemph\vd\textbf{अभाषत}\lem \msCa\msCc\msNc\msPaperA, अभाषतः \msCb, स्वभावत \msNa, मभासतः \msM, 
ह्यभाषत \Ed}}% 

{\devanagarifont त्रय्यारुणिस्तु विप्रेन्द्रो धनंजयमभाषत \thinspace{\dandab} \dontdisplaylinenum }%
     \var{{\devanagarifont \numemph\va\textbf{त्रय्यारुणि॰}\lem \corr, त्र्यैयारुणि॰ \mssCaCbCc\msNc\msPaperA, त्रैयारुणि॰ \msNa\Ed, त्र्यैर्यारुणि॰ \msM\oo 
\textbf{विप्रेन्द्रो}\lem \mssALL, विप्रेन्द \msCc\msM}}% 
    \var{{\devanagarifont \numnoemph\vb\textbf{धनंजय॰}\lem \mssALL, धन॰ \msNaacorr\oo 
\textbf{॰भाषत}\lem \msCa\msCc\msNa\msNc\msPaperA, ॰भाषतः \msCb\msM\Ed}}% 

%Verse 1:69

{\devanagarifont द्वादशानि सहस्राणि संक्षिप्य पुनरब्रवीत् {॥ १:६९॥} \veg\dontdisplaylinenum }%
 
{\devanagarifont कृतंजयाय सम्प्राप्तो धनंजयमहामुनिः \thinspace{\dandab} \dontdisplaylinenum }%
     \var{{\devanagarifont \numemph\vb\textbf{॰मुनिः}\lem \mssALL, ॰मुणि \msM}}% 

%Verse 1:70

{\devanagarifont कृतंजयाद्द्विजश्रेष्ठ ऋणंजयमहात्मने {॥ १:७०॥} \veg\dontdisplaylinenum }%
     \var{{\devanagarifont \numnoemph\vc\textbf{कृतंजयाद्द्वि॰}\lem \msCa\msNa\msPaperA\Ed, कृतंजया द्वि॰ \msCb\msCc\msNc, धनञ्जय द्वि॰ \msM\oo 
\textbf{॰श्रेष्ठ}\lem \mssALL, ॰श्रेष्ठो \Ed}}% 
    \var{{\devanagarifont \numnoemph\vd\textbf{ऋणंजय॰}\lem \mssALL, ऋणंजाय॰ \msCb\oo 
\textbf{॰महात्मने}\lem \mssALL, ॰मभाशतः \msM}}% 

{\devanagarifont ऋणञ्जयात्पुनः प्राप्तो गौतमाय महर्षिणे \thinspace{\dandab} \dontdisplaylinenum }%
     \var{{\devanagarifont \numemph\va\textbf{प्राप्तो}\lem \mssALL, प्राप्तः \msM, प्राप्तौ \Ed}}% 
    \var{{\devanagarifont \numnoemph\vb\textbf{महर्षिणे}\lem \mssALL, महर्षिणः \msM}}% 

%Verse 1:71

{\devanagarifont गौतमाच्च भरद्वाजस्तस्माद्धर्यद्वताय तु {॥ १:७१॥} \veg\dontdisplaylinenum }%
     \var{{\devanagarifont \numnoemph\vc\textbf{गौतमाच्च}\lem \mssCaCbCc\msNa\Ed, गौतमाश्च \msNc\msPaperA, गौतमेन \msM}}% 
    \var{{\devanagarifont \numnoemph\vcd\textbf{भरद्वाजस्तस्माद्धर्यद्वताय}\lem \msCa\msCc\msNa\msNc, 
भरद्वारस्तस्माद्धर्यद्वताय \msCb, 
भरद्वाज तस्मा हर्यद्वताय \msM, 
भरद्वाजस्तस्माद्धर्यद्वनाय \msPaperA, 
भरद्वाजस्तस्माद्दम्याद्दमाय \Ed}}% 

{\devanagarifont राजश्रवास्ततः प्राप्तः सोमशुष्माय वै ततः \thinspace{\dandab} \dontdisplaylinenum }%
     \var{{\devanagarifont \numemph\va\textbf{राजश्रवास्त॰}\lem \eme, राजश्रव त॰ \mssCaCbCc\msNa\msPaperA\Ed, राजश्रवे त॰ \msNc, 
राजर्षव त॰ \msM}}% 
    \var{{\devanagarifont \numnoemph\vab\textbf{प्राप्तः सोम॰}\lem \mssALL, प्राप्त साम॰ \msPaperA}}% 

%Verse 1:72

{\devanagarifont सोमशुष्मात्ततः प्राप्तस्तृणबिन्दुस्तु भो द्विज {॥ १:७२॥} \veg\dontdisplaylinenum }%
     \var{{\devanagarifont \numnoemph\vc\textbf{॰शुष्मात्त॰}\lem \mssALL, ॰शुष्मा त॰ \msNa}}% 
    \var{{\devanagarifont \numnoemph\vcd\textbf{प्राप्तस्तृणबिन्दुस्तु}\lem \mssALL, 
प्रा\uncl{प्त तृ}णबिन्दुस्तु \msCc, 
प्राप्तस्तृणविन्दुन्तु \msPaperA}}% 
    \var{{\devanagarifont \numnoemph\vd\textbf{भो}\lem \mssALL, \om\ \msCb}}% 

{\devanagarifont तृणबिन्दुस्तु वृक्षाय वृक्षः शक्तिमभाषत \thinspace{\dandab} \dontdisplaylinenum }%
     \var{{\devanagarifont \numemph\vb\textbf{वृक्षः}\lem \mssALL, वृक्ष \msM\oo 
\textbf{॰भाषत}\lem \msCa\msCb\msNa\msNc\msPaperA, ॰भाषतः \msCc\msM\Ed}}% 

%Verse 1:73

{\devanagarifont शक्तिः पराशरं प्राह जतुकर्णाय वै ततः {॥ १:७३॥} \veg\dontdisplaylinenum }%
     \var{{\devanagarifont \numnoemph\vc\textbf{शक्तिः पराशरं}\lem \mssALL, 
शपरासर \msMacorr, शक्ति परासर \msMpcorr}}% 
    \var{{\devanagarifont \numnoemph\vd\textbf{जतु॰}\lem \mssALL, तु॰ \msCb, जंतु॰ \msM}}% 

{\devanagarifont द्वैपायनं तु प्रोवाच जतुकर्णो महर्षिणम् \thinspace{\dandab} \dontdisplaylinenum }%
     \var{{\devanagarifont \numemph\va\textbf{द्वैपायनं तु}\lem \eme, द्वैपायनस्तु \mssCaCbCc\msNa\msNc\msM\msPaperA, 
द्वैपायनाय \Ed\ \unmetr}}% 
    \var{{\devanagarifont \numnoemph\vb \lem \msCa\msCb\msNapcorr\msNc, जतुकर्णा महर्षिणः \msCc, 
जकर्णो महर्षिणं \msNaacorr, जंतुकर्ण्णमहर्षिणा \msM, जतुकर्णा महषिण \msPaperA, 
जतुकर्णमहर्षिणा \Ed}}% 

%Verse 1:74

{\devanagarifont रोमहर्षाय सम्प्राप्तो द्वैपायनमहामुनिः {॥ १:७४॥} \veg\dontdisplaylinenum }%
     \var{{\devanagarifont \numnoemph\vd\textbf{॰मुनिः}\lem \mssALL, ॰मुनि \msM\Ed}}% 

{\devanagarifont रोमहर्षेण प्रोवाच पुत्रायामितबुद्धये \thinspace{\dandab} \dontdisplaylinenum }%
     \var{{\devanagarifont \numemph\va\textbf{॰हर्षेण}\lem \msM, ॰हर्षाय \mssCaCbCc\msNa\msNc\msPaperA, ॰हर्षणाय \Ed}}% 
    \var{{\devanagarifont \numnoemph\vb\textbf{॰बुद्धये}\lem \mssALL, ॰बुद्धयः \msM}}% 
    \paral{{\devanagarifontsmall \vab \similar\ {\englishfont \BRAHMANDAPUR\ 3.4.67ab:}
                 मया चैतत्पुनः प्रोक्तं पुत्रायामितबुद्धये }}

{\devanagarifont दश द्वे च सहस्राणि पुराणं सम्प्रकाशितम्  \danda\dontdisplaylinenum }%
     \var{{\devanagarifont \numnoemph\vd \lem \mssALL, 
पुराण सम्प्रकाशितां \msCc}}% 

%Verse 1:75

{\devanagarifont मानुषाणां हितार्थाय किं भूयः श्रोतुमिच्छसि {॥ १:७५॥} \veg\dontdisplaylinenum }%
     \var{{\devanagarifont \numnoemph\ve\textbf{मानुषाणां}\lem \mssALL, मनुषाणां \msCb, मानुषाना \msM\oo 
\textbf{हितार्थाय}\lem \mssALL, हित्यथाय \msM, हिताथयि \msPaperA}}% 
    \var{{\devanagarifont \numnoemph\vf\textbf{भूयः}\lem \mssALL, भूय \msM\Ed}}% 

{\devanagarifont 
\jump
\begin{center}
\ketdanda~इति वृषसारसंग्रहे ब्रह्माण्डसंख्या नामाध्यायः प्रथमः~\ketdanda
\end{center}
\dontdisplaylinenum\vers  }%
     \var{{\devanagarifont \numnoemph{\englishfont \Colo:}\textbf{नामाध्यायः प्रथमः}\lem \mssALL, 
नामाध्यायः प्रथमः श्लोक ७७ \msM, 
नाम प्रथमो ऽध्याय \Ed}}% 
\bekveg\szamveg
\vfill
\phpspagebreak

\versno=0\fejno=2
\thispagestyle{empty}

\centerline{\Large\devanagarifontbold [   द्वितीयो ऽध्यायः  ]}{\vrule depth10pt width0pt} \fancyhead[CO]{{\footnotesize\devanagarifont वृषसारसंग्रहे  }}
\fancyhead[CE]{{\footnotesize\devanagarifont द्वितीयो ऽध्यायः  }}
\fancyhead[LE]{}
\fancyhead[RE]{}
\fancyhead[LO]{}
\fancyhead[RO]{}
\szam\bek


\vers


{\devanagarifont विगतराग उवाच {\dandab}\dontdisplaylinenum  }%
 
{\devanagarifont श्रुतं मया जनाग्रेण ब्रह्माण्डस्य तु निर्णयम् \thinspace{\danda} \dontdisplaylinenum }%
     \var{{\devanagarifont \numemph\va\textbf{जनाग्रेण}\lem \mssALL, जना\lac\  \msCa}}% 

%Verse 2:1

{\devanagarifont प्रमाणं वर्णरूपं च संख्या तस्य समासतः {॥ २:१॥} \veg\dontdisplaylinenum }%
     \lacuna{\devanagarifontsmall {\englishfont Witnesses used for this chapter: \msCa\ ff.\thinspace 195v--197r, 
                                             \msCb\ ff.\thinspace 203v--204v,
                                             \msCc\ ff.\thinspace 270r--270v (it breaks off at 2.21 and resumes at 3.30b),
                                             \msNa\ ff.\thinspace 3v--4v, 
                                             \msNb\ exp.\thinspace 43 and 42 (sic!; it broke off at 1.60d and resumes at 2.23),
                                             \msNc\ ff.\thinspace 211v--213r,
                                             \Ed\ pp.\thinspace 585--588;
                                             \mssCaCbCc\ = \msCa + \msCb + \msCc } }%
  
{\devanagarifont शिवाण्डेति त्वया प्रोक्तो ब्रह्माण्डालयकीर्तितः \thinspace{\dandab} \dontdisplaylinenum }%
     \var{{\devanagarifont \numemph\vb\textbf{ब्रह्माण्डा॰}\lem \mssALL, ब्रह्माण्ड \Ed}}% 

%Verse 2:2

{\devanagarifont कीदृशं लक्षणं ज्ञेयं प्रमाणं तस्य वा कति {॥ २:२॥} \veg\dontdisplaylinenum }%
     \var{{\devanagarifont \numnoemph\vc\textbf{ज्ञेयं}\lem \mssALL, ज्ञेया \msCc}}% 
    \var{{\devanagarifont \numnoemph\vd\textbf{कति}\lem \mssALL, कतिः \msCc}}% 

{\devanagarifont कस्य वा लयनं ज्ञेयं प्रमाणं वात्र वासिनः \thinspace{\dandab} \dontdisplaylinenum }%
     \var{{\devanagarifont \numemph\va\textbf{लयनं ज्ञेयं}\lem \mssALL, लयनं \msCb, लक्षणं ज्ञेयं \Ed}}% 
    \var{{\devanagarifont \numnoemph\vb\textbf{वासिनः}\lem \mssALL, वासिरानः \msCb}}% 

%Verse 2:3

{\devanagarifont का वा तत्र प्रजा ज्ञेया को वा तत्र प्रजापतिः {॥ २:३॥} \veg\dontdisplaylinenum }%
     \var{{\devanagarifont \numnoemph\vc\textbf{का}\lem \eme, को \mssCaCbCc\msNa\msNc, किं \Ed\oo 
\textbf{प्रजा ज्ञेया}\lem \mssALL, प्र\uncl{जा}\lac  या \msCa}}% 


\alalfejezet{शिवाण्डसंख्या}
{\devanagarifont अनर्थयज्ञ उवाच {\dandab}\dontdisplaylinenum  }%
 
{\devanagarifont शिवाण्डलक्षणं विप्र न त्वं प्रष्टुमिहार्हसि \thinspace{\danda} \dontdisplaylinenum }%
     \var{{\devanagarifont \numemph\vb\textbf{न त्वं}\lem \mssALL, तत्वं \Ed\oo 
\textbf{॰र्हसि}\lem \mssALL, ॰हसि \msNc}}% 

%Verse 2:4

{\devanagarifont दैवतैरपि का शक्तिर्ज्ञातुं द्रष्टुं च तत्त्वतः {॥ २:४॥} \veg\dontdisplaylinenum }%
     \var{{\devanagarifont \numnoemph\vc\textbf{दैवतै॰}\lem \msCa\msCb\msNa, देवतै॰ \msCc\msNc\Ed\oo 
\textbf{शक्तिर्}\lem \msCa, शक्ति \msCb\msCc\msNa\msNc\Ed}}% 

{\devanagarifont अगम्यगमनं गुह्यं गुह्यादपि समुद्धितम् \thinspace{\dandab} \dontdisplaylinenum }%
     \var{{\devanagarifont \numemph\va\textbf{अगम्यगमनं}\lem \mssALL, अगम्यगगहनं \msCc, अगम्यगगमनं \msNc}}% 
    \var{{\devanagarifont \numnoemph\vb\textbf{गुह्या॰}\lem \msNc\Ed, गुहा॰ \mssCaCbCc\msNa\oo 
\textbf{समुद्धितं}\lem \mssALL, सम्रद्धितं \msNc, समृद्धिदम् \Ed}}% 
    \paral{{\devanagarifontsmall \vab {\englishfont \compare\ \LINPU\ 1.21.71ab:} नमो गुण्याय गुह्याय अगम्यगमनाय च }}

%Verse 2:5

{\devanagarifont न प्रभुर्नेतरस्तत्र न दण्ड्यो न च दण्डकः {॥ २:५॥} \veg\dontdisplaylinenum }%
     \var{{\devanagarifont \numnoemph\vc\textbf{प्रभुर्ने॰}\lem \mssALL, प्रने॰ \msCc}}% 
    \var{{\devanagarifont \numnoemph\vd\textbf{दण्ड्यो}\lem \msCc\msNa\msNc, दण्डो \msCa\msCb, दण्ड्या \Ed\oo 
\textbf{दण्डकः}\lem \mssALL, ण्डकः \msCbacorr, पण्डकः \msCbpcorr}}% 

{\devanagarifont न सत्यो नानृतस्तत्र सुशीलो नो दुःशीलवान् \thinspace{\dandab} \dontdisplaylinenum }%
     \var{{\devanagarifont \numemph\va\textbf{सत्यो}\lem \mssALL, सत्यौ \Ed\oo 
\textbf{तत्र}\lem \mssALL, तत्रा \Ed}}% 
    \var{{\devanagarifont \numnoemph\vb\textbf{नो}\lem \mssALL, \lac\  \msCa}}% 

%Verse 2:6

{\devanagarifont नानृजुर्न च दम्भित्वं न तृष्णा न च ईर्ष्यता {॥ २:६॥} \veg\dontdisplaylinenum }%
     \var{{\devanagarifont \numnoemph\vc\textbf{नानृजुर्न}\lem \eme, नाऋजुर्न्न \msCa\Ed, नाऋजुर्न \msCb\msNc, 
\uncl{नाऋजु न} \msCc, नाऋजुन्न \msNa}}% 
    \var{{\devanagarifont \numnoemph\vd\textbf{न तृष्णा न च}\lem \mssALL,  न च तृष्णा न \msNa\oo 
\textbf{ईर्ष्यता}\lem \mssALL, ईर्ष्यताः \msCc, इर्ष्यता \Ed}}% 

{\devanagarifont न क्रोधो न च लोभो ऽस्ति न मानो ऽस्ति न सूयकः \thinspace{\dandab} \dontdisplaylinenum }%
     \var{{\devanagarifont \numemph\va\textbf{क्रोधो}\lem \mssALL, क्रोधौ \msCc}}% 
    \var{{\devanagarifont \numnoemph\vb\textbf{सूयकः}\lem \mssALL, सूचकः \msCb, स्तेयकः \Ed\ \unmetr}}% 

%Verse 2:7

{\devanagarifont ईर्ष्या द्वेषो न तत्रास्ति न शठो न च मत्सरः {॥ २:७॥} \veg\dontdisplaylinenum }%
     \var{{\devanagarifont \numnoemph\vd\textbf{शठो}\lem \mssALL, षठो \msCc, शठे \Ed\oo 
\textbf{मत्सरः}\lem \mssALL, मत्सराः \Ed}}% 

{\devanagarifont न व्याधिर्न जरा तत्र न शोको ऽस्ति न विक्लवः \thinspace{\dandab} \dontdisplaylinenum }%
     \var{{\devanagarifont \numemph\va\textbf{व्याधिर्न}\lem \mssALL, व्याधि न \msCc\msNc\oo 
\textbf{जरा तत्र}\lem \msCb\msNc, जरास्तत्र \msCa\msCc\msNa\Ed}}% 
    \var{{\devanagarifont \numnoemph\vb\textbf{विक्लवः}\lem \mssALL, विक्लव \Ed}}% 

%Verse 2:8

{\devanagarifont नाधमः पुरुषस्तत्र नोत्तमो न च मध्यमः {॥ २:८॥} \veg\dontdisplaylinenum }%
 
{\devanagarifont नोत्कृष्टो मानवस्तस्मिन्स्त्रियश्चैव शिवालये \thinspace{\dandab} \dontdisplaylinenum }%
     \var{{\devanagarifont \numemph\va\textbf{मानव॰}\lem \mssALL, मा\lac  व॰ \msCa}}% 

%Verse 2:9

{\devanagarifont न निन्दा न प्रशंसास्ति मत्सरी पिशुनो न च {॥ २:९॥} \veg\dontdisplaylinenum }%
     \var{{\devanagarifont \numnoemph\vc\textbf{प्रशंसास्ति}\lem \mssALL, प्रशंसाश्च \Ed}}% 

{\devanagarifont गर्वदर्पं न तत्रास्ति क्रूरमायादिकं तथा \thinspace{\dandab} \dontdisplaylinenum }%
 
%Verse 2:10

{\devanagarifont याचमानो न तत्रास्ति दाता चैव न विद्यते {॥ २:१०॥} \veg\dontdisplaylinenum }%
     \var{{\devanagarifont \numemph\vc\textbf{तत्रास्ति}\lem \mssALL, तत्रा \msNaacorr}}% 

{\devanagarifont अनर्थी व्रज तत्रस्थः कल्पवृक्षसमाश्रितः \thinspace{\dandab} \dontdisplaylinenum }%
     \var{{\devanagarifont \numemph\va\textbf{व्रज त॰}\lem \mssALL, व्रजस्त॰ \msNc}}% 

%Verse 2:11

{\devanagarifont न कर्म नाप्रियस्तत्र न कलिः कलहो न च {॥ २:११॥} \veg\dontdisplaylinenum }%
     \var{{\devanagarifont \numnoemph\vc\textbf{कर्म ना॰}\lem \eme, कर्म न \mssCaCbCc\msNa\msNc, कर्मणा \Ed}}% 
    \var{{\devanagarifont \numnoemph\vd\textbf{कलिः}\lem \mssALL, कलि \msNcacorr\Ed}}% 

{\devanagarifont द्वापरो न च न त्रेता कृतं चापि न विद्यते \thinspace{\dandab} \dontdisplaylinenum }%
     \var{{\devanagarifont \numemph\va\textbf{च न त्रेता}\lem \mssALL, च न त्रेत्रा \msCa, च त्रेता न \msCb}}% 
    \var{{\devanagarifont \numnoemph\vb\textbf{कृतं चा॰}\lem \msCc\msNa, कृतश्चा॰ \msCa\msCb\msNc\Ed}}% 

%Verse 2:12

{\devanagarifont मन्वन्तरं न तत्रास्ति कल्पश्चैव न विद्यते {॥ २:१२॥} \veg\dontdisplaylinenum }%
     \var{{\devanagarifont \numnoemph\vc \lem \mssALL, मन्वन्तत्रास्ति \msCc, 
मन्वन्तरनन्त तत्रास्ति \msNc}}% 
    \var{{\devanagarifont \numnoemph\vd\textbf{कल्पश्चैव}\lem \mssALL, कल्पं चैव \msNa}}% 

{\devanagarifont आहूतसम्प्लवं नास्ति ब्रह्मरात्रिदिनं तथा \thinspace{\dandab} \dontdisplaylinenum }%
     \var{{\devanagarifont \numemph\va\textbf{आहूत॰}\lem \mssALL, आभूत॰ \Ed}}% 
    \var{{\devanagarifont \numnoemph\vb\textbf{ब्रह्मरात्रिदिनं}\lem \mssALL, ब्रह्मरात्रिदिवस् \Ed}}% 

%Verse 2:13

{\devanagarifont न जन्ममरणं तत्र आपदं नाप्नुयात्क्वचित् {॥ २:१३॥} \veg\dontdisplaylinenum }%
     \var{{\devanagarifont \numnoemph\vc\textbf{जन्ममरणं तत्र}\lem \msCc\msNa\Ed, जन्मरणं तत्र \msCa\msCb, 
जन्ममरणन्त्रत \msNc}}% 
    \var{{\devanagarifont \numnoemph\vd\textbf{आपदं}\lem \mssALL, अपदं \Ed}}% 

{\devanagarifont न चाशापाशबद्धो ऽस्ति रागमोहं न विद्यते \thinspace{\dandab} \dontdisplaylinenum }%
     \var{{\devanagarifont \numemph\va\textbf{चाशापाश॰}\lem \msCb\msNcpcorr, च सायाश॰ \msCa\msCc\msNa\msNcacorr\Ed\oo 
\textbf{॰बद्धो}\lem \mssALL, ॰द्धो \msCc, ॰वृद्धो \Ed}}% 
    \var{{\devanagarifont \numnoemph\vb\textbf{॰मोहं}\lem \mssALL, ॰मोहो \msCa}}% 

%Verse 2:14

{\devanagarifont न देवा नासुरास्तत्र न यक्षोरगराक्षसाः {॥ २:१४॥} \veg\dontdisplaylinenum }%
     \var{{\devanagarifont \numnoemph\vc\textbf{देवा}\lem \mssALL, देवो \msCb}}% 

{\devanagarifont न भूता न पिशाचाश्च गन्धर्वा ऋषयस्तथा \thinspace{\dandab} \dontdisplaylinenum }%
     \var{{\devanagarifont \numemph\vb\textbf{गन्धर्वा}\lem \mssALL,  गन्धर्वो \Ed}}% 

%Verse 2:15

{\devanagarifont ताराग्रहं न तत्रास्ति नागकिंनरगारुडम् {॥ २:१५॥} \veg\dontdisplaylinenum }%
 
{\devanagarifont न जपो नाह्निकस्तत्र नाग्निहोत्री न यज्ञकृत् \thinspace{\dandab} \dontdisplaylinenum }%
     \var{{\devanagarifont \numemph\va\textbf{जपो}\lem \mssALL, जयो \msCa\oo 
\textbf{नाह्निकस्त॰}\lem \mssALL, नाह्निक त॰ \msCb}}% 

%Verse 2:16

{\devanagarifont न व्रतं न तपश्चैव न तिर्यङ्नरकं तथा {॥ २:१६॥} \veg\dontdisplaylinenum }%
     \var{{\devanagarifont \numnoemph\vd\textbf{न तिर्यङ्नरकं}\lem \eme, नातिर्यन्नरकस् \msCa\msCc\msNa, 
नातिर्यनरकन् \msCb, नात्रिर्यं नरकस् \msNc, न तीर्थन्नरकन् \Ed}}% 
    \paral{{\devanagarifontsmall \vd {\englishfont \compare\ 19.49cd:} विसृष्टे त्विन्द्रियग्रामे तिर्यङ्नरकसाधनम् }}

{\devanagarifont तस्येशानस्य देवस्य ऐश्वर्यगुणविस्तरम् \thinspace{\dandab} \dontdisplaylinenum }%
     \paral{{\devanagarifontsmall \vc {\englishfont \compare\ \MBH\ Suppl. 14.4.2743:} ऐश्वर्यगुणसंपन्नाः क्रीडन्ति च यथासुखम्, 
                               {\englishfont and \BRAHMANDAPUR\ 1.26.1:} महादेवस्य महात्म्यं प्रभुत्वं च महात्मनः\thinspace{\devanagarifontsmall ।}  
                                                             श्रोतुमिच्छामहे सम्यगैश्वर्यगुणविस्तरम्\thinspace{\devanagarifontsmall ॥} }}

%Verse 2:17

{\devanagarifont अपि वर्षशतेनापि शक्यं वक्तुं न केनचित् {॥ २:१७॥} \veg\dontdisplaylinenum }%
 
{\devanagarifont हरेच्छाप्रभवाः सर्वे पर्यायेण ब्रवीमि ते \thinspace{\dandab} \dontdisplaylinenum }%
     \var{{\devanagarifont \numemph\va\textbf{हरेच्छाप्रभवाः}\lem \msNc, हरेच्छप्रभवाः \mssCaCbCc\msNa, हरेच्छाप्रभवा \Ed}}% 

%Verse 2:18

{\devanagarifont देवमानुषवर्ज्यानि वृक्षगुल्मलतादयः {॥ २:१८॥} \veg\dontdisplaylinenum }%
     \var{{\devanagarifont \numnoemph\vc\textbf{वर्ज्यानि}\lem \mssALL, वज्ज्ञानि \Ed}}% 

{\devanagarifont परार्धद्विगुणोत्सेधो विस्तारश्च तथाविधः \thinspace{\dandab} \dontdisplaylinenum }%
     \var{{\devanagarifont \numemph\va\textbf{॰गुणोत्सेधो}\lem \conj, ॰गुणोच्छेधा \msCa\msCb\msNa\msNc, ॰गुणेच्छेधा \msCc, ॰गुणाच्छ्रेधा \Ed}}% 
    \var{{\devanagarifont \numnoemph\vb\textbf{विस्तारश्च}\lem \msNc, विस्तारं च \mssCaCbCc\msNa\Ed\oo 
\textbf{॰विधः}\lem \msNc, ॰विधा \mssCaCbCc\msNa\Ed}}% 

%Verse 2:19

{\devanagarifont अनेकाकारपुष्पाणि फलानि च मनोहरम् {॥ २:१९॥} \veg\dontdisplaylinenum }%
     \var{{\devanagarifont \numnoemph\vc\textbf{अनेकाकार॰}\lem \mssALL, अनेकार॰ \msCa}}% 

{\devanagarifont अन्ये काञ्चनवृक्षाणि मणिवृक्षाण्यथापरे \thinspace{\dandab} \dontdisplaylinenum }%
     \var{{\devanagarifont \numemph\va\textbf{अन्ये}\lem \mssALL, बहु॰ \Ed}}% 

%Verse 2:20

{\devanagarifont प्रवालमणिषण्डाश्च पद्मरागरुहाणि च {॥ २:२०॥} \veg\dontdisplaylinenum }%
     \var{{\devanagarifont \numnoemph\vc\textbf{षण्डाश्च}\lem \mssALL, घण्टाश्च \Ed}}% 
    \var{{\devanagarifont \numnoemph\vd\textbf{॰रुहाणि}\lem \msCc, ॰रुहानि \msCa\msCb\msNa\msNc, ॰सहानि \Ed}}% 

{\devanagarifont स्वादुमूलफलाः स्कन्धलताविटपपादपाः \thinspace{\dandab} \dontdisplaylinenum }%
     \var{{\devanagarifont \numemph\va\textbf{स्वादु॰}\lem \mssALL, स्वाधु॰ \msCa\oo 
\textbf{॰मूल॰}\lem \mssALL, ॰मूला \msNa\oo 
\textbf{॰फलाः}\lem \conj, ॰फला \mssCaCbCc\msNa\msNc\Ed}}% 
    \var{{\devanagarifont \numnoemph\vb\textbf{स्कन्ध॰}\lem \conj, स्कन्द॰ \mssCaCbCc\msNa\msNc\Ed}}% 

%Verse 2:21

{\devanagarifont कामरूपाश्च ते सर्वे कामदाः कामभाषिणः {॥ २:२१॥} \veg\dontdisplaylinenum }%
     \lacuna{\devanagarifontsmall \vc {\englishfont After }कामरू॰, {\englishfont \msCc\ has two folios missing (ff.\ 271--272) and resumes only at 3.30b} }%
  
{\devanagarifont तत्र विप्र प्रजाः सर्वे अनन्तगुणसागराः \thinspace{\dandab} \dontdisplaylinenum }%
 
%Verse 2:22

{\devanagarifont तुल्यरूपबलाः सर्वे सूर्यायुतसमप्रभाः {॥ २:२२॥} \veg\dontdisplaylinenum }%
     \var{{\devanagarifont \numemph\vc\textbf{॰बलाः}\lem \mssALL, ॰वराः \Ed}}% 

{\devanagarifont परार्धद्वयविस्तारं परार्धद्वयमायतम् \thinspace{\dandab} \dontdisplaylinenum }%
 
%Verse 2:23

{\devanagarifont परार्धद्वयविक्षेपं योजनानां द्विजोत्तम {॥ २:२३॥} \veg\dontdisplaylinenum }%
     \var{{\devanagarifont \numemph\vc\textbf{॰द्वय॰}\lem \mssALL, ॰द्व॰ \msNaacorr\oo 
\textbf{विक्षेपं}\lem \eme, विक्षेपा \msCa\msCb\msNa\msNb\msNc, विज्ञेया \Ed}}% 
    \var{{\devanagarifont \numnoemph\vd\textbf{॰त्तम}\lem \mssALL, ॰त्तमः \msNa}}% 

{\devanagarifont ऐश्वर्यत्वं न संख्यास्ति बलशक्तिश्च भो द्विज \thinspace{\dandab} \dontdisplaylinenum }%
     \var{{\devanagarifont \numemph\vb \lem \mssALL, 
\om\ \msNaacorr, तव शक्तिश्च भो द्विज \Ed}}% 

%Verse 2:24

{\devanagarifont अधोर्ध्वो न च संख्यास्ति न तिर्यञ्चैति कश्चन {॥ २:२४॥} \veg\dontdisplaylinenum }%
     \var{{\devanagarifont \numnoemph\vc \lem \mssALL, \om\ \msNaacorr}}% 
    \var{{\devanagarifont \numnoemph\vd \lem \msNapcorr\msNc, 
न तिर्यञ्चेति कश्चन \msCa\msCb\msNb\Ed, 
न तिर्यं चेति कश्चन \msNaacorr}}% 

{\devanagarifont शिवाण्डस्य च विस्तारमायामं च न वेद्म्यहम् \thinspace{\dandab} \dontdisplaylinenum }%
 
%Verse 2:25

{\devanagarifont भोगमक्षय तत्रैव जन्ममृत्युर्न विद्यते {॥ २:२५॥} \veg\dontdisplaylinenum }%
     \var{{\devanagarifont \numemph\vc\textbf{भोगमक्षय त॰}\lem \eme, भोगमक्षयस्त॰ \msCa\msCb\msNa\msNb\msNc\ \unmetr, 
भोगमयास्तु त॰ \Ed}}% 
    \var{{\devanagarifont \numnoemph\vd\textbf{॰मृत्युर्न}\lem \mssALL, ॰मृत्यु न \msNb}}% 

{\devanagarifont शिवाण्डमध्यमाश्रित्य गोक्षीरसदृशप्रभाः \thinspace{\dandab} \dontdisplaylinenum }%
     \var{{\devanagarifont \numemph\vb\textbf{प्रभाः}\lem \mssALL, प्रभा \Ed}}% 

%Verse 2:26

{\devanagarifont परार्धपरकोटीनामीशानानां स्मृतालयः {॥ २:२६॥} \veg\dontdisplaylinenum }%
     \var{{\devanagarifont \numnoemph\vd\textbf{॰शानानां}\lem \mssALL, ॰शानाना \msNb, ॰गानानां \msNc\oo 
\textbf{स्मृतालयः}\lem \msCa\msNb\msNc, स्मृतालय \msCb, स्मृतालयं \msNa, स्मृतालया \Ed}}% 

{\devanagarifont बालसूर्यप्रभाः सर्वे ज्ञेयास्तत्पुरुषालये \thinspace{\dandab} \dontdisplaylinenum }%
     \var{{\devanagarifont \numemph\va\textbf{॰भाः}\lem \mssALL, ॰भा \Ed}}% 
    \var{{\devanagarifont \numnoemph\vb\textbf{ज्ञेयास्त॰}\lem \mssALL, ज्ञेया त॰ \msNa\Ed\oo 
\textbf{॰आलये}\lem \mssALL, ॰आलयं \Ed}}% 

%Verse 2:27

{\devanagarifont परार्धपरकोटीनां पूर्वस्यां दिशमाश्रिताः {॥ २:२७॥} \veg\dontdisplaylinenum }%
     \var{{\devanagarifont \numnoemph\vd\textbf{दिश॰}\lem \mssALL, दिशि॰ \msNb}}% 

{\devanagarifont भिन्नाञ्जनप्रभाः सर्वे दक्षिणां दिशमाश्रिताः \thinspace{\dandab} \dontdisplaylinenum }%
     \var{{\devanagarifont \numemph\va\textbf{॰प्रभाः}\lem \mssALL, ॰प्रभा \Ed}}% 
    \var{{\devanagarifont \numnoemph\vb\textbf{दक्षिणां}\lem \mssALL, दक्षिण॰ \Ed\oo 
\textbf{दिशम्}\lem \mssALL, दिशिम् \msCb\Ed}}% 

%Verse 2:28

{\devanagarifont परार्धपरकोटीनामघोरालयमाश्रिताः {॥ २:२८॥} \veg\dontdisplaylinenum }%
     \var{{\devanagarifont \numnoemph\vd\textbf{॰घोरा॰}\lem \mssALL, ॰धोरा॰ \Ed\oo 
\textbf{॰श्रिताः}\lem \mssALL, ॰श्रिता \Ed}}% 

{\devanagarifont कुन्देन्दुहिमशैलाभाः पश्चिमां दिशमाश्रिताः \thinspace{\dandab} \dontdisplaylinenum }%
     \var{{\devanagarifont \numemph\vb\textbf{पश्चिमां}\lem \mssALL, पश्चिमा \msCb\oo 
\textbf{दिश॰}\lem \mssALL, दिशि॰ \msNc\oo 
\textbf{॰श्रिताः}\lem \mssALL, ॰श्रिता \Ed}}% 

%Verse 2:29

{\devanagarifont परार्धपरकोटीनां सद्यमिष्टालयः स्मृतः {॥ २:२९॥} \veg\dontdisplaylinenum }%
     \var{{\devanagarifont \numnoemph\vd\textbf{सद्यमिष्टा॰}\lem \mssALL, सद्यमिष्ट्वा॰ \msNa\oo 
\textbf{स्मृतः}\lem \mssALL, स्मृताः \msCb}}% 

{\devanagarifont कुङ्कुमोदकसंकाशा उत्तरां दिशमाश्रिताः \thinspace{\dandab} \dontdisplaylinenum }%
     \var{{\devanagarifont \numemph\vb\textbf{उत्तरां}\lem \mssALL, उत्तरा \msCb\oo 
\textbf{दिशम्}\lem \mssALL, दिशिम् \msCa}}% 

%Verse 2:30

{\devanagarifont परार्धपरकोतीनां वामदेवालयः स्मृतः {॥ २:३०॥} \veg\dontdisplaylinenum }%
     \var{{\devanagarifont \numnoemph\vd\textbf{॰लयः}\lem \mssALL, ॰लय \msNc}}% 

{\devanagarifont ईशानस्य कलाः पञ्च वक्त्रस्यापि चतुष्कलाः \thinspace{\dandab} \dontdisplaylinenum }%
     \var{{\devanagarifont \numemph\va\textbf{कलाः}\lem \mssALL, कला \Ed}}% 
    \var{{\devanagarifont \numnoemph\vb\textbf{चतुष्कलाः}\lem \mssALL, चतुष्तके \Ed}}% 

%Verse 2:31

{\devanagarifont अघोरस्य कला अष्टौ वामदेवास्त्रयोदश {॥ २:३१॥} \veg\dontdisplaylinenum }%
     \var{{\devanagarifont \numnoemph\vd\textbf{वामदेवा॰}\lem \mssALL, वामदेव॰ \msNb}}% 

{\devanagarifont सद्यश्चाष्टौ कला ज्ञेयाः संसारार्णवतारकाः \thinspace{\dandab} \dontdisplaylinenum }%
     \var{{\devanagarifont \numemph\va\textbf{ज्ञेयाः}\lem \mssALL, ज्ञेया \Ed}}% 
    \var{{\devanagarifont \numnoemph\vb\textbf{संसारा॰}\lem \mssALL, संसा॰ \msCbacorr}}% 

%Verse 2:32

{\devanagarifont अष्टत्रिंशत्कला ह्येताः कीर्तिता द्विजसत्तम {॥ २:३२॥} \veg\dontdisplaylinenum }%
     \var{{\devanagarifont \numnoemph\vc\textbf{॰त्रिंशत्क॰}\lem \corr, ॰त्रिंशक॰ \msCa\msCb\msNa\msNb\msNc\Ed\oo 
\textbf{ह्येताः}\lem \mssALL, ज्ञेयाः \Ed}}% 
    \var{{\devanagarifont \numnoemph\vd\textbf{॰सत्तम}\lem \mssALL, ॰सत्तमः \msNb\Ed}}% 

{\devanagarifont संख्या वर्णा दिशश्चैव एकैकस्य पृथक्पृथक् \thinspace{\dandab} \dontdisplaylinenum }%
     \var{{\devanagarifont \numemph\va\textbf{संख्या वर्णा}\lem \msCb\msNc, संख्या वर्ण्णो \msCa\msNb, संख्या वण्णा \msNa, संध्या वर्णा \Ed}}% 
    \var{{\devanagarifont \numnoemph\vb\textbf{एकैकस्य}\lem \mssALL, ऐकैकस्य \msCb\msNa}}% 

%Verse 2:33

{\devanagarifont पूर्वोक्तेन विधानेन बोधव्यास्तत्त्वचिन्तकैः {॥ २:३३॥} \veg\dontdisplaylinenum }%
     \var{{\devanagarifont \numnoemph\vd\textbf{बोधव्यास्त॰}\lem \eme, बोधव्या त॰ \msCa\msCb\msNa\msNb\msNc\Ed}}% 

{\devanagarifont शिवाण्डगमनाकृष्ट्या शिवयोगं सदाभ्यसेत् \thinspace{\dandab} \dontdisplaylinenum }%
     \var{{\devanagarifont \numemph\va\textbf{॰कृष्ट्या}\lem \mssALL, कृष्टा \msNa\msNc}}% 
    \var{{\devanagarifont \numnoemph\vb\textbf{योगं सदाभ्यसेत्}\lem \mssALL, योग समभ्यसेत् \msNb}}% 

%Verse 2:34

{\devanagarifont शिवयोगं विना विप्र तत्र गन्तुं न शक्यते {॥ २:३४॥} \veg\dontdisplaylinenum }%
     \var{{\devanagarifont \numnoemph\vc\textbf{॰योगं}\lem \mssALL, ॰योग \Ed}}% 

{\devanagarifont अश्वमेधादियज्ञानां कोट्यायुतशतानि च \thinspace{\dandab} \dontdisplaylinenum }%
 
{\devanagarifont कृच्छ्रादितप सर्वाणि कृत्वा कल्पशतानि च  \danda\dontdisplaylinenum }%
     \var{{\devanagarifont \numemph\vc\textbf{॰तप}\lem \Ed, ॰तपः \msCa\msCb\msNa\msNb\msNc\ \unmetr}}% 

%Verse 2:35

{\devanagarifont तत्र गन्तुं न शक्येत देवैरपि तपोधन {॥ २:३५॥} \veg\dontdisplaylinenum }%
     \var{{\devanagarifont \numnoemph\ve\textbf{शक्येत}\lem \mssALL, शक्यैत \msCb, शक्येते \Ed}}% 
    \var{{\devanagarifont \numnoemph\vf\textbf{देवै॰}\lem \mssALL, देवे॰ \msNc\oo 
\textbf{॰धन}\lem \mssALL, ॰धनम् \msCb}}% 

{\devanagarifont गङ्गादिसर्वतीर्थेषु स्नात्वा तप्त्वा च वै पुनः \thinspace{\dandab} \dontdisplaylinenum }%
 
%Verse 2:36

{\devanagarifont तत्र गन्तुं न शक्येत ऋषिभिर्वा महात्मभिः {॥ २:३६॥} \veg\dontdisplaylinenum }%
     \var{{\devanagarifont \numemph\vc\textbf{गन्तुं}\lem \mssALL, गन्तु \msNb\msNc\oo 
\textbf{शक्येत}\lem \mssALL, शक्यन्ते \Ed}}% 

{\devanagarifont सप्तद्वीपसमुद्राणि रत्नपूर्णानि भो द्विज \thinspace{\dandab} \dontdisplaylinenum }%
     \var{{\devanagarifont \numemph\va\textbf{॰द्वीप॰}\lem \mssALL, ॰दीप॰ \msNc\oo 
\textbf{॰समुद्राणि}\lem \mssALL, ॰समुद्राय \msNb}}% 
    \paral{{\devanagarifontsmall \vab {\englishfont Cf. \SDHU\ 2.104:} त्रिः प्रदत्वा महीं पूर्णां{\englishfont ...} }}

{\devanagarifont दत्त्वा वा वेदविदुषे श्रद्धाभक्तिसमन्वितः  \danda\dontdisplaylinenum }%
 
%Verse 2:37

{\devanagarifont तत्र गन्तुं न शक्येत विना ध्यानेन निश्चयः {॥ २:३७॥} \veg\dontdisplaylinenum }%
     \var{{\devanagarifont \numnoemph\ve\textbf{गन्तुं}\lem \mssALL, गन्तु \msNb, गंन्तु \msNc\oo 
\textbf{शक्येत}\lem \mssALL, शक्यन्ते \Ed}}% 

{\devanagarifont स्वदेहान्मांसमुद्धृत्य दत्त्वार्थिभ्यश्च निश्चयात् \thinspace{\dandab} \dontdisplaylinenum }%
     \var{{\devanagarifont \numemph\va\textbf{स्वदेहान्मांस॰}\lem \mssALL, स्वदेहात्मांस॰ \msNc, स्वदेहात्मां स॰ \Ed}}% 

{\devanagarifont स्वदारपुत्रसर्वस्वं शिरो ऽर्थिभ्यश्च यो ददेत्  \danda\dontdisplaylinenum }%
     \var{{\devanagarifont \numnoemph\vc\textbf{॰स्वं}\lem \mssALL, ॰स्व \msNb}}% 

%Verse 2:38

{\devanagarifont न तत्र गन्तुं शक्येत अन्यैर्वापि सुदुष्करैः {॥ २:३८॥} \veg\dontdisplaylinenum }%
     \var{{\devanagarifont \numnoemph\ve\textbf{न तत्र गन्तुं}\lem \mssALL, न तत्र गन्तुं न \msCb}}% 
    \var{{\devanagarifont \numnoemph\vf\textbf{॰दुष्करैः}\lem \mssALL, ॰दुष्कृतः \msNb}}% 

{\devanagarifont यज्ञतीर्थतपोदानवेदाध्ययनपारगः \thinspace{\dandab} \dontdisplaylinenum }%
     \var{{\devanagarifont \numemph\va\textbf{॰दान॰}\lem \mssALL, ॰दानं \msNa, ॰दानै \msNb}}% 
    \var{{\devanagarifont \numnoemph\vb\textbf{॰पारगः}\lem \mssALL, ॰पारगाः \msCa\msNb}}% 

%Verse 2:39

{\devanagarifont ब्रह्माण्डान्तस्य भोगांस्तु भुङ्क्ते कालवशानुगः {॥ २:३९॥} \veg\dontdisplaylinenum }%
     \var{{\devanagarifont \numnoemph\vc \lem \mssALL, 
ब्रह्माण्डान्तस्य भोगास्तु \msNb, 
ब्रह्माण्डात्तस्य भोगास्तु \Ed}}% 
    \var{{\devanagarifont \numnoemph\vd\textbf{भुङ्क्ते}\lem \mssALL, \uncl{भुङ्क्ते} \msNc, भुक्त्वा \Ed\oo 
\textbf{॰गः}\lem \mssALL, ॰गाः \msNaacorr}}% 

\vfill
\pageparbreak
\vers

{\devanagarifont कालेन समप्रेष्येण धर्मो याति परिक्षयम् \thinspace{\dandab} \dontdisplaylinenum }%
     \var{{\devanagarifont \numemph\vb\textbf{धर्मो}\lem \mssALL, धर्मे \msNc}}% 

{\devanagarifont अलातचक्रवत्सर्वं कालो याति परिभ्रमन्  \danda\dontdisplaylinenum }%
 
%Verse 2:40

{\devanagarifont त्रैकाल्यकलनात्कालस्तेन कालः प्रकीर्तितः {॥ २:४०॥} \veg\dontdisplaylinenum }%
     \var{{\devanagarifont \numnoemph\ve\textbf{॰कलनात्काल॰}\lem \mssALL, ॰कलना काल॰ \msNb}}% 

{\devanagarifont 
\jump
\begin{center}
\ketdanda~इति वृषसारसंग्रहे शिवाण्डसंख्या नामाध्यायो द्वितीयः~\ketdanda
\end{center}
\dontdisplaylinenum\vers  }%
     \var{{\devanagarifont \numnoemph{\englishfont \Colo:}\textbf{नामाध्यायो द्वितीयः}\lem \mssALL, 
नामाध्याय द्वितीयः \msNb, 
नाम द्वितीयो ऽध्यायः \Ed}}% 
\bekveg\szamveg
\vfill
\phpspagebreak

\versno=0\fejno=3
\thispagestyle{empty}

\centerline{\Large\devanagarifontbold [   तृतीयो ऽध्यायः  ]}{\vrule depth10pt width0pt} \fancyhead[CO]{{\footnotesize\devanagarifont वृषसारसंग्रहे  }}
\fancyhead[CE]{{\footnotesize\devanagarifont तृतीयो ऽध्यायः  }}
\fancyhead[LE]{}
\fancyhead[RE]{}
\fancyhead[LO]{}
\fancyhead[RO]{}
\szam\bek



\alalfejezet{धर्मप्रवचनम्}
\vers


{\devanagarifont विगतराग उवाच {\dandab}\dontdisplaylinenum  }%
 
{\devanagarifont किमर्थं धर्ममित्याहुः कतिमूर्तिश्च कीर्त्यते \thinspace{\danda} \dontdisplaylinenum }%
     \var{{\devanagarifont \numemph\va\textbf{आहुः}\lem \mssALL, आहु \Ed}}% 
    \lacuna{\devanagarifontsmall {\englishfont Witnesses used for this chapter: \msParis\ exp.\thinspace 215r--215v (breaks off after 3.14d and resumes at 4.8a),
                                             \msCa\ ff.\thinspace 197r--198v, 
                                             \msCb\ ff.\thinspace 204v--206r, 
                                             \msCc\ ff.\thinspace 273r--273v (broke off at 2.21 and resumes at 3.30b),
                                             \msNa\ ff.\thinspace 4v--6r, 
                                             \msNb\ exp.\thinspace 42, 47 (upper), 48 (lower),
                                             \msNc\ ff.\thinspace 213r--214v,
                                             \Ed\ pp.\thinspace 588--591;
                                        \mssCaCbCc\ = \msCa + \msCb + \msCc } }%
  
%Verse 3:1

{\devanagarifont कतिपादवृषो ज्ञेयो गतिस्तस्य कति स्मृताः {॥ ३:१॥} \veg\dontdisplaylinenum }%
     \var{{\devanagarifont \numnoemph\vd\textbf{स्मृताः}\lem \mssALL, स्मृता \msCb, स्मृतः \Ed}}% 

{\devanagarifont कौतूहलं ममोत्पन्नं संशयं छिन्धि तत्त्वतः \thinspace{\dandab} \dontdisplaylinenum }%
     \var{{\devanagarifont \numemph\va\textbf{कौतूहलं}\lem \mssALL, कौतुहल \Ed\oo 
\textbf{ममोत्पन्नं}\lem \mssALL, समोत्पन्नं \msNc}}% 
    \var{{\devanagarifont \numnoemph\vb\textbf{संशयं}\lem \mssALL, सशयं \msCa}}% 

%Verse 3:2

{\devanagarifont कस्य पुत्रो मुनिश्रेष्ठ प्रजास्तस्य कति स्मृताः {॥ ३:२॥} \veg\dontdisplaylinenum }%
 
{\devanagarifont अनर्थयज्ञ उवाच {\dandab}\dontdisplaylinenum  }%
 
{\devanagarifont धृतिरित्येष धातुर्वै पर्यायः परिकीर्तितः \thinspace{\danda} \dontdisplaylinenum }%
 
%Verse 3:3

{\devanagarifont आधारणान्महत्त्वाच्च धर्म इत्यभिधीयते {॥ ३:३॥} \veg\dontdisplaylinenum  }%
     \var{{\devanagarifont \numemph\vc\textbf{आधारणान्म॰}\lem \msParis\msCa\msNb, आधारणात्प॰ \msCb, आधारणात्म॰ \msNa\msNc, आधारेण म॰ \Ed}}% 
    \var{{\devanagarifont \numnoemph\vd\textbf{इत्यभिधीयते}\lem \msCa\msNa\msNc\Ed, इ\uncl{त्यभिधीयते} \msParis, 
इत्यविधीयते \msCb\msNb}}% 
    \paral{{\devanagarifontsmall \vcd {\englishfont \compare\ \LINPU\ 1.10.12cd--13ab:}
                         धारणार्थे महान्ह्येष धर्मशब्दः प्रकीर्तितः\thinspace{\devanagarifontsmall ॥}
                         अधारणे ऽमहत्त्वे च अधर्म इति चोच्यते\thinspace{\devanagarifontsmall ।}
                \vo\ {\englishfont \compare\ \BRAHMANDAPUR\ 1.32.29:}
                         धारणार्थो धृतिश्चैव धातुः शब्दे प्रकीर्तितः\thinspace{\devanagarifontsmall ।}
                         अधारणामहत्त्वे च अधर्म इति चोच्यते\thinspace{\devanagarifontsmall ॥}
                     {\englishfont \compare\ \VAYUP\ 1.59.28:}
                         धारणा धृतिरित्यर्थाद्धातोर्धर्मः प्रकीर्तितः\thinspace{\devanagarifontsmall ।}
                         अधारणे ऽमहत्त्वे च अधर्म इति चोच्यते\thinspace{\devanagarifontsmall ॥}
                     {\englishfont \compare\ \MATSP\ 145.27:}  
                         धर्मेति धारणे धातुर्महत्वे चैव उच्यते\thinspace{\devanagarifontsmall ।}
                         आधारणे महत्त्वे वा धर्मः स तु निरुच्यते\thinspace{\devanagarifontsmall ।} }}

{\devanagarifont श्रुतिस्मृतिद्वयोर्मूर्तिश्चतुष्पादवृषः स्थितः \thinspace{\dandab} \dontdisplaylinenum }%
     \var{{\devanagarifont \numemph\vab\textbf{॰स्मृतिद्वयोर्मूर्तिश्च॰}\lem \msCa, ॰स्मृतिद्वयो मूर्त्तिश्च॰ \msParis\msCb\msNb, 
॰स्मृतिद्वयो मूर्त्ति च॰ \msNa\msNc, 
॰स्मृतिर्द्वयो मूर्तिश्च \Ed}}% 
    \var{{\devanagarifont \numnoemph\vb\textbf{॰वृषः}\lem \mssALL, ॰वृष \msNc}}% 

%Verse 3:4

{\devanagarifont चतुराश्रम यो धर्मः कीर्तितानि मनीषिभिः {॥ ३:४॥} \veg\dontdisplaylinenum }%
     \var{{\devanagarifont \numnoemph\vc\textbf{चतुरा॰}\lem \mssALL, चातुरा॰ \msCa\msNc}}% 
    \paral{{\devanagarifontsmall \vo {\englishfont \compare\ 4.74 below:}
                 चतुष्पादः स्मृतो धर्मश्चतुराश्रममाश्रितः\thinspace{\devanagarifontsmall ।}
                 गृहस्थो ब्रह्मचारी च वानप्रस्थो ऽथ भैक्षुकः\thinspace{\devanagarifontsmall ॥} }}

{\devanagarifont गतिश्च पञ्च विज्ञेयाः शृणु धर्मस्य भो द्विज \thinspace{\dandab} \dontdisplaylinenum }%
     \var{{\devanagarifont \numemph\va\textbf{विज्ञेयाः}\lem \eme, विज्ञेयः \msParis\msCa\msNa\msNb\msNc\Ed, \om\ \msCb}}% 
    \lacuna{\devanagarifontsmall \vab {\englishfont \msCb\ reads here } गतिश्च पौत्राश्च अनेकाश्च बभूव ह,
                        {\englishfont skipping to 3.7cd, omitting 3.5--7ab.} }%
  
%Verse 3:5

{\devanagarifont देवमानुषतिर्यं च नरकस्थावरादयः {॥ ३:५॥} \veg\dontdisplaylinenum }%
     \var{{\devanagarifont \numnoemph\vc\textbf{॰मानुष॰}\lem \mssALL, ॰मानुषि॰ \msParis}}% 

{\devanagarifont ब्रह्मणो हृदयं भित्त्वा जातो धर्मः सनातनः \thinspace{\dandab} \dontdisplaylinenum }%
     \var{{\devanagarifont \numemph\va\textbf{ब्रह्मणो}\lem \mssALL, \om\ \msCb, ब्राह्मणो \Ed\oo 
\textbf{भित्त्वा}\lem \mssALL, वित्त्वा \msNb}}% 
    \var{{\devanagarifont \numnoemph\vb\textbf{धर्मः}\lem \mssALL, धर्म \msNb}}% 
    \paral{{\devanagarifontsmall \vab {\englishfont \compare\ \DEVIP\ 4.59cd:} ब्रह्मणो हृदयाज्जातः पुत्रो धर्म इति स्मृतः \oo 
                     {\englishfont \compare\ also \MBH\ 1.60.40ab:} ब्रह्मणो हृदयं भित्त्वा निःसृतो भगवान्भृगुः }}

%Verse 3:6

{\devanagarifont तस्य पत्नी महाभागा त्रयोदश सुमध्यमाः {॥ ३:६॥} \veg\dontdisplaylinenum }%
     \var{{\devanagarifont \numnoemph\vd\textbf{॰मध्यमाः}\lem \mssALL, \om\ \msCb}}% 

{\devanagarifont दक्षकन्या विशालाक्षी श्रद्धाद्या सुमनोहराः \thinspace{\dandab} \dontdisplaylinenum }%
     \var{{\devanagarifont \numemph\va\textbf{॰आक्षी}\lem \mssALL, \om\ \msCb, ॰आक्षि \Ed}}% 
    \var{{\devanagarifont \numnoemph\vb\textbf{॰आद्या}\lem ॰आद्या \msParis\msNb\msNc\Ed, ॰आढ्या \msCa, \om\ \msCb, ॰आढ्याः \msNa\oo 
\textbf{॰हराः}\lem \msNb\Ed, ॰हरा \msParis\msCa\msNc,  \om\ \msCb, ॰\lk \uncl{माः} \msNa}}% 

{\devanagarifont तस्य पुत्राश्च पौत्राश्च अनेकाश्च बभूव ह  \danda\dontdisplaylinenum }%
     \var{{\devanagarifont \numnoemph\vcd \lem \msParis\msCa\msNb, 
गतिश्च पौत्राश्च अनेकाश्च बभूव ह {\englishfont (eyeskip to 3.5a)} \msCb, 
तस्य पुत्राश्च योत्राश्च अनेकाश्च बभूव ह \msNa\msNc, 
तस्य पुत्रा अनेकाश्च तथा पौत्रा बभूवहः \Ed}}% 

%Verse 3:7

{\devanagarifont एष धर्मनिसर्गो ऽयं किं भूयः श्रोतुमिच्छसि {॥ ३:७॥} \veg\dontdisplaylinenum }%
 
{\devanagarifont विगतराग उवाच {\dandab}\dontdisplaylinenum  }%
     \var{{\devanagarifont \numemph\vo\textbf{विगतराग उवाच}\lem \msCb\msNapcorr\msNc\Ed, विगतराग उ \msParis\msCa\msNb, \om\ \msNaacorr}}% 

{\devanagarifont धर्मपत्नी विशेषेण पुत्रस्तेभ्यः पृथक्पृथक् \thinspace{\danda} \dontdisplaylinenum }%
 
%Verse 3:8

{\devanagarifont श्रोतुमिच्छामि तत्त्वेन कथयस्व तपोधन {॥ ३:८॥} \veg\dontdisplaylinenum }%
 
\vfill
\pageparbreak
\vers

{\devanagarifont अनर्थयज्ञ उवाच {\dandab}\dontdisplaylinenum  }%
 
{\devanagarifont श्रद्धा लक्ष्मीर्धृतिस्तुष्टिः पुष्टिर्मेधा क्रिया लज्जा \thinspace{\danda} \dontdisplaylinenum }%
     \var{{\devanagarifont \numemph\va\textbf{लक्ष्मीर्धृतिस्तुष्टिः}\lem \msCa, 
लक्ष्मी धृतिस्तुष्टिः \msParis\msNc, 
लक्ष्मीर्धृतिस्तुष् \msCb, 
लक्ष्मी द्धृतिर्द्धृतिस्तुष्टिः \msNaacorr, 
लक्ष्मीर्द्धृतिस्तुष्टिः \msNapcorr, 
लक्ष्मीं धृति तुष्टिः \msNb, 
लक्ष्मी धृतिस्तुष्टी \Ed}}% 
    \var{{\devanagarifont \numnoemph\vb\textbf{पुष्टिर्मे॰}\lem \mssALL, पुष्टि मे॰ \Ed\oo 
\textbf{लज्जा}\lem \mssALL, लजा \msNa}}% 

%Verse 3:9

{\devanagarifont बुद्धिः शान्तिर्वपुः कीर्तिः सिद्धिः प्रसूतिसम्भवाः {॥ ३:९॥} \veg\dontdisplaylinenum }%
     \var{{\devanagarifont \numnoemph\vc\textbf{बुद्धिः}\lem \mssALL, बुद्धि \msCa}}% 
    \var{{\devanagarifont \numnoemph\vd \lem \conj, सिद्धिश्चाभूतिसम्भवाः \msParis, 
सिद्धिश्चाभूतिसम्भवा \msCa\msNa\msNb\msNc, 
सिद्धिश्चातिसम्भवा \msCb, सिद्धिश्च भूतिसम्भवा \Ed}}% 

{\devanagarifont श्रद्धा कामः सुतो जातो दर्पो लक्ष्मीसुतः स्मृतः \thinspace{\dandab} \dontdisplaylinenum }%
     \var{{\devanagarifont \numemph\va\textbf{कामः}\lem \msNa, काम॰ \msParis\msCa\msCb\msNb\msNc, धर्म॰ \Ed}}% 

%Verse 3:10

{\devanagarifont धृत्यास्तु नियमः पुत्रः संतोषस्तुष्टिजः स्मृतः {॥ ३:१०॥} \veg\dontdisplaylinenum }%
     \paral{{\devanagarifontsmall \vo {\englishfont See a passage similar to \VSS\ 3.10--13,
         e.g., in \KURMP\ 1.8.20 ff.:}
         श्रद्धाया आत्मजः कामो दर्पो लक्ष्मीसुतः स्मृतः\thinspace{\devanagarifontsmall ।}
         धृत्यास्तु नियमः पुत्रस्तुष्ट्याः संतोष उच्यते\thinspace{\devanagarifontsmall ॥} 
         पुष्ट्या लाभः सुतश्चापि मेधापुत्रः श्रुतस्तथा\thinspace{\devanagarifontsmall ।} 
         क्रियायाश्चाभवत्पुत्रो दण्डः समय एव च\thinspace{\devanagarifontsmall ॥}  
         बुद्ध्या बोधः सुतस्तद्वदप्रमादो व्यजायत\thinspace{\devanagarifontsmall ।} 
         लज्जाया विनयः पुत्रो वपुषो व्यवसायकः\thinspace{\devanagarifontsmall ॥}  
         क्षेमः शान्तिसुतश्चापि सुखं सिद्धिरजायत\thinspace{\devanagarifontsmall ।}
         यशः कीर्तिसुतस्तद्वदित्येते धर्मसूनवः\thinspace{\devanagarifontsmall ॥}   
         कामस्य हर्षः पुत्रो ऽभूद्देवानन्दो व्यजायत\thinspace{\devanagarifontsmall ।}  
         इत्येष वै सुखोदर्कः सर्गो धर्मस्य कीर्तितः\thinspace{\devanagarifontsmall ॥} }}

{\devanagarifont पुष्ट्या लाभः सुतो जातो मेधापुत्रः श्रुतस्तथा \thinspace{\dandab} \dontdisplaylinenum }%
     \var{{\devanagarifont \numemph\va\textbf{लाभः}\lem \mssALL, लाभ॰ \msNa\Ed\oo 
\textbf{जातो}\lem \mssALL, \om\ \msParis}}% 
    \var{{\devanagarifont \numnoemph\vb\textbf{॰पुत्रः}\lem \eme, ॰पुत्र \msParis\msCa\msCb\msNa\msNb\msNc\Ed\oo 
\textbf{श्रुत॰}\lem \mssALL, श्रुति॰ \msParis, श्रत॰ \msCb}}% 

%Verse 3:11

{\devanagarifont क्रियायास्त्वभवत्पुत्रो दण्डः समय एव च {॥ ३:११॥} \veg\dontdisplaylinenum }%
     \var{{\devanagarifont \numnoemph\vc\textbf{त्वभवत्पुत्रो}\lem \eme, त्वभयः पुत्रो \msParis\msCa\msCb\msNa\msNb\msNc, तूभयः पुत्रौ \Ed}}% 
    \var{{\devanagarifont \numnoemph\vd\textbf{दण्डः}\lem \corr, दण्डे \msCa\msNaacorr दण्ड॰ \msParis\msNapcorr\msNb\msNc\Ed, दण्डो \msCb\oo 
\textbf{च}\lem \mssALL, तु \Ed}}% 
    \paral{{\devanagarifontsmall \vcd {\englishfont \similar\ \LINPU\ 1.70.295ab:}क्रियायामभवत्पुत्रो दण्डः समय एव च;
                     {\englishfont \similar\ \KURMP\ 1.8.22cd:   }क्रियायाश्चाभवत्पुत्रो दण्डः समय एव च;
                     {\englishfont \compare\ \LINPU\ 1,5.37:     }धर्मस्य वै क्रियायां तु दण्डः समय एव च }}

{\devanagarifont लज्जाया विनयः पुत्रो बुद्ध्या बोधःसुतः स्मृतः \thinspace{\dandab} \dontdisplaylinenum }%
     \var{{\devanagarifont \numemph\va\textbf{लज्जाया विनयः}\lem \mssALL, लज्जायाः विनय॰ \Ed}}% 
    \var{{\devanagarifont \numnoemph\vb\textbf{सुतः स्मृतः}\lem \mssALL, सुतः \lk\lk\ \msCa, सुतःस्तथा \msCb}}% 

%Verse 3:12

{\devanagarifont लज्जायाः सुधियः पुत्र अप्रमादश्च तावुभौ {॥ ३:१२॥} \veg\dontdisplaylinenum }%
     \var{{\devanagarifont \numnoemph\vc\textbf{सुधियः}\lem \Ed, सुधिय \msParis\msCa\msCb\msNa\msNb\msNc\oo 
\textbf{पुत्र}\lem \mssALL, पुत्रः \Ed}}% 
    \var{{\devanagarifont \numnoemph\vd\textbf{अप्रमाद॰}\lem \mssALL, अप्रमादा॰ \msNa}}% 

\vfill
\pageparbreak
\vers

{\devanagarifont क्षेमः शान्तिसुतो विन्द्याद्व्यवसायो वपोः सुतः \thinspace{\dandab} \dontdisplaylinenum }%
     \var{{\devanagarifont \numemph\vb\textbf{वपोः}\lem \mssALL, वपो \msNa}}% 

{\devanagarifont यशः कीर्तिसुतो ज्ञेयः सुखं सिद्धेर्व्यजायत  \danda\dontdisplaylinenum }%
     \var{{\devanagarifont \numnoemph\vd\textbf{सिद्धे॰}\lem \msParis\msCb\msNa\msNb, सिद्धि \msCa\msNc\Ed\oo 
\textbf{व्यजायत}\lem \msParis\msCa\msCb\msNa, व्यजायते \msNb\Ed, व्यजायतः \msNc}}% 

%Verse 3:13

{\devanagarifont स्वायम्भुवे ऽन्तरे त्वासन्कीर्तिता धर्मसूनवः {॥ ३:१३॥} \veg\dontdisplaylinenum }%
     \var{{\devanagarifont \numnoemph\ve\textbf{स्वायम्भुवे}\lem \msParis\msCa\msNa\msNc, स्वायम्भुवो \msCb, स्वयम्भुवे \msNb\Ed\oo 
\textbf{ऽन्तरे त्वासन्}\lem \conj, ऽन्तरे त्वासि \msParis\msCa\msCb\msNa, 
ऽन्तरे त्वासीत् \msNb, ऽन्तरे त्वासं \msNc, ऽन्तरेवासि \Ed}}% 

{\devanagarifont विगतराग उवाच {\dandab}\dontdisplaylinenum  }%
 
{\devanagarifont मूर्तिद्वयं कथं धर्मं कथयस्व तपोधन \thinspace{\danda} \dontdisplaylinenum }%
     \var{{\devanagarifont \numemph\va\textbf{धर्मं}\lem \mssALL, द्धर्म \msNc, धर्मः \Ed}}% 

%Verse 3:14

{\devanagarifont कौतूहलमतीवं मे कर्तय ज्ञानसंशयम् {॥ ३:१४॥} \veg\dontdisplaylinenum }%
     \var{{\devanagarifont \numnoemph\vc\textbf{कौतूहल॰}\lem \mssALL, कोतूहल॰ \msCb\oo 
\textbf{॰तीवं मे}\lem \mssALL, ॰तीव मे \msCb}}% 
    \var{{\devanagarifont \numnoemph\vd\textbf{कर्तय}\lem \eme, कीर्तय \msCa\msCb\msNa\msNb\msNc\Ed\oo 
\textbf{॰संशयम्}\lem \mssALL, ॰संशयः \msCb\msNb}}% 
    \lacuna{\devanagarifontsmall \vc {\englishfont In \msParis, folio 215v ends with} कौतूहलमती {\englishfont and the next available 
                      folio side (217r) starts with} त्यमिष्टगतिः प्रोक्तं {\englishfont  in 4.8a. Thus one folio (f. 216), 
                      containing 3.14d--4.7, is missing.} }%
  
{\devanagarifont अनर्थयज्ञ उवाच {\dandab}\dontdisplaylinenum  }%
 
{\devanagarifont श्रुतिस्मृतिद्वयोर्मूर्तिर्धर्मस्य परिकीर्तिता \thinspace{\danda} \dontdisplaylinenum }%
     \var{{\devanagarifont \numemph\va\textbf{श्रुति॰}\lem \mssALL, श्रुतिः \msCb\Ed}}% 
    \var{{\devanagarifont \numnoemph\vab\textbf{॰द्वयोर्मूर्तिर्ध॰}\lem \msCa, ॰द्वयो मूर्ति ध॰ \msCb\msNa\msNb, ॰द्वयी मूर्ति ध॰ \msNc, 
॰द्वयोर्मूर्ति ध॰ \Ed}}% 
    \var{{\devanagarifont \numnoemph\vb\textbf{॰कीर्तिता}\lem \mssALL, ॰कीर्त्तितः \msNb, कीर्त्तिताः \msNc}}% 

{\devanagarifont दाराग्निहोत्रसम्बन्ध इज्या श्रौतस्य लक्षणम्  \danda\dontdisplaylinenum }%
     \var{{\devanagarifont \numnoemph\vcd\textbf{॰बन्ध इ॰}\lem \eme, ॰बद्ध इ॰ \msCa\msCb\msNa\msNc, ॰बन्ध इ॰ \msNb\Ed}}% 
    \var{{\devanagarifont \numnoemph\vd\textbf{श्रौतस्य}\lem \eme, श्रोतस्य \msCa\msCb\msNc, श्रौत्रस्य \msNa, स्रोत्रस्य \msNb, श्रुतस्य \Ed}}% 
    \paral{{\devanagarifontsmall \vcd {\englishfont \compare\ \MANU\ 3.171ab:}दाराग्निहोत्रसंयोगं कुरुते यो ऽग्रजे स्थिते; 
                         {\englishfont and also \MATSP\ 142.41:} 
                         दाराग्निहोत्रसम्बन्धमृग्यजुःसामसंहिताः\thinspace{\devanagarifontsmall ।}
                         इत्यादिबहुलं श्रौतं धर्मं सप्तर्षयो ऽब्रुवन्\thinspace{\devanagarifontsmall ॥} }}

%Verse 3:15

{\devanagarifont स्मार्तो वर्णाश्रमाचारो यमैश्च नियमैर्युतः {॥ ३:१५॥} \veg\dontdisplaylinenum }%
     \var{{\devanagarifont \numnoemph\ve\textbf{स्मार्तो}\lem \eme, स्मार्त \msCa\msCb\msNa\msNb\msNc\Ed}}% 
    \paral{{\devanagarifontsmall \vcdef\ {\englishfont  \similar\ \MBH\ Suppl. 1.36.10: 
                                 }दानाग्निहोत्रमिज्या च श्रौतस्यैतद्धि लक्षणम्\thinspace{\devanagarifontsmall ।}
                                 स्मार्तो वर्णाश्रमाचारो यमैश्च नियमैर्युतः\thinspace{\devanagarifontsmall ॥}
                          \similar\ {\englishfont \MATSP\ 145.30cd--31ab:
                                 }दाराग्निहोत्रसम्बन्धमिज्या श्रौतस्य लक्षणम्\thinspace{\devanagarifontsmall ।}
                                 स्मार्तो वर्णाश्रमाचारो यमैश्च नियमैर्युतः\thinspace{\devanagarifontsmall ॥}
                          \similar\ {\englishfont \BRAHMANDAPUR\ 1.32.33cd--34ab:}
                                 दाराग्निहोत्रसम्बन्धाद् द्विधा श्रौतस्य लक्षणम्\thinspace{\devanagarifontsmall ।}
                                 स्मार्तो वर्णाश्रमाचारैर्यमैः स नियमैः स्मृतः\thinspace{\devanagarifontsmall ॥} }}


\alalfejezet{यमनियमभेदः}
{\devanagarifont यमश्च नियमश्चैव द्वयोर्भेदमतः शृणु \thinspace{\dandab} \dontdisplaylinenum }%
     \var{{\devanagarifont \numemph\va\textbf{नियम॰}\lem \mssALL, नियमै॰ \msNa}}% 

{\devanagarifont अहिंसा सत्यमस्तेयमानृशंस्यं दमो घृणा  \danda\dontdisplaylinenum }%
     \var{{\devanagarifont \numnoemph\vd\textbf{॰मानृशंस्यं}\lem \eme, ॰मनृशंस्यो \msCa\msCb\msNa\msNb\Ed, ॰मानृशंस्या \msNc}}% 
    \paral{{\devanagarifontsmall \vcd {\englishfont \similar\ \MBH\ 12.8.17ab:} अहिंसा सत्यवचनमानृशंस्यं दमो घृणा
                 \vo {\englishfont \similar\ \VDHU\ 3.233.203: 
                         }आनृशंस्यं क्षमा सत्यमहिंसा च दमः स्पृहा\thinspace{\devanagarifontsmall ।}
                         ध्यानं प्रसादो माधुर्यं चार्जवं च यमा दश\thinspace{\devanagarifontsmall ॥} }}

%Verse 3:16

{\devanagarifont धन्याप्रमादो माधुर्यमार्जवं च यमा दश {॥ ३:१६॥} \veg\dontdisplaylinenum }%
     \var{{\devanagarifont \numnoemph\ve\textbf{धन्या॰}\lem \Ed, धन्यः \msCa\msCb\msNb\msNc, ध्यन्यं \msNa\oo 
\textbf{माधुर्य॰}\lem \Ed, माधूर्य॰ \msCa\msCb\msNa\msNb\msNc}}% 
    \var{{\devanagarifont \numnoemph\vf\textbf{आर्जवं च}\lem \mssALL, आर्जवश्च \Ed}}% 

{\devanagarifont एकैकस्य पुनः पञ्चभेदमाहुर्मनीषिणः \thinspace{\dandab} \dontdisplaylinenum }%
     \var{{\devanagarifont \numemph\vb\textbf{॰माहुर्म॰}\lem \mssALL, ॰माहु म॰ \msNc}}% 

%Verse 3:17

{\devanagarifont अहिंसादि प्रवक्ष्यामि शृणुष्वावहितो द्विज {॥ ३:१७॥} \veg\dontdisplaylinenum }%
     \var{{\devanagarifont \numnoemph\vd\textbf{शृणुष्वा॰}\lem \mssALL, शृणुष्व॰ \msNa\msNb}}% 


\alalfejezet{यमेष्वहिंसा (१)}

\alalalfejezet{पञ्चविधा हिंसा}

{\devanagarifont त्रासनं ताडनं बन्धो मारणं वृत्तिनाशनम् \thinspace{\dandab} \dontdisplaylinenum }%
     \var{{\devanagarifont \numemph\va\textbf{बन्धो}\lem \mssALL, बद्धो \msNb, बन्ध \Ed}}% 

%Verse 3:18

{\devanagarifont हिंसां पञ्चविधामाहुर्मुनयस्तत्त्वदर्शिनः {॥ ३:१८॥} \veg\dontdisplaylinenum }%
     \var{{\devanagarifont \numnoemph\vc\textbf{हिंसां}\lem \msCa\msNa\msNc, हिंसा \msCb\msNb\Ed\oo 
\textbf{॰विधामाहु॰}\lem \msCb\msNa\msNc, ॰विधमाहु॰ \msCa, 
॰विधान्याहु॰ \msNb, ॰विध प्राहु॰ \Ed}}% 

{\devanagarifont काष्ठलोष्टकशाद्यैस्तु ताडयन्तीह निर्दयाः \thinspace{\dandab} \dontdisplaylinenum }%
     \var{{\devanagarifont \numemph\va\textbf{काष्ठलोष्ट॰}\lem \mssALL, का\uncl{ष्ठ}\lac\  \msNb}}% 
    \var{{\devanagarifont \numnoemph\vb\textbf{निर्दयाः}\lem \mssALL, निर्दया \Ed}}% 

%Verse 3:19

{\devanagarifont तत्प्रहारविभिन्नाङ्गो मृतवध्यमवाप्नुयात् {॥ ३:१९॥} \veg\dontdisplaylinenum }%
     \var{{\devanagarifont \numnoemph\vc\textbf{॰भिन्नाङ्गो}\lem \mssALL, ॰भिन्नाङ्गा \Ed}}% 
    \var{{\devanagarifont \numnoemph\vd\textbf{॰वध्यमवा॰}\lem \mssALL, ॰वध्यववा॰ \msCa}}% 

{\devanagarifont बद्ध्वा पादौ भुजोरश्च शिरोरुक्कण्ठपाशिताः \thinspace{\dandab} \dontdisplaylinenum }%
     \var{{\devanagarifont \numemph\va\textbf{भुजोरश्च}\lem \mssALL, भुजौरश्च \msNa\Ed}}% 
    \var{{\devanagarifont \numnoemph\vb\textbf{शिरोरुक्कण्ठ॰}\lem \eme, शिरोरुकण्ठ॰ \msCa\msCb\msNa\msNb\msNc, शिरोरुः कण्ठ॰ \Ed}}% 

%Verse 3:20

{\devanagarifont अनाहता म्रियन्त्येवं वधो बन्धनजः स्मृतः {॥ ३:२०॥} \veg\dontdisplaylinenum }%
     \var{{\devanagarifont \numnoemph\vc \lem \mssALL, अनाहत म्रियंत्येष \msNb}}% 
    \var{{\devanagarifont \numnoemph\vd\textbf{॰नजः स्मृतः}\lem \conj, ॰नजाः स्मृताः \msCa\msCb\msNa\msNb, 
॰नजाः स्मृता \msNc, ॰नज स्मृतः \Ed}}% 

\vfill
\pageparbreak
\vers

{\devanagarifont शत्रुचौरभयैर्घोरैः सिंहव्याघ्रगजोरगैः \thinspace{\dandab} \dontdisplaylinenum }%
     \var{{\devanagarifont \numemph\va\textbf{॰चौरभयैर्घोरैः}\lem \mssALL, ॰चोरभयै घोरै \msNb}}% 

%Verse 3:21

{\devanagarifont त्रासनाद्वधमाप्नोति अन्यैर्वापि सुदुःसहैः {॥ ३:२१॥} \veg\dontdisplaylinenum }%
     \var{{\devanagarifont \numnoemph\vd\textbf{अन्यैर्वापि}\lem \mssALL, अन्ये चापि \msNc}}% 

{\devanagarifont यस्य यस्य हरेद्वित्तं तस्य तस्य वधः स्मृतः \thinspace{\dandab} \dontdisplaylinenum }%
     \var{{\devanagarifont \numemph\va\textbf{हरेद्वि॰}\lem \mssALL, हरे वि॰ \msNb}}% 
    \var{{\devanagarifont \numnoemph\vb\textbf{वधः}\lem \mssALL, वध \Ed}}% 

%Verse 3:22

{\devanagarifont वृत्तिजीवाभिभूतानां तद्द्वारा निहतः स्मृतः {॥ ३:२२॥} \veg\dontdisplaylinenum }%
     \var{{\devanagarifont \numnoemph\vc\textbf{॰भिभूतानां}\lem \mssALL, ॰विभूतानां \msNb}}% 
    \var{{\devanagarifont \numnoemph\vd\textbf{तद्द्वारा नि॰}\lem \conj, तद्वारान्नि॰ \msCa\msCb\msNa\msNb\msNc, तद्द्वारान्नि॰ \Ed}}% 

{\devanagarifont विषवह्निशरशस्त्रैर्मायायोगबलेन वा \thinspace{\dandab} \dontdisplaylinenum }%
     \var{{\devanagarifont \numemph\vab\textbf{॰शस्त्रैर्माया॰}\lem \mssALL, ॰शस्त्रै मा॰ \msNc, ॰शस्त्रैर्म्मया॰ \Ed}}% 

%Verse 3:23

{\devanagarifont हिंसकान्याहु विप्रेन्द्र मुनयस्तत्त्वदर्शिनः {॥ ३:२३॥} \veg\dontdisplaylinenum }%
     \var{{\devanagarifont \numnoemph\vc\textbf{हिंसकान्याहु वि॰}\lem \msCb\msNb\msNc, 
हिंसकान्याहुर्वि॰ \msCa\msNa\ \unmetr, हिंसकेत्याहु वि॰ \Ed}}% 


\alalalfejezet{अहिंसाप्रशंसा}

{\devanagarifont अहिंसा परमं धर्मं यस्त्यजेत्स दुरात्मवान् \thinspace{\dandab} \dontdisplaylinenum }%
     \var{{\devanagarifont \numemph\va\textbf{परमं धर्मं}\lem \mssALL, परमं धर्म \msNb, परमो धर्मं \msNc}}% 
    \var{{\devanagarifont \numnoemph\vb\textbf{त्यजेत्स दुरात्मवान्}\lem \msCb\msNc\Ed, त्यजेच्छ दुरात्म\lk\ \msCa, त्यजेत्सुदुरात्मवान् \msNa, 
त्यजेत्स दुरात्मनम् \msNb}}% 

%Verse 3:24

{\devanagarifont क्लेशायासविनिर्मुक्तं सर्वधर्मफलप्रदम् {॥ ३:२४॥} \veg\dontdisplaylinenum }%
 
{\devanagarifont नातः परतरो मूर्खो नातः परतरं तमः \thinspace{\dandab} \dontdisplaylinenum }%
     \var{{\devanagarifont \numemph\vb\textbf{॰तरं}\lem \mssALL, ॰तन् \msCbacorr\Ed}}% 

%Verse 3:25

{\devanagarifont नातः परतरं दुःखं नातः परतरो ऽयशः {॥ ३:२५॥} \veg\dontdisplaylinenum }%
 
{\devanagarifont नातः परतरं पापं नातः परतरं विषम् \thinspace{\dandab} \dontdisplaylinenum }%
 
%Verse 3:26

{\devanagarifont नातः परतराविद्या नातः परं तपोधन {॥ ३:२६॥} \veg\dontdisplaylinenum }%
     \var{{\devanagarifont \numemph\vd\textbf{परं तपोधन}\lem \mssALL, पर तपोद्यमाः \Ed}}% 

{\devanagarifont यो हिनस्ति न भूतानि उद्भिज्जादि चतुर्विधम् \thinspace{\dandab} \dontdisplaylinenum }%
     \var{{\devanagarifont \numemph\va\textbf{यो हिनस्ति न}\lem \mssALL, यो न हिन्सन्ति \msNb, यो हि नास्ति न \Ed}}% 
    \var{{\devanagarifont \numnoemph\vb\textbf{उद्भिज्जादि}\lem \eme, उद्भिजादि \msCa\msCb\msNb\msNc\Ed, उद्भिजानि \msNa\oo 
\textbf{॰विधम्}\lem \mssALL, ॰विधिं \msNc}}% 

%Verse 3:27

{\devanagarifont स भवेत्पुरुषः श्रेष्ठः सर्वभूतदयान्वितः {॥ ३:२७॥} \veg\dontdisplaylinenum }%
     \var{{\devanagarifont \numnoemph\vc\textbf{पुरुषः}\lem \mssALL, पुरुष॰ \Ed}}% 

\vfill
\pageparbreak
\vers

{\devanagarifont सर्वभूतदयां नित्यं यः करोति स पण्डितः \thinspace{\dandab} \dontdisplaylinenum }%
     \var{{\devanagarifont \numemph\va\textbf{॰दयां नित्यं}\lem \msCa\msNa\Ed, ॰दया नित्यं \msCb\msNb, ॰दया नित्य \msNc}}% 

%Verse 3:28

{\devanagarifont स यज्वा स तपस्वी च स दाता स दृढव्रतः {॥ ३:२८॥} \veg\dontdisplaylinenum }%
     \var{{\devanagarifont \numnoemph\vc\textbf{यज्वा}\lem \mssALL, यज्या \msNb}}% 

{\devanagarifont अहिंसा परमं तीर्थमहिंसा परमं तपः \thinspace{\dandab} \dontdisplaylinenum }%
     \var{{\devanagarifont \numemph\va\textbf{परमं ती॰}\lem \mssALL, परन्ती॰ \msCb}}% 

%Verse 3:29

{\devanagarifont अहिंसा परमं दानमहिंसा परमं सुखम् {॥ ३:२९॥} \veg\dontdisplaylinenum }%
     \paral{{\devanagarifontsmall \vo {\englishfont This and the following verses are similar to MBh 13.117.37--38} }}
    \lacuna{\devanagarifontsmall \vd {\englishfont \msCc\ resumes here in exp.\ 189, f. 273r (sic!) with }रमं सुखम् }%
  
{\devanagarifont अहिंसा परमो यज्ञः अहिंसा परमं व्रतम् \thinspace{\dandab} \dontdisplaylinenum }%
     \var{{\devanagarifont \numemph\va\textbf{यज्ञः}\lem \msCb\msCc\msNb\Ed, यज्ञर् \msCa, यज्ञ \msNa\msNc}}% 

%Verse 3:30

{\devanagarifont अहिंसा परमं ज्ञानमहिंसा परमा क्रिया {॥ ३:३०॥} \veg\dontdisplaylinenum }%
     \var{{\devanagarifont \numnoemph\vc\textbf{परमं}\lem \mssALL, परमो \Ed}}% 
    \var{{\devanagarifont \numnoemph\vd\textbf{परमा}\lem \mssALL, परमां \msNb}}% 

{\devanagarifont अहिंसा परमं शौचमहिंसा परमो दमः \thinspace{\dandab} \dontdisplaylinenum }%
     \var{{\devanagarifont \numemph\vab\textbf{(अहिंसा{\englishfont ...} दमः)}\lem \mssALL, \om\ \Ed}}% 

%Verse 3:31

{\devanagarifont अहिंसा परमो लाभः अहिंसा परमं यशः {॥ ३:३१॥} \veg\dontdisplaylinenum }%
     \var{{\devanagarifont \numnoemph\vc\textbf{लाभः}\lem \msNc, लाभ \msCa\msCb\msNa\msNb\Ed, लाभो \msCc}}% 
    \var{{\devanagarifont \numnoemph\vd\textbf{परमं}\lem \mssALL, परमा \msNa}}% 
    \lacuna{\devanagarifontsmall \vcd {\englishfont After pādas cd, \Ed\ inserts this: }अहिंसा परमा कीर्ति अहिंसा परमो दमः,
                 {\englishfont which is not to be found in \mssCaCbCc\msNa\msNb\msNc\ (or in \msPaperA)} }%
  
{\devanagarifont अहिंसा परमो धर्मः अहिंसा परमा गतिः \thinspace{\dandab} \dontdisplaylinenum }%
     \var{{\devanagarifont \numemph\va\textbf{धर्मः}\lem \msNa\msNc, धर्म \msCa\msCb\Ed, धर्मो \msCc, ध\lac\  \msNb}}% 
    \var{{\devanagarifont \numnoemph\vb \lem \mssALL, \lac\  \msNb, अहिंसा परमो गतिः \Ed}}% 

%Verse 3:32

{\devanagarifont अहिंसा परमं ब्रह्म अहिंसा परमः शिवः {॥ ३:३२॥} \veg\dontdisplaylinenum }%
     \var{{\devanagarifont \numnoemph\vc \lem \mssALL, 
\uncl{अहिंसा परमं ब्रह्म} \msNb, अहिंसा परंमं ब्रह्म \msNc}}% 


\alalalfejezet{मांसाहारः}

{\devanagarifont मांसाशनान्निवर्तेत मनसापि न काङ्क्षयेत् \thinspace{\dandab} \dontdisplaylinenum }%
     \var{{\devanagarifont \numemph\va\textbf{मांसाशनान्नि॰}\lem \msCa\msCb\Ed, मान्साशन नि॰ \msCc, 
मांसाशनन्नि॰ \msNa, मन्सासनन्नि॰ \msNb, \uncl{मांसशानान्नि}॰ \msNc}}% 

%Verse 3:33

{\devanagarifont स महत्फलमाप्नोति यस्तु मांसं विवर्जयेत् {॥ ३:३३॥} \veg\dontdisplaylinenum }%
     \var{{\devanagarifont \numnoemph\vd\textbf{मांसं}\lem \mssCaCbCc\msNa, मांस \msNb\Ed, मासं \msNc}}% 

\vfill
\pageparbreak
\vers

{\devanagarifont स्वमांसं परमांसेन यो वर्धयितुमिच्छति \thinspace{\dandab} \dontdisplaylinenum }%
     \var{{\devanagarifont \numemph\va\textbf{॰मांसेन}\lem \mssALL, ॰मासेन \msNc}}% 
    \var{{\devanagarifont \numnoemph\vb\textbf{वर्धयितु॰}\lem \mssALL, वर्द्धयति \msNb}}% 
    \paral{{\devanagarifontsmall \vab {\englishfont  = \MBH\ 13.116.14ab and 13.116.34ab \similar\ \UUMS\ 2.48cd:
                          }स्वमांसं परमांसेन यो देहे वृद्धिमिच्छति }}

%Verse 3:34

{\devanagarifont अनभ्यर्च्य पितॄन्देवान्न ततो ऽन्यो ऽस्ति पापकृत् {॥ ३:३४॥} \veg\dontdisplaylinenum }%
     \var{{\devanagarifont \numnoemph\vc\textbf{पितॄन्}\lem \msCa\msCb\msNa\msNc, पितृन् \msCc\Ed, \uncl{पितॄन्} \msNb}}% 
    \var{{\devanagarifont \numnoemph\vd\textbf{ततो ऽन्यो}\lem \mssALL, तदन्यो \Ed}}% 
    \paral{{\devanagarifontsmall \vo {\englishfont \similar\ \MANU\ 5.52 (Olivelle's edition):} 
                 स्वमांसं परमांसेन यो वर्धयितुमिच्छति\thinspace{\devanagarifontsmall ।}
                 अनभ्यर्च्य पितॄन्देवान्न ततो ऽन्यो स्त्यपुण्यकृत्\thinspace{\devanagarifontsmall ॥} }}

{\devanagarifont मधुपर्के च यज्ञे च पितृदैवतकर्मणि \thinspace{\dandab} \dontdisplaylinenum }%
     \var{{\devanagarifont \numemph\vb\textbf{॰दैवत॰}\lem \mssALL, ॰देवत॰ \msCc\msNb}}% 

%Verse 3:35

{\devanagarifont अत्रैव पशवो हिंस्या नान्यत्र मनुरब्रवीत् {॥ ३:३५॥} \veg\dontdisplaylinenum }%
     \var{{\devanagarifont \numnoemph\vc \lem \msCa\msCc\msNc\Ed, 
अत्रैव पशवो हिंसा \msCb, अत्रैव पशवो हिंस्यान् \msNa, 
\lac\  \msNb}}% 
    \var{{\devanagarifont \numnoemph\vd \lem \mssALL, 
\lac \uncl{त्र मनुरब्रवीत्} \msNb}}% 
    \paral{{\devanagarifontsmall \vo {\englishfont \similar\ \MANU\ 5.41 (Olivelle's edition):}
                         मधुपर्के च यज्ञे च पितृदैवतकर्मणि\thinspace{\devanagarifontsmall ।}
                         अत्रैव पशवो हिंस्या नान्यत्रेत्यब्रवीन्मनुः\thinspace{\devanagarifontsmall ॥} }}

{\devanagarifont क्रीत्वा स्वयं वाप्युत्पाद्य परोपहृतमेव वा \thinspace{\dandab} \dontdisplaylinenum }%
     \var{{\devanagarifont \numemph\va\textbf{क्रीत्वा}\lem \mssALL, कृत्वा \Ed\oo 
\textbf{॰प्युत्पाद्य}\lem \mssALL, ॰प्युत्पाद्या॰ \Ed}}% 
    \var{{\devanagarifont \numnoemph\vb\textbf{॰हृत॰}\lem \mssALL, ॰हित॰ \Ed\oo 
\textbf{वा}\lem \mssALL, च \Ed}}% 

%Verse 3:36

{\devanagarifont देवान्पितॄंश्चार्चयित्वा खादन्मांसं न दोषभाक् {॥ ३:३६॥} \veg\dontdisplaylinenum }%
     \var{{\devanagarifont \numnoemph\vc\textbf{पितॄंश्चार्चयित्वा}\lem \mssALL, पितॄश्चार्चयित्वा \msNb, पितृश्चार्पयित्वा \Ed}}% 
    \var{{\devanagarifont \numnoemph\vd\textbf{मांसं}\lem \mssALL, मासं \msNc}}% 
    \paral{{\devanagarifontsmall \vo {\englishfont = \MANU\ 5.32 (in Olivelle's critical edition; other editions read}
                          परोपकृत॰ {\englishfont in pāda b)} }}

{\devanagarifont वेदयज्ञतपस्तीर्थदानशीलक्रियाव्रतैः \thinspace{\dandab} \dontdisplaylinenum }%
     \var{{\devanagarifont \numemph\vb\textbf{॰शील॰}\lem \mssALL, ॰शल॰ \msCc\oo 
\textbf{॰व्रतैः}\lem \mssALL, ॰व्र\uncl{तः} \msCb}}% 

%Verse 3:37

{\devanagarifont मांसाहारनिवृत्तानां षोडशांशं न पूर्यते {॥ ३:३७॥} \veg\dontdisplaylinenum }%
     \var{{\devanagarifont \numnoemph\vc\textbf{॰वृत्तानां}\lem \mssALL, ॰वृत्ताना \msNb, ॰वृत्तीनां \Ed}}% 
    \var{{\devanagarifont \numnoemph\vd\textbf{न}\lem \mssALL, त \msCb}}% 

{\devanagarifont मृगाः पर्णतृणाहारादजमेषगवादिभिः \thinspace{\dandab} \dontdisplaylinenum }%
     \var{{\devanagarifont \numemph\va\textbf{पर्ण॰}\lem \mssALL, पण्ण॰ \msNa, पर्णा॰ \Ed}}% 
    \var{{\devanagarifont \numnoemph\vab\textbf{॰हाराद॰}\lem \msCa\msCc\msNbpcorr\msNc\Ed, ॰हारा अ॰ \msCb\msNa, ॰हाद॰ \msNbacorr}}% 

%Verse 3:38

{\devanagarifont सुखिनो बलवन्तश्च विचरन्ति महीतले {॥ ३:३८॥} \veg\dontdisplaylinenum }%
 
{\devanagarifont वानराः फलमाहारा राक्षसा रुधिरप्रियाः \thinspace{\dandab} \dontdisplaylinenum }%
     \var{{\devanagarifont \numemph\vab\textbf{॰हारा रा॰}\lem \msCb\msNa\msNb, ॰हाराद्रा॰ \msCa\msCc\msNc\Ed}}% 

%Verse 3:39

{\devanagarifont निहता राक्षसाः सर्वे वानरैः फलभोजिभिः {॥ ३:३९॥} \veg\dontdisplaylinenum }%
     \var{{\devanagarifont \numnoemph\vd\textbf{॰भोजिभिः}\lem \mssALL, ॰भोगिभिः \Ed}}% 

\vfill
\pageparbreak
\vers

{\devanagarifont तस्मान्मांसं न हीहेत बलकामेन भो द्विज \thinspace{\dandab} \dontdisplaylinenum }%
     \var{{\devanagarifont \numemph\va\textbf{मांसं}\lem \mssALL, मासं \msNc}}% 
    \var{{\devanagarifont \numnoemph\vb\textbf{हीहेत}\lem \mssALL, हीयेत \msNa\msNb}}% 

%Verse 3:40

{\devanagarifont बलेन च गुणाकर्षात्परतो भयभीरुणा {॥ ३:४०॥} \veg\dontdisplaylinenum }%
     \var{{\devanagarifont \numnoemph\vc\textbf{गुणाकर्षा॰}\lem \conjTorzsok, गुणाकाशा॰ \mssCaCbCc\msNa\msNb\msNc, गुणा कुर्या॰ \Ed}}% 

{\devanagarifont अहिंसकसमो नास्ति दानयज्ञसमीहया \thinspace{\dandab} \dontdisplaylinenum }%
     \var{{\devanagarifont \numemph\vb\textbf{॰यज्ञसमीहया}\lem \msCa\msCb\msNa\msNb, ॰धर्मसमीहया \msCc, 
॰यज्ञसमीहयाः \msNc, ॰धर्मसमीहय \Ed}}% 

%Verse 3:41

{\devanagarifont इह लोके यशः कीर्तिः परत्र च परा गतिः {॥ ३:४१॥} \veg\dontdisplaylinenum }%
     \var{{\devanagarifont \numnoemph\vc\textbf{यशः}\lem \mssALL, य\uncl{शं} \msCc}}% 
    \var{{\devanagarifont \numnoemph\vd\textbf{परा गतिः}\lem \msCc\msNa\msNc, \uncl{परा गतिः} \msCa, 
पराङ्गतिम् \msCb\msNb, परां गतिः \Ed}}% 

\ujvers\nemsloka {
{\devanagarifont त्रैलोक्यं मणिरत्नपूर्णमखिलं दत्त्वोत्तमे ब्राह्मणे }%
  \dontdisplaylinenum}    \var{{\devanagarifont \numemph\va\textbf{त्रैलोक्यं}\lem \mssALL, त्रैलोक्य \msNb\oo 
\textbf{अखिलं दत्त्वोत्तमे ब्राह्मणे}\lem \mssALL, 
अ\uncl{खिलं}\lk\lk \lk\lk \lk\lk \lk\ \msCa, अखिलं दत्तोत्तमे ब्राह्मणे \msNa}}% 
    \paral{{\devanagarifontsmall \va {\englishfont \SDHS\ 11.91:}
                    त्रैलोक्यमपि यो दद्यादखिलं रत्नपूरितम्\thinspace{\devanagarifontsmall ।}
                    चरेत्तपांसि सर्वाणि न तत्तुल्यमहिंसया\thinspace{\devanagarifontsmall ॥} }}


\nemslokab

{\devanagarifont कोटीयज्ञसहस्रपद्ममयुतं दत्त्वा महीं दक्षिणाम्  \danda\dontdisplaylinenum }%
     \var{{\devanagarifont \numnoemph\vb\textbf{कोटीयज्ञसहस्रपद्मम्}\lem \mssALL, \lk\lk \lk\lk \lk\lk \lk\lk \lk\  \msCa\oo 
\textbf{महीं}\lem \mssALL, मही \msCc}}% 

\nemslokac

{\devanagarifont तीर्थानां च सहस्रकोटिनियुतं स्नात्वा सकृन्मानव }%
  \dontdisplaylinenum    \var{{\devanagarifont \numnoemph\vc\textbf{॰कोटि॰}\lem \mssALL, ॰कोटी॰ \Ed\ \unmetr\oo 
\textbf{स्नात्वा}\lem \mssALL, स्ना ऽ \msCb}}% 

%Verse 3:42


\nemslokad

{\devanagarifont एतत्पुण्यफलमहिंसकजनः प्राप्नोति निःसंशयः {॥ ३:४२॥} \veg\dontdisplaylinenum }%
     \var{{\devanagarifont \numnoemph\vd\textbf{॰फलमहिंस॰}\lem \mssALL, ॰फलं त्वहिंस॰ \msNc\oo 
\textbf{निःसंशयः}\lem \msCc\msNa\msNb\msNc, \lk\lk \lk\lk\ \msCa, निःसंशय\lk\ \msCb, निःसंशयं \Ed}}% 

\vers


{\devanagarifont 
\jump
\begin{center}
\ketdanda~इति वृषसारसंग्रहे अहिंसाप्रशंसा नामाध्यायस्तृतीयः~\ketdanda
\end{center}
\dontdisplaylinenum\vers  }%
     \var{{\devanagarifont \numnoemph{\englishfont \Colo:}\textbf{नामाध्यायस्तृतीयः}\lem \mssALL, नामाध्यायस्तृतीय \msNc, 
नामस्तृतीयो ऽध्यायः \Ed}}% 
\bekveg\szamveg
\vfill
\phpspagebreak

\versno=0\fejno=4
\thispagestyle{empty}

\centerline{\Large\devanagarifontbold [   चतुर्थो ऽध्यायः  ]}{\vrule depth10pt width0pt} \fancyhead[CO]{{\footnotesize\devanagarifont वृषसारसंग्रहे  }}
\fancyhead[CE]{{\footnotesize\devanagarifont चतुर्थो ऽध्यायः  }}
\fancyhead[LE]{}
\fancyhead[RE]{}
\fancyhead[LO]{}
\fancyhead[RO]{}
\szam\bek



\alalfejezet{यमेषु सत्यम् (२)}
\vers


{\devanagarifont अनर्थयज्ञ उवाच {\dandab}\dontdisplaylinenum  }%
     \lacuna{\devanagarifontsmall {\englishfont Witnesses used for this chapter: \msCa\ ff.\thinspace 198v--201v, 
                                              \msCb\ ff.\thinspace 206r--208v, 
                                              \msCc\ ff.\thinspace 273v--277r,
                                              \msNa\ ff.\thinspace 6r--9r, 
                                              \msNb\ exp.\thinspace 48--50 (lower--upper),
                                              \msNc\ ff.\thinspace 214v--217r,
                                              \Ed\ pp.\thinspace 591--597;
                                        \mssCaCbCc\ = \msCa + \msCb + \msCc} }%
  
{\devanagarifont सद्भावः सत्यमित्याहुर्दृष्टप्रत्ययमेव वा \thinspace{\danda} \dontdisplaylinenum }%
     \var{{\devanagarifont \numemph\va\textbf{सद्भावः}\lem \mssALL, सद्भाव॰ \msNb\Ed}}% 
    \var{{\devanagarifont \numnoemph\vab\textbf{सत्यमित्याहुर्दृ॰}\lem \msCb\msNa\msNc\Ed, सत्य\uncl{मि}$\-$त्याहु दृ॰ \msCa, 
सत्यमित्याहु दृ॰ \msCc, सत्यामित्याहुर्दृ॰ \msNb}}% 
    \var{{\devanagarifont \numnoemph\vb\textbf{॰प्रत्यय॰}\lem \msCa\msCb\msNa\msNb, ॰प्रत्य॰ \msCc, ॰प्रत्येय॰ \msNc, प्रत्यक्ष॰ \Ed}}% 
    \paral{{\devanagarifontsmall \va {\englishfont \similar\ \MBH\ 12.288.45d:} 
                         सद्भावः सत्यमुच्यते 
                    {\englishfont \compare\  also \BRAHMANDAPUR\ 3.3.86ab:}
                         असद्भावो ऽनृतं ज्ञेयं सद्भावः सत्यमुच्यते  }}

%Verse 4:1

{\devanagarifont यथाभूतार्थकथनं तत्सत्यकथनं स्मृतम् {॥ ४:१॥} \veg\dontdisplaylinenum }%
     \var{{\devanagarifont \numnoemph\vc \lem \mssALL, 
यथाभूतार्थ \msCcacorr, 
यथाभूतार्थनं क्त \msCcpcorr}}% 
    \var{{\devanagarifont \numnoemph\vd\textbf{तत्सत्यकथनं}\lem \msCa\msNa\msNb\msNc\Ed, 
तत्सत्यकथकं \msCb, 
कथनं स्मृतं \msCcacorr, 
\uncl{सत्यक ज}कथनं स्मृतं \msCcpcorr}}% 
    \paral{{\devanagarifontsmall \vcd {\englishfont \compare\ \SDHS\ 11.105:} 
                 स्वानुभूतं स्वदृष्टं च यः पृष्टार्थं न गूहति\thinspace{\devanagarifontsmall ।}
                 यथाभूतार्थकथनमित्येतत्सत्यलक्षणम्\thinspace{\devanagarifontsmall ॥} }}

{\devanagarifont आक्रोशताडनादीनि यः सहेत सुदुःसहम् \thinspace{\dandab} \dontdisplaylinenum }%
     \var{{\devanagarifont \numemph\va\textbf{॰ताडना॰}\lem \mssALL, ॰नाडना॰ \msCb}}% 
    \var{{\devanagarifont \numnoemph\vb\textbf{सुदुःसहम्}\lem \mssALL, सुदुसहं \msCc}}% 

%Verse 4:2

{\devanagarifont क्षमते यो जितात्मा तु स च सत्यमुदाहृतम् {॥ ४:२॥} \veg\dontdisplaylinenum }%
     \var{{\devanagarifont \numnoemph\vd\textbf{सत्यमुदाहृतम्}\lem \mssALL, 
\uncl{सत्य}मु\uncl{दा}हृतम् \msCa}}% 
    \paral{{\devanagarifontsmall \vo {\englishfont \compare\ \SDHS\ 11.82:}
                 आक्रुष्टस्ताडितो वापि यो नाक्रोशेन्न ताडयेत्\thinspace{\devanagarifontsmall ।}
                 वागाद्यविकृतः स्वस्थं क्षान्तिरेषा सुनिर्मला\thinspace{\devanagarifontsmall ॥} }}

{\devanagarifont वधार्थमुद्यतः शस्त्रं यदि पृच्छेत कर्हिचित् \thinspace{\dandab} \dontdisplaylinenum }%
     \var{{\devanagarifont \numemph\va\textbf{॰द्यतः}\lem \mssALL, ॰द्यत \msNa\oo 
\textbf{शस्त्रं}\lem \msCa\msNa\msNb\msNc, सत्य \msCb\Ed, शस्त्र \msCc}}% 
    \var{{\devanagarifont \numnoemph\vb\textbf{कर्हिचित्}\lem \mssCaCbCc\Ed, कर्हचित् \msNa\msNb\msNc}}% 

%Verse 4:3

{\devanagarifont न तत्र सत्यं वक्तव्यमनृतं सत्यमुच्यते {॥ ४:३॥} \veg\dontdisplaylinenum }%
     \var{{\devanagarifont \numnoemph\vc\textbf{सत्यं}\lem \mssALL, सत्य \msCb\Ed}}% 

\vfill
\pageparbreak
\vers

{\devanagarifont वधार्हः पुरुषः कश्चिद्व्रजेत्पथि भयातुरः \thinspace{\dandab} \dontdisplaylinenum }%
     \var{{\devanagarifont \numemph\vb\textbf{॰तुरः}\lem \mssALL, ॰तुर \msCb}}% 

%Verse 4:4

{\devanagarifont पृच्छतो ऽपि न वक्तव्यं सत्यं तद्वापि उच्यते {॥ ४:४॥} \veg\dontdisplaylinenum }%
     \var{{\devanagarifont \numnoemph\vc\textbf{पृच्छतो}\lem \mssALL, पृच्छते \Ed}}% 
    \var{{\devanagarifont \numnoemph\vd\textbf{तद्वापि}\lem \mssALL, तदपि \msNb}}% 

\ujvers\nemsloka {
{\devanagarifont न नर्मयुक्तमनृतं हिनस्ति }%
  \dontdisplaylinenum}    \var{{\devanagarifont \numemph\va\textbf{हिनस्ति}\lem \msCa\msCb\msNb\msNc, हि नास्ति \msCc\msNa\Ed}}% 


\nemslokab

{\devanagarifont न स्त्रीषु राजन्न विवाहकाले  \danda\dontdisplaylinenum }%
     \var{{\devanagarifont \numnoemph\vb\textbf{राजन्न}\lem \mssALL, राज न \msCc, राज्यं न \msNa}}% 

\nemslokac

{\devanagarifont प्राणात्यये सर्वधनापहारे }%
  \dontdisplaylinenum    \var{{\devanagarifont \numnoemph\vc\textbf{॰त्यये}\lem \mssALL, ॰त्यजे \msNb\oo 
\textbf{॰पहारे}\lem \mssALL, ॰प्रहारे \msCc\msNb}}% 

%Verse 4:5


\nemslokad

{\devanagarifont पञ्चानृतं सत्यमुदाहरन्ति {॥ ४:५॥} \veg\dontdisplaylinenum }%
     \paral{{\devanagarifontsmall \vo {\englishfont \similar\ \MBH\ 1.77.16:} न नर्मयुक्तं वचनं हिनस्ति न स्त्रीषु राजन्न विवाहकाले\thinspace{\devanagarifontsmall ।}
                                                प्राणात्यये सर्वधनापहारे पञ्चानृतान्याहुरपातकानि\thinspace{\devanagarifontsmall ॥};
                            {\englishfont \MBH\ 12.159.28:} न नर्मयुक्तं वचनं हिनस्ति न स्त्रीषु राजन्न विवाहकाले\thinspace{\devanagarifontsmall ।}
                                                न गुर्वर्थे नात्मनो जीवितार्थे पञ्चानृतान्याहुरपातकानि\thinspace{\devanagarifontsmall ॥};
                              {\englishfont \MATSP\ 31.16:} न नर्मयुक्तं वचनं हिनस्ति न स्त्रीषु राजन्न विवाहकाले\thinspace{\devanagarifontsmall ।}
         {\englishfont Abhidharmakośabhāṣya 24114--24117 (introduced by } मोहजो मृषावादो यथाह{\englishfont ):}
                                                न नर्मयुक्तमनृतं हि नास्ति न स्त्रीषु राजन्न विवाहकाले\thinspace{\devanagarifontsmall ।}
                                                प्राणात्यये सर्वधनापहारे पञ्चानृतान्याहुरपातकानि\thinspace{\devanagarifontsmall ॥} {\englishfont etc.} }}

\vers


{\devanagarifont देवमानुषतिर्येषु सत्यं धर्मः परो यतः \thinspace{\dandab} \dontdisplaylinenum }%
     \var{{\devanagarifont \numemph\vb\textbf{॰मानुष॰}\lem \mssALL, ॰मानुष्य॰ \msNc\oo 
\lem  \msCb\msCc, सत्यं धर्मः पयतः \msCa, 
सत्यं धर्म परो यतः \msNa\msNc, सत्यधर्म परो यतः \msNb, सत्यधर्मपरायणः \Ed}}% 

%Verse 4:6

{\devanagarifont सत्यं श्रेष्ठं वरिष्ठं च सत्यं धर्मः सनातनः {॥ ४:६॥} \veg\dontdisplaylinenum }%
     \var{{\devanagarifont \numnoemph\vc\textbf{श्रेष्ठं}\lem \mssALL, श्रेष्ठ \msNb\Ed\oo 
\textbf{वरिष्ठं च}\lem \mssALL, वरिष्ठम्वरिष्ठम्वञ्च \msCbacorr}}% 
    \var{{\devanagarifont \numnoemph\vd\textbf{सत्यं}\lem \mssALL, सत्य॰ \msCb\msNb\oo 
\textbf{धर्मः}\lem \mssALL, धर्म \msCc\Ed}}% 

{\devanagarifont सत्यं सागरमव्यक्तं सत्यमक्षयभोगदम् \thinspace{\dandab} \dontdisplaylinenum }%
     \var{{\devanagarifont \numemph\va\textbf{सत्यं}\lem \mssALL, सत्य \msCc}}% 
    \var{{\devanagarifont \numnoemph\vb \lem \msCa\msNa\msNb\msNc, सत्यंमक्षयभोगदम् \msCb\msCc, 
सत्यमक्षयते नरं \Ed}}% 

%Verse 4:7

{\devanagarifont सत्यं पोतः परत्रार्थं सत्यं पन्थान विस्तरम् {॥ ४:७॥} \veg\dontdisplaylinenum }%
     \var{{\devanagarifont \numnoemph\vc\textbf{पोतः}\lem \mssALL, पोत \msNa, प्रोक्तः \Ed}}% 
    \var{{\devanagarifont \numnoemph\vd\textbf{पन्थान विस्तरम्}\lem \mssALL, यज्ज्ञानविस्तरम् \Ed}}% 

{\devanagarifont सत्यमिष्टगतिः प्रोक्तं सत्यं यज्ञमनुत्तमम् \thinspace{\dandab} \dontdisplaylinenum }%
     \var{{\devanagarifont \numemph\va\textbf{॰ष्टगतिः}\lem \mssALL, ॰\uncl{ष्टा}गतिः \msNb}}% 

%Verse 4:8

{\devanagarifont सत्यं तीर्थं परं तीर्थं सत्यं दानमनन्तकम् {॥ ४:८॥} \veg\dontdisplaylinenum }%
     \var{{\devanagarifont \numnoemph\vc\textbf{तीर्थं}\lem \mssCaCbCc\msNa, तीर्थ \msNb\msNc, तीर्थात् \Ed}}% 

\vfill
\pageparbreak
\vers

{\devanagarifont सत्यं शीलं तपो ज्ञानं सत्यं शौचं दमः शमः \thinspace{\dandab} \dontdisplaylinenum }%
     \var{{\devanagarifont \numemph\va\textbf{सत्यं}\lem \mssALL, सत्य \msCb}}% 
    \var{{\devanagarifont \numnoemph\vb\textbf{शमः}\lem \mssALL, शमम् \msNb}}% 

%Verse 4:9

{\devanagarifont सत्यं सोपानमूर्ध्वस्य सत्यं कीर्तिर्यशः सुखम् {॥ ४:९॥} \veg\dontdisplaylinenum }%
     \var{{\devanagarifont \numnoemph\vc\textbf{सत्यं}\lem \mssALL, संत्यं \msCb, सत्य \msNc}}% 
    \var{{\devanagarifont \numnoemph\vd\textbf{सुखम्}\lem \mssALL, सुखः \Ed}}% 
    \paral{{\devanagarifontsmall \vc {\englishfont \similar\ \VARP\ 193.36cd:} सत्यं स्वर्गस्य सोपानं पारावारस्य नौरिव }}

{\devanagarifont अश्वमेधसहस्रं च सत्यं च तुलया धृतम् \thinspace{\dandab} \dontdisplaylinenum }%
     \var{{\devanagarifont \numemph\va\textbf{॰सहस्रं च}\lem \mssALL, ॰सहस्रस्य \msCc}}% 
    \var{{\devanagarifont \numnoemph\vb\textbf{तुलया}\lem \mssALL, तुल्यया \msCc}}% 

%Verse 4:10

{\devanagarifont अश्वमेधसहस्राद्धि सत्यमेव विशिष्यते {॥ ४:१०॥} \veg\dontdisplaylinenum }%
     \var{{\devanagarifont \numnoemph\vc\textbf{॰सहस्राद्धि}\lem \mssALL, ॰सहस्रा हि \msCc}}% 
    \var{{\devanagarifont \numnoemph\vd\textbf{एव}\lem \mssALL, एवं \msCc\Ed}}% 
    \paral{{\devanagarifontsmall \vo {\englishfont  = \MBH\ 1.69.22 = \MBH\ Suppl. 13.20.330 = \MARKP\ 8.42 = \VDHU\ 3.265.7
                        \similar\ \MBH\ 12.156.26 (pāda d reads } सत्यमेवातिरिच्यते{\englishfont ) \similar\ \VDH\ 55.6 
                            (pāda d reads} सत्यमेतद्विशिष्यते{\englishfont )};
                    {\englishfont \compare\ \SDHS\ 11.107:}
                         अश्वमेधायुतं पूर्णं सत्यञ्च तुलितं पुरा\thinspace{\devanagarifontsmall ।}
                         अश्वमेधायुतात्सत्यमधिकं बहुभिर्गुणैः\thinspace{\devanagarifontsmall ॥} }}

{\devanagarifont सत्येन तपते सूर्यः सत्येन पृथिवी स्थिता \thinspace{\dandab} \dontdisplaylinenum }%
     \var{{\devanagarifont \numemph\vab\textbf{सूर्यः सत्येन पृथिवी स्थिता}\lem \msNa\msNc, सू\uncl{र्यः स}त्येन पृथि स्थिताः \msCa, 
सूर्यः सत्यैन पृथिवी स्थिता \msCb, सूर्य सत्येन पृथिवी स्थिताः \msCc, 
सूर्य \uncl{सत्ये} \lac\  वी स्थिता \msNb, सूर्यः सत्येन पृथिवी स्थिताः \Ed}}% 

%Verse 4:11

{\devanagarifont सत्येन वायवो वान्ति सत्ये तोयं च शीतलम् {॥ ४:११॥} \veg\dontdisplaylinenum }%
     \var{{\devanagarifont \numnoemph\vc\textbf{वायवो}\lem \mssALL, वात्यवो \msNb}}% 
    \var{{\devanagarifont \numnoemph\vd\textbf{सत्ये}\lem \mssALL, सत्यात् \Ed}}% 
    \paral{{\devanagarifontsmall \vo {\englishfont \similar\ \VARP\ 193.37:} 
                         सूर्यस्तपति सत्येन वातः सत्येन वाति च\thinspace{\devanagarifontsmall ।}  
                         अग्निर्दहति सत्येन सत्येन पृथिवी स्थिता\thinspace{\devanagarifontsmall ॥} 
                    {\englishfont \similar\ \VDHU\ 3.265.4cd--5ab:}
                         सत्येन वायुरभ्येति सत्येनाभासते रविः\thinspace{\devanagarifontsmall ॥} 
                         सत्येन चाग्निर्दहति स्वर्गं सत्येन गच्छति\thinspace{\devanagarifontsmall ।}  }}

{\devanagarifont तिष्ठन्ति सागराः सत्ये समयेन प्रियव्रतः \thinspace{\dandab} \dontdisplaylinenum }%
     \var{{\devanagarifont \numemph\va\textbf{सागराः}\lem \mssALL, सागरा \msCc}}% 
    \var{{\devanagarifont \numnoemph\vb\textbf{समयेन}\lem \mssALL, सत्येन च \Ed}}% 

%Verse 4:12

{\devanagarifont सत्ये तिष्ठति गोविन्दो बलिबन्धनकारणात् {॥ ४:१२॥} \veg\dontdisplaylinenum }%
 
\vfill
\pageparbreak
\vers

{\devanagarifont अग्निर्दहति सत्येन सत्येन शशिनश्चरः \thinspace{\dandab} \dontdisplaylinenum }%
     \var{{\devanagarifont \numemph\vab\textbf{सत्येन सत्येन}\lem \mssALL, सत्येन \msNaacorr\msNc}}% 
    \var{{\devanagarifont \numnoemph\vb\textbf{शशिनश्चरः}\lem \conj, सशि\uncl{भाचरः} \msCa, 
श\uncl{सि}\lk चरः \msCb, 
स शिरा वरः \msCc, 
शशिराचरः \msNa\msNb\msNc, 
शशिभाष्करः \Ed}}% 
    \paral{{\devanagarifontsmall \vc {\englishfont \similar\ \VARP\ 193.37cd:} 
                 अग्निर्दहति सत्येन सत्येन पृथिवी स्थिता }}
    \paral{{\devanagarifontsmall \vd {\englishfont \compare\ \VARP\ 155.30cd:}
                         सत्येन सूर्यस्तपति सोमः सत्येन राजते;
                  {\englishfont \compare\ \LAKSMINARS\  1.345.50ab:}
                         सत्येन सूर्यस्तपति चन्द्रः सत्येन वर्धते\thinspace{\devanagarifontsmall ।}
                 {\englishfont \compare\ \MBH\ Suppl. 13.587:}
                         मुचुकुन्देन मान्धात्रा हरिश्चन्द्रेण चाभिभो\thinspace{\devanagarifontsmall ।}
                         सत्यं वदत मासत्यं सत्यं धर्मः सनातनः\thinspace{\devanagarifontsmall ।}
                         हरिश्चन्द्रश्चरति वै दिवि सत्येन चन्द्रवत्\thinspace{\devanagarifontsmall ॥} }}

%Verse 4:13

{\devanagarifont सत्येन विन्ध्यास्तिष्ठन्ति वर्धमानो न वर्धते {॥ ४:१३॥} \veg\dontdisplaylinenum }%
     \var{{\devanagarifont \numnoemph\vc\textbf{विन्ध्यास्तिष्ठन्ति}\lem \msCa\msNa\msNc, 
विन्ध्यस्तिष्ठन्ति \msCb\msNb, विन्ध्या तिष्ठन्ति \msCc, तिष्ठते विन्ध्यो \Ed}}% 

{\devanagarifont लोकालोकः स्थितः सत्ये मेरुः सत्ये प्रतिष्ठितः \thinspace{\dandab} \dontdisplaylinenum }%
     \var{{\devanagarifont \numemph\va\textbf{॰लोकः}\lem \Ed, ॰लोक \mssCaCbCc\msNa\msNb\msNc\oo 
\textbf{स्थितः}\lem \mssALL, स्थिः \msNc\oo 
\textbf{सत्ये}\lem \mssALL, सत्यं \Ed}}% 
    \var{{\devanagarifont \numnoemph\vb\textbf{मेरुः}\lem \mssALL, मेरु \msCc\Ed}}% 

%Verse 4:14

{\devanagarifont वेदास्तिष्ठन्ति सत्येषु धर्मः सत्ये प्रतिष्ठति {॥ ४:१४॥} \veg\dontdisplaylinenum }%
     \var{{\devanagarifont \numnoemph\vc\textbf{वेदास्ति॰}\lem \mssALL, देवास्ति॰ \msCb, वेदा ति॰ \Ed}}% 
    \var{{\devanagarifont \numnoemph\vd\textbf{सत्ये}\lem \mssALL, धर्मे \msCc\oo 
\textbf{प्रतिष्ठति}\lem \mssALL, प्रतिष्ठिति \msNcacorr, प्रतिष्ठितः \msNcpcorr}}% 

{\devanagarifont सत्यं गौः क्षरते क्षीरं सत्यं क्षीरे घृतं स्थितम् \thinspace{\dandab} \dontdisplaylinenum }%
     \var{{\devanagarifont \numemph\va\textbf{गौः}\lem \mssALL, गौ \msCc\msNb}}% 
    \var{{\devanagarifont \numnoemph\vab\textbf{क्षीरं सत्यं}\lem \mssALL, क्षीत्यं \msCbacorr, क्सी\lk  नित्यं \msCbpcorr}}% 
    \var{{\devanagarifont \numnoemph\vb\textbf{क्षीरे घृतं स्थितम्}\lem \msCa\msCb\msNa\msNc, क्षीरं घृतं स्थितम् \msCc, क्षीरे घृत स्थितम् \msNb, 
क्षीरं स्थितं घृतम् \Ed}}% 

%Verse 4:15

{\devanagarifont सत्ये जीवः स्थितो देहे सत्यं जीवः सनातनः {॥ ४:१५॥} \veg\dontdisplaylinenum }%
     \var{{\devanagarifont \numnoemph\vc\textbf{सत्ये जीवः}\lem \mssALL, सत्ये जीव \msNc, सत्यं जीव \Ed}}% 
    \var{{\devanagarifont \numnoemph\vd\textbf{जीवः}\lem \mssALL, जीव \msCc}}% 

{\devanagarifont सत्यमेकेन सम्प्राप्तो धर्मसाधननिश्चयः \thinspace{\dandab} \dontdisplaylinenum }%
     \var{{\devanagarifont \numemph\va\textbf{सत्यमेकेन}\lem \mssALL, सत्यमेकैन \msCb, सत्येमेकेन \msNb}}% 
    \var{{\devanagarifont \numnoemph\vb\textbf{धर्म॰}\lem \Ed, धर्मः \mssCaCbCc\msNa\msNb\msNc\oo 
\textbf{॰निश्चयः}\lem \mssALL, ॰निश्चः \msCa}}% 

%Verse 4:16

{\devanagarifont रामराघववीर्येण सत्यमेकं सुरक्षितम् {॥ ४:१६॥} \veg\dontdisplaylinenum }%
     \var{{\devanagarifont \numnoemph\vd\textbf{सत्यमेकं}\lem \mssALL, सत्येमेकं \msNb\oo 
\textbf{सुरक्षितम्}\lem \mssALL, सुरिक्षितम् \msCb, सुरक्षितः \msNa}}% 

{\devanagarifont एवं सत्यविधानस्य कीर्तितं तव सुव्रत \thinspace{\dandab} \dontdisplaylinenum }%
     \var{{\devanagarifont \numemph\va\textbf{एवं सत्य॰}\lem \msCb, एतत्सत्य॰ \msCa\msCc\msNa\msNb\msNc\Ed}}% 
    \var{{\devanagarifont \numnoemph\vb\textbf{सुव्रत}\lem \msCa\msNa\msNc, सुव्रते \msCb\msNb, सुव्र\uncl{तः} \msCc, सुव्रतं \Ed}}% 

%Verse 4:17

{\devanagarifont सर्वलोकहितार्थाय किमन्यच्छ्रोतुमिच्छसि {॥ ४:१७॥} \veg\dontdisplaylinenum }%
 
\vfill
\pageparbreak
\vers


\alalfejezet{यमेष्वस्तेयम् (३)}
{\devanagarifont विगतराग उवाच {\dandab}\dontdisplaylinenum  }%
 
{\devanagarifont न हि तृप्तिं विजानामि श्रुत्वा धर्मं तवाप्यहम् \thinspace{\danda} \dontdisplaylinenum }%
     \var{{\devanagarifont \numemph\va\textbf{तृप्तिं}\lem \mssALL, तृप्ति \msCc\oo 
\textbf{विजानामि}\lem \mssALL, विनामि \msNb}}% 
    \var{{\devanagarifont \numnoemph\vb \lem \mssALL, 
श्रु धर्मन्तवाप्यहम् \msCa, 
धर्मं श्रुत्वा तथाप्यहम् \Ed}}% 

%Verse 4:18

{\devanagarifont उपरिष्टादतो भूयः कथयस्व तपोधन {॥ ४:१८॥} \veg\dontdisplaylinenum }%
     \var{{\devanagarifont \numnoemph\vd\textbf{॰धन}\lem \msCc\msNa\msNb\Ed, ॰धून \msCa, ॰धनः \msCb\msNc}}% 

{\devanagarifont अनर्थयज्ञ उवाच {\dandab}\dontdisplaylinenum  }%
 
{\devanagarifont स्तेयं शृण्वथ विप्रेन्द्र पञ्चधा परिकीर्तितम् \thinspace{\danda} \dontdisplaylinenum }%
     \var{{\devanagarifont \numemph\vb\textbf{॰कीर्तितम्}\lem \mssALL, ॰कीर्त्तिताम् \msCb}}% 

{\devanagarifont अदत्तादानमादौ तु उत्कोचं च ततः परम्  \danda\dontdisplaylinenum }%
     \var{{\devanagarifont \numnoemph\vd\textbf{उत्कोचं च ततः}\lem \mssALL, त्कोच ततः \msCb, उत्कोचं चानृतः \Ed}}% 

%Verse 4:19

{\devanagarifont प्रस्थव्याजस्तुलाव्याजः प्रसह्यस्तेय पञ्चमम् {॥ ४:१९॥} \veg\dontdisplaylinenum }%
     \var{{\devanagarifont \numnoemph\ve\textbf{तुलाव्याजः}\lem \msCb\msNc\Ed, तुलाव्याज \msCa\msCc\msNa\msNb}}% 
    \var{{\devanagarifont \numnoemph\vf\textbf{॰सह्य॰}\lem \mssALL, ॰सह्ये \msNb\oo 
\textbf{॰स्तेय}\lem \mssALL, ॰स्तेन \msCa\msNc\oo 
\textbf{पञ्चमम्}\lem \mssALL, पञ्चमः \msCc\Ed}}% 

{\devanagarifont धृष्टदुष्टप्रभावेन परद्रव्यापकर्षणम् \thinspace{\dandab} \dontdisplaylinenum }%
     \var{{\devanagarifont \numemph\va\textbf{धृष्टदुष्ट॰}\lem \msCa\msNa\msNc\Ed, धृष्टदुम्न॰ \msCb, धृतदुष्ट॰ \msCc, दृष्टदुष्ट॰ \msNb}}% 
    \var{{\devanagarifont \numnoemph\vb\textbf{॰कर्षणम्}\lem \mssALL, ॰कर्षण \msNa}}% 

%Verse 4:20

{\devanagarifont वार्यमाणो ऽपि दुर्बुद्धिरदत्तादानमुच्यते {॥ ४:२०॥} \veg\dontdisplaylinenum }%
     \var{{\devanagarifont \numnoemph\vc\textbf{वार्यमाणो ऽपि}\lem \mssALL, वार्यमानो वि॰ \msCb}}% 

{\devanagarifont उत्कोचं शृणु विप्रेन्द्र धर्मसंकरकारकम् \thinspace{\dandab} \dontdisplaylinenum }%
     \var{{\devanagarifont \numemph\va\textbf{उत्कोचं}\lem \mssALL, उत्कोच \msCa\oo 
\textbf{विप्रेन्द्र}\lem \mssALL, विद्रेन्द्र \msNb}}% 
    \var{{\devanagarifont \numnoemph\vb\textbf{॰संकर॰}\lem \msCc\msNa, ॰शङ्कर॰ \msCa\msCb\msNb, ॰सकर॰ \msNc, ॰संहार॰ \Ed\oo 
\textbf{॰कारकम्}\lem \mssALL, ॰कारकः \msNa}}% 

{\devanagarifont मूल्यं कार्यविनाशार्थमुत्कोचः परिगृह्यते  \danda\dontdisplaylinenum }%
     \var{{\devanagarifont \numnoemph\vc\textbf{मूल्यं}\lem \conj, मूल \mssCaCbCc\msNa\msNb\msNc\Ed\oo 
\textbf{॰विनाशार्थ॰}\lem \mssALL, ॰विनार्थ॰ \msNaacorr}}% 
    \var{{\devanagarifont \numnoemph\vd\textbf{॰त्कोचः}\lem \mssALL, ॰त्कोचं \msNb, ॰त्कोच \Ed}}% 

%Verse 4:21

{\devanagarifont तेन चासौ विजानीयाद्द्रव्यलोभबलात्कृतम् {॥ ४:२१॥} \veg\dontdisplaylinenum }%
     \var{{\devanagarifont \numnoemph\vef\textbf{विजानीयाद्द्र॰}\lem \mssALL, विजानीया द्र॰ \msCc}}% 

{\devanagarifont प्रस्थव्याज-उपायेन कुटुम्बं त्रातुमिच्छति \thinspace{\dandab} \dontdisplaylinenum }%
 
%Verse 4:22

{\devanagarifont तं च स्तेनं विजानीयात्परद्रव्यापहारकम् {॥ ४:२२॥} \veg\dontdisplaylinenum }%
     \var{{\devanagarifont \numemph\vc\textbf{तं च स्तेनं}\lem \msCa, तञ्च स्तेन \msCb, 
सो ऽपि तेन \msCc\Ed, तं च स्तेयं \msNa, तञ्च तेय \msNb, तञ्च तेन \msNc}}% 
    \var{{\devanagarifont \numnoemph\vd\textbf{॰हारकम्}\lem \msCa\msCb\msNapcorr\msNc\Ed, ॰हारकः \msCc, ॰हारका \msNaacorr ॰हारकाः \msNb}}% 

{\devanagarifont तुलाव्याज-उपायेन परस्वार्थं हरेद्यदि \thinspace{\dandab} \dontdisplaylinenum }%
     \var{{\devanagarifont \numemph\va\textbf{परस्वार्थं}\lem \msCa\msCc\msNa\msNc, परस्वार्थ \msCb\msNb, परस्यार्थं \Ed\oo 
\textbf{हरेद्यदि}\lem \mssALL, हरेद्यति \msCb}}% 

%Verse 4:23

{\devanagarifont चौरलक्षणकाश्चान्ये कूटकापटिका नराः {॥ ४:२३॥} \veg\dontdisplaylinenum }%
     \var{{\devanagarifont \numnoemph\vd\textbf{कूटकापटिका}\lem \msNb, \uncl{कु}टका यटिका \msCa, कूटकायटिका \msCb\msCc\msNaacorr\msNc, 
कूटकार्यटिका \msNapcorr\Ed}}% 
    \paral{{\devanagarifontsmall \vcd {\englishfont \compare\ \UMS\ 8.3cd:} कूटकापटिकाश्चैव सत्यार्जवविवर्जिताः }}

{\devanagarifont दुर्बलार्जवबालेषु च्छद्मना वा बलेन वा \thinspace{\dandab} \dontdisplaylinenum }%
     \var{{\devanagarifont \numemph\va\textbf{॰र्जव॰}\lem \mssALL, ॰जव॰ \msNb}}% 
    \var{{\devanagarifont \numnoemph\vb\textbf{च्छद्मना}\lem \Ed, च्छन्मना \mssCaCbCc\msNa\msNb, च्छत्माना \msNc}}% 

%Verse 4:24

{\devanagarifont अपहृत्य धनं मूढः स चौरश्चोर उच्यते {॥ ४:२४॥} \veg\dontdisplaylinenum }%
     \var{{\devanagarifont \numnoemph\vcd\textbf{मूढः स}\lem \mssALL, मूढास्स \msNb}}% 
    \var{{\devanagarifont \numnoemph\vd\textbf{चौरश्चोर}\lem \msNc, चोरश्चोर \msCa\msCc\msNb\Ed, चौर चोर \msCb, चौरश्चौर \msNa}}% 

{\devanagarifont नास्ति स्तेयसमं पापं नास्त्यधर्मश्च तत्समः \thinspace{\dandab} \dontdisplaylinenum }%
     \var{{\devanagarifont \numemph\va\textbf{स्तेय॰}\lem \msNa\msNc, तेन \msCa, स्तेन॰ \msCb\msCc\msNb}}% 
    \var{{\devanagarifont \numnoemph\vb\textbf{॰समः}\lem \mssALL, ॰समं \msCc}}% 
    \lacuna{\devanagarifontsmall \vo {\englishfont This verse is missing in \Ed.} }%
  
%Verse 4:25

{\devanagarifont नास्ति स्तेनसमाकीर्तिर्नास्ति स्तेनसमो ऽनयः {॥ ४:२५॥} \veg\dontdisplaylinenum }%
     \var{{\devanagarifont \numnoemph\vc\textbf{स्तेन॰}\lem \mssALL, तेन \msCc, स्तेय॰ \msNc\oo 
\textbf{॰समा॰}\lem \msCb\msCc\msNb, ॰समो \msCa\msNa\msNc}}% 
    \var{{\devanagarifont \numnoemph\vd\textbf{स्तेन॰}\lem \mssALL, स्तेय॰ \msNa\msNc}}% 

{\devanagarifont नास्ति स्तेयसमाविद्या नास्ति स्तेनसमः खलः \thinspace{\dandab} \dontdisplaylinenum }%
     \var{{\devanagarifont \numemph\va\textbf{स्तेय॰}\lem \msNa\msNc\Ed, स्तेन॰ \mssCaCbCc\msNb\oo 
\textbf{॰समा}\lem \msCc\msNb, ॰समो \msCa\msCb\msNa\msNc\Ed}}% 
    \var{{\devanagarifont \numnoemph\vb\textbf{स्तेन॰}\lem \mssCaCbCc\msNb, स्तेय॰ \msNa\msNc, तेन \Ed}}% 

%Verse 4:26

{\devanagarifont नास्ति स्तेनसम अज्ञो नास्ति स्तेनसमो ऽलसः {॥ ४:२६॥} \veg\dontdisplaylinenum }%
     \var{{\devanagarifont \numnoemph\vc\textbf{स्तेन॰}\lem \msCa\msCb\msNb\msNc, स्तेय॰ \msCc\msNa\Ed\oo 
\textbf{॰सम}\lem \mssALL, ॰समं \msNb\oo 
\textbf{अज्ञो}\lem \msCb, अज्ञ\lk\ \msCa, अज्ञ \msCc\msNa\msNb\msNc, अज्ञः \Ed}}% 
    \var{{\devanagarifont \numnoemph\vd\textbf{स्तेन॰}\lem \msCa\msCb\msNb, स्तेय॰ \msCc\msNa\msNc, तेन \Ed}}% 

{\devanagarifont नास्ति स्तेनसमो द्वेष्यो नास्ति स्तेनसमो ऽप्रियः \thinspace{\dandab} \dontdisplaylinenum }%
     \var{{\devanagarifont \numemph\va\textbf{स्तेन॰}\lem \msCa\msCb\msNb, स्तेय॰ \msCc\msNa\msNc, तेन \Ed}}% 
    \var{{\devanagarifont \numnoemph\vb\textbf{स्तेन॰}\lem \msNb, स्तेय॰ \mssCaCbCc\msNa\msNc\Ed}}% 

%Verse 4:27

{\devanagarifont नास्ति स्तेयसमं दुःखं नास्ति स्तेयसमो ऽयशः {॥ ४:२७॥} \veg\dontdisplaylinenum }%
     \var{{\devanagarifont \numnoemph\vc\textbf{स्तेय॰}\lem \msCc, स्तेन॰ \msCa\msCb\msNa\msNb, स्तेन्य॰ \msNc, तेन \Ed}}% 
    \var{{\devanagarifont \numnoemph\vd\textbf{स्तेय॰}\lem \msCc\msNc, स्तेन॰ \msCa\msCb\msNa\msNb, तेन \Ed}}% 

\vfill
\pageparbreak
\vers

\nemslokalong


\ujvers\nemsloka {
{\devanagarifont प्रच्छन्नो ह्रियते ऽर्थमन्यपुरुषः प्रत्यक्षमन्यो हरेत् }%
  \dontdisplaylinenum}    \var{{\devanagarifont \numemph\va\textbf{प्रच्छन्नो}\lem \mssALL, प्रस्थन्नो \msCb\oo 
\textbf{ऽर्थमन्यपुरुषः}\lem \msCb\msNc, 
वित्तम् \msCa\msNaacorr\msNb, 
चित्त \msCc, च वित्तमथवा \msNapcorr\Ed\oo 
\textbf{प्रत्यक्षमन्यो}\lem \mssALL, प्रत्यक्षमनो \msCb, 
प्रत्यक्ष्यमन्ये \Ed}}% 


\nemslokab

{\devanagarifont निक्षेपाद्धनहारिणो ऽन्यमधमो व्याजेन चान्यो हरेत्  \danda\dontdisplaylinenum }%
     \var{{\devanagarifont \numnoemph\vb\textbf{निक्षेपाद्धन॰}\lem \msCa\msCb\msNa, निक्षेपा धन॰ \msCc\msNb\msNc, निक्षेपात्रय॰ \Ed\oo 
\textbf{॰हारिणो}\lem \mssALL, ॰हारिण्यो \msCb, ॰हारिणा \msNb\oo 
\textbf{ऽन्यमधमो}\lem \mssALL, ऽन्यमधनो \msCc, ऽन्यविधयो \Ed\oo 
\textbf{चान्यो}\lem \mssALL, चान्या \Ed\oo 
\textbf{हरेत्}\lem \mssALL, हरे \msNa}}% 

\nemslokac

{\devanagarifont अन्ये लेख्यविकल्पनाहृतधना †अन्यो हृताद्वै हृता† }%
  \dontdisplaylinenum    \var{{\devanagarifont \numnoemph\vc\textbf{अन्ये लेख्य॰}\lem \corr, अन्या लेख॰ \msCb\msCc, 
अन्यो ले\uncl{ख्य}॰ \msCa, अन्यो लेख्य॰ \msNa\msNb\msNc, अन्योल्लेख्य \Ed\oo 
\textbf{॰धना अन्यो}\lem \mssALL, ॰धन्यो \msCb\oo 
\textbf{हृताद्वै}\lem \mssALL, हृतद्वै \msNa, हृताद्वे \msNb}}% 

%Verse 4:28


\nemslokad

{\devanagarifont अन्यः क्रीतधनो ऽपरो धयहृत एते जघन्याः स्मृताः {॥ ४:२८॥} \veg\dontdisplaylinenum }%
     \var{{\devanagarifont \numnoemph\vd\textbf{अन्यः क्रीतधनो}\lem \mssALL, अन्य क्रीतधनो \msNc, अनाश्रीतधनं \Ed\oo 
\textbf{ऽपरो धयहृत}\lem \msCa\msCc\msNb, परो धयह्यत \msCb, परो धन\uncl{हृत} \msNa, 
परोधप्रहृत \msNc, मदा ह्यपहृतं \Ed\oo 
\textbf{जघन्याः}\lem \mssALL, जघन्यः \Ed}}% 

\ujvers\nemsloka {
{\devanagarifont स्तेनतुल्य न मूढमस्ति पुरुषो धर्मार्थहीनो ऽधमः }%
  \dontdisplaylinenum}    \var{{\devanagarifont \numemph\va\textbf{स्तेनतुल्य}\lem \msCa\msCb\msNc\ \unmetr, स्तेयस्तुल्य \msCc, 
स्तेयतुल्य \msNa\ \unmetr, तेन तुल्य \msNb\ \unmetr, स्तेनस्तुल्य \Ed}}% 


\nemslokab

{\devanagarifont यावज्जीवति शङ्कया नरपतेः संत्रस्यमानो रटन्  \danda\dontdisplaylinenum }%
     \var{{\devanagarifont \numnoemph\vb\textbf{यावज्जीवति}\lem \mssALL, यावत्तज्जीवति \Ed\oo 
\textbf{॰पतेः}\lem \msCb\msNb\msNc, ॰पतिः \msCa\msCc\msNa\Ed\oo 
\textbf{संत्रस्यमानो रटन्}\lem \mssALL, संत्रास्यमानो शठः \Ed}}% 
    \lacuna{\devanagarifontsmall \vo {\englishfont The lower folio side in exposure 49 in \msNb\ is rather blurred and seems to be partly erased,
                        therefore all the readings in this MS for verses 4.29--46 are rather uncertain,
                        even if not indicated explicitly.} }%
  
\nemslokac

{\devanagarifont प्राप्तःशासन तीव्रसह्यविषमं प्राप्नोति कर्मेरितः }%
  \dontdisplaylinenum    \var{{\devanagarifont \numnoemph\vc\textbf{प्राप्तः॰}\lem \mssALL, प्राप्त॰ \msNa\oo 
\textbf{॰सह्य॰}\lem \mssALL, \lac\  \msNb, ॰सद्य॰ \Ed\oo 
\textbf{॰विषमं}\lem \eme, ॰विषमः \mssCaCbCc\msNa\msNc\Ed, \lac\  \msNb\oo 
\textbf{कर्मेरितः}\lem \mssALL, कर्मे\uncl{रित} \msCa, 
\lac \uncl{रितः} \msNb}}% 

%Verse 4:29


\nemslokad

{\devanagarifont कालेन म्रियते स याति निरयमाक्रन्दमानो भृशम् {॥ ४:२९॥} \veg\dontdisplaylinenum }%
     \var{{\devanagarifont \numnoemph\vd\textbf{निरयमाक्रन्दमानो}\lem \mssCaCbCc\msNa, \uncl{निर}यमाक्रन्दमा\uncl{नो} \msNb, 
निरयं स क्रन्दमानो \msNc, नियममाक्रन्द्रमानो \Ed}}% 

\vfill
\pageparbreak
\vers

\nemslokalong


\ujvers\nemsloka {
{\devanagarifont नीत्वा दुर्गतिकोटिकल्प निरयात्तिर्यत्वमायान्ति ते }%
  \dontdisplaylinenum}    \var{{\devanagarifont \numemph\va\textbf{निरयात्तिर्यत्व॰}\lem \msCb\msNa, निरयान्तिर्यत्व॰ \msCa, निरया तिर्यत्व॰ \msCc, 
नि\uncl{रया$\-$त्तिर्यत्व}॰ \msNb, निरयान्तिर्यक्ष॰ \msNc, निरयान्तिर्यक्त्व॰ \Ed}}% 


\nemslokab

{\devanagarifont तिर्यत्वे च तथैवमेकशतिकं प्रभ्रम्य वर्षार्बुदम्  \danda\dontdisplaylinenum }%
     \var{{\devanagarifont \numnoemph\vb\textbf{तिर्यत्वे}\lem \mssALL, \uncl{तिर्यत्वे} \msNb, तिर्यक्त्वं \Ed\oo 
\textbf{तथैवमेकशतिकं}\lem \msCb, तथैकमेकशतिकं \msCa\msNa\msNc, 
तथैकमेकशतिक \msCc, \uncl{तथै}कमेकशतिकं \msNb, तथैकमेकसकिकं \Ed\oo 
\textbf{॰भ्रम्य॰}\lem \mssALL, ॰भ्राम्य \msNa, \lac  म्य \msNb\oo 
\textbf{वर्षार्बुदम्}\lem \msNcpcorr, वर्षाम्बुदम् \msCa\msCb\msNa\msNb\msNcacorr, वर्षाम्बुदः \msCc\Ed}}% 

\nemslokac

{\devanagarifont मानुष्यं तदवाप्नुवन्ति विपुले दारिद्र्यरोगाकुलं }%
  \dontdisplaylinenum    \var{{\devanagarifont \numnoemph\vc\textbf{मानुष्यं}\lem \mssALL, 
मानुष्य \msCb\ \unmetr, \uncl{मानुष्य} \msNb\ \toplost\oo 
\textbf{विपुले}\lem \mssALL, विपु\uncl{ल} \msNb\ \toplost, विपुलं \Ed\oo 
\textbf{दारिद्र्य॰}\lem \mssALL, \lk रि\lk\ \msNb, दारिध्र॰ \Ed}}% 

%Verse 4:30


\nemslokad

{\devanagarifont तस्माद्दुर्गतिहेतु कर्म सकलं त्यक्त्वा शिवं चाश्रयेत् {॥ ४:३०॥} \veg\dontdisplaylinenum }%
     \var{{\devanagarifont \numnoemph\vd\textbf{तस्माद्दु॰}\lem \mssALL, तस्मा दु॰ \msCc, \uncl{तस्मा दु}॰ \msNb\oo 
\textbf{चाश्रयेत्}\lem \mssALL, चाश्रत् \msNa}}% 

\nemslokanormal



\alalfejezet{यमेष्वानृशंस्यम् (४)}
\vers


{\devanagarifont अष्टमूर्तिशिवद्वेष्टा पितुर्मातुश्च यो द्विषेत् \thinspace{\dandab} \dontdisplaylinenum }%
     \var{{\devanagarifont \numemph\va\textbf{॰शिव॰}\lem \mssALL, ॰शिवं \msNc}}% 

%Verse 4:31

{\devanagarifont गवां वा अतिथेर्द्वेष्टा नृशंसाः पञ्च एव ते {॥ ४:३१॥} \veg\dontdisplaylinenum }%
     \var{{\devanagarifont \numnoemph\vc\textbf{गवां वा}\lem \mssALL, अवाम्वा \msCb, \lk\lk \uncl{म्वा} \msNb\oo 
\textbf{अतिथेर्द्वे॰}\lem \mssALL, अतिथिद्वे॰ \msCc, अतिथे द्वे॰ \msNa}}% 
    \var{{\devanagarifont \numnoemph\vd\textbf{नृशंसाः}\lem \msCa\msCc\msNa\msNb, नृशंसा \msCb\msNc\Ed}}% 

{\devanagarifont अष्टमूर्तिः शिवः साक्षात्पञ्चव्योमसमन्वितः \thinspace{\dandab} \dontdisplaylinenum }%
     \var{{\devanagarifont \numemph\va\textbf{॰मूर्तिः}\lem \mssALL, ॰मूर्ति॰ \Ed}}% 
    \var{{\devanagarifont \numnoemph\vb\textbf{॰न्वितः}\lem \mssALL, ॰न्विताः \msCc\msNb}}% 

%Verse 4:32

{\devanagarifont सूर्यः सोमश्च दीक्षश्च दूषकः स नृशंसकः {॥ ४:३२॥} \veg\dontdisplaylinenum }%
     \var{{\devanagarifont \numnoemph\vc\textbf{सूर्यः}\lem \mssCaCbCc\msNa, \uncl{सूर्य}॰ \msNb\msNc, सूर्य॰ \Ed\oo 
\textbf{दीक्ष॰}\lem \mssALL, \uncl{दी}\lk\ \msNb, दीक्षु॰ \Ed}}% 
    \paral{{\devanagarifontsmall \vo {\englishfont \compare\ \SDHS\ 12.17:}
                 मूर्तयो याः शिवस्याष्टौ तासु निन्दां विवर्जयेत्\thinspace{\devanagarifontsmall ।}
                 गुरोश्च शिवभक्तानां नृपसाधुतपस्विनां\thinspace{\devanagarifontsmall ॥} }}

{\devanagarifont पिताकाशसमो ज्ञेयो जन्मोत्पत्तिकरः पिता \thinspace{\dandab} \dontdisplaylinenum }%
     \var{{\devanagarifont \numemph\vb\textbf{॰करः पिता}\lem \mssALL, ॰करपिताः \msCc, ॰\uncl{करः पिता} \msNb}}% 

%Verse 4:33

{\devanagarifont पितृदैवत†मादिश्चमानृशंस तमन्वितः† {॥ ४:३३॥} \veg\dontdisplaylinenum }%
     \var{{\devanagarifont \numnoemph\vc\textbf{॰दैवत॰}\lem \mssALL, ॰देवत॰ \msCb, \lk वत॰ \msNb}}% 
    \var{{\devanagarifont \numnoemph\vcd\textbf{॰दिश्चमानृशंस तमन्वितः}\lem \msCa\msCb, 
॰दित्यमनृशंस तमन्वितः \msCc\msNb, 
॰दिश्च अनृशंस तमान्वितः \msNa, 
॰दिश्चमनृशंस तमान्वितः \msNc, 
॰दित्यम्मानृशंस ततो ऽन्वितः \Ed}}% 

{\devanagarifont पृथ्व्या गुरुतरी माता को न वन्देत मातरम् \thinspace{\dandab} \dontdisplaylinenum }%
     \var{{\devanagarifont \numemph\va\textbf{पृथ्व्या}\lem \msCa\msCb\msNc, \uncl{पृथ्व्या} \msCc\msNa, पृथ्वी \msNb, 
पृथ्व्यां \Ed}}% 
    \var{{\devanagarifont \numnoemph\vb\textbf{वन्देत}\lem \mssALL, वन्देन वन्देत \msCb, वन्द्येत \msCc}}% 

%Verse 4:34

{\devanagarifont यज्ञदानतपोवेदास्तेन सर्वं कृतं भवेत् {॥ ४:३४॥} \veg\dontdisplaylinenum }%
     \var{{\devanagarifont \numnoemph\vd\textbf{सर्वं}\lem \eme, सर्व \mssCaCbCc\msNa\msNb\msNc\Ed}}% 

{\devanagarifont गावः पवित्रं मङ्गल्यं देवतानां च देवताः \thinspace{\dandab} \dontdisplaylinenum }%
     \var{{\devanagarifont \numemph\va\textbf{पवित्रं}\lem \mssALL, \uncl{पवित्र} \msNb\oo 
\textbf{मङ्गल्यं}\lem \msCa\msCb\msNa, माङ्गल्यं \msCc\msNc\Ed, \uncl{मङ्गल्यं} \msNb\oo 
\textbf{देवताः}\lem \mssCaCbCc\msNc, दैवताः \msNa, \uncl{देवताः} \msNb, देवता \Ed}}% 
    \paral{{\devanagarifontsmall \va {\englishfont \similar\ \VISNUS\ 23.57c:} गावः पवित्रमङ्गल्यं (गोषु लोकाः प्रतिष्ठिता)\oo
                 {\englishfont \compare\ also \MBH\ Suppl. 13.15.33:} गावः पवित्रं परमं गोषु लोकाः प्रतिष्ठिताः 
                 {\englishfont and \AGNIP\ 291.1cd:} गावः पवित्रा माङ्गल्या गोषु लोकाः प्रतिष्ठिताः }}

%Verse 4:35

{\devanagarifont सर्वदेवमया गावस्तस्मादेव न हिंसयेत् {॥ ४:३५॥} \veg\dontdisplaylinenum }%
     \var{{\devanagarifont \numnoemph\vd\textbf{॰स्मादेव}\lem \mssALL, ॰स्मादुव \msCb, ॰स्माद्गावं \Ed}}% 
    \paral{{\devanagarifontsmall \vc {\englishfont = \VDHU\ 3.291.25c} }}

{\devanagarifont जातमात्रस्य लोकस्य गावस्त्राता न संशयः \thinspace{\dandab} \dontdisplaylinenum }%
     \var{{\devanagarifont \numemph\va \lem \msCa\msCc\msNa\msNc\Ed, सतसातस्य \msCbacorr, 
सतसातस्य नोकस्य \msCbpcorr, 
जातमात्र\uncl{स्य लोकस्य} \msNb}}% 

%Verse 4:36

{\devanagarifont घृतं क्षीरं दधि मूत्रं शकृत्कर्षणमेव च {॥ ४:३६॥} \veg\dontdisplaylinenum }%
     \var{{\devanagarifont \numnoemph\vd\textbf{शकृत्क॰}\lem \mssALL, क्षत्क॰ \msCb, \uncl{शकृत्क}॰ \msNb}}% 
    \paral{{\devanagarifontsmall \vo {\englishfont \compare\ \SDHU\ 12.92ff} }}

\ujvers\nemsloka {
{\devanagarifont पञ्चामृतं पञ्चपवित्रपूतं }%
  \dontdisplaylinenum}    \var{{\devanagarifont \numemph\va\textbf{॰पवित्रपूतम्}\lem \msCc\msNa\Ed, ॰पवित्रपूतन \msCa\ \unmetr, 
॰पवित्रं \msCb\ \unmetr, ॰पवित्रपूत \msNb, 
॰पवित्रपूतंनं \msNc\ \unmetr}}% 


\nemslokab

{\devanagarifont ये पञ्चगव्यं पुरुषाः पिबन्ति  \danda\dontdisplaylinenum }%
     \var{{\devanagarifont \numnoemph\vb\textbf{॰गव्यं}\lem \mssALL, ॰गव्या \msCc, ॰\uncl{गव्यां} \msNb\oo 
\textbf{पुरुषाः}\lem  \mssALL, पुरुषा \msCc, पुरुषः \Ed\oo 
\textbf{पिबन्ति}\lem  \mssALL, विवन्ति \msCc}}% 

\nemslokac

{\devanagarifont ते वाजिमेधस्य फलं लभन्ति }%
  \dontdisplaylinenum    \var{{\devanagarifont \numnoemph\vc\textbf{लभन्ति}\lem \mssALL, भवन्ति \msCc}}% 

%Verse 4:37


\nemslokad

{\devanagarifont तदक्षयं स्वर्गमवाप्नुवन्ति {॥ ४:३७॥} \veg\dontdisplaylinenum }%
     \var{{\devanagarifont \numnoemph\vd\textbf{स्वर्ग॰}\lem \mssALL, स्व॰ \msCb}}% 

\ujvers\nemsloka {
{\devanagarifont गोभिर्न तुल्यं धनमस्ति किंचिद् }%
  \dontdisplaylinenum}    \var{{\devanagarifont \numemph\va\textbf{गोभिर्न तु॰}\lem \msNc, न गोभिस्तु॰ \mssCaCbCc\msNa\msNb\ \unmetr, न गावतु॰ \Ed}}% 
    \paral{{\devanagarifontsmall \va {\englishfont = \SDHU\ 12.102d, 103d, 104d; } 
                    {\englishfont \compare\ \MBH\ 13.51.26cd:} गोभिस्तुल्यं न पश्यामि धनं किंचिदिहाच्युत }}


\nemslokab

{\devanagarifont दुह्यन्ति वाह्यन्ति बहिश्चरन्ति  \danda\dontdisplaylinenum }%
 
\vfill
\pageparbreak
\vers

\nemslokac

{\devanagarifont तृणानि भुक्त्वा अमृतं स्रवन्ति }%
  \dontdisplaylinenum
%Verse 4:38


\nemslokad

{\devanagarifont विप्रेषु दत्ताः कुलमुद्धरन्ति {॥ ४:३८॥} \veg\dontdisplaylinenum }%
     \var{{\devanagarifont \numnoemph\vd\textbf{दत्ताः}\lem \mssALL, \uncl{दत्ता} \msCc, दत्ता \Ed}}% 
    \paral{{\devanagarifontsmall \vo {\englishfont \compare\ \SDHU\ 12.92:}
                         तृणानि खादन्ति वसन्त्यरण्ये पिबन्ति तोयान्यपरिग्रहाणि\thinspace{\devanagarifontsmall ।}
                         दुह्यन्ति वाह्यन्ति पुनन्ति पापं गवां रसैर्जीवति जीवलोकः\thinspace{\devanagarifontsmall ॥} }}

\ujvers\nemsloka {
{\devanagarifont गवाह्निकं यश्च करोति नित्यं }%
  \dontdisplaylinenum}    \var{{\devanagarifont \numemph\va\textbf{गवाह्निकं}\lem \mssALL, गवांह्निकं \msCa\oo 
\textbf{यश्च करोति}\lem \mssALL, यः प्रकरोति \Ed}}% 


\nemslokab

{\devanagarifont शुश्रूषणं यः कुरुते गवां तु  \danda\dontdisplaylinenum }%
     \var{{\devanagarifont \numnoemph\vb\textbf{गवां तु}\lem \msCb\msNc, गवान्तु \msCa\msCc\msNa\msNb, गवानाम् \Ed}}% 

\nemslokac

{\devanagarifont अशेषयज्ञतपदानपुण्यं }%
  \dontdisplaylinenum    \var{{\devanagarifont \numnoemph\vc\textbf{॰तप॰}\lem \mssALL, ॰\uncl{तप}॰ \msNb, ॰जप॰ \Ed}}% 

%Verse 4:39


\nemslokad

{\devanagarifont लभत्यसौ तामनृशंसकर्ता {॥ ४:३९॥} \veg\dontdisplaylinenum }%
     \var{{\devanagarifont \numnoemph\vd \lem \eme, 
लभत्यसौ तमनृशंसकर्ता \msCb\msNa\msNb\msNc, 
लभत्यसौ भमनृशंसकर्त्ता \msCa, 
लभत्यसौ तमनृतं स कर्त्ता \msCc, 
भवत्यसौ धर्ममशेषकर्ता \Ed}}% 

\vers


{\devanagarifont अतिथिं यो ऽनुगच्छेत अतिथिं यो ऽनुमन्यते \thinspace{\dandab} \dontdisplaylinenum }%
 
%Verse 4:40

{\devanagarifont अतिथिं यो ऽनुपूज्येत अतिथिं यः प्रशंसते {॥ ४:४०॥} \veg\dontdisplaylinenum }%
     \var{{\devanagarifont \numemph\vd\textbf{प्रशंसते}\lem \mssALL, प्रशंस्यते \msCc}}% 

{\devanagarifont अतिथिं यो न पीड्येत अतिथिं यो न दुष्यति \thinspace{\dandab} \dontdisplaylinenum }%
     \var{{\devanagarifont \numemph\va\textbf{न पीड्येत}\lem \msCa\msCb\msNa\Ed, न गच्छेत ({\englishfont eyeskip to 4.40c}) \msCc, 
\uncl{न पी}\lk\lk\ \msNb, निपीड्येत \msNc}}% 
    \var{{\devanagarifont \numnoemph\vb\textbf{अतिथिं}\lem \mssALL, अतिं \msCc, \lk\lk\lk\ \msNb\oo 
\textbf{न दुष्यति}\lem \mssALL, नुदुष्यति \msCb, \lk\ दुष्यति \msNb}}% 

{\devanagarifont अतिथिप्रियकर्ता यः अतिथेः परिचारकः  \danda\dontdisplaylinenum }%
     \var{{\devanagarifont \numnoemph\vc\textbf{अतिथि॰}\lem \msCa\msNa, अतिथिं \msCb\msCc\msNc\Ed, अति\uncl{थिं} \msNb\oo 
\textbf{॰प्रिय॰}\lem \mssALL, प्रियः \msCc\oo 
\textbf{यः}\lem \mssALL, यर् \msCa, य \msNa}}% 

%Verse 4:41

{\devanagarifont अतिथेः कृतसंतोषस्तस्य पुण्यमनन्तकम् {॥ ४:४१॥} \veg\dontdisplaylinenum }%
     \var{{\devanagarifont \numnoemph\ve\textbf{अतिथेः}\lem \msCb\msCc\msNc, अतिथि॰ \msCa\msNa\msNb, अतिथिं \Ed}}% 
    \var{{\devanagarifont \numnoemph\vef\textbf{॰संतोषस्तस्य}\lem \mssALL, ॰संता यस्य \msCb}}% 
    \var{{\devanagarifont \numnoemph\vf\textbf{पुण्य॰}\lem \mssALL, पून॰ \msNc}}% 

{\devanagarifont आसनेनार्घपात्रेण पादशौचजलेन च \thinspace{\dandab} \dontdisplaylinenum }%
     \var{{\devanagarifont \numemph\va\textbf{॰आर्घ॰}\lem \mssALL, ॰आर्ध्य॰ \Ed\oo 
\textbf{॰पात्रेण}\lem \conj, ॰पाद्येन \mssCaCbCc\msNa\msNb\msNc\Ed}}% 

%Verse 4:42

{\devanagarifont अन्नवस्त्रप्रदानैर्वा सर्वं वापि निवेदयेत् {॥ ४:४२॥} \veg\dontdisplaylinenum }%
     \var{{\devanagarifont \numnoemph\vc\textbf{अन्नव॰}\lem \mssALL, अन्नम्व॰ \msCc, \uncl{अन्न}व॰ \msNb}}% 
    \var{{\devanagarifont \numnoemph\vd\textbf{निवेदयेत्}\lem \mssALL, प्रदापयेत् \Ed}}% 

{\devanagarifont पुत्रदारात्मनो वापि यो ऽतिथिमनुपूजयेत् \thinspace{\dandab} \dontdisplaylinenum }%
     \var{{\devanagarifont \numemph\va\textbf{॰दारात्मनो}\lem \mssALL, ॰\uncl{दारा}त्मनो \msCa, ॰दारात्मको \Ed}}% 
    \var{{\devanagarifont \numnoemph\vb\textbf{॰पूजयेत्}\lem \msCa\msNa\Ed, ॰पूज्यते \msCb\msCc\msNb, ॰पूजते \msNc}}% 

%Verse 4:43

{\devanagarifont श्रद्धया चाविकल्पेन अक्लीबमानसेन च {॥ ४:४३॥} \veg\dontdisplaylinenum }%
     \var{{\devanagarifont \numnoemph\vc\textbf{श्रद्धया}\lem \mssALL, श्रद्धाया \msCc\oo 
\textbf{चाविकल्पेन}\lem \mssALL, चापि कल्पेन \msCa}}% 

{\devanagarifont न पृच्छेद्गोत्रचरणं स्वाध्यायं देशजन्मनी \thinspace{\dandab} \dontdisplaylinenum }%
     \var{{\devanagarifont \numemph\va\textbf{॰चरणं}\lem \mssALL, ॰प्रवरं \Ed}}% 
    \var{{\devanagarifont \numnoemph\vb\textbf{देशजन्मनी}\lem \mssALL, देशजन्मना \msCa}}% 
    \paral{{\devanagarifontsmall  {\englishfont \vab = \UUMS\ 10.7ab = \UMS\ 6.11ab \similar\ \MBH\ 13.62.18ab:
                 }न पृच्छेद्गोत्रचरणं स्वाध्यायं देशमेव वा }}

%Verse 4:44

{\devanagarifont चिन्तयेन्मनसा भक्त्या धर्मः स्वयमिहागतः {॥ ४:४४॥} \veg\dontdisplaylinenum }%
     \var{{\devanagarifont \numnoemph\vc\textbf{चिन्तयेन्म॰}\lem \mssALL, चित्तयेत्म॰ \msCb, चिन्तयेत्म॰ \msNc}}% 
    \var{{\devanagarifont \numnoemph\vd\textbf{॰गतः}\lem \mssALL, ॰गताः \msCc, ग\uncl{तम्} \msNb}}% 
    \paral{{\devanagarifontsmall \vcd {\englishfont \compare\ 12.37cd: }द्विजरूपधरो धर्मः स्वयमेव इहागतः }}

{\devanagarifont अश्वमेधसहस्राणि राजसूयशतानि च \thinspace{\dandab} \dontdisplaylinenum }%
     \var{{\devanagarifont \numemph\vb\textbf{॰सूय॰}\lem \msCa\msNa\msNc\Ed, ॰सूर्य॰ \msCb\msCc, ॰सू\uncl{र्य}॰ \msNb}}% 

%Verse 4:45

{\devanagarifont पुण्डरीकसहस्रं च सर्वतीर्थतपःफलम् {॥ ४:४५॥} \veg\dontdisplaylinenum }%
     \var{{\devanagarifont \numnoemph\vd\textbf{॰तपः॰}\lem \mssALL, ॰तप॰ \msNc\ \unmetr}}% 

{\devanagarifont अतिथिर्यस्य तुष्येत नृशंसमतमुत्सृजेत् \thinspace{\dandab} \dontdisplaylinenum }%
     \var{{\devanagarifont \numemph\vb \lem \msCa\msNa\msNc, नृशंसमत उत्सृजेत् \msCb, 
नृशंसकमममुत्सृजेत् \msCc, नृससमतमुत्सृजेत् \msNb, न संशय समश्नुते \Ed}}% 

%Verse 4:46

{\devanagarifont स तस्य सकलं पुण्यं प्राप्नुयान्नात्र संशयः {॥ ४:४६॥} \veg\dontdisplaylinenum }%
 
{\devanagarifont †न गतिमतिथिज्ञस्य† गतिमाप्नोति कर्हचित् \thinspace{\dandab} \dontdisplaylinenum }%
     \var{{\devanagarifont \numemph\va\textbf{न गतिम॰}\lem \msCa\msCb\msNb\msNc, न तिथिम॰ \msCc\Ed, न गति ना॰ \msNa}}% 
    \var{{\devanagarifont \numnoemph\vb\textbf{कर्हचित्}\lem \mssALL, कर्हिचित् \msCa\Ed}}% 

%Verse 4:47

{\devanagarifont तस्मादतिथिमायान्तमभिगच्छेत्कृताञ्जलिः {॥ ४:४७॥} \veg\dontdisplaylinenum }%
     \var{{\devanagarifont \numnoemph\vc\textbf{॰यान्त॰}\lem \mssALL, ॰यान्ति॰ \msCc}}% 
    \paral{{\devanagarifontsmall \vcd {\englishfont = \VAYUP\ 2.17.8 = \BRAHMANDAPUR\ 2.15.8 
                         \similar\ \SDHU\ 4.44ab:}
                         तस्मादतिथिमायान्तमनुगच्छेत्कृताञ्जलिः }}

{\devanagarifont सक्तुप्रस्थेन चैकेन यज्ञ आसीन्महाद्भुतः \thinspace{\dandab} \dontdisplaylinenum }%
     \var{{\devanagarifont \numemph\va\textbf{सक्तु॰}\lem \eme, शन्कु॰ \msCa\msCb, शंक्तु॰ \msCc, शक्तु॰ \msNa\msNc, शक्थु॰ \msNb, शक्ति॰ \Ed\oo 
\textbf{चैकेन}\lem \mssALL, चेकेन \msNc}}% 
    \var{{\devanagarifont \numnoemph\vb\textbf{आसीन्महाद्भुतः}\lem \corr, आसीन्महद्भुतः \msCa\msCb\msNa\msNb, आसी महद्भुतः \msCc, 
आसीत्महाद्भुतः \msNc, आसीन्महद्भुतम् \Ed}}% 

%Verse 4:48

{\devanagarifont अतिथिप्राप्तदानेन स्वशरीरं दिवं गतम् {॥ ४:४८॥} \veg\dontdisplaylinenum }%
     \var{{\devanagarifont \numnoemph\vc\textbf{॰दानेन}\lem \mssALL, ॰प्रादानेन \msCc}}% 
    \var{{\devanagarifont \numnoemph\vd\textbf{स्व॰}\lem \mssALL, \uncl{स}॰ \msNc, स॰ \Ed\oo 
\textbf{॰गतम्}\lem \mssALL, ॰गतः \msCc}}% 

\vfill
\pageparbreak
\vers

{\devanagarifont नकुलेन पुराधीतं विस्तरेण द्विजोत्तम \thinspace{\dandab} \dontdisplaylinenum }%
     \var{{\devanagarifont \numemph\vb\textbf{॰त्तम}\lem \mssALL, ॰त्तमम् \msCc, ॰त्तमः \Ed}}% 

%Verse 4:49

{\devanagarifont विदितं च त्वया पूर्वं प्रस्थवार्त्ता च कीर्तिता {॥ ४:४९॥} \veg\dontdisplaylinenum }%
     \var{{\devanagarifont \numnoemph\vd\textbf{कीर्तिता}\lem \mssALL, कीर्तितम् \msCc, कीर्तिताः \Ed}}% 


\alalfejezet{यमेषु दमः (५)}
{\devanagarifont दम एव मनुष्याणां धर्मसारसमुच्चयः \thinspace{\dandab} \dontdisplaylinenum }%
     \var{{\devanagarifont \numemph\vb\textbf{धर्मसार॰}\lem \eme, धर्मः सार॰ \mssCaCbCc\msNa\msNb\msNc, धर्मभार॰ \Ed}}% 
    \paral{{\devanagarifontsmall \vb {\englishfont \compare, e.g., \MBH\ Suppl. 14.4.2477: }श्रोतुमिच्छामि कार्त्स्न्येन धर्मसारसमुच्चयम् }}

%Verse 4:50

{\devanagarifont दमो धर्मो दमः स्वर्गो दमः कीर्तिर्दमः सुखम् {॥ ४:५०॥} \veg\dontdisplaylinenum }%
     \var{{\devanagarifont \numnoemph\vc\textbf{स्वर्गो}\lem \mssALL, स्वर्ग \msCc}}% 
    \var{{\devanagarifont \numnoemph\vd\textbf{कीर्तिर्द॰}\lem \msCa\msCb\msNb\Ed, कीर्ति द॰ \msCc\msNa\msNc}}% 

{\devanagarifont दमो यज्ञो दमस्तीर्थं दमः पुण्यं दमस्तपः \thinspace{\dandab} \dontdisplaylinenum }%
     \var{{\devanagarifont \numemph\va\textbf{दमस्ती॰}\lem \mssALL, दम ती॰ \msCb}}% 

%Verse 4:51

{\devanagarifont दमहीनमधर्मश्च दमः कामकुलप्रदः {॥ ४:५१॥} \veg\dontdisplaylinenum }%
     \var{{\devanagarifont \numnoemph\vd\textbf{दमः}\lem \mssALL, दम \msCc, दमं \Ed\oo 
\textbf{काम॰}\lem \mssALL, कामं \msNc}}% 

{\devanagarifont निर्दमः करि मीनश्च पतङ्गभ्रमरमृगाः \thinspace{\dandab} \dontdisplaylinenum }%
     \var{{\devanagarifont \numemph\va\textbf{॰दमः}\lem \mssALL, ॰दम \msCc}}% 
    \var{{\devanagarifont \numnoemph\vb\textbf{॰भ्रमर॰}\lem \mssALL, ॰भ्रम\uncl{रा}॰ \msNc}}% 

%Verse 4:52

{\devanagarifont त्वग्जिह्वा च तथा घ्राणा चक्षुः श्रवणमिन्द्रियाः {॥ ४:५२॥} \veg\dontdisplaylinenum }%
     \var{{\devanagarifont \numnoemph\vc\textbf{घ्राणा}\lem \mssALL, घ्राणं \msCb, घ्राण \msCc}}% 
    \var{{\devanagarifont \numnoemph\vd\textbf{॰न्द्रियाः}\lem \mssALL, ॰न्द्रियः \Ed}}% 

{\devanagarifont दुर्जयेन्द्रियमेकैकं सर्वे प्राणहराः स्मृताः \thinspace{\dandab} \dontdisplaylinenum }%
     \var{{\devanagarifont \numemph\vb\textbf{सर्वे}\lem \mssALL, सर्व॰ \msCb\oo 
\textbf{॰हराः}\lem \mssALL, ॰हरा \Ed}}% 

%Verse 4:53

{\devanagarifont दमं यो जयते ऽसम्यग्निर्दमो निधनं व्रजेत् {॥ ४:५३॥} \veg\dontdisplaylinenum }%
     \var{{\devanagarifont \numnoemph\vd\textbf{व्रजेत्}\lem \mssALL, व्रजे\lac\  \msCa}}% 

{\devanagarifont मृगे श्रोत्रवशान्मृत्युः पतङ्गाश्चक्षुषोर्मृताः \thinspace{\dandab} \dontdisplaylinenum }%
     \var{{\devanagarifont \numemph\va\textbf{मृगे}\lem \mssALL, मृगो \msNb\Ed\oo 
\textbf{श्रोत्र॰}\lem \mssALL, शोत्र॰ \msCc, श्रोत॰ \msNc\oo 
\textbf{॰वशा॰}\lem \mssALL, ॰वचशा॰ \msCb}}% 
    \var{{\devanagarifont \numnoemph\vb\textbf{पतङ्गाश्च॰}\lem \mssALL, पतङ्गा च॰ \Ed\oo 
\textbf{॰षोर्मृताः}\lem \mssALL, ॰सो मृताः \msCc, ॰षो मृताः \msNc}}% 

%Verse 4:54

{\devanagarifont घ्राणया भ्रमरो नष्टो नष्टो मीनश्च जिह्वया {॥ ४:५४॥} \veg\dontdisplaylinenum }%
     \var{{\devanagarifont \numnoemph\vc\textbf{घ्राणया}\lem \mssALL, घ्रातया \msCb}}% 
    \var{{\devanagarifont \numnoemph\vcd\textbf{नष्टो नष्टो}\lem \mssALL, नष्टो \msCb}}% 
    \paral{{\devanagarifontsmall \vo {\englishfont \compare\ \BUDDHACARITA\ 11.35:} 
                गीतैर्ह्रियन्ते हि मृगा वधाय रूपार्थमग्नौ शलभाः पतन्ति\thinspace{\devanagarifontsmall ।} 
                मत्स्यो गिरत्यायसमामिषार्थी तस्मादनर्थं विषयाः फलन्ति\thinspace{\devanagarifontsmall ॥} }}

\vfill
\pageparbreak
\vers

{\devanagarifont स्पर्शेन च करी नष्टो बन्धनावासदुःसहः \thinspace{\dandab} \dontdisplaylinenum }%
     \var{{\devanagarifont \numemph\vb\textbf{॰सदुःसहः}\lem \mssALL, ॰सदुःसह \msCb, ॰सुदुस्सहः \msNb}}% 

%Verse 4:55

{\devanagarifont किं पुनः पञ्चभुक्तानां मृत्युस्तेभ्यः किमद्भुतम् {॥ ४:५५॥} \veg\dontdisplaylinenum }%
     \var{{\devanagarifont \numnoemph\vc\textbf{पुनः}\lem \mssALL, पुन \msCaacorr}}% 
    \var{{\devanagarifont \numnoemph\vd\textbf{तेभ्यः}\lem \mssALL, तेभ्य \Ed}}% 

{\devanagarifont पुरूरवो ऽतिलोभेन अतिकामेन दण्डकः \thinspace{\dandab} \dontdisplaylinenum }%
     \var{{\devanagarifont \numemph\va\textbf{पुरूरवो}\lem \mssALL, पुरोरवे \msCc, पुरुरवा॰ \Ed}}% 
    \var{{\devanagarifont \numnoemph\vab\textbf{तिलोभेन अतिकामेन}\lem \mssALL, तिकामेन अतिलोभेन \Ed}}% 
    \var{{\devanagarifont \numnoemph\vb\textbf{दण्डकः}\lem \mssALL, पुण्डकः \Ed}}% 

%Verse 4:56

{\devanagarifont सागराश्चातिदर्पेण अतिमानेन रावणः {॥ ४:५६॥} \veg\dontdisplaylinenum }%
     \var{{\devanagarifont \numnoemph\vc\textbf{सागरा॰}\lem \eme, सगर॰ \msCa\msCb\msNa\msNb\msNc\Ed, सागर॰ \msCc}}% 
    \paral{{\devanagarifontsmall \vd {\englishfont \compare\ \MAHASUBHS\ 563cd:}
                         विनष्टो रावणो लौल्यादति सर्वत्र वर्जयेत् }}

{\devanagarifont अतिक्रोधेन सौदास अतिपानेन यादवाः \thinspace{\dandab} \dontdisplaylinenum }%
     \var{{\devanagarifont \numemph\vb\textbf{अतिपानेन}\lem \mssALL, अतिपापेन \Ed}}% 

%Verse 4:57

{\devanagarifont अतितृष्णाच्च मान्धाता नहुषो द्विजवज्ञया {॥ ४:५७॥} \veg\dontdisplaylinenum }%
     \var{{\devanagarifont \numnoemph\vc \lem \conj, 
अतितृष्णा च मान्दातो \msCa, 
अतितृष्णा च मान्धातो \msCb\msCc\msNa\msNc, 
अतितृष्णा च मन्धातो \msNb, 
अतितृष्णा च मानाच्च च \Ed}}% 
    \var{{\devanagarifont \numnoemph\vd\textbf{नहुषो}\lem \mssALL, नघुषो \msNb}}% 

{\devanagarifont अतिदानाद्बलिर्नष्ट अतिशौर्येण अर्जुनः \thinspace{\dandab} \dontdisplaylinenum }%
     \var{{\devanagarifont \numemph\va\textbf{॰र्नष्ट}\lem \mssALL, ॰र्नष्टो \msCb, नष्टो \msCc}}% 
    \paral{{\devanagarifontsmall \va {\englishfont \compare\ \MAHASUBHS\ 563ab:}
                         अतिदानाद्बलिर्बद्धो नष्टो मानात्सुयोधनः }}

%Verse 4:58

{\devanagarifont अतिद्यूतान्नलो राजा नृगो गोहरणेन तु {॥ ४:५८॥} \veg\dontdisplaylinenum }%
     \var{{\devanagarifont \numnoemph\vc\textbf{अतिद्यूतान्नलो}\lem \msCa\msCc\msNb\msNc, अतिद्यूतान्नरो \msCb\msNa, अतिख्यातान्नलो \Ed}}% 
    \var{{\devanagarifont \numnoemph\vd\textbf{नृगो गो॰}\lem \Ed, नृगङ्गो॰ \msCa\msCc\msNb\msNc, नृगं गो॰ \msCb\msNa}}% 
    \lacuna{\devanagarifontsmall \vo {\englishfont After this verse, \Ed\ adds:} 
                        तस्माद्दम सदा स रक्षेत् अति सर्वत्र वर्जयेत्   
                {\englishfont (understand:} तस्माद्दमं सदा रक्षेत् अति सर्वत्र वर्जयेत् {\englishfont )};
                {\englishfont \compare\ \MAHASUBHS\ 563cd:}
                        विनष्टो रावणो लौल्यादति सर्वत्र वर्जयेत्  }%
  
\ujvers\nemsloka {
{\devanagarifont दमेन हीनः पुरुषो द्विजेन्द्र }%
  \dontdisplaylinenum}    \var{{\devanagarifont \numemph\va \lem \mssALL, 
हीन पुरुषो द्विजेन्द्र \msNb, हीनं पुरुषं द्विजेन्द्रः \Ed}}% 


\nemslokab

{\devanagarifont स्वर्गं च मोक्षं च सुखं च नास्ति  \danda\dontdisplaylinenum }%
 
\nemslokac

{\devanagarifont विज्ञानधर्मकुलकीर्तिनाश }%
  \dontdisplaylinenum    \var{{\devanagarifont \numnoemph\vc\textbf{॰नाश}\lem \msCb, ॰नाशो \Ed ॰नाम \msCa\msCc\msNa, ॰नश्च \msNb, ॰नागा \msNc}}% 

%Verse 4:59


\nemslokad

{\devanagarifont भवन्ति विप्र दमया विहीनाः {॥ ४:५९॥} \veg\dontdisplaylinenum }%
     \var{{\devanagarifont \numnoemph\vd\textbf{विप्र}\lem \mssALL, विप्रा \msNapcorr\msNc\oo 
\textbf{दमया}\lem \mssALL, दया \msCbacorr}}% 

\vfill
\pageparbreak
\vers


\alalfejezet{यमेषु घृणा (६)}
\vers


{\devanagarifont निर्घृणो न परत्रास्ति निर्घृणो न इहास्ति वै \thinspace{\dandab} \dontdisplaylinenum }%
     \var{{\devanagarifont \numemph\va\textbf{निर्घृणो}\lem \msCa\msCb\msNb, निघृणो \msCc\msNc, निर्घृण \msNaacorr, 
निर्घृ\uncl{णे} \msNapcorr, निर्घृणे \Ed}}% 
    \var{{\devanagarifont \numnoemph\vb\textbf{निर्घृणो}\lem \msCa\msCb\msNaacorr\msNb, निघृणो \msCc\msNc, निर्घृणे \msNapcorr\Ed}}% 

%Verse 4:60

{\devanagarifont निर्घृणे न च धर्मो ऽस्ति निर्घृणे न तपो ऽस्ति वै {॥ ४:६०॥} \veg\dontdisplaylinenum }%
     \var{{\devanagarifont \numnoemph\vc\textbf{निर्घृणे}\lem \msCa\msCb\msNb\Ed, निघृणे \msCc\msNa\msNc}}% 
    \var{{\devanagarifont \numnoemph\vd\textbf{निर्घृणे}\lem \mssALL, निघृणे \msCc\msNc}}% 

{\devanagarifont परस्त्रीषु परार्थेषु परजीवापकर्षणे \thinspace{\dandab} \dontdisplaylinenum }%
     \var{{\devanagarifont \numemph\vb\textbf{॰जीवापकर्षणे}\lem \mssALL, ॰जीवापर्कणे \msCb, ॰जीवोपकर्षणे \Ed}}% 

%Verse 4:61

{\devanagarifont परनिन्दापरान्नेषु घृणां पञ्चसु कारयेत् {॥ ४:६१॥} \veg\dontdisplaylinenum }%
     \var{{\devanagarifont \numnoemph\vc\textbf{परनिन्दा॰}\lem \mssALL, परनि$\-$न्द\lk\ \msCa\oo 
\textbf{॰परान्नेषु}\lem \mssALL, ॰परांनेषु \msNb}}% 
    \var{{\devanagarifont \numnoemph\vd\textbf{घृणां}\lem \msCa\msCb\msNa\msNc, घृणा \msCc\msNb\Ed}}% 

{\devanagarifont परस्त्री शृणु विप्रेन्द्र घृणीकार्या सदा बुधैः \thinspace{\dandab} \dontdisplaylinenum }%
     \var{{\devanagarifont \numemph\va\textbf{घृणी॰}\lem \mssALL, घृणा \msCb}}% 

%Verse 4:62

{\devanagarifont राज्ञी विप्री परिव्राजा स्वयोनिपरयोनिषु {॥ ४:६२॥} \veg\dontdisplaylinenum }%
     \var{{\devanagarifont \numnoemph\vc\textbf{॰व्राजा}\lem \mssCaCbCc\msNc, ॰व्राजी \msNa\msNb, ॰व्राज्या \Ed}}% 
    \var{{\devanagarifont \numnoemph\vd\textbf{॰पर॰}\lem \mssALL, ॰पशु॰ \msNb}}% 

{\devanagarifont परार्थे शृणु भूयो ऽन्य अन्यायार्थमुपार्जनम् \thinspace{\dandab} \dontdisplaylinenum }%
     \var{{\devanagarifont \numemph\vb\textbf{अन्याया॰}\lem \mssALL, अन्यया॰ \msNb\oo 
\textbf{॰र्जनम्}\lem \mssALL, ॰र्ज्जवम् \msNb}}% 
    \paral{{\devanagarifontsmall \vb {\englishfont \compare\ \BHG\ 16.12:}
                 आशापाशशतैर्बद्धाः कामक्रोधपरायणाः\thinspace{\devanagarifontsmall ।}
                 ईहन्ते कामभोगार्थमन्यायेनार्थसंचयान्\thinspace{\devanagarifontsmall ॥} }}

%Verse 4:63

{\devanagarifont आढप्रस्थतुलाव्याजैः परार्थं यो ऽपकर्षति {॥ ४:६३॥} \veg\dontdisplaylinenum }%
     \var{{\devanagarifont \numnoemph\vc\textbf{॰तुला॰}\lem \mssALL, ॰तुल॰ \msNb}}% 
    \var{{\devanagarifont \numnoemph\vd\textbf{॰र्थं}\lem \msCa\msCb\msNa\Ed, ॰र्थ \msCc, ॰\uncl{र्थ} \msNb, ॰र्थे \msNc}}% 

{\devanagarifont जीवापकर्षणे विप्र घृणीकुर्वीत पण्डितः \thinspace{\dandab} \dontdisplaylinenum }%
     \var{{\devanagarifont \numemph\va\textbf{विप्र}\lem \mssALL, वि\uncl{प्र} \msCa, विप्रे \msCc}}% 
    \var{{\devanagarifont \numnoemph\vb\textbf{घृणी॰}\lem \mssALL, घृणां \Ed}}% 

%Verse 4:64

{\devanagarifont वनजावनजा जीवा विलगाश्चरणाचराः {॥ ४:६४॥} \veg\dontdisplaylinenum }%
     \var{{\devanagarifont \numnoemph\vc\textbf{वनजावनजा}\lem \msCa\msCc\msNa\msNb\Ed, 
वनजाव\lk जा \msCbacorr, वनजा व\uncl{नि}जा \msCbpcorr, वनज विनजा \msNc}}% 
    \var{{\devanagarifont \numnoemph\vd \lem \corr, 
विलगाचरणाचराः \msCa\msCb\msNc, विलगोचरगोचरः \msCc\Ed, विलगोचरगोचराः \msNa, 
\uncl{विलगाचर}णाचराः \msNb}}% 

\vfill
\pageparbreak
\vers

{\devanagarifont परनिन्दा च का विप्र शृणु वक्ष्ये समासतः \thinspace{\dandab} \dontdisplaylinenum }%
     \var{{\devanagarifont \numemph\vb\textbf{वक्ष्ये}\lem \mssALL, वक्ष्या \Ed}}% 

%Verse 4:65

{\devanagarifont देवानां ब्राह्मणानां च गुरुमातातिथिद्विषः {॥ ४:६५॥} \veg\dontdisplaylinenum }%
     \lacuna{\devanagarifontsmall \vcd {\englishfont These two pādas are illegible in \msNb} }%
  
{\devanagarifont परान्नेषु घृणा कार्या अभोज्येषु च भोजनम् \thinspace{\dandab} \dontdisplaylinenum }%
     \var{{\devanagarifont \numemph\vb\textbf{अभोज्येषु}\lem \mssALL, अभोज्ये \msCb}}% 

%Verse 4:66

{\devanagarifont सूतके मृतके शौण्डे वर्णभ्रष्टकुले नटे {॥ ४:६६॥} \veg\dontdisplaylinenum }%
     \var{{\devanagarifont \numnoemph\vc\textbf{शौण्डे}\lem \msNa, सौण्ड्ये \msCa\msCc\msNc, शोण्ड्ये \msCb, \uncl{सौण्डे} \msNb, सौण्डो \Ed}}% 
    \lacuna{\devanagarifontsmall \vo {\englishfont This verse is mostly illegible in \msNb} }%
  
\nemslokalong


\ujvers\nemsloka {
{\devanagarifont एते पञ्चघृणासु सक्तपुरुषाः स्वर्गार्थमोक्षार्थिनो }%
  \dontdisplaylinenum}    \var{{\devanagarifont \numemph\va\textbf{॰पुरुषाः}\lem \msNc, ॰पुरुषः \mssCaCbCc\msNa\msNb\Ed\oo 
\textbf{॰र्थिनो}\lem \eme, ॰र्थिनः \msNcpcorr, ॰र्थिनां \mssCaCbCc\msNa\msNb\Ed, ॰र्थिना \msNcacorr}}% 


\nemslokab

{\devanagarifont लोके ऽनिन्दनमाप्नुवन्ति सततं कीर्तिर्यशोऽलंकृतम्  \danda\dontdisplaylinenum }%
     \var{{\devanagarifont \numnoemph\vb\textbf{ऽनिन्दनमाप्नुवन्ति}\lem \mssALL, 
ऽनिन्दनवाप्नुवन्ति \msCc, नन्दनवायुवान्ति \Ed}}% 

\nemslokac

{\devanagarifont प्रज्ञाबोधश्रुतिं स्मृतिं च लभते मानं च नित्यं लभेद् }%
  \dontdisplaylinenum    \var{{\devanagarifont \numnoemph\vc\textbf{॰श्रुतिं}\lem \msNc, ॰श्रुति॰ \mssCaCbCc\msNa\msNb\Ed\oo 
\textbf{नित्यं}\lem \mssALL, नित्य \msCb}}% 

%Verse 4:67


\nemslokad

{\devanagarifont दाक्षिण्यं सभवेत्स आयुष परं प्राप्नोति निःसंशयः {॥ ४:६७॥} \veg\dontdisplaylinenum }%
     \var{{\devanagarifont \numnoemph\vd\textbf{स आयुष}\lem \eme, समायुष \mssCaCbCc\msNc, समायुषः \msNa\ \unmetr, 
\uncl{समायुष} \msNb, स मानुष \Ed\oo 
\textbf{निःसंशयः}\lem \mssALL, निसंशयः \msNa}}% 

\nemslokanormal



\alalfejezet{यमेषु पञ्चविधो धन्यः (७)}
\vers


{\devanagarifont चतुर्मौनं चतुःशत्रुश्चतुरायतनं तथा \thinspace{\dandab} \dontdisplaylinenum }%
     \var{{\devanagarifont \numemph\va\textbf{चतुर्मौनं च॰}\lem \corr, चतुर्मौनश्च॰ \msCa\msCb\msNa\msNc\Ed, चतुर्मोणश्च॰ \msCc, 
\uncl{चतुर्मौनश्च}॰ \msNb}}% 
    \var{{\devanagarifont \numnoemph\vab\textbf{॰तुःशत्रुश्च॰}\lem \mssALL, 
॰तुशत्रु च॰ \msCc, ॰तुःशत्रु च॰ \Ed}}% 
    \var{{\devanagarifont \numnoemph\vb\textbf{॰तुरायतनं}\lem \mssALL, ॰\uncl{तु}रायतनं \msCa, 
॰\uncl{तुरायतनम्} \msNb}}% 

%Verse 4:68

{\devanagarifont चतुर्ध्यानं चतुष्पादं पञ्चधन्यविधोच्यते {॥ ४:६८॥} \veg\dontdisplaylinenum }%
     \var{{\devanagarifont \numnoemph\vc\textbf{॰पादं}\lem \mssALL, ॰पादः \msNa, \lk\lk\ \msNb}}% 
    \var{{\devanagarifont \numnoemph\vd\textbf{पञ्चधन्य॰}\lem \mssALL, धन्यपञ्च॰ \Ed}}% 

\vfill
\pageparbreak
\vers

{\devanagarifont चतुर्मौनस्य वक्ष्यामि शृणुष्वावहितो भव \thinspace{\dandab} \dontdisplaylinenum }%
     \var{{\devanagarifont \numemph\va\textbf{॰मौनस्य}\lem \mssALL, ॰मोनस्य \msCb}}% 

%Verse 4:69

{\devanagarifont पारुष्यपिशुनामिथ्या सम्भिन्नानि च वर्जयेत् {॥ ४:६९॥} \veg\dontdisplaylinenum }%
     \var{{\devanagarifont \numnoemph\vc\textbf{पारुष्य॰}\lem \mssALL, पारुष्यं \msNa\oo 
\textbf{॰पिशुना॰}\lem \mssALL, ॰पिण्डाना॰ \Ed}}% 
    \paral{{\devanagarifontsmall \vcd {\englishfont \compare\ \DIVYAV\ 186.21:}
                     आर्य, किमेभिः कर्म कृतम्येनैवंविधानि दुःखानि प्रत्यनुभवन्तीति? 
                     स कथयति\thinspace{\devanagarifontsmall ।} एते प्राणातिपातिका अदत्तादायिकाः काममिथ्याचारिका मृषावादिकाः पैशुनिकाः पारुषिकाः 
                     संभिन्नप्रलापिका अभिध्यालवो व्यापन्नचित्ता मिथ्यादृष्टिकाः\thinspace{\devanagarifontsmall ।};
                     {\englishfont \compare\ \DHARMP\ 1.31cd--32ab:}
                         मिथ्या पिशुनसम्भिन्नपारुष्यवचनानि च\thinspace{\devanagarifontsmall ॥}
                         जल्पतः सम्भवन्त्येते तस्मान्मौनं प्रशस्यते\thinspace{\devanagarifontsmall ।} }}

{\devanagarifont कामः क्रोधश्च लोभश्च मोहश्चैव चतुर्विधः \thinspace{\dandab} \dontdisplaylinenum  }%
 
%Verse 4:70

{\devanagarifont चतुःशत्रुर्निहन्तव्यः सो ऽरिहा वीतकल्मषः {॥ ४:७०॥} \veg\dontdisplaylinenum }%
     \var{{\devanagarifont \numemph\vc\textbf{चतुःशत्रुर्नि॰}\lem \msCa\msCb\Ed, चतुशत्रु नि॰ \msCc\msNa\msNb\msNc}}% 
    \var{{\devanagarifont \numnoemph\vd\textbf{सो ऽरिहा}\lem \mssALL, स्रोरिहा \msCb, सर्वथा \Ed}}% 

{\devanagarifont चतुरायतनं विप्र कथयिष्यामि तच्छृणु \thinspace{\dandab} \dontdisplaylinenum }%
 
%Verse 4:71

{\devanagarifont करुणा मुदितोपेक्षा मैत्री चायतनं स्मृतम् {॥ ४:७१॥} \veg\dontdisplaylinenum }%
     \var{{\devanagarifont \numemph\vc\textbf{मुदितो॰}\lem \mssALL, मुदितौ॰ \Ed}}% 
    \var{{\devanagarifont \numnoemph\vd\textbf{चायतनं}\lem \mssALL, चायतन \msCa, चायत\uncl{न} \msCb}}% 

{\devanagarifont चतुर्ध्यानाधुना वक्ष्ये संसारार्णवतारणम् \thinspace{\dandab} \dontdisplaylinenum }%
 
%Verse 4:72

{\devanagarifont आत्मविद्याभवः सूक्ष्मं ध्यानमुक्तं चतुर्विधम् {॥ ४:७२॥} \veg\dontdisplaylinenum }%
     \var{{\devanagarifont \numemph\vc\textbf{॰भवः}\lem \msCb\msCcpcorr\msNa\msNb\msNc, ॰भव \msCa\msCcacorr, ॰भवं \Ed}}% 
    \var{{\devanagarifont \numnoemph\vcd\textbf{सूक्ष्मं ध्या॰}\lem \msCa\msNa\msNc\Ed, 
सूक्ष्मा\uncl{न्या}॰ \msCb, 
सू\uncl{क्ष्म}ध्या॰ \msCc, सूक्ष्मध्यान॰ \msNb}}% 
    \var{{\devanagarifont \numnoemph\vd\textbf{॰नमुक्तं चतुर्विधम्}\lem \msCc\msNb, ॰नमुक्तश्चतुर्विधम् \msCa, 
॰नमुक्तश्चतुर्विधः \msCb\msNa, 
॰नमुक्तं चतुर्विधिं \msNc, ॰नयज्ञश्च \Ed}}% 

{\devanagarifont आत्मतत्त्वः स्मृतो धर्मो विद्या पञ्चसु पञ्चधा \thinspace{\dandab} \dontdisplaylinenum }%
     \var{{\devanagarifont \numemph\va\textbf{स्मृतो}\lem \mssALL, स्मृता \msCc\Ed\oo 
\textbf{धर्मो}\lem \mssALL, धन्या \Ed}}% 

%Verse 4:73

{\devanagarifont षट्त्रिंशाक्षरमित्याहुः सूक्ष्मतत्त्वमलक्षणम् {॥ ४:७३॥} \veg\dontdisplaylinenum }%
     \var{{\devanagarifont \numnoemph\vcd\textbf{आहुः सू॰}\lem \mssALL, आ\lk\lk\ \msCa}}% 

{\devanagarifont चतुष्पादः स्मृतो धर्मश्चतुराश्रममाश्रितः \thinspace{\dandab} \dontdisplaylinenum }%
     \var{{\devanagarifont \numemph\vab\textbf{धर्मश्च॰}\lem \mssALL, धर्म च॰ \msCc\msNb}}% 
    \var{{\devanagarifont \numnoemph\vb\textbf{॰श्रितः}\lem \mssALL, ॰श्रिताः \msNc}}% 

%Verse 4:74

{\devanagarifont गृहस्थो ब्रह्मचारी च वानप्रस्थो ऽथ भैक्षुकः {॥ ४:७४॥} \veg\dontdisplaylinenum }%
     \var{{\devanagarifont \numnoemph\vd\textbf{भैक्षुकः}\lem \mssALL, भक्षकः \Ed}}% 
    \paral{{\devanagarifontsmall \vcd {\englishfont  = \MBH\ 12.234.13ab \similar\ \MBH\ 14.4513ab etc. }
                 \vo {\englishfont \compare\ 3.4 above:}
                 श्रुतिस्मृतिद्वयोर्मूर्तिश्चतुष्पादवृषः स्थितः\thinspace{\devanagarifontsmall ।}
                 चतुराश्रम यो धर्मः कीर्तितानि मनीषिभिः\thinspace{\devanagarifontsmall ॥} }}

{\devanagarifont धन्यास्ते यैरिदं वेत्ति निखिलेन द्विजोत्तम \thinspace{\dandab} \dontdisplaylinenum }%
     \var{{\devanagarifont \numemph\va\textbf{यैरिदं}\lem \mssALL, येरिदं \msCb\msCc\oo 
\textbf{वेत्ति}\lem \mssALL, वेति \msCc}}% 

%Verse 4:75

{\devanagarifont पावनं सर्वपापानां पुण्यानां च प्रवर्धनम् {॥ ४:७५॥} \veg\dontdisplaylinenum }%
     \var{{\devanagarifont \numnoemph\vd\textbf{प्रवर्धनम्}\lem \mssALL, प्रवर्धनः \Ed}}% 

{\devanagarifont आयुः कीर्तिर्यशः सौख्यं धन्यादेव प्रवर्धते \thinspace{\dandab} \dontdisplaylinenum }%
     \var{{\devanagarifont \numemph\vb\textbf{धन्यादेव}\lem \mssALL, धर्मादेव \Ed}}% 

%Verse 4:76

{\devanagarifont शान्तिः पुष्टिः स्मृतिर्मेधा जायते धन्यमानवे {॥ ४:७६॥} \veg\dontdisplaylinenum }%
     \var{{\devanagarifont \numnoemph\vc\textbf{पुष्टिः}\lem \mssALL, \lk ष्टिः \msCa\oo 
\textbf{स्मृतिर्मेधा}\lem \\mssALL, स्मृति मेधा \msCc\msNa}}% 
    \var{{\devanagarifont \numnoemph\vd\textbf{॰मानवे}\lem \eme, ॰मानवः \mssCaCbCc\msNa\msNb\msNc\Ed}}% 


\alalfejezet{यमेष्वप्रमादः (८)}
{\devanagarifont प्रमादस्थान पञ्चैव कीर्तयिष्यामि तच्छृणु \thinspace{\dandab} \dontdisplaylinenum }%
     \var{{\devanagarifont \numemph\va\textbf{॰स्थान}\lem \msCa\msCc\msNa\msNb, ॰स्थानं \msCb\msNc\Ed\ \unmetr\oo 
\textbf{पञ्चैव}\lem \mssALL, पञ्चैवं \Ed}}% 
    \var{{\devanagarifont \numnoemph\vb\textbf{कीर्तयिष्यामि}\lem \mssALL, कीर्तियिष्यामि \msNb}}% 

{\devanagarifont ब्रह्महत्या सुरापानं स्तेयो गुर्वङ्गनागमम्  \danda\dontdisplaylinenum }%
     \paral{{\devanagarifontsmall \vcdef {\englishfont \similar\ \MBH\ Suppl. 12.30:}
                     ब्रह्महत्यां सुरापानं स्तेयं गुर्वङ्गनागमम्\thinspace{\devanagarifontsmall ।}
                     महान्ति पातकान्याहुः संयोगं चैव तैः सह\thinspace{\devanagarifontsmall ॥}
                     {\englishfont  \similar\ \MANU\ 11.55 (in Olivelle's edition):}
                     ब्रह्महत्या सुरापानं स्तेयं गुर्वङ्गनागमः\thinspace{\devanagarifontsmall ।}
                     महान्ति पातकान्याहुः संसर्गश्चापि तैः सह\thinspace{\devanagarifontsmall ॥}
                 {\englishfont \compare\ also \YAJNS\ 3.228:}
                         ब्रह्महा मद्यपः स्तेनस्तथैव गुरुतल्पगः\thinspace{\devanagarifontsmall ।}
                         एते महापातकिनो यश्च तैः सह संवसेत्\thinspace{\devanagarifontsmall ॥}  }}

%Verse 4:77

{\devanagarifont महापातकमित्याहुस्तत्संयोगी च पञ्चमः {॥ ४:७७॥} \veg\dontdisplaylinenum }%
 
{\devanagarifont अनृतं च समुत्कर्षे राजगामी च पैशुनः \thinspace{\dandab} \dontdisplaylinenum }%
     \var{{\devanagarifont \numemph\va\textbf{समुत्कर्षे}\lem \eme, समुत्कर्षं \msCa\msNa, 
समुत्क\uncl{र्ष} \msCb, 
समुत्कर्ष \msCc\msNb\msNc\Ed}}% 
    \var{{\devanagarifont \numnoemph\vb\textbf{राज॰}\lem \mssALL, राज्ञी॰ \Ed}}% 

%Verse 4:78

{\devanagarifont गुरोश्चालीकनिर्बन्धः समानि ब्रह्महत्यया {॥ ४:७८॥} \veg\dontdisplaylinenum }%
     \var{{\devanagarifont \numnoemph\vc\textbf{॰निर्बन्धः}\lem \eme, ॰निर्बद्धः \msCb\msNc, निबद्धस् \msCa\msCc\msNa\msNb, निर्वद्धस् \Ed}}% 
    \var{{\devanagarifont \numnoemph\vd\textbf{ब्रह्महत्यया}\lem \mssALL, ब्र\lk\lk \lk या \msCa}}% 
    \paral{{\devanagarifontsmall \vo \similar\ {\englishfont \MBH\ 5.40.3 and \MANU\ 11.56:}
                  अनृतं च समुत्कर्षे राजगामि च पैशुनम्\thinspace{\devanagarifontsmall ।}
                  गुरोश्चालीकनिर्बन्धः समानि ब्रह्महत्यया\thinspace{\devanagarifontsmall ॥}
                 {\englishfont \similar\ \VISNUS\ 37.1--4 \similar\ \AGNIP\ 168.25} }}

\vfill
\pageparbreak
\vers

{\devanagarifont ब्रह्मोज्झं वेदनिन्दा च कूटसाक्षी सुहृद्वधः \thinspace{\dandab} \dontdisplaylinenum }%
     \var{{\devanagarifont \numemph\va\textbf{ब्रह्मोज्झं}\lem \eme, ब्रह्मो ऋग्॰ \mssCaCbCc\msNa\msNb\msNc, ब्रह्म ऋग्॰ \Ed}}% 
    \var{{\devanagarifont \numnoemph\vb\textbf{सुहृद्वधः}\lem \mssALL, सकृद्बुधः \Ed}}% 

%Verse 4:79

{\devanagarifont गर्हितानाद्ययोर्जग्धिः सुरापानसमानि षट् {॥ ४:७९॥} \veg\dontdisplaylinenum }%
     \var{{\devanagarifont \numnoemph\vc\textbf{॰नाद्ययोर्जग्धिः}\lem \eme, ॰न्नञ्च यो जग्धिस् \msCa, ॰न्नञ्च यो जग्धि \msCb, 
॰न्नञ्च योद्विग्नः \msCc, ॰न्नं च यो जग्धिः \msNa, ॰न्नं च यो जग्धिः \msNb, 
॰न्नञ्च यो जवे \msNc, ॰न्नश्च यो विप्रः \Ed}}% 
    \paral{{\devanagarifontsmall \vo \similar\ {\englishfont \MANU\ 11.57:}
                         ब्रह्मोज्झता वेदनिन्दा कौटसाक्ष्यं सुहृद्वधः\thinspace{\devanagarifontsmall ।}
                         गर्हितानाद्ययोर्जग्धिः सुरापानसमानि षट्\thinspace{\devanagarifontsmall ॥}
                 {\englishfont \compare\ \YAJNS\ 3.229:}
                         गुरूणामध्यधिक्षेपो वेदनिन्दा सुहृद्वधः\thinspace{\devanagarifontsmall ।}
                         ब्रह्महत्यासमं ज्ञेयमधीतस्य च नाशनम्\thinspace{\devanagarifontsmall ॥} }}

{\devanagarifont रेतोत्सेकः स्वयोन्यासु कुमारीष्वन्त्यजासु च \thinspace{\dandab} \dontdisplaylinenum }%
     \var{{\devanagarifont \numemph\va\textbf{स्वयोन्यासु}\lem \mssALL, सुतोन्यासु \msCb}}% 

%Verse 4:80

{\devanagarifont सख्युः पुत्रस्य च स्त्रीषु गुरुतल्पसमः स्मृतः {॥ ४:८०॥} \veg\dontdisplaylinenum }%
     \var{{\devanagarifont \numnoemph\vc\textbf{सख्युः}\lem \eme, सख्य \mssCaCbCc\msNa\Ed, \lk\lk\ \msNb, स\uncl{ख्यु} \msNc\oo 
\textbf{पुत्रस्य च स्त्रीषु}\lem \mssALL, 
\lk\lk\lk\lk\lk\lk\ \msNb, पुत्रीषु चास्त्रीषु \Ed}}% 
    \var{{\devanagarifont \numnoemph\vd\textbf{॰समः}\lem \mssALL, \lk\lk\ \msNb, ॰सम \Ed}}% 
    \paral{{\devanagarifontsmall \vo \similar\ {\englishfont \MANU\ 11.59:}
                                 रेतःसेकः स्वयोनीषु कुमारीष्वन्त्यजासु च\thinspace{\devanagarifontsmall ।}
                                 सख्युः पुत्रस्य च स्त्रीषु गुरुतल्पसमं विदुः\thinspace{\devanagarifontsmall ॥} }}

{\devanagarifont निक्षेपस्यापहरणं नराश्वरजतस्य च \thinspace{\dandab} \dontdisplaylinenum }%
     \var{{\devanagarifont \numemph\va\textbf{निक्षेप॰}\lem \mssALL, निखेप॰ \msCb, \uncl{निक्षेप}॰ \msNb}}% 
    \var{{\devanagarifont \numnoemph\vb\textbf{नराश्वरजतस्य}\lem \mssALL, नराणां स्वजनस्य \msCb, 
\uncl{नराश्वरजतस्य} \msNb}}% 

%Verse 4:81

{\devanagarifont भूमिवज्रमणीनां च रुक्मस्तेयसमः स्मृतः {॥ ४:८१॥} \veg\dontdisplaylinenum }%
     \var{{\devanagarifont \numnoemph\vd\textbf{रुक्मस्तेय॰}\lem \eme, \uncl{रूग्य}\lk य॰ \msCa, 
रुग्मस्तेय॰ \msCb\msCc\msNa\msNc, \lk\lk \lk\lk\ \msNb, हृतस्तेय॰ \Ed\oo 
\textbf{॰समः}\lem \mssALL, सः \msCbacorr, ॰सम \Ed}}% 
    \paral{{\devanagarifontsmall \vo {\englishfont = \MANU\ 11.58 } }}

{\devanagarifont चत्वार एते सम्भूय यत्पापं कुरुते नरः \thinspace{\dandab} \dontdisplaylinenum }%
     \var{{\devanagarifont \numemph\va\textbf{एते}\lem \mssALL, \uncl{एते} \msNb, एव \Ed\oo 
\textbf{सम्भूय}\lem \mssALL, संभूयो \msCc, \uncl{संभूयो} \msNb}}% 

{\devanagarifont महापातक पञ्चैतत् तेन सर्वं प्रकाशितम्  \danda\dontdisplaylinenum }%
     \var{{\devanagarifont \numnoemph\vc\textbf{॰पञ्चैतत्}\lem \corr, ॰पञ्चैतन् \mssCaCbCc\Ed, ॰पञ्चैते \msNa, 
॰पञ्चैतम् \msNb, ॰पञ्चेतन् \msNc}}% 

%Verse 4:82

{\devanagarifont पञ्चप्रमादमेतानि वर्जनीयं द्विजोत्तम {॥ ४:८२॥} \veg\dontdisplaylinenum }%
     \var{{\devanagarifont \numnoemph\ve\textbf{॰मादम्}\lem \mssALL, ॰माद \Ed}}% 
    \var{{\devanagarifont \numnoemph\vf\textbf{वर्जनीयं}\lem \mssALL, वर्जनीयो \msCc}}% 

\vfill
\pageparbreak
\vers


\alalfejezet{यमेषु माधुर्यम् (९)}
{\devanagarifont कायवाङ्मनमाधुर्यश्चक्षुर्बुद्धिश्च पञ्चमः \thinspace{\dandab} \dontdisplaylinenum }%
     \var{{\devanagarifont \numemph\vab\textbf{मनमाधुर्यश्च॰}\lem \eme, ॰मनसा धूर्यश्च॰ \msCa\msCc\msNa\msNc, 
॰मन\uncl{मा}धूर्यश्च॰ \msCb, 
॰मन\lk धूर्य\lk ॰ \msNb, ॰मनसा भूयश्च॰ \Ed}}% 
    \var{{\devanagarifont \numnoemph\vb\textbf{॰क्षुर्बुद्धि॰}\lem \msCa\msCb\msNc\Ed, ॰क्षु बुद्धि॰ \msCc\msNa, \lk\lk \lk\  \msNb}}% 

%Verse 4:83

{\devanagarifont सौम्यदृष्टिप्रदानं च क्रूरबुद्धिं च वर्जयेत् {॥ ४:८३॥} \veg\dontdisplaylinenum }%
     \var{{\devanagarifont \numnoemph\vc\textbf{॰दानं च}\lem \mssALL, \lk\lk\ \msNb, ॰दानश्च \Ed}}% 
    \var{{\devanagarifont \numnoemph\vd\textbf{॰बुद्धिं च}\lem \msCa\msNa\msNc, बुद्धिश्च \msCb, ॰दृष्टिं च \msCc\Ed, \lk\lk \lk\ \msNb}}% 

{\devanagarifont प्रसन्नमनसा ध्यायेत्प्रियवाक्यमुदीरयेत् \thinspace{\dandab} \dontdisplaylinenum }%
     \var{{\devanagarifont \numemph\va\textbf{प्रसन्न॰}\lem \mssALL, \uncl{प्रसन्न}॰ \msNb, प्रसंन॰ \msNc}}% 

%Verse 4:84

{\devanagarifont यथाशक्तिप्रदानं च स्वाश्रमाभ्यागतो गुरुः {॥ ४:८४॥} \veg\dontdisplaylinenum }%
     \var{{\devanagarifont \numnoemph\vc\textbf{यथा॰}\lem \mssALL, यस्य \Ed\oo 
\textbf{॰दानं}\lem \mssALL, ॰दातश् \Ed}}% 
    \var{{\devanagarifont \numnoemph\vd\textbf{स्वाश्रमा॰}\lem \mssALL, स्वासमा॰ \msCc\oo 
\textbf{॰गतो}\lem \mssALL, ॰सतो \msNc}}% 

{\devanagarifont इन्धनोदकदानं च जातवेदमथापि वा \thinspace{\dandab} \dontdisplaylinenum }%
     \var{{\devanagarifont \numemph\vb\textbf{इन्धनो॰}\lem \mssALL, इत्वनो॰ \msNc\oo 
\textbf{जात॰}\lem \mssALL, जा॰ \msCb}}% 

{\devanagarifont सुलभानि न दत्तानि इन्धनाग्न्युदकानि च  \danda\dontdisplaylinenum }%
     \var{{\devanagarifont \numnoemph\vc\textbf{सुलभानि न}\lem \mssALL, सुरभानि च \Ed}}% 
    \var{{\devanagarifont \numnoemph\vd\textbf{॰दकानि}\lem \mssALL, ॰\uncl{त}कानि \msNb}}% 

%Verse 4:85

{\devanagarifont क्षुते जीवेति वा नोक्तं तस्य किं परतः फलम् {॥ ४:८५॥} \veg\dontdisplaylinenum }%
     \var{{\devanagarifont \numnoemph\ve\textbf{क्षुते}\lem \conj, क्षुतं \mssCaCbCc\msNa\msNb\msNc, शतं \Ed}}% 


\alalfejezet{यमेष्वार्जवम् (१०)}
{\devanagarifont पञ्चार्जवाः प्रशंसन्ति मुनयस्तत्त्वदर्शिनः \thinspace{\dandab} \dontdisplaylinenum }%
     \var{{\devanagarifont \numemph\va\textbf{पञ्चार्जवाः}\lem \msCa\msCb\msNa\msNc, पञ्चार्जवः \msCc, \lk\lk \lk\lk\ \msNb, पञ्चार्जवा \Ed\oo 
\textbf{प्रशंसन्ति}\lem \mssCaCbCc\msNc, प्रशसन्ति \msNa\Ed, \uncl{प्रससन्ति} \msNb}}% 

{\devanagarifont कर्मवृत्त्याभिवृद्धिं च पारितोषिकमेव च  \danda\dontdisplaylinenum }%
     \var{{\devanagarifont \numnoemph\vc\textbf{कर्म॰}\lem \mssALL, \lk र्म्म॰ \msCa, \uncl{कम्मा}॰ \msNb\oo 
\textbf{॰वृत्त्याभिवृद्धिं च}\lem \mssALL, 
॰वृत्तिभिवृद्धिञ्च \msNb, ॰वृत्याभिवृद्धिश्च \Ed}}% 
    \var{{\devanagarifont \numnoemph\vd\textbf{पारितोषिक॰}\lem \eme, पारतोषिक॰ \mssCaCbCc\msNa\msNb\msNc\Ed}}% 

%Verse 4:86

{\devanagarifont स्त्रीधनोत्कोचवित्तं च आर्जवो नाभिनन्दति {॥ ४:८६॥} \veg\dontdisplaylinenum }%
     \var{{\devanagarifont \numnoemph\ve\textbf{स्त्रीधनोत्कोच॰}\lem \mssALL, स्त्रीधनङ्गो च \Ed\oo 
\textbf{॰वित्तं च}\lem \mssALL, ॰वित्तिञ्च \msNb}}% 
    \var{{\devanagarifont \numnoemph\vf\textbf{आर्जवो ना॰}\lem \mssALL, आर्जवञ्च \msCc, आर्ज्जवेना॰ \Ed}}% 

\vfill
\pageparbreak
\vers

{\devanagarifont आर्जवो न वृथा यज्ञ आर्जवो न वृथा तपः \thinspace{\dandab} \dontdisplaylinenum }%
     \var{{\devanagarifont \numemph\vab \lem \mssCaCbCc\msNb\msNc, \om\ \msNaacorr, 
आर्जवो न वृथा यज्ञ आर्जवो न वृथा तप \msNapcorr, 
आर्जवो न वृथा यज्ञश्चार्र्जवो न वृथा तपः \Ed}}% 

%Verse 4:87

{\devanagarifont आर्जवो न वृथा दानमार्जवो न वृथाग्नयः {॥ ४:८७॥} \veg\dontdisplaylinenum }%
     \var{{\devanagarifont \numnoemph\vcd\textbf{(आर्जवो{\englishfont ...} वृथाग्नयः)}\lem \mssALL, \om\ \Ed}}% 

{\devanagarifont आर्जवस्येन्द्रियग्रामः सुप्रसन्नो ऽपि तिष्ठति \thinspace{\dandab} \dontdisplaylinenum }%
     \var{{\devanagarifont \numemph\vab\textbf{(आर्जव॰{\englishfont ...} तिष्ठति)}\lem \mssALL, \om\ \Ed}}% 
    \var{{\devanagarifont \numnoemph\va\textbf{॰ग्रामः}\lem \msCa\msCb\msNc\Ed, ॰ग्रामात् \msCc\msNb, ॰ग्रामाः \msNa}}% 

%Verse 4:88

{\devanagarifont आर्जवस्य सदा देवाः काये तस्य चरन्ति ते {॥ ४:८८॥} \veg\dontdisplaylinenum }%
     \var{{\devanagarifont \numnoemph\vd\textbf{तस्य चरन्ति}\lem \mssALL, त\lk\lac  न्ति \msCa, तस्य रमन्ति \Ed}}% 

\ujvers\nemsloka {
{\devanagarifont इति यमप्रविभागः कीर्तितो ऽयं द्विजेन्द्र }%
  \dontdisplaylinenum}    \var{{\devanagarifont \numemph\va\textbf{यमप्रविभागः}\lem \msCa\msCb\msNb\msNc, यमविभागः \msCc, 
यमप्ररिभागः \msNa, नियमपरिभागः \Ed\oo 
\textbf{द्विजेन्द्र}\lem \mssALL, नरेन्द्र \Ed}}% 


\nemslokab

{\devanagarifont इह परत सुखार्थं कारयेत्तं मनुष्यः  \danda\dontdisplaylinenum }%
     \var{{\devanagarifont \numnoemph\vb\textbf{॰येत्तं मनुष्यः}\lem \corr, ॰येत्तन्मनुष्यः \msCa\msNa\msNb\msNc\Ed, ॰येत्त मनुष्यः \msCb, 
॰येत्तत्मनुष्यः \msCc}}% 

\nemslokac

{\devanagarifont दुरितमलपहारी शङ्करस्याज्ञयास्ते }%
  \dontdisplaylinenum    \var{{\devanagarifont \numnoemph\vc\textbf{दुरित॰}\lem \mssALL, इरित॰ \Ed\oo 
\textbf{॰पहारी}\lem \mssALL, ॰पलपहारी \msCc\oo 
\textbf{॰ज्ञयास्ते}\lem \mssALL, ॰ज्ञयाते \msNa}}% 

%Verse 4:89


\nemslokad

{\devanagarifont भवति पृथिविभर्ता ह्येकछत्रप्रवर्ता {॥ ४:८९॥} \veg\dontdisplaylinenum }%
     \var{{\devanagarifont \numnoemph\vd\textbf{॰वर्ता}\lem \conj, ॰वृत्ता \mssCaCbCc\msNb\msNc, ॰वृत्ताः \msNa\Ed}}% 

\vers


{\devanagarifont 
\jump
\begin{center}
\ketdanda~इति वृषसारसंग्रहे यमविभागो नामाध्यायश्चतुर्थः~\ketdanda
\end{center}
\dontdisplaylinenum\vers  }%
     \var{{\devanagarifont \numnoemph{\englishfont \Colo:}\textbf{नामाध्यायश्चतुर्थः}\lem \mssALL, 
नामश्चतुर्थो ऽध्यायः \Ed}}% 
\bekveg\szamveg
\vfill
\phpspagebreak

\versno=0\fejno=5
\thispagestyle{empty}

\centerline{\Large\devanagarifontbold [   पञ्चमो ऽध्यायः  ]}{\vrule depth10pt width0pt} \fancyhead[CO]{{\footnotesize\devanagarifont वृषसारसंग्रहे  }}
\fancyhead[CE]{{\footnotesize\devanagarifont पञ्चमो ऽध्यायः  }}
\fancyhead[LE]{}
\fancyhead[RE]{}
\fancyhead[LO]{}
\fancyhead[RO]{}
\szam\bek



\alalfejezet{नियमाः}
\vers


{\devanagarifont विगतराग उवाच {\dandab}\dontdisplaylinenum  }%
     \var{{\devanagarifont \numemph\vo\textbf{विगतराग उवाच}\lem \mssALL, 
विगत\uncl{राग उवा}च \msCa}}% 
    \lacuna{\devanagarifontsmall {\englishfont Witnesses used for this chapter: \msCa\ ff.\thinspace 201v--202r, 
                                                  \msCb\ ff.\thinspace 208v--209r, 
                                                  \msCc\ ff.\thinspace 277r--278r,
                                                  \msNa\ ff.\thinspace 9r--9v, 
                                                  \msNb\ exp.\thinspace 50 (upper) and 51 (lower),
                                                  \msNc\ ff.\thinspace 217r--218r,
                                                  \msM\ ff.\thinspace 9r--10r,
                                                  \Ed\ pp.\thinspace 597--599;  
                                                  \mssCaCbCc\ = \msCa + \msCb + \msCc} }%
  
\nemsloka 
{\devanagarifont कथय नियमतत्त्वं साम्प्रतं त्वं विशेषाद् }%
  \dontdisplaylinenum    \var{{\devanagarifont \numnoemph\va\textbf{कथय नि॰}\lem \mssALL, कथयति \Ed\oo 
\textbf{॰तत्त्वं}\lem \mssALL, तं \msCb\oo 
\textbf{साम्प्रतं त्वं विशेषाद्}\lem \msCa\msNa\msNc\Ed, त्वां वशेषात् \msCb, 
सांप्रत त्वं विसेषात् \msCc\msNb, साम्प्रतं त्वं विशेषा \msM}}% 


\nemslokab

{\devanagarifont अमृतवचनतुल्यं श्रोतुकामो गतो ऽस्मि  \danda\dontdisplaylinenum }%
     \var{{\devanagarifont \numnoemph\vb\textbf{॰वचनतुल्यं श्रो॰}\lem \msM, वदनतुल्यं श्रो॰ \msCa\msCc\msNapcorr\msNb\msNc\Ed, 
वदनतुल्यां श्रो॰ \msCb, 
वदन\uncl{तुल्यं श्रो} तुल्यं स्रो॰ \msNaacorr\oo 
\textbf{॰कामो}\lem \mssALL, ॰कामा \msM\Ed}}% 

\nemslokac

{\devanagarifont प्रकृतिदहनदग्धं ज्ञानतोयैर्निषिक्तम् }%
  \dontdisplaylinenum    \var{{\devanagarifont \numnoemph\vc\textbf{॰दहन॰}\lem \mssALL, ॰वदन॰ \Ed\oo 
\textbf{॰दग्धं}\lem \mssALL, ॰दग्ध \msM\oo 
\textbf{॰र्निषिक्तम्}\lem \mssALL, ॰र्विमुक्तम् \msCb, ॰र्निशिक्तः \msM}}% 

%Verse 5:1


\nemslokad

{\devanagarifont अपर वदमतज्ज्ञं नास्ति धर्मेषु तृप्तिः {॥ ५:१॥} \veg\dontdisplaylinenum }%
     \var{{\devanagarifont \numnoemph\vd\textbf{अपर॰}\lem \mssALL, अपरं \msNa\ \unmetr, अर॰ \msMacorr\oo 
\textbf{मतज्ज्ञं नास्ति}\lem \conj, मतज्ञा नास्ति \msCapcorr\msCb\msNa\msNc\msM, 
तज्ञा नास्ति \msCaacorr, 
मतज्ञा\uncl{न्ना}स्ति \msCc, 
\uncl{मे} \lk\lk \lk\lk\ \msNb, 
॰न तज्ज्ञान्नास्ति \Ed\oo 
\textbf{धर्मेषु तृप्तिः}\lem \mssALL, मे धर्मतृप्तिः \msM}}% 

\vers


{\devanagarifont अनर्थयज्ञ उवाच {\dandab}\dontdisplaylinenum  }%
     \var{{\devanagarifont \numemph\vo\textbf{अनर्थ॰}\lem \mssALL, अर्थ॰ \msM}}% 

\nemsloka 
{\devanagarifont श्रवणसुखमतो ऽन्यत्कीर्तयिष्ये द्विजेन्द्र }%
  \dontdisplaylinenum    \var{{\devanagarifont \numnoemph\va\textbf{॰सुख॰}\lem \mssALL, ॰मुख॰ \msNaacorr\oo 
\textbf{॰मतो ऽन्यत्}\lem \mssCaCbCc\msNa\msNc, ॰मतो ऽन्य \msNb, 
॰मतो न्यः \msM, ॰मनो ऽन्यत् \Ed\oo 
\textbf{कीर्त॰}\lem \mssALL, कीर्ति॰ \msNa\msNb}}% 


\nemslokab

{\devanagarifont नियमकलविशेषः पञ्च पञ्च प्रकारः  \danda\dontdisplaylinenum }%
     \var{{\devanagarifont \numnoemph\vb\textbf{॰विशेषः}\lem \mssALL, विशे\lk\ \msCa, ॰विशेष \msCb\oo 
\textbf{प्रकारः}\lem \mssALL, पकारः \msNc}}% 

\vfill
\pageparbreak
\vers

\nemslokac

{\devanagarifont हरिहरमुनिभीष्टं धर्मसारं द्विजेन्द्र }%
  \dontdisplaylinenum
%Verse 5:2


\nemslokad

{\devanagarifont कलिकलुषविनाशं प्रायमोक्षप्रसिद्धम् {॥ ५:२॥} \veg\dontdisplaylinenum }%
     \var{{\devanagarifont \numnoemph\vd\textbf{॰विनाशं}\lem \mssALL, ॰विनाश॰ \msCc\Ed}}% 

\vers


{\devanagarifont शौचमिज्या तपो दानं स्वाध्यायोपस्थनिग्रहः \thinspace{\dandab} \dontdisplaylinenum }%
     \var{{\devanagarifont \numemph\va\textbf{इज्या}\lem \msCa\msCb\msNa\msNc\Ed, ईज्या \msCc\msNb\msM\oo 
\textbf{दानं}\lem \mssALL, दान॰ \msNb}}% 

%Verse 5:3

{\devanagarifont व्रतोपवासमौनं च स्नानं च नियमा दश {॥ ५:३॥} \veg\dontdisplaylinenum }%
     \var{{\devanagarifont \numnoemph\vc\textbf{॰पवास॰}\lem \mssALL, ॰प्रवाष॰ \msM}}% 
    \paral{{\devanagarifontsmall \vo {\englishfont  = \LINPU\ 1.8.29cd--30ab = \VDHU\ 3.233.202} }}


\alalfejezet{नियमेषु शौचम् (१)}
{\devanagarifont तत्र शौचादिनिर्देशं वक्ष्यामीह द्विजोत्तम \thinspace{\dandab} \dontdisplaylinenum }%
     \var{{\devanagarifont \numemph\va\textbf{॰निर्देशं}\lem \mssALL, ॰नियमं \msNa, ॰र्द्देशं \msNb}}% 

%Verse 5:4

{\devanagarifont शारीरशौचमाहारो मात्रा भावश्च पञ्चमः {॥ ५:४॥} \veg\dontdisplaylinenum }%
     \var{{\devanagarifont \numnoemph\vc\textbf{शारीर॰}\lem \mssALL, शरीर॰ \msNb\oo 
\textbf{॰शौचमाहारो}\lem \mssALL, ॰शौच\lk हारो \msCa, 
॰स्रोतमाहार \msM}}% 
    \var{{\devanagarifont \numnoemph\vd\textbf{मात्रा भावश्च}\lem \mssALL, मात्रा भावं च \msCa, 
\uncl{सात्राभा}वश्च \msNb}}% 


\alalalfejezet{शरीरशौचम्}

{\devanagarifont ताडयेन्न च बन्धेत न च प्राणैर्वियोजयेत् \thinspace{\dandab} \dontdisplaylinenum }%
     \var{{\devanagarifont \numemph\va\textbf{ताडयेन्न}\lem \mssALL, ताडये न \msNc\msM\oo 
\textbf{बन्धेत}\lem \mssALL, बन्धेन \msM}}% 

%Verse 5:5

{\devanagarifont परस्त्रीपरद्रव्येषु शौचं कायिकमुच्यते {॥ ५:५॥} \veg\dontdisplaylinenum }%
     \var{{\devanagarifont \numnoemph\vc\textbf{॰द्रव्येषु}\lem \mssALL, ॰द्रवेषु \msM}}% 
    \var{{\devanagarifont \numnoemph\vd\textbf{शौचं}\lem \mssALL, शौच \msNc\oo 
\textbf{कायिकमुच्यते}\lem \mssALL, कायिकमुमुच्येते \msNc}}% 

{\devanagarifont श्रोत्रशौचं द्विजश्रेष्ठ गुदोपस्थमुखादयः \thinspace{\dandab} \dontdisplaylinenum }%
     \var{{\devanagarifont \numemph\va\textbf{श्रोत्र॰}\lem \msM, श्रोत॰ \mssCaCbCc\msNa\msNb\msNc\Ed}}% 
    \var{{\devanagarifont \numnoemph\vb\textbf{गुदोपस्थ॰}\lem \mssALL, गुदोप्रस्थ॰ \msNc, गुदापस्थ॰ \Ed}}% 

%Verse 5:6

{\devanagarifont मुखस्याचमनं शौचमाहारवचनेषु च {॥ ५:६॥} \veg\dontdisplaylinenum }%
     \var{{\devanagarifont \numnoemph\vc\textbf{मुखस्या॰}\lem \mssALL, मुखस्था॰ \msCb}}% 
    \var{{\devanagarifont \numnoemph\vcd\textbf{शौचमा॰}\lem \msCa\msCc\msNa\msNc\Ed, शौचंमा॰ \msCb\msNb, शौच आ॰ \msM}}% 
    \var{{\devanagarifont \numnoemph\vd\textbf{॰वचनेषु}\lem \mssALL, ॰वषनेषु \msM}}% 

{\devanagarifont मूत्रविष्टासमुत्सर्गे देवताराधनेषु च \thinspace{\dandab} \dontdisplaylinenum }%
     \var{{\devanagarifont \numemph\va\textbf{॰विष्टा॰}\lem \mssALL, ॰विष्ट॰ \msNb\msM}}% 

%Verse 5:7

{\devanagarifont मृत्तोयैस्तु गुदोपस्थं शौचयीत विचक्षणः {॥ ५:७॥} \veg\dontdisplaylinenum }%
     \var{{\devanagarifont \numnoemph\vc\textbf{मृत्तोयैस्तु}\lem \msCc\msNa\msNb\Ed, \uncl{मृ}\lk\lk \lk\ \msCa, 
मृतोयैस्तु \msCb\msM, मृत्तोयेस्तु \msNc\oo 
\textbf{॰पस्थं}\lem \msCa\msCb\msNa\msNb\msNc, ॰पस्थ \msCc\Ed, ॰पस्थः \msM}}% 
    \var{{\devanagarifont \numnoemph\vd\textbf{शौचयीत}\lem \mssALL, शौचये च \msM}}% 

{\devanagarifont एकोपस्थे गुदे पञ्च तथैकत्र करे दश \thinspace{\dandab} \dontdisplaylinenum }%
     \var{{\devanagarifont \numemph\va\textbf{॰पस्थे}\lem \msCa\msCb\msNa\msNc\Ed, ॰पस्थ॰ \msCc\msNb\msM\oo 
\textbf{गुदे}\lem \msCa\msCb\msNa\msNc\Ed, गुदो \msCc\msNb, गुद \msM}}% 
    \var{{\devanagarifont \numnoemph\vb\textbf{तथैकत्र}\lem \msCa\msCc\msNa\msNb\msNc, तथैक\uncl{त्र} \msCb, 
तथैकत्रे \msM, तथैकश्च \Ed\oo 
\textbf{दश}\lem \mssALL, दशः \msCc}}% 
    \paral{{\devanagarifontsmall \vo {\englishfont \similar\ \MANU\ 5.136:} एका लिङ्गे गुदे तिस्रस्तथैकत्र करे दश\thinspace{\devanagarifontsmall ।}
                                               उभयोः सप्त दातव्या मृदः शुद्धिमभीप्सता\thinspace{\devanagarifontsmall ॥} }}

%Verse 5:8

{\devanagarifont उभयोः सप्त दातव्या मृदः शुद्धिं समीहता {॥ ५:८॥} \veg\dontdisplaylinenum }%
     \var{{\devanagarifont \numnoemph\vc\textbf{उभयोः}\lem \mssALL, उभय \msM\oo 
\textbf{दातव्या}\lem \msCa\msCb\msNa\msNb\msNc, दातव्यो \msCc\Ed, दातव्य \msM}}% 
    \var{{\devanagarifont \numnoemph\vd\textbf{मृदः}\lem \mssCaCbCc\msNc\Ed, मृतः \msNa\msM, मृदा \msNb\oo 
\textbf{शुद्धिं समीहता}\lem \msCa\msCb\msNa, शुद्धिसमीहया \msCc, शु\uncl{द्धि} समीहता \msNb, 
शुद्धिः समीहता \msNc, शुद्धि समीहता \msM, शुद्धिं समाहिता \Ed}}% 

{\devanagarifont एतच्छौचं गृहस्थानां द्विगुणं ब्रह्मचारिणाम् \thinspace{\dandab} \dontdisplaylinenum }%
     \var{{\devanagarifont \numemph\va\textbf{एतच्छौचं}\lem \msCa\msCb\msNa\msNc\msM, चेतच्हौच \msCc\Ed, एत\lk\lk\ \msNb}}% 
    \var{{\devanagarifont \numnoemph\vb\textbf{॰गुणं}\lem \mssALL, ॰गुण \msCc}}% 
    \paral{{\devanagarifontsmall \vab {\englishfont \similar\ \MANU\ 5.137:}
                 एतच्छौचं गृहस्थानां द्विगुणं ब्रह्मचारिणाम्\thinspace{\devanagarifontsmall ।}
                 त्रिगुणं स्याद्वनस्थानां यतीनां तु चतुर्गुणम्\thinspace{\devanagarifontsmall ॥} }}

%Verse 5:9

{\devanagarifont वानप्रस्थस्य त्रिगुणं यतीनां तु चतुर्गुणम् {॥ ५:९॥} \veg\dontdisplaylinenum }%
     \var{{\devanagarifont \numnoemph\vc\textbf{वानप्रस्थस्य}\lem \mssALL, वानप्रस्थे तु \msM\oo 
\textbf{त्रि॰}\lem \mssALL, द्वि॰ \msCc}}% 


\alalalfejezet{आहारशौचम्}

{\devanagarifont आहारशौचं वक्ष्यामि शृणुष्वावहितो भव \thinspace{\dandab} \dontdisplaylinenum }%
     \var{{\devanagarifont \numemph\va\textbf{॰शौचं}\lem \mssALL, ॰शौच \msM}}% 
    \var{{\devanagarifont \numnoemph\vb\textbf{शृणुष्वावहितो}\lem \mssALL, शृणु\uncl{ष्वाव}\lk\lk\ \msCa, 
शृणुष्ववहितो \msNb}}% 

{\devanagarifont भागद्वयं तु भुञ्जीत भागमेकं जलं पिबेत्  \danda\dontdisplaylinenum }%
     \var{{\devanagarifont \numnoemph\vd\textbf{॰कं जलं}\lem \mssALL, ॰कोदकं \msM\oo 
\textbf{पिबेत्}\lem \mssALL, पिबे \msCb}}% 

%Verse 5:10

{\devanagarifont वायुसंचारदानार्थं चतुर्थमवशेषयेत् {॥ ५:१०॥} \veg\dontdisplaylinenum }%
     \var{{\devanagarifont \numnoemph\ve\textbf{॰चारदानार्थं}\lem \mssALL, ॰चरदानार्थं \msM, ॰चारणार्थाय \Ed}}% 
    \paral{{\devanagarifontsmall \vo {\englishfont \similar\ Śaṅkara's commentary ad \BHG\ 6.16:}
                                 उक्तं हि\thinspace{\devanagarifontsmall ।} 
                                 अर्धं सव्यञ्जनान्नस्य तृतीयमुदकस्य च\thinspace{\devanagarifontsmall ।} 
                                 वायोः संचरणार्थं तु चतुर्थमवशेषयेत्\thinspace{\devanagarifontsmall ॥};
                    {\englishfont \compare\ \ASTANGHR\ 8.46cd--47ab:}
                                              अन्नेन कुक्षेर्द्वावंशौ पानेनैकं प्रपूरयेत्\thinspace{\devanagarifontsmall ॥} 
                                              आश्रयं पवनादीनां चतुर्थमवशेषयेत्\thinspace{\devanagarifontsmall ।};
                    {\englishfont \compare\ \SANNYASUP\ 59:}
                                              आहारस्य च भागौ द्वौ तृतीयमुदकस्य च\thinspace{\devanagarifontsmall ।} 
                                              वायोः संचरणार्थाय चतुर्थमवशेषयेत्\thinspace{\devanagarifontsmall ॥} }}

\vfill
\pageparbreak
\vers

{\devanagarifont स्निग्धस्वादुरसैः षड्भिराहारषड्रसैर्बुधः \thinspace{\dandab} \dontdisplaylinenum }%
     \var{{\devanagarifont \numemph\va\textbf{॰स्वादुरसैः}\lem \mssCaCbCc\msNa\msNc, ॰स्वा\lk रसैः \msNb, ॰स्वादुरसं \msM, ॰स्वादरसैः \Ed}}% 
    \var{{\devanagarifont \numnoemph\vb\textbf{॰हारषड्रसैर्बु॰}\lem \msCb\Ed, ॰हारसद्रवैर्बु॰ \msCa\msNa\msNc, 
॰हारसद्रवै बु॰ \msCc, ॰हारषड्रसै बु॰ \msNb, ॰हारे सद्रवद्बु॰ \msM}}% 

%Verse 5:11

{\devanagarifont धातुवैषम्यनाशो ऽस्ति न च रोगाः सुदारुणाः {॥ ५:११॥} \veg\dontdisplaylinenum }%
     \var{{\devanagarifont \numnoemph\vc\textbf{॰वैषम्यनाशो ऽस्ति}\lem \msCa\msCc\msNa\msNb\msNc, 
॰\uncl{दै}षम्य$\-$नाशास्ति \msCb, ॰वैशम्य नस्यास्ति \msM, ॰वैषम्य नश्यन्ति \Ed}}% 
    \var{{\devanagarifont \numnoemph\vd\textbf{रोगाः}\lem \mssALL, रोग \msM\oo 
\textbf{सुदारुणाः}\lem \mssALL, स्वदारुणाः \msM, सुदारुणः \Ed}}% 

{\devanagarifont अभक्ष्यं च न भक्षेत अपेयं न च पाययेत् \thinspace{\dandab} \dontdisplaylinenum }%
     \var{{\devanagarifont \numemph\va\textbf{अभक्ष्यं}\lem \mssCaCbCc\msNa\msNc, \lk\lk \lk\ \msNb, अभक्षं \msM\Ed\oo 
\textbf{च न भक्षेत}\lem \mssALL, न च भक्षेतः \msM}}% 
    \var{{\devanagarifont \numnoemph\vb\textbf{न च}\lem \mssALL, च न \msNc\Ed}}% 

%Verse 5:12

{\devanagarifont अगम्यं न च गम्येत अवाच्यं न च भाषयेत् {॥ ५:१२॥} \veg\dontdisplaylinenum }%
     \var{{\devanagarifont \numnoemph\vc\textbf{गम्येत}\lem \mssALL, गम्येतः \msM}}% 
    \var{{\devanagarifont \numnoemph\vd\textbf{अवाच्यं}\lem \mssALL, अवाचं \msCc}}% 

{\devanagarifont लशुनं च पलाण्डुं च गृञ्जनं कवकानि च \thinspace{\dandab} \dontdisplaylinenum }%
     \var{{\devanagarifont \numemph\va\textbf{पलाण्डुं}\lem \Ed, पलण्डुं \mssCaCbCc\msNb\msNc\msM, पलडुं \msNa}}% 
    \var{{\devanagarifont \numnoemph\vb\textbf{कवकानि}\lem \mssALL, च कचानि \Ed}}% 
    \paral{{\devanagarifontsmall \vab {\englishfont \similar\ \MANU\ 5.5ab: } लशुनं गृञ्जनं चैव पलाण्डुं कवकानि च }}

%Verse 5:13

{\devanagarifont गौरं च सूकरं मांसं वर्जयेच्च विधानतः {॥ ५:१३॥} \veg\dontdisplaylinenum }%
     \var{{\devanagarifont \numnoemph\vc\textbf{गौरं च}\lem \eme, गोरस्व \msCa\msNb, गोरश्च \msCb\msCc\msNa\msNc\msM, गौरश्च \Ed\oo 
\textbf{मांसं}\lem \mssALL, मांसः \msM, मासं \Ed}}% 
    \var{{\devanagarifont \numnoemph\vd\textbf{विधानतः}\lem \mssALL, विधानत् \msM}}% 

{\devanagarifont छत्त्राकं विड्वराहं च गोमांसं च न भक्षयेत् \thinspace{\dandab} \dontdisplaylinenum }%
     \var{{\devanagarifont \numemph\va\textbf{छत्त्राकं}\lem \mssALL, छत्त्राक \msCc\oo 
\textbf{विड्व॰}\lem \mssALL, विद्व॰ \msNa\msNc}}% 
    \var{{\devanagarifont \numnoemph\vb\textbf{गोमांसं}\lem \mssALL, गोमाञ् \msCbacorr}}% 
    \paral{{\devanagarifontsmall \vab {\englishfont \compare\ \MANU\ 5.19ab:} छत्राकं विड्वराहं च लशुनं ग्रामकुक्कुटम्  }}

%Verse 5:14

{\devanagarifont चटकं च कपोतं च जालपादांश्च वर्जयेत् {॥ ५:१४॥} \veg\dontdisplaylinenum }%
     \var{{\devanagarifont \numnoemph\vc\textbf{चटकं}\lem \mssALL, चटकाम् \msCc\msNb}}% 
    \var{{\devanagarifont \numnoemph\vd\textbf{॰पादांश्च}\lem \mssALL, जालपादञ्च \msM}}% 

{\devanagarifont हंससारसचक्राह्वकुक्कुटान्शुकश्येनकान् \thinspace{\dandab} \dontdisplaylinenum }%
     \var{{\devanagarifont \numemph\va\textbf{॰चक्राह्व॰}\lem \mssALL, ॰चक्राह्वा॰ \msM}}% 
    \var{{\devanagarifont \numnoemph\vb\textbf{॰कुक्कुटान्शु॰}\lem \mssCaCbCc\msNc\Ed, ॰कुक्कुटा शु॰ \msNa, ॰कुक्कुटां शु॰ \msNb, ॰कुर्कुटा शु॰ \msM\oo 
\textbf{॰श्येनकान्}\lem \msCa\msCc\msNc\Ed, ॰शोनकान् \msCb, ॰श्येनका \msNa, ॰श्येनकां \msNb, ॰श्येनकम् \msM}}% 

%Verse 5:15

{\devanagarifont काकोलूकं बलाकं च मत्स्यादींश्चापि वर्जयेत् {॥ ५:१५॥} \veg\dontdisplaylinenum }%
     \var{{\devanagarifont \numnoemph\vc \lem \msCb\msNc, काकोलूक\uncl{स्व}\lk\lk ञ्च \msCa, 
काकोलूकबलाकं च \msCc\msNa\msM\Ed, 
\uncl{काकोलूकं बलाकं च} \msNb}}% 
    \var{{\devanagarifont \numnoemph\vd \lem \mssALL, मत्स्यादीनि च वर्जये \msM}}% 

{\devanagarifont अमेध्यांश्चापवित्रांश्च सर्वानेव विवर्जयेत् \thinspace{\dandab} \dontdisplaylinenum }%
     \var{{\devanagarifont \numemph\va \lem \mssCaCbCc\msNa\msNc, 
\uncl{अमेध्याश्चापवित्रांश्च} \msNb, 
अमेध्याश्च पवित्राश्च \msM, अमेध्यश्चापवित्रांश्च \Ed}}% 
    \var{{\devanagarifont \numnoemph\vb \lem \mssALL, सर्वान्येतानि वर्जयेत् \msM}}% 

%Verse 5:16

{\devanagarifont शाकमूलफलानां च अभक्ष्यं परिवर्जयेत् {॥ ५:१६॥} \veg\dontdisplaylinenum }%
 
{\devanagarifont मानवेषु पुराणेषु शैवभारतसंहिते \thinspace{\dandab} \dontdisplaylinenum }%
 
{\devanagarifont कीर्तितानि विशेषेण शौचाचारमशेषतः  \danda\dontdisplaylinenum }%
     \var{{\devanagarifont \numemph\vc\textbf{विशेषेण}\lem \mssALL, मशेषेण \msM}}% 

%Verse 5:17

{\devanagarifont त्वया जिज्ञासितो ऽस्म्यद्य संक्षिप्तः कथितो मया {॥ ५:१७॥} \veg\dontdisplaylinenum }%
     \var{{\devanagarifont \numnoemph\ve\textbf{जिज्ञासितो}\lem \mssALL, जिज्ञासनो \msNc, जिज्ञासतो \Ed}}% 
    \var{{\devanagarifont \numnoemph\vf\textbf{॰क्षिप्तः}\lem \msCa\msCc\msNa\msNc\Ed, ॰क्षिप्य \msCb, ॰क्षिप्त \msNb\msM\oo 
\textbf{कथितो}\lem \mssALL, कथितं \Ed}}% 

{\devanagarifont सत्यवादी शुचिर्नित्यं ध्यानयोगरतः शुचिः \thinspace{\dandab} \dontdisplaylinenum }%
     \var{{\devanagarifont \numemph\va\textbf{॰वादी}\lem \mssALL, ॰वादि \msM\oo 
\textbf{॰रतः शुचिर्}\lem \msCa\msCb\Ed, ॰रतः शुचि \msCc\msNc, रतः शुचिन् \msNa\msNb, ॰रत शुचि \msM}}% 

%Verse 5:18

{\devanagarifont अहिंसकः शुचिर्दान्तो दयाभूतक्षमा शुचिः {॥ ५:१८॥} \veg\dontdisplaylinenum }%
     \var{{\devanagarifont \numnoemph\vc\textbf{अहिंसकः}\lem \mssALL, अहिंसक \msCb\msM\oo 
\textbf{शुचिर्दान्तो}\lem \msCa\msCb\msNa\msNb, शुचि दान्तो \msCc\msNc\msM, शुचिर्दान्तौ \Ed}}% 
    \var{{\devanagarifont \numnoemph\vd\textbf{॰भूत॰}\lem \mssALL, ॰भुत॰ \msM\oo 
\textbf{शुचिः}\lem \mssALL, शुचि \msM}}% 

{\devanagarifont सर्वेषामेव शौचानामर्थशौचं परं स्मृतम् \thinspace{\dandab} \dontdisplaylinenum }%
     \var{{\devanagarifont \numemph\vb\textbf{॰शौचं परं स्मृतम्}\lem \msCa\msNa\msNb\msNc, ॰शौचं पर स्मृतम् \msCb\msCc, 
॰शौच पर स्मृतः \msM, 
॰शौचयनं स्मृतः \Ed}}% 
    \paral{{\devanagarifontsmall \vab {\englishfont \similar\ \MANU\ 5.106:}
                         सर्वेषामेव शौचानामर्थशौचं परं स्मृतम्\thinspace{\devanagarifontsmall ।}
                         यो ऽर्थे शुचिर्हि स शुचिर्न मृद्वारिशुचिः शुचिः\thinspace{\devanagarifontsmall ॥} }}

{\devanagarifont यो ऽर्थे हि शुचिः स शुचिर्न मृद्वारिशुचिः शुचिः  \danda\dontdisplaylinenum }%
     \var{{\devanagarifont \numnoemph\vcd\textbf{यो ऽर्थे हि शुचिः स शुचिर्न}\lem \mssCaCbCc\msNc\ \unmetr, 
यो ऽर्थे हि शुचिः स शुचि न \msNa\msNb, 
यो र्थे शुचि हि स शुद्धि \msM, 
यो ऽर्थे हि सुशुचिर्विप्र न \Ed}}% 
    \var{{\devanagarifont \numnoemph\vd\textbf{॰शुचिः शुचिः}\lem \mssCaCbCc\msNa\msNc, शुचि शुचिः \msNb, ॰शुचि शुचि \msM, ॰शुचिः शुचि \Ed}}% 

%Verse 5:19

{\devanagarifont कायवाङ्मनसां शौचं स शुचिः सर्ववस्तुषु {॥ ५:१९॥} \veg\dontdisplaylinenum }%
     \var{{\devanagarifont \numnoemph\ve\textbf{वाङ्मनसां शौचं}\lem \mssALL, वाङ्मणसा शुद्धि \msM}}% 
    \var{{\devanagarifont \numnoemph\vf\textbf{शुचिः}\lem \mssALL, शुचि \msCc\msM\oo 
\textbf{वस्तुषु}\lem \mssALL, वस्तुषुः \msNc, वस्तुशु \msM}}% 
    \lacuna{\devanagarifontsmall \vcd {\englishfont \Ed\ adds here, after pādas cd:} शौचाशौचविधिर्ज्ञात्वा मुच्यते सर्वकिल्बिषात् }%
  
\vfill
\pageparbreak
\vers

\nemslokalong


\ujvers\nemsloka {
{\devanagarifont शौचाशौचविधिज्ञमानव यदि कालक्षये निश्चयः }%
  \dontdisplaylinenum}    \var{{\devanagarifont \numemph\va\textbf{शौचाशौच॰}\lem \mssALL, शौचाशुच \msCb\oo 
\textbf{यदि}\lem \mssALL, यदिः \msM\oo 
\textbf{कालक्षये निश्चयः}\lem \msNaacorr\msNc, 
कालक्षयैर्निश्चयः \msCa\msCb\msNapcorr, 
कालक्षयेन्निश्चयः \msCc\msNb, 
कालक्षयानिश्चयः \msM, 
कालक्षयेतिश्च यः \Ed}}% 


\nemslokab

{\devanagarifont सौभाग्यत्वमवाप्नुवन्ति सततं कीर्तिर्यशोऽलङ्कृतम्  \danda\dontdisplaylinenum }%
     \var{{\devanagarifont \numnoemph\vb\textbf{कीर्तिर्यशो॰}\lem \msCb\msNa\msNb\msNc\Ed, कीर्तियशो॰ \msCa\msCc \unmetr, कीर्तिर्यषा॰ \msM\oo 
\textbf{॰लंकृतम्}\lem \msM, ॰लङ्कृतः \msCa\msCc\msNa\msNb\msNc\Ed, ॰लकृतः \msCb}}% 
    \paral{{\devanagarifontsmall \vb {\englishfont \similar\ 4.67b above:}
                         लोके ऽनिन्दनमाप्नुवन्ति सततं कीर्तिर्यशोऽलंकृतम् }}

\nemslokac

{\devanagarifont प्राप्तं तेन इहैव पुण्यसकलं सद्धर्मशास्त्रेरितं }%
  \dontdisplaylinenum    \var{{\devanagarifont \numnoemph\vc\textbf{सद्धर्म॰}\lem \mssALL, य धर्म॰ \msM\oo 
\textbf{॰एरितम्}\lem \mssALL, ॰ओदितः \Ed}}% 

%Verse 5:20


\nemslokad

{\devanagarifont जीवान्ते च परत्रमीहितगतिं प्राप्नोति निःसंशयम् {॥ ५:२०॥} \veg\dontdisplaylinenum }%
     \var{{\devanagarifont \numnoemph\vd\textbf{परत्रमीहित॰}\lem \mssALL, 
परत्रमीहत॰ \msM, पवित्रमीहित॰ \Ed\oo 
\textbf{॰गतिं}\lem \eme, ॰गतिः \mssCaCbCc\msNa\msNb\msNc\msM\Ed\oo 
\textbf{निःसंशयम्}\lem \msCa\msNb\msNc, निःसंशयः \msCb\msCc\msNa\msM\Ed}}% 

\vers


{\devanagarifont 
\jump
\begin{center}
\ketdanda~इति वृषसारसंग्रहे शौचाचारविधिर्नामाध्यायः पञ्चमः~\ketdanda
\end{center}
\dontdisplaylinenum\vers  }%
     \var{{\devanagarifont \numnoemph{\englishfont \Colo:}\textbf{॰विधिर्नमा॰}\lem \msCa, ॰विधिनामा॰ \msCb\msCc\msNa\msNc\msM, \uncl{विंधि}नामा॰ \msNb, ॰विधिर्नाम \Ed\oo 
\textbf{॰ध्ययः पञ्चमः}\lem \mssALL, ॰ध्यायः पञ्चमः श्लोक २५ \msM, 
पञ्चमो ऽध्यायः \Ed}}% 
\bekveg\szamveg
\vfill
\phpspagebreak

\versno=0\fejno=6
\thispagestyle{empty}

\centerline{\Large\devanagarifontbold [   षष्ठो ऽध्यायः  ]}{\vrule depth10pt width0pt} \fancyhead[CO]{{\footnotesize\devanagarifont वृषसारसंग्रहे  }}
\fancyhead[CE]{{\footnotesize\devanagarifont षष्ठो ऽध्यायः  }}
\fancyhead[LE]{}
\fancyhead[RE]{}
\fancyhead[LO]{}
\fancyhead[RO]{}
\szam\bek


\nemslokanormal



\alalfejezet{नियमेष्विज्या (२)}
\vers


{\devanagarifont अथ पञ्चविधामिज्यां प्रवक्ष्यामि द्विजोत्तम \thinspace{\dandab} \dontdisplaylinenum }%
     \var{{\devanagarifont \numemph\va\textbf{॰मिज्यां}\lem \msCb, ॰मीज्यां \msCa\msCc\msNa\msNb\msNc\Ed}}% 
    \var{{\devanagarifont \numnoemph\vb\textbf{॰त्तम}\lem \mssALL, ॰त्तमः \msNb\msNc}}% 
    \lacuna{\devanagarifontsmall {\englishfont Witnesses used for this chapter: \msCa\ ff.\thinspace 202r--203r, 
                                              \msCb\ ff.\thinspace 209r--209v, 
                                              \msCc\ ff.\thinspace 278r--279r,
                                              \msNa\ ff.\thinspace 9v--10v, 
                                              \msNb\ exp.\thinspace 51 (lower--upper) -- 52 (lower),
                                              \msNc\ ff.\thinspace 218r--218v,
                                              \Ed\ pp.\thinspace 599--601;  
                                              \mssCaCbCc\ = \msCa + \msCb + \msCc} }%
  
%Verse 6:1

{\devanagarifont धर्ममोक्षप्रसिद्ध्यर्थं शृणुष्वावहितो द्विज {॥ ६:१॥} \veg\dontdisplaylinenum }%
     \var{{\devanagarifont \numnoemph\vc\textbf{॰मोक्षप्रसिद्ध्यर्थं}\lem \mssCaCbCc\msNc, ॰मोक्षप्रसिद्ध्यर्थ \msNa\msNb, 
॰मोक्षेशसिद्ध्यर्थं \Ed}}% 
    \var{{\devanagarifont \numnoemph\vd\textbf{द्विज}\lem \mssALL, भव \Ed}}% 

{\devanagarifont अर्थयज्ञः क्रियायज्ञो जपयज्ञस्तथैव च \thinspace{\dandab} \dontdisplaylinenum }%
     \var{{\devanagarifont \numemph\va\textbf{अर्थयज्ञः}\lem \msCa\msCc\msNa, अनर्थयज्ञः \msCb, 
अर्थयज्ञ \msNb\msNc, अर्थयज्ञ॰ \Ed}}% 

%Verse 6:2

{\devanagarifont ज्ञानं ध्यानं च पञ्चैतत्प्रवक्ष्यामि पृथक्पृथक् {॥ ६:२॥} \veg\dontdisplaylinenum }%
     \var{{\devanagarifont \numnoemph\vc\textbf{ज्ञानं}\lem \mssALL, ज्ञान \msCc\msNc}}% 


\alalalfejezet{अर्थयज्ञः}

{\devanagarifont अग्न्युपासनकर्मादि अग्निहोत्रक्रतुक्रिया \thinspace{\dandab} \dontdisplaylinenum }%
     \var{{\devanagarifont \numemph\vb\textbf{अग्नि॰}\lem \mssALL, \uncl{अ}\lac\  \msCa, \lk\lk\ \msNb\oo 
\textbf{॰क्रिया}\lem \mssALL, ॰क्रियाः \msCb\msCc}}% 

%Verse 6:3

{\devanagarifont अष्टका पार्वणी श्राद्धं द्रव्ययज्ञः स उच्यते {॥ ६:३॥} \veg\dontdisplaylinenum }%
     \var{{\devanagarifont \numnoemph\vc\textbf{पार्वणी}\lem \mssALL, पर्वणी \msCb, \uncl{पर्वणी} \msNb}}% 
    \var{{\devanagarifont \numnoemph\vd\textbf{॰यज्ञः}\lem \mssALL, ॰यज्ञ \msCc, \lk\lk\ \msNb}}% 


\alalalfejezet{क्रियायज्ञः}

{\devanagarifont आरामोद्यानवापीषु देवतायतनेषु च \thinspace{\dandab} \dontdisplaylinenum }%
     \var{{\devanagarifont \numemph\vb\textbf{॰यतनेषु}\lem \msCb\msCc\Ed, ॰लयनेषु \msCa\msNa\msNc, ॰यत\lk\lk\ \msNb}}% 

%Verse 6:4

{\devanagarifont स्वहस्तकृतसंस्कारः क्रियायज्ञ स उच्यते {॥ ६:४॥} \veg\dontdisplaylinenum }%
     \var{{\devanagarifont \numnoemph\vc\textbf{॰हस्त॰}\lem \mssALL, \lk\lk\ \msNb, ॰हस्तैः \Ed}}% 

\vfill
\pageparbreak
\vers


\alalalfejezet{जपयज्ञः}

{\devanagarifont जपयज्ञं ततो वक्ष्ये स्वर्गमोक्षफलप्रदम् \thinspace{\dandab} \dontdisplaylinenum }%
     \var{{\devanagarifont \numemph\va\textbf{॰यज्ञं ततो}\lem \mssALL, ॰यज्ञं तपो \msCb ॰यज्ञस्ततो \msCc}}% 

{\devanagarifont वेदाध्ययन कर्तव्यं शिवसंहितमेव च  \danda\dontdisplaylinenum }%
     \var{{\devanagarifont \numnoemph\vc\textbf{वेदा॰}\lem \mssALL, अदा॰ \msNb}}% 

%Verse 6:5

{\devanagarifont इतिहासपुराणं च जपयज्ञः स उच्यते {॥ ६:५॥} \veg\dontdisplaylinenum }%
     \var{{\devanagarifont \numnoemph\ve\textbf{॰पुराणं च}\lem \mssALL, ॰पुराणश्च \Ed}}% 
    \var{{\devanagarifont \numnoemph\vf\textbf{॰यज्ञः}\lem \mssALL, ॰यज्ञ \msCc}}% 


\alalalfejezet{ज्ञानयज्ञः}

{\devanagarifont इदं कर्म अकर्मेदमूहापोहविशारदः \thinspace{\dandab} \dontdisplaylinenum }%
     \var{{\devanagarifont \numemph\va\textbf{कर्म}\lem \mssALL, क्रमम् \Ed}}% 

%Verse 6:6

{\devanagarifont शास्त्रचक्षुः समालोक्य ज्ञानयज्ञः स उच्यते {॥ ६:६॥} \veg\dontdisplaylinenum }%
     \var{{\devanagarifont \numnoemph\vc\textbf{॰चक्षुः}\lem \mssALL, ॰चक्षु \msCc}}% 
    \var{{\devanagarifont \numnoemph\vd\textbf{॰यज्ञः}\lem \mssALL, ॰यज्ञ \msCc, ॰\uncl{यज्ञस्} \msNb}}% 


\alalalfejezet{ध्यानयज्ञः}

{\devanagarifont ध्यानयज्ञं समासेन कथयिष्यामि ते शृणु \thinspace{\dandab} \dontdisplaylinenum }%
     \var{{\devanagarifont \numemph\va\textbf{॰यज्ञं}\lem \mssALL, ॰यज्ञ \msCc\msNb}}% 

{\devanagarifont ध्यानं पञ्चविधं चैव कीर्तितं हरिणा पुरा  \danda\dontdisplaylinenum }%
     \var{{\devanagarifont \numnoemph\vc\textbf{ध्यानं}\lem \mssALL, ध्यान \msNa\msNc}}% 

%Verse 6:7

{\devanagarifont सूर्यः सोमो ऽग्नि स्फटिकः सूक्ष्मं तत्त्वं च पञ्चमम् {॥ ६:७॥} \veg\dontdisplaylinenum }%
     \var{{\devanagarifont \numnoemph\ve\textbf{सोमो}\lem \msCa\msCc\msNa\msNc, सोमा॰ \msCb\msNb\Ed}}% 
    \var{{\devanagarifont \numnoemph\vf \lem \msCb, 
सूक्ष्मं त\uncl{त्व}\lac  ञ्चमम् \msCa, 
सूक्ष्मतत्त्वं च पञ्चमः \msCc\msNa\msNb, 
सूक्ष्मं तत्त्वञ्च पञ्चमः \msNc, 
सूक्ष्मां तत्त्वश्च पञ्चमम् \Ed}}% 

{\devanagarifont सूर्यमण्डलमादौ तु तत्त्वं प्रकृतिरुच्यते \thinspace{\dandab} \dontdisplaylinenum }%
 
%Verse 6:8

{\devanagarifont तस्य मध्ये शशिं ध्यायेत्तत्त्वं पुरुष उच्यते {॥ ६:८॥} \veg\dontdisplaylinenum }%
     \var{{\devanagarifont \numemph\vc\textbf{शशिं}\lem \mssALL, शशि \msNb, शशिंन् \msNc}}% 
    \var{{\devanagarifont \numnoemph\vcd\textbf{ध्यायेत्त॰}\lem \mssALL, ध्याये त॰ \msCc}}% 

{\devanagarifont चन्द्रमण्डलमध्ये तु ज्वालामग्निं विचिन्तयेत् \thinspace{\dandab} \dontdisplaylinenum }%
     \var{{\devanagarifont \numemph\vb\textbf{ज्वालामग्निं}\lem \mssALL, जालामग्नि \msNc}}% 

%Verse 6:9

{\devanagarifont प्रभुतत्त्वः स विज्ञेयो जन्ममृत्युविनाशनः {॥ ६:९॥} \veg\dontdisplaylinenum }%
     \var{{\devanagarifont \numnoemph\vc\textbf{॰तत्त्वः}\lem \mssCaCbCc\msNc, ॰तत्व \msNa, ॰तत्वं \msNb\Ed}}% 
    \var{{\devanagarifont \numnoemph\vd\textbf{॰नाशनः}\lem \mssALL, ॰नाशनम् \msCc\Ed}}% 

{\devanagarifont अग्निमण्डलमध्ये तु ध्यायेत्स्फटिक निर्मलम् \thinspace{\dandab} \dontdisplaylinenum }%
     \var{{\devanagarifont \numemph\vb\textbf{ध्यायेत्स्फटिक}\lem \msCapcorr\msCb\msNa\msNb\msNc, ध्यायेत्स्फटि \msCaacorr, 
ध्याये स्फटिक \msCc\Ed\oo 
\textbf{॰मलम्}\lem \mssALL, ॰मलः \msNa, ॰\uncl{मलः} \msNc}}% 

%Verse 6:10

{\devanagarifont विद्यातत्त्वः स विज्ञेयः कारणमजमव्ययम् {॥ ६:१०॥} \veg\dontdisplaylinenum }%
     \var{{\devanagarifont \numnoemph\vc\textbf{तत्त्वः स}\lem \msCb\msNa\msNb\msNc, त\uncl{त्वन्}\lac\  \msCa, तत्व स \msCc, तत्वं स \Ed}}% 
    \var{{\devanagarifont \numnoemph\vd\textbf{॰जमव्ययम्}\lem \mssALL, ॰मव्ययं \msCc}}% 

{\devanagarifont विद्यामण्डलमध्ये तु ध्यायेत्तत्त्वमनुत्तमम् \thinspace{\dandab} \dontdisplaylinenum }%
     \var{{\devanagarifont \numemph\vab\textbf{ध्यायेत्त॰}\lem \mssALL, ध्याये त॰ \msCc}}% 

{\devanagarifont अकीर्तितमनौपम्यं शिवमक्षयमव्ययम्  \danda\dontdisplaylinenum }%
     \paral{{\devanagarifontsmall \vcd {\englishfont \DHARMP\ 4.14ab: } अकीर्तितमनौपम्यं पञ्चमं शिवमण्डलम् }}

%Verse 6:11

{\devanagarifont पञ्चमं ध्यानयज्ञस्य तत्त्वमुक्तं समासतः {॥ ६:११॥} \veg\dontdisplaylinenum }%
     \var{{\devanagarifont \numnoemph\ve\textbf{॰यज्ञस्य}\lem \mssALL, ॰यज्ञञ्च \msCc\Ed}}% 
    \var{{\devanagarifont \numnoemph\vf\textbf{समासतः}\lem \mssALL, सनातनः \Ed}}% 

{\devanagarifont विगतराग उवाच {\dandab}\dontdisplaylinenum  }%
 
{\devanagarifont एकैकस्य तु तत्त्वस्य फलं कीर्तय कीदृशम् \thinspace{\danda} \dontdisplaylinenum }%
     \var{{\devanagarifont \numemph\va\textbf{तु}\lem \conj, त्रि॰ \mssCaCbCc\msNa\msNb\msNc, हि \Ed}}% 

%Verse 6:12

{\devanagarifont कानि लोकाः प्रपद्यन्ते कालं वास्य तपोधन {॥ ६:१२॥} \veg\dontdisplaylinenum }%
     \var{{\devanagarifont \numnoemph\vc\textbf{लोकाः}\lem \msCa\msNa\msNc, लोका \msCb\msCc\msNb\Ed\oo 
\textbf{प्रपद्यन्ते}\lem \mssALL, प्र\lk\lk\lk\ \msCa}}% 
    \var{{\devanagarifont \numnoemph\vd\textbf{॰धन}\lem \mssALL, ॰धनः \msCb\msNc}}% 

{\devanagarifont अनर्थयज्ञ उवाच {\dandab}\dontdisplaylinenum  }%
 
{\devanagarifont ब्रह्मलोकं तु प्रथमं तत्त्वप्रकृतिचिन्तया \thinspace{\danda} \dontdisplaylinenum }%
     \var{{\devanagarifont \numemph\vab\textbf{प्रथमं तत्त्व॰}\lem \mssALL, 
\om\ \msNaacorr, प्रथमं तत्त्वं \Ed\oo 
\textbf{प्रकृतिचिन्तया}\lem \mssALL, च कृतिचिन्तय \Ed}}% 

%Verse 6:13

{\devanagarifont कल्पकोटिसहस्राणि शिववन्मोदते सुखी {॥ ६:१३॥} \veg\dontdisplaylinenum }%
     \var{{\devanagarifont \numnoemph\vd\textbf{सुखी}\lem \mssALL, सुखम् \Ed}}% 

{\devanagarifont द्वितीयं तत्त्व पुरुषं ध्यायमानो मृतो यदि \thinspace{\dandab} \dontdisplaylinenum }%
 
%Verse 6:14

{\devanagarifont विष्णुलोकमितो याति कल्पकोट्ययुतं सुखी {॥ ६:१४॥} \veg\dontdisplaylinenum }%
     \var{{\devanagarifont \numemph\vc\textbf{याति}\lem \mssALL, यान्ति \Ed}}% 

{\devanagarifont प्रभुतत्त्वं तृतीयं तु ध्यायमानो मरिष्यति \thinspace{\dandab} \dontdisplaylinenum }%
     \var{{\devanagarifont \numemph\va\textbf{॰तत्त्वं}\lem \mssALL, ॰तत्व \msCc\oo 
\textbf{तृतीयं}\lem \mssALL, तृतीयस् \Ed}}% 
    \var{{\devanagarifont \numnoemph\vb \lem \mssALL, ध्याय\lk\lk \lk रिष्यति \msCa, 
धयायामानो मरिष्यति \Ed}}% 

%Verse 6:15

{\devanagarifont शिवलोके वसेन्नित्यं कल्पकोट्ययुतं शतम् {॥ ६:१५॥} \veg\dontdisplaylinenum }%
     \var{{\devanagarifont \numnoemph\vc\textbf{शिवलोके}\lem \mssALL, शिवलोक \msCb, रुद्रलोके \Ed\oo 
\textbf{वसेन्नि॰}\lem \mssALL, वसे नि॰ \msCc}}% 
    \var{{\devanagarifont \numnoemph\vd\textbf{॰युतं}\lem \mssALL, ॰युत \msNb}}% 

{\devanagarifont विद्यातत्त्वामृतं ध्यायेत्सदाशिवमनामयम् \thinspace{\dandab} \dontdisplaylinenum }%
     \var{{\devanagarifont \numemph\va\textbf{॰तत्त्वामृतं}\lem \mssALL, ॰तत्वमृतन् \msCc, ॰तत्त्वामतं \Ed}}% 

%Verse 6:16

{\devanagarifont अक्षयं लोकमाप्नोति कल्पानान्तपरं तथा {॥ ६:१६॥} \veg\dontdisplaylinenum  }%
     \var{{\devanagarifont \numnoemph\vc\textbf{अक्षयं}\lem \mssALL, अक्षय॰ \Ed}}% 

{\devanagarifont पञ्चमं शिवतत्त्वं तु सूक्ष्मं चात्मनि संस्थितम् \thinspace{\dandab} \dontdisplaylinenum }%
 
%Verse 6:17

{\devanagarifont न कालसंख्या तत्रास्ति शिवेन सह मोदते {॥ ६:१७॥} \veg\dontdisplaylinenum }%
 
\nemslokalong


\ujvers\nemsloka {
{\devanagarifont पञ्चध्यानाभियुक्तो भवति च न पुनर्जन्मसंस्कारबन्धः }%
  \dontdisplaylinenum}    \var{{\devanagarifont \numemph\va\textbf{॰युक्तो}\lem \mssALL, ॰यु\lk\ \msCa\ \toplost, ॰युक्तौ \Ed\oo 
\textbf{च}\lem \mssALL, \om\ \msCb\Ed\oo 
\textbf{पुनर्जन्म॰}\lem \mssALL, 
पुन\uncl{ज}न्म॰ \msCa\ \toplost, पुनजन्म॰ \msCc}}% 


\nemslokab

{\devanagarifont जिज्ञास्यन्तां द्विजेन्द्र भवदहनकरः प्रार्थनाकल्पवृक्षः  \danda\dontdisplaylinenum }%
     \var{{\devanagarifont \numnoemph\vb\textbf{जिज्ञास्यन्तां}\lem \msCa\msNb\msNc\Ed, जिज्ञास्यतां \msCb\msNa\ \unmetr, जिज्ञास्यन्ता \msCc}}% 

\nemslokac

{\devanagarifont जन्मेनैकेन मुक्तिर्भवति किमु न वा मानवाः साधयन्तु }%
  \dontdisplaylinenum    \var{{\devanagarifont \numnoemph\vc\textbf{जन्मेनैकेन}\lem \msCb\msNb\msNc\Ed, जन्मनैकेन \msCa\msCc\msNa\ \unmetr\oo 
\textbf{मुक्तिरभ्॰}\lem \mssALL, मुक्ति भ्॰ \msCc\oo 
\textbf{न वा}\lem \mssALL, भवा \msNa\oo 
\textbf{मानवाः}\lem \msCa\msNa\msNb\msNc, मानमानवाः \msCb, मानवा \msCc, मानव \Ed}}% 

%Verse 6:18


\nemslokad

{\devanagarifont प्रत्यक्षान्नानुमानं सकलमलहरं स्वात्मसंवेदनीयम् {॥ ६:१८॥} \veg\dontdisplaylinenum }%
     \var{{\devanagarifont \numnoemph\vd\textbf{प्रत्यक्षा॰}\lem \mssALL, प्रत्यक्ष॰ \msNa\oo 
\textbf{॰वेदनीयम्}\lem \msCb\msNa\msNb, ॰वेदनीयः \msCa\msCc\msNc, ॰वेदनीय \Ed}}% 

\nemslokanormal


\vers



\alalfejezet{नियमेषु तपः (३)}
{\devanagarifont मानसं तप आदौ तु द्वितीयं वाचिकं तपः \thinspace{\dandab} \dontdisplaylinenum }%
     \var{{\devanagarifont \numemph\va\textbf{॰तप}\lem \mssALL, ॰तपम् \Ed}}% 

{\devanagarifont कायिकं च तृतीयं तु मनोवाक्कर्म तत्परम्  \danda\dontdisplaylinenum }%
     \var{{\devanagarifont \numnoemph\vc \lem \mssALL, 
मानसं तप आदौ तु \msNb\ {\englishfont (eyeskip)}}}% 
    \var{{\devanagarifont \numnoemph\vd\textbf{मनोवाक्कर्म}\lem \msCa\msNc\Ed, मनोक्कर्म \msCb, म्मनोवाकर्म॰ \msCc, मनोवाक्काय॰ \msNa\msNb\oo 
\textbf{॰परम्}\lem \msCc, ॰परः \msCa\msCb\msNa\msNb\msNc\Ed}}% 

%Verse 6:19

{\devanagarifont कायिकं वाचिकं चैव तपो मिश्रक पञ्चमम् {॥ ६:१९॥} \veg\dontdisplaylinenum }%
     \var{{\devanagarifont \numnoemph\ve\textbf{कायिकं}\lem \mssALL, कायिक \msNa}}% 

{\devanagarifont मनःसौम्यं प्रसादश्च आत्मनिग्रहमेव च \thinspace{\dandab} \dontdisplaylinenum }%
     \var{{\devanagarifont \numemph\va\textbf{॰सौम्यं}\lem \msNc, ॰सौम्य॰ \msCa\msCb\msNa\msNb\Ed, ॰सौम्\uncl{य}॰ \msCc\ \toplost\oo 
\textbf{प्रसादश्च}\lem \msCa\msCc\msNa\msNc, प्रसादं च \msCb\Ed, प्रदानश्च \msNb}}% 

%Verse 6:20

{\devanagarifont मौनं भावविशुद्धिश्च पञ्चैतत्तप मानसम् {॥ ६:२०॥} \veg\dontdisplaylinenum }%
     \var{{\devanagarifont \numnoemph\vc\textbf{मौनं}\lem \mssALL, मौन\lk  \Ed\oo 
\textbf{॰शुद्धिश्च}\lem \mssALL, ॰शुद्धिं च \msCc\Ed}}% 
    \var{{\devanagarifont \numnoemph\vd\textbf{पञ्चैतत्}\lem \msCa\msNb\msNc, पञ्चैते \msCb\msNa, पञ्चेतत् \msCc, पञ्चैतन् \Ed}}% 
    \paral{{\devanagarifontsmall \vo {\englishfont \similar\ \MBH\ 6.39.16 (\BHG\ 17.16):}
                 मनःप्रसादः सौम्यत्वं मौनमात्मविनिग्रहः\thinspace{\devanagarifontsmall ।}
                 भावसंशुद्धिरित्येतत्तपो मानसमुच्यते\thinspace{\devanagarifontsmall ॥} }}

\vfill
\pageparbreak
\vers

{\devanagarifont अनुद्वेगकरा वाणी प्रियं सत्यं हितं च यत् \thinspace{\dandab} \dontdisplaylinenum }%
 
%Verse 6:21

{\devanagarifont स्वाध्यायाभ्यसनं चैव वाचिकं तप उच्यते {॥ ६:२१॥} \veg\dontdisplaylinenum }%
     \var{{\devanagarifont \numemph\vc\textbf{॰भ्यसनं चैव}\lem \mssALL, ॰भ्यसन\lk\lk\ \msCa, 
॰भ्यस\uncl{नं} चैव \msNb}}% 
    \paral{{\devanagarifontsmall \vcd {\englishfont \similar\ \MBH\ 6.39.15cd (\BHG\ 17.15):}
                                  अनुद्वेगकरं वाक्यं सत्यं प्रियहितं च यत्\thinspace{\devanagarifontsmall ।}
                                  स्वाध्यायाभ्यसनं चैव वाङ्मयं तप उच्यते\thinspace{\devanagarifontsmall ॥} }}

{\devanagarifont आर्जवं च अहिंसा च ब्रह्मचर्यं सुरार्चनम् \thinspace{\dandab} \dontdisplaylinenum }%
     \var{{\devanagarifont \numemph\va \lem \mssALL, आर्जवत्वमहिंसाश्च \Ed}}% 
    \var{{\devanagarifont \numnoemph\vb\textbf{॰चर्यं}\lem \mssALL, ॰चर्य \msCc\Ed}}% 

%Verse 6:22

{\devanagarifont शौचं पञ्चममित्येतत्कायिकं तप उच्यते {॥ ६:२२॥} \veg\dontdisplaylinenum }%
     \var{{\devanagarifont \numnoemph\vc\textbf{शौचं}\lem \mssALL, शौच \Ed}}% 
    \paral{{\devanagarifontsmall \vo {\englishfont \compare\ \MBH\ 6.39.14 (\BHG\ 17.14):}
                          देवद्विजगुरुप्राज्ञपूजनं शौचमार्जवम्\thinspace{\devanagarifontsmall ।}
                          ब्रह्मचर्यमहिंसा च शारीरं तप उच्यते\thinspace{\devanagarifontsmall ॥} }}

{\devanagarifont इष्टं कल्याणभावं च धन्यं पथ्यं हितं वदेत् \thinspace{\dandab} \dontdisplaylinenum }%
     \var{{\devanagarifont \numemph\va\textbf{इष्टं}\lem \mssALL, इष्ट \msCc\msNb\oo 
\textbf{॰भावं}\lem \mssALL, ॰भावश् \Ed}}% 
    \var{{\devanagarifont \numnoemph\vb\textbf{पथ्यं}\lem \mssALL, सत्यं \Ed}}% 

%Verse 6:23

{\devanagarifont मनोमिश्रक पञ्चैतत्तप उक्तं महर्षिभिः {॥ ६:२३॥} \veg\dontdisplaylinenum }%
     \var{{\devanagarifont \numnoemph\vc\textbf{मनो॰}\lem \mssALL, मन॰ \Ed\oo 
\textbf{पञ्चैतत्}\lem \mssALL, पञ्चेतत् \msNc, पञ्चैतान् \Ed}}% 
    \var{{\devanagarifont \numnoemph\vd \lem \mssALL, तपमुक्तं महिर्षिभिः \Ed}}% 

{\devanagarifont स्वस्ति मङ्गलमाशीर्भिरतिथिगुरुपूजनम् \thinspace{\dandab} \dontdisplaylinenum }%
     \var{{\devanagarifont \numemph\va\textbf{॰शीर्भि॰}\lem \msCa\Ed, ॰शीभि॰ \msCb\msCc\msNa\msNb\msNc}}% 
    \var{{\devanagarifont \numnoemph\vb\textbf{॰तिथि॰}\lem \mssALL, ॰तिथिं \Ed}}% 
    \paral{{\devanagarifontsmall \vab {\englishfont \compare\ \SDHS\ 11.79:}
                 नमस्काराभिवादेषु स्वस्तिमङ्गलवाचकैः\thinspace{\devanagarifontsmall ।}
                 शिवं भवतु सर्वत्र प्रब्रूयात्सर्वकर्मसु\thinspace{\devanagarifontsmall ॥} }}

%Verse 6:24

{\devanagarifont कायमिश्रक पञ्चैतत्तप उक्तं महात्मभिः {॥ ६:२४॥} \veg\dontdisplaylinenum }%
     \var{{\devanagarifont \numnoemph\vc\textbf{॰मिश्रक}\lem \mssALL, ॰\lk\lk क \msCa, ॰मित्यश्रक \msCb\oo 
\textbf{पञ्चैतत्}\lem \mssALL, पञ्चैतन् \Ed}}% 
    \var{{\devanagarifont \numnoemph\vd\textbf{तप उक्तं}\lem \mssALL, तपमुक्तं \Ed}}% 

{\devanagarifont मण्डूकयोगी हेमन्ते ग्रीष्मे पञ्चतपास्तथा \thinspace{\dandab} \dontdisplaylinenum }%
     \var{{\devanagarifont \numemph\vb\textbf{ग्रीष्मे}\lem \mssALL, गृष्मे \Ed}}% 
    \paral{{\devanagarifontsmall \vab {\englishfont \similar\ \MBH\ Appendices 15.801:}
                                 मण्डूकशायी हेमन्ते ग्रीष्मे पञ्चतपा भवेत
                     {\englishfont \similar\ \UMS\ 6.26ab:}मण्डूकयोगो हेमन्ते ग्रीष्मे पञ्चतपास्तथा;
                     {\englishfont \compare\ \SDHSAMGR\ 9.32ab:}
                         अभ्रावकाश्यं शीतोष्णे पञ्चाग्निर्जलशायिता }}

%Verse 6:25

{\devanagarifont अभ्रावकाशो वर्षासु तपःसाधनमुच्यते {॥ ६:२५॥} \veg\dontdisplaylinenum }%
     \var{{\devanagarifont \numnoemph\vc\textbf{॰वकाशो}\lem \eme, ॰वकाशे \mssCaCbCc\msNa\msNb\msNc\Ed}}% 
    \var{{\devanagarifont \numnoemph\vd\textbf{तप॰}\lem \mssALL, तप \msCc\oo 
\textbf{साधनमु॰}\lem \msCa\msNa\msNc\Ed, साधन उ॰ \msCb\msCc\msNb}}% 

{\devanagarifont स्वमांसोद्धृत्य दानं च हस्तपादशिरस्तथा \thinspace{\dandab} \dontdisplaylinenum }%
     \var{{\devanagarifont \numemph\va\textbf{दानं}\lem \mssALL, \uncl{दान} \msNb\ \toplost, दानश् \Ed}}% 

%Verse 6:26

{\devanagarifont पुष्पमुत्पाद्य दानंच सर्वे ते तपसाधनाः {॥ ६:२६॥} \veg\dontdisplaylinenum }%
     \var{{\devanagarifont \numnoemph\vc\textbf{दानं}\lem \mssALL, दानश् \Ed}}% 
    \var{{\devanagarifont \numnoemph\vd\textbf{तप॰}\lem \Ed, तपः \mssCaCbCc\msNa\msNb\msNc\ \unmetr}}% 

{\devanagarifont कृच्छ्रातिकृच्छ्रं नक्तं च तप्तकृच्छ्रमयाचितम् \thinspace{\dandab} \dontdisplaylinenum }%
     \var{{\devanagarifont \numemph\va\textbf{कृच्छ्रातिकृच्छ्रं}\lem \msCa\msCb\msNa\Ed, 
कृच्छ्रादिकृच्छ्र \msCc, कृच्छ्रातिकृच्छ्र \msNb, कृच्छातिकृच्छं \msNc}}% 
    \var{{\devanagarifont \numnoemph\vb\textbf{॰याचितम्}\lem \mssALL, ॰याचितः \Ed}}% 

%Verse 6:27

{\devanagarifont चान्द्रायणं पराकं च तपः सांतपनादयः {॥ ६:२७॥} \veg\dontdisplaylinenum }%
     \var{{\devanagarifont \numnoemph\vc\textbf{चान्द्रायणं पराकं}\lem \msCa\msCc\msNb\msNc, चान्द्रायनं पराकं \msCb, 
चन्द्रायणं पराकं \msNa, चान्द्रायणवराकश् \Ed}}% 
    \var{{\devanagarifont \numnoemph\vd \lem \mssALL, तपसान्तपनादयः \msCc\Ed}}% 

\nemslokalong


\ujvers\nemsloka {
{\devanagarifont येनेदं तप तप्यते सुमनसा संसारदुःखच्छिदम् }%
  \dontdisplaylinenum}    \var{{\devanagarifont \numemph\va\textbf{तप त॰}\lem \Ed, तपस्त॰ \mssCaCbCc\msNa\msNb\msNc\ \unmetr\oo 
\textbf{॰मनसा}\lem \eme, ॰मनसः \mssCaCbCc\msNa\msNb\msNc\Ed}}% 


\nemslokab

{\devanagarifont आशापाश विमुच्य निर्मलमतिस्त्यक्त्वा जघन्यं फलम्  \danda\dontdisplaylinenum }%
     \var{{\devanagarifont \numnoemph\vb\textbf{निर्मलमति॰}\lem \mssALL, निर्मलर्मति॰ \msCb\oo 
\textbf{जघन्यं}\lem \mssALL, जगत्यं \Ed}}% 

\nemslokac

{\devanagarifont स्वर्गाकाङ्क्ष्यनृपत्वभोगविषयं सर्वान्तिकं तत्फलं }%
  \dontdisplaylinenum    \var{{\devanagarifont \numnoemph\vc\textbf{॰काङ्क्ष्य॰}\lem \mssALL, ॰कांक्ष॰ \Ed\oo 
\textbf{सर्वान्तिकं}\lem \mssALL, सर्वार्त्तिकं \msCb}}% 

%Verse 6:28


\nemslokad

{\devanagarifont जन्तुः शाश्वतजन्ममृत्युभवने तन्निष्ठसाध्यं वहेत् {॥ ६:२८॥} \veg\dontdisplaylinenum }%
     \var{{\devanagarifont \numnoemph\vd\textbf{॰भवने}\lem \mssALL, ॰भवेने \msNc\oo 
\textbf{॰साध्यं वहेत्}\lem \msCc\msNa\msNb\msNc, ॰\uncl{साध्यम्}\lk\lk\ \msCa, 
॰साध्य वहेत् \msCb, ॰साध्यं वदेत् \Ed}}% 

\vers


{\devanagarifont 
\jump
\begin{center}
\ketdanda~इति वृषसारसंग्रहे षष्ठो ऽध्यायः~\ketdanda
\end{center}
\dontdisplaylinenum\vers  }%
 