\fejno=0\versno=0
\centerline{\Huge\devanagarifontbold वृषसारसंग्रहः  }

 
{\vrule depth10pt width0pt}

\nemslokanormal

\versno=0\fejno=11
\thispagestyle{empty}

\centerline{\Large\devanagarifontbold [   एकादशमो ऽध्यायः  ]}{\vrule depth10pt width0pt} \fancyhead[CO]{{\footnotesize\devanagarifont वृषसारसंग्रहे  }}
\fancyhead[CE]{{\footnotesize\devanagarifont एकादशमो ऽध्यायः  }}
\fancyhead[LE]{}
\fancyhead[RE]{}
\fancyhead[LO]{}
\fancyhead[RO]{}
\szam\bek


\vers



\alalfejezet{चतुराश्रमधर्मविधानः}
{\devanagarifont देव्युवाच {\dandab}\dontdisplaylinenum  }%
 
{\devanagarifont सर्वयज्ञः परश्रेष्ठ अस्ति अन्यः सुरोत्तम \thinspace{\danda} \dontdisplaylinenum  }%
     \var{{\devanagarifont \numemph\vb\textbf{अन्यः}\lem \msCb\msNa\msNc, अन्य \msCa\msCc\msNb, चान्या \Ed\oo 
\textbf{॰त्तम}\lem \mssALL, ॰त्तमः \msNc}}% 
    \paral{{\devanagarifontsmall {\englishfont Witnesses used for this chapter:    \msCa\ ff.\thinspace 208v--210r,
                                                     \msCb\ ff.\thinspace 214r--215v,
                                                     \msCc\ ff.\thinspace 285v--287v,
                                                     \msNa\ ff.\thinspace 15v--17v,
                                                     \msNb\ ff.\thinspace 221v--223v 
                                                         (exp.\thinspace 56 lower -- 58 lower),
                                                     \msNc\ ff.\thinspace 223v--225v;
                                                     \Ed\ pp.\thinspace 613--617; 
                                                     \mssCaCbCc~= \msCa + \msCb + \msCc } }}

%Verse 11:1

{\devanagarifont अल्पक्लेशमनायास अर्थप्रायं विनेश्वर {॥ ११:१॥} \veg\dontdisplaylinenum }%
     \var{{\devanagarifont \numnoemph\vc\textbf{॰नायास}\lem \mssALL, 
॰नाया\uncl{सं} \msNa, ॰\uncl{नाया}सं \msNb}}% 
    \var{{\devanagarifont \numnoemph\vd\textbf{॰र्थप्रायं}\lem \msNapcorr\msNc, ॰र्थप्राय \mssCaCbCc, 
॰र्थप्रार्थप्रायं \msNaacorr, ॰\uncl{र्थप्राय} \msNb, 
॰थाम्नाय \Ed\oo 
\textbf{विनेश्वर}\lem \mssALL, \uncl{विनेश्वर} \msNb, सुरेश्वर \Ed}}% 

{\devanagarifont सर्वयज्ञफलावाप्ति दैवतैश्चापि पूजितम् \thinspace{\dandab} \dontdisplaylinenum }%
     \var{{\devanagarifont \numemph\va\textbf{दैवतै॰}\lem \msCa\msCb\msNa\Ed, देवतै॰ \msCc\msNc, \uncl{देवतै} \msNb}}% 

%Verse 11:2

{\devanagarifont कथयस्व सुरश्रेष्ठ मानुषाणां हिताय वै {॥ ११:२॥} \veg\dontdisplaylinenum }%
     \var{{\devanagarifont \numnoemph\vcd\textbf{॰श्रेष्ठ मानुषाणां हिताय वै}\lem \mssALL, 
॰श्रे\lac\  \msNb}}% 

{\devanagarifont महेश्वर उवाच {\dandab}\dontdisplaylinenum  }%
     \var{{\devanagarifont \numemph\vo\textbf{महे॰}\lem \mssALL, मेहे॰ \msNc}}% 

{\devanagarifont न तुल्यं तव पश्यामि दया भूतेषु भामिनि \thinspace{\danda} \dontdisplaylinenum }%
     \var{{\devanagarifont \numnoemph\va\textbf{तुल्यं तव}\lem \mssALL, \lac\  \msCa}}% 
    \var{{\devanagarifont \numnoemph\vb\textbf{भामिनि}\lem \mssALL, भामि \msCc}}% 

%Verse 11:3

{\devanagarifont किमन्यत्कथयिष्यामि दया यत्र न विद्यते {॥ ११:३॥} \veg\dontdisplaylinenum }%
     \var{{\devanagarifont \numnoemph\vc\textbf{किमन्य॰}\lem \mssALL, किम्यन्य॰ \msNb}}% 

{\devanagarifont सदाशिवमुखात्पूर्वं श्रुतं मे वरसुन्दरि \thinspace{\dandab} \dontdisplaylinenum }%
 
%Verse 11:4

{\devanagarifont शृणु देवि प्रवक्ष्यामि धर्मसारमनुत्तमम् {॥ ११:४॥} \veg\dontdisplaylinenum }%
     \var{{\devanagarifont \numemph\vc\textbf{देवि प्रवक्ष्यामि}\lem \msCb\msCc\msNa\msNb, ते देवि वक्ष्यामि \msCa\msNc\Ed}}% 
    \var{{\devanagarifont \numnoemph\vd\textbf{॰सारमनुत्तमम्}\lem \mssALL, ॰सारसमुच्चयम् \msCc}}% 


\alalfejezet{गृहस्थः(?)}
{\devanagarifont विनार्थेन तु यो यज्ञः स यज्ञः सार्वकामिकः \thinspace{\dandab} \dontdisplaylinenum }%
     \var{{\devanagarifont \numemph\vb\textbf{यज्ञः}\lem \mssALL, यज्ञ \Ed\oo 
\textbf{सार्वकामिकः}\lem \msCb\Ed, सर्वकालिकः \msCa\msNc, 
सर्वकामिक \msCc, सार्वकालिकः \msNa, सार्वकामिकाः \msNb}}% 
    \paral{{\devanagarifontsmall \vab {\englishfont See a sequence or list of the four āśramas in 4.75 above:}
                 गृहस्थो ब्रह्मचारी च वानप्रस्थो ऽथ भैक्षुकः;
                 {\englishfont see also 5.9:} 
                 एतच्छौचं गृहस्थानां द्विगुणं ब्रह्मचारिणाम्\thinspace{\devanagarifontsmall ।}
                 वानप्रस्थस्य त्रिगुणं यतीनां तु चतुर्गुणम्\thinspace{\devanagarifontsmall ॥} }}

%Verse 11:5

{\devanagarifont अक्षयश्चाव्ययश्चैव सर्वपातकनाशनः {॥ ११:५॥} \veg\dontdisplaylinenum }%
     \var{{\devanagarifont \numnoemph\vc\textbf{अक्षयश्चाव्ययश्}\lem \msCb\msNb\msNc\Ed, अक्षयं चाव्ययं \msCa\msCc\msNa}}% 
    \var{{\devanagarifont \numnoemph\vd\textbf{॰नाशनः}\lem \msCa\msNa\msNb\msNc, ॰नाशनम् \msCb\Ed, ॰नाशन \msCc}}% 

{\devanagarifont बहुविघ्नकरो ह्यर्थो बह्वायासकरस्तथा \thinspace{\dandab} \dontdisplaylinenum }%
     \var{{\devanagarifont \numemph\va\textbf{॰करो}\lem \mssALL, ॰करा \msCc\Ed\oo 
\textbf{ह्यर्थो}\lem \mssALL, ह्येर्थो \Ed}}% 
    \var{{\devanagarifont \numnoemph\vb\textbf{करस्तथा}\lem \mssALL, करतस्था \Ed}}% 

%Verse 11:6

{\devanagarifont ब्रह्महत्या इवेन्द्रस्य प्रविभागफला स्मृता {॥ ११:६॥} \veg\dontdisplaylinenum }%
     \var{{\devanagarifont \numnoemph\vd\textbf{प्रविभाग॰}\lem \msCb, प्रविभोग॰ \msCa\msCc(?)\msNa\msNc\Ed, प्रतिभोग॰ \msNb\oo 
\textbf{॰फला स्मृता}\lem \msCc, ॰फलः स्मृतः \msCapcorr\msCb\msNa\msNb\msNc, 
॰फल स्मृतः \msCaacorr, ॰प्रदः स्मृतः \Ed}}% 

{\devanagarifont पञ्चशोध्येन शोध्येत अर्थयज्ञो वरानने \thinspace{\dandab} \dontdisplaylinenum }%
     \var{{\devanagarifont \numemph\vb\textbf{॰यज्ञो}\lem \mssALL, ॰यज्ञ \msCc}}% 

%Verse 11:7

{\devanagarifont शोधिते तु फलं शुद्धमशुद्धे निष्फलं भवेत् {॥ ११:७॥} \veg\dontdisplaylinenum }%
     \var{{\devanagarifont \numnoemph\vcd\textbf{शुद्धमशुद्धे}\lem \mssALL, 
शुद्धंमशुद्धे \msNa, शुद्धमशुद्धं \Ed}}% 

{\devanagarifont देव्युवाच {\dandab}\dontdisplaylinenum  }%
     \var{{\devanagarifont \numemph\vo\textbf{देव्युवाच}\lem \mssALL, \om\ \msNbacorr}}% 

{\devanagarifont पञ्चशोध्ये सुरश्रेष्ठ संशयो ऽत्र भवेन्मम \thinspace{\danda} \dontdisplaylinenum }%
     \var{{\devanagarifont \numnoemph\va\textbf{॰शोध्ये}\lem \mssCaCbCc\msNa, ॰शोध्य \msNb\msNc, ॰शोध्यः \Ed\oo 
\textbf{॰श्रेष्ठ}\lem \mssALL, ॰स्रे\uncl{म्न} \msCc}}% 
    \var{{\devanagarifont \numnoemph\vb\textbf{ऽत्र भवे॰}\lem \mssALL, ऽत्रा भव॰ \Ed}}% 

%Verse 11:8

{\devanagarifont कथयस्व विभागेन श्रोतुमिच्छामि तत्त्वतः {॥ ११:८॥} \veg\dontdisplaylinenum }%
 
{\devanagarifont रुद्र उवाच {\dandab}\dontdisplaylinenum  }%
 
{\devanagarifont मनःशुद्धिस्तु प्रथमं द्रव्यशुद्धिरतः परम् \thinspace{\danda} \dontdisplaylinenum }%
     \var{{\devanagarifont \numemph\vb\textbf{॰शुद्धिरतः}\lem \mssALL, ॰शुद्धिगतः \msNb}}% 

{\devanagarifont मन्त्रशुद्धिस्तृतीया तु कर्मशुद्धिरतः परम्  \danda\dontdisplaylinenum }%
     \var{{\devanagarifont \numnoemph\vc\textbf{मन्त्रशुद्धिस्तृतीया}\lem \mssALL, मन्त्रद्धि तृतीया \msNc}}% 
    \var{{\devanagarifont \numnoemph\vd\textbf{कर्मशुद्धि॰}\lem \mssALL, कर्मसिद्धि \msNc}}% 

%Verse 11:9

{\devanagarifont पञ्चमी सत्त्वशुद्धिस्तु क्रतुशुद्धिश्च पञ्चधा {॥ ११:९॥} \veg\dontdisplaylinenum }%
     \var{{\devanagarifont \numnoemph\ve\textbf{पञ्चमी}\lem \mssALL, पञ्चमं \Ed\oo 
\textbf{॰शुद्धिस्तु}\lem \mssALL, ॰शुद्धिश्च \msNa\Ed}}% 
    \var{{\devanagarifont \numnoemph\vf\textbf{॰शुद्धिश्च पञ्चधा}\lem \mssALL, ॰शुद्धिस्तु पञ्चधा \msCc, 
॰शुद्धिरतः परम् \msNa}}% 

{\devanagarifont मनःशुद्धिर्नाम अविपरीतभावनया \thinspace{\dandab} \dontdisplaylinenum  }%
     \var{{\devanagarifont \numemph\vab\textbf{॰शुद्धिर्ना॰}\lem \mssALL, ॰शुद्धि ना॰ \msCc\oo 
\textbf{॰भावनया}\lem \mssALL, ॰भावनवा \msNa, ॰भावनतया \msNb}}% 

%Verse 11:10

{\devanagarifont द्रव्यशुद्धिर्नाम अनन्यायोपार्जितद्रव्येन {॥ ११:१०॥} \veg\dontdisplaylinenum  }%
     \var{{\devanagarifont \numnoemph\vcd\textbf{॰शुद्धिर्ना॰}\lem \mssALL, ॰शुद्धि ना॰ \msCc\msNc\oo 
\textbf{अनन्यायो॰}\lem \msCb\msNa\msNb\msNc, अन\lac  यो॰ \msCa, अन्यायो॰ \msCc, स्वल्पोन्यायो॰ \Ed\oo 
\textbf{॰द्रव्येन}\lem \mssALL, ॰व्येन \msNb}}% 

{\devanagarifont मन्त्रशुद्धिर्नाम स्वरव्यञ्जनयुक्ततया \thinspace{\dandab} \dontdisplaylinenum  }%
     \var{{\devanagarifont \numemph\vab\textbf{मन्त्रशुद्धिर्ना॰}\lem \msCa\msCb\msNb\Ed, मन्त्रशुद्धि ना॰ \msCc\msNc, 
मन्त्रस्तुद्दिना॰ \msNa\oo 
\textbf{॰युक्ततया}\lem \mssALL, ॰युक्तया \msCb}}% 

{\devanagarifont क्रियाशुद्धिर्नाम यथाक्रमाविपरीततया  \danda\dontdisplaylinenum  }%
     \var{{\devanagarifont \numnoemph\vcd\textbf{॰शुद्धिर्ना॰}\lem \mssALL, ॰शुद्धि ना॰ \msCc\msNb\oo 
\textbf{॰क्रमा॰}\lem \mssALL, ॰क्रम॰ \msCc\oo 
\textbf{॰रीततया}\lem \mssALL, ॰रीतया \msCb, \lac  तया \msNc}}% 

%Verse 11:11

{\devanagarifont सत्त्वशुद्धिर्नाम रजस्तम-अप्रधानतया {॥ ११:११॥} \veg\dontdisplaylinenum  }%
     \var{{\devanagarifont \numnoemph\vef\textbf{॰शुद्धिर्ना॰}\lem \mssALL, ॰शुद्धि ना॰ \msCa\msCc\oo 
\textbf{॰धानतया}\lem \mssALL, ॰धानत \msNc}}% 

\vers


{\devanagarifont विधिमेवं यदा शुध्येद्यदि यज्ञं करोति हि \thinspace{\dandab} \dontdisplaylinenum }%
     \var{{\devanagarifont \numemph\va\textbf{॰धिमेवं यदा}\lem \msCb\Ed, ॰धिमेव यदा \msCa\msCc\msNa, ॰धिमेव य \msNb, 
॰धिमेवं यथा \msNc}}% 
    \var{{\devanagarifont \numnoemph\vab\textbf{शुध्येद्यदि}\lem \conj, सूयेद्यदि \msCa\msNa, पूर्य यदि \msCb, 
सूर्येद्यदि \msCc, सूयेद्यति \msNb, पूयेद्यदि \msNc, शूद्ध्य यदि \Ed}}% 
    \var{{\devanagarifont \numnoemph\vb\textbf{यज्ञं}\lem \msCa\msCb\msNa\Ed, यज्ञ \msCc\msNc, संज्ञ \msNb\oo 
\textbf{हि}\lem \mssALL, \om\ \msNb}}% 

%Verse 11:12

{\devanagarifont तस्य यज्ञफलावाप्तिर्जन्ममृत्युश्च नो भवेत् {॥ ११:१२॥} \veg\dontdisplaylinenum }%
     \var{{\devanagarifont \numnoemph\vcd\textbf{॰वाप्तिर्ज॰}\lem \msCa\msCb\Ed, ॰वाप्ति ज \msCc\msNb\msNc, ॰वापि ज॰ \msNa}}% 

{\devanagarifont विनार्थेन तु यो यज्ञं करोति वरसुन्दरि \thinspace{\dandab} \dontdisplaylinenum }%
     \var{{\devanagarifont \numemph\vb\textbf{॰सुन्दरि}\lem \mssALL, ॰सुन्दरी \Ed}}% 

%Verse 11:13

{\devanagarifont न तस्य तत्फलावाप्तिः सर्वयज्ञेष्वशेषतः {॥ ११:१३॥} \veg\dontdisplaylinenum }%
     \var{{\devanagarifont \numnoemph\vd\textbf{॰यज्ञेष्वशेषतः}\lem \mssALL, ॰यज्ञेषु शेषतः \Ed}}% 

{\devanagarifont यज्ञवाट कुरुक्षेत्रं सत्त्वावासकृतालयः \thinspace{\dandab} \dontdisplaylinenum }%
     \var{{\devanagarifont \numemph\va\textbf{॰वाट कुरु॰}\lem \mssALL, ॰वाटङ्कुरु॰ \msCb, ॰वाटकृत॰ \Ed\oo 
\textbf{॰क्षेत्रं}\lem \mssALL, ॰क्षेत्र \msNc}}% 
    \var{{\devanagarifont \numnoemph\vb\textbf{सत्त्वा॰}\lem \mssALL, सत्वासत्वा॰ \msCbacorr\oo 
\textbf{॰लयः}\lem \mssALL, ॰लयम् \msCc}}% 

%Verse 11:14

{\devanagarifont प्रत्याहार महावेदि कुशप्रस्तर संयमः {॥ ११:१४॥} \veg\dontdisplaylinenum }%
     \var{{\devanagarifont \numnoemph\vc\textbf{॰वेदि}\lem \mssALL, ॰देवि \Ed}}% 

{\devanagarifont विधि नियमविस्तारो ध्यानवह्निः प्रदीपितः \thinspace{\dandab} \dontdisplaylinenum }%
     \var{{\devanagarifont \numemph\va\textbf{विधि नि॰}\lem \mssALL, विधिर्नि॰ \Ed\oo 
\textbf{॰विस्तारो}\lem \mssALL, ॰विस्तारौ \msCb}}% 
    \var{{\devanagarifont \numnoemph\vb \lem \msNc, ध्यानवह्निप्रदीपितः \msCa\msNa, 
ध्यानं वह्निप्रदीपितः \msCb, ध्यानमग्निप्रदीपितः \msCc, 
ध्यान अग्निप्रदीपनः \msNb, ध्यानवृद्धिर्प्रदीपिनः \Ed}}% 

%Verse 11:15

{\devanagarifont योगेन्धनसमिज्ज्वालतपोधूमसमाकुलः {॥ ११:१५॥} \veg\dontdisplaylinenum }%
     \var{{\devanagarifont \numnoemph\vcd\textbf{॰न्धनसमिज्ज्वालतपोधूम॰}\lem \msNb\msNc, ॰न्धनसमिज्ज्वालतपोधूप॰ \msCa, 
॰\uncl{न्ध}$\-$सत्वमिज्ज्वालतपोधूम॰ \msCb, ॰न्धनसमिज्वालतपोधूम॰ \msCc, 
॰न्धनशमि\uncl{त}ज्वाल$\-$तयोधूय॰ \msNa, ॰न्धनसमिज्ज्वाला तपोधूम॰ \Ed}}% 

{\devanagarifont पात्रन्यास शिवज्ञानं स्थालीपाक शिवात्मकः \thinspace{\dandab} \dontdisplaylinenum }%
     \var{{\devanagarifont \numemph\va\textbf{पात्र॰}\lem \mssALL, पात्रा॰ \msNc}}% 

%Verse 11:16

{\devanagarifont आज्याहुतिमविच्छिन्नं लम्बकस्रुवपातितः {॥ ११:१६॥} \veg\dontdisplaylinenum }%
     \var{{\devanagarifont \numnoemph\vc\textbf{॰च्छिन्नं}\lem \mssALL, ॰च्छिन्न \msNc}}% 
    \var{{\devanagarifont \numnoemph\vd\textbf{लम्बक॰}\lem \mssALL, \uncl{ल}म्बक॰ \msCc, त्र्यम्बक॰ \Ed\oo 
\textbf{॰पातितः}\lem \mssALL, ॰पातितम् \Ed}}% 

{\devanagarifont धारणाध्वर्युवत्कृत्वा प्राणायामश्च ऋत्विजः \thinspace{\dandab} \dontdisplaylinenum }%
     \var{{\devanagarifont \numemph\va\textbf{॰ध्वर्युव॰}\lem \msNb, ॰ध्वर्यव॰ \mssCaCbCc, ॰\uncl{ध्व}र्यव॰ \msNa, 
ध्व\lk\lk\ \msNc, धर्मव॰ \Ed}}% 

%Verse 11:17

{\devanagarifont तर्कयुक्तः सविस्तारः समाधिर्वयतापनः {॥ ११:१७॥} \veg\dontdisplaylinenum }%
     \var{{\devanagarifont \numnoemph\vc\textbf{॰युक्तः}\lem \mssALL, ॰युक्त \msCc, ॰युक्तिः \msNa\oo 
\textbf{॰विस्तारः}\lem \mssALL, ॰विस्तारो \msCc}}% 

{\devanagarifont ब्रह्मविद्यामयो यूपः पशुबन्धो मनोन्मनः \thinspace{\dandab} \dontdisplaylinenum }%
     \var{{\devanagarifont \numemph\vb\textbf{॰न्मनः}\lem \msCa\msNa\msNb\Ed, ॰त्मनः \msCb\msCc\msNc}}% 

%Verse 11:18

{\devanagarifont श्रद्धा पत्नी विशालाक्षि संकल्प पद शाश्वतम् {॥ ११:१८॥} \veg\dontdisplaylinenum }%
     \var{{\devanagarifont \numnoemph\vc\textbf{पत्नी}\lem \mssALL, \uncl{पत्नी} \msCa\oo 
\textbf{विशालाक्षि}\lem \mssALL, विशालाक्षी \msNc\Ed}}% 
    \var{{\devanagarifont \numnoemph\vd\textbf{पद शाश्वतम्}\lem \mssALL, प\uncl{द}\lac  श्वतम् \msCa}}% 

{\devanagarifont पञ्चेन्द्रियजयोत्पन्नः पुरोडाशो ऽमृताशनः \thinspace{\dandab} \dontdisplaylinenum }%
     \var{{\devanagarifont \numemph\vb\textbf{॰डाशो}\lem \mssCaCbCc\msNb\msNc, ॰भा \msNaacorr, ॰भासे \msNapcorr, ॰भागे \Ed\oo 
\textbf{मृता॰}\lem \mssALL, मृगा॰ \msCc}}% 

%Verse 11:19

{\devanagarifont ब्रह्मनादो महामन्त्रः प्रायश्चित्तानिलो जयः {॥ ११:१९॥} \veg\dontdisplaylinenum }%
     \var{{\devanagarifont \numnoemph\vd\textbf{॰त्तानिलो}\lem \mssALL, ॰त्तनिलो \msCc\msNb\oo 
\textbf{जयः}\lem \mssALL, जलाः \Ed}}% 

{\devanagarifont सोमपान परिज्ञानमुपाकर्म चतुर्यमः \thinspace{\dandab} \dontdisplaylinenum }%
     \var{{\devanagarifont \numemph\va\textbf{परि॰}\lem \mssALL, पर॰ \msCc}}% 

%Verse 11:20

{\devanagarifont इतिहास जलस्नानं पुराणकृतमम्बरः {॥ ११:२०॥} \veg\dontdisplaylinenum }%
     \var{{\devanagarifont \numnoemph\vc\textbf{॰स्नानं}\lem \mssALL, ॰स्नान \msCb}}% 
    \var{{\devanagarifont \numnoemph\vd\textbf{पुराण॰}\lem \mssALL, पुराणं \Ed\oo 
\textbf{॰कृतमम्बरः}\lem \mssALL, ॰कृतम्बरम् \msCb\ \unmetr}}% 

{\devanagarifont इडासुषुम्नासंवेद्ये स्नानमाचमनं सकृत् \thinspace{\dandab} \dontdisplaylinenum }%
     \var{{\devanagarifont \numemph\va\textbf{॰सुषुम्ना॰}\lem \mssALL, ॰सुषुम्न॰ \msCc\oo 
\textbf{॰वेद्ये}\lem  \msCa\Ed, ॰वेद्य \msCb\msNb, ॰वेद्येः \msCc, ॰वैद्य \msNa, ॰भेदो \msNc}}% 
    \var{{\devanagarifont \numnoemph\vb\textbf{सकृत्}\lem \mssALL, विदुः \msCc}}% 

%Verse 11:21

{\devanagarifont संतोषातिथिमादृत्य दयाभूतद्विजार्चितः {॥ ११:२१॥} \veg\dontdisplaylinenum }%
     \var{{\devanagarifont \numnoemph\vc\textbf{॰तोषातिथिमादृत्य}\lem \mssALL, ॰तोषतिथिमावृत्य \msNb}}% 
    \var{{\devanagarifont \numnoemph\vd\textbf{॰द्विजा॰}\lem \mssALL, ॰दया॰ \msCb}}% 

{\devanagarifont ब्रह्मकूर्च गुणातीत हविर्गन्ध निरञ्जनः \thinspace{\dandab} \dontdisplaylinenum }%
     \var{{\devanagarifont \numemph\vb\textbf{॰हविर्ग॰}\lem \mssALL, ॰हवि\uncl{र्ग}॰ \msCb, ॰हविग \msNa}}% 

%Verse 11:22

{\devanagarifont ब्रह्मसूत्रं त्रयस्तत्त्वं बोधना मुण्डितं शिरः {॥ ११:२२॥} \veg\dontdisplaylinenum }%
     \var{{\devanagarifont \numnoemph\vc\textbf{॰सूत्रं त्रयस्}\lem \msCb\msNb\msNc\Ed, ॰सूत्रन्त्रयस्तयस् \msCa, 
॰सूत्रं त्रय \msCc, ॰सूत्रत्रयं \msNa}}% 
    \var{{\devanagarifont \numnoemph\vd\textbf{मुण्डितं}\lem \mssALL, मुण्डित॰ \msCb\msNc\unmetr}}% 

{\devanagarifont निवृत्त्यादि चतुर्वेदश्चतुःप्रकरणासनः \thinspace{\dandab} \dontdisplaylinenum }%
     \var{{\devanagarifont \numemph\va\textbf{निवृत्त्या॰}\lem \eme, निवृत्या॰ \mssCaCbCc\msNa\msNb\msNc, निर्वृत्या॰ \Ed}}% 
    \var{{\devanagarifont \numnoemph\vb\textbf{॰प्रकरणासनः}\lem \mssALL, 
प्रकरनाशनः \msCc, प्रकरशासनः \Ed}}% 

%Verse 11:23

{\devanagarifont दक्षिणामभयं भूते दत्त्वा यज्ञं यजेत्सदा {॥ ११:२३॥} \veg\dontdisplaylinenum }%
     \var{{\devanagarifont \numnoemph\vc\textbf{॰भयं भूते}\lem \mssALL, ॰भक्षयम्भूतै \msCb}}% 
    \var{{\devanagarifont \numnoemph\vd\textbf{यज्ञं यजेत्}\lem \mssALL, यज्ञ ददत् \Ed}}% 
    \paral{{\devanagarifontsmall \vc {\englishfont \compare\ \VSS\ 22.14ab:} दक्षिणाभय भूतेभ्यः पशुबन्धः स्वयंकृतः }}

{\devanagarifont विनार्थं यज्ञसम्प्राप्तिः कथिता ते वरानने \thinspace{\dandab} \dontdisplaylinenum }%
     \var{{\devanagarifont \numemph\va\textbf{विनार्थं}\lem \mssALL, विनार्थ \msCc}}% 
    \var{{\devanagarifont \numnoemph\vb\textbf{कथिता ते}\lem \mssALL, 
कथि\uncl{तो} स्मि \msCc, कथितस्ते \Ed\oo 
\textbf{वरानने}\lem \mssALL, व\uncl{रा}नने \msCc}}% 

%Verse 11:24

{\devanagarifont आसहस्रस्य यज्ञानां फलं प्राप्नोति नित्यशः {॥ ११:२४॥} \veg\dontdisplaylinenum }%
     \var{{\devanagarifont \numnoemph\vd\textbf{प्राप्नोति}\lem \mssALL, प्रा\lac  ति \msCa\oo 
\textbf{नित्यशः}\lem \mssALL, मानवः \msNb}}% 

{\devanagarifont आश्रमः प्रथमस्तुभ्यं कथितो ऽस्ति वरानने \thinspace{\dandab} \dontdisplaylinenum }%
     \var{{\devanagarifont \numemph\va\textbf{आश्रमः}\lem \mssALL, आश्रम \msCb\msCc\oo 
\textbf{॰स्तुभ्यं}\lem \mssALL, ॰स्येष \msCc, ॰स्यैवं \Ed}}% 
    \var{{\devanagarifont \numnoemph\vb\textbf{ऽस्ति}\lem \msCa\msCb\msNa\msNc, स्मि \msCc\msNb\Ed}}% 

%Verse 11:25

{\devanagarifont सदाशिवेन सद्धर्मं दैवतैरपि पूजितम् {॥ ११:२५॥} \veg\dontdisplaylinenum }%
     \var{{\devanagarifont \numnoemph\vc\textbf{॰धर्मं}\lem \mssALL, ॰ध\uncl{र्मं} \msCb, ॰धर्मे \Ed}}% 
    \var{{\devanagarifont \numnoemph\vd\textbf{दैव॰}\lem \mssALL, देव॰ \msNb\Ed\oo 
\textbf{पूजितम्}\lem \mssALL, पूपूजितम् \msCb}}% 


\alalfejezet{ब्रह्मचारी}
{\devanagarifont ब्रह्मचर्यं निबोधेदं शृणुष्वावहिता शुभे \thinspace{\dandab} \dontdisplaylinenum }%
     \var{{\devanagarifont \numemph\va\textbf{॰चर्यं}\lem \mssALL, ॰चर्य \msNa}}% 
    \var{{\devanagarifont \numnoemph\vb\textbf{॰वहिता शुभे}\lem \mssALL, 
॰वहितो भव \msCc, ॰वहितो शुभे \msNb}}% 

%Verse 11:26

{\devanagarifont द्वितीयमाश्रमं देवि सर्वपापविनाशनम् {॥ ११:२६॥} \veg\dontdisplaylinenum }%
     \var{{\devanagarifont \numnoemph\vd\textbf{॰विनाशनम्}\lem \mssALL, ॰प्रनाशनम् \msNb}}% 
    \paral{{\devanagarifontsmall \vcd {\englishfont  \compare\ \MBH\ 12.184.10A:} गार्हस्थ्यं खलु द्वितीयमाश्रमं वदन्ति }}

{\devanagarifont व्रतं ब्रह्मपरं ध्यानं सावित्री प्रकृतिर्लयम् \thinspace{\dandab} \dontdisplaylinenum }%
     \var{{\devanagarifont \numemph\va\textbf{॰परं ध्यानं}\lem \mssALL, ॰परिज्ञानं \Ed}}% 
    \var{{\devanagarifont \numnoemph\vb\textbf{॰कृतिर्लयम्}\lem \msCa\msNa\msNc\Ed, ॰कृतालयम् \msCb, ॰कृतीलयम् \msCc, ॰कृतिलः \msNb}}% 
    \paral{{\devanagarifontsmall \vab {\englishfont cf.\ \VSS\ 16.8cd} }}

%Verse 11:27

{\devanagarifont ब्रह्मसूत्राक्षरं सूक्ष्मं त्रिगुणालय मेखलम् {॥ ११:२७॥} \veg\dontdisplaylinenum }%
     \var{{\devanagarifont \numnoemph\vd\textbf{॰लय}\lem \mssALL, ॰ल\lac\  \msCa\oo 
\textbf{मेखलम्}\lem \mssALL, यत्फलम् \Ed}}% 

{\devanagarifont दम दण्ड दया पात्रं भिक्षा संसारमोचनम् \thinspace{\dandab} \dontdisplaylinenum }%
     \var{{\devanagarifont \numemph\va\textbf{दण्ड दया}\lem \mssALL, दण्डादया \msNa, दण्डादयो \Ed\oo 
\textbf{पात्रं}\lem \mssALL, पात्र \msNb}}% 

%Verse 11:28

{\devanagarifont त्र्यायुषं द्व्यक्षरातीतं ज्ञानभस्म-अलङ्कृतम् {॥ ११:२८॥} \veg\dontdisplaylinenum }%
     \var{{\devanagarifont \numnoemph\vc\textbf{॰युषं}\lem \mssALL, ॰युष \msNa}}% 
    \var{{\devanagarifont \numnoemph\vd\textbf{भस्म}\lem \mssALL, भष्मम् \Ed}}% 

{\devanagarifont स्नानव्रतं सदासत्यं शीलशौचसमन्वितम् \thinspace{\dandab} \dontdisplaylinenum }%
     \var{{\devanagarifont \numemph\va\textbf{॰व्रतं}\lem \msCa\msCc\msNa\msNb, ॰व्रत \msCb\msNc\Ed}}% 

%Verse 11:29

{\devanagarifont अग्निहोत्र त्रयस्तत्त्वं जप ब्रह्मबिलस्वरः {॥ ११:२९॥} \veg\dontdisplaylinenum }%
     \var{{\devanagarifont \numnoemph\vc\textbf{॰होत्र त्रयस्तत्त्वं}\lem \msNa\msNc\Ed, ॰होत्रन्त्रयस्तत्वं \msCa, 
॰होत्र$\-$\uncl{त}यस्तत्वं \msCb, ॰होत्रत्रयं तत्वा \msCc, 
॰होत्रं त्रयंस्तत्वं \msNb}}% 
    \var{{\devanagarifont \numnoemph\vd\textbf{॰बिलस्वरः}\lem \corr, ॰बिलश्वरः \mssCaCbCc\msNa\msNb, ॰बिलेश्वर \msNc\Ed}}% 

{\devanagarifont द्वितीय आश्रमो देवि यथाह भगवान्शिवः \thinspace{\dandab} \dontdisplaylinenum }%
     \var{{\devanagarifont \numemph\va\textbf{द्वितीय आश्रमो}\lem \mssALL, द्वितीयमाश्रमो \msCc, 
द्वितीयमाश्रमं \Ed}}% 
    \var{{\devanagarifont \numnoemph\vb\textbf{यथाह}\lem \msCa\msCb\msNa\msNc, यथाहं \msCc\msNb, यदाह \Ed}}% 

%Verse 11:30

{\devanagarifont ममापि कथितं तुभ्यं जन्ममृत्युविनाशनम् {॥ ११:३०॥} \veg\dontdisplaylinenum }%
     \var{{\devanagarifont \numnoemph\vc\textbf{ममापि कथितं तु॰}\lem \mssALL, 
ममापि कथितस्तु॰ \msNc, मयापि कथितो तु॰ \Ed}}% 
    \var{{\devanagarifont \numnoemph\vd\textbf{॰मृत्यु॰}\lem \mssALL, ॰मृ\lac\  \msCa\oo 
\textbf{॰नाशनं}\lem \mssALL, ॰नाशनः \msNc}}% 


\alalfejezet{वानप्रस्थः}
{\devanagarifont वानप्रस्थविधिं वक्ष्ये शृणुष्वायतलोचने \thinspace{\dandab} \dontdisplaylinenum }%
     \var{{\devanagarifont \numemph\va\textbf{॰विधिं}\lem \mssALL, ॰विधि \msCb}}% 

%Verse 11:31

{\devanagarifont यथाश्रुतं यथातथ्यमृषिदैवतपूजितम् {॥ ११:३१॥} \veg\dontdisplaylinenum }%
     \var{{\devanagarifont \numnoemph\vd\textbf{॰दैवत॰}\lem \mssALL, ॰देवत॰ \msCc}}% 

{\devanagarifont वैराग्यवनमाश्रित्य नियमाश्रममाहरेत् \thinspace{\dandab} \dontdisplaylinenum }%
     \var{{\devanagarifont \numemph\va\textbf{वैराग्य॰}\lem \mssALL, वैराग्या \Ed}}% 
    \var{{\devanagarifont \numnoemph\vb\textbf{नियमा॰}\lem \mssALL, मा॰ \msNaacorr\oo 
\textbf{॰श्रममा॰}\lem \mssALL, ॰श्रमनो हरेत् \msCa}}% 

%Verse 11:32

{\devanagarifont शीलशैलदृढद्वारे प्राकारे विजितेन्द्रियः {॥ ११:३२॥} \veg\dontdisplaylinenum }%
     \var{{\devanagarifont \numnoemph\vc\textbf{॰दृढ॰}\lem \mssALL, ॰दृष॰ \Ed}}% 
    \var{{\devanagarifont \numnoemph\vd\textbf{॰कारे}\lem \mssALL, ॰कार॰ \msCc}}% 

{\devanagarifont अधिभूतः स्मृतो माता अध्यात्मश्च पिता तथा \thinspace{\dandab} \dontdisplaylinenum }%
     \var{{\devanagarifont \numemph\va\textbf{स्मृतो}\lem \mssALL, \lac\  \msCb, स्मृतौ \Ed}}% 
    \paral{{\devanagarifontsmall \vab {\englishfont cf.\ \VSS\ 22.10ab:} अध्यात्मनगरस्फीतः अधिभूतजनाकुलः }}

%Verse 11:33

{\devanagarifont अधिदैविकमाचार्यो व्यवसायाश्च भ्रातरः {॥ ११:३३॥} \veg\dontdisplaylinenum }%
     \var{{\devanagarifont \numnoemph\vc\textbf{अधिदैविक॰}\lem \emeGoodall, 
\uncl{अ}\lac \uncl{भौ}\lac  क॰ \msCa, 
अधिभौतिक॰ \msCb\msCc\msNa\msNc\Ed, 
अधिभौक्तिक॰ \msNb}}% 
    \var{{\devanagarifont \numnoemph\vd\textbf{व्यवसायाश्च}\lem \mssALL, व्यवसायश्च \Ed}}% 

{\devanagarifont श्रुतिः स्मृतिः स्मृता भार्या प्रज्ञा पुत्रः क्षमानुजः \thinspace{\dandab} \dontdisplaylinenum }%
     \var{{\devanagarifont \numemph\va\textbf{स्मृता}\lem \mssALL, स्मृतो \msCb}}% 

{\devanagarifont मैत्री बन्धुर्जटा चापं करुणा सुपवित्रकम्  \danda\dontdisplaylinenum }%
     \var{{\devanagarifont \numnoemph\vc\textbf{बन्धुर्ज॰}\lem \mssALL, बन्धु ज॰ \msCc\msNb}}% 

%Verse 11:34

{\devanagarifont मुदिता मौन चत्वारः सर्वकार्यमुपेक्षका {॥ ११:३४॥} \veg\dontdisplaylinenum }%
     \var{{\devanagarifont \numnoemph\ve\textbf{मौन चत्वारः}\lem \mssALL, 
मौनश्चत्वारः \msCb, मौन चत्वार \msCc}}% 
    \var{{\devanagarifont \numnoemph\vf\textbf{॰कार्यमु॰}\lem \mssALL, ॰कार्यामु॰ \msNa\oo 
\textbf{॰पेक्षका}\lem \mssALL, ॰पेक्षया \Ed}}% 

{\devanagarifont यमवल्कलसंवीतस्तपःकृष्णाजिनाधरः \thinspace{\dandab} \dontdisplaylinenum }%
     \var{{\devanagarifont \numemph\va\textbf{॰संवीत॰}\lem \mssALL, ॰सान्वीत॰ \Ed}}% 
    \var{{\devanagarifont \numnoemph\vb\textbf{॰कृष्णा॰}\lem \mssALL, ॰कृष्णां \msCc\oo 
\textbf{॰जिनाधरः}\lem \msNc, ॰जिनधरः \mssCaCbCc\msNa\msNb\ \unmetr, ॰जिनं पुरः \Ed}}% 

%Verse 11:35
 
{\devanagarifont उत्तरासङ्गमासीनो योगपट्टदृढव्रतः {॥ ११:३५॥} \veg\dontdisplaylinenum }%
     \var{{\devanagarifont \numnoemph\vd\textbf{॰दृढ॰}\lem \mssALL, ॰दृष्ट॰ \msNb\oo 
\textbf{॰व्रतः}\lem \mssALL, \lac\  \msCa}}% 

{\devanagarifont वेदघोषेण घोषेण प्राणायामो ऽग्निहावनम् \thinspace{\dandab} \dontdisplaylinenum }%
     \var{{\devanagarifont \numemph\va\textbf{वेद॰}\lem \mssALL, \lac  द॰ \msCa\oo 
\textbf{॰ण घोषेण}\lem \mssALL, ॰ण घोषीण \msCc}}% 
    \var{{\devanagarifont \numnoemph\vb\textbf{॰हावनम्}\lem \mssALL, ॰\uncl{हावनम्} \msCb, ॰हावन \msCc}}% 

%Verse 11:36

{\devanagarifont जितप्राण मृगाकूलो धृति यज्ञः क्रिया जपः {॥ ११:३६॥} \veg\dontdisplaylinenum }%
     \var{{\devanagarifont \numnoemph\vd\textbf{॰जपः}\lem \mssALL, ॰जिणः \msCc}}% 

{\devanagarifont अर्थसंग्रह शास्त्रेषु सखा दमदयादयः \thinspace{\dandab} \dontdisplaylinenum }%
     \var{{\devanagarifont \numemph\vb\textbf{सखा}\lem \mssALL, सखो \msNb\oo 
\textbf{दमद॰}\lem \mssALL, 
दम॰ \msCaacorr, दयद॰ \msCc}}% 

%Verse 11:37

{\devanagarifont शिवयज्ञं प्रयुञ्जीत साधनाष्टकपूजनम् {॥ ११:३७॥} \veg\dontdisplaylinenum }%
     \var{{\devanagarifont \numnoemph\vc\textbf{॰यज्ञं}\lem \mssALL, ॰यज्ञ \msCc\msNc}}% 
    \var{{\devanagarifont \numnoemph\vd\textbf{॰पूजनम्}\lem \mssALL, ॰पूजिकं \msCc}}% 
    \paral{{\devanagarifontsmall \vd {\englishfont \compare\ \DHARMP\ 2.1:} 
                 अष्टभिः साधनैरेभिश्चित्तं कायञ्च यत्नतः\thinspace{\devanagarifontsmall ।}
                 शोधयित्वा ततो योगी योगाभ्यासं समाचरेत्\thinspace{\devanagarifontsmall ॥} }}

{\devanagarifont पञ्चब्रह्मजलैः पूतः सत्यतीर्थशिवह्रदे \thinspace{\dandab} \dontdisplaylinenum }%
     \var{{\devanagarifont \numemph\va\textbf{॰ब्रह्मजलैः पूतः}\lem \mssALL, ब्र\lac\  \msNb}}% 
    \var{{\devanagarifont \numnoemph\vb\textbf{॰तीर्थ॰}\lem \mssALL, ॰तीर्थं \Ed}}% 

%Verse 11:38

{\devanagarifont स्नानमाचमनं कृत्वा संध्यात्रयमुपासयेत् {॥ ११:३८॥} \veg\dontdisplaylinenum }%
     \var{{\devanagarifont \numnoemph\vc\textbf{॰चमनं}\lem \mssALL, ॰चनं \msCb}}% 
    \var{{\devanagarifont \numnoemph\vd\textbf{॰सयेत्}\lem \eme, ॰श्रयेत् \mssCaCbCc\msNa\msNb\msNc\Ed}}% 
    \paral{{\devanagarifontsmall \vd {\englishfont \compare\ \VSS\ 11.59cd:} शिवस्य हृदयं संध्या तस्मात्संध्यामुपासयेत् }}

{\devanagarifont अक्षमाला पुराणार्थं जप शान्तं दिवानिशम् \thinspace{\dandab} \dontdisplaylinenum }%
     \var{{\devanagarifont \numemph\va\textbf{अक्षमाला}\lem \mssALL, \uncl{अक्ष}\lac  ला \msCa\oo 
\textbf{पुराणार्थं}\lem \mssALL, पुराणाञ्च \msNb, 
पुराणा\uncl{र्था} \msNc}}% 
    \var{{\devanagarifont \numnoemph\vb\textbf{॰शान्तं}\lem \mssALL, ॰शन्ति \msCaacorr\msNa}}% 

%Verse 11:39

{\devanagarifont ज्ञानसलिलसम्पूर्णमितिहासकमण्डलुः {॥ ११:३९॥} \veg\dontdisplaylinenum }%
     \var{{\devanagarifont \numnoemph\vc\textbf{॰सलिल॰}\lem \mssALL, ॰सलील॰ \Ed}}% 
    \var{{\devanagarifont \numnoemph\vd\textbf{॰कमण्डलुः}\lem \mssALL, ॰कमण्डलु \Ed}}% 

{\devanagarifont पञ्चकर्मक्रियोत्क्रान्ति जप पञ्चविधः सुखम् \thinspace{\dandab} \dontdisplaylinenum }%
     \var{{\devanagarifont \numemph\vab\textbf{॰त्क्रान्ति ज॰}\lem \msCa\msCb\msNb, ॰क्रान्तिज॰ \msCc, ॰त्क्रान्तिर्ज॰ \msNa, 
॰त्कान्तिज॰ \msNc, ऽक्रान्ति ज॰ \Ed}}% 

%Verse 11:40

{\devanagarifont साधनं शिवसंकल्पो योगसिद्धिफलप्रदः {॥ ११:४०॥} \veg\dontdisplaylinenum }%
     \var{{\devanagarifont \numnoemph\vd\textbf{॰दः}\lem \mssALL, ॰दम् \Ed}}% 

{\devanagarifont संतोषफलमाहारः कामक्रोधपराजितः \thinspace{\dandab} \dontdisplaylinenum }%
 
{\devanagarifont आशापाशजयाभ्यासो ध्यानयोगरतिप्रियः  \danda\dontdisplaylinenum }%
     \var{{\devanagarifont \numemph\vc\textbf{॰भ्यासो}\lem \mssALL, ॰भ्यास \Ed}}% 
    \var{{\devanagarifont \numnoemph\vd\textbf{॰रति॰}\lem \msCc\msNa\msNb\msNc, \lac\  \msCa, ॰रिति॰ \msCb, ॰रतिः \Ed}}% 

%Verse 11:41

{\devanagarifont अतिथिभ्यो ऽभयं दत्त्वा वानप्रस्थश्चरेद्व्रतम् {॥ ११:४१॥} \veg\dontdisplaylinenum }%
     \var{{\devanagarifont \numnoemph\ve\textbf{अतिथिभ्यो ऽभयं}\lem \mssALL, 
आर्तिभ्यश्चाभयं \Ed\oo 
\textbf{दत्त्वा}\lem \mssALL, दारा \msCc}}% 
    \var{{\devanagarifont \numnoemph\vf\textbf{॰प्रस्थश्च॰}\lem \mssALL, ॰प्रस्थ च॰ \msCc\msNb}}% 

\nemslokalong


\ujvers\nemsloka {
{\devanagarifont वानप्रस्थमयं धर्मं गदित यत्पूर्वमवधारितं }%
  \dontdisplaylinenum}    \var{{\devanagarifont \numemph\va\textbf{गदित यत्पूर्वमवधारितम्}\lem \conj, गदितं पूर्वधारितम् \msCa\msCb, 
यत्पूर्वमवधारितं \msCc\Ed, 
गदितं यत्पूर्वधारितं \msNaacorr, 
गदितं यत्पूर्व\uncl{मवधा}रितं \msNapcorr, 
गदित पूर्वधारितं \msNb, 
गदितं यत्पूर्वमेधारितं \msNc}}% 


\nemslokab

{\devanagarifont संसारोद्धरणमनित्यहरणमज्ञाननिर्मूलनम्  \danda\dontdisplaylinenum }%
     \var{{\devanagarifont \numnoemph\vb\textbf{॰हरणमनित्यहरणमज्ञा॰}\lem \msCa\msCb\msNaacorr\msNb\msNc, 
॰हरणंमनित्यहरणमज्ञा॰ \msCc\Ed, 
॰हरणंम् अनित्यहरणन्तज्ञा॰ \msNapcorr}}% 

\nemslokac

{\devanagarifont प्रज्ञावृद्धिकरममोघकरणं क्लेशार्णवोत्तारणं }%
  \dontdisplaylinenum    \var{{\devanagarifont \numnoemph\vc\textbf{(प्रज्ञा॰{\englishfont ...} ॰त्तारणम)}\lem \mssALL, \om\ \msNb\oo 
\textbf{॰करममोघ॰}\lem \mssCaCbCc\msNa\ \unmetr, \om\ \msNb, ॰कममोघ॰ \msNc, 
॰करं प्रबोध॰ \Ed\oo 
\textbf{क्लेशार्णवो॰}\lem \mssCaCbCc\msNc, क्लेशाण्णवो॰ \msNa, 
\om\ \msNb, शोकार्णवो॰ \Ed}}% 

%Verse 11:42


\nemslokad

{\devanagarifont जन्मव्याधिहरमकर्मदहनं सेवेत्स धर्मोत्तमम् {॥ ११:४२॥} \veg\dontdisplaylinenum }%
     \var{{\devanagarifont \numnoemph\vd\textbf{सेवेत्स}\lem \mssALL, 
सेवे स \msCc, सेवेत्त \msNb}}% 
    \lacuna{\devanagarifontsmall \vd {\englishfont \Ed\ (and \msPaperA) add here a Śārdūlavikrīḍita line:}
                 श्रद्धापूर्वकमेव यः सनियमं साक्षाच्च जीवन्शिवः
                 (शुद्धापूर्व्वकमेव यः सनियतं साक्षाच्च जीवने शिवः {\englishfont \msPaperA}) }%
  
\nemslokanormal



\alalfejezet{परिव्राजकः}
\vers


{\devanagarifont परिव्राजकधर्मो ऽयं कीर्तयिष्यामि तच्छृणु \thinspace{\dandab} \dontdisplaylinenum }%
     \var{{\devanagarifont \numemph\vb\textbf{कीर्तयिष्यामि}\lem \mssALL, 
कीर्तयि\lac  मि \msCa}}% 

%Verse 11:43

{\devanagarifont सुखदुःखं समं कृत्वा लोभमोहविवर्जितः {॥ ११:४३॥} \veg\dontdisplaylinenum }%
     \var{{\devanagarifont \numnoemph\vc\textbf{॰दुःखं}\lem \msCb, ॰दुःख \msCa\msCc\msNa\msNb\msNc\Ed}}% 
    \var{{\devanagarifont \numnoemph\vd\textbf{लोभमोह॰}\lem \msCb, लाभालोभ॰ \msCa\msNa\msNb\msNc, 
लाभलोभ॰ \msCc, लाभालाभ॰ \Ed\oo 
\textbf{॰वर्जितः}\lem \mssALL, ॰वर्जिताः \msNb}}% 
    \paral{{\devanagarifontsmall \vd {\englishfont \compare\ \VSS\ 4.71:}  
                      कामः क्रोधश्च लोभश्च मोहश्चैव चतुर्विधः\thinspace{\devanagarifontsmall ।}
                      चतुःशत्रुर्निहन्तव्यः सर्वथा वीतकल्मषः\thinspace{\devanagarifontsmall ॥} }}

{\devanagarifont वर्जयेन्मधु मांसानि परदारांश्च वर्जयेत् \thinspace{\dandab} \dontdisplaylinenum }%
     \var{{\devanagarifont \numemph\va\textbf{वर्जयेन्}\lem \msCa\msNb, वर्जयेत् \msCb\msCc\msNa\msNc\Ed}}% 
    \paral{{\devanagarifontsmall \vab {\englishfont \compare\ \MANU\ 2.177:}
                 वर्जयेन्मधु मांसं च गन्धं माल्यं रसान्स्त्रियः\thinspace{\devanagarifontsmall ।}
                 शुक्तानि यानि सर्वाणि प्राणिनां चैव हिंसनम्\thinspace{\devanagarifontsmall ॥} }}

%Verse 11:44

{\devanagarifont वर्जयेच्चिरवासं च परवासं च वर्जयेत् {॥ ११:४४॥} \veg\dontdisplaylinenum }%
     \var{{\devanagarifont \numnoemph\vc\textbf{॰वासं}\lem \mssALL, ॰वासश् \Ed}}% 
    \var{{\devanagarifont \numnoemph\vd\textbf{॰वासं}\lem \mssALL, ॰वासश् \Ed}}% 

{\devanagarifont वर्जयेत्सृष्टभोज्यानि भिक्षामेकां च वर्जयेत् \thinspace{\dandab} \dontdisplaylinenum }%
     \var{{\devanagarifont \numemph\va\textbf{वर्जयेत्सृष्ट॰}\lem \msCc(?)\msNa\msNc, वर्जयेत्मृष्ट॰ \msCa, 
वर्ज्जन्मृष्ट॰ \msNb, वर्जयेन्मृष्ट॰ \Ed\oo 
\textbf{॰भोज्यानि}\lem \mssALL, ॰भोजालि(?) \msNc}}% 
    \var{{\devanagarifont \numnoemph\vb\textbf{॰क्षामेकां}\lem \msCa\msNb, ॰क्षामेकं \msCc\msNa, 
॰क्षमेकञ् \msNc, ॰क्षामेकश् \Ed}}% 
    \lacuna{\devanagarifontsmall \vab {\englishfont Omitted in \msCb} }%
      \paral{{\devanagarifontsmall \vb {\englishfont \compare\ \MANU\ 2.188ab:}
                          भैक्षेण वर्तयेन्नित्यं नैकान्नादी भवेद्व्रती }}

%Verse 11:45

{\devanagarifont वर्जयेत्संग्रहं नित्यमभिमानं च वर्जयेत् {॥ ११:४५॥} \veg\dontdisplaylinenum }%
 
{\devanagarifont सुसूक्ष्मं मनसा ध्यात्वा दृशौ पादं विनिक्षिपेत् \thinspace{\dandab} \dontdisplaylinenum }%
     \var{{\devanagarifont \numemph\vb\textbf{दृशौ}\lem \conj, शुचौ \mssCaCbCc\msNa\msNb\msNc\Ed\oo 
\textbf{पादं}\lem \msCb\msCc\msNa\msNc, पा\uncl{दं} \msCa, पाद \msNb\Ed\oo 
\textbf{विनिक्षि॰}\lem \mssALL, \lac  निक्षि॰ \msCa, 
विनिक्ष॰ \msNc}}% 

%Verse 11:46

{\devanagarifont न कुप्येत अनालाभे लाभे वापि न हर्षयेत् {॥ ११:४६॥} \veg\dontdisplaylinenum }%
     \var{{\devanagarifont \numnoemph\vc\textbf{कुप्येत}\lem \mssALL, कुपेत \msCc\oo 
\textbf{अनालाभे}\lem \msNa, मनोलाभे \msCa\msCb\msNb\msNc, 
मनोलाभो \msCc, मनालाभे \Ed}}% 
    \paral{{\devanagarifontsmall \vcd {\englishfont \similar\ \MANU\ 6.57:}
                         अलाभे न विषदी स्याल्लाभे चैव न हर्षयेत् = 
                     {\englishfont \VASISTHADHS\ 10.22} }}

{\devanagarifont अर्थतृष्णास्वनुद्विग्नो रोषे वापि सुदारुणे \thinspace{\dandab} \dontdisplaylinenum }%
     \var{{\devanagarifont \numemph\va\textbf{अर्थ॰}\lem \msCb\msCc\msNc, अर्था॰ \msCa\msNa\msNb, अथ \Ed\oo 
\textbf{॰नुद्विग्नो}\lem \mssALL, ॰नुदिग्नो \msCc}}% 

%Verse 11:47

{\devanagarifont स्तुतिनिन्दा समं कृत्वा प्रियं वाप्रियमेव वा {॥ ११:४७॥} \veg\dontdisplaylinenum }%
 
{\devanagarifont नियमास्तु परीधानं संयमावृतमेखलः \thinspace{\dandab} \dontdisplaylinenum }%
     \var{{\devanagarifont \numemph\va\textbf{॰धानं}\lem \mssALL, 
॰धाना \msCc, ॰\uncl{धानं} \msNc}}% 
    \var{{\devanagarifont \numnoemph\vb\textbf{॰वृत॰}\lem \mssALL, ॰मृत॰ \msNb, ॰नृत॰ \Ed\oo 
\textbf{॰मेखलः}\lem \mssALL, 
॰मेखलाः \msCc, ॰मेखला \msNb}}% 

%Verse 11:48

{\devanagarifont निरालम्बं मनः कृत्वा बुद्धिं कृत्वा निरञ्जनाम् {॥ ११:४८॥} \veg\dontdisplaylinenum }%
     \var{{\devanagarifont \numnoemph\vc\textbf{॰बं मनः कृत्वा}\lem \msNc, ॰बमसत्कृत्वा \msCa\msNa, 
॰बमसंकृत्वा \msCb, ॰बमनंकृत्वा \msCc, 
॰ब मनस्कृत्वा \msNb, ॰बमनङ्कृत्वा \Ed}}% 
    \var{{\devanagarifont \numnoemph\vd\textbf{बुद्धिं}\lem \mssALL, बुद्धि \msCb\Ed\oo 
\textbf{निरञ्जनाम्}\lem \eme, निरञ्जनम् \mssCaCbCc\msNb\msNc\Ed, निरञ्जनः \msNa}}% 

{\devanagarifont आत्मानं पृथिवीं कृत्वा खं च कृत्वा मनोन्मनम् \thinspace{\dandab} \dontdisplaylinenum }%
     \var{{\devanagarifont \numemph\vab\textbf{कृत्वा खं च}\lem \mssALL, 
कृ\uncl{त्वा}\lac  ञ्च \msCa}}% 
    \var{{\devanagarifont \numnoemph\vb\textbf{मनोन्मनम्}\lem \mssALL, मनोन्मनः \msNc, मनोन्मनैः \Ed}}% 

%Verse 11:49

{\devanagarifont त्रिदण्डं त्रिगुणं कृत्वा पात्रं कृत्वाक्षरो ऽव्ययः {॥ ११:४९॥} \veg\dontdisplaylinenum }%
     \var{{\devanagarifont \numnoemph\vd\textbf{॰क्षरो}\lem \mssALL, ॰करो \msNb\oo 
\textbf{व्ययः}\lem \msCa\msCb\msNa\msNb, व्ययं \msCc, व्यय \msNc, द्वयम् \Ed}}% 

{\devanagarifont न्यसेद्धर्ममधर्मं च ईर्ष्याद्वेषं परित्यजेत् \thinspace{\dandab} \dontdisplaylinenum }%
     \var{{\devanagarifont \numemph\va\textbf{॰धर्मं च}\lem \mssALL, ॰धर्मं वा \msNa}}% 
    \var{{\devanagarifont \numnoemph\vb\textbf{ईर्ष्या॰}\lem \msNa\msNc\Ed, ईर्षा॰ \mssCaCbCc\msNb\oo 
\textbf{॰द्वेषं}\lem \mssALL, ॰द्वेष \msCc}}% 

%Verse 11:50

{\devanagarifont निर्द्वन्द्वो नित्यसत्यस्थो निर्ममो निरहंकृतः {॥ ११:५०॥} \veg\dontdisplaylinenum }%
     \var{{\devanagarifont \numnoemph\vc\textbf{निर्द्वन्द्वो}\lem \mssALL, निवंद्वो \msCc\oo 
\textbf{॰सत्य॰}\lem \mssALL, ॰संत्य॰ \msCc}}% 
    \var{{\devanagarifont \numnoemph\vd\textbf{निर्ममो}\lem \msNc\Ed, निर्मांसो \mssCaCbCc\msNa, निर्मंसो \msNb\oo 
\textbf{॰कृतः}\lem \mssALL, ॰कृतं \msNa, ॰कृतिः \Ed}}% 
    \paral{{\devanagarifontsmall \vcd {\englishfont \compare\ \BHG\ 2.45cd: 
                         }निर्द्वन्द्वो नित्यसत्त्वस्थो निर्योगक्षेम आत्मवान् }}

{\devanagarifont दिवसस्याष्टमे भागे भिक्षां सप्तगृहं चरेत् \thinspace{\dandab} \dontdisplaylinenum }%
     \var{{\devanagarifont \numemph\va\textbf{दिवसस्या॰}\lem \mssALL, दिवसत्या॰ \msCb}}% 
    \var{{\devanagarifont \numnoemph\vb\textbf{भिक्षां}\lem \mssALL, भिक्षा \msNb}}% 
    \paral{{\devanagarifontsmall \vb {\englishfont \compare\ \GAUTDHS\ 23.18:}
                 तस्याजिनमूर्ध्वबालं परिधाय लोहितपत्रः सप्त गृहान्भक्षं चरेत् }}

%Verse 11:51

{\devanagarifont न चासीत न तिष्ठेत न च देहीति वा वदेत् {॥ ११:५१॥} \veg\dontdisplaylinenum }%
 
{\devanagarifont यथालाभेन वर्तेत अष्टौ पिण्डान्दिने दिने \thinspace{\dandab} \dontdisplaylinenum }%
     \var{{\devanagarifont \numemph\va\textbf{यथालाभेन}\lem \mssALL, यथाला\lac\  \msCa}}% 
    \var{{\devanagarifont \numnoemph\vb\textbf{अष्टौ}\lem \mssALL, अष्ट \Ed}}% 

%Verse 11:52

{\devanagarifont वस्त्रभोजनशय्यासु न प्रसज्येत विस्तरम् {॥ ११:५२॥} \veg\dontdisplaylinenum }%
     \var{{\devanagarifont \numnoemph\vc\textbf{॰शय्यासु}\lem \mssALL, ॰शय्याञ्च \msNb, ॰शैय्यासु \Ed}}% 
    \var{{\devanagarifont \numnoemph\vd\textbf{॰सज्येत}\lem \msCa\msCc\msNa\msNb, ॰युज्ये \msCb, ॰सहेत \msNc, ॰सह्येत \Ed\oo 
\textbf{विस्तरम्}\lem \mssALL, विस्तरः \Ed}}% 

{\devanagarifont नाभिनन्देत मरणं नाभिनन्देत जीवितम् \thinspace{\dandab} \dontdisplaylinenum }%
     \paral{{\devanagarifontsmall \vab {\englishfont = \MBH\ 12.237.15ab = \MANU\ 6.45ab = \NARADAPARIVR\ 3.61cd} }}

%Verse 11:53

{\devanagarifont इन्द्रियाणि वशंकृत्वा कामं हत्वा यतव्रतः {॥ ११:५३॥} \veg\dontdisplaylinenum }%
     \var{{\devanagarifont \numemph\vc\textbf{वशंकृ॰}\lem \mssALL, वसंत्कृ॰ \msCc}}% 
    \var{{\devanagarifont \numnoemph\vd\textbf{हत्वा यतव्रतः}\lem \mssALL, 
कृत्वा यतः व्रतः \msNb}}% 

{\devanagarifont अतीतं च भविष्यं च न भिक्षुश्चिन्तयेत्सदा \thinspace{\dandab} \dontdisplaylinenum }%
     \var{{\devanagarifont \numemph\vb\textbf{भिक्षुश्चि॰}\lem \mssALL, 
भिक्षुंश्चि॰ \msNa, भिक्षु चि॰ \Ed\oo 
\textbf{सदा}\lem \mssALL, \om\ \msCb}}% 

%Verse 11:54

{\devanagarifont क्रोधमानमददर्पान्परिव्राड्वर्जयेत्सदा {॥ ११:५४॥} \veg\dontdisplaylinenum }%
     \var{{\devanagarifont \numnoemph\vcd\textbf{॰दर्पान्प॰}\lem \mssALL, ॰दर्पात्प॰ \msCb}}% 

{\devanagarifont विरागं तु धनुः कृत्वा प्राणायामगुणैर्युतम् \thinspace{\dandab} \dontdisplaylinenum }%
     \var{{\devanagarifont \numemph\va\textbf{धनुः}\lem \mssALL, धनुष् \Ed}}% 
    \var{{\devanagarifont \numnoemph\vb\textbf{प्राणायामगु॰}\lem \mssALL, प्राणायामङ्गु॰ \msCa\oo 
\textbf{युतम्}\lem \mssALL, युतः \msNa, वृतं \Ed}}% 

%Verse 11:55

{\devanagarifont धारणाशरतीक्ष्णेन मृगं हत्वा मनेन्द्रियम् {॥ ११:५५॥} \veg\dontdisplaylinenum }%
     \var{{\devanagarifont \numnoemph\vc\textbf{॰तीक्ष्णेन}\lem \msNb\Ed, ॰तीक्ष्णेण \mssCaCbCc\msNc, ॰तीक्षेण \msNa}}% 

{\devanagarifont मैत्रीखड्गसुतीक्ष्णेन संसारारिं निकृन्तयेत् \thinspace{\dandab} \dontdisplaylinenum }%
     \var{{\devanagarifont \numemph\va\textbf{सुतीक्ष्णेन}\lem \msCa\msNb\msNc\Ed, सुतीक्ष्णेण \msCb\msCc\msNapcorr, ण \msNaacorr}}% 
    \var{{\devanagarifont \numnoemph\vb\textbf{॰सारारिं}\lem \mssALL, ॰सारारि \msCc\msNc}}% 

{\devanagarifont करुणावर्तचक्रेण क्रोधमत्तगजं जयेत्  \danda\dontdisplaylinenum }%
 
%Verse 11:56

{\devanagarifont मुदितावर्मबद्धाङ्गस्तूणं पूर्णमुपेक्षया {॥ ११:५६॥} \veg\dontdisplaylinenum }%
     \var{{\devanagarifont \numnoemph\vf\textbf{तूणं पूर्णमु॰}\lem \emeGoodall, तूण्णापूर्ण्णमु॰ \msCa, 
तूणापूर्ण्णमु॰ \msCb, तू$\-$\uncl{न}पूर्ण्णमु॰ \msCc, 
तूण्णापूण्णामु॰ \msNa, तूर्ण्णापूर्ण्णमु॰ \msNb\msNc, तूणीपूर्णमु॰ \Ed}}% 

{\devanagarifont अनक्षरं परं ब्रह्म चिन्तयेत्सततं द्विज \thinspace{\dandab} \dontdisplaylinenum }%
     \var{{\devanagarifont \numemph\va\textbf{अनक्षरं}\lem \msCb, अनाक्षरं \msCa\msNa, 
अनाक्षर॰ \msCc\msNc\Ed, अनक्षर॰ \msNb\oo 
\textbf{परं}\lem \mssALL, पर \msCb\msNc}}% 

{\devanagarifont ब्रह्मणो हृदयं विष्णुर्विष्णोश्च हृदयं शिवः  \danda\dontdisplaylinenum }%
     \var{{\devanagarifont \numnoemph\vc\textbf{हृदयं}\lem \mssALL, 
\lac  दयं \msCa, हृदये \msNc}}% 
    \var{{\devanagarifont \numnoemph\vcd\textbf{विष्णुर्वि॰}\lem \msCa\msNa\Ed, विष्णुम्वि॰ \msCb, 
विष्णु वि॰ \msCc\msNb\msNc}}% 
    \var{{\devanagarifont \numnoemph\vd\textbf{शिवः}\lem \Ed, शिवं \mssCaCbCc\msNa\msNb\msNc}}% 

%Verse 11:57

{\devanagarifont शिवस्य हृदयं संध्या तस्मात्संध्यामुपासयेत् {॥ ११:५७॥} \veg\dontdisplaylinenum }%
     \var{{\devanagarifont \numnoemph\vf\textbf{॰सयेत्}\lem \msCa\msCc\msNb, ॰शयेत् \msCb\msNa, ॰श्रयेत् \msNc\Ed}}% 
    \paral{{\devanagarifontsmall \vo {\englishfont \similar\ Saubhāgyabhāskara of Bhāskararāya ad Lalitāsahasranāmastotra 302:}
                 ब्रह्मणो हृदयं विष्णुर्विष्णोरपि शिवः स्मृतः\thinspace{\devanagarifontsmall ।}
                 शिवस्य हृदयं सन्ध्या तेनोपास्या द्विजातिभिः\thinspace{\devanagarifontsmall ॥}
                 इति कश्यपादिवचनैः कौर्मपाद्मस्कान्दादिनिखिलपुराणेषु च तत्र 
                 तत्र देवीकालिकाब्रह्माण्डमार्कण्डेयादिपुराणेषु बहुशः 
                 शक्तिरहस्य-देवीभागवत-तृतीयस्कन्धादिषु
                 च इदंपर्येण सर्वत्र ज्ञानार्णवकुलार्णवादितन्त्रेषु त्वपरिमितत्या वर्णितम् }}

\nemslokalong


\ujvers\nemsloka {
{\devanagarifont संसारार्णवतारणं शुभगतिः स ब्रह्म संध्याक्षरं }%
  \dontdisplaylinenum}    \var{{\devanagarifont \numemph\va\textbf{॰गतिः}\lem \msCc\Ed, ॰गति \msCa\msCb\msNa\msNb\ \unmetr, ॰गतिं \msNc\oo 
\textbf{॰क्षरं}\lem \mssALL, ॰क्षर \msCb}}% 


\nemslokab

{\devanagarifont ध्यायेन्नित्यमतन्द्रितो ह्यनुपमं व्यक्तात्मवेद्यं शिवम्  \danda\dontdisplaylinenum }%
     \var{{\devanagarifont \numnoemph\vb\textbf{॰तन्द्रितो}\lem \msCa\msNa\msNc\Ed, ॰नन्द्रितो \msCb, ॰तन्द्रिय \msCc, ॰तन्द्रियं \msNb\oo 
\textbf{॰वेद्यं}\lem \mssALL, ॰वेद्य \msNb\ \unmetr}}% 

\nemslokac

{\devanagarifont रूपैर्वर्णगुणादिभिश्च विहितं दुर्लक्ष्यलक्ष्योत्तमं }%
  \dontdisplaylinenum    \var{{\devanagarifont \numnoemph\vc\textbf{रूपैर्व॰}\lem \msCa\msNa\msNc\Ed, रूपै व॰ \msCb\msCc\msNb\oo 
\textbf{विहितं}\lem \mssALL, रहितं \msNapcorr(?)\Ed\oo 
\textbf{दुर्लक्ष्यलक्ष्योत्तमम्}\lem \msCa\msNb, 
दुर्लक्ष्यलक्षोत्तमम् \msCb\msCc\msNc\Ed, 
दुलक्ष्यलक्ष्योत्तमम् \msNa}}% 

%Verse 11:58


\nemslokad

{\devanagarifont यत्नोद्धृत्य समाश्रयेत्सुरगुरुं सर्वार्तिहर्ता हरम् {॥ ११:५८॥} \veg\dontdisplaylinenum }%
     \var{{\devanagarifont \numnoemph\vd\textbf{यत्नोद्धृत्य}\lem \mssALL, यत्नाद्धृत्य \Ed\oo 
\textbf{समाश्रये॰}\lem \mssALL, मणाश्रये॰ \msNb\oo 
\textbf{सर्वार्तिहर्ता हरम्}\lem \mssCaCbCc\msNb, सर्वार्त्तिह\uncl{र्त्ता} हरं \msNa, 
सर्वात्तिहर्त्ता हरं \msNc, 
सर्वार्तिहन् शङ्करम् \Ed}}% 

\nemslokanormal


\vers


{\devanagarifont 
\jump
\begin{center}
\ketdanda~इति वृषसारसंग्रहे चतुराश्रमधर्मविधानो नामाध्याय एकादशमः~\ketdanda
\end{center}
\dontdisplaylinenum\vers  }%
     \var{{\devanagarifont \numnoemph{\englishfont \Colo:}\textbf{नामाध्याय एकादशमः}\lem \mssALL, नामाध्याय एकादश \msNc, 
नाम एकादशो ऽध्यायः \Ed}}% 
