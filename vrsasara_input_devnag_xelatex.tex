\fejno=0\versno=0
\centerline{\Huge\devanagarifontbold वृषसारसंग्रहः  }

 
{\vrule depth10pt width0pt}
\versno=0\fejno=3
\thispagestyle{empty}

\centerline{\Large\devanagarifontbold [   तृतीयो ऽध्यायः  ]}{\vrule depth10pt width0pt} \fancyhead[CO]{{\footnotesize\devanagarifont वृषसारसंग्रहे  }}
\fancyhead[CE]{{\footnotesize\devanagarifont तृतीयो ऽध्यायः  }}
\fancyhead[LE]{}
\fancyhead[RE]{}
\fancyhead[LO]{}
\fancyhead[RO]{}
\szam\bek



\alalfejezet{धर्मप्रवचनम्}
\vers


{\devanagarifont विगतराग उवाच {\dandab}\dontdisplaylinenum  }%
 
{\devanagarifont किमर्थं धर्ममित्याहुः कतिमूर्तिश्च कीर्त्यते \thinspace{\danda} \dontdisplaylinenum }%
     \var{{\devanagarifont \numemph\va\textbf{आहुः}\lem \msParis\msCa\msCb\msNa\msNb\msNc, आहु \Ed}}% 
    \lacuna{\devanagarifontsmall {\englishfont Witnesses used for this chapter: \msParis\ exp.\thinspace 215r--215v (breaks off after 3.14d and resumes at 4.8a),
                                             \msCa\ ff.\thinspace 197r--198v, 
                                             \msCb\ ff.\thinspace 204v--206r, 
                                             \msCc\ ff.\thinspace 273r--273v (broke off at 2.21 and resumes at 3.30b),
                                             \msNa\ ff.\thinspace 4v--6r, 
                                             \msNb\ exp.\thinspace 42, 47 (upper), 48 (lower),
                                             \msNc\ ff.\thinspace 213r--214v,
                                             \Ed\ pp.\thinspace 588--591;
                                        \mssCaCbCc\ = \msCa + \msCb + \msCc } }%
  
%Verse 3:1

{\devanagarifont कतिपादवृषो ज्ञेयो गतिस्तस्य कति स्मृताः {॥३:१॥} \veg\dontdisplaylinenum }%
     \var{{\devanagarifont \numnoemph\vd\textbf{स्मृताः}\lem \msParis\msCa\msNa\msNb\msNc, स्मृता \msCb, स्मृतः \Ed}}% 

{\devanagarifont कौतूहलं ममोत्पन्नं संशयं छिन्धि तत्त्वतः \thinspace{\dandab} \dontdisplaylinenum }%
     \var{{\devanagarifont \numemph\va\textbf{कौतूहलं}\lem \msParis\msCa\msCb\msNa\msNb\msNc, कौतुहल \Ed\oo 
\textbf{ममोत्पन्नं}\lem \msParis\msCa\msCb\msNa\msNb\Ed, समोत्पन्नं \msNc}}% 
    \var{{\devanagarifont \numnoemph\vb\textbf{संशयं}\lem \msParis\msCb\msNa\msNb\msNc\Ed, सशयं \msCa}}% 

%Verse 3:2

{\devanagarifont कस्य पुत्रो मुनिश्रेष्ठ प्रजास्तस्य कति स्मृताः {॥३:२॥} \veg\dontdisplaylinenum }%
 
{\devanagarifont अनर्थयज्ञ उवाच {\dandab}\dontdisplaylinenum  }%
 
{\devanagarifont धृतिरित्येष धातुर्वै पर्यायः परिकीर्तितः \thinspace{\danda} \dontdisplaylinenum }%
 
%Verse 3:3

{\devanagarifont आधारणान्महत्त्वाच्च धर्म इत्यभिधीयते {॥३:३॥} \veg\dontdisplaylinenum  }%
     \var{{\devanagarifont \numemph\vc\textbf{आधारणान्म॰}\lem \msParis\msCa\msNb, आधारणात्प॰ \msCb, आधारणात्म॰ \msNa\msNc, आधारेण म॰ \Ed}}% 
    \var{{\devanagarifont \numnoemph\vd\textbf{इत्यभिधीयते}\lem \msCa\msNa\msNc\Ed, इ\uncl{त्यभिधीयते} \msParis, 
इत्यविधीयते \msCb\msNb}}% 
    \paral{{\devanagarifontsmall \vcd {\englishfont \compare\ \LINPU\ 1.10.12cd--13ab:}
                         धारणार्थे महान्ह्येष धर्मशब्दः प्रकीर्तितः\thinspace{\devanagarifontsmall ॥}
                         अधारणे ऽमहत्त्वे च अधर्म इति चोच्यते\thinspace{\devanagarifontsmall ।}
                \vo\ {\englishfont \compare\ \BRAHMANDAPUR\ 1.32.29:}
                         धारणार्थो धृतिश्चैव धातुः शब्दे प्रकीर्तितः\thinspace{\devanagarifontsmall ।}
                         अधारणामहत्त्वे च अधर्म इति चोच्यते\thinspace{\devanagarifontsmall ॥};
                     {\englishfont \compare\ \VAYUP\ 1.59.28:}
                         धारणा धृतिरित्यर्थाद्धातोर्धर्मः प्रकीर्तितः\thinspace{\devanagarifontsmall ।}
                         अधारणे ऽमहत्त्वे च अधर्म इति चोच्यते\thinspace{\devanagarifontsmall ॥};
                     {\englishfont \similar\ \MATSP\ 145.27:}  धर्मेति धारणे धातुर्महत्वे चैव उच्यते\thinspace{\devanagarifontsmall ।}
                                                   आधारणे महत्त्वे वा धर्मः स तु निरुच्यते\thinspace{\devanagarifontsmall ।} }}

{\devanagarifont श्रुतिस्मृतिद्वयोर्मूर्तिश्चतुष्पादवृषः स्थितः \thinspace{\dandab} \dontdisplaylinenum }%
     \var{{\devanagarifont \numemph\vab\textbf{॰स्मृतिद्वयोर्मूर्तिश्च॰}\lem \msCa, ॰स्मृतिद्वयो मूर्त्तिश्च॰ \msParis\msCb\msNb, 
॰स्मृतिद्वयो मूर्त्ति च॰ \msNa\msNc, 
॰स्मृतिर्द्वयो मूर्तिश्च \Ed}}% 
    \var{{\devanagarifont \numnoemph\vb\textbf{॰वृषः}\lem \msParis\msCa\msCb\msNa\msNb\Ed, ॰वृष \msNc}}% 

%Verse 3:4

{\devanagarifont चतुराश्रम यो धर्मः कीर्तितानि मनीषिभिः {॥३:४॥} \veg\dontdisplaylinenum }%
     \var{{\devanagarifont \numnoemph\vc\textbf{चतुरा॰}\lem \msParis\msCb\msNa\msNb\Ed, चातुरा॰ \msCa\msNc}}% 
    \paral{{\devanagarifontsmall \vo {\englishfont \compare\ 4.74 below:}
                 चतुष्पादः स्मृतो धर्मश्चतुराश्रममाश्रितः\thinspace{\devanagarifontsmall ।}
                 गृहस्थो ब्रह्मचारी च वानप्रस्थो ऽथ भैक्षुकः\thinspace{\devanagarifontsmall ॥} }}

{\devanagarifont गतिश्च पञ्च विज्ञेयाः शृणु धर्मस्य भो द्विज \thinspace{\dandab} \dontdisplaylinenum }%
     \var{{\devanagarifont \numemph\va\textbf{विज्ञेयाः}\lem \eme, विज्ञेयः \msParis\msCa\msNa\msNb\msNc\Ed, \om\ \msCb}}% 
    \lacuna{\devanagarifontsmall \vab {\englishfont \msCb\ reads here } गतिश्च पौत्राश्च अनेकाश्च बभूव ह,
                        {\englishfont skipping to 3.7cd, omitting 3.5--7ab.} }%
  
%Verse 3:5

{\devanagarifont देवमानुषतिर्यं च नरकस्थावरादयः {॥३:५॥} \veg\dontdisplaylinenum }%
     \var{{\devanagarifont \numnoemph\vc\textbf{॰मानुष॰}\lem \msParis\msCa\msCb\msNa\msNb\msNc\Ed, ॰मानुषि॰ \msParis}}% 

{\devanagarifont ब्रह्मणो हृदयं भित्त्वा जातो धर्मः सनातनः \thinspace{\dandab} \dontdisplaylinenum }%
     \var{{\devanagarifont \numemph\va\textbf{ब्रह्मणो}\lem \msParis\msCa\msNa\msNb\msNc, \om\ \msCb, ब्राह्मणो \Ed\oo 
\textbf{भित्त्वा}\lem \msParis\msCa\msCb\msNa\msNc\Ed, वित्त्वा \msNb}}% 
    \var{{\devanagarifont \numnoemph\vb\textbf{धर्मः}\lem \msParis\msCa\msCb\msNa\msNc\Ed, धर्म \msNb}}% 
    \paral{{\devanagarifontsmall \vab {\englishfont \compare\ \DEVIP\ 4.59cd:} ब्रह्मणो हृदयाज्जातः पुत्रो धर्म इति स्मृतः \oo 
                     {\englishfont \compare\ also \MBH\ 1.60.40ab:} ब्रह्मणो हृदयं भित्त्वा निःसृतो भगवान्भृगुः }}

%Verse 3:6

{\devanagarifont तस्य पत्नी महाभागा त्रयोदश सुमध्यमाः {॥३:६॥} \veg\dontdisplaylinenum }%
     \var{{\devanagarifont \numnoemph\vd\textbf{॰मध्यमाः}\lem \msParis\msCa\msNa\msNb\msNc\Ed, \om\ \msCb}}% 

{\devanagarifont दक्षकन्या विशालाक्षी श्रद्धाद्याः सुमनोहराः \thinspace{\dandab} \dontdisplaylinenum }%
     \var{{\devanagarifont \numemph\va\textbf{॰आक्षी}\lem \msParis\msCa\msNa\msNb\msNc, \om\ \msCb, ॰आक्षि \Ed}}% 
    \var{{\devanagarifont \numnoemph\vb\textbf{॰आद्याः}\lem \eme, ॰आद्या \msParis\msNb\msNc\Ed, ॰आढ्या \msCa, \om\ \msCb, ॰आढ्याः \msNa\oo 
\textbf{॰हराः}\lem \msNb\Ed, ॰हरा \msParis\msCa\msNc,  \om\ \msCb, ॰\lk \uncl{माः} \msNa}}% 

{\devanagarifont तस्य पुत्राश्च पौत्राश्च अनेकाश्च बभूव ह  \danda\dontdisplaylinenum }%
     \var{{\devanagarifont \numnoemph\vcd\textbf{तस्य पुत्राश्च पौत्राश्च अनेकाश्च बभूव ह}\lem \msParis\msCa\msNb, 
गतिश्च पौत्राश्च अनेकाश्च बभूव ह {\englishfont (eyeskip to 3.5a)} \msCb, 
तस्य पुत्राश्च योत्राश्च अनेकाश्च बभूव ह \msNa\msNc, 
तस्य पुत्रा अनेकाश्च तथा पौत्रा बभूवहः \Ed}}% 

%Verse 3:7

{\devanagarifont एष धर्मनिसर्गो ऽयं किं भूयः श्रोतुमिच्छसि {॥३:७॥} \veg\dontdisplaylinenum }%
 
{\devanagarifont विगतराग उवाच {\dandab}\dontdisplaylinenum  }%
     \var{{\devanagarifont \numemph\vo\textbf{विगतराग उवाच}\lem \msCb\msNapcorr\msNc\Ed, विगतराग उ \msParis\msCa\msNb, \om\ \msNaacorr}}% 

{\devanagarifont धर्मपत्नी विशेषेण पुत्रस्ताभ्यः पृथक्पृथक् \thinspace{\danda} \dontdisplaylinenum }%
     \var{{\devanagarifont \numnoemph\vb\textbf{ताभ्यः}\lem \eme, तेभ्यः \msCa\msCb\msNa\msNb\msNc\Ed}}% 

%Verse 3:8

{\devanagarifont श्रोतुमिच्छामि तत्त्वेन कथयस्व तपोधन {॥३:८॥} \veg\dontdisplaylinenum }%
 
{\devanagarifont अनर्थयज्ञ उवाच {\dandab}\dontdisplaylinenum  }%
 
{\devanagarifont श्रद्धा लक्ष्मीर्धृतिस्तुष्टिः पुष्टिर्मेधा क्रिया लज्जा \thinspace{\danda} \dontdisplaylinenum }%
     \var{{\devanagarifont \numemph\va\textbf{लक्ष्मीर्धृतिस्तुष्टिः}\lem \msCa, 
लक्ष्मीर्धृतिस्तुष् \msCb, 
लक्ष्मी द्धृतिर्द्धृतिस्तुष्टिः \msNaacorr, 
लक्ष्मीर्द्धृतिस्तुष्टिः \msNapcorr, 
लक्ष्मीं धृति तुष्टिः \msNb, 
लक्ष्मी धृतिस्तुष्टिः \msParis\msNc, 
लक्ष्मी धृतिस्तुष्टी \Ed}}% 
    \var{{\devanagarifont \numnoemph\vb\textbf{पुष्टिर्मे॰}\lem \msParis\msCa\msCb\msNa\msNb\msNc, पुष्टि मे॰ \Ed\oo 
\textbf{लज्जा}\lem \msParis\msCa\msCb\msNb\msNc\Ed, लजा \msNa}}% 

%Verse 3:9

{\devanagarifont बुद्धिः शान्तिर्वपुः कीर्तिः सिद्धिः प्रसूतिसम्भवाः {॥३:९॥} \veg\dontdisplaylinenum }%
     \var{{\devanagarifont \numnoemph\vc\textbf{बुद्धिः}\lem \msParis\msCb\msNa\msNb\msNc\Ed, बुद्धि \msCa}}% 
    \var{{\devanagarifont \numnoemph\vd\textbf{सिद्धिः प्रसूतिसम्भवाः}\lem \conj, सिद्धिश्चाभूतिसम्भवाः \msParis, 
सिद्धिश्चाभूतिसम्भवा \msCa\msNa\msNb\msNc, 
सिद्धिश्चातिसम्भवा \msCb, सिद्धिश्च भूतिसम्भवा \Ed}}% 

{\devanagarifont श्रद्धा कामः सुतो जातो दर्पो लक्ष्मीसुतः स्मृतः \thinspace{\dandab} \dontdisplaylinenum }%
     \var{{\devanagarifont \numemph\va\textbf{कामः}\lem \msNa, काम॰ \msParis\msCa\msCb\msNb\msNc, धर्म॰ \Ed}}% 

%Verse 3:10

{\devanagarifont धृत्यास्तु नियमः पुत्रः संतोषस्तुष्टिजः स्मृतः {॥३:१०॥} \veg\dontdisplaylinenum }%
     \paral{{\devanagarifontsmall \vo {\englishfont For 3.10--13, see a rather similar 
         passage e.g.\ in \KURMP\ 1.8.20 ff.:}
         श्रद्धाया आत्मजः कामो दर्पो लक्ष्मीसुतः स्मृतः\thinspace{\devanagarifontsmall ।}
         धृत्यास्तु नियमः पुत्रस्तुष्ट्याः संतोष उच्यते\thinspace{\devanagarifontsmall ॥} 
         पुष्ट्या लाभः सुतश्चापि मेधापुत्रः श्रुतस्तथा\thinspace{\devanagarifontsmall ।} 
         क्रियायाश्चाभवत्पुत्रो दण्डः समय एव च\thinspace{\devanagarifontsmall ॥}  
         बुद्ध्या बोधः सुतस्तद्वदप्रमादो व्यजायत\thinspace{\devanagarifontsmall ।} 
         लज्जाया विनयः पुत्रो वपुषो व्यवसायकः\thinspace{\devanagarifontsmall ॥}  
         क्षेमः शान्तिसुतश्चापि सुखं सिद्धिरजायत\thinspace{\devanagarifontsmall ।}
         यशः कीर्तिसुतस्तद्वदित्येते धर्मसूनवः\thinspace{\devanagarifontsmall ॥}   
         कामस्य हर्षः पुत्रो ऽभूद्देवानन्दो व्यजायत\thinspace{\devanagarifontsmall ।}  
         इत्येष वै सुखोदर्कः सर्गो धर्मस्य कीर्तितः\thinspace{\devanagarifontsmall ॥} }}

{\devanagarifont पुष्ट्या लाभः सुतो जातो मेधापुत्रः श्रुतस्तथा \thinspace{\dandab} \dontdisplaylinenum }%
     \var{{\devanagarifont \numemph\va\textbf{लाभः}\lem \msCa\msCb\msNb\msNc, लाभ॰ \msNa\Ed}}% 
    \var{{\devanagarifont \numnoemph\vb\textbf{॰पुत्रः}\lem \eme, ॰पुत्र \msCa\msCb\msNa\msNb\msNc\Ed\oo 
\textbf{श्रुत॰}\lem \msCa\msNa\msNb\msNc\Ed, श्रत॰ \msCb}}% 

%Verse 3:11

{\devanagarifont क्रियायास्त्वभवत्पुत्रो दण्डः समय एव च {॥३:११॥} \veg\dontdisplaylinenum }%
     \var{{\devanagarifont \numnoemph\vc\textbf{त्वभवत्पुत्रो}\lem \eme, त्वभयः पुत्रो \msCa\msCb\msNa\msNb\msNc, तूभयः पुत्रौ \Ed}}% 
    \var{{\devanagarifont \numnoemph\vd\textbf{दण्डः}\lem \corr, दण्डे \msCa\msNaacorr, दण्डो \msCb, दण्ड॰ \msNapcorr\msNb\msNc\Ed\oo 
\textbf{च}\lem \msCa\msCb\msNa\msNb\msNc, तु \Ed}}% 
    \paral{{\devanagarifontsmall \vcd {\englishfont \similar\ \LINPU\ 1.70.295ab:}क्रियायामभवत्पुत्रो दण्डः समय एव च;
                     {\englishfont \similar\ \KURMP\ 1.8.22cd:   }क्रियायाश्चाभवत्पुत्रो दण्डः समय एव च;
                     {\englishfont \compare\ \LINPU\ 1,5.37:     }धर्मस्य वै क्रियायां तु दण्डः समय एव च }}

{\devanagarifont लज्जाया विनयः पुत्रो बुद्ध्या बोधःसुतः स्मृतः \thinspace{\dandab} \dontdisplaylinenum }%
     \var{{\devanagarifont \numemph\va\textbf{लज्जाया विनयः}\lem \msCa\msCb\msNa\msNb\msNc, लज्जायाः विनय॰ \Ed}}% 
    \var{{\devanagarifont \numnoemph\vb\textbf{सुतः स्मृतः}\lem \msNa\msNb\msNc\Ed, सुतः \lk\lk\ \msCa, सुतःस्तथा \msCb}}% 

%Verse 3:12

{\devanagarifont लज्जायाः सुधियः पुत्र अप्रमादश्च तावुभौ {॥३:१२॥} \veg\dontdisplaylinenum }%
     \var{{\devanagarifont \numnoemph\vc\textbf{सुधियः}\lem \Ed, सुधिय \msCa\msCb\msNa\msNb\msNc\oo 
\textbf{पुत्र}\lem \msCa\msCb\msNa\msNb\msNc, पुत्रः \Ed}}% 
    \var{{\devanagarifont \numnoemph\vd\textbf{अप्रमाद॰}\lem \msCa\msCb\msNb\msNc\Ed, अप्रमादा॰ \msNa}}% 

{\devanagarifont क्षेमः शान्तिसुतो विन्द्याद्व्यवसायो वपोः सुतः \thinspace{\dandab} \dontdisplaylinenum }%
     \var{{\devanagarifont \numemph\vb\textbf{वपोः}\lem \msCa\msCb\msNb\msNc\Ed, वपो \msNa}}% 

{\devanagarifont यशः कीर्तिसुतो ज्ञेयः सुखं सिद्धेर्व्यजायत  \danda\dontdisplaylinenum }%
     \var{{\devanagarifont \numnoemph\vd\textbf{सिद्धे॰}\lem \msCb\msNa\msNb, सिद्धि \msCa\msNc\Ed\oo 
\textbf{व्यजायत}\lem \msCa\msCb\msNa, व्यजायते \msNb\Ed, व्यजायतः \msNc}}% 

%Verse 3:13

{\devanagarifont स्वायम्भुवे ऽन्तरे त्वासन्कीर्तिता धर्मसूनवः {॥३:१३॥} \veg\dontdisplaylinenum }%
     \var{{\devanagarifont \numnoemph\ve\textbf{स्वायम्भुवे}\lem \msCa\msNa\msNc, स्वायम्भुवो \msCb, स्वयम्भुवे \msNb\Ed\oo 
\textbf{ऽन्तरे त्वासन्}\lem \conj, ऽन्तरे त्वासि \msCa\msCb\msNa, 
ऽन्तरे त्वासीत् \msNb, ऽन्तरे त्वासं \msNc, ऽन्तरेवासि \Ed}}% 

{\devanagarifont विगतराग उवाच {\dandab}\dontdisplaylinenum  }%
 
{\devanagarifont मूर्तिद्वयं कथं धर्मं कथयस्व तपोधन \thinspace{\danda} \dontdisplaylinenum }%
     \var{{\devanagarifont \numemph\va\textbf{धर्मं}\lem \msCa\msCb\msNa\msNb, द्धर्म \msNc, धर्मः \Ed}}% 

%Verse 3:14

{\devanagarifont कौतूहलमतीवं मे कर्तय ज्ञानसंशयम् {॥३:१४॥} \veg\dontdisplaylinenum }%
     \var{{\devanagarifont \numnoemph\vc\textbf{कौतूहल॰}\lem \msCa\msNa\msNb\msNc\Ed, कोतूहल॰ \msCb\oo 
\textbf{॰तीवं मे}\lem \msCa\msNa\msNb\msNc\Ed, ॰तीव मे \msCb}}% 
    \var{{\devanagarifont \numnoemph\vd\textbf{कर्तय}\lem \eme, कीर्तय \msCa\msCb\msNa\msNb\msNc\Ed\oo 
\textbf{॰संशयम्}\lem \msCa\msNa\msNc\Ed, ॰संशयः \msCb\msNb}}% 
    \lacuna{\devanagarifontsmall \vc {\englishfont In \msParis, folio 215v ends with }कौतूहलमती {\englishfont and the next available folio side (217r) starts with 
                    } त्यमिष्टगतिः प्रोक्तं {\englishfont  in 4.8a. Thus one folio (f. 216), containing
                      3.14d--4.7, is missing.} }%
  
{\devanagarifont अनर्थयज्ञ उवाच {\dandab}\dontdisplaylinenum  }%
 
{\devanagarifont श्रुतिस्मृतिद्वयोर्मूर्तिर्धर्मस्य परिकीर्तिता \thinspace{\danda} \dontdisplaylinenum }%
     \var{{\devanagarifont \numemph\va\textbf{श्रुति॰}\lem \msCa\msNa\msNb\msNc, श्रुतिः \msCb\Ed}}% 
    \var{{\devanagarifont \numnoemph\vab\textbf{॰द्वयोर्मूर्तिर्ध॰}\lem \msCa, ॰द्वयो मूर्ति ध॰ \msCb\msNa\msNb, ॰द्वयी मूर्ति ध॰ \msNc, 
॰द्वयोर्मूर्ति ध॰ \Ed}}% 
    \var{{\devanagarifont \numnoemph\vb\textbf{॰कीर्तिता}\lem \msCa\msCb\msNa\Ed, ॰कीर्त्तितः \msNb, कीर्त्तिताः \msNc}}% 

{\devanagarifont दाराग्निहोत्रसम्बन्धमिज्या श्रौतस्य लक्षणम्  \danda\dontdisplaylinenum }%
     \var{{\devanagarifont \numnoemph\vcd\textbf{॰बन्धमि॰}\lem \eme, ॰बद्ध इ॰ \msCa\msCb\msNa\msNc, ॰बन्ध इ॰ \msNb\Ed}}% 
    \var{{\devanagarifont \numnoemph\vd\textbf{श्रौतस्य}\lem \eme, श्रोतस्य \msCa\msCb\msNc, श्रौत्रस्य \msNa, स्रोत्रस्य \msNb, श्रुतस्य \Ed}}% 
    \paral{{\devanagarifontsmall \vcd {\englishfont \compare\ \MANU\ 3.171ab:}दाराग्निहोत्रसंयोगं कुरुते यो ऽग्रजे स्थिते; 
                         {\englishfont and also \MATSP\ 142.41:} 
                         दाराग्निहोत्रसम्बन्धमृग्यजुःसामसंहिताः\thinspace{\devanagarifontsmall ।}
                         इत्यादिबहुलं श्रौतं धर्मं सप्तर्षयो ऽब्रुवन्\thinspace{\devanagarifontsmall ॥} }}

%Verse 3:15

{\devanagarifont स्मार्तो वर्णाश्रमाचारो यमैश्च नियमैर्युतः {॥३:१५॥} \veg\dontdisplaylinenum }%
     \var{{\devanagarifont \numnoemph\ve\textbf{स्मार्तो}\lem \eme, स्मार्त \msCa\msCb\msNa\msNb\msNc\Ed}}% 
    \paral{{\devanagarifontsmall \vcdef\ {\englishfont  \similar\ \MBH\ Indices 1.36.10: 
                                 }दानाग्निहोत्रमिज्या च श्रौतस्यैतद्धि लक्षणम्\thinspace{\devanagarifontsmall ।}
                                 स्मार्तो वर्णाश्रमाचारो यमैश्च नियमैर्युतः\thinspace{\devanagarifontsmall ॥}
                          \similar\ {\englishfont \MATSP\ 145.30cd--31ab:
                                 }दाराग्निहोत्रसम्बन्धमिज्या श्रौतस्य लक्षणम्\thinspace{\devanagarifontsmall ।}
                                 स्मार्तो वर्णाश्रमाचारो यमैश्च नियमैर्युतः\thinspace{\devanagarifontsmall ॥}
                          \similar\ {\englishfont \BRAHMANDAPUR\ 1.32.33cd--34ab:}
                                 दाराग्निहोत्रसम्बन्धाद् द्विधा श्रौतस्य लक्षणम्\thinspace{\devanagarifontsmall ।}
                                 स्मार्तो वर्णाश्रमाचारैर्यमैः स नियमैः स्मृतः\thinspace{\devanagarifontsmall ॥} }}


\alalfejezet{यमनियमभेदः}
{\devanagarifont यमश्च नियमश्चैव द्वयोर्भेदमतः शृणु \thinspace{\dandab} \dontdisplaylinenum }%
     \var{{\devanagarifont \numemph\va\textbf{नियम॰}\lem \msCa\msCb\msNb\msNc\Ed, नियमै॰ \msNa}}% 

{\devanagarifont अहिंसा सत्यमस्तेयमानृशंस्यं दमो घृणा  \danda\dontdisplaylinenum }%
     \var{{\devanagarifont \numnoemph\vd\textbf{॰मानृशंस्यं}\lem \eme, ॰मनृशंस्यो \msCa\msCb\msNa\msNb\Ed, ॰मानृशंस्या \msNc}}% 
    \paral{{\devanagarifontsmall \vcd {\englishfont \similar\ \MBH\ 12.8.17ab:} अहिंसा सत्यवचनमानृशंस्यं दमो घृणा
                 \vo {\englishfont \similar\ \VDHU\ 3.233.203: 
                         }आनृशंस्यं क्षमा सत्यमहिंसा च दमः स्पृहा\thinspace{\devanagarifontsmall ।}
                         ध्यानं प्रसादो माधुर्यं चार्जवं च यमा दश\thinspace{\devanagarifontsmall ॥} }}

%Verse 3:16

{\devanagarifont धन्याप्रमादो माधुर्यमार्जवं च यमा दश {॥३:१६॥} \veg\dontdisplaylinenum }%
     \var{{\devanagarifont \numnoemph\ve\textbf{धन्या॰}\lem \Ed, धन्यः \msCa\msCb\msNb\msNc, ध्यन्यं \msNa\oo 
\textbf{माधुर्य॰}\lem \Ed, माधूर्य॰ \msCa\msCb\msNa\msNb\msNc}}% 
    \var{{\devanagarifont \numnoemph\vf\textbf{आर्जवं च}\lem \msCa\msCb\msNa\msNb\msNc, आर्जवश्च \Ed}}% 

{\devanagarifont एकैकस्य पुनः पञ्चभेदमाहुर्मनीषिणः \thinspace{\dandab} \dontdisplaylinenum }%
     \var{{\devanagarifont \numemph\vb\textbf{॰माहुर्म॰}\lem \msCa\msCb\msNa\msNb\Ed, ॰माहु म॰ \msNc}}% 

%Verse 3:17

{\devanagarifont अहिंसादि प्रवक्ष्यामि शृणुष्वावहितो द्विज {॥३:१७॥} \veg\dontdisplaylinenum }%
     \var{{\devanagarifont \numnoemph\vd\textbf{शृणुष्वा॰}\lem \msCa\msCb\msNc\Ed, शृणुष्व॰ \msNa\msNb}}% 


\alalfejezet{यमेष्वहिंसा (१)}

\alalalfejezet{पञ्चविधा हिंसा}

{\devanagarifont त्रासनं ताडनं बन्धो मारणं वृत्तिनाशनम् \thinspace{\dandab} \dontdisplaylinenum }%
     \var{{\devanagarifont \numemph\va\textbf{बन्धो}\lem \msCa\msCb\msNa\msNc, बद्धो \msNb, बन्ध \Ed}}% 

%Verse 3:18

{\devanagarifont हिंसां पञ्चविधामाहुर्मुनयस्तत्त्वदर्शिनः {॥३:१८॥} \veg\dontdisplaylinenum }%
     \var{{\devanagarifont \numnoemph\vc\textbf{हिंसां}\lem \msCa\msNa\msNc, हिंसा \msCb\msNb\Ed\oo 
\textbf{॰विधामाहु॰}\lem \msCb\msNa\msNc, ॰विधमाहु॰ \msCa, 
॰विधान्याहु॰ \msNb, ॰विध प्राहु॰ \Ed}}% 

{\devanagarifont काष्ठलोष्टकशाद्यैस्तु ताडयन्तीह निर्दयाः \thinspace{\dandab} \dontdisplaylinenum }%
     \var{{\devanagarifont \numemph\va\textbf{काष्ठलोष्ट॰}\lem \msCa\msCb\msNa\msNc\Ed, का\uncl{ष्ठ}\lac\  \msNb}}% 
    \var{{\devanagarifont \numnoemph\vb\textbf{निर्दयाः}\lem \msCa\msCb\msNa\msNb\msNc, निर्दया \Ed}}% 

%Verse 3:19

{\devanagarifont तत्प्रहारविभिन्नाङ्गो मृतवध्यमवाप्नुयात् {॥३:१९॥} \veg\dontdisplaylinenum }%
     \var{{\devanagarifont \numnoemph\vc\textbf{॰भिन्नाङ्गो}\lem \msCa\msCb\msNa\msNb\msNc, ॰भिन्नाङ्गा \Ed}}% 
    \var{{\devanagarifont \numnoemph\vd\textbf{॰वध्यमवा॰}\lem \msCb\msNa\msNb\msNc\Ed, ॰वध्यववा॰ \msCa}}% 

{\devanagarifont बद्ध्वा पादौ भुजोरश्च शिरोरुक्कण्ठपाशिताः \thinspace{\dandab} \dontdisplaylinenum }%
     \var{{\devanagarifont \numemph\va\textbf{भुजोरश्च}\lem \msCa\msCb\msNb\msNc, भुजौरश्च \msNa\Ed}}% 
    \var{{\devanagarifont \numnoemph\vb\textbf{शिरोरुक्कण्ठ॰}\lem \eme, शिरोरुकण्ठ॰ \msCa\msCb\msNa\msNb\msNc, शिरोरुः कण्ठ॰ \Ed}}% 

%Verse 3:20

{\devanagarifont अनाहता म्रियन्त्येवं वधो बन्धनजः स्मृतः {॥३:२०॥} \veg\dontdisplaylinenum }%
     \var{{\devanagarifont \numnoemph\vc\textbf{अनाहता म्रियन्त्येवं}\lem \msCa\msCb\msNa\msNc\Ed, अनाहत म्रियंत्येष \msNb}}% 
    \var{{\devanagarifont \numnoemph\vd\textbf{वधो बन्धनजः स्मृतः}\lem \conj, ॰नजाः स्मृताः \msCa\msCb\msNa\msNb, 
॰नजाः स्मृता \msNc, ॰नज स्मृतः \Ed}}% 

{\devanagarifont शत्रुचौरभयैर्घोरैः सिंहव्याघ्रगजोरगैः \thinspace{\dandab} \dontdisplaylinenum }%
     \var{{\devanagarifont \numemph\va\textbf{॰चौरभयैर्घोरैः}\lem \msCa\msCb\msNa\msNc\Ed, ॰चोरभयै घोरै \msNb}}% 

%Verse 3:21

{\devanagarifont त्रासनाद्वधमाप्नोति अन्यैर्वापि सुदुःसहैः {॥३:२१॥} \veg\dontdisplaylinenum }%
     \var{{\devanagarifont \numnoemph\vd\textbf{अन्यैर्वापि}\lem \msCa\msCb\msNa\msNb\Ed, अन्ये चापि \msNc}}% 

{\devanagarifont यस्य यस्य हरेद्वित्तं तस्य तस्य वधः स्मृतः \thinspace{\dandab} \dontdisplaylinenum }%
     \var{{\devanagarifont \numemph\va\textbf{हरेद्वि॰}\lem \msCa\msCb\msNa\msNc\Ed, हरे वि॰ \msNb}}% 
    \var{{\devanagarifont \numnoemph\vb\textbf{वधः}\lem \msCa\msCb\msNa\msNb\msNc, वध \Ed}}% 

%Verse 3:22

{\devanagarifont वृत्तिजीवाभिभूतानां तद्द्वारा निहतः स्मृतः {॥३:२२॥} \veg\dontdisplaylinenum }%
     \var{{\devanagarifont \numnoemph\vc\textbf{॰भिभूतानां}\lem \msCa\msCb\msNa\msNc\Ed, ॰विभूतानां \msNb}}% 
    \var{{\devanagarifont \numnoemph\vd\textbf{तद्द्वारा नि॰}\lem \conj, तद्वारान्नि॰ \msCa\msCb\msNa\msNb\msNc, तद्द्वारान्नि॰ \Ed}}% 

{\devanagarifont विषवह्निशरशस्त्रैर्मायायोगबलेन वा \thinspace{\dandab} \dontdisplaylinenum }%
     \var{{\devanagarifont \numemph\vab\textbf{॰शस्त्रैर्माया॰}\lem \msCa\msCb\msNa\msNb, ॰शस्त्रै मा॰ \msNc, ॰शस्त्रैर्म्मया॰ \Ed}}% 

%Verse 3:23

{\devanagarifont हिंसकान्याहु विप्रेन्द्र मुनयस्तत्त्वदर्शिनः {॥३:२३॥} \veg\dontdisplaylinenum }%
     \var{{\devanagarifont \numnoemph\vc\textbf{हिंसकान्याहु वि॰}\lem \msCb\msNb\msNc, 
हिंसकान्याहुर्वि॰ \msCa\msNa\ \unmetr, हिंसकेत्याहु वि॰ \Ed}}% 


\alalalfejezet{अहिंसाप्रशंसा}

{\devanagarifont अहिंसा परमं धर्मं यस्त्यजेत्स दुरात्मवान् \thinspace{\dandab} \dontdisplaylinenum }%
     \var{{\devanagarifont \numemph\va\textbf{परमं धर्मं}\lem \msCa\msCb\msNa\Ed, परमं धर्म \msNb, परमो धर्मं \msNc}}% 
    \var{{\devanagarifont \numnoemph\vb\textbf{त्यजेत्स दुरात्मवान्}\lem \msCb\msNc\Ed, त्यजेच्छ दुरात्म\lk\ \msCa, त्यजेत्सुदुरात्मवान् \msNa, 
त्यजेत्स दुरात्मनम् \msNb}}% 

%Verse 3:24

{\devanagarifont क्लेशायासविनिर्मुक्तं सर्वधर्मफलप्रदम् {॥३:२४॥} \veg\dontdisplaylinenum }%
 
{\devanagarifont नातः परतरो मूर्खो नातः परतरं तमः \thinspace{\dandab} \dontdisplaylinenum }%
     \var{{\devanagarifont \numemph\vb\textbf{॰तरं}\lem \msCa\msCbpcorr\msNa\msNb\msNc, ॰तन् \msCbacorr\Ed}}% 

%Verse 3:25

{\devanagarifont नातः परतरं दुःखं नातः परतरो ऽयशः {॥३:२५॥} \veg\dontdisplaylinenum }%
 
{\devanagarifont नातः परतरं पापं नातः परतरं विषम् \thinspace{\dandab} \dontdisplaylinenum }%
 
%Verse 3:26

{\devanagarifont नातः परतराविद्या नातः परं तपोधन {॥३:२६॥} \veg\dontdisplaylinenum }%
     \var{{\devanagarifont \numemph\vd\textbf{परं तपोधन}\lem \msCa\msCb\msNa\msNb\msNc, पर तपोद्यमाः \Ed}}% 

{\devanagarifont यो हिनस्ति न भूतानि उद्भिज्जादि चतुर्विधम् \thinspace{\dandab} \dontdisplaylinenum }%
     \var{{\devanagarifont \numemph\va\textbf{यो हिनस्ति न}\lem \msCa\msCb\msNa\msNc, यो न हिन्सन्ति \msNb, यो हि नास्ति न \Ed}}% 
    \var{{\devanagarifont \numnoemph\vb\textbf{उद्भिज्जादि}\lem \eme, उद्भिजादि \msCa\msCb\msNb\msNc\Ed, उद्भिजानि \msNa\oo 
\textbf{॰विधम्}\lem \msCa\msCb\msNa\msNb\Ed, ॰विधिं \msNc}}% 

%Verse 3:27

{\devanagarifont स भवेत्पुरुषः श्रेष्ठः सर्वभूतदयान्वितः {॥३:२७॥} \veg\dontdisplaylinenum }%
     \var{{\devanagarifont \numnoemph\vc\textbf{पुरुषः}\lem \msCa\msCb\msNa\msNb\msNc, पुरुष॰ \Ed}}% 

{\devanagarifont सर्वभूतदयां नित्यं यः करोति स पण्डितः \thinspace{\dandab} \dontdisplaylinenum }%
     \var{{\devanagarifont \numemph\va\textbf{॰दयां नित्यं}\lem \msCa\msNa\Ed, ॰दया नित्यं \msCb\msNb, ॰दया नित्य \msNc}}% 

%Verse 3:28

{\devanagarifont स यज्वा स तपस्वी च स दाता स दृढव्रतः {॥३:२८॥} \veg\dontdisplaylinenum }%
     \var{{\devanagarifont \numnoemph\vc\textbf{यज्वा}\lem \msCa\msCb\msNa\msNc\Ed, यज्या \msNb}}% 

{\devanagarifont अहिंसा परमं तीर्थमहिंसा परमं तपः \thinspace{\dandab} \dontdisplaylinenum }%
     \var{{\devanagarifont \numemph\va\textbf{परमं ती॰}\lem \msCa\msNa\msNb\msNc\Ed, परन्ती॰ \msCb}}% 

%Verse 3:29

{\devanagarifont अहिंसा परमं दानमहिंसा परमं सुखम् {॥३:२९॥} \veg\dontdisplaylinenum }%
     \paral{{\devanagarifontsmall \vo {\englishfont This and the following verses are similar to MBh 13.117.37--38} }}
    \lacuna{\devanagarifontsmall \vd {\englishfont \msCc\ resumes here in exp.\ 189, f. 273r (sic!) with }रमं सुखम् }%
  
{\devanagarifont अहिंसा परमो यज्ञः अहिंसा परमं व्रतम् \thinspace{\dandab} \dontdisplaylinenum }%
     \var{{\devanagarifont \numemph\va\textbf{यज्ञः}\lem \msCb\msCc\msNb\Ed, यज्ञर् \msCa, यज्ञ \msNa\msNc}}% 

%Verse 3:30

{\devanagarifont अहिंसा परमं ज्ञानमहिंसा परमा क्रिया {॥३:३०॥} \veg\dontdisplaylinenum }%
     \var{{\devanagarifont \numnoemph\vc\textbf{परमं}\lem \mssCaCbCc\msNa\msNb\msNc, परमो \Ed}}% 
    \var{{\devanagarifont \numnoemph\vd\textbf{परमा}\lem \mssCaCbCc\msNa\msNc\Ed, परमां \msNb}}% 

{\devanagarifont अहिंसा परमं शौचमहिंसा परमो दमः \thinspace{\dandab} \dontdisplaylinenum }%
     \var{{\devanagarifont \numemph\vab\textbf{(अहिंसा{\englishfont ...} दमः)}\lem \mssCaCbCc\msNa\msNb\msNc, \om\ \Ed}}% 

%Verse 3:31

{\devanagarifont अहिंसा परमो लाभः अहिंसा परमं यशः {॥३:३१॥} \veg\dontdisplaylinenum }%
     \var{{\devanagarifont \numnoemph\vc\textbf{लाभः}\lem \msNc, लाभ \msCa\msCb\msNa\msNb\Ed, लाभो \msCc}}% 
    \var{{\devanagarifont \numnoemph\vd\textbf{परमं}\lem \mssCaCbCc\msNb\msNc\Ed, परमा \msNa}}% 
    \lacuna{\devanagarifontsmall {\englishfont After pādas cd, \Ed\ inserts this: }अहिंसा परमा कीर्ति अहिंसा परमो दमः,
                 {\englishfont which is not to be found in \mssCaCbCc\msNa\msNb\msNc} }%
  
{\devanagarifont अहिंसा परमो धर्मः अहिंसा परमा गतिः \thinspace{\dandab} \dontdisplaylinenum }%
     \var{{\devanagarifont \numemph\va\textbf{धर्मः}\lem \msNa\msNc, धर्म \msCa\msCb\Ed, धर्मो \msCc, ध\lac\  \msNb}}% 
    \var{{\devanagarifont \numnoemph\vb\textbf{अहिंसा परमा गतिः}\lem \mssCaCbCc\msNa\msNc, \lac\  \msNb, अहिंसा परमो गतिः \Ed}}% 

%Verse 3:32

{\devanagarifont अहिंसा परमं ब्रह्म अहिंसा परमः शिवः {॥३:३२॥} \veg\dontdisplaylinenum }%
     \var{{\devanagarifont \numnoemph\vc \lem \mssCaCbCc\msNa\Ed, 
\uncl{अहिंसा परमं ब्रह्म} \msNb, अहिंसा परंमं ब्रह्म \msNc}}% 


\alalalfejezet{मांसाहारः}

{\devanagarifont मांसाशनान्निवर्तेत मनसापि न काङ्क्षयेत् \thinspace{\dandab} \dontdisplaylinenum }%
     \var{{\devanagarifont \numemph\va\textbf{मांसाशनान्नि॰}\lem \msCa\msCb\Ed, मान्साशन नि॰ \msCc, 
मांसाशनन्नि॰ \msNa, मन्सासनन्नि॰ \msNb, \uncl{मांसशानान्नि}॰ \msNc}}% 

%Verse 3:33

{\devanagarifont स महत्फलमाप्नोति यस्तु मांसं विवर्जयेत् {॥३:३३॥} \veg\dontdisplaylinenum }%
     \var{{\devanagarifont \numnoemph\vd\textbf{मांसं}\lem \mssCaCbCc\msNa, मांस \msNb\Ed, मासं \msNc}}% 

{\devanagarifont स्वमांसं परमांसेन यो वर्धयितुमिच्छति \thinspace{\dandab} \dontdisplaylinenum }%
     \var{{\devanagarifont \numemph\va\textbf{॰मांसेन}\lem \mssCaCbCc\msNa\msNb\Ed, ॰मासेन \msNc}}% 
    \var{{\devanagarifont \numnoemph\vb\textbf{वर्धयितु॰}\lem \mssCaCbCc\msNa\msNc\Ed, वर्द्धयति \msNb}}% 
    \paral{{\devanagarifontsmall \vab {\englishfont  = \MBH\ 13.116.14ab and 13.116.34ab \similar\ \UUMS\ 2.48cd:
                          }स्वमांसं परमांसेन यो देहे वृद्धिमिच्छति }}

%Verse 3:34

{\devanagarifont अनभ्यर्च्य पितॄन्देवान्न ततो ऽन्यो ऽस्ति पापकृत् {॥३:३४॥} \veg\dontdisplaylinenum }%
     \var{{\devanagarifont \numnoemph\vc\textbf{पितॄन्}\lem \msCa\msCb\msNa\msNc, पितृन् \msCc\Ed, \uncl{पितॄन्} \msNb}}% 
    \var{{\devanagarifont \numnoemph\vd\textbf{ततो ऽन्यो}\lem \mssCaCbCc\msNa\msNb\msNc, तदन्यो \Ed}}% 
    \paral{{\devanagarifontsmall \vo {\englishfont \similar\ \MANU\ 5.52} }}

{\devanagarifont मधुपर्के च यज्ञे च पितृदैवतकर्मणि \thinspace{\dandab} \dontdisplaylinenum }%
     \var{{\devanagarifont \numemph\vb\textbf{॰दैवत॰}\lem \msCa\msCb\msNa\msNc\Ed, ॰देवत॰ \msCc\msNb}}% 

%Verse 3:35

{\devanagarifont अत्रैव पशवो हिंस्या नान्यत्र मनुरब्रवीत् {॥३:३५॥} \veg\dontdisplaylinenum }%
     \var{{\devanagarifont \numnoemph\vc\textbf{अत्रैव पशवो हिंस्या}\lem \msCa\msCc\msNc\Ed, 
अत्रैव पशवो हिंसा \msCb, अत्रैव पशवो हिंस्यान् \msNa, 
\lac\  \msNb}}% 
    \var{{\devanagarifont \numnoemph\vd\textbf{नान्यत्र मनुरब्रवीत्}\lem \mssCaCbCc\msNa\msNc\Ed, \lac \uncl{त्र मनुरब्रवीत्} \msNb}}% 
    \paral{{\devanagarifontsmall \vo {\englishfont \similar\ \MANU\ 5.41:}
                         मधुपर्के च यज्ञे च पितृदैवतकर्मणि\thinspace{\devanagarifontsmall ।}
                         अत्रैव पशवो हिंस्या नान्यत्रेत्यब्रवीन्मनुः\thinspace{\devanagarifontsmall ॥} }}

{\devanagarifont क्रीत्वा स्वयं वाप्युत्पाद्य परोपहृतमेव वा \thinspace{\dandab} \dontdisplaylinenum }%
     \var{{\devanagarifont \numemph\va\textbf{क्रीत्वा}\lem \mssCaCbCc\msNa\msNb\msNc, कृत्वा \Ed\oo 
\textbf{॰प्युत्पाद्य}\lem \mssCaCbCc\msNa\msNb\msNc, ॰प्युत्पाद्या॰ \Ed}}% 
    \var{{\devanagarifont \numnoemph\vb\textbf{॰हृत॰}\lem \mssCaCbCc\msNa\msNb\msNc, ॰हित॰ \Ed\oo 
\textbf{वा}\lem \mssCaCbCc\msNa\msNb\msNc, च \Ed}}% 

%Verse 3:36

{\devanagarifont देवान्पितॄंश्चार्चयित्वा खादन्मांसं न दोषभाक् {॥३:३६॥} \veg\dontdisplaylinenum }%
     \var{{\devanagarifont \numnoemph\vc\textbf{पितॄंश्चार्चयित्वा}\lem \mssCaCbCc\msNa\msNc, पितॄश्चार्चयित्वा \msNb, पितृश्चार्पयित्वा \Ed}}% 
    \var{{\devanagarifont \numnoemph\vd\textbf{मांसं}\lem \mssCaCbCc\msNa\msNb\Ed, मासं \msNc}}% 
    \paral{{\devanagarifontsmall \vo {\englishfont = \MANU\ 5.32 (in Olivelle's critical edition; other editions read 
                          } परोपकृत॰{\englishfont  in pāda b) } }}

{\devanagarifont वेदयज्ञतपस्तीर्थदानशीलक्रियाव्रतैः \thinspace{\dandab} \dontdisplaylinenum }%
     \var{{\devanagarifont \numemph\vb\textbf{॰शील॰}\lem \msCa\msCb\msNa\msNb\msNc\Ed, ॰शल॰ \msCc\oo 
\textbf{॰व्रतैः}\lem \msCa\msCc\msNa\msNb\msNc\Ed, ॰व्र\uncl{तः} \msCb}}% 

%Verse 3:37

{\devanagarifont मांसाहारनिवृत्तानां षोडशांशं न पूर्यते {॥३:३७॥} \veg\dontdisplaylinenum }%
     \var{{\devanagarifont \numnoemph\vc\textbf{॰वृत्तानां}\lem \mssCaCbCc\msNa\msNc, ॰वृत्ताना \msNb, ॰वृत्तीनां \Ed}}% 
    \var{{\devanagarifont \numnoemph\vd\textbf{न}\lem \msCa\msCc\msNa\msNb\msNc\Ed, त \msCb}}% 

{\devanagarifont मृगाः पर्णतृणाहारादजमेषगवादिभिः \thinspace{\dandab} \dontdisplaylinenum }%
     \var{{\devanagarifont \numemph\va\textbf{पर्ण॰}\lem \mssCaCbCc\msNb\msNc, पण्ण॰ \msNa, पर्णा॰ \Ed}}% 
    \var{{\devanagarifont \numnoemph\vab\textbf{॰हाराद॰}\lem \msCa\msCc\msNbpcorr\msNc\Ed, ॰हारा अ॰ \msCb\msNa, ॰हाद॰ \msNbacorr}}% 

%Verse 3:38

{\devanagarifont सुखिनो बलवन्तश्च विचरन्ति महीतले {॥३:३८॥} \veg\dontdisplaylinenum }%
 
{\devanagarifont वानराः फलमाहारा राक्षसा रुधिरप्रियाः \thinspace{\dandab} \dontdisplaylinenum }%
     \var{{\devanagarifont \numemph\vab\textbf{॰हारा रा॰}\lem \msCb\msNa\msNb, ॰हाराद्रा॰ \msCa\msCc\msNc\Ed}}% 

%Verse 3:39

{\devanagarifont निहता राक्षसाः सर्वे वानरैः फलभोजिभिः {॥३:३९॥} \veg\dontdisplaylinenum }%
     \var{{\devanagarifont \numnoemph\vd\textbf{॰भोजिभिः}\lem \mssCaCbCc\msNa\msNb\msNc, ॰भोगिभिः \Ed}}% 

{\devanagarifont तस्मान्मांसं न हीहेत बलकामेन भो द्विज \thinspace{\dandab} \dontdisplaylinenum }%
     \var{{\devanagarifont \numemph\va\textbf{मांसं}\lem \mssCaCbCc\msNa\msNb\Ed, मासं \msNc}}% 
    \var{{\devanagarifont \numnoemph\vb\textbf{हीहेत}\lem \mssCaCbCc\msNc\Ed, हीयेत \msNa\msNb}}% 

%Verse 3:40

{\devanagarifont बलेन च गुणाकर्षात्परतो भयभीरुणा {॥३:४०॥} \veg\dontdisplaylinenum }%
     \var{{\devanagarifont \numnoemph\vc\textbf{गुणाकर्षा॰}\lem \conjTorzsok, गुणाकाशा॰ \mssCaCbCc\msNa\msNb\msNc, गुणा कुर्या॰ \Ed}}% 

{\devanagarifont अहिंसकसमो नास्ति दानयज्ञसमीहया \thinspace{\dandab} \dontdisplaylinenum }%
     \var{{\devanagarifont \numemph\vb\textbf{॰यज्ञसमीहया}\lem \msCa\msCb\msNa\msNb, ॰धर्मसमीहया \msCc, 
॰यज्ञसमीहयाः \msNc, ॰धर्मसमीहय \Ed}}% 

%Verse 3:41

{\devanagarifont इह लोके यशः कीर्तिः परत्र च परा गतिः {॥३:४१॥} \veg\dontdisplaylinenum }%
     \var{{\devanagarifont \numnoemph\vc\textbf{यशः}\lem \msCa\msCb\msNa\msNb\msNc\Ed, य\uncl{शं} \msCc}}% 
    \var{{\devanagarifont \numnoemph\vd\textbf{परा गतिः}\lem \msCc\msNa\msNc, \uncl{परा गतिः} \msCa, 
पराङ्गतिम् \msCb\msNb, परां गतिः \Ed}}% 

\ujvers\nemsloka {
{\devanagarifont त्रैलोक्यं मणिरत्नपूर्णमखिलं दत्त्वोत्तमे ब्राह्मणे }%
  \dontdisplaylinenum}    \var{{\devanagarifont \numemph\va\textbf{त्रैलोक्यं}\lem \mssCaCbCc\msNa\msNc\Ed, त्रैलोक्य \msNb\oo 
\textbf{अखिलं दत्त्वोत्तमे ब्राह्मणे}\lem \msCb\msCc\msNb\msNc\Ed, 
अ\uncl{खिलं}\lk\lk \lk\lk \lk\lk \lk\ \msCa, अखिलं दत्तोत्तमे ब्राह्मणे \msNa}}% 


\nemslokab

{\devanagarifont कोटीयज्ञसहस्रपद्ममयुतं दत्त्वा महीं दक्षिणाम्  \danda\dontdisplaylinenum }%
     \var{{\devanagarifont \numnoemph\vb\textbf{कोटीयज्ञसहस्रपद्मम्}\lem \msCb\msCc\msNa\msNb\msNc\Ed, \lk\lk \lk\lk \lk\lk \lk\lk \lk\  \msCa\oo 
\textbf{महीं}\lem \msCa\msCb\msNa\msNb\msNc\Ed, मही \msCc}}% 

\nemslokac

{\devanagarifont तीर्थानां च सहस्रकोटिनियुतं स्नात्वा सकृन्मानव }%
  \dontdisplaylinenum    \var{{\devanagarifont \numnoemph\vc\textbf{॰कोटि॰}\lem \mssCaCbCc\msNa\msNb\msNc, ॰कोटी॰ \Ed\ \unmetr\oo 
\textbf{स्नात्वा}\lem \msCa\msCc\msNa\msNb\msNc\Ed, स्ना ऽ \msCb}}% 

%Verse 3:42


\nemslokad

{\devanagarifont एतत्पुण्यफलमहिंसकजनः प्राप्नोति निःसंशयः {॥३:४२॥} \veg\dontdisplaylinenum }%
     \var{{\devanagarifont \numnoemph\vd\textbf{॰फलमहिंस॰}\lem \mssCaCbCc\msNa\msNb\Ed, ॰फलं त्वहिंस॰ \msNc\oo 
\textbf{निःसंशयः}\lem \msCc\msNa\msNb\msNc, \lk\lk \lk\lk\ \msCa, निःसंशय\lk\ \msCb, निःसंशयं \Ed}}% 

\vers


{\devanagarifont 
\jump
\begin{center}
\ketdanda~इति वृषसारसंग्रहे अहिंसाप्रशंसा नामाध्यायस्तृतीयः~\ketdanda
\end{center}
\dontdisplaylinenum\vers  }%
     \var{{\devanagarifont \numnoemph{\englishfont \Colo:}\textbf{नामाध्यायस्तृतीयः}\lem \mssCaCbCc\msNa\msNb, नामाध्यायस्तृतीय \msNc, 
नामस्तृतीयो ऽध्यायः \Ed}}% 
