\documentclass[11pt]{book} % use larger type; default would be 10pt

\usepackage[utf8x]{inputenx}
\usepackage{graphicx} 

\usepackage{hyperref}
\input{../vrsasara_macros.tex}
\begin{document}



\begin{center}
\textsc{\Huge introduction\\ to the\vskip.5em vṛṣasārasaṃgraha}
\end{center}
\vfill
\tableofcontents
\bigskip


\chapter{Summary of the text}

\chapter{The role of the VSS in the Śivadharma corpus and the role of Anarthayaj\~na}

- (a bit like Arjuna)

\chapter{Dating the VSS}
Dating: 

- note the tattva chapter (no tanmātras) 
 
- the archaic yoga of chapter 10 (no Piṅgalā)

- order of asramas?

\chapter{The Language of the VSS}

Special vocabulary/language: 

karhacit, hṛdi as nominative 10.27cd, tirya

Special structures:

caturmaunasya vakṣyāmi

singular next to numerals

me as mayā

a more or less full collation is important: we cannot automatically
reject `ungrammatical' or unmetrical forms because they may well be the
`original' ones

\chapter{Editorial Policies}








\chapter{Vṛṣasārasaṃgraha MSS used so far
(Conspectus Siglorum)}
\chaptermark{MSS}




\parindent0em


\section{Palm-leaf manuscripts in the University Library of Cambridge (ULC)}

{\large\textbf{C$_{\scriptscriptstyle 94}$} 
(N$^{\scriptscriptstyle C}_{\scriptscriptstyle 94}$) = 
ULC Add. 1694}%
     \footnote{\url{https://cudl.lib.cam.ac.uk/view/MS-ADD-01694-00001/381}}
\bigskip

{\large\textbf{C$_{\scriptscriptstyle 45}$} 
(N$^{\scriptscriptstyle C}_{\scriptscriptstyle 45}$) = 
ULC Add. 1645}%
     \footnote{\url{https://cudl.lib.cam.ac.uk/view/MS-ADD-01645/404}}
\bigskip

259 Nepāla / 1139 CE.


{\large\textbf{C$_{\scriptscriptstyle 02}$} 
(N$^{\scriptscriptstyle C}_{\scriptscriptstyle 02}$)  =  
ULC Add. 2102.%
     \footnote{\url{https://cudl.lib.cam.ac.uk/view/MS-ADD-02102/181}}
In this multiple-text manuscript, the \titl{Vṛṣasārasaṃgraha} is trasmitted in an incomplete form, that is to say, 
a number of folios are missing (most notably chapters 15--17). The text starts on a folio which is labelled 237r in 
the online Digital Library of the University of Cambridge (image no.\thinspace 181). This folio in fact has no visible 
foliation and is written in a hand that is clearly different from that of the previous one.
(That folio, image 180, ended with verse 7.122cd of the \titl{Śivopaniṣad}:
\skt{yauvanasthā gṛhasthāś ca [\skt{prāsā}]dasthāś ca ye nṛpāḥ}.)
In image 183 (folio 268r, according to the ULC website), the hand changes back to one that is similar to that 
in image 180. In image 184, the characters for folio number 200+60 are visible (268v, according to the ULC website).
In image 186, the folio number 269 is clearly visible (f.\ 269v). In folio 270v, the continuous text is broken at
verse 2.21c (\skt{kāmarū°}), folios 271 and 272 are missing, and the text resumes on folio 273r with verse 3.30b 
([\skt{ahiṃsā pa}]\skt{ramaṃ sukham}). In folio 296v (image no.\thinspace 234) the text breaks off again at 
\skt{vātaśūlair upadrutā} | \skt{śukro} (verse 14.XX CHECK), the next folio being 306r (\skt{carmatāś ca dvijasundarīṣu} 
(verse 18.XX CHECK) (nine folios and chapters 15--17 are completely missing). 
Again, there two missing folios after \skt{bandhus sarvva°} in
verse 18.XX CHECK in folio 306v. The text resumes in folio 309r (image 237) with \skt{°ṇeṣu ca sarvveṣu vidvān sreṣṭha sa ucyate}
(verse 19.XX CHECK). Another folio is missing between \skt{iṣṭāniṣṭadvaya°} (verse 20.XX CHECK, folio 309v) 
and \skt{snāyu majjā sirā tathā} (verse 20.XX CHECK, folio 311r). The \titl{Vṛṣasārasaṃgraha} ends on folio 322v 
(image no.\thinspace 262) with the concluding colophon \skt{vṛṣasārasaṅgraha samāpta iti}. This folio also contains the beginning
of the \titl{Dharmaputrikā}, but this multiple-text manuscript contains no more folios.

\medskip

{\large\textbf{C} } = 
{\large\textbf{C$_{\scriptscriptstyle 94}$}}
+
{\large\textbf{C$_{\scriptscriptstyle 45}$}} 
+
{\large\textbf{C$_{\scriptscriptstyle 02}$} }







\section{Palm-leaf manuscripts in Kathmandu}

%\msNa
{\large \textbf{K$_{\scriptscriptstyle 82}$}
(N$^{\scriptscriptstyle K}_{\scriptscriptstyle 82}$) =
NAK 3-393, NGMPP A 1082/3
\bigskip


%\msNb
{\large \textbf{K$_{\scriptscriptstyle 10}$}
(N$^{\scriptscriptstyle K}_{\scriptscriptstyle 10}$) =
NGMPP A 10/5, NAK 1-1261. 
The leaves are damaged and slightly disordered. The folio numbers are rarely visible.
The \titl{Vṛṣasārasaṃgraha} starts on exp.\ 44 (upper leaf, no folio number is visible here).
It continues on the lower leaf and then on the upper leaf on exp.\ 43
up to 1.62 (\skt{viṃśakoṭiṣu gulmeṣu ūrdhva°}). 
Verses 1.62cd--2.22 seem to be missing.
The lower leaf on exp.\ 43 contains verses 2.23--2.39.
The single leaf in exp.\ 42 contains verses 2.40--3.16a.
Exp.\ 41 contains a single leaf of the \titl{Umāmaheśvarasaṃvāda},
ending in a colophon for its 22 chapter, 
and the folios preceding continue transmitting the \titl{Umāmaheśvarasaṃvāda}.
Exploring the presence of the \titl{Vṛṣasārasaṃgraha} in 
this manuscript further, one should look at the expositions after no.\ 44.
Exp.\ 45 contains the end of the \titl{Śivopaniṣad}.
The single leaf on exp.\ 46 is almost illegible but most probably contains
a fragment of the \titl{Gautamadharmasūtra}. The second line just above the hole
reads \skt{\dots vīrud vanaspatīnāṃ ca puṣpāṇi svavad ādadīte\dots},
which is a fragment of \titl{Gautamadharmasūtra} 2.3.25 (12.28).
The remaining parts of the \titl{Vṛṣasārasaṃgraha} are to be found 
on exp.\ 47ff. The upper leaf on exp.\ 47 continues with \titl{Vṛṣasārasaṃgraha} 3.16b-36ab,
while the lower leaf contains a text that I have not been able to identify.
The lower leaf in exp.\ 48 transmits 3.36cd--4.11ab, the upper one 4.11b--30a.
The lower leaf in exp.\ 49 contains 4.30ab--47ab, the upper one 47d--68a. And so on so forth.
Thus when reading the text from these images, after exp.\ 48, 
one has to start with the lower leaf and continue with the upper one. 

\bigskip



%\msNc
{\large \textbf{K$_{\scriptscriptstyle 07}$}
(N$^{\scriptscriptstyle K}_{\scriptscriptstyle 07}$) =
NGMPP B 7/3 = A 1082/2, NAK 1-1075

corrections in red in a second hand

%\msNd
{\large \textbf{K$_{\scriptscriptstyle 03}$}
(N$^{\scriptscriptstyle K}_{\scriptscriptstyle 03}$) =
NGMPP A3/3, NAK  5-737. Collated only for the first folio of 
chapter one. The scribe seems to be careless and 
the text seems to be in a rather corrupt state. This manuscript
seems to be related to \msCc, as well as, to a lesser degree,
to \msNa, \msNb\ and
\hbox{\msNc.}%
\footnote{NGMCP catalogue entry: \\
\url{http://catalogue-old.ngmcp.uni-hamburg.de/mediawiki/index.php/A\_3-3\_\%C5\%9Aivadharma}}


\section{The Paris manuscript}
Thanks to...
\textbf{P} Very good quality. Used only for...


\section{Paper manuscripts}

{\large\textbf{L}} (used only for 22.1--35) = Manuscript no.\thinspace 657 
in the Wellcome Institute for the History of Medicine.
Shelved at $\delta$ 16 (vii). 
Listed and its chapter titles given in: Dominik Wujastyk (1985). 
\textit{A Handlist of the Sanskrit and Prakrit Manuscripts in the 
Library of the Wellcome Institute for the History of Medicine}.
Vol.\ 1. London: The Wellcome Institute for the History of Medicine.


\section{Naraharinātha's edition}


{\large \textbf{E$^{\scriptscriptstyle N}$}} =
Naraharinath, Yogin. \textit{Śivadharma Paśupatimatam Śivadharmamahāśāstram Paśupatināthadarśanam}. 
Ed.\ by Yogin Naraharinatha. Kathmandu, saṃvat 2055 (1998 CE). 







\chapter{Critical Edition}

\chapter{Notes on the Constitution of the Sanskrit Text}

\chapter{Annotated Translation}


\end{document}
