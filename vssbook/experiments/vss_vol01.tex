% Created 2021-12-04 szo 12:36
% Intended LaTeX compiler: pdflatex
\documentclass[11pt]{article}
\usepackage[utf8]{inputenc}
\usepackage[T1]{fontenc}
\usepackage{graphicx}
\usepackage{grffile}
\usepackage{longtable}
\usepackage{wrapfig}
\usepackage{rotating}
\usepackage[normalem]{ulem}
\usepackage{amsmath}
\usepackage{textcomp}
\usepackage{amssymb}
\usepackage{capt-of}
\usepackage{hyperref}
\usepackage[utf8x]{inputenx}\input{/home/csaba/indology/dharma_project/vrsa_edition/vrsasara_macros.tex}
\author{Csaba}
\date{\today}
\title{Vṛṣasārasaṃgraha Volume 1: chapters 1--12}
\hypersetup{
 pdfauthor={Csaba},
 pdftitle={Vṛṣasārasaṃgraha Volume 1: chapters 1--12},
 pdfkeywords={},
 pdfsubject={},
 pdfcreator={Emacs 26.3 (Org mode 9.1.9)}, 
 pdflang={English}}
\begin{document}

\maketitle
\tableofcontents



\section{Introduction}
\label{sec:org3570d5f}
\subsection{ERC note}
\label{sec:org84f31c6}
The present publication is a result of the project DHARMA 
`The Domestication of ``Hindu'' Asceticism and the Religious Making of South and Southeast 
Asia'. This project has received funding from the European Research Council (ERC) 
under the European Union's Horizon 2020 research and innovation programme (grant agreement no 809994).

\subsection{Acknowledgements}
\label{sec:orgff6a7a9}
\begin{itemize}
\item NAK
\end{itemize}
Jyoti Neupane, Manita Neupane, Saubhagya Pradhananga (Chief, National Archives), Rubin Shrestha, Sahan Ranjitkar[?] a morc?,
Sushmita Das (VSS MSS in Calcutta)
sribin7501@gmail.com
\begin{itemize}
\item Naples
\end{itemize}
Florinda De Simini, Nirajan Kafle, Kengo Harimoto, Giulia Buriola, Alessandro Battistini,
Lucas Boer, Torsten \ldots{}, Kenji Takahashi, Francesco Sferra, Dorotea \ldots{} , Martina Dello Buono, Chiara \ldots{}
\begin{itemize}
\item Other
\end{itemize}
Judit Törzsök, Dominic Goodall, Harunaga Isaacson, SAS Sarma, Sathyanarayan, Gergely Hidas, my family

\subsection{On the contents of the VSS}
\label{sec:orgaed5074}
\subsubsection{On the title}
\label{sec:org8a3a12b}
ŚDhU:
īśvarāyatanasyādhaḥ śrīmān dharmavṛṣaḥ sthitaḥ ||
yatra vīravṛṣastatra kṣityāṃ gomātaraḥ sthitā || 12.87 ||

filtered\(_{\text{temp}}\)/kosa/10otherkosas:   dharmo vṛṣo vṛṣaḥ śreṣṭho vṛṣo gaurmūṣiko vṛṣaḥ *
filtered\(_{\text{temp}}\)/kosa/10otherkosas-   vṛṣo balaṃ vṛṣaḥ kāmo vṛṣalo vṛṣa ucyate \textbf{* 48 *}

Manusmṛti 8.16a (vṛṣo hi bhagavān dharma). 

visnusmrḍn:ViS 86.15a/ vṛṣo hi bhagavān dharmaś catuṣ-pādaḥ prakīrtitaḥ /

smrti/dharma/krtyaratnākaraḍn: dharmo 'yaṃ vṛṣarūpeṇa nāmnā nandīśavaro vibhuḥ |
smrti/dharma/krtyaratnākaraḍn: dharmān māheśvarān vakṣyaty ataḥ prabhṛti nārada||

tak2015/AtmapujaT55Muktabodha.dn:dharmas tatra vṛṣākāro jñānaḥ sihmasvarūpakaḥ | vairāgyaṃ 

On the title:
MMW 'vṛṣa':  "Justice or Virtue personified as a bull or as "Siva's bull Mn. viii, 16 Pur. Kāvyād.;
just or virtuous act, virtue, moral merit "Siś. Vās.;"

On the title, see De Simini 2016:238 n. 13: ''As noted by Sanderson [\ldots{}], this title can have a double meaning,
since the ‘bull’ (vṛṣa) is both a synonym of ‘religious practice’ and the traditional mount (vāhana) of Śiva.

\subsubsection{The structure of the VSS}
\label{sec:org03981d2}
Matryoshka, dialogues, affiliations
\begin{enumerate}
\item Contents:
\label{sec:org76141ca}
24 chapters

\begin{enumerate}
\item brahmāṇḍasaṃkhyā
\item śivāṇḍasaṃkhyā
\item ahiṃsāpraśaṃsā
\item yamavibhāga
\item śaucācāravidhi
\item yajñavidhi (also lokāḥ)
\item dānapraśaṃsā
\item niyamapraśaṃsā (p. 603: types of svādhyāyana: śaiva, sāṃkhya, purāṇa,
\end{enumerate}
smārta, bhārata)
\begin{enumerate}
\item traiguṇyaviśeṣaṇīya
\item kāyatīrthavivarṇana
\item caturāśramadharmavidhāna
\end{enumerate}
¤12. vipulopākhyāna  (narrative)
On birth:
\begin{enumerate}
\item garbhotpatti (on conception)
\item praśnavyākaraṇa (why people are tall/short etc.)
\end{enumerate}
¤15. jīvanirṇaya 
¤16. adhyātmanirṇaya (yoga) 
\begin{enumerate}
\item dānadharma
\item pūrvakarmavipāka
\item dānayajñaviśeṣa
\item pañcaviṃśatitattvanirṇaya
\item kalpanirṇaya
\item varṇagotrāśrama
\item nidrotpatti
\item śāstravarṇana
\end{enumerate}

\item Summary of the contents of all 24 chapters of the VSS
\label{sec:org3d61159}
\end{enumerate}
\subsubsection{References to other works}
\label{sec:org97c0b77}
Mahābhārata
nakule
vipule
etc.  
\subsection{The role of the VSS in the Śivadharma corpus}
\label{sec:orgd02f304}
\subsubsection{general ideas}
\label{sec:orgde0ae63}
\begin{itemize}
\item is this text really Śaiva? why in this collection?
\item niśvāsa as sadāśiva in ch. 16; Niśvāsa uttarasūtra 5.50-51; see also Kafle Niśvāsamukha p.11ff; ibid. p.12: "The term niśvāsa             means sighing. Thus, an alternative
meaning of the Niśvāsatattvasaṃhitā could also be a ‘‘sighing tantra.’’ To be more precise,
a tantra that originated from the sighing of Śiva. This is to say, the speech of Śiva."
\item tattva-system: mati and suśira (ch. 20)
\item parallels: MBh, Bṛhatkālottara,
\item ch. 21: Viṣṇu; is this a Śaiva text?
\item āśramas are in an order different from usual; compare this to NĀT; "Variations on the āśrama-system"
\end{itemize}

History of Dharmasastra 2.1
pp. 416ff on āśramas

n. 988! see Āpastamba-dharma-sūtra ii.9.21.1: catvāra āśramā gārhasthyam ācāryakulaṃ maunaṃ vānaprasthyam iti| Quoted by Śankara
But the chapters in Āpastamba follow the traditional order.
"Āp. places the householder first among the āśramas, probably on account of the importance of that stage to all other āśramas." Kane ibid.

ibid p. 417: person in last āśrama is called: parivrāṭ, parivrājaka(!), bhikṣu, muni, yati.
See 
Olivelle, Patrick. The Āśrama System. The History and Hermeneutics of a Religious Institution. New York, Oxford: Oxford University Press, 1993.  [megvan] p.82ff: The Order of Āśramas; 
	ibid: "In later texts the usual order is student, householder, hermit, and renouncer, reflecting the sequence of the passage from one \uline{āśrama} to another\ldots{} In the Dharmasūtras, however, only Baudhāyana and Vasiṣṭha follow that order\ldots{} A specific order becomes insignificant when the \uline{āśramas} are taken as four alternative adult vocations." 
Are they alternative adult vocations here in the Vṛṣasārasaṃgraha? They are numbered.

\textit\{Gṛhastha. The Householder in Ancient Indian 
Religious Culture.\} Edited by Patrick Olivelle. OUP, 2019.
Especially Csaba Dezső's article in it.

\%\%\%\%\%\%
\%dscn 8034.jpg ff in folder \emph{home/csaba/mmedia/images/scan/saiva/sivadharmacorpus/pasupatimatam4}
\% in Naraharinātha's Paśupatimatam pp. 580ff
\% CHECK if Naraharinath seems to be better at Sanskrit in other texts
\% the edition seems problematic at many places
\% a dialogue between Janamejaya and Vaiśampāyana, the latter of whom relates
dialogues between Vigatarāga and Anarthayajña
\% revise ¤s and lost/ill
Bisschop in "Universal Śaivism": " -- En-dashes indicate a lost or illegible syllable in the manuscript."

\%N. of a celebrated king to whom Vaiśampāyana recited the [MBh.] (greatgrandson to Arjuna, as being son and, successor to Parikshit who was the son of Arjuna's son Abhimanyu) ["SBr.] xi, xīi AitBr. "Sāṅkhir. xvi [MBh.] \&c.;

Bisschop 2018:2:
	``The full text of the corpus was first published by Naraharinātha in 1998, while over the past few years several scholars have started to work on individual parts of the corpus or referred to them in their studies. See, in particular, Acharya 2009; Bisschop 2010, 2014; De Simini 2013, 2016a, 2016b, 2017; De Simini \& Mirnig 2017; Goodall 2011; Kafle 2013, 2015; Magnone 2005; Sanderson 2003/04, 2012/13; Schwartz 2012. An edition of the Śivadharmaśāstra alone, based on a single manuscript in the Adyar Library, has been published more recently as well (Jugnu \& Sharma 2014). The Śivopaniṣad, which also forms part of the Śivadharma corpus, was already published much earlier but was not recognised as such, being included in a collection of Upaniṣads (Kunhan Raja 1933).''

What MS did Naraharinātha used? See Biscchop 2018:58--59.

Palm leaf: /home/csaba/mmedia/images/scan/saiva/sivadharmacorpus/mss\(_{\text{florinda}}\)/newari/ngmpp/palm\(_{\text{leaf}}\)\(_{\text{mtm}}\)/A 3:3/fr.8493.0.A 0003-03\(_{\text{3}}\)/A3-03+65851+177\(_{\text{vss}}\)\(_{\text{start.jpg}}\)
Paper MS /home/csaba/mmedia/images/scan/saiva/sivadharmacorpus/mss\(_{\text{florinda}}\)/newari/ngmpp/paper\(_{\text{mtm}}\)/A 1341-06/DSCN0331 fol. 204\(_{\text{vss.JPG}}\)
\subsubsection{Vipula}
\label{sec:org830304a}
Vipula in the MBh:

MBh 13040016aff

Devaśarman and his wife Ruci
13040017a tasya rūpeṇa --> 13040017a tasyā rūpeṇa

all gods, esp. Indra, are in love with her
but Devaśarman guards her
wants to perform yajña: how to guard her during the ritual?
calls his pupil, Vipula
tells him that Indra can assume various forms
Vipula decides that the only way to protect her from Indra is to magically 'enter' her (with yoga)
he tells her stories and enters her 

MBh 13041001ff
Indra sees the opportunity and enters the āśrama as a beautiful man
he sees Vipula's lifeless body
Ruci fancies Indra, but Vipula in his body stops her from standing up
Indra sings to her beautiful songs
he says "I have come for you, I am Devendra, I am in love"
Vipula stops her from doing anything
Indra is a bit shocked by her not being moved, gets angry and can see now that Vipula is in her
Vipula leaves her, enters his own body, and abuses Indra and tells Indra how wicked he is
Indra is ashamed and disappears
Devaśarman returns to the āśrama, Vipula tells him what happened and Devaśarman praises him

ETC., see translation here:
\url{https://www.sacred-texts.com/hin/m13/m13b005.htm}

See summary also here:
V. S. Sukthankar. Critical Studies in the Mahābhārata.
	Poona, V. S. Sukthankar Memorial Edition Committee, 1944. 317--318
\url{https://archive.org/details/in.ernet.dli.2015.281344/page/n333}

\subsection{Dating and provenance}
\label{sec:org06fa85b}
\begin{itemize}
\item note the tattva chapter 20 (no tanmātras)
\item the archaic yoga of chapter 10 (no Piṅgalā)
\item check lists of deities such as Vasus
\item order of asramas?
\end{itemize}
\begin{enumerate}
\item Place of composition: geographical names and persons mentioned
\label{sec:org8e5e470}
\end{enumerate}

\subsection{Interpretation of chapters}
\label{sec:org93b9b29}
\subsubsection{Chapter 12}
\label{sec:orgf298b3e}
everybody is donating to everybody, 
the final donor is Brahmā
lot of testing going on in the frame story and also
in chapter 12
also the disguise thing is recurring: 12.37 and ch 1 and
when Viṣṇu reveals his identity
\subsection{Misc}
\label{sec:orgdf4797a}
\subsubsection{susūkṣma:}
\label{sec:orgd27f794}
Śivadharmottara 10.45cd--46: rudraḥ ṣaḍviṃśakaḥ proktaḥ śivaś ca paratas tataḥ || 45 || saptaviṃśatimaḥ śāntaḥ susūkṣmaḥ parameśvaraḥ | svargāpavargayor dātā taṃ vijñāya vimucyate || 46 ||.
yamas-niyamas: see table in Śaiva Utopia p17
\subsubsection{other}
\label{sec:org0efcb93}
Why is this mentioned at
\url{http://cudlḷib.cam.ac.uk/view/MS-ADD-01694-00001/403} :
C., Kunhan Raja, Un-published Upanishads (Adyar: The Adyar Library, 1933).
Ahhh, Śivopaniṣat is in there!
cf. śivasaṃkalpa in pp 319 ff. (Śivasaṃkalpopaniṣat)
Bonazzoli, Giorgio, "Introducing Śivadharma and Śivadharmottara", Altorientalische Forschungen vol. 20 issue. 2 pp. 342-349 (1993).
"There is no raw data." EdX Harvard Digital Humanities

CHECK out Kenji on the Umāmaheśvarasaṃvāda in the MBh, his summary looks similar to the VSS

Kenji:
''BDhS 2: Discussion of gṛhastha. but BDh 2.11.9--34 is a digression on the topic of caturāśrama (vikalpa
type, not krama type), and the author denies caturāśrama idea.''

MSS: see Bisschop Universal \ldots{} pp. 52--53; De Simini \& Mirnig pp. 587, 591
\% ``a stable element of the corpus''

Vindicate your edition: look at the apparatus, all the \Ed entries
\subsection{Notes on language}
\label{sec:org0db1d18}
\begin{itemize}
\item Special vocabulary/language: karhacit, hṛdi as nominative 10.27cd, tirya, me as mayā
\item Special structures: 
caturmaunasya vakṣyāmi
indreṇāsmi phalaṃ dattaṃ
\item Number: singular next to numerals, and general confusion (CHECK)
\end{itemize}

'Muta cum liquida' (Balogh 2018?:33 note 6) find sources on metrical licenses / -ces
Apte Appendix A p. 1: pra, bra, hra, kra are exceptions
In the text below, śra, śya, śva, sva, dva seem additional ones (plus tra and vra? and rp? 11.5X)

''According to Kedārabhaṭṭa in Vṛttaratnākara, a short vowel followed by a consonant cluster 
(i.e., multiple consonants), although guru by default, can optionally be treated as laghu if
that consonant cluster happens to be the beginning of a new word. Implement this option.
(With required UI changes, e.g. saying: "this verse can be read as \ldots{} if an option is used in places x, y, z.") ''
SOURCE: ( \url{https://github.com/shreevatsa/sanskrit/issues/1} )

Vṛttaratnākara 1.10! (etext downloaded):
padādāv iha varṇasya saṃyogaḥ kramasaṃjñikaḥ /
puraḥsthitena tena syāl laghutā 'pi kvacid guroḥ \emph{/ KVrk\(_{\text{1.10}}\) /}

SOURCE: ( \url{https://github.com/shreevatsa/sanskrit/issues/1} ):
``The traditional rule is that the presence of a conjunct consonant (consonant cluster) makes
  the preceding syllable a guru syllable. For purists, this is an inviolable rule, and there are no exceptions.

However, under the influence of Prākṛta and deśa-bhāṣā prosody, and also music, 
later prosodists give poets the option to either conform to this rule, or to occasionally indulge in an exception.
This exception is stated differently by different authors:

In Kannada and Telugu prosody, under the name of śithila-dvitva, the exception is that sometimes in
a consonant cluster of the form [consonant + "r"], the "r" (repha) can be ignored, so that it is not a conjunct consonant anymore.
Kedāra-bhaṭṭa in his Vṛtta-ratnākara states the exception as: if the consonant cluster is at the
beginning of a word, then it may be treated as a single consonant.
Yet another way of stating the exception is that a consonant cluster can be optionally 
treated as a single consonant if the effort of pronouncing it is quick 

or with less effort

All ways of stating the exception cover the example of

but Kedārabhaṭṭa's way doesn't cover the example of "राज्ञां मध्ये सपदि जह्रिषे मित्रविन्दामवन्तीम्" from the Nārāyaṇīyam. So far I don't see a reason to prefer Kedārabhaṭṭa's formulation over the others.

But anyway, to return to the main point:

The exception is not accepted by purists: Shatavadhani Ganesh says that the Sanskrit masters like
Kālidāsa, Bhāravi, Māgha, Śrīharṣa, and Viśākhadatta have not freely used this exception (though the masters in Kannada
and other languages have).
Being more of a "poetic licence" and a violation of the standard rule (only found in later poetry),
it is extremely unlikely that any sane poet would have indulged in that exception in all four pāda-s of a verse.
Thus it is very unlikely that the program will miss identifying a verse that indulges in this exception.
At least, I haven't seen any example from real life so far. (Just saying that because
it would be easy to intentionally compose a verse to disproves this!)

   Reference: see comments by Dr. Ganesh and Nityananda Misra in this thread started by Vishvas Vasuki:
   \url{https://groups.google.com/forum/\#!topic/bvparishat/ya1cGLuhc14/discussion}
''

Nirajan's advice was to look into Chandomañjarī, and here you are:
p. 2: 
atra pādāntago laghur gurur bhaved| yathā| 
taruṇaṃ va?rṣapaśākaṃ navaudanaṃ picchilāni ca dadhīni?|
alpavyayena sundari grāmyajano miṣṭam aśnāti||
sundarīti grāmyaśabde pare [because it is followed by the word grāmya] vikalpena [option] laghutvam|
tathā bhaṭṭiḥ| 
atha 
 . . . . -  . - -   -  - . . - . - . . . - -
lulitapatatrimālaṃ rugṇāśanavāṇakesaratamālam|
.   . .   . -  . - -  -  -    -! -   . - - -
sa vanaṃ viviktamālaṃ sītāṃ draṣṭuṃ jagāmālam|
atra prathamapādāntaguror laghutvam| tathā matpituḥ pārijātaharaṇanāṭake| etc.
OK, it is also in the Vṛttaratnākara, check it there
\%\%\%\%\%\%\%\%\%\%\%\%\%\%\%\%\%\%\%\%\%\%\%\%\%\%\%\%\%\%\%\%\%\%\%\%\%\%\%\%\%\%\%\%\%\%\%\%\%\%\%\%\%\%\%\%\%\%\%
MBh on the 4-legged bull:
03188010c vṛṣaḥ pratiṣṭhito dharmo manuṣyeṣv abhavat purā
03188011a adharmapādaviddhas tu tribhir aṃśaiḥ pratiṣṭhitaḥ
03188011c tretāyāṃ dvāpare 'rdhena vyāmiśro dharma ucyate
03188012a tribhir aṃśair adharmas tu lokān ākramya tiṣṭhati
03188012c caturthāṃśena dharmas tu manuṣyān upatiṣṭhati
03188013a āyur vīryam atho buddhir balaṃ tejaś ca pāṇḍava
03188013c manuṣyāṇām anuyugaṃ hrasatīti nibodha me
\%\%\%\%\%\%\%\%\%\%\%\%\%\%\%\%\%\%\%\%\%\%\%\%\%\%\%\%\%\%\%\%\%\%\%\%\%\%\%\%\%\%\%\%\%\%\%\%\%\%\%\%\%\%\%\%\%\%\%

\begin{itemize}
\item How to deal with the problem of info distributed in two volumes?
\item metre
\begin{itemize}
\item muta cum liquida
\item final -am etc. counts as long (reverse of muta cum liquida)
\end{itemize}
\item stem form nouns (prātipadika)
\item Special vocabulary/language: 
karhacit, hṛdi as nominative 10.27cd, tirya
\item Special structures:
caturmaunasya vakṣyāmi
\item number (e.g. singular next to numerals)
\item gender
\item me as mayā
\item a more or less full collation is important: we cannot automatically 
reject `ungrammatical' or unmetrical forms because they may well be the
`original' one
\end{itemize}
\subsection{Manuscripts consulted}
\label{sec:org81773e7}
In the manuscript descriptions below, in addition to some general remarks,
I will mainly focus on information relevant to the VSS. For much more
detail on the overall features of these manuscripts, see De Simini 2016 
and the catalogues I mention at each individual manuscript.

I owe thanks Florinda De Simini for sharing with me most of the manuscripts
listed here, to Kengo Harimoto and Gudrun Melzer (Munich) for providing
photos of the Munich MS, and to Nirajan Kafle for sharing the Paris MS. 
\subsubsection{The Cambridge MSS}
\label{sec:org333b616}
\begin{enumerate}
\item \msCa (NC94): University Library of Cambridge Add. 1694.1.
\label{sec:orgaf70b57}
I used this manuscript extensively in the critical edition. 
See a detailed description of this MS at
	\url{https://cudl.lib.cam.ac.uk/view/MS-ADD-01694-00001/382}
Date of creation: 12th century.

The script is Newari/Nepālākṣara.

It is a palm-leaf MTM containing 258 folios.
Contents:
\begin{enumerate}
\item Śivadharmaśāstra
\item Śivadharmottara
\item Śivadharmasaṃgraha
\item Umāmaheśvarasaṃvāda
\item Uttarottaramahāsaṃvāda
\item Vṛṣasārasaṃgraha
\item Dharmaputrikā
\item Śivopaniṣad
\end{enumerate}
The VSS occupies 45 folios: 
	it starts on f. 193 (the recto side, online image no. 381,      
	is an empty folio side, the text itself starts on the verso side);
	it ends on f. 239r (and not 193r, as the online description says; online image no. 472).
	The text of the VSS is transmitted fully, without any folios or major sections of the text missing.
The leaves transmitting the VSS are well-preserved. Some folio
      sides are faded and most folios are somewhat damaged on the right side, 
      sometimes at other parts, and it seems from the images that some opaque-looking tape has been
      applied to protect these damaged sections. In my critical edition the broken off, completely lost,
      /akṣara/s are represented by ×, the illegible /akṣara/s under the tape by ¤. CHECK
The quality of the readings of this manuscript is one of the best,
comparable only to \msNa$\backslash$ and \msP, making it one of the most
important sources for the VSS.

\item \msCb (NC45): University Library of Cambridge Add. 1645.
\label{sec:org9de9905}
I used this manuscript extensively in the critical edition. 
See a detailed description of this MS at
      \url{https://cudl.lib.cam.ac.uk/view/MS-ADD-01645/404} 
Dated to Nepali Samvat 259 (1138--39 CE).

The script is Newari/Nepālākṣara.
It is a palm-leaf MTM containing 247 folios.
Contents: 
\begin{enumerate}
\item Śivadharmaśāstra
\item Śivadharmottara
\item Śivadharmasaṃgraha
\item Śivopaniṣad
\item Umāmaheśvarasaṃvāda
\item Uttarottaramahāsaṃvāda
\item Vṛṣasārasaṃgraha
\item Dharmaputrikā
\end{enumerate}
The VSS occupies 37 folios plus one folio side: 
	it starts on f. 201v line 4 (online image no. 404),
	it ends on f. 238v line 3 (onine image no. 478).
The readings of this manuscript seem to follow those of \msNa
      remarkably closely in the Śivadharmottara (as observed by De Simini
      and Harimoto.[fn. Personal communication 01-12-2021]
      This is more difficult to see in the VSS, but indeed, they
      closely related. CHECK

\item \msCc (NC02) CHECK zero; Cambridge University Library Add. 2102. Palm-leaf, 96 folios.
\label{sec:org6d5d072}
I used this manuscript extensively in the critical edition. 

The script is Newari/Nepālākṣara.
Contents: 
\begin{enumerate}
\item Śivadharmottara
\item Śivadharmasaṃgraha
\item Umāmaheśvarasaṃvāda
\item Śivopaniṣad
\item Vṛṣasārasaṃgraha
\item Dharmaputrikā (only fol. 322v).
\end{enumerate}
For a general description of this manuscript, see the online record on the Cambridge Digital Library website:
      \url{https://cudl.lib.cam.ac.uk/view/MS-ADD-02102/181}.
      f. 237r
      Remark: the Vṛṣasārasaṃgraha seems to start in a different hand, but then it changes.
      In this MTM, the VSS is trasmitted in an incomplete form, that is to say,
	   a number of folios are missing (most notably chapters 15--17).
      The text starts on a folio which is labelled 237r in the online Digital Library of 
	   the University of Cambridge (image no. 181). 
       This folio in fact has no visible foliation and is written in a hand that is clearly different 
	   from that of the previous one.
     (That folio, image no. 180, ended with verse 7.122cd of the \emph{Śivopaniṣad}:
	   \emph{yauvanasthā gṛhasthāś ca} [/prāsā/]/dasthāś ca ye nṛpāḥ/.)
     In image 183 (folio 268r, according to the ULC website), the hand changes back to one that is similar to that
	   in image 180. 
     In image 184, the characters for folio number 200+60 are visible (268v, according to the ULC website).
     In image 186, the folio number 269 is clearly visible (f. 269v). 
     In folio 270v, the continuous text is broken at verse 2.21c (\emph{kāmarū°}), 
	   folios 271 and 272 are missing, and the text resumes on folio 273r with verse 3.30b
	   (\emph{ahiṃsā pa/]/ramaṃ sukham}). 
     In folio 296v (image no. 234) the text breaks off again at
	   \emph{vātaśūlair upadrutā} | \emph{śukro} (verse 14.XX CHECK), the
	   next folio being 306r (\emph{carmatāś ca dvijasundarīṣu}\} (verse 18.XX CHECK) 
	   (nine folios and chapters 15--17 are completely missing).
     Again, there are two missing folios after \skt{bandhus sarvva°} in
	   verse 18.XX CHECK in folio 306v. 
     The text resumes in folio 309r (image 237) with \emph{°ṇeṣu ca sarvveṣu vidvān sreṣṭha sa ucyate}
	   (verse 19.XX CHECK). 
     Another folio is missing between \emph{iṣṭāniṣṭadvaya°} (verse 20.XX CHECK, folio 309v)
	   and \emph{snāyu majjā sirā tathā} (verse 20.XX CHECK, folio 311r). 
     The VSS ends on folio 322v (image no. 262) with the concluding
	   colophon \emph{vṛṣasārasaṅgraha samāpta iti}.
     This folio also contains the beginning of the \emph{Dharmaputrikā}, but this multiple-text manuscript contains no more folios.
\end{enumerate}

\subsubsection{The Kathmandu MSS}
\label{sec:org25dc199}
\begin{enumerate}
\item \msNa (NK82): NGMPP A 1082/3, NAK 3-393
\label{sec:org5500a28}
Palm-leaf, dated to NS 189 (1068--69 CE), 274 folios.
I used this manuscript extensively in the critical edition. 
See a brief description of this MS at
     \url{https://catalogue.ngmcp.uni-hamburg.de/receive/aaingmcp\_ngmcpdocument\_00098499}

The script is Newari/Nepālākṣara.
It is a palm-leaf MTM containing 274 folios.
Contents:
\begin{enumerate}
\item Śivadharmaśāstra
\item Śivadharmottara
\item Śivadharmasaṃgraha
\item Umāmaheśvarasaṃvāda
\item Śivopaniṣad
\item Vṛṣasārasaṃgraha
\item Dharmaputrikā
\item Uttarottaramahāsaṃvāda
\end{enumerate}
The foliation for the VSS restarts from 1v (f. 1r is a cover) and
the text spans fols. 1v--46r. 
This a beautifully written and well-preserved manuscript which give
very useful readings and proved to be essential for the
reconstruction of the VSS.
\item \msNb (NK10): NGMPP A 10/5, NAK 1-1261
\label{sec:org3d1adc0}
Palm-leaf.
See a brief description of this MS at:
      \url{https://catalogue.ngmcp.uni-hamburg.de/receive/aaingmcp\_ngmcpdocument\_00085264}
I used this manuscript extensively in the critical edition. 
It is a MTM containing 74 folios.
Contents: 
\begin{enumerate}
\item Śivadharmottara
\item Umāmaheśvarasaṃvāda
\item Śivopaniṣad
\item Vṛṣasārasaṃgraha
\end{enumerate}
Some folios feature monochrome drawings.
A great number of the leaves that transmit the VSS
are damaged, faded and slightly disordered. The folio numbers are rarely visible.
The VSS starts on exp. 44 (upper leaf, no folio number is visible here).
It continues on the lower leaf and then on the upper leaf on exp. 43
      (going backwards) up to 1.62 (\emph{viṃśakoṭiṣu gulmeṣu ūrdhva°}).
Verses 1.62cd--2.22 seem to be missing.
The lower leaf on exp. 43 contains verses 2.23--2.39.
The single leaf in exp. 42 contains verses 2.40--3.16a.
Exp. 41 contains a single leaf of the \emph{Umāmaheśvarasaṃvāda},
      ending in a colophon for its chapter twenty-two,
      and still going backwards, the preceding folios continue transmitting the \emph{Umāmaheśvarasaṃvāda}.
Exploring the presence of the VSS in
      this manuscript further, one should look at the expositions after no. 44.
Exp. 45 contains the end of the \emph{Śivopaniṣad}.
The single leaf on exp. 46 is almost illegible but most probably contains
      a fragment of the \emph{Gautamadharmasūtra}. 
The second line just above the string hole on the left reads 
      \emph{\ldots{} vīrud vanaspatīnāṃ ca puṣpāṇi svavad ādadīte\ldots{}},
      which is a fragment of \emph{Gautamadharmasūtra} 2.3.25 (12.28).
The remaining parts of the VSS are to be found on exp. 47ff. 
The upper leaf on exp. 47 continues with VSS 3.16b-36ab,
      while the lower leaf contains a text that I have not been able to identify.
The lower leaf in exp. 48 transmits 3.36cd--4.11ab, the upper one 4.11b--30a.
The lower leaf in exp. 49 contains 4.30ab--47ab, the upper one 47d--68a.
And so on so forth.
Thus when reading the text from these images, after exp. 48,
      one has to start with the lower leaf and continue with the upper one.

\item \msNc (NK07): NGMPP B 7/3 = A 1082/2, NAK 1-1075
\label{sec:org50cb4b4}
See a brief description of this MS at:
  \url{https://catalogue.ngmcp.uni-hamburg.de/receive/aaingmcp\_ngmcpdocument\_00062373}
Palm-leaf, dated to NS 290 (1169--70 CE), 289 folios.
Contents:
\begin{enumerate}
\item Śivadharmaśāstra
\item Śivadharmottara
\item Śivadharmasaṃgraha
\item Umāmaheśvarasaṃvāda
\item Śivopaniṣad
\item Vṛṣasārasaṃgraha
\item Uttarottaramahāsaṃvāda
\item Dharmaputrikā
\end{enumerate}
Fol. 209v--264v contain the VSS.

\item \msNd (NK03): NGMPP A 3/3 (= A 1081/5), NAK 5-737
\label{sec:org0227172}
Palm-leaf, dated to NS 321 (1200--01 CE), 215 folios.
Contents: 
\begin{enumerate}
\item Śivadharmaśāstra
\item Śivadharmottara
\item Śivadharmasaṃgraha missing (only a few folios extant, e.g. ff. 124 and 143)
\item Umāmaheśvarasaṃvāda
\item Śivopaniṣad
\item Uttarottaramahāsaṃvāda
\item Vṛṣasārasaṅgraha (fols. 227v--264v)
\item Dharmaputrikā
\end{enumerate}
         For a brief  description of this manuscript, see the record in the NGMCP online catalogue: 
\url{http://catalogue.ngmcp.uni-hamburg.de/wiki/A\_3-3(1)\_Śivadharma}.
VSS starts on 177.jpg

\begin{enumerate}
\item -- NAK 5--738 (NGMPP A 11/3): Palm-leaf, dated to NS 516 (1395--96 CE), 253 folios.  Contents: 
Śivadharmaśāstra (fols. 1v--43r); Śivadharmottara (fols. 4v--95r); Śivadharmasaṃgraha (fols. 96v--139v); 
Umāmaheśvarasaṃvāda (fols. 140v-- 171r); Śivopaniṣad (fols. 172v--189r); Uttarottaramahāsaṃvāda (fols. 
190v-- 211v); Vṛṣasārasaṃgraha (fols. 212v--257v). For a description of this manuscript, also see the 
record in the NGMCP online catalogue: \url{http://cata- logue.ngmcp.uni-hamburg.de/wiki/A\_11-3\_Śivadharmottara}.

\item -- NAK 4--1604 (NGMPP A 1365/3). Paper, 90 folios. Contents: Śivopaniṣad (fols.
166v--184r); Uttarottaramahāsaṃvāda (fols. 185v--210r); Vṛṣasārasaṃgraha
(fols. 211v--255r). For a description of this manuscript, see the record in the
NGMCP online catalogue: \url{http://catalogue.ngmcp.uni-hamburg.de/wiki/A\_1365-3(1)\_Śivopaniṣad}  ASK

\item -- NAK 4--2537 (NGMPP B 219/3). Paper, 339 folios. Contents: Śivadharmaśāstra (fols. 1v--58r); 
Śivadharmottara (fols. 59v--123v); Śivadharmasaṃgraha (fols.  124v--161v); Umāmaheśvarasaṃvāda (fols. 
162v--238v); Vṛṣasārasaṃgraha (fols. 239v--338v). GOTIT

\item -- NAK 4--93 (NGMPP A 1341/6). Paper, 82 folios. Contents: Śivadharmasaṃgraha (fols. 91r¤--135v); 
Vṛṣasārasaṃgraha (fols. 204r¤--243v). GOTIT
\end{enumerate}


\item Kesar 218 BLURRED (NGMPP C 25/1). Palm-leaf, 298 folios. Contents: Śivadharmaśāstra (fols. 1v--57r);
\label{sec:org8a2ae20}
Śivadharmottara (fols. 57v--134v); Śivadharmasaṃgraha (fols.  135r--215v); Umāmaheśvarasaṃvāda (fols. 
216v--255r); Śivopaniṣad (fols. 256v--278r); Umottara°/ Uttarottaramahāsaṃvāda (fols. 279v--299v¤); 
Vṛṣasārasaṃgraha (?¤--?¤); (?--?¤).

\item Kesar 537 (NGMPP C 107/7). Paper, dated to NS 803 (1682--83 CE), 174 folios.  Contents:
\label{sec:org29bc9ef}
Śivadharmasaṃgraha (fols. 89r--133v); Umāmaheśvarasaṃvāda (fols. 134r--163v); Śivopaniṣad (fols. 
164r--181r); Uttarottaramahāsaṃvāda (fols. 182r--206v); Vṛṣasārasaṃgraha (fols. 207r--251v); Dharmaputrikā 
(fols. 252r--262v).

\item Kesar 597 (NGMPP C 57/5). Paper, dated to NS 863 (1742--43 CE), 257 folios.  Contents:
\label{sec:orgef67c2c}
Śivadharmaśāstra (fols. 1v--41v); Śivadharmottara (fols. 42v--92r); Śivadharmasaṃgraha (fols. 93v--138v); 
Umāmaheśvarasaṃvāda (fols. 139v-- 170v); Śivopaniṣad (fols. 171v--188r); Uttarottaramahāsaṃvāda (fols. 
189v-- 213r); Vṛṣasārasaṃgraha (fols. 214v--257r).
\end{enumerate}

\subsubsection{The Paris MS}
\label{sec:org7a56046}
Nirajan using this MS:
The Manuscript NP 57

It is a multiple-text palm-leaf manuscript written in Newari script and preserved in de la collection Sylvain Lévi à
l’Institut d’études indiennes, collège de France. The manuscript number is: MS. Skt. 57-B. 23. It contains 249
palm leaves, each folio containing six lines. The following palm leaves are missing: 3, 8, 47, 48, 135, 197, 214 et
\begin{enumerate}
\item Foliation is in the verso: on middle of the lef-hand margin in combination of Newari letters and in the
\end{enumerate}
middle of right-hand in roman numbers by a second hand. There are two binding holes: one in the centre lef and
one in the centre right. The manuscript is some times damaged in margins with considerable loos of the text. The
text is written in a clear hand and contains few mistakes. Although it is undated manuscript, it could be dated to
the 11th century AD on the palaeographical grounds.3 It contains the following text in the order they are
presented in the manuscript: Śivadharmaśāstra (fols. 1--40), Śivadharmottaraśāstra (fols. 40--93),
Śivadharmasaṃgraha (fols. 94--142), Umāmaheśvarasaṃvāda (fols. 143--172), Śivopaniṣad (fols. 173-- 189),
Uttarottaramahāsaṃvāda (fols. 190--211), Vṛṣasārasaṃgraha (fols. 212--252), Dharmaputrikā (fols. 253--262). This
source contains reliable readings and contains few scribble mistakes.Śivāśramā- dhyāya covers fols. 33v4--37r3.
Nirajan says it reads close to Naraharinātha's edition

\subsubsection{The Munich MS}
\label{sec:orgb6d199b}
Kengo got it in Munich on 16 Nov 2021. VSS starts in 411.jpg
'cover' [411.jpg]: ||w|| vṛṣasārasaṃgraha 50 patra ||w||
Text starts in 412.jpeg, f.1r 
Ends on image 455.jpeg
Has interesting readings, but mostly very corrupt and useless.
Hand different from that of some of the other texts in this bundle.
Collated chapter 1, will probably not go on. Gives number of verses in colophons

\emph{\msM$\backslash$ 412.jpg, f. 1r start; ten folios are missing:}
\begin{itemize}
\item f. 5 (VSS 3.4-3.33)*
\item ff. 11-13 (VSS 6.20-8.45)
\item ff. 24 (VSS 13.9-13.36)
\item ff. 39-43 (VSS 20.38-22.35)
\textbf{416.jpg lower image is Dharmaputrikā 4.22-39); 
 417.jpg upper is Dharmaputrikā 4.39-55}
\end{itemize}
Kengo writes: ``411.jpg forms a cover that says vṛṣasārasaṅgraha
			but it is actually 50 verso''
samvat 282? [that would be 1161 CE, or is it 292? = 1171 CE] 
No, maybe 192! see Kengo's notes! = 1070 CE
\subsubsection{The Oxford MS}
\label{sec:orgcf5dac5}
 Bodl. Or. B 125[? Sansk. a. 15]. Palm-leaf, dated to NS 307 (1186--87 CE), 335 folios. Contents: Śivadharmaśāstra 
(fols. 1v 1--15v1 / 12r--49v); Śivadharmottara (fols. 50v--113v); Śivadharmasaṃgraha (fols. 114v--159v); 
Umāmaheśvarasaṃvāda (fols. 160v--197v); Śivopaniṣad (fols. 198v--219v); Uttarottaramahāsaṃvāda (fols. 
220v--247r); Vṛṣasārasaṃgraha (fols. 248v--299r); Dharmaputrikā (fols. 300v--312r). 

\subsubsection{Kolkata TO BE OBTAINED}
\label{sec:org32ef672}
\begin{enumerate}
\item The Asiatic Society, Kolkata, G. 4076 (only the Vṛṣasārasaṃgraha, but once part of a larger corpus)
\item The Asiatic Society, Kolkata, G. 3852 (Śivadharma corpus)
\item The Asiatic Society, Kolkata, G 4077. Palm leaf, Newari script, dated [Nepāla] Saṃvat 156 (1035--36 CE). 52 folios
\end{enumerate}

\subsubsection{The Oxford MS}
\label{sec:orgc0b038c}
Bodl. [Or. B 125? cancelled] Sansk. a. 15; GOTIT

\subsubsection{The London MS}
\label{sec:org7cc2162}
-- WI δ 16 (I--VIII). Paper, 406 folios. Contents: Śivadharmaśāstra (serial no. 634), fols. 1v--63r; 
Śivadharmottara (s. no. 635), fols. 64r--143v; Śivadharmasaṃgraha (s. no. 633), fols. 144r--217v; 
Umāmaheśvarasaṃvāda (s. no. 652), fols. 218v-- 263v; Śivopaniṣad (s. no. 636), fols. 264r--297v; 
Uttarottarama-hāsaṃvāda (s. no.  654), fols. 298r--324r; Vṛṣasārasaṃgraha (s. no. 657), fols. 325r--390r; 
Dharmapu- trikā (s. no. 608), fols. 391r--406r. Described in: Dominik Wujastyk (1985). A Handlist of the 
Sanskrit and Prakrit Manuscripts in the Library of the Wellcome Institute for the History of Medicine, 
vol. 1. London, The Wellcome Institute for the History of Medicine.
DIRECT COPY of CHECK A82

\subsubsection{The Edition}
\label{sec:org96d4df5}
\begin{itemize}
\item \Ed Naraharinath's edition
\end{itemize}

\subsubsection{CHECK remaining ones}
\label{sec:orgea9f34f}
De Simini 2016:240 n. 19 Śivadharma MSS:
\begin{enumerate}
\item ASC G 3852 (cat. no. 4085); GOTIT
\item ASC G 4077 (cat. no. 4084); GOTIT
\item Kesar 218 (NGMPP C 25/1) GOTIT
\item NAK 1--1261 (A 10/5); GOTIT
National Archives, Kathmandu, 3/393, 274 folios. Microfilmed by the NGMPP, A 1082/3. 
115 Palm leaf, Newari script, dated [Nepāla] Saṃvat 189 (1069 CE): Florinda sent it to me!
\item NAK 5--737 (NGMPP A 3/3=A 1081/5); GOTIT
\item NAK 5--738 (NGMPP A 11/3); GOTIT
\end{enumerate}
10  NAK 5--841 (NGMPP B 12/4); GOTIT 
\begin{enumerate}
\item NAK 6--7 (NGMPP A 1028/4);  GOTIT NO VṚṢA!
\item UBT Ma I 582; GOTIT (Tübingen)
\begin{itemize}
\item Cambridge, Cambridge University Library: Add.1599 no!
\item Add.2836 no!
\item Or.726.
\end{itemize}
\end{enumerate}

\subsection{Editorial policies}
\label{sec:orge8737ee}
\begin{itemize}
\item avagrahas usually supplied but sometimes found in the MSS
\item usually 4 pādas to a verse, but I have made arbitrary decisions based on sense-units 
because none of the sources really indicate where a verse ends (||).
\item falsifications everywhere on purpose and accidentally
\end{itemize}
\section{The critically edited Sanskrit text of VSS 1--12}
\label{sec:orgd12d76b}
Roman or Devanāgarī?
\subsection{Beginning of MSS}
\label{sec:orgf43c8cc}
\%not in \msCc$\backslash$ MS Add.2102
<> śrīgaṇeśāya namaḥ<>
	<AP> \vo \om$\backslash$ \mssCaCbCc</AP>

\% \msCa f.193v line 1                   image 382
\% \msCb f.201v line 4                   image 404
\% \msCc f.267r line 1 (not 237r as online shown) image 181
		It is incomplete. It starts on exp. 181.jpg, f. 237v(??! different hand) (1.1)
				it breaks off at 2.21 (270v??) and resumes in 189.jpg, 273r (sic!)
\% \msNa f.1v (numbering restarts) image 195.jpg upper
\% \msNb image 44.jpg 
\% \msNc f.209v image 212.jpg upper  
\% \msNd A 3/3, 177.jpg, f. 227v
\% \msL 2014-06-04 13.24.06.jpg (Wellcome Institute)
\section{Translation and Notes}
\label{sec:org568ba99}
Separately or in one go?
\section{Bibliography}
\label{sec:org71f678a}

\begin{itemize}
\item Bisschop 2018: Bisschop, Peter C. Universal Śaivism. The Appeasement of All Gods and Powers in the Śāntyadhyāya of the Śivadharmaśāstra. In: Gonda Indological Studies no. 18. Leiden: Brill.

\item Bisschop, Peter C., Early Śaivism and the Skandapurāṇa. Sects and Centres (Groningen: Egbert Forsten, 2006).

\item Bisschop, Peter C., "Once Again on the Identity of Caṇḍeśvara in Early Śaivism: A rare Caṇḍeśvara in the British Museum?", Indo-Iranian Journal vol. 53 pp. 233-249 (2010).

\item Bonazzoli, Giorgio, "Introducing Śivadharma and Śivadharmottara", Altorientalische Forschungen vol. 20 issue. 2 pp. 342-349 (1993).

\item De Simini, Florinda, A Critical Edition of Śivadharmottara’s Chapter Two, ‘On The Gift of Knowledge’ (vidyādānādhyāya). With Introduction and English Translation (Forth.).

\item Goodall, Dominic, "The Throne of Worship. An Archaeological Tell of Religious Rivalries", Studies in Indian History vol. 27 issue. 2 pp. 221-250 (2013).

\item Hazra, Rajendra Chandra, "The Śivadharma", Journal of the Ganganath Jha Research Institute vol. 10 pp. 1-20 (1952-3).

\item Hazra, Rajendra Chandra, "The Śivadharmottara", Journal of the Ganganath Jha Research Institute vol. 13 pp. 19-50 (1983).

\item Magnone, Paolo, "Śivadharmottara Purāṇa: a Survey", in Petteri Koskikallio (ed.), Epics, Khilas, and Purāṇas: Continuities and Ruptures. Proceedings of the Third Dubrovnik International Conference on the Sanskrit Epics and Purāṇas. September 2002 (Zagreb: 2005) pp. 575-596.

\item Sanderson, Alexis, "The Śaiva Religion among the Khmers. Part I", Bulletin de l'École Française d'Extrême-Orient vol. 90-91 pp. 349-462 (2003-2004).

\item Sanderson, Alexis, Religion and the State: Initiating the Monarch in Śaivism and the Buddhist Way of Mantras (Heidelberg: Harrassowitz, Forth.).

\item De Simini 2016: De Simini, Florinda. ‘Śivadharma Manuscripts from Nepal and the Making of a Śaiva Corpus’. In: Michael Friedrich \& Cosima Schwarke (eds), One-Volume Libraries: Composite and Multiple-Text Manuscripts. Berlin, De Gruyter. Studies in Manuscript Cultures, pp. 233--286.

\item De Simini \& Mirnig 2017: De Simini, Florinda \& Nina Mirnig. ‘Umā and Śiva’s Playful Talks in Detail (Lalitavistara): On the Production of Śaiva Works and their Manuscripts in Medieval Nepal. Studies on the Śivadharma and the Mahābhārata 1.’ in: V. Vergiani, D. Cuneo, C.A. Formigatti (eds.), Indic Manuscript Cultures through the Ages, Berlin: De Gruyter, 587--653.

\item Naraharinath 1998: Naraharinath, Yogin. Śivadharma Paśupatimatam Śivadharmamahāśāstram Paśupatināthadarśanam. Edit. by Yogin Naraharinatha. Kathmandu, saṃvat 2055 (1998 CE).

\item Olivelle, Patrick. The Āśrama System. The History and Hermeneutics of a Religious Institution. New York, Oxford: Oxford University Press, 1993.

\item Sanderson 2009: Sanderson, Alexis. ‘The Śaiva Age: The Rise and Dominance of Śaivism during the Early Medieval Period.’ In: Genesis and Development of Tantrism. ed. Shingo Einoo. Tokyo: Institute of Oriental Culture, University of Tokyo. Institute of Oriental Culture Special Series, 23, pp. 41--350.

\item Sanderson 2014: Sanderson, Alexis. ‘The Śaiva Literature.’ Journal of Indological Studies (Kyoto), Nos. 24 \& 25 (2012--2013), pp. 1--113.

\item Shastri 1905, 1915: Śāstri, H. P. A Catalogue of Palm-leaf \& Selected Paper MSS belonging to the Durbar Library, Nepal. Vol I--II. Calcutta: Baptist Mission Press, 1905, 1915.

\item Shaman Hatley BraYā vol. i.: a chapter on Śivadharmaśāstra's origin of liṅga story

\item Kenji sent me:
Schlingloff 1969: Schlingloff, Dieter. 'The Oldest Extant Parvan-List of the Mahābhārata.'
Journal of the American Oriental Society, Vol. 89, No. 2 (Apr. - Jun., 1969), pp. 334-338
Stable URL: \url{http://www.jstor.org/stable/596517}
\end{itemize}

\section{Index}
\label{sec:org10d9891}
A must
\end{document}
