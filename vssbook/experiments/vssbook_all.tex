\PassOptionsToPackage{unicode=true}{hyperref} % options for packages loaded elsewhere
\PassOptionsToPackage{hyphens}{url}
%
\documentclass[]{article}
\usepackage{lmodern}
\usepackage{amssymb,amsmath}
\usepackage{ifxetex,ifluatex}
\usepackage{fixltx2e} % provides \textsubscript
\ifnum 0\ifxetex 1\fi\ifluatex 1\fi=0 % if pdftex
  \usepackage[T1]{fontenc}
  \usepackage[utf8x]{inputenx}
  \usepackage{textcomp} % provides euro and other symbols
\else % if luatex or xelatex
  \usepackage{unicode-math}
  \defaultfontfeatures{Ligatures=TeX,Scale=MatchLowercase}
\fi
% use upquote if available, for straight quotes in verbatim environments
\IfFileExists{upquote.sty}{\usepackage{upquote}}{}
% use microtype if available
\IfFileExists{microtype.sty}{%
\usepackage[]{microtype}
\UseMicrotypeSet[protrusion]{basicmath} % disable protrusion for tt fonts
}{}
\IfFileExists{parskip.sty}{%
\usepackage{parskip}
}{% else
\setlength{\parindent}{0pt}
\setlength{\parskip}{6pt plus 2pt minus 1pt}
}
\usepackage{hyperref}
\hypersetup{
            pdfborder={0 0 0},
            breaklinks=true}
\urlstyle{same}  % don't use monospace font for urls
\setlength{\emergencystretch}{3em}  % prevent overfull lines
\providecommand{\tightlist}{%
  \setlength{\itemsep}{0pt}\setlength{\parskip}{0pt}}
\setcounter{secnumdepth}{0}
% Redefines (sub)paragraphs to behave more like sections
\ifx\paragraph\undefined\else
\let\oldparagraph\paragraph
\renewcommand{\paragraph}[1]{\oldparagraph{#1}\mbox{}}
\fi
\ifx\subparagraph\undefined\else
\let\oldsubparagraph\subparagraph
\renewcommand{\subparagraph}[1]{\oldsubparagraph{#1}\mbox{}}
\fi

% set default figure placement to htbp
\makeatletter
\def\fps@figure{htbp}
\makeatother


\date{}

\begin{document}

\hypertarget{acknowledgements}{%
\section{Acknowledgements}\label{acknowledgements}}

For the staff in the NAK: Jyoti Neupane, Manita Neupane, Saubhagya
Pradhananga (Chief, National Archives), Rubin Shrestha, Sahan
Ranjitkar{[}?{]} a morc?, Sushmita Das (VSS MSS in Calcutta)
sribin7501@gmail.com

For colleagues and friends in Naples: Florinda De Simini, Nirajan Kafle,
Kengo Harimoto, Giulia Buriola, Alessandro Battistini, Lucas Boer,
Torsten \ldots{}, Kenji Takahashi, Francesco Sferra, Dorotea \ldots{} ,
Martina Dello Buono, Chiara Margherita Trento, students in Naples?
\ldots{}

Others: Judit Törzsök, Dominic Goodall, Harunaga Isaacson, SAS Sarma,
Sathyanarayan, Gergely Hidas, my family

Mention COVID-19?

The present publication is a result of the project DHARMA `The
Domestication of ``Hindu'' Asceticism and the Religious Making of South
and Southeast Asia'. This project has received funding from the European
Research Council (ERC) under the European Union's Horizon 2020 research
and innovation programme (grant agreement no. 809994).

OR latest 2022 March: ``This project has received funding from the
European Research Council (ERC) under the European Union's Horizon 2020
research and innovation programme (grant agreement no. 809994). This
article/book/communication (to choose) reflects the views of the author
only. The funding body is not responsible for any use that may be made
of the information contained therein.''

CHECK what to add for the Śivadharma project?

\pagebreak

\hypertarget{preface}{%
\section{Preface}\label{preface}}

How to deal with the problem of info distributed in two volumes?

\hypertarget{introduction}{%
\section{Introduction}\label{introduction}}

\hypertarget{on-the-contents-of-the-vss}{%
\subsection{On the contents of the
VSS}\label{on-the-contents-of-the-vss}}

\hypertarget{on-the-title}{%
\subsubsection{On the title}\label{on-the-title}}

The main question concerning the title Vṛṣasārasaṃgraha (`A Compendium
on the Essence of the Bull {[}of Dharma{]}') is weather the bull in the
title is only a reference to a representation of Dharma or also a hint
at Śiva's vehicle, sometimes named Nandi.

Dharma has frequently been referred to as a bull in Sanskrit literature
from at least the time of the MBh, e.g.~in this passage:

\begin{verbatim}
    MBh on the 4-legged bull:
    03188010a kṛte catuṣpāt sakalo nirvyājopādhivarjitaḥ
    03188010c vṛṣaḥ pratiṣṭhito dharmo manuṣyeṣv abhavat purā
    03188011a adharmapādaviddhas tu tribhir aṃśaiḥ pratiṣṭhitaḥ
    03188011c tretāyāṃ dvāpare 'rdhena vyāmiśro dharma ucyate
    03188012a tribhir aṃśair adharmas tu lokān ākramya tiṣṭhati
    03188012c caturthāṃśena dharmas tu manuṣyān upatiṣṭhati
    03188013a āyur vīryam atho buddhir balaṃ tejaś ca pāṇḍava
    03188013c manuṣyāṇām anuyugaṃ hrasatīti nibodha me
\end{verbatim}

Manusmṛti also confirms this (8.16a): vṛṣo hi bhagavān dharma.

MMW `vṛṣa':\\
``Justice or Virtue personified as a bull or as''Siva's bull Mn. viii,
16 Pur. Kāvyād.; just or virtuous act, virtue, moral merit ``Siś.
Vās.;''

Mahākṣapaṇaka's koṣa (CHECK date), the Anekārthadhvanimañjarī, places
the meaning `dharma' as first when defining the word `vṛṣa':

\begin{verbatim}
    dharmo vṛṣo vṛṣaḥ śreṣṭho vṛṣo gaurmūṣiko vṛṣaḥ |
    vṛṣo balaṃ vṛṣaḥ kāmo vṛṣalo vṛṣa ucyate || 1.48
    
\end{verbatim}

The ŚDhU also mentions the `Dharma bull':

\begin{verbatim}
    īśvarāyatanasyādhaḥ śrīmān dharmavṛṣaḥ sthitaḥ ||
    yatra vīravṛṣas tatra kṣityāṃ gomātaraḥ sthitā || 12.87
\end{verbatim}

visnusmrḍn:ViS 86.15a/ vṛṣo hi bhagavān dharmaś catuṣ-pādaḥ prakīrtitaḥ
/

smrti/dharma/krtyaratnaakara.dn: !!! dharmo 'yaṃ vṛṣarūpeṇa nāmnā
nandīśavaro vibhuḥ \textbar{} dharmān māheśvarān vakṣyaty ataḥ prabhṛti
nārada\textbar{}\textbar{}

tak2015/AtmapujaT55Muktabodha.dn: dharmas tatra vṛṣākāro jñānaḥ
siṃhasvarūpakaḥ \textbar{} vairāgyaṃ

On the title, see De Simini 2016:238 n.~13: `'As noted by Sanderson
{[}\ldots{}{]}, this title can have a double meaning, since the `bull'
(vṛṣa) is both a synonym of `religious practice' and the traditional
mount (vāhana) of Śiva. i.e.~Sanderson (Forthc. b), Śaivism and
Brahmanism. (can't find it)

Sanderson Tolerance p.~210 n.~136: ``{[}\ldots{}{]} To laud the bull
(\emph{vṛṣa}) {[}on the coins of the Hephthalite Hun Mihirakula, CsK{]}
would be surprising if the intended meaning were the bull that is Śiva's
mount, but not if the word is intended in its figurative meaning, namely
\emph{dharmaḥ}, or \emph{sukṛtam} `the virtuous actions {[}prescribed by
the Veda{]}.' For this meaning of \emph{vṛṣaḥ} see, for example,
Amarasiṃha, \emph{Nāmaliṅgānuśāsana} 1.4.25b (\emph{sukṛtam vṛṣaḥ}),
3.3.220 (\emph{sukṛte vṛṣabhe vṛṣaḥ}); Halāyudha,
\emph{Abhidhānaratnamālā} 1.125cd (\emph{dharmaḥ puṇyaṃ vṛṣaḥ śreyaḥ
sukṛtaṃ ca samaṃ smṛtam}); \emph{Manusmṛti} 8{[}. CsK{]}16a CHECK
(\emph{vṛṣo hi bhagavān dharmas \ldots{}}); and the Gwalior Museum Stone
Inscription of Pataṅgaśambhu (\textsc{Mirashi} 1962), l. 15,
\emph{vṛṣaikaniṣṭho `pi jitasmaro 'pi yaḥ śaṅkaro 'bhūd bhuvi ko 'py
apūrvvaḥ,* concerning the Śaiva ascetic Vyomaśambhu: 'He was in the
world an extraordinary new Śiva, since he too was \emph{vṛṣaikaniṣṭhaḥ}
('devoted solely to pious observance'; in Śiva's case `riding only on
the Bull') and he too was }jitasmaraḥ* (`one who had defeated sensual
urges'; in Śiva's case `the defeater of the Love god Kāmadeva'). This is
also the meaning of \emph{vṛṣaḥ} in the title \emph{Vṛṣasārasaṃgraha},
one of the works of the Śivadharma corpus (see, e.g., \textsc{Sanderson}
2014, p.~2), i.e., `Summary of the Essentials of the
{[}Śiva{]}dharma'.''

\hypertarget{the-structure-of-the-vss}{%
\subsubsection{The structure of the
VSS}\label{the-structure-of-the-vss}}

\begin{verbatim}
- Matryoshka 
- dialogues
- affiliations
- lotus diagramme
- ch. 2 misplaced? 
\end{verbatim}

\hypertarget{contents}{%
\subsubsection{Contents}\label{contents}}

\begin{verbatim}
  24 chapters

  1. brahmāṇḍasaṃkhyā 
  2. śivāṇḍasaṃkhyā 
  3. ahiṃsāpraśaṃsā 
  4. yamavibhāga
  5. śaucācāravidhi
  6. yajñavidhi (also lokāḥ)
  7. dānapraśaṃsā 
  8. niyamapraśaṃsā (p. 603: types of svādhyāyana: śaiva, sāṃkhya, purāṇa,
                    smārta, bhārata)
  9. traiguṇyaviśeṣaṇīya
  10. kāyatīrthavivarṇana
  11. caturāśramadharmavidhāna 
  12. vipulopākhyāna  (narrative)
  13. garbhotpatti (on conception)
  14. praśnavyākaraṇa (why people are tall/short etc.)
  15. jīvanirṇaya 
  16. adhyātmanirṇaya (yoga) 
  17. dānadharma
  18. pūrvakarmavipāka
  19. dānayajñaviśeṣa
  20. pañcaviṃśatitattvanirṇaya
  21. kalpanirṇaya
  22. varṇagotrāśrama
  23. nidrotpatti
  24. śāstravarṇana
\end{verbatim}

\begin{itemize}
\tightlist
\item
  Summary of the contents of all 24 chapters of the VSS
\item
  References to other works - Mahābhārata - nakule - vipule etc.
\end{itemize}

\hypertarget{the-role-of-the-vss-in-the-ux15bivadharma-corpus}{%
\subsubsection{The role of the VSS in the Śivadharma
corpus}\label{the-role-of-the-vss-in-the-ux15bivadharma-corpus}}

\begin{itemize}
\item
  general ideas

  \begin{itemize}
  \tightlist
  \item
    is this text really Śaiva? why in this collection?
  \item
    niśvāsa as sadāśiva in ch.~16; Niśvāsa uttarasūtra 5.50-51; see also
    Kafle Niśvāsamukha p.11ff; ibid. p.12: ``The term niśvāsa means
    sighing. Thus, an alternative meaning of the Niśvāsatattvasaṃhitā
    could also be a ``sighing tantra.'' To be more precise, a tantra
    that originated from the sighing of Śiva. This is to say, the speech
    of Śiva.''
  \item
    tattva-system: mati and suśira (ch.~20)
  \item
    parallels: MBh, Bṛhatkālottara,
  \item
    ch.~21: Viṣṇu; is this a Śaiva text?
  \item
    āśramas are in an order different from usual; compare this to NĀT;
    ``Variations on the āśrama-system''
  \end{itemize}
\item
  History of Dharmasastra 2.1 pp.~416ff on āśramas
\item
  \begin{enumerate}
  \def\labelenumi{\alph{enumi}.}
  \setcounter{enumi}{13}
  \tightlist
  \item
    988! see Āpastamba-dharma-sūtra ii.9.21.1: catvāra āśramā
    gārhasthyam ācāryakulaṃ maunaṃ vānaprasthyam iti\textbar{} Quoted by
    Śankara But the chapters in Āpastamba follow the traditional order.
    ``Āp. places the householder first among the āśramas, probably on
    account of the importance of that stage to all other āśramas.'' Kane
    ibid.
  \end{enumerate}
\item
  ibid p.~417: person in last āśrama is called: parivrāṭ,
  parivrājaka(!), bhikṣu, muni, yati. See Olivelle, Patrick. The Āśrama
  System. The History and Hermeneutics of a Religious Institution. New
  York, Oxford: Oxford University Press, 1993. {[}megvan{]} p.82ff: The
  Order of Āśramas; ibid: ``In later texts the usual order is student,
  householder, hermit, and renouncer, reflecting the sequence of the
  passage from one \emph{āśrama} to another\ldots{} In the Dharmasūtras,
  however, only Baudhāyana and Vasiṣṭha follow that order\ldots{} A
  specific order becomes insignificant when the \emph{āśramas} are taken
  as four alternative adult vocations.'' Are they alternative adult
  vocations here in the Vṛṣasārasaṃgraha? They are numbered.
\item
  \textit{Gṛhastha. The Householder in Ancient Indian 
      Religious Culture.} Edited by Patrick Olivelle. OUP, 2019.
  Especially Csaba Dezső's article in it.
\item
  \%dscn 8034.jpg ff in folder
  /home/csaba/mmedia/images/scan/saiva/sivadharmacorpus/pasupatimatam4/
  \% in Naraharinātha's Paśupatimatam pp.~580ff \% CHECK if Naraharinath
  seems to be better at Sanskrit in other texts \% the edition seems
  problematic at many places \% a dialogue between Janamejaya and
  Vaiśampāyana, the latter of whom relates dialogues between Vigatarāga
  and Anarthayajña \% revise ¤s and lost/ill Bisschop in ``Universal
  Śaivism'': " -- En-dashes indicate a lost or illegible syllable in the
  manuscript."
\item
  \%N. of a celebrated king to whom Vaiśampāyana recited the {[}MBh.{]}
  (greatgrandson to Arjuna, as being son and, successor to Parikshit who
  was the son of Arjuna's son Abhimanyu) {[}"SBr.{]} xi, xīi AitBr.
  "Sāṅkhir. xvi {[}MBh.{]} \&c.;
\item
  Bisschop 2018:2: ``The full text of the corpus was first published by
  Naraharinātha in 1998, while over the past few years several scholars
  have started to work on individual parts of the corpus or referred to
  them in their studies. See, in particular, Acharya 2009; Bisschop
  2010, 2014; De Simini 2013, 2016a, 2016b, 2017; De Simini \& Mirnig
  2017; Goodall 2011; Kafle 2013, 2015; Magnone 2005; Sanderson 2003/04,
  2012/13; Schwartz 2012. An edition of the Śivadharmaśāstra alone,
  based on a single manuscript in the Adyar Library, has been published
  more recently as well (Jugnu \& Sharma 2014). The Śivopaniṣad, which
  also forms part of the Śivadharma corpus, was already published much
  earlier but was not recognised as such, being included in a collection
  of Upaniṣads (Kunhan Raja 1933).''
\item
  What MS did Naraharinātha used? See Biscchop 2018:58--59.
\item
  Palm leaf:
  /home/csaba/mmedia/images/scan/saiva/sivadharmacorpus/mss\_florinda/newari/ngmpp/palm\_leaf\_mtm/A
  3:3/fr.8493.0.A 0003-03\_3/A3-03+65851+177\_vss\_start.jpg Paper MS
  /home/csaba/mmedia/images/scan/saiva/sivadharmacorpus/mss\_florinda/newari/ngmpp/paper\_mtm/A
  1341-06/DSCN0331 fol.~204\_vss.JPG
\item
  Vipula

  Vipula in the MBh:

  MBh 13040016aff

  Devaśarman and his wife Ruci 13040017a tasya rūpeṇa --\textgreater{}
  13040017a tasyā rūpeṇa

\begin{verbatim}
  all gods, esp. Indra, are in love with her but Devaśarman guards her
  wants to perform yajña: how to guard her during the ritual?  calls his
  pupil, Vipula tells him that Indra can assume various forms Vipula
  decides that the only way to protect her from Indra is to magically
  'enter' her (with yoga) he tells her stories and enters her 

  MBh 13041001ff Indra sees the opportunity and enters the āśrama as a
  beautiful man he sees Vipula's lifeless body Ruci fancies Indra, but
  Vipula in his body stops her from standing up Indra sings to her
  beautiful songs he says "I have come for you, I am Devendra, I am in
  love" Vipula stops her from doing anything Indra is a bit shocked by
  her not being moved, gets angry and can see now that Vipula is in her
  Vipula leaves her, enters his own body, and abuses Indra and tells
  Indra how wicked he is Indra is ashamed and disappears Devaśarman
  returns to the āśrama, Vipula tells him what happened and Devaśarman
  praises him
\end{verbatim}
\item
  ETC., see translation here:
  https://www.sacred-texts.com/hin/m13/m13b005.htm
\item
\begin{verbatim}
  See summary also here:
 V. S. Sukthankar. Critical Studies in the Mahābhārata.  Poona, V. S.
 Sukthankar Memorial Edition Committee, 1944. 317--318
 https://archive.org/details/in.ernet.dli.2015.281344/page/n333
\end{verbatim}
\end{itemize}

\hypertarget{dating-and-provenance}{%
\subsection{Dating and provenance}\label{dating-and-provenance}}

\begin{itemize}
\tightlist
\item
  Dating

  \begin{itemize}
  \tightlist
  \item
    the archaic yoga of chapter 10 (no Piṅgalā), Śaiva
  \item
    order of āśramas, cf.~Śaiva Utopia p23, Chapter 11, Śaiva
  \item
    11.23a: 4 kalās (nivṛttyādi caturvedaś), instead of the later 5,
    Śaiva
  \item
    the tattvas (no tanmātras), Chapter 20, Vaiṣṇava
  \item
    varṇas and the Liṅgapurāṇa
  \item
    check lists of deities such as Vasus
  \end{itemize}
\item
  Place of composition: geographical names and persons mentioned
\end{itemize}

\hypertarget{interpretation-of-chapters}{%
\subsection{Interpretation of
chapters}\label{interpretation-of-chapters}}

\begin{itemize}
\tightlist
\item
  Chapter 12

  \begin{itemize}
  \tightlist
  \item
    everybody is donating to everybody,
  \item
    the final donor is Brahmā
  \item
    lot of testing going on in the frame story and also
  \item
    in chapter 12
  \item
    also the disguise thing is recurring: 12.37 and ch 1 and
  \item
    when Viṣṇu reveals his identity
  \end{itemize}
\end{itemize}

\hypertarget{misc}{%
\subsection{Misc}\label{misc}}

\begin{itemize}
\item
  susūkṣma: Śivadharmottara 10.45cd--46: rudraḥ ṣaḍviṃśakaḥ proktaḥ
  śivaś ca paratas tataḥ \textbar{}\textbar{} 45 \textbar{}\textbar{}
  saptaviṃśatimaḥ śāntaḥ susūkṣmaḥ parameśvaraḥ \textbar{}
  svargāpavargayor dātā taṃ vijñāya vimucyate \textbar{}\textbar{} 46
  \textbar{}\textbar{}. yamas-niyamas: see table in Śaiva Utopia p17
\item
  other Why is this mentioned at
  http://cudlḷib.cam.ac.uk/view/MS-ADD-01694-00001/403 : C., Kunhan
  Raja, Un-published Upanishads (Adyar: The Adyar Library, 1933). Ahhh,
  Śivopaniṣat is in there! cf.~śivasaṃkalpa in pp 319 ff.
  (Śivasaṃkalpopaniṣat) Bonazzoli, Giorgio, ``Introducing Śivadharma and
  Śivadharmottara'', Altorientalische Forschungen vol.~20 issue. 2
  pp.~342-349 (1993). ``There is no raw data.'' EdX Harvard Digital
  Humanities
\item
  CHECK out Kenji on the Umāmaheśvarasaṃvāda in the MBh, his summary
  looks similar to the VSS
\item
  Kenji: `'BDhS 2: Discussion of gṛhastha. but BDh 2.11.9--34 is a
  digression on the topic of caturāśrama (vikalpa type, not krama type),
  and the author denies caturāśrama idea.''
\item
  MSS: see Bisschop Universal \ldots{} pp.~52--53; De Simini \& Mirnig
  pp.~587, 591 \% ``a stable element of the corpus''
\item
  Vindicate your edition: look at the apparatus, all the \Ed entries
\end{itemize}

\hypertarget{notes-on-the-language}{%
\section{Notes on the language}\label{notes-on-the-language}}

The language of the VSS goes beyond the idiosyncrasies of epic Sanskrit.
It exhibits strong similarities to Śaiva Aiśa Sanskrit,\footnote{See
  Goodall, Törzsök, Hatley, Kiss, Meyr?} and it applies particular
metrical licences and uses a special vocabulary, morphology and syntax.
The analysis of this language may lead us to the author(s) and to the
place of composition. Here I only give a brief overview of the most
important phenomena. For details see the Notes on the constitution of
the Sanskrit text and the Index.

\hypertarget{metre}{%
\subsection{Metre}\label{metre}}

As regards metrical licences, the first striking feature is the generous
use of the poetic licence sometimes labelled `muta cum
liquida,'\footnote{For recent contributions on this phenomenon, see
  e.g.~Ranjan Sen 2006 (discussing it as appearing in Latin) and Balogh
  2018, note 6 (discussing Sanskrit metre).} namely that some consonant
clusters that would normally turn the previous short (laghu) syllable
long (guru) may in some cases do not do so. The syllables pra, bra, hra,
kra, especially at the beginning of words CHECK, are well-known
candidates for this licence.\footnote{See e.g.~Apte' XXX Dictionary
  Appendix A p.~1. ADD real life examples.} In the VSS, tra, vra, śra,
pra, and also śya, śva, sva, dva, all involving conjunct consonants with
a liquid sounds or semi-vowels in second position, and possibly also
rpa,CHECK! seem additional ones.

The well-known author on prosody, Kedārabhaṭṭa, who must definitely be
much later than the time of composition of the VSS since he is
frequently quoted by Mallinātha, gives the following definition in his
Vṛttaratnākara (here given together with Sulhana's Sukavihṛdayanandinī
commentary):

\begin{quote}
\emph{padādāv iha varṇasya saṃyogaḥ kramasaṃjñikaḥ} \textbar{}\\
\emph{puraḥsthitena tena syāl laghutā 'pi kvacid guroḥ}
\textbar{}\textbar{} 1.10 \textbar{}\textbar{}
\end{quote}

\begin{quote}
\begin{quote}
\emph{vibhaktyantaṃ padaṃ tasya padasyādau vartamāno yo varṇas tasya
saṃyogaḥ} \textbar{} \emph{sa iha śāstre kramasaṃjño jñeyaḥ} \textbar{}
\emph{tena krameṇa purovartinā prākpadānte vartamānasya
prāptagurubhāvasyāpi laghutā syāt} \textbar{} \emph{kvacil
lakṣānurodhena} \textbar{} \emph{nanu ka eṣaḥ kramo nāma saṃyoga ucyate}
\textbar{} \emph{pūrvācāryāṇāṃ piṅgalanāgaprabhṛtīnāṃ kālidāsādīnāṃ ca
kavīnāṃ samayaḥ parigṛhītaḥ} \textbar{} \emph{saṃyogaḥ kramasaṃyogaḥ}
\textbar{}\textbar{} 10 \textbar{}\textbar{} \emph{tatra gra-saṃyogena
yathā} \textbar{} \emph{idam asyodāharaṇam} \textbar{}
\end{quote}
\end{quote}

\begin{quote}
\emph{taruṇaṃ sarṣapaśākaṃ navaudanaṃ picchalāni ca dadhīni}
\textbar{}\\
\emph{alpavyayena sundari grāmyajano miṣṭam aśnāti} \textbar{}\textbar{}
1.11 \textbar{}\textbar{}
\end{quote}

TRANSLATE the whole passage!

\begin{quote}
In this {[}work{]}, a consonant cluster at the beginning of a word is
called krama. A {[}preceding{]} long syllable can sometimes be turned
short by that initial {[}syllable{]}.
\end{quote}

\begin{quote}
\begin{quote}
\begin{verbatim}
   ... TRANSLATE 
\end{verbatim}
\end{quote}
\end{quote}

\begin{quote}
Tender mustard seed, fresh porridge, and slimy curds: village people eat
this kind of dishes, O beautiful goddess, due to lack of money.
\end{quote}

To sum this example up: the verse given above (1.11) is in āryā, and the
metric pattern of the second half-verse is the following: - - . - . - .
- - . . - - . - - - This yields 28 morae, instead of the expected 27. By
treating the final syllable of sundari short, in spite of the following
grā, the pattern conforms to the expected pattern: - - . - . - . . - . .
- - . - - -

The commentator gives several more examples (involving the syllables
gra, hra, bhra), including this Śārdūla line:

\begin{quote}
nidravyo hriyam eti hrīparigataḥ prabhraśyate tejasaḥ
\end{quote}

In this line, the last syllable of eti should count as short, in spite
of the fact that the beginning of the next word (hrī°) would normally
turn it long.

The VSS abounds in this phenomenon of `muta cum liquida'. EXAMPLES from
the VSS and from my emendations. CHECK

\begin{itemize}
\item
  only beginning of words in the VSS?
\item
  final -am etc. counts as long (reverse of muta cum liquida)
\item
  In Kannada and Telugu prosody, under the name of śithila-dvitva, the
  exception is that sometimes in a consonant cluster of the form
  {[}consonant + ``r''{]}, the ``r'' (repha) can be ignored, so that it
  is not a conjunct consonant anymore.
\item
  The exception is not accepted by purists: Shatavadhani Ganesh says
  that the Sanskrit masters like Kālidāsa, Bhāravi, Māgha, Śrīharṣa, and
  Viśākhadatta have not freely used this exception (though the masters
  in Kannada and other languages have). Being more of a ``poetic
  licence'' and a violation of the standard rule (only found in later
  poetry), it is extremely unlikely that any sane poet would have
  indulged in that exception in all four pāda-s of a verse. Thus it is
  very unlikely that the program will miss identifying a verse that
  indulges in this
\item
  Reference: see comments by Dr.~Ganesh and Nityananda Misra in this
  thread started by Vishvas Vasuki:
  \%https://groups.google.com/forum/\#!topic/bvparishat/ya1cGLuhc14/discussion
\end{itemize}

\hypertarget{vocabulary}{%
\subsection{Vocabulary}\label{vocabulary}}

\begin{itemize}
\tightlist
\item
  Special vocabulary/language: karhacit, hṛdi as nominative 10.27cd,
  tirya, me as mayā
\item
  generate list from index
\end{itemize}

\hypertarget{number-and-gender}{%
\subsection{Number and gender}\label{number-and-gender}}

\begin{itemize}
\tightlist
\item
  Number: singular next to numerals, and general confusion (CHECK)
\end{itemize}

\hypertarget{syntax}{%
\subsection{Syntax}\label{syntax}}

\begin{itemize}
\tightlist
\item
  Special structures: caturmaunasya vakṣyāmi indreṇāsmi phalaṃ dattaṃ
\end{itemize}

\hypertarget{stem-form-nouns}{%
\subsection{Stem form nouns}\label{stem-form-nouns}}

\begin{itemize}
\item
  stem form nouns (prātipadika)
\item
  a more or less full collation is important: we cannot automatically
  reject `ungrammatical' or unmetrical forms because they may well be
  the `original' one Manuscripts consulted ---------------------
\end{itemize}

\hypertarget{witnesses}{%
\subsection{Witnesses}\label{witnesses}}

In the manuscript descriptions below, in addition to some general
remarks, I will mainly focus on information relevant to the VSS. For
much more detail on the overall features of these manuscripts, see De
Simini 2016 {[}CHECK SDh MSS from Nepal{]} and the catalogues I mention
at each individual manuscript.\footnote{I owe thanks to Florinda De
  Simini for sharing with me most of the manuscripts listed here, to
  Kengo Harimoto and Gudrun Melzer (Munich) for providing photos of the
  Munich MS, and to Nirajan Kafle for sharing the Paris MS with me.}

In recently published and forthcoming critical editions of and articles
on the Śivadharma corpus (CHECK examples), the sigla of the manuscripts
used are made up of a letter signifying the script (e.g.~`N' for
Nepālākṣara/Newari), a superscript letter for the current location where
the manuscript is deposited (e.g.~`C' for Cambridge), and two (sometimes
only one or even three) subscript digits echoing the last digit(s) of
the reference number of the manuscript in the library where it is
located or, in the case of NGMPP reel numbers, the last two digits of
the first part of the reel number. For details of this system and for
the underlying reasons, see Bisschop 2018b, 50--51 (UnivŚ). Since in the
case of the Vṛṣasārasaṃgraha all available manuscripts use some variants
of the Nepālākṣara script, in this publication I have left the first
letter out, making the letter for the current location non-superscript.
This makes the apparatus more readable. In the manuscript descriptions
below, I give this implied `N' in brackets as a reminder.

\href{The\%20Edition\%20by\%20Naraharinath}{The Edition by Naraharinath}

\href{CHECK\%20remaining\%20ones}{CHECK remaining ones}

\hypertarget{how-to-describe-a-ms}{%
\section{How to describe a MS?}\label{how-to-describe-a-ms}}

In general these categories should be included: - {[}X{]} Siglum - {[}
{]} Location where it is deposited - {[} {]} Ms no. - {[} {]} How much
do I use it? - {[} {]} Catalogued by and where and under what no., with
what title - {[} {]} what does the catalogue says - {[} {]} Physical: -
{[} {]} material - {[} {]} dimensions - {[} {]} no of folios - {[} {]}
condition - {[} {]} format, binding - {[} {]} Text - {[} {]} script -
{[} {]} contains these texts - {[} {]} complete - {[} {]} language
(implied) - {[} {]} MTM? - {[} {]} foliation - {[} {]} hands - {[} {]}
initial rubric, incipit, explicit, final rubric (in edition) - {[} {]}
colophons (in edition) - {[} {]} dating - {[} {]} description in detail,
remarks

\hypertarget{the-cambridge-manuscripts}{%
\section{The Cambridge Manuscripts}\label{the-cambridge-manuscripts}}

\hypertarget{nc94}{%
\subsection{(N)C94}\label{nc94}}

Cambridge University Library, Add. 1694.1. Fully collated for chapters
1--12 of the critical edition in this volume. See a detailed description
of this manuscript at the CUL online catalogue.\footnote{https://cudl.lib.cam.ac.uk/view/MS-ADD-01694-00001/382}
According to this catalogue, the date of creation is the 12th century,
the dimensions of the manuscript are 5cm × ca. 53.5cm. The script is
Nepālākṣara. It is a palm-leaf multiple-text manuscript containing 258
folios. Eight texts are transmitted in this manuscript: 1)
Śivadharmaśāstra 2) Śivadharmottara 3) Śivadharmasaṃgraha 4)
Umāmaheśvarasaṃvāda 5) Uttarottaramahāsaṃvāda 6) Vṛṣasārasaṃgraha 7)
Dharmaputrikā 8) Śivopaniṣad.

The VSS occupies 45 folios: it starts on f.~193 (the recto side, online
image no. 381, is an empty folio side, the text itself starts on the
verso side); it ends on f.~239r (and not 193r, as the online description
says; online image no. 472). The text of the VSS is transmitted fully,
without any folios or major sections of the text missing. The leaves
transmitting the VSS are well-preserved. Some folio sides are faded and
most folios are somewhat damaged on the right side, sometimes at other
parts, and it seems from the images that some opaque-looking tape has
been applied to protect these damaged sections. In my critical edition
the broken off, completely lost, \emph{akṣara}s are represented by ×,
the illegible \emph{akṣara}s under the tape by ¤ (`illegible'). The
quality of the readings of this manuscript is one of the best,
comparable only to msNa and msP, making it one of the most important
sources for the VSS.

\hypertarget{nc45}{%
\subsection{(N)C45}\label{nc45}}

Cambridge University Library, Add. 1645. Fully collated for chapters
1--12 of the critical edition in this volume. See a detailed description
of this manuscript at the CUL online catalogue.\footnote{https://cudl.lib.cam.ac.uk/view/MS-ADD-01645/404}
According to this catalogue, the dimensions of the manuscript are 4.4cm
× 61.7cm. The manuscript is dated to (Nepali) `\emph{samvat} 259
\emph{śrāvaṇa śukla dvādaśiyādi}(?) \textless{} \emph{trayodaśyām},'
which converts to July 10/11 Monday/Tuesday, 1139 CE.\footnote{F. 247r
  line 6. The Cambridge site transcribes this colophon as: \emph{saṃvat}
  259 \emph{śrāvaṇaśukladvādaśi{[}pyaḍi} 8 \emph{trayodaśyāṃ} (retrived
  8 Dec 2021). The element \emph{dvādaśipyaḍi} might be read as
  \emph{dvādaśiyā} \emph{di}, perhaps a mistake for \emph{dvādaśyāṃ}
  \emph{di} (\emph{di} for a misplaced \emph{diva/divā}?), and the
  symbol that does look like a figure `8' of a slightly later period
  than the manuscript itself (resembling the mathematical symbol
  \textless{}) might also be a \emph{kākapada}. Another faint
  \emph{kākapada} is perhaps to be seen under \emph{daśi}, therefore it
  is possible that the scribe's intention was to delete \emph{dvādaśi}°
  and correct it to \emph{trayodaśyām}, and then the date becomes the
  11th of July. Kengo Harimoto has suggested that the unclear element
  (\emph{yādi/pyaḍi}) is in fact \emph{ghaṭi}, and after comparing these
  two syllables to other instances of \emph{gha} and \emph{ṭa}, one
  could not be agree. In this case this should be an indication of the
  exact time the scribe finished copying the text. It is still not clear
  if we should take \emph{dvādaśi} or \emph{trayodaśyām} as the date.
  For help on the conversion of the date and for a detailed discussion
  on the colophon I am indebted to Kengo Harimoto.} The script is
Nepālākṣara. It is a palm-leaf multiple-text manuscript containing 247
folios. Eight texts are transmitted in this manuscript: 1)
Śivadharmaśāstra 2) Śivadharmottara 3) Śivadharmasaṃgraha 4) Śivopaniṣad
5) Umāmaheśvarasaṃvāda 6) Uttarottaramahāsaṃvāda 7) Vṛṣasārasaṃgraha 8)
Dharmaputrikā.

The VSS occupies 37 folios plus one folio side: it starts on f.~201v
line 4 (online image no. 404), it ends on f.~238v line 3 (online image
no. 478). The readings of this manuscript seem to follow those of msNa
remarkably closely transmitting the Śivadharmottara (as observed by De
Simini and Harimoto.\footnote{Personal communication, 1 Dec 2021.} This
is more difficult to see in the VSS, but indeed, they seem closely
related. CHECK MORE on this

\hypertarget{nc02}{%
\subsection{(N)C02}\label{nc02}}

Cambridge University Library, Add. 2102. Palm-leaf, 96 folios. Fully
collated all available folios for chapters 1--12 of the critical edition
in this volume. See a detailed description of this manuscript at the CUL
online catalogue.\footnote{https://cudl.lib.cam.ac.uk/view/MS-ADD-02102/181}
According to this catalogue site, the date of creation is the 12th
century, the dimensions of the manuscript are 4.8cm × ca. 52.5cm. The
script is Nepālākṣara. It is a palm-leaf multiple-text manuscript
containing 96 folios. Six texts are transmitted in this manuscript: 1)
Śivadharmottara 2) Śivadharmasaṃgraha 3) Umāmaheśvarasaṃvāda 4)
Śivopaniṣad 5) Vṛṣasārasaṃgraha 6) Dharmaputrikā (only fol.~322v).

The Vṛṣasārasaṃgraha starts on f.~267r line 1\footnote{Not on f.~237r as
  the online description claims. This first folio in fact has no visible
  foliation.} (online image 181) in a hand which is different from the
rest of the manuscript, but it changes back in the next folio.\footnote{Cf.
  the metadata on the CUL site: `1 folio of the same dimensions is a
  modern supply for the beginning of the \emph{Vṛṣasārasaṃgraha}.' A
  hardly readable note in pencil to the same effect is visible at the
  top of the first folio side (f.~267r, `mode\ldots{}\ldots{} supply beg
  of Vṛṣasāra-saṃgr.'). I am not sure how `modern' this supplement is,
  but it seems indeed likely that a lost first folio was supplemented
  with a later copy. To match the end of this new copy with the
  beginning of the next, older, folio, a scribe more or less erased the
  beginning of the first line in the old folio, rather than the other
  way round. This slightly illogical decision may mean that the younger
  copy was not tailor-made for the old portion, but rather that it was
  taken from a younger manuscript which was perhaps considered more
  legible. Otherwise it would have been more practical to stop copying
  the first folio at the point where the next begins. (The folio coming
  before this supplement, image no. 180, ends with verse 7.122cd of the
  \emph{Śivopaniṣad}: \emph{yauvanasthā gṛhasthāś ca}
  {[}\emph{prāsā}{]}\emph{dasthāś ca ye nṛpāḥ}.)}

In this multiple-text manuscript, the VSS is trasmitted in an incomplete
form, that is to say, a number of folios are missing (most notably
chapters 15--17). The first partially visible folio number is in image
184: the numeral characters 200+60 are visible (268v, according to the
CUL online catalogue). In image 186, the folio number 269 is clearly
visible (f.~269v). In folio 270v, the continuous text is broken at verse
2.21c (\emph{kāmarū°}), folios 271 and 272 are missing, and the text
resumes on folio 273r with verse 3.30b ({[}\emph{ahiṃsā
pa}{]}\emph{ramaṃ} \emph{sukham}). Folio 291 is missing (verses
12.87cd--12.113). In folio 296v (image no. 234) the text breaks off
again at \emph{vātaśūlair upadrutā} \textbar{} \emph{śukro} (verse 14.XX
CHECK), the next folio being 306r (\emph{carmatāś ca dvijasundarīṣu})
(verse 18.XX CHECK) (nine folios and chapters 15--17 are completely
missing).

Again, there are two missing folios after \emph{bandhus sarvva°} in
verse 18.XX CHECK in f.~306v. The text resumes in folio 309r (image 237)
with \emph{°ṇeṣu ca sarvveṣu vidvān sreṣṭha sa ucyate} (verse 19.XX
CHECK). Another folio is missing between \emph{iṣṭāniṣṭadvaya°} (verse
20.XX CHECK, folio 309v) and \emph{snāyu majjā sirā tathā} (verse 20.XX
CHECK, folio 311r). The VSS ends on folio 322v (image no. 262) with the
concluding colophon \emph{vṛṣasārasaṅgraha samāpta iti}. This folio also
contains the beginning of the \emph{Dharmaputrikā}, but this
multiple-text manuscript contains no more folios.

\hypertarget{the-kathmandu-manuscripts}{%
\section{The Kathmandu Manuscripts}\label{the-kathmandu-manuscripts}}

\hypertarget{nk82}{%
\subsection{(N)K82}\label{nk82}}

NGMPP A 1082/3, NAK 3/393. Fully collated for chapters 1--12 of the
critical edition in this volume. See a brief description of this MS in
the NGMCP online catalogue.\footnote{https://catalogue.ngmcp.uni-hamburg.de/receive/aaingmcp\_ngmcpdocument\_00098499}
According to this catalogue, the dimensions of the manuscript are 55.6cm
x 5.5cm. Dated to NS 189 (1068--69 CE).\footnote{F. 12r line 2 of the
  \emph{Dharmaputrikā}: \emph{navottarāsītiyute sate bde āsāḍhaśuklasya
  tithau tṛtīye}, translated by De Simini 2016, 252 n.~49 (NMSS) as: `in
  {[}the year{]} 189, in the 3rd lunar day of the bright {[}fortnight{]}
  of {[}the month{]} Āṣādha.' She adds that the date is verified in
  Petech 1984, 46 as May 24, 1069 CE.} The script is Nepālākṣara. It is
a palm-leaf multiple-text manuscript containing 274 folios. Eight texts
are transmitted in this manuscript: 1) Śivadharmaśāstra 2)
Śivadharmottara 3) Śivadharmasaṃgraha 4) Umāmaheśvarasaṃvāda 5)
Śivopaniṣad 6) Vṛṣasārasaṃgraha 7) Dharmaputrikā 8)
Uttarottaramahāsaṃvāda.

As for each text in this collection, the foliation for the
Vṛṣasārasaṃgraha restarts from f.~1v (f.~1r is a cover) and the text
spans fols.~1v--46r. This is a beautifully written and well-preserved
manuscript which gives very useful readings and has proved to be
essential for the reconstruction of the Vṛṣasārasaṃgraha.

\hypertarget{nk10}{%
\subsection{(N)K10}\label{nk10}}

NGMPP A 10/5, NAK 1/1261. Fully collated for chapters 1--12 of the
critical edition in this volume. See a brief description of this MS in
the NGMCP online catalogue.\footnote{https://catalogue.ngmcp.uni-hamburg.de/receive/aaingmcp\_ngmcpdocument\_00085264}
According to this catalogue, the dimensions of the manuscript are 55 x
5.5cm. Undated. It is a palm-leaf multiple-text manuscript containing 74
folios. Four text are transmitted in this manuscript: 1) Śivadharmottara
2) Umāmaheśvarasaṃvāda 3) Śivopaniṣad 4) Vṛṣasārasaṃgraha.

Some folios feature monochrome drawings. A great number of the leaves
that transmit the VSS are damaged and, at least judging from the
microfilm images, faded and slightly disordered. The folio numbers are
rarely visible. The VSS starts on exp. 44 (upper leaf, no folio number
is visible here). It continues on the lower leaf and then on the upper
leaf on exp. 43 (going backwards) up to 1.62 (\emph{viṃśakoṭiṣu gulmeṣu
ūrdhva°}). Verses 1.62cd--2.22 seem to be missing. The lower leaf on
exp. 43 contains verses 2.23--2.39. The single leaf in exp. 42 contains
verses 2.40--3.16a. Exp. 41 contains a single leaf of the
\emph{Umāmaheśvarasaṃvāda}, ending in a colophon for its chapter
twenty-two, and still going backwards, the preceding folios continue
transmitting the \emph{Umāmaheśvarasaṃvāda}. Exploring the presence of
the VSS in this manuscript further, one should look at the expositions
after no. 44. Exp. 45 contains the end of the \emph{Śivopaniṣad}. The
single leaf on exp. 46 is almost illegible but most probably contains a
fragment of the \emph{Gautamadharmasūtra}. The second line just above
the string hole on the left reads \emph{\ldots{} vīrud vanaspatīnāṃ ca
puṣpāṇi svavad} \emph{ādadīte\ldots{}}, which is a fragment of
\emph{Gautamadharmasūtra} 2.3.25 (12.28). The remaining parts of the VSS
are to be found on exp. 47ff. The upper leaf on exp. 47 continues with
VSS 3.16b-36ab, while the lower leaf contains a text that I have not
been able to identify. The lower leaf in exp. 48 transmits
3.36cd--4.11ab, the upper one 4.11b--30a. The lower leaf in exp. 49
contains 4.30ab--47ab, the upper one 47d--68a. And so on so forth. Thus
when reading the text from these images, after exp. 48, one has to start
with the lower leaf and continue with the upper one.

\hypertarget{nk7}{%
\subsection{(N)K7}\label{nk7}}

NGMPP B 7/3 = A 1082/2, NAK 1/1075. Fully collated for chapters 1--12 of
the critical edition in this volume. See a brief description of this MS
in the NGMCP online catalogue.\footnote{https://catalogue.ngmcp.uni-hamburg.de/receive/aaingmcp\_ngmcpdocument\_00062373}
According to this catalogue, the dimensions of the manuscript are 58cm x
6cm. The script is Nepālākṣara. Dated to NS 290 (1169--70 CE). It is a
palm-leaf multiple-text manuscript containing 289 folios. Eight texts
are transmitted in this manuscript: 1) Śivadharmaśāstra 2)
Śivadharmottara 3) Śivadharmasaṃgraha 4) Umāmaheśvarasaṃvāda 5)
Śivopaniṣad 6) Vṛṣasārasaṃgraha 7) Uttarottaramahāsaṃvāda 8)
Dharmaputrikā.

It is a nicely written manuscript, giving generally good readings. Fols.
209v--264v contain the VSS.

\hypertarget{nk3}{%
\subsection{(N)K3}\label{nk3}}

NGMPP A 3/3 (= A 1081/5), NAK 5-737. Collated only for verses 1.1--15
CHECK. See a brief description of this MS in the NGMCP online
catalogue.\footnote{http://catalogue-old.ngmcp.uni-hamburg.de/mediawiki/index.php/A\_3-3\_\%C5\%9Aivadharma
} According to this catalogue, the dimensions of the manuscript are
58.5cm x 5.5cm. The script is Nepālākṣara. Dated to NS 321 (1200--01
CE). It is a palm-leaf multiple-text manuscript containing 215 folios.
Eight texts are transmitted in this manuscript: 1) Śivadharmaśāstra 2)
Śivadharmottara 3) Śivadharmasaṃgraha missing (only a few folios extant,
e.g. ff.~124 and 143) 4) Umāmaheśvarasaṃvāda 5) Śivopaniṣad 6)
Uttarottaramahāsaṃvāda 7) Vṛṣasārasaṅgraha 8) Dharmaputrikā

VSS starts in f.~227 (image no. 177) and seems to end after it begins
transmitting chapter 23 on f.~264 (image no. 218). The microfilm images
are somewhat blurred and the readings do not seem promising.

-- GOTIT -- NAK 5--738 (NGMPP A 11/3): Palm-leaf, dated to NS 516
(1395--96 CE), 253 folios. Contents: Śivadharmaśāstra (fols.~1v--43r);
Śivadharmottara (fols.~4v--95r); Śivadharmasaṃgraha (fols.~96v--139v);
Umāmaheśvarasaṃvāda (fols.~140v-- 171r); Śivopaniṣad (fols.~172v--189r);
Uttarottaramahāsaṃvāda (fols.~190v-- 211v); Vṛṣasārasaṃgraha
(fols.~212v--257v). For a description of this manuscript, also see the
record in the NGMCP online catalogue: \textless{}http://cata-
logue.ngmcp.uni-hamburg.de/wiki/A\_11- 3\_Śivadharmottara\textgreater{}.

-- GOTIT Kesar 218 BLURRED (NGMPP C 25/1). Palm-leaf, 298 folios.
Contents: Śivadharmaśāstra (fols.~1v--57r); Śivadharmottara (fols.
57v--134v); Śivadharmasaṃgraha (fols.~135r--215v); Umāmaheśvarasaṃvāda
(fols.~216v--255r); Śivopaniṣad (fols. 256v--278r); Umottara°/
Uttarottaramahāsaṃvāda (fols. 279v--299v¤); Vṛṣasārasaṃgraha (?¤--?¤);
(?--?¤).

**** Kesar 537 (NGMPP C 107/7). Paper, dated to NS 803 (1682--83 CE),
174 folios. Contents: Śivadharmasaṃgraha (fols.~89r--133v);
Umāmaheśvarasaṃvāda (fols.~134r--163v); Śivopaniṣad (fols. 164r--181r);
Uttarottaramahāsaṃvāda (fols.~182r--206v); Vṛṣasārasaṃgraha
(fols.~207r--251v); Dharmaputrikā (fols. 252r--262v).

**** Kesar 597 (NGMPP C 57/5). Paper, dated to NS 863 (1742--43 CE), 257
folios. Contents: Śivadharmaśāstra (fols.~1v--41v); Śivadharmottara
(fols.~42v--92r); Śivadharmasaṃgraha (fols. 93v--138v);
Umāmaheśvarasaṃvāda (fols.~139v-- 170v); Śivopaniṣad (fols.~171v--188r);
Uttarottaramahāsaṃvāda (fols.~189v-- 213r); Vṛṣasārasaṃgraha
(fols.~214v--257r).

-- NAK 4--2537 (NGMPP B 219/3). Paper, 339 folios. Contents:
Śivadharmaśāstra (fols.~1v--58r); Śivadharmottara (fols.~59v--123v);
Śivadharmasaṃgraha (fols.~124v--161v); Umāmaheśvarasaṃvāda (fols.
162v--238v); Vṛṣasārasaṃgraha (fols.~239v--338v). GOTIT

-- NAK 4--93 (NGMPP A 1341/6). Paper, 82 folios. Contents:
Śivadharmasaṃgraha (fols.~91r¤--135v); Vṛṣasārasaṃgraha
(fols.~204r¤--243v). GOTIT

-- NAK 4--1604 (NGMPP A 1365/3). Paper, 90 folios. Contents: Śivopaniṣad
(fols.~166v--184r); Uttarottaramahāsaṃvāda (fols.~185v--210r);
Vṛṣasārasaṃgraha (fols.~211v--255r). For a description of this
manuscript, see the record in the NGMCP online catalogue:
\textless{}http://catalogue.ngmcp.uni-hamburg.de/wiki
/A\_1365-3(1)\_Śivopaniṣad\textgreater{} ASK*

\hypertarget{the-kolkata-mss}{%
\section{The Kolkata MSS}\label{the-kolkata-mss}}

Should get hold of the VSS part because Peter B and Kengo says a bit
diff. transmission The Tübingen MS

\begin{verbatim}
1) The Asiatic Society, Kolkata, G. 4076 (only the
Vṛṣasārasaṃgraha, but once part of a larger
corpus) 2) The Asiatic Society, Kolkata, G. 3852
(Śivadharma corpus) 3) The Asiatic Society, Kolkata,
G 4077. Palm leaf, Newari script, dated [Nepāla]
Saṃvat 156 (1035--36 CE). 52 folios
\end{verbatim}

\hypertarget{the-london-ms}{%
\section{The London MS}\label{the-london-ms}}

\href{Manuscripts\%20consulted}{BACK}

-- WI δ 16 (I--VIII). Paper, 406 folios. Contents: Śivadharmaśāstra
(serial no. 634), fols.~1v--63r; Śivadharmottara (s. no. 635),
fols.~64r--143v; Śivadharmasaṃgraha (s. no. 633), fols.~144r--217v;
Umāmaheśvarasaṃvāda (s. no. 652), fols.~218v-- 263v; Śivopaniṣad (s. no.
636), fols.~264r--297v; Uttarottarama-hāsaṃvāda (s. no. 654),
fols.~298r--324r; Vṛṣasārasaṃgraha (s. no. 657), fols.~325r--390r;
Dharmapu- trikā (s. no. 608), fols.~391r--406r. Described in: Dominik
Wujastyk (1985). A Handlist of the Sanskrit and Prakrit Manuscripts in
the Library of the Wellcome Institute for the History of Medicine,
vol.~1. London, The Wellcome Institute for the History of Medicine.
DIRECT COPY of CHECK A82

\hypertarget{the-oxford-ms}{%
\section{The Oxford MS}\label{the-oxford-ms}}

Bodl. Or. B 125{[}? Sansk. a. 15{]}. Palm-leaf, dated to NS 307
(1186--87 CE), 335 folios. Contents: Śivadharmaśāstra (fols.~1v 1--15v1
/ 12r--49v); Śivadharmottara (fols.~50v--113v); Śivadharmasaṃgraha
(fols.~114v--159v); Umāmaheśvarasaṃvāda (fols.~160v--197v); Śivopaniṣad
(fols.~198v--219v); Uttarottaramahāsaṃvāda (fols.~220v--247r);
Vṛṣasārasaṃgraha (fols.~248v--299r); Dharmaputrikā (fols.~300v--312r).

\hypertarget{the-munich-ms}{%
\section{The Munich MS}\label{the-munich-ms}}

Kengo got it in Munich on 16 Nov 2021. VSS starts in 411.jpg `cover'
{[}411.jpg{]}: \textbar{}\textbar{}w\textbar{}\textbar{}
vṛṣasārasaṃgraha 50 patra \textbar{}\textbar{}w\textbar{}\textbar{} Text
starts in 412.jpeg, f.1r Ends on image 455.jpeg Has interesting
readings, but mostly very corrupt and useless. Hand different from that
of some of the other texts in this bundle. Collated chapter 1, will
probably not go on. Gives number of verses in colophons

msM 412.jpg, f.~1r start; ten folios are missing: - f.~5 (VSS 3.4-3.33)*
- ff.~11-13 (VSS 6.20-8.45) - ff.~24 (VSS 13.9-13.36) - ff.~39-43 (VSS
20.38-22.35) *416.jpg lower image is Dharmaputrikā 4.22-39); 417.jpg
upper is Dharmaputrikā 4.39-55 Kengo writes: ``411.jpg forms a cover
that says vṛṣasārasaṅgraha but it is actually 50 verso'' samvat 282?
{[}that would be 1161 CE, or is it 292? = 1171 CE{]} No, maybe 192! see
Kengo's notes! = 1070 CE

\hypertarget{np57}{%
\section{NP57}\label{np57}}

\href{Manuscripts\%20consulted}{BACK}

It is a multiple-text palm-leaf manuscript written in Newari script and
preserved in de la collection Sylvain Lévi à l'Institut d'études
indiennes, collège de France. The manuscript number is: MS. Skt. 57-B.
23. It contains 249 palm leaves, each folio containing six lines. The
following palm leaves are missing: 3, 8, 47, 48, 135, 197, 214 et 216.
Foliation is in the verso: on middle of the lef-hand margin in
combination of Newari letters and in the middle of right-hand in roman
numbers by a second hand. There are two binding holes: one in the centre
lef and one in the centre right. The manuscript is some times damaged in
margins with considerable loos of the text. The text is written in a
clear hand and contains few mistakes. Although it is undated manuscript,
it could be dated to the 11th century AD on the palaeographical
grounds.3 It contains the following text in the order they are presented
in the manuscript: Śivadharmaśāstra (fols.~1--40), Śivadharmottaraśāstra
(fols. 40--93), Śivadharmasaṃgraha (fols.~94--142), Umāmaheśvarasaṃvāda
(fols.~143--172), Śivopaniṣad (fols.~173-- 189), Uttarottaramahāsaṃvāda
(fols. 190--211), Vṛṣasārasaṃgraha (fols.~212--252), Dharmaputrikā
(fols.~253--262). This source contains reliable readings and contains
few scribble mistakes.Śivāśramādhyāya covers fols.~33v4--37r3. Nirajan
says it reads close to Naraharinātha's edition

\hypertarget{naraharinaths-edition}{%
\section{Naraharinath's edition}\label{naraharinaths-edition}}

\begin{itemize}
\item
  De Simini 2016:240 n.~19 Śivadharma MSS:

\begin{verbatim}
  + 1) ASC G 3852 (cat. no. 4085); GOTIT 
  + 2) ASC G 4077 (cat. no. 4084); GOTIT

  + NAK 5--841 (NGMPP B 12/4); GOTIT 
  + NAK 6--7 (NGMPP A 1028/4);  GOTIT NO VṚṢA! 
  + UBT Ma I 582; GOTIT (Tübingen)
\end{verbatim}
\item
  Plus: + Cambridge, Cambridge University Library: Add.1599 no! +
  Add.2836 no! + Or.726.
\end{itemize}

\hypertarget{editorial-policies}{%
\subsubsection{Editorial policies}\label{editorial-policies}}

\begin{verbatim}
- orthography: deviant orth, sandhi, punctuation?
- avagrahas usually supplied but sometimes found in the MSS, not used by me for crasis (e.g. a+a=ā)
- daṇḍas: usually 4 pādas to a verse, but I have made arbitrary decisions based on sense-units 
  because none of the sources really indicate where a verse ends (||).
- falsifications everywhere on purpose and accidentally
\end{verbatim}

SDh MSS from Nepal

\hypertarget{a-critical-edition-of-vss-112}{%
\section{A Critical Edition of VSS
1--12}\label{a-critical-edition-of-vss-112}}

\ldots{} TO BE SUPPLIED

\hypertarget{notes-on-the-constituted-text}{%
\section{Notes on the Constituted
Text}\label{notes-on-the-constituted-text}}

\ldots{} TO BE SUPPLIED

\hypertarget{an-annotated-translation}{%
\section{An Annotated Translation}\label{an-annotated-translation}}

\ldots{} TO BE SUPPLIED

\hypertarget{abbreviations-and-bibliography}{%
\section{Abbreviations and
Bibliography}\label{abbreviations-and-bibliography}}

\begin{itemize}
\tightlist
\item
  CUL = Cambridge University Library 
\end{itemize}

\ldots{} TO BE SUPPLIED

\begin{itemize}
\item
  Balogh 2018? ON THE SAME TOPIC
\item
  Ranjan Sen 2006. `Vowel-weakening before muta cum liquidā sequences in
  Latin. A problem of syllabification?' In: Oxford University Working
  Papers in Linguistics, Philology \& Phonetics 11: 143-61.
\end{itemize}

\end{document}
