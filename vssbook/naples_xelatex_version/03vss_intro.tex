
\mychapter{Introduction}

\thispagestyle{empty}
\frenchspacing

\section{Śivadharma corpus}
\fancyhead[CE]{{\footnotesize \textit{Vṛṣasārasaṃgraha}}}
\fancyhead[CO]{{\footnotesize \textit{Introduction}}}
\fancyhead[LE]{}
\fancyhead[RE]{}
\fancyhead[LO]{}
\fancyhead[RO]{}

The \Vss\ (\VSS), a 24-chapter-long Sanskrit Śaiva text,
has always%
		\footnote{For cases that seem exceptions 
		(\msKoa\ and \msPaperA\ \CHECK\ if more)
						see the manusctipt descriptions
						on pp.~\pageref{mss_descr}ff.}
			%	As already remarked in \mycitep{KissVolume2021}{185, n.~9}, MS G 4076 at the Asiatic Society, Calcutta, while seemingly an independent MS}
been transmitted as part of the so-called Śivadharma corpus,
in multiple-text manuscripts that usually contain 
eight texts.%
	\footnote{Typically, the \SDhS\ (\SDHS), \SDhU\ (\SDHU), \SDhSangr\ (\SDHSANGR),
	\Ums\ (\UMS), \Uums\ (\UUMS), \VSS, \DharmP\ (\DHARMP), and the \SivaUp\ (\SIVAUP).}
By now, much has been written
on the corpus itself and on the individual
texts included. 
For an introduction, an overview of secondary 
literature, an up-to-date bibliography, and the results of 
recent Śivadharma-related research, see \mycite{SivadharmamrtaVolume2021}. 
Since the \VSS's links to other texts of the corpus, 
except possibly the \DharmP, 
are relatively weak, I will refer to other
Śivadharma texts only when they are relevant
for the present inquiry.
%		\footnote{Mainly in section `\CHECK' on p.~\pageref{vss_connection_other_sd_texts}}



\section{Title}\label{title}
The title \Vss%
	\footnote{Read \Vss\ for \titleface{Vṛttasārasaṅgraha}
	in \mycitep{PetechHistory}{84}.}
can be translated as
`Compendium on the Essence of the Bull [of Dharma].'
The last two elements (\skt{sāra-saṃgraha}) need
little explanation: this work is a 
`compendium' on, a `collection' or `summary' of (\skt{saṃgraha})
the `essence' (\skt{sāra}), of its topic, that is, a distilled
version of relevant teachings.
The words `compendium' and `collection' clearly reflect the composite nature of
the \VSS; see details on the structure of the text and
on its possible sources on pp.~\pageref{structure}ff.
The\label{bull} remaining question is whether the bull in the title 
is only a reference to a representation of Dharma 
or if it also hints at Śiva's bull, his vehicle or mount,
sometimes called Nandi or Nandin in other works.%
		\footnote{There is no trace of Nandi/Nandin
		as identified with the bull in the \Vss.
		On the possible time after which 
		Nandi or Nandin, originally a \cskt{gaṇa}{gana},
		was considered a \csindex{bull}, see 
		\mycite{bhattacharya_nandin_1977} and 
		\mycitep{Pancavaranastava}{100--108 and 171--172}.}

Dharma is frequently referred to as a
bull, often depicted as losing a leg in every Kalpa.
This portrayal appears in Dharma literature from at least the time of the \MBh;
see, e.g., \MBH\ 3.188.10--12,%
	\footnote{\skt{kṛte catuṣpāt sakalo nirvyājopādhivarjitaḥ |      \\
	              vṛṣaḥ pratiṣṭhito dharmo manuṣyeṣv abhavat purā || \\
				  adharmapādaviddhas tu tribhir aṃśaiḥ pratiṣṭhitaḥ |\\
				  tretāyāṃ dvāpare 'rdhena vyāmiśro dharma ucyate || \\
				  tribhir aṃśair adharmas tu lokān ākramya tiṣṭhati |\\
   				  caturthāṃśena dharmas tu manuṣyān upatiṣṭhati ||}}
and \Manu\ 1.81a
(\skt{catuṣpāt sakalo dharmaḥ}) and 8.16a
(\skt{vṛṣo hi bhagavān dharma}).%
	 \footnote{See, e.g., \mycite{CoutureDharma}.
	 \citeauthor{GutierrezEmbodiment} (\citeyear{GutierrezEmbodiment})
	 sums the trope up thus
	 (in the section `In animal terms'): 
	 `The emphasis on the whole body, with all four legs, assures 
	 the maintenance of stability in dharma's structure, which in 
	 turn structured Brahmanical society.'}
In addition, in Śaiva contexts, the bull of Dharma
does feature as Śiva's vehicle. See, e.g., the argument in
\mycitep{BakkerWorld2014}{68ff}, especially p.~69,
where, after analysing seals containing images of
bulls, Bakker remarks:


\begin{quote}
The topicality of the Śaiva accommodation 
of the Dharma in the second half of the 
sixth century is nicely illustrated by a myth found in the
original \titleface{Skandapurāṇa}[; \dots]
the uncontrollable, wild bull \ie{vṛṣa} is 
domesticated by Śiva's Gaṇapa Prabhākara [\dots]
In this way the bull is transformed into Śiva’s vehicle \ie{vāhana}. 
\end{quote}

\noindent
To put the same argument more bluntly:

\begin{quote}
Making the bull Śiva's vehicle implies that Śiva has become
the supreme lord of the Dharma, or that the Dharma has 
been accommodated in [Ś]aivism.%
	\footnote{\mycitep{SkandaIIb}{65 n.~210}.
	\citeauthor{bhattacharya_nandin_1977} 		
			(\citeyear{bhattacharya_nandin_1977}, {1552}) 
			suggests that `In the Purāṇas the bull
		(\csindexxx{Vṛṣabha}{vrsabha}{\skt{vṛṣabha}} or 
		\csindexxx{Vṛṣa}{vrsa}{\skt{vṛṣa}}) 
	of Śiva is identified with Dharma, ``virtue personified''. 
		This is a new development to sanctify the animal 
		vehicle of the god. This new situation took place with the 		
		religious rite when an offering of a bull to a Brahmin   
		deemed to be	of a high religious merit.'}
\end{quote}

\noindent
The possibility that the bull in the title \Vss\ refers 
not only to Dharma as a bull, but also
to Śiva's \skt{vāhana} has been mentioned
in \mycitep{UmaSivaPlay}{238 n.~13}, and briefly
discussed in \mycitep{KissVolume2021}{185--186}, 
with the conclusion that although 

\begin{quote} 
[W]hile the bull as a synonym of Dharma is mentioned in the text repeatedly,
somewhat surprisingly, and perhaps significantly, there is no
clear reference to Śiva's mount in the \Vss. [\dots\ Nevertheless, it]
is not inconceivable that the redactors of the \Vss\ had
the same association in mind, namely that the bull in 
question is both Dharma and Śi­va's mount.%
	\footnote{Note that \SDhU\  12.87
				also mentions the `Dharma bull':
	    \skt{īśvarāyatanasyādhaḥ śrīmān dharmavṛṣaḥ sthitaḥ} |
    	\skt{yatra vīravṛṣas tatra kṣityāṃ gomātaraḥ sthitā} ||. }
\end{quote} 

%Śiva got his bull, MBh:
%13076027a vṛṣabhaṃ ca dadau tasmai saha tābhiḥ prajāpatiḥ
%13076027c prasādayām āsa manas tena rudrasya bhārata
%13076028a prītaś cāpi mahādevaś cakāra vṛṣabhaṃ tadā
%13076028c dhvajaṃ ca vāhanaṃ caiva tasmāt sa vṛṣabhadhvajaḥ
%13076029a tato devair mahādevas tadā paśupatiḥ kṛtaḥ
%13076029c īśvaraḥ sa gavāṃ madhye vṛṣāṅka iti cocyate

%MMW `vṛṣa':\\
%``Justice or Virtue personified as a bull or as''Siva's bull Mn. viii,
%16 Pur. Kāvyād.; just or virtuous act, virtue, moral merit ``Siś.
%Vās.;''

%Mahākṣapaṇaka's koṣa (CHECK date), the Anekārthadhvanimañjarī, places
%the meaning `dharma' as first when defining the word `vṛṣa':
%
%\begin{quote}
%    \skt{dharmo vṛṣo vṛṣaḥ śreṣṭho vṛṣo gaur mūṣiko vṛṣaḥ} |\\
%    \skt{vṛṣo balaṃ vṛṣaḥ kāmo vṛṣalo vṛṣa ucyate} || 1.48
%    \end{quote}

% Śivapurāṇa:
% śuddhasphaṭikasaṃkāśo vṛṣabhaḥ sarvasundaraḥ ||
% yo dharma ucyate vedaiḥ śāstraiḥ siddhamaharṣibhiḥ ||54||
% tam ārūḍho mahādevo vṛṣabhaṃ dharmavatsalaḥ||
% śuśubhe 'tīva devarṣisevitaḥ sakalair vrajan ||55||


%visnusmrḍn:ViS 86.15a/ vṛṣo hi bhagavān dharmaś catuṣ-pādaḥ prakīrtitaḥ
%/
%
%Śivapurāṇa 2.3.40.54--55:
%
%\begin{quote}
%\skt{śuddhasphaṭikasaṃkāśo vṛṣabhaḥ sarvasundaraḥ} |\\
%\skt{yo dharma ucyate vedaiḥ śāstraiḥ siddhamaharṣibhiḥ} ||\\
%\skt{tam ārūḍho mahādevo vṛṣabhaṃ dharmavatsalaḥ} |\\
%\skt{śuśubhe 'tīva devarṣisevitaḥ sakalair vrajan} ||
%\end{quote}
%also quoted by \mycitep{bhattacharya_nandin_1977}{1553}	

%smrti/dharma/krtyaratnaakara.dn: !!! dharmo 'yaṃ vṛṣarūpeṇa nāmnā
%nandīśavaro vibhuḥ \textbar{} dharmān māheśvarān vakṣyaty ataḥ prabhṛti
%nārada\textbar{}\textbar{}
%
%tak2015/AtmapujaT55Muktabodha.dn: dharmas tatra vṛṣākāro jñānaḥ
%siṃhasvarūpakaḥ \textbar{} vairāgyaṃ

%On the title, see 
%\mycite{DeSiminiMSSFromNepal2016} (238, n.\thinspace 13):
% `'As noted by \Sanderson\
%{[}\ldots{}{]}, this title can have a double meaning, since the `bull'
%(vṛṣa) is both a synonym of `religious practice' and the traditional
%mount (vāhana) of Śiva. i.e.~Sanderson (Forthc. b), Śaivism and
%Brahmanism. (can't find it)

\noindent
Sanderson (\citeyear{SandersonTolerance2015}, 210 n.~136) has
the following to say on \cskt{vṛṣa}{vrsa} being Dharma
in general, and on the bull appearing on the coins of the 
Hephthalite Hun Mihirakula in particular, also
referencing the \VSS: 

\begin{quote}
To laud the bull (\cskt{vṛṣa}{vrsa}) 
would be surprising if the intended meaning were 
the bull that is Śiva's mount, but not if the word is intended in its figurative meaning, namely \skt{dharmaḥ}, 
or \skt{sukṛtam} `the virtuous actions [prescribed by
the Veda].' For this meaning of \skt{vṛṣaḥ} see, for example,
Amarasiṃha, \skttitle{Nāmaliṅgānuśāsana}{Namalinganusasana} 
1.4.25b (\skt{sukṛtam vṛṣaḥ}),
3.3.220 (\skt{sukṛte vṛṣabhe vṛṣaḥ}); 
Halāyudha,
\skttitle{Abhidhānaratnamālā}{Abhidhanaratnamala} 1.125cd (\skt{dharmaḥ puṇyaṃ vṛṣaḥ śreyaḥ
sukṛtaṃ ca samaṃ smṛtam}); 
\Manu\ 8[.]16a
(\skt{vṛṣo hi bhagavān dharmas}\dots); 
and the Gwalior Museum Stone
Inscription of Pataṅgaśambhu (\mycite{MirashiGwalior1962}), l. 15,
\skt{vṛṣaikaniṣṭho `pi jitasmaro 'pi yaḥ śaṅkaro 'bhūd 
bhuvi ko 'py apūrvvaḥ}, 
concerning the Śaiva ascetic Vyomaśambhu: 
`He was in the
world an extraordinary new Śiva, since he too was 
\skt{vṛṣaikaniṣṭhaḥ}
(`devoted solely to pious observance'; 
in Śiva's case `riding only on the Bull') and he too was 
\skt{jitasmaraḥ} (`one who had defeated sensual
urges'; in Śiva's case `the defeater of the Love god Kāmadeva'). 
This is also the meaning of \skt{vṛṣaḥ} in the title \Vss,
one of the works of the Śivadharma corpus 
(see, e.g., \mycitep{SandersonSaivaLit2014}{p.~2}), i.e., 
`Summary of the Essentials of the [Śiva]dharma'. 
\end{quote}

\noindent
In the last sentence here, \Sanderson\ implies that the
\VSS\ is organically part of the teachings that we
may collectively call the Śivadharma, and he
thus supplies `Śiva' when translating the title 
\Vss. A closer examination of the \VSS\ 
reveals no direct references to either Śiva's bull or
to the bull embodying the Śivadharma. Instead, the bull
in the \VSS\ is repeatedly associated with the Dharma which
is the four \asrama s (see, e.g., \VSS\ 3.1--5 and 4.74).
My conclusion here is that while the word \cskt{vṛṣa}{vrsa} in the
title may indeed refer to Śiva's bull, 
this reference is always implied and never explicitly stated,
whereas the bull as the personification of Dharma as the four
\asrama s appears explicitly and repeatedly. Thus
the title lacks any explicit hint to Śaivism,%
	\footnote{In contrast, see an explicit equation of the bull
	of Dharma with Śiva's mount in the \UUMS\
					(\msCa\ fol.~184r ll. 3--4; see
					\mycitep{KissVolume2021}{185--186}):
		\skt{īśvara uvāca |
  			na jānanti ca loke 'smin mānavā mūḍhacetasaḥ |
 			catuṣpādo bhaved dharmaḥ śuklo 'yaṃ mama vāhanaḥ ||};
		    `Īśvara spoke: In this world, foolish people do not know that
		    the four-legged Dharma is this bright mount of mine.'}
which aligns well with the text's blurred and multi-layered
affiliation of the text to Dharmaśāstra, Vaiṣṇavism, and Śaivism.%
		 \footnote{See p.~\pageref{structure}.}
%		\footnote{See also \mycitep{BakkerWorld2014}{69}, who 
%			while discussing a seal of Śarvavarman that 
%			features a beautifully carved bull representing Dharma,
%			remarks (italics mine): `The reader \textit{may} also see in the 
%			image the thriving Śaiva religion, represented
%			by the Bull, the vāhana of Śiva [\dots]'} 




%\mysubsubsection{Vṛṣadeva's commission?}{vrsadevas-commission}
Finally, as a fanciful experiment, and if one accepts 
that the \VSS\ originated in Nepal,%
		\footnote{See pp.~\pageref{provenance}ff.}
one could wonder if the title \Vss\ 
has anything to do with the Licchavī king Vṛṣadeva.
\citeauthor{SandersonSaivaAge} 
(\citeyear{SandersonSaivaAge}, 74) mentions that  
Vṛṣadeva is `described in an inscription of his eighth-century 
descendant Jayadeva as having inclined towards Buddhism;%
			 \footnote{See \mycitep{Vajracarya1973}{148, l. 9}: 
					\skt{sugataśāsanapakṣapātī}.}
a view confirmed by a local chronicle, which attributes to
him the establishing of Buddhist images,'
%fn: Lévi 1990, vol. 2, p. 98.
and that this king established 
`the Caitya of the Sı̄nagu-vihāra (the Svayambhūnāth Caitya).'
More importantly, Sanderson summarises the 
information found in the Changu Narayana Pillar Inscription (east shaft),%
		\footnote{\mycitep{GnoliNepInscr}{1},  and 
		https://siddham.network/inscription/in02001/} 
noting that Vṛṣadeva was the great-grandfather of Mānadeva, whose
`dated inscriptions range in date from 459 to 505/6' [\CE]
(\mycitep{SandersonSaivaAge}{75}).
%	   \footnote{Vṛṣadeva was succeeded by Śaṅkaradeva and   		            Dharmadeva.}
This would place 
the reign of Vṛṣadeva around 400 \CE. 
The early fifth century may look too early for the date of composition
of the \VSS, and any connection between this king
and the text is impossible to prove at the moment.
However, it is equally impossible to dismiss it entirely.
If such a connection exists, 
it might explain the slightly unusual nature of the title (`\dots\ the essence of the bull').
\hide{
%https://siddham.network/inscription/in02087/
%https://siddham.network/inscription/in02087/?section=translation
%https://siddham.network/inscription/in02002/
%https://siddham.network/inscription/in02001/
%https://infogalactic.com/info/Licchavi_(kingdom):
}

%
%Gopālarājavaṃśāvalī p. 124 Dharmadeva and a vṛṣa statue? Text mentions vṛṣadhvaja though...
%
%Pañcāvaraṇastava 71:
%pratyag āśāsthitaṃ vande vṛṣaṃ ca vṛṣabhākṛtim|
%sākṣād dharmaṃ sitaṃ tryakṣaṃ parameśasya vāhanam||
%+ notes to this verse on p. 171







\section{Genre}

Some texts of the Śivadharma corpus have been recognised as Purāṇas 
or Upapurāṇas at certain points in their textual history 
(see, e.g., \mycite{HazraSDh} and \citeyear{HazraSDhU}).
Could the \VSS\ be considered a Purāṇa? There are at least two reasons to support this idea.
One is the section spanning \VSS\ 1.62--75, which provides a list of so-called \skt{vedavyāsa}s,
transmitters of Purāṇas, from Brahmā to Vyāsa Dvaipāyana, Romaharṣa,
and his son.
Why would a text include such a list in its first chapter 
if not to suggest that it is describing its own origins?

Another argument is that the topics dealt with in the \VSS\ are exactly what
we expect from a Purāṇa. The famous \skt{purāṇapañcalakṣaṇa} includes,
following Wilson's translation (see \mycitep{RocherPuranas1986}{26}), the following:
(1) primary creation, cosmogony and chronology (\skt{sarga}); 
(2) creation, destruction of the world (\skt{pratisarga});
(3) genealogies (\skt{vaṃśa}); 
(4) Manu eras (\skt{manvantara}s);
(5) history (\skt{vaṃśānucarita}).%
		\footnote{See, e.g., \SIVP\ 7.1.41: 
				\skt{sargaś ca pratisargaś ca 
						vaṃśo manvantarāṇi ca |
                        vaṃśānucaritaṃ caiva 
                        purāṇaṃ paṃcalakṣaṇam} ||.}
Arguably, all of these elements are present in the \VSS,
with most appearing in chapter one, and again in chapters twenty-one and
twenty-four, along with narratives of the deeds of gods
(e.g. in chapter twenty-three), and more. It is possible
that certain sections of the \VSS\ were originally intended
to form a separate \skt{purāṇa}. The part in question could
be the outermost layer of the text (see pp.~\pageref{structure}ff).

%This leads us to the examination of the structure of the \VSS.

%Hazra. \verify\ Brahmāṇḍapurāṇa is similar \verify

Could the VSS alternatively be classified as a Dharmaśāstra?
The \VSS\ does contain features characteristic of Dharmaśāstric texts,
such as descriptions of rules of conduct (chapters 3--8) and discussions of the 
\skt{varṇa}s and \skt{āśrama}s (chapters 11 and 19).
However, other elements---such as narratives (chapter 12),
yogic teachings (chapter 16), lists of \skt{tīrtha}s (chapter 10), 
and the frequent use of poetic metres (e.g. \skt{upajāti} and
\skt{śārdūlavikrīḍita})---are less obviously Dharmaśāstric.

\Fol251v of paper MS \msPaperA\ includes a scribal addition 
that provides a richer and more nuanced definition of the genre of the \VSS, 
paraphrasing \MBh\ 1.56.21:%
        \footnote{\MBh\ 1.56.21 reads:
                 \skt{arthaśāstram idaṃ puṇyaṃ dharmaśāstram idaṃ param~|
                      mokṣaśāstram idaṃ proktaṃ vyāsenāmitabuddhinā ||}. 
                      The parallel between the scribal verses in \msPaperA\ and the 
                      \MBH\ has already been noted in 
                      \mycitep{DeSiminiMSSFromNepal2016}{253 n.~51}.}

\begin{quote}
\skt{pādam ādyam}% 
		\footnote{Understand \skt{pādamātram}?}
							\skt{idaṃ śāstraṃ yo 'dhīyīta jitendriyaḥ |}\\
\skt{tenādhītaṃ sarvvadharmmam iti nāsty atra saṃśayaḥ ||}\\
\skt{arthaśāstram idaṃ puṇyaṃ dharmmaśāstram idaṃ paraṃ |}\\
\skt{mokṣaśāstram idaṃ proktaṃ śivenāmitatejasā ||}

Should someone read [only as much as] the first \skt{pāda} [of]
this \skt{śāstra} with his senses subdued, [it would count as if]
they had read all the Dharmi[c teachings]. There is no doubt about this.
This virtuous Arthaśāstra, this excellent Dharmaśāstra, 
this \skt{śāstra} on Liberation was taught by Śiva, whose splendour is
immeasurable.
\end{quote}

\noindent
According to this definition, the \VSS\ is both an Arthaśāstra and a
Dharmaśāstra, and also a yogic text offering instructions on \skt{mokṣa}.
One could cautiously characterise the \VSS\ as a heterogeneous
text containing Dharmaśāstric, Purāṇic, yogic, and narrative elements,
similar to its starting point and model, the \MBh\
(see the summary of \VSS\ chapter 1 on p.~\pageref{contents_of_ch01}).




\section{Structure}\label{structure}

As described in more detail in \mycite{KissVolume2021},
the \VSS\ contains at least three discernible structural layers:
a general Dharmaśāstric layer; a more or less Vaiṣṇa\-va layer;
and a Śaiva layer. Figure \ref{fig:struct2021} is
a diagramme reproduced from 
\mycitep{KissVolume2021}{188}
showing the textual divisions more precisely.

\begin{figure}
\begin{tikzpicture}
\path  (0,0) coordinate(A);
\draw [scale=1,shift={(0,0)}] (-.9,.6) arc [start angle=160, end angle=20, radius=2];
\draw [scale=1,shift={(0,0)}] (-.2,.5) arc [start angle=160, end angle=20, radius=1.3];
\draw [scale=1,shift={(0,0)}] (-.2,-.5) arc [start angle=200, end angle=340, radius=1.3];
\draw [scale=1,shift={(0,0)}] (-.9,-.6) arc [start angle=200, end angle=340, radius=2];
% circles
%\draw [scale=1,shift={(0,0)}] (1,0) circle (2);
%\draw [scale=1,shift={(0,0)}] (1,0) circle (1.3);
\draw [scale=1,shift={(0,0)}] (1,0) circle (.6);
\draw[decoration={text along path, text={General Dharma{ś}{ā}stric},raise=.8,text align=center}, decorate] (-0.5,0) arc [start angle=180,end angle=0,radius=1.5];
\draw[decoration={text along path, text={General Dharma{ś}{ā}stric},raise=.8,text align=center}, decorate] (2.5,0) arc [start angle=360,end angle=181,radius=1.5];
\draw[decoration={text along path, text={Vai{ṣ}{ṇ}ava},raise=.8,text align=center}, decorate] (0,-.2) arc [start angle=180,end angle=0,radius=1];
\draw[decoration={text along path, text={Vai{ṣ}{ṇ}ava},raise=-.5,text align=center}, decorate] (2,.1) arc [start angle=360,end angle=180,radius=1];
\node at (1,0) {{Ś}aiva};

        % arrows
        \draw [->, line width=0.25mm] (2,1.5) -- (5,1.5);
        \draw [->, line width=0.25mm] (1.9,.4) -- (5,0.4);
        \draw [->, line width=0.25mm] (1.2,-0.3) -- (5,-0.3);
        \draw [->, line width=0.25mm] (1.55,-1) -- (5,-1);
        \draw [->, line width=0.25mm] (1.25,-1.8) -- (5,-1.8);

\node[text width=5cm] at (8,1.5)  {verses 1.1--8};
\node[text width=5cm] at (8,0.4)  {verses 1.9--10.3};
\node[text width=5cm] at (8,-0.3) {verses 10.4--18.46};
\node[text width=5cm] at (8,-1)   {verses 19.1--21.29};
\node[text width=5cm] at (8,-1.8) {verses 21.30--24.83};

%vertical lines
        %\draw [line width=0.5mm] (-1,0) -- (0.4,0); 
        %\draw [line width=0.5mm] (1.6,0) -- (3,0); 

% \path  (0,0) coordinate(A); \draw [scale=1,shift={(0,0)}] (6,0) circle (2); \draw [scale=1,shift={(0,0)}] (6,0) circle (1.3); \draw [scale=1,shift={(0,0)}] (6,0) circle (.6); \draw[decoration={text along path, text={As a legendary yogin},raise=.8,text align=center}, decorate] (4.5,0) arc [start angle=180,end angle=0,radius=1.5]; \draw[decoration={text along path, text={As an interlocutor},raise=.8,text align=center}, decorate] (5,-.2) arc [start angle=180,end angle=0,radius=1]; \node at (6,0) {As sacrifice};

\end{tikzpicture}
\caption[Structure of the \VSS]{The structure of the \VSS\ (reproduced from \mycitep{KissVolume2021}{188})\label{fig:struct2021}}
\end{figure}

Each layer is characterised by a dialogue between
two interlocutors. The layer that I label general 
Dharmaśāstric is a dialogue between king Janamejaya and
Vaiśampāyana; the Vaiṣṇava layer is presented as
a dialogue between Vigatarāga, who is
Viṣṇu in disguise, and Anarthayajña,\label{anarthayajna_person} the ascetic;
the Śaiva layer is a dialogue between Śiva and Devī,
as related by Nandikeśvara. The transitions between the
layers are smooth, that is to say, Nandikeśvara's narrative
is mentioned, introduced, and told by Anarthayajña, whose dialogue
with Vigatarāga is in turn narrated to Janamejaya by Vaiśampāyana.

Another way to represent the overall structure of the \VSS\
visually is shown by Figure \ref{fig:structlotus} 
on p.~\pageref{fig:structlotus}. 
The \VSS\ is represented
as a lotus whose petals represent chapters. White petals indicate chapters within
the general Dharmaśāstric layer; light grey colour
indicates the Vaiṣṇava layer; dark grey colour indicates
Śaiva chapters. The divisions are not clear-cut: 
the first few verses of chapter one belong to
the general layer, and transitions also occur within chapters.
Additionally, the layers are not hermetically
sealed, and there is some `leaking' between the chapters.
Śaiva chapters may contain Vaiṣṇava material, and vice versa.
The labels beside the petals represent keywords
indicating the main topics of each chapter.
Big check marks indicate the presence of Anarthayajña the ascetic in
the given chapter, while smaller check marks indicate
references in the given chapters to Anarthayajña's
ascetic practice repeatedly called \skt{anartha-yajña}, 
i.e.\ `non-material\thinspace /\thinspace internali\-sed sacrifice/worship.'\label{nonmaterial}
Anarthayajña in both senses seems to be one of the 
main foci of the \VSS.

The main theme of the Dharmaśāstric layer is Janamejaya's desire 
to hear the condensed and ultimate Dharmic teachings of the \MBh\ from Vaiśampāyana.
A brief overview of the Vaiṣṇava chapters would be the following.
An\-artha\-yajña, a Vaiṣṇava ascetic, who propagates
a system of internalised \skt{āśrama}s\thinspace /\thinspace
a system beyond the traditional \skt{āśrama}s, 
and who was born into an obscure or fluid \skt{varṇa} 
(\skt{brāhmaṇa\thinspace /\thinspace kṣatriya}),
is being tested by Viṣṇu; he passes the test
and follows Viṣṇu to Viṣṇuloka.
The Śaiva layer is a collection of chapters addressing internalised
pilgrimage places, relating a tale on donating a wife to a Brahmin,
embryology, karma, the \skt{jīva}, yoga, and more.

Another general observation is that roughly one-fourth 
of the text elaborates on rules of religious conduct
\ie{yama-niyama}. Also, chapter two seems slightly
out of place, being a clearly Śaiva chapter inserted
into the Vaiṣṇava layer and in the corresponding 
dialogue of the Vai\-ṣṇa\-va interlocutors.
It is not inconceivable that the Śaiva layer, which
contains a teaching on non-material sacrifice
(\skt{vinārthena tu yo yajñaḥ}, \VSS\ 11.5a)
is the oldest part of the \VSS. The Vaiṣṇava layer may have 
been developed later, with the legend of Anarthayajña
constructed around that concept and phrase.



%{\Huge\textsc{The V\textsubring{r}ṣasārasaṀgraha: Structure}}\hfill Csaba Kiss, 2 Mar 2022, Dharma Workshop, Berlin
\begin{figure}[!]

\begin{center}
\thispagestyle{empty}
\vspace{0em}
\leftskip1em
\begin{tikzpicture}[scale=.6]
%\bigcircle
\drawpetalswitharrows  
\colorpetalsOne
\drawtopicsTwo
\textTwo
\end{tikzpicture}
\end{center}

\caption[Structure and topics of the \VSS]{The structure and topics of the \VSS\ 
   \label{fig:structlotus}}
   
\end{figure}



\section{Connection to other texts}
\label{vss_connection_other_texts}

The \VSS's indebtedness to the \MBh\ (\MBH) is evident 
from its very first verses. As already noted,
the frame story in the \VSS\  comprises

\begin{quote}
a dialogue between Janamejaya and Vaiśampāyana, 
echoing the setting of the frame story of the \MBh. 
Janamejaya is the king at whose snake-sacrifice 
Vaiśampāyana recited the whole \MBh\ for the first
time. This important moment is where the frame story 
of the \Vss\ takes off: Janamejaya has 
listened to the entire \MBh,
but having had the desire to hear the ultimate 
teaching on Dharma, he is bound to remain unsatisfied.
Asked by Janamejaya for a higher teaching
on Dharma which can lead to liberation, 
Vaiśampāyana relates a dialogue between Vigatarāga 
(in fact Viṣṇu in disguise) and Anarthayajña, an ascet­ic.%
				\footnote{\mycitep{KissVolume2021}{187}}
\end{quote}
 
\noindent
Thus the frame story in the \VSS\ suggests
that the text is to be ideally read as a summary 
or higher synthesis of the Dharmic teachings found
in the \MBH. The \VSS's connection to the \MBH\
is also evident from quotations from and paraphrases
of \MBH\ passages. EXAMPLES (tattvasystem).
  References to other works - Mahābhārata - nakule - vipule etc.
MBh VSS 8.21
BhG 17.16 and 15 and 14: VSS 6.20--22

VSS 9.40--42
	
Moreover, a significant number of passages in 
the \VSS\ derive from Purāṇas and from \Manu. EXAMPLES.

Manu: \VSS\ 4.77--81, 5.8--9, 5.13ab, 5.14ab

The possibility of influence from Śaiva tantric works is
minimal, but not to be excluded. EXAMPLES.
Niśvāsakārikā



Śivadharma texts:
\label{vss_connection_other_sd_texts}


Embryology

yoga \DharmP\ see below
Dhyāna in the \VSS\ and the \DHARMP
\label{dharmaputrika}

Compare, borrowings

Buddhacarita 

Bṛhatkālottara,

Skanda







\section{Dating and provenance}
\label{provenance}
There are several reasons to believe that
Nepal, specifically the Kathmandu valley, is the most
likely location for the composition
or final redaction of the \VSS. 
The most probable period for this composition is 
the first half of the poorly documented `transitional period'%
	\footnote{\mycitep{PetechHistory}{31}}
in the history of Nepal. 
This is a `relatively obscure period [\dots]
[b]etween the Licchavis, who last appear 
in epigraphical record in 737 [\CE], 
and the Malla kings, who ruled from 1200--1768.'%
	\footnote{\mycitep{SandersonSaivaAge}{77}}

To support these assumptions, we can consider the following:
the location of the manuscript evidence;
place names and individuals mentioned;
and a possible influence of any local language on
the style and grammar of the text.

All MSS known to us that transmit the \VSS\
hail from Nepal. This in itself is not strong evidence
but it stands in stark contrast with the MSS situation
of the \SDHS\ and the \SDHU.%
		\footnote{See, e.g., \mycitep{UmaSivaPlay}{589}.}

The geographical locations 
mentioned in the \VSS\ include the
\skt{tīrtha}s mentioned in chapter ten:
Himavat (the Himālayas),
Kurukṣetra,
Prayāga,
Vārāṇasī,
Yamunā,
Gaṅgā,
Agnitīrtha, % north
Somatīrtha, % north
Sūryatīrtha, % north
Puṣkara, % north
Mānasa, % north
Naimiṣa, % north
Bindusāra (= Bindusaras), % north
Setubandha, % Śrīlaṅkā ?
Suradraha or Sura\-hrada, % tīrthaśreṣṭhaḥ suradrahaḥ 15.18d and it is then in the heart ! could be key
Ghaṇṭikeśvara,
and Vāgīśa.
These may hint at the area where the \VSS\ was
composed by giving more significance to 
locations nearby and by being more specific when
mentioning local sacred places.
Some names on the list above are 
easy to indentify and at the same
time probably too often included in lists such as this one
to be indicative enough: 
Himavat, Kurukṣetra,%
	\footnote{Generally thought to be the area
		around Thaneswar\thinspace /\thinspace
		Thanesar (\mycitep{DeyGeography}{45}), 
		160km northwest of Delhi.}	
Prayāga, Vārāṇasī, Yamunā,
Gaṅgā, Puṣkara (modern Pushkar), and Naimiṣa.%
		\footnote{\mycitep{BisschopEarly}{217}: 
		`Naimiśa has been identified with the region around 
		modern Nimsar on the Gomatī river in Uttar Pradesh
		(SP vol. I, p.~67, n.~23).
		This identification is doubted by Mirashi (\citeyear{MirashiNaimisa}).'}
All these are locations in modern North India, or in the case of Himavat,
for our purposes and more precisely, in North India and Nepal.
Agnitīrtha, Somatīrtha and Sūryatīrtha could also
be locations in North India, although they are
more obscure than the ones above.
For Agnitīrtha, see, e.g., \PadmaP\ 3.45.27ab:      
\skt{agnitīrtham iti khyātaṃ yamunādakṣiṇe taṭe}; and  
\PadmaP\  6.139.1ab:    
\skt{sābhramaty\-uttare kūle agnitīrtham iti śrutam}; 
therefore Agnitīrtha may be placed at the southern
banks of the Yamunā or at the northern banks of
the Sābhramatī river (modern Sabarati) in the area of
Ahmedabad.
Somatīrtha is also sometimes placed on the banks
of the Sabarmati, see, e.g., \PadmaP\  6.161.1ab:
\skt{somatīrthaṃ tato gacched guptaṃ sābhramatītaṭe}.
Sūryatīrtha is sometimes placed in Kurukṣetra.%
		\footnote{See 
		\mycitep{PuranicEnc}{s.v. `\skt{sūryatīrtha}.'}}
Going further in the list, Mānasa is generally thought
to be `[a] lake on the peak of the Himālayas',%, 
	\footnote{\mycitep{PuranicEnc}{s.v. `\skt{mānasa} IV.'}}
modern Manasarovar.%
	\footnote{\mycitep{DeyGeography}{57}.} 
Bindusāra, which most probably stands for
Bindusaras, can be a sacred place north of Mount
Kailāsa,%
	\footnote{\mycitep{PuranicEnc}{s.v. `\skt{bindusaras}.'}}
two miles south of Gaṅgotri,%
	\footnote{\mycitep{DeyGeography}{11}.}
or alternatively Sitpur in Gujarat, north-west of 
Ahmedabad.%
	\footnote{\mycitep{DeyGeography}{ibid.}.}
	
In contrast with these, Setubandha is the
traditional name for the ridge of rocks between
South India and Śrī Laṅkā, and Ghaṇṭikeśvara could
be a sacred place in Orissa.%
	\footnote{\mycitep{SandersonSaivaAge}{113 n.~241}.}
Vāgīśa seems difficult to locate, but it is most probably
a sacred place east of Kathmandu. The name comes
up in \NepMah\ 3.21--25 as a location south of and 
not far from the Hanumadīśvara-liṅga,
which is in the southern outskirts of 
Bhaktapur in Nepal, at the confluence
of two rivers (according to 
\mycitep{AcharyaNepalaMahatmya}{37--38 and 298}):

\begin{quote}
\skt{kiṃciddūre saṅgamasya yajñabhūmiṃ manoharām} |\\
\skt{vidhāya munibhir sārddhaṃ vājapeyam athākarot} ||\\
\skt{yajñaṃ samāpya vālmīkir navanāḍīmayaṃ girim} |\\
\skt{āruroha dvijaśreṣṭho munibhir munisattamaḥ} ||\\
\skt{kaṭake tasya śailasya nānānirjharaśobhite} |\\
\skt{liṅgaṃ saṃsthāpayām āsa vālmīkīśvarasaṃjñitam} ||\\
\skt{sthāpayitvā mahāliṅgaṃ vālmīkir munisattamaḥ} |\\
\skt{svāśrame tamasātīre yayau munigaṇair vṛtaḥ} ||\\
\skt{vālmīkīśvaram ālokya vāgvibhūtiḥ prajāyate} |\\
\skt{ato vāgīśvaraṃ liṅga pravadanti manīṣiṇaḥ} ||

Not far from the confluence, [Vālmīki] prepared a nice
sacrificial ground together with the sages, and 
performed the Vājapeya sacrifice. After having
completed the sacrifice, Vālmīki, the best of
the twice-born, the truest of sages, climbed 
the mountain on which fresh grass was growing(?)%
	\footnote{\skt{navanāḍīmayaṃ}. Emend to 
	\skt{navanadīmayaṃ} (`having nine rivers')?}
together with the sages. In a valley of the mountain
which was embellished with various waterfalls, he
installed a \skt{liṅga} called Vālmīkīśvara. Having 
installed that great \skt{liṅga}, Vālmīki, the truest of
sages, surrounded by groups of sages,
returned to his own \skt{āśrama} on the banks
of the river Tamasā. If one sees the Vālmīkīśvara [\skt{liṅga}],
one will have the power of speech \ie{vāg-vibhūti}. 
That is why the wise call it the Vāgīśvara-liṅga.
\end{quote}

\noindent
I have reproduced a map from the beginning of 
\mycite{AcharyaNepalaMahatmya} as Figure~\ref{fig:map02}
on p.~\pageref{fig:map02} below. On this map, Vāgīśvara is placed north-east 
of Bhaktapur. 

The only toponym left from the list with which
we started this section is Suradraha.%
	\footnote{Always spelt \skt{surahrada} in Naraharinath's
			edition.}
This sacred place is mentioned as the most important
one in \VSS\ 18.15, in a chapter that lists personal names,
placenames, rivers, deities, etc., that are considered the
best \ie{śreṣṭha} of all others in the same category:

\begin{quote}
\skt{devatānāṃ hariḥ śreṣṭhaḥ śreṣṭhā gaṅgā nadīṣu ca} |\\
\skt{anāśanas tapaḥśreṣṭhas tīrthaśreṣṭhaḥ suradrahaḥ} || 18.15 

The best god is Hari.%
		\footnote{This is so, somewhat confusingly, still in the
		layer that I call Śaiva.}
The best river is the Ganges.
The best austerity is fasting. The best pilgrimage-place is Suradraha.
\end{quote}

\noindent
This suggests that the location of Suradraha could potentially
give us a hint on the geographic centre of 
the community in which the \VSS\ was commissioned
or composed. Unfortunately, up to this moment, I have
not been able to obtain any useful information on Suradraha. Nevertheless,
I suppose that it should be  a sacred place in the 
Kathmandu valley. The word \skt{draha} is attested in
Monier-Williams' Sanskrit-English Dictionary 
as a variant of \skt{hrada} (`pond').
In classical Newar the corresponding form is \skt{daha}
(\mycitep{MallaDict}{218}). Suradaha may stand for Sūradaha,
that is Sūryadaha, which is a `pond situated in Devakuru' 
according to \citeauthor{PrakritProperNames2} (\citeyear{PrakritProperNames2}, 850).%
	\footnote{The references given are the Jaina 
			\titleface{Jambūdvīpaprajñapti} and \titleface{Sthānāṅga\-sūtra}.} 
In fact, Sūryahrada, that is, Sūradaha, is one of the 
so-called \skt{yamaka}-lakes in Devakuru in the North
in Jaina cosmography (see \mycitep{KirfelKosmo}{235--236}).
 % Taudaha pond near Kathmandu?
 
All the above is based on \VSS\ chapter ten. All locations have
 been entered in the map which is Figure \ref{fig:map01} on
 p.~\pageref{fig:map01}.
 The impression one gets is clearly a north-Indian or
 Nepalese focus.
 
 Chapter twelve also contains toponyms that could
 refine or contradict what we have seen so far.
 The narrative of this chapter starts in Kusumanagara
 at the confluence of the Gaṅgā and the Gaṇḍakī rivers, 
 which is Pāṭali\-putra (12.4 and 12). As the story
 develops, Vipula, our hero, departs from Ku\-su\-ma\-na\-gara
 to travel to a far-away land, which is identified by a 
 fellow traveller as the city of Naravīrapura in the 
 Deccan (12.60).
\label{naravirapura} It is difficult to find a Naravīrapura that would fit
 the context. I suspect that what could have been meant
 is Karavīrapura, possibly modern Kolhapur
 in Maharashtra.%
	\footnote{The city we are looking for is clearly in
						the South, therefore Karavīrapura as
						`the Pīṭha of the North' in Kashmir is
						not a good candidate (see, e.g., 
						\mycitep{SandersonExegesis}{261}). Rather,
						as \citeauthor{DeyGeography} 
						(\citeyear{DeyGeography}, {35}) puts it, it is
						`[a] town situated on the north of the
						Western Gh\^ats near Jooner [Junnar?], 
						on the bank of the Vená [Venna], 
						a branch of the Krishná,
						where Krishna met Parasuráma and
						killed its king named Srigála (\textit{Harivansa)}.'
						See \Harivamsa\ App.~I. 18.352--355:\\
						\skt{pūrvajais tava govinda
									 pūrvaṃ puram idaṃ kṛtam} |\\
						\skt{karavīrapuraṃ nāma  
									rāṣṭraṃ caiva niveśitam} ||\\
						\skt{pure 'smin nṛpatiḥ kṛṣṇa  
									vāsudevo mahāyaśāḥ} |\\
						\skt{sṛgāla iti vikhyāto  
								nityaṃ paramakopanaḥ} ||\\
					See also \PadmaP\ 6.106.3:\\
						\skt{āsīt sahyādriviṣaye 
									karavīrapure purā |}  \\
					    \skt{brāhmaṇo dharmavit kaścid 
					    			dharmadatto 'tiviśrutaḥ} ||}
Since this placename, and 
the Sahya mountains (12.93),%
 	\footnote{`The northern part of the Western Gháts
north of the river Káveri' (\mycitep{DeyGeography}{78}).}
come up in the framework of a dreamlike, fanciful part of
the narrative, playing the role of `the far-away, magical
land,' a Nepalese origin of the \VSS\ is still tenable.%
		\footnote{On the area of the Sahya mountain as `the southernmost limit of the authors' map'
                in the `the Skandapurāṇa's literary imagining of a Pāśupata landscape,' see 
                \mycitep{CecilMapping}{161ff}.}
 
Perhaps the most telling of all toponyms found
in the \VSS\ is Mṛgendraśikhara,
where Anarthayajña's \skt{āśrama} is situated,
`on the southern slopes of
					the Himalayas.'%
		\footnote{\label{mrgendrasikhara}See \VSS\ 22.4--5:\\
     		\skt{vaiśampāyana uvāca~|\\
			śṛṇu rājann avahito yogendrasya mahātmanaḥ~|\\
		    āśramaṃ varṇajātīnāṃ vakṣyāmy eva narādhipa~||\\
			himavaddakṣiṇe pārśve mṛgendraśikhare nṛpa~|\\
			mahendrapathagānāmanadītīre narādhipa}~||\\
			`Vaiśampāyana spoke: Listen, O King, attentively.
		  	I shall tell you about the \skt{āśrama}, 
		  	the \skt{varṇa}, and the \skt{jāti} of the
		  	great and noble yogin, O king.
		  	In the southern region of the Himālaya, 
		  	on the Mṛgendra peak, O king,
		  	on the banks of the river Mahendrapathaga, O King[,
		  	was his \skt{āśrama}]'.}
This name comes up
several times in the \NepMah\ and thus features on
the map in \mycite{AcharyaNepalaMahatmya}
(Figure~\ref{fig:map02}). Mṛgendraśikhara is a mountain
situated north of Kathmandu. Today the area is
called Śivapurī. See details on the identification and
on legends connected to Mṛgendraśikhara in
\mycitep{GoggeVisnuKathmandu}{114ff}. The \VSS\
specifies that Anarthayajña's \skt{āśrama} was
on the banks of the Mahendrapathaga,%
		 \footnote{See fn.~\ref{mrgendrasikhara}.}
but I have not been able to identify this river.



\begin{figure}[!]
\leftskip-6em\includegraphics[scale=.4]{images/simplemap.png}
\caption[Geography of the \VSS]{A possible reconstruction of the  geography of the \VSS. Toponyms in italics are uncertain. Map constructed using a simple hydrographic map made by Daniel Dalet (d-maps.com).\label{fig:map01}}
\end{figure}

\begin{figure}[!]
\includegraphics[scale=.43]{images/map_in_jayaraj.png}
\caption[Map in \mycite{AcharyaNepalaMahatmya}]{Map in \mycite{AcharyaNepalaMahatmya}
\label{fig:map02}}
\end{figure}


The location with which the ascetic Anarthayajña
is connected strongly suggests the Kathmandu 
valley as the geographical focus of the \VSS\
because he is a key figure and 
main interlocutor in the \VSS,
possibly the reason behind the composition of the text.%
	\footnote{On Anarthayajña's central role in the \VSS,
			see more in \mycite{KissVolume2021}.}


Turning to names of individuals mentioned in the \VSS,
those that might betray anything about the place or
time of composition of the text include King Siṃhajaṭa
and queen Kekayī, rulers of Nara- or Karavīrapura
in the narrative of chapter twelve. Unfortunately,
so far I have not been able to link these names to
any historical or legendary persons. The name of the
hero of the same chapter, Vipula,\label{Vipula} may be familiar 
from \MBH\ 13.40.16--13.43.16.: 

\begin{quote}
Devaśarman asks his disciple,
Vipula, to protect his wife, Ruci, primarily from Indra's
amorous advances, while he is away from home.
Vipula decides that the only way he can protect Ruci
is from within, i.e., by entering her body by yogic powers.
Vipula succeeds in protecting Ruci's reputation and 
departs to practise extreme austerities. Later he 
encounters several people (in fact,
as we learn later, Day and Night,
and the six seasons) who mention `Vipula's path leading to
the other world' (\skt{vipulasya pare loke yā gatis}, 
\MBH\ 13.42.27cd) as something horrible. He 
wonders what sins he may have committed that
could yield such unfortunate consequences. He
realizes that by not telling Devaśarman that he
actually entered Ruci's body, he lied and thus
may have commited a horrible sin. When Devaśarman learns
about this, he praises Vipula for his services instead, 
and all three, Devaśarman, his wife, and Vipula,
go to heaven.%
		\footnote{See a summary of Vipula's story in the 
			\MBH\ also in 
			\mycitep{SukthankarCriticalStudies}{317--318}.}
%(\MBH\ 13.43.16)
\end{quote}

\noindent
Thus, ironically, while the Vipula of the \MBH\ is famous
for protecting somebody else's wife,  
a rather different Vipula
in \VSS\ chapter twelve donates
his own wife to a Brahmin as soon as the latter expresses
interest in her. It is more than possible that
the two characters have no connection at all.%
	\footnote{Nevertheless, see the word \skt{vipule} used
	in \VSS\ 12.155b potentially referring to the famous
	story in the \MBh.}

Other characters in \VSS\ chapter twelve---Kapila, 
Vipula's father;
Bhīmabala, a traveller; Puṇḍaka, the foreman of the guild;
and Caṇḍa and Vicaṇḍa, two royal envoys---seem 
to be of little use for us to ascertain the time and place of composition or redaction of the \VSS. 

Going further, as mentioned above, any discernible influence
of a local, vernacular language on the style or grammar of
a Sanskrit work could also be useful to
locate the text in question geographically. 
The language of the \VSS\
displays numerous oddities that could be
explained by the interference of some other 
language, most likely early classical Newar.
On this, see a separate section below on 
pp.~\pageref{newar}ff.

In addition, the quotes from \Manu\ in the \VSS\
usually contain variants that can be found in the apparatus
in Olivelle's critical edition of \Manu\ (\citeyear{OlivelleManu})
as belonging overwhelmingly to
what Olivelle calls the `Northern Transmission.'%
		\footnote{See, e.g., \skt{pāpakṛt} in \VSS\ 3.34d 
		(${\approx}$\ \Manu\ 5.52) attested in Olivelle's
		Devanāgarī MSS Pu$^{5}$, Pu$^{7}$, Pu$^{9}$;
				\skt{nānyatra manur abravı̄t} in \VSS\ 3.35d 
				(${\approx}$\ \Manu\ 5.41) attested in
			Śāradā MSS {\tiny S}Ox$^{1}$, {\tiny S}Pu$^{6}$ and 
			Devanāgarī MS Tr$^{2}$;
			\skt{kūṭa} in \VSS\ 4.79 (${\approx}$ \Manu\ 11.57) in
			a MS from Kathmandu ({\tiny B}Kt$^5$), 
			in Devanāgarī/Old Nāgarī MSS
			(Lo$^{4}$, {\tiny N}Pu$^{1}$, Pu$^{2}$, Pu$^{4}$, Pu$^{10}$),
			as well as in two South-Indian MSS ({\tiny G}Md$^1$, 
			{\tiny T}Md$^3$).}
This again confirms a North-Indian or Nepalese
origin for the \VSS.

\medskip
\label{dating}The obvious \textit{terminus ante quem} for the
composition or redaction of the \VSS\ 
is the date of the earliest MSS that transmits it.
The earliest dated MS containing the \VSS\ is \msKoa,
dated to Nepal Saṃvat 156, i.e., 1035-36 \CE.% 
	\footnote{See \mycitep{SastriCatalogue5}{721} and
		\mycitep{UmaSivaPlay}{591}. The date
	 	is clearly visible as `\skt{samvat} 156' 
	 	in the last line of the penultimate folio side 
	 	of \msKoa/8.}
In a multiple-text MS%
	\footnote{See more detail on this MS, which is
						now to be found in Munich, in	
					    \mycite{HarimotoMunichMS}.}
that is potentially earlier than \msKoa,
the \VSS\ is written in a hand that appears later than
that used for some of the other texts in that MS.%
		\footnote{\mycitep{HarimotoMunichMS}{597--598}:
		`This Śivadharma ms consists 
		of two major parts, easily distinguishable by different 		
		hands: one that appears to be produced in
		9th-c.\ Nepal [\dots], and another seemingly from 
		a century or so later [\dots] 
		The next set of folios making up this Śivadharma ms 	
		consists of three titles: the 
		\textit{Uttaromāmaheśvarasaṃvāda}* (24 folios), 
		the \textit{Vṛṣasārasaṃgraha} (50 folios), and the
		\textit{Dharmaputrikā} (11 folios). We do
		not know the original order of these three works 
		because each section starts with folio 1. Moreover, even 
		though these three titles appear to be written by the same 
		hand (probably somewhat later than the first part), there 
		is no certainty that these folios were produced to 	
		complement the first part.'}
The final colophon of the \VSS\ (and the \DHARMP) in
this MS  (\fol50r) is followed by the date
[Nepāla] `\skt{samvat} 192,' i.e., 1071-1072 \CE.

These two MSS make it impossible to date the \VSS\ later 
than the first half of the 11th century \CE, and parts of the text
may be considerably older.
Archaic features that may indicate 
that the \VSS, or parts of it, were
composed much earlier than the early 11th century
include the following. Chapter ten,%
		\footnote{Also verse 11.21.} 
while it teaches the yogic tubes
\ie{nāḍī} Suṣumnā and Iḍā, is silent on Piṅgalā, 
which is a situation similar to that in 
the 6-7-century \Nisvnaya%
	\footnote{\mycitep{NisvasaGoodall}{33--35}.}  
(see details in the notes to the translation).
Similarly, 11.23a (\skt{nivṛttyādi caturvedaś}) mentions four
Śaiva \skt{kalā}s, instead of the expected and 
somewhat later, and in character tantric, five, namely
\skt{nivṛtti}, \skt{pratiṣṭhā}, \skt{vidyā}, 
\skt{śānti}, and \skt{śāntyatīta}. In the same chapter,
the order in which the \skt{āśrama}s are taught
(\skt{gṛhastha, brahmacārin, vānaprastha, parivrājaka}) 
is reminiscent of \Apastambadharmasutra\ 2.9.21.1,
and is relatively rare,
as opposed to the traditional order (\skt{brahmacārin,
gṛhastha, vānaprastha, parivrājaka}) found, e.g., in
\MANU. (See \mycitep{KissVolume2021}{195--196}.)
%cf. \mycitep{SaivaUtopia}{23}, Chapter 11, Śaiva
Another feature that might point towards a date
considerably earlier than the 11th century is the 
system of \skt{tattva}s in chapter 20:
the \skt{mahābhūta}s of classical Sāṅkhya are called 
\skt{dhātu}s here, the \skt{tanmātra}s of
classical Sāṅkhya are called \skt{guṇa}s,%
		\footnote{In contrast with, e.g.\ \SDHU\ 10.40--46 and
					\UUMS\ chapter 5, \DHARMP\ 1.42--43, or the \SIVAUP.}
the \skt{buddhi} of classical Sāṅkhya
is called \skt{mati}, and the highest \skt{tattva} 
is singular unlike the multiple \skt{puruṣa}s of classical 
Sāṅkhya. These may well be archaisms 
included in the \VSS\ consciously, but they could also
indicate that the time of composition of the \VSS\
is much closer to pre-classical Sāṅkhya than what the MS
evidence suggests.%
	\footnote{There are also numerous borrowings in \VSS\ 20
					from the Śāntiparvan of the \MBH. See more details
					at the analysis of \VSS\ chapter 20 in volume two.}

All in all, in light of all the above,
it is difficult to be more precise on the dating
of the \VSS\ than saying that its production must have
happened before the end of the 10th century, or the beginning
of the 11th century \CE\ if our oldest dated MS that transmits
the \VSS\ is close in time to the actual composition or
redaction of the text. The date could also be considerably
earlier than the 10th century, and therefore a tentative dating
for the \VSS\ would consider the 7th to 10th centuries \CE.
%    varṇas and the Liṅgapurāṇa
%    check lists of deities such as Vasus
%  bull, Nandi

\section{Authors, redactors and target audience}


\section{Why was the \VSS\ included in the Śivadharma corpus?}

One of the objectives of the article 
\mycite{KissVolume2021} was
to find clues about the r\^ole of the \VSS\ in the Śivadharma corpus.
The conclusion  therein (pp.~200--201), focusing on the fusion of 
Vaiṣṇava and Śaiva material in the \VSS, and on the reinterpretations of 
the \skt{āśrama} system in its eleventh chapter, includes the following:

\begin{quote}
The \Vss's role in the Śivadharma corpus is then twofold: 
it provides a text that is suitable for Vaiṣṇavas and Śaivas,
presenting its teachings on different levels of an esoteric scale, 
the Śaiva teachings being closest to the core, and always
providing an internalised, secret version of topics 
discussed in the other layers; and it also reinvents the traditional 
\skt{āśrama} system in a Śaiva way,
but in such a manner that would be acceptable for other religious groups. 
This may be an attempt to further develop an idea that appears in both 
the \SDhS\ and the \SDhU.
\end{quote}

\noindent
Indeed, one of the most striking feature of the \VSS\
is its structure in which Vaiṣṇava material surrounds
Śaiva teachings (see pp.~\pageref{structure}\thinspace ff. above). 
Even the title is not unambiguously Śaiva, as
we have seen (see pp.~\pageref{title}ff. above).
Can we still say that this text is Śaiva? Does it
aim at a sort of balance of Vaiṣṇava and Śaiva
teachings? Does this duality reflect the 
religio\-political reality of the era?

Radicalism in chapters 2, 11, 19

MORE...






\section{Pāśupata and tantric influence}
One of the major questions concerning the Śivadharma corpus is whether it was aware of or influenced by Tantrism. This question is perhaps more important in the case of earlier Śivadharma texts, such as the \SDhS\ and the \SDhU, than for the \VSS, which was likely composed later. Tantric influence in the 7-10th-century \VSS\ would not be surprising; what is more revealing is whether this influence is early (5-8th century) or late (9-11th century), which may help determine the text's date.

The description of Śiva's Universe (\skt{śivāṇḍa}) in chapter two contains clear references to the five Brahma-mantras (usually regarded as Vedic in origin, but possibly entering the Pāśupata and later Śaiva tantric traditions from other sources),%
				\footnote{See \TAKIII, s.v. 
				\skt{pañca brahmāṇi} and \TAKIV, s.v.
				\skt{brahmamantra}.\nocite{TAK3}\nocite{TAK4}}
or five faces of Sadāśiva: Īśāna, Tatpuruṣa, Aghora, Sadyojāta, and Vāmadeva (2.26--33). Their traditional division into \skt{kalā}s also appears (2.31--32). 
Other glimpses into the Pāśupata realm can be seen in chapter eight. In verse 8.2, the Pāśupata tradition is explicitly named alongside the `Śaiva' school. Additionally, the religious observances given in verses 8.13--18, particularly the Dog and Cow Observances (8.15--16) evoke Pāśupata practices.%
			\footnote{See details in the notes to the
								translation of these passages.}
Verses 8.35--43 describe various modes of ritual bathing. The first, Fire Bath, is explicitly referred to as a `Pāśupata observance' (\skt{vrataṃ pāśupataṃ}), and is praised as the most important (\skt{pāśupataṃ śreṣṭhaṃ}) in verse 8.39. (Note that chapter eight, despite these influences, is part of a layer of the text that otherwise could be labelled as Vaiṣṇava.)%
				\footnote{Pāśupatas are also mentioned among 	
					other religious groups in chapter twenty-two.
					See volume two.}

As for any possible Mantramārgic or Saiddhāntika influence, Sadāśiva, Paraśiva, and Śiva as Paramātman are mentioned in 16.34 as corresponding to breaths.%
		\footnote{\VSS\ 16.34: 
			\skt{sadāśivas{ }tu niśvāsa}
						\skt{ūrdhvaśvāsaḥ paraḥ śivaḥ} |
		\skt{tayor{ }madhye tu vijñeyaḥ 
								paramātmā śivo 'vyayaḥ} ||;						
        `Sighing/exhaling is Sadāśiva, a deep breath is 
        supreme Śiva. In between the two, there is Śiva the 	
        supreme and imperishable Self.'
        The word \skt{niśvāsa} evokes the title of 
        the earliest surviving Śaiva tantra, the \Nisv.
        In \Nisvuttara\ 5.50--51, the explanation of
        \skt{niśvāsa} in the title is given as follows:
        \skt{anadhītya tha niśvāsaṃ 
        				niśvasanti punaḥ punaḥ} |
		\skt{adhītvā caiva niśvāsan 
					na punar nniśvasanti te} ||
		\skt{niśvāsa eva vikhyātas 
						sarvatantrasamuccayaḥ} |
		\skt{yaṃ jñātvā mucyate jantuḥ 	
					saṃsārabhavabandhanāt} ||;
        `Now (\skt{'tha}) those who do not study 
        the \skt{Niśvāsa} will go on sighing and sighing.
        And those who do study the \skt{Niśvāsa}, 
        they will not sigh again. [For this reason] it is known
         as the \skt{Niśvāsa}, the compendium of all Tantras, 
         on knowing which a creature will be released from 
         the bondage of being in \skt{saṃsāra}' 
         (tr. \mycitep{NisvasaGoodall}{400}).  
         \citeauthor{KafleNisvasaBook} 	
         (\citeyear{KafleNisvasaBook}, {33}) adds:
         `On the basis of this passage we may render 
         the title of the work as `compendium (\skt{saṃhitā}) 	
         of the essence (\skt{tattva}) of sighing (\skt{niśvāsa}).'
		   One wonders if the connection between breaths
		   and (Sadā)śiva in the \VSS\ may relate to 
		   Saiddhāntika ideas about the connotations
		   of the word \skt{niśvāsa}.}
Sadāśiva appears in a visualisation in \VSS\ 6.16, and is said to be the original teacher of the internalisation of the \skt{āśrama}s, bestowing this knowledge on Maheśvara (11.4, 25).
The term \skt{dhyāna} generally means visualization, similarly to its tantric usage, in verses 4.72--73 (Śaiva), 6.7--18 (mostly Śaiva, but said to be taught by Hari), 
10.23 (a visualisation of the deity in the centre of a lotus), 
10.25--26 (an obscure visualisation possibly echoing \Nisvuttara\ 5.16), and in chapter 16, the main yogic teaching, and in chapter 22.%
		\footnote{In other cases, \skt{dhyāna} does 
		not so clearly involve visualisation; see        
        2.37, 5.18, 9.32, 11.15, 27, 41, and 12.11.}
An obscure reference to a 36-\skt{tattva} system appears in 4.73, possibly indicating familiarity with a full-fledged tantric ontological system, in stark contrast with the highly detailed account and propagation of a 25-\skt{tattva}-system in chapter 20.%
			\footnote{\VSS\ 20.1ab: 
			\skt{pañcaviṃśati yat tattvaṃ 
					jñātum icchāmi tattvataḥ} |
			\skt{kathayasva mamādya tvaṃ
					 chidyate yena saṃśayaḥ} ||
        `I wish to learn about the twenty-five 
        Tattvas truly.' (Note the use of singular alongside
        numerals, and see
        p.~\pageref{singularwithnumerals}.) }
Similary, the terms \skt{sakala-vikala} in 9.5 may betray some knowledge of Śaiva tantric theology. Mantras resembling those of the tantric Mantra\-mārga, apart from \skt{om}, are largely absent in the \VSS, however chapter twenty-two presents an obscurely, perhaps unbreakably, encoded ten-syllable mantra.

Finally, the Pañcarātra tradition is mentioned several times (10.33, 16.36--37), but its presence, similar to some \MBH\ passages,%
			\footnote{Compare, e.g., \MBH\ 12.337.1
			(\skt{sāṃkhyaṃ yogaṃ pañcarātraṃ 
											vedāraṇyakam eva ca} |
		    \skt{jñānāny etāni brahmarṣe 
		    								lokeṣu pracaranti ha} ||) with
		    \VSS\ 16.36 
		    (\skt{śāstrapañcasu yat proktaṃ
				    		 śṛṇu saṃkṣepa nirṇayam} |
			\skt{sāṃkhye yoge pañcarātre 
						śaive vede ca nirmitam} ||).} 
tells us little about the text's date.						

In summary, the Pāśupatas are clearly known and highly regarded in the \VSS, and while tantric influence is subtle, the cumulative evidence suggests that Tantra was present in the vicinity of the text's conception. 






\section{Buddhist influence}

The presence of Buddhist influence in the \Vss\ is subtle but noticeable. The four \skt{brahmavihāra}s---\skt{maitrī, karuṇā, muditā}, and \skt{upekṣā}---are mentioned in 4.71 as `the four \skt{āyatana}s,' possibly indicating only a superficial familiarity with the concept.%
	\footnote{Could this passage have been influenced by 
				the following passage in the \Dharmasamuccaya?		
									\nocite{CaubeDharmasamuccaya}
				\skt{mokṣasy\textbf{āyatanāni} ṣaṭ~|
   				apramādas tathā śraddhā 
   									vīryārambhas tathā dhṛtiḥ~|
			   jñānābhyāsaḥ saṃtāśleṣo 
			   						mokṣasyāyatanāni ṣaṭ}~||1.3||
			  \skt{nava śāntisamprāptihetavaḥ~|
  			  dānaṃ śīlaṃ damaḥ kṣāntir 
  			  \textbf{maitrī} bhūteṣv ahiṃsatā~|
  				\textbf{karuṇāmuditopekṣā} 
  										śāntisamprāptihetavaḥ}~||1.4||.}
They are also referenced in 11.34 and 11.56 in the context of the internalization of the \skt{vānaprastha}'s and \skt{parivrājaka}'s modes of life. Additionally, a rule given in 11.46 concerning begging might echo a Buddhist precept. Viṣṇu, one of the interlocutors in chapters 1--9 and 19--21, assumes the name Vigatarāga (``passionless'') when disguised as a Brahmin, a name that may carry faint Buddhist connotations. A possible influence from the \Buddhacarita\ is seen in 4.55--57 and 70, while the teachings on \skt{mauna} in 4.69 seem similar to Buddhist teachings. In summary, \VSS\ chapter 4, and to some extent chapter 11, display faint signs of Buddhist influence. This may contribute to the text's broader program of offering a foundational Dharma text for devotees of all religions present at the time and place of its composition.


\subsection{Misc}

  susūkṣma: Śivadharmottara 10.45cd--46: rudraḥ ṣaḍviṃśakaḥ proktaḥ
  śivaś ca paratas tataḥ \textbar{}\textbar{} 45 \textbar{}\textbar{}
  saptaviṃśatimaḥ śāntaḥ susūkṣmaḥ parameśvaraḥ \textbar{}
  svargāpavargayor dātā taṃ vijñāya vimucyate \textbar{}\textbar{} 
  46
  







\section{Language}\label{language}

\subsection{Newar influence}
\label{newar}

The oddities of the language of the VSS go beyond the idiosyncrasies of epic Sanskrit.
This dialect exhibits strong similarities to Śaiva Aiśa Sanskrit,%
		\footnote{On Aiśa, see, e.g., \mycitep{GoodallKirana}{lxv\thinspace ff.}, 
								  \mycitep{TorzsokSYMthesis}{xxvi\thinspace ff.},
						          \mycitep{KissBraYa}{77--87}, 
						          \mycite{GerstmayrAisa}, and
						          \mycitep{HatleyBraYaVol1}{28ff.}} 
and it often applies peculiar metrical licences and
uses a special vocabulary, morphology, and syntax.
The analysis of this language, ideally, would help us
define the identity of the author(s) or redactor(s) of the text,
and confirm our views on its place of composition.
To feed a working hypothesis, I will mention parallelisms
between the language of the \VSS\ and early classical 
Newar---since the \VSS\ was most probably produced in the 
Kathmandu valley%
		\footnote{See pp.~\pageref{provenance}\thinspace ff.}%
---whenever possible. 
Of course, the assumable date
of the composition of the \VSS, which is without much doubt
pre-early-11th century, does not allow any direct
comparison with contemporary Newar language texts.%
	\footnote{The earliest dated Newar document is 
			the Ukū Bāhāḥ land grant palmleaf manuscript from
			1114 \CE. See, e.g., \mycite{MallaUku}.}
Therefore I have to project a much later Newar grammar
onto an earlier and less well-known 
state of the language, which is not without risks.

In the following, I will only give a brief overview of the most
important phenomena. For details, see the observations 
on the constitution of the Sanskrit text in the footnotes 
to the translation, as well as the Index.


\subsection{Number and gender}\label{number}
One of the most evident deviation from Pāṇinian grammar in 
the text of the \VSS\ is a general disregard of grammatical concord 
as to number and gender.%
	\footnote{Compare Kölver's introductory remarks in his investigation of
	`Newarized Sanskrit' (\citeyear{KolverErgative}, 202) in the \SvayP\ thus (ibid. 192):
					
					\noindent
	          	`Number is often ignored
	          	
					[\skt{catvāro 'pi maṇḍalañ ca} 429,19 (cf.\thinspace 429, 21), 
					\skt{narāḥ pañcagatiñ ca na labhec ca} 428,12],
					
					\noindent
					as is gender
					
			[\skt{tvam ekam āgataṃ na hi} 464, 10 `only you have not come’; 
			\skt{°nāgakanyā \dots\ vṛṣṭipūrṇaṃ kṛtam} 470, 8 
					`the Nāga girl made (it) full of rain'],
				
				\noindent
				and case
				
				[\skt{manuṣyāḥ \dots\ tasmai \dots\ pūjitam} 426, 2 etc. 
				`men worshipped him; he was worshipped by people'; 
				\skt{bhavatām apy arthāya karomy upāyakam mayā} 452, 5 
				`I am making an expedient for your sake'].'}
See, e.g., a plural verb
(metri causa?) with a singular subject in \VSS\ 1.25ab:

\begin{quote}
\skt{rātryāgame pralīyante jagat sarvaṃ carācaram} 

When [Brahmā's] night falls, the whole moving and unmoving universe dissolve[s].
\end{quote}

\noindent
See a neuter plural participle picking up a 
neuter singular and a feminine singular noun in 1.61ab:

\begin{quote}
\skt{pramāṇaṃ nāma saṃkhyā ca kīrtitāni samāsataḥ}

The numbers [pertaining to] the measurements have been taught in brief.
\end{quote}

\noindent
This confusion, or often metrically forced disregard of standard Sanskrit
grammar, when dealing with number and gender, becomes almost
predictable when the noun phrase involves numerals.%
    \footnote{I am thankful to Judit Törzsök, who first pointed out to me
    the regular nature of the phenomenon itself as seen in the \VSS, and who 
    later drew my attention to the similar Newar grammatical rule
    (personal communication, Nov 29, 2023), which 
    led me to an investigation of a possible link between the Sanskrit of the \VSS\
    and classical Newar.}
See, e.g., verse 1.2cd:


\begin{quote}
\skt{parva cāsya śataṃ pūrṇaṃ śrutvā bhāratasaṃhitām}

Having listened to the \titleface{Mahābhārata},
to all its hundred section[s] \ie{parvan}\dots
\end{quote}

\noindent
Here one would expect either a plural genitive \ie{parvāṇāṃ śataṃ},
a compound \ie{śataparvāṇi}, or a plural accusative \ie{parvāṇi śataṃ}.
Similarly, \skt{gatiś ca pañca vijñeyāḥ} in 3.5a stands for
\skt{gatayaś ca pañca vijñeyāḥ} (`and the paths are to be known as five'), 
partly metri causa; and an interrogative quantifier (\skt{kati}, `how many?') can
trigger the same: \skt{gatis tasya kati smṛtāḥ} (3.1d; `how many are its path[s]?').
It is not without interest that classical Newar rarely applies
any plural marker in noun phrases with numerals.\label{singularwithnumerals}% 
	\footnote{See, e.g., \mycitep{JorgensenGrammar}{18}:
			`The plural ending is wanting where plurality is expressed 
			in other ways; thus always after numerals, and
			mostly after nouns denoting ``many, all'' '.
			 Incidentally, singular after numerals is also the norm in Modern Nepali,
									 and in other, even more distant languages
									 such as Hungarian.} 
Moreover in Newar, `nouns denoting inanimate objects 
are indifferent as to number.'%
	\footnote{\mycitep{JorgensenGrammar}{5 and 17}.}
A further clear example is verse 3.6cd:

\begin{quote}
\skt{tasya patnī mahābhāgā trayodaśa sumadhyamāḥ}

       He has thirteen beautiful wives with nice waists.
\end{quote}

\noindent
Here, with no variants in any of the MSS consulted, only the very end 
of the noun phrase \ie{sumadhyamāḥ} has the required 
plural ending. This again is what we often see in Newar.%
		\footnote{`Any case [\dots] and/or plural markers [\dots], as well as 
		postpositions [\dots], are added to the last constituent of the 
		N[oun ]P[hrase].' (\mycitep{OtterCourse}{11--12}.)
		E.g.: in the Newar phrase \skt{thwo khuṃ-na khaṅ-ā rājā-pani}
		(`these kings seen by the thief'), the only indication that
		multiple kings are involved is the plural marker \skt{-pani}
		at the end (ibid.).}
A good example of total number-blindness is 5.17cd: 

\begin{quote}
\skt{kīrtitāni viśeṣeṇa śaucācāram aśeṣataḥ}

\dots\ the practice of purity is definitely expounded in great detail.
\end{quote}

\noindent
Note that there would have been little problem in composing the same
line in standard Sanskrit, e.g., beginning with \skt{kīrtitaṃ ca\dots}
Instead, this line gives away something about the author's indifference
towards grammatical concord.%
		\footnote{Compare Kölver's remark on the phrase \skt{āgataḥ sarve nāgāḥ}
		in a verse in the \SvayP\ (on p.~459 in \mycite{SastriSvayambhuP}):
		`this is a remarkable lack of sensitivity as to the category of number'
		(\mycitep{KolverErgative}{195}).}
Also, the participle \skt{kīrtitāni} might
function here as a finite verb in the plural: `they teach [the practice of purity].'
In this case there is some sense of number but coupled with a totally 
blurred boundary between active finite verbs and passive participles.


In general, gender confusion is not unusual in epic Sanskrit and in Aiśa.%
		\footnote{See, e.g., \mycitep{OberliesEpicSkt}{XXXVIII--XL}, and
									\mycitep{KissBraYa}{85 and the Index therein}.}
It is its extent in the \VSS\ that suggests a very strong external influence,
supposedly of classical Newar.									





\subsection{Case and syntax}

An extreme example of a total lack of awareness of Sanskrit syntax is
\VSS~17.20:

\begin{quote}
\skt{bhūmipradātā dvija hīnadīnaḥ}\\
\skt{\phantom{aaa}samṛddhasasyo jalasaṃnikṛṣṭaḥ} |\\
\skt{sa yāti lokam amarādhipasya}\\
\skt{\phantom{aaa}vimānayānena manohareṇa} ||

He who donates to a poor and distressed Brahmin land that yields plenty of corn and is in the vicinity of water will go to the world of the king of the immortal ones [i.e.\ of Indra] on a fascinating \ae rial vehicle.
\end{quote}            
            
\noindent            
The translation of this verse, surprising as it may seem, is, judging from the context, rather secure. \skt{Pāda}s ab probably stand for a sentence that would be the following in slightly more standard Sanskrit: \skt{yo dvijāya hīnadīnāya sasyasamṛddha-jalasaṃnikṛṣṭa-bhūmi-pradātā}. This is expressed by a phrase in which a word that should be in the dative or genitive \ie{dvija} is in the vocative, or rather in stem form, and everything else is in the nominative: endings seem but decorations. This is difficult to explain by classical Newar influence since Newar
does have a dative case marker, with animate nouns added to the genitive
marker. Similarly difficult is to explain why then \skt{pāda}s cd
are written in perfect standard Sanskrit.%
		\footnote{See a similarly puzzling situation in the \BraYa, 
						which is briefly described in \mycitep{KissBraYa}{74} as follows:
		`One of the most intriguing questions concerning the Bra[hma]Yā[mala] 
		is not why its language deviates from Pāṇini so often 
		but rather why sometimes it falls back to perfectly standard 
		Pāṇinian language for fairly long passages.'}

There are dozens, or hundreds, of syntactical oddities in the \VSS,
even if not all this baffling.%
		\footnote{Most of them are addressed in the footnotes 
									to the translation.}
Somewhat similarly to what Kölver describes in 
his analysis of the language of the \SvayP, a Nepalese composition (\mycite{KolverErgative}),
there often (but not always!) seems to be a lack of understanding of the
passive,\label{confusedpassive} 
together with the application of the ergative,\label{ergative} one of the
basic syntactical tools of classical Newar. To demonstrate this, a good 
example is 12.113cd:

\begin{quote}
\skt{indreṇāsmi phalaṃ dattaṃ sa phalaṃ datta me bhavān} 

It was Indra who gave me the fruit and I gave that fruit to you.
\end{quote}

\noindent
Again, this is the translation that seems to fit the context. 
Here the skeleton of \skt{pāda} c is a well-constructed passive:
\skt{indreṇa phalaṃ dattaṃ}, but then, instead of adding a dative or 
genitive (e.g., \skt{indreṇa me phalaṃ dattaṃ}), the author chooses 
a finite verb (\skt{asmi}). In \skt{pāda} d, after seemingly 
treating \skt{phalaṃ} as a masculine noun, and leaving
\skt{datta} in \stemform\ metri causa, and using \skt{me} for \skt{mayā},%
		\footnote{This often happens in epic Sanskrit, see 
			\mycitep{OberliesEpicSkt}{4.1.3, pp.~102--103}.}
this time he ends the phrase with a noun in the nominative \ie{bhavān} instead of
the dative or genitive. Why not try to write \skt{dattaṃ tad eva te mayā},%
		\footnote{Although this solution carries the metric fault of 
								being iambic.}
or \skt{dattaṃ tava tad eva ca}?
\label{kathita}Constructions with \skt{datta}/\skt{kathita} plus an expected dative 
are especially prone to confusion. See, e.g., \VSS\ 1.62cd--63ab and 
10.2d:

\begin{quote}
\skt{brahmaṇā kathitaṃ pūrṇaṃ mātariśvā yathātatham}\\
\skt{vāyunā pāda saṃkṣipya prāptaṃ cośanasaṃ purā}

        [The Purāṇas] were taught by Brahmā to 
        Mātariśvan [= Vāyu] in their entirety, in their true form.
        Vāyu abridged the verses and then gave [them] to Uśanas.
        
\skt{bravīmi vaḥ purāvṛttaṃ nandinā kathito 'smy aham}

        I shall teach you an ancient legend that Nandi told me.
\end{quote}

\noindent
Again, there is some struggle first with an expected dative here:
it ends up in the nominative \ie{mātariśvā}. Then an expected 
agent in the instrumental, or rather another dative, 
becomes an accusative \ie{cośanasaṃ}. Thirdly,
\skt{kathito 'smi} stands for \skt{kathitaṃ mama} or
\skt{kathitaṃ mahyam}. 

Somewhat similar are constructions with a 
past participle plus \skt{asmi}
in place of an active finite verb. See, e.g.,
13.68cd, 14.56ab and 15.15cd:

\begin{quote}
\skt{eṣa garbhasamutpattiḥ kathito 'smi varānane}

            This is how I have told you the formation of 
            the embryo, O Varānanā.
            
\skt{āgneyadhātuṃ somaṃ ca kathito 'smi varānane}

			I have taught, O Varānanā, the Fiery constituents 
			and the Soma-ones.
			
\skt{kathito 'smi samāsena kim anyac chrotum icchasi}

		Thus have I briefly described [to you, O Mahādevī, the soul.] 
		What else would you like to hear?
\end{quote}

\noindent
These are also similar to what \citeauthor{JorgensenVicitra}
analyses in a Sanskrit passage in the Newar
\titleface{Vicitra\-karṇikāvadā\-noddhṛta}, namely that
the phrase \skt{na jñāto 'ham} must in that context 
mean `I did not know.'%
		\footnote{\mycitep{JorgensenVicitra}{77 and 328}.
						Compare \skt{tat phalaṃ sa niveditaḥ} (`he gave that fruit') in \VSS\ 12:67d.}

%Vicitrakar :
%ibid and 66 \skt{aṣṭhan/au divasam āgata} should mean `Come on the eighth day'.;
%buddhamārgaṃ abhijñātaṃ: `on the well-known path of the Buddhas'
%80 and 328-329 samagraṃ, stem forms etc.!?

Sometimes the agent of an active construction with a transitive verb
simply imitates an ergative structure: \skt{viṣṇunā\dots\ papraccha} (1.8),
\skt{dhanyās te yair idaṃ vetti} (4.75ab),
\skt{sa}[!] \skt{hovāca pathīkena} (12.60a).%
		\footnote{This happens also in Aiśa. See, e.g., \SiddhYogMata\ 18.23: 
		\skt{pūjayet \dots\ mantriṇā} (\mycitep{TorzsokSYMthesis}{42}).}
%yajec cakre ca vidhivad yoginīsiddhim icchatā 21.12cd

\label{tellplusgen}Another typical syntactical construction in the \VSS\ is a verb
meaning `to tell, teach' plus a noun in the genitive, e.g. 4.69ab:

\begin{quote}
\skt{caturmaunasya vakṣyāmi śṛṇuṣvāvahito bhava}

        I shall tell you about the four cases of observing silence. 
        Listen, be attentive.
\end{quote}

\noindent
One could say that \skt{pāda} a is simply elliptical and that
a verb like \skt{lakṣaṇaṃ} or \skt{svabhāvaṃ} 
(`the caracteristics/\thinspace essence [of X]') is missing. 1.37ab and 4.17ab
also belong to this category:

\begin{quote}

\skt{brahmāṇḍānāṃ prasaṃkhyātuṃ mayā śakyaṃ kathaṃ dvija}

How could I enumerate [all the details of] the Brahmāṇḍa[s], O twice-born?

\skt{evaṃ satyavidhānasya kīrtitaṃ tava suvrata}

Thus have [I] taught you the rules of truth, O virtuous one.

\end{quote}

\noindent
This phenomenon is difficult to explain by any Newar influence since
classical Newar would usually also require an 
extra word (such as \skt{khaṃ} `thing, topic, word, story') in such a sentence.
It might belong to a class of phenomena in Buddhist Hybrid Sanskrit 
that Edgerton labels `genitive with miscellaneous verbs.'%
		\footnote{\mycitep{EdgertonHybrid}{vol.~1, \S 7.65, p.~47}.}

These kinds of deviations from standard Sanskrit make it
necessary that the translation be somewhat intuitive,
driven by the context, rather than forced by an adherence to
standard Sanskrit syntax.





\subsection{Cardinal and ordinal numbers}

Although the \VSS\ does use simple ordinal numbers such 
as \skt{prathama}, \skt{dvi\-tī\-ya}, and \skt{tṛtīya}, with higher 
numbers there seems to be a non-distinction between cardinal
and ordinal numbers, and cardinals are used as ordinals.
See, e.g., 20.8ab and 11ab:

\begin{quote}
\skt{caturviṃśati yat{ }tattvaṃ prakṛtiṃ viddhi niścayam}\\
\skt{dvāviṃśati ahaṃkāras{ }tattvam{ }uktaṃ manīṣibhiḥ}

Know the twenty-fourth Tattva certainly as Prakṛti.
The twenty-second Tattva is Ahaṃkāra according to the wise.
\end{quote}

\noindent
This phenomenon is known to a certain degree
from epic Sanskrit,%
	\footnote{See \mycitep{OberliesEpicSkt}{\S 5.2.2, pp.~127--128}.}
and is even more characteristic of classical Newar.%
		\footnote{See \mycitep{JorgensenGrammar}{42} and
									 \mycitep{OtterCourse}{57}.}





\subsection{Stem form nouns}\label{stemform}

\Stemform\ nouns, or \skt{prātipadika}s, are extremely common in the
language of the \VSS. They are not alien to the Aiśa Sanskrit
of Śaiva Tantras,%
		\footnote{See, e.g., \mycitep{KissBraYa}{75--77} and
								\mycitep{NisvasaGoodall}{126 and 441}.}
but the extent to which they prevail in the \VSS\ is striking and it
reminds one of the zero suffix of the nominative and accusative,
or rather of the `casus indefinitus' or `absolutive case,' of classical Newar.%
		\footnote{\mycitep{JorgensenGrammar}{18 and 21},
		 and \mycitep{OtterCourse}{16}.} 
Often \stemform s are required to restore the metre, 
and they would thus be difficult to emend,
and often they blend in sandhi with the following word. 
See some clearcut examples below with the expected,
but usually unmetrical, form in parentheses:

\begin{quote}
 1.63a: \skt{vāyunā pāda saṃkṣipya} (\skt{pādaṃ}) \\
1.63c: \skt{tenāpi pāda saṃkṣipya} (\skt{pādaṃ}) \\
2.25c: \skt{bhogam akṣaya tatraiva} (\skt{akṣayaṃ}) \\
2.26d: \skt{īśānānāṃ smṛtālayaḥ} (\skt{smṛta ālayaḥ}) \\
4.19f: \skt{prasahyasteya pañcamam} (°\skt{steyaṃ}) \\
4.72a: \skt{caturdhyānādhunā} (°\skt{dhyānam adhunā}) \\
4.77a: \skt{pramādasthāna pañcaiva} (°\skt{sthānaṃ} or °\skt{sthānāni})
 \\
6.5c: \skt{vedādhyayana kartavyaṃ} (\skt{vedādhyayanaṃ}) \\
6.14a: \skt{dvitīyaṃ tattva puruṣaṃ} (\skt{tattvaṃ}) \\
\end{quote}





\subsection{Vocabulary}




  Special vocabulary/language: karhacit, hṛdi as nominative 10.27cd,
  tirya, me as mayā, āhūtaplavana

  generate list from index

Modern Nepali: singular after numerals.

Kölver

No short-long


\subsection{Metre}\label{metre}
\label{muta}
As regards metrical licences, perhaps 
the most striking feature is the generous
use of the poetic licence sometimes labelled `muta cum liquida,'%
	\footnote{I.e. `stop with liquid.' The term `muta' stands for a `plosive' sound or `stop'. 
	 For a recent contribution on this phenomenon, see,
 	 \mycite{SenMutaCum} (discussing it as it appears in Latin).}
 	%and \mycitep{BaloghYati}??  2018, note 6 (discussing 
 	%Sanskrit metre), Balogh 2019, 39 and 115
namely that some consonant clusters that would 
normally turn the previous short \ie{laghu} syllable
long \ie{guru} may in some cases do not do so.%
		\footnote{On its appearance in Śaiva Tantras,
							see, e.g., \mycitep{GoodallParakhya}{lxxxi} and
							\mycitep{NisvasaGoodall}{441}. The latter concerns
							 the syllable \skt{spa} in \skt{sparśan} in \Nisvnaya\ 2.55cd:
							\skt{sparśatanmātra sparśan tu gṛhṇate tvacam āśṛtaḥ}. }
Syllables beginning with \skt{pr, br, kr}, and also \skt{hr},
especially (in theory exclusively) at the beginning of words, are well-known
candidates for this licence.%
			\footnote{See, e.g., \mycitep{ApteDict}{Appendix~A p.~1}.
			Note that even here, the phenomenon extends beyond plosive sounds:
				\skt{h} is rather a fricative.} 
In the \VSS, \skt{tr, dr, bhr, vr, śr}, and also \skt{śy},%
	\footnote{See, e.g., the cadence of 5.15b: \skt{śukaśyenakān} for \ \shortsyllable\
								\shortsyllable - \shortsyllable -}
\skt{śv, sv}, and \skt{dv}, can also trigger this licence.
All these syllables involve conjunct consonants with
a semivowel in second position. Since the sound in 
first position is not always a plosive (`muta'), the term
`muta cum liquida' is actually less than perfect in our case.
I propose the term `\skt{krama} licence.' To give reasons for this,
% and possibly also rpa,CHECK! seem additional ones.
and for context, it is perhaps not useless to briefly show
what a well-known author on prosody, 
Kedārabhaṭṭa (11th or 12th century),%
		\footnote{\mycitep{OllettGana}{333}.}
who is frequently quoted by Mallinātha, has to say on this
phenomenon in his
\skttitle{Vṛttaratnākara}{Vrttaratnakara} (here given together with Sulhaṇa's \skttitle{Sukavihṛdayanandinī}{Sukavihrdayanandini} commentary):%
		\footnote{\mycite{PatelVrttaratnakara}.}

\begin{quote}
\skt{padādāv iha varṇasya saṃyogaḥ kramasaṃjñikaḥ} | \\
\skt{puraḥsthitena tena syāl laghutā 'pi kvacid guroḥ} || 1.10 ||

In this [work], a conjunct [i.e. combination of two or more consonants]
\ie{saṃ\-yoga} in a word-initial syllable \ie{pādādau varṇasya} is
called `sequence' \ie{kra\-ma}. [A syllable that counts as] long because one
such [consonant cluster] stands in front [of it, i.e.\ after it] can 
sometimes be treated as short.

{\footnotesize [Comm.:] 
\skt{vibhaktyantaṃ padaṃ tasya padasyādau vartamāno yo varṇas tasya
saṃ\-yogaḥ} |
\skt{sa iha śāstre kramasaṃjño jñeyaḥ} |
\skt{tena krameṇa purovartinā prākpadānte vartamānasya
prāptagurubhāvasyāpi laghutā syāt} | 
\skt{kvacil lakṣānurodhena} | 
\skt{nanu ka eṣaḥ kramo nāma saṃyoga ucyate} | 
\skt{pūrvācāryāṇāṃ piṅgalanāgaprabhṛtīnāṃ kāli\-dāsādīnāṃ ca
kavīnāṃ samayaḥ parigṛhītaḥ} |
\skt{saṃyogaḥ kramasaṃyogaḥ}
||~ˇ10~||
\skt{tatra gra-saṃyogena yathā} | 
\skt{idam asyodā\-hara\-ṇam}~|

A `word' is [a unit of speach that] ends in an inflection. 
A `conjunct' is in a `syllable' which is
at the beginning of such a word. 
`In this' [i.e.] work it is to be known under the 
term `sequence' \ie{krama}. By that sequence which stands in front, 
[a syllable] at the end of the previous word, even if it acquired
heaviness [by position], may acquire lightness. `Sometimes' [means:]
as shown in the examples.
But then what is this combination of consonants called `sequence'
(\skt{krama})?
The old teachers such as Piṅgalanāga and poets such as Kālidāsa
accepted [this] rule. The conjunct \ie{saṃyoga}
is the sequence[-type] \ie{krama} [i.e. word-initial]
conjunct \ie{saṃyoga} [in this case].
Among [the possibilities,] for example by the conjunct
\skt{gr}.
Here is an example of that:}

\skt{taruṇaṃ sarṣapaśākaṃ navaudanaṃ picchalāni ca dadhīni} |\\
\skt{alpavyayena sundari grāmyajano miṣṭam aśnāti} || 1.11 ||

Tender mustard seed, fresh porridge, and slimy curds: men in the village eat
these kinds of savoury dishes, O pretty girl, because they do not have
much money.%
	\footnote{I.e.: `you are pretty, don't waste your time with poor village men.'}
\end{quote}

\noindent
The example verse given above (1.11) is in \skt{āryā}, and the
metric pattern of the second half-verse is, strictly speaking, the following:
 
- - | \shortsyllable\ - \shortsyllable\ | - \shortsyllable\ -!~| 
- \shortsyllable\ \shortsyllable\ | 
- - | \shortsyllable\ | -  - | - |

For any \skt{āryā}, this is unmetrical for it yields 28 mor\ae, instead
of the expected 27. By treating the final syllable of \skt{sundari} short, 
in spite of the following \skt{grā}, the pattern conforms 
to the expected pattern: 

- - | \shortsyllable\ - \shortsyllable\ | 
- \shortsyllable\ \shortsyllable\ | - \shortsyllable\ \shortsyllable\ | - - | 
\shortsyllable\ | - - | - |

The commentator gives several more examples, involving the syllables
\skt{gra}, \skt{hra}, and \skt{bhra}, and confirms that the rule
applies only to word-initial consonant clusters: 

\begin{quote}
{\footnotesize\skt{padādāv iti kim | anyatra mā bhūt |}

Why `at the beginning of a word'? [Because] elsewhere it should not be.}
\end{quote}

\noindent
Here follow some examples from the \VSS. The syllables 
with the \skt{krama}\index{krama licence@\skt{krama} licence}
conjunct consonant, before which the syllable
is not turned into long, are encircled, and the metre is given in 
parentheses.

\begin{quote}	
1.1c: \skt{harīndra\Circled{br}ahmādibhir āsamagraṃ} (\skt{upajāti})\\
4.67c: \skt{prajñābodha\Circled{śr}utiṃ smṛtiṃ ca labhate mānaṃ ca nityaṃ labhed} (\skt{śārdūlavikrīḍita})\\
4.89a: \skt{iti yama\Circled{pr}avibhāgaḥ kīrtito 'yaṃ dvijendra} (\skt{mālinī})\\
5.5cd: \skt{parastrīpara\Circled{dr}avyeṣu śaucaṃ kāyikam ucyate} (\skt{pathyā})\\
5.9cd: \skt{vānaprasthasya \Circled{tr}iguṇaṃ yatīnāṃ tu caturguṇam}  (\skt{na-vipulā})\\
5.15ab: \skt{haṃsasārasacakrāhvakukkuṭān śuka\Circled{śy}enakān} (\skt{pathyā})\\
6.13ab: \skt{brahmalokaṃ tu \Circled{pra}thamaṃ tattvaprakṛticintayā} (\skt{na-vipulā})\\
8.33a: \skt{tasmān mauna\Circled{vr}ataṃ sadaiva sudṛḍhaṃ kurvīta yo niścitaṃ} 
 (\skt{śārdūla\-vikrīḍita})\\
10.31b: \skt{īśānenābhijuṣṭaṃ hṛdi \Circled{hr}ada vimalaṃ nādaśītāmbupūrṇam} (\skt{srag\-dharā})\\
11.9ab: \skt{manaḥśuddhis tu \Circled{pr}athamaṃ dravyaśuddhir{ }ataḥ param} (\skt{na-vipulā})
\end{quote}

\noindent
These indeed follow the rule of having the special conjunct with the semi\-vowel
at the beginning of a word in the sense that the word can be a member
of a compound.%
		\footnote{There are some problematic verses that I ignore here. They are
						unlikely to change the overall picture.}
As noted above, since conjuncts such as \skt{śr} and \skt{hr} show up
in this phenomenon, the phrase `muta cum liquida' is slightly misleading,
and therefore I use the phrase `\skt{krama} licence' instead.
To understand how unique the \VSS's indulgence in this
\skt{krama} licence is, the epics and the Purāṇas should perhaps 
be examined from this perspective.						

\label{short2long}Another metrical odditity, or rather metrical licence, that is applied
regularly in the \VSS, exclusively in non-\skt{anuṣṭubh} verses,
 is that a word-final short syllable can count 
as long. Here are some examples, with the short syllable now
turned into long encircled:

\begin{quote} 
3:42d: \skt{etatpuṇyapha\Circled{la}m ahiṃsakajanaḥ prāpnoti niḥsaṃśayaḥ} 
 (\skt{śā\-rdū\-la\-vikrī\-ḍita})\\ 
 4.5a: \skt{na narmayu\Circled{kta}m anṛtaṃ hinasti} (\skt{upajāti})%
 		\footnote{Versions of this line in the \MBH\ and the \MATSP\
 								read °\skt{yuktaṃ vacanaṃ}, avoiding the metrical
								problem (see the apparatus
 								at verse 4.5 in my edition below).}\\
4.39c: \skt{aśeṣaya\Circled{jña}tapadānapuṇyaṃ}  (\skt{upajāti})\\
4.59c: \skt{vijñānadha\Circled{rma}kulakīrtināśa} (\skt{upajāti})\\
4.59d: \skt{bhavanti vi\Circled{pra} damayā vihīnāḥ} (\skt{upajāti})\\
5.20a: \skt{śaucāśaucavidhijña mānava ya\Circled{di} kālakṣaye niścayaḥ} (\skt{śā\-rdū\-la\-vikrī\-ḍita})\\
6.18b: \skt{jijñāsyantāṃ dvijen\Circled{dra} bhavadahanakaraḥ prārthanā\-kalpa\-vṛkṣaḥ}\linebreak (\skt{sragdharā})\\
7.13b: \skt{saubhā\Circled{gya}m atulaṃ labheta sa naro rūpaṃ tathā śobhanam} (\skt{śā\-rdū\-la\-vikrīḍita})\\
8.44d: \skt{na bhavati punaja\Circled{nma} kalpakoṭyāyute 'pi} (\skt{mālinī})\\
11.42b: \skt{saṃsāroddhara\Circled{ṇa}m anityahara\Circled{ṇa}m ajñānanirmūlanam} (\skt{śā\-rdū\-la\-vikrīḍita})\\
11.42c: \skt{prajñāvṛddhika\Circled{ra}m amoghakaraṇaṃ kleśārṇavottāraṇaṃ} (\skt{śā\-rdū\-la\-vikrīḍita})\\
11.42d: \skt{janmavyādhiha\Circled{ra}m akarmadahanaṃ sevet sa dharmottamam} (\skt{śā\-rdū\-la\-vikrīḍita})\\
12.150c: \skt{nityaṃ rogādhivā\Circled{sa}m aniyatavapuṣaṃ trāhi māṃ kālapāśāt} (\skt{srag\-dharā})
\end{quote} 

\noindent
When the syllable that is turned into long ends in \skt{-m}
(see 3.42d, 4.5a, 7.13b, 11.42bcd, and 12.150c among the examples above), 
the phenomenon can be treated as the one described in 
\mycitep{EdgertonHybrid}{vol.~1, \S 2.68--69, p.~19--20}:

\begin{quote}
2.68. As in M lndic generally, anusvāra is often used
instead of any final nasal. This seems to be more than a
merely orthographic matter. For it occurs before vowels,
in what must have been close juncture [\dots] 

2.69. Most texts make use of this practice in verses
for metrical convenience. It is absolutely standard practice
in all verses to use final \skt{m} before a following initial vowel
if meter requires a short final syllable, but \skt{ṃ} if a long is
required. No editor has seen this clearly; all editions are
confused and inconsistent in this respect. So are the mss.\ to 
some extent; but they follow the rule in an overwhelming
majority of instances, and there can be no question of its
original validity; the exceptions are mere corruptions of
tradition.
\end{quote}
  
\noindent
Upon re-examination, none of the witnesses of the \VSS\ that were collated, 
or only consulted for this purpose (\msCa\msCb\msCc\allowbreak\msNa\msNb\msNc\msM\msParis\msKoa\msKob),
use an \skt{anusvāra} in the above cases. There is only one exeption:
\msM\ writes \skt{anityaharaṇaṃ}, °\skt{vṛddhkaraṃ} 
and °\skt{vyādhiharaṃ} in 11.42 before vowels (but not
\skt{saṃsāroddharaṇaṃ}!). The same MS has neither 
\skt{ṃ} or \skt{m} in 12.150c (°\skt{vāsa aniyata}°).
One could argue that this lack of awareness of \skt{ṃ} before
a vowel indicating \skt{gurutva} in almost all cases in all MSS
are `mere corruptions of tradition,' and then typesetting
such -\skt{m} + vowel combinations as \skt{-ṃ} + vowel 
would be commendable.
On the other hand there is little evidence that in the transmission
of the \VSS\ \skt{anusvāra}s were used in this way. This is why
I hesitate to apply `Edgerton's rule' in this edition. Another argument
against applying it is all the cases in which the syllable turned into long
ends in a vowel (4.39c, 4.59cd, 5.20a, 6.18b, and 8.44d among the
examples above). There can be no orthographocal indication of \skt{gurutva} there;
there may have not been any need of it in the other cases either.


%\CHECK  a more or less full collation is important: we cannot automatically reject `ungrammatical' or unmetrical forms because they may well be  the `original' one

\CHECK the more original a section the more extreme language? see ch11


\vfill
\pagebreak

%%% CG start %%%
\section{Contents of chapters 1--12}\label{contentsof1_12}

The following are brief descriptions of the topics covered in chapters 
1--12 of the \VSS, which have been edited and translated in this volume. 
These are accompanied by brief discussions and some analytical remarks.%	
		\footnote{See a Sanskrit summary of the 
					contents of the \VSS, based on Naraharinath's edition,
					in \mycitep{AnilkumarBook}{61--72}. }
See more details in the footnotes to the translation.
					
\subsection{Adhyāya 1}\label{contents_of_ch01}
After a \skt{maṅgala}-verse that addresses a 
deity whose identity is obscure (verse 1.1; is it Śiva or the 
impersonal Brahman?), we enter the first layer 
of the text, which comprises a dialogue between Janamejaya and 
Vaiśampāyana, and which could be labelled Dharmaśāstric.
Janamejaya seeks to hear the essence---the ultimate Dharmic 
teaching---of the \MBh. In response, Vaiśampāyana begins relating 
a dialogue in which Viṣṇu, disguised as a Brahmin, 
tests an ascetic named Anarthayajña, renowned for performing 
non-material, i.e., internalised, sacrifice (\skt{anarthayajña}, 
the subject of \skt{adhyāya} eleven), and 
a devotee of Viṣṇu (as revealed in \skt{adhyāya} twenty-one). 
This marks the beginning of the layer one could label Vaiṣṇava (see pp.~\pageref{structure}ff). 

The first topic they discuss is \skt{brahmavidyā} (1.9--10), an
ambiguous definition of the impersonal Brahman and/or the syllable \skt{oṃ}. 
The next topics include \skt{kāla} (`death, time'), the origin of the body, karma (1.11--17), 
and the divisions of time (from \skt{truṭi} and \skt{nimeṣa} up to \skt{kalpa}s, 1.18--30), 
which lead to a teaching on numbers, ranging from one up to 
two hundred quadrillion (\skt{para}, 1.31--35).
Verses 1.36--39 introduce a list of the rulers of the eight 
regions of Brahmā's Egg (\skt{brahmāṇḍa}, that is, the universe, 1.40--48). 
In addition, Viṣṇu is presented as the ruler of the centre of the Brahmāṇḍa (1.49),  
reaffirming the general Vaiṣṇava character of this layer. 
Verses 1.50--57 give the numbers of subordinates to each ruler mentioned above. 
Verses 1.58--61 teach the measurements of the Brahmāṇḍa. 
Finally, verses 1.62--75 list the redactors and transmitters of the Purāṇas, 
from Brahmā to Vyāsa Dvaipāyana, Romaharṣa, and Romaharṣa's son Amitabuddhi.


% Keywords: Brahmā, Brahman
%
% 11:35:18 From somadeva vasudeva To Everyone : There is a very relevant forthcoming study by H. Kondo: “Fault Lines of Samkya History: Struggling with the Conflict between Satkarayvada and the Accumulation Theory” where he analyses exactly the differing accounts of the exact role of the tanmatras, using also Chinese sources. Hopefully our Journal will be able to publish it this year.
%11:40:18 From somadeva vasudeva To Everyone : Kenji can tell you more, but for dating also consider the Carakasamhita’s “samkhya section”.
%12:02:18 From Yuko Yokochi To Everyone : Three grāmas are in the Nāṭyaśāstra.
%12:03:10 From somadeva vasudeva To Everyone : Check maybe also the B.rhadde”sii of Matas”nga
%12:03:16 From Sathyanarayana Sarma To Csaba Kiss(Privately) : Sorry Csaba, I had other reading, so I couldn’t join.
%12:04:55 From serenasaccone To Everyone : Thanks for the talk and lively discussion! Have to go. See you soon!
%12:11:38 From Csaba Kiss To Sathyanarayana Sarma(Privately) : https://filedn.com/lFSw9FGgUBpyrpsGtImyUHh/textprocess_javascript.html
%12:12:05 From Csaba Kiss To Sathyanarayana Sarma(Privately) : (Another approach at the above link.)

 
\subsection{Adhyāya 2}\label{contents_of_ch02}
Seemingly a reaction, counterpart, or addendum to the previous chapter which
discussed Brahmā's Egg, this chapter introduces Śiva's Egg (\skt{śivā\-ṇḍa}),
potentially an innovation of the \VSS. Śiva's Egg is portrayed as an esoteric, mysterious, and
thus superior, part of the universe, accessible only through 
Śaiva yogic practices (\skt{śiva\-yoga}, 2.34). A description of
an idealistic and egalitarian society is given (`There is no master or servant there, 
nobody to be punished and no punisher,' etc., 2.5ff). The text goes on 
deconstructing the `Hindu' religious universe and the Dharmic ritual 
life of the devotee, eliminating the Kalpas and \skt{karma} (2.11--12), 
all mythological creatures (2.14--15), and ritual (2.16).

Following this, the text describes the details of the Śivāṇḍa---its 
height and width, its lovely flowers, fruits, golden trees, 
gem trees, coral gem thickets and ruby plants (2.17--25). 
The chapter then introduces a scheme that divides the Śivāṇḍa 
into five regions, each connected to one of Śiva's five faces, and
subdivided into the thirty-eight \skt{kalā}s of the five Brahmamantras.

This chapter can be perceived as an innovative attempt to reinforce the
Śaiva character of the text, counterbalancing the previous chapter.
It also seems to reflect tantric, or pre-tantric, Pāśupata, ideas and 
it also emphasises the text's yogic character by implementing 
another esoteric, meditative layer of the universe above, or outside
the Brahmāṇḍa (\skt{śivāṇḍābhyantareṇaiva}, 1.39a). One could theorise that
this chapter is a tantric, or Pāśupata, insertion in a non-tantric text, 
but the fact that the Śivāṇḍa was already mentioned in chapter 
one suggests that the two chapters were likely composed at the same time.
 
Overall, the concept of the Śivāṇḍa appears to be
a bold attempt to transcend the fundamentals of \skt{varṇāśramadharma}
in a radical manner by relativising basic social and moral distinctions, perhaps
distantly echoing Pāśupata teachings, and suggesting that Śaivism, or perhaps
tantric Śaivism, is superior to generic Dharmaśāstric tenets. This radicalism,
perhaps the main motive behind the composition of the \VSS, is perceivable
again in chapter eleven, which discusses the internalisation of the \skt{āśrama}-system, and 
in chapter 19, where it is suggested that the \skt{varṇa}s originate from a social contract.
 



\subsection{Adhyāya 3}\label{contents_of_ch03}

This chapter starts with general questions about Dharma
including the etymology of the word \skt{dharma}, 
Dharma's embodiments---especially as a bull---and 
about the family of personified Dharma (3.1--13).
Dharma is declared to be the embodiment of Śruti and Smṛti (3.14--15).
Smṛti is described as concerning the \skt{varṇāśrama}-system, as well as 
rules of conduct, i.e., the \skt{yama} and \skt{niyama} rules, which are the
focus of 3.16--8.44. Each \skt{yama-niyama} rule is five-fold.
The \skt{yama}s are: \skt{ahiṃsā, satya, asteya, ānṛśaṃsya, dama, ghṛṇā,
dhanya, apramāda, mādhurya}, and \skt{ārjava}. This list is more similar to
ones found in the \MBh\ than to yogic lists such as the one in the
\YS,%
		\footnote{See, e.g., \MBh\ 12.8.17ab: 
						\skt{ahiṃsā satyavacanam ānṛśaṃsyaṃ damo ghṛṇā}.
						On \skt{yama}s and \skt{niyama}s in the \SDHS\ and related
					   texts, see also \mycite{SaivaUtopia}{11--17}.}
but the closest parallel is found in the \VDhU.% 
%śaucam ijyā tapo dānaṃ svādhyāyopasthanigrahaḥ ||
%vratopavāso maunaṃ ca snānaṃ ca niyamā daśa ||202 ||
		\footnote{\VDHU\ 3.233.203:
				\skt{ānṛśaṃsyaṃ kṣamā satyam ahiṃsā ca damaḥ spṛhā} |
				\skt{dhyānaṃ prasādo mādhuryaṃ cārjavaṃ ca yamā daśa} ||.
						The \VDhU\ is probably earlier than 1000 \CE\ (see 
						\mycitep{RocherPuranas1986}{252}).}
The rest of this chapter elaborates on the first \skt{yama}, 
non-violence (\skt{ahiṃsā}), focusing particularly on the five kinds of violence (3.18--23).
After a general praise of non-violence (3.24--32), the text discusses restrictions on meat consumption, quoting \Manu\ in 3.34--37.


\subsection{Adhyāya 4}\label{contents_of_ch04}
Verses 4.1--17 discuss the second \skt{yama}, truthfulness (\skt{satya}). 
After defining truth (\skt{satya}, 4.1), rules for speaking the truth are presented,
illustrated with references to mythological stories. 

Verses 4.18--30 cover the third 
\skt{yama}, refraining from stealing (\skt{asteya}). 
The fourth \skt{yama}, absence of hostility (\skt{ānṛśaṃsya}), is 
introduced in verses 4.31--49. It consists of being kind to Śiva, 
to fathers and mothers, cows, and guests,
with particular emphasis on the praise of cows and rules of hospitality.  
The story of the mongoose in the \MBH\ (\MBH\ 14.92--93)
is mentioned in the context of the latter.

Verses 4.50--59 elaborate on the fifth \skt{yama},
self-restraint (\skt{dama}), possibly drawing on the 
\Buddhacarita,  with more mythological stories referenced.
The sixth \skt{yama}, concerning taboos (\skt{ghṛṇā}) is addressed in verses 4.60--67.
These taboos concern restrictions on sexual partners, taking away others' wealth and
lives, hurting others, and commensality.

The seventh \skt{yama} is \skt{dhanya}, which I translate as `virtue' (4.68--76).
Five areas of practising virtue are mentioned here: 
maintaining silence in four situations;
conquering the fourfold enemy desire, anger, greed, and delusion;
the `four sanctuaries' (\skt{caturāyatana}), which are in fact the
Buddhist \skt{brahmavihāra}s; four types of meditations (on \skt{ātman, vidyā,}
Śiva, and the Subtle One); and Dharma as a four-legged bull. The
basic pattern for this \skt{yama} is that each of its five subcategories
has a fourfold structure.

The eighth \skt{yama} provides instructions how to avoid mistakes and committing sins
(\skt{apramāda}, 4.77--82), with verses 4.77--81 following \Manu.
The ninth \skt{yama} is charm (\skt{mādhurya}), which involves being kind both mentally
and through bodily actions (4.83--85). 
The tenth and final \skt{yama} is sincerity (\skt{ārjava}, 4.86--89),
completing the section on the ten \skt{yama}s.



\subsection{Adhyāya 5}\label{contents_of_ch05}

This chapter begins the section on the \skt{niyama} rules, which are
\skt{śauca, ijyā, tapas, dāna, svādhyāya, upasthanigraha,
vrata, upavāsa, mauna,} and \skt{snāna}. This list also appears in the
\LinPu\ (1.8.29cd--30ab) and the \VDhU\ (3.233.202).
The discussion on the first \skt{niyama}, purity (\skt{śauca}, 5.4--20) seems
incomplete. As usual, we are supposed to be given a list of the five sub-types, 
but there seem to be only four here. The third and fourth types
(\skt{mātrā}- and \skt{bhāva-śauca}) are rather vague, and 
no details are given about them. 
While the first two---bodily purity and purity of food---are 
discussed to some extent, partly drawing on \Manu\ in verses 5.5--9 and 5.10--16, 
the rest of the discussion is quite general. It seems likely that
the author of this section borrowed a list of four or five items from 
an external source but felt unable to elaborate on some of them.




\subsection{Adhyāya 6}\label{contents_of_ch06}
The second \skt{niyama}, sacrifice (\skt{ijyā}), is discussed in verses
6.1--18. It includes five types again: material sacrifice, sacrifice through
work and recitation, knowledge, and meditation. Corresponding
or similar teachings on the `five \skt{mahāyajña}s' can be found, in texts
such as the \BhG\ (4.28), \Manu\ (3.69--71), and \SDhU\ (1.10).
The third \skt{niyama}, penance (\skt{tapas}) is the focus of verses 6.19--28.
with verses 6.21--22 echoing the \MBh.


\subsection{Adhyāya 7}\label{contents_of_ch07}
This chapter addresses the fourth \skt{niyama}, donation (\skt{dāna}).
The five subcategories here are donation of food, clothes, gold, land, and cows
(7.1--25). The chapter concludes with praise for the practice of donation (7.26--28).
This chapter is relatively well-written, composed in simple and generally
straightforward language, in contrast to some passages in the previous
chapters that contain radically non-standard Sanskrit. One cannot help feeling that the
author or redactor of this and some of the following chapters is different
from those of chapters one and two, for example.



\subsection{Adhyāya 8}\label{contents_of_ch08}
In a similarly more or less straightforward chapter, six additional 
\skt{niyama} rules are taught. The fifth \skt{niyama}, study (\skt{svādhyāya}) 
is covered first (8.1--6). The five pillars of the intellectual
milieu in which this teaching was likely composed are
Śaivism, Sāṃkhya philosophy, the Purāṇas, Smārta texts (i.e. Dharmaśāstra),
and the \MBh\ (8.1). Śaivism is defined through the dichotomy of the Śaiva and Pāśupata
traditions. Sāṃkhya-tattvas are said to be taught in groups of five,
suggesting a 25-\skt{tattva} system. The \MBh\ is identified as \skt{itihāsa}. 
Verses 8.7--12 list the five types of sexual offences that constitute
the sixth \skt{niyama} rule (\skt{upasthanigraha}).

Verses 8.13--18 address the seventh \skt{niyama} rule, religious
observances (\skt{vrata}). Four of these observances are in principle imitations of 
animal behaviour: cats, herons, dogs, and cows. The fifth is
somewhat obscure but could be an imitation of Bhīṣma's dying scene in the 
\MBh. All of these observances are radical and may be based on Pāśupata practices.

Verses 8.19--24 teach dietary restrictions as the eighth \skt{niyama} rule (\skt{upavāsa}),
 with verse 8.21 drawing on the \MBh.
Verses 8.25--33 describe the ninth \skt{niyama} rule, \skt{mauna}, outlining
when to remain silent and what to avoid saying, including abusive speach and
insults.

Ritual bathing (\skt{snāna}) concludes the chapter in verses 8.34--44. This 
tenth \skt{niyama} rule, and consists of five types: fire-bath, water-bath, Vedic bath,
Wind bath, and divine or heavenly bath.  

This chapter also concludes the entire \skt{yama-niyama} section, which has
taught twenty rules in total, each theoretically consisting of five subcategories.

%%% CG end %%%



\subsection{Adhyāya 9}\label{contents_of_ch09}
This chapter turns to a discussion of the three Guṇas, \skt{sattva, rajas,} and \skt{tamas}.
The treatment of the topic seems less philosophical and more moralising and classificatory.
It categorizes gods, people, animals, plants, activities, and foods into Sāttvika, Rājasa, and Tāmasa, as well as into superior, mediocre, and low variants of Sāttvika, Rājasa, etc. Mixed categories such as Tāmasa-Rājasa are also mentioned. The chapter concludes by introducing the yogic or moral concept of a state of being beyond the Guṇas (9.39--43), again most probably insprired by the \MBH.

\subsection{Adhyāya 10}\label{contents_of_ch10}
At the very beginning of this chapter, our interlocutors, Vigatarāga and Anarthayajña,
hand over the narration to Nandikeśvara, who immediately begins recounting
a dialogue between Śiva and Devī. This marks a shift to a new layer of the text,
which can be labelled Śaiva. The topic discussed is internalised pilgrimage places (\skt{tīrtha}).
The significance of this chapter lies in the possibility that the topographical
names mentioned, and their hierarchy, may provide clues about the text's
place of composition. Another clue of a different nature is that
while the yogic tubes Suṣumnā and Iḍā are mentioned in verses 10.17 and 20--21,
no clear mention of Piṅgalā, the third tube traditionally associated with them,
is seen anywhere in the text. For more details on both topics, see pp.~\pageref{provenance}ff.

\subsection{Adhyāya 11}\label{contents_of_ch11}
This chapter is crucial for understanding what the \VSS\ may have aspired to and 
why the main interlocutor of the Vaiṣṇava chapters is named Anartayajña. The primary
focus here is `non-material' sacrifice, or \skt{anarthayajña}, which essentially
represents internalized sacrifice or worship, or rather the internalisation of all aspects of
the religious life of a `Hindu' devotee, in each of the four social disciplines (\skt{āśrama}).
Given the omnipresence of the name and concept of Anarthayajña/\skt{anartha\-yajña}, 
this chapter could be central to the development of the entire text. 
See pp.~\pageref{nonmaterial}ff and \mycite{KissVolume2021} for more details.
 


\subsection{Adhyāya 12}\label{contents_of_ch12}
Although non-violence is mentioned alongside hospitality as a
topic to be discussed in this chapter, it is clear that hospitality 
is the primary focus of this long chapter.
What we have after verse 12.3 is a charming, fairy-tale-like narrative
about the adventures of Vipula the merchant. 
Vipula is forced to donate his wife to a visiting 
Brahmin to honor his promise to his guest, 
which leads him to leave his home and wander the world.
At this point a series of miraculous events unfolds,
triggered by the fact that a magical fruit with the power of bestowing 
youthfulness is gifted to him by a monkey, and he, instead 
of eating the fruit, gives it away, and the king of Naravīrapura (Karavīrapura?)
orders him to obtain more such fruits. A quest for more fruit
leads Vipula to the Gandharva king, god Sūrya, Soma, Indra, Viṣṇu, and
ultimately to Brahmā's palace.

The story ends abruptly, giving the impression that it was
part of a longer narrative. Although the story's starting point
is the necessity to satisfy a guest's wishes (\skt{ātithya} or 
rules of guest reception), another key focus appears to 
be the rewards of donation (\skt{dāna}): Vipula donates his wife to the Brahmin;
a monkey gives him a magical fruit; he gives the magical fruit to the foreman of 
the guild; the foreman gives the fruit to the king; it turns out that the
fruit was originally given to the monkey by the Gandharva king; he
was given the fruit by Indra; and so forth. 
    
One of the lessons suggested by the story’s conclusion, 
where Vipula is honored by Brahmā and other gods, 
is that donors eventually receive great rewards. 
The narrative also features a recurring theme of testing people while in disguise: Viṣṇu tests Anarthayajña disguised as Vigatarāga (see 1.7–8), and Vipula seems to be tested by a Brahmin who may in fact be Dharma himself (12.37).

\section{Topics in chapters 13--24}\label{contentsof12_24}

\vfill
\pagebreak






