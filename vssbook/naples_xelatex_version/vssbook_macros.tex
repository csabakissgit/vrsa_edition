% PACKAGES
\usepackage{fontspec} % Font selection for XeLaTeX; see fontspec.pdf for documentation
% LetterSpace=1.2,
\defaultfontfeatures{WordSpace=1.1,Ligatures=TeX, Mapping=tex-text} % to support TeX conventions like ``---''
\usepackage{xunicode} % Unicode support for LaTeX character names (accents, European chars, etc)
%\usepackage{hyperref}
%\usepackage{glossaries}
\usepackage{polyglossia}
\usepackage{xltxtra} % Extra customizations for XeLaTeX
%\setsansfont{Deja Vu Sans}
%\setmonofont{Deja Vu Mono}
% other LaTeX packages.....
%\usepackage[total={12cm,21cm},top=2.8cm, left=4.5cm, headsep=1.1cm, footskip=1.5cm, footnotesep=1cm, xetex]{geometry}
\usepackage[textwidth=11cm, height=20cm, includehead, headsep=1.1cm, footnotesep=1cm, xetex, footskip=1.7cm]{geometry}

\usepackage{pdfpages}
\usepackage{todonotes}
\usepackage{circledsteps}


\usepackage{tikz}
\usepackage{color}
\usetikzlibrary{positioning}% To get more advanced positioning options
\usetikzlibrary{decorations.text}


\parskip.1em\parindent1.5em
%%%%%%%%%%%%%%%%%%%%%%%%%%%%%%%%%%%%%%%%%%%%%%%%%%%%%%%%%%%%%%%
% customize the TOC
\usepackage{tocloft}
\renewcommand{\cftsecafterpnum}{\hspace*{0em}}
\renewcommand{\cftsubsecafterpnum}{\hspace*{0em}}
\renewcommand{\cftsubsubsecafterpnum}{\hspace*{0em}}
\renewcommand{\cftchapafterpnum}{\hspace*{0em}}

\renewcommand\cftloftitlefont{\englishfont} % optional

\makeatletter
\newcommand \mydotfill {\leavevmode \cleaders \hb@xt@ .8em{\hss .\hss }\hfill \kern \z@}
\makeatother
%%%%%%%%%%%%%%%%%%%%%%%%%%%%%%%%%%%%%%%%%%%%%%%%%%%%%%%%%%%%%%%




%%%%%%%%%%%%%%%%%%%%%%%%%%%%%%%%%%%%%%%%%%%%%%%%%%%%%%%%%%%%%%%
% bibliography
\usepackage{natbib}

% notesep below is for the sep between year and page num
\setcitestyle{authoryear, round, aysep={ },numbers,open={},close={},notesep={,\thinspace }}
% to modify the appearance of the label (SANDERSON 2009) in the bibliography (get rid of []s):
\makeatletter                         % Reference list option change
\renewcommand\@biblabel[1]{#1:}      %   from [1] to 1:
\makeatother      
% Title of biblio
%For book-classes: 
%\newcommand{\UnexpandableProtect}{}
\newcommand{\mycite}[1]{\citeauthor{#1} \citeyear{#1}}
\newcommand{\mycitep}[2]{\citeauthor{#1} \citeyear{#1}, #2}

% Finally! To hide the numbering in the Biblio 
\makeatletter
\renewcommand\@biblabel[1]{}
\makeatother

% to change indentation in the Bibliography
\usepackage{enumitem}
\makeatletter
\renewenvironment{thebibliography}[1]
     {\section{Secondary Sources and Editions\bigskip}%
     \fancyhead[CE]{}
     \fancyhead[CO]{}
     \@mkboth{header!}{header!}%
      \begin{enumerate}[label={[\arabic{enumi}]},itemindent=-2em,leftmargin=2em]
      \@openbib@code
      \sloppy
      \clubpenalty4000
      \@clubpenalty \clubpenalty
      \widowpenalty4000
      \sfcode`\.\@m}
     {\def\@noitemerr
       {\@latex@warning{Empty `thebibliography' environment}}%
      \end{enumerate}}
\makeatother
%%%%%%%%%%%%%%%%%%%%%%%%%%%%%%%%%%%%%%%%%%%%%%%%%%%%%%%%%%%%%%%





%%%%%%%%%%%%%%%%%%%%%%%%%%%%%%%%%%%%%%%%%%%%%%%%%%%%%%%%%%%%%%%
%INDEX
\newcommand{\csindex}[1]{#1\index{#1}}
\newcommand{\csindexx}[2]{#1\index{#2@#1}}
\newcommand{\csindexxx}[3]{#1\index{#2@#3}}
\newcommand{\cskt}[2]{\skt{#1}\index{#2@\skt{#1}}}
\newcommand{\sktx}[1]{\skt{#1}\index{#1@\skt{#1}}}

\renewenvironment{theindex}
               {%\if@twocolumn
                  %\@restonecolfalse
                %\else
                  %\@restonecoltrue
                %\fi
%               \columnseprule \z@
                \columnseprule 0pt
%               \columnsep 35\p@
                \columnsep 35pt
%               \twocolumn[\@makeschapterhead{\indexname}]%
                \twocolumn[%\makeschapterhead{\indexname}
                \bigskip
                \bigskip
                \bigskip
                \bigskip

                {\englishfont\Large\hfill\textit{Index to Introduction and Translation}\hfill}
                \bigskip
                \bigskip

 REVISE \verify\ In the Index, the surnames of modern authors, as well
 as mantra-syllables, are typeset
 in \textsc{Small Capitals}, Sanskrit words in general
 in \textit{italics}, Sanskrit names of deities, humans, including authors,
 in non-italic normal typeface with capital initial letters,
 English words in non-italic normal typeface, and
 titles of works in \textsl{slanted font}.
\par \vskip 20pt]%
% Two lines below were the original.  I changed them so that the
% heading of the following pages would suit the previous chapters.
% But I am not certain if that was what Dominic and Haru wanted.
%                \markboth{\MakeUppercase\indexname}%
%                        {\MakeUppercase\indexname}%
              %  \markboth{Niśvāsatattvasaṃhitā}{Index to Introduction and Notes}
%               \thispagestyle{plain}\parindent\z@
                \thispagestyle{plain}\parindent0pt
%               \parskip\z@ \@plus .3\p@\relax
                \parskip0pt plus .3pt\relax
                \def\idxitem{\par\hangindent 40pt}
                \let\item\idxitem%
}
%              {\if@restonecol\onecolumn\else\clearpage\fi}
               {\onecolumn}
%%%%%%%%%%%%%%%%%%%%%%%%%%%%%%%%%%%%%%%%%%%%%%%%%%%%%%%%%%%%%%%





\tolerance=100
\hbadness=100
\hfuzz=2pt

% FONT STUFF
%\newfontfamily\devanagarifont[Script=Devanagari]{Sanskrit2003}
\newfontfamily\newarfont{Noto Sans Newa}
\newfontfamily\englishfont[Scale=1]{EB Garamond}
\setmainlanguage{english}
\setotherlanguages{sanskrit}
\setmainfont[
  %SmallCapsFont={Latin Modern Roman Caps}, 
  %Renderer=ICU
%  SmallCapsFeatures={Letters=SmallCaps},
]{EB Garamond}
%\renewcommand{\baselinestretch}{.95}
% Activate to begin paragraphs with an empty line rather than an indent
%\usepackage[parfill]{parskip}
% support the \includegraphics command and options:
\usepackage{graphicx} 
\usepackage{makeidx}



%%%%%%%%%%%%%%%%%%%%%%%%%%%%%%%%%%%%%%%%%%%%%%%%%%%%%%%%%%%%%%%
%HEADER
\usepackage{fancyhdr}
\pagestyle{fancy}

\fancyhead[CE]{}
\fancyhead[CO]{}
\renewcommand{\headrulewidth}{0pt}
\cfoot{{\footnotesize\thepage}}
\addtolength{\topmargin}{-2.49998pt}

%%%%%%%%%%%%%%%%%%%%%%%%%%%%%%%%%%%%%%%%%%%%%%%%%%%%%%%%%%%%%%%

%%%%%%%%%%%%%%%%%%%%%%%%%%%%%%%%%%%%%%%%%%%%%%%%%%%%%%%%%%%%%%%
% TOC

% this is to get rid of the page numbers on the pages of the toc
\fancypagestyle{plain}{%
  \fancyhf{}                          % clear all header and footer fields
  \renewcommand{\headrulewidth}{0pt}
  \renewcommand{\footrulewidth}{0pt}
}

% no section numbering in text:
\setcounter{secnumdepth}{0}
\setcounter{tocdepth}{3}

% section titles
\usepackage{titlesec}

  \titleformat{\chapter}%
  {\centering\normalfont\englishfont\Large\it}{%
    %\hspace*{-4.5em}\rule[-1.35mm]{4.5em}{1.25em}
    %{\color{white}\hspace{-1cm}\normalfont\thesection\hspace{5pt}}
  }{0em}{}

  % old color format I used to have:
%  \titleformat{\section}
%  {\normalfont\Large\itshape\color{red}}{%
%    \hspace*{-4.5em}\rule[-1.35mm]{4.5em}{1.25em}
%    {\color{white}\hspace{-1cm}\normalfont\thesection\hspace{5pt}}
%  }{0em}{}{}

%\titlespacing*{<command>}{<left>}{<before-sep>}{<after-sep>}
\titlespacing*{\section}
{0pt}{5.5ex plus 1ex minus .2ex}{.3em}

  \titleformat{\section}
  {\englishfont\large}{%
    {\englishfont\large\bf\thesection}  }{0em}{}{}


%\titlespacing*{<command>}{<left>}{<before-sep>}{<after-sep>}
\titlespacing*{\subsection}
{0pt}{5.5ex plus 1ex minus .2ex}{.3em}

  \titleformat{\subsection}
  {\normalfont\itshape}{%
    {\englishfont\itshape\thesection}
  }{0em}{}{}

 % \titleformat{\subsection}
 % {\normalfont\large\itshape\color{brown}}{%
 %   %\hspace*{-4.5em}\rule[-1.35mm]{4.5em}{1.25em}
    %{\color{white}\hspace{-1cm}\normalfont\thesubsection\hspace{5pt}}
 % }{0em}{}

  \titleformat{\subsubsection}
  {\normalfont}{%
   % \hspace*{-4.5em}\rule[-1.35mm]{4.5em}{1.25em}
   % {\color{white}\hspace{-1cm}\normalfont\thesubsection\hspace{5pt}}
  }{0em}{}
  
%\newcommand{\mychapter}[1]{\pagebreak\thispagestyle{empty}\ \vskip10em\begin{center}{\huge#1}\label{#2}\end{center}\vfill\pagebreak}
\newcommand{\mychapter}[1]{\chapter*{#1}
\addtocontents{toc}{\protect\contentsline{chapter}{\protect\numberline{\hspace{2em}}{\englishfont#1}}{}{}}}

%\newcommand{\mysection}[2]{\vskip1em\begin{center}{\textsc{\large #1}}\end{center}\label{#2}\vskip1em}
\newcommand{\mysection}[1]{\section{#1}
\addtocontents{toc}{\protect\contentsline{section}{\protect\numberline{}#1}{}}}


\newcommand{\mysubsection}[2]{\noindent\textit{#1}\label{#2}\\%
\noindent%
}
\newcommand{\mysubsubsection}[2]{\noindent{\color{black}{\textbf{#1}}}\label{#2}\hskip2em}
\newcommand{\msdescr}[2]{\medskip\noindent{\color{black}{\large\textbf{#1}}}\label{#2}\hskip2em}
%%%%%%%%%%%%%%%%%%%%%%%%%%%%%%%%%%%%%%%%%%%%%%%%%%%%%%%%%%%%%%%




%%%%%%%%%%%%%%%%%%%%%%%%%%%%%%%%%%%%%%%%%%%%%%%%%%%%%%%%%%%%%%%
% FOOTNOTES
\renewcommand{\footnoterule}{}
\usepackage[]{footmisc}
% Here you can modify the spacing between the footnote number and the text of footnote; keep this value and \hangfootparindent value the same
\renewcommand{\footnotelayout}{\hspace{.5em}}
\renewcommand{\hangfootparindent}{\hspace{.5em}}

\newcommand\blankfootnote[1]{%
  \let\thefootnote\relax\footnotetext{#1}%
  \let\thefootnote\svthefootnote%
}

%%%%%%%%%%%%%%%%%%%%%%%%%%%%%%%%%%%%%%%%%%%%%%%%%%%%%%%%%%%%%%%




%MISC COMMANDS
\newcommand{\skt}[1]{\textit{#1}}
\newcommand{\ie}[1]{(\textit{#1})}
\newcommand{\skttitle}[2]{\textsl{#1}\index{#2@\textsl{#1}}}
\newcommand{\CHECK}{{\englishfont\color{red}{CHECK}}}
\newcommand{\uncl}[1]{${\wr}$#1${\wr}$}
%\newcommand{\shortsyllable}{\raise.15em\hbox{˘}}
\newcommand{\shortsyllable}{$\cup$}
\newcommand{\similar}{${\approx}$}
\newcommand{\compare}{{cf.}}
\newcommand{\asrama}{\cskt{āśrama}{asrama}}
\newcommand{\CE}{{\textsc{ce}}}
\newcommand{\BC}{{\textsc{bc}}}
\newcommand{\AD}{{\textsc{ad}}}
\newcommand{\fol}{f.\thinspace}
\newcommand{\fols}{ff.\thinspace}
\newcommand{\Fol}{F.\thinspace}
\newcommand{\Fols}{Ff.\thinspace}
\newcommand{\verbalroot}[1]{$\sqrt{#1}$}
\newcommand{\verify}{{\color{red}{CHECK}}}
\newcommand{\hide}[1]{}
\newcommand{\dbldanda}{||}
\newcommand{\il}{${\CHECK}$}
\newcommand{\lost}{\englishfont{×}}
\newcommand{\mutacumliquida}{\skt{krama} licence\index{krama licence@\skt{krama} licence}}
\newcommand{\stemform}{stem form\index{stem form (\skt{prātipadika})}}
\newcommand{\Stemform}{Stem form\index{stem form (\skt{prātipadika})}}

\newcommand{\recto}{$^r$}
\newcommand{\verso}{$^v$}

%SMALL CAPS HACKS
\newcommand\scR{r\kern-.35em\raisebox{-.17em}{.}\kern.15em}
\newcommand\scS{s\kern-.35em\raisebox{-.17em}{.}\kern.13em}
\newcommand\scM{m\kern-.5em\raisebox{-.17em}{.}\kern.25em}

% for the translation: chapter and verse number; translation; notes
\newcommand{\slokawithfn}[3]{\blankfootnote{\kern-1em#1: #3}%
%\textbf
\noindent{#1} #2\medskip}

\newcommand{\slokawithoutfn}[2]{%
%\textbf
\noindent{#1} #2\medskip}

\renewcommand{\labelitemi}{--}

% INPUT titles, authors, sigla
% SANSKRIT WORKS
\newcommand{\AstangHr}{\textsl{A\-ṣṭā\-ṅga\-hṛ\-da\-ya}\index{Astangahrdaya@\textsl{Aṣṭāṅgahṛdaya}}}
\newcommand{\ASTANGHR}{AṣṭāṅgHṛ\index{Astangahrdaya@\textsl{Aṣṭāṅgahṛdaya}}}

\newcommand{\AgniP}{\textsl{A\-gni\-pu\-rā\-ṇa}\index{Agnipurana@\textsl{Agnipurāṇa}}}
\newcommand{\AGNIP}{AgniP\index{Agnipurana@\textsl{Agnipurāṇa}}}

\newcommand{\Amara}{\textsl{Amarakośa}\index{Amarakosa@\textsl{Amarakośa}}}
\newcommand{\AMARA}{Amara\index{Amarakosa@\textsl{Amarakośa}}}

\newcommand{\Uums}{\textsl{Utta\-ro\-tta\-ma\-ma\-hā\-saṃ\-vā\-da}\index{Uttarottaramahasamvada@\textsl{Uttarottaramahāsaṃvāda}}}
\newcommand{\UUMS}{UUMS\index{Uttarottaramahasamvada@\textsl{Uttarottaramahāsaṃvāda}}}

\newcommand{\Ums}{\textsl{Umāmaheśvarasaṃvāda}\index{Umamahesvarasamvada@\textsl{Umāmaheśvarasaṃvāda}}}
\newcommand{\UMS}{UMS\index{Umamahesvarasamvada@\textsl{Umāmaheśvarasaṃvāda}}}


\newcommand{\KurmP}{\textsl{Kū\-rma\-pu\-rā\-ṇa}\index{Kurmapurana@\textsl{Kūrmapurāṇa}}}
\newcommand{\KURMP}{KūrmP\index{Kurmapurana@\textsl{Kūrmapurāṇa}}}

\newcommand{\GautDhS}{\textsl{Gau\-ta\-ma\-dha\-rma\-sū\-tra}\index{Gautamadharmasutra@\textsl{Gautadharmasūtra}}}
\newcommand{\GAUTDHS}{GautDhS\index{Gautamadharmasutra@\textsl{Gautadharmasūtra}}}

\newcommand{\Caraka}{\textsl{Ca\-ra\-ka}\index{Caraka@\textsl{Caraka}}}
\newcommand{\CARAKA}{Caraka\index{Caraka@\textsl{Caraka}}}

\newcommand{\TakI}{\textsl{Tā\-ntri\-kā\-bhi\-dhā\-na\-ko\-śa I}\index{Tantrikabhidhanakosa@\textsl{Tāntrikābhidhānakośa}}}
\newcommand{\TAKI}{TAK I\index{Tantrikabhidhanakosa@\textsl{Tāntrikābhidhānakośa}}}

\newcommand{\TakII}{\textsl{Tā\-ntri\-kā\-bhi\-dhā\-na\-ko\-śa II}\index{Tantrikabhidhanakosa@\textsl{Tāntrikābhidhānakośa}}}
\newcommand{\TAKII}{TAK II\index{Tantrikabhidhanakosa@\textsl{Tāntrikābhidhānakośa}}}

\newcommand{\TakIII}{\textsl{Tā\-ntri\-kā\-bhi\-dhā\-na\-ko\-śa III}\index{Tantrikabhidhanakosa@\textsl{Tāntrikābhidhānakośa}}}
\newcommand{\TAKIII}{TAK III\index{Tantrikabhidhanakosa@\textsl{Tāntrikābhidhānakośa}}}

\newcommand{\Diksottara}{\textsl{Dī\-kṣo\-tta\-ra}\index{Diksottara@\textsl{Dīkṣottara}}}
\newcommand{\DIKSOTTARA}{DīkṣU\index{Diksottara@\textsl{Dīkṣottara}}}

\newcommand{\Divyav}{\textsl{Di\-vyā\-va\-dā\-na}\index{Divyavadana@\textsl{Divyāvadāna}}}
\newcommand{\DIVYAV}{Divyāv\index{Divyavadana@\textsl{Divyāvadāna}}}

\newcommand{\DharmP}{\textsl{Dha\-rma\-pu\-tri\-kā}\index{Dharmaputrika@\textsl{Dharmaputrikā}}}
\newcommand{\DHARMP}{DharmP\index{Dharmaputrika@\textsl{Dharmaputrikā}}}

\newcommand{\NaradaP}{\textsl{Nā\-ra\-da\-pu\-rā\-ṇa}\index{Naradapurana@\textsl{Nāradapurāṇa}}}
\newcommand{\NARADAP}{NāradaP\index{Naradapurana@\textsl{Nāradapurāṇa}}}

\newcommand{\Nisv}{\textsl{Ni\-śvā\-sa\-tattva\-saṃhi\-tā}\index{Nisvasatattvasamhita@\textsl{Niśvāsatattva\-saṃhitā}}}
\newcommand{\NISV}{Niśv\index{Nisvasatattvasamhita@\textsl{Niśvāsatattva\-saṃhitā}}}

\newcommand{\Nisvmukh}{\textsl{Ni\-śvā\-sa mu\-kha}\index{Nisvasamukha@\textsl{Niśvāsa mukha}}}
\newcommand{\NISVMUKH}{NiśvMukha\index{Nisvasamukha@\textsl{Niśvāsa mukha}}}

\newcommand{\Nisvmul}{\textsl{Ni\-śvā\-sa mū\-la}\index{Nisvasamula@\textsl{Niśvāsa mūla}}}
\newcommand{\NISVMUL}{NiśvMūl\index{Nisvasamukha@\textsl{Niśvāsa mūla}}}

\newcommand{\Nisvnaya}{\textsl{Ni\-śvā\-sa na\-ya}\index{Nisvasanaya@\textsl{Niśvāsa naya}}}
\newcommand{\NISVNAYA}{NiśvNaya\index{Nisvasamukha@\textsl{Niśvāsa naya}}}

\newcommand{\Nisvuttara}{\textsl{Ni\-śvā\-sa u\-tta\-ra}\index{Nisvasauttara@\textsl{Niśvāsa uttara}}}
\newcommand{\NISVUTTARA}{NiśvUttara\index{Nisvasauttara@\textsl{Niśvāsa uttara}}}

\newcommand{\NisvK}{\textsl{Ni\-śvā\-sa kā\-ri\-kā}\index{Nisvasakarika@\textsl{Niśvāsa kārikā}}}
\newcommand{\NISVK}{NiśvK\index{Nisvasakarika@\textsl{Niśvāsa kārikā}}}

\newcommand{\Padmap}{\textsl{Padma\-pu\-rā\-ṇa}\index{Padmapurana@\textsl{Padmapurāṇa}}}
\newcommand{\PADMAP}{PadmaP\index{Padmapurana@\textsl{Padmapurāṇa}}}

\newcommand{\Buddhacarita}{\textsl{Bu\-ddha\-ca\-ri\-ta}\index{Buddhacarita@\textsl{Buddhacarita}}}
\newcommand{\BUDDHACARITA}{BuddhCar\index{Buddhacarita@\textsl{Buddhacarita}}}

\newcommand{\BrahmaP}{\textsl{Bra\-hma\-pu\-rā\-ṇa}\index{Brahmapurana@\textsl{Brahmapurāṇa}}}
\newcommand{\BRAHMAP}{BrahmaP\index{Brahmapurana@\textsl{Brahmapurāṇa}}}

\newcommand{\BraYa}{\textsl{Bra\-hma\-yā\-ma\-la}\index{Brahmayamala@\textsl{Brahmayāmala}}}
\newcommand{\BRAYA}{BraYā\index{Brahmayamala@\textsl{Brahmayāmala}}}

\newcommand{\BrahmaVP}{\textsl{Bra\-hma\-vai\-va\-rta\-pu\-rā\-ṇa}\index{Brahmavaivartapurana@\textsl{Brahma\-vaivarta\-purāṇa}}}
\newcommand{\BRAHMAVP}{BrahmaVP\index{Brahmavaivartapurana@\textsl{Brahma\-vaivarta\-purāṇa}}}

\newcommand{\BhG}{\textsl{Bha\-ga\-va\-dgī\-tā}\index{Bhagavadgita@\textsl{Bhagavadgītā}}}
\newcommand{\BHG}{BhG\index{Bhagavadgita@\textsl{Bhagavadgītā}}}


\newcommand{\BhelaS}{\textsl{Bhe\-la\-saṃ\-hi\-tā}\index{Bhelasamhita@\textsl{Bhelasaṃhitā}}}
\newcommand{\BHELAS}{BhelaS\index{Bhelasamhita@\textsl{Bhelasaṃhitā}}}

\newcommand{\MBh}{\textsl{Mahābhārata}\index{Mahabharata@\textsl{Mahābhārata}}}
\newcommand{\MBH}{MBh\index{Mahabharata@\textsl{Mahābhārata}}}

\newcommand{\MahaSubhS}{\textsl{Ma\-hā\-su\-bhā\-ṣi\-ta\-saṃ\-gra\-ha}\index{Mahasubhasitasamgraha@\textsl{Mahā\-subhāṣita\-saṃgraha}}}
\newcommand{\MAHASUBHS}{MahāSubhS\index{Mahasubhasitasamgraha@\textsl{Mahā\-subhāṣita\-saṃgraha}}}

\newcommand{\MarkP}{\textsl{Mārkaṇḍeyapurāṇa}\index{Markandeyapurana@\textsl{Mārkaṇḍeyapurāṇa}}}
\newcommand{\MARKP}{MarkP\index{Markandeyapurana@\textsl{Mārkaṇḍeyapurāṇa}}}

\newcommand{\MatsP}{\textsl{Matsyapurāṇa}\index{Matsyapurana@\textsl{Matsyapurāṇa}}}
\newcommand{\MATSP}{MatsP\index{Matsyapurana@\textsl{Matsyapurāṇa}}}

\newcommand{\Mitaksara}{\textsl{Mi\-tā\-kṣa\-ra}\index{Mitaksara@\textsl{Mitākṣara}}}
\newcommand{\MITAKSARA}{Mitākṣara\index{Mitaksara@\textsl{Mitākṣara}}}

\newcommand{\Manu}{Manu\index{Manu@\textsl{Mānavadharmaśāstra}}}
\newcommand{\MANU}{Manu\index{Manu@\textsl{Mānavadharmaśāstra}}}

\newcommand{\BrahmandaPur}{\textsl{Bra\-hmā\-ṇḍa\-pu\-rā\-ṇa}\index{Brahmandapurana@\textsl{Brahmāṇḍapurāṇa}}}
\newcommand{\BRAHMANDAPUR}{BrahmāṇḍaP\index{Brahmandapurana@\textsl{Brahmāṇḍapurāṇa}}}

\newcommand{\BhavP}{\textsl{Bha\-vi\-ṣya\-pu\-rā\-ṇa}\index{Bhavisyapurana@\textsl{Bhaviṣyapurāṇa}}}
\newcommand{\BHAVP}{BhavP\index{Bhavisyapurana@\textsl{Bhaviṣyapurāṇa}}}

\newcommand{\BhagP}{\textsl{Bhā\-ga\-va\-ta\-pu\-rā\-ṇa}\index{Bhagavatapurana@\textsl{Bhāgavatapurāṇa}}}
\newcommand{\BHAGP}{BhāgP\index{Bhagavatapurana@\textsl{Bhāgavatapurāṇa}}}

\newcommand{\YajnS}{\textsl{Yājñavalkyasmṛti}\index{Yajnavalkyasmrti@\textsl{Yājñavalkyasmṛti}}}
\newcommand{\YAJNS}{YājñS\index{Yajnavalkyasmrti@\textsl{Yājñavalkyasmṛti}}}

\newcommand{\LinPu}{\textsl{Li\-ṅga\-pu\-rā\-ṇa}\index{Lingapurana@\textsl{Liṅgapurāṇa}}}
\newcommand{\LINPU}{LiṅP\index{Sivapurana@\textsl{Śivapurāṇa}}}

\newcommand{\Ramayana}{\textsl{Rā\-mā\-ya\-ṇa}\index{Ramayana@\textsl{Rāmāyaṇā}}}
\newcommand{\RAMAYANA}{\textsl{Rāmāyaṇa}\index{Ramayana@\textsl{Rāmāyaṇā}}}

\newcommand{\Vagmati}{\textsl{Vā\-gma\-tī\-māhātmya\-praśaṃsā}\index{Vagmatimahatmyaprasamsa@\textsl{Vāgmatīmāhātmyapraśaṃsā}}}
\newcommand{\VAGMATI}{VāgMāPr\index{Vagmatimahatmyaprasamsa@\textsl{Vāgmatīmāhātmyapraśaṃsā}}}

\newcommand{\VisnuP}{\textsl{Vi\-ṣṇu\-pu\-rā\-ṇa}\index{Visnupurana@\textsl{Viṣṇupurāṇa}}}
\newcommand{\VISNUP}{ViṣṇuP\index{Visnupurana@\textsl{Viṣṇupurāṇa}}}

\newcommand{\VDhU}{\textsl{Vi\-ṣṇu\-dha\-rmo\-tta\-ra}\index{Visnudharmottara@\textsl{Viṣṇudharmottara}}}
\newcommand{\VDHU}{VDhU\index{Visnudharmottara@\textsl{Viṣṇudharmottara}}}

\newcommand{\VamP}{\textsl{Vā\-ma\-na\-pu\-rā\-ṇa}\index{Vamanapurana@\textsl{Vāmanapurāṇa}}}
\newcommand{\VAMP}{VarP\index{Vamanapurana@\textsl{Vāmanapurāṇa}}}

\newcommand{\VayuP}{\textsl{Vā\-yu\-pu\-rā\-ṇa}\index{Vayupurana@\textsl{Vāyupurāṇa}}}
\newcommand{\VAYUP}{VāyuP\index{Vayupurana@\textsl{Vāyupurāṇa}}}

\newcommand{\VarP}{\textsl{Va\-rā\-ha\-pu\-rā\-ṇa}\index{Varahapurana@\textsl{Varāhapurāṇa}}}
\newcommand{\VARP}{VarP\index{Varahapurana@\textsl{Varāhapurāṇa}}}

\newcommand{\VasDh}{\textsl{Vā\-si\-ṣṭha\-dha\-rma\-śā\-stra}\index{Vasisthadharmasastra@\textsl{Vāsiṣṭhadharmaśāstra}}}
\newcommand{\VASDH}{VasDh\index{Vasisthadharmasastra@\textsl{Vāsiṣṭhadharmaśāstra}}}

\newcommand{\VDh}{\textsl{Vi\-ṣṇu\-dha\-rma}\index{Visnudharma@\textsl{Viṣṇudharma}}}
\newcommand{\VDH}{VDh\index{Visnudharma@\textsl{Viṣṇudharma}}}

\newcommand{\VisnuS}{\textsl{Vi\-ṣṇu\-smṛ\-ti}\index{Visnusmrti@\textsl{Viṣṇusmṛti}}}
\newcommand{\VISNUS}{ViṣṇuS\index{Visnusmrti@\textsl{Viṣṇusmṛti}}}

\newcommand{\SivPu}{\textsl{Śi\-va\-pu\-rā\-ṇa}\index{Sivapurana@\textsl{Śivapurāṇa}}}
\newcommand{\SIVPU}{ŚivP\index{Sivapurana@\textsl{Śivapurāṇa}}}

\newcommand{\Vss}{\textsl{Vṛ\-ṣa\-sā\-ra\-saṃ\-gra\-ha}\index{Vrsasarasamgraha@\textsl{Vṛṣasārasaṃgraha}}}
\newcommand{\Vsssc}{{V\scR\scS a\-sā\-ra\-sa\scM\-gra\-ha}%
								\index{Vrsasarasamgraha@\textsl{Vṛṣasārasaṃgraha}}}
\newcommand{\VSS}{{VSS}\index{Vrsasarasamgraha@\textsl{Vṛṣasārasaṃgraha}}}

\newcommand{\SivP}{\textsl{Śivapurāṇa}\index{Sivapurana@\textsl{Śivapurāṇa}}}
\newcommand{\SIVP}{SivP\index{Sivapurana@\textsl{Śivapurāṇa}}}

\newcommand{\SDhS}{\textsl{Śi\-va\-dhar\-ma\-śās\-tra}\index{Sivadharmasastra@\textsl{Śivadharmaśāstra}}}
\newcommand{\SDHS}{ŚDhŚ\index{Sivadharmasastra@\textsl{Śivadharmaśāstra}}}

\newcommand{\SDhSangr}{\textsl{Śi\-va\-dhar\-ma\-saṅ\-gra\-ha}\index{Sivadharmasangraha@\textsl{Śivadharmasaṅgraha}}}
\newcommand{\SDHSANGR}{ŚDhSaṅgr\index{Sivadharmasangraha@\textsl{Śivadharmasaṅgraha}}}

\newcommand{\SDhU}{\textsl{Śi\-va\-dhar\-mo\-tta\-ra}\index{Sivadharmottara@\textsl{Śivadharmottara}}}
\newcommand{\SDHU}{ŚDhU\index{Sivadharmottara@\textsl{Śivadharmottara}}}

\newcommand{\SannyasUp}{\textsl{Sa\-nnyā\-so\-pa\-ni\-ṣad}\index{Sannyasopanisad@\textsl{Sanyāsopaniṣad}}}
\newcommand{\SANNYASUP}{SannyāsUp\index{Sannyasopanisad@\textsl{Sanyāsopaniṣad}}}

\newcommand{\SkandaP}{\textsl{Ska\-nda\-pu\-rā\-ṇa}\index{Skandapurana@\textsl{Skandapurāṇa}}}
\newcommand{\SKANDAP}{SkandaP\index{Skandapurana@\textsl{Skandapurāṇa}}}

\newcommand{\Hyp}{\textsl{Ha\-ṭha\-yo\-ga\-pra\-dī\-pi\-kā}\index{Hathayogapradipika@\textsl{Haṭhayogapradīpikā}}}
\newcommand{\HYP}{HYP\index{Hathayogapradipika@\textsl{Haṭhayogapradīpikā}}}



% INDIVIDUALS
\newcommand{\authorfont}[1]{{#1}}
%\newcommand{\authorfont}[1]{\textsc{#1}}
\newcommand{\Kiss}{\textsc{Kiss}\index{Kiss, Csaba}}
\newcommand{\Bader}{\textsc{Bader}}%\index{Bader}}
\newcommand{\Clark}{\textsc{Clark}}%\index{Clark}}
\newcommand{\Jowett}{\textsc{Jowett}}%\index{Jowett}}
\newcommand{\Goodall}{{\authorfont{Good\-all}}\index{Goodall, Dominic}}
\newcommand{\Kloppenborg}{\textsc{Kloppenborg}}%\index{Kloppenborg}}
\newcommand{\Dehejia}{\textsc{Dehejia}}%\index{Dehejia}}
\newcommand{\Pandeya}{\textsc{P\={a}\d{n}\d{d}eya}}%\index{{P\={a}\d{n}\d{d}eya}}}
\newcommand{\Sanderson}{\authorfont{Sand\-erson}\index{Sanderson, Alexis@\authorfont{Sanderson}, Alexis}}
\newcommand{\Hatley}{\textsc{Hat\-ley}\index{Hatley, Shaman}}
\newcommand{\Wise}{\textsc{Wise}}%\index{Wise}}
\newcommand{\Dey}{\textsc{Dey}}%\index{Dey}}
\newcommand{\Torzsok}{\textsc{T\"or\-zs\"ok}\index{Törzsök, Judit}}
\newcommand{\Haru}{\textsc{Isa{a}cson}\index{Isa{a}cson, Harunaga}}
\newcommand{\Nirajan}{{\rm \textsc{Kafle}}\index{Kafle, Nirajan}}
\newcommand{\Kafle}{{\rm \textsc{Kafle}}\index{Kafle, Nirajan}}
\newcommand{\Mallinson}{\textsc{Mallin\-son}\index{Mallin\-son, James}}
\newcommand{\White}{\textsc{White}\index{White}}
\newcommand{\Goudr}{\textsc{Goudria{a}n}\index{Goudria{a}n, Teun}}
\newcommand{\Kale}{\textsc{Kale}\index{Kale}}
\newcommand{\Bagchi}{\textsc{Bagchi}}%\index{Bagchi}}
\newcommand{\Briggs}{\textsc{Briggs}}%\index{Briggs}}
\newcommand{\Brunner}{\textsc{Brunner}\index{Brunner, H\'el\`ene}}
\newcommand{\Dyczkowski}{\textsc{Dyczkowski}}%\index{Dyczkowski}}
\newcommand{\Finn}{\textsc{Finn}}%\index{Finn}}
\newcommand{\Garzilli}{\textsc{Garzilli}}%\index{Garzilli}}
\newcommand{\Heilijgers}{\textsc{Heilijgers}}%\index{Heilijgers}}
\newcommand{\Mallik}{\textsc{Mallik}}%\index{Mallik}}
\newcommand{\McGregor}{\textsc{McGregor}}%\index{McGregor}}
\newcommand{\Rao}{\textsc{Rao}}%\index{Rao}}
\newcommand{\Vasudeva}{\textsc{Vasudeva}\index{Vasudeva, Somadeva}}
\newcommand{\Vyas}{\textsc{Vyas \&\ Kshirsagar}}%\index{Vyas \&\ Kshirsagar}}
\newcommand{\Bouy}{\textsc{Bouy}}%\index{Bouy}}
\newcommand{\Sensharma}{\textsc{Sensharma}}%\index{Sensharma}}
\newcommand{\Bisschop}{\textsc{Bisschop}}%\index{Bisschop}}
\newcommand{\Schoterman}{\textsc{Schoterman}}%\index{Schoterman}}
\newcommand{\Gonda}{\textsc{Gonda}}%\index{Gonda}}
\newcommand{\Brooks}{\textsc{Brooks}}%\index{Brooks}}
\newcommand{\Buhnemann}{\textsc{B\"uhnemann}}%\index{B\"uhnemann}}
\newcommand{\Khanna}{\textsc{Khanna}}%\index{Khanna}}
\newcommand{\Gupta}{\textsc{Gupta}}%\index{Gupta}}
\newcommand{\Hoens}{\textsc{Hoens}}%\index{Hoens}}
\newcommand{\Padoux}{\textsc{Padoux}}%\index{Padoux}}
\newcommand{\Avalon}{\textsc{Avalon}}%\index{Avalon}}
\newcommand{\MookerjeeKhanna}{\textsc{Mookerjee \&\ Khanna}}%\index{Mookerjee \&\ Khanna}}
\newcommand{\Rawson}{\textsc{Rawson}}%\index{Rawson}}
\newcommand{\Uberoi}{\textsc{Uberoi}}%\index{Uberoi}}
\newcommand{\Tapasyananda}{\textsc{Tapasyananda}}%\index{Tapasyananda}}
\newcommand{\Dvivedi}{\textsc{Dvivedi}}%\index{Dvivedi}}
\newcommand{\Mallman}{\textsc{Mallman}}%\index{Mallman}}
\newcommand{\West}{\textsc{West}}%\index{West}}
\newcommand{\Dearing}{\textsc{Dearing}}%\index{Dearing}}
\newcommand{\Chaudhuri}{\textsc{Chaudhuri}}%\index{Chaudhuri}}
\newcommand{\Deshpande}{\textsc{Deshpande}}%\index{Deshpande}}
\newcommand{\Digby}{\textsc{Digby}}%\index{Digby}}
\newcommand{\Levi}{\textsc{Levi}}%\index{Levi}}
\newcommand{\Dupuche}{\textsc{Dupuche}}%\index{Dupuche}}
\newcommand{\Rukmini}{\textsc{Rukmini}}%\index{Rukmini}}
\newcommand{\Rose}{\textsc{Rose}}%\index{Rose}}
\newcommand{\Bunce}{\textsc{Bunce}}%\index{Bunce}}
\newcommand{\Antarkar}{\textsc{Antarkar}}%\index{Antarkar}}
\newcommand{\Locke}{\textsc{Locke}}%\index{Locke}}
\newcommand{\Upadhyay}{\textsc{Upadhyay}}%\index{Upadhyay}}
\newcommand{\Diwakar}{\textsc{Diwakar}}%\index{Diwakar}}
\newcommand{\Wallis}{\textsc{Wallis}\index{Wallis, Christopher}}

\newcommand{\eme}{\textit{em.}}



%% Sigla for vss_book_xetex.tex
\newcommand{\msCa}{{\englishfont C$_{\scriptscriptstyle 94}$}}
\newcommand{\msCaacorr}{{\englishfont C$^{\scriptscriptstyle ac}_{\scriptscriptstyle 94}$}}
\newcommand{\msCapcorr}{{\englishfont C$^{\scriptscriptstyle pc}_{\scriptscriptstyle 94}$}}

\newcommand{\mssCaCbCc}{{\englishfont C$^{\scriptscriptstyle\Sigma}$}}

% original
%\newcommand{\msCb}{{\englishfont C$_\textrm{b}$}}
%\newcommand{\msCbacorr}{{\englishfont C$_\textrm{b}$\kern-.4em$^{ac}$\/}}
%\newcommand{\msCbpcorr}{{\englishfont C\raisebox{-.1em}{$_\textrm{b}$}\kern-.57em$^{pc}$\/}}
%\newcommand{\msCb}{{\rm N$^{\scriptscriptstyle C}_{\scriptscriptstyle 45}$}}
%\newcommand{\msCbacorr}{{\rm N$^{\scriptscriptstyle Cac}_{\scriptscriptstyle 45}$}}
%\newcommand{\msCbpcorr}{{\rm N$^{\scriptscriptstyle Cpc}_{\scriptscriptstyle 45}$}}
\newcommand{\msCb}{{\englishfont C$_{\scriptscriptstyle 45}$}}
\newcommand{\msCbacorr}{{\englishfont C$^{\scriptscriptstyle ac}_{\scriptscriptstyle 45}$}}
\newcommand{\msCbpcorr}{{\englishfont C$^{\scriptscriptstyle pc}_{\scriptscriptstyle 45}$}}

%\newcommand{\msCc}{{\englishfont C$_\textrm{c}$}}
%\newcommand{\msCcacorr}{{\englishfont C$_\textrm{c}$\kern-.16cm$^{ac}$\/}}
%\newcommand{\msCcpcorr}{{\englishfont C$_\textrm{c}$\kern-.16cm$^{pc}$\/}}
%\newcommand{\msCc}{{\rm N$^{\scriptscriptstyle C}_{\scriptscriptstyle 02}$}}
%\newcommand{\msCcacorr}{{\rm N$^{\scriptscriptstyle Cac}_{\scriptscriptstyle 02}$}}
%\newcommand{\msCcpcorr}{{\rm N$^{\scriptscriptstyle Cpc}_{\scriptscriptstyle 02}$}}
\newcommand{\msCc}{{\englishfont C$_{\scriptscriptstyle 02}$}}
\newcommand{\msCcacorr}{{\englishfont C$^{\scriptscriptstyle ac}_{\scriptscriptstyle 02}$}}
\newcommand{\msCcpcorr}{{\englishfont C$^{\scriptscriptstyle pc}_{\scriptscriptstyle 02}$}}

%\newcommand{\msNa}{{\englishfont N$_\textrm{a}$}}
%\newcommand{\msNaacorr}{{\englishfont N$_\textrm{a}$\kern-.10cm$^{ac}$\/}}
%\newcommand{\msNapcorr}{{\englishfont N$_\textrm{a}$\kern-.13cm$^{pc}$\/}}
%\newcommand{\msNa}{{\rm N$^{\scriptscriptstyle K}_{\scriptscriptstyle 82}$}}
%\newcommand{\msNaacorr}{{\rm N$^{\scriptscriptstyle Kac}_{\scriptscriptstyle 82}$}}
%\newcommand{\msNapcorr}{{\rm N$^{\scriptscriptstyle Kpc}_{\scriptscriptstyle 82}$}}
\newcommand{\msNa}{{\englishfont K$_{\scriptscriptstyle 82}$}}
\newcommand{\msNaacorr}{{\englishfont K$^{\scriptscriptstyle ac}_{\scriptscriptstyle 82}$}}
\newcommand{\msNapcorr}{{\englishfont K$^{\scriptscriptstyle pc}_{\scriptscriptstyle 82}$}}

\newcommand{\msNb}{{\englishfont K$_{\scriptscriptstyle 10}$}}
\newcommand{\msNbacorr}{{\englishfont K$^{\scriptscriptstyle ac}_{\scriptscriptstyle 10}$}}
\newcommand{\msNbpcorr}{{\englishfont K$^{\scriptscriptstyle pc}_{\scriptscriptstyle 10}$}}

\newcommand{\msNc}{{\englishfont K$_{\scriptscriptstyle 7}$}}
\newcommand{\msNcacorr}{{\englishfont K$^{\scriptscriptstyle ac}_{\scriptscriptstyle 7}$}}
\newcommand{\msNcpcorr}{{\englishfont K$^{\scriptscriptstyle pc}_{\scriptscriptstyle 7}$}}

\newcommand{\msNd}{{\englishfont K$_{\scriptscriptstyle 3}$}}
\newcommand{\msNdacorr}{{\englishfont K$^{\scriptscriptstyle ac}_{\scriptscriptstyle 03}$}}
\newcommand{\msNdpcorr}{{\englishfont K$^{\scriptscriptstyle pc}_{\scriptscriptstyle 03}$}}

\newcommand\msBod{{\englishfont O$_{\scriptscriptstyle 15}$}}
\newcommand\msBodacorr{{\englishfont O$_{\scriptscriptstyle 15}$}$^{\scriptscriptstyle ac}$}
\newcommand\msBodpcorr{{\englishfont O$_{\scriptscriptstyle 15}$}$^{\scriptscriptstyle pc}$}


\newcommand\msP{{\englishfont P$_{\scriptscriptstyle 57}$}}
\newcommand\msPacorr{{\englishfont P$_{\scriptscriptstyle 57}$}$^{\scriptscriptstyle ac}$}
\newcommand\msPpcorr{{\englishfont P$_{\scriptscriptstyle 57}$}$^{\scriptscriptstyle pc}$}

\newcommand\msL{{\englishfont L$_{\scriptscriptstyle 16}$}}
\newcommand\msLacorr{{\englishfont L$_{\scriptscriptstyle 16}$}$^{\scriptscriptstyle ac}$}
\newcommand\msLapcorr{{\englishfont L$_{\scriptscriptstyle 16}$}$^{\scriptscriptstyle pc}$}

\newcommand\msKoa{{\englishfont Ko$_{\scriptscriptstyle 76}$}}
\newcommand\msKoaacorr{{\englishfont Ko$_{\scriptscriptstyle 76}$}$^{\scriptscriptstyle ac}$}
\newcommand\msKoapcorr{{\englishfont Ko$_{\scriptscriptstyle 76}$}$^{\scriptscriptstyle pc}$}

\newcommand\msKob{{\englishfont Ko$_{\scriptscriptstyle 52}$}}
\newcommand\msKobacorr{{\englishfont Ko$_{\scriptscriptstyle 52}$}$^{\scriptscriptstyle ac}$}
\newcommand\msKobpcorr{{\englishfont Ko$_{\scriptscriptstyle 52}$}$^{\scriptscriptstyle pc}$}

\newcommand\msKoc{{\englishfont Ko$_{\scriptscriptstyle 77}$}}
\newcommand\msKocacorr{{\englishfont Ko$_{\scriptscriptstyle 77}$}$^{\scriptscriptstyle ac}$}
\newcommand\msKocpcorr{{\englishfont Ko$_{\scriptscriptstyle 77}$}$^{\scriptscriptstyle pc}$}

\newcommand\msM{{\englishfont M}}
\newcommand\msMacorr{{\englishfont M}$^{\scriptscriptstyle ac}$}
\newcommand\msMpcorr{{\englishfont M}$^{\scriptscriptstyle pc}$}

%\newcommand{\Ed}{{\englishfont Ed$_\textrm{N}$}}
%\newcommand{\Ed}{{\rm E$^{\scriptscriptstyle N}$}}
\newcommand{\Ed}{{\englishfont E}}







\newcommand{\blankpage}{%
\thispagestyle{empty}

\       \


\vspace{8cm}

{\large This page intentionally left blank...}

\vfill\pagebreak
}


% for translation with Sanskrit
\newcommand{\chptr}[1]{\vfill\pagebreak\thispagestyle{empty} \centerline{\large[{\thinspace#1\thinspace}]}}
\newcommand{\subchptr}[1]{%
	\centerline{{[\textit{\thinspace#1\thinspace}---}}}
\newcommand{\subsubchptr}[1]{\centerline{[\textit{\thinspace{\footnotesize#1}\thinspace}---}}

\newcommand{\trchptr}[1]{\centerline{\large[{#1}]}\indent}
\newcommand{\trsubchptr}[1]{\centerline{#1 ]}%
                 \addcontentsline{toc}{subsection}{#1}%
                        }
\newcommand{\trsubsubchptr}[1]{\centerline{{\footnotesize#1} ]}}

\newcommand{\maintext}[1]{
\leftskip0em
\noindent
%\smallskip
\textit{\englishfont #1}}

\newcommand{\nonanustubhindent}{\noindent\hspace{1.5em}}
			                

% redefine quote environment
\let\origquote\quote
\let\endorigquote\endquote
\renewenvironment{quote}{%
  \vspace{-.6em}
    \origquote
    \leftskip-.9em
    \rightskip-1.8em
    \small
    }%
{\endorigquote}


\newcommand{\translation}[1]{
%\noindent2
\smallskip
\begin{quote}
%\vskip-.8em
%\normalsize
  %\leftskip-.7em
  %\rightskip-1.5em
{\englishfont #1}
\end{quote}
%\vskip-.4em
%\smallskip
%\leftskip0em
}


% START
% FOR DRAWING MANDALAS TO SHOW STRUCTURE OF VSS




%based on http://www.texample.net/tikz/examples/mandala/


%\definecolor{csabacolor}{HTML}{e6fff0}
%\definecolor{vaisnava}{HTML}{ff661a}
%\definecolor{saiva}{HTML}{003399}
%\definecolor{general}{HTML}{1f7a7a}

% Monochrome version:
\definecolor{csabacolor}{HTML}{e6fff0}
\definecolor{vaisnava}{HTML}{cccccc}
\definecolor{saiva}{HTML}{666666}
\definecolor{general}{HTML}{ffffff}

\def\petal {
[rounded corners=0.5] %
(-1,0) -- (-.7,-1.7) -- (-1,0)%
.. controls (-1,0.6) and (-0.07,0.8).. (0,1)%
.. controls (0.07,0.7) and (1,0.6).. (1,0);
  %.. controls (0.7,-1) and (-0.7,-1).. (-1,0)%
          }

\def\petalwitharrow {
[rounded corners=0.5] %
(-1,0) -- (-.7,-1.7) -- (-1,0)%
.. controls (-1,0.6) and (-0.07,0.8).. (0,1)%
.. controls (0.07,0.7) and (1,0.6).. (1,0);
  %.. controls (0.7,-1) and (-0.7,-1).. (-1,0)%
\draw [->, very thick] (0,1) -- (0,1.3);
          }


\newcommand{\drawpetals}{%
\newcount\nums
\nums=0

\foreach \a in {0,-15,-30,...,-345}
{
\begin{scope}[rotate=\a,shift={(0,6)},xscale=1,yscale=1]
        \draw    [scale=.8] \petal;
\end{scope}

\begin{scope}[rotate=\a,shift={(0,6.3)},xscale=1,yscale=1]
        {\global\advance\nums by1}
        \node [rotate=0, scale=1]{{\the\nums}};
\end{scope}

}
}

\newcommand{\drawpetalswitharrows}{%
\newcount\nums
\nums=0

\foreach \a in {0,-15,-30,...,-345}
{
\begin{scope}[rotate=\a,shift={(0,6)},xscale=1,yscale=1]
        \draw    [scale=.8] \petalwitharrow;
\end{scope}

\begin{scope}[rotate=\a,shift={(0,6.3)},xscale=1,yscale=1]
        {\global\advance\nums by1}
        \node [rotate=0, scale=1]{{\the\nums}};
\end{scope}

}
}


\newcommand{\colorpetals}[2]{%
\begin{scope}[rotate=(#2-1)*-15,shift={(0,6)},xscale=1,yscale=1, opacity=.5]
        \filldraw    [scale=.8, draw=#1, color=#1] \petal;
\end{scope}
}


\newcommand{\colorpetalstransition}[3]{%
\begin{scope}[rotate=(#3-1)*-15,shift={(0,6)},xscale=1,yscale=1, opacity=.5]
        \shadedraw [scale=.8, draw=#1, left color=#1, right color=#2]  \petal;
\end{scope}
}


\newcommand{\drawtopicsOne}{%
\def\topics{{ {"{Brahmāṇḍa}"}, {"{\ \ Śivāṇḍa}"}, {"{\hspace{12em}Dharma = \textit{yama-niyama + varṇāśrama}}"},
{"{\textit{yama}s}"},
{"{\textit{niyama}s}"},
{"{\hspace{1em}\textit{niyama}s}"},
{"{\hspace{1em}\textit{niyama}s}"},
{"{\hspace{1em}\textit{niyama}s}"},
{"{\textit{trikāla} = 3\ \textit{guṇa}s}"},
%SAIVA
{"{}"}, {"{}"},
{"{}"},
{"{}"},
{"{}"},
{"{}"},
{"{}"},
{"{}"},
{"{}"},
% Anarth
{"{\hspace{-2em}\textit{varṇa}s}"},
{"{\textit{tattva}s}"}, {"{\hspace{-2em}Viṣṇu's theophany}"}, {"{\hspace{-6em}Anarthayaj\~na's life}"}, {"{\hspace{-6em}\textit{dharmādharma, nidrotpatti}}"},
{"{\textit{trailokya}}"} }}

\foreach \a in {0,1,2,...,23} {
        \begin{scope}[rotate=\a*-15,shift={(0,8.5)},xscale=1,yscale=1]
        \node[scale=1.3,] {\pgfmathparse{\topics[\a]}{\Large\pgfmathresult}};
        \end{scope}
    }
}


\newcommand{\drawtopicsTwo}{%
\def\topics{{ {"{\ \ Brahmāṇḍa}"}, {"{\ \ Śivāṇḍa}"}, {"{\hspace{9em}Dharma, \textit{yama}s-\textit{niyama}s}"},
{"{\textit{yama}s}"},
{"{\textit{niyama}s}"},
{"{\hspace{2em}\textit{niyama}s}"},
{"{\hspace{2em}\textit{niyama}s}"},
{"{\hspace{2em}\textit{niyama}s}"},
{"{3\ \textit{guṇa}s}"},
%SAIVA
{"{internal \textit{tīrtha}s}"},
{"{\hspace{6em}{\textit{āśrama}s:}}\ \textit{vinārthena yaj\~naḥ}"},
{"{Vipula's story}"},
{"{\hspace{6em}\phantom{\Huge Ā}summary of 10--12, \textit{garbhotpatti}}"},
{"{the body}"},
{"{\textit{jīva, śreṣṭha}}"},
{"{\textit{yoga}}"},
{"{\textit{dāna}}"},
{"{\textit{karman}}"},
% Anarth
{"{\hspace{-2em}{\textit{varṇa}s}}"},
{"{\hspace{-1em}\textit{tattva}s}"}, {"{\hspace{-2em}Viṣṇu's theophany}"}, {"{{\hspace{-6em}Anarthayaj\~na's life}}"}, {"{\textit{dharmādharma, nidrā\ \ \ \ \ \ }}"},
{"{\textit{trailokya}}"} }}

\foreach \a in {0,1,2,...,23} {
        \begin{scope}[rotate=\a*-15,shift={(0,8)},xscale=1,yscale=1]
        \node[scale=1.3,] {\pgfmathparse{\topics[\a]}{\pgfmathresult}};
        \end{scope}
    }
}


\newcommand{\drawtopicsThree}{%
\def\topics{{ {"{}"}, {"{}"}, {"{}"},{"{}"},{"{}"},{"{}"},{"{}"},{"{}"},{"{}"},{"{}"},
{"{\hspace{5em}\fbox{\textit{āśrama}s:}}\ \textit{vinārthena yaj\~naḥ}"},
{"{}"},{"{}"},{"{}"},{"{}"},{"{}"},{"{}"},{"{}"},
% Anarth
{"{\fbox{\textit{varṇa}s}}"},
{"{}"},{"{}"},{"{\fbox{Anarthayaj\~na's life}}"},{"{}"},{"{}"}} }

\foreach \a in {0,1,2,...,23} {
        \begin{scope}[rotate=\a*-15,shift={(0,9)},xscale=1,yscale=1]
        \node[scale=1.3,] {\pgfmathparse{\topics[\a]}{\Large\pgfmathresult}};
        \end{scope}
    }
}

\newcommand{\drawtopicsDating}{%
\def\topics{{ {"{}"}, {"{}"}, {"{}"},{"{}"},{"{}"},{"{}"},{"{}"},{"{}"},{"{}"},
{"{no \textit{piṅgalā}}"}, {"{order of \textit{āśrama}s, 4 \textit{kalā}s, no \textit{piṅgalā}}"},
{"{}"},{"{}"},{"{}"},{"{}"},{"{}"},{"{}"},{"{}"},
% Anarth
{"{}"},
{"{no \textit{tanmātra}s}"},{"{}"},{"{}"},{"{}"},{"{}"}} }

\foreach \a in {0,1,2,...,23} {
        \begin{scope}[rotate=\a*-15,shift={(0,9)},xscale=1,yscale=1]
        \node[scale=1.3,] {\pgfmathparse{\topics[\a]}{\Large\pgfmathresult}};
        \end{scope}
    }
}



\newcommand{\bigcircle}{%
\thispagestyle{empty}
\hspace{-3cm}
\draw [scale=1] (0,-1) circle (4.7cm);
}

\newcommand{\titlecircle}{%
\thispagestyle{empty}
\hspace{-3cm}
\draw [scale=1] (0,-1) circle (4.7cm);
}

\newcommand{\markAnarthaPresence}[1]{%
\begin{scope}[rotate=(#1-1)*-15,shift={(0,5.3)},xscale=.7,yscale=.7, opacity=.5]
        \node    [scale=2, color=black] {{\checkmark}};
\end{scope}
}

\newcommand{\markvinarthaPresence}[1]{%
\begin{scope}[rotate=(#1-1)*-15,shift={(0,5.6)},xscale=.7,yscale=.7, opacity=.5]
        \node    [scale=1.5, color=black] {{\checkmarksmall}};
\end{scope}
}

\newcommand{\centeredtext}[2]{
\node [color=#1 ]{{\Huge #2}};
}

\newcommand{\uptext}[2]{
\node [shift={(0,1)}, color=#1 ]{{{\huge #2}}};
}

\newcommand{\upuptext}[2]{
\node [shift={(0,2)}, color=#1 ]{{{\huge #2}}};
}

\newcommand{\downtext}[2]{
\node [shift={(0,-1)}, color=#1 ]{{{\huge #2}}};
}

\newcommand{\downdowntext}[2]{
\node [shift={(0,-2)}, color=#1 ]{{{\huge #2}}};
}

% or this:
\def\checkmark{\tikz\fill[scale=0.5, color=black](0,.35) -- (.25,0) -- (1,.7) -- (.25,.15) -- cycle;}
\def\checkmarksmall{\tikz\fill[scale=0.3, color=black](0,.35) -- (.25,0) -- (1,.7) -- (.25,.15) -- cycle;}


\newcommand{\colorpetalsOne}{%
% General
\colorpetals{general}{22}
\colorpetals{general}{23}
\colorpetals{general}{24}
% Vaisnava
\colorpetalstransition{general}{vaisnava}{1}
\colorpetals{saiva}{2}
\colorpetals{vaisnava}{3}
\colorpetals{vaisnava}{4}
\colorpetals{vaisnava}{5}
\colorpetals{vaisnava}{6}
\colorpetals{vaisnava}{7}
\colorpetals{vaisnava}{8}
\colorpetals{vaisnava}{9}
\colorpetalstransition{saiva}{vaisnava}{10}
\colorpetals{vaisnava}{19}
\colorpetals{vaisnava}{20}
\colorpetalstransition{vaisnava}{general}{21}
% Saiva
\colorpetals{saiva}{11}
\colorpetals{saiva}{12}
\colorpetals{saiva}{13}
\colorpetals{saiva}{14}
\colorpetals{saiva}{15}
\colorpetals{saiva}{16}
\colorpetals{saiva}{17}
\colorpetals{saiva}{18}
}

\newcommand{\textOne}{%
%\upuptext{black}{Vṛṣasārasaṃgraha chapters:}
\upuptext{teal}{Vaiśampāyana $\rightarrow$ Janamejaya}
\uptext{vaisnava}{Anarthayajña $\rightarrow$ Viṣṇu}
\centeredtext{saiva}{Nandikeśvara: Śiva $\rightarrow$ Devī}

}

\newcommand{\textTwo}{%
%\upuptext{black}{Vṛṣasārasaṃgraha chapters:}
%\upuptext{teal}{Vaiśampāyana $\rightarrow$ Janamejaya}
%\uptext{vaisnava}{Anarthayajña $\rightarrow$ Viṣṇu}
\centeredtext{black}{VSS}
%\downtext{black}{{\Large\checkmark} Anarthayajña present}
%\downdowntext{black}{\checkmarksmall\textit{yajño 'rthaṃ vinā}}

%Anarthayajña's presence:
\markAnarthaPresence{1}
\markAnarthaPresence{2}
\markAnarthaPresence{3}
\markAnarthaPresence{4}
\markAnarthaPresence{5}
\markAnarthaPresence{6}
\markAnarthaPresence{7}
\markAnarthaPresence{8}
\markAnarthaPresence{9}
\markAnarthaPresence{10}
\markAnarthaPresence{19}
\markAnarthaPresence{20}
\markAnarthaPresence{21}
\markAnarthaPresence{22}

% vinārthena yajñaḥ presence
\markvinarthaPresence{11}
\markvinarthaPresence{22}
}

\newcommand{\textDating}{%
%\upuptext{black}{Vṛṣasārasaṃgraha chapters:}
\centeredtext{saiva}{Dating the VSS}
}

% FOR DRAWING MANDALAS TO SHOW STRUCTURE OF VSS
% END




\newcommand{\lac}{\kern0.1em{\englishfont-{}-{}-{}}\kern0.1em}
\newcommand{\lk}{{\englishfont{\kern.05em\englishfont\textendash}\kern-.25em{\large\raise.15em\hbox{˘}}\kern.05em}}


% not used, just an experiment
%\newcommand{\BisschopetalUtopia}{%
%{Bisschop \&\ Kafle \&\ Lubin 2021}%
%\newwrite\outputstream%
%\immediate\openout\outputstream=myfile.tmp%
%\immediate\write\outputstream{%
%Bisschop \&\ Kafle \&\ Lubin 2021: Bisschop, Peter C. \&\
%Kafle, Nirajan \&\ Lubin, Timothy.
%\string\textsl{A Śaiva Utopia. The Śivadharma’s Revision of Brahmanical Varṇāśramadharma.
%Critical Edition, Translation \&\
%Study of the Śivāśramādhyāya of the Śivadharmaśāstra}.
%No. I in Studies in the History of Śaivism.
%Napoli: Università degli Studi di Napoli
%L’Orientale, Dipartimento Asia, Africa e Mediterraneo, 2021.}%
%\immediate\closeout\outputstream
%}
