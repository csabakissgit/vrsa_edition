
\mychapter{Preface}

\section{Aims and problems}
\frenchspacing

\noindent
What is the \textit{raison d'\^etre} of this edition? 
It is essentially a new copy, and carefully prepared new version
of a text called \Vss, based on multiple witnesses,
augmented with an analysis of the contents, with
contextualisation, and with an English translation.
As for the critical edition, while I went to great 
lengths to understand the textual history behind 
the manuscripts used, it is obviously a deeply contaminated 
version of a text transmitted through contaminated witnesses.
Nevertheless, I hope that this version is as 
close as possible to the authors' and redactors' original intentions at the time of assembling these chapters together, approximately in the seventh to tenth centuries. 
Of course we do not know if there was a single moment
when the intention to compose a new text on Dharma
under the title \Vss\ was conceived or if there was one single
`original copy,'%
		\footnote{This reminds one of James McLaverty's
                  question (as quoted in
   				  \mycitep{McGannTextualCondition}{9}):
                 ``If the Mona Lisa is in the Louvre in Paris, where is Hamlet?''}
but it this edition aim to be the most meaningful and most readable among all available copies. 

Still, the present book is just a
version of a text that likely never existed exactly 
in this form, inevitably showing
signs of being an eclectic edition. 
Moreover, it may unintentionally exhibit 
characteristics of the 21th century 
(beyond the modern Devanāgarī typeface
or occasional choices based on our contemporary
understandings and misunderstandings) 
mixed with characteristics of the first millenium. 
We know that `[a]ll editing is an act of interpretation.'% 		
		\footnote{\mycitep{McGannTextualCondition}{27}.}
Many of the editorial decisions I made were based
on opinions expressed by colleagues during our
regular reading sessions. Thus this edition is a result
of the interpretative efforts of a group of scholars, and
this may sometimes, though hopefully rarely, have caused contradictions.

To complicate matters further, we are publishing this long text
in two volumes, with the second volume still in progress
when the first is released. This may produce various problems:
of interpretation, of internal references, of repetition, 
and most importantly, of presenting a text with
embedded and recurring layers cut in half. To counteract
some of these issues, I finished editing and 
studying the most significant chapters in 
the second part of the text (although all chapters seem 
increasingly significant as the editorial process progresses)
by the time I completed the first part. 
Relevant passages from the second part can be found in 
the Appendices. \CHECK\ A further minor issue arises when
I discuss topics that I have already covered in
\mycite{KissVolume2021}: some overlaps are inevitable.

What is the purpose of this edition? The main 
objective of the \textsc{śiva\-dharma project} 
has been to better understand the function of 
individual texts within the so-called Śivadharma corpus, 
as well as their relations and interconnectedness, or
lack thereof, and thus to grasp 
the \emph{raison d'être} of the corpus itself. 
My attempt is rather simplistic: to understand
what the \Vss\ tried to convey when it was composed, and
to explore why this text got inserted in those multiple-text 
manuscripts that transmit the so-called Śivadharma corpus; but even if we do not fully understand
the purpose and function of the \Vss, 
to make a pre-eleventh-century
Sanskrit text easily accessible in the twenty-first century is,
I believe, a noble aspiration.

\vfill
\pagebreak





