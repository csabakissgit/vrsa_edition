
\thispagestyle{empty}
\ \vskip3em

\chapter*{Acknowledgements}
\label{acknowledgements}
\vskip1cm

\noindent
I am grateful to Alexis Sanderson, Dominic Goodall and 
Harunaga Isaacson for initiating me into the philological 
study of Śaivism, and to Florinda De Simini for
encouraging me to apply for a position in her 
\textsc{\hbox{śivadharma} project} (ERC no.\thinspace 803624),
for sharing with me all the relevant manuscript material and
in general leading the project in the most friendly and 
generous way through difficult Covid-affected years.  
While working on the \textsl{Vṛṣa\-sāra\-saṃgraha},
I have also been affiliated with another ERC project, 
the \textsc{dharma project} (ERC no.\thinspace 809994), 
and I'm grateful to all my colleagues involved in that enterprise,
including Arlo Griffith, Emmanuel Francis, Annette Schmiedchen, 
Astrid Zotter, and Dániel Balogh.

During my visit to the National Archives in Kathmandu, 
the staff were helpful and professional. I wish to express my thanks to Jyoti Neupane, Manita Neupane, Saubhagya Pradhananga, Rubin Shrestha, Sahan Ranjitkar, and all %sribin7501@gmail.com
other members. Sushmita Das made great efforts to acquire the manuscripts in Calcutta.

My colleagues and friends working in Naples or visiting Naples for shorter periods have helped me on a daily basis, during our regular reading sessions and in every other possible way, and I am thankful to them: to Florinda De Simini, Nirajan Kafle, Kengo Harimoto, Giulia Buriola, Alessandro Battistini, Lucas den Boer, Torsten Gerloff, Kenji Takahashi, Francesco Sferra, Dorotea Operato, Daniela Cappello, Michael Bluett, Marco Franceschini, Martina Dello Buono, Chiara Livio, Margherita Trento, Nina Mirnig, Timothy Lubin, S.A.S. Sarma, R.\thinspace Sathyanarayanan, and others.
% students in Naples?

Colleagues I have known for countless years, such as Judit Törzsök, Dominic Goodall, Harunaga Isaacson, Csaba Dezső 
and Gergely Hidas, are always the first
to help my work and support me in every possible way.

I am infinitely grateful to my family for always supporting me unwaveringly.

\bigskip

\noindent
{\footnotesize
The present publication is a result of the project \textsc{dharma} `The
Domestication of ``Hindu'' Asceticism and the Religious Making of South and Southeast Asia'.  This project has received funding from the European Research Council (ERC) under the European Union's Horizon 2020 research and innovation programme (grant agreement no.\thinspace 809994).  This book reflects the views of the author only.  The funding body is not responsible for any use that
may be made of the information contained therein.}

%CHECK what to add for the Śivadharma project?

\vfill
\pagebreak
