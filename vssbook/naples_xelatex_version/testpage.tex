
\fancyhead[CE]{\textit{\footnotesize ‘...not satisfied with the Mahābhārata...’ (śrutvā bhāratasaṃhitām atṛptaḥ)}}
\fancyhead[CO]{\textit{\footnotesize ‘...not satisfied with the Mahābhārata...’ (śrutvā bhāratasaṃhitām atṛptaḥ)}}

\noindent
primary mission of the Vṛṣasārasaṃgraha must have been similar to that of
the Lalitavistara, another, less successfully surviving, text of the Śivadhar\-ma 
corpus: the Vṛṣasārasaṃgraha too must have been aiming at ‘harmo\-nising aspects of Śaiva and Vaiṣṇava dharma’ (De Simini and Mirnig 2017,
649) and probably of a number of related philosophical schools and reli\-gious currents.

There seems to be even more to the Vṛṣasārasaṃgraha’s aspirations. It
would appear difficult to find any further leitmotif in this impressively rich
material, in which innumerable traditions intermingle, or to understand
what other role this text could have played in the formation of the Śivadha\-rma corpus, if one thing did not stand out clearly: the figure of Anarthayajña.

\vspace{1em}
\noindent
2. \textit{Anarthayajña’s sacrifice and the āśramas}
\vspace{.3em}

\noindent
As we have seen, Anarthayajña is the interlocutor of the sections that can be
labelled Vaiṣṇava and his name also appears in other parts of the text. That
he is part of a Vaiṣṇava setting in chapters one to ten and nineteen to twen­
ty-one is also certain from the observation that when he has answered all of
Vigatarāga’s (Viṣṇu’s) questions in detail, and when Viṣṇu reveals himself,
they are described as departing to Viṣṇuloka together,27 thus offering the
impression that Anarthayajña is a devotee of Viṣṇu. One could argue that
Viṣṇu’s position as a pupil and the fact that he is being taught Śaiva material
(in the Śaiva chapters) point towards the possibility that Anarthayajña is
a Śaiva who converts Viṣṇu, thus turning most of the Vṛṣasārasaṃgraha
into a Śaiva-oriented text; but the episode in which Viṣṇu steps forward and
Anarthayajña praises him throughout thirteen jagatī stanzas (21.9–21) be-

The title of the commentary and everything supplied between
double square brackets is from the editors. The verse numbers of
the verses that are commented upon in the commentary are given
between double square brackets; the beginning and end is also
indicated by double square brackets. Similarly, line numbers of the
manuscript are given in double square brackets. The pratīkas of the
root text are marked in bold. Application of sandhi rules are silently
standardised here. In general we follow the placement of the dañḍas
by the manuscript, but we have occasionally placed them according
to our own understanding as well.

\vspace{1em}
\noindent
4. \textit{A Note on the Translation}

\vspace{.3em}
\noindent
In order to make the commentary accessible to a broader readership,
we have included a running English translation of it. In translating



\vfill
\pagebreak
