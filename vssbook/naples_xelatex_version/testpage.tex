
\fancyhead[CO]{\textit{\footnotesize Introduction to the Commentary}}
\fancyhead[CE]{\textit{\footnotesize Introduction to the Commentary}}

\noindent
of the text. In the following two cases, the \textit{pratiīka}s quoted by the
commentary are found only in G$^Ki$: \textit{bhāñḍagūḍhakaṃ} (after 4d) and
\textit{jñānapāragam} (83b).
Furthermore, the following is a pratīka found
only in G$^{\scriptscriptstyle Ki}$ and G$^{\scriptscriptstyle Lac}_{\scriptscriptstyle 40}$ : \textit{śāntilakṣaṇam} (99d).
There is also an instance
where the quoted \textit{pratīka} is shared by two South Indian manuscripts
(G$^{\scriptscriptstyle Ki}$, M$^{\scriptscriptstyle Tr}_{\scriptscriptstyle 63}$)
as well as one Nepalese manuscript (N$^{\scriptscriptstyle Ko}_{\scriptscriptstyle 77}$): \textit{ratau} (114d).

In another case, the \textit{pratīka} is found in one South Indian manuscript
(G$^{\scriptscriptstyle Ki}$) and three Nepalese sources
(N$^{\scriptscriptstyle K}_{\scriptscriptstyle 82}$, N$^{\scriptscriptstyle P}_{\scriptscriptstyle 57}$, and E$^{\scriptscriptstyle N}$):
\textit{nāryyo'pi} (72c).
Furthermore, another \textit{pratīka} is found in two South Indian sources (G$^{\scriptscriptstyle Ki}$
and P$^{\scriptscriptstyle T}_{\scriptscriptstyle 72}$)
and in the Kashimiri manuscript (Ś$^{\scriptscriptstyle S}_{\scriptscriptstyle 67}$):
\textit{mokṣaḥ satye pratiṣṭhitaḥ} (108d). The commentary furthermore provides two
\textit{pratīka}s that are
not found in any of the South Indian sources, but which occur in two
Nepalese sources (N$^{\scriptscriptstyle K}_{\scriptscriptstyle 28}$
and N$^{\scriptscriptstyle K}_{\scriptscriptstyle 12a}$): \textit{gūhate} (105b).
Similarly, the following \textit{pratīka}
only features in one Nepalese source (N$^{\scriptscriptstyle C}_{\scriptscriptstyle 45}$):
\textit{śivakathābhaktaḥ}~(53a).

\vspace{1em}
\noindent
3. \textit{Editorial Policies for the Commentary}

\vspace{.3em}
\noindent
The critically edited text is presented as the running text. The apparatus
is divided into two registers. On pages that display both registers, the
upper register records testimonia and notes while the bottom register
reports the variants found in the manuscript. Each entry begins with a
line number in boldface (e.g. 21). Then follows the reading adopted
in the main text, followed by a lemma sign (]) and the source of the
adopted reading. A semicolon separates the adopted reading (to its
left) from the variants (to its right). Any siglum that is followed by
superscript ac indicates the reading of a source before correction
(=\textit{ante correctionem}) and a siglum followed by superscript pc indicates
the reading of a source after correction (=\textit{post correctionem}).

The title of the commentary and everything supplied between
double square brackets is from the editors. The verse numbers of
the verses that are commented upon in the commentary are given
between double square brackets; the beginning and end is also
indicated by double square brackets. Similarly, line numbers of the
manuscript are given in double square brackets. The pratīkas of the
root text are marked in bold. Application of sandhi rules are silently
standardised here. In general we follow the placement of the dañḍas
by the manuscript, but we have occasionally placed them according
to our own understanding as well.

\vspace{1em}
\noindent
4. \textit{A Note on the Translation}

\vspace{.3em}
\noindent
In order to make the commentary accessible to a broader readership,
we have included a running English translation of it. In translating



\vfill
\pagebreak
