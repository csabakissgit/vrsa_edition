%CGPT 11/04/2025
\mychapter{Preface}

\section{Aims and problems}
\frenchspacing

\noindent
What is the \textit{raison d'\^etre} of this edition? 
It is essentially a new copy, a carefully prepared new version
of a medi\ae val Sanskrit text called \Vss, based on multiple witnesses,
augmented with an analysis of its contents,
contextualisation, and an annotated English translation.
As for the critical edition, while I went to great 
lengths to understand the textual history behind 
the manuscripts used, it is, quite obviously, a deeply contaminated 
version of a text transmitted through contaminated witnesses.
Nevertheless, I hope that this version comes as 
close as possible to what the authors' and redactors'
original intentions might have been at the time of assembling
these chapters---approximately between the seventh and tenth centuries.

Of course, we do not know if there was a single moment
when the intention to compose a new text on Dharma---i.e. `Hindu' religious duties---under
the title \Vss\ was conceived, or whether there was one single `original copy,'%
		\footnote{This reminds us of James McLaverty's famous question (as quoted in
   				  \mycitep{McGannTextualCondition}{9}):
                 `If the Mona Lisa is in the Louvre in Paris, 	
                 where is Hamlet?'}
but the present edition definitely aims to be the most meaningful and
most readable among all available copies.

Still, this book is only a
version of a text that likely never existed exactly 
in this form, inevitably displaying
signs of being an eclectic edition. 
Moreover, it may unintentionally exhibit 
characteristics of the twenty-first century 
(beyond the modern Devanāgarī typeface or occasional choices\linebreak
shaped by our contemporary understandings---and misunderstandings)\linebreak
mixed with characteristics of the first millennium.
We know that `[a]ll editing is an act of interpretation.'% 		
		\footnote{\mycitep{McGannTextualCondition}{27}.}
Many of the editorial decisions I made were influenced by,
sometimes based on, opinions expressed by colleagues during our
regular Śivadharma reading sessions. Thus, this edition is a result
of the interpretative efforts of a group of scholars---and
this may sometimes, though hopefully rarely, have led to contradictions.
All remaining shortcomings are, of course, my responsibility.

To complicate matters further, we are publishing this long text
in two volumes, with the second volume still in progress
when the first is released. This may produce various problems:
of interpretation, of internal references, of repetition, 
and, most importantly, of presenting a text with
embedded and recurring layers cut in half. To mitigate
some of these issues, I completed the editing and
study of the most significant chapters in 
the second part of the text
before finishing the first part
(although, as the editorial process progresses, all chapters seem increasingly significant). 
A further, minor issue arises when I discuss topics that I have already covered in \mycite{KissVolume2021}: 
some overlap is inevitable.
%Relevant passages from the second part can be found in the Appendices.

What, then, is the purpose of this edition? The main 
objective of the\linebreak \textsc{śiva\-dharma project} 
has been to better understand the function of 
individual texts within the so-called Śivadharma corpus---as
well as their relations and interconnectedness, or their
lack thereof---and thus to grasp 
the \emph{raison d'être} of the corpus itself. 
My attempt here is rather modest: to understand
what the \Vss\ tried to convey when it was composed, and
to explore why this text came to be inserted into the multiple-text 
manuscripts that transmit the so-called Śivadharma corpus.
But even if we do not fully understand the purpose and function of the \Vss, 
I believe that to make a pre-eleventh-century
Sanskrit text easily accessible in the twenty-first century is,
I believe, a noble aspiration.

And as a bonus, the \Vss\ is a colourful and fascinating text that never fails to intrigue and entertain its reader:
it contains philosophical and yogic teachings, and fanciful narratives, in a lovely dialect of Sanskrit,
clues for understanding the history of Śaivism and its intermingling with Vaiṣṇavism,
as well as swearing and humour. Enjoy!

\vfill
\pagebreak





