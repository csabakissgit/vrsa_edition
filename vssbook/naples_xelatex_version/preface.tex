
\mychapter{Preface}

\section{Aims and problems}
\frenchspacing

\noindent
What is this edition? It is not much more than
 a new copy, and carefully prepared new version
of a text called \Vss, based on a number of witnesses,
augmented with an analysis of the contents, with
contextualisation, and with an English translation.
As for the critical edition, while I went to great 
lengths to understand the textual history behind 
the manuscripts used, it is obviously a deeply contaminated 
version of a text transmitted through contaminated witnesses.
Nevertheless, it is hopefully a version that is as 
close to the authors' and redactors' original intention
around the time they assembled these chapters together, 
approximately in the seventh to tenth centuries, as possible. 
Of course we do not know if there was a single moment
when the intention to compose a new text on Dharma
under the title \Vss\ was born or if there was one single
`original copy',%
		\footnote{This reminds one of James McLaverty's
                  question (as quoted in
   				  \mycitep{McGannTextualCondition}{??}):
                 ``If the Mona Lisa is in the Louvre in Paris, where is Hamlet?''}
but it is hopefully the most meaningful and most readable
among all available copies. Still, the present book is just a
version of a text that surely has never existed exactly 
in this very form, inevitably showing
signs of being an eclectic edition. 
Furthermore, it may show unintentional 
characteristics of the 21th century 
(ones that go beyond the modern Devanāgarī typeface
or occasional choices based on our modern understandings and
misunderstandings) 
mixed with characteristics of the first millenium. 
We know that `[a]ll editing is an act of interpretation.'% 		
		\footnote{\mycitep{McGannTextualCondition}{??}.}
		\todo{Find a hard copy of McGann's Textual Condition.}
And many of the editorial decisions I made were based
on opinions expressed by colleagues during our
regular reading sessions. Thus this edition is a result
of the interpretative efforts of a group of scholars, and
this may sometimes, but hopefully rarely, have caused contradictions.

And as to complicate things, we are publishing this long text
in two volumes, and the second volume is still in the making
when the first comes out. This may produce various problems:
of interpretation, of internal references, of repetition, 
and most importantly of presenting a text of
embedded and recurring layers cut in half. To counteract
some of these problems, I had finished editing and 
studying some of the most significant passages in 
the second part of the text by the time I let the first one
out of my hands; some of these the reader can find in 
the Appendices. A further minor problem arises when
I discuss topics that I have already touched upon in
\mycite{KissVolume2021}: some overlaps are inevitable.

And what is the purpose of this edition? The main 
objective of the \textsc{śiva\-dharma project} 
has been to understand better the function of 
individual texts within the so-called Śivadharma corpus,
and thus the \emph{raison d'être} of the corpus itself. 
My attempt is rather simplistic: it is to understand
what the \Vss\ tried to convey when when it was composed and
to try to see why this text got inserted in those multi-text 
manuscripts that usually transmit the so-called Śivadharma corpus.
But even without this ideal to fully understand the purpose
and function of the \Vss, to make a pre-eleventh-century
Sanskrit text easily available in the twenty-first century is,
I believe, a noble aspiration.

\vfill
\pagebreak





