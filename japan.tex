\fejno=0\versno=0
\centerline{\Huge\devanagarifont वृषसारसंग्रहः  }

 
\vers


\vers


\vers


\vers


\vers


\vers


\vers


\vers


\vers


\vers


\vers


\vers


\vers


\vers


\vers


\vers


\vers


\vers


\vers


\vers


\vers


\vers


\vers


\vers


\vers


\vers


\vers


\vers


\vers


\vers


\vers


\vers


\vers


\vers


\vers


\vers


\vers


\vers


\nemslokalong


\nemslokanormal


\vers


\vers


\vers


\vers


\vers


\vers


\vers

\versno=0\fejno=18
\thispagestyle{empty}

\fancyhead[CO]{{\footnotesize\devanagarifont वृषसारसंग्रहे  }}
\fancyhead[CE]{{\footnotesize\devanagarifont अष्टादशमो ऽध्ययः  }}
\fancyhead[LE]{}
\fancyhead[RE]{}
\fancyhead[LO]{}
\fancyhead[RO]{}
\szam\bek

\centerline{\Large\devanagarifont [   अष्टादशमो ऽध्ययः  ]}{\vrule depth10pt width0pt} 

\alalfejezet{पूर्वकर्मविपाकः }
 
\vers


{\devanagarifont देव्युवाच {\dandab}\dontdisplaylinenum  }%
 
\nemsloka 
{\devanagarifont भुक्त्वा तु भोगान्सुचिरं यथेष्टं }%
  \dontdisplaylinenum    \var{{\devanagarifont \numemph\va तु\lem \msCa\msPaperA; \om\ \msNa}}% 

\nemslokab

{\devanagarifont पुण्यक्षयान्मर्त्यमुपागतानाम्  \danda\dontdisplaylinenum }%
 
\nemslokac

{\devanagarifont चिह्नानि तेषां कथयस्व मे ऽद्य }%
  \dontdisplaylinenum

\nemslokad

{\devanagarifont यथाक्रमं कर्मफलं विशेषात् {॥१८:१॥} \veg\dontdisplaylinenum }%
 

\alalfejezet{स्वर्गान्मर्त्यमुपागताः }
 
\vers


{\devanagarifont महेश्वर उवाच {\dandab}\dontdisplaylinenum  }%
     \var{{\devanagarifont \numemph\vo महेश्वर\lem \msCa; भगवान् \msNa\msPaperA}}% 

\nemsloka 
{\devanagarifont सदान्नदाता कृपणार्तिदीनां }%
  \dontdisplaylinenum
\nemslokab

{\devanagarifont स वर्षकोट्यायुतमीशलोके  \danda\dontdisplaylinenum }%
     \var{{\devanagarifont \numnoemph\vb \raise.15em\hbox{॰}युतमीशलोके\lem \msCapcorr\msNa\msPaperA; 
\raise.15em\hbox{॰}युतमीनशलोके \msCaacorr}}% 

\nemslokac

{\devanagarifont भुक्त्वा च भोगान्सममप्सरोभिः }%
  \dontdisplaylinenum    \var{{\devanagarifont \numnoemph\vc \raise.15em\hbox{॰}न्सममप्सरोभिः\lem \msCa\msNa; \raise.15em\hbox{॰}न्सरोभिः \msPaperA}}% 


\nemslokad

{\devanagarifont प्रक्षीणपुण्यः पुनरेति मर्त्यम् {॥१८:२॥} \veg\dontdisplaylinenum }%
     \var{{\devanagarifont \numnoemph\vd मर्त्यम्\lem \msCa\msPaperA; मर्त्ये \msNa}}% 

\ujvers\nemsloka {
{\devanagarifont जायन्ति दिव्येषु कुलेषु पुंसः }%
  \dontdisplaylinenum}    \var{{\devanagarifont \numemph\va दिव्येषु\lem \msCa\msNa; ते दिव्ये \msPaperA\ \unmetr}}% 

\nemslokab

{\devanagarifont सस्त्रीसमृद्धे बहुभृत्यपूर्णे  \danda\dontdisplaylinenum }%
 
\nemslokac

{\devanagarifont गौरश्वरत्नादिधनाकुलेषु }%
  \dontdisplaylinenum    \var{{\devanagarifont \numnoemph\vc गौर\raise.15em\hbox{॰}\lem \msCapcorr\msNa\msPaperA; गौरव\raise.15em\hbox{॰} \msCaacorr\oo 
\raise.15em\hbox{॰}रत्ना\raise.15em\hbox{॰}\lem \msNa; \raise.15em\hbox{॰}रन्ना\raise.15em\hbox{॰} \msCa\msPaperA}}% 


\nemslokad

{\devanagarifont रूपोज्ज्वलः कान्तिसमायुतश्च {॥१८:३॥} \veg\dontdisplaylinenum  }%
     \var{{\devanagarifont \numnoemph\vd रूपोज्ज्वलः\lem \eme; रूपोज्ज्वल\raise.15em\hbox{॰} \msCa\msNa, रूपर्ज्वल\raise.15em\hbox{॰} \msPaperA\oo 
\raise.15em\hbox{॰}समायुतश्च\lem \msCapcorr\msNa; ˚समायुतञ्च \msCaacorr}}% 

\ujvers\nemsloka {
{\devanagarifont वस्त्रं सुसत्कृत्य द्विजस्य दानात् }%
  \dontdisplaylinenum}    \var{{\devanagarifont \numemph\va \raise.15em\hbox{॰}सत्कृत्य\lem \msCa; \raise.15em\hbox{॰}संस्कृत्य \msNa, ˚संकृत्य \msPaperA}}% 

\nemslokab

{\devanagarifont स्वर्गेषु मोदन्ति स वर्षकोट्यः  \danda\dontdisplaylinenum }%
 
\nemslokac

{\devanagarifont पुनश्च ते मर्त्यमुपागताश्च }%
  \dontdisplaylinenum

\nemslokad

{\devanagarifont चिह्नं महच्छ्रीपदमाप्नुवन्ति {॥१८:४॥} \veg\dontdisplaylinenum }%
     \var{{\devanagarifont \numnoemph\vd चिह्नं म\raise.15em\hbox{॰}\lem \msNa\msPaperA; चिह्न\uncl{म्म\raise.15em\hbox{॰}}}}% 

\ujvers\nemsloka {
{\devanagarifont कूपप्रपापुष्करणीप्रदाता }%
  \dontdisplaylinenum}    \var{{\devanagarifont \numemph\va \raise.15em\hbox{॰}पुष्करिणी\raise.15em\hbox{॰}\lem \msNa\msPaperA; \raise.15em\hbox{॰}पुष्करणी\raise.15em\hbox{॰} \msCa}}% 

\nemslokab

{\devanagarifont स लोकमाप्नोति जलेश्वरस्य  \danda\dontdisplaylinenum }%
 
\nemslokac

{\devanagarifont ततः स तस्माच्च्युतिमाप्य लोकात् }%
  \dontdisplaylinenum    \var{{\devanagarifont \numnoemph\vc लोकात्\lem \msCa\msPaperA; लोके \msNa}}% 


\nemslokad

{\devanagarifont सुखी सुतृप्तेषु कुलेषु जायेत् {॥१८:५॥} \veg\dontdisplaylinenum }%
     \var{{\devanagarifont \numnoemph\vd \raise.15em\hbox{॰}तृप्तेषु\lem \msCa\msPaperA; \raise.15em\hbox{॰}तप्तेषु \msNa}}% 

\ujvers\nemsloka {
{\devanagarifont रत्निप्रमाणादपि हेमदानात् }%
  \dontdisplaylinenum}    \var{{\devanagarifont \numemph\va रत्नि\raise.15em\hbox{॰}\lem \msCa\msNa; रत्न\raise.15em\hbox{॰} \msPaperA}}% 

\nemslokab

{\devanagarifont सुरेन्द्रलोकं समवाप्नुवन्ति  \danda\dontdisplaylinenum }%
 
\nemslokac

{\devanagarifont तस्माच्च्युतो मर्त्यमुपागतानां }%
  \dontdisplaylinenum

\nemslokad

{\devanagarifont चिह्नं समृद्धिर्धनधान्यलक्ष्म्याः {॥१८:६॥} \veg\dontdisplaylinenum }%
     \var{{\devanagarifont \numnoemph\vd चिह्नं\lem \msNa\msPaperA; चिह्न \msCa\oo 
\raise.15em\hbox{॰}लक्ष्म्याः\lem \msNa; लक्ष्याः \msCa, ल\uncl{क्ष्या}}}% 

\ujvers\nemsloka {
{\devanagarifont अदूष्य भूमीवरविप्रदानात् }%
  \dontdisplaylinenum}
\nemslokab

{\devanagarifont स लोकमाप्नोति सुरेश्वरस्य  \danda\dontdisplaylinenum }%
     \var{{\devanagarifont \numnoemph\vb लोकमाप्नोति\lem \msCa\msNa; लोक प्राप्नोति \msPaperA}}% 

\nemslokac

{\devanagarifont भुक्त्वा तु भोगान्च्युत मर्त्यलोके }%
  \dontdisplaylinenum    \var{{\devanagarifont \numnoemph\vc भोगान्च्युत\lem \msCa\msNa; भोगानच्युत \msPaperA}}% 


\nemslokad

{\devanagarifont चिह्नं लभेद्वै विषयाधिपत्वम् {॥१८:७॥} \veg\dontdisplaylinenum }%
     \var{{\devanagarifont \numnoemph\vd लभे\raise.15em\hbox{॰}\lem \msCa\msPaperA; भवे\raise.15em\hbox{॰} \msNa}}% 

\ujvers\nemsloka {
{\devanagarifont द्विजस्य सत्कृत्य तिलप्रदाता }%
  \dontdisplaylinenum}
\nemslokab

{\devanagarifont स लोकमाप्नोति च केशवस्य  \danda\dontdisplaylinenum }%
 
\nemslokac

{\devanagarifont भ्रष्टस्ततो मर्त्यमुपागतस्तु }%
  \dontdisplaylinenum

\nemslokad

{\devanagarifont चिह्नं लभेदक्षयमर्थलाभम् {॥१८:८॥} \veg\dontdisplaylinenum }%
     \var{{\devanagarifont \numnoemph\vd लभे\raise.15em\hbox{॰}\lem \msCa\msPaperA; भवे\raise.15em\hbox{॰} \msNa}}% 

\ujvers\nemsloka {
{\devanagarifont गवां सुरूपां विधिवद्द्विजानाम् }%
  \dontdisplaylinenum}    \var{{\devanagarifont \numemph\va सु\raise.15em\hbox{॰}\lem \msPaperA; स्व\raise.15em\hbox{॰} \msCa\msNa}}% 

\nemslokab

{\devanagarifont दत्त्वा च गोकोलमवाप्नुवन्ति  \danda\dontdisplaylinenum }%
 
\nemslokac

{\devanagarifont कल्पावसाने समुपेत्य मर्त्ये }%
  \dontdisplaylinenum

\nemslokad

{\devanagarifont चिह्नं गवाढ्यं शतगोयुतं च {॥१८:९॥} \veg\dontdisplaylinenum }%
     \var{{\devanagarifont \numnoemph\vd चिह्नं\lem \msCa\msNa; चिह्न \msPaperA}}% 

\ujvers\nemsloka {
{\devanagarifont स्वर्गं गतानां पुरुषस्य चिह्नं }%
  \dontdisplaylinenum}
\nemslokab

{\devanagarifont धनाढ्यता श्री सुखभोगलाभम्  \danda\dontdisplaylinenum }%
 
\nemslokac

{\devanagarifont आयुर्यशोरूपकलत्रपुत्रं }%
  \dontdisplaylinenum

\nemslokad

{\devanagarifont सम्पद्विभूतिकुलकीर्तिमर्थम् {॥१८:१०॥} \veg\dontdisplaylinenum }%
 
\vers


\vers


\vers

\versno=19

\vers


\vers


\vers


\vers


\vers


\vers


\vers


\vers


\vers


\vers


\vers


\vers


\vers


\vers


\vers


\vers


\vers


\vers


\vers

\bekveg\szamveg
\vfill
\phpspagebreak

\versno=0\fejno=24
\thispagestyle{empty}

\fancyhead[CO]{{\footnotesize\devanagarifont वृषसारसंग्रहे  }}
\fancyhead[CE]{{\footnotesize\devanagarifont चतुर्विंशतिमो ऽध्यायः  }}
\fancyhead[LE]{}
\fancyhead[RE]{}
\fancyhead[LO]{}
\fancyhead[RO]{}
\szam\bek

\centerline{\Large\devanagarifont [   चतुर्विंशतिमो ऽध्यायः  ]}{\vrule depth10pt width0pt} 
\vers


{\devanagarifont ... \thinspace{\dandab} \dontdisplaylinenum  }%
 
%Verse 24:1

{\devanagarifont ...  {॥२४:१॥} \veg\dontdisplaylinenum }%
 \versno=60

\vers


{\devanagarifont एतद्भूर्लोकविस्तारो ह्यत ऊर्ध्वं भुवः स्मृतः \thinspace{\dandab} \dontdisplaylinenum }%
     \var{{\devanagarifont \numemph\vb ह्यत ऊर्ध्वं\lem \mssCaCbCc; ह्यतद्र्ध्व \Ed}}% 

%Verse 24:61

{\devanagarifont स्वर्लोकस्य परेणैव महर्लोकमतः परम् {॥२४:६१॥} \veg\dontdisplaylinenum }%
     \var{{\devanagarifont \numnoemph\vc स्वर्लोकस्य\lem \mssCaCbCc; स्वर्ल्लोकास्य \Ed}}% 

{\devanagarifont जनर्लोकस्तपः सत्यं क्रमशः परिकीर्तितम् \thinspace{\dandab} \dontdisplaylinenum }%
     \var{{\devanagarifont \numemph\va जनर्लोकस्त\raise.15em\hbox{॰}\lem \msCa\msCb; जनलोक त\raise.15em\hbox{॰} \msCc, जनलोकस{  }त\raise.15em\hbox{॰} \Ed}}% 
    \var{{\devanagarifont \numnoemph\vb \raise.15em\hbox{॰}कीर्तितम्\lem \msCa\msCb\Ed; \raise.15em\hbox{॰}कीर्तितः \msCc}}% 

%Verse 24:62

{\devanagarifont ब्रह्मलोकः स्मृतः सत्यं विष्णुलोकमतः परम् {॥२४:६२॥} \veg\dontdisplaylinenum }%
     \var{{\devanagarifont \numnoemph\vc \raise.15em\hbox{॰}लोकः\lem \msCa\msCb\Ed; \raise.15em\hbox{॰}लोक \msCc}}% 


\alalfejezet{शिवलोकः }
 
{\devanagarifont तस्मात्परेण बोधव्यं दिव्यध्यानपुरं महत् \thinspace{\dandab} \dontdisplaylinenum }%
     \var{{\devanagarifont \numemph\va तस्मात्परेण बोधव्यं\lem \msCb\msCc\Ed; त{\lost}{\lost}{\lost}{\lost}{\lost}धव्यन् \msCa}}% 
    \var{{\devanagarifont \numnoemph\vb दिव्य\raise.15em\hbox{॰}\lem \msCaiEd; दिव्यं \msCb}}% 

%Verse 24:63

{\devanagarifont सहस्रभौमप्रासादं वैडूर्यमणितोरणम् {॥२४:६३॥} \veg\dontdisplaylinenum }%
     \var{{\devanagarifont \numnoemph\vd वैडूर्य\raise.15em\hbox{॰}\lem \msCa\msCb; वैदूर्य\raise.15em\hbox{॰} \Ed}}% 

{\devanagarifont नानारत्नविचित्राणि नानाभूतगणाकुलम् \thinspace{\dandab} \dontdisplaylinenum }%
 
%Verse 24:64

{\devanagarifont सर्वकामसमृद्धानि पूर्णं तत्र मनोहरैः {॥२४:६४॥} \veg\dontdisplaylinenum }%
 
{\devanagarifont तत्र सिंहासने दिव्ये सर्वरत्नविभूषिते \thinspace{\dandab} \dontdisplaylinenum }%
 
%Verse 24:65

{\devanagarifont तत्रास्ते भगवान्रुद्रः सोमाङ्कितजटाधरः {॥२४:६५॥} \veg\dontdisplaylinenum }%
     \var{{\devanagarifont \numemph\vd \raise.15em\hbox{॰}धरः\lem \msCb\Ed; \uncl{ध}{\lost} \msCa}}% 

{\devanagarifont त्र्यक्षस्त्रिभुवनश्रेष्ठस्त्रिशूली त्रिदशाधिपः \thinspace{\dandab} \dontdisplaylinenum }%
     \var{{\devanagarifont \numemph\va त्र्यक्षस्त्रि\raise.15em\hbox{॰}\lem \corr; {\lost}{\lost}स्त्रि˚ \msCa, त्र्यक्षर\raise.15em\hbox{॰} \msCb\ \unmetr, त्र्यक्षत्रि\raise.15em\hbox{॰} \Ed}}% 

%Verse 24:66

{\devanagarifont देव्या सह महाभागो गणैश्च परिवारितः {॥२४:६६॥} \veg\dontdisplaylinenum }%
 
{\devanagarifont स्कन्दनन्दिपुरोगश्च गणकोटीशताकुलः \thinspace{\dandab} \dontdisplaylinenum }%
     \var{{\devanagarifont \numemph\vb \raise.15em\hbox{॰}कोटी\raise.15em\hbox{॰}\lem \msCa; ˚कोटि\raise.15em\hbox{॰} \msCb\Ed\oo 
\raise.15em\hbox{॰}कुलः\lem \msCa\Ed; \raise.15em\hbox{॰}कुलम् \msCb}}% 

%Verse 24:67

{\devanagarifont अनेकरुद्रकन्याभी रूपिणीभिरलङ्कृतः {॥२४:६७॥} \veg\dontdisplaylinenum }%
     \var{{\devanagarifont \numnoemph\vc अनेकरुद्रकन्याभी\raise.15em\hbox{॰}\lem \msCb; {\lost}{\lost}{\lost}{\lost}{\lost}{\lost}न्याभी \msCa, अनेकरुद्रकन्यभि\raise.15em\hbox{॰} \Ed}}% 

{\devanagarifont तत्र पुण्यनदी सप्त सर्वपापापनोदनी \thinspace{\dandab} \dontdisplaylinenum }%
 
%Verse 24:68

{\devanagarifont सुवर्णवालुका दिव्या रत्नपाषाणशोभिता {॥२४:६८॥} \veg\dontdisplaylinenum }%
     \var{{\devanagarifont \numemph\vcd दिव्या रत्नपाषाणशोभिता\lem \msCb\Ed; दि{\lost}{\lost}{\lost}{\lost}{\lost}णशोभिता \msCa}}% 

{\devanagarifont पावनी च वरेण्या च वरार्हा वरदा वरा \thinspace{\dandab} \dontdisplaylinenum }%
     \var{{\devanagarifont \numemph\va पावनी\lem \msCa\Ed; पावणी \msCb}}% 
    \var{{\devanagarifont \numnoemph\vb वरार्हा\lem \msCa\Ed; वराहा \msCb}}% 

%Verse 24:69

{\devanagarifont वरेशा वरभद्रा च सुप्रसन्नजला शिवा {॥२४:६९॥} \veg\dontdisplaylinenum }%
     \var{{\devanagarifont \numnoemph\vd \raise.15em\hbox{॰}प्रसन्न\raise.15em\hbox{॰}\lem \msCa\msCb; \raise.15em\hbox{॰}प्रसन्ना\raise.15em\hbox{॰} \Ed}}% 

{\devanagarifont अनेककुसुमारामा रत्नपुष्पफलद्रुमाः \thinspace{\dandab} \dontdisplaylinenum }%
     \var{{\devanagarifont \numemph\vb \raise.15em\hbox{॰}द्रुमाः\lem \msCb\Ed; \raise.15em\hbox{॰}\uncl{द्रुमाः}}}% 

%Verse 24:70

{\devanagarifont अनेकरत्नप्राकारा योजनायुतमुच्छ्रिताः {॥२४:७०॥} \veg\dontdisplaylinenum }%
     \var{{\devanagarifont \numnoemph\vcd अनेकरत्नप्राकारा योजनायुतमुच्छ्रिताः\lem \Ed; 
\om\ \mssCaCbCc\msNa\msNb\msNc ({\englishfont \msNa\ marks the omission})}}% 

{\devanagarifont अहिंसासत्यनिरताः कामक्रोधविवर्जिताः \thinspace{\dandab} \dontdisplaylinenum }%
     \var{{\devanagarifont \numemph\va अहिंसासत्य\raise.15em\hbox{॰}\lem \msCb\Ed; {\lost}{\lost}{\lost}{\lost}त्य\raise.15em\hbox{॰} \msCa\oo 
\raise.15em\hbox{॰}रताः\lem \msCa\Ed; \raise.15em\hbox{॰}रता \msCb}}% 

%Verse 24:71

{\devanagarifont ध्यानयोगरता नित्यं तत्र मोदन्ति ते नराः {॥२४:७१॥} \veg\dontdisplaylinenum  }%
 
{\devanagarifont तत्र गोमातरस् सर्वा निवसन्ति यतव्रताः \thinspace{\dandab} \dontdisplaylinenum }%
     \var{{\devanagarifont \numemph\vab गोमातरः सर्वा निवसन्ति\lem \msCb\Ed; {\lost}{\lost}{\lost}{\lost}{\lost}{\lost}{\lost}{\lost}सन्ति \msCa}}% 

%Verse 24:72

{\devanagarifont गोलोकः शिवलोकश्च एक एव विधीयते {॥२४:७२॥} \veg\dontdisplaylinenum }%
     \paral{{\devanagarifont \vcd {\englishfont \similar\ \SDhU\ 12.88ab:}
                     गोलोकः शिवलोकश्च एक एव ततः स्मृतः }}
    \lacuna{\devanagarifont {\englishfont \Ed\ and at least two paper NGMCP MSS (A1341-6 and C107-7) 
                 add two anuṣṭubh verses here (minor variations ignored;
                 in \msNa, there is an omission mark here):}
                 तस्मादूर्ध्वं परं ज्ञेयं स्थानत्रयमनुत्तमम् {\devanagarifont ।}
                 स्कन्दगौरीमहेशानां नित्यशुद्धं परं शिवम् {\devanagarifont ॥}
                 दिनकृत्कोटिसङ्कासमनोपम्यं सनातनम् {\devanagarifont ।}
                 आदित्यादिशिवान्तश्च द्विस्थानोर्ध्वक्रमः स्मृतः {\devanagarifont ॥} }%
  
\vers

