% about 70% of these commands are not used now

\parindent2em
\usepackage{pdfpages}
\usepackage{dev}
% for ring below r (\textsubring) and l (Dharma input/IAST):
% \usepackage{tipa}
\cfoot{{\devanagarifont\thepage}}
\renewcommand{\headrulewidth}{0pt}
\renewcommand{\baselinestretch}{1.2}
%\headheight14.5pt
\usepackage[noeledmac]{ledmac}
%\usepackage{index}
%\usepackage{amsmath,amssymb,amsfonts,textcomp}
%\usepackage{ledpar}
%\usepackage{boxedminipage}
%%%%%%%%%%%%%%%%%%%%%%%

\newcommand{\blankpage}{%
\pagebreak
\thispagestyle{empty}
\ \vskip3cm
{\huge {This page is  itnentionally left blank.}}
\vfill\pagebreak}

\newcommand{\twopageedbreak}{%
\bekveg\szam
\newpage\addtocounter{page}{1}
\szam\bek}

% to emphasise number of first entry of a verse in apparatus
% this puts a box around the number (change position of ~)
%\newcommand{\numemph}[1]{{\color{red}\fbox{#1}}~}
\newcommand{\numemph}[1]{{\englishfont\color{red}{#1}}}
\newcommand{\numnoemph}[1]{{#1}}

% pādas in apparatus: verse no. + pāda sign
\newcommand{\vo}{\textbf{\englishfont \vs~}}
\newcommand{\va}{{\textbf{\englishfont \vs a~}}}
\newcommand{\vb}{{\textbf{\englishfont \vs b~}}}
\newcommand{\vc}{{\textbf{\englishfont \vs c~}}}
\newcommand{\vd}{{\textbf{\englishfont \vs d~}}}
\newcommand{\ve}{{\textbf{\englishfont \vs e~}}}
\newcommand{\vf}{{\textbf{\englishfont \vs f~}}}
\newcommand{\vab}{{\textbf{\englishfont \vs ab~}}}
\newcommand{\vcd}{{\textbf{\englishfont \vs cd~}}}
\newcommand{\vef}{{\textbf{\englishfont \vs ef~}}}
\newcommand{\vabce}{{\textbf{\englishfont \vs a--d~}}}
\newcommand{\vcdef}{{\textbf{\englishfont \vs c--f~}}}

%%%%%%%%%%verse no. counting and printing
\newcount\fejno
\newcount\versno
\newcount\ajsa


\fejno=1
\versno=1
\newcommand{\ujvers}{\global\advance\versno by1}
%\renewcommand{\va}{{\textbf{\englishfont \vs a~}}}%
%\renewcommand{\vb}{{\textbf{\englishfont \vs a~}}}%
%\renewcommand{\vc}{{\textbf{\englishfont \vs c~}}}%
%\renewcommand{\vd}{{\textbf{\englishfont \vs a~}}}%
%}
\newcommand{\ujfej}{\versno=0\global\advance\fejno by1}
\newcommand{\newchapter}{\ujfej}
%\newcommand{\fejversnoird}{\oldstylenums{\the\fejno} .\oldstylenums{\the\versno}}
%\newcommand{\versnolabj}{\textbf{\oldstylenums{\the\fejno}}{\textbf{.}}
%\the\versno\rm\ }
%\newcommand{\ljvs}{\versnolabj}
%\newcommand{\trvs}{\norm \the\fejno .\the\versno\ }
%\newcommand{\ujtr}{\ujvers\trvs\hskip 1cm}
\newcommand{\vsveg}{{\the\fejno.\the\versno}}
\newcommand{\vs}{{\englishfont\the\versno}}

\newcommand{\phpspagebreak}{\pagebreak}

%%%%% Sections: from--to
%\newcount\innen%Pl. 14.24-46
%\newcommand{\tolig}{{\bf\the\fejno .\the\innen --\the\versno\ }}
%\newcommand{\innentol}{\global\innen=\versno}
%\newcount\innenn%Pl. 14.24-46
%\newcommand{\toligg}{{\bf\the\fejno .\the\innenn --\the\versno\ }}
%\newcommand{\innentoll}{\global\innenn=\versno}

%%%%  Types of paragraph
%\newcommand{\proza}{\leftskip 3em\noindent}
%\newcommand{\ujproza}{\global\advance\versno by1\leftskip 3em\noindent}
\newcommand{\vers}{\leftskip4em}
% alternative name:
\newcommand{\sloka}{\vers}

\newcommand{\dnvers}{\leftskip4em}

% indentation for non-anustubhs
\newcommand{\nemsloka}{\leftskip4em} % for "Saarduulavikrii.dita etc. stanzas: pada a
\newcommand{\nemslokab}{\leftskip6em} % pada b
\newcommand{\nemslokac}{\leftskip4em} % pada c
\newcommand{\nemslokad}{\leftskip6em} % pada d

\newcommand{\dnnemsloka}{\leftskip 3em} % for "Saarduulavikrii.dita etc. stanzas: pada a
\newcommand{\dnnemslokab}{\leftskip 4em} % pada b
\newcommand{\dnnemslokac}{\leftskip 3em} % pada c
\newcommand{\dnnemslokad}{\leftskip 4em} % pada d

\newcommand{\nemslokalong}{
\renewcommand{\nemsloka}{\leftskip 2em} % for "Saarduulavikrii.dita etc. stanzas: pada a
\renewcommand{\nemslokab}{\leftskip 3em} % pada b
\renewcommand{\nemslokac}{\leftskip 2em} % pada c
\renewcommand{\nemslokad}{\leftskip 3em} % pada d
\renewcommand{\dnnemsloka}{\leftskip 1em} % for "Saarduulavikrii.dita etc. stanzas: pada a
\renewcommand{\dnnemslokab}{\leftskip 2em} % pada b
\renewcommand{\dnnemslokac}{\leftskip 1em} % pada c
\renewcommand{\dnnemslokad}{\leftskip 2em} % pada d
}

\newcommand{\nemslokanormal}{
\renewcommand{\nemsloka}{\leftskip 3em} % for "Saarduulavikrii.dita etc. stanzas: pada a
\renewcommand{\nemslokab}{\leftskip 4em} % pada b
\renewcommand{\nemslokac}{\leftskip 3em} % pada c
\renewcommand{\nemslokad}{\leftskip 4em} % pada d
\renewcommand{\dnnemsloka}{\leftskip 2em} % for "Saarduulavikrii.dita etc. stanzas: pada a
\renewcommand{\dnnemslokab}{\leftskip 3em} % pada b
\renewcommand{\dnnemslokac}{\leftskip 2em} % pada c
\renewcommand{\dnnemslokad}{\leftskip 3em} % pada d
}
%%%%%%%%%%%%%%%%%%%%%%%%%%%%%%%%%%%%%%%%%%%%%%%%%%%%%%

%\newcommand{\prose}{\noindent} % pada d
\newcommand{\prose}{\leftskip 0em\noindent\advanceline{0}}
%\newcommand{\dontdisplaylinenum}{\advanceline{1}\skipnumbering}
\newcommand{\dontdisplaylinenum}{\skipnumbering}
%%%% Apparatus

%\newcommand{\var}[1]{\edtext{}{\Dfootnote{#1}}}


% when 2-col app requires grouping of variants of one single verse:
%\newcommand{\var}[1]{\ \ #1}
\newcommand{\var}[1]{\edtext{}{\Cfootnote{#1}}}
\newcommand{\paral}[1]{\edtext{}{\Bfootnote{#1}}}
\newcommand{\marginalia}[1]{\edtext{}{\Bfootnote{#1}}}
\newcommand{\prosevar}[1]{\edtext{}{\Afootnote{#1}}}
\newcommand{\lacuna}[1]{\edtext{}{\Afootnote{#1}}}

%%%%%%%  Dandas
\newcommand{\dandab}{\ujvers\danda}%\renewcommand{\padaA}{a}\renewcommand{\padaB}{b}\renewcommand{\padaAB}{ab}}
\newcommand{\dandad}{\danda}%\renewcommand{\padaA}{c}\renewcommand{\padaB}{d}\renewcommand{\padaAB}{cd}}
\newcommand{\danda}{।}
\newcommand{\ketdanda}{॥}
\newcommand{\ketdandadn}{{\dn ..}}
\newcommand{\dnd}{\danda}
\newcommand{\dnddnd}{\ketdanda}
%\newcommand{\veg}{}%{{\rm\ketdanda\vsveg\ketdanda}\renewcommand{\padaA}{c}\renewcommand{\padaB}{d}\renewcommand{\padaAB}{cd}}
\newcommand{\veg}{\vrule depth7pt width0pt}% this now only controls the vertical skip at the end of verse
\newcommand{\vsvegdn}{\the\fejno\kern-0.3em{\rm\ :}\kern0.1em\the\versno\kern-0.15em}
\newcommand{\vegdn}{॥$~${\vsvegdn}॥ \vrule depth10pt width0pt}
\newcommand{\sixlsloka}{\renewcommand{\padaA}{e}\renewcommand{\padaB}{f}\renewcommand{\padaAB}{ef}}
\newcommand{\vegg}[1]{{\rm \ketdanda#1\ketdanda}}
\newcommand{\dandadn}{।}
\newcommand{\dandabdn}{\ujvers{।}}


%%%%%%%%%%%
\newcommand{\tart}[1]{\ugras
\leftskip5em

\noindent
[= {\footnotesize #1}]
\ugras
\vers}
\newcommand{\jump}{\ugras}
\newcommand{\thdanda}{{\thinspace$\cal j$}}
\newcommand{\thketdanda}{\thinspace$\cal k$}
\newcommand{\subb}[1]{\bigskip\textbf{#1}\medskip}
%\newcommand{\jati}[1]{{\fbox{\parbox{\linewidth}{#1}}}}
%\newcommand{\jati}[2][]{\tikz[overlay]\node[fill=blue!20,inner sep=2pt, anchor=text, rectangle, rounded corners=1mm,#1] {#2};\phantom{#2}}
\newcommand{\jati}[1]{\textbf{#1}}
\newcommand{\Notes}{\textit{Textual Notes}}
\newcommand{\Colo}{{\englishfont\textbf{Colophon}}}
\newcommand{\notelemm}[2]{#1: \textbf{#2}~]}
%\newcommand{\lem}{{\devanagarifont\thinspace\textbf{]} }}
\newcommand{\lem}{ }

\newcommand{\note}[1]{#1:}
%\newcommand{\AD}{\textsc{ad}}
\newcommand{\CE}{\textsc{ce}}
\newcommand{\dik}{\textsuperscript{th}}
%\newcommand{\rovidb}{\char23\kern-.25em}%abbrev in Skt words left
%\newcommand{\rovidj}{\kern-.25em\char23\kern-.45em}%abbrev in Skt words right
\newcommand{\rovidb}{{\textdegree}}%abbrev in Skt words left % I dont use these any more
\newcommand{\rovidj}{{\textdegree}}%abbrev in Skt words right % I dont use these any more
\newcommand{\rovid}{{\textdegree}}%abbrev in Skt words right
\newcommand{\pusp}{${\oplus}$}
\newcommand{\puspika}{${\otimes}$}
\newcommand{\abbr}{\char23}
\newcommand{\Patala}{\skt{Pa{\d t}ala}}
\newcommand{\patala}{\skt{pa{\d t}ala}}
\newcommand{\sasm}[1]{!SĀŚM: #1!}


%\newcommand{\New}[1]{\begin{center}\textbf{+++++\hbox{N\lower0.2em\hbox{E}\hbox{W}}!!!+++++}#1
%\textbf{+++++ END of \hbox{N\lower0.2em\hbox{E}\hbox{W}} +++++}\end{center}}

\newcommand{\New}{\textbf{\hbox{N\lower0.2em\hbox{E}\hbox{W}}!!!}}
\newcommand{\oo}{{\englishfont\ \ ${\bullet}$\ \ }}
\newcommand{\unmetr}{{\englishfont (unmetr.)}}
\newcommand{\hypermetr}{{\englishfont (hy\-per\-metr.)}}
\newcommand{\hypometr}{{\englishfont (hy\-po\-metr.)}}
\newcommand{\conj}{{\englishfont conj.}}
\newcommand{\corr}{{\englishfont corr.}}
\newcommand{\eme}{{\englishfont em.}}
\newcommand{\CHECK}{{\englishfont\color{red}{CHECK}}}
\newcommand{\kozep}[1]{\begin{center}{#1}\end{center}}
\newcommand{\paralid}{}
\newcommand{\crux}[1]{{\englishfont$\dag$}{\thinspace #1}{\englishfont$\dag$}}
\newcommand{\cruxdn}[1]{{$\dag\thinspace$#1$\dag$}}
\newcommand{\fejezet}[1]{\subsection{#1}
				
}


% Chapter, subchapter etc. in the edition
\newcommand{\alfejezet}[1]{\begin{center}
        {{{{\textbf{\large[ }}{\devanagarifont\large#1}{{\englishfont\large]}}}}}\dontdisplaylinenum
\addcontentsline{toc}{subsection}{#1}
\end{center}

}

\newcommand{\alalfejezet}[1]{\begin{center}
{{{\englishfont{\small [ }}{\devanagarifont\small#1}{\englishfont{\small ]}}}}\dontdisplaylinenum
\addcontentsline{toc}{subsubsection}{#1}
\end{center}

}

\newcommand{\alalalfejezet}[1]{\begin{center}
{{\englishfont\footnotesize [ }{\devanagarifont\footnotesize#1}{\englishfont\footnotesize ]}}\dontdisplaylinenum
\addcontentsline{toc}{subsubsection}{#1}
\end{center}

}


\newcommand{\alfejezetdn}[1]{\begin{center}
        {{\englishfont\Large [}{\dn#1}{\englishfont\Large  ]}}\dontdisplaylinenum 
\addcontentsline{toc}{subsection}{#1}
\end{center}

}

\newcommand{\alalfejezetdn}[1]{\begin{center}
\textbf{{{\englishfont\large [}{\dn#1}{\englishfont\large  ]}}}\dontdisplaylinenum 
\addcontentsline{toc}{subsubsection}{#1}
\end{center} 

}


%\newcommand{\alfejezet}[1]{\subsubsection[#1]{{\textbf{{#1}}}}\vers}
%\newcommand{\fejezet}[1]{\begin{center}{\textbf{\suppl{#1}}}\end{center}}

\newcommand{\atugras}[2]{\begin{center}{\dots}\end{center}\ugras\ugras\begin{center}
(\textit{#1})
\end{center}\begin{center}{\dots}\end{center}}
\newcommand{\atugr}[2]{\begin{center}(\textit{#1})\end{center}}
\newcommand{\urescim}[1]{\thispagestyle{empty}\ \vskip8cm{\begin{center}\huge\textsc{#1}\end{center}}}
\newcommand{\gap}{${\sqcup}$}
\newcommand{\Exeme}{\textit{Ex em.}}
\newcommand{\Exconj}{\textit{Ex conj.}}
\newcommand{\exeme}{\textit{ex em.}}
\newcommand{\exconj}{\textit{ex conj.}}
\newcommand{\nagy}{}
 \sidenotemargin{outer}
%\newcommand{\folio}[1]{\ledsidenote{\tiny #1}\vers}

%\newcommand{\folio}[1]{{\tiny {\textsc{$\rightarrow$}} #1}

%\vers}

%\newcommand{\folio}[1]{}
%\newcommand{\folioo}[1]{\textbf{[f.~#1]}}
%\newcommand{\folio}[1]{\textbf{[#1]}}
\newcommand{\folioo}[1]{}
%\newcommand{\folio}[1]{}

\newcommand{\folio}[1]{\begin{center}{\tiny {\textsc{$\rightarrow$}} #1}
\end{center}

\vers}

% old Velthuis commands
\newcommand{\csbreve}{} %CHECK for Roman version
\newcommand{\csa}{{A}} %only mātrā
\newcommand{\csr}{{\dn \0}} %only superscript r
\newcommand{\csi}{{\dn E}} %only i
\newcommand{\csI}{{\dn F}} %only ī
\newcommand{\cse}{{\dn \?}} %only e
\newcommand{\csai}{{\dn \4}} %only ai
\newcommand{\cso}{{\dn o}} %only o
\newcommand{\csuil}{{\dn \7{\il}}} % u plus \il
\newcommand{\csdh}{{\dn @}} % dh 
\newcommand{\cst}{{\dn (}} % t 


\newcommand{\rem}[1]{\ledrightnote{\tiny [#1]}\vers}
\newcommand{\frecto}[1]{\ledleftnote{\tiny{#1\recto}}\vers}
\newcommand{\fverso}[1]{\ledleftnote{\tiny{#1\verso}}\vers}
\newcommand{\kkicsi}{}
%\newcommand{\toplost}{{\englishfont(\textit{top of akṣaras lost})}}
\newcommand{\toplost}{{\englishfont(top of akṣaras lost)}}
\newcommand{\vverso}{\verso}
\newcommand{\rrecto}{\recto}
\newcommand{\trvers}[1]{$^{{\textnormal{\scriptsize [#1]}}}$}
\newcommand{\prefn}[3]{$^{\scriptsize\textnormal{[}}$\footnote[#1]{$^{\scriptsize 
		\textnormal{#2}}$#3}{$^{\textnormal{\scriptsize #2]}}$}}
%\newcommand{\slokawithfn}[4]{
%
%$^{\scriptsize\textnormal{[}}$\footnote[#1]{$^{\scriptsize 
%		\textnormal{#2}}$#4}{$^{\textnormal{\scriptsize#2]}}$}#3}

%\newcommand{\slokawithfn}[4]{
%
%$^{\scriptsize\textnormal{}}$\footnote[#1]{$^{\scriptsize 
%		\textnormal{#2}}$#4}{$^{\textnormal{\scriptsize#2}}$}#3}


%\newcommand{\slokawithfn}[4]{$^{\the\fejno.}$\footnote[#1]{$^{\scriptsize\textnormal{#2}}$#4}$^{\textnormal{#2}}$#3}
%\newcommand{\slokawithoutfn}[2]{$^{\textnormal{\the\fejno.#1}}$#2}

\newcommand{\slokawithfn}[4]{\footnote[#1]{$^{\scriptsize\textnormal{#2}}$#4}$^{\textnormal{#2}}$#3}
\newcommand{\slokawithoutfn}[2]{$^{\textnormal{#1}}$#2}


%\newcommand{\slokawithoutfn}[3]{
%
%{$^{\scriptsize\textnormal{[}}${$^{\textnormal{\scriptsize #1]}}$}}#2}

%\newcommand{\slokawithoutfn}[3]{
%
%{$^{\scriptsize\textnormal{}}${$^{\textnormal{\scriptsize #1}}$}}#2}


\newcommand{\trlemm}[1]{\textbf{\textit{#1}}}
\newcommand{\trlemmindex}[1]{\textbf{\textit{#1}}\index{#1}}

%\newcommand{\Grammar}{${\rhd}$ }
%\newcommand{\Meaning}{${\bigtriangleup}$ }
%\newcommand{\Grammar}{\Moon\ }
%\newcommand{\Meaning}{\Sun\ }
\newcommand{\Grammar}{}
\newcommand{\Meaning}{}

\newcommand{\upadhm}{f}


\newcommand{\dnapp}[1]{}
\newcommand{\rmapp}[1]{#1}

%\newcommand{\trvers}[1]{$^{\textcircled{#1}}$}
\newcommand{\trfejezet}[1]{\begin{center}{\large \textbf{[{Chapter #1}]}}\end{center}\bigskip}
%\newcommand{\tralfejezet}[1]{\begin{center}\textsc{[#1]}\end{center}}
\newenvironment{trnemsloka}{\begin{quotation}\noindent}{\end{quotation}}
\newcommand{\ie}[1]{(\skt{#1})\index{#1}}
\newcommand{\ienoindex}[1]{(\skt{#1})}%no indexing
\newcommand{\suppl}[1]{$<${#1}$>$}
\newcommand{\error}[1]{{\sout{#1}}}
%\newcommand{\om}{{\englishfont\textit{omitted in}}}
\newcommand{\om}{{\englishfont om.}}
\newcommand{\uncl}[1]{${\wr}$#1${\wr}$}
\newcommand{\cancelled}{{\englishfont(\textit{cancelled})}}
\newcommand{\isitcancelled}{{\englishfont(\textit{cancelled?})}}
%REVISE
\newcommand{\eyeskip}[1]{{\englishfont(eyeskip to #1)}}
\newcommand{\eyeskipto}[1]{{\englishfont (eyeskip to #1)}}
\newcommand{\eyeskipfrom}[2]{{\englishfont (eyeskip from #1)}}

\newcommand{\recto}{$^r$\/}
\newcommand{\verso}{$^v$\/}
\newcommand{\pcorr}{{\englishfont $^{pc}$\/}}
\newcommand{\acorr}{{\englishfont $^{ac}$\/}}
\newcommand{\sadhaka}{S{\=a}\-dha\-ka\index{sādhaka}}
\newcommand{\mantrin}{Mantrin\index{mantrin}}
%\newcommand{\kalpa}{Kalpa\index{kalpa}}
%\newcommand{\Skt}[1]{#1}\index{#1}}%\index{#1}}%Sanskrit in italics
\newcommand{\skt}[1]{\textit{#1}}%}%\index{#1}}%Sanskrit in italics
\newcommand{\sktindex}[1]{\textit{#1}\index{#1}}%}%\index{#1}}%Sanskrit in italics
\newcommand{\pskt}[1]{\textit{#1}}%parallels
\newcommand{\cim}[1]{\textsl{#1}\index{#1}}%Titles in slanted
\newcommand{\titl}[1]{\textsl{#1}}%Titles in slanted
\newcommand{\cimnoindex}[1]{\textsl{#1}}%Titles in slanted
%\newcommand{\il}{${\star}$}
% illegible syllable:
%\def\langkort{\setbox0=\hbox{{\bf --}}\dimen0=\ht0
     %  \advance\dimen0 by-2.5ex
      %  \copy0\kern -1\wd0
       % \raise\dimen0\hbox to\wd0{\hss{\bf \char'025}\hss}}
%\newcommand{\il}{\raise.2em\hbox{\englishfont{ -\kern-.27em\raise-.5em\hbox{˘}$\,$}}}
%\newcommand{\il}{\hspace{.08em}\raise.2em\hbox{-}\kern-.33em\raise-.0em\hbox{\englishfont{˘}$\,$}}
\newcommand{\il}{\englishfont{×}} % note that I use × for lost in the base file!
                                  % this here should be an anceps

\newcommand{\lost}{\englishfont{$\_$\thinspace }}
\newcommand{\chapp}[1]{\textit{#1}}
\newcommand{\att}[1]{\underline{#1}}
\newcommand{\allpatalas}[1]{\textit{Pa\d{t}ala} #1}
\newcommand{\mntr}[1]{{\textsc{#1}}}%}%\index{#1}}%Mantras in small capitals
\newcommand{\mntrdn}[1]{#1}%Mantras in devnag
\newcommand{\mntrindex}[1]{\textsc{#1}\index{#1}}%Mantras in small capitals
\newcommand{\dialog}[1]{{\textsc{#1}}}%Dialogue in small capitals
\newcommand{\edmntrsmall}[1]{{\textsc{#1}}}%Mantras in small capitals
\newcommand{\vegeadalnak}{\bekveg\szamveg\vfill\pagebreak}
\newcommand{\kb}{${\approx}$}
\newcommand\compare{~{\englishfont cf.}}
\newcommand{\similar}{~${\approx}$}
\newcommand{\scsmall}{\mntr} % outdated
\newcommand{\verbalroot}[1]{$\sqrt{#1}$}
\newcommand{\fb}{${\Rsh}$} %folio break
\newcommand{\upperfol}{${\uparrow}$} %folio break
\newcommand{\lowerfol}{${\downarrow}$} %folio break
\newcommand{\aisa}[1]{$^\textnormal{\tiny #1}$}
\newcommand{\Aisa}[1]{~${\rightarrow}~\S$#1\index{Ai{\'sa} #1}}
\newcommand{\kep}[1]{\includegraphics[scale=2.5]{$HOME/indology/early_tantra_project/edition/byeditionmainfiles/images/#1}}
\newcommand{\kepp}[1]{\includegraphics[scale=2]{$HOME/indology/early_tantra_project/edition/byeditionmainfiles/images/#1}}
\newcommand{\keppp}[1]{\includegraphics[scale=1]{$HOME/indology/early_tantra_project/edition/byeditionmainfiles/images/#1}}
%%%AUTHORS%%%%%%%%%%%%%%%%%%%%%%%%%%%%%
\newcommand{\csindex}[1]{#1\index{#1}}



%%%%%%%%%%%%%%%%%%%%%%%%%%%%%%%%%%%%%


\newcommand{\szam}{\beginnumbering}
\newcommand{\bek}{\pstart}
\newcommand{\szamveg}{\endnumbering}
\newcommand{\bekveg}{\pend}
\newcommand{\ugras}{%

\  
\dontdisplaylinenum

}

\newcommand{\edmntr}[1]{\textsc{#1}}

% EXTERNAL files
\newcommand{\mssALL}{{\allowbreak\englishfont$\Sigma$}}

% original:
%\newcommand{\msCa}{{\englishfont C$_\textrm{a}$}}
%\newcommand{\msCaacorr}{{\englishfont C$_\textrm{a}$\kern-.4em$^{ac}$\/}}
%\newcommand{\msCapcorr}{{\englishfont C$_\textrm{a}$\kern-.5em$^{pc}$\/}}
% Peter Bisschop etc.:
%\newcommand{\msCa}{{\rm N$^{\scriptscriptstyle C}_{\scriptscriptstyle 94}$}}
%\newcommand{\msCaacorr}{{\rm N$^{\scriptscriptstyle Cac}_{\scriptscriptstyle 94}$}}
%\newcommand{\msCapcorr}{{\rm N$^{\scriptscriptstyle Cpc}_{\scriptscriptstyle 94}$}}
% my try:
\newcommand{\msCa}{{\allowbreak\englishfont C$_{\scriptscriptstyle 94}$}}
\newcommand{\msCaacorr}{{\allowbreak\englishfont C$^{\scriptscriptstyle ac}_{\scriptscriptstyle 94}$}}
\newcommand{\msCapcorr}{{\allowbreak\englishfont C$^{\scriptscriptstyle pc}_{\scriptscriptstyle 94}$}}

% original
%\newcommand{\msCb}{{\englishfont C$_\textrm{b}$}}
%\newcommand{\msCbacorr}{{\englishfont C$_\textrm{b}$\kern-.4em$^{ac}$\/}}
%\newcommand{\msCbpcorr}{{\englishfont C\raisebox{-.1em}{$_\textrm{b}$}\kern-.57em$^{pc}$\/}}
%\newcommand{\msCb}{{\rm N$^{\scriptscriptstyle C}_{\scriptscriptstyle 45}$}}
%\newcommand{\msCbacorr}{{\rm N$^{\scriptscriptstyle Cac}_{\scriptscriptstyle 45}$}}
%\newcommand{\msCbpcorr}{{\rm N$^{\scriptscriptstyle Cpc}_{\scriptscriptstyle 45}$}}
\newcommand{\msCb}{{\allowbreak\englishfont C$_{\scriptscriptstyle 45}$}}
\newcommand{\msCbacorr}{{\allowbreak\englishfont C$^{\scriptscriptstyle ac}_{\scriptscriptstyle 45}$}}
\newcommand{\msCbpcorr}{{\allowbreak\englishfont C$^{\scriptscriptstyle pc}_{\scriptscriptstyle 45}$}}

%\newcommand{\msCc}{{\englishfont C$_\textrm{c}$}}
%\newcommand{\msCcacorr}{{\englishfont C$_\textrm{c}$\kern-.16cm$^{ac}$\/}}
%\newcommand{\msCcpcorr}{{\englishfont C$_\textrm{c}$\kern-.16cm$^{pc}$\/}}
%\newcommand{\msCc}{{\rm N$^{\scriptscriptstyle C}_{\scriptscriptstyle 02}$}}
%\newcommand{\msCcacorr}{{\rm N$^{\scriptscriptstyle Cac}_{\scriptscriptstyle 02}$}}
%\newcommand{\msCcpcorr}{{\rm N$^{\scriptscriptstyle Cpc}_{\scriptscriptstyle 02}$}}
\newcommand{\msCc}{{\allowbreak\englishfont C$_{\scriptscriptstyle 02}$}}
\newcommand{\msCcacorr}{{\allowbreak\englishfont C$^{\scriptscriptstyle ac}_{\scriptscriptstyle 02}$}}
\newcommand{\msCcpcorr}{{\allowbreak\englishfont C$^{\scriptscriptstyle pc}_{\scriptscriptstyle 02}$}}

%\newcommand{\msNa}{{\englishfont N$_\textrm{a}$}}
%\newcommand{\msNaacorr}{{\englishfont N$_\textrm{a}$\kern-.10cm$^{ac}$\/}}
%\newcommand{\msNapcorr}{{\englishfont N$_\textrm{a}$\kern-.13cm$^{pc}$\/}}
%\newcommand{\msNa}{{\rm N$^{\scriptscriptstyle K}_{\scriptscriptstyle 82}$}}
%\newcommand{\msNaacorr}{{\rm N$^{\scriptscriptstyle Kac}_{\scriptscriptstyle 82}$}}
%\newcommand{\msNapcorr}{{\rm N$^{\scriptscriptstyle Kpc}_{\scriptscriptstyle 82}$}}
\newcommand{\msNa}{{\allowbreak\englishfont K$_{\scriptscriptstyle 82}$}}
\newcommand{\msNaacorr}{{\allowbreak\englishfont K$^{\scriptscriptstyle ac}_{\scriptscriptstyle 82}$}}
\newcommand{\msNapcorr}{{\allowbreak\englishfont K$^{\scriptscriptstyle pc}_{\scriptscriptstyle 82}$}}

\newcommand{\msNb}{{\allowbreak\englishfont K$_{\scriptscriptstyle 10}$}}
\newcommand{\msNbacorr}{{\allowbreak\englishfont K$^{\scriptscriptstyle ac}_{\scriptscriptstyle 10}$}}
\newcommand{\msNbpcorr}{{\allowbreak\englishfont K$^{\scriptscriptstyle pc}_{\scriptscriptstyle 10}$}}

\newcommand{\msNc}{{\allowbreak\englishfont K$_{\scriptscriptstyle 7}$}}
\newcommand{\msNcacorr}{{\allowbreak\englishfont K$^{\scriptscriptstyle ac}_{\scriptscriptstyle 7}$}}
\newcommand{\msNcpcorr}{{\allowbreak\englishfont K$^{\scriptscriptstyle pc}_{\scriptscriptstyle 7}$}}

\newcommand{\msNd}{{\allowbreak\englishfont K$_{\scriptscriptstyle 3}$}}
\newcommand{\msNdacorr}{{\allowbreak\englishfont K$^{\scriptscriptstyle ac}_{\scriptscriptstyle 03}$}}
\newcommand{\msNdpcorr}{{\allowbreak\englishfont K$^{\scriptscriptstyle pc}_{\scriptscriptstyle 03}$}}

\newcommand\msBod{{\allowbreak\englishfont B}}
\newcommand\msBodacorr{{\allowbreak\englishfont B}$^{\scriptscriptstyle ac}$}
\newcommand\msBodpcorr{{\allowbreak\englishfont B}$^{\scriptscriptstyle pc}$}

\newcommand\msL{{\rm L}}
\newcommand\msLacorr{{\englishfont L$^{\scriptscriptstyle ac}$}}
\newcommand\msLpcorr{{\englishfont L$^{\scriptscriptstyle pc}$}}

\newcommand\msPaperA{\allowbreak{\englishfont K$_{\scriptscriptstyle 41}$}}
\newcommand\msPaperAacorr{\allowbreak{\englishfont K$^{\scriptscriptstyle ac}_{\scriptscriptstyle 41}$}}
\newcommand\msPaperApcorr{\allowbreak{\englishfont K$^{\scriptscriptstyle pc}_{\scriptscriptstyle 41}$}}

\newcommand\msPaperC{\allowbreak{\englishfont K$_{\scriptscriptstyle 107}$}}
\newcommand\msPaperCacorr{\allowbreak{\englishfont K$^{\scriptscriptstyle ac}_{\scriptscriptstyle 107}$}}
\newcommand\msPaperCpcorr{\allowbreak{\englishfont K$^{\scriptscriptstyle pc}_{\scriptscriptstyle 107}$}}

\newcommand\msParis{{\allowbreak\englishfont P$_{\scriptscriptstyle 57}$}}
\newcommand\msParisacorr{{\allowbreak\englishfont P$_{\scriptscriptstyle 57}^{\scriptscriptstyle ac}$}}
\newcommand\msParispcorr{{\allowbreak\englishfont P$_{\scriptscriptstyle 57}^{\scriptscriptstyle pc}$}}

\newcommand\msM{{\allowbreak\englishfont M}}
\newcommand\msMacorr{{\allowbreak\englishfont M$^{\scriptscriptstyle ac}$}}
\newcommand\msMpcorr{{\allowbreak\englishfont M$^{\scriptscriptstyle pc}$}}

\newcommand{\msKOa}{{\allowbreak\englishfont K$^{\scriptscriptstyle {\rm o}}_{\scriptscriptstyle 77}$}}
\newcommand{\msKOaacorr}{{\allowbreak\englishfont K$^{\scriptscriptstyle {\rm o}ac}_{\scriptscriptstyle 77}$}}
\newcommand{\msKOapcorr}{{\allowbreak\englishfont K$^{\scriptscriptstyle {\rm o}pc}_{\scriptscriptstyle 77}$}}

%\newcommand{\Ed}{{\englishfont Ed$_\textrm{N}$}}
%\newcommand{\Ed}{{\rm E$^{\scriptscriptstyle N}$}}
\newcommand{\Ed}{{\allowbreak\englishfont E}}




\newcommand{\msCaNa}{{\englishfont C$_\textrm{a}$N$_\textrm{a}$}}
\newcommand{\msCbNa}{{\englishfont C$_\textrm{b}$N$_\textrm{a}$}}
\newcommand{\msCcNa}{{\englishfont C$_\textrm{c}$N$_\textrm{a}$}}
\newcommand{\msCabNa}{{\englishfont C$_\textrm{a}$C$_\textrm{b}$N$_\textrm{a}$}}
\newcommand{\msCbcNa}{{\englishfont C$_\textrm{b}$C$_\textrm{c}$N$_\textrm{a}$}}
\newcommand{\msCabcNa}{{\englishfont C$_\textrm{a}$C$_\textrm{b}$C$_\textrm{c}$N$_\textrm{a}$}}
\newcommand{\msCab}{{\englishfont C$_\textrm{a}$C$_\textrm{b}$}}
\newcommand{\msCac}{{\englishfont C$_\textrm{a}$C$_\textrm{c}$}}
\newcommand{\msCbc}{{\englishfont C$_\textrm{b}$C$_\textrm{c}$}}
\newcommand{\msCabc}{{\englishfont C$_\textrm{a}$C$_\textrm{b}$$_\textrm{c}$}}




\newcommand{\mssCaCbCc}{{\englishfont C$_{\scriptscriptstyle\Sigma}$}}
%\newcommand{\mssCaCbCc}{{\englishfont C\kern-.4em\raise.1em\hbox{C}\kern-.4em\raise.2em\hbox{C}}}
%\newcommand{\mssNaNbNc}{{\englishfont N$^{\scriptscriptstyle\Sigma}$}}
\newcommand{\mssNaNbNc}{{\englishfont K\kern-.4em\raise.1em\hbox{K}\kern-.4em\raise.2em\hbox{K}}}

\newcommand{\Cod}{\textit{Cod.}}
\newcommand{\Codd}{$\Sigma$}

\newcommand{\AP}{AP\index{Agnipurana@\textsl{Agnipur\=a\d na}}}
\newcommand{\APP}{ĀPP\index{Atmarthapujapaddhati@\textsl{\=Atm\=arthap\=uj\=apaddhati}}}
\newcommand{\RV}{\d RV\index{Rigveda@\textsl{\d Rgveda}}}
\newcommand{\BhP}{BhP}%\index{Bhagavatapurana@\textsl{Bh\=agavatapur\=a\d na}}}
\newcommand{\GS}{GS\index{Gherandasamhita@\textsl{Ghera\d n\d dasa\d mhit\=a}}}
\newcommand{\GorakSatBriggs}{{G\'S$_\textrm{B}$}\index{Goraksasataka@\textsl{Gorak\d sa\'sataka}}}
\newcommand{\GorakSatLonavla}{{G\'{S}}$_\textrm{L}$\index{Goraksasataka@\textsl{Gorak\d sa\'sataka}}}
\newcommand{\GorakSatNowotny}{{G\'{S}}$_\textrm{N}$\index{Goraksasataka@\textsl{Gorak\d sa\'sataka}}}
%\newcommand{\HYP}{HYP\index{Ha\d thayogaprad\=\i pik\=a@\textsl{Ha\d thayogaprad\=\i pik\=a}}}
\newcommand{\JAT}{JAT\index{Jnanarnavatantra@\textsl{J\~n\=an\=ar\d navatantra}}}
\newcommand{\KAN}{K\=AN\index{Kaulavalinirnaya@\textsl{Kaul\=aval\=\i nir\d naya}}}
\newcommand{\KAT}{KAT\index{Kularnavatantra@\textsl{Kul\=ar\d navatantra}}}
\newcommand{\KMT}{KMT\index{Kubjikamatatantra@\textsl{Kubjik\=amatatantra}}}
\newcommand{\KJN}{KJN\index{Kaulajnananirnaya@\textsl{Kaulaj\~n\=ananir\d naya}}}
\newcommand{\KRU}{KRU\index{Kularatnodaya@\textsl{Kularatnodaya}}}
\newcommand{\KP}{KP\index{Kulapradipa@\textsl{Kulaprad\=\i pa}}}
\newcommand{\KUp}{KUp\index{Kubjikopanisad@\textsl{Kubjikopani\d sad}}}
\newcommand{\KhV}{KhV\index{Khecarividya@\textsl{Khecar\=\i vidy\=a}}}

\newcommand{\MKS}{MKS\index{Mahakalasamhita@\textsl{Mah\=ak\=alasa\d mhit\=a}}}
%\newcommand{\MP}{MatPu}%\index{Matsyapurana@\textsl{Matsyapur\=a\d na}}}
\newcommand{\MUT}{MUT\index{Matottaratantra@\textsl{Matottaratantra}}}
\newcommand{\MVUT}{MVUT\index{Malinivijayottaratantra@\textsl{M\=alin\=\i vijayottaratantra}}}
\newcommand{\MaSam}{MaSa\d m} %\index{Matsyendrasamhita@\textsl{Matsyendrasa\d mhit\=a}}}
%\newcommand{\MBh}{MBh}%\index{Mahabharata@\textsl{Mah\=abh\=arata}}}
\newcommand{\NAT}{N\=AT\index{Nityahnikatilaka@\textsl{Nity\=ahnikatilaka}}}
%\newcommand{\Nisv}{Ni\'sv\index{Nisvasatattvasamhita@\textsl{Ni\'sv\=asatattvasa\d mhit\=a}}}
\newcommand{\NSA}{N\d SA\index{Nityasodasikarnava@\textsl{Nity\=a\d so\d da\'sik\=ar\d nava} alias\\ \textsl{V\=amake\'svar\=\i mata}}}
\newcommand{\PrS}{PrS\index{Prapancasara@\textsl{Prapa\~ncas\=ara}}}
\newcommand{\PS}{PS}%\index{Pasupatasutra@\textsl{P\=a\'supatas\=utra}}}
\newcommand{\RY}{RY\index{Rudrayamala@\textsl{Rudray\=amala}}}
\newcommand{\SNT}{\'SNT\index{Sambhunirnayatantra@\textsl{\'Sambhunir\d nayatantra}}}
\newcommand{\SP}{SP\index{Svacchandapaddhati@\textsl{Svacchandapaddhati}}}
\newcommand{\SSK}{\d S\'SK\index{Sadanvayasambhavakrama@\textsl{\d Sa\d danvaya\'s\=ambhavakrama}}}
\newcommand{\SSP}{S\'SP\index{Somasambhupaddhati@\textsl{Soma\'sambhupaddhati}}}
\newcommand{\SSS}{\d SSS\index{Satsahasrasamhita@\textsl{\d Sa\d ts\=ahasrasa\d mhit\=a}}}
\newcommand{\SYM}{SYM\index{Siddhayogesvarimata@\textsl{Siddhayoge\'svar\=\i mata}}}
\newcommand{\ST}{\'ST\index{Saradatilaka@\textsl{\'S\=arad\=atilaka}}}
\newcommand{\STKU}{STKU\index{Sardhatrisatikalottara@\textsl{S\=ardhatri\'satik\=alottara}}}
\newcommand{\STcomm}{\'ST$_\textrm{comm}$\index{Saradatilaka@\textsl{\'S\=arad\=atilaka}!Raghavabhatta's comm.@ R\=aghavabha\d t\d ta's comm.}}
\newcommand{\SU}{SU\index{Subhagodaya@\textsl{Subhagodaya}}}
\newcommand{\Skandapurana}{\textsl{Skanda\-pur\=a\d na}\index{Skandapurana@\textsl{Skandapur\=a\d na}}}
\newcommand{\SvT}{SvT\index{Svacchandatantra@\textsl{Svacchandatantra}}}
\newcommand{\YHcomm}{YH$_\textrm{c}$\index{Yoginihridaya@\textsl{Yogin\=\i h\d rdaya}!comm.}}
\newcommand{\YH}{YH\index{Yoginihridaya@\textsl{Yogin\=\i h\d rdaya}}}
\newcommand{\YBh}{YBh\index{Yogabhāṣya@\textsl{Yogabh\=a\d sya}}}
\newcommand{\YSS}{YSS\index{Yogasarasamgraha@\textsl{Yogas\=arasa\d mgraha}}}
\newcommand{\TAK}{TAK} %\index{Tantrikabhidhanakosa@\textsl{T\=antrik\=abhidh\=anako\'sa}}}

\newcommand{\AgamaKL}{\textsl{Ā\-ga\-ma\-ka\-lpa\-la\-tā}}
\newcommand{\AGAMAKL}{ĀgamaKL}

\newcommand{\AgniP}{\textsl{A\-gni\-pu\-rā\-ṇa}}
\newcommand{\AGNIP}{AgniP}

\newcommand{\AstangHr}{\textsl{A\-ṣṭā\-ṅga\-hṛ\-da\-ya}}
\newcommand{\ASTANGHR}{AṣṭāṅgHṛ}

\newcommand{\Amara}{\textsl{Amarakośa}}
\newcommand{\AMARA}{Amara}

\newcommand{\Uums}{\textsl{Uttarottaramahāsaṃvāda}}
\newcommand{\UUMS}{UUMS}

\newcommand{\Ums}{\textsl{Umāmaheśvarasaṃvāda}}
\newcommand{\UMS}{UMS}

\newcommand{\KurmP}{\textsl{Kū\-rma\-pu\-rā\-ṇa}}
\newcommand{\KURMP}{KūrmP}

\newcommand{\GarPu}{\textsl{Ga\-ru\-ḍa\-pu\-rā\-ṇa}}
\newcommand{\GARPUR}{GarP}

\newcommand{\GautDhS}{\textsl{Gau\-ta\-ma\-dha\-rma\-sū\-tra}}
\newcommand{\GAUTDHS}{GautDhS}

\newcommand{\Caraka}{\textsl{Ca\-ra\-kasaṃhitā}}
\newcommand{\CARAKA}{Caraka}

\newcommand{\TakI}{\textsl{Tā\-ntri\-kā\-bhi\-dhā\-na\-ko\-śa I}}
\newcommand{\TAKI}{TAK I}

\newcommand{\TakII}{\textsl{Tā\-ntri\-kā\-bhi\-dhā\-na\-ko\-śa II}}
\newcommand{\TAKII}{TAK II}

\newcommand{\TakIII}{\textsl{Tā\-ntri\-kā\-bhi\-dhā\-na\-ko\-śa III}}
\newcommand{\TAKIII}{TAK III}

\newcommand{\Diksottara}{\textsl{Dī\-kṣo\-tta\-ra}}
\newcommand{\DIKSOTTARA}{DīkṣU}

\newcommand{\Divyav}{\textsl{Di\-vyā\-va\-dā\-na}}
\newcommand{\DIVYAV}{Divyāv}

\newcommand{\DeviP}{\textsl{De\-vī\-pu\-rā\-ṇa}}
\newcommand{\DEVIP}{DevīP}

\newcommand{\DharmP}{\textsl{Dha\-rma\-pu\-tri\-kā}}
\newcommand{\DHARMP}{DharmP}
 
\newcommand{\NaradaParivr}{\textsl{Nārada\-pari\-vrājako\-paniṣad}}
\newcommand{\NARADAPARIVR}{NāradParivrUp}

\newcommand{\Patimokkha}{\textsl{Pāti\-mokkha}}
\newcommand{\PATMOKHA}{Pātimokkha}

\newcommand{\Buddhacarita}{\textsl{Bu\-ddha\-ca\-ri\-ta}}
\newcommand{\BUDDHACARITA}{BuddhCar}

\newcommand{\Bodhisattvabhumi}{\textsl{Bodhi\-sattva\-bhūmi}}
\newcommand{\BODHISBH}{BodhisattvaBh}

\newcommand{\BrahmaVP}{\textsl{Bra\-hma\-vai\-va\-rta\-pu\-rā\-ṇa}}
\newcommand{\BRAHMAVP}{BrahmaVP}

\newcommand{\BrahmaP}{\textsl{Bra\-hma\-pu\-rā\-ṇa}}
\newcommand{\BRAHMAP}{BrahmaP}

\newcommand{\BraYa}{\textsl{Bra\-hma\-yā\-ma\-la}}
\newcommand{\BRAYA}{Brahmayāmala}

\newcommand{\BrahmandaPu}{\textsl{Bra\-hmā\-ṇḍa\-pu\-rā\-ṇa}}
\newcommand{\BRAHMANDAPUR}{BrahmāṇḍaP}

\newcommand{\BhavP}{\textsl{Bha\-vi\-ṣya\-pu\-rā\-ṇa}}
\newcommand{\BHAVP}{BhavP}

\newcommand{\NaradaP}{\textsl{Nā\-ra\-da\-pu\-rā\-ṇa}}
\newcommand{\NARADAP}{NāradaP}

\newcommand{\BhagP}{\textsl{Bhā\-ga\-va\-ta\-pu\-rā\-ṇa}}
\newcommand{\BHAGP}{BhāgP}

\newcommand{\Nisv}{\textsl{Ni\-śvā\-sa\-tattva\-saṃhi\-tā}}
\newcommand{\NISV}{Niśv}

\newcommand{\Nisvmukh}{\textsl{Ni\-śvā\-sa mu\-kha}}
\newcommand{\NISVMUKH}{NiśvMukha}

\newcommand{\Nisvmul}{\textsl{Ni\-śvā\-sa mū\-la}}
\newcommand{\NISVMUL}{NiśvMūl}

\newcommand{\Nisvnaya}{\textsl{Ni\-śvā\-sa na\-ya}}
\newcommand{\NISVNAYA}{NiśvNaya}

\newcommand{\Nisvuttara}{\textsl{Ni\-śvā\-sa u\-tta\-ra}}
\newcommand{\NISVUTTARA}{NiśvUttara}

\newcommand{\Nisvguhya}{\textsl{Ni\-śvā\-sa gu\-hya}}
\newcommand{\NISVGUHYA}{NiśvGuhya}

\newcommand{\NisvK}{\textsl{Ni\-śvā\-sa kā\-ri\-kā}}
\newcommand{\NISVK}{NiśvK}


\newcommand{\Padmap}{\textsl{Padma\-pu\-rā\-ṇa}}
\newcommand{\PADMAP}{PadmaP}

\newcommand{\BhG}{\textsl{Bhagavadgītā}}
\newcommand{\BHG}{BhG}

\newcommand{\BaudhDhS}{\textsl{Bau\-dhā\-ya\-na\-dha\-rma\-sū\-tra}}
\newcommand{\BAUDHDHS}{BaudhDhS}

\newcommand{\BhelaS}{\textsl{Bhe\-la\-saṃ\-hi\-tā}}
\newcommand{\BHELAS}{BhelaS}

\newcommand{\MatsP}{\textsl{Ma\-tsya\-pu\-rā\-ṇa}}
\newcommand{\MATSP}{MatsP}

\newcommand{\Manu}{\textsl{Manu}}
\newcommand{\MANU}{Manu}

\newcommand{\Mitaksara}{\textsl{Mitākṣarā}}
\newcommand{\MITAKSARA}{\textsl{Mitākṣarā}}

\newcommand{\MahaSubhS}{\textsl{Ma\-hā\-su\-bhā\-ṣi\-ta\-saṃ\-gra\-ha}}
\newcommand{\MAHASUBHS}{MahāSubhS}

\newcommand{\MarkP}{\textsl{Mārkaṇḍeyapurāṇa}}
\newcommand{\MARKP}{MarkP}

\newcommand{\MBh}{\textsl{Ma\-hā\-bhā\-ra\-ta}}
\newcommand{\MBH}{MBh}

\newcommand{\YajnS}{\textsl{Yājñavalkyasmṛti}}
\newcommand{\YAJNS}{YājñS}

\newcommand{\YogaS}{\textsl{Yogasūtra}}
\newcommand{\YS}{YS}

\newcommand{\YogaBh}{\textsl{Yogabhāṣya}}
\newcommand{\YBH}{YBh}

\newcommand{\Raghu}{\textsl{Ra\-ghu\-va\-ṃśa}}
\newcommand{\RAGHU}{\textsl{Ra\-ghu\-vaṃ\-śa}}

\newcommand{\Ramayana}{\textsl{Rā\-mā\-ya\-ṇa}}
\newcommand{\RAMAYANA}{\textsl{Rāmāyaṇa}}

\newcommand{\LaksmiNarS}{\textsl{Lakṣmīnārāyaṇasaṃhitā}}
\newcommand{\LAKSMINARS}{LakṣmīNārS}

\newcommand{\LinPu}{\textsl{Li\-ṅga\-pu\-rā\-ṇa}}
\newcommand{\LINPU}{LiṅP}


\newcommand{\Vagmati}{\textsl{Vā\-gma\-tī\-māhātmya\-praśaṃsā}}
\newcommand{\VAGMATI}{VāgMāPr}

\newcommand{\VajasaneyiS}{\textsl{Vā\-ja\-sa\-ne\-yi\-saṃ\-hi\-tā}}
\newcommand{\VAJASANEYIS}{VājasaneyiS}

\newcommand{\VayuP}{\textsl{Vā\-yu\-pu\-rā\-ṇa}}
\newcommand{\VAYUP}{VāyuP}

\newcommand{\VasisthaDhS}{\textsl{Vā\-si\-ṣṭha\-dha\-rma\-sū\-tra}}
\newcommand{\VASISTHADHS}{VāsiṣṭhaDhS}

\newcommand{\VamP}{\textsl{Vā\-ma\-na\-pu\-rā\-ṇa}}
\newcommand{\VAMP}{VamP}

\newcommand{\VarP}{\textsl{Va\-rā\-ha\-pu\-rā\-ṇa}}
\newcommand{\VARP}{VarP}

\newcommand{\VasDh}{\textsl{Vā\-si\-ṣṭha\-dha\-rma\-śā\-stra}}
\newcommand{\VASDH}{VasDh}

\newcommand{\VDhU}{\textsl{Vi\-ṣṇu\-dha\-rmo\-tta\-ra}}
\newcommand{\VDHU}{VDhU}

\newcommand{\VDh}{\textsl{Vi\-ṣṇu\-dha\-rma}}
\newcommand{\VDH}{VDh}

\newcommand{\VisnuP}{\textsl{Vi\-ṣṇu\-pu\-rā\-ṇa}}
\newcommand{\VISHNUP}{ViṣṇuP}

\newcommand{\VisnuS}{\textsl{Vi\-ṣṇu\-smṛ\-ti}}
\newcommand{\VISNUS}{ViṣṇuS}

\newcommand{\Vss}{\textsl{Vṛ\-ṣa\-sā\-ra\-saṃ\-gra\-ha}}
\newcommand{\Vsssc}{{V\scR\scS a\-sā\-ra\-sa\scM\-gra\-ha}}
\newcommand{\VSS}{{VSS}}

\newcommand{\SamkhK}{\textsl{Sāṃkhyakārikā}}
\newcommand{\SK}{SāṃkhyK}

\newcommand{\SDhS}{\textsl{Śi\-va\-dhar\-ma\-śās\-tra}}
\newcommand{\SDHS}{ŚDhŚ}

\newcommand{\SDhSamgr}{\textsl{Śi\-va\-dhar\-ma\-saṅ\-gra\-ha}}
\newcommand{\SDHSAMGR}{ŚDhSaṅgr}

\newcommand{\SDhU}{\textsl{Śi\-va\-dhar\-mo\-tta\-ra}}
\newcommand{\SDHU}{ŚDhU}

\newcommand{\SannyasUp}{\textsl{Sa\-nnyā\-so\-pa\-ni\-ṣad}}
\newcommand{\SANNYASUP}{SannyāsUp}

\newcommand{\SivP}{\textsl{Śi\-va\-pu\-rā\-ṇa}}
\newcommand{\SIVP}{ŚivP}

\newcommand{\Sivaup}{\textsl{Śi\-vo\-pa\-ni\-ṣad}}
\newcommand{\SIVAUP}{ŚivaUp}

\newcommand{\Sivasamkalpa}{\textsl{Śi\-va\-saṃ\-ka\-lpa}}
\newcommand{\SIVAKALPA}{ŚivaSaṃkalpa}

\newcommand{\SkandaP}{\textsl{Ska\-nda\-pu\-rā\-ṇa}}
\newcommand{\SKANDAP}{SkandaP}

\newcommand{\Hyp}{\textsl{Ha\-ṭha\-yo\-ga\-pra\-dī\-pi\-kā}}
\newcommand{\HYP}{HYP}

\newcommand{\Hitop}{\textsl{Hi\-to\-pa\-de\-śa}}
\newcommand{\HITOP}{Hitop}

%%%%%%%%%%%%%%%%%%%%%%%%%%%%%%%%%%%%%
%Śaṅkaradigvijayas

\newcommand{\VSV}{ŚVV\index{Sankaravijaya@\textsl{\'Sa\.nkaravijaya} of Vy\=as\=acala}}
\newcommand{\SDV}{ŚDV\index{Sankaradigvijaya@\textsl{\'Sa\.nkaradigvijaya} of\\ M\=adhva/Vidy\=ara\d nya}}
\newcommand{\ASV}{ŚVA\index{Sankaravijaya@\textsl{\'Sa\.nkaravijaya} of (Anant)\=anandagiri}}
\newcommand{\RSA}{ŚAR\index{Sankarabhyudaya@\textsl{\'Sa\.nkar\=abhyudaya} of R\=ajac\=u\d d\=ama\d ni D\=\i k\d sita}}
\newcommand{\SSD}{SŚD\index{Samksiptasankaradigvijaya@\textsl{Sa\d mk\d sipta\'sa\.nkaradigvijaya} of\\ \=Aditya\'sa\.nkar\=ac\=arya}}


%%%%%%%%%%%%%%%%%%%%%%%%%%%
% I am using this:
% no lemma, no number
%%%%%%%%%%%%%%%%%%%%%%%%%%%
%\makeatletter
% my old version
%\def\lemmaesszamnelkulfmt#1#2#3{%
%\tolerance=99\normal@pars\rightskip=0pt\leftskip=0pt
%\parindent=0pt \parfillskip=0pt plus 1fil {#3}\penalty-10}
%\makeatother

% https://djdekker.net/ledmac/#mult
% Q 24: Help! My footnotes are running off the bottom of the page!
\renewcommand{\footfudgefiddle}{90}

% updated current version
\makeatletter
 \newcommand*{\lemmaesszamnelkulfmt}[3]{%
\ledsetnormalparstuff%
%\enspace
#3\allowbreak
}
\makeatother


%%%%%%%%%%%%%% inactive %%%%%%%%%%%%%%%%%%%%%%
\makeatletter
% I'd like a spaced out colon after the lemma:
\def\spacedcolon{{\rm\thinspace:\thinspace}}
\def\normalfootfmt#1#2#3{%
  \normal@pars
  \parindent=0pt \parfillskip=0pt plus 1fil
  {\notenumfont\printlines#1|}\strut\enspace
  {\select@lemmafont#1|#2}\spacedcolon\enskip#3\strut\par}

% And I'd like the 3-col notes printed with a hanging indent:
\def\threecolfootfmt#1#2#3{%
  \normal@pars
  \hsize .3\hsize
  \parindent=0pt
  \tolerance=5000       % high, but not infinite
  \raggedright
  \hangindent1.5em \hangafter1
  \leavevmode
  \strut\hbox to 1.5em{\notenumfont\printlines#1|\hfil}\ignorespaces
  {\select@lemmafont#1|#2}\rbracket\enskip
  #3\strut\par}

\makeatletter
\def\myparafootfmt#1#2#3{%
\tolerance=9999\normal@pars\rightskip=0pt\leftskip=0pt
\parindent=0pt \parfillskip=0pt plus 1fil{\bf l}$_{\bf\printlines#1|}$\thinspace #3\penalty-10}
\makeatother

% And I'd like the 2-col notes printedf with a double colon:
\def\doublecolon{{\rm\thinspace::\thinspace}}
\def\twocolfootfmt#1#2#3{%
  \normal@pars
  \hsize .45\hsize
  \parindent=0pt
  \tolerance=5000
  \raggedright
  \leavevmode
  \strut{\notenumfont\printlines#1|}\enspace
  {\select@lemmafont#1|#2}\doublecolon\enskip
  #3\strut\par}

% And in the paragraphed footnotes, I'd like a colon too:
\def\parafootfmt#1#2#3{%
  \normal@pars
  \parindent=0pt \parfillskip=0pt plus 1fil
  {\notenumfont\printlines#1|}%
  {\select@lemmafont#1|#2}:%
  #3\penalty-10 }
\makeatother

% I'd like the line numbers picked out in bold.
%\let\notenumfont=\eightbf
%\lineation{page}
%\linenummargin{inner}
%\firstlinenum=3       % just because I can
%\linenumincrement=1

\makeatletter
 \def\lemmaesszamnelkulparallelsfmt#1#2#3{%
\tolerance=9999\normal@pars\rightskip=0pt\leftskip=0pt
\parindent=0pt \parfillskip=0pt plus 1fil #3\penalty-10}
\makeatother
%%%%%%%%%%%%%%%%%%%%%%%%%%%%%%%%%%%%%%%%%

% NAPLES : two-column apparatus
\makeatletter
 \def\lemmaesszamnelkultwocolfmt#1#2#3{%
\normal@pars\parindent=-10pt\rightskip=0pt\leftskip=10pt
\tolerance=9999
  \hsize .46\hsize
  %\raggedright
  %\leavevmode
  \strut{}%\enspace
  {\select@lemmafont#1|#2}%\enskip
  \textit{#3}\strut\par}
\makeatother

%%%% no lemma, but with number(?) %%%%%%%%%%%%%%%%%%%%%%%%%%%
\makeatletter
\def\lemmanelkulfmt#1#2#3{%
 \normal@pars\rightskip=0pt\leftskip=0pt
\parindent=0pt \parfillskip=0pt plus 1fil
{{\bf\printlines#1| }}%
{\select@lemmafont#1|#2}#3 \penalty-10}

\makeatother
\footparagraph{A}
\footparagraph{B}
\footparagraph{C}
%\footnormal{C}
%\footparagraph{D}
%\foottwocol{D} % if you want 2-col apparatus STEP 1
\footparagraph{E}

% register separators, could be Alex Watson-style
\usepackage{bbding} % for the lohere symbol
\newcommand{\anormalfootnoterule}{\begin{center}{\tiny\CrossClowerTips}\end{center}}
%\newcommand{\bnormalfootnoterule}{\begin{center}{\tiny\CrossClowerTips}\end{center}\vskip.5em}
\newcommand{\bnormalfootnoterule}{\begin{center}{\vskip-0em\rule{3em}{.1pt}}\end{center}\vskip0em}
%\newcommand{\cnormalfootnoterule}{\begin{center}{\vskip.5em\rule{3em}{.1pt}}\end{center}\vskip.5em}
%\newcommand{\dnormalfootnoterule}{\begin{center}{\vskip-.5em\rule{3em}{.1pt}}\end{center}\vskip.5em}
\newcommand{\cnormalfootnoterule}{\begin{center}{\tiny\CrossClowerTips}\end{center}}
\newcommand{\dnormalfootnoterule}{\begin{center}{\vskip-0em}\end{center}\vskip0em}
%\newcommand{\paranormalfootnoterule}{\begin{center}{\rule{3em}{.1pt}}\end{center}\vskip-.1em}

\let\Afootnoterule=\anormalfootnoterule
\let\Bfootnoterule=\bnormalfootnoterule
\let\Cfootnoterule=\cnormalfootnoterule
\let\Dfootnoterule=\dnormalfootnoterule


%%%%%%%%%%%%%%%%%%%%%%%%%%%%%%%%%%%%%%%%%%%%%%%%%%%%
%\let\Afootfmt\lemmaesszamnelkulfmt % without automatic lemma and any number;
% these you add manually; for Sanskrit, automatic lemma is a nightmare
\let\Afootfmt\lemmaesszamnelkulfmt
\let\Bfootfmt\lemmaesszamnelkulfmt
\let\Cfootfmt\lemmaesszamnelkulfmt
\let\Dfootfmt\lemmaesszamnelkulfmt % original 1-col apparatus
%\let\Dfootfmt\parafootfmt % another type of apparatus layer

\lineation{page}
\linenummargin{outer}
\firstlinenum{10000} % 10000 to eliminate
\linenumincrement{1}


\newcommand{\Kiss}{{\englishfont\textsc{Kiss}}}%\index{Kiss, Csaba}}
\newcommand{\Goodall}{{\englishfont\textsc{Good\-all}}}%\index{Goodall, Dominic}}
\newcommand{\Sanderson}{{\englishfont \textsc{Sand\-erson}}}%\index{Sanderson, Alexis}}
\newcommand{\Torzsok}{{\englishfont\textsc{T\"or\-zs\"ok}}}%\index{Törzsök, Judit}}
\newcommand{\Szanto}{{\englishfont\textsc{Szántó}}}
\newcommand{\Kafle}{{\englishfont\textsc{Kafle}}}
\newcommand{\Yokochi}{{\englishfont\textsc{Yokochi}}}
\newcommand{\emeYokochi}{{\eme~\Yokochi}}

\newcommand{\Kloppenborg}{\textsc{Kloppenborg}}%\index{Kloppenborg}}
\newcommand{\Bader}{\textsc{Bader}}%\index{Bader}}
\newcommand{\Clark}{\textsc{Clark}}%\index{Clark}}
\newcommand{\Jowett}{\textsc{Jowett}}%\index{Jowett}}
\newcommand{\Dehejia}{\textsc{Dehejia}}%\index{Dehejia}}
\newcommand{\Pandeya}{\textsc{P\={a}\d{n}\d{d}eya}}%\index{{P\={a}\d{n}\d{d}eya}}}
\newcommand{\Hatley}{\textsc{Hat\-ley}\index{Hatley, Shaman}}
\newcommand{\Wise}{\textsc{Wise}}%\index{Wise}}
\newcommand{\Dey}{\textsc{Dey}}%\index{Dey}}
\newcommand{\Mallinson}{\textsc{Mallin\-son}\index{Mallin\-son, James}}
\newcommand{\White}{\textsc{White}\index{White}}
\newcommand{\Goudr}{\textsc{Goudria{a}n}\index{Goudria{a}n, Teun}}
\newcommand{\Kale}{\textsc{Kale}\index{Kale}}
\newcommand{\Bagchi}{\textsc{Bagchi}}%\index{Bagchi}}
\newcommand{\Briggs}{\textsc{Briggs}}%\index{Briggs}}
\newcommand{\Brunner}{\textsc{Brunner}\index{Brunner, H\'el\`ene}}
\newcommand{\Dyczkowski}{\textsc{Dyczkowski}}%\index{Dyczkowski}}
\newcommand{\Finn}{\textsc{Finn}}%\index{Finn}}
\newcommand{\Garzilli}{\textsc{Garzilli}}%\index{Garzilli}}
\newcommand{\Heilijgers}{\textsc{Heilijgers}}%\index{Heilijgers}}
\newcommand{\Mallik}{\textsc{Mallik}}%\index{Mallik}}
\newcommand{\McGregor}{\textsc{McGregor}}%\index{McGregor}}
\newcommand{\Rao}{\textsc{Rao}}%\index{Rao}}
\newcommand{\Vasudeva}{\textsc{Vasudeva}\index{Vasudeva, Somadeva}}
\newcommand{\Vyas}{\textsc{Vyas \&\ Kshirsagar}}%\index{Vyas \&\ Kshirsagar}}
\newcommand{\Bouy}{\textsc{Bouy}}%\index{Bouy}}
\newcommand{\Sensharma}{\textsc{Sensharma}}%\index{Sensharma}}
\newcommand{\Bisschop}{\textsc{Bisschop}}%\index{Bisschop}}
\newcommand{\Schoterman}{\textsc{Schoterman}}%\index{Schoterman}}
\newcommand{\Gonda}{\textsc{Gonda}}%\index{Gonda}}
\newcommand{\Brooks}{\textsc{Brooks}}%\index{Brooks}}
\newcommand{\Buhnemann}{\textsc{B\"uhnemann}}%\index{B\"uhnemann}}
\newcommand{\DeSimini}{\textsc{De Simini}}%\index{Brooks}}
\newcommand{\Khanna}{\textsc{Khanna}}%\index{Khanna}}
\newcommand{\Gupta}{\textsc{Gupta}}%\index{Gupta}}
\newcommand{\Hoens}{\textsc{Hoens}}%\index{Hoens}}
\newcommand{\Padoux}{\textsc{Padoux}}%\index{Padoux}}
\newcommand{\Avalon}{\textsc{Avalon}}%\index{Avalon}}
\newcommand{\MookerjeeKhanna}{\textsc{Mookerjee \&\ Khanna}}%\index{Mookerjee \&\ Khanna}}
\newcommand{\Rawson}{\textsc{Rawson}}%\index{Rawson}}
\newcommand{\Uberoi}{\textsc{Uberoi}}%\index{Uberoi}}
\newcommand{\Tapasyananda}{\textsc{Tapasyananda}}%\index{Tapasyananda}}
\newcommand{\Dvivedi}{\textsc{Dvivedi}}%\index{Dvivedi}}
\newcommand{\Mallman}{\textsc{Mallman}}%\index{Mallman}}
\newcommand{\West}{\textsc{West}}%\index{West}}
\newcommand{\Dearing}{\textsc{Dearing}}%\index{Dearing}}
\newcommand{\Chaudhuri}{\textsc{Chaudhuri}}%\index{Chaudhuri}}
\newcommand{\Deshpande}{\textsc{Deshpande}}%\index{Deshpande}}
\newcommand{\Digby}{\textsc{Digby}}%\index{Digby}}
\newcommand{\Levi}{\textsc{Levi}}%\index{Levi}}
\newcommand{\Dupuche}{\textsc{Dupuche}}%\index{Dupuche}}
\newcommand{\Rukmini}{\textsc{Rukmini}}%\index{Rukmini}}
\newcommand{\Rose}{\textsc{Rose}}%\index{Rose}}
\newcommand{\Bunce}{\textsc{Bunce}}%\index{Bunce}}
\newcommand{\Antarkar}{\textsc{Antarkar}}%\index{Antarkar}}
\newcommand{\Locke}{\textsc{Locke}}%\index{Locke}}
\newcommand{\Upadhyay}{\textsc{Upadhyay}}%\index{Upadhyay}}
\newcommand{\Diwakar}{\textsc{Diwakar}}%\index{Diwakar}}
\newcommand{\Wallis}{\textsc{Wallis}\index{Wallis, Christopher}}

%%% Apparatus signs
\newcommand{\corrKiss}{{\englishfont\corr}}
\newcommand{\emeKiss}{{\englishfont\eme}}
\newcommand{\conjKiss}{{\englishfont\conj}}
\newcommand{\corrHaru}{\corr~\Haru}
\newcommand{\emeHaru}{{\englishfont\eme~\Haru}}
\newcommand{\conjHaru}{\conj~\Haru}
%\newcommand{\corrHidas}{\corr~\textsc{Hidas}}
%\newcommand{\emeHidas}{\eme~\textsc{Hidas}}
%\newcommand{\conjHidas}{\conj~\textsc{Hidas}}
\newcommand{\corrHidas}{\corr}
\newcommand{\emeHidas}{\eme}
\newcommand{\conjHidas}{\conj}
\newcommand{\corrTorzsok}{{\englishfont\corr~\Torzsok}}
\newcommand{\emeTorzsok}{{\englishfont\eme~\Torzsok}}
\newcommand{\conjTorzsok}{{\englishfont\conj~\Torzsok}}
\newcommand{\emeDeSimini}{{\englishfont\eme~\DeSimini}}
\newcommand{\conjSzanto}{{\englishfont\conj~\Szanto}}
\newcommand{\corrHatley}{\corr~\Hatley}
\newcommand{\emeHatley}{\eme~\Hatley}
\newcommand{\conjHatley}{\conj~\Hatley}
\newcommand{\corrGoodall}{{\englishfont\corr~\Goodall}}
\newcommand{\emeGoodall}{{\englishfont\eme~\Goodall}}
\newcommand{\conjGoodall}{{\englishfont\conj~\Goodall}}
\newcommand{\corrKafle}{{\englishfont\corr~\Kafle}}
\newcommand{\emeKafle}{{\englishfont\eme~\Kafle}}
\newcommand{\conjKafle}{{\englishfont\conj~\Kafle}}
\newcommand{\corrSanderson}{\corr~\Sanderson}
\newcommand{\emeSanderson}{\eme~\Sanderson}
\newcommand{\conjDiwakar}{\conj~Diwakar}
\newcommand{\corrDiwakar}{\corr~Diwakar}
\newcommand{\emeDiwakar}{\eme~Diwakar}
\newcommand{\conjSanderson}{\conj~\Sanderson}



% Linux Tamil font
%\newfontfamily\tamilfont{Lohit Tamil}


% Redefining section, subsection etc.
\makeatletter
\renewcommand\section{\@startsection{section}{1}{\z@}%
                                   {-3.5ex \@plus -1ex \@minus -.2ex}%
                                   {2.3ex \@plus.2ex}%
                                   {\englishfont\LARGE\bfseries}}% from \Large
\renewcommand\subsection{\@startsection{subsection}{2}{\z@}%
                                     {-3.25ex\@plus -1ex \@minus -.2ex}%
                                     {1.5ex \@plus .2ex}%
                                     {\englishfont\Large\bfseries}}% from \large
\renewcommand\subsubsection{\@startsection{subsubsection}{3}{\z@}%
                                     {-0ex\@plus -1ex \@minus -.2ex}%
                                     {1.5ex \@plus .2ex}%
                                     {\englishfont\large\bfseries}}% from \normalsize
\makeatother

