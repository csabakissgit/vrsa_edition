\newcommand{\danda}{\thinspace$\cal j$ }
\newcommand{\twodanda}{\thinspace$\cal k$ }
\newcommand{\msCa}{{\rm C$_{\scriptscriptstyle 94}$}}
\newcommand{\msCaacorr}{{\rm C$^{\scriptscriptstyle ac}_{\scriptscriptstyle 94}$}}
\newcommand{\msCapcorr}{{\rm C$^{\scriptscriptstyle pc}_{\scriptscriptstyle 94}$}}
\newcommand{\msCb}{{\rm C$_{\scriptscriptstyle 45}$}}
\newcommand{\msCbacorr}{{\rm C$^{\scriptscriptstyle ac}_{\scriptscriptstyle 45}$}}
\newcommand{\msCbpcorr}{{\rm C$^{\scriptscriptstyle pc}_{\scriptscriptstyle 45}$}}
\newcommand{\msCc}{{\rm C$_{\scriptscriptstyle 02}$}}
\newcommand{\msCcacorr}{{\rm C$^{\scriptscriptstyle ac}_{\scriptscriptstyle 02}$}}
\newcommand{\msCcpcorr}{{\rm C$^{\scriptscriptstyle pc}_{\scriptscriptstyle 02}$}}
\newcommand{\msNa}{{\rm K$_{\scriptscriptstyle 82}$}}  
\newcommand{\msNaacorr}{{\rm K$^{\scriptscriptstyle ac}_{\scriptscriptstyle 82}$}}
\newcommand{\msNapcorr}{{\rm K$^{\scriptscriptstyle pc}_{\scriptscriptstyle 82}$}}
\newcommand{\msNb}{{\rm K$_{\scriptscriptstyle 10}$}}  
\newcommand{\msNbacorr}{{\rm K$^{\scriptscriptstyle ac}_{\scriptscriptstyle 10}$}}
\newcommand{\msNbpcorr}{{\rm K$^{\scriptscriptstyle pc}_{\scriptscriptstyle 10}$}}
\newcommand{\msNc}{{\rm K$_{\scriptscriptstyle 7}$}}  
\newcommand{\msNcacorr}{{\rm K$^{\scriptscriptstyle ac}_{\scriptscriptstyle 7}$}}
\newcommand{\msNcpcorr}{{\rm K$^{\scriptscriptstyle pc}_{\scriptscriptstyle 7}$}}
\newcommand\msBod{{\rm B}}
\newcommand\msBodac{{\rm B}$^{\scriptscriptstyle ac}$}
\newcommand\msBodpc{{\rm B}$^{\scriptscriptstyle pc}$}
\newcommand\msP{{\rm P}}
\newcommand\msPac{{\rm P}$^{\scriptscriptstyle ac}$}
\newcommand\msPpc{{\rm P}$^{\scriptscriptstyle pc}$}
\newcommand\Ed{{\rm E$^{\scriptscriptstyle N}$}}
\newcommand{\msCaNa}{{\normalfont C$_\textrm{a}$N$_\textrm{a}$}}
\newcommand{\msCbNa}{{\normalfont C$_\textrm{b}$N$_\textrm{a}$}}
\newcommand{\msCcNa}{{\normalfont C$_\textrm{c}$N$_\textrm{a}$}}
\newcommand{\msCabNa}{{\normalfont C$_\textrm{a}$C$_\textrm{b}$N$_\textrm{a}$}}
\newcommand{\msCbcNa}{{\normalfont C$_\textrm{b}$C$_\textrm{c}$N$_\textrm{a}$}}
\newcommand{\msCabcNa}{{\normalfont C$_\textrm{a}$C$_\textrm{b}$C$_\textrm{c}$N$_\textrm{a}$}}
\newcommand{\msCab}{{\normalfont C$_\textrm{a}$C$_\textrm{b}$}}
\newcommand{\msCac}{{\normalfont C$_\textrm{a}$C$_\textrm{c}$}}
\newcommand{\msCbc}{{\normalfont C$_\textrm{b}$C$_\textrm{c}$}}
\newcommand{\msCabc}{{\normalfont C$_\textrm{a}$C$_\textrm{b}$$_\textrm{c}$}}
\newcommand{\mssCaCbCc}{{\normalfont C}}
\newcommand{\msL}{{\rm L}\allowbreak}
\newcommand{\msLacorr}{{\rm L}$^{\scriptscriptstyle ac}$\allowbreak}
\newcommand{\msLpcorr}{{\rm L}$^{\scriptscriptstyle pc}$\allowbreak}
\newcommand{\msM}{{\rm L}\allowbreak}
\newcommand{\Cod}{\textit{Cod.}}
\newcommand{\Codd}{$\Sigma$}

\newcommand{\msA}{{\rm A}}
\newcommand{\msB}{{\rm A}}
\newcommand{\msC}{{\rm A}}
\newcommand{\msD}{{\rm A}}
\newcommand{\msE}{{\rm A}}
\newcommand{\msF}{{\rm A}}

\newcommand{\KKT}{KKT}
\newcommand{\LP}{{\englishfont Li\.nPu}}
\newcommand{\msNKeightytwo}{{\englishfont N$^{\scriptscriptstyle K}_{\scriptscriptstyle 82}$}\allowbreak}
\newcommand{\msNKeightytwoac}{{\englishfont N$^{\scriptscriptstyle Kac}_{\scriptscriptstyle 82}$}\allowbreak}
\newcommand{\msNKeightytwopc}{{\englishfont N$^{\scriptscriptstyle Kpc}_{\scriptscriptstyle 82}$}\allowbreak}

\newcommand{\msNKtwelve}{{\englishfont N$^{\scriptscriptstyle K}_{\scriptscriptstyle 12}$}\allowbreak}
\newcommand{\msNKtwelveac}{{\englishfont N$^{\scriptscriptstyle Kac}_{\scriptscriptstyle 12}$}\allowbreak}
\newcommand{\msNKtwelvepc}{{\englishfont N$^{\scriptscriptstyle Kpc}_{\scriptscriptstyle 12}$}\allowbreak}

\newcommand{\msNKtwelveb}{{\englishfont N$^{\scriptscriptstyle K}_{\scriptscriptstyle 12b}$}\allowbreak}
\newcommand{\msNKtwelvebac}{{\englishfont N$^{\scriptscriptstyle Kac}_{\scriptscriptstyle 12b}$}\allowbreak}
\newcommand{\msNKtwelvebpc}{{\englishfont N$^{\scriptscriptstyle Kpc}_{\scriptscriptstyle 12b}$}\allowbreak}

\newcommand{\msNCfortyfive}{{\englishfont N$^{\scriptscriptstyle C}_{\scriptscriptstyle 45}$}\allowbreak}
\newcommand{\msNCfortyfiveac}{{\englishfont N$^{\scriptscriptstyle Cac}_{\scriptscriptstyle 45}$}\allowbreak}
\newcommand{\msNCfortyfivepc}{{\englishfont N$^{\scriptscriptstyle Cpc}_{\scriptscriptstyle 45}$}\allowbreak}

\newcommand{\msNCninetyfour}{{\englishfont N$^{\scriptscriptstyle C}_{\scriptscriptstyle 94}$}\allowbreak}
\newcommand{\msNCninetyfourac}{{\englishfont N$^{\scriptscriptstyle Cac}_{\scriptscriptstyle 94}$}\allowbreak}
\newcommand{\msNCninetyfourpc}{{\englishfont N$^{\scriptscriptstyle Cpc}_{\scriptscriptstyle 94}$}\allowbreak}
\newcommand{\msNCninetyfouracorr}{{\englishfont N$^{\scriptscriptstyle Cac}_{\scriptscriptstyle 94}$}\allowbreak}
\newcommand{\msNCninetyfourpcorr}{{\englishfont N$^{\scriptscriptstyle Cpc}_{\scriptscriptstyle 94}$}\allowbreak}

\newcommand{\msNKtwentyeight}{{\englishfont N$^{\scriptscriptstyle K}_{\scriptscriptstyle 28}$}\allowbreak}  
\newcommand{\msNKtwentyeightac}{{\englishfont N$^{\scriptscriptstyle Kac}_{\scriptscriptstyle 28}$}\allowbreak}  
\newcommand{\msNKtwentyeightpc}{{\englishfont N$^{\scriptscriptstyle Kpc}_{\scriptscriptstyle 28}$}\allowbreak}  

\newcommand{\msNKoseventyseven}{{\englishfont N$^{\scriptscriptstyle Ko}_{\scriptscriptstyle 77}$}\allowbreak}  
\newcommand{\msNKoseventysevenac}{{\englishfont N$^{\scriptscriptstyle Koac}_{\scriptscriptstyle 77}$}\allowbreak}  
\newcommand{\msNKoseventysevenpc}{{\englishfont N$^{\scriptscriptstyle Kopc}_{\scriptscriptstyle 77}$}\allowbreak}  





% SDHS10
\newcommand{\msGa}{{\englishfont G$^{\scriptscriptstyle Ki}$}\allowbreak}
\newcommand{\msGaac}{{\englishfont G$^{\scriptscriptstyle Kiac}$}\allowbreak}
\newcommand{\msGapc}{{\englishfont G$^{\scriptscriptstyle Kipc}$}\allowbreak}



\thispagestyle{empty}\label{startoftranslation}
\ \vskip6em\begin{center}{\Huge \textit{Vṛṣasārasaṃgrahaḥ
}}\vskip1em {\Large (translation)}\bigskip\\ {\large\today}\end{center}
\vfill\pagebreak

\thispagestyle{empty}\addcontentsline{toc}{section}{Chapter 1}
\begin{center}
{\large{Chapter One}}
\end{center}




\begin{center}
{{[Invocation]}}
\end{center}




\slokawithfn{1.1}{  Having bowed to [Him] whose boundaries are limitless, who has no beginning, no middle part and no end, [to Him] who is very subtle and who is the unmanifest and fine essence of the world, [to Him] who is wholly complete with Hari, Indra, Brahmā and the other [gods], I shall recite [the work called] `A Compendium on the Essence of the Bull [of Dharma]'.}
{ \skt{Pāda} a is reminiscent of, among other famous passages, Bhagavadgītā 11.19:  

 \skt{anādimadhyāntam anantavīryam 
 anantabāhuṃ śaśisūryanetram \danda 
 paśyāmi tvāṃ dīptahutāśavaktraṃ 
 svatejasā viśvam idaṃ tapantam \twodanda} 

  See also Bhagavadgītā 10.20cd: 

 \skt{aham ādiś ca madhyaṃ ca bhūtānām anta eva ca} \twodanda 

  A faint reference to the Bhagavadgītā seems proper at the beginning of a work that claims to deliver a teaching  based on, but also to surpass, the Mahābhārata (see following verses). See also e.g. Kūrmapurāṇa 1.11.237:  

         \skt{rūpaṃ tavāśeṣakalāvihīnam
         agocaraṃ nirmalam ekarūpam \danda 
         anādimadhyāntam anantam ādyaṃ  
         namāmi satyaṃ tamasaḥ parastāt} \twodanda 

  To say that a god has no beginning and no end in a temporal or spacial sense is natural (\skt{anādi°...°antam}), but to have no `middle part' (\skt{°madhya°}) in these senses is slightly less so.  Thus the rather commonly occuring phrase \skt{anādimadhyāntam} is probably  a fixed expression usually referring to  a formless, abstrace deity that is endless, eternal and immaterial. As to which deity or what form of a deity this stanza  refers to, it may be Śiva, his name not being listed explicitely in pāda c, but the phrasing of the verse is vague enough to keep the question somewhat open: the impersonal Brahman might be another option, even more so if we look at 1.9--10, two verses nearby  discussing \skt{brahmavidyā}.                  

 In \skt{pāda} b \skt{jagat-susāraṃ} is most probably not  to be interpreted as \skt{jagatsu sāraṃ} (`the essence in the worlds').  

 Strictly speaking, \skt{pāda} c is unmetrical, but it is better to  simply acknowledge here the phenomenon of `muta cum liquida', namely that syllables followed by consonant clusters such as  \skt{ra, bra, hra, kra, śra, śya, śva, sva, dva} can be treated as short. (See Introduction CHECK) Thus \skt{harīndrabrahmā°} can be treated as a regular beginning of an \skt{upajāti} (. - . - -), the syllable  \skt{bra} not turning the previous syllable long.  

 The reading \skt{āsamagraṃ} in \skt{pāda} c is suspect, although the initial \skt{ā-} might convey the meaning of completeness (See e.g. Kale Higher Grammar, 126). The fact that we could percieve the ends of \skt{pāda}s a and b, as well as \skt{pāda}s c and d, as rhyming pairs suggests that accepting the reading \skt{āsamagram} can be the right decision (as suggested by Alessandro Battistini). I translate this verse accordingly. \msM\ gives an exciting, albeit unmetrical, alternative (\skt{yat samagraṃ}), but this seems more like a guess to me than the correct reading. For some time I was considering emending \skt{āsamagraṃ}. The most tempting of all the possible options  (\skt{arcyam/arhyam/arghyam/īḍyam/āḍhyam agraṃ, āsamastaṃ})  seemed to be \skt{āptam agraṃ}, meaning `appointed/received/respected [by Hari, Indra, Brahmā etc.] as the foremost one'. The fact that  the \skt{akṣara}s \skt{āsam} and \skt{āptam} look similar in most of the scripts used in our manuscripts could support this conjecture. \skt{Āptam} could also possibly refer to the text itself, although then the syntax becomes slightly confusing: `I shall recite the \skt{Vṛṣasārasaṃgraha} that was first received by Hari...' etc. Another candidate was \skt{āḍhyam agram}: `Having bowed to [Him] who contains Hari, Indra, Brahmā etc.' I have not emended the text because it is difficult to know if any change is required and if yes, which reading  to chose. There was no consensus when this verse was discussed in our extended Śivadharma reading group.  

 Pāda d seems hypermetrical, but it can be interpreted as a \skt{vaṃśastha} line, a change from \skt{triṣṭubh} to \skt{jagatī} (as suggested by Dominic Goodall).    }




\begin{center}
{{[The dialogue of Janamejaya and Vaiśampāyana]}}
\end{center}




\slokawithfn{1.2}{ Having listened to the Bhāratasaṃhitā [i.e. the Mahābhārata], the supreme book of a hundred thousand [verses], a thousand chapters (\skt{adhyāya}) with all its hundred sections (\skt{parvan}),}
{ The dialogue of Janamejaya and Vaiśampāyana make up the outermost layer of the VSS                  (except for the introductory stanzas 1.1-3), mostly containing                 general \skt{dharmaśāstric} material.                 

                 The hundred \skt{parvan}s of the Mahābhārata are listed in MBh 1.2.33--70. }





\slokawithfn{1.3}{ Janamejaya remained unsatisfied and what he asked Vaiśampāyana in the past, listen to that unweariedly.}
{ For a similar unsatisfaction or dissatisfaction with previous                  teachings, see Niśvāsa mūla 1.9:                 

                 <skt>vedāntaṃ viditaṃ deva sāṃkhyaṃ vai pañcaviṃśakam \danda                      na ca tṛptiṃ gamiṣyāmo hy ṛte śaivād anugrahāt \twodanda</skt>                 

                  and Śivadharmaśāstra... CHECK.  Vaiśampāyana, a Ṛṣi, the disciple of Vyāsa, great-grandson to Arjuna,                   recited the Mahābhārata at the snake sacrifice of                  Janamejaya. This setting is an echo of the starting point of the Mahābhārata, see MBh 1.1.8ff.                 In fact the next few verses in the VSS make it clear that we the VSS                 picks up where the Mahābhārata left off: Janamejaya has heard the whole Mahābhārata from                 Vaiśampāyana, but he is eager to hear more.                 

                 Note how we are forced to emend \skt{pāda} c to contain a stem form proper noun (\skt{janamejaya})                 to maintain the metre, and note how the manuscripts struggle with this \skt{pāda}.                 Stem form nouns, \skt{prātipadika}s, abound in the VSS, see Introdcution p. XXCHECK.          }





\slokawithfn{1.4}{ Janamejaya spoke: O venerable sir, O knower of the entire Dharma, O you who are well-versed in all the sciences (\skt{śāstra})! Is there a supreme and secret Dharma which liberates [us] from the ocean of mundane existence (\skt{saṃsāra})?}
{ Note \skt{dharma} as a neuter noun in \skt{pāda} c and in the next verse. }





\slokawithfn{1.5}{ Teach me the Dharma that emerged from [Vyāsa] Dvaipāyana's mouth, O best of Brahmins. Help me find satisfaction at all cost, O great ascetic!}
{ The majority of the MSS consulted include a \skt{vā} in \skt{pāda} b,                  and although \msCb's reading seems a bit smoother, that manuscript rarely gives superior readings.                 Therefore I have chosen \skt{dharmaṃ vā yad}, in which \skt{vā} is probably in a weak sense.                 That the secret Dharma Janamejaya is seeking is the one taught by Vyāsa Dvaipāyana,                 thus no real options are involved here, becomes clear in 1.6cd.                 The reading of \msM\ is tempting but could be a later correction.  \msM's readings here are unique but probably secondary. \skt{tṛptiṃ kuru} seems more                 attractive than \skt{prasādena} because it echoes \skt{atṛptaḥ} in 1.3a }





\slokawithoutfn{1.6}{ Vaiśampāyana spoke: Listen with great attention, O king, to this unsurpassed narration of Dharma. Hear the secret Dharma that I received by Vyāsa's favour.}





\slokawithfn{1.7--8}{  Viṣṇu, the great Lord, assuming the form of a twice-born [Brahmin], wanted to test the one (i.e. Anarthayajña) who performed nonmaterial sacrifices (i.e. \skt{anarthayajña}), the one who focused on his austerities and observances, the one whose conduct was virtuous and pure, and who was intent on compassion towards all living beings, and therefore he (Viṣṇu) humbly asked him a question.}
{ Note the odd syntax here: \skt{viṣṇunā... dvijarūpadharo bhūtvā papraccha}.                The agent of the active verb is in the instrumental case.                 

                On Anarthayajña, the interlocutor of VSS 1.9--10.2 and 19.1--21.22, and                an important figure discussed in 22.3ff, as well as a concept (`nonmaterial sacrifice'),                see Kiss 2022 and Introduction XXCHECK. }





\begin{center}
{{[The knowledge of Brahman]}}
\end{center}




\slokawithfn{1.9}{ [Vigatarāga spoke:] ``How is the knowledge of the Brahman to be understood if [that knowledge] is devoid of [definitions of the] form and colour [of the Brahman]? [And] the syllable that is devoid of vowels and consonants: is there anything higher than that?''}
{ The translation of this verse, and the reconstruction and interpretation                         of \skt{pāda} d, which is echoed in 1.10d, is slightly tentative.                         I doubt if \skt{kimu} could have the standard meaning `how much more/less'                         here. Rather \skt{u} is probably just an expletive. }





\slokawithoutfn{1.10}{ Anarthayajña replied: ``That syllable is not to be pronounced, is unquestionable, non-dividable, consistent, spotless, all-pervading and subtle: what could be higher than that?''}




\begin{center}
{{[The noose of death and time]}}
\end{center}




\slokawithfn{1.11}{ Vigatarāga spoke: When the body disintegrates in the ground, in water, in fire or [is torn apart] by jackals and other [animals], how is the supportless and spotless soul led [to the netherworld] by Yama's messengers?}
{ The word \skt{°śivā°} in \skt{pāda} b is slightly suspect, and could be the result                 of metathesis, from \skt{°viṣā°} (`by poison'). Nevertheless,                  jackals seems appropriate in this context, for they                  are commonly associated with human corpses, death and the cremation ground                 (see e.g. \mycite{Ohnuma2019}). }





\slokawithfn{1.12}{ How is it bound by the nooses of death/time? And if it is bodiless, how can it move? And how does the [soul of a] virtuous [person] (\skt{bahudharmakṛt}) reach heaven if it has no body? This is my doubt. Teach me. I want to know the truth.}
{ The word \skt{kāla} has, as usual, a double meaning in this verse: \skt{kālapāśa}                         is both Yama's noose, and also the limitation caused by time,                          as becomes clear at the discussion on the different time units in verses 1.18--31. }





\slokawithoutfn{1.13}{ Anarthayajña spoke: You are asking me about an extremely doubtful and problematic matter, O truest of the twice-born. It is difficult to understand by humans, and [even] by gods (\skt{deva}), demons (\skt{dānava}) and serpents (\skt{pannaga}).}





\slokawithoutfn{1.14}{ The cause of both the birth and death of the body is karma. Good and bad deeds are called the two nooses.}





\slokawithoutfn{1.15}{ [Man] goes to hell or heaven accordingly. Happiness and suffering, both arising from karma, are to be experienced by the body.}





\slokawithoutfn{1.16}{ O great Brahmin, the body is produced for humans for this reason. Now learn about that which they call the noose of time, I shall teach you, O you of great observances.}





\slokawithfn{1.17}{ [If] you don't know anything, how could you start your investigation, O twice-born? O great Brahmin, you should know the noose of time in its entirety.}
{ The variant \skt{jijñāsyasi} seems to be the lectio difficilior as opposed to                         \skt{vijñāsyasi}, but the latter could also work fine here.  Note how \msM\ (agreeing with \Ed) gives a reading that is clearly wrong.               This confirms that while \msM\ comes up with interesting readings,                  they are mostly to be ignored. }





\slokawithfn{1.18}{ Learn about time which is divided into digits (\skt{kalā}), [i.e. about] the division[s] (\skt{kalā}) of the entity [called] Time (\skt{kālatattva}). Two atomic units of time (\skt{truṭi}) is one twinkling (\skt{nimeṣa}). One digit (\skt{kalā}) is twice a twinkling.}
{ 1.18d and 1.19a are problematic in the light of 1.19b, which                          redefines \skt{kalā} in harmony with the traditional                         interpretaion, see e.g. Arthaśāstra 2.20.33: \skt{trimśatkāṣṭhāḥ kalāḥ}.                         On divisions of time, see also, e.g., Manu 1.64ff. }





\slokawithfn{1.19}{ Two digits (\skt{kalā}) form one bit (3.2 seconds; \skt{kāṣṭhā}). Thirty bits (\skt{kāṣṭhā}) is one digit (1.6 minutes; \skt{kalā}?). Thirty digits (\skt{kalā}) make up one section (48 minutes; \skt{muhūrta}) according to mankind, O great Brahmin.}
{ I have calculated 3.2 seconds for one \skt{kāṣṭhā} backwards, starting from one day (see 1.20ab). }





\slokawithoutfn{1.20}{ Thirty sections (\skt{muhūrta}) are known to the wise as night and day [i.e. a full day]. Thirty days and nights are taught by the wise ones to be one month.}





\slokawithoutfn{1.21}{ One year is twelve months [according to] people who know the entity of time. The time span of three hundred and sixty thousand years}





\slokawithfn{1.22}{ by human standards is said to be the Kali era. The Dvāpara era is known to be twice as long as the Kali era.}
{ Note the stem form noun \skt{yuga} metri causa. }





\slokawithfn{1.23}{ The Tretā era is thrice [as long], the Kṛta era four [times as long as the Kali]. Taking these numbers related to the Four Yugas [= a \skt{mahāyuga}] seventy-one [times],}
{ The element \skt{°yugā°} seems to stand for °yuga° metri causa.                 If \skt{°yugā} and \skt{saṃkhyā} are to be separated, \skt{eṣā} becomes                  problematic to interpret. }





\slokawithfn{1.24}{ the knowledge about one time-span of Manu is being taught briefly [i.e.\ 71 four-fold \skt{mahāyuga}s make up a \skt{manvantara}]. One Kalpa is fourteen \skt{manvantara}s in total.}
{ See 21.34ff. }





\slokawithoutfn{1.25}{ Brahmā's day is made up of ten thousand Kalpas. [Brahmā's] night is of the same [length] according to the wise who know the truth.}





\slokawithfn{1.26}{ When [Brahmā's] night falls, the whole moving and unmoving universe dissolves. And when [his] daylight comes, the moving and unmoving [universe] is born.}
{ The plural form \skt{pralīyante} in \skt{pāda} a is metri causa for \skt{pralīyate},                 perhaps also influencing \skt{utpadyante} (for \skt{utpadyate}) in \skt{pāda} d,                 which in turn is used here to avoid an iambic pattern (- - . - . - . -). }





\slokawithfn{1.27}{ One \skt{para} times \skt{parārdha} [number of, i.e. two hundred quadrillion times a hundred quadrillion] \skt{kalpas} have passed [so far], O great Brahmin. Bhṛgu and the other sages say that the future is the same [time span].}
{ Note the peculiar compound \skt{bhṛgu-r-ādi-maharṣayaḥ}. }





\slokawithoutfn{1.28}{ Just as the sun, the planets, the stars and the moon are percieved in this world as wandering around, the wheel of time (\skt{kālacakra}) keeps spinning and we never experience its halting.}





\slokawithoutfn{1.29}{ Time creates living beings and time destroys them again. Everything is under the control of time. There is nothing that can bring time under control.}





\slokawithfn{1.30}{ Fourteen \skt{parārdha}s is [the number of] the kings of the gods [i.e. Indras?], O Brahmin, who passed by over time, for time is difficult to overcome.}
{ Note that \skt{samatītāni} (neuter) most probably picks up \skt{devarājāḥ}                 (masculine) in this verse, or rather \skt{devarājā} stands for                 \skt{devarājānāṃ} and \skt{samatītāni} picks up \skt{°parārdhāni}. }





\slokawithoutfn{1.31}{ Time is [manifest] as a great yogin, as Brahmā, Viṣṇu and supreme Śiva, it is beginningless and endless, it is the creator, the great soul. Pay homage [to Time].}




\begin{center}
{{[The parārdha etc.: numbers]}}
\end{center}




\slokawithfn{1.32}{ Vigatarāga spoke: I have just heard [the term] `wheel of time' (\skt{kālacakra}) uttered from [your] lotus mouth, as well as \skt{parārdha} and \skt{para}. You have made these things appear as exciting, as things to hear.}
{ The reading of all manuscripts consulted, \skt{vinisṛtam},                  may be considered metrical if we interpret it, loosely, as \skt{vinisritam}.         

                   \skt{Pāda} d is suspicious and my translation is tentative.                   \msM s reading in pāda d (\skt{srotuṃ naḥ pratidīyatāṃ}) might make sense                    ("give it back/repeat it for us again"), but it sounds forced,                   as if the scribe tried to come up with a reading that he understood                   better than \skt{srotuṃ vaḥ pratidīpitam}, which is in fact not easy to interpret. }





\slokawithoutfn{1.33}{ Anarthayajña spoke: One, ten, a hundred, a thousand, and ten thousand (\skt{ayuta}), a hundred thousand (\skt{prayuta}), a million (\skt{niyuta}), ten millions (\skt{koṭi}), a hundred millions (\skt{arbuda}), and a billion (\skt{vṛnda}, 10 to the power of 9),}





\slokawithfn{1.34}{ ten billion (\skt{kharva}), a hundred billion (\skt{nikharva}), one trillion (\skt{śaṅku}, 10 to the power of 12), and ten trillion (\skt{padma}), a hundred trillion (\skt{samudra}), one quadrillion (\skt{madhya} 10 to the power of 15), ten quadrillion (\skt{[an]anta}), a hundred quadrillion (\skt{parārdha}), and two hundred quadrillion (\skt{para}).}
{ For \skt{anta} meaning \skt{ananta}, see 1.58cd-59ab. \msM's reading in pāda d                         may be a result of an eyeskip to 1.35c. }





\slokawithoutfn{1.35}{ All should be known as powers of ten up to \skt{parārdha}. The number corresponding to \skt{para} is double the \skt{parārdha}.}





\slokawithoutfn{1.36}{ There is no higher number than \skt{para}. This is my firm conviction, which is based on my readings of the Purāṇas and the Vedas and [which I have now] taught [to you], O great Brahmin.}




\begin{center}
{{[Brahmā's Egg]}}
\end{center}




\slokawithfn{1.37}{ Vigatarāga spoke: How many eggs of Brahmā are there? And are its measurements available anywhere? From how many finger's breadths high does the sun heat the earth?}
{ The word \skt{prāpitaṃ} is a conjecture for \skt{cāpitaṃ}, which I find unintelligible.                  Another possibility could be \skt{jñāpitaṃ}.  The purport of \skt{pāda}s c and d is slightly obscure to me. }





\slokawithfn{1.38}{ Anarthayajña spoke: How could I enumerate all the eggs of Brahmā, O twice-born? Even the gods don't know [all the details], not to mention mortals.}
{ One would expect \skt{brahmāṇḍāni} in \skt{pāda} a instead of \skt{brahmāṇḍānāṃ},                 but we should probably understand \skt{brahmāṇḍānāṃ viśeṣān prasaṃkhyātuṃ...} }





\slokawithfn{1.39}{ I shall teach [these details to you] one by one, as far as I can, O great Brahmin, in the manner in which Brahmā taught Mātariśvan in the past, truthfully.}
{ Note that in \skt{pāda} d \skt{mātariśvan} stands for the accusative \skt{mātariśvānaṃ} or                         the dative \skt{mātariśvane} or the genitive \skt{mātariśvanaḥ}.                         The claim that Brahmā taught Mātariśvan is confirmed in 1.64cd,                         again using the nominative for the accusative, dative or genitive, and                         also e.g. in Brahmāṇḍapurāṇa 3.4.58cd (see the apparatus). }





\slokawithfn{1.40}{ Ten names of all the [cosmic] rulers of each of the eight directions in Brahmā's Egg, [which is] inside Śiva's Egg, are being taught now, listen.}
{ My conjecture in pāda b is based on the fact that the                             readings transmitted in the MSS seem unintelligible and more importantly that                            these names are said to belong to \skt{nāyaka}s in the subsequent verses,                            a possible synonym of \skt{bhūbhṛt}, ('a king'), and also that                            it is a minute intervention.                                     In \skt{pāda} c, understand \skt{diśāṣṭānāṃ} as \skt{diśām aṣṭānāṃ}                                      or \skt{digaṣṭakānāṃ} }




\begin{center}
{{[The names of the cosmic rulers]}}
\end{center}



\begin{center}{{[East]}}\end{center}




\slokawithfn{1.41}{ [1] Saha, [2] Asaha, [3] Sahas, [4] Sahya, [5] Visaha, [6] Saṃhata, [7] Asahā, [8] Prasaha, [9] Aprasaha, [10] Sānu: [these are] the ten Leaders in the East.}
{ I chose to supply an \skt{avagraha} before \skt{sahā} only because all the sources                  consulted read \skt{saṃhato} as the previous word, making the \skt{sandhi}                 \skt{o-s} suspicious.                  Note that many of the names here and in the following verses are,                 in the absence of any parallel passage, rather insecure.                 What is clear here is that the names evoke the name Sahasrākṣa,                 one of the appellations of                 Indra, the guadrian of the eastern direction. }




\begin{center}{{[South-East]}}\end{center}




\slokawithoutfn{1.42}{ [1] Prabhāsa, [2] Bhāsana, [3] Bhānu, [4] Pradyota, [5] Dyutima, [6] Dyuti, [7] Dīptatejas, [8] Tejas, [9] Tejā, [10] Tejavaho: [these are] the ten}





\slokawithoutfn{1.43}{ [leaders] in the direction of Agni [SE]. Now listen to [the names for] the direction of Yama [S], O twice-born. [1] Yama, [2] Yamunā, [3] Yāma, [4] Saṃyama, [5] Yamuna, [6] Ayama,}




\begin{center}{{[South]}}\end{center}




\slokawithfn{1.44}{ [7] Saṃyana, [8] Yamanoyāna, [9] Yaniyugmā, [10] Yanoyana. [1] Nagaja, [2] Naganā, [3] Nanda, [4] Nagara, [5] Naga, [6] Nandana,}
{ I have choosen the variant \skt{saṃyano} in \skt{pāda} c only to avoid the repetition of                         the name \skt{saṃyama}, and the variant \skt{yanoyanaḥ} because I suspect that                         most of the names here should begin with \skt{ya}. All the name forms                         in this verse are to be taken as tentative. The only                          guiding light is the presence of \skt{ya}, reinforcing a connection with Yama. }




\begin{center}{{[South-West]}}\end{center}




\slokawithfn{1.45}{ [7] Nagarbha, [8] Gahana, [9] Guhyo, [10] Gūḍhaja: [these are] the ten associated with [the South-West]. I shall teach you the [names] in Varuṇa's direction [in the west]. Listen, O Brahmin, learn from me.}
{ Note that the reconstruction of these names are tentative. What is clear here is that the                   initials should be \skt{na} and \skt{ga}, probably suggesting a connection with                                     \skt{nirṛti}, \skt{naraka} and \skt{nāga}s. }




\begin{center}{{[West]}}\end{center}




\slokawithfn{1.46}{ [1] Babhra, [2] Setu, [3] Bhava, [4] Udbhadra, [5] Prabhava, [6] Udbhava, [7] Bhājana, [8] Bharaṇa, [9] Bhuvana, and [10] Bhartṛ: these ten dwell in Varuṇa's direction [in the west].}
{ Varuṇa upholds the sky and the earth. That could be the reason why                    these names include \skt{bharaṇa} and \skt{bhartṛ}. }




\begin{center}{{[North-West]}}\end{center}




\slokawithoutfn{1.47}{ [1] Nṛgarbha, [2] Asuragarbha, [3] Devagarbha, [4] Mahīdhara, [5] Vṛṣabha, [6] Vṛṣagarbha, [7] Vṛṣāṅka, [8] Vṛṣabhadhvaja,}





\slokawithoutfn{1.48}{ and [9] Vṛṣaja and [10] Vṛṣanandana: these are to be known properly as the ten leaders in Vāyu's direction [in the north-west], as I taught them, O twice-born.}




\begin{center}{{[North]}}\end{center}




\slokawithfn{1.49}{ [1] Sulabha, [2] Sumana, [3] Saumya, [4] Supraja, [5] Sutanu, [6] Śiva, [7] Sata, [8] Satya, [9] Laya, [10] Śambhu: [these are] the ten leaders in the north.}
{ Note how \skt{daśanāyakam} is a singular collective noun in pāda d. }




\begin{center}{{[North-East]}}\end{center}




\slokawithfn{1.50}{ [1] Indu, [2] Bindu, [3] Bhuva, [4] Vajra, [5] Varada, [6] Vara, [7] Varṣaṇa, [8] Ilana, [9] Valina, [10] Brahmā: [these are] the ten leaders in the Īśāna direction [in the north-east].}
{ The North-East seems to be occupied by Brahmā, and by kings whose names should                                     somehow evoke Brahmā's name. }




\begin{center}{{[Center]}}\end{center}




\slokawithfn{1.51}{ [1] Apara, [2] Vimala, [3] Moha, [4] Nirmala, [5] Mana, [6] Mohana, [7] Akṣaya, [8] Avyaya, [9] Viṣṇu, [10] Varada: [these are] the ten [leaders] in the centre.}
{ Note how the center of the universe seems to be occupied by Viṣṇu and                            notice that the last three lists above have been associated                             with Śiva, Brahmā and Viṣṇu, respectively. }





\slokawithoutfn{1.52}{ Each of the ten deities[?] has a retinue of a hundred [deities]. Each one in [these groups of] a hundred [deities] is surrounded by a thousand.}





\slokawithfn{1.53}{ Each one in these [groups of] a thousand [deities] is surrounded by ten thousand [deities]. The ten thousand by a multitude of a hundred thousand. The hundred thousand is surrounded by a million,}
{ We are forced to follow \Ed's readings here to make sense of this passage.               Note that \skt{vṛnda} is not a number here. Elsewhere in this chapter it is the word that                                         signifies `a billion'. }





\slokawithfn{1.54}{ [that is] each one has a retinue of a million [deities] (\skt{niyuta}). [Then] each [of those] is surrounded by ten million [deities] (\skt{koṭi}), [they] by a hundred million (\skt{daśakoṭi} = \skt{arbuda}).}
{ Note how the scribe of \msM\ gets confused due to an eye-skip at 1.54c and                  fully regains control only at 1.56b. }





\slokawithoutfn{1.55}{ Each one of the hundred million (\skt{daśakoṭi} = \skt{arbuda}) is surrounded by a billion (\skt{vṛnda}) bhṛta??? Each of those billion (\skt{vṛnda}) is surrounded by ten billion (\skt{kharva}) [deities].}





\slokawithoutfn{1.56}{ Each of those ten billion (\skt{kharva}) is surrounded by a hundred billion (\skt{daśakharva} = \skt{nikharva}). Each of those hundred billion (\skt{daśakharva} = \skt{nikharva}) is surrounded by a trillion (\skt{śaṅku}) [deities].}





\slokawithfn{1.57}{ Each of those one trillion (\skt{śaṅku}) is surrounded by ten trillion (\skt{padma}). Each of those ten trillion (\skt{padma}) is surrounded by a hundred trillion (\skt{samudra}).}
{ Note \skt{śaṅkubhiḥ pṛthag...}: it stands for \skt{śaṅkūṣu pṛthag...} (instrumental for locative). }





\slokawithoutfn{1.58}{ And each of those hundred trillion (\skt{samudra}) is surrounded by those whose number is one quadrillion (\skt{madhya}). Each of those quadrillion (\skt{madhya}) is surrounded by ten quadrillion (\skt{ananta}).}





\slokawithoutfn{1.59}{ Each of those ten quadrillion (\skt{ananta}) is surrounded by a hundred quadrillion (\skt{parārdha}). Each of those hundred quadrillion (\skt{parārdha}) is surrounded by two hundred quadrillion (\skt{para}). This is how it is taught, O Brahmin. [All] the possible numbers have been taught.}




\begin{center}
{{[Measurements]}}
\end{center}




\slokawithoutfn{1.60}{ Hear about the measurements [of the universe] briefly, O Brahmin, from me, I shall teach [you]. Listen to the extent [of the Brahmāṇḍa], O Brahmin! I shall teach it to you in a concise manner. The body of the Egg is like that of the full moon at moonrise.}





\slokawithfn{1.61}{ The whole circumference of the Eggs has been declared by Brahmā to be \skt{koṭi} times a thousand \skt{koṭi} yojanas.}
{ aṇḍānāṃ plural...: a new egg in every mahākalpa? CHECK }





\slokawithoutfn{1.62}{ The Sun shines from above from seven thousand and seven hundred \skt{koṭi} [height] ... twenty \skt{koṭi} gulma?? mūrdha?}





\slokawithfn{1.63}{ In brief the numbers pertaining to the measurements have been taught. The characteristics of the unmeasurable Brahmāṇḍa[s] have been taught.}
{ Note the mixture of different grammatical genders and numbers here.                  Understand \skt{pramāṇeṣu saṃkhyāḥ kīrtitāḥ samāsataḥ}. }




\begin{center}
{{[The Redactors (of the Purāṇas)]}}
\end{center}




\slokawithfn{1.64}{ O truest of the twice-born, the Purāṇa[s of] 8,000,000 [verses] were taught by [1] Brahmā to [2] Mātariśvan [= Vāyu] in their entirety, in their true form.}
{ \skt{Pāda} a should probably be analysed and interpreted as                          \skt{purāṇam (purāṇānām aśītisahasrāṇi śatāni ślokāni) brahmaṇā kathitam}.                         Alternatively, pāda a may have originally read \skt{purāṇāni sahasrāṇi},                         and then the inital number of verses transmitted by Brahmā is                         a hundred thousand.                 

                 Compare this list to Viṣṇupurāṇa 3.3.11--19:                 

                 \skt{dvāpare prathame vyastaḥ svayaṃ vedaḥ svayaṃbhuvā\danda
                 dvitīye dvāpare caiva vedavyāsaḥ prajāpati\twodanda
                 tṛtīye cośanā vyāsaś caturthe ca bṛhaspatiḥ\danda 
                 savitā pañcame vyāsaḥ ṣaṣṭhe mṛtyuḥ smṛtaḥ prabhuḥ\twodanda 
                 saptame ca tathaivendro vasiṣṭhaś cāṣṭame smṛtaḥ\danda 
                 sārasvataś ca navame tridhāmā daśame smṛtaḥ\twodanda 
                 ekādaśe tu triśikho bharadvājas tataḥ paraḥ\danda 
                 trayodaśe cāntarikṣo varṇī cāpi caturdaśe\twodanda 
                 trayyāruṇaḥ pañcadaśe ṣoḍaśe tu dhanañjayaḥ\danda 
                 kratuñjayaḥ saptadaśe tadūrdhvaṃ ca jayaḥ smṛtaḥ\twodanda 
                 tato vyāso bharadvājo bharadvājāc ca gautamaḥ\danda 
                 gautamād uttaro vyāso haryātmā yo 'bhidhīyate\twodanda 
                 atha haryātmanonte ca smṛto vājaśravāmuniḥ\danda 
                 somaśuṣkāyaṇas tasmāt tṛṇabindur iti smṛtaḥ\twodanda 
                 ṛkṣobhūdbhārgavas tasmād vālmīkir yo 'bhidhīyate\danda 
                 tasmād asmatpitā śaktir vyāsas tasmād ahaṃ mune\twodanda 
                 jātukarṇo 'bhavan mattaḥ kṛṣṇadvaipāyanas tataḥ\danda 
                 aṣṭaviṃśatir ity ete vedavyāsāḥ purātanāḥ\twodanda  
}                 

                 Another relevant passage is Brahmāṇḍapurāṇa 3.4.58cd--67:                 

                 \skt{brahmā dadau śāstram idaṃ purāṇaṃ mātariśvane\twodanda 
                 tasmāc cośanasā prāptaṃ tasmāc cāpi bṛhaspatiḥ\danda   
                 bṛhaspatis tu provāca savitre tadanantaram\twodanda   
                 savitā mṛtyave prāha mṛtyuś cendrāya vai punaḥ\danda 
                 indraś cāpi vasiṣṭāya so 'pi sārasvatāya cai\twodanda 
                 sārasvatas tridhāmne 'tha tridhāmā ca śaradvate\danda 
                 śaradvāṃs tu triviṣṭāya so 'ntarikṣāya dattavān\twodanda 
                 carṣiṇe cāntarikṣo vai so 'pi trayyāruṇāya ca\danda 
                 trayyāruṇād dhanañjayaḥ sa vai prādāt kṛtañjaye\twodanda 
                 kṛtañjayāt tṛṇañjayo bharadvājāya so 'py atha\danda   
                 gautamāya bharadvājaḥ so 'pi niryyantare punaḥ\twodanda 
                 niryyantaras tu provāca tathā vājaśravāya vai\danda   
                 sa dadau somaśuṣmāya sa cādāt tṛṇabindave\twodanda 
                 tṛṇabindus tu dakṣāya dakṣaḥ provāca śaktaye\danda 
                 śakteḥ parāśaraś cāpi garbhasthaḥ śrutavānidam\twodanda 
                 parāśarāj jātukarṇyas tasmād dvaipāyanaḥ prabhuḥ\danda   
                 dvaipāyanāt punaś cāpi mayā prāptaṃ dvijottama\twodanda 
                 mayā caitat punaḥ proktaṃ putrāyāmitabuddhaye\danda   
                 ity eva vākyaṃ brahmādiguruṇāṃ samudāhṛtam\twodanda  } 
                  }





\slokawithoutfn{1.65}{ Vāyu abridged the verses and then gave [the Purāṇas] to [3] Uśanas. He [Uśanas] also abridged the verses, and [4] Bṛhaspati received them.}





\slokawithoutfn{1.66}{ Bṛhaspati taught 30,000 [verses] to [5] Sūrya [the Sun]. Divākara [= the Sun] taught 25,000 [verses] to [6] Mṛtyu [Death].}





\slokawithoutfn{1.67}{ Death taught 21,000 [verses] to [7] Indra. Indra taught 20,000 verses to [8] Vasiṣṭha.}





\slokawithoutfn{1.68}{ And he[, Vasiṣṭha taught] 18,000 [verses] to [9] Sārasvata. Sārasvata [taught] 17,000 [verses] to [10] Tridhāman.}





\slokawithoutfn{1.69}{ [Tridhāman] taught 16,000 verses to [11] Bharadvāja. [Bharadvāja] taught 15,000 verses to [12] Trivṛṣa.}





\slokawithoutfn{1.70}{ [Trivṛṣa] then [taught] 14,000 verses to [13] Antarīkṣa. [Antarīkṣa] taught 13,000 [verses] to [14] Trayyāruṇi.}





\slokawithoutfn{1.71}{ Trayyāruṇi, the great Brahmin, having abridged them again, taught 12,000 [verses] to [15] Dhanaṃjaya.}





\slokawithoutfn{1.72}{ Dhanaṃjaya, the great sage, handed [them] over to [16] Kṛtaṃjaya. [This recension  was transmitted] from Kṛtaṃjaya, O best of the twice-born, to [17] noble Ṛṇaṃjaya.}





\slokawithfn{1.73}{ Then from Ṛṇaṃjaya it was given to [18] Gautama, the great sage, from Gautama to [19] Bharadvāja, from him to [20] Dharmadvata.}
{ The name \skt{harmyadvata} is probably a variant or a corrupted form                         of \skt{harmyātman}, who appears in lists of \skt{vedavyāsa}s                         in the Purāṇas (see note to 1.64). }





\slokawithoutfn{1.74}{ Then [21] Rājaśravas received it, then [22] Somaśuṣma. Then from Somaśuṣma [23] Tṛṇabindu received it, O twice-born.}





\slokawithfn{1.75}{ Tṛṇabindu taught it to [24] Vṛkṣa, Vṛkṣa to [25] Śakti [the father of Parāśara]. Śakti taught it to [26] Parāśara, then [Parāśara] to [27] Jātūkarṇa.}
{ Perhaps keep jatu°. }





\slokawithoutfn{1.76}{ Jātukarṇa taught it to [28] [Vyāsa] Dvaipāyana, the great sage. Dvaipāyana, the great sage, gave it to Romaharṣa.}





\slokawithoutfn{1.77}{ He [Dvaipāyana] taught the Purāṇa[s] [consisting of] 12,000 [verses] to Romaharṣa, his brilliant son, [in the form that] has been revealed [to us] for the benefit of humankind. What else do you wish to know?}



\vfill\pagebreak

\thispagestyle{empty}\addcontentsline{toc}{section}{Chapter 2}
\begin{center}
{\large{Chapter Two}}
\end{center}




\slokawithfn{2.1}{ Vigatarāga spoke: I the best of men(? phps accept it) [rather: through you, the best of men], have listened to the concise description of the Brahmāṇḍa, it's extent, colour, form and the numbers associated with it.}
{ Manuscripts \msCc\ and \msM\ place the \skt{iti} of the colophon at the end of the last śloka, before                 the daṇḍas, thus: \skt{icchasīti \twodanda@\twodanda} (\msCc) and \skt{icchasi iti \twodanda o\twodanda} (\msM).                 Note also that \msM\ gives the number of ślokas in this chapter, 77, which is exactly                 the number of verses this critical edition has produced. The scribe of \msM\ struggled                  with eyeskips in this chapter, therefore it seems unlikely that he himself                 counted the number of verses he had copied and arrived at this very figure.                 Rather, he copied the number from his exemplar. }





\slokawithoutfn{2.2}{ You mentioned the Śivāṇḍa as taught to be the receptacle of the Brahmāṇḍa [see 1.40ab]. What are its characteristics and how much is its extent?}





\slokawithoutfn{2.3}{ Whose dwelling/resting place is it [phps ālayana for ālaya] and [what] is the extent/proof of the one who dwells there? [maybe the number of inhabitants Flo] [Or: what is its extent and [who are its] inhabitants]? Who are the people there? And who is Prajāpati there?}




\begin{center}
{{[Summary of the Śivāṇḍa]}}
\end{center}




\slokawithoutfn{2.4}{ Anarthayajña spoke: Please don't ask me about the characteristics of the Śivāṇḍa, O Brahmin. How could even the gods have the power to really know and see...}





\slokawithoutfn{2.5}{ The path leading to it is not to be trodden, it is extremely secret and [...] There is no master or the opposite there, nobody to be punished and no punisher.}





\slokawithoutfn{2.6}{ There are no truthful or untruthful people there, no moral or immoral people, no wicked people, no hypocrisy, no thirst or envy.}





\slokawithoutfn{2.7}{ There is no anger or desire, no arrogance or discontent ([a]sūyaka). No envy or hatred, no cheaters and no jealousy.}





\slokawithoutfn{2.8}{ There is no disease, no aging, no grief and no agitation there. There are no inferior or superior people and there is nobody in-between.}





\slokawithoutfn{2.9}{ There are no privileged men or women there in Śiva's abode, no reproach or praise, no selfish or treacherous people.}





\slokawithoutfn{2.10}{ There is no pride or arrogance there, no cruelty or trickery and so on. There are no beggars and no donors there.}





\slokawithoutfn{2.11}{ Go without material desires (\skt{anarthin}), being there you'll be resting under a wishing tree. There is no karma there and no enemy. The era of strife [the Kali era] is not there and there is no fight.}





\slokawithoutfn{2.12}{ There is no Dvāpara era or Tretā or Kṛta. There are no Manvantaras (1 Manvantara = 1000 Kalpas) there and no Kalpas.}





\slokawithoutfn{2.13}{ No universal floods of destruction come, and there are no days and nights of Brahmā. There is no birth and death there and one never encounters catastrophes.}





\slokawithoutfn{2.14}{ Nobody is tied to the noose of hope and there is no passion or delusion. There are no gods and demons there and no Yakṣas, Serpents and Rākṣasas.}





\slokawithoutfn{2.15}{ There are no Ghosts nor Piśācas, no Gandharvas and no Ṛṣis. There are no asterisms and planets there, no Nāgas, Kiṃnaras or Garuḍa-like creatures.}





\slokawithoutfn{2.16}{ There is no recitation there or daily rituals, nobody performs the Agnihotra and there is no sacrificer. There are no religious observances and no austerities and no 'animal hell' [or: on animals and no hell].}





\slokawithoutfn{2.17}{ Nobody would be able to tell the extent of the god Īśāna's[??] powers starting with aiśvarya, not even in a hundred years.}





\slokawithoutfn{2.18}{ [Instead] I shall teach you all that are produced by Hara's wish one by one, excluding the gods and people, starting with the trees, the bushes and creepers.}





\slokawithoutfn{2.19}{ [Their?] height is two Parārdha, and [their?] width is the same. There are lovely flowers of different forms [there] and also lovely fruits.}





\slokawithoutfn{2.20}{ There are also golden trees and also gem trees, coral gem thickets and ruby plants.}





\slokawithfn{2.21}{ There are trees with twigs on which creepers with tasty roots reach for the tasty fruits. [REVISE] All of them can change their shapes on their own accord [just bending etc.?] and they fulfill man's desires and they whisper in a lovely way[?] [any language? maybe not].}
{ After kāmarū°, MS \msCc\ has some folios missing and resumes only at 3.XX. CHECK Florinda's pics! }





\slokawithoutfn{2.22}{ There [in the Śivāṇḍa], O Brahmin, all the subjects are the oceans of endless virtues. They are all equally beautiful and strong, and they shine like millions of suns.}





\slokawithoutfn{2.23}{ ... is two Parārdha [yojanas] long and two Parārdha [yojanas] wide, and two Parārdha yojanas is its extension[?], O great Brahmin.}





\slokawithoutfn{2.24}{ Authority is not a number [cannot be expressed by a number? OR: there is no question of....?] neither is the Power of strength, O twice-born. Down and up are no numbers [no question of going to heaven or hell?], and nobody goes to the Tiryañc [hell] [??? OR with iti: there is no horizontal extension?].}





\slokawithfn{2.25}{ I do not know the length and width of the Śivāṇḍa. Enjoyment is undecaying there, and there is no birth or death there.}
{ Pāda c is unmetrical, or rather, a ra-vipulā with licence                  (tatraiva as SHORT-LONG). Note also the gender problem                  (\skt{bhogam akṣayas}), or rather take \skt{-m-} as a sandhi-bridge                  (\skt{bhoga-m-akṣayas}, for \skt{bhogo 'kṣayas}). }





\slokawithoutfn{2.26}{ Inside the Śivāṇḍa, there is the dwelling-place of Īśāna's people [= Īśāna's region] [on] one and a half Para krore [yojanas? or that many people?], who shine like cow's milk [or the region shines?].}





\slokawithoutfn{2.27}{ They are all like the rising sun in the House of Tatpuruṣa [on] one and a half Para krore [yojanas? or that many people?] in the east.}





\slokawithfn{2.28}{ All of them are like collyrium in the southern direction, in the House of Aghora, [on] one and a half Para krore [yojanas?].}
{ Note the Aiśa form <i>diśiṃ</i> in <ms>C<sub>45</sub></ms>. }





\slokawithfn{2.29}{ In the western direction, in Sadyojāta's beloved House, [on] one and a half krore [yojanas?] they are like jasmine, the moon, like snowy rocks.}
{ Note the Aiśa form <i>diśiṃ</i> in <ms>K<sub>07</sub></ms> in pāda b.                 In pāda d, we may suppose the presence of a sandhi-bridge:                 <i>sadya-m-iṣṭālayaḥ</i>. }





\slokawithfn{2.30}{ In the northern direction, in Vāmadeva's House of one and a half krore [yojanas?] they are like saffron and water.}
{ Note the Aiśa form <i>diśiṃ</i> in <ms>C<sub>95</sub></ms> in pāda b. }





\slokawithfn{2.31}{ Īśāna has five parts (kalā), [his Tatpuruṣa] face has four. Aghora has eight, and there are thirteen Vāmadeva[-kalā]s.}
{ Note how <i>vaktrasya</i> should refer to Śiva's Tatpuruṣa-face,                  given that the text lists Śiva's five faces: Īśāna, Tatpuruṣa, Aghora, Vāmadeva, Sadyojāta. }





\slokawithoutfn{2.32}{ Sadyojāta has eight parts. These parts, altogether thirty-eight, which liberate us from the ocean of existence, have been taught, O truest Brahmin.}





\slokawithoutfn{2.33}{ Those who explore the Truth should know the numbers, the colours and directions associated with each one [of Śiva's faces] in the way taught above.}





\slokawithoutfn{2.34}{ If one has the intention to go to the Śivāṇḍa [if he is 'pulled' towards it], one should practise Śiva yoga regularly. Without Śiva yoga, O Brahmin, it is impossible to go there.}





\slokawithfn{2.35}{ [Even] by [performing] millions of sacrifices such as the Aśvamedha, or all the difficult austerities, for a hundred Kalpas, it is impossible to get there even for the gods, O great ascetic.}
{ Understand \skt{kṛcchrāditapa sarvāṇi} as \skt{kṛcchrāditapāṃsi sarvāṇi}. }





\slokawithoutfn{2.36}{ By [merely] bathing and performing austerities at all the sacred places such as the Gaṅgā, even the honorable Ṛṣis will not be able to get there.}





\slokawithoutfn{2.37}{ Or by donating the oceans of the seven islands with all their gems to a Veda expert, O Brahmin, having faith and devotion, one will not be able to go there without meditation. [This is a] certainty.}





\slokawithoutfn{2.38}{ He who destroys his own body and gives it without hesitation to those who are in need of it, or gives away his wife, his son and his possessions or his own head to those in need, or by [performing] other difficult deeds, will not be able to go there [by merely doing these].}





\slokawithoutfn{2.39}{ He who has completed the sacrifices, the pilgrimages, the austerities, the donations, the study of the Vedas, will experience those enjoyments that the Brahmāṇḍa offers, still being subject to time and death.}





\slokawithoutfn{2.40}{ Dharma decays with time that is sent by... Like a circle of burning coal, time goes round and round. Time is called \skt{kāla} because of the waves (kalana) of the three divisions of time [past, present, future].}



\vfill\pagebreak

\thispagestyle{empty}\addcontentsline{toc}{section}{Chapter 3}
\begin{center}
{\large{Chapter Three}}
\end{center}




\begin{center}
{{[An Exposition of Dharma]}}
\end{center}




\slokawithfn{3.1}{ Vigatarāga spoke: Why do they call [Dharma] Dharma? And how many embodiments (\skt{mūrti}) is he known to have? He is known as a bull: how many legs does it/he have? How many are his paths?}
{ For the correct interpretation of \skt{pāda} a, namely to decide whether these questions                 focus on the bull of Dharma or Dharma itself/himself, see                  the end of the previous chapter, where \skt{dharma} was mentioned (2.40b),                 and to which the present verse is a reaction; and also                 MBh 12.110.10--11:                 

                 \skt{prabhāvārthāya bhūtānāṃ dharmapravacanaṃ kṛtam\danda
                  yat syād ahiṃsāsaṃyuktaṃ sa dharma iti niścayaḥ\twodanda
                  dhāraṇād dharma ity āhur dharmeṇa vidhṛtāḥ prajāḥ\danda
                  yat syād dhāraṇasaṃyuktaṃ sa dharma iti niścayaḥ\twodanda}                 

                 Note the similarities with this chapter: the phrase \skt{dharma ity āhur},                 the fact that the present chapter from verse 18 on is actually a chapter on \skt{ahiṃsā},                 and that the etimological explanation involves the word [\skt{ā}]\skt{dhāraṇa} in                 both cases. These lead me to think that in \skt{pāda}s ab of this verse in the VSS,                 it is Dharma that is the focus of the inquiry and not the bull.  Understand \skt{pāda} d as \skt{gatayas tasya kati smṛtāḥ}. I have accepted                         \skt{smṛtāḥ} because this plural signals that \skt{gatis} is meant to be plural,                         similarly to what happens in 3.6cd (\skt{tasya patnī... mahābhāgāḥ}). }





\slokawithoutfn{3.2}{ I have become curious [about these questions]. Put an end to my doubts for good. Whose son is [Dharma], O best of sages? How many children does he have?}





\slokawithfn{3.3}{ Anarthayajña spoke: Well, the root [sic!] \skt{dhṛti} (`resolution') is said to be a synonym [of \skt{dharma}]. It is called Dharma because it supports (\skt{āDHĀRaṇa}) and because it is great (\skt{MAhattva}).}
{ On a non-verbal stem being a \skt{dhātu}, see e.g.                                                  Vāyupurāṇa 3.17cd:                         \skt{bhāvya ity eṣa dhātur vai bhāvye kāle vibhāvyate};                                                 Vāyupurāṇa 3.19cd (= Brahmāṇḍapurāṇa 1.38.21ab):                         \skt{nātha ity eṣa dhātur vai dhātujñaiḥ pālane smṛtaḥ};                                                 Liṅgapurāṇa 2.9.19:                 \skt{bhaja ity eṣa dhātur vai sevāyāṃ parikīrtitaḥ};                                         \skt{ etc.} }





\slokawithfn{3.4}{ The four-legged Bull is the embodiment of both Śruti and Smṛti. The four \skt{āśrama}s are taught by the wise to be [the four legs of] Dharma. [or rather: ... which is Dharma as made up of the four āśramas... kīrtitaḥ!]}
{ A similar image of the legs of the Bull of Dharma being the four \skt{āśrama}s                  is hinted at MBh 12.262.19--21:                  

                     \skt{dharmam{ }ekaṃ catuṣpādam{ }āśritās{ }te nararṣabhāḥ\danda                
taṃ santo vidhivat{ }prāpya gacchanti paramāṃ gatim\twodanda                
gṛhebhya eva niṣkramya vanam{ }anye samāśritāḥ\danda                
gṛham{ }evābhisaṃśritya tato 'nye brahmacāriṇaḥ\twodanda                
dharmam{ }etaṃ catuṣpādam{ }āśramaṃ brāhmaṇā viduḥ\danda                
ānantyaṃ brahmaṇaḥ sthānaṃ brāhmaṇā nāma niścayaḥ\twodanda}         

         On the more frequently quoted interpretation of the four legs, see Olivelle `Āśrama', 235:         ``Dharma and truth possess all four feet and are whole during the Kṛta yuga,          and people did not obtain anything unrighteously (\skt{adharmeṇa}).          By obtaining, however, \skt{dharma} has lost one foot during each of the other \skt{yuga}s          and righteousness (\skt{dharma}) likewise has diminished by one quarter due to theft,          falsehood, and deceit. (MDh 1.81--82)''         

         Understand \skt{pāda}s c and d as \skt{catvāri āśramāṇi kīrtitāni dharmo manīṣibhiḥ} or                 \skt{yo dharmaḥ kīrtitaś caturāśramāṇi manīṣibhiḥ} or                  \skt{yo dharmaś caturāśramaḥ kīrtito manīṣibhiḥ}. }





\slokawithfn{3.5}{ And the paths of Dharma are five. Listen, O Brahmin: [existence as] gods, men, animals, [existence in] hell and [as] immovable things [such as plants and rocks] etc.}
{ Understand \skt{gatiś} as \skt{gatayaś} and note that \skt{vijñeyāḥ} is an emendation from                 \skt{vijñeyaḥ} following the logic of 3.1d.                  \skt{tirya} seems to be an acceptable nominal stem in this text for \skt{tiryañc}. See                 e.g. 4.6a: \skt{devamānuṣatiryeṣu}.  \skt{°ādayaḥ} in \skt{pāda} d seems superfluous. }





\slokawithfn{3.6}{ Eternal Dharma was born after splitting Brahmā's heart. He has beautiful wives, thirteen in number, with nice waists.}
{ Note the use of the singular in \skt{pāda}s c and d. I have left \skt{sumadhyamāḥ} as the         manuscripts transmit it: it signals the presence of the plural. And consider          correcting \skt{mahābhāgā} to \skt{mahābhāgās}. In sum,                  understand \skt{tasya patnyo mahābhāgās trayodaśa sumadhyamāḥ}. }





\slokawithfn{3.7}{ They are Dakṣa's daughters, [called] Śraddhā and so on. They have huge eyes and they are beautiful. and they are charming. Numerous sons and grandsons were born to him. This is the emergence of Dharma. What more do you wish to hear?}
{ \skt{śraddhāḍhyāḥ} in \skt{pāda} b is an attractive lectio difficilior (`they were rich in faith/devotion'), but I have finally                  decided to accept the easier and better-attested \skt{śraddhādyā}[\skt{ḥ}].  Again, I have chosen/applied the plural forms \skt{°ādyāḥ} and \skt{sumanoharāḥ} in \skt{pāda} b to hint at the fact                         that the presence of the plural is to be preferred here; thus only \skt{viśālākṣī} is                          problematic. As \skt{patnī} in the previous verse, it should be treated as a plural.         Note the use of the singular for the plural also in \skt{pāda}s cd, especially \skt{babhūva ha} for \skt{babhūvuḥ}.          

         MMW on Dakṣa:         ``daughters of whom 27 become the Moon's wives, forming the lunar asterisms, and 13          [or 17 BhP.; or 8 R.] those of Kaśyapa, becoming by this latter the mothers          of gods, demons, men, and animals, while 10 are married to Dharma, Mn. ix, 128f.'' CHECK }





\slokawithfn{3.8}{ Vigatarāga spoke: I would like to hear about Dharma's wives according to the truth[?] and about each one of the sons born to them. Teach me, O great ascetic.}
{ Consider emending \skt{tebhyaḥ} to the correct feminine form \skt{tābhyaḥ}.                 Note again the use of the singular (nominative) for the plural (accusative) in \skt{pāda}s ab.                 Alternatively, emend \skt{dharmapatnī} to \skt{dharmapatnīr} (plural accusative) and                  \skt{putras} to \skt{putrān} to make them work with \skt{śrotum icchāmi}. }





\slokawithfn{3.9}{ Anarthayajña spoke: [Dharma's wives are:] [1] Śraddhā (`Faith'), [2] Lakṣmī (`Prosperity'), [3] Dhṛti (`Resolution'), [4] Tuṣṭi (`Satisfaction'), [5] Puṣṭi (`Growth'), [6] Medhā (`Wisdom'), [7] Kriyā (`Ritual'), [8] Lajjā (`Modesty'), [9] Buddhi (`Intelligence'), [10] Śānti (`Tranquillity'), [11] Vapus (`Beauty'), [12] Kīrti (`Fame'), [13] Siddhi (`Success'), [all] born to Prasūti [Dakṣa's wife].}
{ For Dharma's thirteen wives and their sons, see Liṅgapurāṇa 1.5.34-37 (note the                         similarity between the first line and VSS 3.6cd--7ab above):         

         \skt{dharmasya patnyaḥ śraddhādyāḥ kīrtitā vai trayodaśa\danda         
tāsu dharmaprajāṃ vakṣye yathākramam{ }anuttamam\twodanda         
kāmo darpo 'tha niyamaḥ saṃtoṣo lobha eva ca\danda         
śrutas{ }tu daṇḍaḥ samayo bodhaś{ }caiva mahādyutiḥ\twodanda         
apramādaś{ }ca vinayo vyavasāyo dvijottamāḥ\danda         
kṣemaṃ sukhaṃ yaśaś{ }caiva dharmaputrāś{ }ca tāsu vai\twodanda          
dharmasya vai kriyāyāṃ tu daṇḍaḥ samaya eva ca\danda         
apramādas{ }tathā bodho buddher{ }dharmasya tau sutau\twodanda}           

\skt{prasūtisambhavāḥ} is a rather bold conjecture that can be supported by two facts:                         firstly, the readings of the manuscripts are difficult to make sense of and thus are                                          probably corrupt; secondly, a corruption from the name Prasūti,                                          that of Dakṣa's wife, to \skt{ābhūti}                                          is relatively easily to explain, \skt{sū} and \skt{bhū} being close enough in some scripts                                           (e.g. in \msCa) to cause confusion. Another option would be to accept                                           Ābhūti as the name of Dakṣa's wife.                 

                 For Prasūti being Dakṣa's wife in other sources,                          see e.g. Liṅgapurāṇa 1.5.20--21 (but also note the presence of the name Sambhūti...):                                \skt{prasūtiḥ suṣuve dakṣāc{ }caturviṃśatikanyakāḥ\danda                                 śraddhāṃ lakṣmīṃ dhṛtiṃ puṣṭiṃ tuṣṭiṃ medhāṃ kriyāṃ tathā\twodanda                                 buddhi lajjāṃ vapuḥ śāntiṃ siddhiṃ kīrtiṃ mahātapāḥ\danda                                 khyātiṃ śāntiś{ }ca saṃbhūtiṃ smṛtiṃ prītiṃ kṣamāṃ tathā\twodanda} }





\slokawithfn{3.10}{ Śraddhā's son is Kāma (`Desire'), Darpa (`Pride') is said to be Lakṣmī's son. Dhṛti's son is Niyama (`Rule'), Saṃtoṣa (`Satisfaction') is Tuṣṭi's son.}
{ Understand \skt{śraddhā} as a stem form noun for \skt{śraddhāyāḥ} (gen./abl.). }





\slokawithfn{3.11}{ To Puṣṭi was born a son [called] Lābha (`Profit'). Medhā's son is Śruta (`Sacred Knowledge'). Kriyā's sons are Abhaya (`Freedom from danger'), Daṇḍa (`Punishment') and Samaya (`Law').}
{ It is tempting to emend \skt{abhayaḥ} to \skt{ubhayaḥ}, thus matching the relevant line in the Kūrmapurāṇa cited above:                         \skt{kriyāyāś cābhavat putro daṇḍaḥ samaya eva ca} and allotting only two sons to Kriyā, but                         in a number of sources Kriyā actually has three sons, see e.g. Viṣṇupurāṇa 1.7.29(ab? CHECK in book),                         where they are named as Daṇḍa, Naya and Vinaya:                                 \skt{medhā śrutaṃ kriyā daṇḍaṃ nayaṃ vinayam eva ca}.                          Perhaps read \skt{kriyāyās tu nayaḥ putro} in pāda c? Compare Vāyupurāṇa 1.10.34cd                                         \skt{kriyāyās tu nayaḥ prokto daṇḍaḥ samaya eva ca}                                  with Brahmāṇḍapurāṇa 1.9.60ab:                                         \skt{kriyāyās tanayau proktau damaś ca śama eva ca} }





\slokawithfn{3.12}{ Lajjā's son is Vinaya (`Discipline'), Buddhi's son is Bodha (`Intelligence'). Lajjā has two [more] sons: Sudhiya[/Sudhī] (`Wise') and Apramāda (`Cautiousness'). [or one more son only: the wise Apramāda?]}
{ In a very similar passages in Kūrmapurāṇa 1.8.20 ff., Apramāda is Buddhi's son and                   Lajjā has only one son, Vinaya. In the above verse (VSS 3.12), \skt{sudhiyaḥ} (for \skt{sudhīḥ}) may only be                          qualifying \skt{apramāda}, thus Lajjā may have two sons: Vinaya and the wise Apramāda. }





\slokawithfn{3.13}{ Kṣema (`Peace') is to be known as Śānti's son, Vyavasāya (`Resolution') is Vapus' son. Yaśas (`Fame') is Kīrti's son, Sukha (`Joy') was born to Siddhi. [This is how] the sons of Dharma in the era of Svāyambhuva [Manu] were known.}
{ Note that \skt{sukhaṃ} in \skt{pāda} d is probably meant to be masculine (\skt{sukhaḥ}), but e.g. in the                  Kūrmapurāṇa passage quoted above it is also neuter. For the emendation in \skt{pāda} e,                  see Matsyapurāṇa 9.2cd:                                        \skt{yāmā nāma purā devā āsan svāyambhuvāntare}                 and Bhāgavatapurāṇa 6.4.1:                                             \skt{devāsuranṛṇāṃ sargo nāgānāṃ mṛgapakṣiṇām\danda                                       sāmāsikas tvayā prokto yas tu svāyambhuve 'ntare\twodanda}. }





\slokawithfn{3.14}{ Vigatarāga spoke: How does Dharma have two embodiments? Tell me, O great ascetic. I am extremely intrigued. Cut my doubts concerning [this] knowledge.}
{ Note \skt{dharma} as a neuter noun and the form \skt{atīvaṃ} for \skt{atīva} metri causa.          My emendation from \skt{kīrtaya} (`declare') to \skt{kartaya} (`cut') was influenced by the combination         of \skt{chindhi} and \skt{saṃśaya}, often with \skt{kautūhala}, elsewhere in the VSS:                 3.2b: \skt{saṃśayaṃ chindhi tattvataḥ};                  10.XXcd: \skt{kautūhalaṃ mahaj jātaṃ chindhi saṃśayakārakam};                 15.2ab: \skt{etat kautūhalaṃ chindhi saṃśayaṃ parameśvara}.                  The reading \skt{kīrtaya} may have been the result of the influence of \skt{kīrtitā} in 3.13b above                  (De Simini's observation). }





\slokawithfn{3.15}{ Anarthayajña spoke: Dharma's embodiment is said to consist of Śruti and Smṛti. The characteristics of the Śrauta [tradition] are an association with a wife [i.e.\ marriage] and with the fire ritual, and sacrifice. The Smārta [tradition] [focuses on] the conduct (\skt{ācāra}) of the classes (\skt{varṇa}) and life-stages (\skt{āśrama}) which is connected to rules and regulations (\skt{yama-niyama}).}
{ The reading \skt{°dvayī} in \msNc\ in \skt{pāda} a is attractive, but as Judit                          Törzsök has pointed out to me, it is probable that                         the slightly less convincing but widespread variant \skt{°dvayor} is original.  To state that the Smārta tradition is connected to \skt{yama}s and \skt{niyama}s and the \skt{āśrama}s and         then to discuss these at length (principally in chapters 3--8 and 11) can be seen                                  as a clear self-identification with the Smārta tradition. }




\begin{center}
{{[Yama and Niyama rules]}}
\end{center}




\slokawithfn{3.16}{ Now hear the classification of both the \skt{yama} and \skt{niyama} rules. Non-violence, truthfulness, not stealing, kindness, self-restraint, the rule of taboos,}
{ \skt{Pāda} a should be understood as \skt{yamaniyamayoś caiva}, but the author of this line                 may have tried to avoid the metrical fault of having two short syllables in the second and third positions.  Note that this is the beginning of a long section in our text         that describes the \skt{yama-niyama} rules, reaching up to the end of chapter eight.          The title given in the colophon of the next chapter, chapter four, namely \skt{yamavibhāga},         would fit this locus better than the beginning of that chapter, which          commences with a discussion on the second of the \skt{yama}s, \skt{satya}. }





\slokawithfn{3.17}{ virtue, carefulness, charm, honesty: these are the ten \skt{yama}s. The wise say that there are five subclasses to each.}
{ Note how all witnesses read \skt{mādhūrya} instead of \skt{mādhurya}. The former may have been                 acceptable originally in this text. }




\begin{center}
{{[The first yama-rule: Non-violence]}}
\end{center}




\slokawithoutfn{3.18}{ I shall teach you about non-violence and the other [\skt{yama}-rules]. Listen carefully, O Brahmin. Frightening and beating [other people], tying [someone] up, killing and the destruction of [other people's] livelihood: violence is said by the wise who see the truth to be of [these] five types.}





\slokawithfn{3.19}{ Cruel people beat [other people] with sticks, clods of earth [understand: they stone them], whips and other [objects] in the everyday world. Their bodies broken by the same blows, they receive the capital punishment.}
{ Note the use of the singular in \skt{pāda}s cd referring back to the agents of the previous sentence.                 Most probably, °\skt{vadhyam} is to be understand as °\skt{vadham} and the form                          \skt{vadhyam} serves only to avoid two \skt{laghu} syllables in \skt{pāda} d. }





\slokawithfn{3.20}{ [Others,] tie up [people] at their feet and their arms and chests. [These,] bound by [with?] their hair and [on their?] necks, die without being wounded. This is the capital punishment for tying up [other people].}
{ Understand \skt{bhujoraś ca} in \skt{pāda} a as \skt{bhuje urasi ca}, in this case with an instance of double sandhi:                 \skt{bhuje urasi ca} -- \skt{bhuja urasi ca} -- \skt{bhujorasi ca}. Alternatively, understand it as a compound:                                      \skt{bhujorasi}. }





\slokawithoutfn{3.21}{ He who frightens [other people] with the terrible danger of enemies and thieves, with lions, tigers, elephants or snakes, will be destroyed [by the above] or by other horrors.}





\slokawithfn{3.22}{ He who robs somebody's money is to be punished by the same person. He is [to be] hit by those whose livelihood got damaged by him as many times [as the victims are].}
{ Understand \skt{vadhaḥ} in \skt{pāda} b as \skt{vadhyaḥ} metri causa. }





\slokawithfn{3.23}{ [Those who kill other people] with poison, fire, arrows, swords, or by the force of magic or yoga are called murderers by the sages who see the truth, O great Brahmin[, and to be killed by the same methods].}
{ \skt{Pāda} a is unmetrical.                Note how elliptical this verse is and that \skt{hiṃsakāni} is neuter although it refers to                  people, perhaps implying \skt{bhūtāni}. Alternatively, take \skt{°ny°} in \skt{hiṃsakāny} as                  rather unusual sandhi-bridge (\skt{hiṃsakā-ny-āhu}).                  Note also that \skt{āhu} stands for \skt{āhur} metri causa. }





\slokawithfn{3.24}{ Non-violence is the highest Dharma. He who abandons it is a wicked person. It is free of pain and trouble, it yields the fruits of all [other] Dharmic teachings [in itself].}
{ Note \skt{dharma} as a neuter noun in \skt{pāda} a and that \skt{°vinirmuktaṃ} and                         \skt{°pradam} are neuter accordingly. }





\slokawithfn{3.25}{ There isn't a bigger fool than he [who abandons it is]. There is no bigger mental darkness [than the abandonment of non-violence]. There is no greater suffering or greater infamy.}
{ Note that \skt{parataro} is masculine in \skt{pāda} d, picking up a neuter \skt{'yaśaḥ}.         This phenomenon is probably the result of \skt{'yaśaḥ} resembling a masculine noun ending in \skt{-aḥ}                 and also of the metrical problem with the grammatically correct                         \skt{nātaḥ parataram ayaśaḥ}. }





\slokawithfn{3.26}{ There is no greater sin or a more effective poison. There is no greater ignorance, there is nothing worse, O great ascetic.}
{ \skt{Pāda} d (\skt{nātaḥ paraṃ tapodhana}) is slightly suspicious.                  The vocative \skt{tapodhana} usually refers to Anarthayajña in these                 passages, and not to Vigatarāga, as here.  The text may have read \skt{nātaḥ paratamo 'dhanaḥ}                  (`There is no bigger loss of wealth') or possibly something starting with                 \skt{nātaḥ paraṃ tapo ...} (`There is no greater austerity...'). }





\slokawithoutfn{3.27}{ He who does not harm the four types of living beings beginning with plants is the best person, having compassion for all creatures.}





\slokawithoutfn{3.28}{ He who always has compassion for all creatures is the [true] Pandit. He is the [true] sacrificer, the [true] ascetic, he is the donor, the one with a firm vow CHECK.}





\slokawithoutfn{3.29}{ Non-violence is the supreme sacred place. Non-violence is the highest austerity. Non-violence is the highest donation. Non-violence is the highest joy.}





\slokawithoutfn{3.30}{ Non-violence is the supreme sacrifice. Non-violence is the supreme religious observance. Non-violence is supreme knowledge. Non-violence is the supreme ritual.}





\slokawithoutfn{3.31}{ Non-violence is the highest purity. Non-violence is the highest self-restraint. Non-violence is the highest profit. Non-violence is the greatest fame.}





\slokawithoutfn{3.32}{ Non-violence is the supreme Dharma. Non-violence is the supreme path. Non-violence is the supreme Brahman. Non-violence is the supreme welfare (\skt{śiva}).}





\slokawithoutfn{3.33}{ One should refrain from meat-consumption. One should not even desire it mentally. He who abandons meat will receive a great reward.}





\slokawithfn{3.34}{ He who wishes to nourish his own flesh with the flesh of other [beings], outside of worshipping the ancestors and the gods, is the biggest sinner of all.}
{ See Uttarottara chapter two for a similar section on meat-consumption. }





\slokawithoutfn{3.35}{ During the \skt{madhuparka} offering and during a sacrifice, during rituals for the ancestors and the gods: only in these cases are animals to be slaughtered and not in any other case. [This is what] Manu taught.}





\slokawithoutfn{3.36}{ Should he buy it or procure it himself or should it be offered by others, if he eats meat, he will not sin if he first worships the gods and the ancestors.}





\slokawithfn{3.37}{ [People who know] the Vedas and [perform] sacrifices and austerities and [visit] sacred places, donate, [are of] good conduct, [perform] rituals and [keep] religious vows [but eat meat] will not [be able to] enjoy even a tiny portion of [such rewards that] [those] people [receive] who have given up meat.}
{ See a similarly phrased comparison in Manu 2.86:                 

                        \skt{ye pākayajñās catvāro vidhiyajñasamanvitāḥ \danda
                         sarve te japayajñasya kalāṃ nārhanti ṣoḍaśīm \twodanda} }





\slokawithoutfn{3.38}{ The deer and the goats, the sheep, the cows and other [animals] wander in the world happily and in great strength [just] from eating leaves and grass.}





\slokawithfn{3.39}{ Monkeys eat fruits, Rākṣasas prefer blood. The fruit-eating monkeys defeated all the Rākṣasas [as the Rāmāyaṇa tells us].}
{ Understand \skt{phalam āhārā} as \skt{phalāhārā} (-m- is a sandhi-bridge). }





\slokawithfn{3.40}{ Therefore one should not crave meat in the hope of gaining strength, O Brahmin,  in order to be able to draw a bow with force, or out of fear of the danger coming from the enemy.}
{ \skt{guṇākāśāt} in pāda c is difficult to interpret and                  \skt{guṇākarṣāt} is a conjecture by Judit Törzsök which fits the context well,                 although the polysemy of \skt{guṇa} may allow for other solutions.         

         Verses 3.40--42 may be echoing Brahmapurāṇa 216.64--66:         

                       \skt{ māṃsān miṣṭataraṃ nāsti bhakṣyabhojyādikeṣu ca \danda
                         tasmān māṃsaṃ na bhuñjīta nāsti miṣṭaiḥ sukhodayaḥ \twodanda 
                         gosahasraṃ tu yo dadyād yas tu māṃsaṃ na bhakṣayet \danda
                         samāv etau purā prāha brahmā vedavidāṃ varaḥ \twodanda
                         sarvatīrtheṣu yat puṇyaṃ sarvayajñeṣu yat phalam \danda
                         amāṃsabhakṣaṇe viprās tac ca tac ca ca tatsamam \twodanda }          }





\slokawithfn{3.41}{ One cannot be equal to someone who refrains from violence by [merely] wishing to make donations and perform sacrifices. [He will have] fame and glory in this world and the supreme path in the other.}
{ Pādas ab probably stand for \skt{ahiṃsako nāsti samo dānayajñasamīhaiḥ puruṣaiḥ} CHECK                         and are reminescent of Śivadharmaśāstra 11.92:

                               \skt{  ahiṃsaikā paro dharmaḥ śaktānāṃ parikīrtitam\danda
                                 aśaktānām ayaṃ dharmo dānayajñādipūrvakaḥ \twodanda }                 

                 Note the variant \skt{°dharma°} in both \msCc\ and \Ed\ in \skt{pāda} b. }





\slokawithfn{3.42}{ A person who refrains from violence will gain, no doubt about it, the [same] meritorious rewards that others would get by donating the three worlds filled with jewels and gems in their entirety to an excellent Brahmin, by performing a thousand [times] ten trillion (\skt{padma}) [times] ten thousand (\skt{ayuta}) \skt{koṭīyajña} (= koṭihoma?) sacrifices, by donating the earth [to a priest] as sacrificial fee, and by bathing [at] a thousand times ten million times a million (\skt{niyuta}) sacred places at once,}
{ On \skt{padma} meaning `ten trillion', and on other words for numbers, see 1.32--35.          

                 \skt{koṭīyajña} in pāda d may refer to a special kind of sacrifice,                  mostly known as \skt{koṭihoma} in the Purāṇas and in inscriptions                  (see e.g. Fleming 2010 and 2013)                 It probably involves a hundred fire-pits                  and a hundred times one thousand brāhmaṇas (hence the name `the ten-million sacrifice').                 See e.g. Bhaviṣyapurāṇa uttaraparvan 4.142.54--58:                         

                              \skt{  śatānano daśamukho dvimukhaikamukhas tathā \danda                                 caturvidho mahārāja koṭihomo vidhīyate \twodanda                                  kāryasya gurutāṃ jñātvā naiva kuryād aparvaṇi \danda                                 yathā saṃkṣepataḥ kāryaḥ koṭihomas tathā śṛṇu \twodanda                                 kṛtvā kuṇḍaśataṃ divyaṃ yathoktaṃ hastasaṃmitam \danda                                 ekaikasmiṃs tataḥ kuṇḍe śataṃ viprān niyojayet \twodanda                                 sadyaḥ pakṣe tu viprāṇāṃ sahasraṃ parikīrtitam \danda                                 ekasthānapraṇīte 'gnau sarvataḥ paribhāvite \twodanda                                  homaṃ kuryur dvijāḥ sarve kuṇḍe kuṇḍe yathoditam \danda                                 yathā kuṇḍabahutve 'pi rājasūye mahākratau \twodanda  }                         

                 Note that the second syllable of \skt{phalam} in \skt{pāda} d is treated as a long syllable: this                 happens often at word-boundaries in this text; and                  note how \msNc\ aims to restore the metre by inserting \skt{tv} after its \skt{phalaṃ}. }



\vfill\pagebreak

\thispagestyle{empty}\addcontentsline{toc}{section}{Chapter 4}
\begin{center}
{\large{Chapter Four}}
\end{center}




\begin{center}
{{[The second yama-rule: Truthfulness]}}
\end{center}




\slokawithfn{4.1}{ Anarthayajña spoke: The state of being real (\skt{sad-bhāva}) is called Truth (\skt{sat-ya}). Alternatively, it is also a notion that originates in perception. [Also, it is] relating things that correspond to reality. This is how Truth is discussed. REVISE}
{ Should we read \skt{satyalakṣaṇaṃ} in pāda d, following the rather similar                          Śivadharmaśāstra 11.105cd? }





\slokawithfn{4.2}{ He who endures severe abuse and beating etc. but keeps quiet, his self being conquered, is said to be [an example of] truth.}
{ \skt{suduḥsaham} (singular) in \skt{pāda} b picks up \skt{°ādīni} (plural) in \skt{pāda} a.         The \skt{-m} in \skt{satyam} may be a sandhi-bridge and the phrase may refer to a         masculine subject thus: \skt{sa ca satya -m- udāhṛtaḥ}. }





\slokawithoutfn{4.3}{ If one is being interrogated any time with a sword lifted to strike him down, then it is not the truth that is to be spoken. [In this case,] a lie is called truth.}





\slokawithoutfn{4.4}{ A person who is walking on the road and is afraid of being killed, should not reply [to people who are potentially dangerous] even if they ask him. That is also called Truth.}





\slokawithoutfn{4.5}{ A lie does not hurt when it is connected with joking, with women, O king[!], at the time of marriage, at the departure from life and when one's entire wealth is about to be taken away. They call these five kinds of lies Truth.}





\slokawithoutfn{4.6}{ Since Truth is the supreme Dharma with respect to gods, humans and animals[?], Truth is the best, the most preferable. Truth is the eternal Dharma.}





\slokawithfn{4.7}{ Truth is an unmanifest ocean. Truth yields imperishable pleasures. Truth is the ship that carries you to the other world. Truth is the wide path.}
{ \skt{Pāda} d is slightly problematic because it is difficult to ascertain if some of the                 MSS actually read \skt{panthāna} or \skt{pasthāna} (or \skt{yasthāna}).                 I suspect that \skt{panthāna} is a stem form noun formed (metri causa) to stand for an irregular nominative                 of \skt{pathin}. }





\slokawithoutfn{4.8}{ Truth is said to be the desired path. Truth is the supreme sacrifice. Truth is a pilgrimage place, a supreme pilgrimage place. Truth is an endless donation.}





\slokawithoutfn{4.9}{ Truth is morality, austerity, knowledge. Truth is purity, self-control and tranquillity. Truth is the ladder upwards. Truth is fame and glory and happiness.}





\slokawithoutfn{4.10}{ [When] a thousand Aśvamedha sacrifices and Truth are measured on a pair of scales, Truth indeed surpasses a thousand Aśvamedha sacrifices.}





\slokawithfn{4.11}{ The Sun shines because of Truth. The Earth stays in place by Truth. The winds blow because of Truth. Water is cooling through Truth.}
{ Here and several times below, \skt{satye} is probably to be taken as standing for \skt{satyena}. }





\slokawithfn{4.12}{ The oceans dwell in Truth because of their encounter[?] with Priyavrata [Manu's son]. Govinda abides in Truth because He [as Vāmana] stopped [Mahā]Bali [in spite of the fact that this was achieved by a trick].}
{ \skt{Pāda} b, \skt{samayena priyavrataḥ}, probably stand for \skt{samayena priyavratasya} although         it is unclear to me what exactly \skt{samaya} refers to here.         

         For Priyavrata's story, in which he wanted to turn nights into days by          circling aroung Mount Meru in a chariot, and by this produced the seven oceans,          see e.g. Bhāgavatapurāṇa 5.1.30--31:
          \skt{yāvad avabhāsayati suragirim anuparikrāman bhagavān ādityo          vasudhātalam ardhenaiva pratapaty ardhenāvacchādayati, tadā hi [priyavrataḥ]          bhagavadupāsanopacitātipuruṣaprabhāvas tad anabhinandan samajavena          rathena jyotirmayena rajanīm api dinaṃ kariṣyāmīti saptakṛtvas           taraṇim anuparyakrāmad dvitīya iva pataṅgaḥ\danda          ye vā u ha tadrathacaraṇanemikṛtaparikhātās te sapta sindhava āsan          yata eva kṛtāḥ sapta bhuvo dvīpāḥ\danda}            

Pādas cd: for a somewhat similar reference to the story of Mahābali, see e.g. Vāmanapurāṇa 65.66:         \skt{evaṃ purā cakradhareṇa viṣṇunā baddho balir vāmanarūpadhāriṇā \danda         śakrapriyārthaṃ surakāryasiddhaye hitāya viprarṣabhagodvijānām \twodanda}  }





\slokawithfn{4.13}{ Fire burns with Truth. The Moon rises by Truth. It is because of Truth that the Vindhya mountain stands in place and that although is was growing it is not growing [anymore].}
{ Since \skt{śaśi} (instead of \skt{śaśin}) is a possible stem in this text,                  \skt{śaśir ācaraḥ} could also be possible here in pāda b (see \msNa\msNb\msNc), perhaps standing for                  \skt{śaśinaś caraṇam} or \skt{śaśiś carati}. My emendation (\skt{śaśinācaraḥ})                  could stand for \skt{śaśinā/śaśinaś cāraḥ} metri causa.  

Pādas cd refer to the story of Agastya and the Vindhya mountain:         Vindhya became jealous of the Sun's revolving around Mount Meru and when the Sun          refused to him the same favour, he decided to grow higher and obstruct the Sun's movement.         As a solution to this situation,          Agastya asked Vidhya to bend down to make it easier for him to reach the south and         to remain thus until he retured. Vindhya agreed to do what Agastya asked him to do          but Agastya never returned. See Mahābhārata 3.102.1--14 (see in the word \skt{samaya} in verse 13                 and compare it to VSS 4.12b):                 

         \skt{                yudhiṣṭhira uvāca \danda
                          kimarthaṃ sahasā vindhyaḥ pravṛddhaḥ krodhamūrchitaḥ \danda
                          etad icchāmy ahaṃ śrotuṃ vistareṇa mahāmune \twodanda
                          lomaśa uvāca \danda
                          adrirājaṃ mahāśailaṃ meruṃ kanakaparvatam \danda
                          udayāstamaye bhānuḥ pradakṣiṇam avartata \twodanda
                          taṃ tu dṛṣṭvā tathā vindhyaḥ śailaḥ sūryam athābravīt \danda
                          yathā hi merur bhavatā nityaśaḥ parigamyate \twodanda
                          pradakṣiṇaṃ ca kriyate mām evaṃ kuru bhāskara \danda
                          evam uktas tataḥ sūryaḥ śailendraṃ pratyabhāṣata \twodanda
                          nāham ātmecchayā śaila karomy enaṃ pradakṣiṇam \danda
                          eṣa mārgaḥ pradiṣṭo me yenedaṃ nirmitaṃ jagat \twodanda
                          evam uktas tataḥ krodhāt pravṛddhaḥ sahasācalaḥ \danda
                          sūryācandramasor mārgaṃ roddhum icchan paraṃtapa \twodanda
                          tato devāḥ sahitāḥ sarva eva; sendrāḥ samāgamya mahādrirājam \danda
                          nivārayām āsur upāyatas taṃ; na ca sma teṣāṃ vacanaṃ cakāra \twodanda
                          athābhijagmur munim āśramasthaṃ; tapasvinaṃ dharmabhṛtāṃ variṣṭham \danda
                          agastyam atyadbhutavīryadīptaṃ; taṃ cārtham ūcuḥ sahitāḥ surās te \twodanda
                          devā ūcuḥ \danda
                          sūryācandramasor mārgaṃ nakṣatrāṇāṃ gatiṃ tathā \danda
                          śailarājo vṛṇoty eṣa vindhyaḥ krodhavaśānugaḥ \twodanda
                          taṃ nivārayituṃ śakto nānyaḥ kaś cid dvijottama \danda
                          ṛte tvāṃ hi mahābhāga tasmād enaṃ nivāraya \twodanda
                          lomaśa uvāca \danda
                          tac chrutvā vacanaṃ vipraḥ surāṇāṃ śailam abhyagāt \danda
                          so 'bhigamyābravīd vindhyaṃ sadāraḥ samupasthitaḥ \twodanda
                          mārgam icchāmy ahaṃ dattaṃ bhavatā parvatottama \danda
                          dakṣiṇām abhigantāsmi diśaṃ kāryeṇa kena cit \twodanda
                          yāvadāgamanaṃ mahyaṃ tāvat tvaṃ pratipālaya \danda
                          nivṛtte mayi śailendra tato vardhasva kāmataḥ \twodanda
                          evaṃ sa samayaṃ kṛtvā vindhyenāmitrakarśana \danda
                          adyāpi dakṣiṇād deśād vāruṇir na nivartate \twodanda
                          etat te sarvam ākhyātaṃ yathā vindhyo na vardhate \danda
                          agastyasya prabhāvena yan māṃ tvaṃ paripṛcchasi \twodanda }  }





\slokawithoutfn{4.14}{ The [mythical] Lokāloka mountains are located in Truth. Mount Meru stands by Truth. The Vedas abide in Truth. Dharma is rooted in Truth.}





\slokawithoutfn{4.15}{ The milk a cow yields is Truth. Ghee in milk is there as Truth. The soul dwells in the body in Truth. The eternal soul is Truth.}





\slokawithfn{4.16}{  If Truth alone (ekena) is obtained, Dharma is surely accomplished. By the heroism of Rāma Rāghava, Truthfulness was well-guarded, more than anything else.}
{ Another way to translate \skt{ekena} in pāda a would turn the sentence into this:                `If Truth is obtained by somebody, he will be one for whom Dharma is surely accomplished.' }





\slokawithoutfn{4.17}{ This is how [I] taught the rules of Truth to you, O virtuous one, to favour the whole world. What else do you wish to hear?}





\slokawithfn{4.18}{ Vigatarāga spoke: I can't have enough of learning about [this teaching of] your[s on] Dharma. Teach me further than this, O great ascetic.}
{ It is not inconceivable that \skt{tava} is meant to carry the sense of an ablative,                 as Kenji Takahashi has suggested to me:                 `I can't have enough of learning about Dharma from you.'  }




\begin{center}
{{[The third yama-rule: Refraining from stealing]}}
\end{center}




\slokawithoutfn{4.19}{ Anarthayajña spoke: Now listen to [my teaching about] stealing, O great Brahmin, which is taught to be of five kinds. Firstly, [listen to] theft [lit. `taking what has not been given'], then bribery, cheating with weights, cheating with scales, and the fifth kind, robbery.}





\slokawithoutfn{4.20}{ Theft is when somebody else's wealth is taken away through a bold/impudent crime. [A person who commits such a crime] is foolish even if he remains unnoticed [or: kept back from the crime?].}





\slokawithfn{4.21}{ O great Brahmin, listen to bribery, which defiles Dharma. A sum of money taken in order to annul a punishment [or something that is to be done, in order to become exempt from a duty] is a bribe. Therefore this [also] should be considered as such [i.e.\ as stealing because] it is committed out of greed.}
{ Note \skt{asau} in pāda c as an accusative form. }





\slokawithoutfn{4.22}{ [Even if] somebody wants to protect families by the method of cheating with weights, that person should be considered a thief, because he takes away other people's wealth.}





\slokawithfn{4.23}{ [The case is similar] if somebody takes away somebody else's belongings by the method of cheating with scales. Other people, deceitful swindlers (\skt{kūṭa-kāpaṭika}) [can also] have the characteristics of thieves.}
{ A line may have dropped out after pāda b, perhaps because a line                  similar to 4.22cd caused an eyeskip. Alternatively, this line may simply be                 elliptical. }





\slokawithoutfn{4.24}{ [If] someone, by deceit or by force, snatches away the wealth of weak and honest people or children [and women and simpletons?], that morally corrupt thief is [rightly] called a thief.}





\slokawithoutfn{4.25}{ There is no sin equal to stealing. There is no crime (\skt{adharma}) equal to it. There is no ill-fame comparable to that of being a thief. There is no bad-conduct comparable to being a thief.}





\slokawithoutfn{4.26}{ There is no such ignorance as stealing. There are no bigger rouges than thieves. There is nobody as ignorant as a thief. There is not a lazy person who is comparable to a thief.}





\slokawithfn{4.27}{ There is nobody as detestable as a thief. There is nobody as much of an enemy as a thief. There is no such suffering as stealing. There is nobody more disgraced than a thief.}
{ Note how \skt{stena} and \skt{steya} are used interchangeably (or chaotically)                         in the above passages in the MSS to denote both `thief' and 'theft/stealing'.                          The scribe of \msNc\ ends up writing \skt{stenya} in 4.27e. }





\slokawithfn{4.28}{ Some [thieves] take away [other people's] wealth in disguise, some in broad daylight. Other wicked people take money from deposits, and some people steal through fraud. Some gather wealth by forging documents, others steal from stolen money??? Some people's wealth is from a purchased [child?? (\skt{krīta})]. .............. These are considered the vilest.}
{ It appears that \skt{hriyate} in pāda a is to be taken as an active verb (\skt{harate}).                  Note also how \msCb\ and \msNc\ read the same here.  Take \skt{°hariṇo} in pāda b as singular and \skt{m} in \skt{'nya-m-adhamo} as a sandhi-bridge. }





\slokawithfn{4.29}{ There are no bigger idiots than thieves, who are wicked people without Dharma and Artha. As long as he lives, he trembles in fear of the king, wailing. Having received his punishment, he gets into severe and [in]tolerable difficulties, propelled by [his] karma. When his time comes, he dies and goes to hell, weeping vehemently.}
{ Understand \skt{stenastulya na mūḍham{ }asti} (the reading of \Ed!) as a `metri causa' version of                 \skt{stenatulyo na mūḍho 'sti}, and see a similar case of a nominative ending                 inside of compound in pāda c below. One major concern remains here:                 the accepted reading here is that of \Ed, an edition that rarely emerges as                  the sole transmitter of the best reading. A solution could be                  to emend to \skt{stenaṃtulya...}, meaning `There is no bigger foolishness than theft',                 but then the second part of pāda a is difficult to connect.                  

   Understand \skt{prāptaḥśāsana tīvrasahyaviṣamaṃ} in pāda c as \skt{prāptaśāsanas tīvram asahyaṃ ca viṣamaṃ prāpnoti}.                 Alternatively, understand \skt{tīvrasahya°} as \skt{duḥsahya°} (suggested by Törzsök).                 

                 The actual reading of \msCa, \skt{prāptaś} (lost in the process of normalization and standing                         in contrast with that of all other MSS that read \skt{prāptaḥ}) may suggest                         a doubling of the \skt{ś} of \skt{śāsana} metri causa (suggestion by Törzsök).                         More likely is that a licence of having a nominative ending inside of a compound                         is applied here, as probably above in pāda a (also remarked by Törzsök). }





\slokawithfn{4.30}{ Having spent ten million aeons of suffering, they emerge from hell to the state of animal existence. Similarly [CHECK eka], after roaming about in animal existence for a hundred and one times ten million years, then they reach the status of human existence on earth which is full of poverty and disease. Then abandoning all one's karmas, the causes of suffering, one seeks refuge in Śiva.}
{ Note °\skt{kalpa} for °\skt{kalpaṃ} metri causa.  I understand \skt{vipule} as \skt{vipulāyāṃ}, \skt{vipulā} appearing in Amarakośa 2.1.7 as a synonym of                         \skt{dhātrī}, `earth'.  Note the switch from plural to singular in pāda d. }




\begin{center}
{{[The fourth yama-rule: Absence of cruelty]}}
\end{center}




\slokawithfn{4.31}{ The one who is hostile towards the eight-formed Śiva, he who hurts his mother or father, he who is hostile towards cows or guests: these are the five types of cruel people.}
{ Note \skt{pitur} and \skt{mātur} used as accusative forms in \skt{pāda} b, or alternatively                         understand: `who are hateful towards their fathers and mothers'. }





\slokawithfn{4.32}{ Śiva in his manifest form (\skt{sākṣāt}) is of eight forms, with the five elements (vyoman! NOTE), and the Sun, the Moon, and the sacrificer. [He who] disgraces [any of these] is a cruel person.}
{ See Śakuntalā 1.1:         

                         \skt{yā sṛṣṭiḥ sraṣṭur ādyā [1] vahati vidhihutaṃ yā havir [2] yā ca hotrī [3]
                         ye dve kālaṃ vidhattaḥ [4,5] śruti-viṣaya-guṇā yā [6] sthitā vyāpya viśvam \danda
                         yām āhuḥ sarva-bīja-prakṛtir [7] iti yayā prāṇinaḥ prāṇavantaḥ [8]
                         pratyakṣābhiḥ prapannas tanubhir avatu vas tābhir aṣṭābhir īśaḥ \twodanda}
                         
                  The eight \skt{tanu}s here are: [1] jala [2] agni [3] yajamāna [4,5] sūrya + candra [6] ākāśa [7] bhūmi [8] vāyu                 

                 For a similar interpretation of \skt{aṣṭamūrti}, see e.g. Īśānaśivagurudevapaddhati 2.29.34 (\skt{mantrapāda};                                 note \skt{yajamāna} for our \skt{dīkṣa}):                                         \skt{kṣmā-vahni-yajamānārka-jala-vāyv-indu-puṣkaraiḥ}\danda                                         \skt{aṣṭābhir mūrtibhiḥ śambhor dvitīyāvaraṇaṃ smṛtam}\twodanda                 (For \skt{puṣkara} as `sky, atmosphere', see e.g. Amarakośa 1.2.167:                         \skt{dyodivau dve striyām abhraṃ vyoma puṣkaram ambaram}.)                 A closely related Aṣṭamūrti-hymn appears in Niśv mukha 1.30--41 (I owe thanks to Nirajan Kafle                         for drawing my attention to this); see Kafle 2018: 62, 63, 116, 119. Kafle notes                         that this hymn is closely parallel to some passages in the Prayogamañjarī (1.19--26),                         the Tantrasamuccaya (1.16--23), and the Īśānaśivagurudevapaddhati (kriyāpāda 26.56--63).                                         See also TAK I s.v. \skt{aṣṭamūrti}. }





\slokawithoutfn{4.33}{ The father is to be considered similar to the sky, he is the cause of one's birth. ....}





\slokawithoutfn{4.34}{ The mother is more venerable than the earth. Who would not praise a mother? By that [praise], sacrifices, donations, austerities and [the study of] the Vedas, all will be completed.}





\slokawithoutfn{4.35}{ Cows are a sacred [auspicious/purifying Judit] blessing, they are the gods of the gods. Cows contain in themselves all the gods. That's exactly why one should not hurt them.}





\slokawithoutfn{4.36}{ Cows are the protectors of the world as if the world were their new-born [calf], there is no doubt about it. The collection of [the five products of the cow, the \skt{pañcagavya},] ghee, milk, curd, and [the cow's] urine and dung [is auspicious].}





\slokawithoutfn{4.37}{ People who drink the five products of the cow, the five nectars, the five holy and pure [substances] [or: clarified with a strainer??], will obtain the fruits of a horse sacrifice, and then reach the undecaying heavens.}





\slokawithoutfn{4.38}{ There is no wealth comparable to [having] a cow. They yield milk, they draw [a plough etc.]. [As] they roam under the sky, feeding on grass, they issue nectar. When given to Brahmins, they deliver the family [from \skt{saṃsāra}/the suffering experienced in hell].}





\slokawithoutfn{4.39}{ He who never fails to serve the cow daily [e.g. with a handful of grass], and he who tends to the cows' service, will obtain the merits of all sacrifices, austerities and donation [because] he is one who is kind to it (\skt{tām}?) [i.e. to the cow].}





\slokawithfn{4.40}{ He who looks after a guest, he who respects a guest, he who worships a guest, he who praises a guest,}
{ Not the peculiar verb forms \skt{anugaccheta} and \skt{anupūjyeta}) in this verse. }





\slokawithoutfn{4.41}{ he who does not harm a guest, he who does not commit a fault towards a guest, he who does kind things to a guest, he who attends to the needs of a guest, he who makes a guest satisfied: his merits are endless.}





\slokawithfn{4.42}{ He should offer [the guest] a seat, water-offering, feet-washing water [or: °pātreṇa?], water for washing his feet[?], or gifts of food and clothes, or all [of these].}
{ Pāda b seems to awkwardly repeat what \skt{arghapādyena} in pāda a signifies.                 Some emendation may be required here, perhaps taking into account bathing (\skt{snāna}) or                          an unguent (\skt{abhyaṅga}). }





\slokawithfn{4.43}{ He who worships the guest by [offering him] his own son, wife or himself with willingness and with a brave and non-hesitating mind,}
{ For the requirement that one could part with his wife or son, or his own life,                 for the benefit of someone else, see VSS 2.38 and the narrative in VSS chapter 12;                 these influenced my decision to emend \skt{°ātmano} to \skt{°ātmanā} in pāda a. }





\slokawithoutfn{4.44}{ and does not ask [the guests about their] lineage, Vedic affiliation (\skt{caraṇa}), studies, country or birth, and imagines mentally, with devotion, that it is Dharma himself who has arrived,}





\slokawithoutfn{4.45}{ [will obtain all the fruits of] thousands of Aśvamedha sacrifices and hundreds of Rājasūya sacrifices, a thousand Puṇḍarīka sacrifices and the fruit of [visiting] all the pilgrimage places and [performing] all the austerities;}





\slokawithfn{4.46}{ he whose guest is satisfied [and] he who can abandon the sentiment of cruelty, will obtain all the merits of [the above], there is no doubt about it.}
{ The demonstrative pronoun \skt{tasya} in pāda c may refer to the guest:               `he will obtain all his [i.e. the guest's] merits', hinting at some sort of karmic exchange.                 Nevertheless, I think that \skt{tasya} points at the merits one can obtain by rituals listed                  in the previous verse. This is suggested by passages such as the following:                 

                 Mahābhārata Supp. 13.14.379 ff.:
                    <skt>ahany ahani yo dadyāt kapilāṃ dvādaśīḥ samāḥi\danda
                         māsi māsi ca satreṇa yo yajeta sadā naraḥ\twodanda 
                         gavāṃ śatasahasraṃ ca yo dadyāj jyeṣṭhapuṣkare\danda 
                         na taddharmaphalaṃ tulyam <b>atithir yasya tuṣyati</b>\twodanda</skt>                 

                  Brahmavaivartapurāṇa 3.44--46:
                    <skt>atithiḥ pūjito yena pūjitāḥ sarvadevatāḥ\danda
                         <b>atithir yasya santuṣṭas</b> tasya tuṣṭo hariḥ svayam\twodanda
                         snānena sarvatīrtheṣu sarvadānena yat phalam\danda
                           sarvavratopavāsena sarvayajñeṣu dīkṣayā\twodanda 
                         sarvais tapobhir vividhair nityair naimittikādibhiḥ\danda  
                         tad evātithisevāyāḥ kalāṃ nārhanti ṣoḍaśīm\twodanda</skt>           }





\slokawithoutfn{4.47}{ ... he who [does not] know [how to greet his] guests ... will never reach the path ... ? Therefore one should go up to the arriving guest with respectfully joined palms.}





\slokawithfn{4.48}{ By one \skt{prastha} of coarsely ground grains given to a guest, an extremely great sacrifice was performed [so to say], and his [the Brahmin's and his family members'] bodies (\skt{svaśarīraṃ}) reached heaven.}
{ This verse is a reference to the story related by a mongoose in MBh 14.92--93:                  A Brahmin who practises the vow of gleaning (\skt{uñcha}) and his family                 receive a guest. They feed the guest with the last morsels of the little food                 they have. In the end, the guest reveals that he is in fact Dharma (14.93.80cd) and as                  a reward the family departs to heaven. The noble act of the poor Brahmin and his family                 is depicted as yielding greater rewards than Yudhiṣṭhira's grandiose horse-sacrifice.                  (See some remarks on this story in Takahashi 2021.)  

We would be forced to accept the reading of \Ed\ in pāda d if the expression                 were in the masculine (\skt{saśarīro divaṃ gataḥ}). This would make sense                 and it would also echo expressions occuring e.g. in the Mahābhārata:                 3.164.33cd: <skt>paśya puṇyakṛtāṃ lokān saśarīro divaṃ vraja</skt>;                 14.5.10cd:  <skt>saṃjīvya kālam iṣṭaṃ ca saśarīro divaṃ gataḥ</skt>.                 It is tempting to emend the pāda accordingly, but I have retained                          \skt{svaśarīraṃ divaṃ gatam} and I interpret it as                          referring to the Brahmin's whole family (\skt{sva}).                  }





\slokawithoutfn{4.49}{ The mongoose related [this story in the Mahābhārata] in the past in detail, O great Brahmin, and you've known it already. The story of the \skt{prastha} is well-known.}




\begin{center}
{{[The fifth yama-rule: Self-restraint]}}
\end{center}




\slokawithoutfn{4.50}{ Self-restraint of humans is in itself the collected essence of Dharma. Self-restraint is Dharma, Self-restraint is heaven, Self-restraint is fame, Self-restraint is happiness.}





\slokawithoutfn{4.51}{  Self-restraint is sacrifice, Self-restraint is a pilgrimage-place, Self-restraint is merit, Self-restraint is religious austerity. If one has no Self-restraint, there is no Dharma, [while] Self-restraint yields a multitude of desired objects.}





\slokawithfn{4.52}{ The elephant, the fish, the moth, the bee and the deer are without Self-restraint. The senses are the skin, the tongue, the nose, the eye and the ear.}
{ Note \skt{kari} for \skt{karī} metri causa, and the end of pāda b (\skt{°mṛgāḥ}), which                          should be treated metrically as if it read \skt{°mrigāḥ}. }





\slokawithoutfn{4.53}{ Each of these sense faculties are hard to conquer and all are known to be fatal [if unconquered]. If one masters Self-restraint, the [one with a?] lack of Self-restraint will die.????}





\slokawithoutfn{4.54}{ In the case of the deer, death comes about because of hearing [when hunters use buck grunts]. Moths die because[?] of their eyes [as they are attracted to the light of a lamp]. Bees perish because of their smelling, fish because of their tongues.}





\slokawithoutfn{4.55}{ The elephant perishes because of touch, not being able to tolerate being in fetters [?]. How much more true it is for those who enjoy all five [senses]! Why should death come as a surprise for them?}





\slokawithfn{4.56}{ Purūravas [perished] by excessive greed, Daṇḍaka by excessive desire, Sagara's sons by excessive pride, Rāvaṇa by excessive haughtiness,}
{ Purūravas (double sandhi originally? purūravās ati° -- purūravā ati° -- purūravāti°).         Pāda a may refer to the following passage in the Mahābhārata (1.70.16--18, 20ab):         
                                  <skt>purūravās tato vidvān ilāyāṃ samapadyata\danda
                                  sā vai tasyābhavan mātā pitā ceti hi naḥ śrutam\twodanda
                                  trayodaśa samudrasya dvīpān aśnan purūravāḥ\danda
                                  amānuṣair vṛtaḥ sattvair mānuṣaḥ san mahāyaśāḥ\twodanda
                                  vipraiḥ sa vigrahaṃ cakre vīryonmattaḥ purūravāḥ\danda
                                  jahāra ca sa viprāṇāṃ ratnāny utkrośatām api\twodanda
                                         ... 
                                  tato maharṣibhiḥ kruddhaiḥ śaptaḥ sadyo vyanaśyata\danda</skt>         

                         (``The wise Purūravas was born to Ilā. We heard that Ilā                                  was both his mother and his father.                             The great Purūravas ruled over thirteen islands of the ocean                            and, though human, he was always surrounded by superhuman beings.                            Intoxicated with his power, Purūravas quarrelled with some Brahmins                              and robbed them of their wealth even though they were protesting. [...]                            Therefore, cursed be the great Ṛṣis, he perished.'')         

         See also Buddhacarita 11.15 (Aiḍa = Purūravas):

                 <skt> aiḍaś ca rājā tridivaṃ vigāhya
                          nītvāpi devīṃ vaśam urvaśīṃ tām\danda
                       lobhād ṛṣibhyaḥ kanakaṃ jihīrṣur  
                         jagāma nāśaṃ viṣayeṣv atṛptaḥ\twodanda</skt>         

                 For Daṇḍa(ka)'s story, see Rāmāyaṇa 7.71.31 ff.:                 Daṇḍa meets Arajā, a beautiful girl, in a forest and rapes her. As a consequence, her father, Śukra/Bhārgava,                 destroyes Daṇḍa's kingdom, which thus becomes the desolate Daṇḍaka-forest.                     

For two versions of the destruction of                 Sagara's sons, who were chasing the sacrificial horse of their father's Aśvamedha sacrifice,                 and by doing so disturbed Kapila's meditation, and who in turn burnt them to ashes,                 see Mahābhārata 3.105.9 ff. and Brahmāṇḍapurāṇa 2.52--53.                 

                 As for Rāvaṇa's haughtiness,                 especially the fact that he chose to be invincible by all creatures except humans,                 and its consequences,                 one should recall the story of the Rāmāyaṇa and Rāvaṇa's destruction brought about by Rāma therein. }





\slokawithfn{4.57}{ Saudāsa by excessive anger, the Yādavas by excessive drinking, Māndhātṛ by excessive desire, Nahuṣa by contempt for Brahmins,}
{ Saudāsa, also known as Kalmāṣapāda, hit Śakti, Vasiṣṭha's son, with a whip because                 the latter did not give way to him, and as a consequence Śakti cursed Saudāsa:                 Saudāsa had to roam the world as a Rākṣasa for twelve years.                  See Mahābhārata 1.166.1 ff.                 

                 As for the end of the Yādavas, see the short Mausalaparvan of the Mahābhārata (canto 16):                 cursed by the sages Viśvāmitra, Kaṇva and Nārada, and seeing menacing omens,                 the Yādavas take to drinking in Prabhāsa and destroy each other.           

Most probably, \skt{atitṛṣṇā} in the MSS stand for \skt{atitṛṣṇāt} (intending \skt{atitṛṣṇayā}).                 The form \skt{māndhāto} in \msCb\ stands for \skt{māndhātā} (nominative of \skt{māndhātṛ}).                 I have corrected it in spite of the fact that the authors' knowledge about his story may                 come from Divyāvadāna 17, where it sometimes appears to be an a-stem noun (\skt{māndāta}).                 \skt{dvijavajñayā} in \skt{pāda} d stands for \skt{dvijāvajñayā} metri causa.                 

                 Māndhātṛ was born from his father's body who, being excessively thirsty once,                 had drank some decoction prepared for ritual purposes and as a result become pregnant with him.                 Nevertheless, Buddhacarita 11.13 suggests that Māndhātṛ himself was still unsatisfied                 with wordly objects even after he had obtained half of Indra's throne:                 

                 <skt>                devena vṛṣṭe 'pi hiraṇyavarṣe  
                                         dvīpān samagrāṃś caturo 'pi jitvā\danda  
                                      śakrasya cārdhāsanam apy avāpya   
                                         māndhātur āsīd viṣayeṣv atṛptiḥ\twodanda</skt> 

                 In fact, as Monika Zin points out (2012: 149) Māndhātṛ/Māndhāta's rise and fall is a very popular theme                 in the `Narrative Art of the Amaravati School':                  ``Statistics show that in the Amaravati School the most frequently represented narrative is                   the story of King Māndhātar, which appears 47 times.''         See ibid. p. 151:         ''The story [e.g. <i>Divyāvadāna</i> XVII, see more sources in fn. 17 of this article]          relates that Māndhātar was a miraculously born <i>cakravartin</i> with Seven         Jewels who could cause rain to fall so that his subjects could prosper; not usual rain, but rain         of coins, of grain or of cloth. By virtue of his moral strength alone, Māndhātar conquered the world -         without any weapons. He conquered all the countries on earth, then Uttarakuru,         Pūrvavideha and Aparagodānīya, after which he set out to conquer the heavens. When he was         traversing from one abode of the gods to the next (Nāgas, Sadāmattas, Mālādharas, etc.)         groups of gods pledged obeisance to him and immediately marched in front of his troops.         Māndhātar reached the splendid city of the Trayastriṃśa gods atop Sumeru, where Indra, in         the meeting-hall, bequeathed to him half of his own seat and half of his heavenly realm.         Māndhātar then ruled together with Indra for an unimaginable period of time during which 36         Indras changed. One day, shortly after he won a battle against the Asuras, a sinful thought         came to his mind: why should he rule alongside Indra? It was he, after all, who won the war,         not Indra - he was better and should, therefore, rule alone.         At that very moment Māndhatar fell from heaven, down to his former realm, became sick and died.         Shortly before his death, he preached a sermon to his subjects in which <i>gātha</i>s          from the <i>Dhammapada</i> (186--187) appear...''                   

                 Nahuṣa was elevated to the position of Indra for a period of time and he also wanted                 to take Śacī, Indra's wife. Indra instructed Śacī to tell Nahuṣa to                  harness some Ṛsis to a vehicle and use this vehicle to take Śacī.                  Agastya, one of the Ṛṣis, was insulted even further by Nahuṣa, therefore                 he cursed Nahuṣa, who then fell from the vehicle. See Mahābhārata 12.329.35 ff. and                 the verse in the Buddhacarita (11.14) that follows the one about Māndhātṛ:         

                                 <skt>        bhuktvāpi rājyaṃ divi devatānāṃ   
                                                 śatakratau vṛtrabhayāt pranaṣṭe\danda
                                              darpān maharṣīn api vāhayitvā  
                                                 kāmeṣv atṛpto nahuṣaḥ papāta\twodanda</skt>                  }





\slokawithfn{4.58}{ [Mahā]bali perished by excessive donations, Arjuna by excessive heroism, King Nala by excessive gambling, Nṛga by taking a cow.}
{ Pāda a is most probably a reference to Mahābali's promises made to Vāmana that caused his fall.                  Arjuna: the exile? Flo Kirātārjunīya?? he killed Bhīṣma? Flo    

 King Nala was an expert in the game of dice and lost his kingdom to Puṣkara in                              a game. See e.g. Mahābhārata 3.56.1 ff.                  

                        As for Nṛga, see Mahābhārata 14.93.74: 
                        <skt>                        gopradānasahasrāṇi dvijebhyo 'dān nṛgo nṛpaḥ\danda
                        ekāṃ dattvā sa pārakyāṃ narakaṃ samavāptavān\twodanda
                                                 </skt>                                (``King Nṛga had made gifts of thousands of cows for the twice-born.                                  By giving away one single cow that belonged to someone else,                                   he fell into hell.'')                  }





\slokawithfn{4.59}{ [For] a person who is without Self-restraint, O great Brahmin, there is no heaven, liberation or happiness. O Brahmin, people without Self-restraint are the destruction of knowledge, Dharma, family and fame.}
{ Note how flexible the gender of most nouns is in pāda b:                          \skt{svarga}, \skt{mokṣa} and \skt{dama} are usually masculine in standard Sanskrit.  The majority of the witnesses suggest that pāda c ends in a stem form noun (\skt{°nāśa}).                 This pāda is unmetrical, or rather it applies the licence of a word-final                 short syllable being counted as potentially long (\skt{°dharMA°}).   Note how \skt{viprā} in pāda d is probably an attempt in some MSS to restore the metre.                 This pāda is also unmetrical, or rather it applies the licence of a word-final                 short syllable being counted as potentially long (\skt{viPRA}). }




\begin{center}
{{[The sixth yama-rule: Taboos]}}
\end{center}




\slokawithoutfn{4.60}{ [For] a person without taboos there is neither the other world, nor this life. In the case of a person without taboos there is no Dharma or religious austerity.}





\slokawithoutfn{4.61}{ These five are taboo: women who are not depending on oneself, others' wealth, taking away others' lives, hurting others and [consuming] others' food.}





\slokawithoutfn{4.62}{ Listen, O great Brahmin, the wise should always treat women who are not dependent on oneself as taboo, [be she] a queen, a Brahmin's wife, a wandering religious mendicant, a relative or of another family.}





\slokawithoutfn{4.63}{ Listen further to something else with regards to others' wealth. [It may include] gaining wealth through unlawful means, when somebody takes away other people's wealth by cheating with [small] weights of an \skt{āḍha[ka]} or a \skt{prastha} and with scales.}





\slokawithfn{4.64}{ O Brahmin, the wise should regard the taking away [of others'] lives as taboo. Wild and domesticated animals, [serpents] that live in holes and those that walk on their feet [are examples of life forms not to destroy].}
{ In pāda d, understand \skt{caraṇācara} as \skt{caraṇacara} (metri causa). }





\slokawithfn{4.65}{ And what is the hurting of others? Listen, O Brahmin, I'll tell you briefly. He who is hostile to the gods, Brahmins, gurus, mothers and guests [hurts others].}
{ Note \skt{mātā} as a stem form. }





\slokawithfn{4.66}{ As regards other people's food, eating together with people whose food is not to be accepted (\skt{abhojyeṣu}) is taboo, [e.g.] after birth or death [in the family], in case there are vendors of alcohol, in the case of a family having lost their caste, and in the case of a Naṭa [dancer caste?].}
{ One should probably understand \skt{śauṇḍe} in pāda c as \skt{śauṇḍike} (alternatively,                 it may be corrupted from \skt{ṣaṇḍhe}); see both in Vāsiṣṭhadharmaśāstra 14.1--3:                 

                 <skt>athāto bhojyābhojyaṃ ca varṇayiṣyāmaḥ\danda                 cikitsaka-mṛgayu-puṃścalī-ḍaṇḍika-stenābhiśastar-ṣaṇḍha-patitānām annam abhojyam\danda                 kadarya-dīkṣita-baddhātura-somavikrayi-takṣa-rajaka-śauṇḍika-sūcaka-vārdhuṣika-carmāvakṛntānām\twodanda</skt> etc.                 

                 In Olivelle's translation (DhSūtras 1999: 285):                         ``Next we will describe food that is fit and food that is                         unfit to be eaten [...] The following are unfit                         to be eaten: food given by a physician, a hunter, a harlot, a law                         enforcement agent, a thief, a heinous sinner [...] a                         eunuch, or an outcaste; as also that given by a miser, a man                         consecrated for a sacrifice, a prisoner, a sick person, a man who                         sells Soma, a carpenter, a washerman, a liquor dealer, a spy, an                         usurer, a leather worker...''                 

                 In support of reading \skt{ṣaṇḍhe}, see Manu 3.239:                 

                         <skt>cāṇḍālaś ca varāhaś ca kukkuṭaḥ śvā tathaiva ca\danda
                         rajasvalā ca ṣaṇḍhaś ca nekṣerann aśnato dvijān\twodanda</skt> }





\slokawithfn{4.67}{ Those people who cling to [the prohibition of] the five kinds of taboo [and thus] seek heaven, wealth and liberation, will reach eternal faultlessness in this world, embellished with fame and glory. [A person like that] will obtain wisdom, intelligence, [knowledge of] the Śruti and Smṛti traditions, and honour forever. He will be kindness itself[?] and he will obtain an extra long life, no doubt.}
{ Understand \skt{kīrtir yaśo°} as \skt{kīrtiyaśo°} ('r' being an intrusive consonant here metri causa).  Understand \skt{āyuṣa} as \skt{āyuṣaṃ} (metri causa). }




\begin{center}
{{[The seventh yama-rule: The five methods of virtue?]}}
\end{center}




\slokawithoutfn{4.68}{ The four cases of observing silence, [victory over] the four enemies, the four sanctuaries/planes, the four meditations, and the four legged [Dharma] are called the five ways of being virtuous[?].}





\slokawithfn{4.69}{ I shall tell you about the four cases of observing silence. Listen, be attentive. One should avoid [1] violent [words], [2] slanderous [words], [3] lies, and [4] idle [talk].}
{ Is \skt{sambhinna} a Buddhist term? See also Dharmaputrikā 1.31. }





\slokawithfn{4.70}{ The fourfold enemy, desire, anger, greed and delusion, is to be destroyed. He who destroys [these] enemies will become sinless.}
{ Possible direct sources for the idea that \skt{kāma} is an enemy to be defeated include                 Buddhacarita 11.17: 

         <skt>cīrāmbarā mūlaphalāmbubhakṣā
               jaṭā vahanto 'pi bhujaṃgadīrghāḥ\danda
               yair nānyakāryā munayo 'pi bhagnāḥ
               kaḥ kāmasaṃjñān mṛgayeta śatrūn\twodanda</skt> 

          and Bhagavadītā 3.43:         

         <skt>evaṃ buddheḥ paraṃ buddhvā saṃstabhyātmānam ātmanā\danda
              jahi śatruṃ mahābāho kāmarūpaṃ durāsadam\twodanda</skt>  }





\slokawithfn{4.71}{ I shall teach you the four sanctuaries/planes. Listen, O Brahmin. Compassion, sympathy in joy, indifference, and benevolence are the four sanctuaries/planes.}
{ Is \skt{āyatana} just a synonym of \skt{vihāra} here or                  could this use of the term \skt{āyatana} for the four Buddhist                  \skt{brahmavihāra}s have been influenced by the following passage in the Dharmasamuccaya (date?)?         

                         <skt>mokṣasyāyatanāni ṣaṭ\danda
                         apramādas tathā śraddhā vīryārambhas tathā dhṛtiḥ\danda
                         jñānābhyāsaḥ saṃtāśleṣo mokṣasyāyatanāni ṣaṭ\twodanda1.3\twodanda
                         nava śāntisamprāptihetavaḥ\danda
                         dānaṃ śīlaṃ damaḥ kṣāntir maitrī bhūteṣv ahiṃsatā\danda
                         karuṇāmuditopekṣā śāntisamprāptihetavaḥ\twodanda1.4\twodanda
</skt>  }





\slokawithfn{4.72}{ I shall now teach you the four meditations, which will liberate you from mundane existence (\skt{saṃsāra}). Meditation is taught to be fourfold: of the Self, \skt{vidyā}, \skt{bhava} [= Śiva] and the subtle one.}
{ Note the stem form \skt{dhyāna} in \skt{°dhyānādhunā} (for \skt{°dhyānam adhunā}) in pāda a.  For contrast, see VSS 6.8:                 

                 <skt>dhyānaṃ pañcavidhaṃ caiva kīrtitaṃ hariṇā purā\danda
                      sūryaḥ somo 'gni sphaṭikaḥ sūkṣmaṃ tattvaṃ ca pañcamam\twodanda</skt> }





\slokawithfn{4.73}{ The \skt{tattva} of the Self is the \skt{ātman}. \skt{Vidyā} in the five in a fivefold way[??]. They call the thirty-sixth the imperishable one, [and] the subtle \skt{tattva} has no attributes.}
{ If pāda c is indeed a reference to a 36-tattva philosophical system,                 it is in striking contrast with the 25-tattva system described in VSS chapter 20. }





\slokawithoutfn{4.74}{ Dharma is said to be four-legged [as] it rests on the four \skt{āśrama}s, [those of] the householder, the chaste one, the forest-dweller and the mendicant.}





\slokawithfn{4.75}{ Virtuous are those who know these thoroughly, O great Brahmin. [They will experience] the purification of all sins and the growth of merits.}
{ Note the plural instrumental (\skt{yair}) with a singular active verb (\skt{vetti}). }





\slokawithoutfn{4.76}{ One's life-span, fame and glory and happiness grow only through virtue (\skt{dhanya}). [In] a virtuous person piece, prosperity, memory/tradition? and intelligence will arise.}




\begin{center}
{{[The eighth yama-rule: Lack of Negligence]}}
\end{center}




\slokawithfn{4.77}{ There are five areas of negligence. I shall teach them to you, listen. Murdering a Brahmin, drinking alcohol, stealing, having sex with the guru's wife: they call these Grievous Sins. The fifth is when one is connected with them [i.e. with these sins or with people involved in these sinful acts].}
{ Note the stem form noun in pāda a (\skt{°sthāna}) metri causa, and also                          that this stem form noun may function as a singular noun                         next to a number (\skt{pañca}), a frequently seen phenomenon in this text.  Note how \skt{pāda} f deviates from Manu. }





\slokawithfn{4.78}{ A lie concerning one's superiority, a slander that reaches the king's ear, and false accusations against an elder are equal to killing a Brahmin.}
{ The translation of this verse is based on Olivelle's (Olivelle Crit Ed. p. 218). }





\slokawithoutfn{4.79}{ Defaming a Brahmin or the Ṛgveda, being a false witness, murdering a friend, eating unfit or forbidden food are six [deeds that are] equal to drinking alcohol.}





\slokawithoutfn{4.80}{ Sexual intercourse with a female relative, with an unmarried girl, with women of the lowest castes, with the wife of a friend or of one's own son are said to be equal to violating the guru's bed.}





\slokawithoutfn{4.81}{ Stealing/taking away deposits, people, horses, silver, land, diamonds, or gems are said to be equal to stealing gold.}





\slokawithfn{4.82}{ If a man takes parts in these four [i.e. \skt{brahmahatyā, surāpāna, stena, gurvaṅganāgama}], that is the fifth Grievous Sin. By this all [of them] have been explained. These five kinds of negligence are to be avoided, O great Brahmin.}
{ Note syntax. }




\begin{center}
{{[The ninth yama-rule: Charm]}}
\end{center}




\slokawithfn{4.83}{ [Charm has five types:] bodily, verbal and mental charm, [charm of] the eyes and [of one's] thoughts pañcamaḥ. Giving [others] a friendly glance [is commendable] and one should avoid cruel thoughts.}
{ My emendation from \skt{°manasā dhūryaś} to \skt{°mana-mādhuryaś} is based on the fact that following the list                 of \skt{yama}s in 3.16cd--17ab, we need some reference to \skt{mādhurya} here and that it is easy to see how this                  corruption came about: \skt{°mano-mādhurya°} would be unmetrical, thus the form \skt{°mana-mādhurya};                         \skt{°mana-mā°} is easily corrupted to \skt{°manasā°} (not to mention the fact                          that \skt{manasā} comes up in the next verse);                          in addition we need five items in this line because of \skt{pañcamaḥ}.                         As always, I correct \skt{mādhūrya} to \skt{mādhurya}, although it seems that                          the former is acceptable in this text.                          I did not correct \skt{mādhuryaś} to \skt{mādhuryaṃ} because of the corresponding                         \skt{pañcamaḥ}. }





\slokawithoutfn{4.84}{ One should meditate with a tranquil mind and should speak [to other people using] gentle words. [When] respectable people arrive at one's own hermitage, [one should] present them with as many gifts as one can,}





\slokawithfn{4.85}{ with gifts of fire-wood, water and fire. [If] fire-wood, fire and water are easily available [but] are not given [as gift] or [if the phrase] `Live [for a hundred years]!' is not uttered [by him] when [somebody else] sneezes, what reward could there be for him in the afterlife?}
{ Understand \skt{jātavedam} in pāda b as \skt{jātavedasam} or \skt{jātavedāḥ},                 or rather as belonging to the compound \skt{°dānaṃ}: \skt{jātavedodānaṃ}.  For pāda e, see Mahāsubhāṣitasaṃgraha 2558:                  <skt>amṛtāyatām iti vadet pīte bhukte kṣute ca śataṃ jīva</skt>                 (`When eating or drinking, one should say: "Let it turn into nectar!";                   and after sneezing: "Live for a hundred years!".')  }




\begin{center}
{{[The tenth yama-rule: Sincerity]}}
\end{center}




\slokawithoutfn{4.86}{ The sages who see the truth praise five types of sincerity. [Sincerity] in action, in livelihood, in prosperity, in gratifying others [and ...?]. A sincere person does not rejoice in women, wealth, bribery and property.}





\slokawithoutfn{4.87}{ Sincerity [means] no sacrifice [performed] idly. Sincerity [means] no austerity [performed] idly. Sincerity [means] no donation [given] idly. Sincerity [means] no fires [kindled] idly.}





\slokawithoutfn{4.88}{ The sense faculties of a sincere person are firm even when he is delighted. The gods always live inside the body of a sincere person.}





\slokawithfn{4.89}{ Thus has been taught this section on the \skt{yama}-rules, O great Brahmin. Humans should follow them to reach happiness here and in the other world. He'll live by Śaṅkara's command with his filth of sins destroyed. He'll become a ruler of the world [that he subjugates] under one royal umbrella.}
{ In pāda a \skt{°pra°} does not make the previous syllable long: this is the phenomenon of                 `muta cum liquida', one of the hallmarks of the \skt{Vṛṣasārasaṃgraha},                  that is, syllables such as \skt{tra, pra, bra, dra} do not necessarily make the                  previous syllable long.  In pāda b, \skt{parata} most probably stands for \skt{paratra} or \skt{parataḥ} metri causa.          We may correct it to \skt{paratra} (`muta cum liquida').  \skt{°malapahārī} in the MSS stands either for \skt{°malāpahārī} or \skt{°malaprahārī} metri causa.                  I could have choosen to emend it to \skt{°malaprahārī} (`muta cum liquida' again),                 but I decided not to because \skt{apahārin}, \skt{apahāra}                 \skt{apahāraka} are used in the text very frequently. See also 8.XX, which contains a very similar expression:                         \skt{sakalamalapahāre dharmapañcāśad etat}. }



\vfill\pagebreak

\thispagestyle{empty}\addcontentsline{toc}{section}{Chapter 5}
\begin{center}
{\large{Chapter Five}}
\end{center}




\begin{center}
{{[The niyama-rules]}}
\end{center}




\slokawithoutfn{5.1}{ Vigatarāga spoke: [Please] now teach me the true nature of the Niyama-rules in detail. It is comparable to a speech of ambrosia. I have become curious to hear [it]. [It was?] burnt by the fire of Prakṛti, sprinkled with the water of knowledge[?!]. There is no satisfaction [yet] in the Dharmas [for me]. ...[perhaph \skt{apara-vadam ataj-jñā... or apara[ṃ] vada me tajjña? mata-jña?}].}





\slokawithfn{5.2}{ Anarthayajña spoke: I shall teach you something else that is nice to hear, O best of the twice-born: the particular part[s, for kalā; or for kalpa?] of Niyama are of five types [each]. It is the essence of Dharma, dear to Hari, Hara and the sages, O great Brahmin, the destruction of the impurity of the Kali age, generally[?] known as liberation.}
{ In \skt{pāda} a, \skt{anyat} is a bit strange, but it could be echoing \skt{apara} above in 5.1d. }





\slokawithoutfn{5.3}{ Purification, sacrifice, penance, donation, Vedic study and the restraint of sexual desire, religious observances, fasting, taciturnity, and bathing: these are the ten Niyamas.}




\begin{center}
{{[The first niyama-rule: Purity]}}
\end{center}




\slokawithoutfn{5.4}{ From among these, now I shall tell you the particulars of purification [first], and [then] the others. [1] Bodily purity, [2] [purity of] food, [3] [purity of] property[?], [4] [purity of] conduct[?], and the fifth, [5]...?}




\begin{center}{{[Purity of the Body]}}\end{center}




\slokawithoutfn{5.5}{ He should not beat or tie or kill [any living being]. When this concerns others' wives and property, it is called bodily purity.}





\slokawithoutfn{5.6}{ The cleanliness of the ears, O great Brahmin, and of the anus, the loins, the mouth etc. [is also bodily purity]. The purity of the mouth [comes from] sipping water when eating, speaking,}





\slokawithfn{5.7}{ [after] the emission of urine and faeces, and [before] the worship of gods. The wise one should clean his anus and his loins with clay and water.}
{ Note [or emend?] the form \skt{śaucayīta}.  }





\slokawithoutfn{5.8}{ One [portion of clay] for the loins, five for the anus, and ten for one [the left] hand. [Then] seven is to be applied for both [hands] by him who wishes cleanliness with clay.}





\slokawithoutfn{5.9}{ This is the purification for the householder (\skt{gṛhastha}), twice as much for the chaste one (\skt{brahmacārin}), three times as much for the forest-dweller (\skt{vānaprastha}), four times as much for the ascetic (\skt{yati}).}




\begin{center}{{[Purity of the food]}}\end{center}




\slokawithfn{5.10}{ I shall teach you the rules of purity with food. Listen, pay great attention. He should eat [as much] food [that fills] two quarters [of the stomach] and drink water [that fills] one quarter. In order to be able to practise breath-control, he should save the remaining quarter.}
{ For similar instructions, see a verse cited in Śaṅkara's commentary ad BhG 6.16:                                 \skt{uktaṃ hi\danda                                  ardhaṃ savyañjanānnasya tṛtīyam{ }udakasya ca\danda                                  vāyoḥ saṃcaraṇārthaṃ tu caturtham{ }avaśeṣayet\twodanda}                                (``Half is for food with sauce, the third part for water,                                but in order to be able to move the air, he should leave the fourth part [empty].'')                 See also e.g. Aṣṭāṅgahṛdaya 8.46cd-47ab:                                         \skt{annena kukṣer dvāv aṃśau pānenaikaṃ prapūrayet\twodanda                                          āśrayaṃ pavanādīnāṃ caturtham avaśeṣayet\danda}                 and Sannyāsopaniṣad 59:                                          \skt{āhārasya ca bhāgau dvau tṛtīyam udakasya ca\danda                                          vāyoḥ saṃcaraṇārthāya caturtham avaśeṣayet\twodanda}  }





\slokawithoutfn{5.11}{ [By] the wise one['s applying] the six soft and sweet juices, [which are] the six juices in food, the disturbances of the \skt{dhātu}s and the terrible illnesses will disappear.}





\slokawithoutfn{5.12}{ He should not eat foods that are forbidden and he should not drink drinks that are forbidden. He should not go where he is not allowed to and he should not say what is improper.}





\slokawithoutfn{5.13}{ He should avoid garlic, onion, \skt{gṛñjana} onion, mushrooms, buffalo meat? and pork, following the rules.}





\slokawithoutfn{5.14}{ He should not eat \skt{chattrāka} mushrooms, village hog, and cow flesh. He should also avoid sparrows, pigeons, and water-birds.}





\slokawithoutfn{5.15}{ He should also avoid [eating] geese, cranes, \skt{cakravāka} birds, dogs, parrots and hawks, crows, owls, \skt{balāka} cranes, fish etc.}





\slokawithoutfn{5.16}{ He should avoid everything that is ritually impure or polluted. He should also completely avoid those vegetables, roots and fruits that are prohibited.}





\slokawithfn{5.17}{ In the books of Manu, in the Purāṇas, in Śaiva texts, and in the Bhāratasaṃhitā (= the Mahābhārata), the practice of purity is definitely expanded in full.}
{ Understant \skt{°śaivabhāratasaṃhite} as  \skt{śaive bhāratasaṃhitāyāṃ}.  }





\slokawithoutfn{5.18}{ Now you have asked me [? about it], and I taught it [to you] in a condensed form. He who speaks the truth is pure. He who engages in yogic meditation is pure.}





\slokawithoutfn{5.19}{ He who avoids violence and is restrained is pure. He whose patience has become compassion is pure[???]. Of all the [ways of] purification, material purification is taught to be the highest.}





\slokawithoutfn{5.20}{ For he who is pure with regards to material things is truly pure, and not he who [only] uses clay and water [i.e. who performs only ordinary baths]. When purification pertains to the body, to speech and to the mind, that is purity of all things.}





\slokawithfn{5.21}{ If a person knows the rules of purity and impurity, he will surely (niścayaṃ?) gain happiness at the end of time, eternally embellished with glory and fame. He has reached here in this world all the merits that the books on true Dharma teach, i       and at the end of his life he will undoubtedly reach the desired path in the other world.}
{ Note the stem form adjective \skt{°jña} and noun \skt{°mānava} metri causa,                  the second syllable of \skt{yadi} as a long syllable at the caesure, the plural \skt{āpnuvanti} where one would expect a verb                     in the singular, \skt{kīrtir} metri causa for a compounded stem form (\skt{kīrti°}),                 and the sandhi-bridge \skt{-m-} in \skt{paratra-m-īhita°}.  }



\vfill\pagebreak

\thispagestyle{empty}\addcontentsline{toc}{section}{Chapter 6}
\begin{center}
{\large{Chapter Six}}
\end{center}




\begin{center}
{{[The second niyama-rule: Sacrifice]}}
\end{center}




\slokawithfn{6.1}{ [Anarthayajña spoke:] Now I shall teach you the five types of sacrifice, O excellent Brahmin, for [your] success in Dharma and liberation. Listen carefully, O Brahmin!}
{ Maybe ījyāṃ is to be accepted. No, see 5.3a. }





\slokawithfn{6.2}{ Material sacrifice, sacrifice through work, sacrifice through recitation, knowledge and meditation: I shall teach you these five one by one.}
{ Note pañcaitat for pañcaitāni or pañcete. }




\begin{center}{{[Material sacrifice]}}\end{center}




\slokawithfn{6.3}{ Material sacrifice includes the following: the worship of fire etc., the performance of the ritual of Agnihotra, oblations on the eight day after full moon, oblations offered at new and full moons, and the rituals for the ancestors.}
{ See Dharmasūtras, Niśv book, Kiraṇa, Svacchanda, Tantrāloka etc. }




\begin{center}{{[Sacrifice through work]}}\end{center}




\slokawithoutfn{6.4}{ The sacrifice through work is the construction of a grove, a park, a pond or a temple with one's own hands.}




\begin{center}{{[Sacrifice with recitation]}}\end{center}




\slokawithfn{6.5}{ Next I shall teach you the sacrifice with recitation, the bestower of the fruits of heaven and liberation. One should recite the Vedas, the Śivasaṃhitā [= Śivasaṃkalpa? or rather śaivaṃ bhāratasaṃhitaṃ ca?],}
{ Note vedādhyayana (stem form) and °saṃhitam for saṃhitāṃ metri causa. }





\slokawithoutfn{6.6}{ the epics and the Purāṇas: this is called sacrifice with recitation. He who is knowledgeable about inference CHECK and reasoning, [and knows that] ``this is [proper] action; the other is improper action'',}




\begin{center}{{[Sacrifice through knowledge]}}\end{center}




\slokawithoutfn{6.7}{ and views [things through?] the eyes of science is called [a person performing] sacrifice through knowledge. I shall teach you concisely about sacrifice through meditation. Listen to me.}




\begin{center}{{[Sacrifice through meditation]}}\end{center}




\slokawithoutfn{6.8}{ Meditation was taught by Hari in the past as of five kinds. [Meditation of] the Sun, the Moon, Fire, Crystal and the subtle Tattva as fifth.}





\slokawithfn{6.9}{ First it is the Sun [that should be meditated upon], which is said to be Prakṛti Tattva. He should visualize the Moon in its centre: that is said to be Puruṣa [Tattva].}
{ Note śaśiṃ for śaśinaṃ. }





\slokawithoutfn{6.10}{ In the centre of the Moon disk, he should visualise a flame, a fire. That is said to be Prabhu Tattva, the destroyer of birth and death.}





\slokawithoutfn{6.11}{ In the centre of the ring of fire, he should visualize a spottless crystal. That is said to be Vidyā Tattva, the never-born, imperishable Cause.}





\slokawithoutfn{6.12}{ In the centre of the disk of Vidyā, he should visualize the highest Tattva, never-heard, unparalleled one, undecaying and imperishable Śiva. The fifth Tattva of the sacrifice through meditation has been taught in short.}





\slokawithfn{6.13}{ Vigatarāga spoke: Teach me: what are the fruits of [reaching] each Tattva? Which worlds can be attained and how much time [can one spend there], O great ascetic?}
{ \skt{tri°} in the MSS is a problem. }





\slokawithfn{6.14}{ Anarthayajña spoke: The first [world to reach] is Brahmaloka, through the meditation on the first Tattva, Prakṛti. He will rejoice [there] happily like Śiva for millions of aeons.}
{ Odd syntax plus gender. }





\slokawithoutfn{6.15}{ If one dies while meditating on the second Tattva, Puruṣa, one goes to Viṣṇuloka from this world, [and will live there] happily for billions of aeons.}





\slokawithoutfn{6.16}{ Should one die while meditating on the third Tattva, Prabhu, one can live in Śivaloka continuously for a hundred billion aeons.}





\slokawithoutfn{6.17}{ If he visualizes Vidyā Tattva, [i.e.] Sadāśiva [or sadā śivam?] he can reach [His] immortal, diseaseless, imperishable world [and can live there] well beyond endless aeons[?].}





\slokawithoutfn{6.18}{ The fifth one, the subtle Śivatattva dwells in the Self. There is no counting of time there and he will be rejoicing [there] together with Śiva.}





\slokawithfn{6.19}{ [If] he practises the five meditations, there is no rebirth and no more fetters of transmigration. O excellent Brahmin, [the Lord] should be seeked, a wishing tree of desires, [as] he burns away existence. Liberation comes within one single birth! People, why should you not strive [for it]! [This is known] as the destroyer of all impurity. [It's ascertainable] by direct perception. It is not inference. It is to be experienced by your own self.}
{ Note how a plural imperative ātmanepada form (jijñāsyantāṃ) stands for the singular                 (jijñāsyatāṃ) metri causa. Note also that the last syllable of                 dvijendra counts here as long: this phenomenon of a word-ending                 syllable becoming long by position is common in the VSS.  Note the form janmena. }




\begin{center}
{{[The third niyama-rule: Penance]}}
\end{center}




\slokawithoutfn{6.20}{ The first [type of penance] is mental penance, the second is verbal penance, the third is the bodily one, the next one[??] is the one which is both mental and verbal action. The fifth type of penance is a mixture of the bodily and the verbal.}





\slokawithfn{6.21}{ Gentleness of the mind, calmness, self-control, taciturnity and the purification of one's state of mind: mental penance comprises these five.}
{ Note that miśraka in pāda b stands for miśrakaṃ metri causa.  ete would be better for etāni? phps no, see 6.24c. }





\slokawithoutfn{6.22}{ Verbal penance is taught as speech that causes no anxiety, which is kind, true and useful, and [it includes] also the practice of recitation.}





\slokawithoutfn{6.23}{ Bodily penance is taught as the following: honesty, harmlessness, chastity, the worship of gods, and purity as the fifth.}





\slokawithoutfn{6.24}{ [Penance] which is a mixture of the mental [and the verbal] is taught by the great Ṛṣis to be these five: he should speak [about things that are] agreeable, virtuous [bhāva?], auspicious, salutary and useful.}





\slokawithoutfn{6.25}{ [Penance] in which bodily [and verbal things] are mixed is taught by the great Ṛṣis to be these five: the worship of the guest and the guru by asking about their well-being, celebrating them and blessing them.[??]}





\slokawithfn{6.26}{ [Being] a [so-called] frog-yogin in the winter, or one with the five fires, or one who has the clouds [i.e. the open sky] for shelter in the rainy season: this kind of penance is called \skt{sādhana}.}
{ CHECK abhrāvakāśa in MBh, Manu and Śivadharmasaṃgraha. }





\slokawithoutfn{6.27}{ Carving out his own flesh as a donation, or [offering his own] hand, feet and head, ... puṣpa as blood? All these kinds of penance is \skt{sādhana},}





\slokawithoutfn{6.28}{ [such as also] the Painful penance and the Extremely Paniful one, [eating only] at night, the Hot and Painful and [the one in which only food obtained] without solicitation [can be eaten], the Cāndrāyaṇa and Parāka penances, the Sāṃtapana etc.}





\slokawithfn{6.29}{ A person who performs with a well-disposed mind this penance that puts an end to the suffering caused by mundane existence, abandoning the trap of hope, with a spotless mind, giving up the lowest rewards [such as] wishing for heaven, being a king and having enjoyments for the senses, can bring that ultimate [? \skt{sarvāntika}] reward which stems from it [i.e. from \skt{tapas}] to [this] home of eternal births and deaths.}
{ Note the stem form \skt{°pāśa} in \skt{pāda} b metri causa. }



\vfill\pagebreak

\thispagestyle{empty}\addcontentsline{toc}{section}{Chapter 7}
\begin{center}
{\large{Chapter Seven}}
\end{center}




\begin{center}
{{[The fourth niyama-rule: Donation]}}
\end{center}




\slokawithfn{7.1}{ In the past the wise declared that there were five kinds of donation ... CHECK Donation of food, clothes, gold, land and the fifth, donation of cows.}
{ \skt{tathety} is suspicious. Note how \skt{annaṃ}, \skt{vastraṃ}, \skt{hiraṇyaṃ} and \skt{bhūmi} (the latter treated as neuter, or given in                                 stem form) are all meant to go with -\skt{dāna} (again, in stem form, metri causa). }




\begin{center}{{[Donation of food]}}\end{center}




\slokawithoutfn{7.2}{ From food [comes] energy, memory, the vital breath, growth, body, happiness. From food arise grace and beauty, heroism, strength.}





\slokawithoutfn{7.3}{ Living beings live on food. Food always satisfies. From food arise desire, rapture, pride and valour.}





\slokawithoutfn{7.4}{ Food drives away hunger and thirst and disease instantly. From donations of food arise happiness, fame and glory.}





\slokawithoutfn{7.5}{ He who donates food donates life. He who donates life donates everything. Therefore nothing is equal to the donation of food, nothing was, nothing will be.}




\begin{center}{{[Donation of clothes]}}\end{center}




\slokawithoutfn{7.6}{ ...  ? A person without clothes may not be respected by his wife, son, friends etc.}





\slokawithoutfn{7.7}{ Be it a learned person from a good family or an intelligent and virtuous one, a person without clothes is subdued and humiliated on every occasion}





\slokawithfn{7.8}{ because a person without clothes receives contempt and disrespect. Even a great soul will try to avoid [him] at the court, among women, in an assembly.}
{ The intention originally may have been this: ``Even if he is a great soul, he will be avoided...'' }





\slokawithoutfn{7.9}{ Therefore the wise praise donations of clothes. One should not give away old, torn or dirty clothes.}





\slokawithoutfn{7.10}{ [Clothes] should be donated [only if they are] new, not worn, soft, delicate and beautiful, well-washed, and [if] accompanied by willingness and devotion.}





\slokawithfn{7.11}{ They say that the reward [of donation/generosity] is in every case dependent on the particular [donor's] willingness and character, the choice of place and time, and on the particular recipient and material.}
{ It seems that \skt{vidhena ca} stands for \skt{vidhinā ca} or rather \skt{vidhānena} metri causa in \skt{pāda} b. }





\slokawithoutfn{7.12}{ The reward received will be similar to the clothes donated. By donating old clothes, one would receive old clothes [as a reward]. By donating beautiful clothes, one would receive beautiful clothes [as a reward].}





\slokawithoutfn{7.13}{ Should one bestow very beautiful clothes on a Brahmin [lit. on a person who is first among the twice-born] in an auspicious time, respectfully. he [i.e. the donor] will receive unequalled happiness and a beautiful appearance. When he departs, he will be given hundreds of millions of items of nice clothes, no doubt about that. Therefore do donate clothes often. It is the way up to the other world.}




\begin{center}{{[Donation of gold]}}\end{center}




\slokawithoutfn{7.14}{ O great Brahmin, now I shall teach you about the donation of gold in a concise manner. It is pure, auspicious and meritorious [act] and it washes off all sins.}





\slokawithfn{7.15}{ Should one hand over [to someone] a golden bracelet or ring, O Brahmin, he will be freed of all sins, just as the moon is freed from [the demon] Rāhu.}
{ I suspect that \skt{aṅguli} is used here in the sense of \skt{aṅgulīya} (`finger-ring'). }





\slokawithfn{7.16}{ If a person donates  gold to Brahmins or gods, O excellent Brahmin, even if it is only in a minute quantity, he will be freed of all sins.}
{ The form \skt{tuṭi} as a widespread variant of \skt{tuṭi}, see e.g. CHECK. }





\slokawithfn{7.17}{ [The amount can be just] one \skt{rakti}, a \skt{māṣaka}, a \skt{karṣa}, half a \skt{pala} or a \skt{pala}: this is exactly how the increase in the [size of the corresponding] reward will be, in proportion to the kind [i.e.\ amount] of the donation.}
{ I suspect that \skt{phalaṃ vṛddhir} stands for \skt{phalavṛddhir} (\skt{phalasya vṛddhiḥ}) metri causa, meaning                                 `the increase of the reward'. }




\begin{center}{{[Donation of land]}}\end{center}




\slokawithoutfn{7.18}{ The wise praise the donation of land as the basis of everything [else]. Food, clothes, gold etc.: all of these originate in the land.}





\slokawithoutfn{7.19}{ O Brahmin, one can obtain all the rewards of donation be donating land. If there is anything that equals the donation of land, O Brahmin, you should really tell me.}





\slokawithfn{7.20}{ [Humans] have the earth as their abode as soon as they get out of their mother's womb. Land is taught as common to all that is mobile and immobile.}
{ I take \skt{sādhāraṇā} as one word, but it is possible that the intention of the author                         was \skt{sā dhāraṇā} in two words, in fact meaning \skt{sādhāraḥ} (\skt{sā ādhāraḥ}, `it is the basis'). }





\slokawithoutfn{7.21}{ Be it [only a land of] one forearm, two forearms, fifty or a hundred, a thousand, ten thousand, a hundred thousand, donations of land are held in great esteem.}





\slokawithoutfn{7.22}{ Should he donate a piece of land of [only] one forearm to an excellent Brahmin, he will enjoy a billion divine years in heaven.}





\slokawithfn{7.23}{ Thus in case of many forearms [of land], the reward is said to be [proportional to the dimensions of the land, i.e.] ... O Brahmin, I have taught you about the rewards of donation that is made willingly.}
{ I think that \skt{guṇāguṇi}, or perhaps \skt{guṇaguṇi} (which would be unmetrical), should refer to the idea                         that e.g. the donation of a piece of land of 2 x 2 \skt{hasta}s would result in                          4 x \skt{koṭiśata} years in heaven, \skt{guṇa} generally meaning `times'. But this is only a guess, and                         it needs to be supported by some similar passage.                         I suspect that \skt{pāda} c is an awkward attempt at saying \skt{śraddhādhikadāna(sya) phalaṃ}. }





\slokawithfn{7.24}{ [Paraśu]rāma, the son of Jamadagni, having donated land to the Brahmin [Kaśyapa], obtained eternal life in this very world, O excellent Brahmin.}
{ See entry `Paraśurāma' in Purāṇic Enc.:                 

                                 To atone for the sin of slaughtering even                                 innocent Kṣatriyas, Paraśurāma gave away all his                                 riches as gifts to brahmins. He invited all the brahmins                                 to Samantapañcaka and conducted a great Yāga there.                                 The chief Ṛtvik (officiating priest) of the Yāga was                                 the sage Kaśyapa and Paraśurāma gave all the lands                                 he conquered till that time to Kaśyapa. Then a plat-                                 form of gold ten yards long and nine yards wide was                                 made and Kaśyapa was installed there and worshipped.                                 After the worship was over according to the instructions                                 from Kaśyapa the gold platform was cut into pieces                                 and the gold pieces were offered to brahmins.                                 When Kaśyapa got all the lands from Paraśurāma he                                 said thus:—“Oh Rāma, you have given me all your                                 land and it is not now proper for you to live in my                                 soil. You can go to the south and live somewhere on                                 the shores of the ocean there.” Paraśurāma walked                                 south and requested the ocean to give him some land to                                 live.  }




\begin{center}{{[Donation of cows]}}\end{center}




\slokawithoutfn{7.25}{ [A cow] with golden horns, silver hooves, garment and bell, O Brahmin, when given to a Veda-knowing Brahmin, [produces] rewards that are said to be endless.}




\begin{center}{{[Praise of donation]}}\end{center}




\slokawithfn{7.26}{ Always rejoicing in the practice of giving as far as his capacities go ... ? one should give food, clothes, gold and silver, water, cows, sesamum [oil?], land, sandals, parasols, seats, jars, cups or anything else. Making the [deed of] giving willingly (\skt{śraddhādāna}) something done with an uninterrupted facial expression of affection, one's mind becomes spotless.}
{ For \skt{śakyānurūpaṃ} in \skt{pāda} a understand \skt{śakyatānurūpaṃ}. }





\slokawithfn{7.27}{ Glory and fortune that makes us happy come about only by donations, and one can gain unequalled fame. The reproach of the enemy will give pleasure and happiness only because of donations[?]. Being invincible comes from donation and also unequalled graciousness. One can reach happiness thought donations. Endless enjoyments surely come only from donations, and heaven is [reached] also because of it.}
{ I suspect that \skt{khyātiś ca tulyaṃ} in the MSS stands for \skt{khyātim atulyāṃ} (`and unequalled fame')                  metri causa. I have corrected those parts of this phrase that could be                                  corrected without violating the metre.  REVISE! ūrja?  Note \skt{svargaṃ} as a neuter in \skt{pāda} d. }





\slokawithfn{7.28}{ The unequalled world of Śakra [i.e. Indra] [can be reached] only by donations. Donations make people happy. Samrāj enjoyed the whole earth in the world only because of donations. CHECK Skanda (\skt{candrānana}) is seen as a handsome and fortunate one with a [good] family[? CHECK] only because of donations. One can reach happiness that lasts countless births only through donations, there is no doubt about that.}
{ Revise. }



\vfill\pagebreak

\thispagestyle{empty}\addcontentsline{toc}{section}{Chapter 8}
\begin{center}
{\large{Chapter Eight}}
\end{center}




\begin{center}
{{[The fifth niyama-rule: Study]}}
\end{center}




\slokawithfn{8.1}{ Five kinds of study are to be pursued by those who wish to be happy in this life and in the other: [one has to study the] Śaiva [teachings], Sāṃkhya [philosophy], the Purāṇa[s], the Smārta [tradition] and the \skt{Bhāratasaṃhitā} [i.e. the \skt{Mahābhārata}].}
{ Note the accusative ending of \skt{°saṃhitām} after a list consisting of words probably in the                         nominative. One may correct it to \skt{°saṃhitā}. }





\slokawithfn{8.2}{ He should reflect on the Śaiva truth in both Śaiva and Pāśupata [teachings]. In those teachings the whole essence of truth is taught extensively.}
{ Note that \skt{śaivatattvaṃ} in pāda a is the result of a conjecture and that the reading \skt{śaivapāśupatadvaye}                        in pāda b is based on one single manuscript (\msP). In spite of this uncertainty,                        I think that this form of the current half-verse is the only one that yields an appropriate meaning. }





\slokawithfn{8.3}{ Those who reflect on the truth (\skt{tattva}) can grasp the truth (\skt{tattva}) of enumeration (\skt{saṃkhyā}) [of ontological principles/reality levels] from Sāṃkhya [texts]. The great sages taught [those twenty-five] \skt{tattva}s [of Sāṃkhya] as being in groups of five.}
{ In pāda d, \skt{kīrtitāni} pick up an implied \skt{tattvāni}. }





\slokawithfn{8.4}{ In the Purāṇas it is the sheaths of the world that are described extensively. One can definitely enter [the realm] of the lower [world, i.e. hell], the upper [world, i.e. heaven], and middle [world, i.e. the human world], and the horizontal [world, i.e. of animals by studying the Purāṇas].}
{ Note that \skt{tirya} seems to be an acceptable nominal stem in this text for \skt{tiryañc}.                I understand the causative form \skt{sampraveśayet} as non-causative, and                interpret °madhya° as the `human world' tentatively. }





\slokawithfn{8.5}{ The Smārta [tradition] deals with the conduct of the classes (\skt{varṇa}) and the conduct in the life-stages (\skt{āśrama}), and with the activities of Dharma and legal proceedings. Good conduct is to be gathered from that [source] without hesitation, with trust.}
{ Compare pāda a with 3.15c. }





\slokawithoutfn{8.6}{ A man who studies the epics (\skt{itihāsa}) will become omniscient. [All his] doubts about Dharma, Artha, Kāma and Mokṣa will be eliminated.}




\begin{center}
{{[The sixth niyama-rule: Sexual restraint]}}
\end{center}




\slokawithoutfn{8.7}{ Listen with great attention, O Brahmin, to the five types of sexual restraint [concerning the following:] women, forbidden ejaculation, and masturbation are mentioned [in this context, as well as] offence while sleeping, O Brahmin, and daydreaming as the fifth.}




\begin{center}{{[Women]}}\end{center}




\slokawithfn{8.8}{ A woman is not to be approached sexually in daytime and on the four days of the changes of the Moon (\skt{parvan}), even if she is one's lawful wife. One should not have sex with a woman who is taboo or with one of those who have lost their class (\skt{varṇa}) or are [of a] superior [\skt{jāti} than oneself].}
{ Understand \skt{parve} as \skt{parvani} (thematisation of the stem in \skt{-an}). }




\begin{center}{{[Forbidden ejaculation]}}\end{center}




\slokawithfn{8.9}{ Intercourse with goats, sheep, cows, mares, buffaloes is called forbidden ejaculation, which is to be avoided at all cost.}
{ Understand \skt{°ādīnāṃ} in pāda a as standing for the locative case.  Understand \skt{°sargam} as neuter nominative (instead of \skt{°sargaḥ}) or alternatively                understand pāda c with a hiatus bridge: \skt{garhitotsarga-m-ity etad}. }




\begin{center}{{[Masturbation]}}\end{center}




\slokawithfn{8.10}{ Rubbing himself against something else than a female sexual organ or rubbing his anus, are called masturbation, therefore these are to be avoided.}
{ The conjecture that changes \skt{anyonya°} to \skt{ayonya°} in pāda a involves                  minimal intervention and makes the sentence much more meaningful than the                  version transmitted. Also consider \skt{ayoni°}.  The variant \skt{strī} for \skt{tāṃ} in pāda d in the \Ed\ may be an example of Naraharināth, the editor's                          conscious interventions. }




\begin{center}{{[Offence while sleeping]}}\end{center}




\slokawithoutfn{8.11}{ Offence while sleeping, O best of Brahmins, has always been [considered] undesirable by the learned. [If] one enjoys women while sleeping, his semen gets spilt.}




\begin{center}{{[Daydreaming]}}\end{center}




\slokawithoutfn{8.12}{ Daydreaming [about women] should always be avoided by those who are intent on Dharma. Women are called `the bolts [that block the gate to] the path to heaven'.}




\begin{center}
{{[The seventh niyama-rule: religious observances]}}
\end{center}




\slokawithfn{8.13}{ [Hear about] the five religious observances [called] the cat, the crane, the dog, the cow, and the earth. <sep/>He buries his own urine and faeces in the ground, O truest Brahmin. He rejoices [seeing] the sun and the moon when performing the cat observance.}
{ Note \skt{°viṣṭha°} for \skt{viṣṭhā} metri causa in pāda c (\skt{ma-vipulā}).                 Alternatively, read \skt{svaviṣṭhāmūtra bhūmīṣu}.  Note the stem form \skt{sūryasoma} for \skt{sūryasomau} in pāda e.                  It is not entirely clear why cats would rejoice seeing the Sun and the Moon.                 Perhaps this remark refers to the fact that cats can be active both                 in the daytime and at night. }




\begin{center}{{[The Cat Vow]}}\end{center}



\begin{center}{{[The Crane Vow]}}\end{center}




\slokawithfn{8.14}{ O great ascetic, one should suppress all of his senses like a crane, and should cultivate the peace of the mind, focusing on achieving liberation.}
{ Cranes are compared to ascetics here probably because of the similarity of                 their tendency of relaxing standing on one leg to ascetics performing penance                  standing on one leg (such as the ascetic depicted on the famous relief in Mahabalipuram).  }




\begin{center}{{[The Dog Vow]}}\end{center}




\slokawithfn{8.15}{ He does not bury his urine and faeces in the ground, and he barks constantly. Lord Śarva [i.e. Śiva] is satisfied when one practises the dog observance.}
{ CITE source on dog being Bhairava's vāhana... }




\begin{center}{{[The Cow Vow]}}\end{center}




\slokawithfn{8.16}{ A person practising the Cow Vow should never hold back his urine and faeces. He is terrifying and he gives satisfaction, [as] stated in the Purāṇas.}
{ I prefer reading \skt{bhīma tuṣṭi°} as two separate words, the first                 one in stem form, to reading it as a compound because                 of the following \skt{caiva}, and to the reading \skt{bhīmas tuṣṭi°}                  because the corresponding witnesses are the ones that usually give inferior readings. }




\begin{center}{{[The Earth Vow]}}\end{center}




\slokawithfn{8.17}{ CHECK Digging [the earth] with spades and collecting [? the soil] with wedges: Goddess Earth bears [this] patiently. This is exactly how one can practise the earth vow.}
{ While \skt{dārayanto} as an active participle in the masculine nominative is acceptable                 as an irregular form, the precise interpretation of pādas a and b is still problematic. }





\slokawithfn{8.18}{ He who practises these five religious observances with his senses subdued will, without doubt, reach this superior world (i.e. heaven?).}
{ Note the neuter \skt{idaṃ} picking up the normally masculine \skt{lokaṃ} in pāda c. }




\begin{center}
{{[The eighth niyama-rule: Fasting]}}
\end{center}




\slokawithoutfn{8.19}{ Eating leftovers, [not] eating in-between [breakfast and dinner], eating [only] at night, eating food obtained without solicitation, and fasting: listen, I shall teach you these five.}




\begin{center}{{[Eating leftovers]}}\end{center}




\slokawithoutfn{8.20}{ [He who eats] the leftovers belonging to all the gods, to guests, and to the ancestors, he who eats the leftovers (śeṣāśin) of servants, sons and wives is the one who consumes the remains of food (\skt{vighasāśana}).}




\begin{center}{{[[Not] eating in-between breakfast and dinner]}}\end{center}




\slokawithfn{8.21}{ He will be regarded as one that is always fasting if he never eats between breakfast and dinner.}
{ My translation here follows the parallel verse in the MBh and                        is based on that of Kisari Mohan Ganguli. The syntax of the version here in the VSS is less                       smooth than that in the MBh, and the VSS's reading \skt{prāntarāśī} definitely required an emendation. }




\begin{center}{{[Eating [only] at night]}}\end{center}




\slokawithfn{8.22}{ One should not eat in the daytime or in the evening, and should eat [only] at midnight if he wishes to follow the order of [eating only at] night.}
{ Note \skt{°vele} for \skt{°velāyāṃ} in pāda c. }




\begin{center}{{[Eating food obtained without solicitation]}}\end{center}




\slokawithfn{8.23}{ He should eat only the unsolicited food of someone who has not yet started eating [this food]. He who eats [only] that which has been given by others [without asking them for it] is called [one who eats] unsolicited [food].}
{ The translation of \skt{anārambhasya} (`of someone who has not yet started eating') is tentative. }




\begin{center}{{[Fasting]}}\end{center}




\slokawithfn{8.24}{ Chewable and unchewable food (\skt{bhakṣyaṃ bhojyaṃ ca}), food to be sipped or sucked or drunk, as the fifth [category]: if one does not long for and does not consume [any of the above], that is called fasting (\skt{upavāsa}).}
{ For a detailed discussion of the categories \skt{bhakṣya, bhojya, lehya} and \skt{coṣya},                         see Kafle 2020:245, n. 534. See also Śivadharmottara 8.13:

                         \skt{bhakṣyaṃ bhojyaṃ ca peyaṃ ca lehyaṃ coṣyaṃ ca picchilam} \danda\\                         \skt{iti bhedāḥ ṣaḍannasya madhurādyāś ca ṣaḍguṇāḥ} \twodanda }




\begin{center}
{{[The ninth niyama-rule: Silence]}}
\end{center}




\slokawithfn{8.25}{ One should keep these five types of taciturnity, always dwelling in religious observances: [in situations where silence is best instead of] deceitful speech, envious speech, abuse, harsh speech, bragging.}
{ \skt{pāruṣya} seems to be the good reading in pāda a because                  in the following a short section on this category is coming up.                 As far as the readings \skt{spṛṣṭavāg} and \skt{pṛṣṭavāg} are concerned, I suppose                  \skt{pṛṣṭavāg} is not inconceivable (as suggested by Judit Törzsök),                  for in 8.29 it is questions that are given as relevant examples.                  Nevertheless I conjectured \skt{tīkṣṇavāg} here, relying on the same verse, 8.29. }




\begin{center}{{[Deceitful speech]}}\end{center}




\slokawithoutfn{8.26}{ Fictitious [speech], [speech on] unknown [things], [speech about things] outside the range of Dharma, meaningless and unfriendly speech: these are called lying.}




\begin{center}{{[Envy]}}\end{center}




\slokawithoutfn{8.27}{ One who does not rejoice in others' fortune or in others' power, one who would like to see something disadvantageous [for others] is called envious [and he should rather remain silent].}




\begin{center}{{[Abuse]}}\end{center}




\slokawithfn{8.28}{ [May your] mother and father be dead! [This is] how a ruined state will befall [you]! Enjoy the love of unclean [women]! [These are] called abuse.}
{ My translation of pāda b, or rather of the whole verse, is tentative. }




\begin{center}{{[Harsh speech]}}\end{center}




\slokawithfn{8.29}{ Won't you burst in your heart, stupid? Will your head not split into two? [If one utters] these or similar [curses], he is said to be one of harsh speech.}
{ Understand \skt{śiro} as standing for the locative (\skt{śirasi}). }




\begin{center}{{[Bragging]}}\end{center}




\slokawithfn{8.30}{ Relating fancy stories about gambling, enjoyments, fights, drinking and women are the five types of bragging, as I teach them, O excellent Brahmin.}
{ I take \skt{°katham} in pāda b as an alternative nominative form of \skt{°kathā} metri causa and as                  belonging to all the categories here thus: \skt{dyūtakathā, bhojanakathā, yuddhakathā, madyakathā,                 strīkathā}.  Understand \skt{me} in pāda d as \skt{mayā}. }





\slokawithoutfn{8.31}{ Taciturnity should always be practised by those who prefer the beauty of speech. One should always speak without abuse and without idle talk.}





\slokawithfn{8.32}{ He who does not practise taciturnity is defiled and he is the black sheep of the family. For a number of rebirths, [his mouth] will stink and he will become mute.}
{ The form \skt{janme} for \skt{janmani} often occurs in Śaiva tantras as a tipically Aiśa phenomenon.                 See XXXXX  }





\slokawithfn{8.33}{ Therefore the speech of a person who always keeps the observance of taciturnity firmly, with resolution, will be impossible to ignore and he will make the community rejoice. The fragrance of lotuses and [other kinds of] strong fragrances will blow from his mouth. Thousands of faultless \skt{śāstra}s will be declared in the words of this person.}
{ To make sense of pāda d, we are forced to take \skt{śāstra} as a stem form noun and                  \skt{naraḥ} as a (regular) genitive from \skt{nṛ}. (I thank Judit Törzsök for this interpretation.)                 Another way of understanding the beginning of this sentence would be to separate \skt{śāstrāneka°} as                 \skt{śāstrān eka°}, treating the word \skt{śāstra} as masculine. }




\begin{center}
{{[The tenth niyama-rule: Bathing]}}
\end{center}




\slokawithoutfn{8.34}{ I shall teach you the five kinds of bathing as they really are: Fire bath, water bath, Vedic bath, wind bath and divine bath.}




\begin{center}{{[Fire bath]}}\end{center}




\slokawithoutfn{8.35}{ Fire bath is [performed] with ashes. Its fruits are a hundred times bigger than [those of] a water [bath]. [Things] purified with ashes are holy. Ashes destroy sin.}





\slokawithoutfn{8.36}{ Therefore one should use ashes for it purifies humans of their defilement. Ashes produce peace for everyone. Ashes are the ultimate protectors.}





\slokawithfn{8.37}{ Drawing [the sectarian marks on their foreheads while reciting] the Tryāyuṣa [mantra], remaining in chastity, all the Ṛṣis purified themselves with ashes.}
{ Note \skt{tryāyuṣa} in the sense of the three \skt{puṇḍra}-lines on the                 forehead and compare with 11.28c. Understand \skt{sthitam} as                  \skt{sthitaḥ} or rather \skt{sthitāḥ} if we are to connect this line                 to the next (8.37cd).  Grammatical notes on kṛtam and ātmanaḥ }





\slokawithfn{8.38}{ The gods, afflicted by their fear of Vīrabhadra, were set free with the help of ashes. Seeing the glory of ashes, Brahmā consented [to the use of this otherwise impure substance].}
{ It is not clear which story concerning Vīrabhadra is referred to here.                  Is it the destruction of Dakṣa's sacrifice, after which the gods were relieved?                  Or, which is a less likely possibility, another in which                  Kaśyapa and other Ṛṣis were burnt to ashes then reanimated by Vīrabhadra in the Śokara forest?                  For the latter, less well-known story, see Padmapurāṇa 5.107.1--14ff:

                         \skt{śucismitovāca \\                          kaśyapaṃ jamadagniṃ ca devānāṃ ca purā katham \danda \\                         rarakṣa bhasma tad brahman samācakṣva mune mama \twodanda1  \\                         dadhīca uvāca  \\                         kaśyapādiyutā devāḥ pūrvam abhyāgaman girim  \danda\\                         śokaraṃ nāma vikhyātaṃ girimadhye suśobhanam \twodanda2  \\                         nānāvihaṃgasaṃkīrṇaṃ nānāmunigaṇāśrayam \danda \\                         vāsudevāśrayaṃ ramyam apsarogaṇasevitam \twodanda3  \\                         vicitravṛkṣasaṃvītaṃ sarvartukusumojjvalam  \danda\\                         tathāvidhaṃ praviśyaite giriṃ vayam athāpare \twodanda4  \\                         stuvaṃtaḥ keśavaṃ tatra gatāḥ sma giriśeśvaram  \danda\\                         dṛṣṭvā tatra mahājvālāṃ praviṣṭāśca vayaṃ ca tām \twodanda5  \\                         māmekaṃ tu tiraskṛtya hy adahad devatā munīn  \danda\\                         māṃ dadāha tataḥ paścād bhasmībhūtā vayaṃ śubhe \twodanda6\\                           asmān etādṛśān dṛṣṭvā vīrabhadraḥ pratāpavān \danda \\                         kenāpikāraṇenāsau gatavān parvataṃ ca tam \twodanda7  \\                         bhasmoddhūlitasarvāṃgo mastakasthaśivaḥ śuciḥ  \danda\\                         ekākī niḥspṛhaḥ śānto hāhāśabdam athāśṛṇot \twodanda8  \\                         atha ciṃtāparaś cāsīn mriyamāṇa śavadhvaniḥ \danda \\                         śavānām iva gaṃdhaś ca dṛśyate tannirīkṣaṇe \twodanda9  \\                         iti niścitya manasā jagāmāgnim atiprabham  \danda\\                         sa vahnir vīrabhadraṃ ca dagdhum ārabdhavān atha \twodanda10 \\                          tṛṇāgnir iva śāṃto 'bhūd āsādya salilaṃ yathā \danda \\                         tato 'parāṃ mahājvālāṃ vīrabhadras tu dṛṣṭavān \twodanda11 \\                          khaṃ gacchaṃtīṃ mahākālo jvālāṃ nipatitām api  \danda\\                         manasā ciṃtayac cāpi vīrabhadraḥ pratāpavān \twodanda12  \\                         sarveṣāṃ nāśinī jvālā prāṇināṃ śatakoṭiśaḥ \danda \\                         tat sarvaṃ rakṣaṇārthaṃ hi pipāsuś cāpy ahaṃ tv imām \twodanda13  \\                         prāśnāmi mahatīṃ jvālāṃ salilaṃ tṛṣito yathā \danda \\                         etasminn aṃtare vīraṃ vāg āha cāśarīriṇī \twodanda14} 

  ``Śucismitā said:\\  1. O brāhmaṇa, O sage, tell me how formerly the sacred ash protected Kaśyapa, Jamadagni of the gods? Dadhīca said:\\  2--6. Formerly gods accompanied by Kaśyapa and others went to a well-known mountain named Śokara. In the middle of the mountain was a very beautiful (forest) which was full of many birds, which was resorted to by various hosts of sages, which was the resort of Vāsudeva, which was charming, which was resorted to by bevies of celestial nymphs, which was crowded with strange trees, which was bright with flowers of all seasons. We and others entered the best mountain (forest) like that and praising Viṣṇu went there to lord Śiva. We saw a great flame there and we entered it. Excepting me that deity (i.e. that flame) burnt (other) sages. After that it (also) burnt me. O auspicious one, we were reduced to ash. \\ 7--14. Seeing us like this, that brave Vīrabhadra went to that mountain for some reason. With his entire body smeared with sacred ash, he remaining at the top, auspicious and pure, all alone, desireless and tranquil, heard the sound of wailing. Then he was full of thought: ‘The sound of the bodies of dead men and the smell as it were of dead bodies, are being perceived.’ Deciding like this in his mind, he went to the fire of great brilliance. Then that fire also started to burn Vīrabhadra. But it went out as the fire of (i.e. burning) grass (i.e. hay) would go out on receiving (i.e. being sprinkled over with) water. Then Vīrabhadra saw a great, mighty flame, which went (up) to the sky even (like) flame falling (i.e. dropped by) Śiva (obscure!). The brave Vīrabhadra thought in his mind: ‘(This) flame is the destroyer of hundreds of crores of beings. So for the protection of all I desire to drink it. As a thirsty man drinks water, I shall consume this great flame.’ In the meanwhile a divine voice said to (Vīrabhadra) the hero [...] (translation by N.A. Deshpande, in: Padma-purāna, Delhi: MLBD, 1951)''   }





\slokawithfn{8.39}{ [Thus] the Pāśupata observance was created, which is above [the system of] the four \skt{āśrama}s. Therefor the Pāśupata [observance] is the best because it involves carrying ashes [on one's body].}
{ One could simply accept the reading of \msCc (\skt{°hetunā}) in pāda d, but all other rejected                  readings hint at an original \skt{hetutaḥ} (as pointed out by Judit Törzsök). }




\begin{center}{{[Water bath]}}\end{center}




\slokawithfn{8.40}{ A water bath (\skt{vāruṇa}) is to be performed with water by people in various ways in the water of rivers, water tanks, streams and ponds.}
{ The reading \skt{vvidhaṃ} in pāda b seems to be the lectio difficilior as opposed to                 the rejected \skt{vidhivat}. }




\begin{center}{{[Vedic bath]}}\end{center}




\slokawithfn{8.41}{ The wise know the Vedic bath as [the one performed with the Vedic mantra beginning] \skt{āpo hi ṣṭhā} [ṚV 10.9.1--3], O excellent Brahmin. It is to be performed at the three junctures of the day (dawn, noon, evening). It is called the Vedic bath.}
{ The Ṛgvedic mantra starting with \skt{āpo hi ṣṭhā} (ṚV 10.9) is traditionally associated with                  \skt{mārjana} (`cleaning, wiping'). According to Kane (A History of Dharmaśāstra, vol. 4, p. 120),                 a Brahmin ``should bathe thrice in the day, should perform \skt{mārjana} (splashing                 or sprinkling water on the head and other limbs by means of \skt{kuśas}                  dipped in water after repeating sacred mantras) with the three verses `apo hi sthā' [sic] (Ṛg. X.9.1--3) [...]''                 This suggests a method of bathing that is more of a ritual than an actual bath. }




\begin{center}{{[Wind bath]}}\end{center}




\slokawithfn{8.42}{ He should go where, on the paths where cows roam, dust is rising, and he should sit down there. This is called [a kind of] bath, [namely the \skt{vāyavya} or wind-bath].}
{ This version of bathing seems to be rather a kind of bathing                  in the holy dust raising from under the hooves of cows. }




\begin{center}{{[Heavenly bath]}}\end{center}




\slokawithoutfn{8.43}{ One should immerse his own body in the water-showers of rain water. The one and only great Lord (\skt{maheśvara}) of the universe calls it heavenly bath.}





\slokawithfn{8.44}{ Thus have I taught you the section on the Niyama-rules [see Chapters 5--8] in divisions of five [sub-categories] because you asked me to, favouring the whole world. [These Niyama-rules] wipe off all the defilement, these fifty Dharma [teachings, i.e. 10 main topics/rules × 5 subcategories]. There will not be rebirth [for one who keeps these rules], not even in millions of aeons.}
{ Understand \skt{sarvalokānukampya} in pāda b as \skt{sarvalokān anukampya}.  Understand \skt{sakalamalapahārī} in pāda c as \skt{sakala-mala-apahārī}, which would be unmetrical.                        Understand \skt{etan/etad} as either picking up °\skt{pahārī} or                         a plural corresponding to °\skt{pañcāśad}. }



\vfill\pagebreak

\thispagestyle{empty}\addcontentsline{toc}{section}{Chapter 9}
\begin{center}
{\large{Chapter Nine}}
\end{center}




\slokawithoutfn{9.1}{ The whole universe with its moving and unmoving elements is divided by the three [divisions of] time and the [three] \skt{guṇa}s [or guṇa not tech term here?]. Therefore the whole world is bound by the fetters of the three \skt{guṇa}s.}





\slokawithfn{9.2}{ Vigatarāga spoke: What does the term `the three divisions of time' mean for the soul in the three worlds[?]? Talk about it in a somewhat more extended manner, O great ascetic.}
{ I have included the element \skt{trai°} in the lemma in pādas ab only because \msCc\                  has a slightly unusual ligature there (\skt{mtrai}) }





\slokawithoutfn{9.3}{ Anarthayajña spoke: The three [divisions of] time are the three \skt{guṇa}s. It[?] is pervading and born from Prakṛti. They support each other, they serve each other.}





\slokawithoutfn{9.4}{ Sattva, Rajas and Tamas; Rajas, Sattva and Tamas; Tamas, Sattva and Rajas; they are each other's pairs.}





\slokawithoutfn{9.5}{ Lord Viṣṇu is Sattvic. [Brahmā], the one who was born on a lotus, is Rājasa. Lord Īśa is Tāmasa, the limbless is all ... [?]}





\slokawithoutfn{9.6}{ Sattva is of the colour of jasmine and the moon. Rajas is of the colour of ruby. Tamas is of the colour of lamp-black ... śaila. [This is what] the wise teach.}





\slokawithoutfn{9.7}{ Sattva is water, Rajas is charcoal, Tamas is full of smoke. All souls are constructed/suffer (\skt{pacyante}) as bound by these \skt{guṇa}s.}





\slokawithoutfn{9.8}{ Vigatarāga spoke: By what sorts of noose of \skt{guṇa}s is [the soul] bound? Teach me the signs connected to them one by one, O great ascetic.}





\slokawithoutfn{9.9}{ Anarthayajña spoke: The souls are bound in many ways and by many conditions by the fetters of the \skt{guṇa}s. Those who are deluded do not recognize [them]. The Śivayogins do recognize [them].}





\slokawithfn{9.10}{ He who is always established in Sattva goes upwards. He who is covered with Rajas goes in the middle. Those lowest of men in the state of Tamas go downward.}
{ Understand \skt{adhogatis} in pāda c as a bahuvrīhi in plural (\skt{adhogatayas}). }





\slokawithoutfn{9.11}{ These three kinds of \skt{guṇa}s are to be acknowledged even in heaven, O great ascetic, and among humans and also among animals.}





\slokawithoutfn{9.12}{ The ten superior Sattva [beings] are: Brahmā, Viṣṇu, Rudra, Dharma, Indra, Prajāpati, Soma, Agni, Varuṇa and Sūrya.}





\slokawithoutfn{9.13}{ ...}





\slokawithoutfn{9.14}{ ...}





\slokawithoutfn{9.15}{ ... ...}





\slokawithoutfn{9.16}{ ... ...}





\slokawithoutfn{9.17}{ ... ...}





\slokawithoutfn{9.18}{ These are the ten superior Tāmasa [animals]: cows, elephants, Gayal oxen, horses, deer, Yaks, Kiṃnaras, lions, tigers, wild boar.}





\slokawithfn{9.19}{ The ten middle ranking Tāmasa [beings] are: rams, sheep, buffaloes, mice, mongooses etc., camels, Raṅku  deer, hares, rhinoceroses. [only 9!]}
{ \skt{°mahiṣyāś} seems to be an equivalent of \skt{°mahiṣāś} metri causa. }





\slokawithoutfn{9.20}{ The ten low-ranking Tāmasa [beings] are: bears, alligators, deer, horned animals[?], cranes, apes, donkeys, boar, dogs and frogs.}





\slokawithfn{9.21}{ The ten Tāmasa-Sāttvika [beings] are: curlews, swans, parrots, falcons, vultures, B[h]āruṇḍa birds, cranes, Cakra[vāka] birds, parrots, and peacocks.}
{ Although all the manuscripts consulted read \skt{kroñca°} in pāda a, I decided                 to accept \Ed's standard spelling in this case. In pāda b, I left \skt{°bāruṇḍa°}                 thus, although what is really meant is probably \skt{bhāraṇḍa}, \skt{bhāruṇḍa} or \skt{bhuruṇḍa}.  Note the repetition of \skt{śuka} in this stanza. }





\slokawithoutfn{9.22}{ The ten Tāmasa-Rājasa [beings] are: Balāka-cranes, cocks, crows, Bengal kites, Lāvakas, partridges, vultures, herons, Bakas and hawks.}





\slokawithfn{9.23}{ The ten lowest Tāmasa [beings] are: cuckoos, owls, Kiñjalkas[?], doves, Śārika birds and sparrows.}
{ This list is problematic for it has only six elements instead of the expected ten                  and \skt{kiñjalka} is difficult to interpret. }





\slokawithfn{9.24}{ Makaras crocodiles, cow-killing alligators and bears are of Tamas-Sattva. Tortoises, Śuśus[?], crocodiles of the Ganges and frogs are of Tamas-Rajas. Conch-shells, pearl-oysters, shells and [...] are Tamas-Tāmasa.}
{ Note that the reading that yields `and bears' (\skt{ṛkṣāś ca}) is my conjecture                         for a problematic \skt{ṛṣā ca}. It is far from satisfactory since bears have already appeared in                          verse 9.20 above.  I have not been able to identify the probably aquatic animal behind the                          word \skt{śuśu} here. }





\slokawithoutfn{9.25}{ ... ...}





\slokawithoutfn{9.26}{ The ten Tamas-Rajas [trees] are: Citron trees, bread-fruit trees, hog-plum trees, pomegranate trees, jujube trees, ratan trees, Neemb trees, Kadamba trees and ...}





\slokawithoutfn{9.27}{ ... ...}





\slokawithoutfn{9.28}{ ... ...}





\slokawithoutfn{9.29}{ [These words describe] the people who are the best among the Sāttvika [type]: compassion, truthfulness, self-control, purity, knowledge, taciturnity, penance, patience, integrity, lack of self-conceit.}





\slokawithoutfn{9.30}{ [These words describe] the people who are the best among the Rājasa [type]: desire, thirst, pleasure, gambling, arrogance, fight, intoxication, delight, cruel, quarrelling.}





\slokawithoutfn{9.31}{ [These words describe] people who are the best among the Tāmasa [type]: harming, envious, incompassionate, stupid, sleepy, lazy, cowardly, idle, angry, greedy, cheating.}





\slokawithoutfn{9.32}{ The Sāttvika can be characterised as follows: light, joyful, bright, always eager for yoga meditation, wise, intelligent and dispassionate.}





\slokawithoutfn{9.33}{ The Rājasa can be characterised as follows: childish, skilful, passionate, proud, arrogant, greedy, desirous, jealous and chattering.}





\slokawithfn{9.34}{ The Tāmasa can be characterised as follows: anxious, lazy, deluded, cruel, a pitiless robber, angry, wicked and sleepy.}
{ In pāda a, \skt{piśuno} might be the right choice: it is a ra-vipulā                          if \skt{dr} in \skt{nidrā} does not make the previous syllable long, a licence                         often occuring in this text (`muta cum liquida'). }





\slokawithoutfn{9.35}{ Vigatarāga spoke: By what signs can the food of all humans be recognized? [?] Teach me about the three \skt{guṇa}s, O great ascetic.}





\slokawithoutfn{9.36}{ Anarthayajña spoke: The Sāttvikas prefer food that yields [long] life, fame, happiness, joy, which increases strength and health, which is savoury and which tastes nice, and which is soft.}





\slokawithoutfn{9.37}{ The best food for the Rājasas is rather warm, acidic, salty, hard, hot and pungent. It gives you pain, a burning sensation and indigestion.}





\slokawithfn{9.38}{ Tāmasas prefer food that is prohibited, impure and foul-smelling, ... stale  ... and tasteless.}
{ Understand \skt{°pūtī} in pāda a as standing for \skt{°pūti} metri causa, and                  note that °āmedhya° in the same pāda is an emendation (correcting \msNc's reading).  Read \skt{āmayārasa} in pāda c? }





\slokawithoutfn{9.39}{ Vigatarāga spoke: How can one recognize [the state of getting] beyond the \skt{guṇa}s, which leads one to the other shore of [the ocean] of mundane existence? Tell me truly about the liberation of those who are [initially] bound by the noose of the \skt{guṇa}s.}





\slokawithoutfn{9.40}{ Anarthayajña spoke: Well, he who looks at all living beings in the correct way, as his own Self, O Brahmin, is to be known as one beyond the \skt{guṇa}s, as one who has departed to the other shore of [the ocean of] mundane existence.}





\slokawithoutfn{9.41}{ He who treats envy and hate[?], happiness and sorrow, praise and reproach as equal is called `one who is beyond the \skt{guṇa}s'.}





\slokawithoutfn{9.42}{ He who is indifferent to pleasant and unpleasant things, to enemy or friend, to respect and contempt is called `one who is beyond the \skt{guṇa}s'.}





\slokawithoutfn{9.43}{ O Brahmin, thus has the exposition of the essence of the \skt{guṇa}s been taught to you. Those who are connected with the \skt{guṇa}s are mundane (\skt{saṃsārin}), those beyond the \skt{guṇa}s are on the supreme path.}



\vfill\pagebreak

\thispagestyle{empty}\addcontentsline{toc}{section}{Chapter 10}
\begin{center}
{\large{Chapter Seven}}
\end{center}




\begin{center}
{{[The description of the pilgrimage places in the body]}}
\end{center}




\slokawithoutfn{10.1}{ Vigatarāga spoke: Which pilgrimage place do the wise consider the best of all? Tell me, O best of sages, if there is one in the world that fulfills [all] desires.}





\slokawithoutfn{10.2}{ Anarthayajña spoke: This question [that I have been] asked is an extremely deep secret. Out of fondness, O excellent Brahmin, I'll teach you an ancient legend that Nandi told me.}





\slokawithoutfn{10.3}{ Nandikeśvara spoke: On a beautiful peak of Mount Kailāsa, which is frequented by Siddhas and celestial singers (\skt{cāraṇa}), there was Śiva himself there, seated, and Devī spoke to him thus:}





\slokawithoutfn{10.4}{ Devī spoke: O Lord, Lord of the chiefs of the gods, O ruler of all beings and all the world, I would like to ask you about one thing that concerns the eternal and secret Dharma,}





\slokawithoutfn{10.5}{ the transcendental and highly secret pilgrimage place by which one can be liberated from Saṃsāra. O Maheśvara, teach me the truth for the benefit of mankind.}





\slokawithoutfn{10.6}{ Maheśvara spoke: Who else would ask me that question if not you, O Sundarī? Listen, I'll expound that question which is difficult to grasp even for the gods.}





\slokawithoutfn{10.7}{ [If one] gets to know Kurukṣetra, Prayāga, Vārāṇasī, Gaṅgā, Agni[tīrtha], Somatīrtha, Sūrya[tīrtha], Puṣkara, Mānasa,}





\slokawithfn{10.8}{ Naimiṣa, Bindusaras, Setubandha, Suradraha, Ghaṇṭikeśvara, and Vāgīśa, he'll certainly be able to destroy his sins.}
{ Note \skt{bindusāraṃ} for \skt{bindusaras/°saraṃ/°sarasaṃ} metri causa. }





\slokawithfn{10.9}{ Umā spoke: This and other [related] things, O Mahādeva, have been [just] taught to me [by you] as previously. Among these[?] the pilgrimage place that yields all enjoyments, O Suranāyaka.}
{ Is perhaps \skt{pūrvavat} used in the sense of \skt{pūrvaṃ} here? }





\slokawithoutfn{10.10}{ [But] how is one liberated from mundane existence merely be knowledge, O Īśvara? Cut [this] great curiosity arising [in me] that causes doubt.}





\slokawithoutfn{10.11}{ Rudra spoke: How could I not know that pilgrimage place which is both easy and difficult to reach? It is easy to reach for those who serve their guru and difficult to reach should one abandon it [i.e. the service of the guru].}




\begin{center}{{[Kurukṣetra]}}\end{center}




\slokawithoutfn{10.12}{ \skt{Kuru} [in \skt{kurukṣetra}] is to be known as the soul (\skt{puruṣa}), \skt{kṣetra} as the body. Kurukṣetra [which] is in the body yields the fruits of all pilgrimage places.}





\slokawithoutfn{10.13}{ [And there will be] the obtaining of the fruits of all sacrifices, the fruits of all [possible] donations, and all the fruits of all religious observances and penance observed.}





\slokawithoutfn{10.14}{ In the same manner [one will obtain] the fruits of those fifteen pilgrimage places [from Kurukṣetra to Vāgīśa, cf. 10.7--8, by only knowing the bodily Kurukṣetra]. ... [this] great pilgrimage place is extremely auspicious and pleasant.}





\slokawithoutfn{10.15}{ Devī spoke: I am extremely thrilled, O Tridaśeśvara. Hearing this which is easy to obtain, easy to perform and is subtle, I am filled with satisfaction.}





\slokawithoutfn{10.16}{ Teach me on, teach me the remaining fourteen pleasant [pilgrimage places], Prayāga and the others, one by one, as they are, O Sureśvara.}




\begin{center}{{[Prayāga and Vārāṇasī]}}\end{center}




\slokawithfn{10.17}{ The Suṣumnā[-tube] is the Honourable Gaṅgā, Iḍā[-tube] is the river Yamunā. ... is called Prayāga.}
{ There seems to be only two yogic tunnel here (and in 10.20--21): Suṣumnā and Iḍā, instead of                 the usual three (Iḍā, Piṅgalā, Suṣumnā). This is strikingly similar to                 what we see in the archaic yoga of the Niśvāsa Naya, see Goodall et al. pp. 33--34.                 

                 Note \Ed's attempt to make pāda a metrical.  Cf.\ MBh Indices 6.3A.41--44:
                  \skt{iḍā bhagavatī gaṅgā piṅgalā yamunā nadī \danda
                  tayor madhye tṛtīyā tu tat prayāgam anusmaret \twodanda
                  iḍā vai vaiṣṇavī nāḍī brahmanāḍī tu piṅgalā \danda
                  suṣumṇā caiśvarī nāḍī tridhā prāṇavahā smṛtā} \danda
         

         See also \skt{Haṭhayogapradīpikā} 3.110:         
         iḍā bhagavatī gaṅgā piṅgalā yamunā nadī \danda
         iḍāpiṅgalayor madhye bālaraṇḍā ca kuṇḍalī \twodanda  }





\slokawithoutfn{10.18}{ The right nostril is [the river] Vāruṇī, the left nostril is known as [the river] Asi. Because [it is] at the confluence of Vāruṇā and Asi, [the city there] is known as Vārāṇasī.}




\begin{center}{{[The Gaṅgā]}}\end{center}




\slokawithoutfn{10.19}{ She is called the ethereal Gaṅgā [because] the nectar of immortality issues from her day and night uninterruptedly. That's why she is called Gaṅgā (perhaps: `ever-goer').}




\begin{center}{{[Somatīrtha]}}\end{center}




\slokawithoutfn{10.20}{ Somatīrtha is the tube Iḍā. It is characterised by the ringing of small bells. Upon hearing that [ringing], all of one's sins will be destroyed.}




\begin{center}{{[Sūryatīrtha]}}\end{center}




\slokawithoutfn{10.21}{ Sūryatīrtha is the [tube] Suṣumnā .... By merely hearing about it one is liberated, even if one has a huge heap of sins.}




\begin{center}{{[Agnitīrtha]}}\end{center}




\slokawithoutfn{10.22}{ Agnitīrtha is the Arjuna tube[??]. It is charming because of the hum of Veda recitation. Upon hearing this or that syllable, one will become immortal.}




\begin{center}{{[Puṣkara]}}\end{center}




\slokawithfn{10.23}{ Puṣkara is [a lotus] with eight petals and a pericarp in the centre of the heart. One should visualize the Subtle One in its centre [and] it'll destroy birth and death.}
{ \skt{hṛdi} might be meant to be a nominative, as in 12.17, here compounded with \skt{madhyastham}. }




\begin{center}{{[Mānasa]}}\end{center}




\slokawithfn{10.24}{ In the centre of the Mānasa lake on a lotus with [the syllables] HAṂ-SA, ...}
{ Understand \skt{mānasasara°} in pāda a as \skt{mānasasaro} (metri causa). }




\begin{center}{{[Naimiṣa]}}\end{center}




\slokawithoutfn{10.25}{ Listen to Naimiṣa, O Deveśī. It presents proof in a moment. One can observe one's own or others' shadow properly[?].}





\slokawithoutfn{10.26}{ ... When he has seen the proof thus, he is called the knower of Naimiṣa.}




\begin{center}{{[Bindusaras]}}\end{center}




\slokawithfn{10.27}{ Listen O Sundarī, I shall teach you the pilgrimage place called Bindusaras. The heart is to be known to be located in the centre of the body. In the centre of the heart, there is a lotus.}
{ Note \skt{hṛdi} as a nominative in pāda c and possibly also in pāda d (and see 10.23a). }





\slokawithoutfn{10.28}{ There is a pericarp in the centre of the lotus, and the subtle sonic matter (\skt{bindu}) in the centre of the pericarp. In the centre of the subtle sonic matter (\skt{bindu}), there is the subtle sound (\skt{nāda}). How is that subtle sound (\skt{nāda}) divided?}





\slokawithoutfn{10.29}{ Divided by the sound U and the sound MA, the subtle sound (\skt{nāda}) departs. Realizing that [subtle sound], O Viśālākṣi, one can obtain immortality.}




\begin{center}{{[Setubandha]}}\end{center}




\slokawithfn{10.30}{ I shall teach you Setubandha, [which sports] a current whose water of subtle sound (\skt{nāda}) cleanses you of the dirt of your sins. The banks [of this river] are the tongue, the throat and the chest, its sandy beaches are the host of gods, it roars with whirlpools and is wavy. It's full of the roar of Ganges crocodiles and full of fish, ten types of sea-monsters [also: makāra?], terrifying alligators and with \skt{visarga}[?] Go to Setubandha, [the pilgrimage place that] tastes like the pleasure of intoxication in the deep ...}
{ Note that \skt{°kaṇṭhora} is a conjecture based on the context: this line                         probably talks about sounds and the production of sounds. For this                          \skt{uraḥ}/\skt{ura} (`chest') seems better that \skt{ūru} (`thigh'). }




\begin{center}{{[Suradraha]}}\end{center}




\slokawithoutfn{10.31}{ O Moon-faced goddess, listen to [the description of Suradraha], the way to the cessation of all sorrow, in the centre of the seven islands. It is frequented by Īśāna, it's a spotless lake in the heart full of the cool water of sound (\skt{nāda}). There is a lotus arising, with Prakṛti as its petals, and divided by its Śakti filaments. It is praised by the five voids, it is the path to the supreme level, and it is to be served if one wishes to obtain [heaven].}




\begin{center}{{[Ghaṇṭikeśvara]}}\end{center}




\slokawithoutfn{10.32}{}




\begin{center}{{[Vāgīśvaratīrtha]}}\end{center}




\slokawithoutfn{10.33}{}





\slokawithoutfn{10.34}{}



\vfill\pagebreak

\thispagestyle{empty}\addcontentsline{toc}{section}{Chapter 11}
\begin{center}
{\large{Chapter Eleven}}
\end{center}




\begin{center}
{{[The regulations on the Dharma of the four āśramas]}}
\end{center}




\slokawithfn{11.1}{ The Goddess spoke: O Paraśreṣṭha, O Surottama! Is there another [form of] universal sacrifice, which is free of pain, which is easy, and which does not require an abundance of materials, O Īśvara?}
{ I understand pāda c as containing a sandhi bridge \skt{alpakleśa-m-anāyāsa}. }





\slokawithoutfn{11.2}{ For the benefit of mankind, teach me, O Suraśreṣṭha, how one obtains the fruits of [this] universal sacrifice, [a sacrifice] praised even by the gods.}





\slokawithfn{11.3}{ Maheśvara spoke: I cannot see anything comparable to your compassion towards living beings, O Bhāminī. What else could I teach concerning which there is no compassion [in you towards living beings]?}
{ I understand \skt{dayā} in pāda b as instrumental: \skt{tava dayayā bhūteṣu na tulyaṃ paśyāmi}. }





\slokawithoutfn{11.4}{ I heard [this] previously from Sadāśiva's mouth, O Varasundarī. Listen, O Goddess, I shall teach you the ultimate essence of Dharma.}





\slokawithoutfn{11.5}{ Immaterial sacrifice satisfies all desires. It is undecaying and imperishable, and it removes all sins.}





\slokawithfn{11.6}{ Material things present many kinds of obstacle and [their acquisition causes] great fatigue, similarly to Indra's murder of the Brahmin [Viśvarūpa], which yielded results that were distributed [among trees, lands etc.].}
{ Context: Viśvarūpa was a son of Tvaṣṭṛ. Viśvarūpa's heads were struck off by Indra.                           In the Bhāgavatapurāṇa, Indra's sin are distributed among the ground,                           water, trees and women. }





\slokawithoutfn{11.7}{ Material sacrifice can be purified by the five purifications, O Varānanā. If it is purified, then the fruits will also be pure. If it is not purified, there is no fruit.}





\slokawithoutfn{11.8}{ The Goddess spoke: I am not sure about the five purifications, O Suraśreṣṭha. Please teach [them to] me one by one, I want to hear [them] as [they] really [are].}





\slokawithoutfn{11.9}{ Rudra spoke: The first is the purification of the mind, then comes the purification of the substances. The third is the purification of the mantras. The next one is the purification of the ritual. The fifth is the purification of Sattva. The purification of the sacrifice is [thus] fivefold.}





\slokawithoutfn{11.10}{ The purification of the mind is [achived] by mentally creating what is not wrong. The purification of the substances is [achieved] by [using] substances that were not obtained by unlawful means.}





\slokawithoutfn{11.11}{ The purification of the mantras is [achived] by [properly] joining vowels to consonants. The purification of the ritual is [achived] by not altering the proper sequence. The purification of Sattva is [achived] by the non-prevalence of Rajas and Tamas.}





\slokawithoutfn{11.12}{ When he has purified the ritual (\skt{vidhi}) thus and performs the sacrifice, he will obtain the fruits of the sacrifice, and will not experience birth and death [again].}





\slokawithoutfn{11.13}{ But he who performs immaterial sacrifice, O Varasundarī, will not obtain [only] its fruits, [but] of all sacrifices, without exception.}





\slokawithoutfn{11.14}{ His sacrificial ground is Kurukṣetra, he has made his abode in the house of Truth/Sattva. His great altar is the withdrawal of the senses. His seat of kuśa grass is self-control.}





\slokawithfn{11.15}{ The injunction is the various .. . He lights the fire of meditation which is flaring up by the fuel of the firewood of yoga and is abounding in the smoke of penance.}
{ Consider emending °\skt{samijjvāla}° to °\skt{samujjvāla}°, which would stand metri causa for °samujjvala°. }





\slokawithoutfn{11.16}{ The placing down of the chalice is knowledge about Śiva. [The oblation of] boiled rice is directed towards[?] Śiva. The continuous oblation of clarified butter is poured with the ladle of Lambaka [uvula, lambikā?].}





\slokawithfn{11.17}{ Transforming concentration into an Adhvaryu [priest], breath control will be the [other] priests. Samādhi which involves Tarka and which is long is the burning of the oblation[? vayas-tāpana?].}
{ Understand: dhāraṇām adhvaryuvat kṛtvā (dhāraṇā is a stem form noun). }





\slokawithfn{11.18}{ The sacrificial post is made up of the knowledge about Brahman. The tying of the sacrificial animal is [the mental state called] Manonmanas. His wife is Faith, O Viśālākṣī. His sacrificial ritual intention/declaration is the eternal abode.}
{ Understand: padaṃ śāśvatam (pada is a stem form noun metri causa). }





\slokawithfn{11.19}{ Rice oblation is the consumption of the nectar of immortality that arises from the victory over the five senses. The great mantra is Brahmā's sound. Expiation is the victory over breath.}
{ Perhaps \skt{brahmanāda} in pāda c refers to the same concept as \skt{brahmabilasvara} does in                          11.29d. }





\slokawithoutfn{11.20}{ The consumption of Soma is complete knowledge. The commencement [of the reading of the Veda] is the four yama-rules[?]. The ritual water-bath is [the reading of] the epics. His garment is made of [his readings of] the Purāṇas.}





\slokawithoutfn{11.21}{ Ritual bathing and sipping water once are [to be performed] at the confluence of the Iḍā and the Suṣumnā [i.e. at the internalized Prayāga, see 10.17]. Having honoured Contentment as a guest, he salutes the Brahmin that is now Compassion.}





\slokawithfn{11.22}{ The Brahmakūrca [penance] is the Guṇātīta [state of mind], the scent of the sacrifice is the Nirañjana [state of mind]. [His] sacred thread is the three Tattvas. For a shaven head he has enlightenment/teaching.}
{ On the guṇātīta state of mind, see 9.39--43.                  Understand guṇātītatvaṃ and nirañjanatvaṃ? }





\slokawithoutfn{11.23}{ The four Vedas are Nivṛtti etc. His seat is the four Prakaraṇas. He should always perform a sacrifice donating the priestly fee of providing being[s] with freedom from danger.}





\slokawithoutfn{11.24}{ The attainment of non-material sacrifice has been taught to you, O Varānanā. [The sacrificer] will in any case obtain the fruits of up to a thousand [ordinary] sacrifices.}





\slokawithoutfn{11.25}{ The first life-stage [life option] has been taught to you, O Varānanā, the true Dharma, which is revered by Sadāśiva and also by the [other] gods.}





\slokawithoutfn{11.26}{ [Now] learn about brahmacarya. Listen with attention, O Śubhā. [This is] the second life-stage, O Devī, the destroyer of all sins.}





\slokawithfn{11.27}{ [Here] religious observance is [now] meditation on Brahman. The Sāvitrī [hymn] is absorption in Prakṛti. The Brahmanical cord is the subtle syllable. His girdle is now contained in the three guṇas.}
{ One could emend \skt{prakṛtir layam} in pāda b to \skt{prakṛtau layaḥ},                 but I retained the reading of \msCa\msNa\msNc\Ed because                 it may have been the original way to make the compound \skt{prakṛtilaya}                 metrical. In other words, I suspect the \skt{-r-} to be only a link                 between the two elements of this compound. I also retained the neuter ending.                 Note 16.8d, where the same expression becomes \skt{prakṛtālayam}. }





\slokawithoutfn{11.28}{ His staff is self-restraint, his bowl compassion. Begging/alms? is liberation from saṃsāra. The tryāyuṣa [mantra] is the one beyond the two syllables[?]. It[?] is embellished with the ashes of knowledge.}





\slokawithfn{11.29}{ The bath-vow is speaking the truth always. It is accompanied by the purity of moral conduct. Sacrifice to Agni is the three tattvas[?]. Recitation is the sound at the aperture of Brahmā.}
{ Perhaps \skt{brahmabilasvara} in pāda d refers to the same concept as \skt{brahmanāda} does in 11.19c. }





\slokawithoutfn{11.30}{ [This is] the second life-stage as Lord Śiva taught it, O Devī. I have also taught [it to] you[,] the destruction of birth and death.}





\slokawithoutfn{11.31}{ Listen, O Long-eyed goddess, I shall teach you the forest-dweller's way of life, which is revered by the Ṛṣis and the gods, as I heard it, as it [really] is.}





\slokawithoutfn{11.32}{ Having taken to the forest of indifference, he should take residence in the Āśrama of niyama-rules, within walls that have the stone-strong gate of moral conduct, with his sense faculties conquered.}





\slokawithoutfn{11.33}{ One's mother is the material realm, one's father the supreme spirit. the divine realm is one's teacher, determination one's brothers.}





\slokawithoutfn{11.34}{ His wives are Śruti and Smṛti, his son is Wisdom, his younger brother Patience. His relative is Benevolence, his twisted hair is his bow, Compassion his sacred thread.}





\slokawithoutfn{11.35}{ Sympathy is the four ways of taciturnity. All his duties are Indifference. He has the yama-rules for a garment made of bark, and he wears Penance for the skin of a black antelope.}





\slokawithfn{11.36}{ He is seated on the highest level of non-attachment, and the firm observance is his yoga-belt. Fire sacrifice accompanied by he sound of murmuring the Vedas is breath-control accompanied by the hissing [of breathing].}
{ hāvana = havana metri causa }





\slokawithfn{11.37}{ He is full of[??] conquered breaths for a deer[??]. [For him] sacrifice is resolution, ritual is recitation. His collection of wealth is in the \skt{śāstra}s, his companons are self-control, compassion etc.}
{ °mṛgākūla for °mṛgākulaḥ metri causa?  See \skt{saṃgraha} used probably in a similar sense in 11.46. }





\slokawithoutfn{11.38}{ He should perform sacrifice to Śiva [with/as?] the worship of the eight [yogic?] practices. He is purified by the water of the five Brahma[-mantras] in the auspicious [\skt{śiva}] pool on the sacred banks of truthfulness.}





\slokawithoutfn{11.39}{ Having bathed and having sipped water [there], he should take honour the three junctures of the day. His rosary is the meaning of the Purāṇas. The pacification of mantras? is? recitation day and night.}





\slokawithfn{11.40}{ His jar of epics is filled with the water of knowledge. [Tentatively:] The actions of the five [medical] procedures are suicide. The five kinds of pleasure are recitation.[?]}
{ pūrṇa-m-itihāsa°: -m- is a filler.  Note that \skt{utkrānti} is a \skt{yogāṅga} in chapter 16. }





\slokawithfn{11.41}{ The Śivasaṃkalpa [hymn] is practice (sādhana), which yields fruits of yoga accomplishments. His food is the fruit of Contentment. He conquered lust and anger.}
{ The Śivasaṃkalpa is Ṛgvedakhila 4.11 ff:         yenedam bhūtaṃ bhuvanaṃ bhaviṣyat parigṛhītam amṛtena sarvam,         yena yajñas tāyate saptahotā tan me manaś śivasaṅkalpam astu, etc.          See also Manu 11.251ab: sakṛt japtvāsyavāmīyaṃ śivasaṃkalpam eva ca. }





\slokawithfn{11.42}{ His practice is the victory over the trap of hope. He prefers the joy of yoga meditation. The forest-dweller should observe his vow by providing his guests with fearlessness. This is how the Dharma of the forest-dweller has been taught and followed in the past.}
{ Gender! }





\slokawithfn{11.43}{ [The yogin] should follow, with faith and self-control, the supreme Dharma, which delivers him from Saṃsāra, removes transient existence, uproots ignorance, increases wisdom, is fruitful, delivers cross him from the flood of affliction, removes rebirth, disease and burns his bad karma.}
{ \msNa\ only corrects °haraṇamanitya° to °haraṇam anitya° (CHECK this),                 but its scribe probably meant an anusvāra at the end of °haraṇaṃ,                 perhaps trying to correct the metre. He tries to correct the metre                also with anityaharaṇan tajñā°. }




\begin{center}
{{[The wandering mendicant]}}
\end{center}




\slokawithoutfn{11.44}{ Here follows the a wandering religious mendicant's Dharma. Listen, I shall teach you about it. Making joy and pain equal, he gets rid of greed and folly.}





\slokawithoutfn{11.45}{ He should avoid honey and meat, as well as others' wives. He should avoid staying [in a place] for long and also staying at others' places.}





\slokawithfn{11.46}{ He should avoid food that has been thrown away and he should avoid a single alms round[?? the same food?]. He should always refrain from accumulating wealth and from self-conceit.}
{ See the term \skt{arthasaṃgraha} in 11.37c }





\slokawithoutfn{11.47}{ Meditating on the subtle he can put his feet into the pure.[??] He should not get angry when [food] in not available, and when it is, he should not rejoice.}





\slokawithfn{11.48}{ He should not be agitated with regards to thirst for material things or to violent anger. He should take praise and reproach equal, as well as pleasant and unpleasant things.}
{ In pāda c, understand \skt{stutinindā} as a dual accusative. }





\slokawithfn{11.49}{ His garment is the Niyama-rules, and he is girded by the girdle of self-control. He makes his mind supportless, his intellect spotless,}
{ Check if saṃyama is a technical term here. }





\slokawithfn{11.50}{ his self Earth, the Manonmana ether[?], his three staffs [of the Parivrājaka] the three guṇas, his bowl the imperishable syllable.}
{ \skt{°kṣaram avyayam} in pāda d would be unmetrical, that is why the nominative appears here. }





\slokawithoutfn{11.51}{ He should throw away [the distinction between?] Dharma and Adharma, and should avoid envy and hatred. He is indifferent to the opposites [such as cold and heat, good and bad], dwells always in truthfulness, unselfish, humble.}





\slokawithoutfn{11.52}{ He should go on his alms round visiting seven houses at the eighth part of the day. He should not sit down, he should not stay, and he should not say `Give me!'.}





\slokawithoutfn{11.53}{ He should live on what is available, on eight bites a day. He should not stick to items of clothes, food or a bed for long.}





\slokawithoutfn{11.54}{ He should nor rejoice in death, he should not rejoice in life. Having conquered his senses, having killed his desire, firm in his observances,}





\slokawithoutfn{11.55}{ the Bhikṣu should never think about the past or the future. The wandering mendicant should always avoid anger, self-conceit, intoxication and pride.}





\slokawithoutfn{11.56}{ Making indifference a bow which is strung by the strings of breath-control, he should kill the beast that is the mind and the sense-faculties with the sharp-pointed arrow of concentration.}





\slokawithfn{11.57}{ He should stab the enemy that is Saṃsāra with the extremely sharp knife of friendliness. He should defeat the rutting elephant of anger with the whirling discus of compassion.}
{ Note the Buddhist terms \skt{maitrī} and \skt{karuṇā} in this verse. }





\slokawithfn{11.58}{ His body is clad in the armour of sympathy, his quiver is full of indifference. He should constantly recall the unutterable syllable which is supreme Brahman, O Brahmin.}
{ Note the Buddhist terms \skt{muditā} and \skt{upekṣā} in this verse. }





\slokawithoutfn{11.59}{ Brahmā's heart is Viṣṇu. Viṣṇu's heart is Śiva. Śiva's heart is the Junctures of the day. Therefore he should worship the Junctures.}





\slokawithfn{11.60}{ [Śiva] is deliverance from the ocean of mundane existence, the path to happiness, the Brahman, the junctures, the [sacred] syllable. [the yogin] should always, unweariedly, meditate on matchless Śiva, who is to be recognized as the manifested soul. He should take refuge in Hara, who is devoid[!] of form, colour, qualities etc., who is the supreme aim which is difficult to discern, ... , the divine guru, who removes all pain.}
{ vihita here in the sense of `devoid'. }



\vfill\pagebreak

\thispagestyle{empty}\addcontentsline{toc}{section}{Chapter 12}
\begin{center}
{\large{Chapter Twelve}}
\end{center}




\begin{center}
{{[The rules of hospitality]}}
\end{center}




\slokawithoutfn{12.1}{ The Goddess spoke: Harmlessness is always praised as the highest Dharma. Also, teach me the ultimate Dharma of those who practise hospitality.}





\slokawithfn{12.2}{ Maheśvara spoke: Hear the ultimate Dharma of the harmless ones and that of the ones who practise hospitality. O beautiful-eyed goddess, [if] all the three worlds, full of wealth,}
{ Understand \skt{ahiṃsātithyakāmāṃ} as \skt{ahiṃsakānām ātithyakānāṃ ca} }





\slokawithfn{12.3}{ [were handed over as] a gift to [a Brahmin who] knows the four Vedas, [that gift] cannot be compared to somebody who avoids doing harm. Hear the Dharma of the hospitable ones. I'll teach it [to you], O beautiful one.}
{ Note that this verse seems to be all that Maheśvara teaches in this chapter on                  \skt{ahiṃsā} and that \skt{tattulyam ahiṃsakaḥ} may contain a sandhi bridge:                 \skt{tattulya-m-ahiṃsakaḥ}  \skt{atithyānāṃ} in pāda c stands for \skt{ātithyānāṃ} or \skt{ātithyakānāṃ} metri causa. }




\begin{center}
{{[The Story of Vipula]}}
\end{center}




\slokawithoutfn{12.4}{ This is an old story of what happened once in a city called Kusuma [i.e.\ Pāṭaliputra]. There was a famous and wise man called Vipula, Kapila's son.}





\slokawithoutfn{12.5}{ He always followed his Dharma, he conquered anger, he spoke only the truth and he conquered his senses. He was friendly to Brahmins. He was grateful and he was my determined devotee.}





\slokawithoutfn{12.6}{ He was rich and he worshipped[?] his guests. He was generous, restrained, and merciful. He wealth always came through just means. He always stayed away from illegal actions.}





\slokawithoutfn{12.7}{ He had a beautiful wife whose face was as pure as the disk of the moon. Her breasts were round and elevated, she was lovely, a source of all pleasure. She was faithful, devoted to her husband and his needs.}





\slokawithoutfn{12.8}{ Now, once there was an eclipse of the sun. Three quarters [of the sun] were eclipsed, and it was in the dark half of the month of Mādhava.}





\slokawithoutfn{12.9}{ Eager to take a ritual bath, the king and all citizens went down [to the river]. They were worshipping the gods and the deceased ancestors according to rule.}





\slokawithoutfn{12.10}{ Some sacrificed in the fire, some fed the Brahmins, some gave donations, others praised the deity.}





\slokawithoutfn{12.11}{ Some people practised yoga meditation, others were engrossed in five-fire penance. While all the royals and other people were doing this all around the place,}





\slokawithfn{12.12}{ Vipula too, there at the confluence of the Gaṅgā and the Gaṇḍakī, together with his wife, performed a bath, and, attired in linen clothes,}
{ Note \skt{gaṇḍaki} metri causa for \skt{gaṇḍakī} in pāda b. }





\slokawithoutfn{12.13}{ was satiating the deities, the gurus, the Brahmins and others. Then, jumping on the possibility, a Brahmin came up [to them] as a guest.}





\slokawithfn{12.14}{ The wife got infatuated with that Brahmin's extreme beauty. The Brahmin [felt] the same. His beauty was unparalleled.[?]}
{ Pāda d is slightly suspect and the translation of pādas cd is                         tentative. The expression \skt{rūpeṇāpratimo/°pratimā bhuvi} is                          common in the Mahābhārata and in the Purāṇas. Is that what was meant here?                         May a dual have been intended? }





\slokawithoutfn{12.15}{ Their gaze got fixed on each other mutually. Vipula joined his hands [and said:] ``O virtuous Brahmin,}





\slokawithoutfn{12.16}{ I am at your service, be gracious to me now, O great Brahmin. [My] wife, servants, cattle, village and all kinds of jewels [are all at your service].''}





\slokawithfn{12.17}{ Having been addressed and greeted hospitably by Vipula, the Brahmin spoke: ``If you really mean to give, your heart is very generous.''}
{ Note that \msCc's omission here is probably due to an eyeskip from \skt{suprasannaṃ} in                 12.17d to \skt{suprasannaṃ} in 12.18a, although this would have lead to an omission of                 the next \skt{vipula uvāca}. }





\slokawithoutfn{12.18}{ Vipula spoke: ``My heart is generous, generousity is the fruit of austerity. Just command me quickly, O Brahmin. What is your desire? There is nothing that should not be donated to a Brahmin, beginning with one's own head, O Brahmin.''}





\slokawithfn{12.19}{ The Brahmin spoke: ``If you talk like this, my dear, give me your beautiful wife. Be happy, may you be fortunate, and may you prosper eternally!''}
{ In pāda d, \skt{bhava} is less than satisfactory. One would normally expect                  \skt{bhavate/bhavatāṃ/bhavatu} in this context. Alternatively, it is possible                 \skt{kalyāṇo bhava} (`be happy') was meant or we could accept \Ed's reading. }





\slokawithoutfn{12.20}{ Vipula spoke: ``Accept my wife who has nice buttocks, and is young and beautiful, blameless, large-eyed and whose face resembles the full-moon.''}





\slokawithfn{12.21}{ The wife spoke: ``How can you abandon me, my lord? How can you leave somebody who is sinless? How can you abandon a wife who is extremely kind and faultless?}
{ sa is problematic CHECK accept tyajet? }





\slokawithoutfn{12.22}{ A wife is a man's friend in this world and in the other world. [Even if] a man gives enormous donations or performs numerous sacrifices,}





\slokawithoutfn{12.23}{ or performs hard penance, he cannot get to heaven without having a son. I have heard that this was taught by the ancestors, and by Brahmins in my presence.}





\slokawithoutfn{12.24}{ The sonless cannot obtain heaven. I have heard this so many times! Mandapāla, the great Brahmin, went to heaven as a reward of his austerities.}





\slokawithfn{12.25}{ That great Brahmin made numerous donations, performed various sacrifices, [recited] the Vedas, and performed sacrifices of recitation.}
{ I have taken \skt{japayajñāṃś} in pāda c as a \skt{tatpuruṣa} compound.                  The same expression occurs e.g. in VSS 6.2ff, MBh 13.102.8c, Manu 2.86 etc.                 By this, \skt{vedāṃś} becomes difficult to interpret (I supply `recited'). It may be possible to take                 \skt{japa} as a form deriving from \skt{japan} (present participle) metri causa:                         \skt{vedāṃś ca japa}[\skt{n}] \skt{yajñāṃś ca kṛtvā}, but in this case                          the notion of performing sacrifices comes up twice in this verse. }





\slokawithoutfn{12.26}{ But when he reached the gate [of heaven], it was blocked by the celestial messengers: ``The sonless cannot get to heaven, not even by hundreds of sacrifices.''}





\slokawithoutfn{12.27}{ Mandapāla, the great sage was thus informed and he fell from heaven. The Brahmin begot four sons with a Śāraṅga-bird.}





\slokawithfn{12.28}{ By the virtue of this, he reached heaven unobstructed. I am a wife (\skt{kalatra}) [because] I protect the family (\skt{kulatrāṇa}), and I am a wife to be supported (\skt{bhārya}) because I bear [sons] (\skt{bharaṇa}).}
{ Note that pāda c is the result of emendations and that \skt{bhārya} in pāda d is                 to be understood as \skt{bhāryā} metri causa (nevertheless I supplied `to be supported'                 in the translation to convey the general meaning of the word \skt{bhārya},                  which seemed to fit the context well). }





\slokawithoutfn{12.29}{ Taking a wife is for the sake of having sons according to the Śāstras. You can give that Brahmin all the wealth at home, all the villages, the stations of herdsmen and the houses,}





\slokawithfn{12.30}{ but please don't give me away this time!'' Having heard his wife's speech, Vipula spoke again.}
{ I have not included \msCcpcorr's \skt{vipula uvāca} (echoed in \Ed)                 because after \skt{punar abravīt} is seems secondary and unnecessary.                 Note that the correction in \msCc\ is in a second hand. }





\slokawithoutfn{12.31}{ ``Alright, my beautiful wife, I know! Good, good, my faithful wife! I am beaten by this speach and I am satisfied with it.}





\slokawithoutfn{12.32}{ Today the Brahmin came up to me at the time of eclipse, and he asked me. I promised him that I would give [you away]. If I don't give [you to him], I shall go to hell.}





\slokawithfn{12.33}{ If I go to hell along with my family/decendants, I will not see release from hell, O brilliant woman, for millions of eons,}
{ The reading \skt{narakastho} (\msNc\Ed) is tempting but it could be a scribal correction and                 \skt{narakasthād} may be original, meaning \skt{narakasthānād}. }





\slokawithoutfn{12.34}{ as long as millions of births. I can see something bad, my Princess, from not giving, O woman with a nice compexion,}





\slokawithoutfn{12.35}{ but from giving I can see something good in heaven that is eternal. I have never ever lied, I always observe the vow of truthfulness.}





\slokawithfn{12.36}{ If I transgressed the law of truth, [by this] I would stop following all other laws [too]. You mentioned earlier that the wife is one's Dharmic friend.}
{ I have emended \skt{tvayi} in pāda d to \skt{tvayā} because it                  seems an early random scribal mistake, rather than some                  linguistic pecularity. }





\slokawithoutfn{12.37}{ If you are indeed my Dharmic friend, then now the time has come. Dharma himself has visited us disguised as a Brahmin.}





\slokawithfn{12.38}{ to test me. O my dear, please don't cause me trouble. The Unmanifest (Prakṛti) is my mother, Brahmā is my father, Intelligence is my wife, self-control is my friend.}
{ In pāda a, \skt{ahaṃ} either stands for \skt{māṃ} or the phrase \skt{jijñāsārtham ahaṃ} can be                  translated as `I am to be tested.' }





\slokawithfn{12.39}{ Dharma is my son, Ritual is my guru. These are my relatives. The best time is the time of the eclipse of the Sun. The best one among the rivers is the Gaṅgā.}
{ I understand \skt{grahaḥ sūryo} in pāda c as \skt{sūryagrahaḥ} (or \skt{sūryagrahaṇam}):                the eclipse of the Sun, which appears to be an auspicious day. See parallels in the apparatus. }





\slokawithfn{12.40}{ The best day is at new moon, the best man is the Brahmin. I have given you to the Brahmin to serve him. Having given everything to the Brahmin, I'll resort to the forest.''}
{ In pāda f, \skt{brāhmaṇe} (loc., in all the witnesses that I have consulted)                                  may have originally read \skt{brahmaṇe} (dat.). }





\slokawithoutfn{12.41}{ Śaṅkara [i.e.\ Śiva] spoke: The wife remained silent, her eyes filled with tears. [Vipula] took her hand and the long-eyed woman was presented to the Brahmin.}





\slokawithoutfn{12.42}{ I am ready to give you all the wealth I have at home, all the gold and the cattle, O great Brahmin, the village, the stations of herdsmen and the houses, and everything else,}





\slokawithoutfn{12.43}{ pearls, gems, clothes and divine ornaments. Accept all these, O best of Brahmins. It's given in good faith and with respect.}





\slokawithfn{12.44}{ May Lord Dharma be pleased and may Maheśvara be pleased. May all the ancestors rejoice if there is reward for meritorious acts.}
{ Note Śivadharmaśāstra 10.11cd, in a similar context of donations:                 \skt{bhojayitvā tato brūyāt prīyatāṃ bhagavān śivaḥ}  Understand \skt{sukṛtaṃ phalam} as \skt{sukṛtaphalam} (metri causa). }





\slokawithfn{12.45}{ Rudra spoke: Having heard Vipula's speech, the ascetic Brahmin blessed the good-souled Vipula a good number of times,}
{ There are several ways to explain the form \skt{āśīḥ} in pāda c.               The easiest is to treat it as a singular accusative neuter.               Alternatively, it could be a plural accusative feminine from \skt{āśī} and               then \skt{suvipulaṃ} is either to be understood adverbially or as \skt{suvipulā}[\skt{s}].               Another way to treat \skt{āśīḥ} would be to take it as a nominative standing               for the accusative. }





\slokawithoutfn{12.46}{ and then went off to live in a nice house, taking Vipula's wife with him. As for Vipula, he said good-bye and circulambulated him.}





\slokawithoutfn{12.47}{ Thus saluting the Brahmin, he departed quickly into the forest. In the forest he lived off roots and fruits and roamed about in the world.}





\slokawithoutfn{12.48}{ But being alone in an abandoned and deserted place, he got overwhelmed with worry. Where should I go? Where should I look for food? From whom? What shall I do?}





\slokawithfn{12.49}{ I don't know these roads, this country, these villages and these cities, towns, mountain settlements. I don't know anybody here.}
{ In pāda d, the reading of all the witnesses, \skt{kaścana}, seems to be               an early scribal mistake for \skt{kañcana}. But note that the same happens at                 12.55d. }





\slokawithoutfn{12.50}{ I can see a nice mountain there with large cavities and caves. I'll climb it and try to figure out if there is a village, town or city [nearby].}





\slokawithfn{12.51}{ Having said this, Vipula climbed the mountain slowly. He caught sight of the shades of a tree, and being exhausted sat down [there].}
{ I have accepted the reading (emendation?) of \Ed in pāda d (\skt{āruhat})               because I think that \skt{āruhet} is an early scribal mistake that               is easy to make and because \skt{°āruhat} comes up again in 12.53d. }





\slokawithoutfn{12.52}{ In the same moment, descending from among the branches of the tree, [a monkey appeared and] carrying an extraordinary, beautiful, fragrant, excellent,}





\slokawithfn{12.53}{ lovely, delightful and pleasant-looking fruit, it put it in front of Vipula and then returned to the tree.}
{ Note how the agent of this sentence is omitted here. That it was a monkey               that gave Vipula the fruit becomes clear in 12.94. }





\slokawithoutfn{12.54}{ Vipula, seeing this wonder, was perplexed. Am I sleeping or is this the fruit of my penance?}





\slokawithfn{12.55}{ I have never seen, smelt, tasted anything like this. I have not even heard of anything like this. I shall let somebody know about it.}
{ I suspect that \skt{śrotā} in pāda c is meant to be feminine participle \skt{śrutā}, but               the metre required the first vowel to be lengthened; understand \skt{me} as \skt{mayā}.               In pāda d, the reading of all the witnesses, \skt{kaścana}, seems to be               an early scribal mistake for \skt{kañcana}. But note that the same happens at                 12.49d. }





\slokawithoutfn{12.56}{ Having said this ... , taking that nice fruit, he kept observing its smell again and again.}





\slokawithoutfn{12.57}{ ``Examining the fruit, ... seeing this country, I have run out of provision, and this fruit must have been sent to me by a god.}





\slokawithoutfn{12.58}{ Therefore, I shall take this fruit and go to that city, and I shall go and seek something to live on.}





\slokawithoutfn{12.59}{ Then leaving that mountain behind, he entered the city. He asked a man on the road what the name of this city was.}





\slokawithfn{12.60}{ That traveller replied: ``Have you never been here? This is the Deccan region, and this is the city of Naravīra.}
{ I understand \skt{pathīkena} as standing for \skt{pathikena} metri causa (see 12.64b) and not               as two words, \skt{pathī kena}. This means that we are forced to accept an instrumental as the agent                of the finite verb \skt{uvāca}. I suspect that \msNc's reading (\skt{pathīko})                is an attempt to correct the syntax, but in this way \skt{apūrvam} becomes                problematic. With \skt{pūrvam} tha sentence may mean: `The traveller replied:                 ``Have you not come here before?'' '  \skt{ayam} as the end of this verse may have been the original reading and                 \msCb\ may have corrected it to \skt{adaḥ}. Another possibility is that                 an original \skt{adaḥ} is preserved in \msCb, and it got corrupted to                 \skt{ayaḥ} (\msCa), and then to \skt{ayaṃ} (\msCc\msNa).                  In any case, I have chosen the reading \skt{adaḥ}                 because it works better; it can be viewed as my editorial correction. }





\slokawithoutfn{12.61}{ The king is called Siṃhajaṭa, his queen is Kekayī. The king is very old, afflicted by old age. The queen likewise.}





\slokawithoutfn{12.62}{ He is generous and he is an expert in the arts and he possesses the power of heroism in battle. He is pious and devoted to his subjects and he is well-versed in the Śāstras.''}





\slokawithfn{12.63}{ Vipula spoke: ``As a matter of fact, I am seeking audience with the foreman of the guild (\skt{śreṣṭhi/śreṣṭhin}). What is his name? Tell me. In which district is his dwelling? Tell me without any hesitation.''}
{ Note the form \skt{śreṣṭhiṃ} from the stem \skt{śreṣṭhi} instead of \skt{śreṣṭhin} (thematisation). }





\slokawithfn{12.64}{ Having been addressed by Vipula thus, the traveller spoke to him again: ``My name is Bhīmabala and I have come to visit the house of the foreman of the guild.}
{ Note the stem form \skt{pathika} in \skt{pathikovāca} in pāda b. Alternatively,               it is an instance of double sandhi (\skt{pathika uvāca} - \skt{pathikovāca}) }





\slokawithoutfn{12.65}{ The foreman of the guild is called Puṇḍaka and he is said to be a famous foreman. If you are eager [to see him], come with me.''}





\slokawithoutfn{12.66}{ ``Alright, let it be.'' Great-souled Vipula spoke thus to him, and he set off to visit the foreman's house together with Vipula.}





\slokawithoutfn{12.67}{ When Vipula saw the foreman who was sitting in his house, he went up to him and offered him that fruit.}





\slokawithfn{12.68}{ ``Wow, what an excellent fruit! And hey, it has been brought here. Wow, what a form, what a smell, wow what a splendid fruit!}
{ Note \skt{ihānitam} for \skt{ihānītam} in pāda b for metrical reasons. }





\slokawithfn{12.69}{ This fruit was not produced on earth, not even on Mount Meru or ... It is clearly from the world of gods, [this kind of fruit] does not grow in the world of humans.}
{ Most probaby, \skt{kandare} (`in a cave') in pāda b is an early                  mistake for \skt{mandare} (`on Mount Mandara'), a location that                  appears frequently in the epics and the Purāṇas next                  to Mount Meru. This is why I conjecture \skt{mandare} here.  Understand \skt{devalokika} in pāda c as a stem form compound (metri causa) for a more standard               \skt{devalaukikaṃ}.               \skt{martya-m-upajāyate} in pāda d might be original, with \skt{m} as a sandhi bridge. Nevertheless,               I emended the pāda to make it clearer. }





\slokawithfn{12.70}{ Ah! I will enjoy [its] profits. It is fit for a king. Offering this divine fruit to the king, I shall please him.''}
{ Pāda a is slightly suspect. It is possible that originally it contained a                negation: \skt{aho 'smi na phalaṃ bhoktā} (`Ah! I will not eat this fruit').               On the other hand, \skt{saphala} seems to be an odd form in this text simply               meaning \skt{phala} (see 12.71--72, 108).               The translation I have chosen is tentative. }





\slokawithfn{12.71}{ Then grabbing that pleasant fruit, he left hastily. He approached the king respectfully, and gave him the fruit.}
{ In pāda a, \skt{tvarita}, for the adverb \skt{tvaritaṃ}, is in stem form metri causa.  As in 12.70, \skt{sa phala}, or rather \skt{saphala} might simply mean \skt{phala}.                  Here in pāda d I have chosen to print this phrase as two words because here                 \skt{sa} can be grammatically/syntactically correct. See also next line (12.72a). }





\slokawithfn{12.72}{ And seeing the fruit, the king was highly amazed. ``O foreman, from where have you brought this charming fruit previously?}
{ On the possibility that \skt{saphala} is a form in this text simply signifying \skt{phala},                 see notes on 12.70 and 72.  \skt{pūrva}[\skt{ṃ}] in pāda d is suspect and difficult to interpret and                \Ed\ is probably trying to silently emend it.                 One possibility is that the pāda originally contained a stem form noun:                 \skt{phalāpūrvaṃ manoharam} (`an unparalleled and charming fruit').                 Alternatively, \skt{pūrva} is an eyeskip to 12.73b. }





\slokawithoutfn{12.73}{ I have never seen such a sweet root or fruit or bulbous root, one with such beauty, fragrance and qualities that gladden one's heart.}





\slokawithfn{12.74}{ I shall eat this fruit that you have given me instantly. What does it taste like? I want to know. Give it to me quickly.''}
{ I take \skt{svāda} as a stem form noun that stands for the accusative metri causa. }





\slokawithoutfn{12.75}{ Then he ate the fruit that looked like the nectar of immortality. The king devoured all of it and it tasted nice, like nectar.}





\slokawithoutfn{12.76}{ In an instant he obtained the youthfulness of a sixteen-year-old person. In a moment, there were no wrinkles and grey hair, no illness and no weakness.}





\slokawithoutfn{12.77}{ His hair, teeth and nails all became smooth and shiny, his teeth and senses strong, he regained his vital powers, his vision, strength and his life energies in a moment.}





\slokawithoutfn{12.78}{ The minister, the domestic chaplain, the counsellor, all the servants, the townswomen, and all the children and all the elderly people, everybody was amazed.}





\slokawithoutfn{12.79}{ The sovereign, king Siṃhajaṭa, became extremely satisfied and very happy.}





\slokawithfn{12.80}{ The king, who was selfish and cruel, spoke to that foreman of the guild: ``Tell Bhīmabala to bring another fruit today.}
{ The syntax of pāda c is confusing. I translate it as if it carried                  a causative meaning (e.g. \skt{kuru bhīmabalaṃ tv evaṃ}: `make Bhīmabala do like this').                 On the other hand, an instrumental would be better (`act like this, together with                 Bhīmabala'), at least 12.82b hints at this solution. }





\slokawithoutfn{12.81}{ I have regained my youthfulness by your kindness, O excellent man. Bring youthfulness also to Kekayī, who is weak and old.''}





\slokawithfn{12.82}{ The foreman and Bhīmabala were addressed by the king thus. [Bhīmabala] replied to the king, joining his hands reverentially and remaining standing with his head bowed down.}
{ I accepted the reading \skt{śreṣṭhī} in pāda b although it may be a                  correction of \skt{śreṣṭhi}, an original \skt{prātipadika} of the thematised                  form of \skt{śreṣṭhin} (see 1.63a). }





\slokawithfn{12.83}{ ``Your majesty, one cannot obtain [such a fruit by wondering] from forest to forest. It cannot be obtained through merchants or by cultivating the land. Some noble man who is seeking your audience}
{   Pāda a could be construed as \skt{na vane na vane rājan}                         (`Your majesty, there is no [such fruit] in any                         forest'), but a similar expression, \skt{vanena                         vanaṃ}, occurs e.g. in MBh                         1.144.1 meaning `from forest to forest' (\skt{te vanena                         vanaṃ vīrā ghnanto mṛgagaṇān bahūn\danda apakramya yayū                         rājaṃs tvaramāṇā mahārathāḥ\twodanda}), and this made me                         choose another option (\skt{na vanena vane rājan}). \Ed's variant                         seems like an attempt to `correct'                         the text.                  }





\slokawithoutfn{12.84}{ gave it to me, and, O king, I gave it to you, your majesty. Your majesty, I cannot tell you who this foreigner is.''}





\slokawithfn{12.85}{ Having heard Bhīmabala's reply, [the king] said: You are the son of a noble family of ministers.  Announce[?] my orders.}
{ Pāda a is unmetrical. It is possible the the original read \skt{°balaṃ} to avoid this,                 still meaning the compound \skt{bhīmabalavākyaṃ}. }





\slokawithfn{12.86}{ If there are no more, why did you give me one? This is what I request from you, sir. Where there is one, there are many, that is for sure.}
{ I have choosen \msCb's reading in pāda c only because it is metrical. This                 does not mean that the unmetrical reading of \msCa\msNa\msNc\ cannot have                  been the original one. }





\slokawithoutfn{12.87}{ [There is a] path by which[?] it arrived. One should go [back] by the same route. By all means, that's the way to go. Track it down by that route.}





\slokawithfn{12.88}{ If you are unable to provide another [fruit], I'll have your head cut off, you fool. Caṇḍa and Vicaṇḍa will slay [you]. Beware, vile Bhīmabala!''}
{ My impression is that Caṇḍa and Vicaṇḍa could be the two royal envoys mentioned                in verse 12.126 (\skt{rājadūtadvayam}), sent along with Bhīmabala to make sure he obeys the king's command.               Compare with Śivadharmottara 7.101 (Kenji and Sathya), where Yamas attendants are               called Caṇḍa and Mahācaṇḍa. }





\slokawithfn{12.89}{ Then Bhīmabala got angry, took his sword that looked like the [crescent] moon, and, obeying the king's orders, went to that son of a noble family [together with Puṇḍaka the foreman].}
{ The reconstruction of pāda d is unsatisfactory and I do not know               how to emend \skt{aram}/\skt{param} at the end of the line. We have to suppose               that Bhīmabala is accompanied by Puṇḍaka the foreman of the guild because               Vipula's answer seems to be directed towards him. }





\slokawithoutfn{12.90}{ O son of a noble family, don't take it as an offence, [but] I'll kill you unless you have more of this fruit. Give one to the king now!}





\slokawithfn{12.91}{ Reveal to me quickly where you found the divine fruit. Without that fruit, my friend, your life is in danger.''}
{ I conjectured \skt{tvaram} for \skt{tava} in pāda b because \skt{tava} is both               unmetrical and meaningless in this context. \skt{tava} might have               been the result of an eyeskip to pāda d or rather to pāda b of 12.92. }





\slokawithfn{12.92}{ Vipula spoke: I regained my hope for life [when I reached?] your house in this foreign country. How could one who does his duty be slain? I would obtain [another fruit] right now.}
{ The translation of pādas ab is tentative. If my interpretation is                correct, the house in question is Puṇḍaka's house.  Perhaps understand \skt{kṛtakartā} in pāda c as \skt{kṛtyakartā}. }





\slokawithoutfn{12.93}{ But there is no other fruit. Nobody can provide any. Up on the rocky peak[?] of Mount Sahya, I sat down, mentally exhausted.}





\slokawithoutfn{12.94}{ It was a monkey that took that fruit, gave it to me and then disappeared. I gave it to you, you gave it to the king.}





\slokawithfn{12.95}{ Let's go to that place, O foreman, to see if the monkey is there. When we get there together, we can ask the monkey king [for more fruit].}
{ I have accepted \msCb's reading in pāda d mainly because the reading               of all the other witnesses is difficult to interpret and because               a similar verb form, \skt{yācasva}, appears in 12.105d. }





\slokawithfn{12.96}{ The foreman said: ``Alright, let's go together to the place where you got that fruit. We shall be saved.''}
{ The foreman uses the plural in his reply correctly: he refers to               Vipula, Bhīmabala and himself. }





\slokawithoutfn{12.97}{ Rudra spoke: Climbing Mount Sahya, searching the place all over, Vipula then caught glimpse of the monkey, the monkey king.}





\slokawithfn{12.98}{ ``It's that extraordinary monkey there lurking in the shade of that tree. This monkey has showed up today merely by the force of my meritious act.}
{ The `meritious act' mentioned here is probably that of giving his wife to                  the Brahmin at the beginning of the story. }





\slokawithoutfn{12.99}{ Hey, monkey, unless you do me a friendly favour I'll perish very quickly. Give me another one of that fruit that you gave me, O monkey, [and thus] keep me alive.''}





\slokawithoutfn{12.100}{ The monkey spoke: It was a Gandharva that had given me the fruit and I gave it to you. How could I give you another one? Go there [where Gandharvas live] if you wish.}





\slokawithfn{12.101}{ Vipula spoke: ``If you cannot give me another fruit, [my] staying alive is doubtful. Another alternative is that we go where Citraratha himself[, the king of the Gandharvas,] dwells.''}
{ I suspect that \skt{tubhyaṃ} in pāda a is used in the sense of \skt{tvayā} and                 that is how I translate this phrase. I doubt if Vipula would                 threaten the monkey (`for you living becomes doubtful'). }





\slokawithoutfn{12.102}{ The monkey replied: ``Let's do it.'' Then, upon reaching the dwelling place of Citraratha and having gone up to him, he said this:}





\slokawithoutfn{12.103}{ ``O king of the Gandharvas, I have come back to you with a request. Give me another of that fruit that you gave me if you can.''}





\slokawithfn{12.104}{ The king of the Gandharvas spoke: ``I went to the world of Sūrya, and it was him who gave me that extraordinary fruit. I gave that fruit to you [because] you are my very best friend.}
{ Understand \skt{suhṛdo} in pāda d as a singular nominative of the rare \skt{suhṛda}. }





\slokawithoutfn{12.105}{ Where could I find another fruit to give you, I don't have one, O monkey. Let's go to the world of Sūrya and ask the Sun there.''}





\slokawithfn{12.106}{ Having been addressed thus by the Gandharva, the monkey consented. They reached the world of Sūrya all together, the Gandharva and the others.}
{ I have emended the correct but unmetrical °\skt{ādayaḥ} in pāda d to \skt{ādaya} to restore the metre. }





\slokawithoutfn{12.107}{ The Gandharva spoke: I have come back to you with a request, O Sky-goer lord. Give me another of that fruit you gave me and spare a life.}





\slokawithfn{12.108}{ Sūrya spoke: I went to Soma's world, and it was he who gave me the magical fruit. I gave you that fruit out of my friendship for you.}
{ Note the odd syntax of pādas cd. \skt{sa phalaṃ} may have been influenced                  by 12.71d and 72a. Here \skt{tat phalaṃ} would work                 better but see \skt{sa phalaṃ} in a similarly odd position in                  12.113d. \skt{dattam evāsi} is also problematic although similar                 structures do appear in this text, e.g. in 12.113c. The original may have read                 \skt{tat phalam datta evāsi}; or take \skt{dattam evāsi} as \skt{datta-m-evāsi},                 with a hiatus breaker \skt{-m-}. }





\slokawithfn{12.109}{ I cannot give you another one. Go now to Soma's city. Ask him, the son of Atri, the lord of planets, without hesitation.}
{ Understand \skt{purādya} as \skt{puram adya} (stem form metri causa) }





\slokawithfn{12.110}{ Rudra spoke: Led by Sūrya, they went to the world of Soma, Sūrya addressed Soma, expecting compassion from the Moon.}
{ Understand \skt{sūryāgrataḥ} in pāda a as \skt{sūryam agrataḥ} (stem form noun).  Note the form \skt{śaśim} for \skt{śaśinam}. }





\slokawithoutfn{12.111}{ Soma spoke: For what purpose have you returned? O Sun, there will be a solution for that. Except for giving another fruit, I shall do anything.}





\slokawithoutfn{12.112}{ Sūrya spoke: ``If you can, give me a fruit, I am not asking for anything else. If you do not give me another fruit, I'll kill you.''}





\slokawithfn{12.113}{ Soma spoke: ``I shall tell you how it arrived. Listen carefully. It was Indra who gave me the fruit and I gave that fruit to you.}
{ Note \skt{sa phalaṃ} for \skt{tat phalaṃ} again, as in 12.108c.                The syntax of pādas cd is rather confused and \skt{datta} in pāda d               is a stem form participle metri causa. }





\slokawithoutfn{12.114}{ Let's go to Indra's palace and ask for another one together. Let's go!'' he said and left for Indra's dwelling residence.}





\slokawithfn{12.115}{ Some said this to Indra: ``We have come here seeking a fruit.'' Give me another of the fruit now that you gave me before, O Śakra.}
{ \skt{soma indram} in pāda a in \msNc\ may be a correction of                 the reading in all the other sources. On the other hand,                 it can be original, and the hiatus may have confused an early scribe. }





\slokawithoutfn{12.116}{ Indra spoke: ``The reason for which you came here does not exist, O Moon. I received only a single one of that nice fruit out of Viṣṇu's hands.}





\slokawithoutfn{12.117}{ Let's go, all of us, to Viṣṇu's world, O lord of the planets.'' They all went to Madhusūdana for the fruit.}





\slokawithoutfn{12.118}{ After he spoke thus, they all left, led by the king of the gods. They reached the world of Viṣṇu in a moment, O Yaśasvinī.}





\slokawithfn{12.119}{ Indra then approached Janārdana, bowing down respectfully. I have a request, O Yaśodhara, that troubles everybody [here].}
{ Note that pāda a is unmetrical. Emend to \skt{tato} (irregular sandhi)?. }





\slokawithfn{12.120}{ Viṣṇu spoke: ``You all have come here for the fruit that I donated previously. I cannot give you the fruit. Otherwise, what else can I do for you?''}
{ The function of \skt{tac ca} in pāda b is unclear. Perhaps understand \skt{atra} (`here').                 Understand \skt{sarvam ihāgatāḥ} as \skt{sarva-m-ihāgatāḥ}, with a hiatus filler \skt{-m-}                 for \skt{sarva} (i.e. \skt{sarve}) \skt{ihāgatāḥ}.  The non-standard form \skt{anyaṃ} transmitted in all witnesses consulted                  might be original but I have not found any more instances of                  it in this text. That is why I have corrected it to the standard \skt{anyat}. }





\slokawithoutfn{12.121}{ Indra spoke: You are even capable of splitting Brahmā's Egg, O you of the banner with Garuḍa on it. I know that there is nothing that you cannot do, O Puruṣottama.''}





\slokawithoutfn{12.122}{ Having been addressed thus, Viṣṇu replied to Purandara (i.e. Indra): ``O Kauśika, I can do everything with the only exception of the fruit.}





\slokawithoutfn{12.123}{ I shall tell you now the means [of obtaining it]. Listen to where it came from, O Gopati. It was Brahmā who gave me that one single piece of fruit, O Purandara.}





\slokawithfn{12.124}{ I have given you one piece of fruit, why do you want me to give you another one [go for icchati?]? Let's now go to the highest creator Prajāpati (Brahmā) and ask him for one.}
{ For the expression \skt{parameṣṭhiprajāpati} see MBh 6.15.35ab:                 \skt{sarvalokeśvarasyeva parameṣṭhiprajāpateḥ} }





\slokawithoutfn{12.125}{ I'll ask Grandfather Brahmā, O king of the gods, to solve your problem.'' After he said this, they all left together, led by Janārdana:}





\slokawithoutfn{12.126}{ Indra, Soma, Sūrya, the Gandharva, the monkey, Vipula, the foreman, and two envoys of the king.}





\slokawithoutfn{12.127}{ They reached Brahmā's world in a moment, O Surasundarī. Seeing Brahmā's beautiful palace filled by all desireable things,}





\slokawithoutfn{12.128}{ the many kinds of brilliant gems, beautified with coral-tree roofs, floors inlaid with cat's-eye gems,}





\slokawithoutfn{12.129}{ the coral-gem pillars and the diamond and golden altar, the coral-gem and crystalline lattice-windows and sapphire windows,}





\slokawithfn{12.130}{ Vipula [also] saw [that there were] various charming trees there, with their tops bent down with [the burden of] the blossom and the fruits,}
{ Note \skt{°vṛkṣa} in pāda b as a stem form noun for \skt{°vṛkṣā} or \skt{°vṛkṣān}                 (\skt{manoramāḥ/-ān}). One could simply correct the pāda to                 \skt{nānāvṛkṣān manoramān}, but then the next line should also                 be altered. }





\slokawithoutfn{12.131}{ all the trees made of gems and the water[?] made of gems, the trees, bushes, creepers, winding plants and bulbous roots and fruits:}





\slokawithfn{12.132}{ Vipula saw all these consisting of jewels with his eyes open wide. [There was] a multi-storeyed palace decorated with garlands of pearls,}
{ Note the odd syntax of pādas ab. Pāda b should be understood as a                  phrase in the instrumental case. }





\slokawithfn{12.133}{ embellished with millions of groups of Apsarases wearing all kinds of ornaments, and millions and millions of floating aerial palaces possessing everything wished for.}
{ I understand pādas ab as if it read \skt{apsarogaṇakoṭībhiḥ sarvābharaṇabhūṣitair bhūṣitam}  Perhaps understand \skt{vimānakoṭikoṭīnāṃ} as \skt{vimānakoṭīnāṃ koṭiḥ} and                 \skt{°samanvitam} as \skt{°samanvitānām}. }





\slokawithoutfn{12.134}{ The assembly hall in Brahmā's world was charming and it shone like millions of suns. Brahmā was sitting there comfortably, decorated[?] with various jewels,}





\slokawithoutfn{12.135}{ with his four embodiments, four heads, four arms and four hands. The god who is the governor of the four social disciplines (\skt{āśrama}) was holding the four Vedas.}





\slokawithoutfn{12.136}{ Gāyatrī, who is the mother of the Vedas, and beautiful Sāvitrī were there, around the Vedas, attending [upon him] in their embodied form,}





\slokawithoutfn{12.137}{ Also Vyāhṛti[s] (Bhur, Bhuvaḥ, Svar) and Praṇava (Oṃ) were serving [him] in their embodied forms, as well as the syllables Vauṣaṭ, Vaṣaṭ and Namaḥ in their embodied forms,}





\slokawithfn{12.138}{ and Śruti and Smṛti and Nīti and Dharmaśāstra in their embodied forms, as well as Itihāsa, Purāṇa and Pātañjala Sāṃkhyayoga,}
{ Note the form \skt{patañjalam} metri causa for \skt{pātañjalam}.                 It is difficult to say if \skt{sāṃkhya yoga} in pāda d signifies one or two                 things. I have chosen to separate them, interpreting \skt{sāṃkhya} as a stem form                 noun, because in other parts of the text, \skt{sāṃkhya} and \skt{yoga} are usually treated as                 two different traditions. See 8.1--3, 16.36--37, and 23.5c.                  Understand \skt{patañjalam} as \skt{pātañjalaḥ} (metri causa and gender confusion).                 Another, less likely, possibility is that here \skt{sāṃkhyayoga} and                 \skt{pātañjalayoga} are contrasted. }





\slokawithfn{12.139}{ Āyurveda, Dhanurveda, and Gāndharvaveda, Arthaveda, and other Vedas, in their embodied forms.}
{ Understand \skt{mūrtimān} in pāda d as \skt{mūrtimantaḥ}. Note also \msCb\ and \msCc's                 attempt to include the Atharvaveda. I find it more likely that by                 \skt{arthaveda} Kauṭilya's Arthaśāstra is being referred to here. }





\slokawithoutfn{12.140}{ Then Brahmā rose and approached Janārdana (i.e. Viṣṇu). Giving him a cow? and guest-water, he said ``Please take a seat.}





\slokawithoutfn{12.141}{ The one of the banner with Garuḍa on it [should please sit] on [this] divine throne made of gems and jewels. The king of the gods (Indra), the Sun, the Moon, the Gandharva, the monkey king}





\slokawithoutfn{12.142}{ and Vipula the great man should sit on [these] gem-thrones. Well done, excellent Vipula! Congratulations for your enormous (\skt{vipula}) austerity!}





\slokawithfn{12.143}{ Well done, you of enourmous wisdom! Well done, you of enormous fortune! We are all pleased: Brahmā, Viṣṇu, Maheśvara,}
{ Understand \skt{°śriya} as the singular vocative masculine of \skt{°śrī}. }





\slokawithoutfn{12.144}{ the Ādityas, the Vasus, the Rudras, the Sādhyas, the Aśvins and the Marut[s]. Dive into the enjoyments in my world as much as you want, as you please.}





\slokawithfn{12.145}{ This one amongst the millions of aerial palaces has been built for you. There are thousands and thousands of sexy Apsarases,}
{ \skt{iyaṃ} (f.) in pāda a stands for either \skt{ayaṃ} (m.) or \skt{idaṃ} (n.), agreeing                 with the gender of \skt{vimāna}. Alternatively, the sentence                  wants, rather clumsily, to convey the meaning `all these millions of aerial palaces...'.  Note that here, as often in this text, nouns stand in the singular                 after numbers such as a thousand. }





\slokawithfn{12.146}{ adorned with all kinds of ornaments, making advances towards you. [This state of affairs will go on] for a thousand hundred quadrillion aeons, O great ascetic. Where there is effort, there one can enjoy [the results]''.}
{ Understand \skt{tavārthīyopasarpanti} as \skt{tavārthīyā upasarpanti} (double sandhi).                 \skt{tavārthāyo°} may work as well (\msCb\ and \msNa) but I consider                  \skt{tavārtīyo°} the lectio difficilior, thus potentially the original reading. }





\slokawithfn{12.147}{ Maheśvara spoke: Listening to his speech, Vipula, with his eyes wild open, shaking, trembling with fear, his eyes filled with tears,}
{ We are forced to accept \Ed's reading of \skt{bhayatrasta} here because it                 if far superior to the readings of all other witnesses.                  The rejected reading (\skt{bhayas tatra}) may have been the result of                 a simple metathesis (\skt{tra-sta} to \skt{sta-tra}). }





\slokawithfn{12.148}{ bowing down his head, prostrating himself on the ground again and again, delivered a sweet speech to [Brahmā,] the Grandfather of Brahmaloka:}
{ The compound \skt{brahmalokapitāmahaḥ} may sound                  slightly odd as an epithet of Brahmā but it does occur                  in the MBh (12.336.30b) and in other texts (Padmasaṃhitā 3.193d, Jayadrathayāmala 3.14.198b). }





\slokawithfn{12.149}{ Vipula spoke: ``Venerable sir, lord of all the worlds, Grandfather of all people, I can see a dream-like wonder, O lord of the thirty[-three] gods. My memory abandons me, my mind's intelligence is darkened.}
{ Note that \Ed\ adds a line here (see the apparatus; translation:                 `I am a fool, how could I praise you? You are beyond knowledge, beyond the ultimate.').                 I have not been able to locate this line in any of the available sources. }





\slokawithoutfn{12.150}{ ... Be my refuge. Protect [me] from[?] terrible transmigration. I am afraid of being in a womb, of the terror of old age and death. Protect me from the fetter of illusions. Dwelling in illness is eternal and the body is uncontollable. Protect me from the noose of time. Animals eating each other[?] for hundreds and hundreds of \skt{yuga}. Protect [me] from the darkness of illusions.}





\slokawithfn{12.151}{ Hearing [this] Brahmā spoke to [Vipula] of huge intellect, honouring [him] as follows. You will live until the universal floods of destruction. You will not have any longing for being reborn any more. There will be no dwelling in a womb for you, no rebirth, no anguish full of weariness. Killing the enemy who is the darkness of illusions, and you will reach the ultimate, the absorption into the Brahman.''}
{ The stem form noun °\skt{mati} of the bahuvrīhi compound                  in pāda a may stand for \skt{matiḥ} (see the unmetrical reading of \msCa\msCb\msNa), and                 then it should refer to Brahmā himself (`Brahmā, the one with a huge intellect...').                 I have choosen to take \skt{mati} as a stem form noun standing for the accusative,                 referring to Vipula. This works better because \skt{mānayitvā} (and \skt{śrutvā}) requires an object.  Note \skt{āhūtasamplava} instead of the more common \skt{ābhūtasamplava} (see also 2.13).                         \skt{me} in pāda b is difficult to interpret.  I take \skt{tvan na} in pāda c as an ablative of \skt{tvad} used as a genitive plus \skt{na}. }





\slokawithoutfn{12.152}{ Maheśvara spoke: When [Vipula] was addressed thus by Brahmā, Lord Viṣṇu (\skt{viṣṇunā prabhaviṣṇunā}) [said:] ``Let it be like that, bless your soul, just as the Grandfather said.''}





\slokawithoutfn{12.153}{ [Then] Indra, Ravi and Soma, the Sādhyas, the Ādityas, the Maruts, the Rudras, the Viśve[śas] and the Vasus[?] [spoke:]}





\slokawithfn{12.154}{ ``Wow, what a divine reward for great-souled Vipula's penance! He has reached heaven in his own [mortal] body by virtue of his worshipping a guest in good faith.''}
{ \skt{svaśarīraṃ} may stand for \skt{svaśarīre} or \skt{svaśarīreṇa} in pāda c. }





\slokawithoutfn{12.155}{ This and many other things are related in the Vipula section [probably of the \skt{Mahābhārata}, see MBh 13.39.1ff]. Viṣṇu, the lord of the whole universe, turned back to Brahmā.}



