\newcommand{\danda}{\thinspace$\cal j$ }
\newcommand{\twodanda}{\thinspace$\cal k$ }
\newcommand{\msCa}{{\rm C$_{\scriptscriptstyle 94}$}}
\newcommand{\msCaacorr}{{\rm C$^{\scriptscriptstyle ac}_{\scriptscriptstyle 94}$}}
\newcommand{\msCapcorr}{{\rm C$^{\scriptscriptstyle pc}_{\scriptscriptstyle 94}$}}
\newcommand{\msCb}{{\rm C$_{\scriptscriptstyle 45}$}}
\newcommand{\msCbacorr}{{\rm C$^{\scriptscriptstyle ac}_{\scriptscriptstyle 45}$}}
\newcommand{\msCbpcorr}{{\rm C$^{\scriptscriptstyle pc}_{\scriptscriptstyle 45}$}}
\newcommand{\msCc}{{\rm C$_{\scriptscriptstyle 02}$}}
\newcommand{\msCcacorr}{{\rm C$^{\scriptscriptstyle ac}_{\scriptscriptstyle 02}$}}
\newcommand{\msCcpcorr}{{\rm C$^{\scriptscriptstyle pc}_{\scriptscriptstyle 02}$}}
\newcommand{\msNa}{{\rm K$_{\scriptscriptstyle 82}$}}  
\newcommand{\msNaacorr}{{\rm K$^{\scriptscriptstyle ac}_{\scriptscriptstyle 82}$}}
\newcommand{\msNapcorr}{{\rm K$^{\scriptscriptstyle pc}_{\scriptscriptstyle 82}$}}
\newcommand{\msNb}{{\rm K$_{\scriptscriptstyle 10}$}}  
\newcommand{\msNbacorr}{{\rm K$^{\scriptscriptstyle ac}_{\scriptscriptstyle 10}$}}
\newcommand{\msNbpcorr}{{\rm K$^{\scriptscriptstyle pc}_{\scriptscriptstyle 10}$}}
\newcommand{\msNc}{{\rm K$_{\scriptscriptstyle 7}$}}  
\newcommand{\msNcacorr}{{\rm K$^{\scriptscriptstyle ac}_{\scriptscriptstyle 7}$}}
\newcommand{\msNcpcorr}{{\rm K$^{\scriptscriptstyle pc}_{\scriptscriptstyle 7}$}}
\newcommand\msBod{{\rm B}}
\newcommand\msBodac{{\rm B}$^{\scriptscriptstyle ac}$}
\newcommand\msBodpc{{\rm B}$^{\scriptscriptstyle pc}$}
\newcommand\msP{{\rm P}}
\newcommand\msPac{{\rm P}$^{\scriptscriptstyle ac}$}
\newcommand\msPpc{{\rm P}$^{\scriptscriptstyle pc}$}
\newcommand\Ed{{\rm E$^{\scriptscriptstyle N}$}}
\newcommand{\msCaNa}{{\normalfont C$_\textrm{a}$N$_\textrm{a}$}}
\newcommand{\msCbNa}{{\normalfont C$_\textrm{b}$N$_\textrm{a}$}}
\newcommand{\msCcNa}{{\normalfont C$_\textrm{c}$N$_\textrm{a}$}}
\newcommand{\msCabNa}{{\normalfont C$_\textrm{a}$C$_\textrm{b}$N$_\textrm{a}$}}
\newcommand{\msCbcNa}{{\normalfont C$_\textrm{b}$C$_\textrm{c}$N$_\textrm{a}$}}
\newcommand{\msCabcNa}{{\normalfont C$_\textrm{a}$C$_\textrm{b}$C$_\textrm{c}$N$_\textrm{a}$}}
\newcommand{\msCab}{{\normalfont C$_\textrm{a}$C$_\textrm{b}$}}
\newcommand{\msCac}{{\normalfont C$_\textrm{a}$C$_\textrm{c}$}}
\newcommand{\msCbc}{{\normalfont C$_\textrm{b}$C$_\textrm{c}$}}
\newcommand{\msCabc}{{\normalfont C$_\textrm{a}$C$_\textrm{b}$$_\textrm{c}$}}
\newcommand{\mssCaCbCc}{{\normalfont C}}
\newcommand{\msL}{{\rm L}\allowbreak}
\newcommand{\msLacorr}{{\rm L}$^{\scriptscriptstyle ac}$\allowbreak}
\newcommand{\msLpcorr}{{\rm L}$^{\scriptscriptstyle pc}$\allowbreak}
\newcommand{\msM}{{\rm L}\allowbreak}
\newcommand{\Cod}{\textit{Cod.}}
\newcommand{\Codd}{$\Sigma$}

\newcommand{\msA}{{\rm A}}
\newcommand{\msB}{{\rm A}}
\newcommand{\msC}{{\rm A}}
\newcommand{\msD}{{\rm A}}
\newcommand{\msE}{{\rm A}}
\newcommand{\msF}{{\rm A}}

\newcommand{\KKT}{KKT}
\newcommand{\LP}{{\englishfont Li\.nPu}}
\newcommand{\msNKeightytwo}{{\englishfont N$^{\scriptscriptstyle K}_{\scriptscriptstyle 82}$}\allowbreak}
\newcommand{\msNKeightytwoac}{{\englishfont N$^{\scriptscriptstyle Kac}_{\scriptscriptstyle 82}$}\allowbreak}
\newcommand{\msNKeightytwopc}{{\englishfont N$^{\scriptscriptstyle Kpc}_{\scriptscriptstyle 82}$}\allowbreak}

\newcommand{\msNKtwelve}{{\englishfont N$^{\scriptscriptstyle K}_{\scriptscriptstyle 12}$}\allowbreak}
\newcommand{\msNKtwelveac}{{\englishfont N$^{\scriptscriptstyle Kac}_{\scriptscriptstyle 12}$}\allowbreak}
\newcommand{\msNKtwelvepc}{{\englishfont N$^{\scriptscriptstyle Kpc}_{\scriptscriptstyle 12}$}\allowbreak}

\newcommand{\msNKtwelveb}{{\englishfont N$^{\scriptscriptstyle K}_{\scriptscriptstyle 12b}$}\allowbreak}
\newcommand{\msNKtwelvebac}{{\englishfont N$^{\scriptscriptstyle Kac}_{\scriptscriptstyle 12b}$}\allowbreak}
\newcommand{\msNKtwelvebpc}{{\englishfont N$^{\scriptscriptstyle Kpc}_{\scriptscriptstyle 12b}$}\allowbreak}

\newcommand{\msNCfortyfive}{{\englishfont N$^{\scriptscriptstyle C}_{\scriptscriptstyle 45}$}\allowbreak}
\newcommand{\msNCfortyfiveac}{{\englishfont N$^{\scriptscriptstyle Cac}_{\scriptscriptstyle 45}$}\allowbreak}
\newcommand{\msNCfortyfivepc}{{\englishfont N$^{\scriptscriptstyle Cpc}_{\scriptscriptstyle 45}$}\allowbreak}

\newcommand{\msNCninetyfour}{{\englishfont N$^{\scriptscriptstyle C}_{\scriptscriptstyle 94}$}\allowbreak}
\newcommand{\msNCninetyfourac}{{\englishfont N$^{\scriptscriptstyle Cac}_{\scriptscriptstyle 94}$}\allowbreak}
\newcommand{\msNCninetyfourpc}{{\englishfont N$^{\scriptscriptstyle Cpc}_{\scriptscriptstyle 94}$}\allowbreak}
\newcommand{\msNCninetyfouracorr}{{\englishfont N$^{\scriptscriptstyle Cac}_{\scriptscriptstyle 94}$}\allowbreak}
\newcommand{\msNCninetyfourpcorr}{{\englishfont N$^{\scriptscriptstyle Cpc}_{\scriptscriptstyle 94}$}\allowbreak}

\newcommand{\msNKtwentyeight}{{\englishfont N$^{\scriptscriptstyle K}_{\scriptscriptstyle 28}$}\allowbreak}  
\newcommand{\msNKtwentyeightac}{{\englishfont N$^{\scriptscriptstyle Kac}_{\scriptscriptstyle 28}$}\allowbreak}  
\newcommand{\msNKtwentyeightpc}{{\englishfont N$^{\scriptscriptstyle Kpc}_{\scriptscriptstyle 28}$}\allowbreak}  

\newcommand{\msNKoseventyseven}{{\englishfont N$^{\scriptscriptstyle Ko}_{\scriptscriptstyle 77}$}\allowbreak}  
\newcommand{\msNKoseventysevenac}{{\englishfont N$^{\scriptscriptstyle Koac}_{\scriptscriptstyle 77}$}\allowbreak}  
\newcommand{\msNKoseventysevenpc}{{\englishfont N$^{\scriptscriptstyle Kopc}_{\scriptscriptstyle 77}$}\allowbreak}  





% SDHS10
\newcommand{\msGa}{{\englishfont G$^{\scriptscriptstyle Ki}$}\allowbreak}
\newcommand{\msGaac}{{\englishfont G$^{\scriptscriptstyle Kiac}$}\allowbreak}
\newcommand{\msGapc}{{\englishfont G$^{\scriptscriptstyle Kipc}$}\allowbreak}



\thispagestyle{empty}
\ \vskip6em\begin{center}{\Huge \textit{Vṛṣasārasaṃgrahaḥ
}}\vskip1em {\Large (translation)}\bigskip\\ {\large\today}\end{center}


\textbf{12.1}The Goddess spoke:Harmlessness is always praised as the highest Dharma.Also, teach me the ultimate Dharma of those who practise hospitality.%


\textbf{12.2}Maheśvara spoke:Hear the ultimate Dharma of the harmless ones                         and that of the ones who practise hospitality.%
\footnote{Understand \skt{ahiṃsātithyakāmāṃ} as \skt{ahiṃsakānām ātithyakānāṃ ca}  }O beautiful-eyed goddess, [if] all the three worlds, full of wealth,%


\textbf{12.3}[were handed over as] a gift to [a Brahmin who] knows                the four Vedas, [that gift] cannot be compared to somebody who avoids doing harm.%
\footnote{Note that this verse seems to be all that Maheśvara teaches in this chapter on                  \skt{ahiṃsā} and that \skt{tattulyam ahiṃsakaḥ} may contain a sandhi bridge:                 \skt{tattulya-m-ahiṃsakaḥ}  }Hear the Dharma of the hospitable ones. I'll teach it [to you], O beautiful one.%
\footnote{\skt{atithyānāṃ} in pāda c stands for \skt{ātithyānāṃ} or \skt{ātithyakānāṃ} metri causa.  }%


\textbf{12.4}This is an old story of what happened once in a city called Kusuma [i.e.\ Pāṭaliputra].There was a famous and wise man called Vipula, Kapila's son.%


\textbf{12.5}He always followed his Dharma, he conquered anger, he spoke only                                 the truth and he conquered his senses.He was friendly to Brahmins. He was grateful and he was my determined devotee.%


\textbf{12.6}He was rich and he worshipped[?] his guests. He was generous, restrained, and merciful.He wealth always came through just means. He always stayed away from illegal actions.%


\textbf{12.7}He had a beautiful wife whose face was as pure as the disk of the moon.Her breasts were round and elevated, she was lovely, a source of all pleasure.She was faithful, devoted to her husband and his needs.%


\textbf{12.8}Now, once there was an eclipse of the sun.Three quarters [of the sun] were eclipsed, and it was in the dark half of the month of Mādhava.%


\textbf{12.9}Eager to take a ritual bath, the king and all citizens went down [to the river].They were worshipping the gods and the deceased ancestors according to rule.%


\textbf{12.10}Some sacrificed in the fire, some fed the Brahmins,some gave donations, others praised the deity.%


\textbf{12.11}Some people practised yoga meditation, others were engrossed in five-fire penance.While all the royals and other people were doing this all around the place,%


\textbf{12.12}Vipula too, there at the confluence of the Gaṅgā and the Gaṇḍakī,%
\footnote{Note \skt{gaṇḍaki} metri causa for \skt{gaṇḍakī} in pāda b.  }together with his wife, performed a bath, and, attired in linen clothes,%


\textbf{12.13}was satiating the deities, the gurus, the Brahmins and others.Then, jumping on the possibility, a Brahmin came up [to them] as a guest.%


\textbf{12.14}The wife got infatuated with that Brahmin's extreme beauty.The Brahmin [felt] the same. His beauty was unparalleled.[?]%
\footnote{Pāda d is slightly suspect and the translation of pādas cd is                         tentative. The expression \skt{rūpeṇāpratimo/°pratimā bhuvi} is                          common in the Mahābhārata and in the Purāṇas. Is that what was meant here?                         May a dual have been intended?  }%


\textbf{12.15}Their gaze got fixed on each other mutually.Vipula joined his hands [and said:] ``O virtuous Brahmin,%


\textbf{12.16}I am at your service, be gracious to me now, O great Brahmin.[My] wife, servants, cattle, village and all kinds of jewels [are all at your service].''%


\textbf{12.17}Having been addressed and greeted hospitably by Vipula, the Brahmin spoke:``If you really mean to give, your heart is very generous.''%
\footnote{Note that \msCc's omission here is probably due to an eyeskip from \skt{suprasannaṃ} in                 12.17d to \skt{suprasannaṃ} in 12.18a, although this would have lead to an omission of                 the next \skt{vipula uvāca}.  }%


\textbf{12.18}Vipula spoke:``My heart is generous, generousity is the fruit of austerity.Just command me quickly, O Brahmin. What is your desire?There is nothing that should not be donated to a Brahmin, beginning with one's own head, O Brahmin.''%


\textbf{12.19}The Brahmin spoke:``If you talk like this, my dear, give me your beautiful wife.Be happy, may you be fortunate, and may you prosper eternally!''%
\footnote{In pāda d, \skt{bhava} is less than satisfactory. One would normally expect                  \skt{bhavate/bhavatāṃ/bhavatu} in this context. Alternatively, it is possible                 \skt{kalyāṇo bhava} (`be happy') was meant or we could accept \Ed's reading.  }%


\textbf{12.20}Vipula spoke:``Accept my wife who has nice buttocks, and is young and beautiful,blameless, large-eyed and whose face resembles the full-moon.''%


\textbf{12.21}The wife spoke:``How can you abandon me, my lord? How can you leave somebody who is sinless?How can you abandon a wife who is extremely kind and faultless?%
\footnote{sa is problematic CHECK accept tyajet?  }%


\textbf{12.22}A wife is a man's friend in this world and in the other world.[Even if] a man gives enormous donations or performs numerous sacrifices,%


\textbf{12.23}or performs hard penance, he cannot get to heaven without having a son.I have heard that this was taught by the ancestors, and by Brahmins in my presence.%


\textbf{12.24}The sonless cannot obtain heaven. I have heard this so many times!Mandapāla, the great Brahmin, went to heaven as a reward of his austerities.%


\textbf{12.25}That great Brahmin made numerous donations, performed various sacrifices,%
\footnote{I have taken \skt{japayajñāṃś} in pāda c as a \skt{tatpuruṣa} compound.                  The same expression occurs e.g. in VSS 6.2ff, MBh 13.102.8c, Manu 2.86 etc.                 By this, \skt{vedāṃś} becomes difficult to interpret (I supply `recited'). It may be possible to take                 \skt{japa} as a form deriving from \skt{japan} (present participle) metri causa:                         \skt{vedāṃś ca japa}[\skt{n}] \skt{yajñāṃś ca kṛtvā}, but in this case                          the notion of performing sacrifices comes up twice in this verse.  }[recited] the Vedas, and performed sacrifices of recitation.%


\textbf{12.26}But when he reached the gate [of heaven], it was blocked by the celestial messengers:``The sonless cannot get to heaven, not even by hundreds of sacrifices.''%


\textbf{12.27}Mandapāla, the great sage was thus informed and he fell from heaven.The Brahmin begot four sons with a Śāraṅga-bird.%


\textbf{12.28}By the virtue of this, he reached heaven unobstructed.I am a wife (\skt{kalatra}) [because] I protect the family (\skt{kulatrāṇa}), and I am a wife to                        be supported (\skt{bhārya}) because I bear [sons] (\skt{bharaṇa}).%
\footnote{Note that pāda c is the result of emendations and that \skt{bhārya} in pāda d is                 to be understood as \skt{bhāryā} metri causa (nevertheless I supplied `to be supported'                 in the translation to convey the general meaning of the word \skt{bhārya},                  which seemed to fit the context well).  }%


\textbf{12.29}Taking a wife is for the sake of having sons according to the Śāstras.You can give that Brahmin all the wealth at home,                     all the villages, the stations of herdsmen and the houses,%


\textbf{12.30}but please don't give me away this time!''Having heard his wife's speech, Vipula spoke again.%
\footnote{I have not included \msCcpcorr's \skt{vipula uvāca} (echoed in \Ed)                 because after \skt{punar abravīt} is seems secondary and unnecessary.                 Note that the correction in \msCc\ is in a second hand.  }%


\textbf{12.31}``Alright, my beautiful wife, I know! Good, good, my faithful wife!I am beaten by this speach and I am satisfied with it.%


\textbf{12.32}Today the Brahmin came up to me at the time of eclipse, and he asked me.I promised him that I would give [you away]. If I don't give [you to him], I shall go to hell.%


\textbf{12.33}If I go to hell along with my family/decendants,I will not see release from hell, O brilliant woman, for millions of eons,%
\footnote{The reading \skt{narakastho} (\msNc\Ed) is tempting but it could be a scribal correction and                 \skt{narakasthād} may be original, meaning \skt{narakasthānād}.  }%


\textbf{12.34}as long as millions of births.I can see something bad, my Princess, from not giving, O woman with a nice compexion,%


\textbf{12.35}but from giving I can see something good in heaven that is eternal.I have never ever lied, I always observe the vow of truthfulness.%


\textbf{12.36}If I transgressed the law of truth, [by this]                                I would stop following all other laws [too].You mentioned earlier that the wife is one's Dharmic friend.%
\footnote{I have emended \skt{tvayi} in pāda d to \skt{tvayā} because it                  seems an early random scribal mistake, rather than some                  linguistic pecularity.  }%


\textbf{12.37}If you are indeed my Dharmic friend, then now the time has come.Dharma himself has visited us disguised as a Brahmin.%


\textbf{12.38}%
\footnote{In pāda a, \skt{ahaṃ} either stands for \skt{māṃ} or the phrase \skt{jijñāsārtham ahaṃ} can be                  translated as `I am to be tested.'  }to test me. O my dear, please don't cause me trouble.The Unmanifest (Prakṛti) is my mother, Brahmā is my father,                 Intelligence is my wife, self-control is my friend.%


\textbf{12.39}Dharma is my son, Ritual is my guru. These are my relatives.The best time is the time of the eclipse of the Sun.                The best one among the rivers is the Gaṅgā.%
\footnote{I understand \skt{grahaḥ sūryo} in pāda c as \skt{sūryagrahaḥ} (or \skt{sūryagrahaṇam}):                the eclipse of the Sun, which appears to be an auspicious day. See parallels in the apparatus.  }%


\textbf{12.40}The best day is at new moon, the best man is the Brahmin.I have given you to the Brahmin to serve him.%
\footnote{In pāda f, \skt{brāhmaṇe} (loc., in all the witnesses that I have consulted)                                  may have originally read \skt{brahmaṇe} (dat.).  }Having given everything to the Brahmin, I'll resort to the forest.''%


\textbf{12.41}Śaṅkara [i.e.\ Śiva] spoke:The wife remained silent, her eyes filled with tears.[Vipula] took her hand and the long-eyed woman was presented to the Brahmin.%


\textbf{12.42}I am ready to give you all the wealth I have at home, all the gold and the cattle,O great Brahmin, the village, the stations of herdsmen and the houses,                        and everything else,%


\textbf{12.43}pearls, gems, clothes and divine ornaments.Accept all these, O best of Brahmins. It's given in good faith and with respect.%


\textbf{12.44}May Lord Dharma be pleased and may Maheśvara be pleased.%
\footnote{Note Śivadharmaśāstra 10.11cd, in a similar context of donations:                 \skt{bhojayitvā tato brūyāt prīyatāṃ bhagavān śivaḥ}  }%
\footnote{Understand \skt{sukṛtaṃ phalam} as \skt{sukṛtaphalam} (metri causa).  }May all the ancestors rejoice if there is reward for meritorious acts.%


\textbf{12.45}Rudra spoke:Having heard Vipula's speech, the ascetic Brahminblessed the good-souled Vipula a good number of times,%
\footnote{There are several ways to explain the form \skt{āśīḥ} in pāda c.               The easiest is to treat it as a singular accusative neuter.               Alternatively, it could be a plural accusative feminine from \skt{āśī} and               then \skt{suvipulaṃ} is either to be understood adverbially or as \skt{suvipulā}[\skt{s}].               Another way to treat \skt{āśīḥ} would be to take it as a nominative standing               for the accusative.  }%


\textbf{12.46}and then went off to live in a nice house, taking Vipula's wife with him.As for Vipula, he said good-bye and circulambulated him.%


\textbf{12.47}Thus saluting the Brahmin, he departed quickly into the forest.In the forest he lived off roots and fruits and roamed about in the world.%


\textbf{12.48}But being alone in an abandoned and deserted place, he got overwhelmed with worry.Where should I go? Where should I look for food? From whom? What shall I do?%


\textbf{12.49}I don't know these roads, this country, these villages and these cities,towns, mountain settlements. I don't know anybody here.%
\footnote{In pāda d, the reading of all the witnesses, \skt{kaścana}, seems to be               an early scribal mistake for \skt{kañcana}. But note that the same happens at                 12.55d.  }%


\textbf{12.50}I can see a nice mountain there with large cavities and caves.I'll climb it and try to figure out if there is a village, town or city [nearby].%


\textbf{12.51}%
\footnote{I have accepted the reading (emendation?) of \Ed in pāda d (\skt{āruhat})               because I think that \skt{āruhet} is an early scribal mistake that               is easy to make and because \skt{°āruhat} comes up again in 12.53d.  }Having said this, Vipula climbed the mountain slowly.He caught sight of the shades of a tree, and being exhausted sat down [there].%


\textbf{12.52}In the same moment, descending from among the branches of the tree,[a monkey appeared and] carrying an extraordinary, beautiful, fragrant, excellent,%


\textbf{12.53}lovely, delightful and pleasant-looking fruit,it put it in front of Vipula and then returned to the tree.%
\footnote{Note how the agent of this sentence is omitted here. That it was a monkey               that gave Vipula the fruit becomes clear in 12.94.  }%


\textbf{12.54}Vipula, seeing this wonder, was perplexed.Am I sleeping or is this the fruit of my penance?%


\textbf{12.55}I have never seen, smelt, tasted anything like this.I have not even heard of anything like this. I shall let somebody know about it.%
\footnote{I suspect that \skt{śrotā} in pāda c is meant to be feminine participle \skt{śrutā}, but               the metre required the first vowel to be lengthened; understand \skt{me} as \skt{mayā}.               In pāda d, the reading of all the witnesses, \skt{kaścana}, seems to be               an early scribal mistake for \skt{kañcana}. But note that the same happens at                 12.49d.  }%


\textbf{12.56}Having said this ... , taking that nice fruit,he kept observing its smell again and again.%


\textbf{12.57}``Examining the fruit, ... seeing this country,I have run out of provision, and this fruit must have been sent to me by a god.%


\textbf{12.58}Therefore, I shall take this fruit and go to that city,and I shall go and seek something to live on.%


\textbf{12.59}Then leaving that mountain behind, he entered the city.He asked a man on the road what the name of this city was.%


\textbf{12.60}That traveller replied: ``Have you never been here?%
\footnote{I understand \skt{pathīkena} as standing for \skt{pathikena} metri causa (see 12.64b) and not               as two words, \skt{pathī kena}. This means that we are forced to accept an instrumental as the agent                of the finite verb \skt{uvāca}. I suspect that \msNc's reading (\skt{pathīko})                is an attempt to correct the syntax, but in this way \skt{apūrvam} becomes                problematic. With \skt{pūrvam} tha sentence may mean: `The traveller replied:                 ``Have you not come here before?'' '  }This is the Deccan region, and this is the city of Naravīra.%
\footnote{\skt{ayam} as the end of this verse may have been the original reading and                 \msCb\ may have corrected it to \skt{adaḥ}. Another possibility is that                 an original \skt{adaḥ} is preserved in \msCb, and it got corrupted to                 \skt{ayaḥ} (\msCa), and then to \skt{ayaṃ} (\msCc\msNa).                  In any case, I have chosen the reading \skt{adaḥ}                 because it works better; it can be viewed as my editorial correction.  }%


\textbf{12.61}The king is called Siṃhajaṭa, his queen is Kekayī.The king is very old, afflicted by old age. The queen likewise.%


\textbf{12.62}He is generous and he is an expert in the arts and he possesses                the power of heroism in battle.He is pious and devoted to his subjects and he is well-versed in the Śāstras.''%


\textbf{12.63}Vipula spoke:``As a matter of fact, I am seeking audience with the foreman of                the guild (\skt{śreṣṭhi/śreṣṭhin}). What is his name? Tell me.%
\footnote{Note the form \skt{śreṣṭhiṃ} from the stem \skt{śreṣṭhi} instead of \skt{śreṣṭhin} (thematisation).  }In which district is his dwelling? Tell me without any hesitation.''%


\textbf{12.64}Having been addressed by Vipula thus, the traveller spoke to him again:%
\footnote{Note the stem form \skt{pathika} in \skt{pathikovāca} in pāda b. Alternatively,               it is an instance of double sandhi (\skt{pathika uvāca} - \skt{pathikovāca})  }``My name is Bhīmabala and I have come to visit the house of the foreman of the guild.%


\textbf{12.65}The foreman of the guild is called Puṇḍaka and he is said to be a famous foreman.If you are eager [to see him], come with me.''%


\textbf{12.66}``Alright, let it be.'' Great-souled Vipula spoke thus to him,and he set off to visit the foreman's house together with Vipula.%


\textbf{12.67}When Vipula saw the foreman who was sitting in his house,he went up to him and offered him that fruit.%


\textbf{12.68}``Wow, what an excellent fruit! And hey, it has been brought here.%
\footnote{Note \skt{ihānitam} for \skt{ihānītam} in pāda b for metrical reasons.  }Wow, what a form, what a smell, wow what a splendid fruit!%


\textbf{12.69}%
\footnote{Most probaby, \skt{kandare} (`in a cave') in pāda b is an early                  mistake for \skt{mandare} (`on Mount Mandara'), a location that                  appears frequently in the epics and the Purāṇas next                  to Mount Meru. This is why I conjecture \skt{mandare} here.  }This fruit was not produced on earth, not even on Mount Meru or ...It is clearly from the world of gods, [this kind of fruit] does not grow in                        the world of humans.%
\footnote{Understand \skt{devalokika} in pāda c as a stem form compound (metri causa) for a more standard               \skt{devalaukikaṃ}.               \skt{martya-m-upajāyate} in pāda d might be original, with \skt{m} as a sandhi bridge. Nevertheless,               I emended the pāda to make it clearer.  }%


\textbf{12.70}Ah! I will enjoy [its] profits. It is fit for a king.%
\footnote{Pāda a is slightly suspect. It is possible that originally it contained a                negation: \skt{aho 'smi na phalaṃ bhoktā} (`Ah! I will not eat this fruit').               On the other hand, \skt{saphala} seems to be an odd form in this text simply               meaning \skt{phala} (see 12.71--72, 108).               The translation I have chosen is tentative.  }Offering this divine fruit to the king, I shall please him.''%


\textbf{12.71}Then grabbing that pleasant fruit, he left hastily.%
\footnote{In pāda a, \skt{tvarita}, for the adverb \skt{tvaritaṃ}, is in stem form metri causa.  }He approached the king respectfully, and gave him the fruit.%
\footnote{As in 12.70, \skt{sa phala}, or rather \skt{saphala} might simply mean \skt{phala}.                  Here in pāda d I have chosen to print this phrase as two words because here                 \skt{sa} can be grammatically/syntactically correct. See also next line (12.72a).  }%


\textbf{12.72}And seeing the fruit, the king was highly amazed.%
\footnote{On the possibility that \skt{saphala} is a form in this text simply signifying \skt{phala},                 see notes on 12.70 and 72.  }``O foreman, from where have you brought this charming fruit previously?%
\footnote{\skt{pūrva}[\skt{ṃ}] in pāda d is suspect and difficult to interpret and                \Ed\ is probably trying to silently emend it.                 One possibility is that the pāda originally contained a stem form noun:                 \skt{phalāpūrvaṃ manoharam} (`an unparalleled and charming fruit').                 Alternatively, \skt{pūrva} is an eyeskip to 12.73b.  }%


\textbf{12.73}I have never seen such a sweet root or fruit or bulbous root,one with such beauty, fragrance and qualities that gladden one's heart.%


\textbf{12.74}I shall eat this fruit that you have given me instantly.%
\footnote{I take \skt{svāda} as a stem form noun that stands for the accusative metri causa.  }What does it taste like? I want to know. Give it to me quickly.''%


\textbf{12.75}Then he ate the fruit that looked like the nectar of immortality.The king devoured all of it and it tasted nice, like nectar.%


\textbf{12.76}In an instant he obtained the youthfulness of a sixteen-year-old person.In a moment, there were no wrinkles and grey hair, no illness and no weakness.%


\textbf{12.77}His hair, teeth and nails all became smooth and shiny, his teeth and senses strong,he regained his vital powers, his vision, strength and his life energies in a moment.%


\textbf{12.78}The minister, the domestic chaplain, the counsellor, all the servants,the townswomen, and all the children and all the elderly people, everybody was amazed.%


\textbf{12.79}The sovereign, king Siṃhajaṭa, became extremely satisfiedand very happy.%


\textbf{12.80}The king, who was selfish and cruel, spoke to that foreman of the guild:%
\footnote{The syntax of pāda c is confusing. I translate it as if it carried                  a causative meaning (e.g. \skt{kuru bhīmabalaṃ tv evaṃ}: `make Bhīmabala do like this').                 On the other hand, an instrumental would be better (`act like this, together with                 Bhīmabala'), at least 12.82b hints at this solution.  }``Tell Bhīmabala to bring another fruit today.%


\textbf{12.81}I have regained my youthfulness by your kindness, O excellent man.Bring youthfulness also to Kekayī, who is weak and old.''%


\textbf{12.82}The foreman and Bhīmabala were addressed by the king thus.%
\footnote{I accepted the reading \skt{śreṣṭhī} in pāda b although it may be a                  correction of \skt{śreṣṭhi}, an original \skt{prātipadika} of the thematised                  form of \skt{śreṣṭhin} (see 1.63a).  }[Bhīmabala] replied to the king, joining his hands reverentially                and remaining standing with his head bowed down.%


\textbf{12.83}``Your majesty, one cannot obtain [such a fruit by wondering] from forest to forest.                It cannot be obtained through merchants or by cultivating the land.%
\footnote{  Pāda a could be construed as \skt{na vane na vane rājan}                         (`Your majesty, there is no [such fruit] in any                         forest'), but a similar expression, \skt{vanena                         vanaṃ}, occurs e.g. in MBh                         1.144.1 meaning `from forest to forest' (\skt{te vanena                         vanaṃ vīrā ghnanto mṛgagaṇān bahūn\danda apakramya yayū                         rājaṃs tvaramāṇā mahārathāḥ\twodanda}), and this made me                         choose another option (\skt{na vanena vane rājan}). \Ed's variant                         seems like an attempt to `correct'                         the text.                   }Some noble man who is seeking your audience%


\textbf{12.84}gave it to me, and, O king, I gave it to you, your majesty.Your majesty, I cannot tell you who this foreigner is.''%


\textbf{12.85}%
\footnote{Pāda a is unmetrical. It is possible the the original read \skt{°balaṃ} to avoid this,                 still meaning the compound \skt{bhīmabalavākyaṃ}.  }Having heard Bhīmabala's reply, [the king] said:You are the son of a noble family of ministers.  Announce[?] my orders.%


\textbf{12.86}If there are no more, why did you give me one? This is what I request from you, sir.%
\footnote{I have choosen \msCb's reading in pāda c only because it is metrical. This                 does not mean that the unmetrical reading of \msCa\msNa\msNc\ cannot have                  been the original one.  }Where there is one, there are many, that is for sure.%


\textbf{12.87}[There is a] path by which[?] it arrived. One should go [back] by the same route.By all means, that's the way to go. Track it down by that route.%


\textbf{12.88}If you are unable to provide another [fruit], I'll have your head cut off, you fool.Caṇḍa and Vicaṇḍa will slay [you]. Beware, vile Bhīmabala!''%
\footnote{My impression is that Caṇḍa and Vicaṇḍa could be the two royal envoys mentioned                in verse 12.126 (\skt{rājadūtadvayam}), sent along with Bhīmabala to make sure he obeys the king's command.               Compare with Śivadharmottara 7.101 (Kenji and Sathya), where Yamas attendants are               called Caṇḍa and Mahācaṇḍa.  }%


\textbf{12.89}Then Bhīmabala got angry, took his sword that looked like the [crescent] moon,and, obeying the king's orders, went to that son of a noble family                [together with Puṇḍaka the foreman].%
\footnote{The reconstruction of pāda d is unsatisfactory and I do not know               how to emend \skt{aram}/\skt{param} at the end of the line. We have to suppose               that Bhīmabala is accompanied by Puṇḍaka the foreman of the guild because               Vipula's answer seems to be directed towards him.  }%


\textbf{12.90}O son of a noble family, don't take it as an offence, [but] I'll kill youunless you have more of this fruit. Give one to the king now!%


\textbf{12.91}%
\footnote{I conjectured \skt{tvaram} for \skt{tava} in pāda b because \skt{tava} is both               unmetrical and meaningless in this context. \skt{tava} might have               been the result of an eyeskip to pāda d or rather to pāda b of 12.92.  }Reveal to me quickly where you found the divine fruit.Without that fruit, my friend, your life is in danger.''%


\textbf{12.92}Vipula spoke:I regained my hope for life [when I reached?] your house in this foreign country.%
\footnote{The translation of pādas ab is tentative. If my interpretation is                correct, the house in question is Puṇḍaka's house.  }%
\footnote{Perhaps understand \skt{kṛtakartā} in pāda c as \skt{kṛtyakartā}.  }How could one who does his duty be slain? I would obtain [another fruit] right now.%


\textbf{12.93}But there is no other fruit. Nobody can provide any.Up on the rocky peak[?] of Mount Sahya, I sat down, mentally exhausted.%


\textbf{12.94}It was a monkey that took that fruit, gave it to me                        and then disappeared.I gave it to you, you gave it to the king.%


\textbf{12.95}Let's go to that place, O foreman, to see if the monkey is there.%
\footnote{I have accepted \msCb's reading in pāda d mainly because the reading               of all the other witnesses is difficult to interpret and because               a similar verb form, \skt{yācasva}, appears in 12.105d.  }When we get there together, we can ask the monkey king [for more fruit].%


\textbf{12.96}The foreman said: ``Alright, let's go togetherto the place where you got that fruit. We shall be saved.''%
\footnote{The foreman uses the plural in his reply correctly: he refers to               Vipula, Bhīmabala and himself.  }%


\textbf{12.97}Rudra spoke:Climbing Mount Sahya, searching the place all over,Vipula then caught glimpse of the monkey, the monkey king.%


\textbf{12.98}``It's that extraordinary monkey there lurking in the shade of that tree.This monkey has showed up today merely by the force of my meritious act.%
\footnote{The `meritious act' mentioned here is probably that of giving his wife to                  the Brahmin at the beginning of the story.  }%


\textbf{12.99}Hey, monkey, unless you do me a friendly favour I'll perish very quickly.Give me another one of that fruit that you gave me, O monkey, [and thus] keep me alive.''%


\textbf{12.100}The monkey spoke:It was a Gandharva that had given me the fruit and I gave it to you.How could I give you another one? Go there [where Gandharvas live] if you wish.%


\textbf{12.101}Vipula spoke:``If you cannot give me another fruit, [my] staying alive is doubtful.%
\footnote{I suspect that \skt{tubhyaṃ} in pāda a is used in the sense of \skt{tvayā} and                 that is how I translate this phrase. I doubt if Vipula would                 threaten the monkey (`for you living becomes doubtful').  }Another alternative is that we go where Citraratha himself[, the king of                the Gandharvas,] dwells.''%


\textbf{12.102}The monkey replied: ``Let's do it.''Then, upon reaching the dwelling place of Citraratha and having gone up to him, he said this:%


\textbf{12.103}``O king of the Gandharvas, I have come back to you with a request.Give me another of that fruit that you gave me if you can.''%


\textbf{12.104}The king of the Gandharvas spoke:``I went to the world of Sūrya, and it was him who gave me that extraordinary fruit.I gave that fruit to you [because] you are my very best friend.%
\footnote{Understand \skt{suhṛdo} in pāda d as a singular nominative of the rare \skt{suhṛda}.  }%


\textbf{12.105}Where could I find another fruit to give you, I don't have one, O monkey.Let's go to the world of Sūrya and ask the Sun there.''%


\textbf{12.106}Having been addressed thus by the Gandharva, the monkey consented.They reached the world of Sūrya all together, the Gandharva and the others.%
\footnote{I have emended the correct but unmetrical °\skt{ādayaḥ} in pāda d to \skt{ādaya} to restore the metre.  }%


\textbf{12.107}The Gandharva spoke:I have come back to you with a request, O Sky-goer lord.Give me another of that fruit you gave me and spare a life.%


\textbf{12.108}Sūrya spoke:I went to Soma's world, and it was he who gave me the magical fruit.I gave you that fruit out of my friendship for you.%
\footnote{Note the odd syntax of pādas cd. \skt{sa phalaṃ} may have been influenced                  by 12.71d and 72a. Here \skt{tat phalaṃ} would work                 better but see \skt{sa phalaṃ} in a similarly odd position in                  12.113d. \skt{dattam evāsi} is also problematic although similar                 structures do appear in this text, e.g. in 12.113c. The original may have read                 \skt{tat phalam datta evāsi}; or take \skt{dattam evāsi} as \skt{datta-m-evāsi},                 with a hiatus breaker \skt{-m-}.  }%


\textbf{12.109}I cannot give you another one. Go now to Soma's city.%
\footnote{Understand \skt{purādya} as \skt{puram adya} (stem form metri causa)  }Ask him, the son of Atri, the lord of planets, without hesitation.%


\textbf{12.110}Rudra spoke:Led by Sūrya, they went to the world of Soma,%
\footnote{Understand \skt{sūryāgrataḥ} in pāda a as \skt{sūryam agrataḥ} (stem form noun).  }Sūrya addressed Soma, expecting compassion from the Moon.%
\footnote{Note the form \skt{śaśim} for \skt{śaśinam}.  }%


\textbf{12.111}Soma spoke:For what purpose have you returned? O Sun, there will be a solution for that.Except for giving another fruit, I shall do anything.%


\textbf{12.112}Sūrya spoke:``If you can, give me a fruit, I am not asking for anything else.If you do not give me another fruit, I'll kill you.''%


\textbf{12.113}Soma spoke:``I shall tell you how it arrived. Listen carefully.%
\footnote{Note \skt{sa phalaṃ} for \skt{tat phalaṃ} again, as in 12.108c.                The syntax of pādas cd is rather confused and \skt{datta} in pāda d               is a stem form participle metri causa.  }It was Indra who gave me the fruit and I gave that fruit to you.%


\textbf{12.114}Let's go to Indra's palace and ask for another one together.Let's go!'' he said and left for Indra's dwelling residence.%


\textbf{12.115}%
\footnote{\skt{soma indram} in pāda a in \msNc\ may be a correction of                 the reading in all the other sources. On the other hand,                 it can be original, and the hiatus may have confused an early scribe.  }Some said this to Indra: ``We have come here seeking a fruit.''Give me another of the fruit now that you gave me before, O Śakra.%


\textbf{12.116}Indra spoke:``The reason for which you came here does not exist, O Moon.I received only a single one of that nice fruit out of Viṣṇu's hands.%


\textbf{12.117}Let's go, all of us, to Viṣṇu's world, O lord of the planets.''They all went to Madhusūdana for the fruit.%


\textbf{12.118}After he spoke thus, they all left, led by the king of the gods.They reached the world of Viṣṇu in a moment, O Yaśasvinī.%


\textbf{12.119}%
\footnote{Note that pāda a is unmetrical. Emend to \skt{tato} (irregular sandhi)?.  }Indra then approached Janārdana, bowing down respectfully.I have a request, O Yaśodhara, that troubles everybody [here].%


\textbf{12.120}Viṣṇu spoke:%
\footnote{The function of \skt{tac ca} in pāda b is unclear. Perhaps understand \skt{atra} (`here').                 Understand \skt{sarvam ihāgatāḥ} as \skt{sarva-m-ihāgatāḥ}, with a hiatus filler \skt{-m-}                 for \skt{sarva} (i.e. \skt{sarve}) \skt{ihāgatāḥ}.  }``You all have come here for the fruit that I donated previously.I cannot give you the fruit. Otherwise, what else can I do for you?''%
\footnote{The non-standard form \skt{anyaṃ} transmitted in all witnesses consulted                  might be original but I have not found any more instances of                  it in this text. That is why I have corrected it to the standard \skt{anyat}.  }%


\textbf{12.121}Indra spoke:You are even capable of splitting Brahmā's Egg, O you of the banner with Garuḍa on it.I know that there is nothing that you cannot do, O Puruṣottama.''%


\textbf{12.122}Having been addressed thus, Viṣṇu replied to Purandara (i.e. Indra):``O Kauśika, I can do everything with the only exception of the fruit.%


\textbf{12.123}I shall tell you now the means [of obtaining it]. Listen to where it came from, O Gopati.It was Brahmā who gave me that one single piece of fruit, O Purandara.%


\textbf{12.124}I have given you one piece of fruit, why do you want me to give you another one [go for icchati?]?Let's now go to the highest creator Prajāpati (Brahmā) and ask him for one.%
\footnote{For the expression \skt{parameṣṭhiprajāpati} see MBh 6.15.35ab:                 \skt{sarvalokeśvarasyeva parameṣṭhiprajāpateḥ}  }%


\textbf{12.125}I'll ask Grandfather Brahmā, O king of the gods, to solve your problem.''After he said this, they all left together, led by Janārdana:%


\textbf{12.126}Indra, Soma, Sūrya, the Gandharva, the monkey,Vipula, the foreman, and two envoys of the king.%


\textbf{12.127}They reached Brahmā's world in a moment, O Surasundarī.Seeing Brahmā's beautiful palace filled by all desireable things,%


\textbf{12.128}the many kinds of brilliant gems,beautified with coral-tree roofs, floors inlaid with cat's-eye gems,%


\textbf{12.129}the coral-gem pillars and the diamond and golden altar,the coral-gem and crystalline lattice-windows and sapphire windows,%


\textbf{12.130}Vipula [also] saw [that there were] various charming trees there,%
\footnote{Note \skt{°vṛkṣa} in pāda b as a stem form noun for \skt{°vṛkṣā} or \skt{°vṛkṣān}                 (\skt{manoramāḥ/-ān}). One could simply correct the pāda to                 \skt{nānāvṛkṣān manoramān}, but then the next line should also                 be altered.  }with their tops bent down with [the burden of] the blossom and the fruits,%


\textbf{12.131}all the trees made of gems and the water[?] made of gems,the trees, bushes, creepers, winding plants and bulbous roots and fruits:%


\textbf{12.132}Vipula saw all these consisting of jewels with his eyes open wide.%
\footnote{Note the odd syntax of pādas ab. Pāda b should be understood as a                  phrase in the instrumental case.  }[There was] a multi-storeyed palace decorated with garlands of pearls,%


\textbf{12.133}embellished with millions of groups of Apsarases wearing all kinds of ornaments,%
\footnote{I understand pādas ab as if it read \skt{apsarogaṇakoṭībhiḥ sarvābharaṇabhūṣitair bhūṣitam}  }and millions and millions of floating aerial palaces possessing everything wished for.%
\footnote{Perhaps understand \skt{vimānakoṭikoṭīnāṃ} as \skt{vimānakoṭīnāṃ koṭiḥ} and                 \skt{°samanvitam} as \skt{°samanvitānām}.  }%


\textbf{12.134}The assembly hall in Brahmā's world was charming and it shone like millions of suns.Brahmā was sitting there comfortably, decorated[?] with various jewels,%


\textbf{12.135}with his four embodiments, four heads, four arms and four hands.The god who is the governor of the four social disciplines (\skt{āśrama})                                                was holding the four Vedas.%


\textbf{12.136}Gāyatrī, who is the mother of the Vedas, and beautiful Sāvitrī                were there, around the Vedas, attending [upon him] in their embodied form,%


\textbf{12.137}Also Vyāhṛti[s] (Bhur, Bhuvaḥ, Svar) and Praṇava (Oṃ) were serving [him] in their embodied forms,as well as the syllables Vauṣaṭ, Vaṣaṭ and Namaḥ in their embodied forms,%


\textbf{12.138}and Śruti and Smṛti and Nīti and Dharmaśāstra in their embodied forms,as well as Itihāsa, Purāṇa and Pātañjala Sāṃkhyayoga,%
\footnote{Note the form \skt{patañjalam} metri causa for \skt{pātañjalam}.                 It is difficult to say if \skt{sāṃkhya yoga} in pāda d signifies one or two                 things. I have chosen to separate them, interpreting \skt{sāṃkhya} as a stem form                 noun, because in other parts of the text, \skt{sāṃkhya} and \skt{yoga} are usually treated as                 two different traditions. See 8.1--3, 16.36--37, and 23.5c.                  Understand \skt{patañjalam} as \skt{pātañjalaḥ} (metri causa and gender confusion).                 Another, less likely, possibility is that here \skt{sāṃkhyayoga} and                 \skt{pātañjalayoga} are contrasted.  }%


\textbf{12.139}Āyurveda, Dhanurveda, and Gāndharvaveda,Arthaveda, and other Vedas, in their embodied forms.%
\footnote{Understand \skt{mūrtimān} in pāda d as \skt{mūrtimantaḥ}. Note also \msCb\ and \msCc's                 attempt to include the Atharvaveda. I find it more likely that by                 \skt{arthaveda} Kauṭilya's Arthaśāstra is being referred to here.  }%


\textbf{12.140}Then Brahmā rose and approached Janārdana (i.e. Viṣṇu).Giving him a cow? and guest-water, he said ``Please take a seat.%


\textbf{12.141}The one of the banner with Garuḍa on it [should please sit] on [this]                         divine throne made of gems and jewels.The king of the gods (Indra), the Sun, the Moon, the Gandharva, the monkey king%


\textbf{12.142}and Vipula the great man should sit on [these] gem-thrones.Well done, excellent Vipula! Congratulations for your enormous (\skt{vipula}) austerity!%


\textbf{12.143}Well done, you of enourmous wisdom! Well done, you of enormous fortune!%
\footnote{Understand \skt{°śriya} as the singular vocative masculine of \skt{°śrī}.  }We are all pleased: Brahmā, Viṣṇu, Maheśvara,%


\textbf{12.144}the Ādityas, the Vasus, the Rudras, the Sādhyas, the Aśvins and the Marut[s].Dive into the enjoyments in my world as much as you want, as you please.%


\textbf{12.145}This one amongst the millions of aerial palaces has been built for you.%
\footnote{\skt{iyaṃ} (f.) in pāda a stands for either \skt{ayaṃ} (m.) or \skt{idaṃ} (n.), agreeing                 with the gender of \skt{vimāna}. Alternatively, the sentence                  wants, rather clumsily, to convey the meaning `all these millions of aerial palaces...'.  }There are thousands and thousands of sexy Apsarases%
\footnote{Note that here, as often in this text, nouns stand in the singular                 after numbers such as a thousand.  }%


\textbf{12.146}%
\footnote{Understand \skt{tavārthīyopasarpanti} as \skt{tavārtāyā upasarpanti} (double sandhi).                 \skt{tavārthāyo°} may work as well (\msCb\ and \msNa) but I consider                  \skt{tavārtīyo°} the lectio difficilior, thus potentially the original reading.  }adorned with all kinds of ornaments approaching you gently.[This state of affairs will go on] for a thousand hundred quadrillion aeons,                                         O great ascetic.Where there is effort, there one can enjoy [the results]''.%


\textbf{12.147}Maheśvara spoke:Listening to his speech, Vipula, with his eyes wild open,shaking, trembling with fear, his eyes filled with tears,%
\footnote{We are forced to accept \Ed's reading of \skt{bhayatrasta} here because it                 if far superior to the readings of all other witnesses.                  The rejected reading (\skt{bhayas tatra}) may have been the result of                 a simple metathesis (\skt{tra-sta} to \skt{sta-tra}).  }%


\textbf{12.148}bowing down his head, prostrating himself on the ground again and again,%
\footnote{The compound \skt{brahmalokapitāmahaḥ} may sound                  slightly odd as an epithet of Brahmā but it does occur                  in the MBh (12.336.30b) and in other texts (Padmasaṃhitā 3.193d, Jayadrathayāmala 3.14.198b).  }delivered a sweet speech to [Brahmā,] the Grandfather of Brahmaloka:%


\textbf{12.149}Vipula spoke:``Venerable sir, lord of all the worlds, Grandfather of all people,I can see a dream-like wonder, O lord of the thirty[-three] gods.My memory abandons me, my mind's intelligence is darkened.%
\footnote{Note that \Ed\ adds a line here (see the apparatus; translation:                 `I am a fool, how could I praise you? You are beyond knowledge, beyond the ultimate.').                 I have not been able to locate this line in any of the available sources.  }%


\textbf{12.150}... Be my refuge. Protect [me] from[?] terrible transmigration.I am afraid of being in a womb, of the terror of old age and death.                                 Protect me from the fetter of illusions.Dwelling in illness is eternal and the body is uncontollable.                                 Protect me from the noose of time.Animals eating each other[?] for hundreds and hundreds of \skt{yuga}.                                Protect [me] from the darkness of illusions.%


\textbf{12.151}Hearing [this] Brahmā spoke to [Vipula] of huge intellect, honouring [him] as follows.%
\footnote{The stem form noun °\skt{mati} of the bahuvrīhi compound                  in pāda a may stand for \skt{matiḥ} (see the unmetrical reading of \msCa\msCb\msNa), and                 then it should refer to Brahmā himself (`Brahmā, the one with a huge intellect...').                 I have choosen to take \skt{mati} as a stem form noun standing for the accusative,                 referring to Vipula. This works better because \skt{mānayitvā} (and \skt{śrutvā}) requires an object.  }You will live until the universal floods of destruction.                        You will not have any longing for being reborn any more.%
\footnote{Note \skt{āhūtasamplava} instead of the more common \skt{ābhūtasamplava} (see also 2.13).                         \skt{me} in pāda b is difficult to interpret.  }There will be no dwelling in a womb for you, no rebirth,                                         no anguish full of weariness.%
\footnote{I take \skt{tvan na} in pāda c as an ablative of \skt{tvad} used as a genitive plus \skt{na}.  }Killing the enemy who is the darkness of illusions,                        and you will reach the ultimate, the absorption into the Brahman.''%


\textbf{12.152}Maheśvara spoke:When [Vipula] was addressed thus by Brahmā,                Lord Viṣṇu (\skt{viṣṇunā prabhaviṣṇunā}) [said:]``Let it be like that, bless your soul, just as the Grandfather said.''%


\textbf{12.153}[Then] Indra, Ravi and Soma,the Sādhyas, the Ādityas, the Maruts, the Rudras, the Viśve[śas] and the Vasus[?] [spoke:]%


\textbf{12.154}``Wow, what a divine reward for great-souled Vipula's penance!He has reached heaven in his own [mortal] body by virtue of                                his worshipping a guest in good faith.''%
\footnote{\skt{svaśarīraṃ} may stand for \skt{svaśarīre} or \skt{svaśarīreṇa} in pāda c.  }%


\textbf{12.155}This and many other things are related in the Vipula section [probably                                of the \skt{Mahābhārata}, see MBh 13.39.1ff].Viṣṇu, the lord of the whole universe, turned back to Brahmā.%
