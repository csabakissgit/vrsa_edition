\documentclass[11pt]{book} % use larger type; default would be 10pt
%\input{vrsasara_macros.tex}
\usepackage[utf8x]{inputenx}
\newcommand{\skt}[1]{\textit{#1}}
\usepackage{background}
\backgroundsetup{contents={DRAFT}, color=gray, opacity=.18}




\begin{document}
\begin{center}
\textsc{\Large summary of the vṛṣasārasaṃgraha}
\medskip

Draft of 27 June 2021, Csaba Kiss
\end{center}

\bigskip

%\begin{itemize}

\thispagestyle{empty}
\textbf{Chapter 1}
\begin{itemize}
\item 1.1 benedictory stanza to the Extremely 
Subtle One (\skt{susūkṣma}) in Upajāti
\end{itemize}


\begin{center}
 DHARMAŚĀSTRIC LAYER STARTS
\end{center}

\begin{itemize}
\item 1.2--8 
   Janamejaya remains unsatisfied after hearing the 
   \skt{Mahābhārata} and asks Vaiśampāyana about
   the Secret Dharma. In reply, Vaiśam\-pāyana starts
   relating a dialogue between Vigatarāga (= Viṣṇu in disguise),
   and Anartha\-yajña, an ascetic, Vigatarāga testing Anartha\-yajña
   with questions
   \end{itemize}
   \begin{center}
 LAYER OF VAIṢṆAVA INTERLOCUTORS STARTS
\end{center}

\begin{itemize}

\item 1.9--10 Vigatarāga's question on \skt{brahmavidyā}
\item 1.11--31 \skt{kālapāśa}: his questions on the human body, 
   death (\skt{kāla}) and
    time (\skt{kāla}), 
   which leads naturally to an enumeration of the 
   divisions of time from \skt{kalā} to years, from \skt{yuga}s
   to \skt{kalpa}s
\item 1.32--36 \skt{parārdhādi}: numbers, the powers of
   ten from one to two hundred quadrillion (\skt{para})
\item 1.37--59 \skt{brahmāṇḍa}: Brahmā's Egg and the names of 
    the cosmic rulers
\item 1.60--63 \skt{pramāṇa}:  dimensions of the universe
\item 1.64--77 the descent of the Purāṇas; 
    names of their redactors from Brahmā to
    Dvaipāyana and Romaharṣa (does this indicate that the VSS 
    considers itself a Purāṇa?)
\end{itemize}

\textbf{Chapter 2}
\begin{itemize}
\item 2.1--3 Vigatarāga's question: the term \skt{śivāṇḍa} has been mentioned,
    what is that?
\item 2.4--40 Anarthayajña replies:
 Śiva's world is a sort of heaven without desires,
    anger, disease, rituals, \skt{yuga}s, etc.; its dimensions are given;
   its divisions belong to Īśāna, Tatpuruṣa, Aghora, Sadyojāta and
   Vāmadeva; it is approachable by \skt{śivayoga} and not
   by rituals; the last stanza mentions that \skt{kāla} destroys even
   Dharma
\end{itemize}

\pagebreak


\textbf{Chapter 3}
\begin{itemize}
\item 3.1--2 Vigatarāga's further questions:  
What is Dharma? 
How many embodiments does he have? 
Why is he known as a bull? 
How many legs does he have? 
How many are his paths? Whose son is Dharma? 
How many children does he have?
\item 3.3--7 Anarthayajña explains the etimology of \skt{dharma}; 
that it is a bull with four legs, the four \skt{āśrama}s; there are five paths
of Dharma; Dharma's father is Brahmā; Dharma has thirteen wives 
\item 3.8--13 Dharma's wives and children enumerated
\item 3.14--15 Dharma's two embodiments are Śruti and Smṛti;
the Smārta tradition focuses on  the conduct (\skt{ācāra}) of the classes (\skt{varṇa}) and life-stages (\skt{āśrama}),
which is connected to rules and regulations (\skt{yama-niyama}):\\
  \phantom{\raisebox{.7em}{Á}ÁÁÁÁ} The section on the Yama-Niyama rules starts\\
  \phantom{\raisebox{.7em}{Á}ÁÁÁÁÁÁÁÁÁÁÁÁÁÁÁ} Yama-rules  

\item \textbf{\skt{yama} 1} Non-violence (\skt{ahiṃsā})
 \begin{itemize}
\item 3.16--23  five types of violence to avoid:
\begin{enumerate}
\item frightening people
\item beating people
\item tying someone up 
\item killing someone 
\item destroying people’s livelihood
\end{enumerate}
\end{itemize}
\item 3.24--32 praise of Non-violence
\item 3.33--42 one should refrain from killing animals and eating meat; praise of Non-violence; some animals live happily without
eating meat; monkeys defeated Rākṣasas without eating meat
\end{itemize}

\pagebreak

 \textbf{Chapter 4}
\begin{itemize}
\item \textbf{\skt{yama} 2}  Truthfulness (\skt{satya})
   \begin{itemize}
  \item 4.1 definition of Truth
  \item 4.2--5 cases in which not telling the truth is right
  \item 4.6--17 praise or Truthfulness
  \end{itemize}

 \item \textbf{\skt{yama} 3}  Refraining from stealing (\skt{asteya})
  \begin{itemize}
  \item 4.18--24 five kinds of theft
  \begin{enumerate}
  \item theft 
  \item bribery
  \item cheating with weights
  \item cheating with scales
  \item robbery
  \end{enumerate}
  \item 4.25--30 praise of refraining from stealing and 
  condemnation of stealing and cheating
  \end{itemize}

\item \textbf{\skt{yama} 4}  Absence of cruelty (\skt{ānṛśaṃsya})
\begin{itemize}
\item 4.31 five kinds of cruelty, towards
  \begin{enumerate}
  \item  \skt{aṣṭamūrti} Śiva 
  \item  one's mother
  \item one's father
  \item cows
  \item guests
  \end{enumerate}
\item 4.32 eight-formed Śiva is the five elements, the Sun, the Moon and the sacrificer 
\item 4.33--34 the father and the mother
\item 4.35--39 praise of the cow and of the five products
     of the cow
\item 4.40--49 praise of the guest and of those who respect the guest;
   mention of the mongoose story of the \textit{Mahābhārata}
\end{itemize}

\pagebreak

\item \textbf{\skt{yama} 5}  Self-restraint (\skt{dama})
\begin{itemize}
\item 4.50--51 praise of Self-restraint
\item 4.52--55 without Self-restraint, 
the five sense-faculties lead one to death, 
as in the case of the elephant (touch), 
the fish (taste), the moth (seeing), the bee (smelling) 
and the deer (hearing) [see Aśvaghoṣa's \textit{Buddhacarita} 11.35]
\item 4.56--58 examples of people perishing because of a lack of
    self-restraint are given: Purūravas, Daṇḍaka, Sagara’s sons, 
    Rāvaṇa, Saudāsa, the Yādavas, Māndhātṛ, Nahuṣa,
    Mahābali, Arjuna, King Nala, Nṛga [cf.\ Aśvaghoṣa's   
    \textit{Buddhacarita} 11.13--17]
\item 4.59 warning against a lack of self-restraint 
\end{itemize}

\item \textbf{\skt{yama} 6}  Taboos (\skt{ghṛṇā})
\begin{itemize}
\item 4.60--67 five taboos
  \begin{enumerate}
  \item approaching women who are not depending on oneself
  \item taking away others' wealth
  \item taking away others' lives
  \item hurting others 
  \item eating others' food
  \end{enumerate}
\end{itemize}

\item \textbf{\skt{yama} 7} Virtues (\skt{dhanya}) 
\begin{itemize}
\item 4.68 the five methods of virtue
  \begin{enumerate}
  \item 4.69 four cases of observing silence (\skt{caturmauna})
      \begin{enumerate}
      \item instead of violent words
      \item slanderous words 
      \item lies
      \item idle talk
      \end{enumerate}
  \item 4.70 the four enemies (\skt{catuḥśatru})
        \begin{enumerate}
      \item desire
      \item anger 
      \item greed
      \item delusion
      \end{enumerate}
      
      \pagebreak

  \item 4.71 the four sanctuaries (\skt{caturāyatana} = the four Buddhist
              \skt{brahmavihāra}s)
          \begin{enumerate}
          \item compassion
          \item sympathy in joy 
          \item indifference
          \item benevolence
          \end{enumerate}
  \item 4.72--73 the four meditations (\skt{caturdhyāna})
            \begin{enumerate}
      \item of the Self
      \item of \skt{vidyā}
      \item of  \skt{bhava} [= Śiva?] 
      \item of the subtle one (\skt{sūkṣma})
            \end{enumerate}
  \item 4.74 the four-legged Dharma (\skt{catuṣpāda})
              \begin{enumerate}
      \item the householder
      \item the chaste student 
      \item the forest-dweller 
      \item the mendicant
            \end{enumerate}
   \end{enumerate}
\item 4.75--76 he who knows these will prosper
\end{itemize}

\item \textbf{\skt{yama} 8}  Absence of Negligence (\skt{apramāda})
\begin{itemize}
\item4.77--82 five areas of negligence (the \skt{mahāpātaka}s, 
                with quotes from Manu)
            \begin{enumerate}
            \item murdering a Brahmin
            \item drinking alcohol
            \item stealing
            \item having sex with the guru’s wife
            \item when one is connected with these sins or with people involved in these sinful acts
            \end{enumerate}
\end{itemize}

\item \textbf{\skt{yama} 9}  Charm (\skt{mādhurya})
\begin{itemize}
\item 4.83--85 five types of Charm
            \begin{enumerate}
            \item bodily
            \item verbal
            \item mental
            \item of the eyes
            \item of one's thoughts
            \end{enumerate}
\end{itemize}

\item \textbf{\skt{yama} 10}  Sincerity (\skt{ārjava})
\begin{itemize}
\item 4.86--88 five types of Sincerity
            \begin{enumerate}
            \item in action
            \item in livelihood
            \item in prosperity
            \item in gratifying others
            \item ... ?
            \end{enumerate}
\end{itemize}
\item 4.89 Śaiva closing verse of the Yama section in Mālinī
\end{itemize}

\bigskip
 \textbf{Chapter 5}\\
 \phantom{\raisebox{.7em}{Á}ÁÁÁÁÁÁÁÁÁÁÁÁÁÁÁÁÁ} 
  Niyama-rules  
\begin{itemize}
\item 5.1--3 introduction to the ten Niyama-rules
\item \textbf{\skt{niyama} 1}  Purity (\skt{śauca})
   \begin{itemize}
    \item  five types of \skt{śauca}
    \begin{enumerate}
    \item 5.4--9 bodily (\skt{śarīraśauca})
    \item 5.10--16 of food (\skt{āhāraśauca})
    \item of property (? \skt{mātrā}, not detailed)
    \item of conduct (? \skt{bhāva}, not detailed)
    \item ?
    \end{enumerate}
    \item 5.17 `See more in Manu, the Purāṇas, the Śaiva texts, and the \textit{Bhāratasaṃhitā} (= the  \textit{Mahābhārata})'
    \item 5.18--21 praise of purity
    \end{itemize}
\end{itemize}

 \textbf{Chapter 6}
\begin{itemize}
\item \textbf{\skt{niyama} 2}  Sacrifice (\skt{ijyā})
  \begin{itemize}
  \item 6.1--2 five types of Sacrifice
  \begin{enumerate}
  \item 6.3 \skt{arthayajña}
   \item 6.4 \skt{kriyāyajña}
   \item 6.5--6ab \skt{japayajña}
   \item 6.6cd--7 \skt{jñānayajña}
   \item 8.8--19 \skt{dhyānayajña}
  \end{enumerate}
\end{itemize}
  
\item \textbf{\skt{niyama} 3} Penance (\skt{tapas})
  \begin{itemize}
    \item 6.20 five types of Penance
      \begin{enumerate}
\item 6.21 mental (\skt{mānasa})
\item 6.22 verbal (\skt{vācika})
\item 6.23 bodily (\skt{kāyika})
\item 6.24 mental and verbal (\skt{manovākkarman}) 
\item 6.25 bodily and verbal (\skt{kāyika + vācika})
  \end{enumerate}
  \item 6.26--28 types of \skt{sādhana}
  \item closing verse in Śārdūlavikrīḍita
    \end{itemize}
\end{itemize}



 \textbf{Chapter 7}
  \begin{itemize}
\item \textbf{\skt{niyama} 4}  Donation (\skt{dāna})
  \begin{itemize}
\item 7.1 five types of Donation
  \begin{enumerate}
\item 7.2--5 of food (\skt{annadāna})
\item 7.6--13 of clothes (\skt{vastradāna})
\item 7.14--17 of gold (\skt{suvarṇadāna})
\item 7.18--24 of land (\skt{bhūmidāna})
\item 7.25 of cows (\skt{godāna})

  \end{enumerate}
\item 7.26--28 praise of donation (\skt{dānapraśaṃsā})
  \end{itemize}
  \end{itemize}

 \textbf{Chapter 8}
\begin{itemize}
\item 8.1--6 \textbf{\skt{niyama} 5}  Study (\skt{svādhyāya}) of
   \begin{enumerate}
    \item  \skt{śaiva} texts
    \item  \skt{sāṃkhya} texts
    \item  \skt{purāṇa}s
    \item  \skt{smārta} texts
    \item  the \skt{Bhāratasaṃhitā}
   \end{enumerate}
   
   \pagebreak

\item 8.7--13ab \textbf{\skt{niyama} 6}  Sexual restraint (\skt{upasthanigraha})
   \begin{enumerate}
   \item \skt{strī} 
   \item \skt{garhitotsarga} 
   \item \skt{svayaṃmukti}
   \item \skt{svapnopaghāta} 
   \item \skt{divāsvapna}
   \end{enumerate}
\item 8.13cd--18 \textbf{\skt{niyama} 7}  Observances (\skt{vrata})
\item 8.19--25 \textbf{\skt{niyama} 8}   Fasting (\skt{upavāsa})
\item 8.26--34 \textbf{\skt{niyama} 9}   Silence (\skt{mauna})
\item 8.35--44 \textbf{\skt{niyama} 10} Bathing  (\skt{snāna})
\item 8.45 closing verse of the Niyama-rules in Mālinī
\end{itemize}

 \textbf{Chapter 9}
\begin{itemize}
\item 9.1--39 the three \skt{guṇa}s \skt{sattva, rajas, tamas}
\item 9.40--44 the \skt{guṇātīta} state
\end{itemize}

 \textbf{Chapter 10}
\begin{itemize}
\item 10.1--2 introduction to \skt{kāyatīrtha}s
 \end{itemize}

   \begin{center}
LAYER OF ŚAIVA INTERLOCUTORS  STARTS
\end{center}

\begin{itemize}


\item 10.3 Nandikeśvara's narration of the dialogue 
   between Maheśvara and Devī starts
\item 10.4--8 external \skt{tīrtha}s 
\item 10.9--26 internal \skt{tīrtha}s
\item 10.27--29 \skt{oṃ} as a \skt{tīrtha}
\item 10.30 \skt{setubandha} (a mantra?)
\item 10.31--34 ?
\end{itemize}

\pagebreak

 \textbf{Chapter 11}
\begin{itemize}
\item the four \skt{āśrama}s and the concept of \skt{anarthayajña}
\end{itemize}

 \textbf{Chapter 12}
\begin{itemize}
\item \skt{ātithyadharma}: a narrative of the Brahmin called Vipula
\end{itemize}

 \textbf{Chapter 13}
\begin{itemize}
\item embryology
\end{itemize}

 \textbf{Chapter 14}
\begin{itemize}
\item ...
\end{itemize}

 \textbf{Chapter 15}
\begin{itemize}
\item the \skt{jīva}
\end{itemize}

\textbf{Chapter 16}
\begin{itemize}
\item yoga
\end{itemize}

\textbf{Chapter 17}
\begin{itemize}
\item \skt{dāna}
\end{itemize}

\textbf{Chapter 18}
\begin{itemize}
\item \skt{karman}
\end{itemize}


\begin{center}
 END OF  LAYER OF ŚAIVA INTERLOCUTORS
\end{center}

\begin{center}
BACK TO LAYER OF VAIṢṆAVA INTERLOCUTORS
\end{center}


\bigskip

\textbf{Chapter 19}
\begin{itemize}
\item \skt{dānayajña}
\end{itemize}

\textbf{Chapter 20}
\begin{itemize}
\item the 25 \skt{tattva}s
\end{itemize}

\pagebreak

\textbf{Chapter 21}
\begin{itemize}
\item Vigatarāga reveals his divine form as Viṣṇu in front of Anarthayajña;
Anarthayajña, now fully satisfied, praises him; they depart to Śvetadvīpa
\end{itemize}

\begin{center}
END OF LAYER OF VAIṢṆAVA INTERLOCUTORS
\end{center}

\begin{center}
BACK TO DHARMAŚĀSTRIC LAYER
\end{center}

\begin{itemize}
\item Vaiśampāyana instructs Janamejaya to follow Anarthayajña in
his devotion to Viṣṇu; Janamejaya's further inquiry on \ae ons (\skt{kalpa})
\end{itemize}


\textbf{Chapter 22}
\begin{itemize}
\item details on Anarthayajña the  yogin's life and practice, \skt{daśayajña}, \skt{daśākṣa\-ra\-mantra} etc. 
\end{itemize}

\textbf{Chapter 23}
\begin{itemize}
\item \skt{dharmādharma} and \skt{nidrotpatti}
\end{itemize}

\textbf{Chapter 24}
\begin{itemize}
\item \skt{trailokya}, \skt{naraka}s, \skt{dvīpa}s
\end{itemize}
%\end{itemize}

\end{document}
