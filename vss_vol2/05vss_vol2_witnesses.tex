\section{Witnesses}

\fancyhead[CE]{{\footnotesize \textit{Vṛṣasārasaṃgraha}}}
\fancyhead[CO]{{\footnotesize \textit{Witnesses}}}
\fancyhead[LE]{}
\fancyhead[RE]{}
\fancyhead[LO]{}
\fancyhead[RO]{}

\noindent
In the pre-modern era, the \VSS\ has been transmitted exclusively in multiple-text manuscripts that were produced in Nepal. Even when a
manuscript of the \VSS\ seems to be a single-text MS, 
chances are high that it originally belonged to a multiple-text
manuscript.%
	\footnote{\label{noteonKolkataMs}As I remarked elsewhere 
	(\mycitep{KissVolume2021}{185, n.~9}):
	`Asiatic Society (Calcutta), Manuscript G 4076, cat. no. 4083, 
	may seem to be an independent manuscript of the 
	\textit{Vṛṣasārasaṃgraha}, but as De Simini has already 
	remarked (2016b, 240 n. 19),  % [= \mycite{DeSiminiMSSFromNepal2016}],
	it is probably from a multiple text manuscript. In fact, from what
	can be gathered from its description in
	\mycitep{SastriCatalogue5}{716ff},
	it seems likely that this manuscript was
	originally part of manuscript Asiatic Society (Calcutta) G 3852, cat.\
	no.\ 4085. See for example the folio numbering in these two 
	manuscripts: ASC G 3852 contains 210 folios, 
	and ASC G 4076 starts on folio 210.'}
In the manuscript descriptions below, in addition to some general
remarks, I will mainly focus on information relevant to the \VSS. For
much more detail on the overall features of these manuscripts, see 
\mycite{DeSiminiMSSFromNepal2016}, \mycite{BisschopUniversal}, 
\mycite{SaivaUtopia}, \mycite{SDhS10_ed},  
and the catalogues I mention
at some of the individual manuscript.%
		\footnote{I owe thanks to Florinda De Simini for 
			sharing with me most of the manuscripts listed here, to
  			Kengo Harimoto and Gudrun Melzer (Munich) for 
  			providing photos of the  Munich MS, and to 
  			Nirajan Kafle for sharing a digital 
  			copy of the Paris MS with me.}

In recently published and forthcoming critical editions of and articles
on the Śivadharma corpus,%
