% !TEX TS-program = xelatex
% !TEX encoding = UTF-8

% So this is the main file for the VSS book (XeLaTeX)
% it requires
% vssbook_vol2_macros.tex and other macro files

\documentclass[11pt]{book}

%%%%%%%%%%%%%%%%%%%%%%
%%%% LOAD MACROS %%%%%
%%%%%%%%%%%%%%%%%%%%%%
\input{vssbook_vol2_macros.tex}

\frenchspacing
\usepackage[cam, center,cross,axes,color=blue]{crop}

\makeindex 

\begin{document}
%\pagenumbering{roman}



%%%%%%%%%%%%%%%%%%%%%%%%%%%%%%%%
%%%%%%%%% FRONT MATTER %%%%%%%%%
%%%%%%%%%%%%%%%%%%%%%%%%%%%%%%%%
\thispagestyle{empty}
\ \vskip5cm

\begin{center}
\textit{\large{\Large\newarfont 𑐰𑐺𑐲𑐳𑐵𑐬𑐳𑑄𑐐𑑂𑐬𑐴𑑅 } \\
\vskip.3em
        Vṛṣasārasaṃgraha\\ \vspace{.2em}Volume 1}\\ 
\end{center}
\vfill
\pagebreak





\thispagestyle{empty}
\ \vskip3cm

\begin{center}
\textsc{Università di Napoli L’Orientale\\
Dipartimento Asia, Africa e Mediterraneo}
\vskip.8cm

\textsc{The Śivadharma Project}

\vskip1cm

\textit{Studies on the History of Śaivism}\\
IV

\vskip1cm


\textit{Editor-in-Chief}\\
        Florinda De Simini\\
        (Università di Napoli L'Orientale)
\bigskip

\textit{Editorial \&\ Scientific Board}\\
Andrea Acri (École Pratique des Hautes Études),
Peter C. Bisschop (Universiteit Leiden), 
Dominic Goodall (École française d’Extrême-Orient),
Kengo Harimoto (Università di Napoli L'Orientale),
Csaba Kiss (Università di Napoli L'Orientale), 
Krishnaswamy Nachimuthu (École française d’Extrême-Orient),
Srilata Raman (University of Toronto),
Annette Schmiedchen (Humboldt-Universität zu Berlin), 
Judit Törzsök (École Pratique des Hautes Études),
Margherita Trento (Institut français de Pondichéry -- École des hautes études en sciences sociales), 
Yuko Yokochi (Kyoto University)

\vskip.6cm

\includegraphics[scale=.3]{shivadharma_project_logo.jpg}
%\includegraphics[scale=.3]{eu.png}
\raisebox{-1.1em}{\includegraphics[scale=.2]{LOGO_ERC-FLAG_FP.png}}

\end{center}
\pagebreak








\thispagestyle{empty}
\ \vskip3cm

\begin{center}
\textsc{Università di Napoli L’Orientale\\
Dipartimento Asia, Africa e Mediterraneo}
\vskip.5cm

\textsc{The Śivadharma Project}

\vskip1cm

\textit{Studies on the History of Śaivism}\\
IV

\vskip2cm

\textit{\large 
	Vṛṣasārasaṃgraha: a Nepalese text of the Śivadharma corpus\\
\vskip.5em	
        Volume 1: Anarthayajña's Sacrifice (Chapters 1--12)}\\
\vskip.8em	
 	A Critical Edition and Annotated Translation 
\vskip.3em	
\vskip1cm

{\footnotesize  Csaba Kiss}

\vfill
\includegraphics[scale=.2]{images/up.png}

Napoli 2026
\end{center}


\pagebreak
\thispagestyle{empty}
\ \vskip5cm

\begin{center}
UniorPress

{\footnotesize Nuova Marina, 59 - 80133, Napoli\\
uniorpress@unior.it}
\bigskip
\bigskip

%\includegraphics[scale=.3]{cc.png}
\includegraphics[scale=1]{by-nc-nd.png}
\smallskip


%This work is licensed under a Creative Commons\\
%Attribution 4.0 International License
This is an open access publication distributed under the terms of the\\
CC BY-NC-ND 4.0 license, which
permits any non-commercial use, distribution, and reproduction in any medium, provided no
alterations are made and the original authors and source are credited.
\bigskip

This book has been realised thanks to the financial support of the \textsc{shivadharma} project
(ERC No.\thinspace 803624) CHECK.

\bigskip
ISBN 978-88-6719-???-?
\medskip

Typeset in EB Garamond and Sanskrit2003 by Csaba Kiss,\\
using \XeLaTeX, Bib\TeX, \textit{MakeIndex}, ledmac, and Python

\bigskip
%        Cover photo (American Institute of Indian Studies): a scene from Cave 3 near the Nīlakaṇṭha temple at Kalinjar Fort
%OR
Cover photo (......): ....................................................
\medskip

Cover design: XXX and YYY 
\medskip

Stampato in Italia

Il presente volume è stato sottoposto al vaglio di due revisori anonimi


\end{center}


\pagebreak




%%%%%%%%%%%%%%%%%%%%
% TABLE OF CONTENT %
%%%%%%%%%%%%%%%%%%%%
\fancyhead[CO]{}
\fancyhead[CE]{}
\fancyhead[RO]{}
\fancyhead[RE]{}
\fancyhead[LO]{}
\fancyhead[LE]{}

\renewcommand{\contentsname}{{\hfill
        {\normalfont\Large\englishfont Table of Contents}\hfill}}
\thispagestyle{empty}
\pagenumbering{gobble}
\tableofcontents

\vfill
\pagebreak

\thispagestyle{empty}
\renewcommand{\listfigurename}{{\hfill
        {\normalfont\Large\englishfont List of Figures}\hfill}}
\listoffigures


%%%%%%%%%%%%%%%%%%%%%%%%%%%%%%%%
% ACKNOWLEDGEMENTS and PREFACE %
%%%%%%%%%%%%%%%%%%%%%%%%%%%%%%%%

%\pagenumbering{arabic}
%\thispagestyle{empty}
\chapter*{Acknowledgements}
\label{acknowledgements}
\vspace{-1em}

\noindent
First of all, I am grateful to Florinda De Simini for
encouraging me to apply for a position in her 
\textsc{\hbox{śivadharma} project} (ERC no.\thinspace 803624),
for sharing all the relevant manuscript material with me, 
giving invaluable advice whenever needed, and in general
leading the project in the most friendly and
generous way through happy times as well as 
difficult Covid-affected years. 
While working on the \Vss, I was also affiliated with 
another ERC project, the \textsc{dharma project} (ERC no.\thinspace 809994).
I am grateful to all my colleagues involved in that endeavour,
including Arlo Griffith, Emmanuel Francis, Annette Schmiedchen, 
Astrid Zotter, and Dániel Balogh.

As always, I must express my gratitude to my former supervisor,
Alexis Sanderson, and to Dominic Goodall and 
Harunaga Isaacson, for initiating me into Sanskrit philology and
the study of Śaivism.

My colleagues and friends working in Naples or visiting for
shorter periods helped me on a daily basis, during our regular reading sessions and countless
other ways. I am thankful to them:
 to Florinda De Simini, Judit Törzsök, Nirajan Kafle, Kengo Harimoto, Csaba Dezső, Gergely Hidas,
Giulia Buriola, Alessandro Battistini, Lucas den Boer, Torsten Gerloff, Kenji Takahashi, Francesco Sferra, Dorotea Operato, Daniela Cappello,
Michael Bluett, Marco Franceschini, Martina Dello Buono, Chiara Livio, Margherita Trento, Nina Mirnig, Timothy Lubin, S.A.S. Sarma,
R.\thinspace Sathyanarayanan, Alexander von Rospatt, Martin Orwin, Luca Piscopo, and others.
% students in Naples?
I am grateful to Daniela Cappello, Marco Franceschini, and Sushmita Das for their great efforts in acquiring manu\-scripts in Calcutta.

During my visit to the National Archives in
Kathmandu, the staff were as helpful and professional as ever.
I wish to express my thanks to Jyoti Neupane, Manita Neupane, Saubhagya Pradhananga, Rubin Shrestha, Sahan Ranjitkar, and all %sribin7501@gmail.com
other members of the team.

I thank my host in Capodimonte, Michele Costagliola, for his generosity. 
I am infinitely grateful to my family for always supporting me unwaveringly.

\vfill
\pagebreak

\noindent

\CHECK\ REVISE!!!

{\footnotesize
The present publication is a result of the project \textsc{dharma} `The
Domestication of ``Hindu'' Asceticism and the Religious Making of South and Southeast Asia'.  This project has received funding from the European Research Council (ERC) under the European Union's Horizon 2020 research and innovation programme (grant agreement no.\thinspace 809994).  This book reflects the views of the author only.  The funding body is not responsible for any use that
may be made of the information contained therein.}

%CHECK what to add for the Śivadharma project?

\vfill
\pagebreak


%\input{02vss_vol2_preface.tex}
%
\fancyhead[CE]{\textit{\footnotesize ‘...not satisfied with the Mahābhārata...’ (śrutvā bhāratasaṃhitām atṛptaḥ)}}
\fancyhead[CO]{\textit{\footnotesize ‘...not satisfied with the Mahābhārata...’ (śrutvā bhāratasaṃhitām atṛptaḥ)}}

\noindent
primary mission of the Vṛṣasārasaṃgraha must have been similar to that of
the Lalitavistara, another, less successfully surviving, text of the Śivadhar\-ma 
corpus: the Vṛṣasārasaṃgraha too must have been aiming at ‘harmo\-nising aspects of Śaiva and Vaiṣṇava dharma’ (De Simini and Mirnig 2017,
649) and probably of a number of related philosophical schools and reli\-gious currents.

There seems to be even more to the Vṛṣasārasaṃgraha’s aspirations. It
would appear difficult to find any further leitmotif in this impressively rich
material, in which innumerable traditions intermingle, or to understand
what other role this text could have played in the formation of the Śivadha\-rma corpus, if one thing did not stand out clearly: the figure of Anarthayajña.

\vspace{1em}
\noindent
2. \textit{Anarthayajña’s sacrifice and the āśramas}
\vspace{.3em}

\noindent
As we have seen, Anarthayajña is the interlocutor of the sections that can be
labelled Vaiṣṇava and his name also appears in other parts of the text. That
he is part of a Vaiṣṇava setting in chapters one to ten and nineteen to twen­
ty-one is also certain from the observation that when he has answered all of
Vigatarāga’s (Viṣṇu’s) questions in detail, and when Viṣṇu reveals himself,
they are described as departing to Viṣṇuloka together,27 thus offering the
impression that Anarthayajña is a devotee of Viṣṇu. One could argue that
Viṣṇu’s position as a pupil and the fact that he is being taught Śaiva material
(in the Śaiva chapters) point towards the possibility that Anarthayajña is
a Śaiva who converts Viṣṇu, thus turning most of the Vṛṣasārasaṃgraha
into a Śaiva-oriented text; but the episode in which Viṣṇu steps forward and
Anarthayajña praises him throughout thirteen jagatī stanzas (21.9–21) be-

The title of the commentary and everything supplied between
double square brackets is from the editors. The verse numbers of
the verses that are commented upon in the commentary are given
between double square brackets; the beginning and end is also
indicated by double square brackets. Similarly, line numbers of the
manuscript are given in double square brackets. The pratīkas of the
root text are marked in bold. Application of sandhi rules are silently
standardised here. In general we follow the placement of the dañḍas
by the manuscript, but we have occasionally placed them according
to our own understanding as well.

\vspace{1em}
\noindent
4. \textit{A Note on the Translation}

\vspace{.3em}
\noindent
In order to make the commentary accessible to a broader readership,
we have included a running English translation of it. In translating



\vfill
\pagebreak




%%%%%%%%%%%%%%%%
% INTRODUCTION %
%%%%%%%%%%%%%%%%
\pagenumbering{arabic}
\fancyfoot[C]{{\thepage}}

%
\mychapter{Introduction}

\thispagestyle{empty}
\frenchspacing

\section{Śivadharma corpus}
\fancyhead[CE]{{\footnotesize \textit{Vṛṣasārasaṃgraha}}}
\fancyhead[CO]{{\footnotesize \textit{Introduction}}}
\fancyhead[LE]{}
\fancyhead[RE]{}
\fancyhead[LO]{}
\fancyhead[RO]{}

The \Vss\ (\VSS), a 24-chapter-long Sanskrit Śaiva text,
has always%

\vfill
\pagebreak










%%%%%%%%%%%%%%%%%%%%
% CRITICAL EDITION %
%%%%%%%%%%%%%%%%%%%%
\mychapter{Introduction to the Critical Edition}
%\input{04vss_vol2_cred_intro}

\bigskip

%\section{Witnesses}

\fancyhead[CE]{{\footnotesize \textit{Vṛṣasārasaṃgraha}}}
\fancyhead[CO]{{\footnotesize \textit{Witnesses}}}
\fancyhead[LE]{}
\fancyhead[RE]{}
\fancyhead[LO]{}
\fancyhead[RO]{}

\noindent
In the pre-modern era, the \VSS\ has been transmitted exclusively in multiple-text manuscripts that were produced in Nepal. Even when a
manuscript of the \VSS\ seems to be a single-text MS, 
chances are high that it originally belonged to a multiple-text
manuscript.%
	\footnote{\label{noteonKolkataMs}As I remarked elsewhere 
	(\mycitep{KissVolume2021}{185, n.~9}):
	`Asiatic Society (Calcutta), Manuscript G 4076, cat. no. 4083, 
	may seem to be an independent manuscript of the 
	\textit{Vṛṣasārasaṃgraha}, but as De Simini has already 
	remarked (2016b, 240 n. 19),  % [= \mycite{DeSiminiMSSFromNepal2016}],
	it is probably from a multiple text manuscript. In fact, from what
	can be gathered from its description in
	\mycitep{SastriCatalogue5}{716ff},
	it seems likely that this manuscript was
	originally part of manuscript Asiatic Society (Calcutta) G 3852, cat.\
	no.\ 4085. See for example the folio numbering in these two 
	manuscripts: ASC G 3852 contains 210 folios, 
	and ASC G 4076 starts on folio 210.'}
In the manuscript descriptions below, in addition to some general
remarks, I will mainly focus on information relevant to the \VSS. For
much more detail on the overall features of these manuscripts, see 
\mycite{DeSiminiMSSFromNepal2016}, \mycite{BisschopUniversal}, 
\mycite{SaivaUtopia}, \mycite{SDhS10_ed},  
and the catalogues I mention
at some of the individual manuscript.%
		\footnote{I owe thanks to Florinda De Simini for 
			sharing with me most of the manuscripts listed here, to
  			Kengo Harimoto and Gudrun Melzer (Munich) for 
  			providing photos of the  Munich MS, and to 
  			Nirajan Kafle for sharing a digital 
  			copy of the Paris MS with me.}

In recently published and forthcoming critical editions of and articles
on the Śivadharma corpus,%

\vfill
\pagebreak

\blankpage

\pagebreak
% set page nember also in the edition's tex source file
%\setcounter{page}{500} % set it in the edition file (.tex, see path below)

%\section{The Sanskrit text}
\mychapter{A Critical Edition of Vṛṣasārasaṃgraha 13--24}

\pagebreak

\blankpage
%\addcontentsline{toc}{chapter}{A Critical Edition of Vṛṣasārasaṃgraha 1--12}
\addcontentsline{toc}{section}{Sanskrit Text}
\includepdf[noautoscale, pages=-]{vrsasara_vol2_ed.pdf}




%%%%%%%%%%%%%%%
% TRANSLATION %
%%%%%%%%%%%%%%%


\mychapter{An Annotated Translation of Vṛṣasārasaṃgraha 13--24}
%\addcontentsline{toc}{section}{\Vss\ 1--12}


%FOOTNOTE
% Reformat footnotes
\makeatletter 
\renewcommand{\@makefntext}[1]{%
  \setlength{\parindent}{20pt}%
  %\begin{list}{}
  {\setlength{\labelwidth}{6mm}% 1.5em
    \setlength{\leftmargin}{0pt}%{\labelwidth}%
    \setlength{\labelsep}{10pt}%
    \setlength{\itemsep}{0pt}%
    \setlength{\parsep}{0pt}%
    \setlength{\topsep}{0pt}%
    \footnotesize}%
  %\item[\@thefnmark\hfil]
  #1% @makefnmark
  %\end{list}%
}
\makeatother 
\let\svthefootnote\thefootnote



\fancyhead[CE]{{\footnotesize \textit{Vṛṣasārasaṃgraha}}}
\fancyhead[CO]{{\footnotesize \textit{Translation of chapter 13}}}
%%%%\includepdf[pages=-]{/home/csaba/indology/dharma_project/vrsa_edition/vss_translation.pdf}


% CORRECT PAGE NUMBER AFTER SKT EDITION !
\setcounter{page}{1001}
%%%%%
  \chptr{prathamo 'dhyāyaḥ}
\fancyhead[CE]{{\footnotesize\textit{Translation of chapter 1}}}%

  \trchptr{Chapter One}%

  \subchptr{stutiḥ}%

  \trsubchptr{Invocation}%

  \maintext{anādimadhyāntam anantapāraṃ}%

 \nonanustubhindent \maintext{susūkṣmam avyaktajagatsusāram |}%

  \maintext{harīndrabrahmādibhir āsamagraṃ}%

 \nonanustubhindent \maintext{praṇamya vakṣye vṛṣasārasaṃgraham }||\thinspace1:1\thinspace||%
\translation{Having bowed to the One who has no beginning, no middle part and no end, whose boundaries are limitless, who is very subtle and who is the unmanifest and fine essence of the world, to the One who is wholly complete with Hari, Indra, Brahmā and the other [gods], I shall recite [the work called] `A Compendium on the Essence of the Bull [of Dharma]'. \blankfootnote{1.1 \textit{Pāda} a is reminiscent of, among other famous passages, \BHG\ 11.19:
  %
 \textit{anādimadhyāntam anantavīryam  
 anantabāhuṃ śaśisūryanetram |  
 paśyāmi tvāṃ dīptahutāśavaktraṃ  
 svatejasā viśvam idaṃ tapantam ||}.
  %
 See also \BHG\ 10.20cd:
  %
 \textit{aham ādiś ca madhyaṃ ca bhūtānām anta eva ca} ||.
  %
 
 A faint reference to the \BHG\ seems proper at the
 beginning of a work that claims to deliver a teaching 
 based on, but also to surpass, the \MBH\ {\rm (}see following verses of the \VSS{\rm )}.
 Compare also, e.g., \KURMP\ 1.11.237:
  %
  \textit{rūpaṃ tavāśeṣakalāvihīnam 
  agocaraṃ nirmalam ekarūpam |  
  anādimadhyāntam anantam ādyaṃ  
  namāmi satyaṃ tamasaḥ parastāt} ||.
  %
 To say that a god has no beginning and no end in a temporal or spacial sense is natural
 {\rm (}\textit{anādi°...°antam}{\rm )}, but to have no `middle part' {\rm (}\textit{°madhya°}{\rm )} in these senses is slightly less so.
 Thus the rather commonly occuring phrase \textit{anādimadhyāntam} is probably a fixed expression usually 
 referring to a deity that is endless, eternal and immaterial. 
 As to which deity or what form of a deity this stanza refers to, 
 it may be Śiva, his name missing in pāda c, but the phrasing of the verse 
 is vague enough to keep the question somewhat open: the impersonal Brahman 
 might be another option, even more so if we look at verses 1.9--10, whose
 topic is \textit{brahmavidyā}.
 
 
  %
 In \textit{pāda} b \textit{jagat-susāraṃ} is most probably not 
 to be interpreted as \textit{jagatsu sāraṃ} {\rm (}`the essence in the worlds'{\rm )}.
 Another way to translate \textit{avyaktajagatsusāraṃ} would be: 
 `who is the fine essence of the unmanifest world.'
 
  %
 Strictly speaking, \textit{pāda} c is unmetrical, but it is better to 
 simply acknowledge here the phenomenon of `muta cum liquida', namely
 that syllables followed by consonant clusters such as 
 \textit{ra, bra, hra, kra, śra, śya, śva, sva, dva} can be treated as short {\rm (}\textit{laghu}{\rm )}.
 {\rm (}See Introduction \CHECK{\rm )}
 Thus \textit{harīndrabrahmā°} can be treated as a regular beginning
 of an \textit{upajāti} {\rm (}\shortsyllable\ - \shortsyllable\ - -{\rm )}, the syllable 
 \textit{bra} not turning the previous syllable long {\rm (}\textit{guru}{\rm )}.
 
  %
 The reading \textit{āsamagraṃ} in \textit{pāda} c is suspect,
 although the initial \textit{ā-} might convey some sort of
 completeness, meaning `all round'
 {\rm (}see e.g. \mycitep{KaleHigherGrammar}{226}{\rm )}.
 The fact that we could percieve
 the ending of \textit{pāda}s a and b {\rm (}\textit{pāraṃ}--\textit{sāram}{\rm )}, 
 as well as \textit{pāda}s c and d, as {\rm (}in the latter case, oddly{\rm )} rhyming pairs {\rm (}\textit{graṃ}-\textit{graham}{\rm )}
 suggests that accepting the reading
 \textit{āsamagram} could be the right decision
 {\rm (}as suggested by Alessandro Battistini{\rm )}.
 I translate this verse accordingly. \msM\ gives an exciting,
 albeit unmetrical, alternative {\rm (}\textit{yat samagraṃ}{\rm )}, but
 this seems more like a guess to me than the correct reading.
 For some time I was considering emending \textit{āsamagraṃ}.
 The most tempting of all the possible options 
 {\rm (}\textit{arcyam/arhyam/arghyam/īḍyam/āḍhyam agraṃ, āsamastaṃ}{\rm )} 
 seemed to be \textit{āptam agraṃ},
 meaning `appointed/received/respected [by Hari, Indra,
 Brahmā etc.] as the foremost one'. The fact that 
 the \textit{akṣara}s \textit{āsam} and \textit{āptam} look similar in most
 of the scripts used in our manuscripts could support this
 conjecture. \textit{āptam} could also
 possibly refer to the text itself, although then the
 syntax becomes slightly confusing: `I shall recite the
 \textit{Vṛṣasārasaṃgraha} that was first received by Hari...' etc.
 Another candidate was \textit{āḍhyam agram}:
 `Having bowed to [Him] who contains/is rich with Hari, Indra, Brahmā
 etc.' I have not emended the text because it is difficult
 to know if there is any need for change and if there is, which reading 
 to chose. There was no consensus when this verse was discussed 
 in our extended Śivadharma reading group.
 
  %
 Pāda d seems hypermetrical, but it can be interpreted as a \textit{vaṃśastha}
 line, a change from \textit{triṣṭubh} to \textit{jagatī} {\rm (}as suggested by Dominic Goodall{\rm )}.
 }}

  \subchptr{janamejayavaiśampāyanasaṃvādaḥ}%

  \trsubchptr{The dialogue of Janamejaya and Vaiśampāyana}%

  \maintext{śatasāhasrikaṃ granthaṃ sahasrādhyāyam uttamam |}%

  \maintext{parva cāsya śataṃ pūrṇaṃ śrutvā bhāratasaṃhitām }||\thinspace1:2\thinspace||%
\translation{Having listened to the \textit{Bhāratasaṃhitā} [i.e. the \textit{Mahābhārata}], the supreme book of a hundred thousand [verses] and a thousand chapters {\rm (}\textit{adhyāya}{\rm )}, with all its hundred sections {\rm (}\textit{parvan}{\rm )}, \blankfootnote{1.2 The dialogue of Janamejaya and Vaiśampāyana makes up the outermost layer of the \VSS\ 
  {\rm (}except for the introductory stanzas 1.1--3; see Introduction \CHECK{\rm )}, mostly containing
  general \textit{dharmaśāstric} material.
   %
  That the \MBH\ should contain a hundred thousand verses is hinted at e.g. in line 19 of
  the Khoh Charter 2 of Śarvanātha, year 214 {\rm (}Siddham IN00088: 
  \textit{uktañ ca mahābhārate śatasāhasryaṃ} {\rm (}understand °\textit{ryāṃ}{\rm )} \textit{saṃhitāyāṃ}... 
  The hundred \textit{parvan}s of the \textit{Mahābhārata} are listed in \MBH\ 1.2.33--70.
 }}

  \maintext{atṛptaḥ puna papraccha vaiśampāyanam eva hi |}%

  \maintext{janamejaya yat pūrvaṃ tac chṛṇu tvam atandritaḥ }||\thinspace1:3\thinspace||%
\translation{Janamejaya remained unsatisfied. Listen unweariedly to what he asked Vaiśampāyana in the past. \blankfootnote{1.3 My emendation from the unmetrical \textit{punaḥ} to the unusual, or rather, Middle Indic 
  {\rm (}\mycitep{EdgertonHybrid}{vol. 2, p. 347}{\rm )}, \textit{puna} is based
  on the assumption that in the original the metre must have overridden 
  morphology, similarily to what may have happened in 8.44d {\rm (}Mālinī metre{\rm )}:
  \textit{na bhavati punajanma kalpakoṭyāyute 'pi}, and in 12.151c {\rm (}Sragdharā metre{\rm )}:
  \textit{garbhāvāsaṃ na ca tvan na ca punamaraṇaṃ kleśam āyāsapūrṇam}.
 
   %
  For an unsatisfaction or dissatisfaction {\rm (}\textit{atṛpti}{\rm )} with previous 
  teachings in a somewhat similar manner to what
  Janamejaya experiences here, see e.g. \textit{Niśvāsa} mūla 1.9:
   %
  \textit{vedāntaṃ viditaṃ deva sāṃkhyaṃ vai pañcaviṃśakam |
  na ca tṛptiṃ gamiṣyāmo hy ṛte śaivād anugrahāt ||};
   %
  and the \SDhS\... \CHECK.
 Vaiśampāyana, a Ṛṣi, disciple of Vyāsa, great-grandson to Arjuna,
  recited the Mahābhārata at the snake sacrifice of 
  Janamejaya. This setting is an echo of the starting point of the Mahābhārata, see \MBH\ 1.1.8ff.
  In fact the next few verses in the \VSS\ make it clear that the \VSS\
  picks up where the Mahābhārata left off: Janamejaya has heard the whole Mahābhārata from
  Vaiśampāyana, but he is eager to hear more.
   %
  Note how we are forced to emend \textit{pāda} c to contain a stem form proper noun {\rm (}\textit{janamejaya}{\rm )}
  to maintain the metre, and note how the manuscripts struggle with this \textit{pāda}.
  Stem form nouns, \textit{prātipadika}s, abound in the \VSS: see Introduction p. \CHECK. 
 }}

  \maintext{janamejaya uvāca |}%

  \maintext{bhagavan sarvadharmajña sarvaśāstraviśārada |}%

  \maintext{asti dharmaṃ paraṃ guhyaṃ saṃsārārṇavatāraṇam }||\thinspace1:4\thinspace||%
\translation{Janamejaya spoke: O venerable sir, O knower of the entire Dharma, O you who are well-versed in all the sciences {\rm (}\textit{śāstra}{\rm )}! There is a supreme and secret Dharma [that causes] liberation from the ocean of mundane existence {\rm (}\textit{saṃsāra}{\rm )}. \blankfootnote{1.4 Note \textit{dharma} as a neuter noun in \textit{pāda} c and in the next verse.
 }}

  \maintext{dvaipāyanamukhodgīrṇaṃ dharmaṃ vā yad dvijottama |}%

  \maintext{kathayasva hi me tṛptiṃ kuru yatnāt tapodhana }||\thinspace1:5\thinspace||%
\translation{Teach me the Dharma that emerged from [Vyāsa] Dvaipāyana's mouth, O best of Brahmins. Help me find satisfaction at all cost, O great ascetic! \blankfootnote{1.5 The majority of the MSS consulted include a \textit{vā} in \textit{pāda} b, 
  and although \msCb's reading seems a bit smoother, that manuscript rarely gives superior readings.
  Therefore I have chosen \textit{dharmaṃ vā yad}, in which \textit{vā} functions probably in a weak sense.
  That the secret Dharma Janamejaya is seeking is the one taught by Vyāsa Dvaipāyana, and
  thus no real options are involved here, becomes clear in 1.6cd.
  The reading of \msM\ in \textit{pāda} b {\rm (}\textit{dharmavākyaṃ}{\rm )} is tempting but could be a later correction.
 In general, \msM's readings here are unique but probably secondary: \textit{hi me tṛptiṃ} in \textit{pāda} c seems more
  attractive than \msM's \textit{prasādena} because it echoes \textit{atṛptaḥ} in 1.3a
 }}

  \maintext{vaiśampāyana uvāca |}%

  \maintext{śṛṇu rājann avahito dharmākhyānam anuttamam |}%

  \maintext{vyāsānugrahasamprāptaṃ guhyadharmaṃ śṛṇotu me }||\thinspace1:6\thinspace||%
\translation{Vaiśampāyana spoke: Listen with great attention, O king, to this unsurpassed narration of Dharma. Hear the secret Dharma that I received by Vyāsa's favour. }

  \maintext{anarthayajñakartāraṃ tapovrataparāyaṇam |}%

  \maintext{śīlaśaucasamācāraṃ sarvabhūtadayāparam }||\thinspace1:7\thinspace||%
{\blankfootnote{1.7 Note the odd syntax here: \textit{viṣṇunā... dvijarūpadharo bhūtvā papraccha}.
  The agent of the active verb is in the instrumental case {\rm (}anacoluthic structure{\rm )}.
   %
  On Anarthayajña, the interlocutor of VSS 1.9--10.2 and 19.1--21.22, and
  an important figure discussed in 22.3ff, as well as a concept {\rm (}`nonmaterial sacrifice'{\rm )},
  see \mycite{KissVolume2021} and Introduction \CHECK.
 }}

  \maintext{jijñāsanārthaṃ praśnaikaṃ viṣṇunā prabhaviṣṇunā |}%

  \maintext{dvijarūpadharo bhūtvā papraccha vinayānvitaḥ }||\thinspace1:8\thinspace||%
\translation{Viṣṇu, the great Lord, assuming the form of a twice-born [Brahmin], wanted to test [Anarthayajña, the ascetic yogin] who performed nonmaterial sacrifices {\rm (}\textit{anarthayajña}{\rm )}, focused on his austerities and observances, whose conduct was virtuous and pure, and who was intent on compassion towards all living beings; therefore he [Viṣṇu] humbly asked him a question. }

  \subchptr{brahmavidyā}%

  \trsubchptr{The knowledge of Brahman}%

  \maintext{{\rm [}vigatarāga uvāca |{\rm ]}}%

  \maintext{brahmavidyā kathaṃ jñeyā rūpavarṇavivarjitā |}%

  \maintext{svaravyañjananirmuktam akṣaraṃ kimu tatparam }||\thinspace1:9\thinspace||%
\translation{[Vigatarāga spoke:] How is the knowledge of the Brahman to be understood if it is devoid of form and colour? The syllable that is devoid of vowels and consonants: is there anything higher than that? \blankfootnote{1.9 The translation of this verse, and the reconstruction and interpretation
  of \textit{pāda} d, which is echoed in 1.10d, is slightly tentative.
  I doubt if \textit{kimu} could have the standard {\rm (}Vedic{\rm )} meaning `how much more/less'
  here. Rather \textit{u} is probably just an expletive. In general it seems that
  this verse references the syllable \textit{oṃ}.
 }}

  \maintext{anarthayajña uvāca |}%

  \maintext{anuccāryam asandigdham avicchinnam anākulam |}%

  \maintext{nirmalaṃ sarvagaṃ sūkṣmam akṣaraṃ kimu tatparam }||\thinspace1:10\thinspace||%
\translation{Anarthayajña replied: That syllable is not to be pronounced, is unquestionable, non-dividable, consistent, spotless, all-pervading and subtle: what could be higher than that? }

  \subchptr{kālapāśaḥ}%

  \trsubchptr{The noose of death and time}%

  \maintext{vigatarāga uvāca |}%

  \maintext{dehī dehe kṣayaṃ yāte bhūjalāgniśivādibhiḥ |}%

  \maintext{yamadūtaiḥ kathaṃ nīto nirālambo nirañjanaḥ }||\thinspace1:11\thinspace||%
\translation{Vigatarāga spoke: When the body disintegrates in the ground, in water, in fire or [is torn apart] by jackals and other [animals], how is the supportless and spotless soul led [to the netherworld] by Yama's messengers? \blankfootnote{1.11 The word \textit{°śivā°} in \textit{pāda} b is slightly suspect, and could be the result
  of metathesis, from \textit{°viṣā°} {\rm (}`by poison'{\rm )}. Nevertheless, 
  jackals seems appropriate in this context, for they 
  are commonly associated with human corpses, death and the cremation ground
  {\rm (}see e.g. \mycite{Ohnuma2019}{\rm )}. Furthermore, \textit{pāda} b lists phenomena
  that cause the body to disintegrate, and not causes of death; thus the reading \textit{śiva}
  is probably correct.
 }}

  \maintext{kālapāśaiḥ kathaṃ baddho nirdehaś ca kathaṃ vrajet |}%

  \maintext{svargaṃ vā sa kathaṃ yāti nirdeho bahudharmakṛt |}%

  \maintext{etan me saṃśayaṃ brūhi jñātum icchāmi tattvataḥ }||\thinspace1:12\thinspace||%
\translation{How is it bound by the nooses of death/time? And if it is bodiless, how can it move? And how does the [soul of a] virtuous [person] {\rm (}\textit{bahudharmakṛt}{\rm )} reach heaven if it has no body? This is my doubt. Teach me. I want to know the truth. \blankfootnote{1.12 The word \textit{kāla} has, as usual, a double meaning here: \textit{kālapāśa}
  is both Yama's noose, and also the limitations and bondage caused by time, 
  as becomes clear at the discussion on the different time units in verses 1.18--31.
 }}

  \maintext{anarthayajña uvāca |}%

  \maintext{atisaṃśayakaṣṭaṃ te pṛṣṭo 'haṃ dvijasattama |}%

  \maintext{durvijñeyaṃ manuṣyais tu devadānavapannagaiḥ }||\thinspace1:13\thinspace||%
\translation{Anarthayajña spoke: You are asking me about an extremely doubtful and problematic matter, O truest of the twice-born. [This is something that] is difficult to understand by humans, and [even] by gods {\rm (}\textit{deva}{\rm )}, demons {\rm (}\textit{dānava}{\rm )} and serpents {\rm (}\textit{pannaga}{\rm )}. \blankfootnote{1.13 Note \textit{te} used for \textit{tvayā} in \textit{pāda} a. Alternatively, taking \textit{te} as genitive, the line
  could be translatied as: `I am being asked about a great 
  problem of yours that originates in doubts\dots'
 }}

  \maintext{karmahetuḥ śarīrasya utpattir nidhanaṃ ca yat |}%

  \maintext{sukṛtaṃ duṣkṛtaṃ caiva pāśadvayam udāhṛtam }||\thinspace1:14\thinspace||%
\translation{The cause of both the birth and death of the body is karma. Good and bad deeds are called the two nooses. \blankfootnote{1.14 The MSS give \textit{karmahetu} in \textit{pāda} a overwhelmingly, which could work as a neuter
  \textit{bahuvrīhi} compound picking up both \textit{utpattir} and \textit{nidhanaṃ} but \textit{karmahetuḥ} is
  grammatically more correct, picking up the feminine \textit{utpatti}.
  I suspect that there may have been a confusion,
  scribes taking \textit{karmahetuśarīrasya} as one single compound; but this would make
  it difficult to interpret the verse.
 }}

  \maintext{tenaiva saha saṃyāti narakaṃ svargam eva vā |}%

  \maintext{sukhaduḥkhaṃ śarīreṇa bhoktavyaṃ karmasambhavam }||\thinspace1:15\thinspace||%
\translation{[The soul] goes to hell or heaven accordingly. Happiness and suffering, both arising from karma, are to be experienced by the body. }

  \maintext{hetunānena viprendra dehaḥ sambhavate nṛṇām |}%

  \maintext{yaṃ kālapāśam ity āhuḥ śṛṇu vakṣyāmi suvrata }||\thinspace1:16\thinspace||%
\translation{It is for this reason, O great Brahmin, that the human body is born. Now learn about that which they call the noose of time, I shall teach you, O you of great observances. }

  \maintext{na tvayā viditaṃ kiñcij jijñāsyasi kathaṃ dvija |}%

  \maintext{kālapāśaṃ ca viprendra sakalaṃ vettum arhasi }||\thinspace1:17\thinspace||%
\translation{[If] you don't know anything, how could you start your investigation, O twice-born? O great Brahmin, you should know the noose of time in its entirety. \blankfootnote{1.17 The variant \textit{jijñāsyasi} seems to be the lectio difficilior as opposed to
  \textit{vijñāsyasi}, but the latter could also work fine here.
 Note how \msM\ {\rm (}agreeing with \Ed{\rm )} gives a reading {\rm (}\textit{vaktum arhasi}{\rm )} that is clearly out
  of context. This confirms that while \msM\ comes up with interesting readings, 
  they are mostly to be ignored.
 }}

  \maintext{kalākalitakālaṃ ca kālatattvakalāṃ śṛṇu |}%

  \maintext{truṭidvayaṃ nimeṣas tu nimeṣadviguṇā kalā }||\thinspace1:18\thinspace||%
\translation{Learn about time {\rm (}\textit{kāla}{\rm )} which is divided into digits {\rm (}\textit{kalā}{\rm )}, [i.e. about] the division[s] {\rm (}\textit{kalā}{\rm )} of the entity [called] time {\rm (}\textit{kālatattva}{\rm )}. Two atomic units of time {\rm (}\textit{truṭi}{\rm )} is one twinkling {\rm (}\textit{nimeṣa}{\rm )}. One digit {\rm (}\textit{kalā}, cca. 1.6 second{\rm )} is twice a twinkling. \blankfootnote{1.18 1.18d and 1.19a are problematic in the light of 1.19b, which 
  redefines \textit{kalā} in harmony with the traditional
  interpretaion, see e.g. \textit{Arthaśāstra} 2.20.33: \textit{trimśatkāṣṭhāḥ kalāḥ}.
  \nocite{Arthasastra1969}
  On divisions of time, see also, e.g., \Manu\ 1.64ff.
  I have calculated 1.6 second for one \textit{kalā} backwards, starting from one day {\rm (}see 1.20ab{\rm )}.
 }}

  \maintext{kalādviguṇitā kāṣṭhā kāṣṭhā vai triṃśatiḥ kalā |}%

  \maintext{triṃśatkalā muhūrtaś ca mānuṣena dvijottama }||\thinspace1:19\thinspace||%
\translation{Two digits {\rm (}\textit{kalā}{\rm )} form one bit {\rm (}\textit{kāṣṭhā}, 3.2 seconds{\rm )}. Thirty bits {\rm (}\textit{kāṣṭhā}{\rm )} is one digit {\rm (}\textit{kalā}?, 1.6 minutes{\rm )}. Thirty digits {\rm (}\textit{kalā}{\rm )} make up one section {\rm (}\textit{muhūrta}, 48 minutes{\rm )} in human terms, O great Brahmin. }

  \maintext{muhūrtatriṃśakenaiva ahorātraṃ vidur budhāḥ |}%

  \maintext{ahorātraṃ punas triṃśan māsam āhur manīṣiṇaḥ }||\thinspace1:20\thinspace||%
\translation{Thirty sections {\rm (}\textit{muhūrta}{\rm )} are known to the wise as night and day [i.e. a full day]. Thirty days and nights are taught by the wise to be one month. }

  \maintext{samā dvādaśa māsāś ca kālatattvavido janāḥ |}%

  \maintext{śataṃ varṣasahasrāṇi trīṇi mānuṣasaṃkhyayā }||\thinspace1:21\thinspace||%
\translation{One year is twelve months [according to] people who know the entity of time. The time span of three hundred \blankfootnote{1.21 Note how a verb {\rm (}e.g. \textit{iti vadanti, iti prāhur}{\rm )} is missing in the first half-verse.
 }}

  \maintext{ṣaṣṭiṃ caiva sahasrāṇi kālaḥ kaliyugaḥ smṛtaḥ |}%

  \maintext{dviguṇaḥ kalisaṃkhyāto dvāparo yuga saṃjñitaḥ }||\thinspace1:22\thinspace||%
\translation{and sixty thousand years by human terms is said to be the Kali age {\rm (}\textit{yuga}{\rm )}. The Dvāpara age is known to be twice as long as the Kali age. \blankfootnote{1.22 Note the stem form noun \textit{yuga} metri causa, and also \msM's unique but confused readings.
 }}

  \maintext{tretā tu triguṇā jñeyā catuḥ kṛtayugaḥ smṛtaḥ |}%

  \maintext{eṣā caturyugā saṃkhyā kṛtvā vai hy ekasaptatiḥ }||\thinspace1:23\thinspace||%
\translation{The Tretā age is thrice [as long], the Kṛta age four [times as long as the Kali age]. This is the figure related to the four ages {\rm (}\textit{yuga}{\rm )}. Taking it seventy-one [times], \blankfootnote{1.23 The `figure' mentioned in this verse is the sum of the duration of the four \textit{yuga}s, 
  which makes up one \textit{mahāyuga}:
  Kaliyuga = 360,000 years,
  Dvāparayuga = 720,000 years,
  Tretāyuga = 1,080,000 years,
  Dvāparayuga = 1,440,000 years; altogether 3,600,000 years. 72 \textit{mahāyuga}s make up
  a \textit{manvantara} {\rm (}= 259,200,000 years{\rm )}. One \textit{kalpa} is 14 \textit{manvantara}s {\rm (}= 3,628,800,000 years{\rm )}. 
  Ten thousand \textit{kalpa}s are one day of Brahmā, and his night is of the same length, which
  makes one full day of Brahmā 72,576,000,000,000 years. See next verses.
 }}

  \maintext{manvantarasya caikasya jñānam uktaṃ samāsataḥ |}%

  \maintext{kalpo manvantarāṇāṃ tu caturdaśa tu saṃkhyayā }||\thinspace1:24\thinspace||%
\translation{the knowledge about one time-span of a Manu {\rm (}\textit{manvantara}{\rm )} has been taught briefly. One aeon {\rm (}\textit{kalpa}{\rm )} is fourteen \textit{manvantara}s in total. \blankfootnote{1.24 See 21.34ff.
 }}

  \maintext{daśa kalpasahasrāṇi brahmāhaḥ parikalpitam |}%

  \maintext{rātrir etāvatī proktā munibhis tattvadarśibhiḥ }||\thinspace1:25\thinspace||%
\translation{Brahmā's day {\rm (}\textit{brahmāhar}{\rm )} is made up of ten thousand Kalpas. [Brahmā's] night is of the same [duration] according to the wise who know the truth. \blankfootnote{1.25 \msM\ has a separator sign {\rm (}|o|{\rm )} at the end of \textit{pāda} b, as if a section ended here.
 }}

  \maintext{rātryāgame pralīyante jagat sarvaṃ carācaram |}%

  \maintext{ahāgame tathaiveha utpadyante carācaram }||\thinspace1:26\thinspace||%
\translation{When [Brahmā's] night falls, the whole moving and unmoving universe dissolves. And when [his] daylight comes, the moving and unmoving [universe] is born. \blankfootnote{1.26 The plural form \textit{pralīyante} in \textit{pāda} a is metri causa for \textit{pralīyate},
  perhaps also influencing \textit{utpadyante} {\rm (}for \textit{utpadyate}{\rm )} in \textit{pāda} d,
  which in turn is used here to avoid an iambic pattern
  {\rm (}- - \shortsyllable\ - \shortsyllable\ - \shortsyllable\ -{\rm )}.
 }}

  \maintext{parārdhaparakalpāni atītāni dvijottama |}%

  \maintext{anāgataṃ tathaivāhur bhṛgurādimaharṣayaḥ }||\thinspace1:27\thinspace||%
\translation{One \textit{para} times \textit{parārdha} [number of, i.e. two hundred quadrillion times a hundred quadrillion] \textit{kalpas} have passed [so far], O great Brahmin. Bhṛgu and the other sages say that the future is the same [time span]. \blankfootnote{1.27 On the definition of the numbers \textit{para} and \textit{parārdha}, see verses 1.32--36.
 Note the peculiar compound \textit{bhṛgu-r-ādi-maharṣayaḥ}.
 }}

  \maintext{yathārkagrahatārendu bhramato dṛśyate tv iha |}%

  \maintext{kālacakraṃ bhramatvaiva viśramaṃ na ca vidmahe }||\thinspace1:28\thinspace||%
\translation{Just as the sun, the planets, the stars and the moon are percieved in this world as wandering around, the wheel of time {\rm (}\textit{kālacakra}{\rm )} keeps spinning and we never experience its halting. \blankfootnote{1.28 \textit{bhramato} {\rm (}gen.{\rm )} in \textit{pāda} b seems to stand for the neuter participle \textit{bhramat}.
  Alternatively, \textit{bhramato} might mean `erroneously' {\rm (}\textit{brama-tas}, abl.{\rm )}, but this
  makes the verse difficult to interpret.
 }}

  \maintext{kālaḥ sṛjati bhūtāni kālaḥ saṃharate punaḥ |}%

  \maintext{kālasya vaśagāḥ sarve na kālavaśakṛt kvacit }||\thinspace1:29\thinspace||%
\translation{Time creates living beings and time destroys them again. Everything is under the control of time. There is nothing that can bring time under control. }

  \maintext{caturdaśaparārdhāni devarājā dvijottama |}%

  \maintext{kālena samatītāni kālo hi duratikramaḥ }||\thinspace1:30\thinspace||%
\translation{Fourteen \textit{parārdha} [fourteen hundred quadrillion] god kings, O Brahmin, have passed by over time, for time is difficult to overcome. \blankfootnote{1.30 Note that \textit{samatītāni} {\rm (}neuter{\rm )} most probably picks up \textit{devarājāḥ}
  {\rm (}masculine{\rm )} in this verse, or rather \textit{devarājā} stands for
  \textit{devarājānāṃ} and \textit{samatītāni} picks up \textit{°parārdhāni}. It is not clear to me
  what \textit{devarāja} {\rm (}`god king'{\rm )} means exactly {\rm (}Indra?{\rm )}.
 }}

  \maintext{eṣa kālo mahāyogī brahmā viṣṇuḥ paraḥ śivaḥ |}%

  \maintext{anādinidhano dhātā sa mahātmā namaskuru }||\thinspace1:31\thinspace||%
\translation{Time is [manifest] as a great yogin, as Brahmā, Viṣṇu and supreme Śiva, is beginningless and endless, is the creator, the great soul. Pay homage [to Time]. }

  \subchptr{parārdhādi}%

  \trsubchptr{The \textit{parārdha} etc.: numbers}%

  \maintext{vigatarāga uvāca |}%

  \maintext{śrutaṃ vai kālacakraṃ tu mukhapadmaviniḥsṛtam |}%

  \maintext{parārdhaṃ ca paraṃ caiva śrotuṃ vaḥ pratidīpitam }||\thinspace1:32\thinspace||%
\translation{Vigatarāga spoke: I have just heard [the term] `wheel of time' {\rm (}\textit{kālacakra}{\rm )} uttered from [your] lotus mouth, as well as \textit{parārdha} and \textit{para}. You have made these things appear as exciting, as things to hear. \blankfootnote{1.32 The reading of all manuscripts consulted, \textit{vinisṛtam}, 
  may be considered metrical if we interpret it, loosely, as \textit{vinisritam}.
   %
  \textit{Pāda} d is suspect and my translation is tentative.
  \msM's reading in \textit{pāda} d {\rm (}\textit{srotuṃ naḥ pratidīyatāṃ}{\rm )} might make sense 
  {\rm (}`give it back/repeat it for us again'{\rm )}, but it sounds forced,
  as if the scribe tried to come up with a reading that he understood
  better than \textit{srotuṃ vaḥ pratidīpitam}, which is in fact not easy to interpret.
  One would expect a phrase meaning `please tell me about these.'
 }}

  \maintext{anarthayajña uvāca |}%

  \maintext{ekaṃ daśaṃ śataṃ caiva sahasram ayutaṃ tathā |}%

  \maintext{prayutaṃ niyutaṃ koṭim arbudaṃ vṛndam eva ca }||\thinspace1:33\thinspace||%
\translation{Anarthayajña spoke: One, ten, a hundred, a thousand, and ten thousand {\rm (}\textit{ayuta}{\rm )}, a hundred thousand {\rm (}\textit{prayuta}{\rm )}, a million {\rm (}\textit{niyuta}{\rm )}, ten million {\rm (}\textit{koṭi}{\rm )}, a hundred million {\rm (}\textit{arbuda}{\rm )}, and one billion {\rm (}\textit{vṛnda}, 10$^{9}${\rm )}, \blankfootnote{1.33 See a similar teaching of numbers in \BRAHMANDAPUR\ 3.2.91ff.
 }}

  \maintext{kharvaṃ caiva nikharvaṃ ca śaṅkuḥ padmaṃ tathaiva ca |}%

  \maintext{samudro madhyam antaṃ ca parārdhaṃ ca paraṃ tathā }||\thinspace1:34\thinspace||%
\translation{ten billion {\rm (}\textit{kharva}{\rm )}, a hundred billion {\rm (}\textit{nikharva}{\rm )}, one trillion {\rm (}\textit{śaṅku}, 10$^{12}${\rm )}, and ten trillion {\rm (}\textit{padma}{\rm )}, a hundred trillion {\rm (}\textit{samudra}{\rm )}, one quadrillion {\rm (}\textit{madhya}, 10$^{15}${\rm )}, ten quadrillion {\rm (}\textit{[an]anta}{\rm )}, a hundred quadrillion {\rm (}\textit{parārdha}{\rm )}, and two hundred quadrillion {\rm (}\textit{para}{\rm )}. \blankfootnote{1.34 For \textit{anta} meaning \textit{ananta}, see 1.58cd--59ab. \msM's reading in \textit{pāda} d
  may be a result of an eyeskip to 1.35c.
 }}

  \maintext{sarve daśaguṇā jñeyāḥ parārdhaṃ yāvad eva hi |}%

  \maintext{parārdhadviguṇenaiva parasaṃkhyā vidhīyate }||\thinspace1:35\thinspace||%
\translation{Each should be known as powers of ten up to \textit{parārdha}. The number corresponding to \textit{para} is double that of \textit{parārdha}. }

  \maintext{parāt parataraṃ nāsti iti me niścitā matiḥ |}%

  \maintext{purāṇavedapaṭhitā mayākhyātā dvijottama }||\thinspace1:36\thinspace||%
\translation{There is no higher number than \textit{para}. This is my firm conviction, which is based on my readings of the Purāṇas and the Vedas and [which I have now] taught [to you], O great Brahmin. \blankfootnote{1.36 Note that \Ed, after omitting three lines, inserts this: \textit{vṛndañ caiva mahāvṛnda dviparānantam eva ca}.
 }}

  \subchptr{brahmāṇḍam}%

  \trsubchptr{Brahmā's Egg}%

  \maintext{vigatarāga uvāca |}%

  \maintext{brahmāṇḍaṃ kati vijñeyaṃ pramāṇaṃ prāpitaṃ kvacit |}%

  \maintext{kati cāṅguli{-}m{-}ūrdhveṣu sūryas tapati vai mahīm }||\thinspace1:37\thinspace||%
\translation{Vigatarāga spoke: How many eggs of Brahmā are there? And are its measurements available anywhere? From how many finger's breadths high does the sun heat the earth? \blankfootnote{1.37 The use of the singular next to numerals is one of the hallmarks of the \VSS\ 
  {\rm (}see p. \verify{\rm )}. As an introduction to this phenomenon, \textit{pāda} a has
  \textit{brahmāṇḍaṃ} in the singular where we would expect a plural form.
  The word \textit{prāpitaṃ} is a conjecture for \textit{cāpitaṃ}, which I find unintelligible. 
  Another possibility could be \textit{jñāpitaṃ}.
 My emendation of \textit{cāṅguli-mūrdheṣu} to \textit{cāṅguli{-}m{-}ūrdhveṣu} {\rm (}with a hiatus filler{\rm )} 
  is based on \textit{ūrdhvatas} in 1.61d, which is part of the reply to the question posed in this line.
  In turn, \textit{aṅguli} here triggered an conjecture in 1.61c.
 }}

  \maintext{anarthayajña uvāca |}%

  \maintext{brahmāṇḍānāṃ prasaṃkhyātuṃ mayā śakyaṃ kathaṃ dvija |}%

  \maintext{devās te 'pi na jānanti mānuṣāṇāṃ ca kā kathā }||\thinspace1:38\thinspace||%
\translation{Anarthayajña spoke: How could I enumerate [all] the eggs of Brahmā, O twice-born? Even the gods don't know [all the details], not to mention humans. \blankfootnote{1.38 One would expect \textit{brahmāṇḍāni} in \textit{pāda} a instead of \textit{brahmāṇḍānāṃ},
  but we should probably understand \textit{brahmāṇḍānāṃ viśeṣān prasaṃkhyātuṃ...}
  The structure noun in genitive + verb meaning 'telling' occurs also in 4.69a and \verify .
 }}

  \maintext{paryāyeṇa tu vakṣyāmi yathāśakyaṃ dvijottama |}%

  \maintext{brahmaṇā yat purākhyāto mātariśvā yathā tathā }||\thinspace1:39\thinspace||%
\translation{I shall teach [you], as far as I can, in due order and truthfully, that, O great Brahmin, which Mātariśvan was taught by Brahmā in the past. \blankfootnote{1.39 The claim that Brahmā taught Mātariśvan is confirmed in 1.64cd, and
  also, e.g., in \BrahmandaPur\ 3.4.58cd {\rm (}see the apparatus{\rm )}.
 }}

  \maintext{śivāṇḍābhyantareṇaiva sarveṣām iva bhūbhṛtām |}%

  \maintext{daśa nāma diśāṣṭānāṃ brahmāṇḍe kīrtitaṃ śṛṇu }||\thinspace1:40\thinspace||%
\translation{Ten names of all the [cosmic] rulers of each of the eight directions in Brahmā's Egg, [which is] inside Śiva's Egg, are being taught now, listen. \blankfootnote{1.40 My conjecture in \textit{pāda} b {\rm (}\textit{bhūbhṛtām}{\rm )} is based on the fact that the 
  readings transmitted in the MSS seem unintelligible and, more importantly, that
  these names are said to belong to \textit{nāyaka}s in the subsequent verses,
  a possible synonym of \textit{bhūbhṛt} {\rm (}`a king'{\rm )}, and also that it is a minute intervention.
  In \textit{pāda} c, understand \textit{diśāṣṭānāṃ} as \textit{diśām aṣṭānāṃ} or \textit{digaṣṭakānāṃ}, 
  and note that one of the hallmarks of the language of the \VSS\ is the use
  of the singular in the proximity of numbers, where a plural would be expected {\rm (}\textit{daśa nāma}{\rm )}.
 }}

  \subchptr{bhūbhṛtāṃ nāmāni}%

  \trsubchptr{The names of the cosmic rulers}%

  \subsubchptr{pūrvataḥ}%

  \trsubsubchptr{East}%

  \maintext{sahāsahaḥ sahaḥ sahyo visahaḥ saṃhato 'sabhā |}%

  \maintext{prasaho 'prasahaḥ sānuḥ pūrvato daśa nāyakāḥ }||\thinspace1:41\thinspace||%
\translation{[1] Sahā, [2] Asaha, [3] Saha, [4] Sahya, [5] Visaha, [6] Saṃhata, [7] Asabhā, [8] Prasaha, [9] Aprasaha, [10] Sānu: [these are] the ten Leaders in the East. \blankfootnote{1.41 Note that many of the names here and in the following verses are,
  in the absence of any parallel passage, rather insecure.
  In order to avoid the repetition of the name Saha, I take the first name here
  as feminine; Asabhā seems also to be a feminine ruler's name. Later on there
  seem to come more feminine names {\rm (}Tejā, Yamunā, Naganā, etc.{\rm )}, therefore it 
  may be correct to interpret some of the names as names of queens.
  What is clear here is that the list evokes the name Sahasrākṣa,
  one of the appellations of Indra, the guadrian of the eastern direction.
 }}

  \subsubchptr{āgneye}%

  \trsubsubchptr{South-East}%

  \maintext{prabhāso bhāsano bhānuḥ pradyoto dyutimo dyutiḥ |}%

  \maintext{dīptatejāś ca tejāś ca tejā tejavaho daśa |}%

  \maintext{āgneye tv etad ākhyātaṃ yāmye śṛṇv atha bho dvija }||\thinspace1:42\thinspace||%
\translation{[1] Prabhāsa, [2] Bhāsana, [3] Bhānu, [4] Pradyota, [5] Dyutima, [6] Dyuti, [7] Dīptatejas, [8] Tejas, [9] Tejā, [10] Tejavaho: [these are] the ten [rulers] in the direction of Agni [SE]. Now listen to [the names for] the Yama's region, O twice-born. \blankfootnote{1.42 Here, in the region of Agni, the names evidently evoke the image of flames.
 }}

  \subsubchptr{yāmye}%

  \trsubsubchptr{South}%

  \maintext{yamo 'tha yamunā yāmaḥ saṃyamo yamuno 'yamaḥ |}%

  \maintext{saṃyano yamanoyāno yaniyugmā yanoyanaḥ }||\thinspace1:43\thinspace||%
\translation{[1] Yama, [2] Yamunā, [3] Yāma, [4] Saṃyama, [5] Yamuna, [6] Ayama, [7] Saṃyana, [8] Yamanoyāna, [9] Yaniyugmā, [10] Yanoyana. \blankfootnote{1.43 I have choosen the variant \textit{saṃyano} in \textit{pāda} c only to avoid the repetition of
  the name \textit{saṃyama}, and the variant \textit{yanoyanaḥ} in \textit{pāda} d because I suspect that
  most of the names here should begin with \textit{ya}. All the name forms
  in this verse are to be taken as tentative. The only 
  guiding light is the presence of \textit{ya}, reinforcing a connection with Yama.
 }}

  \subsubchptr{nairṛte}%

  \trsubsubchptr{South-West}%

  \maintext{nagajo naganā nando nagaro naga nandanaḥ |}%

  \maintext{nagarbho gahano guhyo gūḍhajo daśa tatparaḥ }||\thinspace1:44\thinspace||%
\translation{[1] Nagaja, [2] Naganā, [3] Nanda, [4] Nagara, [5] Naga, [6] Nandana, [7] Nagarbha, [8] Gahana, [9] Guhyo, [10] Gūḍhaja: [these are] the ten associated with [the South-West]. \blankfootnote{1.44 \textit{naga} in \textit{pāda} b is a stem form noun metri causa
 \textit{tatparaḥ} in \textit{pāda} d might be another example of a singular form next to a number {\rm (}see 1.40c above{\rm )}.
  Note that the reconstruction of these names are tentative. What is clear here is that the
  initials should be \textit{na} and \textit{ga}, probably suggesting a connection with \textit{nirṛti}, \textit{naraka}s and \textit{nāga}s.
 }}

  \subsubchptr{vāruṇe}%

  \trsubsubchptr{West}%

  \maintext{vāruṇena pravakṣyāmi śṛṇu vipra nibodha me |}%

  \maintext{babhraḥ setur bhavodbhadraḥ prabhavodbhavabhājanaḥ |}%

  \maintext{bharaṇo bhuvano bhartā daśaite varuṇālayāḥ }||\thinspace1:45\thinspace||%
\translation{I shall teach you the [names] in Varuṇa's region [in the west]. Listen, O Brahmin, learn from me. [1] Babhra, [2] Setu, [3] Bhava, [4] Udbhadra, [5] Prabhava, [6] Udbhava, [7] Bhājana, [8] Bharaṇa, [9] Bhuvana, and [10] Bhartṛ: these ten dwell in Varuṇa's region [in the west]. \blankfootnote{1.45 Varuṇa upholds the sky and the earth. This could be the reason why 
  these names include \textit{bharaṇa} and \textit{bhartṛ}.
 }}

  \subsubchptr{vāyavye}%

  \trsubsubchptr{North-West}%

  \maintext{nṛgarbho 'suragarbhaś ca devagarbho mahīdharaḥ |}%

  \maintext{vṛṣabho vṛṣagarbhaś ca vṛṣāṅko vṛṣabhadhvajaḥ }||\thinspace1:46\thinspace||%
\translation{[1] Nṛgarbha, [2] Asuragarbha, [3] Devagarbha, [4] Mahīdhara, [5] Vṛṣabha, [6] Vṛṣagarbha, [7] Vṛṣāṅka, [8] Vṛṣabhadhvaja, \blankfootnote{1.46 The connection between \textit{vṛṣa} and the north-west or Vāyu is not evident to me. \verify
  In a tantric context, a western position is more standard for \textit{vṛṣa}, see e.g.
  \mycitep{Pancavaranastava}{40}.
 }}

  \maintext{jñātavyaś ca tathā samyag vṛṣajo vṛṣanandanaḥ |}%

  \maintext{nāyakā daśa vāyavye kīrtitā ye mayā dvija }||\thinspace1:47\thinspace||%
\translation{and [9] Vṛṣaja and [10] Vṛṣanandana: these are to be known properly as the ten leaders in Vāyu's region [in the north-west], as I taught them, O twice-born. \blankfootnote{1.47 Note how \msM\ deviates here again in a significant way.
 }}

  \subsubchptr{uttare}%

  \trsubsubchptr{North}%

  \maintext{sulabhaḥ sumanaḥ saumyaḥ suprajaḥ sutanuḥ śivaḥ |}%

  \maintext{sataḥ satya layaḥ śambhur daśa nāyakam uttare }||\thinspace1:48\thinspace||%
\translation{[1] Sulabha, [2] Sumana, [3] Saumya, [4] Supraja, [5] Sutanu, [6] Śiva, [7] Sata, [8] Satya, [9] Laya, [10] Śambhu: [these are] the ten leaders in the north. \blankfootnote{1.48 I prefer the form \textit{sumanaḥ} to the more standard \textit{sumanāḥ} {\rm (}\msNc{\rm )} in \textit{pāda} a, 
  because it suits the slightly irregluar language of the \VSS\ {\rm (}see pp. \verify{\rm )},
  and because the solitary reading of \msNc\ may well only be an attempt to
  standardise. It is also not inconceivable that \textit{sumanaḥ} stands compounded 
  with \textit{saumyaḥ}.
 Note how \textit{daśa nāyakam} could again be an example for the use of the singular 
  next to a number in \textit{pāda d}. It seems that here the northern region
  is associated with Śiva, rather than the north-east, the \textit{īśāna} direction, which
  is occupied by Brahmā: see next verse. 
  In a tantric context, Brahmā is sometimes associated with the north-east, see, e.g.,
  \mycitep{Pancavaranastava}{39}.\verify
  I have left \textit{satya} in stem form.
 }}

  \subsubchptr{īśāne}%

  \trsubsubchptr{North-East}%

  \maintext{indu bindu bhuvo vajra varado vara varṣaṇaḥ |}%

  \maintext{ilano valino brahmā daśeśāneṣu nāyakāḥ }||\thinspace1:49\thinspace||%
\translation{[1] Indu, [2] Bindu, [3] Bhuva, [4] Vajra, [5] Varada, [6] Vara, [7] Varṣaṇa, [8] Ilana, [9] Valina, [10] Brahmā: [these are] the ten rulers in the Īśāna direction [i.e. in the north-east]. \blankfootnote{1.49 I consider \textit{indu, bindu} and \textit{vajra} stem form nouns.
 The north-east seems to be occupied by Brahmā, and by rulers whose names should
  somehow evoke Brahmā's name.
 }}

  \subsubchptr{madhyame}%

  \trsubsubchptr{Center}%

  \maintext{aparo vimalo moho nirmalo mana mohanaḥ |}%

  \maintext{akṣayaś cāvyayo viṣṇur varado madhyame daśa }||\thinspace1:50\thinspace||%
\translation{[1] Apara, [2] Vimala, [3] Moha, [4] Nirmala, [5] Mana, [6] Mohana, [7] Akṣaya, [8] Avyaya, [9] Viṣṇu, [10] Varada: [these are] the ten [leaders] in the centre. \blankfootnote{1.50 Note that the last three lists above have been associated 
  with Śiva, Brahmā and Viṣṇu, respectively, and here, in a layer
  of the text that can be labelled Vaṣṇava {\rm (}see pp. \verify{\rm )}, it is Viṣṇu that
  seems to occupy a central position. \textit{mana mohanaḥ} in \textit{pāda} b may
  sound like one single name, but we are forced to separate these two words
  {\rm (}\textit{mana} being in stem form metri causa{\rm )} to arrive at a list of ten names.
 }}

  \subsubchptr{parivārāḥ}%

  \trsubsubchptr{Subordinates}%

  \maintext{sarveṣāṃ daśa{-}m{-}īśānāṃ parivāraśataṃ śatam |}%

  \maintext{śatānāṃ pṛthag ekaikaṃ sahasraiḥ parivāritam }||\thinspace1:51\thinspace||%
\translation{Each of the ten rulers has a retinue of a hundred subordinates. Each one of [these] hundred is surrounded by a thousand subordinates. \blankfootnote{1.51 I take \textit{daśa-m-īśānāṃ} as a disjointed \verify\ compound {\rm (}\textit{daśeśānāṃ}{\rm )}.
  It is conceivable that each of the above ninety rulers has ten subordinates, 
  therefore each group of ten rulers has a hundred subordinates altogether,
  but the original idea may have been that each one of the above ninety 
  rulers has a hundred subordinates. Alternatively, this verse may only refer to 
  the central group of ten rulers mentioned in 1.50, and each one of them has
  a hundred subordinates.
 }}

  \maintext{sahasreṣu ca ekaikam ayutaiḥ parivāritam |}%

  \maintext{ayutaṃ prayutair vṛndaiḥ prayutaṃ niyutair vṛtam }||\thinspace1:52\thinspace||%
\translation{Each one of the thousand is surrounded by ten thousand [subordinates], the ten thousand is surrounded by a multitude of a hundred thousand, the hundred thousand by a million, \blankfootnote{1.52 We are forced to follow \Ed's reading in \textit{pāda} c in order to make sense of this passage.
  My correction in \textit{pāda} d is motivated by the same. Note that \textit{vṛnda} is not a number in this line. 
  Elsewhere in this chapter \textit{vṛnda} is the word that signifies `a billion'.
 }}

  \maintext{ekaikasya parīvāro niyutaḥ pṛthag eva ca |}%

  \maintext{koṭibhir daśakoṭyena ekaikaḥ parivāritaḥ }||\thinspace1:53\thinspace||%
\translation{[that is] each one has a retinue of a million [subordinates] {\rm (}\textit{niyuta}{\rm )}. [Then those] are surrounded by ten million {\rm (}\textit{koṭi}{\rm )} [subordinates], [they in turn] by a hundred million {\rm (}\textit{daśakoṭi}{\rm )}. \blankfootnote{1.53 It seems that \textit{pāda}s ab repeat what has been stated in 1.52cd.
 \textit{°koṭyena} stands for \textit{°koṭyā} {\rm (}thematisation{\rm )}.
  Note how the scribe of \msM\ gets confused at 1.53c due to an eye-skip and 
  fully regains control only at 1.55b.
 }}

  \maintext{daśakoṭiṣu ekaikaṃ vṛndavṛndabhṛtair vṛtam |}%

  \maintext{vṛndavargeṣu ekaikaṃ kharvabhiḥ parivāritam }||\thinspace1:54\thinspace||%
\translation{Each one of the hundred million is surrounded by a billion {\rm (}\textit{vṛnda}{\rm )} subordinates {\rm (}\textit{bhṛta}{\rm )}. Each one in these groupsof a billion {\rm (}\textit{vṛnda}{\rm )} is surrounded by ten billion {\rm (}\textit{kharva}{\rm )} [subordinates]. }

  \maintext{kharvavargeṣu ekaikaṃ daśakharvagaṇair vṛtam |}%

  \maintext{daśakharveṣu ekaikaṃ śaṅkubhiḥ parivāritam }||\thinspace1:55\thinspace||%
\translation{Each in these gourps of ten billion {\rm (}\textit{kharva}{\rm )} is surrounded by a hundred billion {\rm (}\textit{daśakharva}{\rm )}. Each of those hundred billion {\rm (}\textit{daśakharva}{\rm )} is surrounded by a trillion {\rm (}\textit{śaṅku}{\rm )} [deities]. }

  \maintext{śaṅkubhiḥ pṛthag ekaikaṃ padmena parivāritam |}%

  \maintext{padmavargeṣu ekaikaṃ samudraiḥ parivāritam }||\thinspace1:56\thinspace||%
\translation{Each of those one trillion {\rm (}\textit{śaṅku}{\rm )} is surrounded by ten trillion {\rm (}\textit{padma}{\rm )}. Each of those ten trillion {\rm (}\textit{padma}{\rm )} is surrounded by a hundred trillion {\rm (}\textit{samudra}{\rm )}. \blankfootnote{1.56 Note that \textit{śaṅkubhiḥ} stands for \textit{śaṅkūṣu} {\rm (}instrumental for locative{\rm )}.
 }}

  \maintext{samudreṣu tathaikaikaṃ madhyasaṃkhyais tu tair vṛtam |}%

  \maintext{madhyasaṃkhyeṣu ekaikam anantaiḥ parivāritam }||\thinspace1:57\thinspace||%
\translation{And each of those hundred trillion {\rm (}\textit{samudra}{\rm )} is surrounded by those whose number is one quadrillion {\rm (}\textit{madhya}{\rm )}. Each of those quadrillion {\rm (}\textit{madhya}{\rm )} is surrounded by ten quadrillion {\rm (}\textit{ananta}{\rm )}. }

  \maintext{ananteṣu ca ekaikaṃ parārdhaparivāritam |}%

  \maintext{parārdheṣu ca ekaikaṃ pareṇa parivāritam |}%

  \maintext{eṣa vai kathito vipra śakyaṃ sāṃkhyam udīritam }||\thinspace1:58\thinspace||%
\translation{Each of those ten quadrillion {\rm (}\textit{ananta}{\rm )} is surrounded by a hundred quadrillion {\rm (}\textit{parārdha}{\rm )}. Each of those hundred quadrillion {\rm (}\textit{parārdha}{\rm )} is surrounded by two hundred quadrillion {\rm (}\textit{para}{\rm )}. This is how it is taught, O Brahmin. The enumeration [of the rulers of the Brahmāṇḍa] has been taught as much as it is possible. }

  \subchptr{pramāṇam}%

  \trsubchptr{Measurements}%

  \maintext{pramāṇaṃ śṛṇu me vipra saṃkṣepād bruvato mama |}%

  \maintext{candrodaye pūrṇamāsyāṃ vapur aṇḍasya tādṛśam }||\thinspace1:59\thinspace||%
\translation{Listen to me and learn about the measurements [of the universe], O Brahmin, I shall teach [you] in a concise manner. The body of the Egg is like that of [the moon] at moonrise on the day of the full moon. }

  \maintext{koṭikoṭisahasraṃ tu yojanānāṃ samantataḥ |}%

  \maintext{aṇḍānāṃ ca parīmāṇaṃ brahmaṇā parikīrtitam }||\thinspace1:60\thinspace||%
\translation{The whole circumference of the Eggs has been declared by Brahmā to be ten million {\rm (}\textit{koṭi}{\rm )} times a thousand times ten million \textit{yojana}s. }

  \maintext{saptakoṭisahasrāṇi saptakoṭiśatāni ca |}%

  \maintext{viṃśakoṭiṣv aṅgulīṣu ūrdhvatas tapate raviḥ }||\thinspace1:61\thinspace||%
\translation{The Sun shines from the height of seven thousand seven hundred and twenty \textit{koṭi} finger's breath. \blankfootnote{1.61 This verse is the reply to the question in 1.37cd, which contains the word \textit{aṅguli}:
  this hints at the possibility that the unintelligible \textit{gulmeṣu} transmitted in most of the
  witnesses might be corrupted from \textit{aṅguīṣu}; hence my conjecture, resulting
  in a \textit{ra-vipulā}.
 }}

  \maintext{pramāṇaṃ nāma saṃkhyā ca kīrtitāni samāsataḥ |}%

  \maintext{brahmāṇḍaṃ cāprameyāṇāṃ lakṣaṇaṃ parikīrtitam }||\thinspace1:62\thinspace||%
\translation{The numbers pertaining to the measurements have been taught in brief. The characteristics of the unmeasurable Brahmāṇḍa[s] have been taught. \blankfootnote{1.62 Note the mixture of different grammatical genders and numbers in this verse. 
  Understand \textit{pramāṇeṣu saṃkhyāḥ kīrtitāḥ samāsataḥ} and 
  \textit{brahmāṇḍānām aprameyānāṃ}...
 }}

  \subchptr{vyāsāḥ}%

  \trsubchptr{The redactors {\rm (}of the Purāṇas{\rm )}}%

  \maintext{purāṇāśīsahasrāṇi śatāni dvijasattama |}%

  \maintext{brahmaṇā kathitaṃ pūrṇaṃ mātariśvā yathātatham }||\thinspace1:63\thinspace||%
\translation{O truest of the twice-born, the Purāṇa[s of] 8,000,000 [verses] were taught by [1] Brahmā to [2] Mātariśvan [= Vāyu] in their entirety, in their true form. \blankfootnote{1.63 \textit{Pāda} a should probably be analysed and interpreted as 
  \textit{purāṇam {\rm (}purāṇānām aśītisahasrāṇi śatāni ślokāni{\rm )} brahmaṇā kathitam}.
  Alternatively, \textit{pāda} a may have originally read \textit{purāṇāni sahasrāṇi},
  and then the inital number of verses transmitted by Brahmā is
  a hundred thousand. That the number refers to the number of \textit{śloka}s
  transmitted, and not, e.g., the number of lines, is confirmed in 1.66d:
  \textit{viṃśatślokasahasrikam}.
 
  On the idea that initially there was only one Purāṇa, see, e.g.,
  \mycitep{RocherPuranas1986}{41ff}.
 
   %
  In \textit{pāda} d, either understand \textit{mātariśvā} {\rm (}nom.{\rm )} as \textit{mātariśvānaṃ} {\rm (}acc.{\rm )} or emend
  \textit{kathitaṃ} to \textit{kathitaḥ} in the sense `Mātariśvan was taught,' echoing 1.39cd:
  \textit{brahmaṇā yat purākhyāto mātariśvā yathā tathā}.
 
   %
  Compare this list to a list of twenty-eight \textit{vedavyāsa}s, from
  Brahmā to Vyāsa Dvaipāyana, in \VISNUP\ 3.3.10--19,
  taught by Parāśara, the twenty-sixth \textit{vyāsa}
  of this list and our text {\rm (}in the numbering that I add here I follow
  the translation in \mycitep{Visnupurana_tr}{178--179}{\rm )}:
   %
  \textit{vedavyāsā vyatītā ye aṣṭāviṃśati sattama~|
  caturdhā yaiḥ kṛto vedo dvāpareṣu punaḥ punaḥ~||
  dvāpare prathame vyastāḥ svayaṃ vedāḥ }[1]\textit{ svayaṃbhuvā~| 
  dvitīye dvāpare caiva vedavyāsaḥ }[2]\textit{ prajāpati~|| 
  tṛtīye }[3]\textit{ cośanā vyāsaś caturthe ca }[4]\textit{ bṛhaspatiḥ~|  
  }[5]\textit{ savitā pañcame vyāsaḥ }[6]\textit{ mṛtyuḥ ṣaṣṭhe smṛtaḥ prabhuḥ~||  
  saptame ca }[7]\textit{ tathaivendro }[8]\textit{ vasiṣṭhaś cāṣṭame smṛtaḥ~|  
  }[9]\textit{ sārasvataś ca navame }[10]\textit{ tridhāmā daśame smṛtaḥ~||  
  ekādaśe tu }[11]\textit{ trivṛṣā }[12]\textit{ bhāradvājas tataḥ param~|  
  trayodaśe }[13]\textit{ cāntarikṣo }[14]\textit{ varṇī cāpi caturdaśe~||  
  }[15]\textit{ trayyāruṇaḥ pañcadaśe ṣoḍaśe tu }[16]\textit{ dhanaṃjayaḥ~|  
  }[17]\textit{ kratuṃjayaḥ saptadaśe }[18]\textit{ ṛṇajyo 'ṣṭādaśe smṛtaḥ~||  
  tato vyāso }[19]\textit{ bharadvājo bharadvājāt tu }[20]\textit{ gautamaḥ~|  
  gautamād uttamo vyāso }[21]\textit{ haryātmā yo 'bhidhīyate~||  
  atha haryātmano }[22]\textit{ venaḥ smṛto vājaśravās tu yaḥ~|  
  somaḥ śuṣmāyaṇas tasmāt }[23]\textit{ tṛṇabindur iti smṛtaḥ~||  
  }[24]\textit{ ṛkṣo 'bhūd bhārgavas tasmād vālmīkir yo 'bhidhīyate~|  
  tasmād asmatpitā }[25]\textit{ śaktir vyāsas tasmād }[26]\textit{ ahaṃ mune~||  
  }[27]\textit{ jātukarṇo 'bhavan mattaḥ kṛṣṇadvaipāyanas }[28]\textit{ tataḥ~|  
  aṣṭaviṃśatir ity ete vedavyāsāḥ purātanāḥ~||.  }
 
   %
  Another relevant passage is \BrahmandaPur\ 3.4.58cd--67 {\rm (}\similar\ \VayuP\ 2.41.58--67{\rm )}.
  Note how Tṛṇabindu is, perhaps by mistake, different from Somaśuṣma/Śuṣmāyaṇa here,
  but, more importantly, note Amitabuddhi of \VSS\ 1.76 appear at the end of this list:
   %
  [1] \textit{brahmā dadau śāstram idaṃ purāṇaṃ }[2]\textit{ mātariśvane~||  
  tasmāc }[3]\textit{ cośanasā prāptaṃ tasmāc cāpi }[4]\textit{ bṛhaspatiḥ~|  
  bṛhaspatis tu provāca }[5]\textit{ savitre tadanantaram~||  
  savitā }[6]\textit{ mṛtyave prāha mṛtyuś }[7]\textit{ cendrāya vai punaḥ~|  
  indraś cāpi }[8]\textit{ vasiṣṭāya so 'pi }[9]\textit{ sārasvatāya ca~||  
  sārasvatas }[10]\textit{ tridhāmne 'tha tridhāmā ca }[11]\textit{ śaradvate~|  
  śaradvāṃs tu }[12]\textit{ triviṣṭāya so }[13]\textit{ 'ntarikṣāya dattavān~||  
  }[14]\textit{ carṣiṇe cāntarikṣo vai so 'pi }[15]\textit{ trayyāruṇāya ca~|  
  trayyāruṇād }[16]\textit{ dhanañjayaḥ sa vai prādāt }[17]\textit{ kṛtañjaye~||  
  kṛtañjayāt }[18]\textit{ tṛṇañjayo }[19]\textit{ bharadvājāya so 'py atha~|  
  }[20]\textit{ gautamāya bharadvājaḥ so 'pi }[21]\textit{ niryyantare punaḥ~||  
  niryyantaras tu provāca tathā }[22]\textit{ vājaśravāya vai~|  
  sa dadau }[23]\textit{ somaśuṣmāya sa cādāt }[24]\textit{ tṛṇabindave~||  
  tṛṇabindus tu }[25]\textit{ dakṣāya dakṣaḥ provāca }[26]\textit{ śaktaye~|  
  śakteḥ }[27]\textit{ parāśaraś cāpi garbhasthaḥ śrutavān idam~||  
  parāśarāj }[28]\textit{ jātukarṇyas tasmād }[29]\textit{ dvaipāyanaḥ prabhuḥ~|  
  dvaipāyanāt punaś cāpi }[30]\textit{ mayā prāptaṃ dvijottama~||  
  mayā caitat punaḥ proktaṃ }[31]\textit{ putrāyāmitabuddhaye~|  
  ity eva vākyaṃ brahmādiguruṇāṃ samudāhṛtam~||.}  
 
   %
  The list of \textit{vedavyāsa}s in \LinPu\ 1.7.15--18 includes these twenty-five names:
  Kratu, Satya, Bhārgava, Aṅgiras, Savitṛ,
  Mṛtyu, Śatakratu, Vasiṣṭha, Sārasvata, Tridhāman,
  Trivṛta, Śatatejas, Tarakṣu, Āruṇi, Kṛtaṃjaya,
  Ṛtaṃjayo, Bharadvāja, Gautama, Vācaśravas, Tṛṇabindu,
  Rūkṣa, Śakti, Jātūkarṇya, Kṛṣṇa Dvaipāyano.
 }}

  \maintext{vāyunā pāda saṃkṣipya prāptaṃ cośanasaṃ purā |}%

  \maintext{tenāpi pāda saṃkṣipya prāptavāṃś ca bṛhaspatiḥ }||\thinspace1:64\thinspace||%
\translation{Vāyu abridged the verses and then gave [the Purāṇas] to [3] Uśanas. He [Uśanas] also abridged the verses, and [4] Bṛhaspati received them. \blankfootnote{1.64 Note the stem form noun \textit{pāda} twice in this verse and 
  the slightly odd grammatical structure {\rm (}\textit{purāṇaṃ}{\rm )} \textit{prāptam uśanasam} {\rm (}`the Purāṇa reached Uśanas'{\rm )},
  as opposed to the solution in \textit{pāda} d {\rm (}\textit{prāptavān}{\rm )}.
 }}

  \maintext{bṛhaspatis tu provāca sūryaṃ triṃśatsahasrikam |}%

  \maintext{pañcaviṃśatsahasrāṇi mṛtyuṃ prāha divākaraḥ }||\thinspace1:65\thinspace||%
\translation{Bṛhaspati taught 30,000 [verses] to [5] Sūrya [the Sun]. Divākara [= the Sun] taught 25,000 [verses] to [6] Mṛtyu [Death]. }

  \maintext{ekaviṃśatsahasrāṇi mṛtyunendrāya kīrtitam | }%

  \maintext{indreṇāha vasiṣṭhāya viṃśatślokasahasrikam }||\thinspace1:66\thinspace||%
\translation{Mṛtyu taught 21,000 [verses] to [7] Indra. Indra taught 20,000 verses to [8] Vasiṣṭha. }

  \maintext{aṣṭādaśasahasrāṇi tena sārasvatāya tu |}%

  \maintext{sārasvatas tridhāmāya sahasradaśa sapta ca }||\thinspace1:67\thinspace||%
\translation{And he[, Vasiṣṭha taught] 18,000 [verses] to [9] Sārasvata. Sārasvata [taught] 17,000 [verses] to [10] Tridhāma[n]. }

  \maintext{ṣoḍaśānāṃ sahasrāṇi bharadvājāya vai tataḥ |}%

  \maintext{daśa pañcasahasrāṇi trivṛṣāya abhāṣata }||\thinspace1:68\thinspace||%
\translation{[He taught] 16,000 verses to [11] Bharadvāja. [Bharadvāja] taught 15,000 verses to [12] Trivṛṣa. }

  \maintext{caturdaśasahasrāṇi antarīkṣāya vai tataḥ |}%

  \maintext{trayyāruṇiṃ sahasrāṇi trayodaśa abhāṣata }||\thinspace1:69\thinspace||%
\translation{[Trivṛṣa] then [taught] 14,000 verses to [13] Antarīkṣa. [Antarīkṣa] taught 13,000 [verses] to [14] Trayyāruṇi. }

  \maintext{trayyāruṇis tu viprendro dhanaṃjayam abhāṣata |}%

  \maintext{dvādaśāni sahasrāṇi saṃkṣipya punar abravīt }||\thinspace1:70\thinspace||%
\translation{Trayyāruṇi, the great Brahmin, having abridged them again, taught 12,000 [verses] to [15] Dhanaṃjaya. }

  \maintext{kṛtaṃjayāya samprāpto dhanaṃjayamahāmuniḥ |}%

  \maintext{kṛtaṃjayād dvijaśreṣṭha ṛṇaṃjayamahātmane }||\thinspace1:71\thinspace||%
\translation{Dhanaṃjaya, the great sage, handed [them] over to [16] Kṛtaṃjaya. [That recension was transmitted] from Kṛtaṃjaya, O best of the twice-born, to [17] noble Ṛṇaṃjaya. \blankfootnote{1.71 Note the odd structure in \textit{pāda}s ab: \textit{dhanaṃjayaḥ kṛtaṃjayāya samprāptaḥ}, 
  for a more standard \textit{dhanaṃjayena {\rm (}\textit{purāṇam}{\rm )} samprāpitaṃ kṛtaṃjayam} 
  {\rm (}`the Purāṇa was transmitted to Kṛtaṃjaya'{\rm )}.
 }}

  \maintext{ṛṇañjayāt punaḥ prāpto gautamāya maharṣiṇe |}%

  \maintext{gautamāc ca bharadvājas tasmād dharyātmanāya tu }||\thinspace1:72\thinspace||%
\translation{Then from Ṛṇaṃjaya it was given to [18] Gautama, the great sage, from Gautama to [19] Bharadvāja, from him to [20] Haryātman. \blankfootnote{1.72 The structure of \textit{pāda}s ab is as odd as that of 1.71ab. What was
  intended is probably \textit{ṛṇañjayena prāpitaṃ gautamāya}.
 My emendation in \textit{pāda} d of \textit{haryadvatāya} to \textit{haryātmanāya} {\rm (}for a standard \textit{haryātmane}{\rm )} 
  is based on the list of \textit{vedavyāsa}s in \VISNUP\ 3.3.16--17 {\rm (}see note to 1.63 above{\rm )}.
 }}

  \maintext{rājaśravās tataḥ prāptaḥ somaśuṣmāya vai tataḥ |}%

  \maintext{somaśuṣmāt tataḥ prāptas tṛṇabindus tu bho dvija }||\thinspace1:73\thinspace||%
\translation{Then [21] Rājaśravas received it, then [22] Somaśuṣma. Then from Somaśuṣma [23] Tṛṇabindu received it, O twice-born. \blankfootnote{1.73 The syntax is again slightly odd here. The indention may have been
  \textit{prāpitaṃ rājaśavasā somaśuṣmāya... tatas tṛṇabindunā prāptam}.
 }}

  \maintext{tṛṇabindus tu vṛkṣāya vṛkṣaḥ śaktim abhāṣata |}%

  \maintext{śaktiḥ parāśaraṃ prāha jatukarṇāya vai tataḥ }||\thinspace1:74\thinspace||%
\translation{Tṛṇabindu taught it to [24] Vṛkṣa, Vṛkṣa to [25] Śakti [the father of Parāśara]. Śakti taught it to [26] Parāśara, then [Parāśara] to [27] Jatukarṇa. \blankfootnote{1.74 In other list of \textit{vedavyāsa}s, Tṛṇabindu hands the Purāṇas down to 
  Ṛkṣa, Rūkṣa or Dakṣa {\rm (}see note to 1.63 above{\rm )}. \textit{vṛkṣa} in \textit{pāda} a
  is probably a corrupted form.
 The name Jatukarṇa may be a corrupted form of Jātū- or Jātukarṇa.
 }}

  \maintext{dvaipāyanaṃ tu provāca jatukarṇo maharṣiṇam |}%

  \maintext{romaharṣāya samprāpto dvaipāyanamahāmuniḥ }||\thinspace1:75\thinspace||%
\translation{Jatukarṇa taught it to [28] [Vyāsa] Dvaipāyana, the great sage. Dvaipāyana, the great sage, gave it to [29] Romaharṣa. \blankfootnote{1.75 The syntax of \textit{pāda}s cd echoes that of 1.73ab above.
 }}

  \maintext{romaharṣeṇa provāca putrāyāmitabuddhaye |}%

  \maintext{daśa dve ca sahasrāṇi purāṇaṃ samprakāśitam |}%

  \maintext{mānuṣāṇāṃ hitārthāya kiṃ bhūyaḥ śrotum icchasi }||\thinspace1:76\thinspace||%
\translation{Romaharṣa taught the Purāṇa[s] of 12,000 [verses], now fully revealed, to his son, [30] Amitabuddhi, for the benefit of humankind. What else do you wish to know? \blankfootnote{1.76 Romaharṣa is usually considered to be the same person as Sūta, disciple of Vyāsa Dvaipāyana.
  
   %
  In \BrahmandaPur\ 3.4.67ab {\rm (}\textit{mayā caitat punaḥ proktaṃ putrāyāmitabuddhaye}, see note to 
  1.63 above{\rm )} Amitabuddhi is clearly the name {\rm (}or epithet{\rm )} of Romaharṣa's son. This suggests that the form \textit{romaharṣāya}
  in \textit{pāda} a is a mistake for \textit{romaharṣaś ca}, or similar. MS \msM\ is either transmitting an
  otherwise syntactically problematic reading {\rm (}\textit{romaharṣeṇa}{\rm )} that is more 
  original than that of most other witnesses or \msM's scribe is trying to correct the text.
  Supposing the former, in this case I accepted \msM's reading.
  %
 
  Manuscripts \msCc\ and \msM\ place the \textit{iti} of the colophon at the end of the last \textit{śloka}, before
  the \textit{daṇḍa}s, thus: \textit{icchasīti} ||O|| {\rm (}\msCc{\rm )} and \textit{icchasi iti} ||o|| {\rm (}\msM{\rm )}.
  Note also that \msM\ gives the number of \textit{śloka}s in this chapter, 77, which is almost exactly
  the number of verses this critical edition has produced. The scribe of \msM\ struggled 
  with eyeskips in this chapter, therefore it seems unlikely that he himself
  counted the number of verses he had copied and arrived at this very figure.
  Rather, he copied the number from his exemplar.
 }}
\center{\maintext{\dbldanda\thinspace iti vṛṣasārasaṃgrahe brahmāṇḍasaṃkhyā nāmādhyāyaḥ prathamaḥ\thinspace\dbldanda}}
\translation{Here ends the first chapter in the \textit{Vṛṣasārasaṃgraha} called the Description of the Brahmāṇḍa[s].}

  \chptr{dvitīyo 'dhyāyaḥ}
\fancyhead[CE]{{\footnotesize\textit{Translation of chapter 2}}}%

  \trchptr{  Chapter Two }%

  \maintext{vigatarāga uvāca |}%

  \maintext{śrutaṃ mayā janāgreṇa brahmāṇḍasya tu nirṇayam |}%

  \maintext{pramāṇaṃ varṇarūpaṃ ca saṃkhyā tasya samāsataḥ }||\thinspace2:1\thinspace||%
\translation{Vigatarāga spoke: I have heard the description of the Brahmāṇḍa from [you,] the best of men, its extent, colour, form and the numbers associated with it, in a concise manner. \blankfootnote{2.1 It is unlikely that \textit{janāgreṇa} picks up \textit{mayā} {\rm (}`by me, the best of men'{\rm )}, instead,
  I supposed that this instrumental stands for the ablative or should be 
  understood as `through the best of man.'
 }}

  \maintext{śivāṇḍeti tvayā prokto brahmāṇḍālayakīrtitaḥ |}%

  \maintext{kīdṛśaṃ lakṣaṇaṃ jñeyaṃ pramāṇaṃ tasya vā kati }||\thinspace2:2\thinspace||%
\translation{You mentioned a Śivāṇḍa as taught to be the receptacle of the Brahmāṇḍa. What are its characteristics and how much is its extent? \blankfootnote{2.2 The location where Śivāṇḍa was mentioned is verse 1.40ab above.
 }}

  \maintext{kasya vā layanaṃ jñeyaṃ pramāṇaṃ vātra vāsinaḥ |}%

  \maintext{kā vā tatra prajā jñeyā ko vā tatra prajāpatiḥ }||\thinspace2:3\thinspace||%
\translation{Whose dwelling place is it? And [what] is the scale of the one[s] who dwell there? What kind of people live there? And who is the ruler {\rm (}\textit{prajāpati}{\rm )} there? \blankfootnote{2.3 \textit{vā layanaṃ} in \textit{pāda} a may stand for \textit{vā-ālayanaṃ}, in the sense of \textit{vā-ālayaṃ}.
  The questions in this verse are most probably answered in verses 2.26--33, and if my
  interpretation is correct there, \textit{pramāṇaṃ vātra vāsinaḥ} {\rm (}understand \textit{vāsināṃ}{\rm )} 
  and \textit{pāda} c should refer to the number of inhabitants in the five regions of Īśāna, Tatpuruṣa etc.,
  who are refered to here in \textit{pāda}s a and possibly d.
 }}

  \subchptr{śivāṇḍasaṃkhyā}%

  \trsubchptr{Summary of the Śivāṇḍa}%

  \maintext{anarthayajña uvāca |}%

  \maintext{śivāṇḍalakṣaṇaṃ vipra na tvaṃ praṣṭum ihārhasi |}%

  \maintext{daivatair api kā śaktir jñātuṃ draṣṭuṃ ca tattvataḥ }||\thinspace2:4\thinspace||%
\translation{Anarthayajña spoke: Please don't ask me about the characteristics of the Śivāṇḍa, O Brahmin. How could even the gods have the power to really know and see [the Śivāṇḍa]? }

  \maintext{agamyagamanaṃ guhyaṃ guhyād api samuddhṛtam |}%

  \maintext{na prabhur netaras tatra na daṇḍyo na ca daṇḍakaḥ }||\thinspace2:5\thinspace||%
\translation{The path leading to it is not to be trodden, it is more secret than any secret, and it is lofty. There is no master or servant [lit. the opposite] there, nobody to be punished and no punisher. \blankfootnote{2.5 My emendation to \textit{samuddhṛtam} in \textit{pāda} b is not fully satisfactory, but the 
  readings transmitted in the witnesses are problematic. \msM, a MS not collated for
  this chapter, gives a confusing reading: \textit{sa\uncl{murdhni}dam}. I doubt
  if \Ed's \textit{samṛddhidam} {\rm (}`yielding success'{\rm )} is the correct reading.
  Perhaps \textit{samudāhṛtam} {\rm (}`declared, talked about as'{\rm )} was meant.
  It is not inconceivable that \msCc's {\rm (}and \msM's{\rm )} \textit{agamyagahanaṃ} 
  {\rm (}`it is inaccessible because of its depth'{\rm )} is original and
  it is to be contrasted with \textit{samuddhṛtam} {\rm (}`lofty'{\rm )}. One also wonders if
  \textit{guhād} could be the right reading, and in what sense, in \textit{pāda} b.
 }}

  \maintext{na satyo nānṛtas tatra suśīlo no duḥśīlavān |}%

  \maintext{nānṛjur na ca dambhitvaṃ na tṛṣṇā na ca īrṣyatā }||\thinspace2:6\thinspace||%
\translation{There are no truthful or untruthful people there, no moral or immoral people, no crooked people, no hypocrisy, no thirst or envy. \blankfootnote{2.6 Strictly speaking \textit{duḥśīlavān} in \textit{pāda} b is unmetrical; understand or pronounce \textit{duśīlavān}.
 \textit{īrṣyatā} {\rm (}for \textit{īrṣyā}, see 2.7a{\rm )} is a form rarely attested.
 }}

  \maintext{na krodho na ca lobho 'sti na māno 'sti na sūyakaḥ |}%

  \maintext{īrṣyā dveṣo na tatrāsti na śaṭho na ca matsaraḥ }||\thinspace2:7\thinspace||%
\translation{There is no anger or desire there, no arrogance or discontent {\rm (}[a]sūyaka{\rm )}, no envy or hatred, no cheaters and no jealousy. \blankfootnote{2.7 \textit{na sūyakaḥ} in \textit{pāda} b stands for \textit{na asūyaka} metri causa.
 }}

  \maintext{na vyādhir na jarā tatra na śoko 'sti na viklavaḥ |}%

  \maintext{nādhamaḥ puruṣas tatra nottamo na ca madhyamaḥ }||\thinspace2:8\thinspace||%
\translation{There is no disease, no aging, no grief and no agitation there, there are no inferior or superior people and there is nobody in-between. }

  \maintext{notkṛṣṭo mānavas tasmin striyaś caiva śivālaye |}%

  \maintext{na nindā na praśaṃsāsti matsarī piśuno na ca }||\thinspace2:9\thinspace||%
\translation{There are no privileged men or women there in Śiva's abode, no reproach or praise, no selfish or treacherous people. }

  \maintext{garvadarpaṃ na tatrāsti krūramāyādikaṃ tathā |}%

  \maintext{yācamāno na tatrāsti dātā caiva na vidyate }||\thinspace2:10\thinspace||%
\translation{There is no pride or arrogance there, no cruelty or trickery and so on. There are no beggars and no donors there. }

  \maintext{anarthī vraja tatrasthaḥ kalpavṛkṣasamāśritaḥ |}%

  \maintext{na karma nāpriyas tatra na kaliḥ kalaho na ca }||\thinspace2:11\thinspace||%
\translation{Go without material desires {\rm (}\textit{anarthin}{\rm )}, being there you'll be resting under a wishing tree. There is no karma there and no enemy. No Kali age is there and there is no fighting. \blankfootnote{2.11 Note the term \textit{anartī} in \textit{pāda} a: it might have something to do 
  with non-material sacrifice {\rm (}\textit{anarthayajña}{\rm )}, the topic of chapter 11.
  \textit{vraja} in \textit{pāda} a is suspect.
 }}

  \maintext{dvāparo na ca na tretā kṛtaṃ cāpi na vidyate |}%

  \maintext{manvantaraṃ na tatrāsti kalpaś caiva na vidyate }||\thinspace2:12\thinspace||%
\translation{There is no Dvāpara age or Tretā or Kṛta. There are no \textit{manvantara}s there and no \textit{kalpa}s. \blankfootnote{2.12 On \textit{manvantara}s and \textit{kalpa}s, see 1.23--24 above.
 }}

  \maintext{āhūtasamplavaṃ nāsti brahmarātridinaṃ tathā |}%

  \maintext{na janmamaraṇaṃ tatra āpadaṃ nāpnuyāt kvacit }||\thinspace2:13\thinspace||%
\translation{No universal floods of destruction come, and there are no days and nights of Brahmā. There is no birth and death there and one never encounters catastrophes. \blankfootnote{2.13 \textit{āhūtasamplava} for the more widely attested form \textit{ābhūtasamplava} occurs, e.g.,
  in some MSS transmitting \SDHS\ 10.77 and 81 {\rm (}see \mycite{SDhS10_ed}{\rm )}.
 }}

  \maintext{na cāśāpāśabaddho 'sti rāgamohaṃ na vidyate |}%

  \maintext{na devā nāsurās tatra na yakṣoragarākṣasāḥ }||\thinspace2:14\thinspace||%
\translation{Nobody is tied to the noose of hope and there is no passion or delusion. There are no gods and demons there and no Yakṣas, Serpents and Rākṣasas. }

  \maintext{na bhūtā na piśācāś ca gandharvā ṛṣayas tathā |}%

  \maintext{tārāgrahaṃ na tatrāsti nāgakiṃnaragāruḍam }||\thinspace2:15\thinspace||%
\translation{There are no Ghosts nor Piśācas, no Gandharvas and no Ṛṣis. There are no planets there, no Nāgas, Kiṃnaras or Garuḍa-like creatures. }

  \maintext{na japo nāhnikas tatra nāgnihotrī na yajñakṛt |}%

  \maintext{na vrataṃ na tapaś caiva na tiryannarakaṃ tathā }||\thinspace2:16\thinspace||%
\translation{There are no recitations or daily rituals there, nobody performs the Agnihotra and there are no sacrificers. There are no religious observances and no austerities and no `animal hell'. \blankfootnote{2.16 The phrase of \textit{tiryaṅnaraka} appears in \MBH\ 3.181.18ab: 
  \textit{aśubhaiḥ karmabhiḥ pāpās tiryaṅnarakagāminaḥ}. Here \mycite{GanguliMBh} translates \textit{tiryaṅ} 
  separately as `in a crooked way,' but I suspect that in the \VSS\ \textit{tiryannaraka} has
  more to do with \textit{tiraggati}, being reduced to animal existence, being reborn as an animal or entering a 
  hell in animal form. 
  Cf. \MBH\ 13.134.057 {\rm (}\verify{\rm )}:
  \textit{nṛṣu janma labhante ye karmaṇā madhyamāḥ smṛtāḥ |
  tiryaṅnarakagantāro hy adhamās te narādhamāḥ ||},
  and \textit{Umāmaheśvarasaṃvāda} 6.1: 
  \textit{avamanyanti ye viprān sarvaloke namaskṛtān |
  narakaṃ yānti te sarve tiryagyoniṃ vrajanti ca~||.}
  I suspect that \textit{nātirya°} in the witnesses is only a scribal mistake for \textit{na tirya°}.
 }}

  \maintext{tasyeśānasya devasya aiśvaryaguṇavistaram |}%

  \maintext{api varṣaśatenāpi śakyaṃ vaktuṃ na kenacit }||\thinspace2:17\thinspace||%
\translation{Nobody would be able to tell the extent of the qualities of the god Īśāna's powers, not even in a hundred years. \blankfootnote{2.17 My translation of \textit{aiśvaryaguṇa}° is tentative. It could be taken as a \textit{dvandva} compound
  {\rm (}e.g. `supremacy and qualities'{\rm )}. The expression \textit{sarva}° or \textit{aṣṭaiśvaryaguṇopeta}
  occurs frequently, e.g. in \SIVP\ 7.2.8.28ab and \SKANDAP\ 55.30cd, and \SDHU\ 2.6, 79, 125, 127,
  with \textit{aiśvarya} most probably refering to the eight \textit{siddhi}s \textit{aṇiman, laghiman} etc.
  De Simini {\rm (}2016a, 386{\rm )},\nocite{DeSiminiGods2016} e.g., translates \textit{sarvaiśvaryaguṇopetaḥ} in \SDHU\ 2.127 as
  `endowed with all the qualities of lordship.'
 }}

  \maintext{harecchāprabhavāḥ sarve paryāyeṇa bravīmi te |}%

  \maintext{devamānuṣavarjyāni vṛkṣagulmalatādayaḥ }||\thinspace2:18\thinspace||%
\translation{All are born by Hara's wish. I shall teach [them to] you one by one, excluding gods and people, starting with the trees, the bushes and creepers. \blankfootnote{2.18 Note the gender confusion in this verse, and the way I take \textit{pāda} a as a separate 
  statement to aviod a further confusion of case.
 }}

  \maintext{parārdhadviguṇotsedho vistāraś ca tathāvidhaḥ |}%

  \maintext{anekākārapuṣpāṇi phalāni ca manoharam }||\thinspace2:19\thinspace||%
\translation{The height [of the Śivāṇḍa] is two \textit{parārdha}s, and [its] width is the same. There are lovely flowers of different forms [there] and also lovely fruits. \blankfootnote{2.19 I understand \textit{pāda} a as \textit{parārdhadviguṇa utsedho}, i.e. as an example of double \textit{sandhi}.
  On the other hand, °\textit{sedho} is only my conjecture, and may refer to something else than the Śivāṇḍa.
  Note the number confusion in \textit{pāda} d, and also that two \textit{parārdha}s is one \textit{para}, 
  the highest possible number according to verses 1.35--36 above. The number may refer to
  any unit of length, but 2.23 below suggests that it is \textit{yojana}s.
 }}

  \maintext{anye kāñcanavṛkṣāṇi maṇivṛkṣāṇy athāpare |}%

  \maintext{pravālamaṇiṣaṇḍāś ca padmarāgaruhāṇi ca }||\thinspace2:20\thinspace||%
\translation{There are also golden trees and also gem trees, coral gem thickets and ruby plants. \blankfootnote{2.20 Note that both \textit{anye} and \textit{apare} here pick up neuter nouns {\rm (}gender confusion{\rm )}.
 }}

  \maintext{svādumūlaphalāḥ skandhalatāviṭapapādapāḥ |}%

  \maintext{kāmarūpāś ca te sarve kāmadāḥ kāmabhāṣiṇaḥ }||\thinspace2:21\thinspace||%
\translation{There are tasty roots and fruits and trees with creepers on their twigs. All are shape-shifters and they fulfill man's desires and they whisper seductively. \blankfootnote{2.21 My conjectures in \textit{pāda}s ab result in a compoud split at the caesura, which
  may have been the reason why the line got corrupted.
 }}

  \maintext{tatra vipra prajāḥ sarve anantaguṇasāgarāḥ |}%

  \maintext{tulyarūpabalāḥ sarve sūryāyutasamaprabhāḥ }||\thinspace2:22\thinspace||%
\translation{There [in the Śivāṇḍa], O Brahmin, all the subjects are the oceans of endless virtues. They are all equally beautiful and strong, and they shine like millions of suns. }

  \maintext{parārdhadvayavistāraṃ parārdhadvayam āyatam |}%

  \maintext{parārdhadvayavikṣepā yojanānāṃ dvijottama }||\thinspace2:23\thinspace||%
\translation{[The Śivāṇḍa] is two \textit{parārdha} long and two \textit{parārdha} wide, and two \textit{parārdha yojana}s is its extension, O great Brahmin. \blankfootnote{2.23 I understand \textit{pāda}s cd, tentatively, as \textit{vikṣepaṃ parārdhadvayaṃ yojanānāṃ}
 }}

  \maintext{aiśvaryatvaṃ na saṃkhyāsti balaśaktiś ca bho dvija |}%

  \maintext{adhordhvo na ca saṃkhyāsti na tiryañ caiti kaścana }||\thinspace2:24\thinspace||%
\translation{[Īśāna's] powers cannot be expressed by numbers, neither can [His] powerfulness, O twice-born. [In fact, the extension in the Śivāṇḍa] downwards and upwards cannot be expressed by numbers, neiter can its horizontal extension. \blankfootnote{2.24 This line is a reply to 2.17b.
 }}

  \maintext{śivāṇḍasya ca vistāram āyāmaṃ ca na vedmy aham |}%

  \maintext{bhogam akṣaya tatraiva janmamṛtyur na vidyate }||\thinspace2:25\thinspace||%
\translation{[In reality,] I do not know the length and width of the Śivāṇḍa. Enjoyment is undecaying there, and there is no birth or death there. \blankfootnote{2.25 \textit{Pāda} c is transmitted in an unmetrical form and with a gender problem in the witnesses
  {\rm (}\textit{bhogam akṣayas}{\rm )}, hence my emendation using a stem form noun,
  a phenomenon frequently seen in this text. But note that \textit{bhoga} is normally masculine;
  there might be a hiatus-filler in-between: \textit{bhoga-m-akṣaya}{\rm )}.
 }}

  \maintext{śivāṇḍamadhyam āśritya gokṣīrasadṛśaprabhāḥ |}%

  \maintext{parārdhaparakoṭīnām īśānānāṃ smṛtālayaḥ }||\thinspace2:26\thinspace||%
\translation{In the centre of the Śivāṇḍa, [creatures] shine like cow's milk. [It is] said to be the region {\rm (}\textit{ālaya}{\rm )} of the one and a half \textit{para} crore Īśānas. \blankfootnote{2.26 Note the stem form \textit{smṛta} in \textit{pāda} d {\rm (}cf. 2.29d{\rm )}. I understand \textit{īśānānāṃ} as \textit{aiśānānāṃ}.
 
  Īśāna is traditionally the upward-looking face of Śiva, his region is positioned in the centre here.
  Note that the somewhat cryptic third \textit{pāda}s here and in the coming verses
  may or may not refer to the number of people living in the given region. 
  They may tell us about the extent of the given region, although the numbers are much
  higher than what one would expect after verse 2.23.
 }}

  \maintext{bālasūryaprabhāḥ sarve jñeyās tatpuruṣālaye |}%

  \maintext{parārdhaparakoṭīnāṃ pūrvasyāṃ diśam āśritāḥ }||\thinspace2:27\thinspace||%
\translation{They are all like the rising sun in the region of Tatpuruṣa. They are one and a half \textit{para} crore [in number], living in the east. \blankfootnote{2.27 The genitive of \textit{parārdhaparakoṭīnāṃ} is baffling here and in the coming verses,
  but I suspect that again the expression gives the number of subjects living in the given region.
  \textit{pūrvasyāṃ} is meant to mean \textit{pūrvāṃ} {\rm (}cf. \textit{dakṣiṇāṃ, paścimāṃ,} and \textit{uttarāṃ} in the next verses{\rm )};
  note how \msNb\ tries to save the construction by reading \textit{diśi}.
 
  This verse conforms to the traditional view that Śiva's Tatpuruṣa-face
  is looking to the east.
 }}

  \maintext{bhinnāñjanaprabhāḥ sarve dakṣiṇāṃ diśam āśritāḥ |}%

  \maintext{parārdhaparakoṭīnām aghorālayam āśritāḥ }||\thinspace2:28\thinspace||%
\translation{Everybody is like collyrium in the southern direction, in the region of Aghora, one and a half \textit{para} crore [in number]. \blankfootnote{2.28 Note the Aiśa form \textit{diśiṃ} in \msCb, and that Aghora is indeed usually south-facing.
 }}

  \maintext{kundenduhimaśailābhāḥ paścimāṃ diśam āśritāḥ |}%

  \maintext{parārdhaparakoṭīnāṃ sadya{-}m{-}iṣṭālayaḥ smṛtaḥ }||\thinspace2:29\thinspace||%
\translation{In the western direction, they are like jasmine, the moon, like snowy rocks. Sadyojāta's lovely region is [home] to one and a half \textit{para} crore [people]. \blankfootnote{2.29 Note the Aiśa form \textit{diśiṃ} in \msNc\ in \textit{pāda} b.
  In \textit{pāda} d, we may presuppose the presence of a \textit{sandhi}-bridge: \textit{sadya-m-iṣṭālayaḥ}.
  Sadyojāta is traditionally associated with the western direction.
 }}

  \maintext{kuṅkumodakasaṃkāśā uttarāṃ diśam āśritāḥ |}%

  \maintext{parārdhaparakotīnāṃ vāmadevālayaḥ smṛtaḥ }||\thinspace2:30\thinspace||%
\translation{In the northern direction, they are like saffron in water. Vāmadeva's region is [home] to one and a half \textit{para} crore [people]. \blankfootnote{2.30 Note the Aiśa form \textit{diśiṃ} in \msCa\ in \textit{pāda} b.
 Vāmadeva is traditionally associated with the western direction.
 }}

  \maintext{īśānasya kalāḥ pañca vaktrasyāpi catuṣ kalāḥ |}%

  \maintext{aghorasya kalā aṣṭau vāmadevās trayodaśa }||\thinspace2:31\thinspace||%
\translation{Īśāna has five parts {\rm (}\textit{kalā}{\rm )}, [his Tatpuruṣa] face has four. Aghora has eight, and there are thirteen Vāmadeva[-\textit{kalā}]s. \blankfootnote{2.31 Note how \textit{vaktrasya} should refer to Śiva's Tatpuruṣa-face, 
  given that the text lists Śiva's five faces: Īśāna, Tatpuruṣa, Aghora, Vāmadeva, Sadyojāta.
 }}

  \maintext{sadyaś cāṣṭau kalā jñeyāḥ saṃsārārṇavatārakāḥ |}%

  \maintext{aṣṭatriṃśat kalā hy etāḥ kīrtitā dvijasattama }||\thinspace2:32\thinspace||%
\translation{Sadyojāta has eight parts. These parts, altogether thirty-eight, which liberate us from the ocean of existence, have been taught, O truest Brahmin. \blankfootnote{2.32 Note \textit{sadyaś} in \textit{pāda} a for \textit{sadyasaś} or \textit{sadyojātasya}.
 }}

  \maintext{saṃkhyā varṇā diśaś caiva ekaikasya pṛthak pṛthak |}%

  \maintext{pūrvoktena vidhānena bodhavyās tattvacintakaiḥ }||\thinspace2:33\thinspace||%
\translation{Those who explore the truth should know the numbers, the colours and directions associated with each one [of Śiva's faces] in the way taught above. }

  \maintext{śivāṇḍagamanākṛṣṭyā śivayogaṃ sadābhyaset |}%

  \maintext{śivayogaṃ vinā vipra tatra gantuṃ na śakyate }||\thinspace2:34\thinspace||%
\translation{If one has the intention to go to the Śivāṇḍa, one should practise Śiva-yoga regularly. Without Śiva-yoga, O Brahmin, it is impossible to go there. \blankfootnote{2.34 \textit{ākṛṣṭyā} in \textit{pāda} a might be corrupt.
 }}

  \maintext{aśvamedhādiyajñānāṃ koṭyāyutaśatāni ca |}%

  \maintext{kṛcchrāditapa sarvāṇi kṛtvā kalpaśatāni ca |}%

  \maintext{tatra gantuṃ na śakyeta devair api tapodhana }||\thinspace2:35\thinspace||%
\translation{[Even] by [performing] millions of sacrifices such as the Aśvamedha, or by performing all the difficult austerities for a hundred \textit{kalpa}s, it is impossible to get there even for the gods, O great ascetic. \blankfootnote{2.35 Understand \textit{kṛcchrāditapa sarvāṇi} as \textit{kṛcchrāditapāṃsi sarvāṇi}. It can be 
  considered an instance of the use of a stem form noun.
 }}

  \maintext{gaṅgādisarvatīrtheṣu snātvā taptvā ca vai punaḥ |}%

  \maintext{tatra gantuṃ na śakyeta ṛṣibhir vā mahātmabhiḥ }||\thinspace2:36\thinspace||%
\translation{By [merely] bathing and performing austerities at all the sacred places such as the Gaṅgā, even the honorable Ṛṣis will not be able to get there. }

  \maintext{saptadvīpasamudrāṇi ratnapūrṇāni bho dvija |}%

  \maintext{dattvā vā vedaviduṣe śraddhābhaktisamanvitaḥ |}%

  \maintext{tatra gantuṃ na śakyeta vinā dhyānena niścayaḥ }||\thinspace2:37\thinspace||%
\translation{Or [even] by donating the oceans of the seven islands with all their gems to a Veda expert, O Brahmin, with faith and devotion, one will not be able to go there without meditation. [This is a] certainty. }

  \maintext{svadehān māṃsam uddhṛtya dattvārthibhyaś ca niścayāt |}%

  \maintext{svadāraputrasarvasvaṃ śiro 'rthibhyaś ca yo dadet |}%

  \maintext{na tatra gantuṃ śakyeta anyair vāpi suduṣkaraiḥ }||\thinspace2:38\thinspace||%
\translation{He who destroys his own body and gives it without hesitation to those who are in need of it, or he who gives away his wife, his son and his possessions or his own head to those in need, or he who [performs] other difficult deeds, will not be able to go there [by merely doing these]. }

  \maintext{yajñatīrthatapodānavedādhyayanapāragaḥ |}%

  \maintext{brahmāṇḍāntasya bhogāṃs tu bhuṅkte kālavaśānugaḥ }||\thinspace2:39\thinspace||%
\translation{He who has completed the sacrifices, the pilgrimages, the austerities, the donations, the study of the Vedas, will experience those enjoyments that the Brahmāṇḍa offers, still being subject to time and death. }

  \maintext{kālena samapreṣyeṇa dharmo yāti parikṣayam |}%

  \maintext{alātacakravat sarvaṃ kālo yāti paribhraman |}%

  \maintext{traikālyakalanāt kālas tena kālaḥ prakīrtitaḥ }||\thinspace2:40\thinspace||%
\translation{Dharma decays tossed forward by time. Time flies moving everything round and round like a circle of burning coal. Time is called \textit{kāla} because of the waves {\rm (}\textit{kalana}{\rm )} of the three divisions of time [past, present, future]. \blankfootnote{2.40 Notice the muta cum liquida licence in \textit{pāda} a: \textit{samapre}° renders as short-short-long.
  I take \textit{samapreṣyena} as if it read \textit{sampreṣito}, picking up \textit{dharmo}; otherwise
  it is difficult to make sense of it.
 As Kenji Takahashi pointed out to me, \mycite{Fitzgerald_Alatacakra2012} is
  a good starting point to understand the implication of \textit{alātacakra}, 
  `a single, rapidly twirled torch creat[ing] the illusion of an apparently real, continuous circle'
  {\rm (}ibid., p. 777{\rm )}. The function of \textit{sarvaṃ} in \textit{pāda} a becomes clear only if
  we understand \textit{paribhraman} in a causative sense {\rm (}for \textit{paribhramayan}{\rm )}.
 One cannot help noticing that this verse would be in a more fitting context after verse 1.31,
  at the end of a section on \textit{kāla}. On the other hand, it leads us to the next topic, Dharma,
  smoothly.
 }}
\center{\maintext{\dbldanda\thinspace iti vṛṣasārasaṃgrahe śivāṇḍasaṃkhyā nāmādhyāyo dvitīyaḥ\thinspace\dbldanda}}
\translation{Here ends the second chapter in the \textit{Vṛṣasārasaṃgraha} called the Description of the Śivāṇḍa.}

  \chptr{tṛtīyo 'dhyāyaḥ}
\fancyhead[CE]{{\footnotesize\textit{Translation of chapter 3}}}%

  \trchptr{  Chapter Three }%

  \subchptr{dharmapravacanam}%

  \trsubchptr{An Exposition of Dharma}%

  \maintext{vigatarāga uvāca |}%

  \maintext{kimarthaṃ dharmam ity āhuḥ katimūrtiś ca kīrtyate |}%

  \maintext{katipādavṛṣo jñeyo gatis tasya kati smṛtāḥ }||\thinspace3:1\thinspace||%
\translation{Vigatarāga spoke: Why do they call it Dharma? And how many embodiments {\rm (}\textit{mūrti}{\rm )} is he known to have? He is known as a bull: how many legs does it/he have? How many are his paths? \blankfootnote{3.1 For the correct interpretation of \textit{pāda} a, namely to decide whether these questions
  focus on the bull of Dharma or Dharma itself/himself, see 
  the end of the previous chapter, where \textit{dharma} was mentioned {\rm (}2.40b{\rm )},
  and to which the present verse is a reaction; see also
  \MBH\ 12.110.10--11:
   %
  \textit{prabhāvārthāya bhūtānāṃ dharmapravacanaṃ kṛtam | 
  yat syād ahiṃsāsaṃyuktaṃ sa dharma iti niścayaḥ || 
  dhāraṇād dharma ity āhur dharmeṇa vidhṛtāḥ prajāḥ | 
  yat syād dhāraṇasaṃyuktaṃ sa dharma iti niścayaḥ ||}
   %
  Note the similarities of \MBH\ this passage with this chapter: the phrase \textit{dharma ity āhur},
  the fact that the present chapter from verse 18 on is actually a chapter on \textit{ahiṃsā},
  and that the etimological explanation involves the word [\textit{ā}]\textit{dhāraṇa} in
  both cases. These lead me to think that in \textit{pāda}s ab of this verse in the \VSS,
  it is Dharma that is the focus of the inquiry and not the bull.
 
  
 Understand \textit{pāda} d as \textit{gatayas tasya kati smṛtāḥ}. I have accepted
  \textit{smṛtāḥ} because this plural signals that \textit{gatis} is meant to be plural,
  similarly to what happens in 3.6cd {\rm (}\textit{tasya patnī... mahābhāgāḥ}{\rm )}.
  The use of the singular in a context of numbers and quantities is one of 
  the hallmarks of the language of the \VSS, see p. \verify.
 
  On Dharma as a bull, see Introduction, pp. \verify.
 }}

  \maintext{kautūhalaṃ mamotpannaṃ saṃśayaṃ chindhi tattvataḥ |}%

  \maintext{kasya putro muniśreṣṭha prajās tasya kati smṛtāḥ }||\thinspace3:2\thinspace||%
\translation{I have become curious [about these questions]. Put an end to my doubts for good. Whose son is [Dharma], O best of sages? How many children does he have? }

  \maintext{anarthayajña uvāca |}%

  \maintext{dhṛtir ity eṣa dhātur vai paryāyaḥ parikīrtitaḥ |}%

  \maintext{ādhāraṇān mahattvāc ca dharma ity abhidhīyate  }||\thinspace3:3\thinspace||%
\translation{Anarthayajña spoke: Well, \textit{dhṛti} {\rm (}`firmness'{\rm )} is [of the same] verbal root [as \textit{dharma}], and is said to be [its] synonym. It is called \textit{dharma} because it supports {\rm (}\textit{āDHĀRaṇa}{\rm )} and because it is great {\rm (}\textit{MAhattva}{\rm )}. \blankfootnote{3.3 For similar Purāṇic passages on the etimology of \textit{dharma}, see the apparatus to
  this verse.
 
  The insertion in my translation '[of the same]' solves the problem of a noun {\rm (}\textit{dhṛti}{\rm )} seemingly
  being considered a verbal root {\rm (}\textit{dhātu}{\rm )} here. I owe thanks to Judit Törzsök for this interpretation.
  For similar passages with nominal stems appearently being treated as \textit{dhātu}s, see e.g. 
  \VAYUP\ 3.17cd:
  \textit{bhāvya ity eṣa dhātur vai bhāvye kāle vibhāvyate};
  \VAYUP\ 3.19cd {\rm (}= \BRAHMANDAPUR\ 1.38.21ab{\rm )}:
  \textit{nātha ity eṣa dhātur vai dhātujñaiḥ pālane smṛtaḥ};
  \LINPU\ 2.9.19:
  \textit{bhaja ity eṣa dhātur vai sevāyāṃ parikīrtitaḥ}
 }}

  \maintext{śrutismṛtidvayor mūrtiś catuṣpādavṛṣaḥ sthitaḥ |}%

  \maintext{caturāśrama yo dharmaḥ kīrtitāni manīṣibhiḥ }||\thinspace3:4\thinspace||%
\translation{The four-legged Bull is the embodiment of both Śruti and Smṛti. It is Dharma, as made up of the four \textit{āśrama}s. \blankfootnote{3.4 A similar image of the legs of the Bull of Dharma being the four {\rm (}and not three, at least according to
  \mycitep{OlivelleAsrama}{55} and
  \mycitep{GanguliMBh}{Śāntiparvan CCLXX}{\rm )} 
  \textit{āśrama}s is hinted at \MBH\ 12.262.19--21: 
   %
  \textit{dharmam ekaṃ catuṣpādam āśritās te nararṣabhāḥ |
   taṃ santo vidhivat prāpya gacchanti paramāṃ gatim ||
   gṛhebhya eva niṣkramya vanam anye samāśritāḥ |
   gṛham evābhisaṃśritya tato 'nye brahmacāriṇaḥ ||
   dharmam etaṃ catuṣpādam āśramaṃ brāhmaṇā viduḥ |
   ānantyaṃ brahmaṇaḥ sthānaṃ brāhmaṇā nāma niścayaḥ ||}.
   %
  On the more frequently quoted interpretation of the four legs, see 
  \mycitep{OlivelleAsrama}{235}, a translation of \Manu\ 1.81--82:
  `Dharma and truth possess all four feet and are whole during the Kṛta yuga, 
  and people did not obtain anything unrighteously {\rm (}\textit{adharmeṇa}{\rm )}. 
  By obtaining, however, \textit{dharma} has lost one foot during each of the other \textit{yuga}s 
  and righteousness {\rm (}\textit{dharma}{\rm )} likewise has diminished by one quarter due to theft, 
  falsehood, and deceit. {\rm (}MDh 1.81--82{\rm )}.'
   %
  Understand \textit{pāda}s c and d as \textit{catvāri āśramāṇi kīrtitāni dharmo manīṣibhiḥ} or
  \textit{yo dharmaḥ kīrtitaś caturāśramāṇi manīṣibhiḥ} or 
  \textit{yo dharmaś caturāśramaḥ kīrtito manīṣibhiḥ}. Judit Törzsök suggested
  that \textit{caturāśrama} and \textit{dharmaḥ} may be interpreted as a compound here.
 }}

  \maintext{gatiś ca pañca vijñeyāḥ śṛṇu dharmasya bho dvija |}%

  \maintext{devamānuṣatiryaṃ ca narakasthāvarādayaḥ }||\thinspace3:5\thinspace||%
\translation{And the paths of Dharma are five. Listen, O Brahmin: [existence as] gods, men, animals, [existence in] hell and [as] immovable things [such as plants and rocks] etc. \blankfootnote{3.5 Note the use of the singular next to numbers in \textit{pāda} a, as in 3.1d, and that
  \textit{vijñeyāḥ} is an emendation from \textit{vijñeyaḥ} following the logic of 3.1d.
 \textit{tirya} seems to be an acceptable nominal stem in this text for \textit{tiryañc}. See,
  e.g., 4.6a: \textit{devamānuṣatiryeṣu}. \textit{°ādayaḥ} in \textit{pāda} d seems superfluous.
 }}

  \maintext{brahmaṇo hṛdayaṃ bhittvā jāto dharmaḥ sanātanaḥ |}%

  \maintext{tasya patnī mahābhāgā trayodaśa sumadhyamāḥ }||\thinspace3:6\thinspace||%
\translation{Eternal Dharma was born after splitting Brahmā's heart. He has beautiful wives, thirteen in number, with nice waists. \blankfootnote{3.6 Note the use of the singular in \textit{pāda}s cd. I have left \textit{sumadhyamāḥ} as the
  manuscripts transmit it: it signals the presence of the plural. And consider 
  correcting \textit{mahābhāgā} to \textit{mahābhāgās}. In sum, understand
  \textit{tasya patnyo mahābhāgās trayodaśa sumadhyamāḥ}.
 }}

  \maintext{dakṣakanyā viśālākṣī śraddhādyāḥ sumanoharāḥ |}%

  \maintext{tasya putrāś ca pautrāś ca anekāś ca babhūva ha |}%

  \maintext{eṣa dharmanisargo 'yaṃ kiṃ bhūyaḥ śrotum icchasi }||\thinspace3:7\thinspace||%
\translation{They are Dakṣa's daughters, [called] Śraddhā and so on. They have huge eyes and they are beautiful. Numerous sons and grandsons were born to him. This is the emergence of Dharma. What more do you wish to hear? \blankfootnote{3.7 \textit{śraddhāḍhyāḥ} in \textit{pāda} b is an attractive lectio difficilior {\rm (}`they were rich in faith/devotion'{\rm )}, but I have finally 
  decided to accept the easier and better-attested \textit{śraddhādyā}[\textit{ḥ}].
  Again, I have chosen/applied the plural forms \textit{°ādyāḥ} and \textit{sumanoharāḥ} in \textit{pāda} b to hint at the fact
  that the presence of the plural is to be preferred here; thus only \textit{viśālākṣī} is 
  problematic. As \textit{patnī} in the previous verse, it should be treated as a plural.
  Note the use of the singular for the plural also in \textit{pāda}s cd, especially \textit{babhūva ha} for \textit{babhūvuḥ}
  {\rm (}\textit{babhūva ha} perhaps being a phonetic and metrically `adjusted' equivalent, so to say, of \textit{babhūvuḥ}{\rm )}.
 }}

  \maintext{vigatarāga uvāca |}%

  \maintext{dharmapatnī viśeṣeṇa putras tābhyaḥ pṛthak pṛthak |}%

  \maintext{śrotum icchāmi tattvena kathayasva tapodhana }||\thinspace3:8\thinspace||%
\translation{Vigatarāga spoke: I would like to hear about Dharma's wives truly and about each one of the sons born to them. Teach me, O great ascetic. \blankfootnote{3.8 I have emended \textit{tebhyaḥ} to the correct feminine form \textit{tābhyaḥ}
  because I suspect that it is only the result of some early confusion
  brought about by \textit{putras}, although \textit{tebhyaḥ} might be original.
  Note again the use of the singular {\rm (}nominative{\rm )} for the plural {\rm (}accusative{\rm )} in \textit{pāda}s ab.
  Alternatively, emend \textit{dharmapatnī} to \textit{dharmapatnīr} {\rm (}plural accusative{\rm )} and 
  \textit{putras} to \textit{putrān} to make them work with \textit{śrotum icchāmi}.
 }}

  \maintext{anarthayajña uvāca |}%

  \maintext{śraddhā lakṣmīr dhṛtis tuṣṭiḥ puṣṭir medhā kriyā lajjā |}%

  \maintext{buddhiḥ śāntir vapuḥ kīrtiḥ siddhiḥ prasūtisambhavāḥ }||\thinspace3:9\thinspace||%
\translation{Anarthayajña spoke: [Dharma's wives are] [1] Śraddhā {\rm (}`Faith'{\rm )}, [2] Lakṣmī {\rm (}`Prosperity'{\rm )}, [3] Dhṛti {\rm (}`Resolution'{\rm )}, [4] Tuṣṭi {\rm (}`Satisfaction'{\rm )}, [5] Puṣṭi {\rm (}`Growth'{\rm )}, [6] Medhā {\rm (}`Wisdom'{\rm )}, [7] Kriyā {\rm (}`Labour'{\rm )}, [8] Lajjā {\rm (}`Modesty'{\rm )}, [9] Buddhi {\rm (}`Intelligence'{\rm )}, [10] Śānti {\rm (}`Tranquillity'{\rm )}, [11] Vapus {\rm (}`Beauty'{\rm )}, [12] Kīrti {\rm (}`Fame'{\rm )}, [13] Siddhi {\rm (}`Success'{\rm )}, [all] born to Prasūti [Dakṣa's wife]. \blankfootnote{3.9 Note how \textit{lajjā} in \textit{pāda} b makes the line unumetrical.
 
  For Dharma's thirteen wives and their sons, see, e.g., \LINPU\ 1.5.34--37 {\rm (}note the 
  similarity between the first line and \VSS\ 3.6cd--7ab above{\rm )}:
   %
  \textit{dharmasya patnyaḥ śraddhādyāḥ kīrtitā vai trayodaśa |
   tāsu dharmaprajāṃ vakṣye yathākramam anuttamam ||
   kāmo darpo 'tha niyamaḥ saṃtoṣo lobha eva ca |
   śrutas tu daṇḍaḥ samayo bodhaś caiva mahādyutiḥ ||
   apramādaś ca vinayo vyavasāyo dvijottamāḥ |
   kṣemaṃ sukhaṃ yaśaś caiva dharmaputrāś ca tāsu vai || 
   dharmasya vai kriyāyāṃ tu daṇḍaḥ samaya eva ca |
   apramādas tathā bodho buddher dharmasya tau sutau ||}.
   %
 
  \textit{prasūtisambhavāḥ} in \textit{pāda} d is a rather bold conjecture that can be supported by two facts:
  firstly, the readings of the manuscripts are difficult to make sense of and thus are
  probably corrupt; secondly, a corruption from the name Prasūti,
  traditionally the name of Dakṣa's wife, to \textit{ābhūti}
  is relatively easily to explain, \textit{sū} and \textit{bhū} being close enough in some scripts 
  {\rm (}e.g. in \msCa{\rm )} to cause confusion. Another option would be to accept 
  Ābhūti as the name of Dakṣa's wife.
   %
  For Prasūti being Dakṣa's wife in other sources,
  see, e.g., \LINPU\ 1.5.20--21 {\rm (}but also note the presence of the name Sambhūti{\rm )}:
  \textit{prasūtiḥ suṣuve dakṣāc caturviṃśatikanyakāḥ |
  śraddhāṃ lakṣmīṃ dhṛtiṃ puṣṭiṃ tuṣṭiṃ medhāṃ kriyāṃ tathā ||
  buddhi lajjāṃ vapuḥ śāntiṃ siddhiṃ kīrtiṃ mahātapāḥ |
  khyātiṃ śāntiś ca saṃbhūtiṃ smṛtiṃ prītiṃ kṣamāṃ tathā ||}.
 }}

  \maintext{śraddhā kāmaḥ suto jāto darpo lakṣmīsutaḥ smṛtaḥ |}%

  \maintext{dhṛtyās tu niyamaḥ putraḥ saṃtoṣas tuṣṭijaḥ smṛtaḥ }||\thinspace3:10\thinspace||%
\translation{Śraddhā's son is Kāma {\rm (}`Desire'{\rm )}. Darpa {\rm (}`Pride'{\rm )} is said to be Lakṣmī's son. Dhṛti's son is Niyama {\rm (}`Rule'{\rm )}. Saṃtoṣa {\rm (}`Satisfaction'{\rm )} is Tuṣṭi's son. \blankfootnote{3.10 Understand \textit{śraddhā} as a stem form noun for \textit{śraddhāyāḥ} {\rm (}gen./abl., cf. 3.11a{\rm )}.
  Alternatively, take \textit{śraddhā} and \textit{suto} as elements of a split compound, and understand
  \textit{śraddhāsuto jātaḥ kāmaḥ}.
 }}

  \maintext{puṣṭyā lābhaḥ suto jāto medhāputraḥ śrutas tathā |}%

  \maintext{kriyāyās tv abhavat putro daṇḍaḥ samaya eva ca }||\thinspace3:11\thinspace||%
\translation{To Puṣṭi was born a son [called] Lābha {\rm (}`Profit'{\rm )}. Medhā's son is Śruta {\rm (}`Sacred Knowledge'{\rm )}. Kriyā's sons are Daṇḍa {\rm (}`Punishment'{\rm )} and Samaya {\rm (}`Law'{\rm )}. \blankfootnote{3.11 I have emended \textit{abhayaḥ} to \textit{abhavat} in \textit{pāda} c, following the relevant line in the \KURMP\ cited above
  {\rm (}\textit{kriyāyāś cābhavat putro daṇḍaḥ samaya eva ca}{\rm )} and also \LINPU\ 1.5.37 quoted in the 
  apparatus to this verse, allotting only two sons to Kriyā. Thus I don't think
  that Kriyā is supposed to have a son called Abhaya {\rm (}`Freedom from danger'; \BHAGP\ 4.1.50ab 
  claims that Dayā had a son called Abhaya:
  \textit{śraddhāsūta śubhaṃ maitrī prasādam abhayaṃ dayā}{\rm )}.
  Nevertheless, in a number of sources Kriyā actually has three sons, 
  see, e.g., \VISNUP\ 1.7.26ab,
  where they are named as Daṇḍa, Naya and Vinaya:
  \textit{medhā śrutaṃ kriyā daṇḍaṃ nayaṃ vinayam eva ca}. 
  Perhaps read \textit{kriyāyās tu nayaḥ putro} in \textit{pāda} c? Compare \VAYUP\ 1.10.34cd
  {\rm (}\textit{kriyāyās tu nayaḥ prokto daṇḍaḥ samaya eva ca}{\rm )} 
  with \BRAHMANDAPUR\ 1.9.60ab {\rm (}\textit{kriyāyās tanayau proktau damaś ca śama eva ca}{\rm )}.
 }}

  \maintext{lajjāyā vinayaḥ putro buddhyā bodhaḥ sutaḥ smṛtaḥ |}%

  \maintext{lajjāyāḥ sudhiyaḥ putra apramādaś ca tāv ubhau }||\thinspace3:12\thinspace||%
\translation{Lajjā's son is Vinaya {\rm (}`Discipline'{\rm )}, Buddhi's son is Bodha {\rm (}`Intelligence'{\rm )}. Lajjā has two [more] sons: Sudhiya[/Sudhī] {\rm (}`Wise'{\rm )} and Apramāda {\rm (}`Cautiousness'{\rm )}. \blankfootnote{3.12 In a very similar passages in \KURMP\ 1.8.20 ff., Apramāda is Buddhi's son and 
  Lajjā has only one son, Vinaya. In the above verse {\rm (}\VSS\ 3.12{\rm )}, \textit{sudhiyaḥ} {\rm (}for \textit{sudhīḥ}{\rm )} may only be 
  qualifying \textit{apramāda}, thus Lajjā may have two sons: Vinaya and the wise Apramāda.
  Alternatively, \textit{pāda}s cd might be a extra line inserted accidentally.
 }}

  \maintext{kṣemaḥ śāntisuto vindyād vyavasāyo vapoḥ sutaḥ |}%

  \maintext{yaśaḥ kīrtisuto jñeyaḥ sukhaṃ siddher vyajāyata |}%

  \maintext{svāyambhuve 'ntare tv āsan kīrtitā dharmasūnavaḥ }||\thinspace3:13\thinspace||%
\translation{Kṣema {\rm (}`Peace'{\rm )} is to be known as Śānti's son, Vyavasāya {\rm (}`Resolution'{\rm )} is Vapus' son. Yaśas {\rm (}`Fame'{\rm )} is Kīrti's son, Sukha {\rm (}`Joy'{\rm )} was born to Siddhi. [This is how] the sons of Dharma in the [\textit{manvantara}] era of Svāyambhuva [Manu] were known. \blankfootnote{3.13 Note that \textit{sukhaṃ} in \textit{pāda} d is probably meant to be masculine {\rm (}\textit{sukhaḥ}{\rm )}, but e.g. in the 
  \KURMP\ passage quoted above it is also neuter. For the emendation in \textit{pāda} e, 
  see \MATSP\ 9.2cd: 
  \textit{yāmā nāma purā devā āsan svāyambhuvāntare},
  and \BHAGP\ 6.4.1: 
  \textit{devāsuranṛṇāṃ sargo nāgānāṃ mṛgapakṣiṇām |
  sāmāsikas tvayā prokto yas tu svāyambhuve 'ntare ||}.
 }}

  \maintext{vigatarāga uvāca |}%

  \maintext{mūrtidvayaṃ kathaṃ dharmaṃ kathayasva tapodhana |}%

  \maintext{kautūhalam atīvaṃ me kartaya jñānasaṃśayam }||\thinspace3:14\thinspace||%
\translation{Vigatarāga spoke: How come Dharma has two embodiments? Tell me, O great ascetic. I am extremely intrigued. Cut my doubts concerning [this] knowledge. \blankfootnote{3.14 Note \textit{dharma} as a neuter noun and the form \textit{atīvaṃ} for \textit{atīva} metri causa. 
  My emen\-dation from \textit{kīrtaya} {\rm (}`declare'{\rm )} to \textit{kartaya} {\rm (}`cut'{\rm )} was influenced by the combination
  of \textit{chindhi} and \textit{saṃśaya}, often with \textit{kautūhala}, elsewhere in the \VSS:
  3.2ab: \textit{kautūhalaṃ mamotpannaṃ saṃśayaṃ chindhi tattvataḥ}; 
  10.10cd: \textit{kautūhalaṃ mahaj jātaṃ chindhi saṃśayakārakam};
  15.2ab: \textit{etat kautūhalaṃ chindhi saṃśayaṃ parameśvara}. 
  The reading \textit{kīrtaya} may have been the result of the influence of \textit{kīrtitā} in 3.13b above 
  {\rm (}De Simini's observation{\rm )}.
 }}

  \maintext{anarthayajña uvāca |}%

  \maintext{śrutismṛtidvayor mūrtir dharmasya parikīrtitā |}%

  \maintext{dārāgnihotrasambandham ijyā śrautasya lakṣaṇam |}%

  \maintext{smārto varṇāśramācāro yamaiś ca niyamair yutaḥ }||\thinspace3:15\thinspace||%
\translation{Anarthayajña spoke: Dharma's embodiment is said to consist of Śruti and Smṛti. The characteristics of the Śrauta [tradition] are an association with a wife [i.e.\ marriage] and with the fire ritual, and sacrifice. The Smārta [tradition] [focuses on] the conduct {\rm (}\textit{ācāra}{\rm )} of the classes {\rm (}\textit{varṇa}{\rm )} and life-stages {\rm (}\textit{āśrama}{\rm )} which is connected to rules and regulations {\rm (}\textit{yama-niyama}{\rm )}. \blankfootnote{3.15 The reading \textit{°dvayī} in \msNc\ in \textit{pāda} a is attractive, but as Judit 
  Törzsök has pointed out to me, it is more likely that
  the slightly less convincing but widespread variant \textit{°dvayor} is original.
 
  As for Dharma being based on \textit{śruti} and \textit{smṛti}, see, e.g., \Manu\ 2.10:
  \textit{śrutis tu vedo vijñeyo dharmaśāstraṃ tu vai smṛtiḥ |
  te sarvārtheṣv amīmāṃsye tābhyāṃ dharmo hi nirbabhau ||}.
  In Olivelle's translation {\rm (}\mycitep{OlivelleManu}{94}{\rm )}:
  `\thinspace ``Scripture'' should be recognized as ``Veda,'' and ``tradition''
  as ``Law Treatise.'' These two should never be called into question in any matter,
  for it is from them that the Law shines forth.'
 
  
  There may be a hiatus filler in \textit{pāda}s cd: \textit{°sambandha-m-ijyā} for \textit{°sambandha ijyā}.
 
  To state that the Smārta tradition is connected to \textit{yama}s and \textit{niyama}s and the \textit{āśrama}s and
  then to discuss these at length {\rm (}principally in chapters 3--8 and 11{\rm )} can be seen 
  as a clear self-identification with the Smārta tradition.
 }}

  \subchptr{yamaniyamabhedaḥ}%

  \trsubchptr{Yama and Niyama rules}%

  \maintext{yamaś ca niyamaś caiva dvayor bhedam ataḥ śṛṇu |}%

  \maintext{ahiṃsā satyam asteyam ānṛśaṃsyaṃ damo ghṛṇā |}%

  \maintext{dhanyāpramādo mādhuryam ārjavaṃ ca yamā daśa }||\thinspace3:16\thinspace||%
\translation{Now hear the classification of both the \textit{yama} and \textit{niyama} rules. Non-violence, truthfulness, not stealing, absence of hostility, self-restraint, taboos, virtue, carefulness, charm, honesty: these are the ten \textit{yama}s. \blankfootnote{3.16 \textit{Pāda} a should be understood as \textit{yamaniyamayoś caiva}, but the author of this line
  may have tried to avoid the metrical fault of having two short syllables in the second and third positions.
 Note that this is the beginning of a long section in our text
  that describes the \textit{yama-niyama} rules, reaching up to the end of chapter eight. 
  The title given in the colophon of the next chapter, chapter four, namely \textit{yamavibhāga},
  would fit this locus better than the beginning of that chapter, which 
  commences with a discussion on the second of the \textit{yama}s, \textit{satya}.
 Note how all witnesses read \textit{mādhūrya} in \textit{pāda} e instead of \textit{mādhurya}. The former may have been
  acceptable originally in this text. \textit{Pāda} e is a \textit{ma-vipulā}.
 }}

  \maintext{ekaikasya punaḥ pañcabhedam āhur manīṣiṇaḥ |}%

  \maintext{ahiṃsādi pravakṣyāmi śṛṇuṣvāvahito dvija }||\thinspace3:17\thinspace||%
\translation{The wise say that there are five subclasses to each. I shall teach you about non-violence and the other [\textit{yama}-rules]. Listen carefully, O twice-born. \blankfootnote{3.17 In \textit{pāda} a, \textit{pañca} and \textit{bheda} may be typeset as two separate words since
  the use of the singular after numbers is one of the hallmarks of the text {\rm (}see \verify{\rm )}.
 }}

  \subchptr{yameṣv ahiṃsā {\rm {\rm (}1{\rm )}}}%

  \trsubchptr{The first Yama-rule: Non-violence}%

  \subsubchptr{pañcavidhā hiṃsā}%

  \trsubsubchptr{Five types of violence}%

  \maintext{trāsanaṃ tāḍanaṃ bandho māraṇaṃ vṛttināśanam |}%

  \maintext{hiṃsāṃ pañcavidhām āhur munayas tattvadarśinaḥ }||\thinspace3:18\thinspace||%
\translation{Frightening and beating [other people], tying [someone] up, killing and the destruction of [other people's] livelihood: violence is said by the wise who see the truth to be of [these] five types. }

  \maintext{kāṣṭhaloṣṭakaśādyais tu tāḍayantīha nirdayāḥ |}%

  \maintext{tatprahāravibhinnāṅgo mṛtavadhyam avāpnuyāt }||\thinspace3:19\thinspace||%
\translation{Cruel people beat [other people] with sticks, clods of earth [understand: they stone them], with whips and other [objects] in the everyday world. Their bodies broken by the same blows, they receive the capital punishment. \blankfootnote{3.19 Note the use of the singular in \textit{pāda}s cd referring back to the agents of the previous sentence.
  Most probably, °\textit{vadhyam} is to be understand as °\textit{vadham} and the form 
  \textit{vadhyam} serves only to avoid two \textit{laghu} syllables in \textit{pāda} d.
 }}

  \maintext{baddhvā pādau bhujoraś ca śirorukkaṇṭhapāśitāḥ |}%

  \maintext{anāhatā mriyanty evaṃ vadho bandhanajaḥ smṛtaḥ }||\thinspace3:20\thinspace||%
\translation{[Others,] tie up [people] at their feet and their arms and chests. [These,] hung by their hair and neck, die in this way without being wounded. This is the capital punishment for tying up [other people]. \blankfootnote{3.20 Understand \textit{bhujoraś ca} in \textit{pāda} a as \textit{bhuje, urasi ca}, in this case with an instance of double sandhi,
  and in stem form: \textit{bhuje urasi ca} $\rightarrow$\ \textit{bhuja urasi ca} 
  $\rightarrow$\ \textit{bhujorasi ca} $\rightarrow$\ \textit{bhujoraś ca}.
  Alternatively, understand it as a compound {\rm (}\textit{bhujorasi}{\rm )}. 
  In \textit{pāda} b, my emendation is only one of the possible interpretations. We might accept
  \textit{śiroru}° as consisting of \textit{śira} + \textit{ūru} {\rm (}`head and thigh'{\rm )}, or emend it 
  to \textit{śiroraḥ}° for \textit{śira} + \textit{uraḥ} {\rm (}`head and chest'{\rm )}. Also note my conjecture
  in \textit{pāda} d, without which this \textit{pāda} is difficult to interpret.
 }}

  \maintext{śatrucaurabhayair ghoraiḥ siṃhavyāghragajoragaiḥ |}%

  \maintext{trāsanād vadham āpnoti anyair vāpi suduḥsahaiḥ }||\thinspace3:21\thinspace||%
\translation{He who frightens [other people] with the terrible danger of enemies and thieves, with lions, tigers, elephants or snakes, or by other horrors, will be executed. }

  \maintext{yasya yasya hared vittaṃ tasya tasya vadhaḥ smṛtaḥ |}%

  \maintext{vṛttijīvābhibhūtānāṃ taddvārā nihataḥ smṛtaḥ }||\thinspace3:22\thinspace||%
\translation{He who robs somebody's money is to be punished by the same person. He is [to be] struck down by those whose livelihood got damaged by him. \blankfootnote{3.22 Understand \textit{vadhaḥ} in \textit{pāda} b as \textit{vadhyaḥ} metri causa.
 My translation of the second line of this verse reflects a conjecture {\rm (}\textit{taddvārā}{\rm )}
  understood as connected to both \textit{pāda} c and \textit{nihataḥ} in \textit{pāda} d.
 }}

  \maintext{viṣavahniśaraśastrair māyāyogabalena vā |}%

  \maintext{hiṃsakāny āhu viprendra munayas tattvadarśinaḥ }||\thinspace3:23\thinspace||%
\translation{[Those who kill other people] with poison, fire, arrows, swords, or by the force of magic or yoga are called murderers by the sages who see the truth, O great Brahmin. \blankfootnote{3.23 \textit{Pāda} a is unmetrical.
  Note how elliptical this verse is and that \textit{hiṃsakāni} is neuter although it refers to 
  people, perhaps implying \textit{bhūtāni}. Alternatively, take \textit{y} in \textit{hiṃsakāny} as a 
  rather unusual sandhi-bridge {\rm (}\textit{hiṃsakān-y-āhu}{\rm )}, or simply delete this \textit{y}. 
  Note also that \textit{āhu} stands for \textit{āhur} metri causa.
 }}

  \subsubchptr{ahiṃsāpraśaṃsā}%

  \trsubsubchptr{Praise of non-violence}%

  \maintext{ahiṃsā paramaṃ dharmaṃ yas tyajet sa durātmavān |}%

  \maintext{kleśāyāsavinirmuktaṃ sarvadharmaphalapradam }||\thinspace3:24\thinspace||%
\translation{Non-violence is the highest Dharma. He who abandons it is a wicked person. It is free of pain and trouble, it yields the fruits of all [other] Dharmic teachings [in itself]. \blankfootnote{3.24 Note \textit{dharma} as a neuter noun in \textit{pāda} a and that \textit{°vinirmuktaṃ} and
  \textit{°pradam} are neuter accordingly.
 }}

  \maintext{nātaḥ parataro mūrkho nātaḥ parataraṃ tamaḥ |}%

  \maintext{nātaḥ parataraṃ duḥkhaṃ nātaḥ parataro 'yaśaḥ }||\thinspace3:25\thinspace||%
\translation{There isn't a bigger fool than he [who abandons it]. There is no bigger mental darkness [than the abandonment of non-violence]. There is no greater suffering or greater infamy. \blankfootnote{3.25 Note that \textit{parataro} is masculine in \textit{pāda} d, picking up a neuter \textit{'yaśaḥ}.
  This phenomenon is probably the result of \textit{'yaśaḥ} resembling a masculine noun ending in \textit{-aḥ}
  and also of the metrical problem with a grammatically correct \textit{nātaḥ parataram ayaśaḥ}.
 }}

  \maintext{nātaḥ parataraṃ pāpaṃ nātaḥ parataraṃ viṣam |}%

  \maintext{nātaḥ paratarāvidyā nātaḥ paraṃ tapodhana }||\thinspace3:26\thinspace||%
\translation{There is no greater sin or a more effective poison. There is no greater ignorance, there is nothing worse, O great ascetic. \blankfootnote{3.26 \textit{Pāda} d {\rm (}\textit{nātaḥ paraṃ tapodhana}{\rm )} is slightly suspect. 
  The vocative \textit{tapodhana} usually refers to Anarthayajña in these
  passages, and not to Vigatarāga, as here. The text may have read \textit{nātaḥ paratamo 'dhanaḥ} 
  {\rm (}`There is no bigger loss of wealth'{\rm )} or possibly something starting with
  \textit{nātaḥ paraṃ tapo ...} {\rm (}`There is no greater\dots\ of austerity'{\rm )}.
 }}

  \maintext{yo hinasti na bhūtāni udbhijjādi caturvidham |}%

  \maintext{sa bhavet puruṣaḥ śreṣṭhaḥ sarvabhūtadayānvitaḥ }||\thinspace3:27\thinspace||%
\translation{He who does not harm the four types of living beings beginning with plants is the best person, having compassion for all creatures. }

  \maintext{sarvabhūtadayāṃ nityaṃ yaḥ karoti sa paṇḍitaḥ |}%

  \maintext{sa yajvā sa tapasvī ca sa dātā sa dṛḍhavrataḥ }||\thinspace3:28\thinspace||%
\translation{He who always has compassion for all creatures is the [true] Pandit. He is the [true] sacrificer, the [true] ascetic, he is the donor, the one with a firm vow. }

  \maintext{ahiṃsā paramaṃ tīrtham ahiṃsā paramaṃ tapaḥ |}%

  \maintext{ahiṃsā paramaṃ dānam ahiṃsā paramaṃ sukham }||\thinspace3:29\thinspace||%
\translation{Non-violence is the supreme pilgrimage place. Non-violence is the highest austerity. Non-violence is the highest donation. Non-violence is the highest joy. }

  \maintext{ahiṃsā paramo yajñaḥ ahiṃsā paramaṃ vratam |}%

  \maintext{ahiṃsā paramaṃ jñānam ahiṃsā paramā kriyā }||\thinspace3:30\thinspace||%
\translation{Non-violence is the supreme sacrifice. Non-violence is the supreme religious observance. Non-violence is supreme knowledge. Non-violence is the supreme ritual. }

  \maintext{ahiṃsā paramaṃ śaucam ahiṃsā paramo damaḥ |}%

  \maintext{ahiṃsā paramo lābhaḥ ahiṃsā paramaṃ yaśaḥ }||\thinspace3:31\thinspace||%
\translation{Non-violence is the highest purity. Non-violence is the highest self-restraint. Non-violence is the highest profit. Non-violence is the greatest fame. }

  \maintext{ahiṃsā paramo dharmaḥ ahiṃsā paramā gatiḥ |}%

  \maintext{ahiṃsā paramaṃ brahma ahiṃsā paramaḥ śivaḥ }||\thinspace3:32\thinspace||%
\translation{Non-violence is the supreme Dharma. Non-violence is the supreme path. Non-violence is the supreme Brahman. Non-violence is supreme Śiva. }

  \subsubchptr{māṃsāhāraḥ}%

  \trsubsubchptr{On meet-consumption}%

  \maintext{māṃsāśanān nivarteta manasāpi na kāṅkṣayet |}%

  \maintext{sa mahat phalam āpnoti yas tu māṃsaṃ vivarjayet }||\thinspace3:33\thinspace||%
\translation{One should refrain from meat-consumption. One should not even desire it mentally. He who abandons meat will receive a great reward. }

  \maintext{svamāṃsaṃ paramāṃsena yo vardhayitum icchati |}%

  \maintext{anabhyarcya pitṝn devān na tato 'nyo 'sti pāpakṛt }||\thinspace3:34\thinspace||%
\translation{He who wishes to nourish his own flesh with the flesh of other [beings], outside of worshipping the ancestors and the gods, is the biggest sinner of all. \blankfootnote{3.34 See \UUMS\ chapter two for a similar section on meat-consumption.
 }}

  \maintext{madhuparke ca yajñe ca pitṛdaivatakarmaṇi |}%

  \maintext{atraiva paśavo hiṃsyā nānyatra manur abravīt }||\thinspace3:35\thinspace||%
\translation{During the \textit{madhuparka} offering and during a sacrifice, during rituals for the ancestors and the gods: only in these cases are animals to be slaughtered and not in any other case. [This is what] Manu taught. }

  \maintext{krītvā svayaṃ vāpy utpādya paropahṛtam eva vā |}%

  \maintext{devān pitṝṃś cārcayitvā khādan māṃsaṃ na doṣabhāk }||\thinspace3:36\thinspace||%
\translation{Should he buy it or procure it himself or should it be offered by others, if he eats meat, he will not sin if he first worships the gods and the ancestors. }

  \maintext{vedayajñatapastīrthadānaśīlakriyāvrataiḥ |}%

  \maintext{māṃsāhāranivṛttānāṃ ṣoḍaśāṃśaṃ na pūryate }||\thinspace3:37\thinspace||%
\translation{[People who know] the Vedas and [perform] sacrifices and austerities and [visit] sacred places, donate, [are of] good conduct, [perform] rituals and [keep] religious vows [but eat meat] will not [be able to] enjoy even a tiny portion of [such rewards that] [those] people [receive] who have given up meat. \blankfootnote{3.37 See a similarly phrased comparison in \Manu\ 2.86:
   %
  \textit{ye pākayajñās catvāro vidhiyajñasamanvitāḥ | 
  sarve te japayajñasya kalāṃ nārhanti ṣoḍaśīm ||}.
 }}

  \maintext{mṛgāḥ parṇatṛṇāhārād ajameṣagavādibhiḥ |}%

  \maintext{sukhino balavantaś ca vicaranti mahītale }||\thinspace3:38\thinspace||%
\translation{The deer and the goats, the sheep, the cows and other [animals] wander in the world happily and in great strength [just] from eating leaves and grass. }

  \maintext{vānarāḥ phala{-}m{-}āhārā rākṣasā rudhirapriyāḥ |}%

  \maintext{nihatā rākṣasāḥ sarve vānaraiḥ phalabhojibhiḥ }||\thinspace3:39\thinspace||%
\translation{Monkeys eat fruits, Rākṣasas prefer blood. The fruit-eating monkeys defeated all the Rākṣasas. \blankfootnote{3.39 Understand \textit{phalam āhārā} as \textit{phalāhārā} {\rm (}\textit{-m-} is a sandhi-bridge{\rm )}.
 This verse clearly refers to the story of the \textit{Rāmāyaṇa}.
 }}

  \maintext{tasmān māṃsaṃ na hīheta balakāmena bho dvija |}%

  \maintext{balena ca guṇākarṣāt parato bhayabhīruṇā }||\thinspace3:40\thinspace||%
\translation{Therefore one should not crave meat in the hope of gaining strength, O Brahmin, in order to be able to draw a bow with force, or out of fear of the danger coming from the enemy. \blankfootnote{3.40 \textit{guṇākāśāt} in pāda c is difficult to interpret and 
  \textit{guṇākarṣāt} is a conjecture by Judit Törzsök which fits the context well,
  although the polysemy of \textit{guṇa} may allow for other solutions.
   %
  Verses 3.40--42 may be echoing \BRAHMANDAPUR\ 216.64--66:
   %
  \textit{ māṃsān miṣṭataraṃ nāsti bhakṣyabhojyādikeṣu ca | 
  tasmān māṃsaṃ na bhuñjīta nāsti miṣṭaiḥ sukhodayaḥ ||  
  gosahasraṃ tu yo dadyād yas tu māṃsaṃ na bhakṣayet | 
  samāv etau purā prāha brahmā vedavidāṃ varaḥ || 
  sarvatīrtheṣu yat puṇyaṃ sarvayajñeṣu yat phalam | 
  amāṃsabhakṣaṇe viprās tac ca tac ca ca tatsamam ||}.
 }}

  \maintext{ahiṃsakasamo nāsti dānayajñasamīhayā |}%

  \maintext{iha loke yaśaḥ kīrtiḥ paratra ca parā gatiḥ }||\thinspace3:41\thinspace||%
\translation{By wishing to make donations and perform sacrifices no one will become comparable to someone who refrains from violence. [He will have] fame and glory in this world and the supreme path in the other. \blankfootnote{3.41 \textit{Pāda}s ab are reminescent of \SDHS\ 11.92:  %
  \textit{ahiṃsaikā paro dharmaḥ śaktānāṃ parikīrtitam | 
  aśaktānām ayaṃ dharmo dānayajñādipūrvakaḥ ||}. 
  On this verse see also \mycitep{SaivaUtopia2021}{15--16}.
 
  Note the variant \textit{°dharma°} in both \msCc\ and \Ed\ in \textit{pāda} b.
 }}

  \maintext{trailokyaṃ maṇiratnapūrṇam akhilaṃ dattvottame brāhmaṇe}%

 \nonanustubhindent \maintext{koṭīyajñasahasrapadmam ayutaṃ dattvā mahīṃ dakṣiṇām |}%

  \maintext{tīrthānāṃ ca sahasrakoṭiniyutaṃ snātvā sakṛn mānavaḥ}%

 \nonanustubhindent \maintext{etatpuṇyaphalam ahiṃsakajanaḥ prāpnoti niḥsaṃśayaḥ }||\thinspace3:42\thinspace||%
\translation{A person who refrains from violence will gain, no doubt about it, the [same] meritorious rewards that others would get by donating the three worlds filled with jewels and gems in their entirety to an excellent Brahmin, by [performing] a thousand [times] ten trillion {\rm (}\textit{padma}{\rm )} [times] ten thousand {\rm (}\textit{ayuta}{\rm )} \textit{koṭīyajña} sacrifices, by donating the earth [to a priest] as sacrificial fee, and by bathing [at] a thousand times ten million times a million {\rm (}\textit{niyuta}{\rm )} sacred places at once. \blankfootnote{3.42 Metre: \textit{śārdūlavikrīḍita}. 
  On \textit{padma} meaning `ten trillion', and on other words for numbers, see 1.32--35. 
   %
 
  \textit{koṭīyajña} in \textit{pāda} d may refer to a special kind of sacrifice, 
  mostly known as \textit{koṭihoma} in the Purāṇas and in inscriptions 
  {\rm (}see, e.g., \mycitep{Fleming2010}{and 2013}\nocite{Fleming2013}{\rm )}
  It probably involves a hundred fire-pits 
  and a hundred times one thousand Brahmins {\rm (}hence the name `the ten-million sacrifice'{\rm )}.
  See, e.g., \BHAVP\ \textit{uttaraparvan} 4.142.54--58:
   %
  \textit{śatānano daśamukho dvimukhaikamukhas tathā |
  caturvidho mahārāja koṭihomo vidhīyate || 
  kāryasya gurutāṃ jñātvā naiva kuryād aparvaṇi |
  yathā saṃkṣepataḥ kāryaḥ koṭihomas tathā śṛṇu ||
  kṛtvā kuṇḍaśataṃ divyaṃ yathoktaṃ hastasaṃmitam |
  ekaikasmiṃs tataḥ kuṇḍe śataṃ viprān niyojayet ||
  sadyaḥ pakṣe tu viprāṇāṃ sahasraṃ parikīrtitam |
  ekasthānapraṇīte 'gnau sarvataḥ paribhāvite || 
  homaṃ kuryur dvijāḥ sarve kuṇḍe kuṇḍe yathoditam |
  yathā kuṇḍabahutve 'pi rājasūye mahākratau || }
   %
 
  Note that the second syllable of \textit{phalam} in \textit{pāda} d is treated as long: this
  happens often at word-boundaries in this text; and 
  note how \msNc\ aims to restore the metre by inserting \textit{tv} after its \textit{phalaṃ}.
 }}
\center{\maintext{\dbldanda\thinspace iti vṛṣasārasaṃgrahe ahiṃsāpraśaṃsā nāmādhyāyas{ }tṛtīyaḥ\thinspace\dbldanda}}
\translation{Here ends the third chapter in the \textit{Vṛṣasārasaṃgraha} called the Praise of Non-violence.}

  \chptr{caturtho 'dhyāyaḥ}
\fancyhead[CE]{{\footnotesize\textit{Translation of chapter 4}}}%

  \trchptr{ Chapter Four }%

  \subchptr{yameṣu satyam {\rm {\rm (}2{\rm )}}}%

  \trsubchptr{The second Yama-rule: Truthfulness}%

  \maintext{anarthayajña uvāca |}%

  \maintext{sadbhāvaḥ satyam ity āhur dṛṣṭapratyayam eva vā |}%

  \maintext{yathābhūtārthakathanaṃ tat satyakathanaṃ smṛtam }||\thinspace4:1\thinspace||%
\translation{Anarthayajña spoke: The state of being real {\rm (}\textit{sad-bhāva}{\rm )} is called truth {\rm (}\textit{sat-ya}{\rm )}. Alternatively, it is also a certainty {\rm (}\textit{pratyaya}{\rm )} that originates in perception {\rm (}\textit{dṛṣṭa}{\rm )}. Relating things in a way that corresponds to reality is called `speaking the truth.' \blankfootnote{4.1 Although the rather similar line in the \SDHS\ {\rm (}11.105cd: \textit{yathābhūtārthakathanam ity etat satyalakṣaṇam}{\rm )} 
  makes it tempting to emend \textit{satyakathanaṃ} to \textit{satyalakṣaṇaṃ} in \textit{pāda} d, 
  I rather take this verse to introduce two views on truth: one philosophical, and one ordinary that
  relates to the moral question of truthfulness.
 }}

  \maintext{ākrośatāḍanādīni yaḥ saheta suduḥsaham |}%

  \maintext{kṣamate yo jitātmā tu sa ca satyam udāhṛtam }||\thinspace4:2\thinspace||%
\translation{He who endures severe abuse and beating etc. but keeps quiet, his self being conquered, is said to be [an example of] truth[fulness]. \blankfootnote{4.2 \textit{suduḥsaham} {\rm (}singular{\rm )} in \textit{pāda} b picks up \textit{°ādīni} {\rm (}plural{\rm )} in \textit{pāda} a.
  The \textit{-m} in \textit{satyam} may be a sandhi-bridge and the phrase may refer to a
  masculine subject thus: \textit{sa ca satya-m-udāhṛtaḥ}.
 }}

  \maintext{vadhārtham udyataḥ śastraṃ yadi pṛccheta karhicit |}%

  \maintext{na tatra satyaṃ vaktavyam anṛtaṃ satyam ucyate }||\thinspace4:3\thinspace||%
\translation{If one is being interrogated at any time with a sword lifted to strike him down, in this case the truth is not to be spoken, and a lie is can be called truth. \blankfootnote{4.3 Understand \textit{udyataḥ} {\rm (}nom.{\rm )} in an active sense {\rm (}`holding/lifting'{\rm )}.
 }}

  \maintext{vadhārhaḥ puruṣaḥ kaścid vrajet pathi bhayāturaḥ |}%

  \maintext{pṛcchato 'pi na vaktavyaṃ satyaṃ tad vāpi ucyate }||\thinspace4:4\thinspace||%
\translation{A person who is walking on the road and is afraid of being killed \verify should not reply [to people who are potentially dangerous] even if they ask him. This is also called truth[fulness]. }

  \maintext{na narmayuktam anṛtaṃ hinasti}%

 \nonanustubhindent \maintext{na strīṣu rājan na vivāhakāle |}%

  \maintext{prāṇātyaye sarvadhanāpahāre}%

 \nonanustubhindent \maintext{pañcānṛtaṃ satyam udāharanti }||\thinspace4:5\thinspace||%
\translation{A lie does not hurt when it is connected with joking, with women, O king, at the time of marriage, at the departure from life and when one's entire wealth is about to be taken away. They call these five kinds of lies truths. \blankfootnote{4.5 This \textit{upajāti} verse appears in countless sources, beginning with the \MBH\ {\rm (}see the apparatus{\rm )}. All versions 
  contain a vocative addressing a king, which is out of context in the \VSS, the addressee being Vigatarāga,
  i.e. Viṣṇu diguised as a Brahmin. The redactors did not notice or did not care about this
  small inconsistency. Note the metrical licence that allows the last syllable of °\textit{yuktam} to count as long.
  The same reading with \textit{anṛtaṃ} can be found in the apparatus in the \MBH\ critical edition.
 }}

  \maintext{devamānuṣatiryeṣu satyaṃ dharmaḥ paro yataḥ |}%

  \maintext{satyaṃ śreṣṭhaṃ variṣṭhaṃ ca satyaṃ dharmaḥ sanātanaḥ }||\thinspace4:6\thinspace||%
\translation{Since truth is the supreme Dharma in [the world of] gods, humans and animals, truth is the best, the most preferable. Truth is the eternal Dharma. }

  \maintext{satyaṃ sāgaram avyaktaṃ satyam akṣayabhogadam |}%

  \maintext{satyaṃ potaḥ paratrārthaṃ satyaṃ panthāna vistaram }||\thinspace4:7\thinspace||%
\translation{Truth is an unmanifest ocean. Truth yields imperishable pleasures. Truth is the ship that carries you to the other world. Truth is the wide path. \blankfootnote{4.7 \textit{Pāda} d is slightly problematic because it is difficult to ascertain if some of the
  MSS actually read \textit{panthāna} or \textit{pasthāna} {\rm (}or \textit{yasthāna}{\rm )}. I suspect that \textit{panthāna} 
  is a stem form noun formed {\rm (}metri causa{\rm )} to stand for an irregular nominative of \textit{pathin}.
 }}

  \maintext{satyam iṣṭagatiḥ proktaṃ satyaṃ yajñam anuttamam |}%

  \maintext{satyaṃ tīrthaṃ paraṃ tīrthaṃ satyaṃ dānam anantakam }||\thinspace4:8\thinspace||%
\translation{Truth is said to be the desired path. Truth is the supreme sacrifice. Truth is a pilgrimage place, a supreme pilgrimage place. Truth is an endless donation. \blankfootnote{4.8 The repetition of \textit{tīrthaṃ} in \textit{pāda} c is sightly suspect. Cf., e.g., \MATSP\ 22.79ab:
  \textit{satyaṃ tīrthaṃ dayā tīrthaṃ tīrtham indriyanigrahaḥ}.
 }}

  \maintext{satyaṃ śīlaṃ tapo jñānaṃ satyaṃ śaucaṃ damaḥ śamaḥ |}%

  \maintext{satyaṃ sopānam ūrdhvasya satyaṃ kīrtir yaśaḥ sukham }||\thinspace4:9\thinspace||%
\translation{Truth is morality, austerity, knowledge. Truth is purity, self-control and tranquillity. Truth is the ladder upwards. Truth is fame and glory and happiness. \blankfootnote{4.9 Looking at the similar line in the \VARP\ {\rm (}193.36cd, see the apparatus{\rm )}, one 
  wonders if the slightly odd \textit{ūrdhvasya} in \textit{pāda} c is not a corrupt form of 
  \textit{svargasya}.
 }}

  \maintext{aśvamedhasahasraṃ ca satyaṃ ca tulayā dhṛtam |}%

  \maintext{aśvamedhasahasrād dhi satyam eva viśiṣyate }||\thinspace4:10\thinspace||%
\translation{[When] a thousand Aśvamedha sacrifices and truth are measured on a pair of scales, truth indeed surpasses a thousand Aśvamedha sacrifices. }

  \maintext{satyena tapate sūryaḥ satyena pṛthivī sthitā |}%

  \maintext{satyena vāyavo vānti satye toyaṃ ca śītalam }||\thinspace4:11\thinspace||%
\translation{The Sun shines because of truth. The Earth stays in place by truth. The winds blow because of truth. Water is cooling through truth. \blankfootnote{4.11 Here and several times below, \textit{satye} is probably to be taken as standing for \textit{satyena}.
 }}

  \maintext{tiṣṭhanti sāgarāḥ satye samayena priyavrataḥ |}%

  \maintext{satye tiṣṭhati govindo balibandhanakāraṇāt }||\thinspace4:12\thinspace||%
\translation{The oceans exist by the truthful encounter with Priyavrata. Govinda abides in truth because He [as Vāmana] stopped [Mahā]Bali [in spite of the fact that this was achieved by a trick]. \blankfootnote{4.12 \textit{Pāda} b, \textit{samayena priyavrataḥ}, probably stand for \textit{samayena priyavratasya} although
  it is unclear to me what exactly \textit{samaya} refers to here.
   %
 
  For the story of Priyavrata, Manu's son, in which he wanted to turn nights into days by 
  circling aroung Mount Meru in a chariot, and by this produced the seven oceans,
  see, e.g., \BHAGP\ 5.1.30--31:  
  \textit{yāvad avabhāsayati suragirim anuparikrāman bhagavān ādityo
  vasudhātalam ardhenaiva pratapaty ardhenāvacchādayati, tadā hi [priyavrataḥ]
  bhagavadupāsanopacitātipuruṣaprabhāvas tad anabhinandan samajavena
  rathena jyotirmayena rajanīm api dinaṃ kariṣyāmīti saptakṛtvas 
  taraṇim anuparyakrāmad dvitīya iva pataṅgaḥ |
  ye vā u ha tadrathacaraṇanemikṛtaparikhātās te sapta sindhava āsan
  yata eva kṛtāḥ sapta bhuvo dvīpāḥ |}.
  
  %
  
  \textit{Pāda}s cd: for a somewhat similar reference to the story of Mahābali, see, e.g., \VAMP\ 65.66:
  \textit{evaṃ purā cakradhareṇa viṣṇunā baddho balir vāmanarūpadhāriṇā |
  śakrapriyārthaṃ surakāryasiddhaye hitāya viprarṣabhagodvijānām ||} 
 }}

  \maintext{agnir dahati satyena satyena śaśinā caraḥ |}%

  \maintext{satyena vindhyās tiṣṭhanti vardhamāno na vardhate }||\thinspace4:13\thinspace||%
\translation{Fire burns with truth. The Moon rises by truth. It is because of truth that the Vindhya mountain stands in place and that although is was growing it is not growing [anymore]. \blankfootnote{4.13 Since \textit{śaśi} {\rm (}instead of \textit{śaśin}{\rm )} is a possible stem in this text, 
  \textit{śaśir ācaraḥ} could be acceptable here in \textit{pāda} b {\rm (}see \msNa\msNb\msNc{\rm )}, perhaps standing for 
  \textit{śaśinaś caraṇam} or \textit{śaśiś carati}. My conjecture {\rm (}\textit{śaśinā caraḥ}{\rm )} 
  could stand for \textit{śaśinā/śaśinaś cāraḥ} metri causa. Other possibilities, suggested by
  colleagues, include \textit{śaśibhāskaraḥ}, \textit{śaśigocaraḥ} and \textit{śiśirāmbhasaḥ}.
  %
 
  \textit{Pāda}s cd refer to the story of Agastya and the Vindhya mountain:
  Vindhya became jealous of the Sun's revolving around Mount Meru and when the Sun 
  refused him the same favour, he decided to grow higher and obstruct the Sun's movement.
  As a solution to this situation, Agastya asked Vidhya to bend down to make 
  it easier for him to reach the south and to remain thus until he retured. 
  Vindhya agreed to do what Agastya asked him but Agastya never returned. 
  See \MBH\ 3.102.1--14 {\rm (}see in the word \textit{samaya} in verse 13 and compare it to \VSS\ 4.12b{\rm )}:
   %
  \textit{ yudhiṣṭhira uvāca | 
  kimarthaṃ sahasā vindhyaḥ pravṛddhaḥ krodhamūrchitaḥ | 
  etad icchāmy ahaṃ śrotuṃ vistareṇa mahāmune || 
  lomaśa uvāca | 
  adrirājaṃ mahāśailaṃ meruṃ kanakaparvatam | 
  udayāstamaye bhānuḥ pradakṣiṇam avartata || 
  taṃ tu dṛṣṭvā tathā vindhyaḥ śailaḥ sūryam athābravīt | 
  yathā hi merur bhavatā nityaśaḥ parigamyate || 
  pradakṣiṇaṃ ca kriyate mām evaṃ kuru bhāskara | 
  evam uktas tataḥ sūryaḥ śailendraṃ pratyabhāṣata || 
  nāham ātmecchayā śaila karomy enaṃ pradakṣiṇam | 
  eṣa mārgaḥ pradiṣṭo me yenedaṃ nirmitaṃ jagat || 
  evam uktas tataḥ krodhāt pravṛddhaḥ sahasācalaḥ | 
  sūryācandramasor mārgaṃ roddhum icchan paraṃtapa || 
  tato devāḥ sahitāḥ sarva eva; sendrāḥ samāgamya mahādrirājam | 
  nivārayām āsur upāyatas taṃ; na ca sma teṣāṃ vacanaṃ cakāra || 
  athābhijagmur munim āśramasthaṃ; tapasvinaṃ dharmabhṛtāṃ variṣṭham | 
  agastyam atyadbhutavīryadīptaṃ; taṃ cārtham ūcuḥ sahitāḥ surās te || 
  devā ūcuḥ | 
  sūryācandramasor mārgaṃ nakṣatrāṇāṃ gatiṃ tathā | 
  śailarājo vṛṇoty eṣa vindhyaḥ krodhavaśānugaḥ || 
  taṃ nivārayituṃ śakto nānyaḥ kaś cid dvijottama | 
  ṛte tvāṃ hi mahābhāga tasmād enaṃ nivāraya || 
  lomaśa uvāca | 
  tac chrutvā vacanaṃ vipraḥ surāṇāṃ śailam abhyagāt | 
  so 'bhigamyābravīd vindhyaṃ sadāraḥ samupasthitaḥ || 
  mārgam icchāmy ahaṃ dattaṃ bhavatā parvatottama | 
  dakṣiṇām abhigantāsmi diśaṃ kāryeṇa kena cit || 
  yāvadāgamanaṃ mahyaṃ tāvat tvaṃ pratipālaya | 
  nivṛtte mayi śailendra tato vardhasva kāmataḥ || 
  evaṃ sa samayaṃ kṛtvā vindhyenāmitrakarśana | 
  adyāpi dakṣiṇād deśād vāruṇir na nivartate || 
  etat te sarvam ākhyātaṃ yathā vindhyo na vardhate | 
  agastyasya prabhāvena yan māṃ tvaṃ paripṛcchasi ||}.
 }}

  \maintext{lokālokaḥ sthitaḥ satye meruḥ satye pratiṣṭhitaḥ |}%

  \maintext{vedās tiṣṭhanti satyeṣu dharmaḥ satye pratiṣṭhati }||\thinspace4:14\thinspace||%
\translation{The [mythical] Lokāloka mountains are located in truth. Mount Meru stands by truth. The Vedas abide in truth. Dharma is rooted in truth. }

  \maintext{satyaṃ gauḥ kṣarate kṣīraṃ satyaṃ kṣīre ghṛtaṃ sthitam |}%

  \maintext{satye jīvaḥ sthito dehe satyaṃ jīvaḥ sanātanaḥ }||\thinspace4:15\thinspace||%
\translation{The milk a cow yields is truth. Ghee in milk is present as truth. The soul dwells in the body in truth. The eternal soul is truth. \blankfootnote{4.15 \textit{satye} in \textit{pāda} c may stand for \textit{satyaṃ}: `The soul dwells in the body as truth.'
 }}

  \maintext{satyam ekena samprāpto dharmasādhananiścayaḥ |}%

  \maintext{rāmarāghavavīryeṇa satyam ekaṃ surakṣitam }||\thinspace4:16\thinspace||%
\translation{If truth is obtained by somebody {\rm (}\textit{ekena}{\rm )}, he/she will be one for whom Dharma is surely accomplished. By the heroism of Rāma Rāghava, the only truth was well-guarded. \blankfootnote{4.16 Or: `If truth alone {\rm (}\textit{ekena}{\rm )} is obtained, Dharma is surely accomplished.'
 }}

  \maintext{evaṃ satyavidhānasya kīrtitaṃ tava suvrata |}%

  \maintext{sarvalokahitārthāya kim anyac chrotum icchasi }||\thinspace4:17\thinspace||%
\translation{Thus have [I] taught the rules of truth to you, O virtuous one, to favour the whole world. What else do you wish to hear? }

  \subchptr{yameṣv asteyam {\rm {\rm (}3{\rm )}}}%

  \trsubchptr{The third Yama-rule: Refraining from stealing}%

  \maintext{vigatarāga uvāca |}%

  \maintext{na hi tṛptiṃ vijānāmi śrutvā dharmaṃ tavāpy aham |}%

  \maintext{upariṣṭād ato bhūyaḥ kathayasva tapodhana }||\thinspace4:18\thinspace||%
\translation{Vigatarāga spoke: I can't have enough of learning about [this teaching of] your[s on] Dharma. Teach me further than this, O great ascetic. \blankfootnote{4.18 It is not inconceivable that \textit{tava} is meant to carry the sense of the ablative,
  as Kenji Takahashi has suggested to me:
  `I can't have enough of learning about Dharma from you.' 
 }}

  \maintext{anarthayajña uvāca |}%

  \maintext{steyaṃ śṛṇv atha viprendra pañcadhā parikīrtitam |}%

  \maintext{adattādānam ādau tu utkocaṃ ca tataḥ param |}%

  \maintext{prasthavyājas tulāvyājaḥ prasahyasteya pañcamam }||\thinspace4:19\thinspace||%
\translation{Anarthayajña spoke: Now listen to [my teaching about] stealing, O great Brahmin, which is taught to be of five kinds. Firstly, [listen to] theft , then bribery, cheating with weights, cheating with scales, and the fifth kind, robbery. \blankfootnote{4.19 `Theft' {\rm (}\textit{adattādāna}{\rm )}: literally `taking what has not been given.'
 }}

  \maintext{dhṛṣṭaduṣṭaprabhāvena paradravyāpakarṣaṇam |}%

  \maintext{vāryamāṇāpi durbuddhir adattādānam ucyate }||\thinspace4:20\thinspace||%
\translation{When somebody's wealth is taken away by an impudent and wicked person is called theft. It is a foolish thought even if suppressed. \blankfootnote{4.20 My impression is that \textit{prabhāva} in \textit{pāda} a stands for \textit{bhāva}, \textit{duṣṭabhāva} {\rm (}`vicious'{\rm )}
  being a common expression.
 The implications of \textit{vāryamāṇo} in \textit{pāda} c are unclear to me, hence my emendation to \textit{vāryamāṇā}.
  My translation is thus tentative and still not satisfactory.
 }}

  \maintext{utkocaṃ śṛṇu viprendra dharmasaṃkarakārakam |}%

  \maintext{mūlyaṃ kāryavināśārtham utkocaḥ parigṛhyate |}%

  \maintext{tena cāsau vijānīyād dravyalobhabalāt kṛtam }||\thinspace4:21\thinspace||%
\translation{O great Brahmin, listen to bribery, which defiles Dharma. A sum of money taken in order to exempt somebody from a duty is a bribe. Therefore this [also] should be considered as such [i.e.\ as stealing because] it is committed out of greed for material goods. \blankfootnote{4.21 Note that \textit{mūlyaṃ} in \textit{pāda} c is a conjecture for \textit{mūla}. It is partly based on 
  a relevant passage in the \Mitaksara\ {\rm (}ad \YajnS\ 2.176cd{\rm )}:
  \textit{paṇyasya krītadravyasya yan mūlyaṃ dattam, bhṛtir vetanaṃ kṛtakarmaṇe dattam\dots\ 
  utkocena kāryapratibandhanirāsārtham adhikṛtebhyo dattam}\dots\ 
 Note \textit{asau} in \textit{pāda} e as an accusative form {\rm (}for \textit{amum} or \textit{adaḥ}{\rm )}. It is not unlikely that 
  \textit{tena} is a corruption from \textit{stena}, and the \textit{pāda} may have originally read 
  \textit{stenaṃ taṃ ca vijānīyād} {\rm (}`he should be known as a thief'{\rm )}, or similar {\rm (}cf. 4.22c below{\rm )}. 
  \msM\ {\rm (}f. 7r{\rm )} reads \textit{tena steya vijānīyād} here.
 }}

  \maintext{prasthavyāja-upāyena kuṭumbaṃ trātum icchati |}%

  \maintext{taṃ ca stenaṃ vijānīyāt paradravyāpahārakam }||\thinspace4:22\thinspace||%
\translation{[Even if] somebody wants to protect a family by the method of cheating with weights, that person should be considered a thief, because he takes away other people's wealth. }

  \maintext{tulāvyāja-upāyena parasvārthaṃ hared yadi |}%

  \maintext{cauralakṣaṇakāś cānye kūṭakāpaṭikā narāḥ }||\thinspace4:23\thinspace||%
\translation{[The case is similar] if somebody takes away somebody else's belongings by the method of cheating with scales. Other people, deceitful swindlers {\rm (}\textit{kūṭa-kāpaṭika}{\rm )} share the characteristics of thieves. \blankfootnote{4.23 A line may have dropped out after \textit{pāda} b, perhaps because a line 
  similar to 4.22cd caused an eyeskip. Alternatively, this line may simply be
  elliptical.
 }}

  \maintext{durbalārjavabāleṣu cchadmanā vā balena vā |}%

  \maintext{apahṛtya dhanaṃ mūḍhaḥ sa cauraś cora ucyate }||\thinspace4:24\thinspace||%
\translation{If someone, by deceit or by force, snatches away the wealth of weak and honest people and simpletons, that morally corrupt usurper is [simply] a thief. \blankfootnote{4.24 It is possible that \textit{pāda} d read differently, e.g. \textit{sa coraś cora ucyate}, meaning `that thief is [rightly] called a thief'.
 }}

  \maintext{nāsti steyasamaṃ pāpaṃ nāsty adharmaś ca tatsamaḥ |}%

  \maintext{nāsti stenasamākīrtir nāsti stenasamo 'nayaḥ }||\thinspace4:25\thinspace||%
\translation{There is no sin equal to stealing. There is no crime {\rm (}\textit{adharma}{\rm )} equal to it. There is no ill-fame comparable to that of being a thief. There is no bad-conduct comparable to being a thief. }

  \maintext{nāsti steyasamāvidyā nāsti stenasamaḥ khalaḥ |}%

  \maintext{nāsti stenasama ajño nāsti stenasamo 'lasaḥ }||\thinspace4:26\thinspace||%
\translation{There is no greater ignorance than stealing. There are no bigger rouges than thieves. There is nobody as ignorant as a thief. There is not a lazy person who is comparable to a thief. \blankfootnote{4.26 Note the peculiar sandhi in \textit{pāda} c {\rm (}\textit{°sama ajño}{\rm )}, which still leaves the \textit{pāda} unmetrical.
 }}

  \maintext{nāsti stenasamo dveṣyo nāsti stenasamo 'priyaḥ |}%

  \maintext{nāsti steyasamaṃ duḥkhaṃ nāsti steyasamo 'yaśaḥ }||\thinspace4:27\thinspace||%
\translation{There is nobody as detestable as a thief. There is nobody disliked as much as a thief. There is no greater suffering than stealing. There is no greater disgrace than theft. \blankfootnote{4.27 Note how \textit{stena} and \textit{steya} are used interchangeably {\rm (}or chaotically{\rm )}
  in the above passages in the MSS to denote both `thief' and 'theft/stealing'.
  The scribe of \msNc\ ends up writing \textit{stenya} in 4.27e.
 }}

  \maintext{pracchanno hriyate 'rtham anyapuruṣaḥ pratyakṣam anyo haret}%

 \nonanustubhindent \maintext{nikṣepād dhanahāriṇo 'nya{-}m{-}adhamo vyājena cānyo haret |}%

  \maintext{anye lekhyavikalpanāhṛtadhanā {\rm †}anyo hṛtād vai hṛtā{\rm †}}%

 \nonanustubhindent \maintext{anyaḥ krītadhano 'paro dhayahṛta ete jaghanyāḥ smṛtāḥ }||\thinspace4:28\thinspace||%
\translation{Some [thieves] take away [other people's] wealth in disguise, some in broad daylight. Other wicked people take money from deposits, and some people steal through fraud. Some gather wealth by forging documents, others steal from stolen money[?]. Some people's wealth is from purchased [children?] {\rm (}\textit{krīta}{\rm )}. Others take away others' inheritance[?]. These are considered the vilest. \blankfootnote{4.28 Metre \textit{śārdūlavikrīḍita}. It appears that \textit{hriyate} in \textit{pāda} a is to be taken as an active verb {\rm (}\textit{harate}{\rm )}.
  Note also how \msCb\ and \msNc\ read the same here against the other witnesses.
 Take \textit{°hariṇo} in \textit{pāda} b as singular and \textit{m} in \textit{'nya-m-adhamo} as a sandhi-bridge.
  Alternatively, read as plural: °\textit{hariṇo 'nya adhamo}\dots\ 
 The second half of \textit{pāda} c is difficult to reconstruct.
 The translation of \textit{pāda} d is mostly guesswork. Tentatively, I take \textit{krīta} as \textit{krītaka} {\rm (}`a purchased son', see
  \Manu\ 9.174{\rm )}. \textit{dhayahṛta} makes little sense to me. Florinda De Simini suggested that
  \textit{dhaya} might stand for \textit{daya}, which in turn may stand for \textit{dāya} {\rm (}`inheritance'{\rm )} metri causa.
  Lacking any better solution, I supplied these in my translation, marked with question marks.
  Note also the metrical licence that the last syllable of \textit{dhayahṛta} counts as long.
 }}

  \maintext{stenatulya na mūḍham asti puruṣo dharmārthahīno 'dhamaḥ}%

 \nonanustubhindent \maintext{yāvaj jīvati śaṅkayā narapateḥ saṃtrasyamāno raṭan |}%

  \maintext{prāptaḥśāsana tīvrasahyaviṣamaṃ prāpnoti karmeritaḥ}%

 \nonanustubhindent \maintext{kālena mriyate sa yāti nirayam ākrandamāno bhṛśam }||\thinspace4:29\thinspace||%
\translation{There isn't a bigger idiots than a thief, who is a wicked person without Dharma and Artha. As long as he lives, he trembles in fear of the king, wailing. Having received his punishment, he gets into severe and [in]tolerable difficulties, propelled by [his] karma. When his time comes, he dies and goes to hell, weeping vehemently. \blankfootnote{4.29 For some time I was wondering if one should accept \Ed's reading \textit{stenastulya na mūḍham asti} 
  as a metri causa version of \textit{stenatulyo na mūḍho 'sti}; see a similar case of a nominative ending
  inside of compound in \textit{pāda} c below. One major concern remained:
  the accepted reading would be of an edition that rarely emerges as 
  the sole transmitter of the best reading. Another possible solution could be 
  to emend to \textit{stenaṃtulya}\dots, meaning `There is no bigger foolishness than theft',
  but then the second part of \textit{pāda} a is difficult to connect. In the end,
  I decided to to go for the most widely attested reading {\rm (}\textit{stenatulya}{\rm )},
  which is unmetrical.
 
   % 
 Understand \textit{prāptaḥśāsana tīvrasahyaviṣamaṃ} in \textit{pāda} c as \textit{prāptaśāsanas tīvram asahyaṃ ca viṣamaṃ prāpnoti}.
  Alternatively, understand \textit{tīvrasahya°} as \textit{duḥsahya°} {\rm (}suggested by Törzsök{\rm )}.
 
   %
  The actual reading of \msCa, \textit{prāptaś} {\rm (}lost in the process of normalization and standing
  in contrast with that of all other MSS that read \textit{prāptaḥ}{\rm )} may suggest
  a doubling of the \textit{ś} of \textit{śāsana} metri causa {\rm (}suggestion by Törzsök{\rm )}.
  More likely is that a licence of having a nominative ending inside of a compound
  is applied here, as may have been the case above in \textit{pāda} a {\rm (}also remarked by Törzsök{\rm )}.
 }}

  \maintext{nītvā durgatikoṭikalpa nirayāt tiryatvam āyānti te}%

 \nonanustubhindent \maintext{tiryatve ca tathaivam ekaśatikaṃ prabhramya varṣārbudam |}%

  \maintext{mānuṣyaṃ tad avāpnuvanti vipule dāridryarogākulam}%

 \nonanustubhindent \maintext{tasmād durgatihetu karma sakalaṃ tyaktvā śivaṃ cāśrayet }||\thinspace4:30\thinspace||%
\translation{Having spent ten million \ae ons of suffering, they emerge from hell to the state of animal existence. Thus, they roam about in animal existence for a hundred and one times ten million years. Then they reach the status of human existence on earth which is full of poverty and disease. Then abandoning all one's karmas, the causes of suffering, one seeks refuge in Śiva. \blankfootnote{4.30 Note the stem form °\textit{kalpa} for °\textit{kalpaṃ} metri causa.
 In \textit{pāda} c, \textit{tathaivam}, or \textit{tathaikam}, and \textit{ekaśatikaṃ} are suspect.
 I understand \textit{vipule} as \textit{vipulāyāṃ}, \textit{vipulā} appearing in \Amara\ 2.1.7 as a synonym of
  \textit{dhātrī}, `earth' because it is difficult to interpret it otherwise.
  This is still problematic because both human and
  animal existence takes place on earth, thus, if \textit{tiryatva} {\rm (}i.e. \textit{tiryaktva}{\rm )} 
  indeed means `animal existence,' there is no contrast between \textit{pāda}s b and c as
  regards location. As for \textit{tiryaktva}, see, e.g., \Manu\ 12.40:
  \textit{devatvaṃ sāttvikā yānti manuṣyatvaṃ ca rājasāḥ |
  tiryaktvaṃ tāmasā nityam ity eṣā trividhā gatiḥ ||}
  It is not unlikely that the original form of \textit{dāridryarogākulam} was \textit{dāridryarogākule},
  picking up \textit{vipule}.
 Note the switch from plural to singular in \textit{pāda} d.
 }}

  \subchptr{yameṣv ānṛśaṃsyam {\rm {\rm (}4{\rm )}}}%

  \trsubchptr{The fourth Yama-rule: Absence of hostility}%

  \maintext{aṣṭamūrtiśivadveṣṭā pitur mātuś ca yo dviṣet |}%

  \maintext{gavāṃ vā atither dveṣṭā nṛśaṃsāḥ pañca eva te }||\thinspace4:31\thinspace||%
\translation{The one who is hostile towards the eight-formed Śiva, he who hurts his mother or father, he who is hostile towards cows or guests: these are the five types of cruel people. \blankfootnote{4.31 Note \textit{pitur} and \textit{mātur} used as accusative forms in \textit{pāda} b, or alternatively
  understand: `who are hateful towards their fathers and mothers'.
 }}

  \maintext{aṣṭamūrtiḥ śivaḥ sākṣāt pañcavyomasamanvitaḥ |}%

  \maintext{sūryaḥ somaś ca dīkṣaś ca dūṣakaḥ sa nṛśaṃsakaḥ }||\thinspace4:32\thinspace||%
\translation{Śiva in his manifest form {\rm (}\textit{sākṣāt}{\rm )} is of eight forms, with the five elements {\rm (}\textit{vyoman}{\rm )}, the Sun, the Moon, and the sacrificer. [He who] disgraces [any of these] is a hostile person. \blankfootnote{4.32 Törzsök has suggested emending \textit{sa nṛśaṃsakaḥ} in \textit{pāda} d to \textit{tannṛṃśakaḥ}. I don't think that it is
  inevitably necessary. I think that \textit{pāda}s a-c form a list that is meant to be in the genitive, understanding
  \dots\ \textit{ity eteṣāṃ dūṣakaḥ sa nṛśaṃsakaḥ} or similar. This is clumsy but in a way that is
  more than possible within the style of this text.
 
  I have not been able find any attestation of \textit{vyoman} meaning the five elements. Perhaps it is meant
  to mean \textit{vyomādi} {\rm (}`the atmosphere/sky and the other four elements'{\rm )}. 
  
  For Śiva of eight forms, see, e.g., \textit{Śakuntalā} 1.1:
   %
  \textit{yā sṛṣṭiḥ sraṣṭur ādyā [1] vahati vidhihutaṃ yā havir [2] yā ca hotrī [3] 
  ye dve kālaṃ vidhattaḥ [4,5] śruti-viṣaya-guṇā yā [6] sthitā vyāpya viśvam | 
  yām āhuḥ sarva-bīja-prakṛtir [7] iti yayā prāṇinaḥ prāṇavantaḥ [8] 
  pratyakṣābhiḥ prapannas tanubhir avatu vas tābhir aṣṭābhir īśaḥ ||}. 
  The eight \textit{mūrti}s, or rather, \textit{tanu}s, here are: 
  [1] \textit{jala} [2] \textit{agni} [3] \textit{yajamāna} [4,5] \textit{sūrya} + \textit{candra} [6] \textit{ākāśa} [7] \textit{bhūmi} [8] \textit{vāyu}.
 
  For a similar interpretation of \textit{aṣṭamūrti}, see, e.g., \textit{Īśānaśivagurudevapaddhati} 2.29.34 {\rm (}\textit{mantrapāda};
  note \textit{yajamāna} for our \textit{dīkṣa}{\rm )}:
  \textit{kṣmā-vahni-yajamānārka-jala-vāyv-indu-puṣkaraiḥ} |
  \textit{aṣṭābhir mūrtibhiḥ śambhor dvitīyāvaraṇaṃ smṛtam} ||
  {\rm (}For \textit{puṣkara} as `sky, atmosphere', see, e.g., \Amara\ 1.2.167:
  \textit{dyodivau dve striyām abhraṃ vyoma puṣkaram ambaram}.{\rm )}
 
  A closely related Aṣṭamūrti-hymn appears in \Nisvmukh\ 1.30--41 {\rm (}I owe thanks to Nirajan Kafle
  for drawing my attention to this{\rm )}; see \mycitep{KafleNisvasaBook}{62, 63, 116, 119}. 
  Kafle notes that this hymn is closely parallel to some passages in the \textit{Prayogamañjarī} {\rm (}1.19--26{\rm )},
  the \textit{Tantrasamuccaya} {\rm (}1.16--23{\rm )}, and the \textit{Īśānaśivagurudevapaddhati} {\rm (}\textit{kriyāpāda} 26.56--63{\rm )}. 
  See also \TAKI\ s.v. \textit{aṣṭamūrti}.
 }}

  \maintext{pitākāśasamo jñeyo janmotpattikaraḥ pitā |}%

  \maintext{pitṛdaivata{\rm †}m ādiś cam ānṛśaṃsa tamanvitaḥ{\rm †} }||\thinspace4:33\thinspace||%
\translation{The father is to be considered similar to the [element] sky, he is the cause of one's birth. One should not be hostile to the forefathers, gods\dots[?]. \blankfootnote{4.33 It is difficult to restore \textit{pāda}s cd, although the general meaning of this line is
  predictable. Some questions remain. Is \textit{āditya} a good reading or is \textit{mātṛ} hidden in
  \textit{daivata-mādiśca}? Is \textit{ānṛśaṃsa} right or was it \textit{nṛśaṃsa} that was meant by the author of this line?
  Does \textit{tamanvitaḥ} {\rm (}or \textit{tamānvitaḥ}{\rm )} has anything to do with \textit{tamas} {\rm (}`darkness'{\rm )}?
 }}

  \maintext{pṛthvyā gurutarī mātā ko na vandeta mātaram |}%

  \maintext{yajñadānatapovedās tena sarvaṃ kṛtaṃ bhavet }||\thinspace4:34\thinspace||%
\translation{The mother is more venerable than the earth. Who would not praise a mother? By that [praise], sacrifices, donations, austerities and [the study of] the Vedas, all will be completed. }

  \maintext{gāvaḥ pavitraṃ maṅgalyaṃ devatānāṃ ca devatāḥ |}%

  \maintext{sarvadevamayā gāvas tasmād eva na hiṃsayet }||\thinspace4:35\thinspace||%
\translation{Cows are an auspicious blessing, they are the gods of the gods. Cows contain in themselves all the gods. That is exactly why one should not hurt them. }

  \maintext{jātamātrasya lokasya gāvas trātā na saṃśayaḥ |}%

  \maintext{ghṛtaṃ kṣīraṃ dadhi mūtraṃ śakṛtkarṣaṇam eva ca }||\thinspace4:36\thinspace||%
\translation{Cows are the protectors of the world as if the world were their new-born [calf], there is no doubt about it. Collecting [the five products of the cow, the \textit{pañcagavya},] ghee, milk, curd, and [the cow's] urine and dung [is auspicious]. \blankfootnote{4.36 The use of \textit{karsaṇa} in \textit{pāda} d, most probably in the sense of `collecting,' is slightly odd.
 }}

  \maintext{pañcāmṛtaṃ pañcapavitrapūtaṃ}%

 \nonanustubhindent \maintext{ye pañcagavyaṃ puruṣāḥ pibanti |}%

  \maintext{te vājimedhasya phalaṃ labhanti}%

 \nonanustubhindent \maintext{tad akṣayaṃ svargam avāpnuvanti }||\thinspace4:37\thinspace||%
\translation{People who drink the five products of the cow, the five nectars, purified by the five Pavitras, will obtain the fruits of a horse sacrifice, and then reach the undecaying heavens. \blankfootnote{4.37 The five Pavitras are most probably the five \textit{brahmamantras}, see, e.g., \TAKIII\ s.v. \textit{pavitra} 1.
 }}

  \maintext{gobhir na tulyaṃ dhanam asti kiṃcid}%

 \nonanustubhindent \maintext{duhyanti vāhyanti bahiś caranti |}%

  \maintext{tṛṇāni bhuktvā amṛtaṃ sravanti}%

 \nonanustubhindent \maintext{vipreṣu dattāḥ kulam uddharanti }||\thinspace4:38\thinspace||%
\translation{There is no wealth comparable to [having] a cow. They yield milk, they draw [a plough etc.], they roam under the sky. Feeding on grass, they issue nectar. When given to Brahmins, they deliver the family [from \textit{saṃsāra} or the suffering experienced in hell]. \blankfootnote{4.38 Note that \textit{duhyanti} and \textit{vāhyanti} are supposed to be understood as passive,
  as in the similar verse in \SDHU\ 12.92 {\rm (}see apparatus{\rm )}.
 }}

  \maintext{gavāhnikaṃ yaś ca karoti nityaṃ}%

 \nonanustubhindent \maintext{śuśrūṣaṇaṃ yaḥ kurute gavāṃ tu |}%

  \maintext{aśeṣayajñatapadānapuṇyaṃ}%

 \nonanustubhindent \maintext{labhaty asau tām anṛśaṃsakartā }||\thinspace4:39\thinspace||%
\translation{He who never fails to serve the cow daily [e.g. with a handful of grass], he who tends to the cows' service, he who is kind to her [i.e. to the cow], will obtain the merits of all sacrifices, austerities and donation. \blankfootnote{4.39 Strictly speaking, \textit{pāda} c is unmetrical. The second syllable of \textit{tapa} counts as
  long {\rm (}see Intro \verify{\rm )}.
 Although the accusative with °\textit{kartā} in \textit{pāda} d is still not optimal, my 
  emendation of \textit{tam} to \textit{tām} at least restores the metre and improves 
  upon the meaning of the sentece. Alternatively, as suggested by Törzsök,
  \textit{taṃ} could be understood as \textit{tad}, picking up \textit{puṇyaṃ} in \textit{pāda} c,
  but in this way any reference to cows here is only implied.
 }}

  \maintext{atithiṃ yo 'nugaccheta atithiṃ yo 'numanyate |}%

  \maintext{atithiṃ yo 'nupūjyeta atithiṃ yaḥ praśaṃsate }||\thinspace4:40\thinspace||%
\translation{One who looks after a guest, one who respects a guest, one who worships a guest, one who praises a guest, \blankfootnote{4.40 Note the peculiar active verb forms \textit{anugaccheta} and \textit{anupūjyeta}.
  On this formation, see a remark about \Nisvmul\ 2.8 in \mycitep{NisvasaGoodall}{247}:
  `We have assumed that \textit{pūjyeta} is intended to mean \textit{pūjayet} and is
  perhaps a contraction of \textit{pūjayeta}.'
 }}

  \maintext{atithiṃ yo na pīḍyeta atithiṃ yo na duṣyati |}%

  \maintext{atithipriyakartā yaḥ atitheḥ paricārakaḥ |}%

  \maintext{atitheḥ kṛtasaṃtoṣas tasya puṇyam anantakam }||\thinspace4:41\thinspace||%
\translation{one who does not harm a guest, one who does not commit a fault towards a guest, one who keeps the guest happy, one who attends to the needs of a guest, one who makes a guest satisfied: his merits are endless. \blankfootnote{4.41 On the form \textit{pīḍyeta}, see previous note.
 }}

  \maintext{āsanenārghapātreṇa pādaśaucajalena ca |}%

  \maintext{annavastrapradānair vā sarvaṃ vāpi nivedayet }||\thinspace4:42\thinspace||%
\translation{He should offer [the guest] a seat, a vessel with water-offering, and water for washing his feet, or gifts of food and clothes, or all [of these]. \blankfootnote{4.42 My conjecture in \textit{pāda} a {\rm (}°\textit{pātreṇa} for °\textit{pādyena}{\rm )} was inspired by the fact that 
  \textit{pāda} b seems to awkwardly repeat what °\textit{pādyena} in \textit{pāda} a signifies.
  Other possibilities could include taking into account bathing {\rm (}\textit{snāna}{\rm )} or 
  an unguent {\rm (}\textit{abhyaṅga}{\rm )}.
 }}

  \maintext{putradārātmanā vāpi yo 'tithim anupūjayet |}%

  \maintext{śraddhayā cāvikalpena aklībamānasena ca }||\thinspace4:43\thinspace||%
\translation{He who worships the guest by [offering him] his own son, wife or himself with willingness, without hesitation, and with a brave heart, \blankfootnote{4.43 For the requirement that one could part with his wife or son, or his own life,
  for the benefit of someone else, see \VSS\ 2.38 and the narrative in \VSS\ chapter 12
  which tells about a Brahmin giving away his own wife to a guest;
  these influenced my decision to emend \textit{°ātmano} to \textit{°ātmanā} in \textit{pāda} a.
  Note that in fact \VSS\ 4.44cd below echoes verse 37cd in the above mentioned chapter 12: 
  \textit{dvijarūpadharo dharmaḥ svayam eva ihāgataḥ}.
 }}

  \maintext{na pṛcched gotracaraṇaṃ svādhyāyaṃ deśajanmanī |}%

  \maintext{cintayen manasā bhaktyā dharmaḥ svayam ihāgataḥ }||\thinspace4:44\thinspace||%
\translation{and does not ask [the guests about their] lineage, Vedic affiliation {\rm (}\textit{caraṇa}{\rm )}, studies, country or birth, and imagines mentally, with devotion, that it is Dharma himself who has arrived, }

  \maintext{aśvamedhasahasrāṇi rājasūyaśatāni ca |}%

  \maintext{puṇḍarīkasahasraṃ ca sarvatīrthatapaḥphalam }||\thinspace4:45\thinspace||%
\translation{[will obtain all the fruits of] thousands of Aśvamedha sacrifices and hundreds of Rājasūya sacrifices, a thousand Puṇḍarīka sacrifices and the fruit of [visiting] all the pilgrimage places and [performing] all the austerities; }

  \maintext{atithir yasya tuṣyeta nṛśaṃsamatam utsṛjet |}%

  \maintext{sa tasya sakalaṃ puṇyaṃ prāpnuyān nātra saṃśayaḥ }||\thinspace4:46\thinspace||%
\translation{he whose guest is satisfied [and] he who can abandon the sentiment of cruelty, will obtain all the merits of the above, there is no doubt about it. \blankfootnote{4.46 The demonstrative pronoun \textit{tasya} in \textit{pāda} c may refer to the guest:
  `he will obtain all his [i.e. the guest's] merits,' hinting at some sort of karmic exchange.
  Nevertheless, I think rather that \textit{tasya} points to the merits one can obtain by the rituals listed 
  in the previous verse. This is suggested by passages such as the following:
   %
  \MBH\ Supp. 13.14.379 ff.: 
  \textit{ahany ahani yo dadyāt kapilāṃ dvādaśīḥ samāḥi | 
  māsi māsi ca satreṇa yo yajeta sadā naraḥ ||  
  gavāṃ śatasahasraṃ ca yo dadyāj jyeṣṭhapuṣkare |  
  na taddharmaphalaṃ tulyam atithir yasya tuṣyati ||}
   %
  \BRAHMAVP\ 3.44--46: 
  \textit{atithiḥ pūjito yena pūjitāḥ sarvadevatāḥ | 
  atithir yasya saṃtuṣṭas tasya tuṣṭo hariḥ svayam || 
  snānena sarvatīrtheṣu sarvadānena yat phalam |  
  sarvavratopavāsena sarvayajñeṣu dīkṣayā ||  
  sarvais tapobhir vividhair nityair naimittikādibhiḥ |  
  tad evātithisevāyāḥ kalāṃ nārhanti ṣoḍaśīm ||}.
 }}

  \maintext{{\rm †}na gatim atithijñasya{\rm †} gatim āpnoti karhicit |}%

  \maintext{tasmād atithim āyāntam abhigacchet kṛtāñjaliḥ }||\thinspace4:47\thinspace||%
\translation{\dots\ will ever reach the path. Therefore one should go up to the arriving guest with respectfully joined palms. \blankfootnote{4.47 Something has gone wrong with \textit{pāda}s ab and I am unable to reconstruct the
  meaning. The line may have begun with something like \textit{nāgatātithyavajña}°
  {\rm (}`he who despise a guest that has arrived will not\dots'{\rm )}.
 }}

  \maintext{saktuprasthena caikena yajña āsīn mahādbhutaḥ |}%

  \maintext{atithiprāptadānena svaśarīraṃ divaṃ gatam }||\thinspace4:48\thinspace||%
\translation{By one \textit{prastha} [a small unit of weight] of coarsely ground grains given to a guest, an extremely wonderful sacrifice was performed [so to say], and his body [i.e. the protagonist in his mortal form] reached heaven. \blankfootnote{4.48 This verse is a reference to the story related by a mongoose in \MBH\ 14.92--93: 
  A Brahmin who practises the vow of gleaning {\rm (}\textit{uñcha}{\rm )} and his family
  receive a guest. They feed the guest with the last morsels of the little food
  they have. In the end, the guest reveals that he is in fact Dharma {\rm (}14.93.80cd{\rm )} and as 
  a reward the family departs to heaven. The noble act of the poor Brahmin and his family
  is depicted as yielding greater rewards than Yudhiṣṭhira's grandiose horse-sacrifice. 
  {\rm (}See some remarks on this story in \mycite{TakahashiUnca}.{\rm )}
 
  
 We would be forced to accept the reading of \Ed\ in \textit{pāda} d {\rm (}\textit{saśarīro}{\rm )} 
  if the expression were in the masculine {\rm (}\textit{divaṃ gataḥ}{\rm )}. This would make sense
  and it would also echo expressions occuring, e.g., in the \MBH:
  3.164.33cd: \textit{paśya puṇyakṛtāṃ lokān saśarīro divaṃ vraja};
  14.5.10cd: \textit{saṃjīvya kālam iṣṭaṃ ca saśarīro divaṃ gataḥ}.
  It is tempting to emend accordingly, but instead I have retained 
  \textit{svaśarīraṃ divaṃ gatam}, and I interpret it in a general way.
 }}

  \maintext{nakulena purādhītaṃ vistareṇa dvijottama |}%

  \maintext{viditaṃ ca tvayā pūrvaṃ prasthavārttā ca kīrtitā }||\thinspace4:49\thinspace||%
\translation{The mongoose related [this story in the \textit{Mahābhārata}] in the past in detail, O great Brahmin, and you known it already. The story of the \textit{prastha} is well-known. }

  \subchptr{yameṣu damaḥ {\rm {\rm (}5{\rm )}}}%

  \trsubchptr{The fifth Yama-rule: Self-restraint}%

  \maintext{dama eva manuṣyāṇāṃ dharmasārasamuccayaḥ |}%

  \maintext{damo dharmo damaḥ svargo damaḥ kīrtir damaḥ sukham }||\thinspace4:50\thinspace||%
\translation{Self-restraint is in itself the collected essence of Dharma for humans. Self-restraint is Dharma, self-restraint is heaven, self-restraint is fame, self-restraint is happiness. }

  \maintext{damo yajño damas tīrthaṃ damaḥ puṇyaṃ damas tapaḥ |}%

  \maintext{damahīna{-}m{-}adharmaś ca damaḥ kāmakulapradaḥ }||\thinspace4:51\thinspace||%
\translation{Self-restraint is sacrifice, self-restraint is a pilgrimage-place, self-restraint is merit, self-restraint is religious austerity. If one has no self-restraint, one is a sinner {\rm (}\textit{adharma}{\rm )}, [while] self-restraint yields a multitude of desired objects. \blankfootnote{4.51 I suspect that the final \textit{m} in \textit{dhamahīnam} in \textit{pāda} c is a hiatus filler: \textit{dhamahīna-m-adharmaś ca}.
  \textit{kāmakulapradaḥ} in \textit{pāda} d is suspect, and my translation is 
  unsatisfactory. This compound could be interpreted as `fullfilling desires and giving a family' or 
  it may have originally read \textit{sarvakāmapradaḥ} {\rm (}`fullfilling all desires'{\rm )} or
  \textit{kulakāmapradaḥ} {\rm (}`fullfilling the desires of the family'{\rm )}.
  \SDHS\ 4.28b reads \textit{sarvakāmasukhapradam}, which opens up further possibilities.
 }}

  \maintext{nirdamaḥ kari mīnaś ca pataṅgabhramaramṛgāḥ |}%

  \maintext{tvag jihvā ca tathā ghrāṇā cakṣuḥ śravaṇam indriyāḥ }||\thinspace4:52\thinspace||%
\translation{The elephant, the fish, the moth, the bee and the deer are without self-restraint. The senses are the skin, the tongue, the nose, the eye and the ear. \blankfootnote{4.52 Note \textit{kari} for \textit{karī} metri causa, and the end of \textit{pāda} b {\rm (}\textit{°mṛgāḥ}{\rm )}, which 
  should be treated metrically as if it read \textit{°mrigāḥ}.
 }}

  \maintext{durjayendriyam ekaikaṃ sarve prāṇaharāḥ smṛtāḥ |}%

  \maintext{damaṃ yo jayate 'samyag nirdamo nidhanaṃ vrajet }||\thinspace4:53\thinspace||%
\translation{Each of these sense faculties are hard to conquer and all are known to be fatal [if unconquered]. If one masters self-restraint in a less than proper way, one remains unrestrained and will die . \blankfootnote{4.53 The only way to make sense of \textit{pāda}s cd is to supply and \textit{avagraha} before
  \textit{samyag}. Otherwise some text may have dropped out here.
 }}

  \maintext{mṛge śrotravaśān mṛtyuḥ pataṅgāś cakṣuṣor mṛtāḥ |}%

  \maintext{ghrāṇayā bhramaro naṣṭo naṣṭo mīnaś ca jihvayā }||\thinspace4:54\thinspace||%
\translation{In the case of the deer, death comes about because of hearing [when, e.g., hunters use buck grunts]. Moths die because of their eyes [as they are attracted to the light of a lamp]. Bees perish because of their smelling [as they are attracted to smells], fish because of their tongues [when fishermen feed them]. \blankfootnote{4.54 My comments in square brackets in the translation are tentative.
 }}

  \maintext{sparśena ca karī naṣṭo bandhanāvāsaduḥsahaḥ |}%

  \maintext{kiṃ punaḥ pañcabhuktānāṃ mṛtyus tebhyaḥ kim adbhutam }||\thinspace4:55\thinspace||%
\translation{The elephant perishes because of touch, not being able to tolerate being in fetters. How much more true it is for those who enjoy all five [senses]! Why should death come as a surprise for them? \blankfootnote{4.55 \textit{Mātaṅgalīlā} 11.1 may shed some light on elephants dying in captivity:
  \textit{vānyas tatra sukhoṣitā vidhivaśād grāmāvatīrṇā gajā baddhās tīkṣṇakaṭūgravāgbhir atiśugbhīmohabandhādibhiḥ |
  udvignāś ca manaḥśarīrajanitair duḥkhair atīvākṣamāḥ prāṇān dhārayituṃ ciraṃ naravaśaṃ prāptāḥ svayūthād atha} ||.
  In Edgerton's translation \nocite{EdgertonElephant}{\rm (}1931, 92{\rm )}: 
  `Forest elephants who dwell there happily and by
  the power of fate have been brought to town in bonds, afflicted by harsh, bitter, cruel words,
  by excessive grief, fear, bewilderment, bondage, etc., and by sufferings of mind and body,
  are quite unable for long to sustain life, when from their own herds they have come into
  the control of men.'
 }}

  \maintext{purūravo 'tilobhena atikāmena daṇḍakaḥ |}%

  \maintext{sāgarāś cātidarpeṇa atimānena rāvaṇaḥ }||\thinspace4:56\thinspace||%
\translation{Purūravas [perished] by excessive greed, Daṇḍaka by excessive desire, Sagara's sons by excessive pride, Rāvaṇa by excessive haughtiness, \blankfootnote{4.56 We may treat \textit{purūravo} in \textit{pāda} a as a stem form noun or thematised stem, or imagine that the
  original reading was \textit{purūravā}° with double sandhi:
  \textit{purūravās ati}° $\rightarrow$\ \textit{purūravā ati}° $\rightarrow$\ \textit{purūravāti}°.
 
  \textit{Pāda} a may refer to the following passage in the \MBH\ {\rm (}1.70.16--18, 20ab{\rm )}:
  \textit{purūravās tato vidvān ilāyāṃ samapadyata | 
  sā vai tasyābhavan mātā pitā ceti hi naḥ śrutam || 
  trayodaśa samudrasya dvīpān aśnan purūravāḥ | 
  amānuṣair vṛtaḥ sattvair mānuṣaḥ san mahāyaśāḥ || 
  vipraiḥ sa vigrahaṃ cakre vīryonmattaḥ purūravāḥ | 
  jahāra ca sa viprāṇāṃ ratnāny utkrośatām api || 
  [\dots] 
  tato maharṣibhiḥ kruddhaiḥ śaptaḥ sadyo vyanaśyata |}
   %
  {\rm (}``The wise Purūravas was born to Ilā. We heard that Ilā 
  was both his mother and his father. 
  The great Purūravas ruled over thirteen islands of the ocean
  and, though human, he was always surrounded by superhuman beings.
  Intoxicated with his power, Purūravas quarrelled with some Brahmins 
  and robbed them of their wealth even though they were protesting. [...]
  Therefore, cursed by the great Ṛṣis, he perished.''{\rm )}
  See also \BUDDHACARITA\ 11.15 {\rm (}Aiḍa = Purūravas{\rm )}: %
  \textit{ aiḍaś ca rājā tridivaṃ vigāhya  
  nītvāpi devīṃ vaśam urvaśīṃ tām | 
  lobhād ṛṣibhyaḥ kanakaṃ jihīrṣur  
  jagāma nāśaṃ viṣayeṣv atṛptaḥ ||}
   %
 
  For Daṇḍa{\rm (}ka{\rm )}'s story, see \RAMAYANA\ 7.71.31 ff.:
  Daṇḍa meets Arajā, a beautiful girl, in a forest and rapes her. As a consequence, her father, Śukra/Bhārgava,
  destroyes Daṇḍa's kingdom, which thus becomes the desolate Daṇḍaka-forest.
 
   %
  For two versions of the destruction of
  Sagara's sons, who were chasing the sacrificial horse of their father's Aśvamedha sacrifice,
  and by doing so disturbed Kapila's meditation, and who in turn burnt them to ashes,
  see \MBH\ 3.105.9 ff. and \BRAHMANDAPUR\ 2.52--53.
 
   %
  As for Rāvaṇa's haughtiness,
  especially the fact that he chose to be invincible by all creatures except humans,
  and its consequences,
  one should recall the story of the \RAMAYANA\ and Rāvaṇa's destruction brought about by Rāma therein.
 }}

  \maintext{atikrodhena saudāsa atipānena yādavāḥ |}%

  \maintext{atitṛṣṇāc ca māndhātā nahuṣo dvijavajñayā }||\thinspace4:57\thinspace||%
\translation{Saudāsa by excessive anger, the Yādavas by excessive drinking, Māndhātṛ by excessive desire, Nahuṣa by contempt for Brahmins, \blankfootnote{4.57 Saudāsa, also known as Kalmāṣapāda, hit Śakti, Vasiṣṭha's son, with a whip because
  the latter did not give way to him, and as a consequence Śakti cursed Saudāsa:
  Saudāsa had to roam the world as a Rākṣasa for twelve years. 
  See \MBH\ 1.166.1 ff.
 
   %
  As for the end of the Yādavas, see the short \textit{Mausalaparvan} of the \MBH\ {\rm (}canto 16{\rm )}:
  cursed by the sages Viśvāmitra, Kaṇva and Nārada, and seeing menacing omens,
  the Yādavas take to drinking in Prabhāsa and destroy each other.
  %Most probably, \textit{atitṛṣṇā} in the MSS stand for \textit{atitṛṣṇāt} {\rm (}intending \textit{atitṛṣṇayā}{\rm )}.
  The form \textit{māndhāto} in \msCb\ stands for \textit{māndhātā} {\rm (}nominative of \textit{māndhātṛ}{\rm )}.
  I have corrected it in spite of the fact that the authors' knowledge about his story may
  come from \DIVYAV\ 17, where it sometimes appears to be an a-stem noun {\rm (}\textit{māndāta}{\rm )}.
  \textit{dvijavajñayā} in \textit{pāda} d stands for \textit{dvijāvajñayā} metri causa.
 
   %
  Māndhātṛ was born from his father's body who, being excessively thirsty once,
  had drank some decoction prepared for ritual purposes and as a result become pregnant with him.
  Nevertheless, \BUDDHACARITA\ 11.13 suggests that Māndhātṛ himself was still unsatisfied
  with wordly objects even after he had obtained half of Indra's throne:
  \textit{devena vṛṣṭe 'pi hiraṇyavarṣe  
  dvīpān samagrāṃś caturo 'pi jitvā|  
  śakrasya cārdhāsanam apy avāpya  
  māndhātur āsīd viṣayeṣv atṛptiḥ||}. 
  In fact, as Monika Zin points out {\rm (}\mycitep{ZinMandhatar}{149}{\rm )},
  Māndhātṛ/Māndhāta's rise and fall is a very popular theme
  in the `Narrative Art of the Amaravati School': 
  `Statistics show that in the Amaravati School the most frequently represented narrative is
  the story of King Māndhātar, which appears 47 times.'
  
   %
  Nahuṣa was elevated to the position of Indra for a period of time and he also wanted
  to take Śacī, Indra's wife. Indra instructed Śacī to tell Nahuṣa to 
  harness some Ṛsis to a vehicle and use this vehicle to take Śacī. 
  Agastya, one of the Ṛṣis, was insulted even further by Nahuṣa, therefore
  he cursed Nahuṣa, who then fell from the vehicle. See \MBH\ 12.329.35 ff. and
  a verse in the \BUDDHACARITA\ {\rm (}11.14{\rm )} that follows the one about Māndhātṛ:
   %
  \textit{bhuktvāpi rājyaṃ divi devatānāṃ  
  śatakratau vṛtrabhayāt pranaṣṭe| 
  darpān maharṣīn api vāhayitvā  
  kāmeṣv atṛpto nahuṣaḥ papāta||}.
 }}

  \maintext{atidānād balir naṣṭa atiśauryeṇa arjunaḥ |}%

  \maintext{atidyūtān nalo rājā nṛgo goharaṇena tu }||\thinspace4:58\thinspace||%
\translation{[Mahā]bali perished by excessive donations, Arjuna by excessive heroism, King Nala by excessive gambling, Nṛga by taking a cow. \blankfootnote{4.58 \textit{Pāda} a is most probably a reference to Mahābali's promises made to Vāmana that caused his own fall. 
  The ultimate cause of Arjuna' death while the Pāṇḍavas were on the way to the underworld 
  was summarised by Yudhiṣṭhira thus {\rm (}\MBH\ 17.2.21ab{\rm )}:
  \textit{ekāhnā nirdaheyaṃ vai śatrūn ity arjuno 'bravīt |
  na ca tat kṛtavān eṣa śūramānī tato 'patat} ||.
  {\rm (}`Arjuna claimed that he could destroy the enemy in one single day. He failed to do so.
  He was a boaster, that is why he fell.'{\rm )}
 
   % 
  King Nala was an expert in the game of dice but once he lost his kingdom to Puṣkara.
  See, e.g., \MBH\ 3.56.1 ff. 
 
   %
  As for Nṛga, see \MBH\ 14.93.74:  
  \textit{gopradānasahasrāṇi dvijebhyo 'dān nṛgo nṛpaḥ | 
  ekāṃ dattvā sa pārakyāṃ narakaṃ samavāptavān ||.}
  {\rm (}``King Nṛga had made gifts of thousands of cows for the twice-born.
  By giving away one single cow that belonged to someone else, 
  he fell into hell.''{\rm )} 
 }}

  \maintext{damena hīnaḥ puruṣo dvijendra}%

 \nonanustubhindent \maintext{svargaṃ ca mokṣaṃ ca sukhaṃ ca nāsti |}%

  \maintext{vijñānadharmakulakīrtināśa}%

 \nonanustubhindent \maintext{bhavanti vipra damayā vihīnāḥ }||\thinspace4:59\thinspace||%
\translation{[For] a person who is without self-restraint, O great Brahmin, there is no heaven, liberation or happiness. O Brahmin, people without self-restraint are the destruction of knowledge, Dharma, family and fame. \blankfootnote{4.59 Note how flexible the gender of most nouns is in \textit{pāda} b: 
  \textit{svarga}, \textit{mokṣa} and \textit{dama} are usually masculine in standard Sanskrit.
 The majority of the witnesses suggest that \textit{pāda} c ends in a stem form noun {\rm (}\textit{°nāśa}{\rm )}.
  This \textit{pāda} is unmetrical, or rather it applies the licence of a word-final
  short syllable being counted as potentially long {\rm (}\textit{°dharMA°}{\rm )}. 
 Note how \textit{viprā} in \textit{pāda} d is probably an attempt in some MSS to restore the metre.
  This \textit{pāda} is also unmetrical, or rather the licence of a word-final
  short syllable being counted as potentially long is again applied {\rm (}\textit{viPRA}{\rm )}.
 }}

  \subchptr{yameṣu ghṛṇā {\rm {\rm (}6{\rm )}}}%

  \trsubchptr{The sixth Yama-rule: Taboos}%

  \maintext{nirghṛṇo na paratrāsti nirghṛṇo na ihāsti vai |}%

  \maintext{nirghṛṇe na ca dharmo 'sti nirghṛṇe na tapo 'sti vai }||\thinspace4:60\thinspace||%
\translation{A person without taboos does not exists either in this or the other world. In a person without taboos there is no Dharma or religious austerity. \blankfootnote{4.60 The implications of \textit{pāda}s ab are not crystal clear to me. Perhaps:
  such a person has no right for existence in society and has no place in heaven.
 }}

  \maintext{parastrīṣu parārtheṣu parajīvāpakarṣaṇe |}%

  \maintext{paranindāparānneṣu ghṛṇāṃ pañcasu kārayet }||\thinspace4:61\thinspace||%
\translation{These five should be treated as taboo: women who are not depending on oneself, others' wealth, taking away others' lives, hurting others and [consuming] others' food. }

  \maintext{parastrī śṛṇu viprendra ghṛṇīkāryā sadā budhaiḥ |}%

  \maintext{rājñī viprī parivrājā svayoniparayoniṣu }||\thinspace4:62\thinspace||%
\translation{Listen, O great Brahmin, the wise should always treat women who are not dependent on oneself as taboo, [be she] a queen, a Brahmin's wife, a wandering religious mendicant, a relative or of another caste. \blankfootnote{4.62 The translation of \textit{parayoni} in \textit{pāda} d is tentative.
 }}

  \maintext{parārthe śṛṇu bhūyo 'nya anyāyārtha{-}m{-}upārjanam |}%

  \maintext{āḍhaprasthatulāvyājaiḥ parārthaṃ yo 'pakarṣati }||\thinspace4:63\thinspace||%
\translation{Listen further to something else, with regards to others' wealth. [It may include] gaining wealth through unlawful means, when somebody takes away other people's wealth by cheating with weights of one \textit{āḍha[ka]} or a \textit{prastha} and with scales. \blankfootnote{4.63 Although \textit{'nya} in \textit{pāda} a could be interpreted several ways {\rm (}e.g. \textit{anye} for \textit{anyasmin}, 
  or taken to be the first element of a compound: \textit{anya-anyāyārtha-}{\rm )},
  I think that \textit{bhūyo 'nyat} is a fixed expression meaning `something/anything more.' 
  See, e.g., \BHG\ 7.2cd:
  \textit{yaj jñātvā neha bhūyo 'nyaj jñātavyam avaśiṣyate}.
 }}

  \maintext{jīvāpakarṣaṇe vipra ghṛṇīkurvīta paṇḍitaḥ |}%

  \maintext{vanajāvanajā jīvā vilagāś caraṇācarāḥ }||\thinspace4:64\thinspace||%
\translation{O Brahmin, the wise should regard the taking away [of others'] lives as taboo. Wild and domesticated animals, serpents, [in general,] plants and animals [are examples of life forms not to destroy]. \blankfootnote{4.64 In \textit{pāda} d, I take \textit{caraṇācarāḥ} as standing for \textit{carācarāḥ} {\rm (}\textit{cara-acarāḥ}{\rm )} metri causa.
  Alternatively, one may understand it as \textit{caraṇacarāḥ} {\rm (}metri causa{\rm )}, 
  meaning `those who move on their feet,' perhaps as opposed to snakes {\rm (}\textit{bilaga} or \textit{bilaṃga}{\rm )}.
  Neither solution is fully satisfactory. Note that this \textit{pāda} also involves a small correction.
 }}

  \maintext{paranindā ca kā vipra śṛṇu vakṣye samāsataḥ |}%

  \maintext{devānāṃ brāhmaṇānāṃ ca gurumātātithidviṣaḥ }||\thinspace4:65\thinspace||%
\translation{And what is the hurting of others? Listen, O Brahmin, I'll tell you briefly. He who is hostile to the gods, Brahmins, gurus, mothers and guests [hurts others]. \blankfootnote{4.65 Note \textit{mātā} as a stem form in \textit{pāda} d.
 }}

  \maintext{parānneṣu ghṛṇā kāryā abhojyeṣu ca bhojanam |}%

  \maintext{sūtake mṛtake śauṇḍe varṇabhraṣṭakule naṭe }||\thinspace4:66\thinspace||%
\translation{As regards other people's food, eating together with people whose food is not to be accepted {\rm (}\textit{abhojyeṣu}{\rm )} is taboo, [e.g.] after birth or death [in a family], in case of vendors of alcohol, or a family having lost their caste, and in the case of a [member of the] Naṭa [caste of dancers]. \blankfootnote{4.66 One should probably understand \textit{śauṇḍe} in \textit{pāda} c as \textit{śauṇḍike}, `a distiller,' or, alternatively,
  it may be corrupted from \textit{ṣaṇḍhe}, `a eunuch'; see both in \VasDh\ 14.1--3:
  \textit{athāto bhojyābhojyaṃ ca varṇayiṣyāmaḥ |
  cikitsaka-mṛgayu-puṃścalī-ḍaṇḍika-stenābhiśastar-ṣaṇḍha-patitānām annam abhojyam |
  kadarya-dīkṣita-baddhātura-somavikrayi-takṣa-rajaka-śauṇḍika-sūcaka-vārdhuṣika-carmāvakṛntānām ||} etc.
   %
  Translated in \mycitep{OlivelleDharmasutras}{285} as:
  `Next we will describe food that is fit and food that is
  unfit to be eaten [\dots] The following are unfit
  to be eaten: food given by a physician, a hunter, a harlot, a law
  enforcement agent, a thief, a heinous sinner [...] a
  eunuch, or an outcaste; as also that given by a miser, a man
  consecrated for a sacrifice, a prisoner, a sick person, a man who
  sells Soma, a carpenter, a washerman, a liquor dealer, a spy, an
  usurer, a leather worker\dots'
   %
  In support of reading \textit{ṣaṇḍhe}, one might consult \Manu\ 3.239:
   %
  \textit{cāṇḍālaś ca varāhaś ca kukkuṭaḥ śvā tathaiva ca | 
  rajasvalā ca ṣaṇḍhaś ca nekṣerann aśnato dvijān ||.}
  Translated in \mycitep{OlivelleDharmasutras}{120} as:
  `A Cāṇḍāla, a pig, a cock, a dog, a menstruating woman, or a eunuch must not
  look at the Brahmins while they are eating.'
 }}

  \maintext{ete pañcaghṛṇāsu saktapuruṣāḥ svargārthamokṣārthinaḥ}%

 \nonanustubhindent \maintext{loke 'nindanam āpnuvanti satataṃ kīrtir yaśo'laṃkṛtāḥ |}%

  \maintext{prajñābodhaśrutiṃ smṛtiṃ ca labhate mānaṃ ca nityaṃ labhet}%

 \nonanustubhindent \maintext{dākṣiṇyaṃ sabhavet sa āyuṣa paraṃ prāpnoti niḥsaṃśayaḥ }||\thinspace4:67\thinspace||%
\translation{Those people who stick to the five kinds of taboo [and thus] seek heaven, wealth and liberation, will reach eternal faultlessness in this world, embellished with fame and glory. [A person like that] will obtain wisdom, intelligence, [knowledge of] the Śruti and Smṛti traditions, and honour forever. Kindness will arise and he will obtain an extra long life, no doubt. \blankfootnote{4.67 Understand \textit{kīrtir-yaśo°} as \textit{kīrtiyaśo°} {\rm (}'r' being an intrusive consonant here metri causa{\rm )}, 
  as in 5.20 below. Alternatively, as suggested by Francesco Sferra, emend to \textit{kīrtiṃ yaśo'laṃkṛtām}.
  My emendation of °\textit{kṛtam} to °\textit{kṛtāḥ} is influenced be 5.20b.
 In \textit{pāda} c, note the muta cum liquida licence that allows °\textit{bodhaśrutiṃ}°
  to scan as - \shortsyllable\ \shortsyllable\ - , the consonant cluster 
  \textit{śr} not turning the previous syllable long.
 \textit{Pāda} d has several problems. I take \textit{sabhavet} as standing for \textit{sambhavet} metri causa,
  and I had to emend \textit{samāyuṣa} to \textit{sa āyuṣa} to make sense of it.
  Understand \textit{āyuṣa} as \textit{āyuḥ} {\rm (}metri causa{\rm )}, otherwise emend to \textit{sa mānuṣya}.
  Also consider correcting \textit{niḥsaṃśayaḥ} to \textit{niḥsaṃśayam}.
 }}

  \subchptr{yameṣu pañcavidho dhanyaḥ {\rm {\rm (}7{\rm )}}}%

  \trsubchptr{The seventh Yama-rule: The five methods of virtue?}%

  \maintext{caturmaunaṃ catuḥśatruś caturāyatanaṃ tathā |}%

  \maintext{caturdhyānaṃ catuṣpādaṃ pañcadhanyavidhocyate }||\thinspace4:68\thinspace||%
\translation{The four cases of observing silence, [victory over] the four enemies, the four sanctuariess, the four meditations, and the four legged [Dharma] are called the five ways of being virtuous. \blankfootnote{4.68 Understand \textit{pāda} d as \textit{pañcavidho dhanya ucyate}.
 }}

  \maintext{caturmaunasya vakṣyāmi śṛṇuṣvāvahito bhava |}%

  \maintext{pāruṣyapiśunāmithyāsambhinnāni ca varjayet }||\thinspace4:69\thinspace||%
\translation{I shall tell you about the four cases of observing silence. Listen, be attentive. One should avoid violent and slanderous [words], lies, and idle [talk]. \blankfootnote{4.69 Note the genitive with a verb meaning `to tell' in \textit{pāda} a, similarly to 1.38a and \verify.
 Similar teachings on \textit{mauna} in \DHARMP\ 1.31cd--32ab and \DIVYAV\ 186.21 are quoted in the apparatus.
 }}

  \maintext{kāmaḥ krodhaś ca lobhaś ca mohaś caiva caturvidhaḥ | }%

  \maintext{catuḥśatrur nihantavyaḥ so 'rihā vītakalmaṣaḥ }||\thinspace4:70\thinspace||%
\translation{The fourfold enemy [made up of] desire, anger, greed and delusion is to be destroyed. He who destroys [these] enemies will become sinless. \blankfootnote{4.70 Possible direct sources for the idea that \textit{kāma} is an enemy to be defeated or avoided include
  \BUDDHACARITA\ 11.17:
   %
  \textit{cīrāmbarā mūlaphalāmbubhakṣā  
  jaṭā vahanto 'pi bhujaṃgadīrghāḥ |  
  yair nānyakāryā munayo 'pi bhagnāḥ  
  kaḥ kāmasaṃjñān mṛgayeta śatrūn ||};
   %
  see also \BHG\ 3.43:
   %
  \textit{evaṃ buddheḥ paraṃ buddhvā saṃstabhyātmānam ātmanā | 
  jahi śatruṃ mahābāho kāmarūpaṃ durāsadam ||}.
  As for \textit{arihā} in \textit{pāda} d, the notion that a saint is a `destroyer of the enemies' 
  [that are evil states of mind] {\rm (}\textit{arihanta/arahanta}{\rm )}
  in Jainism, but less so in Buddhism, is discussed in \mycitep{GombrichWhat2013}{57--58}.
 }}

  \maintext{caturāyatanaṃ vipra kathayiṣyāmi tac chṛṇu |}%

  \maintext{karuṇā muditopekṣā maitrī cāyatanaṃ smṛtam }||\thinspace4:71\thinspace||%
\translation{I shall teach you the four sanctuaries. Listen, O Brahmin. Compassion, sympathy in joy, indifference, and benevolence are the four sanctuaries. \blankfootnote{4.71 This verse teaches the four Buddhist \textit{brahmavihāra}s under the label
  \textit{caturāyatana}. Therfore the word \textit{āyatana} seems to be a synonym of \textit{vihāra} here,
  and its use a simple method of appropriating it, turning the list into a Brahmanical one.
 }}

  \maintext{caturdhyānādhunā vakṣye saṃsārārṇavatāraṇam |}%

  \maintext{ātmavidyābhavaḥ sūkṣmaṃ dhyānam uktaṃ caturvidham }||\thinspace4:72\thinspace||%
\translation{I shall now teach you the four meditations, which will liberate you from transmigration. Meditation is taught to be fourfold: of the Self, \textit{vidyā}, \textit{bhava} [= Śiva] and the subtle one {\rm (}\textit{sūkṣma}{\rm )}. \blankfootnote{4.72 Note the stem form \textit{dhyāna} in \textit{°dhyānādhunā} {\rm (}for \textit{°dhyānam adhunā}{\rm )} in \textit{pāda} a.
 For contrast, but also for similarities, see the \textit{dhyānayajña} section in \VSS\ 6.7ff, in which
  five types of related meditations are taught. See analysis on pp. Intro \verify.
 }}

  \maintext{ātmatattvaḥ smṛto dharmo vidyā pañcasu pañcadhā |}%

  \maintext{ṣaṭtriṃśākṣaram ityāhuḥ sūkṣmatattvam alakṣaṇam }||\thinspace4:73\thinspace||%
\translation{The \textit{tattva} of the Self is Dharma. \textit{Vidyā} is in the five in a fivefold way[??]. They call the thirty-sixth the imperishable one, [and] the subtle \textit{tattva} has no attributes. \blankfootnote{4.73 This verse is difficult to interpret. \textit{Pāda}s a to d should define \textit{ātman}, \textit{vidyā}, \textit{bhava}, and \textit{sūkṣma},
  objects of meditation, respectively. In \textit{pāda} a, \textit{dharmo} is suspect: it may be the result of
  an eye-skip to \textit{pāda} a of the next verse. \textit{Pāda} b might refer to \textit{tattva}s in an ontological
  system of 25, 26 or 36 \textit{tattva}s.
 If \textit{pāda} c is in fact a reference to a 36-\textit{tattva} philosophical system,
  it is in striking contrast with the 25-\textit{tattva} system described in \VSS\ chapter 20.
  I take \textit{ṣaṭtriṃśa} as being in stem form.
 }}

  \maintext{catuṣpādaḥ smṛto dharmaś caturāśramam āśritaḥ |}%

  \maintext{gṛhastho brahmacārī ca vānaprastho 'tha bhaikṣukaḥ }||\thinspace4:74\thinspace||%
\translation{The four-legged one is said to be Dharma [as] it rests on the four \textit{āśrama}s, [those of] the householder, the chaste one, the forest-dweller and the mendicant. }

  \maintext{dhanyās te yair idaṃ vetti nikhilena dvijottama |}%

  \maintext{pāvanaṃ sarvapāpānāṃ puṇyānāṃ ca pravardhanam }||\thinspace4:75\thinspace||%
\translation{Virtuous are those who know these thoroughly, O great Brahmin. [They will experience] the purification of all sins and the growth of merits. \blankfootnote{4.75 Note the plural instrumental {\rm (}\textit{yair}{\rm )} with a singular active verb {\rm (}\textit{vetti}; anacoluthic structure{\rm )}.
 }}

  \maintext{āyuḥ kīrtir yaśaḥ saukhyaṃ dhanyād eva pravardhate |}%

  \maintext{śāntiḥ puṣṭiḥ smṛtir medhā jāyate dhanyamānave }||\thinspace4:76\thinspace||%
\translation{One's life-span, fame and glory and happiness grow only through virtue {\rm (}\textit{dhanya}{\rm )}. In a virtuous person piece, prosperity, tradition {\rm (}\textit{smṛti}{\rm )} and intelligence will arise. \blankfootnote{4.76 Emending °\textit{mānavaḥ} to °\textit{mānave} might err by overcorrection, and °\textit{mānavaḥ} may have originally
  been felt like a genitive {\rm (}`for a person\dots'{\rm )}.
 }}

  \subchptr{yameṣv apramādaḥ {\rm {\rm (}8{\rm )}}}%

  \trsubchptr{The eighth Yama-rule: Lack of negligence}%

  \maintext{pramādasthāna pañcaiva kīrtayiṣyāmi tac chṛṇu |}%

  \maintext{brahmahatyā surāpānaṃ steyo gurvaṅganāgamam |}%

  \maintext{mahāpātakam ity āhus tatsaṃyogī ca pañcamaḥ }||\thinspace4:77\thinspace||%
\translation{There are five areas of negligence. I shall teach them to you, listen. Murdering a Brahmin, drinking alcohol, stealing, having sex with the guru's wife: they call these grievous sins. The fifth is when one is connected with them [i.e. with these sins or with people involved in these sinful acts]. \blankfootnote{4.77 Note the stem form noun in \textit{pāda} a {\rm (}°\textit{sthāna}{\rm )} metri causa, and also 
  that this stem form noun may function as a singular noun
  next to a number {\rm (}\textit{pañca}{\rm )}, a frequently seen phenomenon in this text.
 See the apparatus to the Sanskrit text for very similar verses in the \MBH, \Manu\ and 
  the \YAJNS, and note how \textit{pāda} f slightly deviates from \Manu\ 11.55, which is translated in
  \mycitep{OlivelleManu}{217--218} as: 
  `Killing a Brahmin, drinking liquor, stealing, and having sex with an elder's 
  wife---they call these ``grievous sins causing loss of caste''; 
  and so is establishing any links with such individuals.'
 }}

  \maintext{anṛtaṃ ca samutkarṣe rājagāmī ca paiśunaḥ |}%

  \maintext{guroś cālīkanirbandhaḥ samāni brahmahatyayā }||\thinspace4:78\thinspace||%
\translation{A lie concerning one's superiority, a slander that reaches the king's ear, and false accusations against an elder are equal to killing a Brahmin. \blankfootnote{4.78 This verse being a quotation of \Manu\ 11.56, my translation 
  is based on \mycitep{OlivelleManu}{218}.
 }}

  \maintext{brahmojjhaṃ vedanindā ca kūṭasākṣī suhṛdvadhaḥ |}%

  \maintext{garhitānādyayor jagdhiḥ surāpānasamāni ṣaṭ }||\thinspace4:79\thinspace||%
\translation{Abandoning the Vedas, reviling the Vedas, being a false witness, murdering a friend, eating unfit or forbidden food are six [deeds that are] equal to drinking alcohol. \blankfootnote{4.79 This verse continues quoting \Manu. \textit{Pāda} a in the witnesses may actually be no more than the result of 
  misreading of the syllable \textit{jjha} in \Manu\ 11.57. Note the variant \textit{brahmojjhaṃ vedanindā ca}
  in both the `Northern' and `Southern' transmissions in Olivelle's critical edition 
  of \Manu\ {\rm (}\mycitep{OlivelleManu}{847}{\rm )}.
 }}

  \maintext{retotsekaḥ svayonyāsu kumārīṣv antyajāsu ca |}%

  \maintext{sakhyuḥ putrasya ca strīṣu gurutalpasamaḥ smṛtaḥ }||\thinspace4:80\thinspace||%
\translation{Sexual intercourse with a female relative, with an unmarried girl, with women of the lowest castes, with the wife of a friend or of one's own son are said to be equal to violating the guru's bed. \blankfootnote{4.80 The text, and my emendation in \textit{pāda} c, still follow \Manu\ {\rm (}11.59{\rm )}.
 }}

  \maintext{nikṣepasyāpaharaṇaṃ narāśvarajatasya ca |}%

  \maintext{bhūmivajramaṇīnāṃ ca rukmasteyasamaḥ smṛtaḥ }||\thinspace4:81\thinspace||%
\translation{Stealing deposits, people, horses, silver, land, diamonds, or gems are said to be equal to stealing gold. \blankfootnote{4.81 This is \Manu\ 11.58. I have emended \textit{rugma}° to \textit{rukma}° in \textit{pāda} d, although
  \textit{rugma}° is attested in a great number of Southern MSS and one Śāradā MS in \mycitep{OlivelleManu}{847}.
 }}

  \maintext{catvāra ete sambhūya yat pāpaṃ kurute naraḥ |}%

  \maintext{mahāpātakapañcaitat tena sarvaṃ prakāśitam |}%

  \maintext{pañcapramādam etāni varjanīyaṃ dvijottama }||\thinspace4:82\thinspace||%
\translation{If a man is associated with [any of these] four [i.e. \textit{brahmahatyā, surāpāna, stena, gurvaṅganāgama}], he commits sin. By this all the five grievous sins have been explained. These five kinds of negligence are to be avoided, O great Brahmin. \blankfootnote{4.82 Perhaps understand \textit{pāda} c as \textit{etan mahāpātakapañcakaṃ}.
 Note the confusion of number and gender: understand \textit{pañca pramādāḥ etā varjanīyāḥ}
  or \textit{pañca prāmādāny etāni varjanīyāni}.
 }}

  \subchptr{yameṣu mādhuryam {\rm {\rm (}9{\rm )}}}%

  \trsubchptr{The ninth Yama-rule: Charm}%

  \maintext{kāyavāṅmanamādhuryaś cakṣur buddhiś ca pañcamaḥ |}%

  \maintext{saumyadṛṣṭipradānaṃ ca krūrabuddhiṃ ca varjayet }||\thinspace4:83\thinspace||%
\translation{[Charm has five types:] bodily, verbal and mental charm, [charm of] the eyes and [of one's] thoughts as fifth. Giving [others] a friendly glance [is commendable] and one should avoid cruel thoughts. \blankfootnote{4.83 My emendation from \textit{°manasā dhūryaś} to \textit{°mana-mādhuryaś} is based on the fact that following the list
  of \textit{yama}s in 3.16cd--17ab, we need some reference to \textit{mādhurya} here and that it is easy to see how this
  corruption came about: \textit{°mano-mādhurya°} would be unmetrical, hence the form \textit{°mana-mādhurya};
  \textit{°mana-mā°} is easily corrupted to \textit{°manasā°} {\rm (}not to mention the fact 
  that \textit{manasā} comes up in the next verse{\rm )}. 
  In addition, we need five items in this line because of \textit{pañcamaḥ}.
  As always, I correct \textit{mādhūrya} to \textit{mādhurya}, although it seems that 
  the former is acceptable in this text. 
  I did not correct \textit{mādhuryaś} to \textit{mādhuryaṃ} because of the corresponding
  \textit{pañcamaḥ}.
 }}

  \maintext{prasannamanasā dhyāyet priyavākyam udīrayet |}%

  \maintext{yathāśaktipradānaṃ ca svāśramābhyāgato guruḥ }||\thinspace4:84\thinspace||%
\translation{One should meditate with a tranquil mind and should speak [to other people using] gentle words. [When] respectable people arrive at one's own hermitage, [one should] present them with as many gifts as one can, \blankfootnote{4.84 \textit{Pāda}s cd of the previous verse, and \textit{pāda}s ab of the present one cover
  four categories of the above: \textit{cakṣurmādhurya}, \textit{buddhimādhurya}, \textit{dṛṣṭimādhurya} and \textit{vāgmādhurya}.
  This suggests that what follows is on \textit{kāyamādhurya}.
 Emending \textit{pāda} d to \textit{svāśramābhyāgate gurau} would make the line smoother, as
  suggested by Kengo Harimoto.
 }}

  \maintext{indhanodakadānaṃ ca jātavedam athāpi vā |}%

  \maintext{sulabhāni na dattāni indhanāgnyudakāni ca |}%

  \maintext{kṣute jīveti vā noktaṃ tasya kiṃ parataḥ phalam }||\thinspace4:85\thinspace||%
\translation{with gifts of fire-wood, water and fire. [If] fire-wood, fire and water are easily available [but] are not given [as gift] or [if the phrase] `Live [for a hundred years]!' is not uttered when [somebody] sneezes, what reward could there be for such a person in the afterlife? \blankfootnote{4.85 Understand \textit{jātavedam} in \textit{pāda} b as \textit{jātavedasam} or \textit{jātavedāḥ},
  or rather as belonging to the compound °\textit{dānaṃ}: \textit{jātavedodānaṃ}.
 For \textit{pāda} e, see an Āryāgīti verse in the \MAHASUBHS\ {\rm {\rm (}2558{\rm )}}: 
  \textit{amṛtāyatām iti vadet pīte bhukte kṣute ca śataṃ jīva |
  choṭikayā saha jṛmbhāsamaye syātāṃ cirāyurānandau ||}
  {\rm (}`When eating or drinking, one should say: ``May it turn into nectar!''; 
  and after sneezing: ``Live for a hundred years!''
  By snapping the thumb and forefinger when yawning, there will be long life and happiness.'{\rm )}
 }}

  \subchptr{yameṣv ārjavam {\rm {\rm (}10{\rm )}}}%

  \trsubchptr{The tenth Yama-rule: Sincerity}%

  \maintext{pañcārjavāḥ praśaṃsanti munayas tattvadarśinaḥ |}%

  \maintext{karmavṛttyābhivṛddhiṃ ca pāratoṣikam eva ca |}%

  \maintext{strīdhanotkocavittaṃ ca ārjavo nābhinandati }||\thinspace4:86\thinspace||%
\translation{The sages who see the truth praise five types of sincerity. A sincere person does not rejoice in prosperity arising from the operation of karma or by a reward, in riches from women, from property, and bribery. \blankfootnote{4.86 °\textit{ārjavāḥ} should be in the accusative, therefore it is to be taken as feminine {\rm (}rather than neuter{\rm )} or as
  an irregular form for °\textit{ārjavāni}. I have emended \textit{pāratoṣikam} to \textit{pāritoṣikam}.
 My translation of the categories listed here is tentative, the only guiding light being
  that, if the first line is right, there should be five of them. In addition, I have tried to
  find categories that seem to be, more or less, in conflict with `sincerity' or `straightness.'
 }}

  \maintext{ārjavo na vṛthā yajña ārjavo na vṛthā tapaḥ |}%

  \maintext{ārjavo na vṛthā dānam ārjavo na vṛthāgnayaḥ }||\thinspace4:87\thinspace||%
\translation{If one is not sincere, sacrifice is in vain. If one is not sincere, austerity is in vain. If one is not sincere, donation is in vain. If one is not sincere, [sacrificial] fires are in vain. \blankfootnote{4.87 I thank Nirajan Kafle for helping me interpret this verse.
 }}

  \maintext{ārjavasyendriyagrāmaḥ suprasanno 'pi tiṣṭhati |}%

  \maintext{ārjavasya sadā devāḥ kāye tasya caranti te }||\thinspace4:88\thinspace||%
\translation{The sense faculties of a sincere person are firm even when he is delighted. The gods are always present in the body of a sincere person. }

  \maintext{iti yamapravibhāgaḥ kīrtito 'yaṃ dvijendra}%

 \nonanustubhindent \maintext{iha parata sukhārthaṃ kārayet taṃ manuṣyaḥ |}%

  \maintext{duritamalapahārī śaṅkarasyājñayāste}%

 \nonanustubhindent \maintext{bhavati pṛthivibhartā hy ekachatrapravartā }||\thinspace4:89\thinspace||%
\translation{Thus has been taught this section on the \textit{yama}-rules, O great Brahmin. Humans should follow them to reach happiness here and in the other world. One will stand removing one's filth of sins, and shall by Śaṅkara's command become a ruler of the world [that he subjugates] under one royal umbrella. \blankfootnote{4.89 In \textit{pāda} a °\textit{pra}° does not make the previous syllable long: this is the phenomenon of
  `muta cum liquida,' one of the hallmarks of the \VSS, 
  that is, syllables such as \textit{tra, pra, bra, dra} do not necessarily make the 
  previous syllable long.
 In \textit{pāda} b, \textit{parata} most probably stands for \textit{paratra} or \textit{parataḥ} metri causa. 
  We may correct it to \textit{paratra}, presupposing the presence of the licence `muta cum liquida.'
 \textit{°malapahārī} in the MSS stands either for \textit{°malāpahārī} or \textit{°malaprahārī} metri causa. 
  I could have choosen to emend it to \textit{°malaprahārī} {\rm (}again applying the licence `muta cum liquida'{\rm )},
  but I decided not to because \textit{apahārin}, \textit{apahāra}, \textit{apahāraka} are used in the text very frequently. 
  See also 8.44c, which contains a very similar expression: \textit{sakalamalapahāre dharmapañcāśad etat}.
 }}
\center{\maintext{\dbldanda\thinspace iti vṛṣasārasaṃgrahe yamavibhāgo nāmādhyāyaś{ }caturthaḥ\thinspace\dbldanda}}
\translation{Here ends the fourth chapter in the \textit{Vṛṣasārasaṃgraha} called the Section on the Yama-rules.}

  \chptr{pañcamo 'dhyāyaḥ}
\fancyhead[CE]{{\footnotesize\textit{Translation of chapter 5}}}%

  \trchptr{ Chapter Five }%

  \subchptr{niyamāḥ}%

  \trsubchptr{The Niyama-rules}%

  \maintext{vigatarāga uvāca |}%

  \maintext{kathaya niyamatattvaṃ sāmprataṃ tvaṃ viśeṣād}%

 \nonanustubhindent \maintext{amṛtavacanatulyaṃ śrotukāmo gato 'smi |}%

  \maintext{prakṛtidahanadagdhaṃ jñānatoyair niṣiktam}%

 \nonanustubhindent \maintext{apara vada{-}m{-}atajjñaṃ nāsti dharmeṣu tṛptiḥ }||\thinspace5:1\thinspace||%
\translation{Vigatarāga spoke: Now teach me the true nature of the Niyama-rules in detail. I have become desirious to hear [your] teaching that is comparable to ambrosia. Tell me more {\rm (}\textit{apara vada}{\rm )}, [to the one who had been] burnt by the fire of materiality {\rm (}\textit{prakṛti}{\rm )}, [but is now] sprinkled with the water of knowledge, and is ignorant of [the topic]. One can't have enough of the [teaching on] Dharmas {\rm (}\textit{nāsti dharmeṣu tṛptiḥ}{\rm )}. \blankfootnote{5.1 Most witnesses read °\textit{vadana}° in \textit{pāda} b. This is slightly odd in the sense of `speech,' the meaning
  required here, therefore I follow \msM\ here. One wonders if it is not \textit{amṛtasvādana} or
  °\textit{svadana} {\rm (}`tasting nectar'{\rm )} what was meant originally. I translate the phrase in question as if it read
  \textit{amṛtatulyavacanaṃ}.
 The first half of \textit{pāda} d is difficult to interpret safely. \textit{apara vada} {\rm (}`tell me more'{\rm )} might be original,
  with \textit{apara} in stem form. The phrase \textit{matajñā} is now emended to \textit{-m-atajjñaṃ},
  containing a hiatus breaker but making the line metrical.
  Otherwise it could be emended to \textit{matajña} {\rm (}with the last syllable taken as long{\rm )} and translated as a vocative {\rm (}`O knower of [my] thoughts{\rm )}.
  Note \msM's reading for the end of the line {\rm (}\textit{me dharmatṛptiḥ}{\rm )}.
 }}

  \maintext{anarthayajña uvāca |}%

  \maintext{śravaṇasukham ato 'nyat kīrtayiṣye dvijendra}%

 \nonanustubhindent \maintext{niyamakalaviśeṣaḥ pañca pañca prakāraḥ |}%

  \maintext{hariharamunibhīṣṭaṃ dharmasāraṃ dvijendra}%

 \nonanustubhindent \maintext{kalikaluṣavināśaṃ prāyamokṣaprasiddham }||\thinspace5:2\thinspace||%
\translation{Anarthayajña spoke: I shall teach you something more that is nice to hear, O best of the twice-born. The specific sections of the Niyamas are of five types [each]. It is the essence of Dharma, dear to Hari, Hara and the sages, O great Brahmin, the destruction of the impurity of the Kali age, generally known as liberation. \blankfootnote{5.2 My suspicion is that °\textit{kala}° in \textit{pāda} b stands for \textit{kalā} metri causa. 
 Similarly, °\textit{munibhīṣṭaṃ} is metri causa, for °\textit{munyabhīṣṭaṃ} {\rm (}`dear the the sages'{\rm )}.
 In \textit{pāda} d, \textit{prāya}° is suspect. Compare with 6.1c: \textit{dharmamokṣaprasiddhyarthaṃ}.
 }}

  \maintext{śaucam ijyā tapo dānaṃ svādhyāyopasthanigrahaḥ |}%

  \maintext{vratopavāsamaunaṃ ca snānaṃ ca niyamā daśa }||\thinspace5:3\thinspace||%
\translation{Purification, sacrifice, penance, donation, Vedic study and the restraint of sexual desire, religious observances, fasting, taciturnity, and bathing: these are the ten Niyamas. }

  \subchptr{niyameṣu śaucam {\rm {\rm (}1{\rm )}}}%

  \trsubchptr{The first Niyama-rule: Purity}%

  \maintext{tatra śaucādinirdeśaṃ vakṣyāmīha dvijottama |}%

  \maintext{śārīraśaucam āhāro mātrā bhāvaś ca pañcamaḥ }||\thinspace5:4\thinspace||%
\translation{From among these, now I shall tell you the particulars of purification [first], and [then] the others. [1] Bodily purity, [2] [purity of] food, [3] [purity of] property[?] {\rm (}\textit{mātrā}{\rm )}, [4] [purity of] character[?] {\rm (}\textit{bhāva}{\rm )}, and the fifth, [5]...? \blankfootnote{5.4 The chapter deals with \textit{śārīraśauca} {\rm (}5.5--9{\rm )} and \textit{āhāraśauca} {\rm (}5.10--16{\rm )}, therefore \textit{pāda} c is probably correct, 
  and \msM's reading {\rm (}\textit{śārīrasrotam āhāra}{\rm )} is wrong. Even if we could interpret \textit{pāda} d with any certainty, there
  is one missing element of this list of allegedly five items. Something must have dropped out here.
  Oddly enought, the chapter stops after teaching the second type of purity, \textit{āhāraśauca}, so we are left without a clue.
  \MBH\ Indices 14.4.3229--3230 is not very helpful: \textit{manaḥśaucaṃ karmaśaucaṃ kulaśaucaṃ ca bhārata |
  śarīraśaucaṃ vākśaucaṃ śaucaṃ pañcavidhaṃ smṛtam} ||.
 }}

  \subsubchptr{śarīraśaucam}%

  \trsubsubchptr{Purity of the Body}%

  \maintext{tāḍayen na ca bandheta na ca prāṇair viyojayet |}%

  \maintext{parastrīparadravyeṣu śaucaṃ kāyikam ucyate }||\thinspace5:5\thinspace||%
\translation{He should not beat, tie or kill [any living being]. [This and] purity concerning others' wives and property is called bodily purity. \blankfootnote{5.5 Note the application of the licence muta cum liquida in \textit{pāda} c: the first syllable
  of \textit{dravyeṣu} does not make the previous syllable heavy.
 }}

  \maintext{śrotraśaucaṃ dvijaśreṣṭha gudopasthamukhādayaḥ |}%

  \maintext{mukhasyācamanaṃ śaucam āhāravacaneṣu ca }||\thinspace5:6\thinspace||%
\translation{The cleanliness of the ears, O great Brahmin, and of the anus, the loins, the mouth etc. [also contributes to bodily purity]. The purity of the mouth [comes from] sipping water before eating, speaking. }

  \maintext{mūtraviṣṭāsamutsarge devatārādhaneṣu ca |}%

  \maintext{mṛttoyais tu gudopasthaṃ śaucayīta vicakṣaṇaḥ }||\thinspace5:7\thinspace||%
\translation{After the emission of urine and faeces, and before the worship of gods, the wise one should clean his anus and his loins with clay and water. \blankfootnote{5.7 Note the peculiar verb form \textit{śaucayīta} {\rm (}for a more standard \textit{śocayeta}{\rm )}. \msM's \textit{śaucaye}[\textit{c}] \textit{ca}
  may be close to an original reading.
 }}

  \maintext{ekopasthe gude pañca tathaikatra kare daśa |}%

  \maintext{ubhayoḥ sapta dātavyā mṛdaḥ śuddhiṃ samīhatā }||\thinspace5:8\thinspace||%
\translation{One [portion of clay] for the loins, five for the anus, ten for one hand, [then] seven [portions] of clay are to be applied for both [hands] by him who wishes cleanliness. \blankfootnote{5.8 In essence, this verse is \Manu\ 5.136. Olivelle's notes on this verse read:
  `\textit{on one hand:} within the context the meaning is clear: ``one hand'' refers to the left
  hand, with which the person applied the earth and water to the penis and anus. All purifications
  below the navel are carried out using the left hand. Variant reading: ``on the left hand.''\thinspace '
  {\rm (}\mycitep{OlivelleManu}{287}.{\rm )}
 }}

  \maintext{etac chaucaṃ gṛhasthānāṃ dviguṇaṃ brahmacāriṇām |}%

  \maintext{vānaprasthasya triguṇaṃ yatīnāṃ tu caturguṇam }||\thinspace5:9\thinspace||%
\translation{This is the purification for the householder {\rm (}\textit{gṛhastha}{\rm )}. It is twice as much for the chaste one {\rm (}\textit{brahmacārin}{\rm )}, three times as much for the forest-dweller {\rm (}\textit{vānaprastha}{\rm )}, four times as much for the ascetic {\rm (}\textit{yati}{\rm )}. \blankfootnote{5.9 This verse corresponds to \Manu\ 5.137.
 Note the muta cum liquida licence in \textit{pāda} c: \textit{tr} does not turn the previous syllable heavy and 
  the \textit{pāda} becomes a \textit{na-vipulā}.
 }}

  \subsubchptr{āhāraśaucam}%

  \trsubsubchptr{Purity of the food}%

  \maintext{āhāraśaucaṃ vakṣyāmi śṛṇuṣvāvahito bhava |}%

  \maintext{bhāgadvayaṃ tu bhuñjīta bhāgam ekaṃ jalaṃ pibet |}%

  \maintext{vāyusaṃcāradānārthaṃ caturtham avaśeṣayet }||\thinspace5:10\thinspace||%
\translation{I shall teach you the rules of purity concerning food. Listen, pay great attention. One should eat [as much] food [that fills] two quarters [of the stomach] and drink water [that fills] one quarter. In order to give passage to the air, one should save the remaining quarter. \blankfootnote{5.10 Śaṅkara quotes a similar verse in his commentary ad \BHG\ 6.16 {\rm (}see apparatus{\rm )}.
  It translates as:
  `Half is for saucy food, the third part for water, but in order to be able to move the air,
  one should leave the fourth part [empty].' This verse and one in the \SANNYASUP\ {\rm (}see apparatus{\rm )} have
  \textit{saṃcaraṇārthaṃ tu} and \textit{saṃcaraṇārthāya}, respectively, where our verse in the \VSS\ has \textit{saṃcāradānārthaṃ}.
  It would be tempting to emend but the \VSS\ version more or less works fine, therefore 
  there is no need to alter the text.
 }}

  \maintext{snigdhasvādurasaiḥ ṣaḍbhir āhāraṣaḍrasair budhaḥ |}%

  \maintext{dhātuvaiṣamyanāśo 'sti na ca rogāḥ sudāruṇāḥ }||\thinspace5:11\thinspace||%
\translation{[By] the wise one['s applying] the six soft and sweet juices, [which are] the six flavours in food, the disturbances of the \textit{dhātu}s will disappear and the terrible illnesses will not arise. \blankfootnote{5.11 The readings may suggest that \textit{pāda} b contains \textit{sadrava} or maybe \textit{sudrava}, but it is difficult to make
  sense of the sentence. We are lacking a verb; \textit{āhāra} might be wrong for \textit{āharet} {\rm (}see \msM{\rm )}.
  The Āyurvedic implications of this clumsy verse are obscue to me. What is clear is that traditionally there are
  six basic flavours or `juices' in food. See, e.g. \BHELAS\ 1.28.1:
  \textit{yad bhakṣayati bhuṅkte vā vidhivac cāpi mānavaḥ |
  anyac ca kiñcit pibati tat sarvaṃ ṣaḍrasānvitam ||.}
  {\rm (}`All that a human eats or enjoys according to the rules, and furthermore all 
  that he or she drinks, is endowed with the six flavours.'{\rm )}
  To repair \textit{pāda}s ab, one should perhaps imagine that the intended meaning was that the
  six flavours/juices should be present in a harmonious proportion in a wise man's food. Cf. \BHELAS\ 3.1.1:
  \textit{śarīraṃ dhārayantīha ṣaḍrasāḥ samam āhṛtāḥ |
  ato 'nyathā vikārāṃs tu janayanti śarīriṇām ||.}
  {\rm (}`The six flavours will support the body in this world when brought to a balanced state.
  Otherwise they will produce defects to people.'{\rm )}
 On \textit{dhātuvaiṣamya}, see, e.g., \CARAKA\ 1.9.4:
  \textit{vikāro dhātuvaiṣamyaṃ sāmyaṃ prakṛtir ucyate |
  sukhasaṃjñakam ārogyaṃ vikāro duḥkham eva ca ||}
  {\rm (}`The imbalance of the \textit{dhātu}s means defects. Balance is said to be natural.
  Health is happiness, defects are suffering.'{\rm )}
 }}

  \maintext{abhakṣyaṃ ca na bhakṣeta apeyaṃ na ca pāyayet |}%

  \maintext{agamyaṃ na ca gamyeta avācyaṃ na ca bhāṣayet }||\thinspace5:12\thinspace||%
\translation{He should not eat what is forbidden and he should not drink what is forbidden. He should not go where he is not allowed to and he should not say what is improper. \blankfootnote{5.12 Understand the causative \textit{pāyayet} as simplex.
 }}

  \maintext{laśunaṃ ca palāṇḍuṃ ca gṛñjanaṃ kavakāni ca |}%

  \maintext{gauraṃ ca sūkaraṃ māṃsaṃ varjayec ca vidhānataḥ }||\thinspace5:13\thinspace||%
\translation{He should avoid garlic, onion, \textit{gṛñjana} onion, mushrooms, buffalo meat and pork, following the rules. }

  \maintext{chattrākaṃ viḍvarāhaṃ ca gomāṃsaṃ ca na bhakṣayet |}%

  \maintext{caṭakaṃ ca kapotaṃ ca jālapādāṃś ca varjayet }||\thinspace5:14\thinspace||%
\translation{He should not eat \textit{chattrāka} mushrooms, village hog, and cow flesh. He should also avoid sparrows, pigeons, and water-birds. }

  \maintext{haṃsasārasacakrāhvakukkuṭān śukaśyenakān |}%

  \maintext{kākolūkaṃ balākaṃ ca matsyādīṃś cāpi varjayet }||\thinspace5:15\thinspace||%
\translation{He should also avoid [eating] geese, cranes, \textit{cakravāka} birds, cocks, parrots and hawks, crows, owls, \textit{balāka} cranes, fish etc. \blankfootnote{5.15 Note that in \textit{pāda} b the first syllable of \textit{śyenakān} does not turn the previous syllable, \textit{śu},
  heavy. This is an extension of the muta cum liquida licence.
 }}

  \maintext{amedhyāṃś cāpavitrāṃś ca sarvān eva vivarjayet |}%

  \maintext{śākamūlaphalānāṃ ca abhakṣyaṃ parivarjayet }||\thinspace5:16\thinspace||%
\translation{He should avoid everything that is ritually impure or polluted. He should also completely avoid those vegetables, roots and fruits, that are prohibited. }

  \maintext{mānaveṣu purāṇeṣu śaivabhāratasaṃhite |}%

  \maintext{kīrtitāni viśeṣeṇa śaucācāram aśeṣataḥ |}%

  \maintext{tvayā jijñāsito 'smy adya saṃkṣiptaḥ kathito mayā }||\thinspace5:17\thinspace||%
\translation{In the books of Manu, in the Purāṇas, in Śaiva texts, and in the \textit{Bhāratasaṃhitā} {\rm (}i.e. the \textit{Mahābhārata}{\rm )}, the practice of purity is definitely expanded in great detail. Now you have asked me [about it], and I taught it [to you] in a condensed form. \blankfootnote{5.17 In \textit{pāda} b, since °\textit{saṃhite} is not a correct locative of °\textit{saṃhitā}, 
  instead of emending to \textit{śaive bhāratasaṃhite}, we may take the compound as a \textit{samāhāra\-dvandva\-samāsa} in the neuter locative.
 Note the gender and number confusion between \textit{kīrtitāni} and °\textit{ācāram} in \textit{pāda}s cd.
  This and the next verse sound as if the author had been aware of the fact that he 
  left the remaining three categories of purity {\rm (}see 5.4{\rm )} unexplained.
 }}

  \maintext{satyavādī śucir nityaṃ dhyānayogarataḥ śuciḥ |}%

  \maintext{ahiṃsakaḥ śucir dānto dayābhūtakṣamā śuciḥ }||\thinspace5:18\thinspace||%
\translation{He who speaks the truth is pure. He who engages in yogic meditation is pure. He who avoids violence and is restrained is pure. Compassion towards living beings and patience is purity. \blankfootnote{5.18 My impression is that \textit{dayābhūtakṣamā} in \textit{pāda} d may stand for \textit{bhūtadayā kṣamā} {\rm (}\textit{bhūtadayā} occurring in
  1.7 and 3.27--28{\rm )}, and I translate accordingly.
 }}

  \maintext{sarveṣām eva śaucānām arthaśaucaṃ paraṃ smṛtam |}%

  \maintext{yo 'rthe hi śuciḥ sa śucir na mṛdvāriśuciḥ śuciḥ |}%

  \maintext{kāyavāṅmanasāṃ śaucaṃ sa śuciḥ sarvavastuṣu }||\thinspace5:19\thinspace||%
\translation{Of all the [ways of] purification, material purification is taught to be the highest. For he who is pure with regards to material things is truly pure, and not the one who [only] uses clay and water [i.e. the one who performs only ordinary baths]. When purification pertains to the body, to speech and to the mind, he is pure in all respects. \blankfootnote{5.19 \textit{Pāda}s a-d are quoting \Manu\ 5.106 {\rm (}in most witnesses, unmetrically{\rm )}; it is translated in
  \mycitep{OlivelleManu}{144} as:
  `Purifying oneself with respect to wealth, tradition tells us, is the highest of all
  purifications; for the truly pure man is the one who is pure with respect to wealth, not
  the one who becomes pure by using earth and water.'
 }}

  \maintext{śaucāśaucavidhijña mānava yadi kālakṣaye niścayaḥ}%

 \nonanustubhindent \maintext{saubhāgyatvam avāpnuvanti satataṃ kīrtir yaśo'laṅkṛtāḥ |}%

  \maintext{prāptaṃ tena ihaiva puṇyasakalaṃ saddharmaśāstreritam}%

 \nonanustubhindent \maintext{jīvānte ca paratra{-}m{-}īhitagatiṃ prāpnoti niḥsaṃśayam }||\thinspace5:20\thinspace||%
\translation{If a person knows the rules of purity and impurity, he will surely gain happiness at the end of time, eternally embellished with glory and fame. He has reached here in this world all the merits that the books on true Dharma teach, and at the end of his life he will undoubtedly reach the desired path in the other world. \blankfootnote{5.20 Note the stem form adjective \textit{°jña} and noun \textit{°mānava} metri causal and
  the second syllable of \textit{yadi} as a long syllable at the c\ae sura in \textit{pāda} a 
  {\rm (}see \msM's reading{\rm )}, the plural \textit{āpnuvanti} where one would expect a verb in the singular and
  \textit{kīrtir} metri causa for a compounded stem form {\rm (}\textit{kīrti°}{\rm )} in \textit{pāda} b,
  and the sandhi-bridge \textit{-m-} in \textit{paratra-m-īhita°} in \textit{pāda} d. Compare with 4.67b above.
 }}
\center{\maintext{\dbldanda\thinspace iti vṛṣasārasaṃgrahe śaucācāravidhir{ }nāmādhyāyaḥ pañcamaḥ\thinspace\dbldanda}}
\translation{Here ends the fifth chapter in the \textit{Vṛṣasārasaṃgraha} called the Method of Purification.}

  \chptr{ṣaṣṭho 'dhyāyaḥ}
\fancyhead[CE]{{\footnotesize\textit{Translation of chapter 6}}}%

  \trchptr{ Chapter Six }%

  \subchptr{niyameṣv ijyā {\rm {\rm (}2{\rm )}}}%

  \trsubchptr{The second Niyama-rule: Sacrifice}%

  \maintext{atha pañcavidhām ijyāṃ pravakṣyāmi dvijottama |}%

  \maintext{dharmamokṣaprasiddhyarthaṃ śṛṇuṣvāvahito dvija }||\thinspace6:1\thinspace||%
\translation{[Anarthayajña continued:] Now I shall teach you the five types of sacrifice {\rm (}\textit{ijyā}{\rm )}, O excellent Brahmin, for success in Dharma and liberation. Listen carefully, O Brahmin. }

  \maintext{arthayajñaḥ kriyāyajño japayajñas tathaiva ca |}%

  \maintext{jñānaṃ dhyānaṃ ca pañcaitat pravakṣyāmi pṛthak pṛthak }||\thinspace6:2\thinspace||%
\translation{Material sacrifice, sacrifice through work, sacrifice through recitation, knowledge and meditation: I shall teach you these five one by one. \blankfootnote{6.2 Note the singular \textit{etat} after a number {\rm (}see Intro \verify{\rm )}.
 
  Compare this list of five to the somewhat similar \BHG\ 4.28:
  \textit{dravyayajñās tapoyajñā yogayajñās tathāpare |
  svādhyāyajñānayajñāś ca yatayaḥ saṃśita-vratāḥ ||}.
  \SDHU\ chapter 3 can be also relevant since it uses the terms
  \textit{japayajña}, \textit{jñānayajña}, and \textit{dhyānayajña}. See also \SDHU\ 1.10 {\rm (}\msCa\ f.\thinspace 42v l4{\rm )}:
  \textit{karmayajñas tapoyajñaḥ svādhyāyo dhyānam eva ca | 
  jñānayajñaś ca pañcaite mahāyajñāḥ prakīrtitāḥ ||}.
  Note how this definition of the five \textit{mahāyajña}s in the \SDHU\ 
  is different from the one, e.g., in \Manu\ 3.69--71
  {\rm (}\textit{brahma}°, \textit{pitṛ}°, \textit{daiva}°, \textit{bhauta}°, and \textit{nṛyajña}{\rm )}.
 }}

  \subsubchptr{arthayajñaḥ}%

  \trsubsubchptr{Material sacrifice}%

  \maintext{agnyupāsanakarmādi agnihotrakratukriyā |}%

  \maintext{aṣṭakā pārvaṇī śrāddhaṃ dravyayajñaḥ sa ucyate }||\thinspace6:3\thinspace||%
\translation{Material sacrifice includes the following: the domestic ritual fire worship etc., the public performance of the ritual of Agnihotra, [and the so-called \textit{pākayajña}s such as] the Aṣṭakā oblation, the Pārvaṇī oblation, and the ancestral ritual {\rm (}\textit{śrāddha}{\rm )}. \blankfootnote{6.3 By somewhat overtranslating the items in this list, I want to emphasise that
  the text introduces three categories of sacrifical rituals well-known from
  the time of the Gṛhyasūtras and Śrautasūtras: those of the domestic or \textit{aupāsana} fire {\rm (}\textit{gṛhyakarman}{\rm )},
  the Śrauta rituals such as the Agnihotra, and the Smārta \textit{pākayajña}s, such as the \textit{aṣṭakā}, 
  the \textit{pārvaṇī} and the \textit{śrāddha}. For a mention of the \textit{pākayajña}s in a manner similar to 
  our \textit{pāda}s cd here, see, e.g. the \Diksottara\ quoted in \mycitep{NisvasaGoodall}{275}:
  \textit{aṣṭakāḥ pārvaṇī śrāddhaṃ śrāvaṇy āgrāyaṇī tathā |
  caitrī cāśvayujī caiva pākayajñāḥ prakīrtitāḥ ||}.
  For an earlier list of \textit{pākayajña}s, see \GAUTDHS\ 1.8.19: 
  \textit{aṣṭakā pārvaṇaḥ śrāddham śrāvaṇy\-āgrahāyaṇī\-caitry\-āśvayujīti sapta pākayajñasamsthāḥ}.
 }}

  \subsubchptr{kriyāyajñaḥ}%

  \trsubsubchptr{Sacrifice through work}%

  \maintext{ārāmodyānavāpīṣu devatāyataneṣu ca |}%

  \maintext{svahastakṛtasaṃskāraḥ kriyāyajña sa ucyate }||\thinspace6:4\thinspace||%
\translation{Sacrifice through work is taking care of/ cleaning/ embellishing {\rm (}\textit{saṃskāra}{\rm )} a grove, a park, a pond or a temple with one's own hands. }

  \subsubchptr{japayajñaḥ}%

  \trsubsubchptr{Sacrifice through recitation}%

  \maintext{japayajñaṃ tato vakṣye svargamokṣaphalapradam |}%

  \maintext{vedādhyayana kartavyaṃ śivasaṃhitam eva ca |}%

  \maintext{itihāsapurāṇaṃ ca japayajñaḥ sa ucyate }||\thinspace6:5\thinspace||%
\translation{Next I shall teach you the sacrifice through recitation, the bestower of the fruits of heaven and liberation. One should recite the Vedas, Śaiva texts or the \textit{Mahābhārata}, the epics and the Purāṇas: this is called sacrifice with recitation. \blankfootnote{6.5 Note the stem form \textit{vedādhyayana} in \textit{pāda} c metri causa. As for the interpretation of
  \textit{śivasaṃhitam} in \textit{pāda} d, see 5.17b above: \textit{śaivabhāratasaṃhite}. 
  The proximity of these two phrases, and the fact that both give instructions
  on using texts, suggest that we should interpret them similarly. 
  It is then a \textit{samāhāra\-dvandva\-samāsa} again, in the neuter.
  Both \textit{śivasaṃhitam} and \textit{itihāsapurāṇaṃ} should be interpreted as
  being part of the compound in \textit{pāda} c: \textit{śiva\-saṃhitādhyayanaṃ} and 
  \textit{itihāsapurāṇādhyayanaṃ}.
 }}

  \subsubchptr{jñānayajñaḥ}%

  \trsubsubchptr{Sacrifice through knowledge}%

  \maintext{idaṃ karma akarmedam ūhāpohaviśāradaḥ |}%

  \maintext{śāstracakṣuḥ samālokya jñānayajñaḥ sa ucyate }||\thinspace6:6\thinspace||%
\translation{[He who can decide if] `this is [proper] action; the other is improper action' because he is knowledgeable about reasoning pro and contra, and investigates with his eyes on the Śāstras, is called [a person performing] sacrifice through knowledge. \blankfootnote{6.6 For the expression \textit{śāstracakṣuḥ}, see, e.g., \BRAHMAP\ 24.21:
  \textit{tena yajñān yathāproktān mānavāḥ śāstracakṣuṣaḥ |
  kurvate 'harahaś caiva devān āpyāyayanti te ||}.
  In G. P. Bhatt's translation {\rm (}\mycitep{BrahmapuranaTr1}{126}{\rm )}:
  `Day by day men with the sacred scriptures as their guides
  perform sacrifices in the manner they have been laid down and thereby nourish the gods.'
 }}

  \subsubchptr{dhyānayajñaḥ}%

  \trsubsubchptr{Sacrifice through meditation}%

  \maintext{dhyānayajñaṃ samāsena kathayiṣyāmi te śṛṇu |}%

  \maintext{dhyānaṃ pañcavidhaṃ caiva kīrtitaṃ hariṇā purā |}%

  \maintext{sūryaḥ somo 'gni sphaṭikaḥ sūkṣmaṃ tattvaṃ ca pañcamam }||\thinspace6:7\thinspace||%
\translation{I shall teach you concisely about sacrifice through meditation. Listen to me. Meditation was taught by Hari in the past as of five kinds. [Meditation on] the Sun, the Moon, Fire, Crystal and the subtle \textit{tattva} as fifth. \blankfootnote{6.7 For an analysis of this fivefold method of meditation, and this ancient-looking
  \textit{tattva}-system, see Intro \verify, and for different
  versions of the same teaching of meditation, see \VSS\ 22.19--28 and \DHARMP\ 4.5--14.
 }}

  \maintext{sūryamaṇḍalam ādau tu tattvaṃ prakṛtir ucyate |}%

  \maintext{tasya madhye śaśiṃ dhyāyet tattvaṃ puruṣa ucyate }||\thinspace6:8\thinspace||%
\translation{First it is the Sun [that should be meditated upon], which is said to be \textit{prakṛti-tattva}. He should visualize the Moon in its centre: that \textit{tattva} is said to be \textit{puruṣa}. \blankfootnote{6.8 Note the form \textit{śaśiṃ} for \textit{śaśinaṃ}.
 }}

  \maintext{candramaṇḍalamadhye tu jvālām agniṃ vicintayet |}%

  \maintext{prabhutattvaḥ sa vijñeyo janmamṛtyuvināśanaḥ }||\thinspace6:9\thinspace||%
\translation{In the centre of the Moon's disk, he should visualise a flame, a fire. That is said to be \textit{prabhu}-\textit{tattva}, the destroyer of [the circle of] birth and death. }

  \maintext{agnimaṇḍalamadhye tu dhyāyet sphaṭika nirmalam |}%

  \maintext{vidyātattvaḥ sa vijñeyaḥ kāraṇam ajam avyayam }||\thinspace6:10\thinspace||%
\translation{In the centre of the ring of Fire, he should visualize a spottless crystal. That is said to be \textit{vidyā}-\textit{tattva}, the never-born, imperishable cause. \blankfootnote{6.10 Note the stem form \textit{sphaṭika} in \textit{pāda} b metri causa.
 }}

  \maintext{vidyāmaṇḍalamadhye tu dhyāyet tattvam anuttamam |}%

  \maintext{akīrtitam anaupamyaṃ śivam akṣayam avyayam |}%

  \maintext{pañcamaṃ dhyānayajñasya tattvam uktaṃ samāsataḥ }||\thinspace6:11\thinspace||%
\translation{In the centre of the disk of \textit{vidyā}, he should visualize the highest \textit{tattva}, never-heard, unparalleled, undecaying and imperishable Śiva. The fifth \textit{tattva} of the sacrifice through meditation has been taught in short. }

  \maintext{vigatarāga uvāca |}%

  \maintext{ekaikasya tu tattvasya phalaṃ kīrtaya kīdṛśam |}%

  \maintext{kāni lokāḥ prapadyante kālaṃ vāsya tapodhana }||\thinspace6:12\thinspace||%
\translation{Vigatarāga spoke: Teach me: what are the fruits of [reaching] each \textit{tattva}? Which worlds can be attained and how much time [can one spend there], O great ascetic? \blankfootnote{6.12 The reading \textit{tritattvasya} in \textit{pāda} a in the MSS is a problem 
  because we have just finished a section mentioning five \textit{tattva}s. 
  {\rm (}This was probably noticed by \Ed, hence printing \textit{hi} for \textit{tri}°.{\rm )}
  My conjecture {\rm (}\textit{tu}{\rm )} is based on the assumption that \textit{tri} is ofter written as \textit{tṛ} 
  in Nepalese MSS {\rm (}e.g. in \msM\ at this point{\rm )} and that \textit{tṛ} may then easily get corrupted to \textit{tu}.
 }}

  \maintext{anarthayajña uvāca |}%

  \maintext{brahmalokaṃ tu prathamaṃ tattvaprakṛticintayā |}%

  \maintext{kalpakoṭisahasrāṇi śivavan modate sukhī }||\thinspace6:13\thinspace||%
\translation{Anarthayajña spoke: Through meditation on the first \textit{tattva}, \textit{prakṛti}, [one reaches] Brahmaloka. He will rejoice [there] happily like Śiva for millions of \ae ons. \blankfootnote{6.13 Understand \textit{pāda}s ab as \textit{brahmalokaṃ prathamatattvacintayā prakṛtitattvacintayā}. 
  One might take \textit{prathamaṃ} adverbially {\rm (}`firstly': \textit{prathamaṃ brahmalokaṃ prakṛtitattvacintayā}{\rm )},
  but in the next verses, the ordinal numbers {\rm (}\textit{dvitīyaṃ, tṛtīyaṃ, pañcamaṃ}{\rm )}
  always refer to the \textit{tattva}s.
 }}

  \maintext{dvitīyaṃ tattva puruṣaṃ dhyāyamāno mṛto yadi |}%

  \maintext{viṣṇulokam ito yāti kalpakoṭyayutaṃ sukhī }||\thinspace6:14\thinspace||%
\translation{If one dies while meditating on the second \textit{tattva}, \textit{puruṣa}, one goes to Viṣṇuloka from this world, [and will live there] happily for billions of \ae ons. \blankfootnote{6.14 Note the stem form \textit{tattva} in \textit{pāda} a metri causa.
 }}

  \maintext{prabhutattvaṃ tṛtīyaṃ tu dhyāyamāno mariṣyati |}%

  \maintext{śivaloke vasen nityaṃ kalpakoṭyayutaṃ śatam }||\thinspace6:15\thinspace||%
\translation{Should one die while meditating on the third, the \textit{prabhu-tattva}, one can live in Śivaloka continuously for a hundred billion \ae ons. \blankfootnote{6.15 \Ed\ changes \textit{śivaloka} to \textit{rudraloka}, probably for more contrast with
  \textit{sadāśiva} in 6.16 and \textit{śivatattva} in 6.17. \verify
 }}

  \maintext{vidyātattvāmṛtaṃ dhyāyet sadāśivam anāmayam |}%

  \maintext{akṣayaṃ lokam āpnoti kalpānāntaparaṃ tathā  }||\thinspace6:16\thinspace||%
\translation{If he visualizes the nectar of \textit{vidyā-tattva}, [i.e.] Sadāśiva, he can reach [His] diseaseless, imperishable world [and can live there] well beyond endless \ae ons. \blankfootnote{6.16 In \textit{pāda} a, \textit{amṛta} is suspect. It may refer to the world of Sadāśiva and 
  then \textit{vidyātattva} is in stem form. Alternatively, since this verse is the only one in
  this list of worlds {\rm (}6.13--17{\rm )} without an ordinal number, \textit{amṛtaṃ} may mean `four' or possibly `fourth,'
  as suggested by Monier-Williams and Apte in their dictionaries. This meaning would fit in nicely.
  In addition, dying has been mentioned above, thus \textit{amṛtaṃ} might be a corrupted form of 
  a participle from the verbal root \textit{mṛ} {\rm (}\textit{mṛyan} or \textit{maran}?{\rm )}: e.g., 
  \textit{vidyātattvaṃ mṛyan dhyāyet...} {\rm (}`should he meditation upon Vidyātattva while dying...'{\rm )}.
 }}

  \maintext{pañcamaṃ śivatattvaṃ tu sūkṣmaṃ cātmani saṃsthitam |}%

  \maintext{na kālasaṃkhyā tatrāsti śivena saha modate }||\thinspace6:17\thinspace||%
\translation{The fifth one, the subtle \textit{śiva-tattva} dwells in the Self. There is no counting of time there and he will be rejoicing [there] together with Śiva. }

  \maintext{pañcadhyānābhiyukto bhavati ca na punarjanmasaṃskārabandhaḥ}%

 \nonanustubhindent \maintext{jijñāsyantāṃ dvijendra bhavadahanakaraḥ prārthanākalpavṛkṣaḥ |}%

  \maintext{janmenaikena muktir bhavati kimu na vā mānavāḥ sādhayantu}%

 \nonanustubhindent \maintext{pratyakṣān nānumānaṃ sakalamalaharaṃ svātmasaṃvedanīyam }||\thinspace6:18\thinspace||%
\translation{[If] he practises the five meditations, there is no rebirth and no more fetters of transmigration. O excellent Brahmin, [the Lord] should be seeked, a wishing tree of desires, [as] he burns away existence. Liberation comes within one single birth! People, why should you not strive [for it]! [It is known] as the destroyer of all impurity. [It's ascertainable] by direct perception. It is not inference. It is to be experienced by one's own Self. \blankfootnote{6.18 Note how a plural passive imperative form {\rm (}\textit{jijñāsyantāṃ}{\rm )} stands for the singular
  {\rm (}\textit{jijñāsyatāṃ}{\rm )} metri causa. Note also that the last syllable of
  \textit{dvijendra} {\rm (}at the c\ae sura{\rm )} counts here as long: this phenomenon of a word-ending
  syllable becoming long by position is common in the \VSS.
 The non-standard \textit{janmena} in \textit{pāda} d seems superior to \textit{janmanā} for it
  preserves the metre.
 }}

  \subchptr{niyameṣu tapaḥ {\rm {\rm (}3{\rm )}}}%

  \trsubchptr{The third niyama-rule: Penance}%

  \maintext{mānasaṃ tapa ādau tu dvitīyaṃ vācikaṃ tapaḥ |}%

  \maintext{kāyikaṃ ca tṛtīyaṃ tu manovākkarma tatparam |}%

  \maintext{kāyikaṃ vācikaṃ caiva tapo miśraka pañcamam }||\thinspace6:19\thinspace||%
\translation{The first type of penance is mental penance, the second is verbal penance, the third is the bodily one, the next one is the one which is [characterised by] both mental and verbal action. The fifth type of penance is a mixture of the bodily and the verbal ones. \blankfootnote{6.19 Note the stem form \textit{miśraka} in \textit{pāda} f metri causa.
 }}

  \maintext{manaḥsaumyaṃ prasādaś ca ātmanigraham eva ca |}%

  \maintext{maunaṃ bhāvaviśuddhiś ca pañcaitat tapa mānasam }||\thinspace6:20\thinspace||%
\translation{Gentleness of the mind, calmness, self-control, taciturnity and the purification of one's state of mind: mental penance comprises these five. \blankfootnote{6.20 Again, we can see the use of the singular {\rm (}\textit{etat}{\rm )} next to numbers; note also 
  the stem form \textit{tapa} in \textit{pāda} d metri causa. This verse is a paraphrase of \MBH\ 3.39.16 {\rm (}\BHG\ 17.16; see text in the
  apparatus{\rm )}.
 }}

  \maintext{anudvegakarā vāṇī priyaṃ satyaṃ hitaṃ ca yat |}%

  \maintext{svādhyāyābhyasanaṃ caiva vācikaṃ tapa ucyate }||\thinspace6:21\thinspace||%
\translation{Verbal penance is taught as speech that causes no anxiety, which is kind, true and useful, and it includes also the practice of recitation. \blankfootnote{6.21 This verse is a version of \MBH\ 6.39.15 {\rm (}\BHG\ 17.15; see it in the apparatus{\rm )}.
 }}

  \maintext{ārjavaṃ ca ahiṃsā ca brahmacaryaṃ surārcanam |}%

  \maintext{śaucaṃ pañcamam ity etat kāyikaṃ tapa ucyate }||\thinspace6:22\thinspace||%
\translation{Bodily penance is taught as the following: honesty, harmlessness, chastity, the worship of gods, and purity as the fifth. \blankfootnote{6.22 This verse seems to be a paraphrase of \MBH\ 6.39.14 {\rm (}\BHG\ 17.14; see it in the apparatus{\rm )}.
 }}

  \maintext{iṣṭaṃ kalyāṇabhāvaṃ ca dhanyaṃ pathyaṃ hitaṃ vadet |}%

  \maintext{manomiśraka pañcaitat tapa uktaṃ maharṣibhiḥ }||\thinspace6:23\thinspace||%
\translation{[Penance] which is a mixture of the mental [and the verbal] is taught by the great sages to be these five: he should speak [about things that are] agreeable, of a virtuous character, auspicious, salutary and useful. \blankfootnote{6.23 Note the use of the singular {\rm (}\textit{etat}{\rm )} next to a number and the stem form noun in \textit{pāda} c.
 }}

  \maintext{svasti maṅgalam āśīrbhir atithigurupūjanam |}%

  \maintext{kāyamiśraka pañcaitat tapa uktaṃ mahātmabhiḥ }||\thinspace6:24\thinspace||%
\translation{[Penance] in which bodily [and verbal actions] mix is taught by the great-souled ones to be these five: the worship of the guest and the guru, benediction, greetings, and blessings. \blankfootnote{6.24 See \SDHS\ 11.73--79 {\rm (}and \mycitep{SaivaUtopia2021}{91--93 and 120--121}{\rm )} 
  for a somewhat similar discussion on `kind speach.'
 }}

  \maintext{maṇḍūkayogī hemante grīṣme pañcatapās tathā |}%

  \maintext{abhrāvakāśo varṣāsu tapaḥ sādhanam ucyate }||\thinspace6:25\thinspace||%
\translation{[Being] a [so-called] frog-yogin in the winter, or one with the five fires in the summer, or one who has the clouds [i.e. the open sky] for shelter in the rainy season: these kinds of penance is called \textit{sādhana}. \blankfootnote{6.25 \Manu\ 6.23 mentions three kins of penance that corresponds to three seasons:
  \textit{grīṣme pañcatapās tu syād varṣāsv abhrāvakāśikaḥ |
  ārdravāsās tu hemante kramaśo vardhayaṃs tapaḥ ||}. 
  Translated in \mycitep{OlivelleManu}{149} as:
  `[He should] surround himself with the five fires in the summer; live in the open air during the rainy season;
  and wear wet clothes in the winter---gradually intensifying his ascetic toil.'
  This and \SDHSANGR\ 9.32ab {\rm (}quoted in the apparatus{\rm )} may suggest that being 
  a `frog-yogin' could be the same as wearing wet clothes or standing in water for a long time.
  A footnote to verse \MBH\ 12.309.9 in the Kumbakonam edition of the \MBH\ {\rm (}\mycite{MBhKumbakonaEd}{\rm )} suggests otherwise:
  \textit{maṇḍūkavat pāṇipādaṃ saṅkocya nyubjaḥ śete iti maṇḍūkaśāyī}. {\rm (}`The word `frog-sleeper' means
  somebody who sleeps like a frog, with his hands and feet withdrawn and with his back humped.'{\rm )} 
 }}

  \maintext{svamāṃsoddhṛtya dānaṃ ca hastapādaśiras tathā |}%

  \maintext{puṣpam utpādya dānaṃ ca sarve te tapasādhanāḥ }||\thinspace6:26\thinspace||%
\translation{Carving out his own flesh as a donation, or [offering his own] hand, feet and head, or drawing [his own] blood {\rm (}\textit{puṣpa}{\rm )} as a donation: all these are \textit{sādhana}-penances, \blankfootnote{6.26 Note the stem form \textit{svamāṃsa} in \textit{pāda} a for the accusative.
 The translation of \textit{pāda} c is tentative, but taking \textit{puṣpa} as `blood' is not only
  normal e.g. in tantric texts {\rm (}see e.g. \verify{\rm )}, but \VSS\ 17.38--39 suggest the same
  in a similar context:
  \textit{devī uvāca |
  svamāṃsarudhiraṃ dānaṃ dānaṃ putrakalatrayoḥ |
  kiṃ praśasyaṃ mahādeva tattvaṃ vaktum ihārhasi ||
  maheśvara uvāca |
  svamāṃsarudhiraṃ dānaṃ praśaṃsanti manīṣiṇaḥ |
  śrūyatāṃ pūrvavṛttāni saṃkṣipya kathayāmy aham ||}.
  {\rm (}`Devī spoke: Why are one's own flesh and blood and one's son and wife praised as donation, O Mahādeva?
  Tell me the truth please. Maheśvara spoke: The wise praise one's own flesh and blood as donation.
  Let's hear the old legends, I shall tell you briefly.'{\rm )}
 }}

  \maintext{kṛcchrātikṛcchraṃ naktaṃ ca taptakṛcchram ayācitam |}%

  \maintext{cāndrāyaṇaṃ parākaṃ ca tapaḥ sāṃtapanādayaḥ }||\thinspace6:27\thinspace||%
\translation{[as also] the `painful penance' and the `extremely paniful one', [eating only] at night, the `hot and painful' and [the one in which only food obtained] without solicitation [can be eaten], the \textit{cāndrāyaṇa} and \textit{parāka} penances, the `sāṃtapana,' etc. \blankfootnote{6.27 For short descriptions and the loci classici of these penances, see, e.g.,
  \mycitep{KaneHistory}{v. 4, 130--152}.
  For \textit{nakta}/\textit{naktānna} see \VSS\ 8.22 below and, e.g., \SDHS\ chapter 10, and for \textit{ayācita}, \VSS\ 8.23 below.
 }}

  \maintext{yenedaṃ tapa tapyate sumanasā saṃsāraduḥkhacchidam}%

 \nonanustubhindent \maintext{āśāpāśa vimucya nirmalamatis tyaktvā jaghanyaṃ phalam |}%

  \maintext{svargākāṅkṣyanṛpatvabhogaviṣayaṃ sarvāntikaṃ tatphalaṃ}%

 \nonanustubhindent \maintext{jantuḥ śāśvatajanmamṛtyubhavane tanniṣṭhasādhyaṃ vahet }||\thinspace6:28\thinspace||%
\translation{He who performs with a well-disposed mind this penance that puts an end to the suffering caused by transmigration {\rm (}\textit{saṃsāra}{\rm )}, abandoning the trap of hope, with a spotless mind, giving up the lowest rewards [such as] wishing for heaven, being a king and having enjoyments for the senses, will have an ultimate {\rm (}\textit{sarvāntika}{\rm )} reward. In this home of eternal births and deaths, man can bring about an accomplishment that puts an end to them. \blankfootnote{6.28 Note my emendation in \textit{pāda} a {\rm (}\textit{sumanasā} from \textit{sumanasaḥ}{\rm )} and that
  in order to restore the metre, I accepted \Ed's stem form \textit{tapa}.
 Note the stem form \textit{°pāśa} in \textit{pāda} b metri causa.
 }}
\center{\maintext{\dbldanda\thinspace iti vṛṣasārasaṃgrahe ṣaṣṭho 'dhyāyaḥ\thinspace\dbldanda}}
\translation{Here ends the sixth chapter in the \textit{Vṛṣasārasaṃgraha}.}

  \chptr{saptamo 'dhyāyaḥ}
\fancyhead[CE]{{\footnotesize\textit{Translation of chapter 7}}}%

  \trchptr{ Chapter Seven }%

  \subchptr{niyameṣu dānam {\rm {\rm (}4{\rm )}}}%

  \trsubchptr{The fourth Niyama-rule: Donation}%

  \maintext{dānāni ca tathety āhuḥ pañcadhā munibhiḥ purā |}%

  \maintext{annaṃ vastraṃ hiraṇyaṃ ca bhūmi godāna pañcamam }||\thinspace7:1\thinspace||%
\translation{In the past the wise declared that, again, there were five kinds of donation. Donation of food, clothes, gold, land and the fifth, donation of cows. \blankfootnote{7.1 \textit{tathety} in \textit{pāda} a is suspicious and my translation of it {\rm (}`again'{\rm )} is tentative and
  is supposed to refer back to the fact that all \textit{yama}s so far have been 
  devided into five types.
  Note how \textit{annaṃ}, \textit{vastraṃ}, \textit{hiraṇyaṃ} and 
  \textit{bhūmi} {\rm (}the latter treated as neuter, or given in
  stem form{\rm )} are all meant to go with °\textit{dāna} {\rm (}again, in stem form, metri causa{\rm )}.
 }}

  \subsubchptr{annadānam}%

  \trsubsubchptr{Donation of food}%

  \maintext{annāt tejaḥ smṛtiḥ prāṇaḥ annāt puṣṭir vapuḥ sukham |}%

  \maintext{annāc chrīḥ kānti vīryaṃ ca annāt sattvaṃ ca jāyate }||\thinspace7:2\thinspace||%
\translation{From food [comes] energy, memory, the vital breath, growth, body, happiness. From food arise grace and beauty, heroism, strength. \blankfootnote{7.2 Note the stem form noun \textit{kānti} metri causa in \textit{pāda} c.
 }}

  \maintext{annāj jīvanti bhūtāni annaṃ tuṣṭikaraṃ sadā |}%

  \maintext{ānnāt kāmo mado darpaḥ annāc chauryaṃ ca jāyate }||\thinspace7:3\thinspace||%
\translation{Living beings live on food. Food always satisfies. From food arise desire, rapture, pride and valour. }

  \maintext{annaṃ kṣudhātṛṣāvyādhīn sadya eva vināśayet |}%

  \maintext{annadānāc ca saubhāgyaṃ khyātiḥ kīrtiś ca jāyate }||\thinspace7:4\thinspace||%
\translation{Food drives away hunger and thirst and disease instantly. From donations of food arise happiness, fame and glory. }

  \maintext{annadaḥ prāṇadaś caiva prāṇadaś cāpi sarvadaḥ |}%

  \maintext{tasmād annasamaṃ dānaṃ na bhūtaṃ na bhaviṣyati }||\thinspace7:5\thinspace||%
\translation{He who donates food donates life. He who donates life donates everything. Therefore nothing is equal to the donation of food, nothing was, nothing will be. }

  \subsubchptr{vastradānam}%

  \trsubsubchptr{Donation of clothes}%

  \maintext{vastrābhāvān manuṣyasya śriyād api parityajet |}%

  \maintext{vastrahīno na pūjyeta bhāryāputrasakhādibhiḥ }||\thinspace7:6\thinspace||%
\translation{In the absence of [proper] clothes, a man will also lose his fortunes. A person without clothes may not be respected by his wife, son, friends etc. \blankfootnote{7.6 \textit{Pāda} b is difficult to interpret securely. I translate it as if reading
  \textit{śrīs tam api parityajet}. Consider also \BRAHMAP\ 220.139:
  \textit{vastrābhāve kriyā nāsti yajñā vedās tapāṃsi ca |
  tasmād vāsāṃsi deyāni śrāddhakāle viśeṣataḥ ||}.
 }}

  \maintext{vidyāvān sukulīno 'pi jñānavān guṇavān api |}%

  \maintext{vastrahīnaḥ parādhīnaḥ paribhūtaḥ pade pade }||\thinspace7:7\thinspace||%
\translation{Be it a learned person from a good family or an intelligent and virtuous person, anybody without clothes is subdued and humiliated on every occasion }

  \maintext{apamānam avajñāṃ ca vastrahīno hy avāpnuyāt |}%

  \maintext{jugupsati mahātmāpi sabhāstrījanasaṃsadi }||\thinspace7:8\thinspace||%
\translation{because a man without clothes receives contempt and disrespect. Even a great soul will despise [him] at the court, among women, in an assembly. \blankfootnote{7.8 The intention originally may have been this: ``Even if he is a great soul, he will be avoided...''
 }}

  \maintext{tasmād vastrapradānāni praśaṃsanti manīṣiṇaḥ |}%

  \maintext{na jīrṇaṃ sphuṭitaṃ dadyād vastraṃ kutsitam eva vā }||\thinspace7:9\thinspace||%
\translation{Therefore the wise praise donations of clothes. One should not give away old, torn or dirty clothes. }

  \maintext{navaṃ purāṇarahitaṃ mṛdu sūkṣmaṃ suśobhanam |}%

  \maintext{susaṃskṛtya pradātavyaṃ śraddhābhaktisamanvitam }||\thinspace7:10\thinspace||%
\translation{[Clothes] should be donated [only if they are] new, not worn, soft, delicate and beautiful, ornamented, and accompanied by willingness and devotion. }

  \maintext{śraddhāsattvaviśeṣeṇa deśakālavidhena ca |}%

  \maintext{pātradravyaviśeṣeṇa phalam āhuḥ pṛthak pṛthak }||\thinspace7:11\thinspace||%
\translation{They say that the reward [of donation/generosity] is in every case dependent on the particular [donor's] willingness and character, the choice of place and time, and on the particular recipient and material. \blankfootnote{7.11 It seems that \textit{vidhena ca} stands for \textit{vidhinā ca} or rather \textit{vidhānena} metri causa in \textit{pāda} b.
  CHECK also ŚDhU, and Florinda's article, etc.
 }}

  \maintext{yādṛśaṃ dīyate vastraṃ tādṛśaṃ prāpyate phalam |}%

  \maintext{jīrṇavastrapradānena jīrṇavastram avāpnuyāt |}%

  \maintext{śobhanaṃ dīyate vastraṃ śobhanaṃ vastram āpnuyāt }||\thinspace7:12\thinspace||%
\translation{The reward received will similar to the clothes donated. By donating old clothes, one would receive old clothes [as a reward]. By donating beautiful clothes, one would receive beautiful clothes [as a reward]. }

  \maintext{dadyād vastra suśobhanaṃ dvijavare kāle śubhe sādaram}%

 \nonanustubhindent \maintext{saubhāgyam atulaṃ labheta sa naro rūpaṃ tathā śobhanam |}%

  \maintext{tasmin yāti suvastrakoṭi śataśaḥ prāpnoti niḥsaṃśayam}%

 \nonanustubhindent \maintext{tasmāt tvaṃ kuru vastradānam asakṛt pāratrikotkarṣaṇam }||\thinspace7:13\thinspace||%
\translation{Should one bestow very beautiful clothes on a Brahmin at an auspicious time, respectfully, he [i.e. the donor] will receive unequalled happiness and a beautiful appearance. When he departs, he will be given hundreds of millions of items of nice clothes, no doubt about that. Therefore do donate clothes often. It is the way up to the other world. \blankfootnote{7.13 Note the stem form \textit{vastra} in \textit{pāda} a metri causa.
 `on a Brahmin' {\rm (}in \textit{pāda} a{\rm )}: literally, `on a person who is first among the twice-born'
  {\rm (}\textit{dvijavare}{\rm )}.
 The final syllable of \textit{saubhāgyam} in \textit{pāda} b counts as long by licence; see, e.g., 5.20 and 6.18b.
  This time the c\ae sura is not involved.
 In \textit{pāda} c, °\textit{koṭi} is treated as neuter or as a stem form {\rm (}metri causa{\rm )}.
 }}

  \subsubchptr{suvarṇadānam}%

  \trsubsubchptr{Donation of gold}%

  \maintext{suvarṇadānaṃ viprendra saṃkṣipya kathayāmy aham |}%

  \maintext{pavitraṃ maṅgalaṃ puṇyaṃ sarvapātakanāśanam }||\thinspace7:14\thinspace||%
\translation{O great Brahmin, now I shall teach you about the donation of gold in a concise manner. It is a pure, auspicious and meritorious [act] and it washes off all sins. }

  \maintext{dhārayet satataṃ vipra suvarṇakaṭakāṅgulim |}%

  \maintext{mucyate sarvapāpebhyo rāhuṇā candramā yathā }||\thinspace7:15\thinspace||%
\translation{Should one hand over [to someone] a golden bracelet or ring, O Brahmin, he will be freed of all sins, just as the moon is freed from [the demon] Rāhu [after an eclipse]. \blankfootnote{7.15 I suspect that \textit{aṅguli} is used in \textit{pāda} b in the sense of \textit{aṅgulīya} {\rm (}`finger-ring'{\rm )}.
 }}

  \maintext{dattvā suvarṇaṃ viprebhyo devebhyaś ca dvijarṣabha |}%

  \maintext{tuṭimātre 'pi yo dadyāt sarvapāpaiḥ pramucyate }||\thinspace7:16\thinspace||%
\translation{If a person donates gold to Brahmins or gods, O excellent Brahmin, even if it is only in a minute quantity, he will be freed of all sins. \blankfootnote{7.16 The form \textit{tuṭi} as a widespread variant of \textit{truṭi}, see e.g. \verify.
 }}

  \maintext{raktimāṣakakarṣaṃ vā palārdhaṃ palam eva vā |}%

  \maintext{evam eva phalaṃvṛddhir jñeyā dānaviśeṣataḥ }||\thinspace7:17\thinspace||%
\translation{[The amount can be just] one \textit{rakti}, a \textit{māṣaka}, a \textit{karṣa}, half a \textit{pala} or a \textit{pala}: this is exactly how the increase in the [size of the corresponding] reward will be, in proportion to the properties [i.e.\ amount] of the donation. \blankfootnote{7.17 I suspect that \textit{phalaṃ vṛddhir}, or \textit{phalaṃvṛddhir}, stands for 
  \textit{phalavṛddhir} {\rm (}\textit{phalasya vṛddhiḥ}{\rm )} metri causa, meaning `the increase of the reward.'
  \textit{rakti}, \textit{māṣaka}, \textit{karṣa}, and \textit{pala} are units of weight.
 }}

  \subsubchptr{bhūmidānam}%

  \trsubsubchptr{Donation of land}%

  \maintext{sarvādhāraṃ mahīdānaṃ praśaṃsanti manīṣiṇaḥ |}%

  \maintext{annavastrahiraṇyādi sarvaṃ vai bhūmisambhavam }||\thinspace7:18\thinspace||%
\translation{The wise praise the donation of land as the basis of everything [else]. Food, clothes, gold etc., all these originate in the land. }

  \maintext{bhūmidānena viprendra sarvadānaphalaṃ labhet |}%

  \maintext{bhūmidānasamaṃ vipra yady asti vada tattvataḥ }||\thinspace7:19\thinspace||%
\translation{O Brahmin, one can obtain all the rewards of donation by donating land. If there is anything that equals the donation of land, O Brahmin, you should definitely tell me. }

  \maintext{mātṛkukṣivimuktas tu dharaṇīśaraṇo bhavet |}%

  \maintext{carācarāṇāṃ sarveṣāṃ bhūmiḥ sādhāraṇā smṛtā }||\thinspace7:20\thinspace||%
\translation{[Humans] have the earth as their abode as soon as they get out of their mother's womb. Land is said to be common to all that are mobile and immobile. \blankfootnote{7.20 I take \textit{sādhāraṇā} as one word, but it is possible that the intention of the author
  was \textit{sā dhāraṇā} in two words, in fact meaning \textit{sādhāraḥ} {\rm (}\textit{sā ādhāraḥ}, `it is the basis'{\rm )}.
 }}

  \maintext{ekahastaṃ dvihastaṃ vā pañcāśac chatam eva vā |}%

  \maintext{sahasrāyutalakṣaṃ vā bhūmidānaṃ praśasyate }||\thinspace7:21\thinspace||%
\translation{Be it [only a land of] one forearm, two forearms, fifty or a hundred, a thousand, ten thousand, a hundred thousand, donations of land are held in great esteem. }

  \maintext{ekahastāṃ ca yo bhūmiṃ dadyād dvijavarāya tu |}%

  \maintext{varṣakoṭiśataṃ divyaṃ svargaloke mahīyate }||\thinspace7:22\thinspace||%
\translation{Should he donate a piece of land of [only] one forearm to a Brahmin, he will enjoy a billion divine years in heaven. }

  \maintext{evaṃ bahuṣu hasteṣu guṇāguṇi phalaṃ smṛtam |}%

  \maintext{śraddhādhikaṃ phalaṃ dānaṃ kathitaṃ te dvijottama }||\thinspace7:23\thinspace||%
\translation{Thus in case of [donating] many forearms [of land], the reward is said to be proportional to the properties [of the land]. O Brahmin, I have taught you about the rewards of donation that is made willingly. \blankfootnote{7.23 I think that \textit{guṇāguṇi}, or perhaps \textit{guṇaguṇi} {\rm (}which would be unmetrical, containing
  two \textit{laghu}s in both the second and third syllables of the \textit{pāda}{\rm )}, should refer to the idea
  that, e.g., the donation of a piece of land of 2 × 2 \textit{hasta}s would result in 
  2 or 4 × \textit{koṭiśata} years in heaven, \textit{guṇa} generally meaning `times.' 
  I take \textit{guṇā}° as referring to the size of the land donated, and °\textit{guṇi}[\textit{n}] as 
  `amounting to that many times,' but this is only a guess, 
  and it would need to be supported by some similar passage, other than 7.17 above.
 
 
  I suspect that \textit{pāda} c is an awkward attempt at saying \textit{śraddhādhikadāna{\rm (}sya{\rm )} phalaṃ}.
 }}

  \maintext{jāmadagnyena rāmeṇa bhūmiṃ dattvā dvijāya vai |}%

  \maintext{āyur akṣayam āptaṃ tu ihaiva ca dvijottama }||\thinspace7:24\thinspace||%
\translation{[Paraśu]rāma, the son of Jamadagni, having donated land to the Brahmin [Kaśyapa], obtained eternal life in this very world, O excellent Brahmin. \blankfootnote{7.24 See a summary of the corresponding episode \verify\ in the \MBH\ in 
  \mycitep{PuranicEnc}{570--571}, s.v. Paraśurāma:
  `To atone for the sin of slaughtering even
  innocent Kṣatriyas, Paraśurāma gave away all his
  riches as gifts to brahmins. He invited all the brahmins
  to Samantapañcaka and conducted a great Yāga there.
  The chief Ṛtvik {\rm (}officiating priest{\rm )} of the Yāga was
  the sage Kaśyapa and Paraśurāma gave all the lands
  he conquered till that time to Kaśyapa. Then a platform 
  of gold ten yards long and nine yards wide was
  made and Kaśyapa was installed there and worshipped.
  After the worship was over according to the instructions
  from Kaśyapa the gold platform was cut into pieces
  and the gold pieces were offered to brahmins.
 
  When Kaśyapa got all the lands from Paraśurāma he
  said thus:---``Oh Rāma, you have given me all your
  land and it is not now proper for you to live in my
  soil. You can go to the south and live somewhere on
  the shores of the ocean there.'' Paraśurāma walked
  south and requested the ocean to give him some land to
  live.'
 Note that without applying the muta cum liquida licence {\rm (}\textit{ca dvi}°{\rm )}, \textit{pāda} d would be iambic and thus
  metrically problematic.
 }}

  \subsubchptr{godānam}%

  \trsubsubchptr{Donation of cows}%

  \maintext{hemaśṛṅgāṃ raupyakhurāṃ cailaghaṇṭāṃ dvijottama |}%

  \maintext{viprāya vedaviduṣe dattvānantaphalaṃ smṛtam }||\thinspace7:25\thinspace||%
\translation{[A cow] with golden horns, silver hooves, garment and bell, O Brahmin, when given to a Veda-knowing Brahmin, [produces] rewards that are said to be endless. }

  \subsubchptr{dānapraśaṃsā}%

  \trsubsubchptr{Praise of donation}%

  \maintext{dānābhyāsarataḥ pravartanabhavāṃ śakyānurūpaṃ sadā}%

 \nonanustubhindent \maintext{annaṃ vastrahiraṇyaraupyam udakaṃ gāvas tilān medinīm |}%

  \maintext{dadyāt pādukachattrapīṭhakalaśaṃ pātrādyam anyac ca vā}%

 \nonanustubhindent \maintext{śraddhādānam abhinnarāgavadanaṃ kṛtvā mano nirmalam }||\thinspace7:26\thinspace||%
\translation{Always rejoicing in the practice of giving, \dots, as far as one's capacities go, one should give food, clothes, gold and silver, water, cows, sesamum seeds, land, sandals, parasols, seats, jars, cups or anything else. Making the [deed of] giving willingly {\rm (}\textit{śraddhādāna}{\rm )} something done with an unconditioned affection {\rm (}\textit{rāga}{\rm )} and reverence {\rm (}\textit{vadana}{\rm )}, one's mind [becomes] spotless. \blankfootnote{7.26 I am unable to interpret \textit{pravartanabhavāṃ} in \textit{pāda} a and
  I suspect that \textit{śakyānurūpaṃ} in the same \textit{pāda} stands for \textit{śaktyanurūpaṃ}.
 }}

  \maintext{dānād eva yaśaḥ śriyaḥ sukhakarāḥ khyātim atulyāṃ labhet}%

 \nonanustubhindent \maintext{dānād eva nigarhaṇaṃ ripugaṇe ānandadaṃ saukhyadam |}%

  \maintext{dānād ūrjayatā prasādam atulaṃ saubhāgya dānāl labhet}%

 \nonanustubhindent \maintext{dānād eva anantabhoga niyataṃ svargaṃ ca tasmād bhavet }||\thinspace7:27\thinspace||%
\translation{Glory and fortune that makes us happy come about only by donations, and one can gain unequalled fame. Only from donations will reproach [exercised by] the enemy [turn into] pleasure and happiness. Vigour and unequalled graciousness come from donation. One can reach happiness thought donations. Endless enjoyments surely come only from donations, and heaven is [reached] also because of it. \blankfootnote{7.27 I suspect that \textit{khyātiś ca tulyaṃ} in the MSS stands for \textit{khyātim atulyāṃ} {\rm (}`and unequalled fame'{\rm )} and
  that it is not a clumsy attempt to restore the metre, but rather a later correction gone wrong.
  I have emended the phrase believing that the second {\rm (}last{\rm )} syllable of \textit{khyātim} may be treated as \textit{guru}.
  See the same licence applied in non-\textit{anuṣṭubh} verses above,
  e.g., in 5.20a, 6.18b, 7.13b {\rm (}just before \textit{atula}{\rm )}.
 I doubt if \Ed's reading in \textit{pāda} c, \textit{durjayatā} {\rm (}`invincibility'{\rm )} were better than \textit{ūrjayatā} transmitted in
  all the MSS consulted. While \textit{ūrjayatā} is still problematic, it is not inconceivable that it
  stands for \textit{ūrjatā} meaning most probably `being powerful, strength, vigour.' Also, note here
  the stem form noun \textit{saubhāgya} metri causa.
 Note \textit{svargaṃ} as a neuter noun, and the stem form °\textit{bhoga} metri causa in \textit{pāda} d. 
  The lack of sandhi between \textit{eva} and \textit{ananta}° helps restore the metre.
 }}

  \maintext{dānād eva ca śakralokasakalaṃ dānāj janānandanam}%

 \nonanustubhindent \maintext{dānād eva mahīṃ samasta bubhuje samrāḍ mahīmaṇḍale |}%

  \maintext{dānād eva surūpayonisubhagaś candrānano vīkṣyate}%

 \nonanustubhindent \maintext{dānād eva anekasambhavasukhaṃ prāpnoti niḥsaṃśayam }||\thinspace7:28\thinspace||%
\translation{The whole world of Śakra [i.e. Indra can be taken as one's possession] by donations only. Donations make people happy. Supreme ruler[s] enjoyed all the land in the world only because of donations. Skanda {\rm (}\textit{candrānana}{\rm )} appears as handsome and fortunate, with a [good] family[? \verify] only because of donations. One can reach happiness that lasts countless births only through donations, there is no doubt about that. \blankfootnote{7.28 °\textit{lokasakalaṃ} in \textit{pāda} a is suspect and \Ed's silent emendation {\rm (}°\textit{lokam atulaṃ}{\rm )} is
  not without reason.
 I translate \textit{pāda} b as a general statement although \textit{samrāṭ} may 
  refer to a specific figure and story in mythology. The perfect form \textit{bubhuje}, and 
  the next \textit{pāda}, at least point to this direction.
 }}
\center{\maintext{\dbldanda\thinspace iti vṛṣasārasaṃgrahe dānapraśaṃsādhyāyaḥ saptamaḥ\thinspace\dbldanda}}
\translation{Here ends the seventh chapter in the \textit{Vṛṣasārasaṃgraha} called Praise of Donations.}

  \chptr{aṣṭamo 'dhyāyaḥ}
\fancyhead[CE]{{\footnotesize\textit{Translation of chapter 8}}}%

  \trchptr{ Chapter Eight}%

  \subchptr{niyameṣu svādhyāyaḥ {\rm {\rm (}5{\rm )}}}%

  \trsubchptr{The fifth Niyama-rule: Study}%

  \maintext{pañcasvādhyāyanaṃ kāryam ihāmutra sukhārthinā |}%

  \maintext{śaivaṃ sāṃkhyaṃ purāṇaṃ ca smārtaṃ bhāratasaṃhitām }||\thinspace8:1\thinspace||%
\translation{Five kinds of study are to be pursued by those who wish to be happy in this life and in the other: [one has to study the] Śaiva [teachings], Sāṃkhya [philosophy], the Purāṇa[s], the Smārta [tradition] and the \textit{Bhāratasaṃhitā} [i.e. the \textit{Mahābhārata}]. \blankfootnote{8.1 Note the accusative ending of \textit{°saṃhitām} after a list consisting of words probably in the
  nominative. One may correct it to \textit{°saṃhitā} or rather 
  supply an active verb such as \textit{adhigacchet} {\rm (}`he should study'{\rm )}.
 }}

  \maintext{śaivatattvaṃ vicinteta śaivapāśupatadvaye |}%

  \maintext{atra vistarataḥ proktaṃ tattvasārasamuccayam }||\thinspace8:2\thinspace||%
\translation{He should reflect on the Śaiva truth in both Śaiva and Pāśupata [teachings]. In those teachings the whole essence of truth is taught extensively. \blankfootnote{8.2 Note that \textit{śaivatattvaṃ} in \textit{pāda} a is the result of a conjecture and that the reading \textit{śaivapāśupatadvaye} 
  in \textit{pāda} b is based on one single manuscript {\rm (}\msP{\rm )}. In spite of these uncertainties, 
  I think that this form of the current half-verse is the only one that yields the appropriate meaning.
 }}

  \maintext{saṃkhyātattvaṃ tu sāṃkhyeṣu boddhavyaṃ tattvacintakaiḥ |}%

  \maintext{pañcatattvavibhāgena kīrtitāni maharṣibhiḥ }||\thinspace8:3\thinspace||%
\translation{Those who reflect on the truth {\rm (}\textit{tattva}{\rm )} can grasp the truth of enumeration [of ontological principles/reality levels] {\rm (}\textit{saṃkhyātattva}{\rm )} from Sāṃkhya [texts]. The great sages taught [those twenty-five] \textit{tattva}s [of Sāṃkhya] as being in groups of five. \blankfootnote{8.3 In \textit{pāda} d, \textit{kīrtitāni} picks up an implied \textit{tattvāni}.
 }}

  \maintext{purāṇeṣu mahīkoṣo vistareṇa prakīrtitaḥ |}%

  \maintext{adhordhvamadhyatiryaṃ ca yatnataḥ sampraveśayet }||\thinspace8:4\thinspace||%
\translation{In the Purāṇas it is the sheath[s] of the world that are described extensively. One can definitely enter [the realm] of the lower [world, i.e. hell], the upper [world, i.e. heaven], and middle [world, i.e. the human world], and the horizontal [world, i.e. of animals, by studying the Purāṇas]. \blankfootnote{8.4 Note that \textit{tirya} seems to be an acceptable nominal stem in this text for \textit{tiryañc}. 
  I understand the causative form \textit{sampraveśayet} as non-causative, and 
  I interpret °\textit{madhya}° as the `human world' tentatively. \Ed's silent emendation
  to \textit{samprabodhayet} is understandable since to `enter' these worlds 
  {\rm (}especially the hells and the human world{\rm )} through the study of the Purāṇas makes little sense,
  at least when taken literally.
 }}

  \maintext{smārtaṃ varṇāśramācāraṃ dharmanyāyapravartanam |}%

  \maintext{śiṣṭācāro 'vikalpena grāhyas tatra aśaṅkitaḥ }||\thinspace8:5\thinspace||%
\translation{The Smārta [tradition] deals with the conduct of the social classes {\rm (}\textit{varṇa}{\rm )} and disciplines {\rm (}\textit{āśrama}{\rm )}, and with the procedures of Dharma and lawsuits. Good conduct is to be gathered from that [source] without hesitation, with certainty. \blankfootnote{8.5 Compare \textit{pāda} a with 3.15c.
 }}

  \maintext{itihāsam adhīyānaḥ sarvajñaḥ sa naro bhavet |}%

  \maintext{dharmārthakāmamokṣeṣu saṃśayas tena chidyate }||\thinspace8:6\thinspace||%
\translation{A man who studies the epics {\rm (}\textit{itihāsa}{\rm )} will become omniscient. [All his] doubts about Dharma, Artha, Kāma and Mokṣa will be eliminated. }

  \subchptr{niyameṣv upasthanigrahaḥ {\rm {\rm (}6{\rm )}}}%

  \trsubchptr{The sixth Niyama-rule: Sexual restraint}%

  \maintext{śṛṇuṣvāvahito vipra pañcopasthavinigraham |}%

  \maintext{striyo vā garhitotsargaḥ svayaṃmuktiś ca kīrtyate |}%

  \maintext{svapnopaghātaṃ viprendra divāsvapnaṃ ca pañcamaḥ }||\thinspace8:7\thinspace||%
\translation{Listen with great attention, O Brahmin, to the five [spheres of] sexual restraint. Women, forbidden ejaculation, and masturbation are mentioned [in this context, as well as] offence while sleeping, O Brahmin, and sleeping by day as the fifth. }

  \subsubchptr{striyaḥ}%

  \trsubsubchptr{Women}%

  \maintext{agamyā strī divā parve dharmapatny api vā bhavet |}%

  \maintext{viruddhastrīṃ na seveta varṇabhraṣṭādhikāsu ca }||\thinspace8:8\thinspace||%
\translation{A woman is not to be approached sexually in daytime and on the four days of the changes of the Moon {\rm (}\textit{parvan}{\rm )}, even if she is one's lawful wife. One should not have sex with a woman who is taboo or with one of those who have lost their class {\rm (}\textit{varṇa}{\rm )} or are [of a] superior [\textit{varṇa} than oneself]. \blankfootnote{8.8 Understand \textit{parve} as \textit{parvani} {\rm (}thematisation of the stem in \textit{-an}{\rm )}.
 The nominative °\textit{strī} in \textit{pāda} c, now corrected to the accusative, may 
  be the result of an eyeskip to \textit{strī} in \textit{pāda} a.
 }}

  \subsubchptr{garhitotsargaḥ}%

  \trsubsubchptr{Forbidden ejaculation}%

  \maintext{ajameṣagavādīnāṃ vaḍavāmahiṣīṣu ca |}%

  \maintext{garhitotsargam ity etad yatnena parivarjayet }||\thinspace8:9\thinspace||%
\translation{Intercourse with goats, sheep, cows, mares, buffalo-cows is called forbidden ejaculation, which is to be avoided at all cost. \blankfootnote{8.9 Understand \textit{°ādīnāṃ} in \textit{pāda} a as standing for the locative case.
 Understand \textit{°sargam} as neuter nominative {\rm (}instead of \textit{°sargaḥ}{\rm )} or alternatively
  understand \textit{pāda} c with a hiatus bridge: \textit{garhitotsarga-m-ity etad}.
 }}

  \subsubchptr{svayaṃmuktiḥ}%

  \trsubsubchptr{Masturbation}%

  \maintext{ayonyakaṣaṇā vāpi apānakaṣaṇāpi vā |}%

  \maintext{svayaṃmuktir iyaṃ jñeyā tasmāt tāṃ parivarjayet }||\thinspace8:10\thinspace||%
\translation{Rubbing himself against something else than a female sexual organ or rubbing his anus, are called masturbation, therefore these are to be avoided. \blankfootnote{8.10 The conjecture that changes \textit{anyonya°} to \textit{ayonya°} in \textit{pāda} a involves 
  minimal intervention and makes the sentence much more meaningful than the 
  version transmitted. Also consider \textit{ayoni°}.
 The variant \textit{strī} for \textit{tāṃ} in \textit{pāda} d in the \Ed\ may be one example of the numerous
  silent intervention made by Naraharināth in his edition.
 }}

  \subsubchptr{svapnaghātam}%

  \trsubsubchptr{Offence while sleeping}%

  \maintext{svapnaghātaṃ dvijaśreṣṭha aniṣṭaṃ paṇḍitaiḥ sadā |}%

  \maintext{svapne strīṣu ramante ca retaḥ prakṣarate tataḥ }||\thinspace8:11\thinspace||%
\translation{Offence while sleeping, O best of Brahmins, has always been [considered] undesirable by the learned. [If] one enjoys women while sleeping, his semen will issue. }

  \subsubchptr{divāsvapnam}%

  \trsubsubchptr{Sleeping by day}%

  \maintext{divāśayaṃ na kartavyaṃ nityaṃ dharmapareṇa tu | }%

  \maintext{svargamārgārgalā hy etāḥ striyo nāma prakīrtitāḥ }||\thinspace8:12\thinspace||%
\translation{Sleeping by day should always be avoided by those who are intent on Dharma. These women are called `the bolts [that block the gate to] the path to heaven.' \blankfootnote{8.12 It is not crystal clear why `sleeping by day' should count as
  one of the offences against sexual restraint. Even if we translated \textit{divāsvapna} and
  \textit{divāśaya} as `daydreaming,' this category would stil seem out of context.
 \textit{Pāda}s cd are clumsy and out of context. They would fit verse 8.8 better.
 }}

  \subchptr{niyameṣu vratapañcakam {\rm {\rm (}7{\rm )}}}%

  \trsubchptr{The seventh Niyama-rule: religious observances}%

  \maintext{mārjārakabakaśvānagomahīvratapañcakam |}%

  \subsubchptr{mārjārakavratam}%

  \trsubsubchptr{The Cat Vow}%

  \maintext{svaviṣṭhamūtraṃ bhūmīṣu chādayed dvijasattama |}%

  \maintext{sūryasomānumodanti mārjāravratikeṣu ca }||\thinspace8:13\thinspace||%
\translation{[Hear about] the five religious observances [called] the cat, the crane, the dog, the cow, and the earth. He buries his own urine and faeces in the ground, O truest Brahmin. He rejoices [seeing] the sun and the moon when performing the cat observance. \blankfootnote{8.13 Note \textit{°viṣṭha°} for \textit{viṣṭhā} metri causa in \textit{pāda} c {\rm (}\textit{ma-vipulā}{\rm )}.
  Alternatively, read \textit{svaviṣṭhāmūtra bhūmīṣu} {\rm (}\textit{pathyā}{\rm )}.
 Note the stem form \textit{sūryasoma} for \textit{sūryasomau} in \textit{pāda} e. 
  It is not entirely clear why cats would rejoice seeing the Sun and the Moon.
  Perhaps this remark refers to the fact that cats can be active both
  in the daytime and at night.
 }}

  \subsubchptr{bakavratam}%

  \trsubsubchptr{The Crane Vow}%

  \maintext{bakavac cendriyagrāmaṃ suniyamya tapodhana |}%

  \maintext{sādhayec ca manastuṣṭiṃ mokṣasādhanatatparaḥ }||\thinspace8:14\thinspace||%
\translation{O great ascetic, one should suppress all his senses like a crane, and should cultivate the peace of the mind, focusing on achieving liberation. \blankfootnote{8.14 Cranes are compared to ascetics here probably because of the similarity of
  their posture when relaxing standing on one leg to ascetics performing penance 
  standing on one leg {\rm (}such as the ascetic, and a cat, depicted on the famous relief in Mahabalipuram{\rm )}. 
 }}

  \subsubchptr{śvānavratam}%

  \trsubsubchptr{The Dog Vow}%

  \maintext{mūtraviṣṭhe na bhūmīṣu kurute śvānadaḥ sadā |}%

  \maintext{tuṣyate bhagavān śarvaḥ śvānavratacaro yadi }||\thinspace8:15\thinspace||%
\translation{He does not bury his urine and f\ae ces in the ground, and he barks constantly. Lord Śarva [i.e. Śiva] is satisfied when one practises the dog observance. \blankfootnote{8.15 A possible expanation for Śiva being satisfied with an ascetic practising 
  this observance is that Śiva's Bhairava form often has a dog as his mount. See, e.g.,
  \mycitep{BakkerWorld2014}{232--233} on a 5-6th-century image of Bhairava and a dog carved in
  rock at Muṇḍeśvarı̄ Hill not far from Vārāṇasī, and Mirnig 2013, 334 ?\verify 
  This observance has ancient roots. Its practitioner, the \textit{kukkuravatika}
  appears in \textit{Majjhimanikāya} 2.1.7, in the \textit{Kukkuravatiyasutta}, 
  alongside with a practitioner of the \textit{govrata} {\rm (}\textit{govatika}{\rm )}, an observance
  that comes up in the next verse in the \VSS:
  \textit{evaṃ me sutaṃ. ekaṃ samayaṃ bhagavā koliyesu viharati haliddavasanaṃ nāma koliyānaṃ nigamo.
  atha kho puṇṇo ca koliyaputto govatiko, acelo ca seniyo kukkuravatiko yena bhagavā tenupasaṅkamiṃsu...}
  See \mycitep{AcharyaBull}{127--128}. Acharya summarises the \textit{Kukkuravatiyasutta} 
  thus:
 
  `The \textit{Kukkuravatiyasutta} from the \textit{Majjhimanikāya} {\rm (}II.1.7{\rm )} 
  presents a \textit{govatika} together with a \textit{kukkuravatika}. They are observing 
  their vows, and have adopted the behaviour of a bull and a dog respectively. 
  The Buddha tells them that as they are cultivating bullness and dogness, 
  the state of mind of these animals, they will go to hell or become reborn as animal.
  They are alarmed at this and take refuge in the Buddha.'
 }}

  \subsubchptr{govratam}%

  \trsubsubchptr{The Cow Vow}%

  \maintext{mūtravarco na rudhyeta sadā govratiko naraḥ |}%

  \maintext{bhīmas tuṣṭikaraś caiva purāṇeṣu nigadyate }||\thinspace8:16\thinspace||%
\translation{A person practising the Cow Vow should never hold back his urine and f\ae ces. This is a terrifying [observance] that gives satisfaction, [as] stated in the Purāṇas. \blankfootnote{8.16 I prefer reading \textit{bhīma} and \textit{tuṣṭi°} as two separate words, the first
  one either in stem form {\rm (}\msCa\msCb\msNa\msNc\msP{\rm )} or as \textit{bhīmas} {\rm (}\msCc\msNb\Ed{\rm )}
  or \textit{bhīmaṃ} {\rm (}\eme{\rm )},
  to reading these two words as a compound because
  of the following \textit{caiva}.
  I suspect that both \textit{bhīma} and \textit{tuṣṭikara} refer to the \textit{vrata}, rather than its practitioner,
  but I have not emended \textit{bhīmas tuṣṭikaraś} to \textit{bhīmaṃ tuṣṭikaraṃ}
  because \textit{vrata} appears as a masculine noun, e.g., in 8.17d below.
 
  \mycite{AcharyaBull} gives a number of significant clues about the origins 
  of this observance. After exploring its links to Pāśupatas, \mycitep{AcharyaBull}{116--118},
  quotes \textit{Jaiminīyabrāhmaṇa} 2.113, which contains the phrase 
  \textit{yatra yatrainaṃ viṣṭhā vindet tat tad vitiṣṭheta}, in Acharya's translation:
  `Wherever he feels the urge to evacuate f\ae ces, right there he should evacuate.'
  This is an instruction in a Vedic text that is close to what the \VSS\ teaches above.
  Incidentaly, the \textit{Jaiminīyabrāhmaṇa} adds:
  \textit{tena haitenottaravayasy e} [\textit{va}] \textit{yajeta}
  {\rm (}translated in \mycitep{AcharyaBull}{118} as: 
  `One should perform this [sacrifice] in the final years of one's life'{\rm )}.
 }}

  \subsubchptr{mahīvratam}%

  \trsubsubchptr{The Earth Vow}%

  \maintext{kuddālair dārayanto 'pi kīlakoṭiśataiś citaḥ |}%

  \maintext{kṣamate pṛthivī devī evam eva mahīvrataḥ }||\thinspace8:17\thinspace||%
\translation{Splitting [the earth] with spades and piling up [the soil] with wedges: Goddess Earth bears [this] patiently. This is exactly how one can practise the earth vow. \blankfootnote{8.17 While \textit{dārayanto} as an active participle in the masculine nominative is acceptable
  as an irregular form, the precise interpretation of \textit{pāda}s a and b is still problematic therefore
  my translation of this verse is tentative and the description seems too condensed to be
  intelligible. 
 
  In \BHAVP\ 4.121, called `The Description of eighty-five observances' {\rm (}\textit{vratapañcāśītivarṇanam}{\rm )},
  we find this on \textit{mahīvrata}:
  \textit{dadyāt triṃśatpalād ūrdhvaṃ mahīṃ kṛtvā tu kāṃcanīm |
  kulācalādrisahitāṃ tilavastrasamanvitām || 152 || 
  tiladroṇopari gatāṃ brāhmaṇāya kuṭuṃbine | 
  dinaṃ payovratas tiṣṭhed rudraloke mahīyate || 153 || 
  etan mahīvrataṃ proktaṃ saptakalpānuvartakam |}.
 
  A tentative translation of this passage would go as follows: `One should donate a golden [model of] Earth
  that weighs more than thirty \textit{pala}s {\rm (}appr. one kilogram{\rm )}, showing the chief mountain-ranges,
  together with [donations of] sesamum seeds and clothes, the sesamum seeds [weighing] more than
  a \textit{droṇa} {\rm (}appr. ten kilograms{\rm )}, to a householder Brāhmin. One should keep the milk-observance 
  [i.e. subsisting on nothing but milk] for one day, and one will have fun in Rudraloka.
  This is called the Earth Observance whose range is seven \ae ons.' {\rm (}I take the values for weights
  from \mycitep{OlivelleManu}{997}.{\rm )} 
  The descriptions of the \textit{dharāvrata} and the \textit{śubhadvādaśī} observance in 
  \mycitep{KaneHistory}{v. 5, 321 and 429} are similar.
  Unfortunately, the \VSS's \textit{mahīvrata} seems different, and more in line with 
  the somewhat transgressive and wild, perhaps Pāśupata-oriented, nature of the
  four preceding observances.
 }}

  \maintext{vratapañcakam ity etad yaś careta jitendriyaḥ |}%

  \maintext{sa cottamam idaṃ lokaṃ prāpnoti na ca saṃśayaḥ }||\thinspace8:18\thinspace||%
\translation{He who practises these five religious observances with his senses subdued will, without doubt, reach this superior world [i.e. heaven?]. \blankfootnote{8.18 Note the neuter \textit{idaṃ} picking up the normally masculine \textit{lokaṃ} in \textit{pāda} c,
  and that the same \textit{idaṃ} would make more sense if the interlocutor were a deity, e.g.,
  Śiva, referring to his abode, and not Anarthayajña, the ascetic.
 }}

  \subchptr{niyameṣv upavāsaḥ {\rm {\rm (}8{\rm )}}}%

  \trsubchptr{The eighth Niyama-rule: Eating restrictions}%

  \maintext{śeṣānnam antarānnaṃ ca naktāyācitam eva ca |}%

  \maintext{upavāsaṃ ca pañcaitat kathayiṣyāmi tac chṛṇu }||\thinspace8:19\thinspace||%
\translation{Eating leftovers, [not] eating in-between [breakfast and dinner], eating [only] at night, eating food obtained without solicitation, and fasting: listen, I shall teach you these five. \blankfootnote{8.19 Note how this category of \textit{niyama}-rules was called \textit{upavāsa} {\rm (}`fasting'{\rm )} in 5.3c above but how in fact
  \textit{upavāsa} is just the fifth subcategory withing this group of eating restrictions.
 }}

  \subsubchptr{śeṣānnam}%

  \trsubsubchptr{Eating leftovers}%

  \maintext{vaiśvadevātithiśeṣaṃ pitṛśeṣaṃ ca yad bhavet |}%

  \maintext{bhṛtyaputrakalatrebhyaḥ śeṣāśī vighasāśanaḥ }||\thinspace8:20\thinspace||%
\translation{[He who eats] the leftovers belonging to all the gods, to guests, and to the ancestors, he who eats the leftovers {\rm (}śeṣāśin{\rm )} of servants, sons and wives, is [called in general] the one who consumes the remains of food {\rm (}\textit{vighasāśana}{\rm )}. }

  \subsubchptr{antarānnam}%

  \trsubsubchptr{{\rm [}Not{\rm ]} eating in-between breakfast and dinner}%

  \maintext{antarā prātarāśī ca sāyamāśī tathaiva ca |}%

  \maintext{sadopavāsī bhavati yo na bhuṅkte kadācana }||\thinspace8:21\thinspace||%
\translation{He will be regarded as one that is always fasting if he never eats between breakfast and dinner. \blankfootnote{8.21 My translation here follows the parallel verse in the \MBH\ and 
  is based on that of Kisari Mohan Ganguli {\rm (}\mycite{GanguliMBh}{\rm )}. 
  The syntax of the version here in the \VSS\ is less
  smooth than that in the \MBH, and the \VSS's reading \textit{prāntarāśī} 
  definitely required an emendation.
 }}

  \subsubchptr{naktānnam}%

  \trsubsubchptr{Eating {\rm [}only{\rm ]} at night}%

  \maintext{na divā bhojanaṃ kāryaṃ rātrau naiva ca bhojayet |}%

  \maintext{naktavele ca bhoktavyaṃ naktadharmaṃ samīhatā }||\thinspace8:22\thinspace||%
\translation{One should eat neither in the daytime nor in the evening, and should eat [only] at midnight if he wishes to follow the practice of [eating only at] night {\rm (}\textit{naktadharma}{\rm )}. \blankfootnote{8.22 Note \textit{°vele} for \textit{°velāyāṃ} in \textit{pāda} c.
 }}

  \subsubchptr{ayācitānnam}%

  \trsubsubchptr{Eating food obtained without solicitation}%

  \maintext{anārambhya ya āhāraṃ kuryān nityam ayācitam |}%

  \maintext{parair dattaṃ tu yo bhuṅkte tam ayācitam ucyate }||\thinspace8:23\thinspace||%
\translation{He who consumes food only without initiating [the donation], without asking for it, and eats [only] that which has been given by others is called [one who eats] unsolicited [food]. \blankfootnote{8.23 \textit{anārambhasya} {\rm (}`of someone who has not yet started/initiated'{\rm )} in \textit{pāda} a seems suspect, hence
  my conjecture {\rm (}\textit{anārambhya ya}{\rm )} that involves mininal intervention and yields better sense.
  I take \textit{ayācitam} in \textit{pāda} b adverbially.
 }}

  \subsubchptr{upavāsaḥ}%

  \trsubsubchptr{Fasting}%

  \maintext{bhakṣyaṃ bhojyaṃ ca lehyaṃ ca coṣyaṃ peyaṃ ca pañcamam |}%

  \maintext{na kāṅkṣen nopayuñjīta upavāsaḥ sa ucyate }||\thinspace8:24\thinspace||%
\translation{Chewable and unchewable food, food to be sipped or sucked or drunk, as the fifth [category]: if one does not long for and does not consume [any of the above], that is called fasting {\rm (}\textit{upavāsa}{\rm )}. \blankfootnote{8.24 For a detailed discussion of the categories \textit{bhakṣya, bhojya, lehya} and \textit{coṣya},
  see \mycitep{KafleNisvasaBook}{245, n. 534}. 
  See also \SDHU\ 8.13: %
  \textit{bhakṣyaṃ bhojyaṃ ca peyaṃ ca lehyaṃ coṣyaṃ ca picchilam} |
  \textit{iti bhedāḥ ṣaḍannasya madhurādyāś ca ṣaḍguṇāḥ} ||
 }}

  \subchptr{niyameṣu maunavratam {\rm {\rm (}9{\rm )}}}%

  \trsubchptr{The ninth Niyama-rule: Silence}%

  \maintext{mithyāpiśunapāruṣyatīkṣṇavāg apralāpanam |}%

  \maintext{maunapañcakam ity etad dhārayen niyatavrataḥ }||\thinspace8:25\thinspace||%
\translation{One who is disciplined in religious observances should keep taciturnity in [i.e. should avoid] these five: deceitful speech, envious speech, insult, harsh speech and bragging. \blankfootnote{8.25 \textit{pāruṣya} seems to be the good reading in \textit{pāda} a, as opposed
  to \msCc's \textit{saṃbhinnā}, because in the following 
  a short section on the category of \textit{pāruṣya} is coming up {\rm (}in 8.28{\rm )}.
  As far as the readings \textit{spṛṣṭavāg} and \textit{pṛṣṭavāg} are concerned, I suppose 
  \textit{pṛṣṭavāg} is not inconceivable {\rm (}as suggested by Judit Törzsök{\rm )}, 
  for in 8.29 it is, in a way, questions that are given as relevant examples. 
  Nevertheless I conjectured \textit{tīkṣṇavāg} here, relying on the same verse, 8.29.
 }}

  \subsubchptr{mithyāvacanam}%

  \trsubsubchptr{Deceitful speech}%

  \maintext{asambhūtam adṛṣṭaṃ ca dharmāc cāpi bahiṣkṛtam |}%

  \maintext{anarthāpriyavākyaṃ yat tan mithyāvacanaṃ smṛtam }||\thinspace8:26\thinspace||%
\translation{Fictitious [speech], [speech about] unknown [things], [speech about things] outside the range of Dharma, meaningless and unfriendly speech: these are called deceitful speech. }

  \subsubchptr{piśunaḥ}%

  \trsubsubchptr{Envy}%

  \maintext{paraśrīṃ nābhinandanti parasyaiśvaryam eva ca |}%

  \maintext{aniṣṭadarśanākāṅkṣī piśunaḥ samudāhṛtaḥ }||\thinspace8:27\thinspace||%
\translation{One who does not rejoice in others' fortune or in others' power, one who would like to see something disadvantageous [for others] is called envious. }

  \subsubchptr{pāruṣyam}%

  \trsubsubchptr{Insult}%

  \maintext{mṛtā mātā pitā caiva hānisthānaṃ kathaṃ bhavet |}%

  \maintext{bhuṅkṣva kāmam amṛṣṭānāṃ pāruṣyaṃ samudāhṛtam }||\thinspace8:28\thinspace||%
\translation{`[Your] mother and father are dead. How can this be a condition for deficit? Enjoy the love of unclean women!' [These are] called insult. \blankfootnote{8.28 My translation of \textit{pāda} b, or rather of the whole verse, is tentative, and to make
  sense of \textit{pāda} a, I have chosen a reading {\rm (}\textit{mṛtā}{\rm )} that is not well attested.
  I am not at all certain that I understand what these abusive words imply.
 }}

  \subsubchptr{tīkṣṇavāk}%

  \trsubsubchptr{Verbal abuse}%

  \maintext{hṛdi na sphuṭase mūḍha śiro vā na vidāryase |}%

  \maintext{evamādīny anekāni tīkṣṇavādī sa ucyate }||\thinspace8:29\thinspace||%
\translation{`Won't you burst in your heart, stupid? [Why] don't you break your head?' [If one utters] these or similar [curses], he is said to be using verbal abuse. }

  \subsubchptr{asatpralāpaḥ}%

  \trsubsubchptr{Bragging}%

  \maintext{dyūtabhojanayuddhaṃ ca madyastrīkatham eva ca |}%

  \maintext{asatpralāpaḥ pañcaitat kīrtitaṃ me dvijottama }||\thinspace8:30\thinspace||%
\translation{Relating fancy stories about gambling, enjoyments, fights, drinking and women are the five types of bragging. [Thus] have I taught [reasons for taciturnity], O excellent Brahmin. \blankfootnote{8.30 I take \textit{°katham} in \textit{pāda} b as an alternative nominative form of \textit{°kathā} metri causa and as 
  belonging to all the categories here thus: \textit{dyūtakathā, bhojanakathā, yuddhakathā, madyakathā,
  strīkathā}.
 Note the use of the singular next to a number in \textit{pāda} c and understand \textit{me} in \textit{pāda} d as \textit{mayā}. 
  The latter usage appears in the epics, see \mycitep{OberliesEpicSkt}{102--103 {\rm (}4.1.3{\rm )}}.
 }}

  \maintext{maunam eva sadā kāryaṃ vākyasaubhāgyam icchatā |}%

  \maintext{apāruṣyam asambhinnaṃ vākyaṃ satyam udīrayet }||\thinspace8:31\thinspace||%
\translation{Taciturnity should always be practised by those who long for the beauty of speech. One should speak true words without insult and idle talk. }

  \maintext{yas tu maunasya no kartā dūṣitaḥ sa kulādhamaḥ |}%

  \maintext{janme janme ca durgandho mūkaś caivopajāyate }||\thinspace8:32\thinspace||%
\translation{He who does not practise taciturnity is defiled and he is the black sheep of the family. For a number of rebirths, [his mouth] will stink and he will become mute. \blankfootnote{8.32 The form \textit{janme} for \textit{janmani} often occurs in Śaiva tantras as a tipically Aiśa phenomenon.
  See, e.g., \NISVNAYA\ 1.86a {\rm (}\textit{janme janme vimūḍhātmā}, see \mycitep{NisvasaGoodall}{114 and 191}{\rm )}
  and \BRAYA\ 45.8b, 452a, 559a {\rm (}the last reads \textit{janme janme tu yā jātiṃ}, 
  see \mycitep{KissBraYa}{83 and 128ff}{\rm )}.
  Thematisation of stems in \textit{-an} occurs in the epics, see 
  \mycitep{OberliesEpicSkt}{88 {\rm (}3.10{\rm )}}.
 }}

  \maintext{tasmān maunavrataṃ sadaiva sudṛḍhaṃ kurvīta yo niścitaṃ}%

 \nonanustubhindent \maintext{vācā tasya alaṅghyatā ca bhavati sarvāṃ sabhāṃ nandati |}%

  \maintext{vaktrāc cotpalagandham asya satataṃ vāyanti gandhotkaṭāḥ}%

 \nonanustubhindent \maintext{śāstrānekasahasraśo giri naraḥ proccāryate nirmalam }||\thinspace8:33\thinspace||%
\translation{Therefore the speech of a person who always keeps the observance of taciturnity firmly, with resolution, will be impossible to ignore and it will make the community rejoice. The fragrance of lotuses and [other kinds of] rich fragrances will blow from his mouth. Thousands of faultless \textit{śāstra}s will be declared in the words of this person. \blankfootnote{8.33 To make sense of \textit{pāda} d, we are forced to take \textit{śāstra} as a stem form noun and 
  \textit{naraḥ} as a {\rm (}regular{\rm )} genitive from \textit{nṛ}. {\rm (}I thank Judit Törzsök for this interpretation.{\rm )}
  Another way of understanding the beginning of this sentence would be to separate \textit{śāstrāneka°} as
  \textit{śāstrān eka°}, treating the word \textit{śāstra} as masculine.
 }}

  \subchptr{niyameṣu snānam {\rm {\rm (}10{\rm )}}}%

  \trsubchptr{The tenth Niyama-rule: Bathing}%

  \maintext{snānaṃ pañcavidhaṃ caiva pravakṣyāmi yathātatham |}%

  \maintext{āgneyaṃ vāruṇaṃ brāhmyaṃ vāyavyaṃ divyam eva ca }||\thinspace8:34\thinspace||%
\translation{I shall teach you the five kinds of bathing as they really are: fire bath, water bath, Vedic bath, wind bath and divine bath. }

  \subsubchptr{āgneyaṃ snānam}%

  \trsubsubchptr{Fire bath}%

  \maintext{āgneyaṃ bhasmanā snānaṃ toyāc chataguṇaṃ phalam |}%

  \maintext{bhasmapūtaṃ pavitraṃ ca bhasma pāpapraṇāśanam }||\thinspace8:35\thinspace||%
\translation{Fire bath is [performed] with ashes. Its fruits are a hundred times bigger than [those of] a water [bath]. [Things] purified with ashes are holy. Ashes destroy sin. }

  \maintext{tasmād bhasma prayuñjīta dehināṃ tu malāpaham |}%

  \maintext{sarvaśāntikaraṃ bhasma bhasma rakṣakam uttamam }||\thinspace8:36\thinspace||%
\translation{Therefore one should use ash for it purifies humans of their defilement. Ashes yield appeasement for everyone. Ash is the ultimate protector. }

  \maintext{bhasmanā tryāyuṣaṃ kṛtvā brahmacaryavrate sthitam |}%

  \maintext{bhasmanā ṛṣayaḥ sarve pavitrīkṛtam ātmanaḥ }||\thinspace8:37\thinspace||%
\translation{Drawing [the sectarian marks on their foreheads while reciting] the Tryāyuṣa [mantra], observing chastity, all the sages purified themselves with ashes. \blankfootnote{8.37 Note \textit{tryāyuṣa} in the sense of the three \textit{puṇḍra}-lines on the
  forehead and compare with 11.28c. Understand \textit{sthitam} as 
  \textit{sthitaḥ} or rather \textit{sthitāḥ} if we are to connect this line
  to the next {\rm (}8.37cd{\rm )}.
 Understand \textit{pavitrīkṛtam} as \textit{pavitrīkṛtvantaḥ}.
 
  The reference here may be a story in which Kaśyapa and 
  other Ṛṣis are burnt to ashes, to be later reanimated by Vīrabhadra, 
  in the Śokara forest. See \PADMAP\ 5.107.1--14ff: %
  \textit{śucismitovāca~|}
  \textit{kaśyapaṃ jamadagniṃ ca devānāṃ ca purā katham~|}
  \textit{rarakṣa bhasma tad brahman samācakṣva mune mama}~||~1~||
  \textit{dadhīca uvāca~|}
  \textit{kaśyapādiyutā devāḥ pūrvam abhyāgaman girim}~|
  \textit{śokaraṃ nāma vikhyātaṃ girimadhye suśobhanam~}||~2~||
  \textit{nānāvihaṃgasaṃkīrṇaṃ nānāmunigaṇāśrayam}~| 
  \textit{vāsudevāśrayaṃ ramyam apsarogaṇasevitam}~||~3~||
  \textit{vicitravṛkṣasaṃvītaṃ sarvartukusumojjvalam}~|
  \textit{tathāvidhaṃ praviśyaite giriṃ vayam athāpare}~||~4~|| 
  \textit{stuvantaḥ keśavaṃ tatra gatāḥ sma giriśeśvaram}~|
  \textit{dṛṣṭvā tatra mahājvālāṃ praviṣṭāś ca vayaṃ ca tām~}||~5~|| 
  \textit{māmekaṃ tu tiraskṛtya hy adahad devatā munīn~}|
  \textit{māṃ dadāha tataḥ paścād bhasmībhūtā vayaṃ śubhe~}||~6~|| 
  \textit{asmān etādṛśān dṛṣṭvā vīrabhadraḥ pratāpavān~}| 
  \textit{kenāpi kāraṇenāsau gatavān parvataṃ ca tam~}||~7~|| 
  \textit{bhasmoddhūlitasarvāṅgo mastakasthaśivaḥ śuciḥ~}|
  \textit{ekākī niḥspṛhaḥ śānto hāhāśabdam athāśṛṇot~}||~8~|| 
  \textit{atha cintāparaś cāsīn mriyamāṇa śavadhvaniḥ~}| 
  \textit{śavānām iva gandhaś ca dṛśyate tannirīkṣaṇe~}||~9~|| 
  \textit{iti niścitya manasā jagāmāgnim atiprabham~}|
  \textit{sa vahnir vīrabhadraṃ ca dagdhum ārabdhavān atha~}||~10~|| 
  \textit{tṛṇāgnir iva śānto 'bhūd āsādya salilaṃ yathā~}| 
  \textit{tato 'parāṃ mahājvālāṃ vīrabhadras tu dṛṣṭavān~}||~11~||
  \textit{khaṃ gacchantīṃ mahākālo jvālāṃ nipatitām api~}|
  \textit{manasā cintayac cāpi vīrabhadraḥ pratāpavān}~||~12~|| 
  \textit{sarveṣāṃ nāśinī jvālā prāṇināṃ śatakoṭiśaḥ}~| 
  \textit{tat sarvaṃ rakṣaṇārthaṃ hi pipāsuś cāpy ahaṃ tv imām}~||~13~|| 
  \textit{prāśnāmi mahatīṃ jvālāṃ salilaṃ tṛṣito yathā}~| 
  \textit{etasminn antare vīraṃ vāg āha cāśarīriṇī}~||~14~||.
 }}

  \maintext{bhasmanā vibudhā muktā vīrabhadrabhayārditāḥ |}%

  \maintext{bhasmānuśaṃsaṃ dṛṣṭvaiva brahmaṇānumatiḥ kṛtā }||\thinspace8:38\thinspace||%
\translation{The gods, afflicted by their fear of Vīrabhadra, were set free with the help of ashes. Seeing the glory of ashes, Brahmā consented [to the use of this otherwise impure substance]. \blankfootnote{8.38 The verse may refer to the destruction of Dakṣa's sacrifice, after which the gods were 
  relieved. See old \SKANDAP\ 180.1--4ab {\rm (}in which our \textit{pāda} b is echoed{\rm )}:
  \textit{sanatkumāra uvāca |}
  \textit{brahmādyā devatā vyāsa dakṣayajñavadhe purā |}
  \textit{śaṅkaraṃ śaraṇaṃ jagmur vīrabhadrabhayārditāḥ}~||~1~||
  \textit{gaṇendreṇābhiyuktās tu bhasmakūṭāni bhejire}~|
  \textit{yadā bhasma praviṣṭās te tejaḥ śāṅkaram uttamam}~||~2~||
  \textit{abhavan te tadā raudrāḥ paśavo dīkṣitā iva}~| 
  \textit{bhasmābhasitagātrāṇāṃ śaṅkaravratacāriṇām}~||~3~||
  \textit{svaṃ yogaṃ pradadau teṣāṃ tadā deva umāpatiḥ~|.}
 }}

  \maintext{caturāśramato 'dhikyaṃ vrataṃ pāśupataṃ kṛtam |}%

  \maintext{tasmāt pāśupataṃ śreṣṭhaṃ bhasmadhāraṇahetutaḥ }||\thinspace8:39\thinspace||%
\translation{[Thus] the Pāśupata observance was created, which is above [the system of] the four \textit{āśrama}s. Therefore the Pāśupata [observance] is the best because it involves carrying ashes [on one's body]. \blankfootnote{8.39 One could simply accept the reading of \msCc\ {\rm (}\textit{°hetunā}{\rm )} in \textit{pāda} d, but all other rejected 
  readings hint at an original \textit{hetutaḥ} {\rm (}as remarked by Judit Törzsök{\rm )}.
 }}

  \subsubchptr{vāruṇaṃ snānam}%

  \trsubsubchptr{Water bath}%

  \maintext{vāruṇaṃ salilaṃ snānaṃ kartavyaṃ vividhaṃ naraiḥ |}%

  \maintext{nadītoyataḍāgeṣu prasraveṣu hradeṣu ca }||\thinspace8:40\thinspace||%
\translation{A water bath {\rm (}\textit{vāruṇa}{\rm )} is to be performed with water in different ways by [different] people: in the water of rivers, water tanks, streams and ponds. \blankfootnote{8.40 The reading \textit{vividhaṃ} in \textit{pāda} b seems to be the lectio difficilior as opposed to
  the rejected \textit{vidhivat}.
 }}

  \subsubchptr{brāhmyaṃ snānam}%

  \trsubsubchptr{Vedic bath}%

  \maintext{brahmasnānaṃ ca viprendra āpohiṣṭhaṃ vidur budhāḥ |}%

  \maintext{trisaṃdhyam eva kartavyaṃ brahmasnānaṃ tad ucyate }||\thinspace8:41\thinspace||%
\translation{The wise know the Vedic bath as [the one performed with the Vedic mantra beginning] \textit{āpo hi ṣṭhā}, O excellent Brahmin. It is to be performed at the three junctures of the day [dawn, noon, evening]. It is called the Vedic bath. \blankfootnote{8.41 The Ṛgvedic mantra starting with \textit{āpo hi ṣṭhā} {\rm (}ṚV 10.9.1--3{\rm )} is traditionally associated with 
  \textit{mārjana} {\rm (}`cleaning, wiping'{\rm )}. According to \mycitep{KaneHistory}{v. 4, 120},
  a Brahmin ``should bathe thrice in the day, should perform \textit{mārjana} {\rm (}splashing
  or sprinkling water on the head and other limbs by means of \textit{kuśas} 
  dipped in water after repeating sacred mantras{\rm )} with the three verses `apo hi sthā' [sic] {\rm (}Ṛg. X.9.1--3{\rm )} [...]''
  This suggests a method of bathing that is more of a ritual than an actual bath.
 }}

  \subsubchptr{vāyavyaṃ snānam}%

  \trsubsubchptr{Wind bath}%

  \maintext{goṣu saṃcāramārgeṣu yatra godhūlisambhavaḥ |}%

  \maintext{tatra gatvāvasīdeta snānam uktaṃ manīṣibhiḥ }||\thinspace8:42\thinspace||%
\translation{He should go where, on the paths where cows roam, dust is rising, and he should sit down there. This is called [a kind of] bath, [namely the \textit{vāyavya} or wind-bath]. \blankfootnote{8.42 Understand \textit{goṣu} in \textit{pāda} a as \textit{gavāṃ} {\rm (}genitive{\rm )}.
 This version of bathing seems to be a way of taking a shower 
  in the holy dust raising from under the hooves of cows.
 }}

  \subsubchptr{divyaṃ snānam}%

  \trsubsubchptr{Heavenly bath}%

  \maintext{varṣatoyāmbudhārābhiḥ plāvayitvā svakāṃ tanum |}%

  \maintext{snānaṃ divyaṃ vadaty eva jagadādimaheśvaraḥ }||\thinspace8:43\thinspace||%
\translation{One should immerse one's own body in the water-showers of rain water. The one and only great Lord {\rm (}\textit{maheśvara}{\rm )} of the universe calls it heavenly bath. }

  \maintext{iti niyamavibhāgaḥ pañcabhedena vipra}%

 \nonanustubhindent \maintext{nigadita tava pṛṣṭaḥ sarvalokānukampya |}%

  \maintext{sakalamalapahārī dharmapañcāśad etan}%

 \nonanustubhindent \maintext{na bhavati punajanma kalpakoṭyāyute 'pi }||\thinspace8:44\thinspace||%
\translation{Thus have I taught you the section on the Niyama-rules in divisions of five [sub-categories to each] because you asked me to, favouring the whole world. These fifty Dharmic [teachings], wipe off all the defilement. There will not be rebirth [for one who keeps these rules], not even in millions of \ae ons. \blankfootnote{8.44 This verse marks not only the end of a long section on the Niyama rules,
  but also the end of a major part of the text that discusses the ten Yama and ten Niyama rules,
  spanning 3.16--8.44.
 
  
 There are two stem form nouns in \textit{pāda} b: I suspect that \Ed\ is right
  assuming that in order to restore the metre, we must have \textit{nigadita} and not
  \textit{nigaditas}, which is trasmitted in all the witnesses;
  also understand \textit{sarvalokānukampya} in \textit{pāda} b as \textit{sarvalokān anukampya}.
 Understand \textit{sakalamalapahārī} in \textit{pāda} c as \textit{sakala-mala-apahārī}, which would be unmetrical,
  and compare it with \textit{duritamalapahārī} in 4.89c.
  Take \textit{etan/etad} as either picking up °\textit{pahārī} or rather
  a plural corresponding to °\textit{pañcāśad}. The latter phenomenon, namely the use of
  the singular after numbers, is one of the hallmarks of the text.
 
  By `fifty Dharmas,' the text refers to the ten main Niyama-rules × five subcategories.
 
  
 The licence of an word-ultimate short syllable treated as long {\rm (}°\textit{janma} in \textit{pāda} d{\rm )} is
  also freqently seen here. Note also \textit{puna} for \textit{punar} metri causa.
 }}
\center{\maintext{\dbldanda\thinspace iti vṛṣasārasaṃgrahe niyamapraśaṃsā nāmādhyāyo 'ṣṭamaḥ\thinspace\dbldanda}}
\translation{Here ends the eighth chapter in the \textit{Vṛṣasārasaṃgraha} called the Praise of the Niyama-rules}

  \chptr{navamo 'dhyāyaḥ}
\fancyhead[CE]{{\footnotesize\textit{Translation of chapter 9}}}%

  \trchptr{ Chapter Nine}%

  \subchptr{traiguṇyam}%

  \trsubchptr{The system of three qualities}%

  \maintext{trikālaguṇabhedena bhinnaṃ sarvacarācaram |}%

  \maintext{tasmāt triguṇabandhena veṣṭitaṃ nikhilaṃ jagat }||\thinspace9:1\thinspace||%
\translation{The whole universe with its moving and unmoving elements is divided by the three subdivisions {\rm (}\textit{guṇa}{\rm )} of time. Therefore the whole world is bound by the fetters of three qualities {\rm (}\textit{guṇa}{\rm )}. \blankfootnote{9.1 It is only \msM, a MS not collated for this chapter, that inserts, post correctionem,
  \textit{anarthayajña uvāca} at the beginning of this chapter. It is not really needed:
  Anarthayajña's teaching continues without interruption here. 
  Another possibility is that this verse was originally the continuation
  of the end of chapter two {\rm (}2:40ef: \textit{traikālyakalanāt kālas tena kālaḥ prakīrtitaḥ}{\rm )}.
  At least it seems to directly connect there topic-wise.
  My translation of \textit{guṇa} in \textit{pāda} a is tentative.
 }}

  \maintext{vigatarāga uvāca |}%

  \maintext{traikālyam iti kiṃ jñeyaṃ traidhātukaśarīriṇaḥ |}%

  \maintext{kiṃcid vistaram eveha kathayasva tapodhana }||\thinspace9:2\thinspace||%
\translation{Vigatarāga spoke: What does the term `the three times' mean for an embodied creature that is made up of three constituents {\rm (}\textit{dhātuka}{\rm )}? Teach me about this in a somewhat more extended manner, O great ascetic. \blankfootnote{9.2 I have included the element \textit{trai°} in the lemma in \textit{pāda}s ab only because \msCc\ 
  has a slightly unusual ligature there {\rm (}\textit{mtrai}{\rm )}. 
  
  As for the interpretation of
  \textit{traidhātuka} in \textit{pāda} b, an intelligent guess would be a reference to the three so-called
  `humours' of the body, namely \textit{pitta, vāyu} and \textit{śleṣman}. This is
  the only occurence of the term \textit{traidhātuka} in the \VSS. In 5.11cd above, \textit{dhātu} is probably used 
  in the same Ayurvedic sense that I am proposing here {\rm (}\textit{dhātuvaiṣamyanāśo 'sti na ca rogāḥ sudāruṇāḥ}{\rm )}.
  Elsewhere \textit{dhātu} means `verbal root' {\rm (}3.3{\rm )}, `metal' {\rm (}16.6: 
  \textit{yathā vai sarvadhātūnāṃ doṣā dahyanti dhāmyatām | 
  tathā pāpāḥ pradahyante dhruvaṃ prāṇasya nigrahāt} ||{\rm )}
  and `gross element' {\rm (}for Sāṃkhya-style \textit{mahābhūta}s in chapter 20{\rm )}.
  To complicate things, chapter thirteen claims that the human body is made up
  of two \textit{dhātu}s, \textit{somadhātu} and \textit{agnidhātu}. Semen contains \textit{somadhātu},
  menstrual blood \textit{agnidhātu}, and the new-born baby is thus made up of both. See e.g. 13.20cd--13.21:
  \textit{śukraśoṇitasaṃyogād garbhotpattis tataḥ smṛtaḥ~||}
  \textit{agnisomātmakaṃ devi śarīradvayadhātutaḥ~|}
  \textit{somadhātu smṛtaṃ śukram agnidhātu rajaḥ smṛtam~|}
  \textit{agnisomāśrayaṃ devi śarīram iti saṃjñitam~||}.
 }}

  \maintext{anarthayajña uvāca |}%

  \maintext{traikālyaṃ triguṇaṃ jñeyaṃ vyāpī prakṛtisambhavaḥ |}%

  \maintext{anyonyam upajīvanti anyonyam anuvartinaḥ }||\thinspace9:3\thinspace||%
\translation{Anarthayajña spoke: The three times are the three qualities {\rm (}\textit{guṇa}{\rm )}. It is [all-]pervading and is born from Prakṛti. They support each other, they follow each other. \blankfootnote{9.3 Understand \textit{pāda} b as referring to the neuter \textit{traikālyaṃ} or rather \textit{triguṇaṃ} {\rm (}gender confusion{\rm )}.
 Note the number confusion in \textit{pāda}s cd.
 }}

  \maintext{sattvaṃ rajas tamaś caiva rajaḥ sattvaṃ tamas tathā |}%

  \maintext{tamaḥ sattvaṃ rajaś caiva anyonyamithunāḥ smṛtāḥ }||\thinspace9:4\thinspace||%
\translation{Sattva, Rajas and Tamas; Rajas, Sattva and Tamas; Tamas, Sattva and Rajas; they are mutually each other's pairs. }

  \maintext{sāttviko bhagavān viṣṇū rājasaḥ kamalodbhavaḥ |}%

  \maintext{tāmaso bhagavān īśaḥ sakalaṃvikaleśvaraḥ }||\thinspace9:5\thinspace||%
\translation{Lord Viṣṇu is Sattvic. [Brahmā], the one who was born on a lotus, is Rājasa. Lord Īśa is Tāmasa, [both in his] complete {\rm (}\textit{sakala}{\rm )} [form] and [as] formless {\rm (}\textit{vikala}{\rm )} Īśvara. \blankfootnote{9.5 My altering the reading \textit{viṣṇu} to \textit{viṣṇū} in \textit{pāda} a against all witnesses may
  be regarded as an overcorrection and the stem form could be original.
 My translation of \textit{pāda}s cd is tentative. I suspect that \textit{pāda} d is one single compound,
  the \textit{anusvāra} is only inserted to avoid the metric fault of two \textit{laghu} syllables 
  at the second and third position.
  I understand \textit{vikala} as a synonym of \textit{niṣkala}. For the tantric connotations of the pair
  \textit{sakala-niṣkala} see, e.g., \TAKIII\ s.v. \textit{niṣkala}.
 }}

  \maintext{sattvaṃ kundenduvarṇābhaṃ padmarāganibhaṃ rajaḥ |}%

  \maintext{tamaś cāñjanaśailābhaṃ kīrtitāni manīṣibhiḥ }||\thinspace9:6\thinspace||%
\translation{Sattva is of the colour of jasmine and the moon. Rajas is of the colour of ruby. Tamas is of the colour of lamp-black ... śaila. [This is what] the wise teach. }

  \maintext{sattvaṃ jalaṃ rajo 'ṅgāraṃ tamo dhūmasamākulam |}%

  \maintext{etadguṇamayair baddhāḥ pacyante sarvadehinaḥ }||\thinspace9:7\thinspace||%
\translation{Sattva is water, Rajas is charcoal, Tamas is full of smoke. All souls are constructed/suffer {\rm (}\textit{pacyante}{\rm )} as bound by these \textit{guṇa}s. }

  \maintext{vigatarāga uvāca |}%

  \maintext{kena kena prakāreṇa guṇapāśena badhyate |}%

  \maintext{cihnam eṣāṃ pṛthaktvena kathayasva tapodhana }||\thinspace9:8\thinspace||%
\translation{Vigatarāga spoke: By what sorts of noose of \textit{guṇa}s is [the soul] bound? Teach me the signs connected to them one by one, O great ascetic. }

  \maintext{anarthayajña uvāca |}%

  \maintext{anekākārabhāvena badhyante guṇabandhanaiḥ |}%

  \maintext{mohitā nābhijānanti jānanti śivayoginaḥ }||\thinspace9:9\thinspace||%
\translation{Anarthayajña spoke: The souls are bound in many ways and by many conditions by the fetters of the \textit{guṇa}s. Those who are deluded do not recognize [them]. The Śivayogins do recognize [them]. }

  \maintext{ūrdhvaṃgo nityasattvastho madhyago rajasāvṛtaḥ |}%

  \maintext{adhogatis tamo'vasthā bhavanti puruṣādhamāḥ }||\thinspace9:10\thinspace||%
\translation{He who is always established in Sattva goes upwards. He who is covered with Rajas goes in the middle. Those lowest of men in the state of Tamas go downward. \blankfootnote{9.10 Understand \textit{adhogatis} in pāda c as a bahuvrīhi in plural {\rm (}\textit{adhogatayas}{\rm )}.
 }}

  \maintext{svarge 'pi hi trayo vaite bhāvanīyās tapodhana |}%

  \maintext{mānuṣeṣu ca tiryeṣu guṇabhedās trayas trayaḥ }||\thinspace9:11\thinspace||%
\translation{These three kinds of \textit{guṇa}s are to be acknowledged even in heaven, O great ascetic, and among humans and also among animals. }

  \subsubchptr{sāttvikottamāḥ}%

  \maintext{brahmā viṣṇuś ca rudraś ca dharma indraḥ prajāpatiḥ |}%

  \maintext{somo 'gnir varuṇaḥ sūryo daśa sattvottamāḥ smṛtāḥ }||\thinspace9:12\thinspace||%
\translation{The ten superior Sattva [beings] are: Brahmā, Viṣṇu, Rudra, Dharma, Indra, Prajāpati, Soma, Agni, Varuṇa and Sūrya. }

  \subsubchptr{sāttvikamadhyamāḥ}%

  \maintext{rudrādityā vasusādhyā viśveśamaruto dhruvaḥ |}%

  \maintext{ṛṣayaḥ pitaraś caiva daśaite sattvamadhyamāḥ }||\thinspace9:13\thinspace||%
\translation{... }

  \subsubchptr{sāttvikādhamāḥ}%

  \maintext{tārā grahāḥ surā yakṣā gandharvāḥ kiṃnaroragāḥ |}%

  \maintext{rakṣobhūtapiśācāś ca daśaite sāttvikādhamāḥ }||\thinspace9:14\thinspace||%
\translation{... }

  \subsubchptr{rājasottamāḥ}%

  \maintext{ṛtvik purohitācāryayajvāno 'tithivijñanī |}%

  \maintext{rājamantrī vratī vedī daśaite rājasottamāḥ }||\thinspace9:15\thinspace||%
\translation{... ... }

  \subsubchptr{jātayo rājasādhamāḥ}%

  \maintext{sūto 'mbaṣṭavaṇik cograḥ śilpikārukamāgadhāḥ |}%

  \maintext{veṇavaidehakāmātyā daśaite rajamadhyamāḥ }||\thinspace9:16\thinspace||%
\translation{... ... }

  \maintext{carmakṛtkumbhakṛtkolī lohakṛttrapunīlikāḥ |}%

  \maintext{naṭamuṣṭikacaṇḍālā daśaite rajasādhamāḥ }||\thinspace9:17\thinspace||%
\translation{... ... }

  \subsubchptr{tāmasottamāḥ}%

  \maintext{gogajagavayā aśvamṛgacāmarakiṃnarāḥ |}%

  \maintext{siṃhavyāghravarāhāś ca daśaite tāmasottamāḥ }||\thinspace9:18\thinspace||%
\translation{These are the ten superior Tāmasa [animals]: cows, elephants, Gayal oxen, horses, deer, Yaks, Kiṃnaras, lions, tigers, wild boar. }

  \subsubchptr{tāmasamadhyamāḥ}%

  \maintext{ajameṣamahiṣyāś ca mūṣikānakulādayaḥ |}%

  \maintext{uṣṭraraṅkuśaśagaṇḍā daśaite tamamadhyamāḥ }||\thinspace9:19\thinspace||%
\translation{The ten middle ranking Tāmasa [beings] are: rams, sheep, buffaloes, mice, mongooses etc., camels, Raṅku deer, hares, rhinoceroses. [only 9!] \blankfootnote{9.19 \textit{°mahiṣyāś} seems to be an equivalent of \textit{°mahiṣāś} metri causa.
 }}

  \subsubchptr{tāmasādhamāḥ}%

  \maintext{ṛkṣagodhāmṛgaśṛṅgibakavānaragardabhāḥ |}%

  \maintext{sūkaraśvānagomāyur daśaite tāmasādhamāḥ }||\thinspace9:20\thinspace||%
\translation{The ten low-ranking Tāmasa [beings] are: bears, alligators, deer, horned animals[?], cranes, apes, donkeys, boar, dogs and frogs. }

  \subsubchptr{tamasāttvikāḥ}%

  \maintext{krauñcahaṃsaśukaśyenabhāsabāruṇḍasārasāḥ |}%

  \maintext{cakrāhvaśukamāyūrā daśaite tamasāttvikāḥ }||\thinspace9:21\thinspace||%
\translation{The ten Tāmasa-Sāttvika [beings] are: curlews, swans, parrots, falcons, vultures, B[h]āruṇḍa birds, cranes, Cakra[vāka] birds, parrots, and peacocks. \blankfootnote{9.21 Although all the manuscripts consulted read \textit{kroñca°} in pāda a, I decided
  to accept \Ed's standard spelling in this case. In pāda b, I left \textit{°bāruṇḍa°}
  thus, although what is really meant is probably \textit{bhāraṇḍa}, \textit{bhāruṇḍa} or \textit{bhuruṇḍa}.
 Note the repetition of \textit{śuka} in this stanza.
 }}

  \subsubchptr{tamarājasāḥ}%

  \maintext{balākāḥ kukkuṭāḥ kākāś cillalāvakatittirāḥ |}%

  \maintext{gṛdhrakaṅkabakaśyena daśaite tamarājasāḥ }||\thinspace9:22\thinspace||%
\translation{The ten Tāmasa-Rājasa [beings] are: Balāka-cranes, cocks, crows, Bengal kites, Lāvakas, partridges, vultures, herons, Bakas and hawks. }

  \maintext{kokilolūkakiñjalkakapotāḥ pañca eva ca |}%

  \maintext{śārikāś ca kuliṅgāś ca daśaite tamasādhamāḥ }||\thinspace9:23\thinspace||%
\translation{The ten lowest Tāmasa [beings] are: cuckoos, owls, Kiñjalkas[?], doves, Śārika birds and sparrows. \blankfootnote{9.23 This list is problematic for it has only six elements instead of the expected ten 
  and \textit{kiñjalka} is difficult to interpret.
 }}

  \maintext{makaragohanakrāś ca ṛkṣāś ca tamasāttvikāḥ |}%

  \maintext{kacchapa{\rm †}śuśu{\rm †}kumbhīramaṇḍūkās tamarājasāḥ |}%

  \maintext{śaṅkhaśuktikaśambūka{\rm †}kabandhyā{\rm †}s tamatāmasāḥ }||\thinspace9:24\thinspace||%
\translation{Makaras crocodiles, cow-killing alligators and bears are of Tamas-Sattva. Tortoises, Śuśus[?], crocodiles of the Ganges and frogs are of Tamas-Rajas. Conch-shells, pearl-oysters, shells and [...] are Tamas-Tāmasa. \blankfootnote{9.24 Note that the reading that yields `and bears' {\rm (}\textit{ṛkṣāś ca}{\rm )} is my conjecture
  for a problematic \textit{ṛṣā ca}. It is far from satisfactory since bears have already appeared in 
  verse 9.20 above.
 I have not been able to identify the probably aquatic animal behind the 
  word \textit{śuśu} here.
 }}

  \maintext{candanāgarupadmaṃ ca plakṣodumbarapippalāḥ |}%

  \maintext{vaṭadāruśamībilvā daśaite tamasāttvikāḥ }||\thinspace9:25\thinspace||%
\translation{... ... }

  \maintext{jāmbīralakucāmrātadāḍimākolavetasāḥ |}%

  \maintext{nimbanīpo dhravāvaś ca daśaite tamarājasāḥ }||\thinspace9:26\thinspace||%
\translation{The ten Tamas-Rajas [trees] are: Citron trees, bread-fruit trees, hog-plum trees, pomegranate trees, jujube trees, ratan trees, Neemb trees, Kadamba trees and ... }

  \maintext{vṛkṣavallīlatāveṇutvaksāratṛṇabhūruhāḥ |}%

  \maintext{mīrajāś ca śilāśasyā daśaite tamasāttvikāḥ }||\thinspace9:27\thinspace||%
\translation{... ... }

  \maintext{bhramarādipataṅgāś ca krimikīṭajalaukasaḥ |}%

  \maintext{yūkoddaṃśamaśānāṃ ca viṣṭajās tamasāttvikāḥ }||\thinspace9:28\thinspace||%
\translation{... ... }

  \maintext{dayā satyaṃ damaḥ śaucaṃ jñānaṃ maunaṃ tapaḥ kṣamā |}%

  \maintext{śīlaṃ ca nābhimānaṃ ca sāttvikāś cottamā janāḥ }||\thinspace9:29\thinspace||%
\translation{[These words describe] the people who are the best among the Sāttvika [type]: compassion, truthfulness, self-control, purity, knowledge, taciturnity, penance, patience, integrity, lack of self-conceit. }

  \maintext{kāmatṛṣṇāratidyūtamāno yuddhaṃ madaḥ spṛhā |}%

  \maintext{nirghṛṇāḥ kalikartāro rājaseṣūttamā janāḥ }||\thinspace9:30\thinspace||%
\translation{[These words describe] the people who are the best among the Rājasa [type]: desire, thirst, pleasure, gambling, arrogance, fight, intoxication, delight, cruel, quarrelling. }

  \maintext{hiṃsāsūyāghṛṇāmūḍhanidrātandrībhayālasāḥ |}%

  \maintext{krodho matsaramāyī ca tāmaseṣūttamā janāḥ }||\thinspace9:31\thinspace||%
\translation{[These words describe] people who are the best among the Tāmasa [type]: harming, envious, incompassionate, stupid, sleepy, lazy, cowardly, idle, angry, greedy, cheating. }

  \maintext{laghuprītiprakāśī ca dhyānayoge sadotsukaḥ |}%

  \maintext{prajñābuddhivirāgī ca sāttvikaṃ guṇalakṣaṇam }||\thinspace9:32\thinspace||%
\translation{The Sāttvika can be characterised as follows: light, joyful, bright, always eager for yoga meditation, wise, intelligent and dispassionate. }

  \maintext{bālako nipuṇo rāgī māno darpaś ca lobhakaḥ |}%

  \maintext{spṛhā īrṣā pralāpī ca rājasaṃ guṇalakṣaṇam }||\thinspace9:33\thinspace||%
\translation{The Rājasa can be characterised as follows: childish, skilful, passionate, proud, arrogant, greedy, desirous, jealous and chattering. }

  \maintext{udvega ālaso mohaḥ krūras taskaranirdayaḥ |}%

  \maintext{krodhaḥ piśuna nidrā ca tāmasaṃ guṇalakṣaṇam }||\thinspace9:34\thinspace||%
\translation{The Tāmasa can be characterised as follows: anxious, lazy, deluded, cruel, a pitiless robber, angry, wicked and sleepy. \blankfootnote{9.34 In pāda a, \textit{piśuno} might be the right choice: it is a ra-vipulā 
  if \textit{dr} in \textit{nidrā} does not make the previous syllable long, a licence
  often occuring in this text {\rm (}`muta cum liquida'{\rm )}.
 }}

  \maintext{vigatarāga uvāca |}%

  \maintext{kena cihnena vijñeya āhāraḥ sarvadehinām |}%

  \maintext{traiguṇyasya pṛthaktvena kathayasva tapodhana }||\thinspace9:35\thinspace||%
\translation{Vigatarāga spoke: By what signs can the food of all humans be recognized? [?] Teach me about the three \textit{guṇa}s, O great ascetic. }

  \maintext{anarthayajña uvāca |}%

  \maintext{āyuḥ kīrtiḥ sukhaṃ prītir balārogyavivardhanam |}%

  \maintext{hṛdyasvādurasaṃ snigdha āhāraḥ sāttvikapriyaḥ }||\thinspace9:36\thinspace||%
\translation{Anarthayajña spoke: The Sāttvikas prefer food that yields [long] life, fame, happiness, joy, which increases strength and health, which is savoury and which tastes nice, and which is soft. }

  \maintext{atyuṣṇam āmlalavaṇaṃ rūkṣaṃ tīkṣṇaṃ vidāhi ca |}%

  \maintext{rājasaśreṣṭha āhāro duḥkhaśokāmayapradaḥ }||\thinspace9:37\thinspace||%
\translation{The best food for the Rājasas is rather warm, acidic, salty, hard, hot and pungent. It gives you pain, a burning sensation and indigestion. }

  \maintext{abhakṣyāmedhyapūtī ca pūti paryuṣitaṃ ca yat |}%

  \maintext{āyāmarasavisvāda āhāras tāmasapriyaḥ }||\thinspace9:38\thinspace||%
\translation{Tāmasas prefer food that is prohibited, impure and foul-smelling, ... stale ... and tasteless. \blankfootnote{9.38 Understand \textit{°pūtī} in pāda a as standing for \textit{°pūti} metri causa, and 
  note that °āmedhya° in the same pāda is an emendation {\rm (}correcting \msNc's reading{\rm )}.
 Read \textit{āmayārasa} in pāda c?
 }}

  \maintext{vigatarāga uvāca |}%

  \maintext{guṇātītaṃ kathaṃ jñeyaṃ saṃsāraparapāragam |}%

  \maintext{guṇapāśanibaddhānāṃ mokṣaṃ kathaya tattvataḥ }||\thinspace9:39\thinspace||%
\translation{Vigatarāga spoke: How can one recognize [the state of getting] beyond the \textit{guṇa}s, which leads one to the other shore of [the ocean] of mundane existence? Tell me truly about the liberation of those who are [initially] bound by the noose of the \textit{guṇa}s. }

  \maintext{anarthayajña uvāca |}%

  \maintext{ātmavat sarvabhūtāni samyak paśyeta bho dvija |}%

  \maintext{guṇātītaḥ sa vijñeyaḥ saṃsāraparapāragaḥ }||\thinspace9:40\thinspace||%
\translation{Anarthayajña spoke: Well, he who looks at all living beings in the correct way, as his own Self, O Brahmin, is to be known as one beyond the \textit{guṇa}s, as one who has departed to the other shore of [the ocean of] mundane existence. }

  \maintext{īrṣādveṣasamo yas tu sukhaduḥkhasamāś ca ye |}%

  \maintext{stutinindāsamā ye ca guṇātītaḥ sa ucyate }||\thinspace9:41\thinspace||%
\translation{He who treats envy and hate[?], happiness and sorrow, praise and reproach as equal is called `one who is beyond the \textit{guṇa}s'. }

  \maintext{tulyapriyāpriyo yaś ca arimitrasamas tathā |}%

  \maintext{mānāpamānayos tulyo guṇātītaḥ sa ucyate  }||\thinspace9:42\thinspace||%
\translation{He who is indifferent to pleasant and unpleasant things, to enemy or friend, to respect and contempt is called `one who is beyond the \textit{guṇa}s'. }

  \maintext{eṣa te kathito vipra guṇasadbhāvanirṇayaḥ |}%

  \maintext{guṇayuktas tu saṃsārī guṇātītaḥ parāṅgatiḥ }||\thinspace9:43\thinspace||%
\translation{O Brahmin, thus has the exposition of the essence of the \textit{guṇa}s been taught to you. Those who are connected with the \textit{guṇa}s are mundane {\rm (}\textit{saṃsārin}{\rm )}, those beyond the \textit{guṇa}s are on the supreme path. }
\center{\maintext{\dbldanda\thinspace iti vṛṣasārasaṃgrahe traiguṇyaviśeṣaṇīyo nāmādhyāyo navamaḥ\thinspace\dbldanda}}
\translation{Here ends the ninth chapter in the Vṛṣasārasaṃgraha called the Particulars of the Three Guṇas}


\
\pagebreak

\blankpage

\begin{center}{\huge\textbf{\englishfont Śivadharmaśāstra}}\end{center}\vskip2em
  \chptr{trayodaśamo 'dhyāyaḥ}
\fancyhead[CO]{{\footnotesize\textit{Translation of chapter 13}}}%

  \trchptr{Chapter Thirteen}
\fancyhead[CO]{{\footnotesize\textit{Translation of chapter 13}}}%

  \subchptr{kathaṃ sukhopāyo na kriyate}%

  \trsubchptr{Why is the easy method not followed?}%

  \maintext{devy uvāca |}%

  \maintext{ahiṃsātithyakānāṃ ca śruto dharmaḥ suvistaraḥ |}%

  \maintext{kiṃ na kurvanti manujāḥ sukhopāyaṃ mahat phalam }||\thinspace13:1\thinspace||%
\translation{Devī spoke: I have heard the Dharma of non-violence and of guest-reception in detail. Why do people not follow the easy method that brings about great rewards? \blankfootnote{13.1 \textit{Pāda}s ab and cd are back-references to chapters 12 and 11,
  respectively.
 }}

  \maintext{svaśarīre sthito yajñaḥ svaśarīre sthitaṃ tapaḥ |}%

  \maintext{svaśarīre sthitaṃ tīrthaṃ śruto vistarato mayā }||\thinspace13:2\thinspace||%
\translation{I have heard in full detail that worship resides in one's own body, and penance resides in one's body, and that the pilgrimage places reside in one's own body. \blankfootnote{13.2 \textit{Pāda}s ab and cd are back-references to chapters 11 and 10,
  respectively.
 }}

  \maintext{kimarthaṃ bhagavan brūhi sukhopāyaṃ mahat phalam |}%

  \maintext{kiṃ nivṛttās tu deveśa ṛṣidaivatamānuṣāḥ }||\thinspace13:3\thinspace||%
\translation{O Lord, tell me why does the easy method yields great rewards? And why are the sages, the gods, and the people indifferent? }

  \maintext{mahādeva uvāca |}%

  \maintext{adya pṛṣṭena kathitaṃ gopitaṃ mayi sundari |}%

  \maintext{mānuṣāṇāṃ hitārthāya tava ca varavarṇini }||\thinspace13:4\thinspace||%
\translation{Mahādeva spoke: Now, as I am asked to do so, I shall reveal the secret to you, O Sundarī, for the benefit of mankind and to favour you, O Varavarṇinī. \blankfootnote{13.4 Understand \textit{mayi} in \textit{pāda} b as \textit{mayā} metri causa.
 }}

  \maintext{adyaprabhṛti deveśi khyātir loke bhaviṣyati |}%

  \maintext{dhanyā evaṃ cariṣyanti adhanyā na ramanti tam }||\thinspace13:5\thinspace||%
\translation{From this day on, O Deveśī, it will be open knowledge {\rm (}\textit{khyāti}{\rm )} in the world. The fortunate ones will follow, the unfortunate ones will not delight in it. }

  \maintext{triguṇena tu bandhena baddhapāśadṛḍhena tu |}%

  \maintext{tenārthena ramanty atra jānanto 'pi hi mohitāḥ }||\thinspace13:6\thinspace||%
\translation{Because of the threefold bondage, in which the nooses are tied firmly, those who are stuck in it remain deluded, even if they are knowledgable. }

  \subchptr{triguṇabandhaḥ}%

  \trsubchptr{Threefold bondage}%

  \maintext{devy uvāca |}%

  \maintext{kiṃ vā triguṇabandheti brūhi saṃśayachedaka |}%

  \maintext{adyāpi mama deveśa mohotpannas tribandhanaiḥ }||\thinspace13:7\thinspace||%
\translation{Devī spoke: But what is this threefold bondage? Teach me, O you who repel my doubts. Even now, I am still confused, O Deveśa, by the threefold bondage. \blankfootnote{13.7 Understand \textit{mohotpannas} in \textit{pāda} d as
  \textit{moha utpannas} {\rm (}double sandhi{\rm )}.
 }}

  \maintext{bhagavān uvāca |}%

  \maintext{prākṛtaṃ vaikṛtaṃ caiva dakṣiṇābandham eva ca |}%

  \maintext{etenaiva tu bandhena baddhāḥ varṇāśramāḥ sadā }||\thinspace13:8\thinspace||%
\translation{The Lord spoke: The social classes {\rm (}\textit{varṇa}{\rm )} and the social order of disciplines {\rm (}\textit{āśrama}{\rm )} are bound forever by [1] natural {\rm (}\textit{prākṛta}{\rm )}, and [2] modified {\rm (}\textit{vaikṛta}{\rm )} bondage, and [3] the bondage of ritual reward {\rm (}\textit{dakṣiṇā\-bandha}{\rm )}. \blankfootnote{13.8 The three categories of \textit{prākṛta}- \textit{vaikṛta}- and \textit{dakṣiṇābandha} appear in commentaries on
  Sāṃkhya texts. One brief explanation that is close to what could be implied 
  here in the \VSS\ is Māṭhatavṛtti ad Sāṃkhyakārikā 44:
  \textit{tenājñānena manuṣyatiryagdeveṣv ātmānaṃ nibadhnāti\thinspace | na mokṣaṃ
  gacchatīty arthaḥ\thinspace | ato 'jñānaṃ nimittaṃ bandho naimittikāḥ\thinspace | sa ca bandhas
  trividhaḥ\thinspace | prakṛtibandho vaikārikabandho dakṣiṇābandhaś ceti\thinspace | tatra
  prakṛtibandho nāmāṣṭāsu prakṛtiṣu paratvenābhimānaḥ\thinspace | vaikārikabandho nāma
  brahmādisthāneṣu śreyobuddhiḥ\thinspace | dakṣiṇābandho nāma gavādidānejyānimittaḥ\thinspace |} 
 }}

  \maintext{jñānahīnā nivartante paramaṃ prāpya tat padam |}%

  \maintext{iṣṭastrīputrabhṛtyārthe dhanadhānyasamuccaye |}%

  \maintext{snehād ākṛṣṭamanasāṃ bandhaḥ prākṛta ucyate }||\thinspace13:9\thinspace||%
\translation{After they have reached that ultimate realm, being ignorant, [people] fall back for the sake of a desired woman, in order to have sons, servants, to accumulate money and grain. The bondage of minds that are caught by affection is called `natural' {\rm (}\textit{prākṛta}{\rm )}. }

  \maintext{yogayuktena manasā yad yad aiśvaryam āpyate |}%

  \maintext{tadā vaikṛtabandhaṃ tu yadi tatrānurajyate }||\thinspace13:10\thinspace||%
\translation{If one gets attached to the powers that one has obtained by controlling the mind with yoga, then it is the `modified' {\rm (}\textit{vaikṛta}{\rm )} bondage. }

  \maintext{ārāmodyānavāpīṣu dānakratuphaleṣu ca |}%

  \maintext{āsaktamanaso vāso dakṣiṇābandha kathyate }||\thinspace13:11\thinspace||%
\translation{The abiding[?] of one's mind that is attached to [donating] gardens, parks, and wells, to the fruits of donations and rites, is called the bondage of ritual reward {\rm (}\textit{dakṣiṇā\-bandha}{\rm )}. \blankfootnote{13.11 I have corrected °\textit{bandhaḥ} to °\textit{bandha}, a \stemform\ noun, in
  \textit{pāda} d to restore the metre.
 }}

  \maintext{anenaiva tu pāśena baddho vānaravad yathā |}%

  \maintext{mokṣituṃ na ca śaknoti itaś cetaś ca dhāvati }||\thinspace13:12\thinspace||%
\translation{Tied with this bondage, like monkeys not able to get away, people just keep running to and fro. }

  \maintext{devāsuramanuṣyeṣu tiryeṣu narakeṣu ca |}%

  \maintext{bhramate cakrayantravad yāvat tattvaṃ na vindati }||\thinspace13:13\thinspace||%
\translation{They will trasmigrate through [births among the] gods, demons, humans, animals, and in hells, as if on a wheel-machine, until they understand the truth. }

  \maintext{garbhavāsaparikleśo janmamṛtyuḥ punaḥ punaḥ |}%

  \maintext{vyādhiśokabhayāyāsacintayā jarayā hataḥ }||\thinspace13:14\thinspace||%
\translation{There is the pain of being in the womb, birth and death again and again, and then one dies of old age with anxiety about illness, sadness, danger, and fatigue. }

  \subchptr{garbhotpattiḥ}%

  \trsubchptr{Formation of the embryo}%

  \maintext{devy uvāca |}%

  \maintext{garbhotpattiḥ kathaṃ deva yogī labhati kīdṛśīm |}%

  \maintext{kīdṛśaṃ labhate garbhaṃ śrotuṃ naḥ pratidīyatām }||\thinspace13:15\thinspace||%
\translation{Devī spoke: How is the embryo produced, O god? And what kind [of development of an embryo] will the yogin go through? What kind of a womb will he end up? Allow us to hear this. }

  \maintext{bhagavān uvāca |}%

  \maintext{śṛṇu devi pravakṣyāmi garbhotpattiṃ yathākramam |}%

  \maintext{yathā saṃśayavicchedaṃ labhiṣyasi varānane }||\thinspace13:16\thinspace||%
\translation{Listen, O Devī, I shall teach you the formation of the embryo in due order, to put an end to your doubts, O Varānanā. }

  \maintext{akṣarāt prabhavo brahmā karma brahmasamudbhavam |}%

  \maintext{karmato yajñaprabhavo yajñato dhūmasambhavaḥ }||\thinspace13:17\thinspace||%
\translation{From the imperishable is born Brahmā. From Brahmā is born the ritual {\rm (}\textit{karman}{\rm )}. From ritual arises worship {\rm (}\textit{yajña}{\rm )}. From sacrifice arises smoke. \blankfootnote{13.17 Understand \textit{pāda} a as \textit{akṣarāt brahmaprabhavaḥ}.
 \textit{Pāda} c is a \textit{na-vipulā} if the syllabe \textit{pra}
  does not makes the previous syllable heavy {\rm (}\mutacumliquida{\rm )}.
 }}

  \maintext{dhūmrād abhrāṇi jāyante abhrāt parjanyasambhavaḥ  |}%

  \maintext{parjanyād annam{-}utpattir annād bhūtāni jajñire }||\thinspace13:18\thinspace||%
\translation{From smoke are born the clouds. From the clouds come the rain-clouds. From the rain-clouds food arises. From food living beings arise. \blankfootnote{13.18 Understand \textit{annam utpattir} in \textit{pāda} c as a \textit{tatpuruṣa} compound.
 }}

  \maintext{annād rasasamutpattī rasāc choṇitasambhavaḥ |}%

  \maintext{śoṇitān māṃsam{-}utpattir māṃsād medasamudbhavaḥ }||\thinspace13:19\thinspace||%
\translation{From food arises flavour {\rm (}\textit{rasa}{\rm )}. From flavour arises blood. From blood flesh arises. From flesh fat arises. }

  \maintext{medaso 'sthīni jāyante asthibhyo majjasambhavaḥ |}%

  \maintext{majjāyās tu bhavec chukraṃ naraḥ śukrasamudbhavaḥ }||\thinspace13:20\thinspace||%
\translation{From fat bones arise. From the bones marrow arises. From marrow arises semen. And man is born from semen. }

  \maintext{śukraśoṇitasaṃyogād garbhotpattis tataḥ smṛtā |}%

  \maintext{agnisomātmakaṃ devi śarīraṃ dvayadhātutaḥ }||\thinspace13:21\thinspace||%
\translation{The formation of the embryo is known to come from the union of semen and blood. The body is made of Fire and Soma, O Devī, because of the two [corresponding] elements. }

  \maintext{somadhātu smṛtaṃ śukram agnidhātu rajaḥ smṛtam |}%

  \maintext{agnisomāśrayaṃ devi śarīram iti saṃjñitam }||\thinspace13:22\thinspace||%
\translation{Semen is said to be of the Soma element, blood is said to be of the Fire element. The body is taught to be the seat of Fire and Soma, O Devī. }

  \maintext{māse māse ṛtuḥ strīṇāṃ bhavatīha na saṃśayaḥ |}%

  \maintext{ṛtukāle prasarpeta na sukhārthaṃ varānane }||\thinspace13:23\thinspace||%
\translation{Each month women have their periods, no doubt about it. One should approach them at the time of their period, and not for pleasure, O Varānanā. }

  \maintext{putrakāmaḥ prayuñjīta dharmārthaṃ ca yaśasvini |}%

  \maintext{pumān strīṣūpayuñjīta araṇīva hutāśanam }||\thinspace13:24\thinspace||%
\translation{One should enjoy [a woman] if one wishes to have a son, and for the sake of religious duty {\rm (}\textit{dharma}{\rm )}, O Yaśasvinī. The man should enjoy the women[, and produce offspring,] like two sticks [rubbed together produce] fire. }

  \maintext{pumān śukrādhiko jñeyaḥ kanyā raktādhikā bhavet |}%

  \maintext{samaśukre ca rakte ca sa ca jāyen napuṃsakaḥ }||\thinspace13:25\thinspace||%
\translation{Males are to be known as having an excess of semen, while females have an excess of blood. When the semen and the blood are in equal quantitiy, a gender-neutral [child] will be born. }

  \subchptr{dviyamā triyamā ca gurviṇī}%

  \trsubchptr{Pregnancy with twins and triplets}%

  \maintext{devy uvāca |}%

  \maintext{dviyamā triyamā caiva kathaṃ jāyeta gurviṇī |}%

  \maintext{kathaṃ strīdviyamā jāyet kathaṃ vā puruṣadvayam }||\thinspace13:26\thinspace||%
\translation{Devī spoke: How does a woman become pregnant with twins or triplets? How are twin girls born and how are twin boys born? }

  \maintext{bhagavān uvāca |}%

  \maintext{raktādhikā smṛtā kanyā jāyate varavarṇini |}%

  \maintext{vāyunā ca dvidhā bhinnā kanyakādviyamā smṛtā }||\thinspace13:27\thinspace||%
\translation{The Lord spoke: When there is an excess of blood, a daughter is born, O Varavarṇinī. When it is split into two by Wind, it will be twin girls. }

  \maintext{śukrādhikas tu puruṣo dvidhā bhinno 'nilena tu |}%

  \maintext{dviyamā puruṣā jñeyās triyamās tu tridhā kṛte }||\thinspace13:28\thinspace||%
\translation{And when the male, which has an excess of semen, is split into two by Wind, they will be twin boys. When it is split into three, they will be triplets. }

  \maintext{ṛtusnātā yadā nārī yadi garbhād vigṛhṇati |}%

  \maintext{prathame ca dvitīye ca tṛtīye ca na jīvati }||\thinspace13:29\thinspace||%
\translation{When a woman has just bathed after her period, if she conceives[?] in[?] her womb on the first, second, or third day, [the embryo] will not survive. }

  \maintext{sameṣu janayet putraṃ kanyakāṃ viṣame dine |}%

  \maintext{ṣaṣṭy aṣṭamī ca daśamī dvādaśī ca pumān bhavet }||\thinspace13:30\thinspace||%
\translation{On even days, he will beget a son, on odd days, a daughter. On the sixth, eighth, tenth, and twelfth days: it will be a male. }

  \maintext{pañcamī saptamī caiva navamy ekādaśī striyaḥ |}%

  \maintext{samarakte ca śukre ca śyāmaḥ saṃjāyate pumān }||\thinspace13:31\thinspace||%
\translation{On the fifth, seventh, ninth, and eleventh days: a female. When blood and semen are even, the man will be dark-coloured. }

  \maintext{rudhiraṃ tv ekarātreṇa kalalaṃ pratipadyate |}%

  \maintext{kalalaṃ pañcarātreṇa arbudatvaṃ prajāyate }||\thinspace13:32\thinspace||%
\translation{The blood will turn into a spot in one day. This spot will become a lump in five days. }

  \maintext{arbudaḥ saptarātreṇa māṃsapeśīsamudbhavaḥ |}%

  \maintext{dvitīyasaptarātreṇa tat sarvaṃ māṃsaśoṇitam }||\thinspace13:33\thinspace||%
\translation{The lump will become a piece of flesh by the end of the first week. By the end of the second week, the whole thing will be flesh and blood. }

  \maintext{tṛtīyasaptarātreṇa hṛdayaṃ jāyate tataḥ |}%

  \maintext{tataḥ sarvāṇi gātrāṇi śiraś caivopajāyate }||\thinspace13:34\thinspace||%
\translation{By the end of the third week, it will have a heart. Then all the limbs and the head are formed. }

  \maintext{hṛdaye jāyamāne tu mūrcchā tandrir arocakaḥ |}%

  \maintext{striyāś chardiḥ prasekaś ca daurbalyaṃ copajāyate }||\thinspace13:35\thinspace||%
\translation{When the heart is developing, [the women will experience] fainting, exhaustion, and loss of apetite. She will experience nausea, vomiting and weakness. }

  \maintext{tasya hi hṛdayaṃ nārī yadi bhidyati kiṃcana |}%

  \maintext{bhakṣyaṃ lehyaṃ tathā peyam upabhogāṃs tathārthayet }||\thinspace13:36\thinspace||%
\translation{If the woman's heart is longing for something, and she asks for any food, be it chewable, sippable, drinkable, or any kind of delicacy, \blankfootnote{13.36 The translation of this verse is tentative and presupposes
  that the construction \textit{tasya ... hṛdayaṃ nārī} stands for
  \textit{tasyā nāryā hṛdayaṃ}. Perhaps understand \textit{bhidyati} 
  in \textit{pāda} b as \textit{bhedayati} {\rm (}causative{\rm )}.
 }}

  \maintext{śayanāsanadānāni vastrāṇy ābharaṇāni ca |}%

  \maintext{yad yad ākāṅkṣate kiṃcit tadāsyaiva pradāpayet }||\thinspace13:37\thinspace||%
\translation{a bed, a seat, gifts, clothes, or jewellery, anything she desires, one should give that to her immediately. \blankfootnote{13.37 Perhaps understand \textit{tadāsyaiva} {\rm (}or \textit{tad asyaiva}{\rm )} 
  as \textit{tadaiva asyai} {\rm (}or \textit{tad eva asyai}{\rm )}.
 }}

  \maintext{nāyāsaṃ kārayec cāsyā na caivam avamānayet |}%

  \maintext{mukham āpāṇḍuraṃ snigdhaṃ kālatvaṃ stanakakṣayoḥ }||\thinspace13:38\thinspace||%
\translation{One should not cause her trouble, and should not ignore her. Her face is pale and greasy, her breasts and armpits dark. }

  \maintext{śarīraṃ ca śriyā juṣṭaṃ pīnoruśroṇivaṅkṣaṇam |}%

  \maintext{liṅgair ebhir vijānīyād garbhe jīvaṃ pratiṣṭhitam }||\thinspace13:39\thinspace||%
\translation{Her body is inhabited by beauty, her thighs, buttocks, and groin are round and fleshy. By these sings, one should know that there is a [new] life in the womb. }

  \maintext{caturthasaptarātreṇa śiraś caivopajāyate |}%

  \maintext{pañcamasaptarātreṇa grīvā tatropajāyate }||\thinspace13:40\thinspace||%
\translation{By the fourth week, the head develops. By the fifth week, the neck develops. }

  \maintext{ṣaṣṭhamasaptarātreṇa skandhagātraṃ prajāyate |}%

  \maintext{saptamasaptarātreṇa pṛṣṭhavaṃśaḥ prajāyate }||\thinspace13:41\thinspace||%
\translation{By the sixth week, the shoulders and the limbs form. By the seventh week, the backbone develops. }

  \maintext{aṣṭamasaptarātreṇa pāṇī jāyata cobhayau |}%

  \maintext{saptarātraṃ nava prāpya jāyate hy urapañjaram }||\thinspace13:42\thinspace||%
\translation{By the eighth week, the two hands form. By the ninth week, the ribs develop. \blankfootnote{13.42 I conjecture \textit{jāyata} {\rm (}for a standard \textit{parasmaipada} dual \textit{jāyataḥ}{\rm )} in \textit{pāda} b,
  influenced by \msCb's reading in 13.43b, to restore the metre.
 }}

  \maintext{daśame saptarātre ca pādau jāyata cobhau |}%

  \maintext{udaraṃ copajāyeta saptaikādaśarātrike }||\thinspace13:43\thinspace||%
\translation{By the tenth week, the two feet develop. By the eleventh week, the abdomen forms. }

  \maintext{dvādaśasaptarātreṇa kukṣipārśvaḥ prajāyate |}%

  \maintext{saptatraidaśarātreṇa kaṭis tatropajāyate }||\thinspace13:44\thinspace||%
\translation{By the twelfth week, the flanks of the abdomen form. By the thirteenth week, its buttocks develop. }

  \maintext{navaty aṣṭa ca rātreṇa jāyate sūtraviṃśatiḥ |}%

  \maintext{saptapañcadaśāhena sarvamedaḥ prajāyate }||\thinspace13:45\thinspace||%
\translation{By the ninety-eighth night [i.e.\ by the fourteenth week], the ten fingers and ten toes develop. By the fifteenth week, all the fat is developed. \blankfootnote{13.45 The word \textit{sūtra} is unusual in the sense of `a finger/toe'.
  I base my translation on \SDHU\ 8.32cd, which seems to 
  point to the same period of pregnancy:
  \textit{māsaiś caturbhir aṅgulyaḥ prajāyante yathākramam};
  `By the fourth month the fingers develop in due order.'
 }}

  \maintext{ṣoḍaśasaptarātreṇa asthi sarvāṇi jāyate |}%

  \maintext{saptasaptadaśāhena jāyate snāyubandhanam }||\thinspace13:46\thinspace||%
\translation{By the sixteenth week, all the bones are formed. By the seventeenth week, the sinews are fixed. }

  \maintext{saptamāṣṭādaśāhena jāyate mukhamaṇḍalam |}%

  \maintext{saptonaviṃśarātreṇa ghrāṇavaṃśaḥ prajāyate }||\thinspace13:47\thinspace||%
\translation{By the eighteenth week, the face develops. By the nineteenth week, the nasal canal is formed. }

  \maintext{saptaviṃśatirātreṇa netranālaṃ prajāyate |}%

  \maintext{saptaikaviṃśarātreṇa karṇayugmaṃ prajāyate }||\thinspace13:48\thinspace||%
\translation{By the twentieth week, the veins of the eye develop. By the twenty-first week, the two ears form. }

  \maintext{dvāviṃśasaptarātreṇa jāyate dvau bhruvau tataḥ |}%

  \maintext{saptatriviṃśarātreṇa gaṇḍayugmaṃ prajāyate }||\thinspace13:49\thinspace||%
\translation{By the twenty-second week, the two eyebrows form. By the twenty-third week, the two cheeks develop. }

  \maintext{caturviṃśatisaptāhe oṣṭhayugmaṃ prajāyate |}%

  \maintext{pañcaviṃśatisaptāhe jihvā jāyeta sundari }||\thinspace13:50\thinspace||%
\translation{By the twenty-fourth week, the two lips develop. By the twenty-fifth week, the tongue is born, O Sundarī. }

  \maintext{ṣaḍviṃśasaptarātreṇa dantapālī prajāyate |}%

  \maintext{saptaviṃśatisaptāhe jāyate vṛṣaṇadvayam }||\thinspace13:51\thinspace||%
\translation{By the twenty-sixth week, the gums develop. By the twenty-seventh week, the testicles form. }

  \maintext{aṣṭāviṃśatisaptāhe bhagaliṅgaṃ prajāyate |}%

  \maintext{ūnaviṃśatisaptāhe jāyate ca tvag eva ca }||\thinspace13:52\thinspace||%
\translation{By the twenty-eighth week, the womb and the penis develop. By the twenty-ninth week, the skin forms. }

  \maintext{triṃśatisaptarātreṇa jāyate nābhimaṇḍalam |}%

  \maintext{saptaikatriṃśarātreṇa sarvarandhraṃ prajāyate }||\thinspace13:53\thinspace||%
\translation{By the thirtieth week, the navel develops. By the thirty-first week, all the cavities are formed. }

  \maintext{dvātriṃśatsaptarātreṇa nakhaviṃśati jāyate |}%

  \maintext{tretriṃśatsaptarātreṇa roma keśaś ca jāyate }||\thinspace13:54\thinspace||%
\translation{By the thirty-second week, the twenty nails are formed. By the thirty-third week, hair on the body and on the head grow. \blankfootnote{13.54 Note °\textit{viṃśati} for °\textit{viṃśatir} in \textit{pāda} b metri causa.
 }}

  \maintext{saptarātracatustriṃśe sarvasandhiḥ prajāyate |}%

  \maintext{pañcatriṃśatisaptāhe sarvamarma prajāyate }||\thinspace13:55\thinspace||%
\translation{By the thirty-fourth week, all the joints are formed. By the thirty-fifth week, all the \textit{marman}-joints are formed. }

  \maintext{ṣaḍtriṃśasaptarātreṇa vedanā copajāyate |}%

  \maintext{saptatriṃśatisaptāhe īrṣādveṣaḥ prajāyate }||\thinspace13:56\thinspace||%
\translation{By the thirty-six week, consciousness arises. By the thirty-seventh week, envy and hatred arise. }

  \maintext{aṣṭatriṃśatisaptāhe pañcātmakasamanvitam |}%

  \maintext{sarvāṅgam aṅgasampūrṇaḥ paripakvaḥ sa tiṣṭhati }||\thinspace13:57\thinspace||%
\translation{By the thirty-eighth week, being composed of the five elements {\rm (}\textit{pañcātmaka}{\rm )}, having all limbs and a full body, it is fully developed. }

  \maintext{mātuḥ śvāśitapītaṃ ca nābhisūtrāgamena tu |}%

  \maintext{prajātasyopadhāryante garbhasthasyaiva jantavaḥ }||\thinspace13:58\thinspace||%
\translation{It is breathing and drinking from the mother, from a flow [of nutrients] through the umbilical cord. Humans nourish a child that has been born just as they do a baby in the womb. \blankfootnote{13.58 The translation of \textit{pāda}s cd is tentative. I take 
  \textit{upadhāryante} as a causative {\rm (}\textit{upadhārayanti}{\rm )}.
 }}

  \maintext{tataḥ praviśate cittaṃ nidrāsvapnaṃ yathā tathā |}%

  \maintext{nopalabhyati sūkṣmatvād araṇy agnir yathā tathā }||\thinspace13:59\thinspace||%
\translation{Then consciousness enters the mind, as if sleep and dream. It cannot [be?] perceived because of its subtlety, like the fire [that resides inside] the sticks [that produce fire cannot be seen?]. }

  \maintext{garbhodakena siktāṅgo jarāyupariveṣṭitaḥ |}%

  \maintext{jātiṃ smarati tatrastho jantuś cetaḥsamanvitaḥ }||\thinspace13:60\thinspace||%
\translation{Wet with the amniotic fluid, he is surrounded by the fetal membranes. Being in there, the child, now being conscious, starts remembering [previous] births. }

  \maintext{mṛtaś cāhaṃ punarjāto bhūyaś caiva punar mṛtaḥ |}%

  \maintext{sthāvarāṇāṃ sahasreṣu jāto 'smi vividheṣu ca }||\thinspace13:61\thinspace||%
\translation{`I died and was then reborn. And then I died again. I was reborn in thousands of different plants, }

  \maintext{tiryagyonisahasreṣu preteṣu narakeṣu ca |}%

  \maintext{caturvarṇavivarṇeṣu mānuṣeṣu sahasraśaḥ }||\thinspace13:62\thinspace||%
\translation{in thousands of animals, as ghosts in hells, as thousands of humans in the four main social classes {\rm (}\textit{varṇa}{\rm )} and in mixed castes. }

  \maintext{sāmprataṃ ca punar garbhaḥ kleśaprāptaḥ suduḥsahaḥ |}%

  \maintext{idānīṃ jātamātro 'haṃ saṃskāraiś cāpi saṃskṛtaḥ }||\thinspace13:63\thinspace||%
\translation{And now I am again an embryo, in unbearable anguish. Now I am a newborn baby, being purified by rituals {\rm (}\textit{saṃskāra}{\rm )}. }

  \maintext{yogam evābhisevāmi sāṃkhyaṃ vā pañcaviṃśakam |}%

  \maintext{yatra janmajarā nāsti yatra mṛtyuś ca nāsti vai }||\thinspace13:64\thinspace||%
\translation{I will do nothing but practise yoga or the Sāṃkhya of twenty-five [\textit{tattva}s]. Where there is no birth and no aging, and where there is no death, }

  \maintext{yatra brahma paraṃ vaidyaṃ cariṣyāmi yatavrataḥ |}%

  \maintext{evam ādīny anekāni cintayitvā punaḥ punaḥ }||\thinspace13:65\thinspace||%
\translation{where the Brahman is the ultimate doctor, I shall go there, keeping my vows firmly.' It thinks about these and many similar things again and again. \blankfootnote{13.65 That the \textit{brahman} is the ultimate doctor {\rm (}\textit{vaidyaṃ}{\rm )} or
  something to be known {\rm (}\textit{vedyaṃ}{\rm )} here is debateble.
  I have chosen the former as a sort of lectio difficilior.
 }}

  \maintext{yāvat tiṣṭhati garbhastho jātiṃ smarati pūrvikām |}%

  \maintext{tato jāyati kaṣṭena mahākleśena mānavaḥ }||\thinspace13:66\thinspace||%
\translation{While in the womb, it remembers its previous births. Then the human is born with great difficulty and great pain, }

  \maintext{yoniyantrasutīvreṇa pīḍyamānaḥ suduḥkhitaḥ |}%

  \maintext{jātamātre smṛtibhraṃśo bhavatīha acetanaḥ }||\thinspace13:67\thinspace||%
\translation{tormented by the mechanisms of the vagina, in incredible pain. As soon as it is born, it forgets its [previous] lives in this world, unconscious. }

  \maintext{māyāmudgaratīvreṇa hataḥ kiṃ śubham ācaret |}%

  \maintext{eṣa garbhasamutpattiḥ kathito 'smi varānane |}%

  \maintext{duḥkhasaṃsāraprathamaḥ kiṃ bhūyaḥ śrotum icchasi }||\thinspace13:68\thinspace||%
\translation{[The baby] is hit by the hard hammer of illusion {\rm (}\textit{māyā}{\rm )}. How could it act in an auspicious way? This is how I have told you the formation of the embryo, O Varānanā, the first [stage] of the suffering in transmigration {\rm (}\textit{saṃsāra}{\rm )}. What else would you like to hear? \blankfootnote{13.68 In fact, 13.68 in \msCa, \msCb, and \msNa\ end with \textit{icchasīti}, 
  leading into the colophon {\rm (}\textit{vṛṣasārasaṃgrahe...}{\rm )}, but \msCb\
  repeats the \textit{iti} thus: \textit{icchasīti\thinspace || iti vṛṣasārasaṃgrahe...}
 }}

\centerline{\maintext{\dbldanda\thinspace iti vṛṣasārasaṃgrahe garbhotpattir{ }adhyāyas{ }trayadaśamaḥ\thinspace\dbldanda}}
\translation{Here ends the thirteenth chapter in the \textit{Vṛṣasārasaṃgraha} chapter on The development of the embryo.}

  \chptr{caturdaśamo 'dhyāyaḥ}
\fancyhead[CO]{{\footnotesize\textit{Translation of chapter 14}}}%

  \trchptr{Chapter Fourteen}%

  \subchptr{deharūpavarṇabhedāni}%

  \trsubchptr{Differences in body-shapes and skin-colours}%

  \maintext{devy uvāca |}%

  \maintext{atidīrgho 'tihrasvaś ca pumān kenopajāyate |}%

  \maintext{atigauro 'tikṛṣṇaś ca naro bhavati kiṃ prabho }||\thinspace14:1\thinspace||%
\translation{Devī spoke: How does a person become extremely tall or very short? Why does one person have an extremely fair complexion, and the other a very dark one, O Lord? }

  \maintext{bhagavān uvāca |}%

  \maintext{gṛhītagarbhā yā nārī nityam uttānaśālinī |}%

  \maintext{prasāritavibhaktātmā so 'tidīrghaḥ prajāyate }||\thinspace14:2\thinspace||%
\translation{The Lord spoke: If a pregnant woman always stands upright, the son will have an expanded and symmetrical body, and will be extremely tall. }

  \maintext{gṛhītagarbhā yā nārī śete saṃkucitā sadā |}%

  \maintext{rasānnādīni kaṭukaṃ sevanā hrasva jāyate }||\thinspace14:3\thinspace||%
\translation{If a pregnant woman always lies down in a contracted posture, [and] indulges in pungent juices and food, the son will be short. \blankfootnote{14.3 Understand \textit{pāda} c as \textit{rasānnādīni kaṭukāni}, and 
  note the stem-form word \textit{hrasva} metri causa in \textit{pāda} d.
 }}

  \maintext{gṛhītagarbhā yā nārī nityaṃ kṣīropasevinī |}%

  \maintext{varakodravaśālīṃś ca bhuṅkte cāpi yavaudanam |}%

  \maintext{śuklavastrasrajā yuktā sātigauraṃ prasūyate }||\thinspace14:4\thinspace||%
\translation{If a pregnant woman always drinks milk, eats \textit{vara} and \textit{kodrava} grains, and rice, and barley-porrige, wears white clothes and white garlands, she will give birth to [offspring] with extremely pale complexion. }

  \maintext{gṛhītagarbhā yā nārī kāladhānyāni sevate |}%

  \maintext{māṣakṛṣṇatilāmudgaṃ tathā kṛṣṇayavodanam |}%

  \maintext{kṛṣṇavastrasrajādīni tasyāḥ kṛṣṇaḥ prajāyate }||\thinspace14:5\thinspace||%
\translation{If a pregnant woman loves black-coloured grains, black beans, black sesamum, black Mungo beans, and porridge made of black barley, [and wears] dark clothes and garlands, she will have a son with dark skin. }

  \subchptr{jātidoṣāni}%

  \trsubchptr{Birth defects}%

  \maintext{devy uvāca |}%

  \maintext{jātyandho jāyate kasmāt ṣaṇḍho bhīrur hatendriyaḥ |}%

  \maintext{kubjo vā vāmano vāpi paṇḍaḥ sthūlaśiraḥ katham }||\thinspace14:6\thinspace||%
\translation{Devī spoke: Why is a person born blind, gender-neutral, timid, with ruined sense-faculties, hump-backed, dwarfish, a weakling, or with an enourmous head? }

  \maintext{bhagavān uvāca |}%

  \maintext{gṛhītagarbhā yā nārī tīkṣṇoṣṇāny upasevate |}%

  \maintext{laśunāni palāṇḍūni karañjamūlakāni ca }||\thinspace14:7\thinspace||%
\translation{The Lord spoke: If a pregnant woman indulges in hot and pungent [food], in garlic and onion, the root of the wood-apple tree, }

  \maintext{pippalīṃ śṛṅgaveraṃ ca sarṣapān maricāni ca |}%

  \maintext{āsavaṃ ca parikliṣṭā ye cānye kaṭutiktakāḥ |}%

  \maintext{tīkṣṇaṃ tu sevamānā yā jātyandhaṃ janayet sutam }||\thinspace14:8\thinspace||%
\translation{berries, ginger, mustard, and black pepper, alcohol, and other [types of] harmful[?], pungent and bitter [food], [i.e.] she who enjoys hot [food], will give birth to a blind child. }

  \maintext{mithyopacārāḥ strīpuṃso vyāpanne śukraśoṇite |}%

  \maintext{yadā garbhāśaye raktaṃ striyāḥ pūrvaṃ niṣicyate |}%

  \maintext{paścāc chukraṃ raktakāle tadā ṣaṇḍaḥ prasūyate }||\thinspace14:9\thinspace||%
\translation{[If] the conduct[?] of the woman and the man is improper, the semen and blood are spoiled. When the woman's blood flows into the womb first, at the time of her period???, and then the semen, then a gender-neutral child will be born. }

  \maintext{trastodvignā yadā bhītā strī puṃsā sūyate prajā |}%

  \maintext{tatra yo jāyate garbhād bhīruḥ krandanako bhavet }||\thinspace14:10\thinspace||%
\translation{When a woman is frightened, terrified, afraid of the man, and offspring is conceived, then what is born from her womb will be a coward, a crybaby. \blankfootnote{14.10 puṃsāṃ is odd
 }}

  \maintext{visargakāle śukrasya vighna utpadyate yadā |}%

  \maintext{indriyāvartavighne tu tadā jāyed anindriyaḥ }||\thinspace14:11\thinspace||%
\translation{Should any obstacle arise in front of the semen at the moment of ejaculation, then [the child] will be born without any sense faculties because of the blocking of the development of the senses[????]. }

  \maintext{gṛhītagarbhā yā nārī vātalāny upasevate |}%

  \maintext{kaṭukāni kaṣāyāni tiktāni ca viśeṣataḥ }||\thinspace14:12\thinspace||%
\translation{If a pregnant woman indulges in eating pulses, in pungent, astringent, and especially, bitter [food], }

  \maintext{vātaḥ prakupitas tasyā garbham ābhujya tiṣṭhati |}%

  \maintext{kubjas tu jāyate tasmād garbhād vātanipīḍanāt }||\thinspace14:13\thinspace||%
\translation{her \textit{vāta} humour will be agitated, and it will be pressing down the womb. A hump-backed child will be born from that womb because of the compression of \textit{vāta}. }

  \maintext{nityam āsanaśīlā yā tathā cotkuṭakāsanā |}%

  \maintext{tasyāḥ saṃhanyate garbho vāmanas tena jāyate }||\thinspace14:14\thinspace||%
\translation{She who is always sitting or squatting will crush the embryo, which will by this be born a dwarf. }

  \maintext{ativyāyāmaśīlā tu yā nārī viṣamāsanī |}%

  \maintext{garbhaḥ saṃkṣubhyate tasyāḥ paṇḍas tenopajāyate }||\thinspace14:15\thinspace||%
\translation{If a woman exercises too much, [or often] sits in an unbalanced way, her womb will be shaken, and by this a weakling will be born. }

  \maintext{gṛhītagarbhā yā nārī rūḍhadhānyāni sevate |}%

  \maintext{vātaśleṣma śirastho vai tasya garbhasya kupyate |}%

  \maintext{tataḥ sthūlaśirās tena pumān jāyaty asaṃśayaḥ }||\thinspace14:16\thinspace||%
\translation{If a pregnant woman indulges in grain that is sprouting[?], the Wind and Phlegm in the the embryo's head will swell up. Then, because of this, a man with an enormous head will be born, no doubt. }

  \maintext{devy uvāca |}%

  \maintext{karālāṅgā hanuḥ paṅgur mūko gadgadabhāṣakaḥ |}%

  \maintext{vivṛtākṣas tv anakṣo vā bhaved duḥkhagudaḥ katham }||\thinspace14:17\thinspace||%
\translation{Devī spoke: How can [a person] be born with terrible limbs, with a terrible jaw[?], as lame, a mute, stammering, [always] open-eyed, or eyeless, or having a painful anus? }

  \maintext{bhagavān uvāca |}%

  \maintext{karālastanadoṣeṇa jāyate mānavas tathā |}%

  \maintext{atha karālaṃ kurute nārī lamboticūcukā |}%

  \maintext{tasmād etena doṣeṇa karālo jāyate pumān }||\thinspace14:18\thinspace||%
\translation{The Lord spoke: It is by the fault of [the mother's] having terrible[?] breasts that a man is born like that [i.e. with terrible limbs]. Now, a woman with extremely dangling nipples does terrible things[???]. Therefore, by this fault, a terrible man is born. }

  \maintext{gṛhītagarbhā yā nārī raktapittāmayārditā |}%

  \maintext{gohanuṃ janayaty eṣā raktapittaprakopitā }||\thinspace14:19\thinspace||%
\translation{If a pregnant woman is afflicted by diseases of the blood and the Pitta, having her blood and Pitta irritated, she will give birth to a child with a cow's jaw[??]. }

  \maintext{gṛhītagarbhā yā nārī vātaśūlair upadrutā |}%

  \maintext{śukrodāvartanī cāpi paṅguṃ janayate sutam }||\thinspace14:20\thinspace||%
\translation{If a pregnant woman is pained by flatulence, or retains her menstrual discharge, she will give birth to a lame child. }

  \maintext{kṣudhārtā vedanārtā ca satataṃ copavāsinī |}%

  \maintext{mūkaṃ janayate putraṃ dauhṛdaṃ ca vimānitā }||\thinspace14:21\thinspace||%
\translation{If she is afflicted by thirst and pain, and always fasts, she will give birth to a mute, her longings of pregnancy {\rm (}\textit{dauhṛda}{\rm )} having been refused. \blankfootnote{14.21 Cf. Suśruta 3.3.21:
  \textit{yeṣu yeṣv indriyārtheṣu dauhṛde vai vimānanā\thinspace |
  prajāyeta sutasyārtis tasmiṃst asmiṃs tathendriye||.}
  Yājñavalkya 3.79:
  \textit{dauhṛdasyāpradānena garbho doṣam avāpnuyāt\thinspace |
  vairūpyaṃ maraṇaṃ vāpi tasmāt kāryaṃ priyaṃ striyāḥ\thinspace ||}
 }}

  \maintext{gṛhītagarbhā yā nārī visṛjen māsamāsikam |}%

  \maintext{anakṣo jāyate tasyā garbhaśoṇitasaṃkṣayāt }||\thinspace14:22\thinspace||%
\translation{If a pregnant woman discharges [menstrual blood] every month, she will give birth to a blind child due to the damage made by the blood to the embryo. }

  \maintext{arśagrastā yadā nārī vātodāvartapīḍitā |}%

  \maintext{gṛhītagarbhā rūkṣāṇi vātalāny upasevate }||\thinspace14:23\thinspace||%
\translation{If a pregnant woman is afflicted by h\ae morrhoids, and is pained by bowel diseases caused by Wind {\rm (}\textit{vāta}{\rm )}, and indulges in astringent [foods and?] pulses, }

  \maintext{vātasthānaṃ tatas tasyā garbhasyāpīḍitaṃ bhavet |}%

  \maintext{agudo jāyate tasmāj jātaś cāpi na jīvati }||\thinspace14:24\thinspace||%
\translation{Because of this, the location of the Wind in her womb becomes compressed. Therefore [the child] will be born without an anus, but even if it is born, it will not survive. }

  \maintext{devy uvāca |}%

  \maintext{hīnāṅgo jāyate kasmād adhikāṅgo 'pi vā katham |}%

  \maintext{śvetapiṅgekṣaṇaḥ kasmāt kathaṃ lohitalocanaḥ }||\thinspace14:25\thinspace||%
\translation{Devī spoke: Why is a child born without limbs or with extra limbs? Why will it have white or yellowish eyes or red eyes? }

  \maintext{bhagavān uvāca |}%

  \maintext{garbhasya jāyamānasya yad aṅge jāyate 'nilaḥ |}%

  \maintext{vātābhyāṃ śleṣmaṇā tasya tad aṅgaṃ parihīyate |}%

  \maintext{hīnāṅgo jāyate tasmāt pumān vātaprakopataḥ }||\thinspace14:26\thinspace||%
\translation{The Lord spoke: If Wind is produced in any of the limbs of the embryo that is being formed, that limb of his will lack Wind {\rm (}\textit{vāta}{\rm )} and Phlegm {\rm (}\textit{śleṣman}{\rm )}, and therefore that human will be born lacking a limb because of the disturbance of Wind. \blankfootnote{14.26 Understand \textit{vātābhyāṃ śleṣmaṇā} in \textit{pāda} c as
  \textit{śleṣmavātābhyāṃ}?
 }}

  \maintext{gṛhītagarbhā yā nārī madhurāṇy upasevate |}%

  \maintext{śṛṅgāṭakāṅkaloḍyāni śālūkāni bisāni ca }||\thinspace14:27\thinspace||%
\translation{If a pregnant woman indulges in sweets, \textit{śṛṅgāṭaka}-fruits, ginger, lotus-roots, lotus-stalks, }

  \maintext{mocaṃ tālaphalaṃ caiva nārikelaphalaṃ tathā |}%

  \maintext{abhīkṣṇaṃ sevamānā tu adhikāṅgaṃ prasūyate }||\thinspace14:28\thinspace||%
\translation{bananas, the fruit of the fan-palm, and coconut, constantly eating these, she will give birth to [a child] with extra limbs. }

  \maintext{piṅgākṣaḥ śleṣmapittābhyāṃ śvetākṣaḥ śleṣmaṇā bhavet |}%

  \maintext{vātapittena raktākṣaḥ puruṣas tūpajāyate }||\thinspace14:29\thinspace||%
\translation{Phlegm and Pitta will cause the child to have yellowish eyes, Phlegm [in itself] will cause white eyes. }

  \maintext{devy uvāca |}%

  \maintext{kathaṃ vā jāyate putraḥ kanyakā kena jāyate |}%

  \maintext{apumān kena jāyeta dviyamā triyamā tathā }||\thinspace14:30\thinspace||%
\translation{Devī spoke: What determines whether a son or a daughter is born? Why is a non-man [a gender-neutral child] born or twins and triplets? }

  \maintext{bhagavān uvāca |}%

  \maintext{śukrādhikaḥ pumān jñeyaḥ kanyā raktādhikā bhavet |}%

  \maintext{raktaśukrasamatvena jāyate sa napuṃsakaḥ }||\thinspace14:31\thinspace||%
\translation{The Lord spoke: A male is to be known as having an excess of semen. A female has an excess of blood. A gender-neutral child is born because of a balance of blood and semen. }

  \maintext{piṇḍībhūto yadā garbhaṃ māruto vibhajed dvidhā |}%

  \maintext{evaṃ te dviyamā jñeyās triyamāś ca tridhā kṛte }||\thinspace14:32\thinspace||%
\translation{When the embryo is still a lump, the Wind can divide it into two. This is what is to be know as twins. When divided into three, it is triplets. }

  \maintext{devy uvāca |}%

  \maintext{śoṇitaṃ māṃsa medaṃ ca asthi majjā ca pañcamī |}%

  \maintext{śarīrasthāni dṛśyante śukrasthānaṃ na dṛśyate }||\thinspace14:33\thinspace||%
\translation{Devī spoke: The blood, the flesh, the fat, the bones, and the fifth, the marrow, can be seen in the body. The location of semen cannot be seen. }

  \maintext{tasyopapatti sthānaṃ ca jñātum icchāmi tattvataḥ |}%

  \maintext{kathayasva trilokeśa cchettum arhasi saṃśayam }||\thinspace14:34\thinspace||%
\translation{I wish to know about its production and location truly. O Trilokeśa, please put an end to my doubts. }

  \maintext{bhagavān uvāca |}%

  \maintext{manaḥ śukrasya prabhavaṃ ghrāṇaṃ śrotraṃ tathākṣiṇī |}%

  \maintext{sthānaṃ tu sarvāṅgagataṃ sparśāt sparśaḥ pravartate }||\thinspace14:35\thinspace||%
\translation{The Lord spoke: Semen is the origin of the mind {\rm (}\textit{manas}{\rm )}, the nose, the ears, and the eyes. As for its location, it is in everywhere in the body. ...... \blankfootnote{14.35 \textit{prabhava} used oddly?
 }}

  \maintext{yathā niṣiktaṃ kṣīraṃ tu payasā dadhi jāyate |}%

  \maintext{pramathyamānadadhnas tu sarpiso 'pi tathāgamaḥ }||\thinspace14:36\thinspace||%
\translation{Just as milk sprinkled with thickened milk[?] becomes yoghurt, and just as butter comes out of churned coagulated milk, }

  \maintext{evaṃ śarīraṃ nirmanthec chukraṃ śukravahā sirā |}%

  \maintext{pūrayitvānupūrveṇa asthayo pratipadyate }||\thinspace14:37\thinspace||%
\translation{thus does semen churn out the body; the tube that carries the semen fills [the body] in due order, and the bones are formed. }

  \maintext{tatas tu tāḥ śukravahā meḍhranāḍīm anusṛtāḥ |}%

  \maintext{na śukratantu siñcanti tasmād garbhasya sambhavaḥ }||\thinspace14:38\thinspace||%
\translation{Then those [tubes] that carry the semen and join the tube of the penis, .... and the f\oe tus is formed. }

  \maintext{devy uvāca |}%

  \maintext{kathaṃ vedayate jātiṃ kathaṃ jātismaro bhavet |}%

  \maintext{etasmin saṃśayaṃ me 'dya chettum arhasi śaṅkara }||\thinspace14:39\thinspace||%
\translation{Devī spoke: How does one experience birth? How can one remember one's birth? O Śaṅkara, please put an end now to my doubts about this topic. }

  \maintext{bhagavān uvāca |}%

  \maintext{bhāvitātmā ca yo jantur devi bhāgādhikaṃ ca yat |}%

  \maintext{buddhivijñānasaṃyuktaḥ sa jātiṃ smarate pumān }||\thinspace14:40\thinspace||%
\translation{The Lord spoke: A man whose soul is lofty, O Devī, ..... is endowed with intelligence and wisdom, can remember [his previous] birth. }

  \maintext{devy uvāca |}%

  \maintext{kathaṃ sadyogṛhītasya liṅgaṃ garbhasya dṛśyate |}%

  \maintext{etat kathaya deveśa rahaḥ kāle maheśvara }||\thinspace14:41\thinspace||%
\translation{Devī spoke: How can the signs of a f\oe tus just having been conceived be seen? O Deveśa, O Maheśvara, tell me this secret in due time. }

  \maintext{maheśvara uvāca |}%

  \maintext{pipāsāromaharṣaś ca vepanaṃ gātrasīdanam |}%

  \maintext{nidrāsvedaṃ ca tandrī ca muhūrtam upajāyate }||\thinspace14:42\thinspace||%
\translation{Maheśvara spoke: Thirst, bristling of the hairs of the body, trembling, exhaustion of the limbs, sweating in sleep, and lassitude, appears in a moment. }

  \maintext{nikledatvaṃ kharatvaṃ ca yonyāṃ samupajāyate |}%

  \maintext{na cārtavaṃ vai dṛśyeta śukrasya rajaso 'pi vā | }%

  \maintext{sadyogṛhītagarbhāyā liṅgāny etāni tattvataḥ }||\thinspace14:43\thinspace||%
\translation{Wetness[?] and pain[?] appear in the vagina. But no discharge is seen, neither of semen, nor of blood. These are the true signs when a f\oe tus has just been conceived. }

  \maintext{devy uvāca |}%

  \maintext{kena liṅgena vijñeyaṃ putrajanma maheśvara |}%

  \maintext{kanyakā kena liṅgena jāyate kathayasva me }||\thinspace14:44\thinspace||%
\translation{Devī spoke: By what signs can the conception of a son be recognized, O Maheśvara? By what sign is [it clear that] a daughter is going to be born? Teach me. }

  \maintext{bhagavān uvāca |}%

  \maintext{yadorujaṅghapārśvaṃ ca dakṣiṇaṃ yadi hy unnatam |}%

  \maintext{dakṣiṇaṃ vipulaṃ netraṃ tadā putraḥ prajāyate }||\thinspace14:45\thinspace||%
\translation{The Lord spoke: When and if the right foot, thigh, and side are swollen, and the right eye is widened, a son will be born. }

  \maintext{vāmaṃ caiva yadā paśyet tadā jāyeta kanyakā |}%

  \maintext{unnataṃ madhyamasthānaṃ tadā jāyen napuṃsakaḥ }||\thinspace14:46\thinspace||%
\translation{When the left [side] is seen [as swollen], then a daughter will be born. When the middle part of the body is swollen, then a gender-neutral child will be born. }

  \maintext{devy uvāca |}%

  \maintext{puṃsāṃ kapolaromāni khalitaṃ kena jāyate |}%

  \maintext{kathaṃ strīṇāṃ na jāyeta romāṇi khalitaṃ tathā }||\thinspace14:47\thinspace||%
\translation{Devī spoke: Why does beard and baldness arise for men and not for women? }

  \maintext{bhagavān uvāca |}%

  \maintext{tathā vṛṣaṇagā jantor yasya retovahā sirā |}%

  \maintext{nibaddhā mastake tās tu kapolās tu samāśritāḥ }||\thinspace14:48\thinspace||%
\translation{The Lord spoke: Men have a tube that carries the semen from the testicles. They are connected to the head, and the cheeks. }

  \maintext{taiḥ kapoleṣu romāṇi jāyante antaretasaḥ |}%

  \maintext{khalitaṃ śukradoṣeṇa narāṇām upajāyate }||\thinspace14:49\thinspace||%
\translation{There on the cheeks, hair grows because there is semen inside. Baldness arises by the defect of men's semen. }

  \maintext{sirā śukravahā strīṇāṃ na syād yasmān na jāyate |}%

  \maintext{yo tmāṣālo ca kas tv agnir dṛṣṭimaṇḍalasaṃśritaḥ }||\thinspace14:50\thinspace||%
\translation{Women do not have a tube that carries semen, therefore... ... }

  \maintext{śoṇitaṃ soktikoṣṭasthan niśoṣayati tattvataḥ |}%

  \maintext{na vardhante 'kṣipakṣmāṇi tena romāṇi ca bhruvoḥ }||\thinspace14:51\thinspace||%
\translation{... That is why the eyelashes and the eyebrows do not grow. }

  \maintext{aśukratvāc ca nārīṇāṃ khalitaṃ nopajāyate |}%

  \maintext{chāyāvyapagatasnehā rūkṣāgātraśiroruhā |}%

  \maintext{udbhūtosmābhajaṭharā mṛtagarbhaḥ prajāyate }||\thinspace14:52\thinspace||%
\translation{Since they have no semen, women do not experience baldness. ... ... }

  \maintext{devy uvāca |}%

  \maintext{somadhātu kati jñeyā agnidhātus tatheśvara |}%

  \maintext{pṛthagbhāgaviśeṣeṇa kathayasva maheśvara }||\thinspace14:53\thinspace||%
\translation{Devī spoke: How many are the Soma constituents, and how many the Fire ones, O Īśvara? O Maheśvara, tell me with all the characteristics of the distinct elements. }

  \maintext{maheśvara uvāca |}%

  \maintext{śleṣmā medas tathā snāyu asthi danta nakhāni ca |}%

  \maintext{striyāḥ stanyaṃ ca śukraṃ ca yac ca śvetaṃ tathākṣiṣu |}%

  \maintext{eteṣāṃ saumyabhāvatvāc chvetatvam upajāyate }||\thinspace14:54\thinspace||%
\translation{The Maheśvara spoke: Phlegm, fat, sinew, bones, teeth, and nails, women's breast milk, semen, and the white of the eyes: whiteness appears in these because they are Soma-related. }

  \maintext{āgneyabhāvād raktatvaṃ kṛṣṇatvaṃ cāpi gacchati |}%

  \maintext{tvag māṃsa rudhira majjā dṛṣṭiroma tathaiva ca }||\thinspace14:55\thinspace||%
\translation{[Constituents of the body that are] Fire-related become red and black: [such as] skin, flesh, blood, marrow, and eyelashes. }

  \maintext{āgneyadhātuṃ somaṃ ca kathito 'smi varānane |}%

  \maintext{brūhi brūhi viśālākṣi yady asti tava saṃśayaḥ }||\thinspace14:56\thinspace||%
\translation{I have taught, O Varānanā, the Fiery constituents and the Soma-ones. Tell me, tell me, O Viśālākṣī, if you still have doubts. }

\centerline{\maintext{\dbldanda\thinspace iti vṛṣasārasaṃgrahe praśnavyākaraṇo nāmādhyāyaś{ }caturdaśamaḥ\thinspace\dbldanda}}
\translation{Here ends the fourteenth chapter in the \textit{Vṛṣasārasaṃgraha} called The Detailed Reply to Questions.}

  \chptr{pañcadaśamo 'dhyāyaḥ}
\fancyhead[CO]{{\footnotesize\textit{Translation of chapter 15}}}%

  \trchptr{Chapter Fifteen}%

  \subchptr{jīvavarṇanam}%

  \trsubchptr{The description of the soul}%

  \maintext{devy uvāca |}%

  \maintext{jīvabhūteti yat proktaṃ lakṣaṇaṃ kīdṛśaṃ bhavet |}%

  \maintext{sthānam asya na jānāmi rūpaṃ varṇaṃ ca īśvara }||\thinspace15:1\thinspace||%
\translation{The goddess spoke: A certain `soul being' was mentioned. What are [its] characteristics? I do not know about its location or form or colour, O Īśvara. }

  \maintext{etat kautūhalaṃ chindhi saṃśayaṃ parameśvara |}%

  \maintext{na cānyad eva paśyāmi jīvanirṇaya kīrtaya }||\thinspace15:2\thinspace||%
\translation{This is what I'm curious about. Drive my doubts away, Parameśvara. I cannot see anything else [as vitally important]. Teach me the details of the soul. \blankfootnote{15.2 \textit{Pāda} c is suspicious. It may have originally read \textit{na cānyad eva yācāmi} {\rm (}``I am not asking for anything else''{\rm )}.
  Note the stem form noun °\textit{nirṇaya} in pāda d metri causa.
 }}

  \maintext{īśvara uvāca |}%

  \maintext{jīvasya lakṣaṇaṃ devi kathituṃ kena śakyate |}%

  \maintext{na rūpavarṇaṃ jīvasya vidyate sthānam eva ca }||\thinspace15:3\thinspace||%
\translation{Īśvara spoke: O Goddess, who could be able to talk about the characteristics of the soul? There is no such thing as the form or colour of the soul or its location. }

  \maintext{vyāpi sarvagataṃ sūkṣmaṃ sarvam āśritya tiṣṭhati |}%

  \maintext{nirālambam anādhāram anaupamyaṃ nirañjanam }||\thinspace15:4\thinspace||%
\translation{Its pervasive, omnipresent, subtle, it exists dwelling in everything. It is supportless, it is not contained in anything, it is unparalleled and spotless. }

  \maintext{araṇistho yathā vahniḥ kāṣṭheṣu nopalabhyate |}%

  \maintext{tadvaj jīvo na paśyeta śarīrastho 'pi sundari }||\thinspace15:5\thinspace||%
\translation{As fire [hidden] in fire-kindling sticks[?] is not perceivable in the wood, similarly the soul cannot be seen although it dwells in the body, O Sundarī. \blankfootnote{15.5 Note \textit{paśyeta} as a passive form.
 }}

  \maintext{dadhivac ca yathā sarpir dṛśyate na ca dṛśyate |}%

  \maintext{tadvaj jīvaḥ śarīrastho dṛśyate na ca dṛśyate }||\thinspace15:6\thinspace||%
\translation{Just as ghee can and cannot be seen in[?] curd[?], in the same way the soul in the body can and cannot be seen. }

  \maintext{devy uvāca |}%

  \maintext{adṛṣṭapratyayo hy asti nāsti pratyayadarśanam |}%

  \maintext{vyāpī kathaṃ mahādeva sarvatrāvasthitaḥ katham }||\thinspace15:7\thinspace||%
\translation{The goddess spoke: Is it without any direct proof? Is there no way to directly see any proof [of its existence]? How is it pervasive, O Mahādeva? How can it be omnipresent? }

  \maintext{maheśvara uvāca |}%

  \maintext{asaṃśayo mahādevi vyāpī sarvagataḥ śivaḥ |}%

  \maintext{dṛśyatendriyasaṃyogāj jīvapratyayadarśanam }||\thinspace15:8\thinspace||%
\translation{Maheśvara spoke: It is doubtlessly pervasive, omnipresent, it is Śiva. It can be perceived through its contact with the senses. [That is] the direct perception of the evidence of [the existence of] the soul. }

  \maintext{yathākāśasthito vāyuḥ śabdasparśaguṇānvitaḥ |}%

  \maintext{tadvad dehī vijānīyād guṇaceṣṭena nānyathā }||\thinspace15:9\thinspace||%
\translation{As air in the sky is endowed with the qualities of sound and touch, similarly one can perceive the soul through the functioning of its qualities and in no other way. \blankfootnote{15.9 Although it is difficult to be certain whether the majority of the MSS read °\textit{ceṣṭena} or °\textit{veṣṭena},
  I suppose that the somewhat irregular °\textit{ceṣṭena} is the right reading 
  and that it stands for a more standard °\textit{ceṣṭayā}.
 }}

  \maintext{devy uvāca |}%

  \maintext{vyāpīti kathitaḥ pūrvaṃ jīvaḥ sarvagato 'pi ca |}%

  \maintext{taṃ vṛthā kathito 'sy adya mriyate kena hetunā }||\thinspace15:10\thinspace||%
\translation{The goddess spoke: The soul was mentioned earlier as being pervasive and also omnipresent. [I suppose] you said that [only] idly. In this case, why does [the soul] die? \blankfootnote{15.10 pāda c strange structure, but frequent in this text.
 }}

  \maintext{īśvara uvāca |}%

  \maintext{na jīvo mriyate devi sarveṣāṃ surasundari |}%

  \maintext{ghaṭāntastho yathākāśo bahirākāśavad yathā }||\thinspace15:11\thinspace||%
\translation{Īśvara spoke: O Goddess, nobody's soul ever dies, O Surasundarī. As [in the case of] space inside a pot, and space outside it, }

  \maintext{ghaṭabhinne viśālākṣi viśeṣo nopalakṣyate |}%

  \maintext{dehabhinne yadā devi vināśo nopalabhyate }||\thinspace15:12\thinspace||%
\translation{there is no perceivable difference when the pot is broken to pieces, O Viśālākṣī. [Similarly,] when the body perishes, O goddess, there is no perceivable destruction [of the soul]. }

  \maintext{susūkṣmaḥ sarvago vyāpī paramātmānam avyayaḥ |}%

  \maintext{bahir antaś ca bhūtānām acaraś cara eva saḥ }||\thinspace15:13\thinspace||%
\translation{It is extremely subtle, omnipresent, pervasive, it is the supreme soul, it is imperishable. It is outside and inside the living beings. It is immovable and moving. \blankfootnote{15.13 Note \textit{paramātmānam} for \textit{paramātmā}.
 }}

  \maintext{aprameyo 'vināśī ca aprapañcaḥ prapañcakaḥ |}%

  \maintext{sarvendriyaguṇābhāsaḥ sarvendriyavivarjitaḥ }||\thinspace15:14\thinspace||%
\translation{It is immeasurable, imperishable, unmanifest and manifest. It appears as having the qualities of all the senses but is devoid of senses. }

  \maintext{evam eṣa mahādevi jīvasya varavarṇini |}%

  \maintext{kathito 'smi samāsena kim anyac chrotum icchasi }||\thinspace15:15\thinspace||%
\translation{Thus have I briefly described to you, O Mahādevī, the soul. O Varavarṇinī, what else would you like to hear? }

  \subchptr{sāraśreṣṭham}%

  \trsubchptr{The best}%

  \maintext{devy uvāca |}%

  \maintext{sāraśreṣṭhaṃ mahādeva kathayeśāna īśvara |}%

  \maintext{śrotum icchāmi deveśa mānuṣāṇāṃ hitaṃ vada }||\thinspace15:16\thinspace||%
\translation{The goddess spoke: O Mahādeva, O Īśāna, Īśvara! Tell me what is the best in essence. I would like to hear it, O Deveśa. Tell me for the benefit of mankind. \blankfootnote{15.16 Pāda d is a clumsy paraphrase of the common \textit{mānuṣāṇāṃ hitāya ca} or similar phrases.
 }}

  \maintext{īśvara uvāca |}%

  \maintext{āśramāṇāṃ gṛhī śreṣṭho varṇaśreṣṭhā dvijātayaḥ |}%

  \maintext{aśvamedhaḥ kratuśreṣṭho japaśreṣṭho 'ghamarṣaṇaḥ }||\thinspace15:17\thinspace||%
\translation{Īśvara spoke: The best life-stage is that of the householder {\rm (}\textit{gṛhin}{\rm )}. The best social classes are the twice-born ones. The best ritual is the \textit{aśvamedha}. The best recitation is the \textit{aghamarṣaṇa}. }

  \maintext{devatānāṃ hariḥ śreṣṭhaḥ śreṣṭhā gaṅgā nadīṣu ca |}%

  \maintext{anāśanas tapaḥśreṣṭhas tīrthaśreṣṭhaḥ suradrahaḥ }||\thinspace15:18\thinspace||%
\translation{The best god is Hari. The best river is the Ganges. The best austerity is fasting. The best pilgrimage-place is Suradraha. \blankfootnote{15.18 \textit{anāśanas} {\rm (}or \textit{anāsanas} in most MSS{\rm )} stands for \textit{anaśanas} {\rm (}found only in \msNc{\rm )} but the latter would cause 
  a metrical problem, namely both the second and third syllables would be two short. This is why I retained
  the non-standard form \textit{anāśanas}.
 }}

  \maintext{kṣomaṃ vastreṣu ca śreṣṭhaṃ yaśaḥ śreṣṭhaṃ vibhūṣaṇam |}%

  \maintext{bhārataṃ śrutiṣu śreṣṭhaṃ vrataśreṣṭho dayāparaḥ }||\thinspace15:19\thinspace||%
\translation{The best cloth is linen. The best ornament is fame. The best Śruti is the Mahābhārata. The best of vows is compassion. }

  \maintext{dāneṣu cābhayaṃ śreṣṭhaṃ manaḥ śreṣṭhendriyeṣu ca |}%

  \maintext{vidyā saṃgrahaṣu śreṣṭhā satyaṃ śreṣṭhaṃ vacaḥsu ca }||\thinspace15:20\thinspace||%
\translation{The best donation is the freedom from danger. The best sense-faculty is the mind. The best way to accumulate wealth is accumulate knowledge. The best word is the truthful one. \blankfootnote{15.20 Note the form \textit{saṃgrahaṣu} in \textit{pāda} c for \textit{saṃgraheṣu} {\rm (}as in \msNc{\rm )} metri causa
 }}

  \maintext{āyudhānāṃ dhanuḥ śreṣṭhaṃ bāndhaveṣu ca mātaraḥ |}%

  \maintext{jñānam auṣadhiṣu śreṣṭhaṃ vaidyaśreṣṭhaḥ śivākṣaraḥ }||\thinspace15:21\thinspace||%
\translation{The best weapon is the bow. The best relatives are the mothers. The best medicine is knowledge. The best doctor is Śiva's syllable. }

  \maintext{akāraś cākṣaraḥ śreṣṭho dharmaśreṣṭho hy ahiṃsakaḥ |}%

  \maintext{paśuṣu saurabhī śreṣṭhā nareṣu ca narādhipaḥ }||\thinspace15:22\thinspace||%
\translation{The best letter is `a'. The best Dharma is non-violence. The best domestic animal is the cow. The best person is the king. }

  \maintext{māsi mārgaśiraḥ śreṣṭhaṃ kṛtaḥ śreṣṭhaś caturyuge |}%

  \maintext{vasanta ṛtuṣu śreṣṭhaḥ śreṣṭhaṃ cāyanam uttaram }||\thinspace15:23\thinspace||%
\translation{The best month is Mārgaśiras. The best of the four \ae ons is the Kṛta. The best season is spring. The best path of the Sun is the northern one. \blankfootnote{15.23 Understand \textit{māsi} in \textit{pāda} a as \textit{māseṣu}. That it is \textit{mārgaśiras} {\rm (}\textit{mārgaśīrṣa}{\rm )}
  that is regarded as the best month may indicate that the VSS 
  uses a calendar in which the year starts with that month {\rm (}corresponding to 
  November-December{\rm )}. The same seems to be true for a religious observance 
  taught in ŚDhŚ 10.17--34.
 }}

  \maintext{amāvāsyā dinaśreṣṭhā grahaśreṣṭho divākaraḥ |}%

  \maintext{strīṣu lakṣmīr dhṛtiḥ śreṣṭhā vasuśreṣṭho hutāśanaḥ }||\thinspace15:24\thinspace||%
\translation{The best day is the day of the new-moon. The best planet is the Sun. The best among women are Lakṣmī and Dhṛti [two of Dharma's thirteen wives]. The best Vasu is Agni. }

  \maintext{ṛṣiṣu uśaṇā śreṣṭhaḥ kāntiśreṣṭho niśākaraḥ |}%

  \maintext{nakṣatreṣv abhijit śreṣṭhaḥ kālaḥ śreṣṭhaḥ kaleṣu ca  }||\thinspace15:25\thinspace||%
\translation{The best Ṛṣi is Uśanas. The best brightness is the Moon['s]. The best constellation is Abhijit. The best CHECK is time. }

  \maintext{vedeṣu ca varaṃ sāma sthāvareṣu himālayaḥ |}%

  \maintext{aśvattho vaṭa vṛkṣeṣu bhūteṣu vara cetanaḥ }||\thinspace15:26\thinspace||%
\translation{The best Veda is the Sāmaveda. The best mountain is the Himalayas. The [best] among trees are the Aśvattha and Vaṭa. The best beings are the ones with consciousness. }

  \maintext{adhyātma sarvavidyāsu vākya satya vara smṛtaḥ |}%

  \maintext{prahlādo vara daityeṣu yakṣarakṣo dhaneśvaraḥ }||\thinspace15:27\thinspace||%
\translation{The [best] of all knowledge is the spiritual one {\rm (}Sāṃkhya?{\rm )}. The best speech is the truthful one. The best demon is Prahlāda. The guard? of the Yakṣas is Kubera. }

  \maintext{marīcir vara vāteṣu hariḥ śreṣṭho mṛgeṣu ca |}%

  \maintext{sādhya nārāyaṇaḥ śreṣṭhaḥ pitṝṇāṃ ca pitāmahaḥ }||\thinspace15:28\thinspace||%
\translation{The best wind is Marīci?. The best among the deer is the reddish one. The best Sādhya deity is Nārāyaṇa. The best ancestor is Brahmā. }

  \maintext{etat samāsato devi kathito 'si varānane |}%

  \maintext{sarvasāraṃ samuddhṛtya kiṃ bhūyaḥ kathayāmy aham }||\thinspace15:29\thinspace||%
\translation{O goddess, having told you this summary of the essence of everything in an extracted form, O Varānanā, what shall I tell you further? }

\centerline{\maintext{\dbldanda\thinspace iti vṛṣasārasaṃgrahe jīvanirṇayo nāmādhyāyaḥ pañcadaśamaḥ\thinspace\dbldanda}}
\translation{Here ends the fifteenth chapter in the Vṛṣasārasaṃgraha called the Description of the Soul.}

  \chptr{ṣoḍaśamo 'dhyāyaḥ}
\fancyhead[CO]{{\footnotesize\textit{Translation of chapter 16}}}%

  \trchptr{ Chapter Sixteen }%

  \subchptr{yogasadbhāvanirṇayaḥ}%

  \trsubchptr{The exposition of the essence of yoga}%

  \maintext{devy uvāca |}%

  \maintext{adhunā śrotum icchāmi yogasadbhāvanirṇayam |}%

  \maintext{karaṇaṃ ca yathānyāyaṃ kathayasva sureśvara }||\thinspace16:1\thinspace||%
\translation{The goddess spoke: Now I would like to hear the exposition of the essence of yoga. Furthermore teach me about the Karaṇa [exercises, practice?], according to the rules, O Sureśvara. }

  \maintext{īśvara uvāca |}%

  \maintext{śṛṇu devi pravakṣyāmi yogasadbhāvam uttamam |}%

  \maintext{yaṃ viditvā na paśyanti janāḥ saṃsārabandhanam }||\thinspace16:2\thinspace||%
\translation{Īśvara spoke: Listen, o Devī, I shall teach you the supreme essence of yoga, by knowing which people don't have to face the fetters of mundane existence. }

  \maintext{brahmahā gurutalpī vā surāpasteya eva vā |}%

  \maintext{athavā saṃkare jātas tat sarvam apanodati }||\thinspace16:3\thinspace||%
\translation{[One can be] a Brahmin-slayer, a violator of his teacher's bed, a drunkard, a thief or can be born into a mixed caste: it [i.e. yoga] will eliminate all [of his sins]. }

  \maintext{muhūrtārdhe muhūrte vā prāṇāyāmaparāyaṇaḥ |}%

  \maintext{dhyeyaṃ cintayamānasya tatpāpaṃ kṣīyate narāt }||\thinspace16:4\thinspace||%
\translation{He who engages in Prāṇāyāma for [just] half a moment or for a moment, [and] focuses on the object to be visualized {\rm (}dhyeya{\rm )} will have those sins destroyed ... [kṣaṇāt? cf. parallel] }

  \maintext{na yamo nāntakaḥ kruddho na mṛtyur bhīmavigrahaḥ |}%

  \maintext{nāviśanti mahātmāno yogino balavattarāḥ }||\thinspace16:5\thinspace||%
\translation{Mighty [balavat] Yama, the cruel Ender, frightening-looking death will not take possession of the brave yogin. }

  \maintext{yathā vai sarvadhātūnāṃ doṣā dahyanti dhāmyatām |}%

  \maintext{tathā pāpāḥ pradahyante dhruvaṃ prāṇasya nigrahāt }||\thinspace16:6\thinspace||%
\translation{Just as the faults of all metals are burnt out by blowing [the fire that heats] them, in the same way sins are surely burnt away by the control of the breath. }

  \maintext{aśvamedhasahasraṃ ca rājasūyaśataṃ tathā |}%

  \maintext{prāṇāyāmaśataṃ caiva na tattulyaṃ kadācana }||\thinspace16:7\thinspace||%
\translation{There is nothing like a thousand Aśvamedha sacrifices, a hundred Rājasūya rituals or a hundred [rounds of] prāṇāyāma. }

  \maintext{yajñena devān āpnoti rājyaṃ vai tapasaḥ phalam |}%

  \maintext{saṃnyāsād brahmaṇaḥ sthānaṃ vairāgyāt prakṛtālayam }||\thinspace16:8\thinspace||%
\translation{By sacrifice, one can reach the gods [Veda?]. The result of austerities is sovereignty [in yoga?]. By renunciation, one reaches Brahmā's place, and by indifference, Prakṛti's abode. }

  \maintext{jñānāt prāpnoti kaivalyaṃ paraṃ brahma sanātanam |}%

  \maintext{ity etā gatayaḥ pañca vidhivat parikīrtitāḥ }||\thinspace16:9\thinspace||%
\translation{By knowledge, one attains kaivalya and the supreme and eternal Brahman [Sāṃkhya?]. These are taught to be the five paths according to the rules. }

  \maintext{muhūrtārdhaṃ muhūrtaṃ vā yogaṃ yuñjīta yogavit |}%

  \maintext{nistaret sarvapāpāni amṛtatvaṃ ca gacchati }||\thinspace16:10\thinspace||%
\translation{He will get beyond all sins and will attain immortality, if the knower of yoga practises yoga for half a moment or for a moment, }

  \maintext{yuñjāno 'pi prayatnena yāvat tattvaṃ na vindati |}%

  \maintext{brahmaloke dhruvaṃ vāso viṣṇuloke ca sundari }||\thinspace16:11\thinspace||%
\translation{Even if he practises diligently, until he knows the Truth, he will surely abide in Brahmā's and Viṣṇu's homes, O Sundarī, }

  \maintext{bhuktvā karmasahasrāṇi sarvakāmasamanvitaḥ |}%

  \maintext{kṣīṇapuṇyas tato martye jāyate vipule kule }||\thinspace16:12\thinspace||%
\translation{and when his merits are exhausted, he will be born in the world of mortals, in a noble family. He will experience thousands of karmas, while he has all possible desires. }

  \maintext{yogam evābhiseveta pūrvajātismaro naraḥ |}%

  \maintext{saṃsārārṇavam uttīrya sa śivatvam avāpnuyāt }||\thinspace16:13\thinspace||%
\translation{He should practise only yoga, and he will be a man who remembers his own previous births. He should practise only yoga, and he will be a man who remembers his own previous births. Crossing the ocean of mundane existence, he will obtain Śivaness. }

  \subchptr{yogavidhiḥ}%

  \trsubchptr{The technique of yoga}%

  \maintext{devy uvāca |}%

  \maintext{yogasya vidhim icchāmi śrotuṃ me puruṣottama |}%

  \maintext{dhyānadhāraṇasiddhīnāṃ kathayasva sureśvara }||\thinspace16:14\thinspace||%
\translation{The goddess spoke: I wish to hear about the method of yoga. Teach me, O Puruṣottama, O Sureśvara, about meditation, concentration and the Powers. }

  \maintext{maheśvara uvāca |}%

  \maintext{śṛṇu yogavidhiṃ vakṣye bhavapāśanikṛntanam |}%

  \maintext{śucir ekāgracittas tu janaśabdavivarjite |}%

  \maintext{tatrāsīnāsane yogī paramātmāna cintayet }||\thinspace16:15\thinspace||%
\translation{Maheśvara spoke: Listen, I shall teach you the method of yoga, the destroyer of the noose of existence. [With his body] purified and his mind concentrated, the yogin should sit down assuming a sitting posture {\rm (}āsana{\rm )} in a place which is devoid of humans and noise, and he should think of the Supreme Soul. }

  \maintext{padmakaṃ svastikaṃ caiva niṣkalam añjalis tathā |}%

  \maintext{ardhacandraṃ ca daṇḍaṃ ca paryaṅkaṃ bhadram eva ca }||\thinspace16:16\thinspace||%
\translation{[The āsanas are:] padmaka, svastika, niṣkala, añjali, ardhacandra, daṇḍa, paryaṅka, and bhadra. }

  \maintext{etadāsanabandhena baddhvā yogaṃ samabhyaset |}%

  \maintext{samaṃ kāyaśirogrīvaṃ dhārayann acalasthitaḥ }||\thinspace16:17\thinspace||%
\translation{He should practise yoga by assuming [any one of] these āsanas, holding his trunk, head and neck level, staying [in the position] without any movement. }

  \maintext{pratyāhāras tathā dhyānaṃ prāṇāyāmaś ca dhāraṇā |}%

  \maintext{tarkaś caiva samādhiś ca ṣaḍaṅgo yoga ucyate }||\thinspace16:18\thinspace||%
\translation{Withdrawal of the senses {\rm (}pratyāhāra{\rm )}, meditation {\rm (}dhyāna{\rm )}, breat-controll {\rm (}prāṇayāma{\rm )}, concentration {\rm (}dhāraṇā{\rm )}, reflection {\rm (}\textit{tarka}{\rm )}, and samādhi: these are called the six-limbed yoga/ yoga with six ancillaries. }

  \maintext{viṣayāsaktacittānām indriyāṇāṃ prati prati |}%

  \maintext{manasākarṣayed yas tu pratyāhāraḥ sa ucyate }||\thinspace16:19\thinspace||%
\translation{That [method] which draws in the senses that are clinging on to the objects again and again [see DhP] with the help of the mind is called withdrawal of the senses. }

  \maintext{śabdādiviṣayān devi vartulīkṛtya dhārayet |}%

  \maintext{vītarāgaḥ samādhistho dhyeye vastuni yojayet }||\thinspace16:20\thinspace||%
\translation{O Devī, [the yogin] should concentrate on the [five] sense-objects beginning with sound after he has made them into a ball. His passions gone, dwelling in samādhi, he should join the object of meditation with the object[?]. }

  \maintext{ātmā dhyātā mano dhyānaṃ dhyeyaḥ śuddhaḥ paraḥ śivaḥ |}%

  \maintext{yat paraṃ paramaiśvaryam ekaṃ tatra prayojanam }||\thinspace16:21\thinspace||%
\translation{The Self is the meditator {\rm (}dhyātṛ{\rm )}, the mind is meditation {\rm (}dhyāna{\rm )}, the object of meditation {\rm (}dhyeya{\rm )} is Pure Supreme Śiva As regards supreme sovereignty, [that] is the only aim in it [i.e. in dhyāna]. }

  \maintext{pūrakaḥ kumbhakaś caiva recakas tadanantaram |}%

  \maintext{praśāntaś ceti vikhyātaḥ prāṇāyāmaś caturvidhaḥ }||\thinspace16:22\thinspace||%
\translation{Inhalation, breath retention, then exhalation, and the tranquillized one: prāṇāyāma is fourfold. }

  \maintext{pūrake sthāpayed vahniṃ pādāṅguṣṭhena buddhimān |}%

  \maintext{kumbhakena virudhyeta dahyamānaṃ vicintayet }||\thinspace16:23\thinspace||%
\translation{During inhalation, the wise one should establish the fire through his great-toe. By breath retention he should stop it [i.e. the fire] and visualize it [i.e. himself] as being burnt. }

  \maintext{bhasmībhūtaṃ tathātmānaṃ recakena vicintayet |}%

  \maintext{śuddhadehas tataś cātmā śuddhasphaṭikanirmalaḥ }||\thinspace16:24\thinspace||%
\translation{Then, while exhaling, he should imagine himself as reduced to ashes. Now his Self has a purified body, one which is as spotless as a clear crystal. }

  \maintext{tālaśabdas tu nirvāṇaṃ daśa dve ca prakīrtitaḥ |}%

  \maintext{prāṇāyāmān na saṃdeho dviguṇā dhāraṇā smṛtā }||\thinspace16:25\thinspace||%
\translation{[When this is maintained for] twelve measures of time, that is called nirvāṇa/exhalation? Concentration {\rm (}dhāraṇā{\rm )} is twice as long as breath-control {\rm (}prāṇāyāma{\rm )}, there is no doubt about it. }

  \maintext{yoge tu triguṇā proktā saṃkrame ca caturguṇā |}%

  \maintext{tathotkrāntau pañcaguṇā yogasiddhis tu ṣaḍguṇā }||\thinspace16:26\thinspace||%
\translation{As regards yoga, it {\rm (}dhāraṇā?{\rm )} is said to be three times as long, in saṃkrama it is four times longer. In case of ritual suicide {\rm (}utkrānti{\rm )} is concerned, it is five times longer. [To reach] yogic Powers {\rm (}yogasiddhi{\rm )} [it takes] six times longer. }

  \maintext{ṣaḍaṅgena samāyukto yogayuktas tu nityaśaḥ |}%

  \maintext{mānaso yaugapadyaś ca dvirūpo yoga ucyate }||\thinspace16:27\thinspace||%
\translation{[The yogin should] always be practising yoga with the six ancillaries. Yoga is taught as having two forms: mental {\rm (}mānasa{\rm )} and simultaneous?? {\rm (}yaugapadya{\rm )}. }

  \maintext{akṛtvā prāṇasaṃrodhaṃ manasaikena kevalam |}%

  \maintext{dhyāyeta paramaṃ sūkṣmaṃ sa yogo mānasaḥ smṛtaḥ }||\thinspace16:28\thinspace||%
\translation{[The yogin] can meditate on the supreme subtle one only mentally, without performing breath-control: that type of yoga is called mental [yoga] {\rm (}mānasa{\rm )}. }

  \maintext{saṃyamya manasā prāṇaṃ prāṇāyāmān manas tathā |}%

  \maintext{evaṃ dhyāyet paraṃ sūkṣmaṃ yaugapadyaḥ sa ucyate }||\thinspace16:29\thinspace||%
\translation{[If the yogin] controls his breath with his mind, and his mind with breath-control, and thus meditates on the supreme subtle one, that is called simultaneous [yoga] {\rm (}yaugapadya{\rm )}. }

  \subchptr{siddhilakṣaṇam}%

  \maintext{siddhilakṣaṇa yogasya śṛṇu vakṣyāmi sundari |}%

  \maintext{śaṅkhabherīmṛdaṅgaṃ ca veṇudundubhim eva ca |}%

  \maintext{tāḍitaṃ na ca vindeta yadā tanmayatāṃ gataḥ }||\thinspace16:30\thinspace||%
\translation{I shall teach you the signs of success in yoga, listen, O Sundarī. When a conch-shell, kettle-drum, mṛdaṅga-drum, flute or dundubhi-drum is beaten, he will not perceive [the sound] when he has reached such-ness [i.e.\ Śivaness]. }

  \maintext{śītoṣṇaṃ sukhaduḥkhaṃ ca tṛṣṇābhukṣaṃ tathaiva ca |}%

  \maintext{vedanāṃ naiva jānāti yogasiddhas tu sundari }||\thinspace16:31\thinspace||%
\translation{Similarly, he will not be able to tell cold from heat, joy from sadness, he will not experience thirst or hunger or pain, when he attains success in yoga, O Sundarī. }

  \maintext{eṣa yogavidhir devi tava pṛṣṭena sundari |}%

  \maintext{kathito 'smi samāsena kim anyat kathayāmy aham }||\thinspace16:32\thinspace||%
\translation{This is how I taught the technique of yoga in a nutshell, O Devī, as a reply to your question, O Sundarī. What else shall I teach you? \blankfootnote{16.32 Note 'smi.
 }}

  \maintext{devy uvāca |}%

  \maintext{vinā yogena deveśa saṃsāratāraṇaṃ mama |}%

  \maintext{kathayasva mahādeva nirvikalpakaraṃ manaḥ }||\thinspace16:33\thinspace||%
\translation{The goddess spoke: Tell me about the liberation from mundane existence without yoga, O Deveśa! O Mahādeva, [that could] free [one's] mind of doubts/hesitation. \blankfootnote{16.33 Understand \textit{nirvikalpakaraṃ manaḥ} as \textit{manonirvikalpakaraṃ}
 }}

  \maintext{maheśvara uvāca |}%

  \maintext{sadāśivas tu niśvāsa ūrdhvaśvāsaḥ paraḥ śivaḥ |}%

  \maintext{tayor madhye tu vijñeyaḥ paramātmā śivo 'vyayaḥ }||\thinspace16:34\thinspace||%
\translation{Maheśvara spoke: Sighing is Sadāśiva, a deep breath is supreme Śiva. In between the two, there is Śiva the supreme and imperishable Self. }

  \maintext{dhyānayogaṃ na tasyāsti karaṇaṃ ca na vidyate |}%

  \maintext{jñātamātreṇa mucyante kim anyat paripṛcchasi }||\thinspace16:35\thinspace||%
\translation{For one [who knows this], there is neither yoga meditation and nor karaṇa. He is liberated by merely knowing [this]. What else would you like to ask? }

  \subchptr{pañca śāstrāṇi}%

  \trsubchptr{The five Śāstras}%

  \maintext{jñānam anyat pravakṣyāmi śṛṇu devi nibodha me |}%

  \maintext{śāstrapañcasu yat proktaṃ śṛṇu saṃkṣepa nirṇayam |}%

  \maintext{sāṃkhye yoge pañcarātre śaive vede ca nirmitam }||\thinspace16:36\thinspace||%
\translation{I shall teach you another kind of knowledge. Listen, O Devī, listen to me. Listen in short to the [its] exposition as constructed in the five śāstras, in Sāṃkhya, in yoga, in the Pañcarātra, in Śaivism and in the Vedas. \blankfootnote{16.36 Note how there is no question from Devī after \textit{kim anyat paripṛcchasi} and how we begin a new topic instead.
  Note also how \textit{saṃkṣepa} stands either for \textit{saṃkṣipta}° or \textit{saṃkṣepena/saṃkṣepataḥ}.
 }}

  \maintext{yat sāṃkhyasiddhaṃ kathayāmy ahaṃ te}%

 \nonanustubhindent \maintext{saṃsāraghorārṇavayogasāram |}%

  \maintext{yogeṣu sāreṣv atha pañcarātre}%

 \nonanustubhindent \maintext{vedeṣu śaiveṣu ca niścayas te }||\thinspace16:37\thinspace||%
\translation{The quintessential yoga which is established in Sāṃkhya, and which is for [liberation from] the terrible ocean of mundane existence, and which I am teaching you now, is there for you as a certainty in essential yoga [teachings], and in the Pañcarātra, in the Vedas, and in Śaivism. }

  \maintext{ghrāṇendriyādyeṣu ca yat samastam}%

 \nonanustubhindent \maintext{manaś ca līnaṃ bhavatīva yasya |}%

  \maintext{buddhyā niyamya sakalān hi bhāvān}%

 \nonanustubhindent \maintext{sa labdhalakṣyaḥ śivam abhyupaiti }||\thinspace16:38\thinspace||%
\translation{If all of [his] senses beginning with smelling, and also his mind, are dissolved, so to say, and if he suppresses all sensations {\rm (}bhāva{\rm )} with his mind, he will attain his aim and will find refuge in Śiva. }

  \maintext{śrotrādisarvendriyaniścalatve}%

 \nonanustubhindent \maintext{ekāgracittaṃ manasā niyamya |}%

  \maintext{svadehaśūnyaḥ sa bhavec cireṇa}%

 \nonanustubhindent \maintext{saṃyogasiddhiṃ pravadanti tajjñāḥ }||\thinspace16:39\thinspace||%
\translation{[When there is] motionlessness of all senses beginning with hearing, and his attention {\rm (}cittaṃ{\rm )}, controlled by his mind, becomes focused {\rm (}ekāgra{\rm )}, his body will slowly disappear. This is called `success in union' by the experts. }

  \maintext{ādāv eva manaḥ śanair uparamet kṛtvā ca vaśyendriyaṃ}%

 \nonanustubhindent \maintext{yāvat tal layatāṃ vrajeta manasā niḥsaṃjñadehas tathā |}%

  \maintext{etad dhyānasamādhiyogasakalaṃ prāpnoti niḥsaṃśayaṃ}%

 \nonanustubhindent \maintext{kiṃ tac chāstrasahasrakoṭipaṭhitaṃ sāraṃ na yo 'nviṣyati }||\thinspace16:40\thinspace||%
\translation{First, he should slowly stop his mind [or subj. = manas?], subduing the sense[s] until it [the senses] dissolve[s] together with the mind [see above]. Thus [the yogin's] body is rendered unconscious/senseless. [The yogin] certainly attains this yoga in its entirety, namely meditation and samādhi. Why is it [if] somebody does not seek [this] essence extracted[? mathitam might be better] from ten thousand million books? }

  \maintext{ātmārāmajitaḥ samādhinirato vairāgyam apy āśritaḥ}%

 \nonanustubhindent \maintext{cittaṃ yasya parikṣayo yadi bhavet tiṣṭhet tanutvaṃ yathā |}%

  \maintext{taj jñeyaṃ gatim uttamaṃ śivapadaṃ saṃsāraduḥkhacchidaṃ}%

 \nonanustubhindent \maintext{vedānteṣu ca niṣṭha eṣa kathitaḥ kiṃ śāstram anyad viśet }||\thinspace16:41\thinspace||%
\translation{He has conquered his joy in his Self, [instead] he rejoices in samādhi and he has also taken refuge in indifference {\rm (}vairāgya{\rm )}. When the end comes, his mind will remain [in] corporal form??? That is to be known as the highest path, Śiva's abode, which puts an end to mundane suffering. And this is taught as `completion' [niṣṭhā!] in the Vedānta {\rm (}in the Upaniṣads?{\rm )}. Why should anyone resort to any other teaching? }

  \maintext{hṛtpadme karṇikāyām upari ravir avadyotayanto 'ntarālam}%

 \nonanustubhindent \maintext{yattejastejamārgair bahalatamaghanair dyotanād dīptadīpam |}%

  \maintext{bhittvā yat tāludeśe mukham uparigataṃ tāludeśena mūrdhni}%

 \nonanustubhindent \maintext{! mūrdhni dvārāntareṇa śivaparamapadaṃ yānti yogena yuktāḥ }||\thinspace16:42\thinspace||%
\translation{On [upari, here with loc.] the pericarp of the heart-lotus, there is a sun, illuminating the intermediate space. There is a lamp lit by the shining of the most dense mass of rays of its light, which having pierced the mouth at the soft palate, goes upwards through the soft palate towards the top of the head. Those practising yoga leave for Śiva's supreme abode through the door on the top of their heads. }

  \maintext{kṛṣṇaḥ kṛṣṇatamottamo 'timahato yas tejatejātmakaḥ}%

 \nonanustubhindent \maintext{lokālokadharādharaḥ śriyapatiḥ prāṇapraviṣṭālayaḥ |}%

  \maintext{kartā kāraṇam avyayo 'vyayam asau vyāpī vibhaktāvidam}%

 \nonanustubhindent \maintext{viṣṇur bhāvamayo vibhaktaviṣayair viśveśvaro viśvavit }||\thinspace16:43\thinspace||%
\translation{[Tentative:] Kṛṣṇa, the highest of the darkest ones, the extremely great one, who is essentially the splendour of light/who shines/is sharp, the one who has never been born, the supporter of the world and the non-world and of the earth, husband to Śrī, abiding in the breath, the imperishable creator, the imperishable cause,[?] he the [all-]pervading, the arranger/distributor?, ...? Viṣṇu, ..., the lord of the universe, the omniscient one. }

  \maintext{! eṣa tattvavaraḥ parāparamayas tejaḥ parasthānadaḥ}%

 \nonanustubhindent \maintext{buddhyā bhāvanabhāvayendriyamano dehāntar ālokayan |}%

  \maintext{hṛtpadmāyatanasthitaḥ sa puruṣo niśvāsam ucchvāsadaḥ}%

 \nonanustubhindent \maintext{nādas tasya sadā sadā nadati taṃ nādopariṣṭhā haraḥ }||\thinspace16:44\thinspace||%
\translation{is viewing the senses and the mind inside the body through the Buddhi which is transformed by meditation. That Puruṣa is located in the abode in the heart-lotus, he who gives us exhalation and inhalation. }

  \maintext{yas tejas tejate 'jo bahuniviḍaghano granthimālopagūḍhaḥ}%

 \nonanustubhindent \maintext{mūrtir mūrtānusārī bahukaraṇabhṛtaṃ kāraṇād dehabandhaḥ |}%

  \maintext{bhittvā granthiṃ sapāśaṃ viṣam iva viṣayaṃ tyaktasaṅgaikabhāvāḥ}%

 \nonanustubhindent \maintext{paśyanty ete tam īśaṃ guṇakalarahitaṃ nirvikāraṃ prakāśam }||\thinspace16:45\thinspace||%
\translation{He who is intensifying energy, the unborn one, who is a very dense mass, who is hidden in the garland of knots, the embodiment, who follows the embodied form, ... piercing the knot together with the bond, abandoning the objects of the senses and attachment like poison, focusing their states of mind, they can see him, the God, who is devoid of [even] a small portion of the Guṇas, and who is formless light. \blankfootnote{16.45 The correction from °\textit{kara}° to °karaṇa° in \msNc\ {\rm (}in a second hand{\rm )} is indicated,
  rather unusually, within the next line, just below the \textit{kākapada} signalling the omission.
 }}

  \maintext{yo 'sau tejāntarātmā kamalapuṭakuṭīsaṃkaṭasthānalīnaḥ}%

 \nonanustubhindent \maintext{indor bhāsānurūpī vimaladalasadācchāditaḥ karṇikāyām |}%

  \maintext{tatra sthāne sthito 'sau tribhuvananilayaḥ sarvabhūtādhivāsaḥ}%

 \nonanustubhindent \maintext{ākāśād ūrdhvatattvasthitavikasakalāsaṃhato muktabandhaḥ }||\thinspace16:46\thinspace||%
\translation{He whose inner self is energy, and who is hiding in the contracted place in the abode which is the hollow of the lotus [in the heart], who resembles the Moon's light, who is always hidden in the pericarp among the spotless petals, is, while remaining in that place, the abode of the three worlds and the home of all beings, free of bondage, the one with the crescent moon, being at the Tattva above space. }

  \maintext{etāni tattvāny akhilāni devi}%

 \nonanustubhindent \maintext{saṃkṣepataḥ kīrtitaḥ pañcabhedaḥ |}%

  \maintext{śrotuṃ kim anyad vijigīṣitārtham}%

 \nonanustubhindent \maintext{saṃsāramokṣeṇa ca tatparo 'sti }||\thinspace16:47\thinspace||%
\translation{These are all the Tattvas, O Devī. The five-fold classification has been taught in short. What other topic do you wish to hear, [something] related to liberation from saṃsāra? }

  \maintext{devy uvāca |}%

  \maintext{tuṣṭāsmi deva mama saṃśayam adya naṣṭam}%

 \nonanustubhindent \maintext{adya prasannaparameśvara īśvara tvam |}%

  \maintext{adya śrutaṃ tvayi ca puṇyaphalaprabhāvam}%

 \nonanustubhindent \maintext{pūrṇāni cādya mama iṣṭamanorathāni }||\thinspace16:48\thinspace||%
\translation{O God, I am satisfied. Now my doubts have been removed. Now you are a gracious supreme Lord, O Īśvara! Now I have heard FROM you the power of the fruits of merit. }

  \maintext{ajñānapaṅkaghanamadhyanilīyamānām}%

 \nonanustubhindent \maintext{uttārayeśa sakalārtivināśanāya |}%

  \maintext{sarveśa tattvaparamārtha namo namas te}%

 \nonanustubhindent \maintext{adyāpi tṛptir iha nāsti mamāpi śambho }||\thinspace16:49\thinspace||%


  \maintext{pītvāmṛtaṃ cottamavaktrajātam}%

 \nonanustubhindent \maintext{ākhyāhi dānaṃ phaladharmasāram |}%

  \maintext{saṃsārapāraṃ paramaṃ nayasva}%

 \nonanustubhindent \maintext{kṛpāṃ mayīśāna kuru prasīda }||\thinspace16:50\thinspace||%
{\blankfootnote{16.50 Pāda c and pāda d are in the reverse order in \Ed and pāda d {\rm (}\textit{kṛpāṃ...}{\rm )}, and while it is
  missing from all the palm-leaf manuscripts, it appears as the last line of this stanza in 
  two paper manuscripts: NGMPP A 1341-6 {\rm (}NAK 4/93{\rm )} and NGMPP C 107-7 {\rm (}Kesar 537{\rm )}. One of these {\rm (}or both{\rm )} 
  may have been the main source for Naraharinath's edition. Note also that these
  two paper manuscripts seem to have been written by the same hand.
 }}

\centerline{\maintext{\dbldanda\thinspace iti vṛṣasārasaṃgrahe 'dhyātmanirṇayo nāmādhyāyaḥ ṣoḍaśamaḥ\thinspace\dbldanda}}
\translation{Here ends the sixteenth chapter in the Vṛṣasārasaṃgraha called the Description of Spirituality.}

  \chptr{saptadaśamo 'dhyāyaḥ}
\fancyhead[CO]{{\footnotesize\textit{Translation of chapter 17}}}%

  \trchptr{Chapter Seventeen}%

  \subchptr{dānadharmaviśeṣaḥ}%

  \trsubchptr{The particulars of the Dharma of donation}%

  \maintext{devy uvāca |}%

  \maintext{pṛthag dānasya icchāmi śrotuṃ māṃ dātum arhasi |}%

  \maintext{annavastrahiraṇyānāṃ gobhūmikanakasya ca }||\thinspace17:1\thinspace||%
\translation{I wish to hear about [the types of] donation one by one. Please let me [hear about donating] food, clothes, gold, cows, land, and gold[?!]. \blankfootnote{17.1 MMW: `dā: to permit, allow {\rm (}with inf.{\rm )} [MBh.] i Śak. vi, 22;'
  Otter par. 145: `biye {\rm (}bil-{\rm )} ``to give'': Alongside its lexical meaning, this verb can also
  mean ``to allow'' if used with a preceding verb noun. With the converb
  in {-āwo}, it can be used to indicate that the action is performed for the
  benefit of someone else.'
 }}

  \subsubchptr{annapradānam}%

  \trsubsubchptr{Donation of food}%

  \maintext{bhagavān uvāca |}%

  \maintext{susaṃskṛtam annam atipradadyād}%

 \nonanustubhindent \maintext{ghṛtaprabhūtam avadaṃśayuktam |}%

  \maintext{ghṛtaprapakvaṃ sukṛtaṃ ca pūpaṃ}%

 \nonanustubhindent \maintext{sitena khaṇḍena guḍena yuktam }||\thinspace17:2\thinspace||%
\translation{The Lord spoke: One should excel in donating food that is well-cooked, rich in ghee and contains pungent ingredients, well-prepared bread baked? with ghee, white sugar and molasses. }

  \maintext{mārgaṃ khagaṃ codakajaṅgalaṃ ca}%

 \nonanustubhindent \maintext{dadyād vaṭaṃ nāgaravaṃśamūlam |}%

  \maintext{śākaṃ phalaṃ cāmlamadhūratiktaṃ}%

 \nonanustubhindent \maintext{pānaṃ payaḥ śītasugandhatoyam }||\thinspace17:3\thinspace||%
\translation{One should give meat coming from deer, birds, and water[-animals], and [the fruits of the] Banyan-tree, dried ginger {\rm (}\textit{nāgara}{\rm )}, sugarcane, and roots, vegetables, sour, sweet and pungent fruits, and for drinks, milk, and cold and fragrant water. \blankfootnote{17.3 For \textit{nāgara} as `dried ginger' {\rm (}in \textit{pāda} b{\rm )}, see \mycitep{Meulenbeld1974}{567}.
 Note °\textit{madhūra}° for °\textit{madhura}° in \textit{pāda} c metri causa;
  or read °\textit{madhūka}° {\rm (}Madhuca latifolia{\rm )}.
 }}

  \maintext{dadhi pradadyād guḍamiśritaṃ ca}%

 \nonanustubhindent \maintext{mṛṇāla śālūka ca nālakā ca |}%

  \maintext{sadakṣiṇālepapavitrapuṣpaṃ}%

 \nonanustubhindent \maintext{śraddhānvitaḥ satkṛtayā praṇamya }||\thinspace17:4\thinspace||%
\translation{One should give coagulated milk mixed with molasses, lotus-fibre [root?], lotus-roots, lotus-stalks, ointments accompanied by gifts, Kuśa grass {\rm (}\textit{pavitra}{\rm )}, and flowers, with faith and respect, bowing down. }

  \maintext{prayāti lokaṃ jagadīśvarasya}%

 \nonanustubhindent \maintext{vimānayānaiḥ sahito 'psarobhiḥ |}%

  \maintext{ekaikasikthasya sahasravarṣam}%

 \nonanustubhindent \maintext{annaprado modati devaloke }||\thinspace17:5\thinspace||%
\translation{He goes to the world of Jagadīśvara on \ae rial vehicles, together with Apsarases. He who donates food will have fun in the world of gods for a thousand years for each lump of boiled rice [that he gave]. \blankfootnote{17.5 Most MSS read \textit{prayānti} in \textit{pāda} a.
 }}

  \maintext{cyutaś ca martye sa bhaved dhanāḍhyaḥ}%

 \nonanustubhindent \maintext{kulodgataḥ sarvaguṇopapannaḥ |}%

  \maintext{yaśaḥ śriyaṃ sarvakalājñatā ca}%

 \nonanustubhindent \maintext{bhavet sa bhogī sakalatraputraḥ }||\thinspace17:6\thinspace||%
\translation{Descending to the human world, he will become a rich man. He will be born in a noble family and will possess all possible virtues, fame, beauty, and knowledge of all the arts. He will be rich together with his wife and sons. }

  \maintext{dadyād daridrakṛpaṇārtadīnāṃ}%

 \nonanustubhindent \maintext{kālāgatatvāturam āgatānām |}%

  \maintext{tṛṣṇābubhukṣāgatikāgatānāṃ}%

 \nonanustubhindent \maintext{dattvā sa dharmaphalam āśrayeta }||\thinspace17:7\thinspace||%
\translation{One should donate to the poor, the miserable, the oppressed, the wretched, to those suffering of old age, to those whose share is thirst, hunger, who are without resources. By donating, one will be connected to the fruits of Dharma. \blankfootnote{17.7 My emendation in \textit{pāda} a presupposes either that °\textit{dīnāṃ}
  is the result of an original confusion, and what was meant was
  °\textit{dīnānāṃ}, a plural genitive, or that this form was considered acceptable
  for a plural genitive. The latter is what we see in Aiśa Sanskrit EXAMPLES.
  Another possibility could be to read °\textit{dīna}- as part of a longer 
  compound.
  Read \textit{kṛpaṇā}° as \textit{kripaṇā}° in \textit{pāda} a to restore the metre. 
  See a similar case in 17.57b.
 Note the variant \textit{kālāgadatvā}°... 
 }}

  \maintext{deśe ca kāle ca tathā ca pātre}%

 \nonanustubhindent \maintext{dānādidharmasya phalaṃ kaniṣṭham |}%

  \maintext{vāṇijyadharmā hi phalāśritānāṃ}%

 \nonanustubhindent \maintext{dharmo hi tasya na ca nirmalo 'sti }||\thinspace17:8\thinspace||%
\translation{[From among aspects such as] the place, the time and the recipient of the Dharma of donation etc., the fruits are the least significant. For surely those who count on the fruits [of their actions] practise the Dharmas of trade. The Dharma of such a person will not be spotless. }

  \maintext{toyaṃ ca dadyāl laghupūrṇakumbhaṃ}%

 \nonanustubhindent \maintext{śītaṃ sugandhaṃ parivāsitaṃ ca |}%

  \maintext{sa yāti lokaṃ salileśvarasya}%

 \nonanustubhindent \maintext{na saptajanmāni tṛṣābhibhūtaḥ }||\thinspace17:9\thinspace||%
\translation{He should give cool, nice-smelling and scented water [in] a light waterpot[?] filled up to the brim. He will go to the world of the Lord of Waters {\rm (}\textit{salileśvara}{\rm )} [i.e.\ Varuṇa] and will not be overcome by thirst thoroughout seven births. }

  \subchptr{vastrādipradānam}%

  \trsubchptr{Donation of clothes etc.}%

  \maintext{upānahaṃ yo dadati dvijāya}%

 \nonanustubhindent \maintext{suśobhanaṃ tailasudīpitaṃ ca |}%

  \maintext{te yānti lokam amarādhipasya}%

 \nonanustubhindent \maintext{yamālayaṃ kaṣṭapathā na yānti }||\thinspace17:10\thinspace||%
\translation{He who donates a beautiful pair of sandals, polished with oil, to a Brahmin will go to the world of the king of the immortal ones [i.e.\ Indra], and will not approach Yama's abode through a difficult path. }

  \maintext{prakṣīṇapuṇyaḥ punar atra loke}%

 \nonanustubhindent \maintext{jāto bhaved divyakulopapannaḥ |}%

  \maintext{dhanaiḥ samṛddho 'dhipatitvatāṃ ca}%

 \nonanustubhindent \maintext{rathāśvanāgāsanagā bhavanti }||\thinspace17:11\thinspace||%
\translation{When his merits fade away, he will be born again in this world into a noble family. He will be abounding with wealth, will be a king, seated on a throne on a chariot, a horse, or an elephant. }

  \maintext{vastrapradānena bhavanti devi}%

 \nonanustubhindent \maintext{rūpottamāḥ sarvakalājñatā ca |}%

  \maintext{samṛddhisaubhāgyaguṇānvitāś ca}%

 \nonanustubhindent \maintext{svargacyutās te puruṣā bhavanti }||\thinspace17:12\thinspace||%
\translation{By donating clothes, O Devī, they will become most beautiful people, with knowledge of all the arts, endowed with riches, beauty, and virtues, after they have descended from heaven. }

  \maintext{vastrapradānābhiratasya puṃsaḥ}%

 \nonanustubhindent \maintext{anyāṃ pravakṣyāmi tataḥ praśaṃsām |}%

  \maintext{vastraṃ tu lokeṣv abhipūjanīyaṃ}%

 \nonanustubhindent \maintext{vastraṃ narāṇāṃ tv atimānanīyam }||\thinspace17:13\thinspace||%
\translation{I shall then praise further the man who engages in the donation of clothes. Clothes are to be honoured in the worlds, clothes are to be held in extremely high esteem by people. }

  \maintext{vastraṃ tu bhūyo na ca mānalābhaḥ}%

 \nonanustubhindent \maintext{parābhavaś cātijugupsanaṃ ca |}%

  \maintext{tasmād dhi vastraṃ satataṃ pradeyaṃ}%

 \nonanustubhindent \maintext{yaśaḥ śriyaḥ svargam anantalābham }||\thinspace17:14\thinspace||%
\translation{[If] clothes are in abundance, there is no respect,[?] [only] contempt and extreme disgust. Therefore clothes should always be donated, [and by this come] fame, fortune, heaven, and endless profit. }

  \maintext{yāvanti sūtrāṇi bhavanti vastre}%

 \nonanustubhindent \maintext{tāvadyugaṃ gacchati somalokam |}%

  \maintext{puṇyakṣayāj jāyati martyaloke}%

 \nonanustubhindent \maintext{vastraprabhūte dhanadhānyakīrṇe |}%

  \maintext{surūpasaubhāgyayaśasvinaś ca}%

 \nonanustubhindent \maintext{vidyādharā lokaprabhutvatāṃ ca }||\thinspace17:15\thinspace||%
\translation{He will stay in Somaloka for as many \ae ons as there are threads in the clothes [donated]. Because his merits fade away, he is reborn in the human world, with an abundance of clothes and having a lot of riches and corn. They will be beautiful, attractive, and glorious demigods {\rm (}\textit{vidyādhara}{\rm )}, and [they will obtain] supremacy over the world. \blankfootnote{17.15 Note the \mutacumliquida\ in operation in \textit{pāda} f {\rm (}\textit{pra}{\rm )}. 
  \textit{vidyādhara}: see Hidas 2019, 24--25.
 }}

  \maintext{dvijebhyaś chatraṃ sukṛtaṃ pradadyād}%

 \nonanustubhindent \maintext{varṣātapatraṃ dṛḍhaśobhanaṃ ca |}%

  \maintext{aṅgāravarṣaṃ trapukhaḍgam ādyam}%

 \nonanustubhindent \maintext{asaṃśayaṃ trāyati yāmyamārge }||\thinspace17:16\thinspace||%
\translation{One should donate well-made parasols to Brahmins which protect them from rain and sunlight, and are firm and nice. It will no doubt protect [them] from charcoal, [molten] tin, knives, etc.\ rain[ing down on them] on their way to Yama['s abode in hell]. \blankfootnote{17.16 To make sense of \textit{pāda} c, especially \textit{trapukhaḍga} {\rm (}`tin knife'?{\rm )},
  I interpret \textit{aṅgāra}, \textit{trapu}, and \textit{khaḍga} as three types
  of rain, so to say, that awaits people on their way to hell.
  Understand \textit{aṅgāra-trapu-khaḍgādya-varṣāt} [\textit{trāyati}]. See
  molten tin filling up people's bodies in hell in \SDhU\ 7.151, 183, 197, 205, etc.
 }}

  \maintext{svargaṃ ca yāti grahanāyakasya}%

 \nonanustubhindent \maintext{sa varṣakoṭyāyutam antakāle |}%

  \maintext{jāyanti te mānuṣa martyaloke}%

 \nonanustubhindent \maintext{gṛhottame bhogapatir bhavanti }||\thinspace17:17\thinspace||%
\translation{He will go to the heaven of the chief of the planets [i.e.\ the Sun] at the time of his death to stay for millions of years. They will be born as humans in the human world in a superb house, and they will be governors. }

  \maintext{kṛtvā maṭhaṃ śobhana vipradātā}%

 \nonanustubhindent \maintext{dravyeṇa śuddhena tu pūrayitvā |}%

  \maintext{sa yāti devendrasado yatheṣṭaṃ}%

 \nonanustubhindent \maintext{sa varṣakoṭīśata divyasaṃkhye }||\thinspace17:18\thinspace||%
\translation{He who builds and donates a hut to a Brahmin, filling it with pure goods, will go to the abode of the king of the gods [i.e.\ Indra] at pleasure [to stay] for millions of years, in divine calculation [i.e.\ counted in divine years]. }

  \maintext{tadantakāle yadi mānuṣatvaṃ}%

 \nonanustubhindent \maintext{jāyanti te saptamahīprabhoktā |}%

  \maintext{sa saptaratnatrayasamprayukto}%

 \nonanustubhindent \maintext{balādhiko yajñasahasrakartā }||\thinspace17:19\thinspace||%
\translation{After that, when they are born into a human existence, they become the kings of the seven worlds. They will be endowed with seven triads of gems[?], having excessive strength, performing thousands of sacrifices. }

  \subchptr{bhūmipradānam}%

  \trsubchptr{Donation of land}%

  \maintext{bhūmipradātā dvija hīnadīnaḥ}%

 \nonanustubhindent \maintext{samṛddhasasyo jalasaṃnikṛṣṭaḥ |}%

  \maintext{sa yāti lokam amarādhipasya}%

 \nonanustubhindent \maintext{vimānayānena manohareṇa }||\thinspace17:20\thinspace||%
\translation{He who donates to a poor and distressed Brahmin land that yields plenty of corn and is in the vicinity of water will go to the world of the king of the immortal ones [i.e.\ of Indra] on a fascinating \ae rial vehicle. \blankfootnote{17.20 Understand the Sanskrit of \textit{pāda}s ab as follows:
  \textit{dvijāya hīnadīnāya sasyasamṛddha-jalasaṃnikṛṣṭa-bhūmi-pradātā}.
 }}

  \maintext{manvantaraṃ yāvad abhuktabhogān}%

 \nonanustubhindent \maintext{tadantakāle cyuta martyaloke |}%

  \maintext{sa jambukhaṇḍādhipatir bhaveta}%

 \nonanustubhindent \maintext{vīryānvito rājasahasranāthaḥ }||\thinspace17:21\thinspace||%
\translation{[He will experience] never-experienced enjoyments for the period of a Manu era. After that he descends to the human world. He will become the king of the Jambu continent, possessing valour, the overlord of thousands of kings. \blankfootnote{17.21 Most sources read °\textit{ṣaṇḍā}° for °\textit{khaṇḍā}°, which can be
  considered not more than an orthographic variant.
 }}

  \subchptr{gopradānam}%

  \trsubchptr{Donation of cows}%

  \maintext{sacailaghaṇṭāṃ kanakāgraśṛṅgāṃ}%

 \nonanustubhindent \maintext{dogdhrīṃ savatsāṃ payasā dvijānām |}%

  \maintext{dattvā dvijebhyaḥ samalaṅkṛtāṃ gāṃ}%

 \nonanustubhindent \maintext{prayānti lokaṃ surabhīsutānām }||\thinspace17:22\thinspace||%
\translation{Those who give a cow to a Brahmin, along with its calf and milk, one that has been dressed up and has bells, one that has gold on the tip of its horns, one that yields milk, one that has been embellished, will go to the world of cows. \blankfootnote{17.22 Note the odd repetition of \textit{dvija} in \textit{pāda}s cd, and that \textit{samalaṅkṛtāṃ gāṃ} is
  found only in one witness.
 }}

  \maintext{yāvanti romāṇi bhavanti gāvas}%

 \nonanustubhindent \maintext{tāvad yugānām anubhūya bhogān |}%

  \maintext{tasmāc cyutā martya mahībhujās te}%

 \nonanustubhindent \maintext{sahasrarājānugato mahātmā }||\thinspace17:23\thinspace||%
\translation{They will experience enjoyments for that many \ae ons as there are hairs on the cow. Then they will descend to the human world and will become noble rulers controlling a thousand kings. \blankfootnote{17.23 \textit{gāvas} {\rm (}plural nominative{\rm )} in \textit{pāda} a stands 
  for \textit{gavām} {\rm (}plural genitive{\rm )} or \textit{gos} {\rm (}singular genitive{\rm )}.
 Note the stem form \textit{martya} for \textit{martye/martyaṃ} in \textit{pāda} c.
 }}

  \subchptr{suvarṇādipradānam}%

  \trsubchptr{Donation of gold etc.}%

  \maintext{suvarṇakāṃsyāyasaraupyadātā}%

 \nonanustubhindent \maintext{tāmrapravālān maṇimauktikādyān |}%

  \maintext{dattvā dvijebhyo vasusādhyaloke}%

 \nonanustubhindent \maintext{prāpnoti varṣaṃ daśapañcakoṭyaḥ }||\thinspace17:24\thinspace||%
\translation{If one gives golden, brass, iron, or silver objects, copper, coral, gems, pearls, etc., to Brahmins, one will live through 150 million years in the world of the Vasus and Sādhyas. \blankfootnote{17.24 Although to emend °\textit{koṭyo} to °\textit{koṭīḥ} in \textit{pāda} d would be more correct
  than what I have chosen, °\textit{koṭyaḥ}, I have decided to echo 18.4b instead.
 }}

  \maintext{bhuktvā yatheṣṭaṃ krama devalokān}%

 \nonanustubhindent \maintext{cyutaś ca martye sa bhaven narendraḥ |}%

  \maintext{sudurjayaḥ śakrasahasrajetā}%

 \nonanustubhindent \maintext{sudīrgham āyuś ca parākramaś ca }||\thinspace17:25\thinspace||%
\translation{Having enjoyed the divine worlds in due order, according to his wishes, he will descend to the human world and will be a king. He will be extremely difficult to defeat and will be capable to defeat thousands of Śakras [Indras]. Also, [he will have] a very long life and heroism. \blankfootnote{17.25 Note \textit{krama} for \textit{kramāt} in \textit{pāda} a. I have adopted the reading
  °\textit{lokān}, as the object of \textit{bhuktvā}, rejecting °\textit{lokāt}/\textit{lokāc},
  which could also work if we supply \textit{bhogān} as the object of
  \textit{bhuktvā}. My choice is based on 17.47cd: 
  \textit{bhuktvā lokān kramāt sarvān śivaloke pratiṣṭhitaḥ}.
 °\textit{jetā} in \textit{pāda} c may have been the result of metathesis from
  °\textit{tejā}, although the former fits the context perfectly well.
 }}

  \subchptr{vimiśraviṣayāṇi}%

  \trsubchptr{Miscellaneous topics}%

  \maintext{yat prekṣaṇaṃ darśayituṃ pradātā}%

 \nonanustubhindent \maintext{surūpasaubhāgyaphalaṃ labheta |}%

  \maintext{tṛṇāśanāmūlaphalāśanena}%

 \nonanustubhindent \maintext{labheta rājyāni akaṇṭakāni }||\thinspace17:26\thinspace||%
\translation{He who facilitates staging spectacles [for people] to see will obtain the fruits of being handsome and attractive. By eating grass, roots and fruits, one will obtain kingdoms that have no enemies. \blankfootnote{17.26 This section clearly draws on, or paraphrases, \MBH\ 13.7. See some of the 
  most evident parallels in the apparatus.
 
  Note \textit{yat} in \textit{pāda} a...
 The lengthening of the final vowel of \textit{tṛṇāśanā} in \textit{pāda} c is most probably
  metri causa. The repetition of \textit{aśana} is odd, and one wonders how \textit{pāda}s cd,
  and some of the verses below on various ways of fasting,
  connect to our main topic here, donations.
 }}

  \maintext{labheta parṇāśana svargavāsaṃ}%

 \nonanustubhindent \maintext{payaḥprayogena ca devalokam |}%

  \maintext{śuśrūṣaṇe yo gurave ca nityaṃ}%

 \nonanustubhindent \maintext{vidyādharo jāyati martyaloke }||\thinspace17:27\thinspace||%
\translation{He who feeds on leaves will obtain a stay in heaven. By using only milk/water, [he will get] to the divine world. And he who is always [engaged] in obedience towards the guru will be born as a demigod {\rm (}\textit{vidyādhara}{\rm )} in the human world. }

  \maintext{dadyād gavāṃ grāsa tṛṇasya muṣṭiṃ}%

 \nonanustubhindent \maintext{gavāḍhyatāṃ jāyati martyaloke |}%

  \maintext{śrāddhaṃ ca dattvā prayato dvijāya}%

 \nonanustubhindent \maintext{samṛddhasantāna bhaved yugānte }||\thinspace17:28\thinspace||%
\translation{Should one give food, a handful of grass, to cows, one will be reborn in the human world possessing an abundancy of cows. By giving [food] to a Brahmin [after] a Śrāddha ritual piously, [his] lineage will be rich until the end of the \ae on. \blankfootnote{17.28 Note that \textit{gogrāsa} as a technical term denotes the ritual feeding of cows.
  See, e.g., \DEVIBH\ 11.22.16ff.
 }}

  \maintext{ahiṃsako jāyati dīrgham āyuḥ}%

 \nonanustubhindent \maintext{kulottamo jāyati dīkṣitena |}%

  \maintext{kālatrayaṃ snānakṛtena rājyaṃ}%

 \nonanustubhindent \maintext{pītvā ca vāyuṃ tridaśādhipatvam }||\thinspace17:29\thinspace||%
\translation{He who refrains from violence will have a long life. By being initiated, he will have a high-class family. He who performs a bath thrice [a day will have] a kingdom. He who drinks [only] air [i.e.\ fasts] will be the lord of the thirty [gods]. }

  \maintext{anaśnatāyāḥ phalam īśaloke}%

 \nonanustubhindent \maintext{tṛptir bhavet toyapradānaśīlaḥ |}%

  \maintext{annapradātā puruṣaḥ samṛddhaḥ}%

 \nonanustubhindent \maintext{sa sarvakāmān labhatīha loke }||\thinspace17:30\thinspace||%
\translation{The fruit of not eating is in Īśaloka. One whose habit is to give water will have satisfaction. A man who gives food will be rich; he will fulfil all his desires in this world. \blankfootnote{17.30 \textit{Pāda} a may hint at suicide by fasting.
 }}

  \maintext{śraddhāmatir yaḥ praviśed dhutāśaṃ}%

 \nonanustubhindent \maintext{sa yāti lokaṃ prapitāmahasya |}%

  \maintext{satyaṃ vaded yo 'pi ca dharmaśīlo}%

 \nonanustubhindent \maintext{modaty asau devi sahāpsarobhiḥ }||\thinspace17:31\thinspace||%
\translation{He who enters the fire with a trusting mind will go to the world of the Grandfather [i.e.\ Brahmā]. And if a virtuous person speaks the truth, he will rejoice, O Devī, together with Apsarases. \blankfootnote{17.31 Entering the fire as a method of ritual suicide appears, e.g., in \NISVMUKH\ 3.16cd--17:
  \textit{yaḥ tīrthaṃ smarate nityaṃ maraṇaṃ cābhikāṅkṣate\thinspace || 
  agnipraveśaṃ yaḥ kuryān mānavo niyame sthitaḥ\thinspace |
  rudralokam avāpnoti tenaiva saha modate\thinspace ||};
  `He who always remembers [a certain] pilgrimage site and desires to
  die {\rm (}\textit{maraṇaṃ cābhikāṃkṣate}{\rm )} [there] [and] who [therefore] enters the fire
  [there], following the prescribed injunction {\rm (}\textit{niyame sthitaḥ}{\rm )}, [that] man
  {\rm (}\textit{mānavaḥ}{\rm )} will obtain the world of Rudra and rejoice [there] with him.'
  {\rm (}Tr.\ in \mycitep{KafleNisvasaBook}{274}.{\rm )}
 }}

  \maintext{rasāṃs tu ṣaḍ ye parivarjayanti}%

 \nonanustubhindent \maintext{atīva saubhāgya labheta sādhvī |}%

  \maintext{dānena bhogān atulān labheta}%

 \nonanustubhindent \maintext{cirāyutāṃ yāti hi brahmacaryāt }||\thinspace17:32\thinspace||%
\translation{As for someone who completely gives up the six flavours, a virtuous woman {\rm (}\textit{sādhvī}{\rm )} will obtain excessive prettiness. One can obtain matchless enjoyments by donating. By chastity, one can have a long life. }

  \maintext{dhanāḍhyatāṃ yāti hi puṇyakarmā}%

 \nonanustubhindent \maintext{maunena ājñāṃ labhate alaṅghyām |}%

  \maintext{prāpnoti kāmaṃ tapasaḥ sutaptaṃ}%

 \nonanustubhindent \maintext{kīrtiṃ yaśaḥ svargam anantabhogam |}%

  \maintext{āyuḥśriyārogyadhanaprabhutvaṃ}%

 \nonanustubhindent \maintext{jñānādilābhaṃ tapasā labheta }||\thinspace17:33\thinspace||%
\translation{Those who perform meritorious acts will have an abundance of wealth. By observing silence, one can exercise inviolable command. He who practises austerities will fulfil his desires. One will obtain fame, glory, heaven, endless enjoyments, longevity, beauty, health, wealth, sovereignty, knowledge, etc., by asceticism. \blankfootnote{17.33 It is best to understand \textit{tapasaḥ sutaptaṃ} in \textit{pāda} c
  as \textit{tapasi sutapte} or \textit{sutaptatapasaḥ} {\rm (}ablative{\rm )}.
 }}

  \maintext{trailokyādhipatitva śakra{-}m{-}agamat kṛtvā tapo duṣkaraṃ}%

 \nonanustubhindent \maintext{yakṣeśo 'pi tapaḥprabhāvam abhavad guhyādhipatyaṃ mahat |}%

  \maintext{rakṣeśo 'pi vibhīṣaṇas tv amaratāṃ prāptas tapasaiva tu}%

 \nonanustubhindent \maintext{rudrārādhanatatparas tapaphalān nandī gaṇatvaṃ gataḥ }||\thinspace17:34\thinspace||%
\translation{Śakra [i.e.\ Indra] became the ruler of the three worlds by doing arduous penance. The king of the Yakṣas [i.e.\ Kubera], too: [by] the power of [his] austerities mighty sovereignty arose over the Guhya[ka]s. The king of the Rakṣas, Vibhīṣaṇa [Rāvaṇa's brother] also gained immortality merely by penance. Nandin became one of the Gaṇas as the fruit of [his] penance: focusing on the worship of Rudra. \blankfootnote{17.34 \textit{Pāda} a may refer to the penance Indra performed in Malada/Karūṣa after killing Vṛtra. See
  \Ramayana\ 1.23.17ff.:
  \textit{purā vṛtravadhe rāma malena samabhiplutam\thinspace |
  kṣudhā caiva sahasrākṣaṃ brahmahatyā yadāviśat\thinspace ||
  tam indraṃ snāpayan devā ṛṣayaś ca tapodhanāḥ\thinspace |
  kalaśaiḥ snāpayām āsur malaṃ cāsya pramocayan\thinspace ||}, etc.
 
 \textit{Pāda} b probably refers to the penance Kubera performed to obtain favours
  from Brahmā. A hint on this is, e.g., in \Ramayana\ 5.7.10--11:
  \textit{brahmaṇo 'rthe kṛtaṃ divyaṃ divi yad viśvakarmaṇā\thinspace |
  vimānaṃ puṣpakaṃ nāma sarvaratnavibhūṣitam\thinspace ||
  pareṇa tapasā lebhe yat kuberaḥ pitāmahāt\thinspace |
  kuberam ojasā jitvā lebhe tad rākṣaseśvaraḥ\thinspace ||}.
 
 The reference in \textit{pāda} c is to Vibhīṣaṇa's penance by which he
  received the boon to live as a righteous man. See \Ramayana\ 7.9.30ff, especially 7.10.29--30:
  \textit{atha prajāpatiḥ prīto vibhīṣaṇam uvāca ha\thinspace |
  dharmiṣṭhas tvaṃ yathā vatsa tathā caitad bhaviṣyati\thinspace ||
  yasmād rākṣasayonau te jātasyāmitrakarṣaṇa\thinspace |
  nādharme jāyate buddhir amaratvaṃ dadāmi te\thinspace ||}.
 
 On Nandin's becoming a Gaṇa {\rm (}\textit{pāda} d{\rm )}, see, e.g., \SKANDAP\ 21.18ff.
 }}

  \maintext{jñānaṃ dvijānāṃ tapa āha viṣṇuḥ}%

 \nonanustubhindent \maintext{kṣatraṃ tapo rakṣaṇam āha sūryaḥ |}%

  \maintext{vaiśyaṃ tapaś cārjanam āha vāyuḥ}%

 \nonanustubhindent \maintext{śūdraṃ hi śilpaṃ tapa āha indraḥ }||\thinspace17:35\thinspace||%
\translation{Viṣṇu proclaimed that the penance of Brahmins was knowledge. Sūrya taught that the penance of Kṣatriyas was protection. Vāyu said that the penance of Vaiśyas was accumulating [wealth]. With regards to Śūdras, Indra taught handicraft as penance. }

  \maintext{raṇotsahaṃ kṣatriyayajñam iṣṭaṃ}%

 \nonanustubhindent \maintext{vaiśye havir yajñam udāharanti |}%

  \maintext{śūdrasya yajñaḥ paricaryam iṣṭaṃ}%

 \nonanustubhindent \maintext{yajñaṃ dvijānāṃ japam ukta{\rm †}mokṣaḥ{\rm †} }||\thinspace17:36\thinspace||%
\translation{Prowess in battle is regarded as proper worship for Kṣatriyas. With regards to Vaiśyas, fire-oblation is said to be worship. Service is regarded as worship for Śūdras. The Brahmins' worship is recitation... ? \blankfootnote{17.36 Compare verse 19.40.
 }}

  \subchptr{svamāṃsarudhiraputrakalatradānam}%

  \trsubchptr{Donation of one's own flesh and blood, son and wife}%

  \maintext{devy uvāca |}%

  \maintext{svamāṃsarudhiraṃ dānaṃ dānaṃ putrakalatrayoḥ |}%

  \maintext{kiṃ praśasyaṃ mahādeva śrotum icchāmi tattvataḥ }||\thinspace17:37\thinspace||%
\translation{Devī spoke: Are one's own flesh and blood and one's son and wife praised as donation, O Mahādeva? Tell me the truth please. }

  \maintext{maheśvara uvāca |}%

  \maintext{svamāṃsarudhiraṃ dānaṃ praśaṃsanti manīṣiṇaḥ |}%

  \maintext{śrūyatāṃ pūrvavṛttāni saṃkṣipya kathayāmy aham }||\thinspace17:38\thinspace||%
\translation{Maheśvara spoke: The wise praise one's own flesh and blood as donation. Let us hear the old legends. I shall relate them in brief. }

  \maintext{uśīnaras tu rājarṣiḥ kapotārthe svakāṃ tanum |}%

  \maintext{tyaktvā svargam anuprāptaḥ parārthe paratatparaḥ }||\thinspace17:39\thinspace||%
\translation{Uśīnara, the royal saint, by giving up his own body to save a dove, focusing on others for a higher aim, reached heaven. \blankfootnote{17.39 King [Śibi] Uśīnara offered his own flesh to save a dove,
  who was Agni in disguise, from a hawk, who was Indra. 
  See \MBH\ 3.130--131, especially 3.131.22--23:
  \textit{śyena uvāca\thinspace |
  uśīnara kapote te yadi sneho narādhipa\thinspace |
  ātmano māṃsam utkṛtya kapotatulayā dhṛtam\thinspace ||
  yadā samaṃ kapotena tava māṃsaṃ bhaven nṛpa\thinspace |
  tadā pradeyaṃ tan mahyaṃ sā me tuṣṭir bhaviṣyati\thinspace ||.}
  See also \Ramayana\ 2.12.4:
  \textit{saṃśrutya śaibyaḥ śyenāya svāṃ tanuṃ jagatīpatiḥ\thinspace |
  pradāya pakṣiṇo rājañ jagāma gatim uttamām\thinspace ||.}
 }}

  \maintext{putramāṃsaṃ svayaṃ chittvā agnidattaṃ purānaghe |}%

  \maintext{tena dānaprabhāvena alarkas tridivaṃ gataḥ }||\thinspace17:40\thinspace||%
\translation{He himself cut the flesh of his son, and roasted it, in the past, O sinless Goddess. Alarka reached the third heaven by force of the same [type of] donation. \blankfootnote{17.40 Traditionally, it is Śibi [Uśīnara] who is said to have killed and 
  cooked his own son, Bṛhadgarbha, without hesitation,
  on the request of a Brahmin. See \MBH\ Suppl. 3.21:
  \textit{asāv ahaṃ śibinā samo nāsmi}\thinspace ||139||
  \textit{yato brāhmaṇaḥ kaścid enam abravīt\thinspace | śibe annārthy asmīti}\thinspace ||140||
  \textit{tam abravīc chibiḥ\thinspace | kiṃ kriyatām\thinspace | ājñāpayatu bhavān iti}\thinspace ||141|| 
  \textit{athainaṃ brāhmaṇo 'bravīt\thinspace | ya eṣa te putro bṛhadgarbho nāma eṣa pramātavya iti\thinspace | 
  tam enaṃ saṃskuru\thinspace |
  annaṃ copapādaya\thinspace | tato 'haṃ pratīkṣya iti}\thinspace ||142|| 
  \textit{tataḥ putraṃ pramāthya saṃskṛtya vidhinā sādhayitvā pātryām 
  arpayitvā śirasā pratigṛhya brāhmaṇam amṛgayat}\thinspace ||143|| 
  \textit{athāsya mṛgayamāṇasya kaś cid ācaṣṭa\thinspace | eṣa te brāhmaṇo nagaraṃ praviśya dahati 
  te gṛhaṃ kośāgāram āyudhāgāraṃ
  stryagāram aśvaśālāṃ hastiśālāṃ ca kruddha iti}\thinspace ||144|| 
  \textit{atha śibis tathaivāvikṛtamukhavarṇo
  nagaraṃ praviśya brāhmaṇaṃ tam abravīt\thinspace | siddhaṃ bhagavann annam iti}\thinspace ||145|| 
  \textit{brāhmaṇo na kiṃ cid vyājahāra\thinspace | vismayād adhomukhaś cāsīt}\thinspace ||146|| 
  \textit{tataḥ prāsādayad brāhmaṇam\thinspace | bhagavan bhujyatām iti}\thinspace ||147|| 
  \textit{muhūrtād udvīkṣya śibim abravīt\thinspace | tvam evaitad aśāneti}\thinspace ||148||
  \textit{tatrāha\thinspace | tathā\thinspace | iti}\thinspace ||149||
  \textit{śibis tathaivāvimanā mahitvā kapālam abhyuddhārya bhoktum aicchat}\thinspace ||150|| 
  \textit{athāsya brāhmaṇo hastam agṛhṇāt\thinspace | abravīc cainam\thinspace | jitakrodho 'si\thinspace | 
  na te kiṃ cid aparityājyaṃ brāhmaṇārthe}\thinspace ||151|| 
  \textit{brāhmaṇo 'pi taṃ mahābhāgaṃ sabhājayat}\thinspace ||152|| 
  \textit{sa hy udvīkṣyamāṇaḥ putram apaśyad agre tiṣṭhantaṃ 
  devakumāram iva puṇyagandhānvitam alaṃkṛtam}\thinspace ||153|| 
  \textit{sarvaṃ ca tam arthaṃ vidhāya brāhmaṇo 'ntaradhīyata}\thinspace ||154||
  \textit{tasya rājarṣer vidhātā tenaiva veṣeṇa parīkṣārtham āgata iti}\thinspace ||155||.
 
 Alarka, on the other hand, gave his own eyes to a Brahmin. See \Ramayana\ 2.12.5:
  \textit{tatha hy alarkas tejasvī brāhmaṇe vedapārage\thinspace |
  yācamāne svake netre uddhṛtyāvimanā dadau}\thinspace ||.
  The redactor of this verse seems to have considered the above mentioned
  story of the sacrificed son to be connected to Alarka, rather than
  Śibi, or possibly a line may have dropped out.
  In \Ramayana\ 2.12.4--5, the two stories in question,
  that of Uśīnara killing his son and that involving Alarka
  offering his eyes to a Brahmin, are mentioned
  next to each other: 
  \textit{saṃśrutya śaibyaḥ śyenāya svāṃ tanuṃ jagatīpatiḥ\thinspace |
  pradāya pakṣiṇo rājañ jagāma gatim uttamām\thinspace ||
  tathā hy alarkas tejasvī brāhmaṇe vedapārage\thinspace |
  yācamāne svake netre uddhṛtyāvimanā dadau\thinspace ||}.
 }}

  \maintext{svadāradānena sudāsaputra}%

 \nonanustubhindent \maintext{aputrabhūtasya ca putra jātaḥ |}%

  \maintext{svarge svayaṃ cākṣayabhogalābhaṃ}%

 \nonanustubhindent \maintext{prāpto mahaddānaphalaprabhāvāt }||\thinspace17:41\thinspace||%
\translation{By donating his own wife, a son was born to Sudāsa's son, who had not had a son. He himself reached undecaying enjoyments in heaven thanks to the fruition of this great act of giving. \blankfootnote{17.41 Sudāsa's son was Mitrasaha, later, when living as a Rākṣasa, also known as Kalmāṣapāda.
  After Vasiṣṭha's curse turned him into a Rākṣasa, another curse fell upon his head
  pronuonced by a Brāhmaṇī whose husband he had killed. This second curse meant that
  the moment he touched a woman, he would die. For this reason, he later offered
  his own wife, Madayantī, to Vasiṣṭha, in order to beget a son {\rm (}Aśmaka{\rm )}.
  See \BHAGP\ 9.9.18ff. For another example a giving away one's own wife, 
  see \VSS\ chapter 12.
  °\textit{putra/e} in \textit{pāda} a is either a locative, meant to agree with
  °\textit{bhūtasya} in \textit{pāda} b, or is rather to be taken as a stem form noun. In fact
  \textit{sudāsaputra aputrabhūtasya} looks like a noun phrase in which only the last
  element is declined.
 }}

  \maintext{yādavaś cārjuno devi dattvā khāṇḍavabhojanam |}%

  \maintext{tapanasya prasādena saptadvīpeśvaro bhavet }||\thinspace17:42\thinspace||%
\translation{Yādava [i.e.\ Kṛṣṇa] and Arjuna allowed [Agni] to consume the Kāṇḍava[-forest]. By the kindness of Agni {\rm (}\textit{tapana}{\rm )}, [Arjuna] became the lord of the seven islands. \blankfootnote{17.42 In their fight with Indra, Arjuna and Kṛṣṇa set the Khāṇḍava forest on fire
  in order for Agni to consume it, causing the death of many living creatures.
  See \MBH\ 1.214.1ff {\rm (}\textit{khāṇḍavadahaparvan}{\rm )}, and \textit{khāṇḍavadāha} in
  \mycite{PuranicEnc} s.v.
 \textit{tapana} means `burning,' but could also signify Agni.
  Also, consider emending \textit{bhavet} in \textit{pāda} d to \textit{'bhavat}.
 }}

  \maintext{hariṇā ca śirāṃ bhittvā dattaṃ me rudhiraṃ purā |}%

  \maintext{pratīcchitaṃ kapālena brahmasambhavajena me }||\thinspace17:43\thinspace||%
\translation{In the old times, Hari [i.e.\ Viṣṇu] cut one of his veins and gave me his blood. I collected it in the skull that used to belong to Brahmā. \blankfootnote{17.43 For this episode of Viṣṇu donating his blood to Śiva,
  and for support of Yokochi's highly welcome emendation in \textit{pāda} a, see
  \SkandaP\ 6.4--6cd:
  \textit{abhyagāt saṃkrameṇaiva veśma viṣṇor mahātmanaḥ\thinspace |
  tasyātiṣṭhata sa dvāri bhikṣām uccārayañ chubhām\thinspace ||
  sa dṛṣṭvā tadupasthaṃ tu viṣṇur vai yogacakṣuṣā\thinspace |
  śirāṃ lalāṭāt sambhidya raktadhārām apātayat\thinspace |
  papāta sā ca vistīrṇā yojanārdhaśataṃ tadā\thinspace ||
  tayā patantyā viprendrā bahūny abdāni dhārayā\thinspace |
  pitāmahakapālasya nārdham apy abhipūritam\thinspace |}.
 
  See also \MATSP\ 183.87--91 and ff., where Viṣṇu 
  slits open his own side:
  \textit{tato 'haṃ gatavān devi himavantaṃ śiloccayam\thinspace | 
  tatra nārāyaṇaḥ śrīmān mayā bhikṣāṃ prayācitaḥ\thinspace || 
  tatas tena svakaṃ pārśvaṃ nakhāgreṇa vidāritam\thinspace |
  sravato mahatī dhārā tasya raktasya niḥsṛtā\thinspace || 
  prayātā sātivistīrṇā yojanārddhaśatan tadā\thinspace | 
  na saṃpūrṇaṃ kapālan tu ghoram adbhuta darśanam\thinspace || 
  divyaṃ varṣasahasran tu sā ca dhārā pravāhinī\thinspace | 
  provāca bhagavān viṣṇuḥ kapālaṃ kuta īdṛśam\thinspace || 
  āścaryabhūtaṃ deveśa saṃśayo hṛdi vartate\thinspace | 
  kutaś ca sambhavo deva sarvaṃ me brūhi pṛcchataḥ\thinspace ||}
 
 \textit{pratīcchitaṃ} {\rm (}`received'{\rm )} in \textit{pāda} c is a participle that
  \mycitep{EdgertonHybrid}{vol.\ 2, 373} 
  labels as a `non-Sanskrit' {\rm (}Buddhist hybrid{\rm )} form of \textit{pratīcchati}.
 }}

  \maintext{divyavarṣasahasrāṇi dhārā tasya na chidyate |}%

  \maintext{parituṣṭo 'smi tenāhaṃ karmaṇānena sundari }||\thinspace17:44\thinspace||%
\translation{For a thousand divine years, its flow did not stop. I became satisfied with him, with this deed, O beautiful Goddess. }

  \maintext{varaṃ dattaṃ mayā devi purāṇapuruṣo 'vyayaḥ |}%

  \maintext{akṣayaṃ balam ūrjaṃ ca ajarāmaram eva ca }||\thinspace17:45\thinspace||%
\translation{I gave him a boon [which was] an undecaying prim\ae val man, O Goddess, impersishable, powerful, strong, and immortal. \blankfootnote{17.45 To interpret this verse correctly, we have to read on
  \SkandaP\ 6, a possible source for this section:
  \textit{sasarja puruṣaṃ dīptaṃ viṣṇoḥ sadṛśam ūrjitam}\thinspace ||6.10||
  \textit{tam āhāthākṣayaś cāsi ajarāmara eva ca}\thinspace |
  \textit{yuddheṣu cāpratidvandvī sakhā viṣṇor anuttamaḥ}\thinspace |
  \textit{devakāryakaraḥ śrīmān sahānena carasva ca}\thinspace ||6.11||;
  `He [Śiva] created a brilliant Man, who was
  as powerful as Viṣṇu. He said to him: 
  ``Now you are imperishable and immortal,
  unconquerable in battle, Viṣṇu's greatest friend.
  You are the fortunate one who carries out the god's tasks.
  Follow him [i.e.\ Viṣṇu].'' '
 }}

  \maintext{mamādhikaṃ bhaved viṣṇur mām api tvaṃ vijeṣyasi |}%

  \maintext{evamādīny anekāni mayoktāni janārdane }||\thinspace17:46\thinspace||%
\translation{`Viṣṇu will surpass me. You will be able to defeat even me,' I told Janārdana [i.e.\ Viṣṇu], and many similar things. }

  \maintext{niṣkampaniścalamanāḥ sthāṇubhūta iva sthitaḥ |}%

  \maintext{dadhīciḥ svatanuṃ dattvā vibudhānāṃ varānane |}%

  \maintext{bhuktvā lokān kramāt sarvān śivaloke pratiṣṭhitaḥ }||\thinspace17:47\thinspace||%
\translation{With an unwavering and motionless mind, standing [firm] like a pillar, Dadhīci gave the gods his own body, O Varānanā. Enjoying all the worlds in due order, he is now living in Śivaloka. \blankfootnote{17.47 Dadhīci/Dadhīca killed himself when the gods told him
  they needed his bones to forge a weapon against Vṛtra. 
  See \MBH\ 12.329.25A ff:
  \textit{tān brahmovāca\thinspace | `ṛṣir bhārgavas tapas tapyate dadhīcaḥ
  sa yācyatāṃ varaṃ yathā kalevaraṃ jahyāt
  tasyāsthibhir vajraṃ kriyatām' iti\thinspace |
  devās tatrāgacchan yatra dadhīco bhagavān ṛṣis tapas tepe\thinspace |
  sendrā devās tam abhigamyocur `bhagavaṃs tapasaḥ kuśalam avighnaṃ ceti'
  tān dadhīca uvāca `svāgataṃ bhavadbhyaḥ kiṃ kriyatām\thinspace |
  yad vakṣyatha tat kariṣyāmīti'
  te tam abruvañ `śarīraparityāgaṃ lokahitārthaṃ bhagavān kartum arhatīti'\thinspace |
  atha dadhīcas `tathaiva' avimanāḥ sukhaduḥkhasamo mahāyogī ātmānaṃ samādhāya 
  śarīraparityāgaṃ cakāra\thinspace | tasya paramātmany avasṛte 
  tāny asthīni dhātā saṃgṛhya vajram akarot\thinspace |
  tena vajreṇābhedyenāpradhṛṣyeṇa brahmāsthisaṃbhūtena 
  viṣṇupraviṣṭenendro viśvarūpaṃ jaghāna\thinspace |
  śirasāṃ cāsya chedanam akarot\thinspace | tasmād anantaraṃ viśvarūpagātramathanasaṃbhavaṃ
  tvaṣṭrotpāditam evāriṃ vṛtram indro jaghāna\thinspace |}.
 }}

  \subchptr{anyāni dānāni}%

  \trsubchptr{Other types of donation}%

  \maintext{jāmadagnir mahīṃ dattvā kāśyapāya mahātmane |}%

  \maintext{ihaiva sa phalaṃ bhoktā devarājyam avāpsyati }||\thinspace17:48\thinspace||%
\translation{Jāmadagni [i.e.\ Paraśurāma] gave the Earth to the great-souled Kāśyapa. In this very same place [i.e.\ Śivaloka], he is enjoying the fruits [of his actions] and will reach the divine kingdom. \blankfootnote{17.48 After Paraśurāma destroyed the Kṣatriyas, he gave the whole world to Kāśyapa. 
  See, e.g., \Harivamsa\ Appendix 1.21.99--101:
  \textit{abhūś ca jāmadagnyas tvaṃ gṛhītvā paraśuṃ prabho\thinspace |
  hatavāṃs tvaṃ mahāvīryaṃ kṛtavīryasutaṃ raṇe\thinspace |
  niḥkṣatriyam imaṃ lokam adadāḥ kāśyapāya vai\thinspace ||}.
 }}

  \maintext{dattvā gosakalaṃ devi vyāsasyāmitatejasaḥ |}%

  \maintext{yudhiṣṭhiro mahīpālaḥ sadehas tridivaṃ gataḥ }||\thinspace17:49\thinspace||%
\translation{Giving all the world, O Devī, to Vyāsa of boundless glory, king Yudhiṣṭhira went to the third heaven in his bodily form. \blankfootnote{17.49 One could emend \textit{gosakalaṃ} in \textit{pāda} a to
  \textit{gāṃ sakalāṃ}, but \textit{gosakalaṃ} might be original.
  That Yudhiṣṭhira donated the whole world to Vyāsa is
  mentioned in \MBH\ 14.91.7:
  \textit{tato yudhiṣṭhiraḥ prādāt sadasyebhyo yathāvidhi\thinspace |
  koṭīsahasraṃ niṣkāṇāṃ vyāsāya tu vasuṃdharām\thinspace ||};
  `Then Yudhiṣṭhira gave the superintending priests,
  according to rule, a thousand krore of golden coins,
  and the whole world to Vyāsa.' 
  That Yudhiṣṭhira entered heaven in a bodily form
  comes up in \MBH\ 180.3.39--40:
  \textit{gaṅgāṃ devanadīṃ puṇyāṃ pāvanīm ṛṣisaṃstutām\thinspace |
  avagāhya tu tāṃ rājā tanuṃ tatyāja mānuṣīm\thinspace ||
  tato divyavapur bhūtvā dharmarājo yudhiṣṭhiraḥ\thinspace |
  nirvairo gatasaṃtāpo jale tasmin samāplutaḥ\thinspace ||}.
 }}

  \maintext{satyabhāmā svakaṃ bhartrā dattvā nāradasatkṛtam |}%

  \maintext{dānasyāsya prabhāvena akṣayaṃ tridivaṃ gataḥ }||\thinspace17:50\thinspace||%
\translation{Satyabhāmā gave her own wealth {\rm (}\textit{svaka}{\rm )} [equal in weight to the wishing-tree together] with [her] husband [Kṛṣṇa] as a way to honour Nārada. By the force of this donation, he [i.e. Nārada] went to the third heaven. \blankfootnote{17.50 The interpretation of this verse is tentative. It seems to 
  refer to the episode when Kṛṣṇa was given flowers of the heavenly wishing tree by Nārada.
  Kṛṣṇa failed to pass any of them to his favourite wife Satyabhāmā {\rm (}note emendation in 
  \textit{pāda} a: an original -\textit{ā} may have been misread as a visarga{\rm )}. 
  Kṛṣṇa's blunder was remedied by a journey to heaven together, and
  by bringing the wishing tree from the world of gods to Satyabhāmā's garden. 
  Nārada told Satyabhāmā that in order for her to have the tree in 
  all of her births, all she has to do is \textit{tulāpuruṣadāna},
  one of the \textit{mahādāna}s. This involves donating as much gold as the weight of the donor. 
  She gave Nārada as much gold as the weight of her husband, Kṛṣṇa, plus
  the tree. After this, Nārada departed to heaven. 
  See \mycite{PuranicEnc} s.v. `Satyabhāmā', and 
  \PadmaP\ 6.88.15--17:
  \textit{satyabhāmovāca\thinspace | 
  īdṛśaḥ kalpavṛkṣo 'yaṃ patir etādṛśaḥ prabhuḥ\thinspace | 
  bhave bhave kathaṃ prāpyas tad ākhyātu bhavān mama\thinspace || 
  iti pṛṣṭas tadā prāha nārado munisattamaḥ\thinspace | 
  prāpyate satyabhāme 'yaṃ tulāpuruṣadānataḥ\thinspace || 
  satyabhāmā tadā kṛṣṇaṃ kalpavṛkṣasamanvitam\thinspace | 
  nāradāyaiva sā prādāt tolayitvā vidhānataḥ\thinspace | 
  sarvopaskaram ākṛṣya nāradas tridivaṃ yayau\thinspace ||}
 }}

  \maintext{catuḥṣaṣṭisahasrāṇi gavāṃ dattvā dvijanmane |}%

  \maintext{duryodhano mahīpālo gataḥ svargam anantakam }||\thinspace17:51\thinspace||%
\translation{By giving sixty-four thousand cows to the Brahmin, king Duryodhana went to boundless heaven. }

  \maintext{vāsukiḥ sarparājendro dattvā vipra susaṃskṛtām |}%

  \maintext{jaratkāroś ca sā bhāryā sarve nāga vimokṣitāḥ }||\thinspace17:52\thinspace||%
\translation{Vāsuki, the serpent king, gave [his sister to] the Brahmin fully adorned. She was Jaratkāru's wife. All the Nāgas were released. \blankfootnote{17.52 The translation of \textit{pāda} d is tentative and presupposes
  that \textit{vipra} is in stem form for a dative or genitive.
 
  Jaratkāru was Vāsuki's sister. He gave her to Jaratkāru {\rm (}sic!{\rm )} in marriage. She,
  through her son Āstīka, later saved the Serpents from Janamejaya's \textit{sarpasattra},
  a sacrifice to kill the Nāgas. See \MBH\ 1.13.34ff:
  \textit{vāsukir uvāca\thinspace |
  jaratkāro jaratkāruḥ svaseyam anujā mama\thinspace |
  pratigṛhṇīṣva bhāryārthe mayā dattāṃ sumadhyamām\thinspace ||
  tvadarthaṃ rakṣitā pūrvaṃ pratīcchemāṃ dvijottama\thinspace | 
  evam uktvā tataḥ prādād bhāryārthe varavarṇinīm\thinspace ||
  sūta uvāca\thinspace |
  mātrā hi bhujagāḥ śaptāḥ pūrvaṃ brahmavidāṃ vara\thinspace |
  janamejayasya vo yajñe dhakṣyaty anilasārathiḥ\thinspace ||
  tasya śāpasya śāntyarthaṃ pradadau pannagottamaḥ\thinspace |
  svasāram ṛṣaye tasmai suvratāya tapasvine\thinspace ||
  sa ca tāṃ pratijagrāha vidhidṛṣṭena karmaṇā\thinspace |
  āstīko nāma putraś ca tasyāṃ jajñe mahātmanaḥ\thinspace ||};
 
  See also \MBH\ 1.48.1ff, especially 1.51.20:
  \textit{āstīka uvāca\thinspace |
  suvarṇaṃ rajataṃ gāś ca na tvāṃ rājan vṛṇomy aham
  satraṃ te viramatv etat svasti mātṛkulasya naḥ\thinspace ||};
  `Āstīka spoke: I do not ask you for gold, silver or cows,
  O king [Parīkṣit]. Let your sacrifice [of Serpents] stop,
  and let out maternal family prosper!'
 }}

  \subchptr{dānabhūmayaḥ}%

  \trsubchptr{Levels of donation}%

  \maintext{gobhūmikanakādīnāṃ dānaṃ kanyasam ucyate |}%

  \maintext{bhṛtyaputrakalatrāṇāṃ dānaṃ madhyamam ucyate }||\thinspace17:53\thinspace||%
\translation{The donation of cows, land, gold, etc.\ is regarded the lowest. The donation of servants, sons and wives is regarded as mediocre. }

  \maintext{svadehapiśitādīnāṃ dānam uttamam ucyate |}%

  \maintext{etat sarvaṃ yadā dānaṃ tad dānam uttamottamam }||\thinspace17:54\thinspace||%
\translation{The donation of one's own body, flesh, etc. is regarded as superior. When donation consists of all these, that donation is the ultimate one. }

  \maintext{yāvaj janmasahasrāṇi bhoktā bhavati kanyasaḥ |}%

  \maintext{śatajanmasahasrāṇi bhoktā bhavati madhyamaḥ }||\thinspace17:55\thinspace||%
\translation{[One who performs] the lowest will enjoy for a thousand births. [One who performs] the mediocre one will enjoy for a hundred thousand births. }

  \maintext{uttamaḥ phalabhoktā ca janmakoṭiśatatrayam | }%

  \maintext{parārdhadvayajanmānāṃ bhoktā vai cottamottamaḥ }||\thinspace17:56\thinspace||%
\translation{[One who performs] the superior one will enjoy its fruits for three billion births. [One who performs] the ultimate one will enjoy for two half-\textit{para}s. \blankfootnote{17.56 For numbers such as \textit{parārdha} and \textit{para}, see verses 1.31ff.
 }}

  \maintext{bhūtānām anukampayā yadi dhanaṃ dātā sadā tv arthine}%

 \nonanustubhindent \maintext{dīnāndhakṛpaṇeṣv anāthamaline śvānāditiryakṣu ca |}%

  \maintext{yady evaṃ kurute sadārtiharaṇaṃ śraddhānvito bhaktimān}%

 \nonanustubhindent \maintext{tasyānantaphalaṃ vadanti vibudhāḥ saṃyamya saṃdarśanāt }||\thinspace17:57\thinspace||%
\translation{If someone regularly donates, out of compassion for living beings, to those in need, to the miserable, the blind, the poor, the helpless and [religious mendicants wearing] dirty clothes, even to animals such as dogs, and always tries to remove suffering in this way, having faith and devotion, he will have endless rewards, the wise/the gods say, as a result of a full inspection [that they carried out] together ???. \blankfootnote{17.57 Note how the variant \textit{anukampāyā} in \textit{pāda} a must be
  the result of trying to read the line as an \textit{anuṣṭubh}.
 Read \textit{kṛpaṇeṣv} in \textit{pāda} b as \textit{kripaṇeṣv} to restore the metre.
  See a similar case in 17.7a.
 }}

\centerline{\maintext{\dbldanda\thinspace iti vṛṣasārasaṃgrahe dānadharmaviśeṣaṃ nāma saptādaśamo 'dhyāyaḥ\thinspace\dbldanda}}
\translation{Here ends the seventeenth chapter in the Vṛṣasārasaṃgraha called The particulars of the Dharma of donation.}

  \chptr{aṣṭādaśamo 'dhyāyaḥ}
\fancyhead[CO]{{\footnotesize\textit{Translation of chapter 18}}}%

  \trchptr{Chapter Eighteen}%

  \subchptr{svargān martyam upāgatānāṃ cihnāni}%

  \trsubchptr{Marks of those who return from heaven}%

  \maintext{devy uvāca |}%

  \maintext{bhuktvā tu bhogān suciraṃ yatheṣṭaṃ}%

 \nonanustubhindent \maintext{puṇyakṣayān martyam upāgatānām |}%

  \maintext{cihnāni teṣāṃ kathayasva me 'dya}%

 \nonanustubhindent \maintext{yathākramaṃ karmaphalaṃ viśeṣāt }||\thinspace18:1\thinspace||%
\translation{Devī spoke: Please tell me now about the characteristic marks of those who, after having experienced enjoyable things as they please for a long time, their merits thus having worn away, return to the mortal world, and especially about the fruits of their deeds, one by one. }

  \subchptr{dānāṣṭakam}%

  \trsubchptr{Eight kinds of donation}%

  \maintext{maheśvara uvāca |}%

  \maintext{sadānnadātā kṛpaṇārtidīnāṃ}%

 \nonanustubhindent \maintext{sa varṣakoṭyāyutam īśaloke |}%

  \maintext{bhuktvā ca bhogān samam apsarobhiḥ}%

 \nonanustubhindent \maintext{prakṣīṇapuṇyaḥ punar eti martyam }||\thinspace18:2\thinspace||%
\translation{Maheśvara spoke: He who regularly gives food to the poor and to the ones afflicted by pain will experience enjoyments in Īśaloka together with Apsarases for millions of years, before he returns to the world of mortals, his merits having worn away. \blankfootnote{18.2 Note the variant \textit{bhagavān uvāca} here.
 Note \textit{koṭyāyutam} in \textit{pada} b instead of the expected but unmetrical \textit{koṭyayutam}. 
  Cf.\ 18.4b below.
 }}

  \maintext{jāyanti divyeṣu kuleṣu puṃsaḥ}%

 \nonanustubhindent \maintext{sastrīsamṛddhe bahubhṛtyapūrṇe |}%

  \maintext{gauraśvaratnādidhanākuleṣu}%

 \nonanustubhindent \maintext{rūpojjvalaḥ kāntisamāyutaś ca  }||\thinspace18:3\thinspace||%
\translation{[These] men will be [re-]born in divine families, [later] having a wife and wealth and many servants, into families that are stuffed with wealth that consists of cows, horses, jewels etc., he himself possessing shining beauty and loveliness. \blankfootnote{18.3 Note the change from plural {\rm (}\textit{kuleṣu}{\rm )} to singular {\rm (}°\textit{samṛddhe}, °\textit{pūrṇe}{\rm )}
  in \textit{pāda}s a and b, the slightly irregular plural nominative \textit{puṃsaḥ} in \textit{pāda} a,
  and that \textit{sastrī} might have been meant as a separate word, in the sense of
  \textit{sastrīkaḥ} {\rm (}`married'{\rm )}.
 I take \textit{gaur aśva}° in \textit{pāda} c as if it were part of the compound:
  \textit{go'śva}°. See \NARADAP\ 1.71.77ab for a compound similar to the one here:
  \textit{gajāśvaratharatnaiś ca grāmakṣetradhanādibhiḥ}
 }}

  \maintext{vastraṃ susatkṛtya dvijasya dānāt}%

 \nonanustubhindent \maintext{svargeṣu modanti sa varṣakoṭyaḥ |}%

  \maintext{punaś ca te martyam upāgatāś ca}%

 \nonanustubhindent \maintext{cihnaṃ mahacchrīpadam āpnuvanti }||\thinspace18:4\thinspace||%
\translation{[If one] donates clothes to a Brahmin with utmost respect, he will have fun in the heavens for millions of years. They will return to the world of mortals, and their characteristics mark is that they rise to an extremely glorious rank. \blankfootnote{18.4 Note that \textit{pāda} a can be considered metrical only if 
  \textit{dvi} in \textit{dvijasya} does not make the previous syllable heavy 
  {\rm (}\mutacumliquida{\rm )}.
 Note the plural verb \textit{modanti} metri causa and the plural nominative °\textit{koṭyaḥ} in \textit{pāda} b for
  a more standard accusative °\textit{koṭīḥ} {\rm (}from \textit{koṭi} or \textit{koṭī}; cf.\ 18.2b above{\rm )}.
 }}

  \maintext{kūpaprapāpuṣkariṇīpradātā}%

 \nonanustubhindent \maintext{sa lokam āpnoti jaleśvarasya |}%

  \maintext{tataḥ sa tasmāc cyutim āpya lokāt}%

 \nonanustubhindent \maintext{sukhī sutṛpteṣu kuleṣu jāyet }||\thinspace18:5\thinspace||%
\translation{He who donates wells, fountains, or lotus-ponds will reach the world of Jaleśvara [i.e.\ Varuṇa]. Then descending from that world, he will be [re-]born into a very comfortable [well-to-do? happy? tarpaṇa?] family, and will be happy. \blankfootnote{18.5 The phrase \textit{sutṛpteṣu kuleṣu} {\rm (}lit.\ `into satified families'{\rm )} is 
  slightly odd.
 }}

  \maintext{ratnipramāṇād api hemadānāt}%

 \nonanustubhindent \maintext{surendralokaṃ samavāpnuvanti |}%

  \maintext{tasmāc cyuto martyam upāgatānāṃ}%

 \nonanustubhindent \maintext{cihnaṃ samṛddhir dhanadhānyalakṣmyāḥ }||\thinspace18:6\thinspace||%
\translation{By donating as little gold as a cubit[?], people can reach the world of Surendra [i.e.\ Indra]. The characteristic marks of those who descend from there to the world of mortals is prosperity, wealth, crops, and good fortune. \blankfootnote{18.6 \textit{ratni} seems to be too large a measure in this context. Maybe \textit{reṇu} {\rm (}`a grain of dust'{\rm )} was meant?
 I have chosen \textit{lakṣmyāḥ} in \textit{pāda} d against \textit{lakṣyāḥ} as
  a lectio difficilior. It is supposed to stand for a 
  plural nominative.
 }}

  \maintext{adūṣyabhūmīvaravipradānāt}%

 \nonanustubhindent \maintext{sa lokam āpnoti sureśvarasya |}%

  \maintext{bhuktvā tu bhogān cyuta martyaloke}%

 \nonanustubhindent \maintext{cihnaṃ labhed vai viṣayādhipatvam }||\thinspace18:7\thinspace||%
\translation{By donating an excellent piece of land to a Brahmin without corruption[? adūṣita° ?], one will reach the world of Sureśvara [Śiva/Brahmā?]. After experiencing enjoyments, he descends to the world of mortals, And the characteristic mark [will be] that he will obtain the rank of `lord of the land.' \blankfootnote{18.7 Note the stem form \textit{cyuta} in \textit{pāda} c metri causa.
 }}

  \maintext{dvijasya satkṛtya tilapradātā}%

 \nonanustubhindent \maintext{sa lokam āpnoti ca keśavasya |}%

  \maintext{bhraṣṭas tato martyam upāgatas tu}%

 \nonanustubhindent \maintext{cihnaṃ labhed akṣayam arthalābham }||\thinspace18:8\thinspace||%
\translation{He who donates sesame seeds to a Brahmin respectfully will reach the world of Keśava [i.e.\ Viṣṇu]. Then, having fallen and returned to the world of mortals, the characteristic mark [will be] that he will obtain undiminishing acquisition of wealth. }

  \maintext{gavāṃ surūpāṃ vidhivad dvijānāṃ}%

 \nonanustubhindent \maintext{dattvā ca golokam avāpnuvanti |}%

  \maintext{kalpāvasāne samupetya martye}%

 \nonanustubhindent \maintext{cihnaṃ gavāḍhyaṃ śatagoyutaṃ ca }||\thinspace18:9\thinspace||%
\translation{By donating beautiful cows to Brahmins according to rule, people reach Goloka. At the end of the \ae on, they return to the world of mortals. Their characteristic mark will be an abundance of cows, having a hundred cows[?]. \blankfootnote{18.9 It seems that \textit{gavāṃ} is meant to be a singular accusative of \textit{go}.
 }}

  \maintext{svargaṃ gatānāṃ puruṣasya cihnaṃ}%

 \nonanustubhindent \maintext{dhanāḍhyatā śrī sukhabhogalābham |}%

  \maintext{āyuryaśorūpakalatraputraṃ}%

 \nonanustubhindent \maintext{sampadvibhūtikulakīrtim artham }||\thinspace18:10\thinspace||%
\translation{The characteristic marks of those who have been in heaven are: an abundance of wealth, grace, the attainment of happiness and enjoyment, [a long] life, fame, beauty, wife, sons, success, power, family, glory, and riches. \blankfootnote{18.10 Note the discrepancy in grammatical number in \textit{pāda} a.
 Note the seemingly accusative forms °\textit{lābham} and °\textit{kīrtim} {\rm (}for \textit{lābhaḥ}
  and \textit{kīrtir}{\rm )}. The last syllable of \textit{vibhūti} is treated as long.
 }}

  \subchptr{nirayān martyam upāgatānāṃ cihnāni}%

  \trsubchptr{Marks of those who return from hell}%

  \maintext{dānāṣṭakaṃ cottama kīrtitaṃ te}%

 \nonanustubhindent \maintext{cihnaṃ ca lokaṃ ca samāsato me |}%

  \maintext{śṛṇotu devī nirayāgatānāṃ}%

 \nonanustubhindent \maintext{cihnaṃ ca karmaṃ ca vipākatāṃ ca }||\thinspace18:11\thinspace||%
\translation{I have taught you the eight supreme kinds of donation, the characteristic marks, and the [corresponding] worlds in brief. Listen, O Goddess, to the characteristic marks of those who have returned from hell, and to their actions and the fruition [thereof]. \blankfootnote{18.11 For a similar description of the consequences of sins in next lives, see
  \Manu\ 11 and 12, and \YAJNS\ 3.207ff {\rm (}in \mycite{YajnavalkyaOlivelle}{\rm )}, 5 {\rm (}\textit{prāyaścittaprakaraṇa}{\rm )}.
  Note the stem form adjective \textit{uttama} metri causa, in \textit{pāda} a.
 Note \textit{me} for \textit{mayā} in \textit{pāda} b {\rm (}\mycitep{OberliesEpicSkt}{4.1.3 [pp. 102--103]}{\rm )}.
 The slightly odd phrase \textit{śṛṇotu devī}, instead of a vocative with \textit{śṛṇu},
  is metri causa.
 Note the accusative form \textit{karmaṃ}, metri causa, in \textit{pāda} d.
 }}

  \maintext{hatvā ca vipraṃ manasā ca vācā}%

 \nonanustubhindent \maintext{sa yāti pāraṃ nirayasya ghoram |}%

  \maintext{aśītikalpaṃ niraye krameṇa}%

 \nonanustubhindent \maintext{bhuktvā punas tirya śatāyutānām }||\thinspace18:12\thinspace||%
\translation{If one kills a Brahmin, [even if only] mentally or verbally, one goes to the boundaries of terrible hell. Gradually experiencing [his karmas] for eighty \ae on in hell, he will [live] as an animal for millions [of years/lives/\ae ons]. \blankfootnote{18.12 Note the stem form \textit{tirya} in \textit{pāda} d {\rm (}metri causa{\rm )},
  and that the phrase \textit{śatāyutānām} is ambiguous. Perhaps
  \textit{śatāyutābdam} {\rm (}for \textit{śatāyutāny abdāni}{\rm )} or \textit{śatāyutāni janmāni}
  was meant.
 }}

  \maintext{jāyanti te mānuṣa hīnavidyāḥ}%

 \nonanustubhindent \maintext{pratyantavāsāḥ kulavittahīnāḥ |}%

  \maintext{nityaṃ ca tasyā kṣayarogapīḍā}%

 \nonanustubhindent \maintext{idaṃ tu cihnaṃ dvijajīvahartuḥ }||\thinspace18:13\thinspace||%
\translation{Those men will be [re-]born as ignorant, will live on the fringes of town, and will lack a good family and wealth. They will always be tormented by consumption {\rm (}\textit{kṣayaroga}{\rm )}. These are the characteristic marks of one who takes away the life of a Brahmin. \blankfootnote{18.13 In \textit{pāda} a, I take \textit{mānuṣa} as a stem form noun {\rm (}metri causa, for {\rm (}\textit{mānuṣā}[\textit{ḥ}]{\rm )}.
 While \textit{pāda} c suggests `incurable diseases' {\rm (}\textit{tasya} +
  \textit{akṣayaroga}°{\rm )}, the parallels {\rm (}reading \textit{brahmā kṣaya}°{\rm )} are traditionally
  interpreted as reading \textit{kṣaya}° {\rm (}`consumption', see the apparatus{\rm )} as
  opposed to \textit{akṣaya}° Thus the lengthening of the final vowel of 
  \textit{tasya} is best to be taken as metri causa.
 }}

  \maintext{pītvā ca madyaṃ dvija kāmato vā}%

 \nonanustubhindent \maintext{āghrāti gandhaṃ svamanīṣikeṇa |}%

  \maintext{sa yāti ghoraṃ narakam asahyaṃ}%

 \nonanustubhindent \maintext{yāvac ca kalpaṃ daśa atra bhuktvā }||\thinspace18:14\thinspace||%
\translation{If a Brahmin {\rm (}\textit{dvija}{\rm )} drinks alcohol no doubt intentionally, smells [its] odour on his own accord, he will go to the terrible and unbearable hell for ten \ae ons, experiencing [his karmas] there. \blankfootnote{18.14 I take \textit{dvija} in \textit{pāda} a as a stem form noun {\rm (}for \textit{dvijaḥ}{\rm )}.
  If standard \textit{sandhi} is expected between \textit{pāda}s a and b, then \textit{vā} in \textit{pāda} a
  is to be understood to stand for \textit{vai} {\rm (}`definitely, without a doubt'{\rm )}.
 In \textit{pāda} b, °\textit{manīṣikena} stands for the better attested °\textit{manīṣikayā}.
 Strictly speaking, \textit{pāda} c is unmetrical, the last
  syllable of \textit{narakam} ending in a light syllable.
  Word-ending syllables are often treated as heavy in this text.
 In \textit{pāda} d \textit{atra} probably stands for \textit{tatra {\rm (}narake{\rm )}}. 
  It is not clear why \textit{atra} seemed better to the redactors than \textit{tatra}. Note also
  that the use of the singular with numerals is one of the hallmarks of this text.
 }}

  \maintext{tiryaṃ ca sarvam anubhūya duḥkhaṃ}%

 \nonanustubhindent \maintext{sa kaṣṭakaṣṭena manuṣyajanma |}%

  \maintext{caṇḍālaśaunaśvapacatvam eti}%

 \nonanustubhindent \maintext{śyāmaṃ ca tālu bhavatīha cihnam }||\thinspace18:15\thinspace||%
\translation{Experiencing all the pain of animal existence, he will, with great difficulty, [reach] a human birth. He will go through [states of being] a Caṇḍāla, a butcher, and a dog-cooker. In this case, the characteristic mark is that his palate becomes black. \blankfootnote{18.15 The syntax of \textit{pāda} a is obscure. Either understand \textit{tiryaṃ} as \textit{tiryaś} {\rm (}`being an animal,'
  `in an animal form'{\rm )} or \textit{tiryaṃ} as qualifying \textit{duḥkhaṃ} {\rm (}`the pain of animal existence'{\rm )}.
  The last syllable of \textit{sarvam} in \textit{pāda} a is treated as long.
 One may consider emending °\textit{janma} to °\textit{janmā} in \textit{pāda} b, turning it into a \textit{bahuvrīhi} compound,
  to make it agree with \textit{sa} {\rm (}`he [will become] a human with great difficulty'{\rm )}.
 The two syllables of \textit{tālu} scan as long-long. Note the relevant
  remark in \YAJNS\ 3.210b quoted in the apparatus.
 }}

  \maintext{nindanti ye veda {\rm †}sambhūya{\rm †} jihvā}%

 \nonanustubhindent \maintext{yaḥ kūṭasākṣī sa ca khalv alāndhau |}%

  \maintext{suhṛdvadhā mṛtyuśataṃ hi garbhe}%

 \nonanustubhindent \maintext{garhāśanocchiṣṭabhujo bhavanti }||\thinspace18:16\thinspace||%
\translation{Those who despise the Vedas will [be reborn] with their tongues ... He who gives false testimony will [be reborn] blind[? CHECK]. [In case of] the murder of a friend, [one will experience] a hundred deaths in the womb. Those who eat forbidden food will eat [only] leftovers [in their next lives]. \blankfootnote{18.16 I take \textit{veda} as a stem form noun in \textit{pāda} a. 
  I suspect that \textit{pāda} a may have contained a reference to
  \textit{upajihvā}, a disease of the tongue. Alternatively, it may speak about one's
  tongue being cut with a sword, or about a swallen {\rm (}\textit{saṃśūna}{\rm )} tongue.
 Understans \textit{garhāśanocchiṣṭabhujo} in \textit{pāda} d as \textit{garhitāśanā 
  ucchiṣṭabhujo} with double sandhi.
 }}

  \maintext{stainyaṃ tu yaḥ kurvati pāpasattvaṃ}%

 \nonanustubhindent \maintext{te pāpadoṣān narakaṃ vrajanti |}%

  \maintext{manvantarādīny anubhūya duḥkhaṃ}%

 \nonanustubhindent \maintext{punaś ca tiryaṃ śataśo 'nubhūyāt }||\thinspace18:17\thinspace||%
\translation{Those wicked people who steal will, because of this sinful crime, go to hell. Suffering pain for many[?] a Manu-era, one will again and again, for hundreds of times, experience animal existence. \blankfootnote{18.17 Note the discrepancy between \textit{yaḥ kurvati} and \textit{te vrajanti} in
  \textit{pāda}s a and b, and the corresponding attempt in \msCa\ to correct \textit{yaḥ} to
  \textit{ye}. One could also emend °\textit{sattvaṃ} to °\textit{sattvaḥ}.
 Note how \Ed\ echoes the reading of \msPaperA\ in \textit{pāda} d.
 }}

  \maintext{mānuṣyajanmeṣu ca duḥkhabhāgī}%

 \nonanustubhindent \maintext{stenatvam āyāti punaś ca mūḍhaḥ |}%

  \maintext{suvarṇacorī kunakhatva cihnaṃ}%

 \nonanustubhindent \maintext{viśīrṇagātro rajatāpahārī }||\thinspace18:18\thinspace||%
\translation{When born as a human, he will suffer, the fool will become a thief again. If one steals gold, the characteristic mark will be that one will have ugly nails. One who steals silver will have broken limbs. \blankfootnote{18.18 \textit{kunakhatva} in \textit{pāda} c is in stem form.
 }}

  \maintext{tāmrāpahārī sphuṭitāgrapāṇir}%

 \nonanustubhindent \maintext{lohāpahārī bhujacheda cihnam |}%

  \maintext{kāṃsāpahārī karabhagna cihnaṃ}%

 \nonanustubhindent \maintext{hṛtvā ca rītitrapusīsakānām }||\thinspace18:19\thinspace||%
\translation{If one steals copper, the fore part of one's hand will be split. If one steals steel, the characteristic mark will be a broken arm. If one steals brass, the characteristic mark will be a broken hand. Stealing bell-metal, tin or lead \blankfootnote{18.19 Note the stem forms °\textit{cheda} and °\textit{bhagna} in \textit{pāda}s b and c.
 Note \textit{kāṃsa} as an alternative form of \textit{kāṃsya}, and °\textit{bhagna} as a stem form in \textit{pāda} c.
 }}

  \maintext{nāsoṣṭhakarṇaśravaṇasya chedaś}%

 \nonanustubhindent \maintext{cihnaṃ nṛṇāṃ vastraharaḥ kucailaḥ |}%

  \maintext{dhānyāpahārī bhavate 'ṅgahīno}%

 \nonanustubhindent \maintext{dīpāpahārī bhavate 'ndha cihnam }||\thinspace18:20\thinspace||%
\translation{will cause clefts in the nose, lips, and the ears {\rm (}\textit{karṇaśravaṇa}{\rm )}. The characteristic mark of one who stole people's clothes is being badly-dressed. Those who steal grain will have missing limbs. If one steals lamps, the characteristic mark is that he will become blind. \blankfootnote{18.20 I take °\textit{karṇaśravaṇa} as a clumsy expression simply meaning `ear.'
 Note the stem form metri causa in \textit{pāda} d {\rm (}\textit{andha}{\rm )}.
 }}

  \maintext{nirvāpahā kāṇa bhaveta cihnaṃ}%

 \nonanustubhindent \maintext{yaḥ strīṃ haret so 'pi jitaḥ striyā syāt |}%

  \maintext{sasyāpahārī bhavate 'nnahīno}%

 \nonanustubhindent \maintext{hṛtvāyudham astrahatatva cihnam }||\thinspace18:21\thinspace||%
\translation{The characteristic mark of one who takes away sacrifical offerings {\rm (}or: alms{\rm )} is becoming one-eyed. He who abducts women will himself be overcome by a woman. Somebody who steals corn will lack food. If one steals weapons, the characteristic mark is death by a missile. }

  \maintext{annāpahārī paradattabhoktā}%

 \nonanustubhindent \maintext{hṛtvā tu gāvaḥ sa bhaved daridraḥ |}%

  \maintext{hariṃ haret tad dhariṇā dahanti}%

 \nonanustubhindent \maintext{hṛtvā tu meṣān ajagardabhaṃ vā }||\thinspace18:22\thinspace||%
\translation{One who steals food will live on [food] given by others. One who steals cows will become poor. [If] someone steals a horse, then he will be destroyed by a horse. One who steals sheep, goats, donkeys, \blankfootnote{18.22 Understand \textit{gāvaḥ} in \textit{pāda} c as plural accusative {\rm (}for \textit{gāḥ}{\rm )}.
  {\rm (}\mycitep{OberliesEpicSkt}{2.15 [p.~68]}{\rm )}.
 }}

  \maintext{sa bhārabhṛjjīvya{-}m{-}udāharanti}%

 \nonanustubhindent \maintext{ratnāpahārī anapatyatā ca |}%

  \maintext{chatrāpahārī apavitratā ca}%

 \nonanustubhindent \maintext{hṛtvā ca bījaṃ sa bhaved abījaḥ }||\thinspace18:23\thinspace||%
\translation{will lead[?] a burdened life, they say[?]. One who steals jewels: [the mark is] childlessness. One who steals parasols: [the mark is] impurity. Stealing seeds, one becomes seedless. }

  \maintext{godhūmaśāliyavamudgamāṣān}%

 \nonanustubhindent \maintext{hṛtvā masūraṃ vilayaṃ vrajanti |}%

  \maintext{kāmāturo mātara mātṛputrīṃ}%

 \nonanustubhindent \maintext{mātṛsvasāṃ gacchati mātulānīm }||\thinspace18:24\thinspace||%
\translation{Those who steal wheat, rice, barley, mungo beans, wild beans, or lentils, will die. If somebody, being sick with desire, sexually approaches his mother, his mother's daughter, his mother's sister, or the wife of a maternal uncle, \blankfootnote{18.24 Note \textit{mātara} for \textit{mātaraṃ} metri causa.
 }}

  \maintext{rājāṅganāṃ putrasutāṃ snuṣāṃ ca}%

 \nonanustubhindent \maintext{pravrājinīṃ brāhmaṇim antyajāṃ ca | }%

  \maintext{ajāśvameṣaṃ surabhīsutāṃ ca}%

 \nonanustubhindent \maintext{yat kāmayet teṣu vimūḍhacetāḥ }||\thinspace18:25\thinspace||%
\translation{or if he has sex with a royal consort, his son's daughter, a daughter-in-law, a female religious mendicant, a Brahmin's wife, or a low-born woman, a goat, horse, sheep, or a cow, with a foolish mind, \blankfootnote{18.25 Note the form \textit{brāhmaṇim} in \textit{pāda} b metri causa.
 Note \textit{yat} in \textit{pāda} d most probably simply for \textit{yaḥ}.
 }}

  \maintext{sa yāti kṛcchraṃ narakaṃ sughoraṃ}%

 \nonanustubhindent \maintext{sa varṣakoṭīśataśo bhramitvā |}%

  \maintext{tiryaṃ ca bhūyaḥ śataśo vyatītya}%

 \nonanustubhindent \maintext{kaṣṭena vai jāyati mānuṣatvam }||\thinspace18:26\thinspace||%
\translation{he will go to the painful and extremely terrible hell. Wandering [through transmigration] a million times, dying as an animal again and again a hundred times, he will, with great difficulty, be born into a human existence, }

  \maintext{hīnāṅgatāṃ dīnaśarīratāṃ ca}%

 \nonanustubhindent \maintext{yo mātṛgāmī sa bhaved aliṅgaḥ |}%

  \maintext{mātṛsvasātalpaga vātaliṅgo}%

 \nonanustubhindent \maintext{liṅgoparodhaḥ sutaputrikāmaḥ }||\thinspace18:27\thinspace||%
\translation{and lacking some limbs and having a miserable body. He who had sex with his mother will have no penis; one who has sex with his mother's sister will have a damaged penis; he who enjoys his son's daughter will have a non-functional[?] penis. \blankfootnote{18.27 The accusatives in \textit{pāda} a seem to continue the syntax of 18.26d. Alternatively,
  they could be corrected to nominatives, suggesting further consequences of the
  previously mentioned crime.
 I have translated \textit{vātaliṅgo} in \textit{pāda} c as `a damaged penis', although in medical literature
  \textit{vātaliṅga} usually means `the symptoms of the problems with the Vāta-type or
  with the Vāta-humour of the body,' see, e.g. \Caraka\ 6.7.35.
  The former may have been the result of misreading medical texts, the latter
  seems too mild a punishment for such a crime.
 While all consulted witnesses read °\textit{āparodhaḥ} in \textit{pāda} d, °\textit{oparodhaḥ} seems a
  much better choice; hence my emendation. Note \textit{putri} in the same \textit{pāda} metri causa for \textit{putrī}.
 }}

  \maintext{snuṣāṃ ca yaḥ sevati raktamehī}%

 \nonanustubhindent \maintext{dauścarmatāṃ ca dvijasundarīṣu |}%

  \maintext{rājāṅganāyāsu ca liṅgacchedaḥ}%

 \nonanustubhindent \maintext{pravrājinīkāmuka mūtrakṛcchram }||\thinspace18:28\thinspace||%
\translation{He who has sex with a daughter-in-law will pass blood with his urine; with a Brahmin's wife: skin disease; with royal consorts: a cut-off penis; having sex with a female mendicant: painful discharge of urine; \blankfootnote{18.28 Understand °\textit{āṅganāyāsu} as °\textit{āṅganāsu}.
 Note the stem form °\textit{kāmuka} in \textit{pāda} d.
 }}

  \maintext{savyādhiliṅgaṃ labhate 'ntyajāsu}%

 \nonanustubhindent \maintext{vilīnaliṅgaḥ paśuyonigāmī |}%

  \maintext{jāyanti te mūṣika dhānyacaurī}%

 \nonanustubhindent \maintext{kṣīraṃ hared vāyasatāṃ prayāti }||\thinspace18:29\thinspace||%
\translation{[sex] with low-born women: he will have a penis disease; he who has sex with animals will have a dissolved penis. He who steals grain will be born as a rat. If one steals milk, one will become a crow. \blankfootnote{18.29 Note the singular subject with a plural predicate, and the stem form noun \textit{mūṣika} in \textit{pāda} c.
 For this and the next verse, compare \MANU\ 12.62 {\rm (}in Olivelle's translation;
  see the Sanskrit in the apparatus; emphasis mine{\rm )}:
  `\textit{By stealing grain, one becomes a rat;} by stealing bronze, a
  ruddy goose; by stealing water, a Plava coot; by stealing honey, a gnat; 
  \textit{by stealing milk, a crow; by stealing sweets, a dog; by stealing ghee, a
  mongoose'.}
 }}

  \maintext{kāṃsāpahārī sa bhavet tu haṃsaḥ}%

 \nonanustubhindent \maintext{śvānatvam āyāti rasāpahārī |}%

  \maintext{hṛtvā ca sūcīṃ tu bhavet sa daṃśaḥ}%

 \nonanustubhindent \maintext{hṛtvā tu sarpir vṛkatāṃ prayāti }||\thinspace18:30\thinspace||%
\translation{He who steals copper will become a goose. He who steals sweet juices will become a dog. By stealing a needle he becomes a gnat. By stealing ghee, he becomes a wolf. \blankfootnote{18.30 For my emendation of \textit{haṃsāpahārī} to \textit{kāṃsāpahārī} in \textit{pāda} a,
  see \MANU\ 12.62b: \textit{kāṃsyaṃ haṃso} and 18.19c above. Note how \msM\ is closer to
  to reading \textit{kāṃsā}° than any of the other witnesses.
  Since \textit{nihaṃsaḥ} in the same \textit{pāda} is difficult to interpret,
  and we expect \textit{haṃsaḥ} anyway, I conjectured \textit{tu haṃsaḥ} here.
 }}

  \maintext{māṃsaṃ tu hṛtvā sa bhaveta gṛdhras}%

 \nonanustubhindent \maintext{tailāpahārī khagatāṃ prayāti |}%

  \maintext{guḍaṃ ca hṛtvā guḍikā bhavanti}%

 \nonanustubhindent \maintext{śākāpahārī sa bhaven mayūraḥ }||\thinspace18:31\thinspace||%
\translation{If he steals meat, he will be a vulture. If he steals oil, he will be a bird. If he steals sugar, he will become a flying fox. If he steals vegetables, he will become a peacock. \blankfootnote{18.31 For this verse, see \MANU\ 12.63--65 {\rm (}in Olivelle's translation;
  see the relevant excerpts from the Sanskrit in the apparatus; these in italics here{\rm )}:
  `\textit{by stealing meat, a vulture;} by stealing fat, a Madgu cormorant;
  \textit{by stealing oil, a cockroach;} by stealing salt, a cricket; by stealing curd, a Balāka
  flamingo; by stealing silk, a partridge; by stealing linen, a frog; by stealing cotton
  cloth, a Krauñca crane; by stealing a cow, a monitor lizard; 
  \textit{by stealing molasses, a flying fox;} by stealing fine perfumes, a muskrat; 
  \textit{by stealing leafy vegetables, a peacock;}
  by stealing various kinds of cooked food, a porcupine; by stealing uncooked
  food, a hedgehog'.
 Here in \textit{pāda} c, based on \MANU\ 12.64d {\rm (}\textit{... vāggudo guḍam}{\rm )}, 
  what is expected is \textit{guḍaṃ hṛtvā vāggudā bhavanti}, and \textit{guḍikā} {\rm (}`a ball'{\rm )}
  is out of context. I translate what the original intention may have been.
 }}

  \maintext{hṛtvā paśuṃ paṅgura jāyate ha}%

 \nonanustubhindent \maintext{śvitratvam āyāti suvastrahārī |}%

  \maintext{hṛtvā dukūlaṃ sa ca sārasatvaṃ}%

 \nonanustubhindent \maintext{kṣaumaṃ ca hṛtvā sa ca darduratvam }||\thinspace18:32\thinspace||%
\translation{If someone steals cattle, he will be [re-]born lame. One who steals nice clothes will have white leprosy. If one steals fine cloth {\rm (}\textit{dukūla}{\rm )}, one becomes a crane. If one steals linen, one will become a frog. \blankfootnote{18.32 Note \textit{paṅgura} in \textit{pāda} a in stem form, standing for \textit{paṅgula} or \textit{paṅguka} {\rm (}see \msM{\rm )}.
 }}

  \maintext{aurṇāni vastrāṇy apahṛtya meṣaḥ}%

 \nonanustubhindent \maintext{chucchundarī jāyati gandhahārī |}%

  \maintext{brahmasvam alpam api hṛtya bhoktā}%

 \nonanustubhindent \maintext{sa gṛdhra ucchiṣṭabhujo bhavanti }||\thinspace18:33\thinspace||%
\translation{If one steals woolen clothes, one will become a ram. One who steals perfumes will be [re-]born as a [female?] musk-rat. If one steals and enjoys the property of a Brahmin, even if it is only a small amount, one becomes a vulture that eats leftovers. \blankfootnote{18.33 Note how the second syllable of \textit{alpam} is counted as heavy in \textit{pāda} c, and that
  \textit{api} is Törzsök's emendation.
 Note the discrepancy in the use of the singular and the plural in \textit{pāda} d.
 }}

  \maintext{pādena yaḥ sparśayate dvijāṅghriṃ}%

 \nonanustubhindent \maintext{tad vātaraktaṃ caraṇe bhaveta |}%

  \maintext{pādena yaḥ sparśayate ca gāvaḥ}%

 \nonanustubhindent \maintext{sa pādarogān vividhān labheta }||\thinspace18:34\thinspace||%
\translation{He who touches the feet of a Brahmin with his foot, one will have rheumatism in his feet. He who touches a cow with his feet, will have various kinds of foot-diseases. \blankfootnote{18.34 Note the use \textit{tad}, probably for \textit{sa} or possibly for \textit{tadā}, in \textit{pāda} b.
 Understand \textit{gāvaḥ} in \textit{pāda} c as plural accusative {\rm (}for \textit{gāḥ}{\rm )}.
  {\rm (}\mycitep{OberliesEpicSkt}{2.15 [p.~68]}{\rm )}.
 }}

  \maintext{yo mātaraṃ tāḍayate padena}%

 \nonanustubhindent \maintext{pāde tadīye kṛmayaḥ patanti |}%

  \maintext{padā spṛśed yaḥ pitaraṃ durātmā}%

 \nonanustubhindent \maintext{śūnonnapādaḥ sa bhavet paratra }||\thinspace18:35\thinspace||%
\translation{If someone kicks his mother with his foot, worms will settle in his feet. If a wicked person touches his father with his foot, his feet will be swollen and moist in a future life {\rm (}\textit{paratra}{\rm )}. }

  \maintext{padā spṛśet toyam anādareṇa}%

 \nonanustubhindent \maintext{sa ślīpadī pādayuge bhaveta |}%

  \maintext{pādena yaḥ sparśayate hutāśaṃ}%

 \nonanustubhindent \maintext{sa cāgnipādaḥ satataṃ bhaveta }||\thinspace18:36\thinspace||%
\translation{He who touches water with his foot without paying respect will have elephantiasis in both feet. If someone touches fire with his foot, will always remain `fire-footed.' \blankfootnote{18.36 It is not clear to me what `fire-footedness' means.
 }}

  \maintext{pādena yaś cāryam upaspṛśeta}%

 \nonanustubhindent \maintext{sa pādachedaṃ bahuśo labheta |}%

  \maintext{granthāpahārī sa bhaveta mūkaḥ}%

 \nonanustubhindent \maintext{durgandhavaktraḥ parachidravādī }||\thinspace18:37\thinspace||%
\translation{He who touches his teacher with his foot, will break his foot many times. He who steals a book will become mute. He who talks about others' faults will have a stinking mouth. \blankfootnote{18.37 \textit{cāryam} in \textit{pāda} a most probably stands for \textit{cācāryam}. Originally,
  there may have been double sandhi thus: \textit{ya ācāryam} $\rightarrow$\ \textit{yācāryam} 
  {\rm (}as suggested by Judit Törzsök{\rm )}.
 }}

  \maintext{paiśunyavādī sa ca pūtināso}%

 \nonanustubhindent \maintext{nṛ namravaktras tv anṛtāpavādī |}%

  \maintext{pāruṣyavaktā mukhapākarogī}%

 \nonanustubhindent \maintext{asatpralāpī sa ca dantarogaḥ }||\thinspace18:38\thinspace||%
\translation{He who speaks slanderously will have a fetid nose. If a man lies, he will have a curved/disfigured mouth. He who speaks abusively will have a mouth ill with inflammation. He who spreads false rumours CHECK will have tooth-aches. \blankfootnote{18.38 \textit{nṛ} in \textit{pāda} b as a stem form noun is odd, but none of the variant readings
  yield any better sense.
 }}

  \maintext{tīkṣṇapradāyī sa ca vakranāsaḥ}%

 \nonanustubhindent \maintext{sambhinnavaktā sa ca kaṇṭharogī |}%

  \maintext{kruddhekṣaṇaḥ paśyati yas tu vipraṃ}%

 \nonanustubhindent \maintext{tīvrākṣirogī sa tu jāyate hi }||\thinspace18:39\thinspace||%
\translation{He who is abusive will have a crooked nose. He whose talk is idle will have a sore throat. He who beholds a Brahmin with angry eyes, will be [re-]born with severe eye-diseases. \blankfootnote{18.39 Buddhist terms...
 }}

  \maintext{pradveṣayālokayate 'tithīn ya}%

 \nonanustubhindent \maintext{utpāṭitākṣiḥ sa bhavet paratra |}%

  \maintext{vairūpyacakṣus tv atisūkṣmacakṣuḥ}%

 \nonanustubhindent \maintext{sa jāyate kekarapiṅgacakṣuḥ }||\thinspace18:40\thinspace||%
\translation{He who looks at his guests with hatred will have his eyes pulled out in his next life, will have ugly, tiny eyes, and will be born squint-eyed, yellow-eyed. \blankfootnote{18.40 Note \textit{pradveṣayā}, a feminine instrumental, in \textit{pāda} a, instead of a more standard
  masculine \textit{pradveṣeṇa}.
 }}

  \maintext{gartākṣikādīni vipaṇḍulāni}%

 \nonanustubhindent \maintext{netrāmayāny eva ca pāpadoṣāt |}%

  \maintext{śṛṇvanti ye pāpakathāṃ praśastāṃ}%

 \nonanustubhindent \maintext{tān karṇasarpiḥ paripīḍayeta }||\thinspace18:41\thinspace||%
\translation{Eye-diseases such as hollow-eyedness and pale[-eyedness] [will arise] because of [this] sinful crime. Those who listen to wicked tales approvingly[?] {\rm (}\textit{praśasta}{\rm )} will be tormented by earwax[??!]. }

  \maintext{śṛṇoti nindāṃ hariśarvayor yaḥ}%

 \nonanustubhindent \maintext{sa karṇaśūlena tu jīvatīva |}%

  \maintext{mātāpitṝṇāṃ śṛṇute 'pavādaṃ}%

 \nonanustubhindent \maintext{sa karṇaśophena vināśam eti }||\thinspace18:42\thinspace||%
\translation{He who listens to abuse towards Hari or Śarva [i.e.\ Śiva] will barely live {\rm (}\textit{iva}{\rm )}, because of ear-ache. If he listens to abusive words about his parents, he will perish from an ear-tumour. }

  \maintext{śṛṇoti nindāṃ guruviprajāṃ yaḥ}%

 \nonanustubhindent \maintext{sa karṇapūyaṃ sravate saraktam |}%

  \maintext{virūpadāridryakulādhameṣu}%

 \nonanustubhindent \maintext{aniṣṭakarmabhṛtijīvanaṃ ca |}%

  \maintext{akīrtanaṃ darśanavarjanaṃ ca}%

 \nonanustubhindent \maintext{śvapākaḍombādiṣu jāyate saḥ }||\thinspace18:43\thinspace||%
\translation{If he listens to abuse aimed at the guru or Brahmins, he will ooze puss from his ears mixed with blood. [The marks will be] ugliness, poverty, [birth] in the lowest of families, undiserable/shitty work, repulsive employment?, and livelihood, disgrace, loss of eyesight, and he will be born amongst `dog-cookers,' Ḍombas etc. }

  \maintext{etāni cihnaṃ nirayāgatānāṃ}%

 \nonanustubhindent \maintext{mānuṣyaloke kukṛtasya dṛṣṭam |}%

  \maintext{samāsataḥ kīrtita eva devi}%

 \nonanustubhindent \maintext{yathaiva muktas tv iha karmabhaṅgaḥ }||\thinspace18:44\thinspace||%
\translation{These are the characteristic marks of those sinners who have returned from hell, as seen in the human world. In brief, I have proclaimed, O Devī, how one, when one's karmas are destroyed, is liberated in this world. \blankfootnote{18.44 Note the total disregard for grammatical number in \textit{pāda}s ab.
 The somewhat out-of-context claim in \textit{pāda}s cd may be meant to be the concluding remark on the
  entire Śaiva section in the \VSS\ {\rm (}10.3--18.64{\rm )}.
 }}

  \maintext{mātāpitroghatoyā sutaduhitṛvahā bhrātṛgambhīravegā}%

 \nonanustubhindent \maintext{bhāryāvartā vivartā kuṭilagativadhū bāndhavormītaraṅgā |}%

  \maintext{kāmakrodhobhakūlā karimakarajhaṣāgrāhakāmā bhayante}%

 \nonanustubhindent \maintext{mṛtyor ākhyārṇave 'smin na śaraṇa vivaśā kāladaṣṭā prayānti }||\thinspace18:45\thinspace||%
\translation{The water of its torrents is mother and father; its flow is son and daughter; its underwater currents are brothers; its revolving whirlpools the wife; its winding currents the daughter-in-laws; its rising waves the relatives; its two banks Desire and Anger; the elephants and Makaras and fish and sharks are Desires; in this frightening[?] [world of transmigration] that appears as an ocean, there is no refuge from death, [and people] advance helplessly, bitten by Time {\rm (}\textit{kāla}{\rm )}. \blankfootnote{18.45 Compare the style of this verse to that of 10.33:
  \textit{mīmāṃsāratnakūlā kramapadapulinā śaivaśāstrārthatoyā,
  mīnaughā pañcarātraṃ śrutikuṭilagatiḥ smārtavegā taraṅgā\thinspace |
  yogāvartātiśobhā upaniṣadivahā bhāratāvartaphenā,
  pañcāśadvyomarūpī rasabhavananadī tīrtha vāgīśvarīyam\thinspace ||}
 }}

  \maintext{nityaṃ yena vināśa yāti divasaṃ pañcatvam āpadyate}%

 \nonanustubhindent \maintext{tyaktvā deha vanāntareṣu viṣame śvānaśṛgālākule |}%

  \maintext{bandhuḥ sarva nivartate gatadayā dharmaika tatra sthitas}%

 \nonanustubhindent \maintext{tasmād dharmaparo na cānyasuhṛdaḥ sevet paratrārthinaḥ }||\thinspace18:46\thinspace||%
\translation{[And yet, it is] always [like this: when] the day [of] dissolution {\rm (}\textit{pañcatva}{\rm )} comes by which one perishes {\rm (}\textit{vināśa}[\textit{ṃ}] \textit{yāti}{\rm )}, abandoning the body [of the deceased person] in a forest, in a rough place filled with dogs and jackals, all the relatives turn back home, with their compassion gone, and only Dharma stays. Therefore one [should] cling on to Dharma and should not serve any other friends if one seeks the other world. \blankfootnote{18.46 In \textit{pāda} a, \textit{vināśa} is a stem form noun.
 Note the stem form noun \textit{deha} in \textit{pāda} b and that this \textit{pāda} is metrical only if we read \textit{śrigālākule}.
 In \textit{pāda} c, understand \textit{bandhuḥ sarva} as \textit{bandhavaḥ sarve} and \textit{dharmaika} as \textit{dharma ekas}.
 I translate \textit{paratrāthinaḥ} in \textit{pāda} d as if it read \textit{paratrārthī}.
 }}

\centerline{\maintext{\dbldanda\thinspace iti vṛṣasārasaṃgrahe pūrvakarmavipākacihnāṣṭādaśamo 'dhyāyaḥ\thinspace\dbldanda}}
\translation{Here ends the eighteenth chapter in the Vṛṣasārasaṃgraha called Marks of the Fruition of Previous Karma.}

  \chptr{ekonaviṃśatimo 'dhyāyaḥ}
\fancyhead[CO]{{\footnotesize\textit{Translation of chapter 19}}}%

  \trchptr{Chapter Nineteen}%

  \subchptr{gāvaḥ}%

  \trsubchptr{Cows}%

  \maintext{vigatarāga uvāca |}%

  \maintext{kriyāsūkṣmo mahādharmaḥ karmaṇā kena prāpyate |}%

  \maintext{alpopāyaṃ narārthāya pṛcchāmi kathayasva me }||\thinspace19:1\thinspace||%
\translation{Vigatarāga spoke: By what action can the great Dharma, whose rituals are subtle[?], be reached? I ask for an easy method for mankind, tell me about it. }

  \maintext{anarthayajña uvāca |}%

  \maintext{alpopāyaṃ mahādharmaṃ kathayāmi dvijottama |}%

  \maintext{sukhena labhate svargaṃ karmaṇā yena tac chṛṇu }||\thinspace19:2\thinspace||%
\translation{Anarthayajña spoke: I shall teach you the great Dharma that is the easy method, O Brahmin. Listen to that action by which heaven can be reached easily. }

  \maintext{lokānāṃ mātaro gāvo gobhiḥ sarvaṃ jagad dhṛtam |}%

  \maintext{gomayam amṛtaṃ sarvaṃ jātaṃ sarvaṃ śivecchayā }||\thinspace19:3\thinspace||%
\translation{Cows are the mothers of the worlds. Cows hold all the world. All cow-dung is nectar, all are produced by Śiva's will. }

  \maintext{sarvadevamayā gāvaḥ sarvadevamayo dvijaḥ |}%

  \maintext{sarvadevamayī bhūmiḥ sarvadevamayaḥ śivaḥ }||\thinspace19:4\thinspace||%
\translation{Cows contain all the gods. The Brahmin contains all the gods. Earth contains all the gods. Śiva contains all the gods. }

  \maintext{tasmād gāvaḥ sadā sevyā dharmamokṣārthasiddhidāḥ |}%

  \maintext{paricaryā yathāśaktyā grāsavāsajalādibhiḥ }||\thinspace19:5\thinspace||%
\translation{Therefore cows are always to be served because they give religious duties, liberation, financial gain and success. They should be provided with food, shelter, water, etc., with all one's effort. }

  \maintext{tāḍayen nātivegena vācayen mṛdunācaret |}%

  \maintext{pālayeta ghanāḍhyeṣu bhagnodvigneṣu yatnataḥ }||\thinspace19:6\thinspace||%
\translation{One should not beat them too hard...? Protect them in thick [darkness or when in multitude], in case something is broken or they are startled. }

  \maintext{vyādhivraṇaparikleśa oṣadhopakramaṃ caret |}%

  \maintext{kaṇḍūyanaṃ ca kartavyaṃ yathāsaukhyaṃ bhaved gavām }||\thinspace19:7\thinspace||%
\translation{In case of pain from disease or wound, one should apply remedy using medicine. Rubbing should be done as much as it is pleasurable for cows. }

  \maintext{gavāṃ pradakṣiṇaṃ kṛtvā śraddhābhaktisamanvitaḥ |}%

  \maintext{sāgarāntā mahī sarvā pradakṣiṇīkṛtā bhavet }||\thinspace19:8\thinspace||%
\translation{By circumabulating cows with faith and devotion, the whole Earth up to the oceans gets circulambulated. }

  \maintext{spṛṣṭasaṃsparśanādye ca śraddhayā yadi mānavaḥ |}%

  \maintext{ahorātrakṛtaṃ pāpaṃ naśyate nātra saṃśayaḥ }||\thinspace19:9\thinspace||%
\translation{If a man touches a cow with faith ...... his sins, be them committed at daylight or at night, will disappeat, no doubt. }

  \maintext{lāṅgūlenoddhṛtaṃ toyaṃ mūrdhnā gṛhṇāti yo naraḥ |}%

  \maintext{yāvaj jīvakṛtaṃ pāpaṃ naśyate nātra saṃśayaḥ }||\thinspace19:10\thinspace||%
\translation{He who applies the water that has been dispersed by a [cow's] tail onto his head, will have his sins accumulated throughout his life destroyed, no doubt. }

  \maintext{vidhivat snāpayed gāṃś ca mantrayuktena vāriṇā |}%

  \maintext{tenāmbhasā svayaṃ snātvā sarvapāpakṣayo bhavet }||\thinspace19:11\thinspace||%
\translation{One should bathe the cows as prescribed, using water onto which mantras have been recited. If he himself bathes in the same water, he will have all his sins destroyed. }

  \maintext{vyādhir vighnam alakṣmītvaṃ naśyate sadya eva ca |}%

  \maintext{mṛtāpatyānapatyāś ca snānam eva praśasyate }||\thinspace19:12\thinspace||%
\translation{Diseases, obstructing forces, and bad luck will disappear instantly. Those with dead offspring or without offspring praise this very bath. \blankfootnote{19.12 Understand \textit{praśasyate} in \textit{pāda} d as active and plural {\rm (}\textit{prasaṃsanti}{\rm )}.
 }}

  \maintext{gavāṃ śṛṅgodakaṃ gṛhya mūrdhni yo dhārayen naraḥ |}%

  \maintext{sa sarvatīrthasnānasya phalaṃ prāpnoti mānavaḥ }||\thinspace19:13\thinspace||%
\translation{If a man collects the `horn-water' of cows and applies it on his head, he will receive the fruits of bathing at all the sacred pilgrimage places. \blankfootnote{19.13 Applying `horn water' means sprinkling with water filled into a cow's horn, 
  while reciting the Gāyatrī matra a hundred times. See note to \SDHS\ 10.24
  in \mycite{SDhS10_ed}.
 }}

  \maintext{grāsamuṣṭipradānena goṣu bhaktisamanvitaḥ |}%

  \maintext{agnihotraṃ hutaṃ tena sarvadevāḥ sutarpitāḥ }||\thinspace19:14\thinspace||%
\translation{If somebody gives a handful of food to cows with devotion, by this an Agnihotra is being performed and all the gods become satisfied. }

  \maintext{catvāraḥ stanadhārās tu yas tu mūrdhnā pratīcchati |}%

  \maintext{sa catuḥsāgaraṃ gatvā snānapuṇyaphalaṃ labhet }||\thinspace19:15\thinspace||%
\translation{He who collects on his head the four streams [of milk] from the teats will receive the meritous fruits of visiting the four oceans. }

  \maintext{gavārthaṃ yas tyajet prāṇān gograheṣu dvijottama |}%

  \maintext{kalpakoṭiśataṃ divyaṃ śivaloke mahīyate }||\thinspace19:16\thinspace||%
\translation{He who gives his life for cows during an attempt at stealing them, O greatest of Brahmins will prosper in Śivaloka for millions of years. }

  \maintext{cyutabhagnādisaṃskāraṃ sarvaṃ yaḥ kurute naraḥ |}%

  \maintext{bhāryākoṭiśataṃ dānaṃ yat phalaṃ parikīrtitam  }||\thinspace19:17\thinspace||%
\translation{If a man rears all [the cows] that have missing or broken [limbs] CHECK, will get all the fruits that are said to be produced by donating millions of wives[?], }

  \maintext{tat phalaṃ labhate martyaḥ śivalokaṃ ca gacchati |}%

  \maintext{śivalokaparibhraṣṭaḥ pṛthivyām ekarāḍ bhavet }||\thinspace19:18\thinspace||%
\translation{and will go to Śivaloka. When descended from Śivaloka, he will become a universal monarch on Earth. }

  \maintext{samāsataḥ samākhyātaṃ yathātattvaṃ dvijottama |}%

  \maintext{na śakyaṃ vistarād vaktuṃ gomahābhāgyam uttamam }||\thinspace19:19\thinspace||%
\translation{I have taught [about cows] truly, in brief, O supreme Brahmin. It is impossible to talk about the excellence of cows in more detail. }

  \subchptr{cāturvarṇyam}%

  \maintext{vigatarāga uvāca |}%

  \maintext{devā aṣṭavidhāḥ proktās tiryak pañcavidhaḥ smṛtaḥ |}%

  \maintext{mānuṣam ekam evāhuś cāturvarṇaḥ kathaṃ bhavet }||\thinspace19:20\thinspace||%
\translation{Vigatarāga spoke: The gods are of eight kinds, animals are of five kinds. Mankind is said to be only one single [kind]. How come that there is the system of four social classes {\rm (}\textit{varṇa}{\rm )}? \blankfootnote{19.20 cāturvarṇ[y]aṃ
 }}

  \maintext{anarthayajña uvāca |}%

  \maintext{pūrvakalpasṛjas tv eṣa viṣṇunā prabhaviṣṇunā |}%

  \maintext{ekavarṇo dvijaś cāsīt sarvakalpāgram agrataḥ }||\thinspace19:21\thinspace||%
\translation{Anarthayajña spoke: It [i.e. the system of four social classes] was created by Lord Viṣṇu in the previous \ae on[s].  Before the very beginning of all \ae ons, there was a single class {\rm (}\textit{varṇa}{\rm )} of Brahmins {\rm (}\textit{dvija}{\rm )}. \blankfootnote{19.21 See above \textit{sṛja} for \textit{sṛṣṭa} in XXX.
 }}

  \maintext{sarvavedavido viprāḥ sarvayajñavidas tathā |}%

  \maintext{teṣāṃ viprasahasrāṇāṃ yajñotsāhamano bhavet }||\thinspace19:22\thinspace||%
\translation{The Brahmins {\rm (}\textit{vipra}{\rm )} got to know all the Vedas and all the sacrifices. These thousands of Brahmins {\rm (}\textit{vipra}{\rm )} developed an inclination to make a resolution to perform sacrifices. }

  \maintext{vṛddhaviprasahasrāṇāṃ matam ājñāya brāhmaṇaiḥ |}%

  \maintext{kartuṃ karma samārabdhaṃ karma cāpi vibhajyate }||\thinspace19:23\thinspace||%
\translation{Having understood the intention of the thousands of senior Brahmins {\rm (}\textit{vipra}{\rm )}, the Brahmins {\rm (}\textit{brāhmaṇa}{\rm )} commenced performing rituals {\rm (}\textit{karman}{\rm )} and the tasks {\rm (}\textit{karman}{\rm )} were distributed. }

  \maintext{ṛtvijatve sthitāḥ kecit kecit saṃrakṣaṇe sthitāḥ |}%

  \maintext{arthopārjanayuktānye anye śilpe niyojitāḥ }||\thinspace19:24\thinspace||%
\translation{Some took on the function of being priests {\rm (}\textit{ṛtvij}{\rm )}, some took on the task of protection. Some got engaged in the acquisition of materials and others were appointed to do manual crafts. \blankfootnote{19.24 Note the form \textit{ṛtvijatva}.
 Note the double sandhi in °\textit{yuktānye} {\rm (}\textit{yuktā anye}{\rm )}.
 }}

  \maintext{evaṃ yajñavidhānena kartum ārebhire purā |}%

  \maintext{yathoddiṣṭena karmeṇa yajñotsāha{-}m{-}avartata }||\thinspace19:25\thinspace||%
\translation{This is how they started performing sacrifices in the beginning. With the tasks {\rm (}\textit{karman}{\rm )} thus distributed, the will to perform sacrifices increased. \blankfootnote{19.25 Perhaps understand pāda a as \textit{evaṃvidhānena yajñaṃ kartum}.
 }}

  \maintext{āgatā ṛṣayaḥ sarve devatāḥ pitaras tathā |}%

  \maintext{anyonyam abruvan tatra devarṣipitṛdevatāḥ }||\thinspace19:26\thinspace||%
\translation{Then came all the Ṛṣis and all the gods and the Ancestors. They discussed it among themselves, the divine Ṛṣis, the Ancestors and the gods. }

  \maintext{yajñārtham asṛjad varṇaṃ vidhinā kratuhetavaḥ |}%

  \maintext{evam eva pravartantu bhavadbhir dvijasattamāḥ }||\thinspace19:27\thinspace||%
\translation{Brahmā/Viṣṇu {\rm (}\textit{vidhi}{\rm )} created class for the sake of sacrifice. [The classes are] for the purpose of rituals {\rm (}\textit{kratu}{\rm )}. Proceed in this very manner, Sirs, O excellent twice-born! \blankfootnote{19.27 Note the confused syntax both in pādas ab and cd.
 }}

  \maintext{ijyādhyayanasampannā brāhmaṇā ye 'tra kalpitāḥ |}%

  \maintext{suviprā vipratāṃ yāntu ṣaṭkarmaniratāḥ sadā }||\thinspace19:28\thinspace||%
\translation{Those Brahmins {\rm (}\textit{brāhmaṇa}{\rm )} who are now engaged in sacrifice and recitation, those good Brahmins {\rm (}\textit{suvipra}{\rm )} shall become Brahmins {\rm (}\textit{vipratāṃ yāntu}{\rm )}, always engaged in the six duties [of Brahmins] {\rm (}\textit{ṣaṭkarman}{\rm )}. }

  \maintext{rakṣaṇārthaṃ tu ye viprāḥ kalpitāḥ śastrapāṇayaḥ |}%

  \maintext{kṣatatrāṇāya viprāṇāṃ nityakṣatravratodbhavāḥ }||\thinspace19:29\thinspace||%
\translation{As for those Brahmins {\rm (}\textit{vipra}{\rm )} that have been appointed to protect [the sacrifice] with weapons in their hands, to protect the Brāhmins {\rm (}\textit{vipra}{\rm )} from injury, they shall eternally follow[?] the vow of Kṣatras. }

  \maintext{arthopārjanam uddiśya kalpitā ye dvijātayaḥ |}%

  \maintext{te tu vaiśyatvam āyāntu vārttopāyaratodbhavāḥ }||\thinspace19:30\thinspace||%
\translation{As for those twice-born who have been appointed for the acquisition of materials, they shall become Vaiśyas, involved in the means of trade. }

  \maintext{vadhabandhanakarmasu śilpasthānavidheṣu ca |}%

  \maintext{kalpitā ye dvijātīnāṃ sarve śūdrā bhavantu te }||\thinspace19:31\thinspace||%
\translation{Those of the twice-born who have been appointed to the tasks of slaughering and tying [animals] and of manual skills, they all shall become Śūdras. }

  \maintext{prājāpatyaṃ brāhmaṇānām ijyādhyayanatatparāt |}%

  \maintext{sthānam aindraṃ kṣatriyāṇāṃ prajāpālanatatparāt }||\thinspace19:32\thinspace||%
\translation{The [world] of Prajāpati belongs to the Brahmins {\rm (}\textit{brāhmaṇa}{\rm )} [after death] because they are devoted to the sacrifice and to recitation. The [world] of Indra belongs to the Kṣatriyas because they are devoted to the protection of the people. }

  \maintext{vaiśyānāṃ vāsavasthānaṃ vāṇijyakṛṣijīvinām |}%

  \maintext{śūdrāṇāṃ marutaḥ sthānaṃ śuśrūṣāniratātmanām }||\thinspace19:33\thinspace||%
\translation{The [world] of Vāsus belongs to the Vaiśyas who earn their living by trade and agriculture. The [world] of Marut belongs to the Śūdras who devote themselves to sevice. }

  \maintext{maharṣipitṛdevānāṃ matam ājñāya niścitaḥ |}%

  \maintext{eṣa saṃkalpito brahmā padmayoniḥ pitāmahaḥ }||\thinspace19:34\thinspace||%
\translation{Understanding the intention of the great Ṛṣis, the Ancestors and the gods, lotus-born Brahmā, the Grandfather, it [i.e. the system of \textit{varṇa}s] was established firmly. }

  \maintext{saṃkalpaprabhavāḥ sarve devadānavamānavāḥ |}%

  \maintext{paśupakṣimṛgā mukhyā yāvanti jagasambhavāḥ }||\thinspace19:35\thinspace||%
\translation{All the main domestic animals, birds and wild animals that are born in the world, }

  \maintext{bhūtasaṃkalpakaṃ nāma kalpam āsīd dvijottama |}%

  \maintext{kīrtitāni samāsena kim anyac chrotum icchasi }||\thinspace19:36\thinspace||%
\translation{CHECK .... [The social classes] have been taught briefly. What else do you wish to hear? }

  \maintext{vigatarāga uvāca |}%

  \maintext{kiṃ tapaḥ sarvavarṇānāṃ vṛttiṃ vāpi tapodhana |}%

  \maintext{yajñāṃś caiva pṛthaktvena śrotum icchāmi tattvataḥ }||\thinspace19:37\thinspace||%


  \maintext{anarthayajña uvāca |}%

  \maintext{brāhmaṇasya tapo jñānaṃ tapaḥ kṣatrasya rakṣaṇam |}%

  \maintext{vaiśyasya ca tapo vārttā tapaḥ śūdrasya sevanam }||\thinspace19:38\thinspace||%
\translation{Anarthayajña spoke: The Brahmins's penance is knowledge. The Kṣatriya's penance is protection. The Vaiśya's penance is business {\rm (}\textit{vārttā}{\rm )}. The Śudra's penance is service. }

  \maintext{pratigraha dhanaṃ vipraḥ kṣatriyasya dhanur dhanam |}%

  \maintext{kṛṣir dhanaṃ tathā vaiśyaḥ śūdraḥ śuśrūṣaṇaṃ dhanam }||\thinspace19:39\thinspace||%
\translation{The Brahmin's wealth is the acceptance of gifts. The Kṣatriya's wealth is his bow. The Vaiśya's wealth is agriculture. The Śūdra's wealth is obedience. }

  \maintext{ārambhayajñaḥ kṣatrasya haviryajñā viśas tathā |}%

  \maintext{śūdrāḥ paricarāyajñā japayajñā dvijātayaḥ }||\thinspace19:40\thinspace||%
\translation{Undertaking is the Kṣatriya's worship. Fire-oblation is the Vaiśya's worship. Service is the Śūdra's worship. Recitation is the Brahmin's worship. \blankfootnote{19.40 Compare this verse to 17.36.
 }}

  \maintext{satya tīrthaṃ dvijātīnāṃ raṇa tīrthaṃ tu kṣatriyāḥ |}%

  \maintext{āryā tīrthaṃ tu vaiśyānāṃ śūdratīrthaṃ ca vai dvijāḥ }||\thinspace19:41\thinspace||%
\translation{Truth is the pilgrimage place for Brahmins. A battle is a pilgrimage place for Kṣatriyas. Āryā[varta?] is the pilgrimage place for Vaiśyas. The Brahmins are the pilgrimage place of Śūdras. }

  \maintext{nāsti vidyāsamo mitro nāsti dānasamaḥ sakhā |}%

  \maintext{nāsti jñānasamo bandhur nāsti yajño japasamaḥ }||\thinspace19:42\thinspace||%
\translation{There is no friend comparable to knowledge. There is no companion comparable to donation. There is no relative comparable to knowledge. There is no worship comparable to recitation. }

  \maintext{dharmahīno mṛtais tulyo devatulyo jitendriyaḥ |}%

  \maintext{yajñatulyo 'bhayaṃ dātā śivatulyo manonmanaḥ }||\thinspace19:43\thinspace||%
\translation{A person without Dharma is similar to the dead. Someone who conquers his senses is similar to the gods. One who gives protection is similar to worship. Who [reaches the state of] mind-nonmind {\rm (}\textit{manonmana}{\rm )} becomes similar to Śiva. \blankfootnote{19.43 For a possible synonym of \textit{manonmana}, see 19.48b: \textit{līnamanāḥ} {\rm (}`whose mind has dissolved'{\rm )}.
  See also 11.18b and 49b.
 }}

  \maintext{vigatarāga uvāca |}%

  \maintext{dānaṃ yajñas tapas tīrthaṃ saṃnyāsaṃ yoga eva ca |}%

  \maintext{eteṣu katamaḥ śreṣṭhaḥ śrotum icchāmi kīrtaya }||\thinspace19:44\thinspace||%
\translation{Vigatarāga spoke: Donation, worship, penance, pilgrimages, renunciation, and yoga: which is the best among these? }

  \maintext{anarthayajña uvāca |}%

  \maintext{dānadharmasahasrebhyo yajñayājī viśiṣyate |}%

  \maintext{yajñayājisahasrebhyas tīrthayātrī viśiṣyate }||\thinspace19:45\thinspace||%
\translation{Anarthayajña spoke: He who worships with sacrifices is better than a thousand whose Dharma is donation. He who visits pilgrimages places is better than a thousand who worship with sacrifices. }

  \maintext{tīrthayātrisahasrebhyas taponiṣṭho viśiṣyate |}%

  \maintext{taponiṣṭhasahasrebhyaḥ śreṣṭhaḥ saṃnyāsikaḥ smṛtaḥ }||\thinspace19:46\thinspace||%
\translation{He who is devoted to penance is better than a thousand who visit pilgrimages places. He who is a renunciate is better than a thousand who are devoted to penance. }

  \maintext{saṃnyāsīnāṃ sahasrebhyaḥ śreṣṭho yas tu jitendriyaḥ |}%

  \maintext{jitendriyasahasrebhyo yogayukto viśiṣyate }||\thinspace19:47\thinspace||%
\translation{He who controls his senses is better than a thousand renunciates. He who practises yoga is better than a thousand who control their senses. }

  \maintext{yogayuktasahasrebhyaḥ śreṣṭho līnamanāḥ smṛtaḥ |}%

  \maintext{tasmāt sarvaprayatnena mana ādau viśodhayet }||\thinspace19:48\thinspace||%
\translation{He whose mind has dissolved is better that a thousand who practise yoga. Therefore one should first purify the mind by all means. }

  \maintext{nigṛhītendriyagrāmaḥ svargamokṣau tu sādhayet |}%

  \maintext{visṛṣṭe tv indriyagrāme tiryaknarakasādhanam }||\thinspace19:49\thinspace||%
\translation{He who controls all his senses can reach heaven and liberation. If all the senses are let loose, what one achieves is the hell of animal existence. }

  \maintext{vigatarāga uvāca |}%

  \maintext{carācarāṇāṃ bhūtānāṃ śreṣṭhaḥ katama ucyate |}%

  \maintext{kathayasva mamādya tvaṃ chettum arhasi saṃśayam }||\thinspace19:50\thinspace||%
\translation{Vigatarāga spoke: From amongst the moving and unmoving living beings, which is said to be the best? Tell me this now, please cut my doubts. }

  \maintext{anarthayajña uvāca |}%

  \maintext{carācarāṇāṃ bhūtānāṃ tatra śreṣṭhāś carāḥ smṛtāḥ |}%

  \maintext{carāṇāṃ caiva sarveṣāṃ buddhimān śreṣṭha ucyate }||\thinspace19:51\thinspace||%
\translation{Anarthayajña spoke: From amongst the moving and unmoving living beings, the moving ones are taught to be the best. From amongst the moving ones, the sentient ones {\rm (}\textit{buddhimat}{\rm )} are said to be the best. }

  \maintext{buddhimatsu ca sarveṣu tataḥ śreṣṭhā narāḥ smṛtāḥ |}%

  \maintext{narāṇāṃ caiva sarveṣāṃ brāhmaṇaḥ śreṣṭha ucyate }||\thinspace19:52\thinspace||%
\translation{Among all the sentient ones then humans are the best. The best of all humans is said to be the Brahmin. }

  \maintext{brāhmaṇeṣu ca sarveṣu vidvān śreṣṭhaḥ sa ucyate  |}%

  \maintext{vidvatsv api ca sarveṣu kṛtabuddhir viśiṣyate }||\thinspace19:53\thinspace||%
\translation{Among all the Brahmins, the scholar {\rm (}\textit{vidvat}{\rm )} is the best. From amongst all the scholars, the one who knows his religios duties {\rm (}\textit{kṛtabuddhi}{\rm )} is the best. }

  \maintext{kṛtabuddhiṣu sarveṣu śreṣṭhaḥ kartā samucyate |}%

  \maintext{kartṛṣv api ca sarveṣu brahmavedī viśiṣyate }||\thinspace19:54\thinspace||%
\translation{From amongst all those who know their religios duties, the performer [of rituals] {\rm (}\textit{kartṛ}{\rm )} is the best. And amongst the performers [of rituals] the one who knows the Vedic mantras {\rm (}\textit{brahmavedin}{\rm )} is eminent. }

  \maintext{brahmavedi paraṃ vipraḥ nānyaṃ vedmi paraṃ tapaḥ |}%

  \maintext{sa vipraḥ sa tapasvī ca sa yogī sa śivaḥ smṛtaḥ }||\thinspace19:55\thinspace||%
\translation{The knower of Vedic mantras is the best Brahmin. I do not know of any other penance[?]. He is the real Brahmin, the real ascetic, the real yogin, he is said to be Śiva. \blankfootnote{19.55 I suspect that \textit{paraṃ tapaḥ} in \textit{pāda} b might mean \textit{parataram}...
 }}

\centerline{\maintext{\dbldanda\thinspace iti vṛṣasārasaṃgrahe dānayajñaviśeṣo nāma ūnaviṃśatitamo 'dhyāyaḥ\thinspace\dbldanda}}
\translation{Here ends the nineteenth chapter in the Vṛṣasārasaṃgraha called The particulars of donation and worship.}

  \chptr{viṃśatimo 'dhyāyaḥ}
\fancyhead[CO]{{\footnotesize\textit{Translation of chapter 20}}}%

  \trchptr{Chapter Twenty}%

  \maintext{vigatarāga uvāca |}%

  \maintext{pañcaviṃśati yat tattvaṃ jñātum icchāmi tattvataḥ |}%

  \maintext{kathayasva mamādya tvaṃ chidyate yena saṃśayaḥ }||\thinspace20:1\thinspace||%
\translation{Vigatarāga spoke: I would like to learn about the twenty-five Tattvas truely. Teach me now so that my doubts could be dispelled. \blankfootnote{20.1  This chapter echoes and is partly based on MBh 12.247.1-10 {\rm (}Mokṣadharma, see
  parallel passages in the apparatus{\rm )}:
  
 
  \textit{bhīṣma uvāca\thinspace |
  bhūtānāṃ guṇasaṃkhyānaṃ bhūyaḥ putra niśāmaya\thinspace |
  dvaipāyanamukhād bhraṣṭaṃ ślāghayā parayānagha }||1||
  \textit{dīptānalanibhaḥ prāha bhagavān dhūmravarcase\thinspace |
  tato 'ham api vakṣyāmi bhūyaḥ putra nidarśanam }||2||
  \textit{bhūmeḥ sthairyaṃ pṛthutvaṃ ca kāṭhinyaṃ prasavātmatā\thinspace |
  gandho gurutvaṃ śaktiś ca saṃghātaḥ sthāpanā dhṛtiḥ }||3||
  \textit{apāṃ śaityaṃ rasaḥ kledo dravatvaṃ snehasaumyatā\thinspace |
  jihvā viṣyandinī caiva bhaumāpyāsravaṇaṃ tathā }||4||
  \textit{agner durdharṣatā tejas tāpaḥ pākaḥ prakāśanam\thinspace |
  śaucaṃ rāgo laghus taikṣṇyaṃ daśamaṃ cordhvabhāgitā }||5||
  \textit{vāyor aniyamaḥ sparśo vādasthānaṃ svatantratā\thinspace |
  balaṃ śaighryaṃ ca mohaś ca ceṣṭā karmakṛtā bhavaḥ }||6||
  \textit{ākāśasya guṇaḥ śabdo vyāpitvaṃ chidratāpi ca\thinspace |
  anāśrayam anālambam avyaktam avikāritā }||7||
  \textit{apratīghātatā caiva bhūtatvaṃ vikṛtāni ca\thinspace |
  guṇāḥ pañcāśataṃ proktāḥ pañcabhūtātmabhāvitāḥ }||8||
  \textit{calopapattir vyaktiś ca visargaḥ kalpanā kṣamā\thinspace |
  sad asac cāśutā caiva manaso nava vai guṇāḥ }||9||
  \textit{iṣṭāniṣṭavikalpaś ca vyavasāyaḥ samādhitā\thinspace |
  saṃśayaḥ pratipattiś ca buddhau pañceha ye guṇāḥ }||10||.
  
 }}

  \subchptr{tattvanirṇayam}%

  \trsubchptr{Explaining the Tattvas}%

  \maintext{anarthayajña uvāca |}%

  \maintext{sarvapratyakṣadarśitvaṃ kathaṃ māṃ praṣṭum arhasi |}%

  \maintext{pṛṣṭena kathanīyo 'smi eṣa me kṛtaniścayaḥ |}%

  \maintext{śṛṇu te sampravakṣyāmi tattvasadbhāvam uttamam }||\thinspace20:2\thinspace||%
\translation{Anarthayajña spoke: How can you possibly ask me to reveal everything as directly visible? [But] I made a decision that [whenever being] questioned, I am to speak. Listen, I shall teach you the supreme essence of the reality levels/principles {\rm (}\textit{tattva}{\rm )}. }

  \subchptr{puruṣaśivabrahmā {\rm {\rm (}25{\rm )}}}%

  \trsubchptr{The Puruṣa/Śiva/Brahmā {\rm (}25th{\rm )}}%

  \maintext{nādimadhyaṃ na cāntaṃ ca yan na vedyaṃ surair api |}%

  \maintext{atisūkṣmo hy atisthūlo nirālambo nirañjanaḥ }||\thinspace20:3\thinspace||%
\translation{That which has no beginning, no middle part and no end, and is not to be known even by the gods, that which is extremely subtle and extremely large, supportless and spotless, \blankfootnote{20.3 Note that the key terms in this verse {\rm (}\textit{ādi, madhya, anta, sūkṣma}{\rm )} are to be found in VSS 1.1ab:
  
 
  \textit{ anādimadhyāntam anantapāraṃ
  susūkṣmam avyaktajagatsusāram}.
 }}

  \maintext{acintyaś cāprameyaś ca akṣarākṣaravarjitaḥ |}%

  \maintext{sarvaḥ sarvagato vyāpī sarvam āvṛtya tiṣṭhati }||\thinspace20:4\thinspace||%
\translation{inconceivable, immeasurable, imperishable, devoid of syllables, that which is everything and everywhere and that which is pervasive, and exists covering everything. }

  \maintext{sarvendriyaguṇābhāsaḥ sarvendriyavivarjitaḥ |}%

  \maintext{ajarāmarajaḥ śāntaḥ paramātmā śivo 'vyayaḥ }||\thinspace20:5\thinspace||%
\translation{It appears to have the qualities of all the sense faculties but is devoid of all sense faculties. It is not subject to ageing, it is immortal and unborn. It is peaceful, it is the supreme soul, it is undecaying Śiva. \blankfootnote{20.5 I take \textit{ajarāmarajaḥ} in \textit{pāda} a as \textit{ajaro 'maro 'ajaś ca}.
 }}

  \maintext{alakṣyalakṣaṇaḥ svastho brahmā puruṣasaṃjñitaḥ |}%

  \maintext{pañcaviṃśaḥ sa vijñeyo janmamṛtyuharaḥ prabhuḥ }||\thinspace20:6\thinspace||%
\translation{It is characterised by being unobservable, it is self-abiding, it is Brahmā, it is called Puruṣa. It is to be known as the twenty-fifth [Tattva], the Lord {\rm (}\textit{prabhu}{\rm )} who destroys death and rebirth. }

  \maintext{kalākalaṅkanirmukto vyomapañcāśavarjitaḥ |}%

  \maintext{jalapakṣī yathā toyair na lipyeta jale caran |}%

  \maintext{tadvad doṣair na lipyeta pāpakarmaśatair api }||\thinspace20:7\thinspace||%
\translation{He is free of the stain of having parts[?], and is devoid of the fifty voids. As a waterbird is not stained by the water while swimming in it, similarly [the Puruṣa] is not stained even by hundreds of sinful acts. \blankfootnote{20.7 The term \textit{pañcavyoman} elsewhere in the text seem to signify the five gross elements;
  see CHECK. The fifty \textit{vyoman}s might mean the fifty sounds/letters of Sanskrit;
  see CHECK.
 }}

  \subchptr{prakṛtiḥ {\rm {\rm (}24{\rm )}}}%

  \trsubchptr{Prakṛti {\rm (}24th{\rm )}}%

  \maintext{caturviṃśati yat tattvaṃ prakṛtiṃ viddhi niścayam |}%

  \maintext{vikṛtiś ca sa vijñeyas tattvataḥ sa manīṣibhiḥ }||\thinspace20:8\thinspace||%
\translation{Know the twenty-fourth Tattva certainly as Prakṛti. It is in fact to be known as Vikṛti {\rm (}`Modification'{\rm )} by the wise. \blankfootnote{20.8 Understand \textit{caturviṃśati} as \textit{caturviṃśaṃ}, and note that \textit{pāda} b involves conjectures.
  For emending \textit{vidhi} to \textit{viddhi}, see \textit{viddhi} in 20.12b below.
 }}

  \maintext{prakṛtiprabhavāḥ sarve buddhyahaṃkāra-ādayaḥ |}%

  \maintext{vikṛtiṃ pratilīyante bhūmyādi kramaśas tu vai }||\thinspace20:9\thinspace||%
\translation{All [the other Tattvas below Prakṛti], Buddhi, Ahaṃkāra etc. originate in Prakṛti. Earth etc. [up to Buddhi] dissolve in Vikṛti one by one. }

  \subchptr{matiḥ/buddhiḥ {\rm {\rm (}23{\rm )}}}%

  \trsubchptr{Mati {\rm (}Buddhi, intelligence, 23rd{\rm )}}%

  \maintext{matitattva trayoviṃśa dharmādiguṇasaṃyutaḥ |}%

  \maintext{sattvādhikasamutpannaboddhāraṃ viddhi dehinaḥ }||\thinspace20:10\thinspace||%
\translation{The Intelligence {\rm (}\textit{matitattva} [= Buddhi]{\rm )} is the twenty-third. It possesses qualities such as dharmic. Know it as the perceiver of the soul, produced by an abundance of Sattva. \blankfootnote{20.10 For my emendation of \textit{vidhi} to \textit{viddhi}, see \textit{viddhi} in 20.12b below.
 }}

  \subchptr{ahaṃkāraḥ {\rm {\rm (}22{\rm )}}}%

  \trsubchptr{Ahaṃkāra {\rm (}egoity, individualization, 22nd{\rm )}}%

  \maintext{dvāviṃśati ahaṃkāras tattvam uktaṃ manīṣibhiḥ |}%

  \maintext{bhūtādi mama pañcāha rajādhikasamudbhavam }||\thinspace20:11\thinspace||%
\translation{The twenty-second Tattva is Ahaṃkāra according to the wise. [This is the Tattva that] says: `The five [gross elements] etc.\ are mine!' It is produced by an abundance of Rajas. \blankfootnote{20.11 Understand \textit{dvāviṃśati} in \textit{pāda} a as \textit{dvāviṃśa}.
 CHECK in classical Sāṃkhya the bhūtādi and tāmasa ahaṃkāra gives birth to the elements!
 }}

  \subchptr{ākāśaḥ {\rm (}suṣiratvaṃ{\rm )} śabdaś ca {\rm (}21-20{\rm )}}%

  \trsubchptr{Space {\rm (}hollowness{\rm )} and Sound {\rm (}21st--20th{\rm )}}%

  \maintext{ekaviṃśati yat tattvaṃ suṣiraṃ viddhi bho dvija |}%

  \maintext{śabdātītaṃ suṣiratvaṃ saśabdaguṇalakṣaṇam }||\thinspace20:12\thinspace||%
\translation{Know the twenty-first Tattva as Hollowness {\rm (}\textit{suṣira}{\rm )} [= \textit{ākāśa}], O Brahmin. Hollowness is beyond Sound [but] it is characterised by the quality of Sound. \blankfootnote{20.12 Understand \textit{ekaviṃśati} in \textit{pāda} a as \textit{ekaviṃśa}.
 Note that from now on in this chapter, \textit{guṇa} is used in the sense of the \textit{tanmātra} of Sāṃkhya philosophy
  and that the word \textit{tanmātra} does not occur in the VSS.
 }}

  \subsubchptr{śabdaḥ}%

  \trsubsubchptr{Sound}%

  \maintext{saptasvarās trayo grāmā mūrchanās tv ekaviṃśatiḥ |}%

  \maintext{tānā{-}m{-}ekonapañcāśac chabdabhedas tadādayaḥ }||\thinspace20:13\thinspace||%
\translation{The seven [diatonic musical] notes {\rm (}\textit{svara}{\rm )}, the three basic scales {\rm (}\textit{grāma}{\rm )}, and the twenty-one modal scales {\rm (}\textit{mūrchana}{\rm )}; the forty-nine hexatonic and pentatonic scales {\rm (}\textit{tāna}{\rm )}: the classification of Sound includes these and other [classes]. }

  \maintext{evam ādīny anekāni svarabhedā dvijottama |}%

  \maintext{gāndharvasvaratattvajñair munibhiḥ samudāhṛtam }||\thinspace20:14\thinspace||%
\translation{These and many other are the classes of sounds, O Brahmin. [This] has been declared by the experts on musical notes. }

  \maintext{veṇumurajatantrīṇāṃ dundubhīnāṃ svanāni ca |}%

  \maintext{śaṅkhakāhalakāṃsyānāṃ śabdāni vividhāni ca }||\thinspace20:15\thinspace||%
\translation{[Other sounds include] the sounds of flutes, tambourines, lutes, kettle-drums, conch-shells, bass-drums and gongs. }

  \subsubchptr{ākāśaḥ}%

  \trsubsubchptr{Space}%

  \maintext{ākāśadhātu viprendra śṛṇu vakṣyāmi te daśa |}%

  \maintext{pāyūpasthodara kaṇṭha śaṅkhalau mukha nāsikau }||\thinspace20:16\thinspace||%
\translation{Listen, O excellent Brahmin, I shall teach you the ten elements {\rm (}\textit{dhātu}{\rm )} of space. [Space is present in the following ten bodily locations:] the anus, the sexual organs, the stomach, the neck, the two ears[?], the mouth, the two nostrils, }

  \maintext{hṛdiṃ ca daśamaṃ jñeyaṃ deha ākāśasambhavaḥ |}%

  \maintext{punar anyat pravakṣyāmi tac chṛṇuṣva dvijottama }||\thinspace20:17\thinspace||%
\translation{and the tenth, the [cavity of the] heart. The body originates in space. Next I shall teach you something else. Listen to it, O excellent Brahmin. }

  \maintext{daśa dhātuguṇā jñeyāḥ pañcabhūtaḥ pṛthak pṛthak |}%

  \maintext{ākāśasya guṇāḥ śabdo vyāpitvaṃ chidratāpi ca }||\thinspace20:18\thinspace||%
\translation{Ten element-guṇas {\rm (}\textit{dhātuguṇa}{\rm )} are to be known for each of the five elements {\rm (}\textit{bhūta}{\rm )}. The qualities of Space are Sound, pervasion and `perforatedness' [being pervaded], }

  \maintext{anāśrayanirālambam avyaktam avikāritā |}%

  \maintext{apratīghātitā caiva bhūtatvaṃ prakṛtāni ca }||\thinspace20:19\thinspace||%
\translation{its being supportless, independent and unmanifest, invariableness, not being restrainable, being an element, and ..., [?]. }

  \subchptr{vāyuḥ sparśaś ca {\rm (}19-18{\rm )}}%

  \trsubchptr{Wind and Touch {\rm (}19th--18th{\rm )}}%

  \maintext{ākāśadhātor viprendra tato vāyusamudbhavaḥ |}%

  \maintext{śabdapūrvaguṇaṃ gṛhya vāyoḥ sparśaguṇaḥ smṛtaḥ }||\thinspace20:20\thinspace||%
\translation{The birth of Wind is then from the Space \textit{dhātu}. Together with the previous Sound \textit{guṇa}, Wind has Touch as its guṇa. }

  \maintext{śabda pūrvaṃ mayākhyātaṃ śṛṇu sparśaṃ dvijottama |}%

  \maintext{kaṭhinaś cikkaṇaḥ ślakṣṇo mṛdusnigdhakharadravāḥ }||\thinspace20:21\thinspace||%
\translation{I have already described Sound, listen to Touch, O excellent Brahmin. Hard, smooth, slippery, soft, sticky, sharp, fluid, }

  \maintext{karkaśaḥ paruṣas tīkṣṇaḥ śītoṣṇa daśa ca dvayam |}%

  \maintext{iṣṭāniṣṭadvayasparśa vapuṣā parigṛhyate }||\thinspace20:22\thinspace||%
\translation{rough, rugged, pointed[?], cold, hot: these are [the] twelve [\textit{vāyuguṇa}s]. It is the body that senses both pleasant and unpleasant touches. }

  \subsubchptr{prāṇāḥ}%

  \trsubsubchptr{The vital breaths}%

  \maintext{prāṇo 'pānaḥ samānaś ca udāno vyāna eva ca |}%

  \maintext{nāgakūrmo 'tha kṛkaro devadatto dhanaṃjayaḥ }||\thinspace20:23\thinspace||%
\translation{Prāṇa, Apāna, Samāna, Udāna and Vyāna, Nāga, Kūrma, Kṛkara, Devadatta, Dhanaṃjaya: }

  \maintext{daśa vāyupradhānaite kīrtitā dvijasattama |}%

  \maintext{dhanaṃjayo bhaved ghoṣo devadatto vijṛmbhakaḥ }||\thinspace20:24\thinspace||%
\translation{These are said to be the ten main Winds, O excellent Brahmin. Dhanaṃjaya is [responsible for] noise, Devadatta [for] yawning, }

  \maintext{kṛkaraḥ kṣudhakṛn nityaṃ kūrmonmīlitalocanaḥ |}%

  \maintext{nāga udghāṭanaṃ puṣyaṃ karoti satataṃ dvija }||\thinspace20:25\thinspace||%
\translation{Kṛkara constantly causes hunger, Kūrma is [is responsible for] the opening of the eyes. Nāga constantly opens [up things] and nourishes, O Brahmin. \blankfootnote{20.25 Kṛkara in other texts usually performs sneezing {\rm (}\textit{kṣut}{\rm )}, here
  it seems that \textit{kṣudha}° stands for \textit{kṣudhā}° metri causa.
 }}

  \maintext{prāṇaḥ śvasati bhūtānāṃ niśvasanti ca nityaśaḥ |}%

  \maintext{prayāṇaṃ kurute yasmāt tasmāt prāṇa iti smṛtaḥ }||\thinspace20:26\thinspace||%
\translation{Prāṇa makes living beings inhale and exhale. It is called Prāṇa because it sets [beings] in motion {\rm (}\textit{prayāṇaṃ kurute}{\rm )}. }

  \maintext{apanayaty apānas tu āhāraṃ manujām adhaḥ |}%

  \maintext{śukramūtravaho vāyur apānas tena kīrtitaḥ }||\thinspace20:27\thinspace||%
\translation{Apāna takes people's[?] food down. It gets rid of semen and urine, that is why it is called Apāna [the 'down and out' Wind]. }

  \maintext{pītabhakṣitam āghrātaṃ raktapittakaphānilam |}%

  \maintext{samaṃ nayati gātreṣu samāno nāma mārutaḥ }||\thinspace20:28\thinspace||%
\translation{The Wind called Samāna brings into equilibrium that which has been drunk, the food that has been eaten, the blood [and the three humours] Pitta, Kapha and Anila. }

  \maintext{spandayaty adharaṃ vaktraṃ netragātraprakopanam |}%

  \maintext{udvejayati marmāṇi udāno nāma mārutaḥ }||\thinspace20:29\thinspace||%
\translation{The Wind called Udāna causes the lower lip and the mouth to tremble, it irritates the eyes and the limbs and it disturbs the vital organs. }

  \maintext{vyāno vināmayaty aṅgaṃ vyaṅgo vyādhiprakopanaḥ |}%

  \maintext{prītivināśakathitaṃ vārdhikyaṃ vyāna ucyate }||\thinspace20:30\thinspace||%
\translation{Vyāna bends the limbs, [it makes the body] deformed, [it causes] illness and irritaion. It is said to destroy pleasure ... is called Vyāna. }

  \maintext{daśavāyuvibhāge ca kīrtito me dvijottama |}%

  \maintext{daśavāyuguṇāṃś cānyāṃ chṛṇu kīrtayato mama }||\thinspace20:31\thinspace||%
\translation{[Everything] concerning the section on the ten Winds has been taught by me, O excellent Brahmin. [Now] listen as I teach you the ten other \textit{guṇa}s of Wind. }

  \maintext{vāyor aniyama sparśo vādasthānaṃ svatantratā |}%

  \maintext{balaṃ śīghraṃ ca mokṣaṃ ca ceṣṭā karmātmanā bhavaḥ }||\thinspace20:32\thinspace||%
\translation{The Wind has [these \textit{guṇa}s]: unsettledness, touch, presence in speech, independence, strength, quickness, release, movement and performing actions and existence. \blankfootnote{20.32 While I hesitate to emend this verse to fully correspond to the 
  very similar one in the MBh, my translation partly reflects the latter.
 }}

  \subchptr{tejo rūpaś ca {\rm (}17-16{\rm )}}%

  \trsubchptr{Fire and Form {\rm (}17th-16th{\rm )}}%

  \maintext{vāyunāpi sṛjas tejas tadrūpaṃ guṇam ucyate |}%

  \maintext{śabdasparśasama jyotis triguṇaṃ samudāhṛtam }||\thinspace20:33\thinspace||%
\translation{Fire is created by Wind. Its \textit{guṇa} is Form. There are three guṇas of Fire together with Sound and Touch. \blankfootnote{20.33 Understand \textit{sṛjas} as \textit{sṛṣṭas} and \textit{tadrūpaṃ guṇam ucyate} as \textit{tadguṇaṃ rūpam ucyate}
 I understand \textit{śabdasparśasama jyotis triguṇaṃ} as \textit{śabdasparśena saha jyotis triguṇaṃ}.
 }}

  \maintext{śabdaḥ sparśaḥ purā proktaḥ śṛṇu rūpaguṇaṃ tataḥ |}%

  \maintext{hrasvaṃ dīrgham aṇu sthūlaṃ vṛttamaṇḍalam eva ca }||\thinspace20:34\thinspace||%
\translation{Sound and Touch have been discussed before, therefore [now] hear about the Form guṇa. Short, tall, minute, gross and circular, }

  \maintext{caturasraṃ dvirasraṃ ca tryasraṃ caiva ṣaḍasrakam |}%

  \maintext{śuklaḥ kṛṣṇas tathā rakto nīlaḥ pīto 'ruṇas tathā }||\thinspace20:35\thinspace||%
\translation{square, ???, triangle and hexagon. Light, dark, red, blue, yellow, brown, }

  \maintext{śyāmaḥ piṅgala babhruś ca nava raṅgāḥ prakīrtitāḥ |}%

  \maintext{navadhā navaraṅgānām ekāśīti guṇāḥ smṛtāḥ }||\thinspace20:36\thinspace||%
\translation{dark-blue, golden, deep-brown: these are the nine colours. The ninefold guṇas of the nine colours make up 81. }

  \maintext{tejodhātu daśa brūmaḥ śṛṇuṣvāvahito bhava |}%

  \maintext{kāmas tejo kṣaṇaḥ krodho jaṭharāgniś ca pañcamaḥ }||\thinspace20:37\thinspace||%
\translation{I am telling you the ten Fire dhātus, listen and be attentive. Desire, heat, sight, anger, the digestive fire as the fifth, }

  \maintext{jñānaṃ yogas tapo dhyānaṃ viśvāgnir daśamaḥ smṛtaḥ |}%

  \maintext{daśa tejoguṇāṃś cānyān pravakṣyāmi dvijottama }||\thinspace20:38\thinspace||%
\translation{knowledge, yoga, penance, meditation, the fire of the universe[?] as the tenth. I shall teach you the other ten \textit{guṇa}s of Fire, O excellent Brahmin. }

  \maintext{agner durdharṣatāpnoti tāpapākaprakāśanaḥ |}%

  \maintext{śaucaṃ rāgo laghus taikṣṇyaṃ daśamaṃ cordhvabhāgitā }||\thinspace20:39\thinspace||%
\translation{Fire has [the following qualities:] unconquerable, ..., splendour, heat, cooking, illuminating, purity, passion, lightness, sharpness and the tenth, tending upwards. }

  \subchptr{āpo rasaś ca {\rm (}15-14{\rm )}}%

  \trsubchptr{Water and Taste {\rm (}15th--14th{\rm )}}%

  \maintext{jyotiso 'pi sṛjaś cāpaḥ saraso guṇasaṃyutaḥ |}%

  \maintext{caturguṇāḥ smṛtā āpaḥ vijñeyā ca manīṣibhiḥ }||\thinspace20:40\thinspace||%
\translation{Water is produced from Fire and has Taste as its \textit{guṇa}. The wise know Water as having four \textit{guṇa}s: }

  \maintext{śabdaḥ sparśaś ca rūpaṃ ca rasaś ca sa caturguṇaḥ |}%

  \maintext{rūpādiguṇa pūrvokta adhunātha rasaṃ śṛṇu }||\thinspace20:41\thinspace||%
\translation{Sound, Touch, Form, and the fourth, Taste. The \textit{guṇa}s Form etc. have been taught. Listen to Taste now. }

  \maintext{kaṭutiktakaṣāyāś ca lavaṇāmlas tathaiva ca |}%

  \maintext{madhuraś ca rasān ṣaḍ vai pravadanti manīṣiṇaḥ }||\thinspace20:42\thinspace||%
\translation{The wise teach that the six tastes are: pungent, bitter, astringent, salty, sour, and sweet. }

  \maintext{ṣaḍrasāḥ ṣaḍvibhedena ṣaṭtriṃśaguṇa ucyate |}%

  \maintext{āpadhātu daśa tv anyān śṛṇu kīrtayato mama }||\thinspace20:43\thinspace||%
\translation{[These] six tastes have six kinds and thus become thirty-six \textit{guṇa}s. I shall teach you the other ten \textit{dhātu}s of Water, listen to me. }

  \maintext{lālā siṅghāṇikā śleṣmā raktaḥ pittaḥ kaphas tathā |}%

  \maintext{svedam aśru rasaś caiva medaś ca daśamaḥ smṛtaḥ }||\thinspace20:44\thinspace||%
\translation{Saliva, mucus, the phlegmatic humour, blood, bile, phlegm, sweat, tears, chyle, and the tenth, fat. }

  \maintext{daśa āpaguṇāś cānye kīrtayiṣyāmi tān śṛṇu |}%

  \maintext{adbhyaḥ śaityaṃ rasa kledo dravatvaṃ snehasaumyatā |}%

  \maintext{jihvā viṣyandinī caiva bhaumānyaśravaṇādhamaḥ }||\thinspace20:45\thinspace||%
\translation{I shall teach you the ten \textit{guṇa}s of Water, listen to me. From Water [come] coldness, liquidity, wetness, fluidity, oiliness, the tongue, dripping, earthiness, wateriness, and flowing. \blankfootnote{20.45 This verse is a version of \MBH\ 12.247.4 {\rm (}see the apparatus{\rm )}.
 }}

  \subchptr{bhūmir gandhaś ca {\rm (}13-12{\rm )}}%

  \trsubchptr{Earth and Smell {\rm (}13rd--12th{\rm )}}%

  \maintext{āpaś cāpy asṛjad bhūmis tasyā gandhaguṇaḥ smṛtaḥ |}%

  \maintext{caturāpaguṇān gṛhya bhūmer gandhaguṇaḥ smṛtaḥ }||\thinspace20:46\thinspace||%
\translation{Water produced Earth. Its \textit{guṇa} is said to be Smell. Taking the four \textit{guṇa}s of Water, Earth has [the additional] \textit{guṇa} Taste: }

  \maintext{śabdaḥ sparśaś ca rūpaṃ ca raso gandhaś ca pañcamaḥ |}%

  \maintext{āpaḥpūrvaguṇāḥ proktā bhūmer gandhaguṇaṃ śṛṇu }||\thinspace20:47\thinspace||%
\translation{Sound, Touch, Form, Taste, and the fifth, Smell. The \textit{guṇa}s of Water have been taught. Now listen to the Smell-\textit{guṇa}s of Earth. }

  \maintext{iṣṭāniṣṭadvayor gandhaḥ surabhir durabhis tathā |}%

  \maintext{karpūraḥ kasturīkaṃ ca candanāgarum eva ca }||\thinspace20:48\thinspace||%
\translation{Smells are pleasant or unpleasant, fragrant and stinking. Pleasant smells are camphor, musk, sandalwood, and Aloe, }

  \maintext{kuṅkumādisugandhāni ghrāṇam iṣṭaṃ prakīrtitam |}%

  \maintext{viṅmūtrasvedagandhāni vaktragandhaṃ ca duḥsaham |}%

  \maintext{jīrṇasphoṭitagandhāni aniṣṭānīti kīrtitam }||\thinspace20:49\thinspace||%
\translation{saffron and other fragrant substances. Unleasant smells are the smell of urine, f\ae ces and sweat, unbearably bad breath, withered and decaying[?] smells. }

  \maintext{bhūmer dhātu daśa tv anyān kathayiṣyāmi tac chṛṇu |}%

  \maintext{tvacaṃ māṃsaṃ ca medaṃ ca snāyu majjā sirā tathā |}%

  \maintext{nakhadantaruhāś caiva keśaś ca daśamas tathā }||\thinspace20:50\thinspace||%
\translation{I shall teach you the other ten \textit{dhātu}s of Earth, listen to me. skin, flesh, fat, sinew, marrow, vein, nail, tooth, bodily hair, and the tenth, the hair of the head. \blankfootnote{20.50 I take \textit{ruha} as shorthand for \textit{tanuruha} or \textit{aṅgaruha}: `bodily hair'.
 }}

  \maintext{daśa tv anyān pravakṣyāmi śṛṇu bhūmiguṇān dvija |}%

  \maintext{bhūmeḥ sthairyaṃ rajastvaṃ ca kāṭhinyaṃ prasavātmakam |}%

  \maintext{gandho guruś ca śaktiś ca nīhārasthāpanākṛtiḥ }||\thinspace20:51\thinspace||%
\translation{I shall teach the other ten \textit{guṇa}s of Earth, listen, O Brahmin. From Earth [come] firmness, dustiness, rigidity, procreation, smell, heaviness, power, fog, support, shape. \blankfootnote{20.51 This verse is a variant of \MBH\ 12.247.3 {\rm (}see the apparatus{\rm )}.
 }}

  \maintext{guṇadhātuviśeṣaś ca utpattiś ca dvijottama |}%

  \maintext{yathā śrutaṃ mayā pūrvaṃ kīrtitaṃ nikhilena tu }||\thinspace20:52\thinspace||%
\translation{O Brahmin, I have taught you in a complete form the various \textit{guṇa}s and \textit{dhātu}s and [their] origin as I heard it before. }

  \subchptr{buddhīndriyāṇi karmendriyāṇi ca {\rm (}11-2{\rm )}}%

  \trsubchptr{Sense capacities and action capacities {\rm (}11th--2nd{\rm )}}%

  \maintext{vaikārikam ahaṃkāraṃ sattvodriktāt tu sāttvikaḥ |}%

  \maintext{śrotraṃ tvak cakṣuṣī jihvā nāsikā caiva pañcamī }||\thinspace20:53\thinspace||%
\translation{The Ahaṃkāra is subject to modification and because of an abundance of Sattva [in it], it is Sāttvika. Ear, skin, eyes, tongue, and the fifth, nose: \blankfootnote{20.53 CHECK In classical Sāṃkhya the vaikṛta and sāttvika 
  ahaṃkāra gives birth to the karmendriyas! See Ruzsa 127
 }}

  \maintext{buddhīndriyāṇi pañcaiva kīrtitāni dvijottama |}%

  \maintext{hastapādas tathā pāyur upastho vāk ca pañcamaḥ }||\thinspace20:54\thinspace||%
\translation{these are the five sense capacities {\rm (}\textit{buddhīndriya}{\rm )}, O great Brahmin. Hands, feet, anus, the sexual organs, and the fifth, speech[: these are the organs of action {\rm (}\textit{karmendriya}{\rm )}.] \blankfootnote{20.54 A line stating that the second set here is that of 
  the \textit{karmendriya}s may have dropped out.
 }}

  \subsubchptr{śrotram {\rm {\rm (}11{\rm )}}}%

  \trsubsubchptr{The sense capacity of hearing {\rm (}11th{\rm )}}%

  \maintext{śrotreṇa gṛhyate śabdo vividhas tu dvijottama |}%

  \maintext{veṇuvīṇāsvanānāṃ ca tantrīśabdam anekadhā }||\thinspace20:55\thinspace||%
\translation{The ear perceives the various sounds, O great Brahmin, those of flutes and lutes, and of various kinds of strings, }

  \maintext{muraja{\rm †}maunda{\rm †}paṇavabherīpaṭahanisvanam |}%

  \maintext{śaṅkhakāhalaśabdaṃ ca śabdaṃ ḍiṇḍimagomukham |}%

  \maintext{kāṃsikātālamiśraṃ ca gītāni vividhāni ca }||\thinspace20:56\thinspace||%
\translation{the sound of tambourines, ..., \textit{paṇava} drums, kettle-drums, tabors, the sound of conch-shells, bass drums, \textit{ḍiṇḍima} drums, `cow-face' horns, a mix of small gongs and cymbals, and various kinds of songs. }

  \subsubchptr{tvak {\rm {\rm (}10{\rm )}}}%

  \trsubsubchptr{The sense capacity of touch {\rm (}10th{\rm )}}%

  \maintext{tvacayā gṛhyate sparśaḥ sukhaduḥkhasamanvitaḥ |}%

  \maintext{mṛdusūkṣma sukhasparśaḥ vastraśayyāsanādayaḥ }||\thinspace20:57\thinspace||%
\translation{Touch is perceived by the skin, and it can be pleasant or unpleasant. Pleasant touch is soft and delicate, [such as] clothes, beds, seats, etc. }

  \maintext{tīkṣṇaśastrajalaśaityaṃ uṣṇe tapte kṣate kṣaraḥ |}%

  \maintext{evamādīny anekāni jñeyānīṣṭaṃ dvijottama }||\thinspace20:58\thinspace||%
\translation{The coldness of steel {\rm (}\textit{tīkṣṇaśastra}{\rm )} and water on a hot and painful wound, a cloud {\rm (}\textit{kṣara}{\rm )}??? these and others are the pleasant ones????, O great Brahmin. }

  \subsubchptr{cakṣuḥ {\rm {\rm (}9{\rm )}}}%

  \trsubsubchptr{The sense capacity of seeing {\rm (}9th{\rm )}}%

  \maintext{cakṣuṣā gṛhyate rūpaṃ sahasrāṇi śatāni ca |}%

  \maintext{devarūpavikārāṇi nakṣatragrahatārakāḥ }||\thinspace20:59\thinspace||%
\translation{Form are perceived by the Eye, and the hunderds of thousand forms of gods, constellations, planets, and stars, }

  \maintext{mānuṣānāṃ vikārāṇi grāmaṃ nagarapattanam |}%

  \maintext{vṛkṣagulmalatānāṃ ca paśupakṣiśarīsṛpām }||\thinspace20:60\thinspace||%
\translation{variations in people, villages, towns, and cities, of trees, bushes, creepers, cattle, birds, reptiles, }

  \maintext{kṛmikīṭapataṅgānāṃ jalajānām anekadhā |}%

  \maintext{śailadāravahemāni rūpāṇi vividhāni ca |}%

  \maintext{dhātudravyavikārāṇi rūpāṇi dvijasattama }||\thinspace20:61\thinspace||%
\translation{worms, insects, moths, various aquatic animals, and various shapes made of stone, wood, and gold, shapes of modifications of mineral materials, O truest Brahmin. }

  \subsubchptr{jihvā {\rm {\rm (}8{\rm )}}}%

  \trsubsubchptr{The sense capacity of tasting {\rm (}8th{\rm )}}%

  \maintext{jihvayā gṛhyate svādo hṛdyāhṛdyo dvijottama |}%

  \maintext{phalamūlāni śākāni kandāni piśitāni ca }||\thinspace20:62\thinspace||%
\translation{Taste is perceived by the Tongue, and it can be pleasant and unpleasant. Fruits and roots, vegetables, bulbs, and meat, }

  \maintext{pakvāpakvaviśeṣāṇi dadhikṣīraghṛtāni ca |}%

  \maintext{vrīhyauṣadharasānāṃ ca miśrāmiśram anekadhā |}%

  \maintext{ṣaṭkarmapratibhedena rasabhedaśataṃ smṛtam }||\thinspace20:63\thinspace||%
\translation{particular cooked and uncooked [dishes], yoghurt, milk, and ghee, herbs and juices, various mixed and unmixed [drinks]. Classified into six functions [the six basic flavours?], there are a hundred kinds of flavour {\rm (}\textit{rasa}{\rm )}. }

  \subsubchptr{ghrāṇam {\rm {\rm (}7{\rm )}}}%

  \trsubsubchptr{The sense capacity of smelling {\rm (}7th{\rm )}}%

  \maintext{ghrāṇena gṛhyate gandha iṣṭāniṣṭo dvijarṣabha |}%

  \maintext{guḍājyaṃ guggulur bhasmacandanāgarukaṃ tathā |}%

  \maintext{kastūrikuṅkumādīnām iṣṭo gandho manoharaḥ }||\thinspace20:64\thinspace||%
\translation{Smell is perceived by the Nose. Smell can be pleasant or unpleasant, O chief Brahmin. The smell of molasses, clarified butter, bdellium, ashes, sandalwood, Aloe, musk, saffron, etc., is pleasant and charming. }

  \maintext{vraṇamūtrapurīṣāṇāṃ māṃsaparyuṣitāni ca |}%

  \maintext{vātakarmādidurgandha aniṣṭaḥ samudāhṛtaḥ }||\thinspace20:65\thinspace||%
\translation{The bad smell of wounds, urine and f\ae ces, rotten meat, fart, etc., is said to be unpleasant. }

  \subsubchptr{hastakarma {\rm {\rm (}6{\rm )}}}%

  \trsubsubchptr{Function of the hands {\rm (}6th{\rm )}}%

  \maintext{hastena kurute karma vividhāni dvijottama |}%

  \maintext{māhendraṃ vāruṇaṃ caiva vāyavyāgneyam eva ca }||\thinspace20:66\thinspace||%
\translation{One performs various actions with the hands, O great Brahmin, those related to earth {\rm (}\textit{māhendra}{\rm )}, water {\rm (}\textit{vāruṇa}{\rm )}, air {\rm (}\textit{vāyavya}{\rm )}, and fire {\rm (}\textit{āgneya}{\rm )}. \blankfootnote{20.66 \textit{māhendra} means `connected to the earth' because of
  its synonym \textit{pārthiva}, which means both `royal' and `earthly'.
 }}

  \maintext{āgneya pacanādīni kāṃsyo lohas trapus tathā |}%

  \maintext{agnikarmāṇy anekāni yajñahomakriyās tathā }||\thinspace20:67\thinspace||%
\translation{Those related to fire are: cooking, etc., [working with] copper, iron, and tin, various fire-rituals, and rituals of fire-worship. }

  \maintext{sūrpavyajanavātena mukhavātena vai tathā |}%

  \maintext{camaracarmavātena vātayantraṃ ca vāyavam }||\thinspace20:68\thinspace||%
\translation{[Acts moving] air with a winnowing fan, one's breath, a chowrie or parchment, or a pankha, are the ones related to air. }

  \maintext{vāruṇaṃ toyakarmāṇi kurute vividhāni ca |}%

  \maintext{rasoparasakarmāṇi tasya poṣaṇakarma ca }||\thinspace20:69\thinspace||%
\translation{Those related to water perform various acts involving water: acts with flavours and secondary flavours, its [the body's?] nourishing, }

  \maintext{snānācamanakarmāṇi vastraśaucādayas tathā |}%

  \maintext{kāyaśaucaṃ ca kurute tṛṣānāśanam eva ca }||\thinspace20:70\thinspace||%
\translation{bathing, sipping water, washing clothes, etc., washing one's body, and quenching one's thirst. }

  \maintext{vamanāni hy anekāni vāruṇaṃ karma ucyate |}%

  \maintext{māhendraṃ pārthivaṃ karma anekāni dvijottama }||\thinspace20:71\thinspace||%
\translation{Various ways of vomitting [ex conj., accept msCb, or 9th?] are [also] called acts involving water. An earthly act involve various ways of working with earth, O great Brahmin. }

  \maintext{kulālakarma bhūkarma karma pāṣāṇam eva ca |}%

  \maintext{dārudantimaśṛṅgādikarma pārthivam ucyate |}%

  \maintext{catuṣkarma samāsena hastataḥ parikīrtitam }||\thinspace20:72\thinspace||%
\translation{The work of a potter, working with soil or with stones, work with wood, ivory, horns of animals, etc., are called earthly acts. The five actions of the hand have been briefly described. \blankfootnote{20.72 Understand °\textit{dantima}° in \textit{pāda} c as \textit{danta} or \textit{dantidanta}.
 }}

  \subsubchptr{pādakarma {\rm {\rm (}5{\rm )}}}%

  \trsubsubchptr{The action of the feet {\rm (}5th{\rm )}}%

  \maintext{pādābhyāṃ gamanaṃ karma diśaś ca vidiśas tathā |}%

  \maintext{nimnonnatasame deśe śilāsaṃkaṭakoṭare |}%

  \maintext{toyakardamasaṃghāte bahukaṇṭakasaṃkule }||\thinspace20:73\thinspace||%
\translation{The action for the feet is going in all the directions of the cardinal and intermediate point of the compass, up and down slopes, and on flat ground, on rocks, through passages, and into caves, through land flooded with water and filled with mud and covered in thorns. }

  \subsubchptr{pāyukarma {\rm {\rm (}4{\rm )}}}%

  \trsubsubchptr{The action of the anus {\rm (}4th{\rm )}}%

  \maintext{pāyukarma visargaṃ tu kaṭhinadravapicchalam |}%

  \maintext{saraktaphenilādīni pāyuśakti pramuñcati }||\thinspace20:74\thinspace||%
\translation{The function of the anus is the discharging [of urine]. [Whether it is] thick, fluid, slimy, bloody, foamy, etc., the energy of the anus discharges it. }

  \subsubchptr{upasthakarma {\rm {\rm (}3{\rm )}}}%

  \trsubsubchptr{The action of the sexual organs {\rm (}3rd{\rm )}}%

  \maintext{upasthakarma ānandaṃ karoti jananaṃ prajā |}%

  \maintext{strīpuṃnapuṃsakaṃ caiva upasthaṃ kurute dvija }||\thinspace20:75\thinspace||%
\translation{The functioning of the sexual organs cause joy, and produces offspring. The sexual organs produce females, males, and gender-neutral ones, O Brahmin. }

  \subsubchptr{vākkarma {\rm {\rm (}2{\rm )}}}%

  \trsubsubchptr{The action of speaking {\rm (}2nd{\rm )}}%

  \maintext{vācā tu kurute karma navadhā dvijapuṅgava |}%

  \maintext{stuti nindā praśaṃsā ca ākrośaḥ priya eva saḥ }||\thinspace20:76\thinspace||%
\translation{Speech performs nine kinds of action, O chief Brahmin. Praise, scolding, approval, abuse, kind words, }

  \maintext{praśno 'nujñā tathākhyānam āśīś ca vidhayo nava |}%

  \maintext{etā navavidhā vāṇī kīrtitā me dvijottama }||\thinspace20:77\thinspace||%
\translation{asking, permitting, describing events, and blessing are nine methods. The nine kinds of speech have been taught by me as being these, O great Brahmin. }

  \subchptr{manaś conmanaś ca {\rm {\rm (}1{\rm )}}}%

  \trsubchptr{The mind and the non-mind {\rm (}1st{\rm )}}%

  \maintext{adhunā kathayiṣyāmi manaso nava vai guṇān |}%

  \maintext{calopapattiḥ sthairaṃ ca visarga kalpanā kṣamā |}%

  \maintext{sad asac cāśutā caiva manaso nava vai guṇāḥ }||\thinspace20:78\thinspace||%
\translation{Now I shall teach the nine \textit{guṇa}s of the mind. Movement, effecting, firmness, emission, fantasy, patience, truth, untruth, and quickness are the nine \textit{guṇa}s of the mind. }

  \maintext{iṣṭāniṣṭavikalpaś ca vyavasāyaḥ samādhitā |}%

  \maintext{manaso dvividhaṃ rūpaṃ manaś conmana eva ca }||\thinspace20:79\thinspace||%
\translation{[The five \textit{guṇa}s of the mind, or intelligence, include] pleasant and unpleasant notions, intention and being united. The form of the Mind is twofold: mind and non-mind. \blankfootnote{20.79 \textit{Pāda}s ab are identical with \MBH\ 12.247.10ab, which 
  starts listing five \textit{guṇa}s, not of the mind {\rm (}\textit{manas}{\rm )}, but of
  intelligence {\rm (}\textit{buddhi}{\rm )}. The second half of
  the verse in the \MBH\ reads:
  \textit{saṃśayaḥ pratipattiś ca buddhau pañceha ye guṇāḥ}.
  Our 20.79ab may simply be misplaced or rather seems
  like a false start.
 }}

  \maintext{manas tv indriyabhāvatve unmanastvam anindriye |}%

  \maintext{nigṛhītā visṛṣtaṃ ca bandhamokṣau tu sādhanam }||\thinspace20:80\thinspace||%
\translation{It is mind when there are sense faculties, and non-mind when there are no senses. When [the senses] are controlled, liberation is produced. When they are let go, bondage is produced. \blankfootnote{20.80 Note the in \textit{pāda}s cd \textit{nigṛhītā} should connect to
  \textit{mokṣa} and \textit{visṛṣṭa} to \textit{bandha}, not in the order
  that the text suggests.
 }}

  \maintext{nigṛhītendriyagrāmaḥ svargamokṣau tu sādhanam |}%

  \maintext{visṛṣṭa indriyagrāme duḥkhasaṃsārasādhanam }||\thinspace20:81\thinspace||%
\translation{If the senses are under control, it means reaching heaven and liberation. When the senses are let loose, suffering and transmigration are the result. }

  \maintext{sakalaṃ niṣkalaṃ caiva mana eva vidur budhāḥ |}%

  \maintext{sakalaṃ mana nānātve ekatve mana niṣkalam }||\thinspace20:82\thinspace||%
\translation{The wise know that the mind can be form-endowed {\rm (}\textit{sakala}{\rm )} and formless {\rm (}\textit{niṣkala}{\rm )}. When there is multiplicity {\rm (}\textit{nānātva}{\rm )}, the mind is form-endowed {\rm (}\textit{sakala}{\rm )}. When there is oneness, the mind is formless {\rm (}\textit{niṣkala}{\rm )}. }

  \maintext{vigatarāga uvāca |}%

  \maintext{manaḥ svavedyaṃ lokānām unmanas tu na vidyate |}%

  \maintext{unmanaḥ kathayāsmākaṃ kīdṛśaṃ lakṣaṇaṃ bhavet }||\thinspace20:83\thinspace||%
\translation{Vigatarāga spoke: The mind is self-evident for people, but the non-mind {\rm (}\textit{unmanas}{\rm )} is not known. Teach us about the non-mind, what is its defining mark? }

  \maintext{anarthayajña uvāca |}%

  \maintext{unmanastvaṃ gate vipra nibodha daśalakṣaṇam |}%

  \maintext{na śabdaṃ śṛṇute śrotraṃ śaṅkhabherīsvanād api }||\thinspace20:84\thinspace||%
\translation{Anarthayajña spoke: Hear the ten chacteristic marks of someone who has entered the state of the non-mind, O Brahmin. His ears do not hear any sound, not even the sound of conch-shells and kettle-drums. }

  \maintext{tvacaḥ sparśaṃ na jānāti śītoṣṇam api duḥsaham |}%

  \maintext{rūpaṃ paśyati no cakṣuḥ parvatābhyadhiko 'pi vā }||\thinspace20:85\thinspace||%
\translation{His skin does not know touch, not even unbearable cold or hotness. His eyes do not see any forms, not even things bigger than mountains. }

  \maintext{jihvā rasaṃ na vindeta madhurāmlavaṇo 'pi vā |}%

  \maintext{gandhaṃ jighrati na ghrāṇā tīkṣṇaṃ vāpy aśucīny api }||\thinspace20:86\thinspace||%
\translation{His tongue does not sense tastes, not even sweetness or saltiness. His nose does not sense smells, not even strong or impure ones. }

  \maintext{unmanas tv eṣa me khyātaṃ sarvadvaitavināśanam |}%

  \maintext{bhavapāragasuvyaktaṃ niṣkalaṃ śivam avyayam }||\thinspace20:87\thinspace||%
\translation{I have taught the non-mind like this. It destroys all duality [such as hot-cold], which is imperishable Niṣkala Śiva that manifests competely when one reaches the other shore of existence. }

  \maintext{sa śivaḥ sa paro brahmā sa viṣṇuḥ sa paro 'kṣaraḥ |}%

  \maintext{sa sūkṣmaḥ sa paro haṃsaḥ so 'kṣaraḥ kṣaravarjitaḥ }||\thinspace20:88\thinspace||%
\translation{It is Śiva, highest Brahmā, highest and imperishable Viṣṇu. It is subtle, it is the highest Swan [i.e.\ soul], which is imperishable and devoid of decay. }

  \maintext{eṣa unmana jānīhi śivaś ca dvijapuṅgava |}%

  \maintext{kīrtito 'smi samāsena kim anyat paripṛcchasi }||\thinspace20:89\thinspace||%
\translation{Know this as the non-mind, and as Śiva, O chief of the Brahmins. I have taught you [this], what else would you like to ask? }

\centerline{\maintext{\dbldanda\thinspace iti vṛṣasārasaṃgrahe pañcaviṃśatitattvanirṇayo nāma viṃśatimo 'dhyāyaḥ\thinspace\dbldanda}}
\translation{Here ends the twentieth chapter in the Vṛṣasārasaṃgraha called the Description of the twenty-five Tattvas.}

  \chptr{ekaviṃśatimo 'dhyāyaḥ}
\fancyhead[CO]{{\footnotesize\textit{Translation of chapter 21}}}%

  \trchptr{Chapter Twenty-One}%

  \subchptr{viṣṇuḥ svarūpaṃ darśayati}%

  \trsubchptr{Viṣṇu reveals his true form}%

  \maintext{vigatarāga uvāca |}%

  \maintext{aho matimatāṃ śreṣṭha aho dharmabhṛtāṃ vara |}%

  \maintext{aho dama śamaḥ satya aho yajña aho tapaḥ }||\thinspace21:1\thinspace||%
\translation{Vigatarāga spoke: ``Bravo, O best of the wise, bravo, O best of the guardians of Dharma! What self-control, what tranquillity! What a sacrifice, what penance! }

  \maintext{anenāmṛtavākyena vismayo me paro gataḥ |}%

  \maintext{prīto 'smi ca tapādhārajñānādbhutarasena ca }||\thinspace21:2\thinspace||%
\translation{By this nectar-like speech [of yours], my amazement has risen considerably. And I am pleased with the extraordinary flavour of knowledge based on penance. }

  \maintext{kiṃ dadāmi varaṃ brūhi dātāsmi tava cepsitam |}%

  \maintext{etac chrutvā tatas tena pratyuvāca śubhāṃ giram }||\thinspace21:3\thinspace||%
\translation{What kind of boon shall I give you? Tell me. I'll give you anything you desire.'' Having heard this, [Anarthayajña] then replied with appropriate words. }

  \maintext{{\rm [}anarthayajña uvāca |{\rm ]}}%

  \maintext{ko bhavān varadaśreṣṭha devadānavarākṣasāḥ |}%

  \maintext{athavā bhagavān viṣṇur mama jijñāsur āgataḥ }||\thinspace21:4\thinspace||%
\translation{[Anarthayajña spoke:] ``Who are you, O best of benefactors? Are you a god, a Dānava-demon or a Rākṣasa? Or rather [you must be] Lord Viṣṇu, who has come to test me. }

  \maintext{vyaktaṃ tvāṃ puruṣaśreṣṭha jānāmi puruṣottama |}%

  \maintext{rūpaṃ darśaya govinda yady asti tapasaḥ phalam }||\thinspace21:5\thinspace||%
\translation{I recognize you clearly, O best of men, O highest person! Display your [true] Form, O Govinda, if penance can yield fruit.'' }

  \maintext{{\rm [}vaiśampāyana uvāca{\rm ]}}%

  \maintext{tatas tu puṇḍarīkākṣo darśayāmāsa svāṃ tanum |}%

  \maintext{śaṅkhacakragadāpāṇiḥ pītāmbaradharo hariḥ }||\thinspace21:6\thinspace||%
\translation{[Vaiśampāyana spoke:] Then lotus-eyed Hari displayed his own [true] body, holding in his hands a conch-shell, a discus and a mace, wearing yellow garments. }

  \maintext{anarthayajñas taṃ dṛṣṭvā vismayaṃ paramaṃ gataḥ |}%

  \maintext{praharṣam atulaṃ labdhvā aśrupūrṇākulekṣaṇaḥ }||\thinspace21:7\thinspace||%
\translation{Seeing him, Anarthayajña was truly amazed. Thrilled by unequalled delight, his eyes filled with tears, }

  \maintext{vepamānasvareṇātra uvāca ca janārdanam |}%

  \maintext{adya me saphalaṃ janma adya me saphalaṃ tapaḥ }||\thinspace21:8\thinspace||%
\translation{his voice trembling, he began speaking to Janārdana [i.e. Viṣṇu]. ``My birth and my austerities have now borne their fruits. }

  \maintext{namo namas te 'stu janādisambhave}%

 \nonanustubhindent \maintext{namo namas te 'stu ca viśvarūpiṇe |}%

  \maintext{namo namas te 'stu janābhisambhave}%

 \nonanustubhindent \maintext{namo namas te 'stu pitāmahodbhave }||\thinspace21:9\thinspace||%
\translation{Obeisance to you who are the origin of man and other [living beings]! [?] Obeisance to you who are the universe! Obeisance to you who [transforming into a person? DG] Obeisance to you from whom Brahmā was born! }

  \maintext{namo namas te 'stu sahasraśīrṣiṇe}%

 \nonanustubhindent \maintext{namo namas te 'stu sahasracakṣuṣe |}%

  \maintext{namo namas te 'stu sahasraliṅgine}%

 \nonanustubhindent \maintext{namo namas te 'stu sahasravakṣase }||\thinspace21:10\thinspace||%
\translation{Obeisance to you who have a thousand heads! Obeisance to you who have a thousand eyes! Obeisance to you who have a thousand liṅgas! Obeisance to you who have a thousand chests! }

  \maintext{namo namas te 'stu sahasramūrtaye}%

 \nonanustubhindent \maintext{namo namas te 'stu sahasrabāhave |}%

  \maintext{namo namas te 'stu sahasravaktriṇe}%

 \nonanustubhindent \maintext{namo namas te 'stu sahasramāyine }||\thinspace21:11\thinspace||%
\translation{Obeisance to you who have a thousand embodiments! Obeisance to you who have a thousand arms! Obeisance to you who have a thousand faces! Obeisance to you who have a thousand supernatural powers! }

  \maintext{namo namas te 'stu varāharūpiṇe}%

 \nonanustubhindent \maintext{namo namas te 'stu mahīsamuddhṛte |}%

  \maintext{namo namas te 'stu ca bhūtasṛṣṭine}%

 \nonanustubhindent \maintext{namo namas te caturāśramāśraye }||\thinspace21:12\thinspace||%
\translation{Obeisance to you who assumed the form of a boar! Obeisance to you who [in that form] dug out and saved the Earth! Obeisance to you who create all living beings! Obeisance to you on who are the seat of the four life-stages! }

  \maintext{namo namas te narasiṃharūpiṇe}%

 \nonanustubhindent \maintext{namo namas te ditijoradāriṇe |}%

  \maintext{namo namas te 'suracakrasūdane}%

 \nonanustubhindent \maintext{namo namas te 'suradarpanāśane }||\thinspace21:13\thinspace||%
\translation{Obeisance to you who assumed the form of the Man-lion! Obeisance to you who [in that form] tore asunder the chest of Diti's son [Hiraṇyakaśipu]! Obeisance to you who destroyed the armies [conj.] of the Asuras! Obeisance to you who destroyed the Asuras' haughtiness! }

  \maintext{namo namas te ditiputradāmane}%

 \nonanustubhindent \maintext{namo namas te baliyajñasūdane |}%

  \maintext{namo namas te 'stu ṣaḍardhavikrame}%

 \nonanustubhindent \maintext{namo namas te tridaśārtināśane }||\thinspace21:14\thinspace||%
\translation{Obeisance to you who tamed Diti's son [Bali?]! Obeisance to you who destroyed Bali's sacrifice! Obeisance to you of the three steps/Trivikrama! Obeisance to you who drove away the pain of the thirty gods! }

  \maintext{namo namas te 'stu ananta acyute}%

 \nonanustubhindent \maintext{namo namas te jagadartināśane |}%

  \maintext{namo namas te madhukaiṭanāśane}%

 \nonanustubhindent \maintext{namo namas te 'stu trilokabāndhave }||\thinspace21:15\thinspace||%
\translation{Obeisance to you who are imperishable, O endless one! Obeisance to you who drive away the pain of the world! Obeisance to you who killed [the Asuras] Madhu and Kaiṭa[bha]! Obeisance to you who are the friend of the three worlds! }

  \maintext{namo namas te tridaśābhinandane}%

 \nonanustubhindent \maintext{namo namas te 'stu ca divyacakṣuṣe |}%

  \maintext{namo namas te 'stu bhavāntapārage}%

 \nonanustubhindent \maintext{namo namas te 'stu trilokapūjite }||\thinspace21:16\thinspace||%
\translation{Obeisance to you who are the delight of the thirty gods! Obeisance to you who possess divine vision! Obeisance to you who have gone beyond the limits of existence! Obeisance to you who are worshipped by the three worlds! }

  \maintext{namo namas te 'stu gadāgrapāṇaye}%

 \nonanustubhindent \maintext{namo namas te varacakrapāṇaye |}%

  \maintext{namo namas te 'stu ca śaṅkhapāṇaye}%

 \nonanustubhindent \maintext{namo namas te 'stu ca kambupāṇaye }||\thinspace21:17\thinspace||%
\translation{Obeisance to you who hold a mace in [one of] your right[?] hand[s]! Obeisance to you who hold an excellent discus in your hand! Obeisance to you who hold a conch-shell in your hand! Obeisance to you who hold a conch-shell[? rather: lotus] in your hand! }

  \maintext{namo namas te 'stu jalaughaśāyine}%

 \nonanustubhindent \maintext{namo namas te haramardarūpiṇe |}%

  \maintext{namo namas te khagarājaketave}%

 \nonanustubhindent \maintext{namo namas te śaśisūryalocane }||\thinspace21:18\thinspace||%
\translation{Obeisance to you who recline on the ocean! Obeisance to you who have the form that crushed Hara [the Dānava?]! Obeisance to you whose banner has the King of Birds [Garuḍa] [on it]! Obeisance to you whose eyes are the Sun and the Moon! }

  \maintext{namo namas te uragārivāhane}%

 \nonanustubhindent \maintext{namo namas te 'dbhutarūpadarśine |}%

  \maintext{namo namas te 'yutasūryatejase}%

 \nonanustubhindent \maintext{namo namas te 'mṛtamanthanadhruve }||\thinspace21:19\thinspace||%
\translation{Obeisance to you whose vehicle is the Enemy of Serpents [i.e. Garuḍa]! Obeisance to you who display your extraordinary form! Obeisance to you whose splendour is that of a hundred thousand suns! Obeisance to you who was, [in your Kūrma-avatāra] the firm support at the churning out of the divine nectar! }

  \maintext{namo namas te 'maralokasaṃstute}%

 \nonanustubhindent \maintext{namo namas te jagamaṇḍapāśraye |}%

  \maintext{namo namas te jagadekavatsale}%

 \nonanustubhindent \maintext{namo namas te śivasarvade namaḥ }||\thinspace21:20\thinspace||%
\translation{Obeisance to you who are praised in the world of immortals! Obeisance to you who are the seat of the temple of the world! Obeisance to you, the only one affectionate towards the world! Obeisance to you who bestow happiness on everyone, obeisance! }

  \maintext{kṣamasva govinda mamāparādham}%

 \nonanustubhindent \maintext{atīva pṛṣṭena durātmanena |}%

  \maintext{mayeda sarvaṃ kathitaṃ smayena}%

 \nonanustubhindent \maintext{dayāṃ kuru tvaṃ tridaśeśvareṇa }||\thinspace21:21\thinspace||%
\translation{O Govinda, forgive my sin. As you were asking me very actively, I, being a wicked person, told you all this out of arrogance. Have pity on me, Lord of the thirty gods [instr.?].'' }

  \maintext{vaiśampāyana uvāca |}%

  \maintext{stotreṇānena saṃtuṣṭaḥ keśavaḥ paravīrahā |}%

  \maintext{pratyuvāca mahāseno girayā nirupaspṛhā }||\thinspace21:22\thinspace||%
\translation{Vaiśampāyana spoke: Keśava, the destroyer of the heroes of the enemy, was satisfied by this hymn of praise. He, the great general, replied in a ... [nirupaspṛhā/spṛhayā] voice. }

  \maintext{stotreṇānena me tāta tuṣṭo 'smi bhṛśam ejitaḥ |}%

  \maintext{durlabhāny api trailokye dadāmi varam īpsitam }||\thinspace21:23\thinspace||%
\translation{I am satisfied by this hymn of praise of me, dear Sir. I am vehemently trembling [with joy]. I'll grant you any boon you desire even if it is something difficult to obtain in the three worlds. }

  \maintext{anena māṃ stauti nirāśritena}%

 \nonanustubhindent \maintext{tvayoktavedārthamanohareṇa |}%

  \maintext{yāvanti tatrākṣarasaṃkhyam asti}%

 \nonanustubhindent \maintext{tāvanti kalpān divi te vasanti }||\thinspace21:24\thinspace||%
\translation{[He who] praises me with this ....? [hymn] that you recited and which is fascinating because it contains the meaning of the Vedas, will dwell in heaven for as many \ae ons as the number of syllables in it. }

  \maintext{tvaṃ cāpi me brūhi varaṃ yatheṣṭaṃ}%

 \nonanustubhindent \maintext{trailokyarājyād api nirviśaṅkam |}%

  \maintext{dadāmi kiṃ saptamahīśvaratvam}%

 \nonanustubhindent \maintext{athārtharāśiṃ bahukanyakāṃ vā }||\thinspace21:25\thinspace||%
\translation{And you should choose a boon at your pleasure, fearlessly, beginning from sovereignty over the three worlds. Shall I grant you sovereignty over the seven-fold[?] world? Or a heap of gold? Or many girls? }

  \maintext{vaiśampāyana uvāca |}%

  \maintext{śrutvaiva divyaṃ varam acyutasya}%

 \nonanustubhindent \maintext{praṇamya pādadvayapaṅkaje tu |}%

  \maintext{vijñāya viṣṇuṃ varadaṃ vareṇyaṃ}%

 \nonanustubhindent \maintext{? prahṛ cetaḥ pukāncito 'to 'bravīt }||\thinspace21:26\thinspace||%
\translation{Vaiśampāyana spoke: Hearing the divine boons [offered] [em. to vacam?] by the imperishable one, he bowed down to his lotus-feet. Having recognized that Viṣṇu was being most generous, with a delighted heart....[to be reconstructed] }

  \maintext{na kāmaye 'nyapravaraṃ tu deva}%

 \nonanustubhindent \maintext{asaṃśayaṃ bandhanasāram ekam |}%

  \maintext{vimuktabandho bhavataḥ prasādād}%

 \nonanustubhindent \maintext{bhavāmi govinda rataś ca dharme }||\thinspace21:27\thinspace||%
\translation{I do not desire anything else as a gift, O God. Only {\rm (}\textit{eka}{\rm )} the essence of bondage is certain. I have been freed from this bondage by your Lordship's grace, and, O Govinda, I am delighting in Dharma. }

  \maintext{bhagavān uvāca |}%

  \maintext{yathaiva cittaṃ tava suprasannaṃ}%

 \nonanustubhindent \maintext{maharṣidevair api naiva dṛṣṭam |}%

  \maintext{akalmaṣaṃ duḥkhavivarjitatvam}%

 \nonanustubhindent \maintext{bhavārṇavas tīrṇam asaṃśayena }||\thinspace21:28\thinspace||%
\translation{The Lord spoke: The extent to which your mind has been enlightened is something even the great sages and the gods have never seen, [this] spotless freedom from suffering. The ocean of existence has certaily been crossed. }

  \maintext{gacchāma bho sāmprata śvetadvīpam}%

 \nonanustubhindent \maintext{agamya devair api durnirīkṣyam |}%

  \maintext{madbhaktipūtamanasā prayāti}%

 \nonanustubhindent \maintext{ghorārṇave naiva punaś caranti }||\thinspace21:29\thinspace||%
\translation{Well, let's go now to the White Island, which is unattainable and is inaccessible even for the gods. He who departs after his mind has been purified by his devotion towards me, will never again enter the dreadful ocean [of existence]. }

  \maintext{vaiśampāyana uvāca |}%

  \maintext{evam uktvā haris tatra kare gṛhya tapodhanam |}%

  \maintext{tataḥ so 'ntarhitas tatra tenaiva saha keśavaḥ }||\thinspace21:30\thinspace||%
\translation{Vaiśampāyana spoke: Having spoken thus, then Hari took the great ascetic by the hand, who disappeared in that moment, and with him Keśava, too. }

  \maintext{evaṃ hi dharmas tv adhikaprabhāvād}%

 \nonanustubhindent \maintext{gataḥ sa lokaṃ puruṣottamasya |}%

  \maintext{aśeṣabhūtaprabhavāvyayasya}%

 \nonanustubhindent \maintext{sanātanaṃ śāśvatam akṣarasya }||\thinspace21:31\thinspace||%
\translation{Thus, as a consequence of the abundance of Dharma[?? in him?], he [Anarthayajña] reached world of the Highest Person, of the one who is the origin of all living beings, and who is imperishable, the eternal and never-ending [world] of the never-decaying. }

  \maintext{tvam eva bhaktiṃ kuru keśavasya}%

 \nonanustubhindent \maintext{janārdanasyāmitavikramasya |}%

  \maintext{yathā hi tasyaiva dvijarṣabhasya}%

 \nonanustubhindent \maintext{gatiṃ labhasva puruṣottamasya }||\thinspace21:32\thinspace||%
\translation{You yourself should be loyal to Keśava, to Janārdana of unmeasurable heroism, so that you can tread the path of that best among the twice-born [i.e. that Brahmin], [to] that excellent person. }

  \maintext{kim anya bhūyaḥ kathayāmi rājan}%

 \nonanustubhindent \maintext{yad asti kautūhalam anyaśeṣam |}%

  \maintext{pṛcchasva māṃ tāta yathepsitaṃ te}%

 \nonanustubhindent \maintext{bhaviṣyabhūtaṃ bhavato yatheṣṭam }||\thinspace21:33\thinspace||%
\translation{What else should I teach you further, O king? If you have any curiosity remaining, ask me, Sir, whatever you want regarding the future or the past, anything you wish, Sir. }

  \maintext{janamejaya uvāca |}%

  \maintext{kiyanti kalpāni gatāni pūrvam}%

 \nonanustubhindent \maintext{bhaviṣyakalpāni kiyanti vipra |}%

  \maintext{ekaikakalpaṃ kiyad indram uktam}%

 \nonanustubhindent \maintext{pravartamānād api kīrtayasva }||\thinspace21:34\thinspace||%
\translation{Janamejaya spoke: How many kalpas have passed until now? How many are the future kalpas? How many Indras are taught to exist with regard to each \ae on? Tell me one by one[???]. }

  \maintext{vaiśampāyana uvāca |}%

  \maintext{parārdhakalpaṃ gata pūrva rājyam}%

 \nonanustubhindent \maintext{caturdaśaivendra narendra kalpam |}%

  \maintext{tathaiva manvantara kalpam ekam}%

 \nonanustubhindent \maintext{bhaviṣyakalpaṃ ca parārdham eva }||\thinspace21:35\thinspace||%
\translation{Vaiśampāyana spoke: 100,000 billions of Kalpas have passed so far [rājyam? / rājan?]. There are fourteen Indras in one Kalpa, O king. The same [number applies to] Manvantaras per Kalpa. The future Kalpas are again 100,000 billion. }

  \maintext{varāhakalpaḥ prathamo babhūva}%

 \nonanustubhindent \maintext{gatāś ca manvantara ṣaḍ narendra |}%

  \maintext{caturyugaṃ saptati ekayuktaṃ}%

 \nonanustubhindent \maintext{manvantarā saṃkhyam udāharanti }||\thinspace21:36\thinspace||%
\translation{The first Kalpa was the Varāhakalpa. Six Manvantaras have passed, O King. Seventy-one four-fold [Mahā]yugas is the number that applies to a Manvantara. }

  \maintext{manvantarāṇāṃ ca caturdaśaiva}%

 \nonanustubhindent \maintext{kalpasya saṃkhyā munayo vadanti |}%

  \maintext{kalpāyutaś cāha pitāmahasya}%

 \nonanustubhindent \maintext{tathā ca rātriṃ pravadanti tajjñāḥ }||\thinspace21:37\thinspace||%
\translation{Fourteen Manvantara-periods is one Kalpa, according to the sages. Ten thousand Kalpas is Brahmā's day. His night is [of] the same [length] according to the experts. }

  \maintext{ṣaḍlakṣakalpena tu māsam āhus}%

 \nonanustubhindent \maintext{taddvādaśā varṣam udāharanti }||\thinspace21:38\thinspace||%
\translation{Six hundred-thousand Kalpas is called a [cosmic] month. Twelve of them is called a year. }

  \maintext{tenābdena parārdhakalpaguṇitaṃ brahmāyur ity ucyate}%

 \nonanustubhindent \maintext{trailokyādhipatiḥ pradhānapuruṣo brahmāpy anityaḥ smṛtaḥ |}%

  \maintext{śeṣaṃ bhūtacaturvidhasya niyataṃ jīvasya kiṃ śocyate}%

 \nonanustubhindent \maintext{tasmān nāsti jagatsusāravimalaṃ muktvā śivaṃ śāśvatam }||\thinspace21:39\thinspace||%
\translation{Brahmā's life is said to be that year multiplied by 100,000 billions of Kalpas[?]. But even Brahmā, the Lord of the three worlds, the supreme person, is taught to be transient. Why should we grieve over the rest of the four kinds of living beings and the fate[?] of the soul? Therefore there is nothing that is untouched by the fine[?] essence of the world except for eternal Śiva. }

\centerline{\maintext{\dbldanda\thinspace iti vṛṣasārasaṃgrahe kalpanirṇayo nāmaikaviṃśatimo 'dhyāyaḥ\thinspace\dbldanda}}
\translation{Here ends the twenty-first chapter in the Vṛṣasārasaṃgraha called the Description of the \ae ons.}

  \chptr{dvāviṃśo 'dhyāyaḥ}
\fancyhead[CO]{{\footnotesize\textit{Translation of chapter 22}}}%

  \trchptr{Chapter Twenty-two}%

  \maintext{janamejaya uvāca |}%

  \maintext{śruto 'thābjamukhād dharmasārasaṃgraham uttamam |}%

  \maintext{madhuraślakṣṇavāṇībhiḥ samyagvedārthasaṃyutam }||\thinspace22:1\thinspace||%
\translation{Janamejaya spoke: I have heard from [your] lotus-mouth the ultimate compendium on the essence of Dharma, in the proper way, together with the meaning of the Vedas, conveyed by sweet and polished speech. \blankfootnote{22.1 Gender problem or śruto = I heard.
 }}

  \maintext{nyāyayuktaṃ mahāsāraṃ guhyajñānam anuttaram |}%

  \maintext{tṛpto 'smīhāmṛtaṃ pītvā janmamṛtyurujāpaham }||\thinspace22:2\thinspace||%
\translation{This great essence is systematic {\rm (}nyāyayukta{\rm )}, and it is the supreme secret knowledge. I am satisfied now having drunk the nectar of immortality that removes birth, death and disease. }

  \maintext{praśnam ekānya pṛcchāmi nāmahetuṃ tapodhana |}%

  \maintext{varṇagotrāśramaṃ tasmāc chrotum icchāmi te punaḥ }||\thinspace22:3\thinspace||%
\translation{I want to ask you another question, O great ascetic, the reason behind the name [of Anarthayajña]. I would like you to tell [me] about [his] Varṇa, Gotra and Āśrama. \blankfootnote{22.3 Note \textit{te} in pāda d: it should most probably be understood in the sense of an ablative.
 }}

  \maintext{vaiśampāyana uvāca |}%

  \maintext{śṛṇu rājann avahito yogendrasya mahātmanaḥ |}%

  \maintext{āśramaṃ varṇajātīnāṃ vakṣyāmy eva narādhipa }||\thinspace22:4\thinspace||%
\translation{Vaiśampāyana spoke: Listen, O King, attentively. I shall tell you about the Āśrama, the Varṇa and the Jāti of the great and noble yogin, O king. }

  \maintext{himavaddakṣiṇe pārśve mṛgendraśikhare nṛpa |}%

  \maintext{mahendrapathagānāmanadītīre narādhipa }||\thinspace22:5\thinspace||%
\translation{In the southern region of the Himālaya, on the Mṛgendra peak, O king, on the banks of the river Mahendrapathaga, O King, }

  \maintext{tatrāśramapadaṃ tasya puline sumanorame |}%

  \maintext{vasati sma mahābhāgas tattvapāraganispṛhaḥ }||\thinspace22:6\thinspace||%
\translation{there was his hermitage. The illustrous one lived on the beautiful banks [of the river], having reached the other shore of Truth, free from desire, }

  \maintext{śīlaśaucasamācāro jitadvandvo jitaśramaḥ |}%

  \maintext{jitamānabhayakrodho jitasarvaparigrahaḥ }||\thinspace22:7\thinspace||%
\translation{leading a moral and pure life, with all opposites [such as happiness and pain] and weakness conquered, with his arrogance, fear, and anger conquered, with his greediness completely conquered. }

  \maintext{somavaṃśaprasūtās te kṣatriyā dvijatāṃ gatāḥ |}%

  \maintext{tapasā vinayācārair viṣṇunā dvijakalpitāḥ }||\thinspace22:8\thinspace||%
\translation{The Kṣatriyas born in the Soma clan became twice-born [Brahmins]. Because of [the] penance [they performed], [and because of their] discipline and good conduct, Viṣṇu turned them into twice-born [Brahmins]. }

  \maintext{ajitā nāma tatpūrvaṃ kāmakrodhajitena tu |}%

  \maintext{saṃkalpas tasya rājendra kathayiṣyāmi tac chṛṇu }||\thinspace22:9\thinspace||%
\translation{They had been called Ajitas, [and] the one who conquered lust and anger [transformed them]. O great King, I shall tell you about his vow, listen. \blankfootnote{22.9 Tentatively, I take \textit{kāmakrodhajitena} in pāda b as a reference to Vigatarāga, i.e. Viṣṇu
  in disguise, who appears as a Brahmin to test Anarthayajña in 1.8.
 }}

  \maintext{adhyātmanagarasphītaḥ adhibhūtajanākulaḥ |}%

  \maintext{adhidaivatasāṃnidhyaṃ daśāyatana pañca ca }||\thinspace22:10\thinspace||%
\translation{I am flourishing in the town of the Spirit, which is[?] populated by Matter[?], in the vicinity of the divine realm[?], the ten abodes and the five [phps Sāṃkhyatattvas??]. }

  \maintext{daśayajñavrataṃ cīrṇaṃ daśakāmaparājitaḥ |}%

  \maintext{niyamān daśa saṃśritya daśa vāyava ṛtvijaḥ }||\thinspace22:11\thinspace||%
\translation{The vow of ten sacrifices was observed and he conquered the ten desires. He followed the ten Niyama-rules. The ten winds were his priest. }

  \maintext{daśākṣareṇa mantreṇa daśadharmakriyāpadaḥ |}%

  \maintext{daśasaṃyamadīptāgnau jihvātejodaśendriyaḥ }||\thinspace22:12\thinspace||%
\translation{With the ten-syllable mantra, he was at the level of the ten Dharmic rituals[???]. His ten sense faculties had the energy of the flames in the [sacrificial] fire lit by the ten saṃyamas. }

  \maintext{daśayogāsanāsīno daśadhyānaparāyaṇaḥ |}%

  \maintext{buddhir vedī mano yūpaḥ somapāno 'mṛtākṣaraḥ }||\thinspace22:13\thinspace||%
\translation{He practised the ten yogic sitting positions and focused on the ten ways of meditation. His intellect was his altar, his mind the sacrificial post, and Soma consumption was the immortal syllable. }

  \maintext{dakṣiṇābhaya bhūtebhyaḥ paśubandha svayaṃkṛtaḥ |}%

  \maintext{vinārthaṃ yajñam iṣṭvā tu kālaṃ ca kṣapayaty asau |}%

  \maintext{anarthayajñaṃ taṃ prāhur munayas tattvadarśinaḥ }||\thinspace22:14\thinspace||%
\translation{The priestly fee was fearlessness offered to living beings, the tying of the sacrificial animal was performed on[?] himself. He spent his time performing immaterial sacrifice [thus]. The sages, who know the truth, call him Anarthayajña. }

  \maintext{janamejaya uvāca |}%

  \maintext{daśayajñam ahaṃ śrotuṃ dehi māṃ dvijasattama |}%

  \maintext{daśakāmadaśadhyānaṃ daśayogadaśākṣaram }||\thinspace22:15\thinspace||%
\translation{Janamejaya spoke: Please let me hear about the ten sacrifices, O best of Brahmins, and about the ten desires and the ten kinds of meditation, the ten yogas and the ten-syllable [mantra]. }

  \maintext{vaiśampāyana uvāca |}%

  \maintext{brahmadevapitṛyajño yajño bhūtātitheś ca ha |}%

  \maintext{japo yogas tapo dhyānaṃ svādhyāyaś ca daśa smṛtaḥ }||\thinspace22:16\thinspace||%
\translation{Vaiśampāyana spoke: Sacrifice to/with the Brahman [?; = Vedic offering at <i>saṃdhyā</i>], to the Devas, the Ancestors, the Ghosts, the Guests, recitation, yoga, penance, meditation and [Vedic?] study: these are the ten [sacrifices]. \blankfootnote{22.16 The missing bit is broken off in \msNa. \msL\ seems to copy \msNa\ directly.
 }}

  \maintext{patnīputrapaśubhṛtyadhanadhānyayaśaḥśriyaḥ |}%

  \maintext{māna bhoga daśa rājan daśakāma udāhṛtaḥ }||\thinspace22:17\thinspace||%
\translation{Wife, son, cattle, servant, wealth, grain, fame, beauty, respect, and enjoyment as the tenth, O king: the ten desires have been taught. }

  \maintext{mānaso yaugapadyaś ca saṃkṣiptaś ca viśāmpate |}%

  \maintext{viśālā nāma yogaś ca tato dvikaraṇaḥ smṛtaḥ }||\thinspace22:18\thinspace||%
\translation{Mental, simultaneous and condensed [yoga], O king, and the yoga named Viśālā, and also the one known as Dvikaraṇa, }

  \maintext{raviḥ somo hutāśaś ca sphaṭikāmbaram eva ca |}%

  \maintext{daśayogāsanāsīno nityam eva tapodhanaḥ }||\thinspace22:19\thinspace||%
\translation{sun, moon, fire, crystal and sky. Always sitting in [one of] the ten yoga positions, the great ascetic, }

  \maintext{anirodhamanāḥ sūkṣmaṃ dhyāyed yogaḥ sa mānasaḥ |}%

  \maintext{prāṇāyāmair mano ruddhvā yaugapadyaḥ sa ucyate }||\thinspace22:20\thinspace||%
\translation{when his mind is still not under control, should visualize the subtle one. This is mental yoga. When he can control his mind with breath-control, that is called simultaneous [yoga]. }

  \maintext{brahmādistambaparyantaṃ sarvaṃ sthāvarajaṅgamam |}%

  \maintext{pralīyamānaṃ dhyāyeta kramāt sūkṣmaṃ vicintayet }||\thinspace22:21\thinspace||%
\translation{He should visualize the universe with all its moving and motionless [animate and inanimate] parts, from Brahmā to a tuft of grass, as gradually dissolving, and should reflect upon the subtle one: }

  \maintext{saṃkṣipta eṣa ākhyāto viśālāṃ chṛṇu tattvataḥ |}%

  \maintext{brahmādisūkṣmaparyantaṃ cintayīta vicakṣaṇaḥ }||\thinspace22:22\thinspace||%
\translation{this is called condensed [yoga]. Now listen to the Viśālā. The wise one should call to mind [everything] from Brahmā to the subtle. }

  \maintext{saṃkṣiptāṃ ca viśālāṃ ca cintayīta parasparam |}%

  \maintext{eṣā dvikaraṇī nāma yogasya vidhir ucyate }||\thinspace22:23\thinspace||%
\translation{He should visualize both the condensed and the Viśālā mutually [one after the other? DG]: this is the yoga method called Dvikaraṇī. }

  \maintext{dehamadhye hṛdi jñeyaṃ hṛdimadhye tu paṅkajam |}%

  \maintext{paṅkajasya ca madhye tu karṇikāṃ viddhi gopate }||\thinspace22:24\thinspace||%
\translation{He should imagine his heart in the center of his body, and that there is a lotus in his heart. In the center of the lotus, know that there is a pericarp, O king. \blankfootnote{22.24 hṛdi as nominative...
 gopate is slightly odd for `king'.
 }}

  \maintext{karṇikāyās tu madhye tu pañcabinduṃ vidur budhāḥ |}%

  \maintext{ravisomaśikhāṃ caiva sphaṭikāmbaram eva ca }||\thinspace22:25\thinspace||%
\translation{The wise ones know that there are five dots in the center of the pericarp: the sun, the moon, the flame, the crystal and the sky. }

  \maintext{ravimaṇḍalamadhye tu bhāvayec candramaṇḍalam |}%

  \maintext{tasya madhye śikhāṃ dhyāyen nirdhūmajvalanaprabhām }||\thinspace22:26\thinspace||%
\translation{He should visualize the disk of the moon in the centre of the sun. In the centre of that [i.e.\ the moon], he should visualize fire that blazes without smoke. }

  \maintext{agnimadhye maṇiṃ dhyāyec chuddhadhārājalaprabham |}%

  \maintext{tasya madhye 'mbaraṃ dhyāyet susūkṣmaṃ śivam avyayam }||\thinspace22:27\thinspace||%
\translation{In the centre of the fire, he should visualize a gem which has the splendour of a jet of clear water. In its center, he should visualize the sky, subtle and imperishable Śiva. }

  \maintext{daśayogam idaṃ rājan kathitaṃ ca mayā tava |}%

  \maintext{daśadhyānaṃ samāsena kīrtitaṃ śṛṇu tad yathā }||\thinspace22:28\thinspace||%
\translation{This is how I taught you the ten yogas, O king. The ten ways of meditation are taught in short as here follows, listen. }

  \maintext{ghoṣaṇī piṅgalā caiva vaidyutī candramālinī |}%

  \maintext{candrā mano'nugā caiva sukṛtā ca tathāparā }||\thinspace22:29\thinspace||%
\translation{Sound, yellow, lightning, Candramālinī, moon, pleasing, well-done, }

  \maintext{saumyā nirañjanā caiva nirālambā ca kīrtitā |}%

  \maintext{supiṣitvāṅgulau śrotre dhvanim ākarṇayen naraḥ }||\thinspace22:30\thinspace||%
\translation{Saumyā, spotless and supportless. [1] Putting[?] two fingers in his ears, one can hear[!] sounds. }

  \maintext{tat tad akṣaram ākarṇya amṛtatvāya kalpyate |}%

  \maintext{piṅgalāṃ tu śikhādhūmāṃ dhyāyen nityam atandritaḥ }||\thinspace22:31\thinspace||%
\translation{Having heard this and that syllable, he is fit for immortality. [2] He should continuously visualize yellow, smokeless[?] flames tirelessly. \blankfootnote{22.31 Stem forms? śikhām adhūmāṃ ?
 }}

  \maintext{vimuktaḥ sarvapāpebhyo nirdvandvapadam āpnuyāt |}%

  \maintext{vaidyutī tu niśāmadhye lakṣate 'jam anāmayam }||\thinspace22:32\thinspace||%
\translation{He will be freed of all his sins and will reach the level without opposites. [3] The lightning in the middle of the night marks the unborn of no diseases. \blankfootnote{22.32 OR: lakṣyateja a°: the visible energy??
 }}

  \maintext{pañcamāsasadābhyāsād divyacakṣur bhaven naraḥ |}%

  \maintext{bindumālāṃ tataḥ paśyet tarucchāyāsamāśritām }||\thinspace22:33\thinspace||%
\translation{After five months of continuous practice, men will develop divine sight. [4] Then he should visualize the Bindumālā [Candramālā??] which rests in the shadow of a tree. }

  \maintext{jātyasphaṭikasaṃkāśaṃ dṛṣṭvā mucyati bandhanaiḥ |}%

  \maintext{dhyāyen mano'nugā nāma pakṣmīr āpīḍya locane }||\thinspace22:34\thinspace||%
\translation{[5] Seeing it as genuine crystal, he is liberated from the fetters. [6] He should visualize the Pleasing one ... ? pressing it in the eye. }

  \maintext{śvetapītāruṇaṃ binduṃ dṛṣṭvā bhūyo na jāyate |}%

  \maintext{mano'nugādi ṣaṭ tv ete dhyānam uktaṃ mayā tava }||\thinspace22:35\thinspace||%
\translation{When he sees the white and yellow and red drop, he will not be born again. These are the six ways of meditation, the Pleasing one and the others, as I taught them to you. }

  \subchptr{paramāṇuḥ}%

  \maintext{adhunānyat pravakṣyāmi paramāṇu caturvidham |}%

  \maintext{pārthivādicaturbhūtaṃ yair vyāptaṃ nikhilaṃ jagat |}%

  \maintext{lakṣaṇaṃ tasya rājendra śṛṇu vakṣyāmi sāmpratam }||\thinspace22:36\thinspace||%
\translation{Now I shall teach you another thing: the fourfold supreme atom. O supreme sovereign, listen to the characteristics of that[?] by which the whole world, made up of the four elements {\rm (}\textit{bhūta}{\rm )}, Earth etc., is pervaded. I shall tell you [about them] now. }

  \maintext{pārthivordhvagatiḥ sūkṣmaḥ paramāṇu narādhipa |}%

  \maintext{pratyakṣadarśanaṃ dhyānaṃ lakṣayen niyataṃ śuciḥ }||\thinspace22:37\thinspace||%
\translation{The subtle atom of Earth tends upwards, O king. The pure one should observe the meditation that is direct perception firmly. }

  \maintext{mucyate sarvapāpebhyo rāhunā candramā yathā |}%

  \maintext{tena yo 'bhyasate nityaṃ sa yogī bhuvaneśvaraḥ }||\thinspace22:38\thinspace||%
\translation{He will be freed from all his sins, as the Moon is from Rāhu. The yogin who constantly practises by this [method] is the lord of the world. }

  \maintext{adhogati mahārāja paramāṇu jalodbhavaḥ |}%

  \maintext{abhyased yad idaṃ rājan sarvapātakanāśanam }||\thinspace22:39\thinspace||%
\translation{The atom of Water tends downwards, O great king. If one practises this, O king, there will be a destruction of all his sins. }

  \maintext{āgneyaparamāṇūni tiryagūrdhvagatiḥ smṛtā |}%

  \maintext{ya idaṃ dhyāyate nityam uttamāṃ gatim āpnuyāt }||\thinspace22:40\thinspace||%
\translation{The atoms? of Fire tend upwards and horizontally. He who constantly meditates upon this will reach the supreme path. \blankfootnote{22.40 Note how a neuter ending in \textit{pāda} a governs a seemingly feminine ending in \textit{pāda} b REVISE. See the 
  same in 22.41ab. CHECK phenomenon and note gatiḥ.
 }}

  \maintext{vāyavyaparamāṇūni adhordhvatiryag āsmṛtā |}%

  \maintext{na sa muhyati taṃ dṛṣṭvā vāyusambhava bhūpate }||\thinspace22:41\thinspace||%
\translation{The atoms of Air tend downwards and horizontally. [If] he is not perplexed when seeing this, he is Hanumān, O king. \blankfootnote{22.41 Note °\textit{sambhava} instead of the more correct °\textit{sambhavo} metri causa.
 }}

  \maintext{catvāra ete rājendra paramāṇu nirīkṣate |}%

  \maintext{tena sarvamakhair iṣṭaṃ tena taptaṃ tapas tathā }||\thinspace22:42\thinspace||%
\translation{[If] he perceives these four [types of] atoms, O king, by this he has sacrificed with all sacrifices, by this penance is completed. }

  \maintext{tena dattā mahī kṛtsnā saptasāgarasaṃvṛtā |}%

  \maintext{sarvatīrthābhiṣekaś ca sarvavratakriyā tathā }||\thinspace22:43\thinspace||%
\translation{By this the whole Earth with its surrounding seven seas is given [as a sacrificial fee], and he will have received all consecrations at the sacred places and all the religious vows and rituals will have been completed. }

  \maintext{anenaiva vidhānena daśadhyānaṃ narādhipa |}%

  \maintext{kurute avyavacchinnaṃ sarvakāmaphalapradam }||\thinspace22:44\thinspace||%
\translation{If one practises the ten meditations by this method, O king, uninterruptedly, it will yield all the desired fruits. }

  \subchptr{daśākṣaramantraḥ}%

  \maintext{daśākṣaraṃ mahārāja yogīndrasya mahātmanaḥ |}%

  \maintext{kathayāmi samāsena śṛṇuṣvāvahito bhava }||\thinspace22:45\thinspace||%
\translation{O great king, I shall teach you in brief the ten-syllable [mantra] of that great-souled yogin king [Anarthayajña]. Listen attentively. }

  \maintext{praṇavādisvarā trīṇi dīrghabindusamāyutam |}%

  \maintext{pañca pañca cavarge tu vāyubījam adhaḥsthitam }||\thinspace22:46\thinspace||%
\translation{The first three sounds are \textit{praṇava}s {\rm (}\textit{oṃ}{\rm )} endowed with long nasalisation {\rm (}\textit{bindu}{\rm )}. \blankfootnote{22.46 I understand pāda a, \textit{praṇavādisvarā trīṇi} as if it read \textit{praṇavā ādisvarās trayas}.
 }}

  \maintext{trayodaśasvarāyuktaṃ pañcame parikīrtitam |}%

  \maintext{pañcaviṃśatimaḥ ṣaṣṭhaḥ akṣaraḥ parikīrtitaḥ }||\thinspace22:47\thinspace||%


  \maintext{yādṛśaṃ pañcame proktaṃ saptame ca prayojayet |}%

  \maintext{ākārasvarasaṃyuktaṃ sarvapātakanāśanam }||\thinspace22:48\thinspace||%


  \maintext{prathamaṃ pañcame varge tṛtīyasvarayojitam |}%

  \maintext{uttarekārasaṃyuktaṃ navamaṃ parikīrtitam }||\thinspace22:49\thinspace||%


  \maintext{daśamaḥ punar oṃkāraḥ mantraśreṣṭho daśākṣaraḥ |}%

  \maintext{japato dhyāyato vāpi pārthivādikrameṇa tu |}%

  \maintext{mucyate so 'pi saṃsāre saṃśayo nāsti bhūpate }||\thinspace22:50\thinspace||%


  \subchptr{ācāravidhiḥ}%

  \maintext{ācāramūlo dharmas tu dharmamūlo janārdanaḥ |}%

  \maintext{tena sarvajagad vyāptaṃ trailokyaṃ sacarācaraṃ }||\thinspace22:51\thinspace||%


  \maintext{ācārāl labhatīha āyur atulam akṣapyavittaṃ tathā}%

 \nonanustubhindent \maintext{ācārāt sutam īpsitaṃ ca labhate śrīkīrtiprajñāyaśaḥ |}%

  \maintext{ācārāl labhate ca lakṣmim atulāṃ khyātiṃ tathaivottamām}%

 \nonanustubhindent \maintext{ācārād iha mantradharmaparamaṃ prāpnoti niḥsaṃśayam }||\thinspace22:52\thinspace||%


  \maintext{janamejaya uvāca |}%

  \maintext{ācārāt prabhavānusaṃśakathitaṃ suśliṣṭadharmākaram}%

 \nonanustubhindent \maintext{ācārāt kativaṃśa kīrtaya punas tṛptir na me jāyate |}%

  \maintext{sarvajñaḥ tvam ahaṃ śṛṇomi varadaṃ kiñcin na me śāśvatam}%

 \nonanustubhindent \maintext{tan me kīrtaya dharmasāraśubhadam ācāramūlāśrayam }||\thinspace22:53\thinspace||%


  \maintext{vaiśampāyana uvāca |}%

  \maintext{nityaṃ namraśirodvijātiguruṣu śuśrūṣaṇaṃ devatā}%

 \nonanustubhindent \maintext{tiṣṭhetācamanena cāśanakaraṃ vāmāsthinānodade |}%

  \maintext{sūryāgniśaśibandhur āryapurataḥ kuryān na cāvaśyakam}%

 \nonanustubhindent \maintext{śasye bhasmani govraje dvija jalaṃ kuryān na cārkaṃ naraḥ }||\thinspace22:54\thinspace||%


  \maintext{pādenāgnijalaṃ spṛśen na ca guruṃ pādena pādaṃ tathā}%

 \nonanustubhindent \maintext{śaucaṃ kārya jalādinā ca niyataṃ nādho jalaṃ kārayet |}%

  \maintext{kuryān nityabhivādanaṃ dvijaguror mātāpitṛ devatām}%

 \nonanustubhindent \maintext{etācāravidhiḥ samāsaniyamas tubhyaṃ mayā kīrtitam }||\thinspace22:55\thinspace||%


  \subchptr{striyaḥ}%

  \maintext{janamejaya uvāca |}%

  \maintext{strīṇāṃ kiṃ priyam asti tad vada vibho saṃsārasārastriyām}%

 \nonanustubhindent \maintext{kiṃ sadbhāva na vedmi tasya viṣaye kiṃ dveṣya kiṃ tātpriyam |}%

  \maintext{paśyāmi na ca tasya garbhakalayā prāpnoti niḥsaṃśayam}%

 \nonanustubhindent \maintext{māyājālasahasragāpi yuvatī kurvanti bhartā priyam }||\thinspace22:56\thinspace||%


  \maintext{vaiśampāyana uvāca |}%

  \maintext{rājan kiṃ priyam asti arthaparataḥ paśyāmi nānyan nṛpe}%

 \nonanustubhindent \maintext{putrārthaikaprayojanaṃ yuvatayaḥ svāyambhuvoktāmaraiḥ |}%

  \maintext{kāntā nityakalā pravartanakarī dharmasakhāyā satī}%

 \nonanustubhindent \maintext{māyā vāpi karoti sadya manujātyaktānya vā sevate }||\thinspace22:57\thinspace||%


  \maintext{strīsaṅgaṃ parivarjayen narapate āyāsadaṃ duḥkhadam}%

 \nonanustubhindent \maintext{mṛtyudvārabhayākaraṃ viṣagṛham āpatsughorālayam |}%

  \maintext{agniṃ māruta mattavāraṇasamaṃ tasyānugāmī sadā}%

 \nonanustubhindent \maintext{strīhetor hata rāvaṇas tridaśapaindro 'py avasthākṛtaḥ }||\thinspace22:58\thinspace||%


  \maintext{daṇḍakyo hatarāṣṭrapaurasahitaḥ kiṃ bhūya vakṣyāmy aham }||\thinspace22:59\thinspace||%


  \subchptr{vipra-muni-bhikṣu-nirgranthi-parivrājaka-rṣyādayaḥ}%

  \maintext{janamejaya uvāca |}%

  \maintext{vipre kīdṛśalakṣaṇaṃ bhavati bho kīdṛg muniś cocyate}%

 \nonanustubhindent \maintext{tenārthena bhaveta bhikṣu bhagavan nigranthi ko vā dvija |}%

  \maintext{kenārthena bhaved dvijendra bhagavan jñeyaḥ parivrājakaḥ}%

 \nonanustubhindent \maintext{! jñeyāḥ kim ṛṣayaś ca lakṣaṇa muner icchāmi jñātuṃ punaḥ }||\thinspace22:60\thinspace||%


  \maintext{vaiśampāyana uvāca |}%

  \maintext{satyaṃ śaucam ahiṃsatā damaśamau bhūtānukampī sadā}%

 \nonanustubhindent \maintext{ātmārāmajito svadharmanirataḥ sattvastha nityaṃ manaḥ |}%

  \maintext{kāmakrodhayamasvadāranirataḥ saṃtyajya lobhaḥ śanaiḥ}%

 \nonanustubhindent \maintext{evaṃ yaḥ kurute dvijātisuvaraḥ śūdro 'pi yaḥ saṃyamī }||\thinspace22:61\thinspace||%


  \maintext{tasmāc chadmakavarjitaḥ sa bhagavān saṃsārabhībhidyakaḥ}%

 \nonanustubhindent \maintext{yat tat sthānaparaṃ vrajanti puruṣāḥ tasmāt parivrājakaḥ |}%

  \maintext{granthidārasutaṃ dhanaṃś ca virati nirgranthika socyate}%

 \nonanustubhindent \maintext{ramyante ṛṣir āśrame dhṛtimanas tasmād ṛṣiḥ socyate }||\thinspace22:62\thinspace||%


  \maintext{kāyavāṅmanadaṇḍatatparataras te daṇḍikarūcyate}%

 \nonanustubhindent \maintext{saddharmaśravaṇaṃ vadanti śravaṇaḥ saddharmabrahmākṣaraḥ |}%

  \maintext{pāśaprakṣipataṃ paśutvasakalaṃ pāśūpatās te smṛtāḥ}%

 \nonanustubhindent \maintext{vipre pāśupatādibhikṣusakalaṃ pṛṣṭo 'smy ahaṃ lakṣaṇam }||\thinspace22:63\thinspace||%


  \maintext{sarvaṃ tat kathito 'si lakṣaṇa mayā sandhiśvanirnāśanam}%

 \nonanustubhindent \maintext{prajñāsaṃgrahaśītavardhanaparaṃ saṃsāranirmūlanam |}%

  \maintext{}%

 \nonanustubhindent \maintext{etaj jñānaparaṃ prabodham atulaṃ nityaṃ śivaṃ dhāryate }||\thinspace22:64\thinspace||%


\centerline{\maintext{\dbldanda\thinspace iti vṛṣasārasaṃgrahe dvāviṃśatitamo 'dhyāyaḥ\thinspace\dbldanda}}
\translation{Here ends the twenty-third chapter in the Vṛṣasārasaṃgraha.}

  \chptr{trayoviṃśatitamo 'dhyāyaḥ}
\fancyhead[CO]{{\footnotesize\textit{Translation of chapter 23}}}%

  \maintext{janamejaya uvāca |}%

  \maintext{devānāṃ dānavānāṃ ca uttarāraṇim eva ca |}%

  \maintext{vidviṣanti ca te 'nyonyaṃ kāraṇaṃ tasya kīrtaya }||\thinspace23:1\thinspace||%


  \maintext{vaiśampāyana uvāca |}%

  \maintext{pāpapuṇyasvabhāvābhyāṃ devadaityasya bhūpate |}%

  \maintext{dharmapakṣasmṛto devo dānavo 'dharmapakṣataḥ }||\thinspace23:2\thinspace||%


  \maintext{hetunā tena rājendra anyonyaṃ vidviṣanti te |}%

  \maintext{devadveṣṭāsurāḥ sarve vibudhāś cāsuradviṣaḥ }||\thinspace23:3\thinspace||%


  \subchptr{dharmādharmavipakṣatā}%

  \maintext{dharmādharmavipakṣatāṃ śṛṇu parāṃ bhūtānukampodayām}%

 \nonanustubhindent \maintext{satyaṃ śaucam ahiṃsatā damaśamo nirmānam īrṣyāruṣā |}%

  \maintext{tṛṣṇālobharatasya kāmaviṣayaḥ sarvendriyāṇāṃ jayaḥ}%

 \nonanustubhindent \maintext{ādhyātmeṣu ratiḥ prasannamanaso nirdvandvasarvālayaḥ }||\thinspace23:4\thinspace||%


  \maintext{pāpopekṣaṇaśaśvapuṇyamudito dīneṣu kāruṇyatā}%

 \nonanustubhindent \maintext{dānaṃ śīladhṛtikṣamājapatapaḥ svādhyāyamaune ratiḥ |}%

  \maintext{yogābhyāsaratir divaukasagaṇe jñāne ca sāṃkhye tathā}%

 \nonanustubhindent \maintext{akrodhārjavatejayajñam abhayaṃ saṃtoṣa adrohatā }||\thinspace23:5\thinspace||%


  \maintext{tyāgo mārdavahrīr acāpalaratir nyāsābhimāno grahāt}%

 \nonanustubhindent \maintext{maitrībhāvasadārapaiśunamatir brāhmaṇyaśraddhānvitaḥ |}%

  \maintext{etācāra sadā narendra vibudhāḥ saṃkṣepataḥ kīrtitāḥ}%

 \nonanustubhindent \maintext{daityānāṃ śṛṇu kīrtaye svavahito 'sambhāvya teṣāṃ nijam }||\thinspace23:6\thinspace||%


  \maintext{daityāḥ pāparatisvabhāvacapalā nirlajjadarpālasāḥ}%

 \nonanustubhindent \maintext{kāmakrodhavaśāḥ suduṣṭamanasas tṛṣṇādhikā nirdayāḥ |}%

  \maintext{śaucācāravivarjitā gurugirānnānitya kuryuḥ kriyāḥ}%

 \nonanustubhindent \maintext{jīvākarṣaṇajīvanaḥ pratidinaṃ mohāndharāgānvitāḥ }||\thinspace23:7\thinspace||%


  \maintext{nidrā nitya divā prasaktam aśuciḥ sūryodaye supyate}%

 \nonanustubhindent \maintext{āśāpāśaśatair nibaddhahṛdayo hṛtvā parasvaṃ punaḥ |}%

  \maintext{mātsaryāt parapākabhedanirato mūlasya duṣpūratā}%

 \nonanustubhindent \maintext{! nāstīkatvaparāṅganāsvabhirata utkocakāmaḥ sadā }||\thinspace23:8\thinspace||%


  \maintext{devabrāhmaṇa vidviṣanti satataṃ lobhāc ca kāryakriyā}%

 \nonanustubhindent \maintext{dharmaṃ dūṣayate ca mūḍhamanasā āryaṃ ca tīrthaṃ tathā |}%

  \maintext{hantavyāny ahatāś ca manyabahavo visphūrjitam adruvan}%

 \nonanustubhindent \maintext{daityānāṃ kathitaṃ ca cihna katicit sadbhāvataḥ kīrtitam }||\thinspace23:9\thinspace||%


  \maintext{martyeṣv eva narendra mānuṣam abhūd devāsurāṇāṃ nṛpaḥ}%

 \nonanustubhindent \maintext{yo yaṃ proktaḥ svabhāvatām ubhayato mānuṣyaloke tathā |}%

  \maintext{yan me pṛcchitavān narendra kathitaṃ yat tvaṃ purā gopitam}%

 \nonanustubhindent \maintext{vidveṣobhayakāraṇaṃ narapate kiṃ bhūya vakṣyāmy aham }||\thinspace23:10\thinspace||%


  \subchptr{nidrottpattiḥ}%

  \maintext{janamejaya uvāca |}%

  \maintext{asti kautūhalaṃ cānyaṃ pṛcchāmi tvāṃ dvijottama |}%

  \maintext{kathaṃ nidrā samutpannā sarvabhūtavimohanī }||\thinspace23:11\thinspace||%


  \maintext{rātrau prajāyate kasmād divā kasmān na jāyate |}%

  \maintext{kasmāc ca kurute jantor nidrā netrapramīlanam |}%

  \maintext{etan me saṃśayaṃ chindhi sarvajño 'si dvijottama }||\thinspace23:12\thinspace||%


  \maintext{vaiśampāyana uvāca |}%

  \maintext{devī hy eṣā mahābhāgā nidrā netrāśrayā nṛṇām |}%

  \maintext{tasyā vaśaṃ gataṃ sarvaṃ jagatsthāvarajaṅgamam }||\thinspace23:13\thinspace||%


  \maintext{sadevadānavagaṇā gandharvoragarākṣasāḥ |}%

  \maintext{yakṣabhūtapiśācāś ca paśupakṣisarīsṛpāḥ }||\thinspace23:14\thinspace||%


  \maintext{guhyakāś ca mṛgā nāgā kiṃnarā jalajoragāḥ |}%

  \maintext{nidrāvaśagatāḥ sarve pāpmanā tv abhilaṅghitāḥ }||\thinspace23:15\thinspace||%


  \maintext{devadānavakarmānte tasminn amṛtasambhave |}%

  \maintext{mandarotthāpane viṣṇur devāsurasamāgame }||\thinspace23:16\thinspace||%


  \maintext{jāyate vigrahe tv eṣā kṛte hy amṛtamanthane |}%

  \maintext{rajas tamaś cāsuraṃ vai sattvaṃ devakṛtaiḥ śubhaiḥ }||\thinspace23:17\thinspace||%


  \maintext{tataḥ sattvamayī devī rajastamanivāsinī |}%

  \maintext{krodhajā vai sthitā madhye devadānavapakṣayoḥ }||\thinspace23:18\thinspace||%


  \maintext{tām adbhutamayīṃ dṛṣṭvā vismitā devadānavāḥ |}%

  \maintext{tasyāḥ prabhāvābhihatā dudruvas te diśo daśa }||\thinspace23:19\thinspace||%


  \maintext{tatra pītāmbaradharo viṣṇur ekas tu tiṣṭhati |}%

  \maintext{sābhigatvā viśālākṣī nārāyaṇam athābravīt }||\thinspace23:20\thinspace||%


  \maintext{devadānavanāthas tvaṃ tvayi sarvaṃ pratiṣṭhitam |}%

  \maintext{dehi deva mamāvāsaṃ yatrāhaṃ nivase sukham }||\thinspace23:21\thinspace||%


  \maintext{tato nārāyaṇas tuṣṭas tāṃ devīṃ pratyabhāṣata |}%

  \maintext{śarīre mama vastavyaṃ viṣṇur enām athābravīt }||\thinspace23:22\thinspace||%


  \maintext{tatas tāṃ vaiṣṇavaṃ tejaḥ pāpmanā samatiṣṭhata |}%

  \maintext{tataḥ śete sa vaikuṇṭhaḥ pāpmanā tv abhilaṅghitaḥ }||\thinspace23:23\thinspace||%


  \maintext{tasmin śayāne vitrastā devāsuragaṇās tathā |}%

  \maintext{ūcus te paramodvignāḥ śayānaṃ viṣṇum acyutam }||\thinspace23:24\thinspace||%


  \maintext{trātāraṃ nābhigacchāma uttiṣṭhottiṣṭha keśava |}%

  \maintext{tataḥ śaṅkhagadāpāṇir uttiṣṭhata mahābhujaḥ }||\thinspace23:25\thinspace||%


  \maintext{utthitaś ca viśālākṣaḥ pāpmanā tasya pṛṣṭhataḥ |}%

  \maintext{tataḥ sā vigrahavatī sthitā nārāyaṇālaye }||\thinspace23:26\thinspace||%


  \maintext{viṣṇur devāsuragaṇān idaṃ vacanam abravīt |}%

  \maintext{asmākaṃ vai śarīreṣu iyaṃ pāpmā viniḥsṛtā }||\thinspace23:27\thinspace||%


  \maintext{eṣābhisattvārasatā satyena bhaginī mama |}%

  \maintext{viśrutāṃ triṣu lokeṣu tāṃ pūjayatha māṃ yathā }||\thinspace23:28\thinspace||%


  \maintext{tato devāsuragaṇāḥ saptalokāḥ samānuṣāḥ |}%

  \maintext{vibhaktā vaiṣṇavī pāpmā teṣu sarveṣu devatā }||\thinspace23:29\thinspace||%


  \maintext{parvateṣv atha vṛkṣeṣu sāgareṣu saritsu ca |}%

  \maintext{tato nidrāvaśagataṃ jagat sthāvarajaṅgamam }||\thinspace23:30\thinspace||%


  \maintext{eṣotpattiś ca nidrāyā yathā vasati tac chṛṇu |}%

  \maintext{trīṇi sthānāni yasyā vai śarīreṣu śarīriṇām }||\thinspace23:31\thinspace||%


  \maintext{śleṣmapittānilasthāne trīṇi pakṣāṇi vāsinaḥ |}%

  \maintext{tamaḥ śleṣmāśrayā nidrā rajonidrā tu vātikā }||\thinspace23:32\thinspace||%


  \maintext{pittāśrayāṃ smṛtāṃ nidrāṃ sāttvikāṃ viddhi bhūpate |}%

  \maintext{ādityaprabhavaṃ tejas tasmin sattvaṃ pratiṣṭhati }||\thinspace23:33\thinspace||%


  \maintext{nidrā divā na bhavati tasmāt sattvaguṇātmikā |}%

  \maintext{yasmāt somodbhavā nidrā tamāṃsi ca rajāṃsi ca }||\thinspace23:34\thinspace||%


  \maintext{tasmād rātrau bhaven nidrā tāmasī harajātmikā |}%

  \maintext{yadā hi sarvāṅgagatau śrotāṃsi pratipadyate }||\thinspace23:35\thinspace||%


  \maintext{rajas tamaś ca niyatas tadā nidrā pravartate |}%

  \maintext{tamasy ūrdhvagataśroto hy akṣipakṣmāsamāśritā }||\thinspace23:36\thinspace||%


  \maintext{tamaḥ pravartate jantos tatas tv akṣnor nimīlanam |}%

  \maintext{nāsākṣikarṇaśrotāṃsi prayujyante kaphena tu }||\thinspace23:37\thinspace||%


  \maintext{hṛdayaṃ muhyate cāpi tamasā cāvṛtaṃ manaḥ |}%

  \maintext{sparśaṃ na vedayaty eva na śṛṇoti na paśyati }||\thinspace23:38\thinspace||%


  \maintext{nocchvāsayati nāsābhyāṃ vivṛtākṣimukho naraḥ |}%

  \maintext{eṣā nṛṇām antakarī nidrā vai tāmasī smṛtā }||\thinspace23:39\thinspace||%


  \maintext{akarmaṇy apravṛttiś ca mṛtavat svapate kṣitau |}%

  \maintext{nidrotpattiṃ vikāraṃ ca kathito 'smi narādhipa |}%

  \maintext{tasmān nidrāṃ na seveta tamomohapravardhanīm }||\thinspace23:40\thinspace||%


\centerline{\maintext{\dbldanda\thinspace iti vṛṣasārasaṃgrahe nidrotpattis{ }trayoviṃśatimo 'dhyāyaḥ\thinspace\dbldanda}}

  \chptr{caturviṃśatimo 'dhyāyaḥ}
\fancyhead[CO]{{\footnotesize\textit{Translation of chapter 24}}}%

  \trchptr{Chapter Twenty-four}%

  \maintext{janamejaya uvāca |}%

  \maintext{devānāṃ dānavānāṃ ca vaiṣamyāni śrutāni me |}%

  \maintext{nidrāsambhavam āścaryaṃ tvatprasādena veditam }||\thinspace24:1\thinspace||%
\translation{I have heard about the conflicts between the gods and the demons, and learnt, by your kindness, about the miracle that is produced by sleep. }

  \maintext{trailokyavistarāyāmaṃ śrotum icchāmi bho dvija |}%

  \maintext{kasmiṃścin narakaṃ jñeyaṃ pātālaṃ ca dvijottama }||\thinspace24:2\thinspace||%
\translation{Now, I would like to hear about the breadth and length of the three worlds, O Brahmin. Where are the hells and the Pātāla located, O excellent Brahmin. \blankfootnote{24.2 In the light of the following verses, \textit{kasmiṃścid} seems to carry 
  the function of an interrogative {\rm (}\textit{kasmin}{\rm )} here, and the form \textit{narake} 
  in the MSS might be taken as an Aiśa neuter plural {\rm (}for \textit{narakāṇi}{\rm )}, 
  but I have decided to except Naraharinātha's \textit{narakaṃ}.
 }}

  \maintext{saptadvīpaṃ samicchāmi saptasāgaram eva ca |}%

  \maintext{merumūrdhaṃ ca viprendra devālayaṃ nibodha mām }||\thinspace24:3\thinspace||%
\translation{And I want [to learn about] the seven islands and the seven oceans. And teach me about the peak of Mount Meru, O best of Brahmins, the abode of the gods. \blankfootnote{24.3 CHECK samicchāmi
 CHECK mūrdhaṃ as an acc. Or ūrdhvaṃ is meant?
 }}

  \subchptr{trailokyaṃ narakāṇi ca}%

  \maintext{vaiśampāyana uvāca |}%

  \maintext{śṛṇu saṃkṣepato rājan trailokyāyāmavistaram |}%

  \maintext{kālāgniḥ prathamo jñeyaḥ sarvādhastān nareśvara }||\thinspace24:4\thinspace||%
\translation{Vaiśampāyana spoke: Hear about, O king, the breadth and length of the three worlds. The first [level of the universe], beneath everything, is to be known as the fire of [the end of] time {\rm (}kālāgni{\rm )}, O king of the people. }

  \maintext{tasyopari nṛpaśreṣṭha jñeyā narakakoṭayaḥ |}%

  \maintext{rauravādi avīcyantaṃ yātanāsthānam ucyate }||\thinspace24:5\thinspace||%
\translation{Above that, O best of kings, are the divisions of hell to be found. They start with Raurava and end with Avīcī, and they are called the places of torment. \blankfootnote{24.5 On \textit{koṭi}s as divisions of hell, see e.g.\ ŚDhU 7, and also Bhṛgusaṃhitā 36.40 ff.
 See \MITAKSARA: \textit{evaṃ rauravādinarakeṣu}...
 }}

  \subchptr{sapta pātālāḥ}%

  \maintext{upariṣṭāt tu vijñeyāḥ pātālāḥ sapta eva tu |}%

  \maintext{ābhāsatālaḥ prathamaḥ svatālaś ca tataḥ param }||\thinspace24:6\thinspace||%
\translation{Above them are the Pātālas, which are exactly seven in number. The first is Ābhāsatāla, the next one is Svatāla, \blankfootnote{24.6 CHECK Niśv book p. 209 and various lists in Goodall 2004:289-291, fn. 522 {\rm (}Prākhya{\rm )}.
 }}

  \maintext{śītalaś ca gabhastiś ca śarkaraś ca śilātalam |}%

  \maintext{saptamaṃ tu mahātālaṃ śeṣanāgakṛtālayaḥ }||\thinspace24:7\thinspace||%
\translation{[then] Śītala, Gabhasti, Śarkara and Śilātala. The seventh is Mahātāla, the abode of the serpent Śeṣa, \blankfootnote{24.7 Monier-Williams: mahātala; metri causa
 }}

  \maintext{baliś ca daityarājendro rākṣasaś ca viśaṃkhaṇaḥ |}%

  \maintext{ity evam ādayaḥ sarve nāgadānavarākṣasāḥ }||\thinspace24:8\thinspace||%
\translation{[and also of] Bali the Daitya prince and Viśaṃkhaṇa[?] the Rākṣasa. These and all the other Nāgas, Dānavas and Rākṣasas [live in the seven Pātālas]. }

  \subchptr{sapta dvīpāḥ priyavratasutāś ca}%

  \maintext{sapta dvīpās tato jñeyāḥ saptasāgarasaṃvṛtāḥ |}%

  \maintext{priyavratasya putro 'bhūd daśa rājaparākramaḥ  }||\thinspace24:9\thinspace||%
\translation{Then one should learn about the seven islands, which are surrounded by seven oceans. Ten sons of kingly heroism were born to Priyavrata [Manu's son]: \blankfootnote{24.9 See VSS 4.12 for a reference to the myth of Priyavrata dividing the earth into seven parts,
  thus producing the seven seas and the seven islands.
 Note putro for plural. Perhaps the original read putrābhūd {\rm (}with double sandhi{\rm )}?
 }}

  \maintext{agnīdhraś cāgnibāhuś ca medhā medhātithir vasuḥ |}%

  \maintext{jyotiṣmān dyutimān havyaḥ savanaḥ patra eva ca }||\thinspace24:10\thinspace||%
\translation{Agnīdhra, Agnibāhu, Medhas, Medhātithi, Vasu, Jyotiṣmat, Dyutimat, Havya, Savana, and Patra. \blankfootnote{24.10 Agnīdhra is a variant of the form given in Monier-Williams as Āgnīndhra.
 }}

  \maintext{agnibāhuś ca medhā ca patraś caiva trayo janāḥ |}%

  \maintext{saṃsārabhayabhītena mokṣamārgasamāśritāḥ }||\thinspace24:11\thinspace||%
\translation{The three men Agnibāhu, Medhas and Patra resorted to the path of liberation out of their fear of transmigration {\rm (}saṃsāra{\rm )}. }

  \maintext{agnīdhraṃ prathamadvīpe abhyaṣiñcat priyavrataḥ |}%

  \maintext{plakṣadvīpeśvaraṃ cakre nāmnā medhātithiṃ tathā }||\thinspace24:12\thinspace||%
\translation{Priyavrata consecrated Agnidhra [as king of] the first island [Jambudvīpa], and named Medhātithi to be `King of Plakṣadvīpa'. }

  \maintext{vasuś ca śālmalīdvīpe abhiṣikto mahīpatiḥ |}%

  \maintext{jyotiṣmantaṃ kuśadvīpe rājānam abhiṣecayet }||\thinspace24:13\thinspace||%
\translation{Vasu was consecrated as king in Śālmalīdvīpa. He [Priyavrata] consecrated Jyotiṣmat as king in Kuśadvīpa, \blankfootnote{24.13 Note that mahīpatiḥ was probably meant to be the agent of the action, i.e. Priyavrata.
 }}

  \maintext{krauñcadvīpeśvaraṃ cakre dyutimantaṃ nareśvara |}%

  \maintext{śākadvīpeśvaraṃ havyaṃ puṣkare savanaḥ smṛtaḥ }||\thinspace24:14\thinspace||%
\translation{and made Dyutimat the king of Krauñcadvīpa, O king, Havya the king of Śākadvīpa and Savana is said to have been [the king] in Puṣkara[dvīpa]. \blankfootnote{24.14 Note that nareśvara[ḥ] might be the agent of the action, i.e. Priyavrata.
 }}

  \maintext{madhye puṣkaradvīpasya parvato mānasottaraḥ |}%

  \maintext{lokapālāḥ sthitās tatra caturbhiś caturo diśaḥ }||\thinspace24:15\thinspace||%
\translation{On the island of Puṣkara, there is a mountain called Mānasottara. There are ... Lokapālas there ... }

  \maintext{mahāvītaḥ smṛto varṣo dhātakī ca narādhipa |}%

  \maintext{tasya bāhyaḥ samudro 'bhūt svādūdaka iti smṛtaḥ }||\thinspace24:16\thinspace||%
\translation{There is Mahāvīta country there, and Dhātaki[n?], O king! Outside of [Puṣkaradvīpa], an ocean called Sweet-water {\rm (}svādudaka{\rm )} emerged. \blankfootnote{24.16 dhātaki: N. of one of the 2 sons of Vītihotra Praiyavrata {\rm (}king of a Varṣa of Puṣkara-dvīpa{\rm )}, Pur.
 }}

  \maintext{catuḥṣaṣṭi smṛto lakṣo yojanānāṃ narādhipa |}%

  \maintext{puṣkaradvīpam antaś ca kṣīrodo nāma sāgaraḥ }||\thinspace24:17\thinspace||%
\translation{[The extension of this ocean is] 64 lakh yojanas, O king! Within the Puṣkara island, there is an ocean called the Ocean of milk {\rm (}kṣīroda{\rm )}. }

  \maintext{dvātriṃśallakṣavistāraḥ śākadvīpabahirvṛtaḥ |}%

  \maintext{jaladaś ca kumāraś ca sukumāramaṇīcakaḥ }||\thinspace24:18\thinspace||%
\translation{[Its extension is] 32 [yojanas] and it is located around[?] the Śāka island. Jalada, Kumāra, Sukumāra, Maṇīcaka, \blankfootnote{24.18 bahirvahaḥ?? 
 }}

  \maintext{kusumottaramodaś ca saptamaṃ ca mahādrumam |}%

  \maintext{havyaputrāḥ smṛtāḥ sapta varṣanāma tathā smṛtaḥ }||\thinspace24:19\thinspace||%
\translation{Kusuma, Uttaramoda, and Mahādruma are the seven sons of Havya, and the country names [in Śākadvīpa] are the same. }

  \maintext{dvīpāntaṃ dadhimaṇḍodakṣīrodārdhaṃ vinirdiśet |}%

  \maintext{krauñcadvīpasamudrānte sapta varṣās tu te smṛtāḥ }||\thinspace24:20\thinspace||%
\translation{At the [inner] shores of the island, one should point out a half-whey, half-milk ocean. On the seashore of Krauñcadvīpa, these are the seven countries: }

  \maintext{kuśalo manonugaś coṣṇaḥ yāvanaś cāndhakārakaḥ |}%

  \maintext{muniś ca dundubhiś caiva sutā dyutimatas tu vai }||\thinspace24:21\thinspace||%
\translation{Kuśala, Manonuga, Uṣṇa, Yāvana, Andhakāraka, Muni, and Dundubhi, and [these are also the names of] Dyutimat's sons. \blankfootnote{24.21 Note that pāda a is hypermetrical.
 }}

  \maintext{dadhyardhe ghṛtamaṇḍodaḥ kuśadvīpasamāvṛtaḥ |}%

  \maintext{tatrāpi saptavarṣe ca nāmataḥ śṛṇu bhārata }||\thinspace24:22\thinspace||%
\translation{An ocean of half curd, half-scum-of-melted-butter is around the Kuśa island. Hear also the seven counties that are located there by name, O Bhārata! \blankfootnote{24.22 Note °varṣe as neuter plural nominative/accusative.
 }}

  \maintext{udbhimān veṇumāṃś caiva svairannālambano dhṛtiḥ |}%

  \maintext{ṣaṣṭhaḥ prabhākaraś caiva kapilaḥ saptamaḥ smṛtaḥ }||\thinspace24:23\thinspace||%
\translation{Udbhimat?, Dhenumat, Svairanna {\rm (}/Svairatha{\rm )}, Ālambana, Dhṛti, the sixth is Prabhākara, and the seventh Kapila. }

  \maintext{ghṛtamaṇḍas tadardhena tasyānte madirodadhiḥ |}%

  \maintext{samantāc chālmalīdvīpo varṣāḥ saptaiva kīrtitāḥ }||\thinspace24:24\thinspace||%
\translation{... the Ocean of alcohol {\rm (}madirodadhi{\rm )}. around[!] Śālmalīdvīpa[, where] there are said to be seven countries: \blankfootnote{24.24 The term madirodadhi for this ocean seems unique in the VSS.
 }}

  \maintext{śvetaś ca haritaś caiva jīmūto rohitas tathā |}%

  \maintext{vaidyuto mānasaś caiva suprabhaḥ saptamaḥ smṛtaḥ }||\thinspace24:25\thinspace||%
\translation{Śveta, Harita, Jīmūta, Rohita, Vaidyuta, Mānasa, and the seventh, Suprabhaḥ. }

  \maintext{madirodadhito 'rdhena jñeyas tv ikṣurasodadhiḥ |}%

  \maintext{plakṣadvīpo vṛtas tena saptavarṣasamanvitaḥ }||\thinspace24:26\thinspace||%
\translation{... the Ocean of sugar-cane. Plakṣadvīpa with its seven countries is surrounded by it. }

  \maintext{śāntaś ca śiśiraś caiva sukhadānanda eva ca |}%

  \maintext{śivakṣemo dhruvaś caiva sapta medhātitheḥ sutāḥ }||\thinspace24:27\thinspace||%
\translation{Śānta, Śiśira, Sukhada, Ānanda, Śiva, Kṣema and Dhruva: these are Medhātithi's seven sons [and the names of their countries]. \blankfootnote{24.27 Purāṇic Encyclopedia p. 499
  {\rm (} https://www.sanskrit-lexicon.uni-koeln.de/scans/PEScan/2020/web/webtc/servepdf.php?page=499-b {\rm )}:
  ``MEDHĀTITHI I . Grandson of Svāyambhuva Manu.
  Svāyambhuva Manu had two sons named Priyavrata and
  Uttānapāda. Of these Priyavrata married Sarūpā and
  Barhiṣmatī, daughters of Viśvakarmaprajāpati. Medhā-
  tithi was the son born to Priyavrata of Sarūpā. Agnī-
  dhra, and others were the brothers of Medhātithi.
  Medhātithi became the King of Plakṣadvīpa after the
  death of Priyavrata. {\rm (}8th Skandha, Devī Bhāgavata{\rm )}.
  Medhātithi got seven sons named Śāntahaya, Śiśira,
  Sukhodaya, Ānanda, Śiva. Kṣemaka and Dhruva. They
  all became Kings of Plakṣadvīpa. The countries they
  ruled were named after them as Śāntahayavarṣa, Śiśira-
  varṣa, Sukhodayavarṣa, Ānandavarṣa, Śivavarṣa, Kṣema-
  kavarṣa and Dhruvavarṣa. There are seven mountains
  showing the boundaries of these states and they are
  called Gomeda, Cāndra, Nārada, Dundubhi, Somaka,
  Sumana and Vaibhrāja. In these beautiful countries
  and grand mountains live a great many Devas,
  Gandharvas and virtuous men. {\rm (}Chapter 4, Aṃśa 2,
  Viṣṇu Purāṇa{\rm )}.''
 }}

  \maintext{lavaṇodas tu tasyānte jambūdvīpasamāvṛtaḥ |}%

  \maintext{lakṣayojanavistāra upadvīpasamanvitaḥ }||\thinspace24:28\thinspace||%
\translation{At its shores, there is the Salty ocean {\rm (}lavaṇoda{\rm )}, which surrounds[!] Jambudvīpa. Its territory is one lakh yojanas and its contains the following minor islands: }

  \maintext{aṅgadvīpo yavadvīpo malayadvīpa eva ca |}%

  \maintext{śaṅkhadvīpakamudvīpo varāhadvīpa eva ca }||\thinspace24:29\thinspace||%
\translation{Aṅgadvīpa, Yavadvīpa, Malayadvīpa, Śaṅkhadvīpa, Kamudvīpa[?] and Varāhadvīpa, }

  \maintext{siṃha barhiṇadvīpaṃ ca padmaś cakras tathaiva ca |}%

  \maintext{vajraratnākaradvīpo haṃsakaḥ kumudas tathā }||\thinspace24:30\thinspace||%
\translation{Siṃha, Barhiṇadvīpa, Padma, Cakra, Vajraratnākaradvīpa, Haṃsaka, Kumuda, }

  \maintext{lāṅgalo vṛṣadvīpaś ca dvīpo bhadrākaras tathā |}%

  \maintext{candradvīpaś ca sindhuś ca candanadvīpa eva ca |}%

  \maintext{upadvīpasahasrāṇi evamādīni kīrtitam }||\thinspace24:31\thinspace||%
\translation{Lāṅgala, Vṛṣadvīpa, Bhadrākāra, Candradvīpa, Sindhu, Candanadvīpa, and so on so forth. There are said to be thousands of minor islands. \blankfootnote{24.31 Note the discrepancy in the numbers: °sahasrāṇi... °ādīni kīrtitam 
 }}

  \subchptr{agnīdhraputrā jambudvīpe}%

  \maintext{agnīdhro navavarṣeṣu navaputrān asiñcayat |}%

  \maintext{nābhiḥ kiṃpuruṣaś caiva harivarṣa ilāvṛtaḥ }||\thinspace24:32\thinspace||%
\translation{Agnīdhra consecrated [his] nine sons in nine countries. [The names of the countries/sons are:] Nābhi, Kiṃpuruṣa, Hari, Ilāvṛta, }

  \maintext{pañcamaṃ ramyakaṃ varṣaṃ ṣaṣṭhaṃ caiva hiraṇmayam |}%

  \maintext{kuravaḥ saptamo jñeyo bhadrāśvaś cāṣṭamaḥ smṛtaḥ }||\thinspace24:33\thinspace||%
\translation{the fifth, Ramyaka country and the sixth, Hiraṇmaya, }

  \maintext{navamaḥ ketumālo 'bhūn navavarṣāḥ prakīrtitāḥ |}%

  \maintext{himavaddakṣiṇe pārśve varṣo bhāratasaṃjñitaḥ }||\thinspace24:34\thinspace||%
\translation{the seventh, Kurava, the eighth Bhadrāśva, and the ninth was Ketumāla. The nine countries have been taught. }

  \maintext{atrāpi navabhedo 'bhūd bhāratātmajasambhavaḥ |}%

  \maintext{indradvīpaḥ kaśeruś ca tāmravarṇo gabhastimān }||\thinspace24:35\thinspace||%
\translation{South of the Himālaya, there is the country called Bhārata. Again, there emerged a ninefold division there due to Bhārata's sons: }

  \maintext{nāgadvīpas tathā saumyo gāndharvaś cātha vāruṇaḥ |}%

  \maintext{ayaṃ ca navamo dvīpaḥ kumārīdvīpasaṃjñitaḥ |}%

  \maintext{dakṣiṇe hemakūṭasya varṣaḥ kiṃpuruṣaḥ smṛtaḥ }||\thinspace24:36\thinspace||%
\translation{Indradvīpa, Kaśeru, Tāmravarṇa, Gabhastimat, Nāgadvīpa, Saumya, Gāndharva, Vāruṇa, and the ninth island, called Kumāradvīpa. South of Hemakūṭa[?] there is the country called Kiṃpuruṣa. }

  \maintext{niṣadho dakṣiṇapārśve harivarṣa iti smṛtaḥ |}%

  \maintext{merumūle tu rājendra jñeyo varṣa ilāvṛtaḥ }||\thinspace24:37\thinspace||%


  \maintext{uttaraṇeṇa {\rm (}uttareṇa?{\rm )} tu nīlasya varṣa ramyaka ucyate |}%

  \maintext{śveta-uttarato jñeyo varṣaramyahiraṇmayaḥ }||\thinspace24:38\thinspace||%


  \maintext{tasya uttarato jñeyas triśṛṅgavaraparvataḥ |}%

  \maintext{tasya cottarapārśve tu varṣaḥ kuruvale smṛtaḥ }||\thinspace24:39\thinspace||%


  \maintext{pūrvaṃ bhadrāśvato jñeyaḥ ketumālas tu paścime |}%

  \maintext{himaṃvān hemakūṭaś ca niṣadho nīla eva ca }||\thinspace24:40\thinspace||%


  \maintext{śvetaś ca śṛṅgavantaś ca ṣaḍ ete varṣaparvatāḥ |}%

  \maintext{aśītinavatīlakṣaḥ - varṣaparvatam āyatam }||\thinspace24:41\thinspace||%


  \maintext{himavān hemakūṭaś ca niṣadhaś ceti dakṣiṇa |}%

  \maintext{śvetaś caivatriśṛṅgaś ca nīlaś caiva tathottare }||\thinspace24:42\thinspace||%


  \maintext{niṣadho nīlamadhye tu meruḥ śailamanoramaḥ |}%

  \maintext{praviṣṭaṣoḍaśādhas tāṃ caturāśītim ucchṛtaḥ }||\thinspace24:43\thinspace||%


  \maintext{yojanānāṃ sahasrāṇi dvātriṃśad ūrdha ! vistṛtaḥ |}%

  \maintext{brahmāmanovatī nāma pureva satimadhyame }||\thinspace24:44\thinspace||%


  \maintext{devarājo 'marāvatyām agnis tejovatī pure }||\thinspace24:45\thinspace||%


  \maintext{yamaḥ saṃyamanī nāma nityaṃ vasati bhūpate |}%

  \maintext{naiṛtir vasati nityaṃ ramye śuddhavatī pure }||\thinspace24:46\thinspace||%


  \maintext{varuṇo bhogavatyāṃ tu vāyor gandhavatī purī |}%

  \maintext{mahodayāpurī ramyā somasyālayaraṃ smṛtam }||\thinspace24:47\thinspace||%


  \maintext{yaśovatī purī ramyānnityam āste triśūlinaḥ |}%

  \maintext{tatra gaṅgā catuḥbhinnā nipatantī mahītale }||\thinspace24:48\thinspace||%


  \maintext{uttare paścime caiva pūrvadakṣiṇatas tathā |}%

  \maintext{pūrvaṃ gaṅgā sravatyāccālakānandā ca dakṣiṇe }||\thinspace24:49\thinspace||%


  \maintext{śītā paścimagā gaṅgā bhadrasomā tathottare |}%

  \maintext{ṣaṣṭiyojanasāhasraṃ nirālambā nipatya ca }||\thinspace24:50\thinspace||%


  \maintext{bhadrāśvaṃ plāvayitvā tu vanāny upavanāni ca |}%

  \maintext{droṇasthalī girīṇāṃ ca atikramyārṇavaṃ gatā }||\thinspace24:51\thinspace||%


  \maintext{tathaivālakanandā ca gatāśailenanimnagā |}%

  \maintext{gaṅgā bhāratavarṣaṃ ca praviṣṭālavaṇo dadhim }||\thinspace24:52\thinspace||%


  \maintext{plāvayitvā sthalīn sarvān mānuṣākaluṣāpahā |}%

  \maintext{paścimena gatāgaṅgā sītānāmā ca bhārataḥ }||\thinspace24:53\thinspace||%


  \maintext{plāvayet ketumālāṃ ca kṣetraśaivavanasthalīm |}%

  \maintext{atikramyārṇavagatā sthalīdroṇī ca nimnagā }||\thinspace24:54\thinspace||%


  \maintext{bhadrasomanadīty evaṃ plāvayitvottaraṃ kurun |}%

  \maintext{sthalī prasravaṇadroṇīm atikramyārṇavaṃ gatā }||\thinspace24:55\thinspace||%


  \maintext{mero vai dakṣiṇe pārśve jambūvṛkṣaḥ sanātanaḥ |}%

  \maintext{tena nāmāṅkito rājan jambūdvīpa iti śrutam }||\thinspace24:56\thinspace||%


  \maintext{koṭīṣoḍaśabhiś caiva ayutāni trayodaśa |}%

  \maintext{adhordhayāma rājendra kṣityāvaraṇam antataḥ }||\thinspace24:57\thinspace||%


  \maintext{navalakṣādhikaṃ rājan pañcakoṭī mahī smṛtā |}%

  \maintext{yojanānāṃ tu vijñeyaḥ pṛthivyāyām avistarāt }||\thinspace24:58\thinspace||%


  \maintext{svādūdakasya ca bahir lokāloko mahāgiriḥ |}%

  \maintext{kañcanidviguṇābhūmi tasmād giribahi smṛtaḥ }||\thinspace24:59\thinspace||%


  \maintext{tasmād bāhyaḥ samudro bhūd garbhādeti samudrarāṭ |}%

  \maintext{aṣṭāviṃśatikaṃ lakṣaṃ śatalakṣāṇi vistaram }||\thinspace24:60\thinspace||%


  \maintext{etad bhūrlokavistāro hy ata ūrdhvaṃ bhuvaḥ smṛtaḥ |}%

  \maintext{svarlokasya pareṇaiva maharlokam ataḥ param }||\thinspace24:61\thinspace||%
\translation{This is the extent of Bhūrloka. Above it there is Bhuvaḥ, just beyond that Svarloka, and above that Maharloka, }

  \maintext{janarlokas tapaḥ satyaṃ kramaśaḥ parikīrtitam |}%

  \maintext{brahmalokaḥ smṛtaḥ satyaṃ viṣṇulokam ataḥ param }||\thinspace24:62\thinspace||%
\translation{Janaloka, Tapoloka, and Satyaloka, in due order. Satya[loka] is said to be Brahmaloka and above it is located Viṣṇuloka. }

  \subchptr{śivalokaḥ}%

  \maintext{tasmāt pareṇa bodhavyaṃ divyadhyānapuraṃ mahat |}%

  \maintext{sahasrabhaumaprāsādaṃ vaiḍūryamaṇitoraṇam }||\thinspace24:63\thinspace||%
\translation{Beyond that, the great city of divine visions is to be recognized as a thousand-story palace with gates [decorated] with cat's-eye gems and }

  \maintext{nānāratnavicitrāṇi nānābhūtagaṇākulam |}%

  \maintext{sarvakāmasamṛddhāni pūrṇaṃ tatra manoharaiḥ }||\thinspace24:64\thinspace||%
\translation{coloured with different kinds of precious stones, inhabited by different troops of beings. That place is full of charming riches of all desires. \blankfootnote{24.64 Note \textit{samṛddhāni} as a plural instrumental of \textit{samṛddhi}?.
 }}

  \maintext{tatra siṃhāsane divye sarvaratnavibhūṣite |}%

  \maintext{tatrāste bhagavān rudraḥ somāṅkitajaṭādharaḥ }||\thinspace24:65\thinspace||%
\translation{There, on a divine throne which is ornamented with all kinds of precious stones, the Lord Rudra is sitting, the one who wears his matted hair marked with the Moon, }

  \maintext{tryakṣas tribhuvanaśreṣṭhas triśūlī tridaśādhipaḥ |}%

  \maintext{devyā saha mahābhāgo gaṇaiś ca parivāritaḥ }||\thinspace24:66\thinspace||%
\translation{the one with three eyes, chief of the three worlds, holding a trident, the ruler of the thirty [gods], together with Devī, he the illustrious one, surrounded by the Gaṇas, }

  \maintext{skandanandipurogaś ca gaṇakoṭīśatākulaḥ |}%

  \maintext{anekarudrakanyābhī rūpiṇībhir alaṅkṛtaḥ }||\thinspace24:67\thinspace||%
\translation{with Skanda and Nandi standing in front of him, in a crowd of a hundred lakh of Gaṇas, embellished with many beautiful Rudra girls. }

  \maintext{tatra puṇyanadī sapta sarvapāpāpanodanī |}%

  \maintext{suvarṇavālukā divyā ratnapāṣāṇaśobhitā }||\thinspace24:68\thinspace||%
\translation{There are seven sacred and divine rivers there that drive away all sins, with golden sandbanks, embellished with gems for rocks: }

  \maintext{pāvanī ca vareṇyā ca varārhā varadā varā |}%

  \maintext{vareśā varabhadrā ca suprasannajalā śivā }||\thinspace24:69\thinspace||%
\translation{Pāvanī, Vareṇyā, Varārhā, Varadā, Varā, Vareśā, and Varabhadrā. The waters of these auspicious rivers are crystal clear. }

  \maintext{anekakusumārāmā ratnapuṣpaphaladrumāḥ |}%

  \maintext{anekaratnaprākārā yojanāyutam ucchritāḥ }||\thinspace24:70\thinspace||%
\translation{There are gardens with numerous flowers, and trees with gems for flowers and fruits, with fences made of different kinds of precious stones. They are ten thousand {\rm (}\textit{ayuta}{\rm )} \textit{yojana}s in extent. }

  \maintext{ahiṃsāsatyaniratāḥ kāmakrodhavivarjitāḥ |}%

  \maintext{dhyānayogaratā nityaṃ tatra modanti te narāḥ  }||\thinspace24:71\thinspace||%
\translation{People there rejoice delighting in non-violence and truthfulness, avoiding lust and anger, constantly practising yogic meditation. }

  \maintext{tatra gomātaras sarvā nivasanti yatavratāḥ |}%

  \maintext{golokaḥ śivalokaś ca eka eva vidhīyate }||\thinspace24:72\thinspace||%
\translation{All the cow-mothers live there, practising observances. Goloka and Śivaloka are established as being one and the same. }

  \subchptr{śāstravarṇanā}%

  \maintext{abhyantare tat kathito 'dya sāraṃ}%

 \nonanustubhindent \maintext{kim anya rājan kathayāmi sāram |}%

  \maintext{jñānārṇavaṃ kīrtita dharmasāram}%

 \nonanustubhindent \maintext{purāṇavedopaniṣatsusāram }||\thinspace24:73\thinspace||%
\translation{Now the essence has been taught with respect to the inner part[?]. What other essence shall I teach, O king? The essence of Dharma, the ocean of knowledge, has been taught, the very essence of the Purāṇas, the Vedas, and the Upaniṣads. }

  \maintext{yathā hi rājā parivāramadhye}%

 \nonanustubhindent \maintext{yathāntavartī bahivartin eva |}%

  \maintext{bhuñjanti bhogān satatāntavartī}%

 \nonanustubhindent \maintext{kleśādhikaṃ nitya bahiḥsthitānām }||\thinspace24:74\thinspace||%
\translation{For as in a family, ... [there are those] dwelling inside and outside. Those who dwell inside enjoy themselves all the time, and for the outsiders, [there is always just] an abundance of pain. }

  \maintext{yathaiva rājā kariṇo 'ntadantam}%

 \nonanustubhindent \maintext{bhuñjanti bhogān satataṃ narendra |}%

  \maintext{yudhyeta rājā bahirdantabhogair}%

 \nonanustubhindent \maintext{yadantaraṃ paśya samānajātam }||\thinspace24:75\thinspace||%
\translation{Just like the inner teeth of an elephant, O king, enjoys the food all the time, O king, a ruler will fight with [the elephant's] outer teeth [i.e.\ use and damage the tusk], The difference between the two, see, of a similar kind[???]. }

  \maintext{na dānatulyaṃ tv abhayapradasya}%

 \nonanustubhindent \maintext{na yajñatulyaṃ jita-indriyasya |}%

  \maintext{na cārthatulyaṃ jitakāminaś ca}%

 \nonanustubhindent \maintext{na dharmatulyaṃ damakāmitasya }||\thinspace24:76\thinspace||%
\translation{There is nothing like donations for somebody who offers freedom from danger. There is nothing like sacrifices for him who has conquered his senses. There is nothing like wealth for him who has conquered his senses. There is nothing like Dharma[?] for him whose desires are tamed[?]. }

  \maintext{bahvantaraṃ naiva hi dharmayoś ca}%

 \nonanustubhindent \maintext{kleśādhikaṃ bāhyaphalālpasāram |}%

  \maintext{yad atra dharmaṃ phalanaiṣṭhikasya}%

 \nonanustubhindent \maintext{na tulya koṭīśatayājināpi }||\thinspace24:77\thinspace||%
\translation{For there are no big differences between Dharma and a-Dharma. }

  \maintext{etat pavitraṃ paramaṃ sadharmam}%

 \nonanustubhindent \maintext{purā yathoktaṃ parameśvareṇa |}%

  \maintext{mayāpi tulyaṃ kathitaṃ yathāvat}%

 \nonanustubhindent \maintext{purāṇavedopaniṣatsusāram }||\thinspace24:78\thinspace||%
\translation{As this sacred and superior true [sat-?] Dharma was in the past taught by Parameśvara, I too have taught it in the same form, as it is, the very essence of the Purāṇas, the Vedas and the Upaniṣads. }

  \maintext{sadojasaubhāgyam atīva medhā}%

 \nonanustubhindent \maintext{nirutsukaḥ saumyam anuttamaṃ ca |}%

  \maintext{suputrapautraṃ na vichinnagotram}%

 \nonanustubhindent \maintext{bhavanti vidyādharalokapūjyam }||\thinspace24:79\thinspace||%


  \maintext{yaśaśriyaṃ kīrtir atīva tejo}%

 \nonanustubhindent \maintext{janapriyo dhānyadhanāyuvṛddhim |}%

  \maintext{prabodhaprajñārujadharmavṛddhim}%

 \nonanustubhindent \maintext{bhavanti taṃ śāstrasadābhiyogī }||\thinspace24:80\thinspace||%


  \maintext{yaśasvinī āryasuvarṇaśṛṅgī}%

 \nonanustubhindent \maintext{vedāntavipradvijagāyaneṣu |}%

  \maintext{dattvā phalaṃ tīrtham anuttameṣu}%

 \nonanustubhindent \maintext{śṛṇvanti ye tasya bhavet sapuṇyam }||\thinspace24:81\thinspace||%


  \maintext{daśādhikaṃ vācayituś ca puṇyam}%

 \nonanustubhindent \maintext{śatādhikaṃ yaḥ paṭhati prabhāte |}%

  \maintext{sahasraśaḥ pustakṛtasya puṇyam}%

 \nonanustubhindent \maintext{pare 'bhyaste kīrtayate 'yutāni }||\thinspace24:82\thinspace||%


  \maintext{adhītya yasyoragataṃ suśāstram}%

 \nonanustubhindent \maintext{samastam adhyāyam anukramena |}%

  \maintext{daśāyutāṅgo dadatuś ca puṇyam}%

 \nonanustubhindent \maintext{labhaty asaṃdigdhayathādinaikaṃ }||\thinspace24:83\thinspace||%


  \maintext{yenedaṃ śāstrasāram avikalamanasā yo 'bhyaset tatprayatnāt}%

 \nonanustubhindent \maintext{vyakto 'sau siddhayogī bhavati ca niyataṃ yas tu cittaprasannaḥ |}%

  \maintext{pitryaṃ yo gītapūrvaṃ pratidina śataśa uddhriyante ca sarve}%

 \nonanustubhindent \maintext{ātmānaṃ nirvikalpaṃ śivapadam asamaṃ prāpnuvantīha sarve }||\thinspace24:84\thinspace||%


\centerline{\maintext{\dbldanda\thinspace iti vṛṣasārasaṃgrahe śāstravarṇanā nāma caturviṃśatitamo 'dhyāyaḥ samāptaḥ\thinspace\dbldanda}}

\centerline{\maintext{\dbldanda\thinspace vṛṣasārasaṃgrahaḥ samāpta iti\thinspace\dbldanda}}










%\mychapter{Appendices}
%\addcontentsline{toc}{chapter}{\hspace{1.4em}Appendices}

%\thispagestyle{empty}

%\section{passeges from part two}

%\vfill
%\pagebreak







\mychapter{Symbols and Abbreviations}
%\addcontentsline{toc}{chapter}{\hspace{1.4em}Abbreviations and Bibliography}
\thispagestyle{empty}

\section{Symbols}

\begin{description}

\item[\similar]

\item[\compare]

\item[ = ]

\end{description}


\section{Abbreviations}

\begin{description}

\item[CUDL] = University of Cambridge Digital Library  
  (https://cudl.lib.cam.ac.uk)
  
\item[\fol]

\item[\fols]

\item[MGMCP]

\item[MGMPP]

\item[MS(S)] = manuscript(s)

%\nocite{Siddham} 
\item[Siddham] = Siddham, the Asia Inscriptions Database:
https://siddham.network 

\item[ŚDhŚ] = \SDhS

\item[ŚDhU] = \SDhU
  
\item[VSS] = asdfadfasdfadsa
  
\end{description}


TO BE SUPPLIED

\begin{itemize}
\item
  Balogh 2018? ON THE SAME TOPIC
\item
  Ranjan Sen 2006. `Vowel-weakening before muta cum liquidā sequences in
  Latin. A problem of syllabification?' In: Oxford University Working
  Papers in Linguistics, Philology \& Phonetics 11: 143-61.
\end{itemize}
\vfill
\pagebreak
%(\cite[see][13 ff.]{OlivelleGrhastha})



\mychapter{References}
\section{Primary Sources}

\leftskip2em
\parindent-2em
\ 


\skttitle{Arthaśāstra}{Arthasastra}: see \mycite{Arthasastra1969}

\skttitle{Uttarottara}{Uttarottara}: see \verify

\skttitle{Umāmaheśvarasaṃvāda}{Umamahesvarasamvada}: see \verify

\skttitle{Ṛgveda-khila}{Rgvedakhila}: see \mycite{RgvedaKhila}

\skttitle{Kūrmapurāṇa}{Kurmapurana}: see \mycite{Kurmapurana}

\skttitle{Padmapurāṇa}{Padmapurana}: see \verify

\skttitle{Buddhacarita}{Buddhacarita}: see \verify

\Bodhisattvabhumi: see \verify

\skttitle{Brahmāṇḍapurāṇa}{Brahmandapurana}: see \verify

\BhG: see \mycite{MahabharataCriticalEd}  \verify

\Manu: see \mycite{Manu1972}

\MBh: see \mycite{MahabharataCriticalEd}

\MahaSubhS: see \mycite{Mahasubhasitasamgraha}

\skttitle{Mātaṅgalīlā}{Matangalila}: see \mycite{Matangalila}

\YS: see \verify

\Raghu: see %\mycite{Goodall...}

\RasarnavaSK: see \mycite{Rasarnavasudhakara}

\skttitle{Vāgmatīmāhātmyapraśaṃsā}{Vagmatimahatmyaprasamsa}:

\VajasaneyiS: see \mycite{VajasaneyiS}

\skttitle{Viṣṇudharmottara}{Visnudharmottara}:

\skttitle{Viṣṇudharma}{Visnudharma}: see \mycite{VisnudharmaGrunendahl}


\skttitle{Viṣṇupurāṇa}{Visnupurana}: see \mycite{Visnupurana_critical}

OTHER PURANAS



% print biblio:
\bibliographystyle{csaba2022}
%\renewcommand*{\bibname}{References huhu}

%% Add titles without referencing them:
%\nocite{MahabharataCriticalEd}
%\nocite{Kurmapurana}

\label{bibliography}

CHANGE repeated authornames with
\rule[.2em]{1.5cm}{.5pt}
% or rather with 3 em dashes

\bibliography{vssbiblio}



\label{index}
%\renewcommand*{\indexname}{{\normalfont Index}}
\addcontentsline{toc}{chapter}{\protect\textbf{\numberline{}Index}}
\printindex

%\listoftodos
%TODOS
%fix nirṛta ligature

\end{document}
