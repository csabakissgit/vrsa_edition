\fejno=0\versno=0
\centerline{\Huge\devanagarifontbold वृषसारसंग्रहः  }

 
{\vrule depth10pt width0pt}

\vers

\versno=0\fejno=13
\thispagestyle{empty}

\centerline{\Large\devanagarifontbold [   त्रयोदशमो ऽध्यायः  ]}{\vrule depth10pt width0pt} \fancyhead[CE]{{\footnotesize\devanagarifont वृषसारसंग्रहे  }}
\fancyhead[CO]{{\footnotesize\devanagarifont त्रयोदशमो ऽध्यायः  }}
\fancyhead[LE]{}
\fancyhead[RE]{}
\fancyhead[LO]{}
\fancyhead[RO]{}
\szam\bek



\alalfejezet{कथं सुखोपायो न क्रियते}
{\devanagarifont देव्युवाच {\dandab}\dontdisplaylinenum  }%
 
{\devanagarifont अहिंसातिथ्यकानां च श्रुतो धर्मः सुविस्तरः \thinspace{\danda} \dontdisplaylinenum }%
     \var{{\devanagarifontvar \numemph\va\textbf{॰तिथ्य॰}\lem \msCb\msNa\Ed, ॰तिथ्या॰ \msCa\ \unmetr}}% 

%Verse 13:1

{\devanagarifont किं न कुर्वन्ति मनुजाः सुखोपायं महत्फलम् {॥ १३:१॥} \veg\dontdisplaylinenum }%
     \var{{\devanagarifontvar \numnoemph\vd\textbf{॰पायं}\lem \msCb\msNa\Ed, ॰\uncl{प}यम् \msCa}}% 

{\devanagarifont स्वशरीरे स्थितो यज्ञः स्वशरीरे स्थितं तपः \thinspace{\dandab} \dontdisplaylinenum }%
     \var{{\devanagarifontvar \numemph\va\textbf{॰शरीरे}\lem \msCa\msCb\msNa, ॰शरीर॰ \Ed}}% 

%Verse 13:2

{\devanagarifont स्वशरीरे स्थितं तीर्थं श्रुतो विस्तरतो मया {॥ १३:२॥} \veg\dontdisplaylinenum }%
     \var{{\devanagarifontvar \numnoemph\vd\textbf{विस्तरतो}\lem \msCa\msNa\Ed, \om\ \msCb}}% 

{\devanagarifont किमर्थं भगवन्ब्रूहि सुखोपायं महत्फलम् \thinspace{\dandab} \dontdisplaylinenum }%
     \var{{\devanagarifontvar \numemph\va\textbf{॰र्थं}\lem \msCa\msNa\Ed, ॰र्थ \msCb\oo 
\textbf{भगवन्ब्रू॰}\lem \msCb\Ed, भगवान्ब्रू॰ \msCa\msNa}}% 

%Verse 13:3

{\devanagarifont किं निवृत्तास्तु देवेश ऋषिदैवतमानुषाः {॥ १३:३॥} \veg\dontdisplaylinenum }%
 
{\devanagarifont महादेव उवाच {\dandab}\dontdisplaylinenum  }%
     \var{{\devanagarifontvar \numemph\vo\textbf{महादेव}\lem \msCa\msCb, देवेश \msNa, भगवान् \Ed}}% 

{\devanagarifont अद्य पृष्टेन कथितं गोपितं मयि सुन्दरि \thinspace{\danda} \dontdisplaylinenum }%
     \var{{\devanagarifontvar \numnoemph\vb\textbf{मयि}\lem \msCb\msNa, \uncl{म}यि \msCa, ऋषि \Ed}}% 

%Verse 13:4

{\devanagarifont मानुषाणां हितार्थाय तव च वरवर्णिनि {॥ १३:४॥} \veg\dontdisplaylinenum }%
 
{\devanagarifont अद्यप्रभृति देवेशि ख्यातिर्लोके भविष्यति \thinspace{\dandab} \dontdisplaylinenum }%
     \var{{\devanagarifontvar \numemph\vb\textbf{ख्यातिर्लो॰}\lem \msCa\msCb\Ed, ख्याति लो॰ \msNa}}% 

%Verse 13:5

{\devanagarifont धन्या एवं चरिष्यन्ति अधन्या न रमन्ति तम् {॥ १३:५॥} \veg\dontdisplaylinenum }%
 
{\devanagarifont त्रिगुणेन तु बन्धेन बद्धपाशदृढेन तु \thinspace{\dandab} \dontdisplaylinenum }%
     \var{{\devanagarifontvar \numemph\vb\textbf{बद्ध॰}\lem \msCa\msNa, बद्धः \msCb, बद्धा॰ \Ed}}% 

%Verse 13:6

{\devanagarifont तेनार्थेन रमन्त्यत्र जानन्तो ऽपि हि मोहिताः {॥ १३:६॥} \veg\dontdisplaylinenum }%
     \var{{\devanagarifontvar \numnoemph\vd\textbf{हि}\lem \msCa\msCb\msNa, वि॰ \Ed}}% 


\alalfejezet{त्रिगुणबन्धः}
{\devanagarifont देव्युवाच {\dandab}\dontdisplaylinenum  }%
 
{\devanagarifont किं वा त्रिगुणबन्धेति ब्रूहि संशयछेदक \thinspace{\danda} \dontdisplaylinenum }%
     \var{{\devanagarifontvar \numemph\va\textbf{॰बन्धेति}\lem \msCa\msNa\Ed, ॰बन्धेन \msCb}}% 
    \var{{\devanagarifontvar \numnoemph\vb\textbf{॰छेदक}\lem \msCa\msCb\Ed, ॰छेदकः \msNa}}% 

%Verse 13:7

{\devanagarifont अद्यापि मम देवेश मोहोत्पन्नस्त्रिबन्धनैः {॥ १३:७॥} \veg\dontdisplaylinenum }%
 
{\devanagarifont भगवानुवाच {\dandab}\dontdisplaylinenum  }%
 
{\devanagarifont प्राकृतं वैकृतं चैव दक्षिणाबन्धमेव च \thinspace{\danda} \dontdisplaylinenum }%
     \var{{\devanagarifontvar \numemph\va\textbf{प्राकृतं वै॰}\lem \msCa\msNa\Ed, प्राकृम्वै॰ \msCb}}% 

%Verse 13:8

{\devanagarifont एतेनैव तु बन्धेन बद्धाः वर्णाश्रमाः सदा {॥ १३:८॥} \veg\dontdisplaylinenum }%
 
{\devanagarifont ज्ञानहीना निवर्तन्ते परमं प्राप्य तत्पदम् \thinspace{\dandab} \dontdisplaylinenum }%
     \var{{\devanagarifontvar \numemph\vb\textbf{तत्पदम्}\lem \msCa\msCb\msNa, तत्परं \Ed}}% 

{\devanagarifont इष्टस्त्रीपुत्रभृत्यार्थे धनधान्यसमुच्चये  \danda\dontdisplaylinenum }%
     \var{{\devanagarifontvar \numnoemph\vc\textbf{इष्टस्त्रीपुत्रभृत्यार्थे}\lem \msCa\msCb\msNa, 
इष्टस्त्रीषु निवर्तन्ते \Ed}}% 
    \var{{\devanagarifontvar \numnoemph\vd\textbf{॰च्चये}\lem \msCa\msNa\Ed, ॰चये \msCb\ \unmetr}}% 

%Verse 13:9

{\devanagarifont स्नेहादाकृष्टमनसां बन्धः प्राकृत उच्यते {॥ १३:९॥} \veg\dontdisplaylinenum }%
     \var{{\devanagarifontvar \numnoemph\ve\textbf{॰कृष्ट}\lem \msCa\msCb\msNa, ॰कृष्य \Ed}}% 

{\devanagarifont योगयुक्तेन मनसा यद्यदैश्वर्यमाप्यते \thinspace{\dandab} \dontdisplaylinenum }%
 
%Verse 13:10

{\devanagarifont तदा वैकृतबन्धं तु यदि तत्रानुरज्यते {॥ १३:१०॥} \veg\dontdisplaylinenum }%
     \var{{\devanagarifontvar \numemph\vc\textbf{तदा}\lem \conj, तमा॰ \msCa\msNa, तमो \msCb, तच्च \Ed\oo 
\textbf{॰बन्धं तु}\lem \msCa\msCb\msNa, ॰बन्धस्तु \Ed}}% 

{\devanagarifont आरामोद्यानवापीषु दानक्रतुफलेषु च \thinspace{\dandab} \dontdisplaylinenum }%
 
%Verse 13:11

{\devanagarifont आसक्तमनसो वासो दक्षिणाबन्ध कथ्यते {॥ १३:११॥} \veg\dontdisplaylinenum }%
     \var{{\devanagarifontvar \numemph\vc\textbf{॰मनसो वासो}\lem \msCa\msCb\msNa, ॰मनसा वाचा \Ed}}% 
    \var{{\devanagarifontvar \numnoemph\vd\textbf{॰बन्ध}\lem \corr, ॰बन्धः \msCa\msCb\msNa\Ed\ \unmetr}}% 

{\devanagarifont अनेनैव तु पाशेन बद्धो वानरवद्यथा \thinspace{\dandab} \dontdisplaylinenum }%
     \var{{\devanagarifontvar \numemph\va\textbf{पाशेन}\lem \msCb\msNa\Ed, \uncl{पा}\lac न \msCa}}% 
    \var{{\devanagarifontvar \numnoemph\vb\textbf{बद्धो}\lem \msCa\msCb\msNa, बद्धा \Ed}}% 

%Verse 13:12

{\devanagarifont मोक्षितुं न च शक्नोति इतश्चेतश्च धावति {॥ १३:१२॥} \veg\dontdisplaylinenum }%
 
{\devanagarifont देवासुरमनुष्येषु तिर्येषु नरकेषु च \thinspace{\dandab} \dontdisplaylinenum }%
 
%Verse 13:13

{\devanagarifont भ्रमते चक्रयन्त्रवद् यावत्तत्त्वं न विन्दति {॥ १३:१३॥} \veg\dontdisplaylinenum }%
     \var{{\devanagarifontvar \numemph\vcd\textbf{भ्रमते चक्रयन्त्रवद्याव॰}\lem \msCa\msNa, 
भ्रमते  चक्रवत्तावद्याव॰ \msCb 
भ्रमन्ते चक्रयन्त्रेव याव॰ \Ed}}% 

{\devanagarifont गर्भवासपरिक्लेशो जन्ममृत्युः पुनः पुनः \thinspace{\dandab} \dontdisplaylinenum }%
     \var{{\devanagarifontvar \numemph\va\textbf{॰क्लेशो}\lem \msCa\msCb\msNa, ॰क्लेशौ \Ed}}% 
    \var{{\devanagarifontvar \numnoemph\vb\textbf{॰मृत्युः}\lem \msCa\msNa, ॰मृत्यु \msCb\Ed}}% 

%Verse 13:14

{\devanagarifont व्याधिशोकभयायासचिन्तया जरया हतः {॥ १३:१४॥} \veg\dontdisplaylinenum }%
     \var{{\devanagarifontvar \numnoemph\vc\textbf{व्याधि॰}\lem \msCa\msCb\msNa, व्याधिः \Ed\oo 
\textbf{॰भया॰}\lem \msCa\msCb\Ed, ॰सया॰ \msNa}}% 


\alalfejezet{गर्भोत्पत्तिः}
{\devanagarifont देव्युवाच {\dandab}\dontdisplaylinenum  }%
 
{\devanagarifont गर्भोत्पत्तिः कथं देव योगी लभति कीदृशीम् \thinspace{\danda} \dontdisplaylinenum }%
     \var{{\devanagarifontvar \numemph\va\textbf{॰पत्तिः}\lem \msCa\msCb\Ed, ॰पत्तिं \msNa\oo 
\textbf{देव}\lem \msCapcorr\msCb\msNa\Ed, देवि \msCaacorr}}% 
    \var{{\devanagarifontvar \numnoemph\vb\textbf{॰दृशीम्}\lem \msCa\msNa\Ed, ॰दृशम् \msCb}}% 

%Verse 13:15

{\devanagarifont कीदृशं लभते गर्भं श्रोतुं नः प्रतिदीयताम् {॥ १३:१५॥} \veg\dontdisplaylinenum }%
     \var{{\devanagarifontvar \numnoemph\vc\textbf{गर्भं}\lem \msCa\msCbpcorr\msNa, ग\uncl{र्भां} \msCbacorr, गर्भः \Ed}}% 
    \var{{\devanagarifontvar \numnoemph\vd\textbf{प्रतिदीयताम्}\lem \msCa\msCb\msNa, प्रत्युदीर्यताम् \Ed}}% 

{\devanagarifont भगवानुवाच {\dandab}\dontdisplaylinenum  }%
 
{\devanagarifont शृणु देवि प्रवक्ष्यामि गर्भोत्पत्तिं यथाक्रमम् \thinspace{\danda} \dontdisplaylinenum }%
     \var{{\devanagarifontvar \numemph\vb\textbf{॰पत्तिं य॰}\lem \msCa\msNa, ॰पत्ति य॰ \msCb, ॰पत्तिर्य॰ \Ed}}% 

%Verse 13:16

{\devanagarifont यथा संशयविच्छेदं लभिष्यसि वरानने {॥ १३:१६॥} \veg\dontdisplaylinenum }%
     \var{{\devanagarifontvar \numnoemph\vc\textbf{॰च्छेदं}\lem \msCa\msNa\Ed, ॰च्छेद \msCb}}% 
    \var{{\devanagarifontvar \numnoemph\vd\textbf{लभिष्यसि}\lem \msCa\msCb\msNa, भविष्यसि \Ed}}% 

{\devanagarifont अक्षरात्प्रभवो ब्रह्मा कर्म ब्रह्मसमुद्भवम् \thinspace{\dandab} \dontdisplaylinenum }%
     \var{{\devanagarifontvar \numemph\vb\textbf{ब्रह्म॰}\lem \msCa\msCb, ब्रह्मा॰ \msNa, वद्ध॰ \Ed}}% 

%Verse 13:17

{\devanagarifont कर्मतो यज्ञप्रभवो यज्ञतो धूमसम्भवः {॥ १३:१७॥} \veg\dontdisplaylinenum }%
 
{\devanagarifont धूम्रादभ्राणि जायन्ते अभ्रात्पर्जन्यसम्भवः  \thinspace{\dandab} \dontdisplaylinenum }%
     \var{{\devanagarifontvar \numemph\va\textbf{धूम्रादभ्राणि}\lem \msCb, धूमादभ्राणि \msCa, 
धूम्राददभ्राणि \msNa}}% 
    \lacuna{\devanagarifontsmall \vab {\englishfont missing in \Ed} }%
  
%Verse 13:18

{\devanagarifont पर्जन्यादन्नमुत्पत्तिरन्नाद्भूतानि जज्ञिरे {॥ १३:१८॥} \veg\dontdisplaylinenum }%
     \var{{\devanagarifontvar \numnoemph\vd\textbf{जज्ञिरे}\lem \msCa\msNa\Ed, यज्ञिरे \msCb}}% 

{\devanagarifont अन्नाद्रससमुत्पत्ती रसाच्छोणितसम्भवः \thinspace{\dandab} \dontdisplaylinenum }%
     \var{{\devanagarifontvar \numemph\va\textbf{॰त्पत्ती}\lem \msCa\msCb\msNa, ॰त्पत्ति \Ed}}% 

%Verse 13:19

{\devanagarifont शोणितान्मांसमुत्पत्तिर्मांसाद्मेदसमुद्भवः {॥ १३:१९॥} \veg\dontdisplaylinenum }%
     \var{{\devanagarifontvar \numnoemph\vc\textbf{॰तान्मांस॰}\lem \msCa\msNa\Ed, ॰तान्मास॰ \msCb}}% 
    \var{{\devanagarifontvar \numnoemph\vcd\textbf{॰त्पत्तिर्मांसा॰}\lem \msCa\msNa, ॰त्पत्तिर्मासा॰ \msCb, ॰त्पत्ति मांसा॰ \Ed}}% 
    \var{{\devanagarifontvar \numnoemph\vd\textbf{॰दसमुद्भ॰}\lem \msNa\Ed, ॰दस्समुद्भ॰ \msCa, ॰दः समुद्भ॰ \msCb}}% 

{\devanagarifont मेदसो ऽस्थीनि जायन्ते अस्थिभ्यो मज्जसम्भवः \thinspace{\dandab} \dontdisplaylinenum }%
 
%Verse 13:20

{\devanagarifont मज्जायास्तु भवेच्छुक्रं नरः शुक्रसमुद्भवः {॥ १३:२०॥} \veg\dontdisplaylinenum }%
     \var{{\devanagarifontvar \numemph\vc\textbf{मज्जाया॰}\lem \msCa\msNa\Ed, मज्जया॰ \msCb}}% 

{\devanagarifont शुक्रशोणितसंयोगाद्गर्भोत्पत्तिस्ततः स्मृता \thinspace{\dandab} \dontdisplaylinenum }%
     \var{{\devanagarifontvar \numemph\vb\textbf{स्मृता}\lem \msCa\msCb\msNapcorr, स्मृतः \msNaacorr\Ed}}% 

%Verse 13:21

{\devanagarifont अग्निसोमात्मकं देवि शरीरं द्वयधातुतः {॥ १३:२१॥} \veg\dontdisplaylinenum }%
     \var{{\devanagarifontvar \numnoemph\vd\textbf{शरीरं}\lem \msCa\msCb\msNa, शरीर॰ \Ed}}% 

{\devanagarifont सोमधातु स्मृतं शुक्रमग्निधातु रजः स्मृतम् \thinspace{\dandab} \dontdisplaylinenum }%
     \var{{\devanagarifontvar \numemph\vab\textbf{॰क्रमग्नि॰}\lem \msCa\msNa\Ed, ॰क्रमार्ग्ग॰ \msCb}}% 
    \var{{\devanagarifontvar \numnoemph\vb\textbf{रजः}\lem \msCa\msCb\msNa, रज \Ed\oo 
\textbf{स्मृतम्}\lem \msCa\msCb\Ed, स्मृतः \msNa}}% 

%Verse 13:22

{\devanagarifont अग्निसोमाश्रयं देवि शरीरमिति संज्ञितम् {॥ १३:२२॥} \veg\dontdisplaylinenum }%
     \var{{\devanagarifontvar \numnoemph\vd\textbf{संज्ञितम्}\lem \msCb\msNa\Ed, सं\uncl{ज्ञि}\lac\ \msCa}}% 

{\devanagarifont मासे मासे ऋतुः स्त्रीणां भवतीह न संशयः \thinspace{\dandab} \dontdisplaylinenum }%
     \var{{\devanagarifontvar \numemph\va\textbf{मासे मासे ऋतुः}\lem \msCa, मासे मासे ऋतु \msCb\msNa, मासी मासी ऋतुः \Ed}}% 

%Verse 13:23

{\devanagarifont ऋतुकाले प्रसर्पेत न सुखार्थं वरानने {॥ १३:२३॥} \veg\dontdisplaylinenum }%
     \var{{\devanagarifontvar \numnoemph\vc\textbf{॰सर्पेत}\lem \msCb\msNa, ॰स\uncl{र्प्पे}त \msCa, ॰सर्प्येत \Ed}}% 

{\devanagarifont पुत्रकामः प्रयुञ्जीत धर्मार्थं च यशस्विनि \thinspace{\dandab} \dontdisplaylinenum }%
     \var{{\devanagarifontvar \numemph\va\textbf{॰कामः}\lem \msCa\msCb, ॰काम \msNa\Ed}}% 
    \var{{\devanagarifontvar \numnoemph\vb\textbf{॰र्थं च}\lem \msCa\msCb\msNa, र्थश्च \Ed}}% 

%Verse 13:24

{\devanagarifont पुमान्स्त्रीषूपयुञ्जीत अरणीव हुताशनम् {॥ १३:२४॥} \veg\dontdisplaylinenum }%
     \var{{\devanagarifontvar \numnoemph\vc\textbf{॰षूप॰}\lem \msCa\msCb\msNa, ॰पुं प्र॰ \Ed}}% 
    \var{{\devanagarifontvar \numnoemph\vd\textbf{॰शनम्}\lem \msNa, ॰शनः \msCa\msCb\Ed}}% 

{\devanagarifont पुमान्शुक्राधिको ज्ञेयः कन्या रक्ताधिका भवेत् \thinspace{\dandab} \dontdisplaylinenum }%
 
%Verse 13:25

{\devanagarifont समशुक्रे च रक्ते च स च जायेन्नपुंसकः {॥ १३:२५॥} \veg\dontdisplaylinenum }%
 

\alalfejezet{द्वियमा त्रियमा च गुर्विणी}
{\devanagarifont देव्युवाच {\dandab}\dontdisplaylinenum  }%
 
{\devanagarifont द्वियमा त्रियमा चैव कथं जायेत गुर्विणी \thinspace{\danda} \dontdisplaylinenum }%
 
%Verse 13:26

{\devanagarifont कथं स्त्रीद्वियमा जायेत्कथं वा पुरुषद्वयम् {॥ १३:२६॥} \veg\dontdisplaylinenum }%
 
{\devanagarifont भगवानुवाच {\dandab}\dontdisplaylinenum  }%
 
{\devanagarifont रक्ताधिका स्मृता कन्या जायते वरवर्णिनि \thinspace{\danda} \dontdisplaylinenum }%
 
%Verse 13:27

{\devanagarifont वायुना च द्विधा भिन्ना कन्यकाद्वियमा स्मृता {॥ १३:२७॥} \veg\dontdisplaylinenum }%
     \var{{\devanagarifontvar \numemph\vd\textbf{कन्यका॰}\lem \msCa\msCb\msNa, कन्यक॰ \Ed}}% 

{\devanagarifont शुक्राधिकस्तु पुरुषो द्विधा भिन्नो ऽनिलेन तु \thinspace{\dandab} \dontdisplaylinenum }%
     \var{{\devanagarifontvar \numemph\va\textbf{॰धिकस्तु}\lem \msCb\msNa, ॰धि\uncl{क}\lac\ \msCa, ॰धिकास्तु \Ed\oo 
\textbf{पुरुषो}\lem \msCa\msCb\msNa, पुरुष \Ed}}% 
    \var{{\devanagarifontvar \numnoemph\vb\textbf{भिन्नो}\lem \msCa\msCb\msNa, भिन्ना \Ed}}% 

%Verse 13:28

{\devanagarifont द्वियमा पुरुषा ज्ञेयास्त्रियमास्तु त्रिधा कृते {॥ १३:२८॥} \veg\dontdisplaylinenum }%
     \var{{\devanagarifontvar \numnoemph\vc\textbf{पुरुषा}\lem \msCa\msCb, पुरुषो \msNa\Ed}}% 
    \var{{\devanagarifontvar \numnoemph\vcd\textbf{ज्ञेयास्त्रियमास्तु त्रि॰}\lem \msCa\msNapcorr, 
ज्ञेयाःस्त्रियमास्तुस्त्रि॰ \msCb, 
ज्ञेयास्त्रियमा त्रि॰ \msNaacorr, 
ज्ञेया त्रियमास्तु त्रि॰ \Ed}}% 

{\devanagarifont ऋतुस्नाता यदा नारी यदि गर्भाद्विगृह्णति \thinspace{\dandab} \dontdisplaylinenum }%
     \var{{\devanagarifontvar \numemph\vb\textbf{गर्भाद्विगृह्णति}\lem \msCa\msCb\msNa, गर्भादि गृह्यति \Ed}}% 

%Verse 13:29

{\devanagarifont प्रथमे च द्वितीये च तृतीये च न जीवति {॥ १३:२९॥} \veg\dontdisplaylinenum }%
     \var{{\devanagarifontvar \numnoemph\vd\textbf{न}\lem \msCa\msCb\msNa, स \Ed}}% 

{\devanagarifont समेषु जनयेत्पुत्रं कन्यकां विषमे दिने \thinspace{\dandab} \dontdisplaylinenum }%
     \var{{\devanagarifontvar \numemph\va\textbf{॰त्पुत्रं}\lem \corr, ॰त्पुत्रः \msCa\msCb\msNa\Ed}}% 
    \var{{\devanagarifontvar \numnoemph\vb\textbf{कन्यकां वि॰}\lem \corr, कन्यका वि॰ \msCa\msNa\Ed, कन्यका\uncl{द्वि॰} \msCb}}% 

%Verse 13:30

{\devanagarifont षष्ट्यष्टमी च दशमी द्वादशी च पुमान्भवेत् {॥ १३:३०॥} \veg\dontdisplaylinenum }%
     \var{{\devanagarifontvar \numnoemph\vc\textbf{षष्ट्यष्टमी}\lem \msCa\msCb\msNa, षष्ट्याष्टमौ \Ed}}% 
    \var{{\devanagarifontvar \numnoemph\vd\textbf{द्वादशी}\lem \msCa\msNa\Ed, \om\ \msCb}}% 

{\devanagarifont पञ्चमी सप्तमी चैव नवम्येकादशी स्त्रियः \thinspace{\dandab} \dontdisplaylinenum }%
     \var{{\devanagarifontvar \numemph\vb\textbf{नवम्येका॰}\lem \msCa\msCb\msNa, नवमेका॰ \Ed}}% 

%Verse 13:31

{\devanagarifont समरक्ते च शुक्रे च श्यामः संजायते पुमान् {॥ १३:३१॥} \veg\dontdisplaylinenum }%
 
{\devanagarifont रुधिरं त्वेकरात्रेण कललं प्रतिपद्यते \thinspace{\dandab} \dontdisplaylinenum }%
 
%Verse 13:32

{\devanagarifont कललं पञ्चरात्रेण अर्बुदत्वं प्रजायते {॥ १३:३२॥} \veg\dontdisplaylinenum }%
 
{\devanagarifont अर्बुदः सप्तरात्रेण मांसपेशीसमुद्भवः \thinspace{\dandab} \dontdisplaylinenum }%
 
%Verse 13:33

{\devanagarifont द्वितीयसप्तरात्रेण तत्सर्वं मांसशोणितम् {॥ १३:३३॥} \veg\dontdisplaylinenum }%
     \var{{\devanagarifontvar \numemph\vc\textbf{॰द्वितीय}\lem \msCa\msCb\msNa, द्वितीयं \Ed}}% 

{\devanagarifont तृतीयसप्तरात्रेण हृदयं जायते ततः \thinspace{\dandab} \dontdisplaylinenum }%
     \var{{\devanagarifontvar \numemph\va\textbf{तृतीय॰}\lem \msCa\msCb, तृतीयं \msNa\Ed}}% 

%Verse 13:34

{\devanagarifont ततः सर्वाणि गात्राणि शिरश्चैवोपजायते {॥ १३:३४॥} \veg\dontdisplaylinenum }%
 
{\devanagarifont हृदये जायमाने तु मूर्च्छा तन्द्रिररोचकः \thinspace{\dandab} \dontdisplaylinenum }%
     \var{{\devanagarifontvar \numemph\vb\textbf{मूर्च्छा त॰}\lem \msCa\msCb\msNa, मूर्च्छन्त॰ \Ed\oo 
\textbf{तन्द्रिररोचकः}\lem \msCa\Ed, तंद्रिरवोचकः \msCb, तन्द्रीररोचकः \msNa}}% 

%Verse 13:35

{\devanagarifont स्त्रियाश्छर्दिः प्रसेकश्च दौर्बल्यं चोपजायते {॥ १३:३५॥} \veg\dontdisplaylinenum }%
     \var{{\devanagarifontvar \numnoemph\vd\textbf{दौर्बल्यं}\lem \msCa\msNa\Ed, दौबल्य \msCb}}% 

{\devanagarifont तस्य हि हृदयं नारी यदि भिद्यति किंचन \thinspace{\dandab} \dontdisplaylinenum }%
     \var{{\devanagarifontvar \numemph\va\textbf{तस्य}\lem \msCa\msCb\msNa, तस्या \Ed}}% 
    \var{{\devanagarifontvar \numnoemph\vb\textbf{भिद्यति}\lem \msCb\msNa, भिद्याति \msCa, भक्ष्यति \Ed}}% 

%Verse 13:36

{\devanagarifont भक्ष्यं लेह्यं तथा पेयमुपभोगांस्तथार्थयेत् {॥ १३:३६॥} \veg\dontdisplaylinenum }%
     \var{{\devanagarifontvar \numnoemph\vc\textbf{लेह्यं त॰}\lem \msCa\msNa\Ed, लेह्यंन्त॰ \msCb}}% 
    \var{{\devanagarifontvar \numnoemph\vcd\textbf{पेयमु॰}\lem \msCb\Ed, पे\lac\ \msCa, पेयंमु॰ \msNa}}% 
    \var{{\devanagarifontvar \numnoemph\vd\textbf{॰भोगांस्त॰}\lem \msCa, ॰भोगांत॰ \msCb, ॰भोगास्त॰ \msNa\Ed\oo 
\textbf{॰र्थयेत्}\lem \msCa\msCb\msNa, ॰ययत् \Ed}}% 

{\devanagarifont शयनासनदानानि वस्त्राण्याभरणानि च \thinspace{\dandab} \dontdisplaylinenum }%
     \var{{\devanagarifontvar \numemph\va\textbf{॰दानानि}\lem \msCa\msCb\msNa, ॰यानानि \Ed}}% 

%Verse 13:37

{\devanagarifont यद्यदाकाङ्क्षते किंचित्तदास्यैव प्रदापयेत् {॥ १३:३७॥} \veg\dontdisplaylinenum }%
     \var{{\devanagarifontvar \numnoemph\vd\textbf{तदास्यैव}\lem \msCb\msNa, तदास्यै च \msCa, 
तत्तदास्यै \Ed}}% 

{\devanagarifont नायासं कारयेच्चास्या न चैवमवमानयेत् \thinspace{\dandab} \dontdisplaylinenum }%
     \var{{\devanagarifontvar \numemph\va\textbf{॰स्या}\lem \msCa\msCb\Ed, ॰स्य \msNa}}% 

%Verse 13:38

{\devanagarifont मुखमापाण्डुरं स्निग्धं कालत्वं स्तनकक्षयोः {॥ १३:३८॥} \veg\dontdisplaylinenum }%
     \var{{\devanagarifontvar \numnoemph\vd\textbf{कालत्वं स्तनकक्षयोः}\lem \msCb\msNa, 
कालंत्व स्तनकक्षयोः \msCa, 
कपोलस्तनकेशयोः \Ed}}% 

{\devanagarifont शरीरं च श्रिया जुष्टं पीनोरुश्रोणिवङ्क्षणम् \thinspace{\dandab} \dontdisplaylinenum }%
     \var{{\devanagarifontvar \numemph\va\textbf{शरीरं च}\lem \msCa\msCb\msNa, शरीरश्च \Ed}}% 
    \var{{\devanagarifontvar \numnoemph\vb\textbf{॰श्रोणिवङ्क्षणम्}\lem \corr, 
॰शोणिवंक्षणम् \msCa\msNa, ॰शोणिवक्षणम् \msCb, ॰श्रोणिवक्षसम् \Ed}}% 

%Verse 13:39

{\devanagarifont लिङ्गैरेभिर्विजानीयाद्गर्भे जीवं प्रतिष्ठितम् {॥ १३:३९॥} \veg\dontdisplaylinenum }%
     \var{{\devanagarifontvar \numnoemph\vc\textbf{लिङ्गैरे॰}\lem \msCa\msCb\msNa, लिङ्गेरे॰ \Ed}}% 
    \var{{\devanagarifontvar \numnoemph\vcd\textbf{॰जानीयाद्ग॰}\lem \msCa\msCb\msNa, ॰जानीयाङ्ग॰ \Ed}}% 

{\devanagarifont चतुर्थसप्तरात्रेण शिरश्चैवोपजायते \thinspace{\dandab} \dontdisplaylinenum }%
     \var{{\devanagarifontvar \numemph\va\textbf{चतुर्थ॰}\lem \corr, चतुर्थे \msCa\msCb\msNa\Ed}}% 

%Verse 13:40

{\devanagarifont पञ्चमसप्तरात्रेण ग्रीवा तत्रोपजायते {॥ १३:४०॥} \veg\dontdisplaylinenum }%
 
{\devanagarifont षष्ठमसप्तरात्रेण स्कन्धगात्रं प्रजायते \thinspace{\dandab} \dontdisplaylinenum }%
 
%Verse 13:41

{\devanagarifont सप्तमसप्तरात्रेण पृष्ठवंशः प्रजायते {॥ १३:४१॥} \veg\dontdisplaylinenum }%
     \var{{\devanagarifontvar \numemph\vd\textbf{॰वंशः}\lem \msCa\msNa, ॰शशः \msCb, ॰वंश \Ed}}% 

{\devanagarifont अष्टमसप्तरात्रेण पाणी जायत चोभयौ \thinspace{\dandab} \dontdisplaylinenum }%
     \var{{\devanagarifontvar \numemph\vb\textbf{जायत}\lem \conj, जायतः चो॰ \msCapcorr\ \unmetr, 
जायति चो॰ \msCaacorr, 
जायतश्चो॰ \msCb\msNa\ \unmetr, 
जायत चो॰ \Ed}}% 

%Verse 13:42

{\devanagarifont सप्तरात्रं नव प्राप्य जायते ह्युरपञ्जरम् {॥ १३:४२॥} \veg\dontdisplaylinenum }%
     \var{{\devanagarifontvar \numnoemph\vd\textbf{ह्युर॰}\lem \corr, ह्युरः॰ \msCa\msCb\msNa\ \unmetr, हृदि॰ \Ed}}% 

{\devanagarifont दशमे सप्तरात्रे च पादौ जायत चोभौ \thinspace{\dandab} \dontdisplaylinenum }%
     \var{{\devanagarifontvar \numemph\vb\textbf{जायत}\lem \msCb, जायेत \msCa\msNa\Ed}}% 

%Verse 13:43

{\devanagarifont उदरं चोपजायेत सप्तैकादशरात्रिके {॥ १३:४३॥} \veg\dontdisplaylinenum }%
     \var{{\devanagarifontvar \numnoemph\vc\textbf{उदरं चो॰}\lem \msCa\msCb\msNa, उदरश्चो॰ \Ed}}% 

{\devanagarifont द्वादशसप्तरात्रेण कुक्षिपार्श्वः प्रजायते \thinspace{\dandab} \dontdisplaylinenum }%
     \var{{\devanagarifontvar \numemph\vb\textbf{॰पार्श्वः}\lem \msCa\msCb\msNa, ॰पार्श्वेः \Ed}}% 

%Verse 13:44

{\devanagarifont सप्तत्रैदशरात्रेण कटिस्तत्रोपजायते {॥ १३:४४॥} \veg\dontdisplaylinenum }%
     \var{{\devanagarifontvar \numnoemph\vd\textbf{कटिस्तत्रो॰}\lem \msCa\msNa, कटिकस्तत्रो॰ \msCbacorr, 
ककस्तत्रो॰ \msCbpcorr, 
कुटि सूत्रो॰ \Ed}}% 

{\devanagarifont नवत्यष्ट च रात्रेण जायते सूत्रविंशतिः \thinspace{\dandab} \dontdisplaylinenum }%
     \var{{\devanagarifontvar \numemph\va\textbf{॰ष्ट च}\lem \msCa\msCb\msNa, ॰ष्टम॰ \Ed}}% 
    \var{{\devanagarifontvar \numnoemph\vb\textbf{सूत्रविंशतिः}\lem \msCb\msNa, सूतविंशतिः \msCa, सूत्रविंशति \Ed}}% 

%Verse 13:45

{\devanagarifont सप्तपञ्चदशाहेन सर्वमेदः प्रजायते {॥ १३:४५॥} \veg\dontdisplaylinenum }%
 
{\devanagarifont षोडशसप्तरात्रेण अस्थि सर्वाणि जायते \thinspace{\dandab} \dontdisplaylinenum }%
 
%Verse 13:46

{\devanagarifont सप्तसप्तदशाहेन जायते स्नायुबन्धनम् {॥ १३:४६॥} \veg\dontdisplaylinenum }%
 
{\devanagarifont सप्तमाष्टादशाहेन जायते मुखमण्डलम् \thinspace{\dandab} \dontdisplaylinenum }%
 
%Verse 13:47

{\devanagarifont सप्तोनविंशरात्रेण घ्राणवंशः प्रजायते {॥ १३:४७॥} \veg\dontdisplaylinenum }%
 
{\devanagarifont सप्तविंशतिरात्रेण नेत्रनालं प्रजायते \thinspace{\dandab} \dontdisplaylinenum }%
     \var{{\devanagarifontvar \numemph\va\textbf{सप्त॰}\lem \msCa\Ed, सप्ता॰ \msCb\msNa}}% 
    \var{{\devanagarifontvar \numnoemph\vb\textbf{नेत्रनालं}\lem \msCa\msCb\msNa, नैत्रनालिं \Ed}}% 

%Verse 13:48

{\devanagarifont सप्तैकविंशरात्रेण कर्णयुग्मं प्रजायते {॥ १३:४८॥} \veg\dontdisplaylinenum }%
 
{\devanagarifont द्वाविंशसप्तरात्रेण जायते द्वौ भ्रुवौ ततः \thinspace{\dandab} \dontdisplaylinenum }%
 
%Verse 13:49

{\devanagarifont सप्तत्रिविंशरात्रेण गण्डयुग्मं प्रजायते {॥ १३:४९॥} \veg\dontdisplaylinenum }%
 
{\devanagarifont चतुर्विंशतिसप्ताहे ओष्ठयुग्मं प्रजायते \thinspace{\dandab} \dontdisplaylinenum }%
 
%Verse 13:50

{\devanagarifont पञ्चविंशतिसप्ताहे जिह्वा जायेत सुन्दरि {॥ १३:५०॥} \veg\dontdisplaylinenum }%
 
{\devanagarifont षड्विंशसप्तरात्रेण दन्तपाली प्रजायते \thinspace{\dandab} \dontdisplaylinenum }%
     \var{{\devanagarifontvar \numemph\va\textbf{षड्विंश॰}\lem \msCa\Ed, षट्त्रिंश॰ \msCb, षड्विंशति॰ \msNa}}% 
    \var{{\devanagarifontvar \numnoemph\vb\textbf{॰पाली}\lem \msCa\msCb\msNa, ॰पङ्क्ति \Ed}}% 

%Verse 13:51

{\devanagarifont सप्तविंशतिसप्ताहे जायते वृषणद्वयम् {॥ १३:५१॥} \veg\dontdisplaylinenum }%
     \var{{\devanagarifontvar \numnoemph\vc\textbf{सप्त॰}\lem \msCa\msNa\Ed, सप्ता॰ \msCb}}% 
    \var{{\devanagarifontvar \numnoemph\vd\textbf{वृषण॰}\lem \msCa\msCb\msNa, वृषल॰ \Ed}}% 

{\devanagarifont अष्टाविंशतिसप्ताहे भगलिङ्गं प्रजायते \thinspace{\dandab} \dontdisplaylinenum }%
 
%Verse 13:52

{\devanagarifont ऊनविंशतिसप्ताहे जायते च त्वगेव च {॥ १३:५२॥} \veg\dontdisplaylinenum }%
     \var{{\devanagarifontvar \numemph\vc\textbf{ऊन॰}\lem \msCa\msCb\msNa, उन॰ \Ed\oo 
\textbf{॰विंशति॰}\lem \msCa\msNa\Ed, ॰त्रिंशति॰ \msCb}}% 
    \var{{\devanagarifontvar \numnoemph\vd\textbf{जायते च}\lem \msCa\msCb\Ed, जायते त्व \msNa}}% 

{\devanagarifont त्रिंशतिसप्तरात्रेण जायते नाभिमण्डलम् \thinspace{\dandab} \dontdisplaylinenum }%
     \var{{\devanagarifontvar \numemph\va\textbf{॰शति॰}\lem \msNa, ॰शता \msCa\msCb, ॰शत॰ \Ed}}% 

%Verse 13:53

{\devanagarifont सप्तैकत्रिंशरात्रेण सर्वरन्ध्रं प्रजायते {॥ १३:५३॥} \veg\dontdisplaylinenum }%
 
{\devanagarifont द्वात्रिंशत्सप्तरात्रेण नखविंशति जायते \thinspace{\dandab} \dontdisplaylinenum }%
     \var{{\devanagarifontvar \numemph\va\textbf{॰त्रिंशत्स॰}\lem \msCa\msCb\msNa, ॰त्रिंशस॰ \Ed}}% 

%Verse 13:54

{\devanagarifont त्रेत्रिंशत्सप्तरात्रेण रोम केशश्च जायते {॥ १३:५४॥} \veg\dontdisplaylinenum }%
     \var{{\devanagarifontvar \numnoemph\vc\textbf{त्रेत्रिंशत्॰}\lem \msCa\msNa, त्रितिंश॰ \msCb\Ed}}% 
    \var{{\devanagarifontvar \numnoemph\vd\textbf{रोम}\lem \msCa\msCb, लोम॰ \msNa\Ed}}% 

{\devanagarifont सप्तरात्रचतुस्त्रिंशे सर्वसन्धिः प्रजायते \thinspace{\dandab} \dontdisplaylinenum }%
     \var{{\devanagarifontvar \numemph\va\textbf{सप्तरात्रचतुस्त्रिंशे}\lem \msCa, 
सप्तरात्रचतुत्रिंशे \msCb, 
सप्तरात्रचतुस्त्रिंशत् \msNa, 
चतुस्त्रिंशसप्तरात्रेण \Ed\ \unmetr}}% 
    \var{{\devanagarifontvar \numnoemph\vb\textbf{सर्व॰}\lem \msCa\msCb\msNa, सर्व्वे \Ed}}% 

%Verse 13:55

{\devanagarifont पञ्चत्रिंशतिसप्ताहे सर्वमर्म प्रजायते {॥ १३:५५॥} \veg\dontdisplaylinenum }%
     \var{{\devanagarifontvar \numnoemph\vc\textbf{॰शति॰}\lem \msCb\msNa\Ed, ॰श॰ \msCa\ \unmetr}}% 

{\devanagarifont षड्त्रिंशसप्तरात्रेण वेदना चोपजायते \thinspace{\dandab} \dontdisplaylinenum }%
 
%Verse 13:56

{\devanagarifont सप्तत्रिंशतिसप्ताहे ईर्षाद्वेषः प्रजायते {॥ १३:५६॥} \veg\dontdisplaylinenum }%
     \var{{\devanagarifontvar \numemph\vd\textbf{ईर्षाद्वेषः}\lem \msCa\msNa, ईर्षाद्वेष \msCb, ईर्ष्याद्वेषः \Ed}}% 

{\devanagarifont अष्टत्रिंशतिसप्ताहे पञ्चात्मकसमन्वितम् \thinspace{\dandab} \dontdisplaylinenum }%
 
%Verse 13:57

{\devanagarifont सर्वाङ्गमङ्गसम्पूर्णः परिपक्वः स तिष्ठति {॥ १३:५७॥} \veg\dontdisplaylinenum }%
     \var{{\devanagarifontvar \numemph\vd\textbf{॰पक्वः}\lem \msCa\msCb\msNa, ॰पक्व \Ed}}% 
    \paral{{\devanagarifontsmall \vo {\englishfont \compare\ \SDHU\ 8.36cd:}
                 पञ्चात्मकसमायुक्तः परिपक्वः स तिष्ठति
                    {\englishfont = \PADMAP\ 2.66.37cd} }}

{\devanagarifont मातुः श्वाशितपीतं च नाभिसूत्रागमेन तु \thinspace{\dandab} \dontdisplaylinenum }%
     \var{{\devanagarifontvar \numemph\va\textbf{मातुः}\lem \eme, मातु \msCa\msCb\msNa\Ed\oo 
\textbf{॰पीतं च}\lem \msCa\msCb\msNa, ॰पीतश्च \Ed}}% 
    \paral{{\devanagarifontsmall \vab {\englishfont \compare\ \SDHU\ 8.37:}
                         मातुराहारवीर्येण षड्विधेन रसेन च\thinspace{\devanagarifontsmall ।}
                         नाभिसूत्रनिबन्धेन वर्धते स दिने दिने\thinspace{\devanagarifontsmall ॥}
                     {\englishfont = \PADMAP\ 2.66.38} }}

%Verse 13:58

{\devanagarifont प्रजातस्योपधार्यन्ते गर्भस्थस्यैव जन्तवः {॥ १३:५८॥} \veg\dontdisplaylinenum }%
     \var{{\devanagarifontvar \numnoemph\vc\textbf{प्रजात॰}\lem \msCa\msCb\msNa, अजात॰ \Ed}}% 

{\devanagarifont ततः प्रविशते चित्तं निद्रास्वप्नं यथा तथा \thinspace{\dandab} \dontdisplaylinenum }%
     \var{{\devanagarifontvar \numemph\va\textbf{चित्तं}\lem \msCa\msCb\msNa, देहे \Ed}}% 
    \var{{\devanagarifontvar \numnoemph\vb\textbf{॰स्वप्नं}\lem \msCa\msCb\msNa, ॰स्वप्न \Ed}}% 
    \paral{{\devanagarifontsmall \vab {\englishfont \compare\ \SDHU\ 8.38:}
                 ततः स्मृतिं लभेज्जीवः संपूर्णे ऽस्मिन् शरीरके\thinspace{\devanagarifontsmall ।} 
                 सुखदुःखं विजानाति निद्रास्वप्नं पुराकृतम्\thinspace{\devanagarifontsmall ॥}
                     {\englishfont \similar\ \PADMAP\ 2.66.39} }}

%Verse 13:59

{\devanagarifont नोपलभ्यति सूक्ष्मत्वादरण्यग्निर्यथा तथा {॥ १३:५९॥} \veg\dontdisplaylinenum }%
     \var{{\devanagarifontvar \numnoemph\vd\textbf{॰रण्यग्नि॰}\lem \msCb\msNa\Ed, \lac ण्यग्नि॰ \msCa\oo 
\textbf{तथा}\lem \msCapcorr\msCb\msNa\Ed, \om\ \msCaacorr}}% 

{\devanagarifont गर्भोदकेन सिक्ताङ्गो जरायुपरिवेष्टितः \thinspace{\dandab} \dontdisplaylinenum }%
     \var{{\devanagarifontvar \numemph\va\textbf{॰ङ्गो}\lem \msCa\msCb\msNa, ॰ङ्ग \Ed}}% 
    \var{{\devanagarifontvar \numnoemph\vb\textbf{जरायु॰}\lem \eme, जरया \msCa\msCb\msNa,  जराया \Ed}}% 
    \paral{{\devanagarifontsmall \vab {\englishfont \compare\ \SDHU\ 8.43--44 (=\PADMAP\ 2.66.44-45):}
                 यथा गिरिवराक्रान्तः कश्चिद्दुःखेन तिष्ठति\thinspace{\devanagarifontsmall ।}
                 तथा जरायुणा देही दुःखं तिष्ठति दुःखितः\thinspace{\devanagarifontsmall ॥} 
                 पतितः सागरे यद्वद्दुःखमास्ते समाकुलः\thinspace{\devanagarifontsmall ।} 
                 गर्भोदकेन सिक्ताङ्गस्तथास्ते व्याकुलात्मकः\thinspace{\devanagarifontsmall ॥} }}

%Verse 13:60

{\devanagarifont जातिं स्मरति तत्रस्थो जन्तुश्चेतःसमन्वितः {॥ १३:६०॥} \veg\dontdisplaylinenum }%
     \var{{\devanagarifontvar \numnoemph\vc\textbf{जातिं}\lem \msCa\msCb\msNa, जाति \Ed}}% 
    \var{{\devanagarifontvar \numnoemph\vd\textbf{॰श्चेतः॰}\lem \msCa\msNa\Ed, ॰श्चैतः॰ \msCb}}% 

{\devanagarifont मृतश्चाहं पुनर्जातो भूयश्चैव पुनर्मृतः \thinspace{\dandab} \dontdisplaylinenum }%
     \var{{\devanagarifontvar \numemph\va\textbf{पुनर्जातो}\lem \msCa\msNa\Ed, पुनजातो \msCb}}% 
    \paral{{\devanagarifontsmall \vab {\englishfont \similar\ \NISVK\ 26.17ab:}
                 मृतश्चाहं पुनर्जातो जातश्चाहं पुनर्मृतः
                      {\englishfont = \SDHU\ 8.39ab = \PADMAP\ 2.66.40ab} }}

%Verse 13:61

{\devanagarifont स्थावराणां सहस्रेषु जातो ऽस्मि विविधेषु च {॥ १३:६१॥} \veg\dontdisplaylinenum }%
 
{\devanagarifont तिर्यग्योनिसहस्रेषु प्रेतेषु नरकेषु च \thinspace{\dandab} \dontdisplaylinenum }%
     \lacuna{\devanagarifontsmall \vab {\englishfont missing in \Ed} }%
  
%Verse 13:62

{\devanagarifont चतुर्वर्णविवर्णेषु मानुषेषु सहस्रशः {॥ १३:६२॥} \veg\dontdisplaylinenum }%
 
{\devanagarifont साम्प्रतं च पुनर्गर्भः क्लेशप्राप्तः सुदुःसहः \thinspace{\dandab} \dontdisplaylinenum }%
     \var{{\devanagarifontvar \numemph\va\textbf{च पुनर्ग॰}\lem \msCa\msNa\Ed, पुनग॰ \msCb}}% 
    \var{{\devanagarifontvar \numnoemph\vb\textbf{क्लेश॰}\lem \msCa\msNa, क्लेशं \msCb, क्लेशः \Ed\oo 
\textbf{॰दुःसहः}\lem \Ed, ॰दुहः \msCa, ॰दुःसहं \msCb, ॰दुःसह \msNa}}% 

%Verse 13:63

{\devanagarifont इदानीं जातमात्रो ऽहं संस्कारैश्चापि संस्कृतः {॥ १३:६३॥} \veg\dontdisplaylinenum }%
     \paral{{\devanagarifontsmall \vcd {\englishfont \similar\ \SDHU\ 8.40ab:}
                 अधुना जातमात्रो ऽहं प्राप्तसंस्कार एव वा
                     {\englishfont = \PADMAP\ 2.66.41ab} }}

{\devanagarifont योगमेवाभिसेवामि सांख्यं वा पञ्चविंशकम् \thinspace{\dandab} \dontdisplaylinenum }%
     \var{{\devanagarifontvar \numemph\vb\textbf{सांख्यं}\lem \msCa\msCb\msNa, साख्यं \Ed}}% 

%Verse 13:64

{\devanagarifont यत्र जन्मजरा नास्ति यत्र मृत्युश्च नास्ति वै {॥ १३:६४॥} \veg\dontdisplaylinenum }%
 
{\devanagarifont यत्र ब्रह्म परं वैद्यं चरिष्यामि यतव्रतः \thinspace{\dandab} \dontdisplaylinenum }%
     \var{{\devanagarifontvar \numemph\va\textbf{ब्रह्म}\lem \msCa\msNa\Ed, ब्रह्मा \msCb\oo 
\textbf{वैद्यं }\lem\msCa\msNa, वेद्यं \msCb\Ed}}% 
    \var{{\devanagarifontvar \numnoemph\vb\textbf{यत॰}\lem \msCa\msCb\Ed, यतः \msNa}}% 

%Verse 13:65

{\devanagarifont एवमादीन्यनेकानि चिन्तयित्वा पुनः पुनः {॥ १३:६५॥} \veg\dontdisplaylinenum }%
 
{\devanagarifont यावत्तिष्ठति गर्भस्थो जातिं स्मरति पूर्विकाम् \thinspace{\dandab} \dontdisplaylinenum }%
     \var{{\devanagarifontvar \numemph\vb\textbf{जातिं}\lem \msCa\msNa, जाति \msCb\Ed}}% 

%Verse 13:66

{\devanagarifont ततो जायति कष्टेन महाक्लेशेन मानवः {॥ १३:६६॥} \veg\dontdisplaylinenum }%
 
{\devanagarifont योनियन्त्रसुतीव्रेण पीड्यमानः सुदुःखितः \thinspace{\dandab} \dontdisplaylinenum }%
     \var{{\devanagarifontvar \numemph\vb\textbf{॰मानः}\lem \msCa\msNa, ॰मान॰ \msCb\Ed}}% 
    \paral{{\devanagarifontsmall \vab {\englishfont \compare\ \SDHU\ 8.49ab:}
                  गर्भात्कोटिगुणं दुःखं योनियन्त्रप्रपीडनात
                     {\englishfont and \SDHU\ 8.52cd:}
                  तथा शरीरं निःसारं योनियन्त्रप्रपीडितम् }}

%Verse 13:67

{\devanagarifont जातमात्रे स्मृतिभ्रंशो भवतीह अचेतनः {॥ १३:६७॥} \veg\dontdisplaylinenum }%
     \var{{\devanagarifontvar \numnoemph\vc\textbf{॰मात्रे}\lem \msCa\msCb\msNa, ॰मात्रो \Ed}}% 

{\devanagarifont मायामुद्गरतीव्रेण हतः किं शुभमाचरेत् \thinspace{\dandab} \dontdisplaylinenum }%
     \var{{\devanagarifontvar \numemph\va\textbf{॰मुद्गर॰}\lem \msCa\msNa\Ed, ॰मुद्गल॰ \msCb\oo 
\textbf{॰तीव्रेण}\lem \msCa\msCb\msNapcorr\Ed, ॰ती \msNaacorr}}% 

{\devanagarifont एष गर्भसमुत्पत्तिः कथितो ऽस्मि वरानने  \danda\dontdisplaylinenum }%
 
%Verse 13:68

{\devanagarifont दुःखसंसारप्रथमः किं भूयः श्रोतुमिच्छसि {॥ १३:६८॥} \veg\dontdisplaylinenum }%
     \var{{\devanagarifontvar \numnoemph\ve\textbf{॰सार॰}\lem \msCa\Ed, ॰सारः \msCb\msNa\oo 
\textbf{॰प्रथमः}\lem \msCa\msCb\msNa, ॰प्रशमं \Ed}}% 

{\devanagarifont 
\jump
\begin{center}
\ketdanda~इति वृषसारसंग्रहे गर्भोत्पत्तिरध्यायस्त्रयदशमः~\ketdanda
\end{center}
\dontdisplaylinenum\vers  }%
     \var{{\devanagarifontvar \numnoemph{\englishfont \Colo:}\textbf{॰त्पत्तिरध्यायस्त्रयदशमः}\lem \msCa\msCb\msNa, 
॰त्पत्तिर्नाम त्रयदशो ऽध्यायः \Ed}}% 
\bekveg\szamveg
\vfill
\phpspagebreak

\versno=0\fejno=14
\thispagestyle{empty}


\vers

\centerline{\Large\devanagarifontbold [   चतुर्दशमो ऽध्यायः  ]}{\vrule depth10pt width0pt} \fancyhead[CE]{{\footnotesize\devanagarifont वृषसारसंग्रहे  }}
\fancyhead[CO]{{\footnotesize\devanagarifont चतुर्दशमो ऽध्यायः  }}
\fancyhead[LE]{}
\fancyhead[RE]{}
\fancyhead[LO]{}
\fancyhead[RO]{}
\szam\bek



\alalfejezet{देहरूपवर्णभेदानि}
{\devanagarifont देव्युवाच {\dandab}\dontdisplaylinenum  }%
 
{\devanagarifont अतिदीर्घो ऽतिह्रस्वश्च पुमान्केनोपजायते \thinspace{\danda} \dontdisplaylinenum }%
     \var{{\devanagarifontvar \numemph\va\textbf{॰दीर्घो}\lem \msCa\msCb\msNa, ॰दीर्घा॰ \Ed}}% 

%Verse 14:1

{\devanagarifont अतिगौरो ऽतिकृष्णश्च नरो भवति किं प्रभो {॥ १४:१॥} \veg\dontdisplaylinenum }%
 
{\devanagarifont भगवानुवाच {\dandab}\dontdisplaylinenum  }%
 
{\devanagarifont गृहीतगर्भा या नारी नित्यमुत्तानशालिनी \thinspace{\danda} \dontdisplaylinenum }%
     \var{{\devanagarifontvar \numemph\vb\textbf{॰शालिनी}\lem \msCa\msNa\Ed, ॰शायिनी \msCb}}% 

%Verse 14:2

{\devanagarifont प्रसारितविभक्तात्मा सो ऽतिदीर्घः प्रजायते {॥ १४:२॥} \veg\dontdisplaylinenum }%
     \var{{\devanagarifontvar \numnoemph\vc\textbf{॰भक्ता॰}\lem \msCa\msCb\msNa, ॰मुक्ता॰ \Ed}}% 

{\devanagarifont गृहीतगर्भा या नारी शेते संकुचिता सदा \thinspace{\dandab} \dontdisplaylinenum }%
 
%Verse 14:3

{\devanagarifont रसान्नादीनि कटुकं सेवना ह्रस्व जायते {॥ १४:३॥} \veg\dontdisplaylinenum }%
     \var{{\devanagarifontvar \numemph\vc\textbf{रसान्नादीनि}\lem \eme, रसोन्नादीनि \msCa\msNa\Ed, रसोन्नादी \msCb}}% 
    \var{{\devanagarifontvar \numnoemph\vd\textbf{सेवना}\lem \msCa, सेवनाः \msCb\Ed, सेवनात् \msNa}}% 

{\devanagarifont गृहीतगर्भा या नारी नित्यं क्षीरोपसेविनी \thinspace{\dandab} \dontdisplaylinenum }%
     \var{{\devanagarifontvar \numemph\vb\textbf{॰सेविनी}\lem \msCa\msCb\msNa, ॰सेविता \Ed}}% 

{\devanagarifont वरकोद्रवशालींश्च भुङ्क्ते चापि यवौदनम्  \danda\dontdisplaylinenum }%
     \var{{\devanagarifontvar \numnoemph\vc\textbf{॰शालींश्च}\lem \msCa\msCb\msNa, ॰शाली च \Ed}}% 
    \var{{\devanagarifontvar \numnoemph\vd\textbf{भुङ्क्ते}\lem \msCa\msCb\msNa, भुक्ता \Ed\oo 
\textbf{यवौ॰}\lem \msCa\msCb, यवो॰ \msNa\Ed}}% 

%Verse 14:4

{\devanagarifont शुक्लवस्त्रस्रजा युक्ता सातिगौरं प्रसूयते {॥ १४:४॥} \veg\dontdisplaylinenum }%
     \var{{\devanagarifontvar \numnoemph\vf\textbf{॰सूयते}\lem \msCa\msNa, ॰जायते \msCb\Ed}}% 

{\devanagarifont गृहीतगर्भा या नारी कालधान्यानि सेवते \thinspace{\dandab} \dontdisplaylinenum }%
     \var{{\devanagarifontvar \numemph\vb\textbf{कालधान्यानि॰}\lem \eme, कलधान्यानि॰ \msCa\msCb, 
कलध्यानानि \msNa, 
वलधान्यानि॰ \Ed}}% 

{\devanagarifont माषकृष्णतिलामुद्गं तथा कृष्णयवोदनम्  \danda\dontdisplaylinenum }%
     \var{{\devanagarifontvar \numnoemph\vcd\textbf{माषकृष्णतिलामुद्गं तथा कृष्णयवोदनम्}\lem \msCa\msCb\msNa, 
कृष्णकोद्रवतैलादि माषकृष्णयवोदनम् \Ed}}% 

%Verse 14:5

{\devanagarifont कृष्णवस्त्रस्रजादीनि तस्याः कृष्णः प्रजायते {॥ १४:५॥} \veg\dontdisplaylinenum }%
 

\alalfejezet{जातिदोषानि}
{\devanagarifont देव्युवाच {\dandab}\dontdisplaylinenum  }%
 
{\devanagarifont जात्यन्धो जायते कस्मात्षण्ढो भीरुर्हतेन्द्रियः \thinspace{\danda} \dontdisplaylinenum }%
 
%Verse 14:6

{\devanagarifont कुब्जो वा वामनो वापि पण्डः स्थूलशिरः कथम् {॥ १४:६॥} \veg\dontdisplaylinenum }%
     \var{{\devanagarifontvar \numemph\vc\textbf{कुब्जो}\lem \msCa\msCb\msNa, कुजो \Ed}}% 
    \var{{\devanagarifontvar \numnoemph\vd\textbf{पण्डः}\lem \msCa\msCb\msNa, पङ्ग्वः \Ed\oo 
\textbf{स्थूल॰}\lem \msCa\msNa\Ed, स्तूलं \msCb}}% 

{\devanagarifont भगवानुवाच {\dandab}\dontdisplaylinenum  }%
 
{\devanagarifont गृहीतगर्भा या नारी तीक्ष्णोष्णान्युपसेवते \thinspace{\danda} \dontdisplaylinenum }%
 
%Verse 14:7

{\devanagarifont लशुनानि पलाण्डूनि करञ्जमूलकानि च {॥ १४:७॥} \veg\dontdisplaylinenum }%
     \var{{\devanagarifontvar \numemph\vc\textbf{॰पलाण्डूनि}\lem \Ed, ॰पलण्डूनि \msCa\msCb\msNa}}% 

{\devanagarifont पिप्पलीं शृङ्गवेरं च सर्षपान्मरिचानि च \thinspace{\dandab} \dontdisplaylinenum }%
     \var{{\devanagarifontvar \numemph\va\textbf{पिप्पलीं}\lem \msCa\msNa, पिप्पली॰ \msCb\Ed\oo 
\textbf{॰वेरं च}\lem \msCb\msNa\Ed, ॰वेर\lac\ \msCa}}% 

{\devanagarifont आसवं च परिक्लिष्टा ये चान्ये कटुतिक्तकाः  \danda\dontdisplaylinenum }%
     \var{{\devanagarifontvar \numnoemph\vc\textbf{आसवं च}\lem \msCa\msCb\msNa, आसवश्च \Ed}}% 

%Verse 14:8

{\devanagarifont तीक्ष्णं तु सेवमाना या जात्यन्धं जनयेत्सुतम् {॥ १४:८॥} \veg\dontdisplaylinenum }%
     \var{{\devanagarifontvar \numnoemph\vf\textbf{॰न्धं जनयेत्सुतम्}\lem \msCa\msCb\msNa, ॰न्धो जायते सुतः \Ed}}% 

{\devanagarifont मिथ्योपचाराः स्त्रीपुंसो व्यापन्ने शुक्रशोणिते \thinspace{\dandab} \dontdisplaylinenum }%
     \var{{\devanagarifontvar \numemph\va\textbf{मिथ्यो॰}\lem \msCa\msCb\msNa, मिथ्या॰ \Ed}}% 
    \var{{\devanagarifontvar \numnoemph\vb\textbf{॰पन्ने}\lem \msCb\Ed, ॰पने \msCa\msNa}}% 

{\devanagarifont यदा गर्भाशये रक्तं स्त्रियाः पूर्वं निषिच्यते  \danda\dontdisplaylinenum }%
     \var{{\devanagarifontvar \numnoemph\vc\textbf{यदा}\lem \msCa\msCb\msNapcorr\Ed, यदि \msNaacorr}}% 

%Verse 14:9

{\devanagarifont पश्चाच्छुक्रं रक्तकाले तदा षण्डः प्रसूयते {॥ १४:९॥} \veg\dontdisplaylinenum }%
     \var{{\devanagarifontvar \numnoemph\vf\textbf{॰सूयते}\lem \msCa\msCb\msNa, ॰जायते \Ed}}% 

{\devanagarifont त्रस्तोद्विग्ना यदा भीता स्त्री पुंसा सूयते प्रजा \thinspace{\dandab} \dontdisplaylinenum }%
     \var{{\devanagarifontvar \numemph\va\textbf{॰द्विग्ना}\lem \msCa\msCb\msNa, ॰द्विग्नो \Ed\oo 
\textbf{भीता}\lem \msCa\msCb\msNa, भीत॰ \Ed}}% 
    \var{{\devanagarifontvar \numnoemph\vb\textbf{पुंसा}\lem \msCb, पुंसां \msCa\msNa\Ed\oo 
\textbf{सूयते प्रजा}\lem \msCa\msCb\msNa, सूपजायते \Ed}}% 

%Verse 14:10

{\devanagarifont तत्र यो जायते गर्भाद्भीरुः क्रन्दनको भवेत् {॥ १४:१०॥} \veg\dontdisplaylinenum }%
     \var{{\devanagarifontvar \numnoemph\vcd\textbf{गर्भाद्भीरुः}\lem \msCa\msCb\msNa, गर्भभिरुः \Ed}}% 

{\devanagarifont विसर्गकाले शुक्रस्य विघ्न उत्पद्यते यदा \thinspace{\dandab} \dontdisplaylinenum }%
     \var{{\devanagarifontvar \numemph\va\textbf{विसर्ग॰}\lem \msCa\msCb\msNa, निसर्ग॰ \Ed}}% 

%Verse 14:11

{\devanagarifont इन्द्रियावर्तविघ्ने तु तदा जायेदनिन्द्रियः {॥ १४:११॥} \veg\dontdisplaylinenum }%
     \var{{\devanagarifontvar \numnoemph\vd\textbf{॰निन्द्रियः}\lem \msCa\msCb\msNa, ॰तिन्द्रियः \Ed}}% 

{\devanagarifont गृहीतगर्भा या नारी वातलान्युपसेवते \thinspace{\dandab} \dontdisplaylinenum }%
 
%Verse 14:12

{\devanagarifont कटुकानि कषायानि तिक्तानि च विशेषतः {॥ १४:१२॥} \veg\dontdisplaylinenum }%
 
{\devanagarifont वातः प्रकुपितस्तस्या गर्भमाभुज्य तिष्ठति \thinspace{\dandab} \dontdisplaylinenum }%
     \var{{\devanagarifontvar \numemph\vb\textbf{॰भुज्य}\lem \msCa\msCb\msNa, ॰आतुह्य \Ed}}% 

%Verse 14:13

{\devanagarifont कुब्जस्तु जायते तस्माद्गर्भाद्वातनिपीडनात् {॥ १४:१३॥} \veg\dontdisplaylinenum }%
 
{\devanagarifont नित्यमासनशीला या तथा चोत्कुटकासना \thinspace{\dandab} \dontdisplaylinenum }%
     \var{{\devanagarifontvar \numemph\va\textbf{नित्यमासन॰}\lem \msCa\msNa, नित्यमानस॰ \msCb, नित्यसासव॰ \Ed}}% 
    \var{{\devanagarifontvar \numnoemph\vb\textbf{चोत्कुटका॰}\lem \eme, चोत्कटुका॰ \msCa\msNa\Ed, चोत्कटका॰ \msCb}}% 

%Verse 14:14

{\devanagarifont तस्याः संहन्यते गर्भो वामनस्तेन जायते {॥ १४:१४॥} \veg\dontdisplaylinenum }%
     \var{{\devanagarifontvar \numnoemph\vc\textbf{तस्याः}\lem \msCa\msCb\msNa, तस्या \Ed}}% 

{\devanagarifont अतिव्यायामशीला तु या नारी विषमासनी \thinspace{\dandab} \dontdisplaylinenum }%
     \var{{\devanagarifontvar \numemph\vb\textbf{॰सनी}\lem \msCa\msCbpcorr\msNa\Ed, ॰सना \msCbacorr}}% 

%Verse 14:15

{\devanagarifont गर्भः संक्षुभ्यते तस्याः पण्डस्तेनोपजायते {॥ १४:१५॥} \veg\dontdisplaylinenum }%
     \var{{\devanagarifontvar \numnoemph\vc\textbf{गर्भः}\lem \msCa\msCb\Ed, गर्भ \msNa}}% 
    \var{{\devanagarifontvar \numnoemph\vd\textbf{पण्ड॰}\lem \msCa\msCb\msNa, पषण्ड॰ \Ed}}% 

{\devanagarifont गृहीतगर्भा या नारी रूढधान्यानि सेवते \thinspace{\dandab} \dontdisplaylinenum }%
     \var{{\devanagarifontvar \numemph\vb\textbf{रूढधान्यानि}\lem \msCa\msNa, रूढधानानि \msCb, रूक्षधान्यानि \Ed}}% 

{\devanagarifont वातश्लेष्म शिरस्थो वै तस्य गर्भस्य कुप्यते  \danda\dontdisplaylinenum }%
     \var{{\devanagarifontvar \numnoemph\vc\textbf{॰श्लेष्मा}\lem \eme, ॰श्लेष्म \msCa\msCb\msNa\Ed}}% 
    \var{{\devanagarifontvar \numnoemph\vd\textbf{तस्य}\lem \msCa\msCb\msNa, तस्या \Ed}}% 

%Verse 14:16

{\devanagarifont ततः स्थूलशिरास्तेन पुमान्जायत्यसंशयः {॥ १४:१६॥} \veg\dontdisplaylinenum }%
 
{\devanagarifont देव्युवाच {\dandab}\dontdisplaylinenum  }%
 
{\devanagarifont करालाङ्गा हनुः पङ्गुर्मूको गद्गदभाषकः \thinspace{\danda} \dontdisplaylinenum }%
     \var{{\devanagarifontvar \numemph\va\textbf{॰लाङ्गा}\lem \msCa\Ed, ॰लोगो \msCb, ॰लागो \msNa\oo 
\textbf{पङ्गु॰}\lem \msCa\msCb\msNa, पङ्गू॰ \Ed}}% 

%Verse 14:17

{\devanagarifont विवृताक्षस्त्वनक्षो वा भवेद्दुःखगुदः कथम् {॥ १४:१७॥} \veg\dontdisplaylinenum }%
     \var{{\devanagarifontvar \numnoemph\vc\textbf{विवृता॰}\lem \msCa\msCb\msNa, विकृता॰ \Ed}}% 
    \var{{\devanagarifontvar \numnoemph\vd\textbf{भवे॰}\lem \msCa\msCb\msNa, भव॰ \Ed\oo 
\textbf{॰द्दुःख॰}\lem \msCa\msCb\msNa, ॰द्रस्व॰ \Ed}}% 

{\devanagarifont भगवानुवाच {\dandab}\dontdisplaylinenum  }%
 
{\devanagarifont करालस्तनदोषेण जायते मानवस्तथा \thinspace{\danda} \dontdisplaylinenum }%
     \var{{\devanagarifontvar \numemph\va\textbf{॰स्तन॰}\lem \msCa\msCb\msNa, ॰स्तेन॰ \Ed}}% 

{\devanagarifont अथ करालं कुरुते नारी लम्बोतिचूचुका  \danda\dontdisplaylinenum }%
 
%Verse 14:18

{\devanagarifont तस्मादेतेन दोषेण करालो जायते पुमान् {॥ १४:१८॥} \veg\dontdisplaylinenum }%
     \var{{\devanagarifontvar \numnoemph\ve\textbf{॰देतेन}\lem \msCa\msCb\msNa, ॰दनेन \Ed}}% 

{\devanagarifont गृहीतगर्भा या नारी रक्तपित्तामयार्दिता \thinspace{\dandab} \dontdisplaylinenum }%
 
%Verse 14:19

{\devanagarifont गोहनुं जनयत्येषा रक्तपित्तप्रकोपिता {॥ १४:१९॥} \veg\dontdisplaylinenum }%
     \var{{\devanagarifontvar \numemph\vc\textbf{जनय॰}\lem \msCa\msCb\msNa, जनये॰ \Ed}}% 
    \var{{\devanagarifontvar \numnoemph\vd\textbf{॰कोपिता}\lem \eme, कोपिताः \msCa, ॰कोपितः \msCb\msNa\Ed}}% 

{\devanagarifont गृहीतगर्भा या नारी वातशूलैरुपद्रुता \thinspace{\dandab} \dontdisplaylinenum }%
     \var{{\devanagarifontvar \numemph\vb\textbf{॰द्रुता}\lem \msCa\msCb\Ed, ॰द्रुतैः \msNa}}% 

%Verse 14:20

{\devanagarifont शुक्रोदावर्तनी चापि पङ्गुं जनयते सुतम् {॥ १४:२०॥} \veg\dontdisplaylinenum }%
     \var{{\devanagarifontvar \numnoemph\vd\textbf{पङ्गुं}\lem \msCa\msCb\msNa, पङ्गू \Ed}}% 

{\devanagarifont क्षुधार्ता वेदनार्ता च सततं चोपवासिनी \thinspace{\dandab} \dontdisplaylinenum }%
     \var{{\devanagarifontvar \numemph\vb\textbf{सततं}\lem \msCa\msCb\msNa, \Ed}}% 

%Verse 14:21

{\devanagarifont मूकं जनयते पुत्रं दौहृदं च विमानिता {॥ १४:२१॥} \veg\dontdisplaylinenum }%
     \var{{\devanagarifontvar \numnoemph\vc\textbf{मूकं}\lem \msCa\msNa\Ed, मूलकं \msCb\oo 
\textbf{पुत्रं}\lem \msCa\msCb\msNa, बालं \Ed}}% 
    \var{{\devanagarifontvar \numnoemph\vd\textbf{॰हृदं च}\lem \msCa\msNa, ॰र्हृदं \msCb, ॰हृदश्च \Ed}}% 

{\devanagarifont गृहीतगर्भा या नारी विसृजेन्मासमासिकम् \thinspace{\dandab} \dontdisplaylinenum }%
     \var{{\devanagarifontvar \numemph\vb\textbf{मासमासिकम्}\lem \msCa\msCb\Ed, मासिमासिकं \msNa}}% 

%Verse 14:22

{\devanagarifont अनक्षो जायते तस्या गर्भशोणितसंक्षयात् {॥ १४:२२॥} \veg\dontdisplaylinenum }%
     \var{{\devanagarifontvar \numnoemph\vc\textbf{तस्या}\lem \msCa\msCb\Ed, तस्याः \msNa}}% 

{\devanagarifont अर्शग्रस्ता यदा नारी वातोदावर्तपीडिता \thinspace{\dandab} \dontdisplaylinenum }%
     \var{{\devanagarifontvar \numemph\va\textbf{अर्श॰}\lem \conjSzanto, अर्ष॰ \msCa\msCb\msNa, अथ \Ed}}% 

%Verse 14:23

{\devanagarifont गृहीतगर्भा रूक्षाणि वातलान्युपसेवते {॥ १४:२३॥} \veg\dontdisplaylinenum }%
     \var{{\devanagarifontvar \numnoemph\vc\textbf{रूक्षाणि}\lem \msCa\msCb, या नारी \msNa, रुक्षाणि \Ed}}% 
    \var{{\devanagarifontvar \numnoemph\vd\textbf{॰लान्युप॰}\lem \msCa\msCbpcorr\msNa\Ed, ॰लान्यु॰ \msCbacorr}}% 

{\devanagarifont वातस्थानं ततस्तस्या गर्भस्यापीडितं भवेत् \thinspace{\dandab} \dontdisplaylinenum }%
 
%Verse 14:24

{\devanagarifont अगुदो जायते तस्माज्जातश्चापि न जीवति {॥ १४:२४॥} \veg\dontdisplaylinenum }%
     \var{{\devanagarifontvar \numemph\vd\textbf{॰तश्चापि}\lem \msCa\Ed, ॰तं चा \msNa}}% 
    \lacuna{\devanagarifontsmall \vcd {\englishfont missing in \msCb} }%
  
{\devanagarifont देव्युवाच {\dandab}\dontdisplaylinenum  }%
 
{\devanagarifont हीनाङ्गो जायते कस्मादधिकाङ्गो ऽपि वा कथम् \thinspace{\danda} \dontdisplaylinenum }%
 
%Verse 14:25

{\devanagarifont श्वेतपिङ्गेक्षणः कस्मात्कथं लोहितलोचनः {॥ १४:२५॥} \veg\dontdisplaylinenum }%
 
{\devanagarifont भगवानुवाच {\dandab}\dontdisplaylinenum  }%
 
{\devanagarifont गर्भस्य जायमानस्य यदङ्गे जायते ऽनिलः \thinspace{\danda} \dontdisplaylinenum }%
     \var{{\devanagarifontvar \numemph\vb\textbf{निलः}\lem \msCa\msNa\Ed, निः \msCb}}% 

{\devanagarifont वाताभ्यां श्लेष्मणा तस्य तदङ्गं परिहीयते  \danda\dontdisplaylinenum }%
 
%Verse 14:26

{\devanagarifont हीनाङ्गो जायते तस्मात्पुमान्वातप्रकोपतः {॥ १४:२६॥} \veg\dontdisplaylinenum }%
 
{\devanagarifont गृहीतगर्भा या नारी मधुराण्युपसेवते \thinspace{\dandab} \dontdisplaylinenum }%
 
%Verse 14:27

{\devanagarifont शृङ्गाटकाङ्कलोड्यानि शालूकानि बिसानि च {॥ १४:२७॥} \veg\dontdisplaylinenum }%
     \var{{\devanagarifontvar \numemph\vc\textbf{॰काङ्कलोड्यानि}\lem \conj, ॰ककलोन्त्यानि \msCa\msCb, ॰कलोन्त्यानि \msNa, ॰ककलोत्यानि \Ed}}% 

{\devanagarifont मोचं तालफलं चैव नारिकेलफलं तथा \thinspace{\dandab} \dontdisplaylinenum }%
 
%Verse 14:28

{\devanagarifont अभीक्ष्णं सेवमाना तु अधिकाङ्गं प्रसूयते {॥ १४:२८॥} \veg\dontdisplaylinenum }%
     \var{{\devanagarifontvar \numemph\vc\textbf{अभीक्ष्णं}\lem \msCa\msCb\msNa, अतीक्ष्णं \Ed}}% 

{\devanagarifont पिङ्गाक्षः श्लेष्मपित्ताभ्यां श्वेताक्षः श्लेष्मणा भवेत् \thinspace{\dandab} \dontdisplaylinenum }%
     \var{{\devanagarifontvar \numemph\va\textbf{पिङ्गाक्षः}\lem \msCa\msCb\Ed, पिङ्गाक्षं \msNa}}% 

%Verse 14:29

{\devanagarifont वातपित्तेन रक्ताक्षः पुरुषस्तूपजायते {॥ १४:२९॥} \veg\dontdisplaylinenum }%
     \var{{\devanagarifontvar \numnoemph\vd\textbf{पुरुषस्तू॰}\lem \msCa\msCb\msNa, पुरुषसू॰ \Ed}}% 

{\devanagarifont देव्युवाच {\dandab}\dontdisplaylinenum  }%
 
{\devanagarifont कथं वा जायते पुत्रः कन्यका केन जायते \thinspace{\danda} \dontdisplaylinenum }%
 
%Verse 14:30

{\devanagarifont अपुमान्केन जायेत द्वियमा त्रियमा तथा {॥ १४:३०॥} \veg\dontdisplaylinenum }%
 
{\devanagarifont भगवानुवाच {\dandab}\dontdisplaylinenum  }%
 
{\devanagarifont शुक्राधिकः पुमान्ज्ञेयः कन्या रक्ताधिका भवेत् \thinspace{\danda} \dontdisplaylinenum }%
     \var{{\devanagarifontvar \numemph\va\textbf{॰धिकः}\lem \msCb\msNa\Ed, ॰धिक \msCa}}% 

%Verse 14:31

{\devanagarifont रक्तशुक्रसमत्वेन जायते स नपुंसकः {॥ १४:३१॥} \veg\dontdisplaylinenum }%
 
{\devanagarifont पिण्डीभूतो यदा गर्भं मारुतो विभजेद्द्विधा \thinspace{\dandab} \dontdisplaylinenum }%
     \var{{\devanagarifontvar \numemph\va\textbf{॰भूतो}\lem \msCa\msNa\Ed, ॰भूत \msCb\oo 
\textbf{गर्भं}\lem \msCa\msCb\msNa, गर्भ \Ed}}% 
    \var{{\devanagarifontvar \numnoemph\vb\textbf{मारुतो}\lem \msCa\msCb\msNa, मारुतौ \Ed\oo 
\textbf{॰भजेद्द्वि॰}\lem \msCa\msCb\msNa, ॰भवेद्द्वि॰ \Ed}}% 

%Verse 14:32

{\devanagarifont एवं ते द्वियमा ज्ञेयास्त्रियमाश्च त्रिधा कृते {॥ १४:३२॥} \veg\dontdisplaylinenum }%
     \var{{\devanagarifontvar \numnoemph\vd\textbf{॰यमाश्च}\lem \msCa\msCb\msNa, ॰यमा च \Ed}}% 

{\devanagarifont देव्युवाच {\dandab}\dontdisplaylinenum  }%
 
{\devanagarifont शोणितं मांस मेदं च अस्थि मज्जा च पञ्चमी \thinspace{\danda} \dontdisplaylinenum }%
     \var{{\devanagarifontvar \numemph\va\textbf{मेदं च}\lem \msCa\msCb\msNa, मेदश्च \Ed}}% 

%Verse 14:33

{\devanagarifont शरीरस्थानि दृश्यन्ते शुक्रस्थानं न दृश्यते {॥ १४:३३॥} \veg\dontdisplaylinenum }%
     \var{{\devanagarifontvar \numnoemph\vc\textbf{शरीर॰}\lem \msCapcorr\msCb\msNa\Ed, शरी॰ \msCaacorr}}% 

{\devanagarifont तस्योपपत्ति स्थानं च ज्ञातुमिच्छामि तत्त्वतः \thinspace{\dandab} \dontdisplaylinenum }%
     \var{{\devanagarifontvar \numemph\va\textbf{तस्योत्पत्तिश्च स्थानं च}\lem \msCa\msCb\msNa, तस्योत्पत्तिश्च स्थानं च \Ed}}% 

%Verse 14:34

{\devanagarifont कथयस्व त्रिलोकेश च्छेत्तुमर्हसि संशयम् {॥ १४:३४॥} \veg\dontdisplaylinenum }%
     \var{{\devanagarifontvar \numnoemph\vd\textbf{॰शयम्}\lem \msCa\msCb\msNa, ॰शयः \Ed}}% 

{\devanagarifont भगवानुवाच {\dandab}\dontdisplaylinenum  }%
 
{\devanagarifont मनः शुक्रस्य प्रभवं घ्राणं श्रोत्रं तथाक्षिणी \thinspace{\danda} \dontdisplaylinenum }%
     \var{{\devanagarifontvar \numemph\vb\textbf{घ्राणं श्रोत्रं तथाक्षिणी}\lem \msCb\Ed, घ्रा\lk\lk\lk\lk\lk\uncl{kṣiṇī} \msCa, 
घ्राणं स्रोत्र स्रवाक्षिणी \msNa}}% 

%Verse 14:35

{\devanagarifont स्थानं तु सर्वाङ्गगतं स्पर्शात्स्पर्शः प्रवर्तते {॥ १४:३५॥} \veg\dontdisplaylinenum }%
     \var{{\devanagarifontvar \numnoemph\vc\textbf{स्थानं तु सर्वाङ्गसतं}\lem \msNa, \uncl{स्थानन्तु सर्व्वा}\lk तं\lk\ \msCa, 
स्थानन्तु सर्वाङ्गतं \msCb, 
स्थानं तु सर्वाङ्गसम॰ \Ed}}% 
    \var{{\devanagarifontvar \numnoemph\vd\textbf{स्पर्शात्स्प॰}\lem \msNa\Ed, \lk र्शात्स्प॰ \msCa, स्पर्शाः स्प॰ \msCb}}% 

{\devanagarifont यथा निषिक्तं क्षीरं तु पयसा दधि जायते \thinspace{\dandab} \dontdisplaylinenum }%
     \var{{\devanagarifontvar \numemph\vb\textbf{पयसा दधि}\lem \msCa\msCb, पयसो दधि \msNa, पयसाद्दधि॰ \Ed}}% 

%Verse 14:36

{\devanagarifont प्रमथ्यमानदध्नस्तु सर्पिसो ऽपि तथागमः {॥ १४:३६॥} \veg\dontdisplaylinenum }%
     \var{{\devanagarifontvar \numnoemph\vc\textbf{॰मान॰}\lem \msCa\msNa\Ed, ॰मानं \msCb}}% 

{\devanagarifont एवं शरीरं निर्मन्थेच्छुक्रं शुक्रवहा सिरा \thinspace{\dandab} \dontdisplaylinenum }%
     \var{{\devanagarifontvar \numemph\va\textbf{शरीरं}\lem \msCa, शरीर \msCb\msNa\Ed}}% 
    \var{{\devanagarifontvar \numnoemph\vab\textbf{निर्मन्थेच्छुक्रं}\lem \msCa\msCb\msNa, निर्गच्चेत् शुक्रं \Ed}}% 
    \var{{\devanagarifontvar \numnoemph\vb\textbf{सिरा}\lem \msCa\msCb\msNa, शिराः \Ed}}% 

%Verse 14:37

{\devanagarifont पूरयित्वानुपूर्वेण अस्थयो प्रतिपद्यते {॥ १४:३७॥} \veg\dontdisplaylinenum }%
     \var{{\devanagarifontvar \numnoemph\vc\textbf{॰नु॰}\lem \msCb\Ed, तु \msCa\msNa}}% 

{\devanagarifont ततस्तु ताः शुक्रवहा मेढ्रनाडीमनुसृताः \thinspace{\dandab} \dontdisplaylinenum }%
     \var{{\devanagarifontvar \numemph\vb\textbf{मेढ्रनाडीमनुसृताः}\lem \msNa, \lk\lk\lk\lk\lk \uncl{नुसृताः} \msCa, 
मेढ्रानाडीमनुस्मृताः \msCb, 
मेढ्रनाभीमनुसृताः \Ed}}% 

%Verse 14:38

{\devanagarifont न शुक्रतन्तु सिञ्चन्ति तस्माद्गर्भस्य सम्भवः {॥ १४:३८॥} \veg\dontdisplaylinenum }%
     \var{{\devanagarifontvar \numnoemph\vc\textbf{न शुक्रतन्तु}\lem \msCa, न शुक्रन्तन्तु \msCb\msNa, 
नाशुक्रं तत्तु \Ed}}% 

{\devanagarifont देव्युवाच {\dandab}\dontdisplaylinenum  }%
 
{\devanagarifont कथं वेदयते जातिं कथं जातिस्मरो भवेत् \thinspace{\danda} \dontdisplaylinenum }%
     \var{{\devanagarifontvar \numemph\va\textbf{कथं}\lem \msCa\msNa\Ed, कथ \msCb\oo 
\textbf{जातिं}\lem \msCa\msCb\msNa, जाति \Ed}}% 

%Verse 14:39

{\devanagarifont एतस्मिन्संशयं मे ऽद्य छेत्तुमर्हसि शङ्कर {॥ १४:३९॥} \veg\dontdisplaylinenum }%
 
{\devanagarifont भगवानुवाच {\dandab}\dontdisplaylinenum  }%
 
{\devanagarifont भावितात्मा च यो जन्तुर्देवि भागाधिकं च यत् \thinspace{\danda} \dontdisplaylinenum }%
     \var{{\devanagarifontvar \numemph\va\textbf{॰त्मा}\lem \msCa\msCb\msNa, ॰त्मां \Ed}}% 
    \var{{\devanagarifontvar \numnoemph\vab\textbf{जन्तुर्दे॰}\lem \msCa\msCb\Ed, जन्तु दे॰ \msNa}}% 
    \var{{\devanagarifontvar \numnoemph\vb\textbf{भागा॰}\lem \msCa\msCb\msNa, भोगा॰ \Ed}}% 

%Verse 14:40

{\devanagarifont बुद्धिविज्ञानसंयुक्तः स जातिं स्मरते पुमान् {॥ १४:४०॥} \veg\dontdisplaylinenum }%
     \var{{\devanagarifontvar \numnoemph\vc\textbf{बुद्ध्विज्ञान॰}\lem \msCa\msCb\msNa, ब्रह्मविद्ज्ञान॰ \Ed\oo 
\textbf{॰युक्तः}\lem \msCa\msCb\Ed, ॰युक्तं \msNa}}% 
    \var{{\devanagarifontvar \numnoemph\vd\textbf{पुमान्}\lem \msCb\msNa\Ed, \uncl{पुमन्} \msCa}}% 

{\devanagarifont देव्युवाच {\dandab}\dontdisplaylinenum  }%
 
{\devanagarifont कथं सद्योगृहीतस्य लिङ्गं गर्भस्य दृश्यते \thinspace{\danda} \dontdisplaylinenum }%
     \var{{\devanagarifontvar \numemph\vb\textbf{लिङ्गं}\lem \msCa\msCb\msNa, लिङ्ग॰ \Ed}}% 

%Verse 14:41

{\devanagarifont एतत्कथय देवेश रहः काले महेश्वर {॥ १४:४१॥} \veg\dontdisplaylinenum }%
     \var{{\devanagarifontvar \numnoemph\vd\textbf{रहः काले}\lem \msCb\Ed, रह\uncl{त्का}ले \msCa, रहं काले \msNa}}% 

{\devanagarifont महेश्वर उवाच {\dandab}\dontdisplaylinenum  }%
     \var{{\devanagarifontvar \numemph\vo\textbf{महेश्वर}\lem \msCa\msCb, भगवान् \Ed}}% 

{\devanagarifont पिपासारोमहर्षश्च वेपनं गात्रसीदनम् \thinspace{\danda} \dontdisplaylinenum }%
     \var{{\devanagarifontvar \numnoemph\va\textbf{॰हर्षश्च}\lem \msCa\msCb\msNa, ॰हर्षं च \Ed}}% 

%Verse 14:42

{\devanagarifont निद्रास्वेदं च तन्द्री च मुहूर्तमुपजायते {॥ १४:४२॥} \veg\dontdisplaylinenum }%
     \var{{\devanagarifontvar \numnoemph\vc\textbf{तन्द्री}\lem \msCa\msCb\msNa, तन्द्रा \Ed}}% 

{\devanagarifont निक्लेदत्वं खरत्वं च योन्यां समुपजायते \thinspace{\dandab} \dontdisplaylinenum }%
     \var{{\devanagarifontvar \numemph\vb\textbf{योन्यां स॰}\lem \msCa\msCb\msNa, योन्यात्स॰ \Ed}}% 

{\devanagarifont न चार्तवं वै दृश्येत शुक्रस्य रजसो ऽपि वा  \danda\dontdisplaylinenum  }%
     \var{{\devanagarifontvar \numnoemph\vc\textbf{चार्तवं वै}\lem \msCa\msCb, चार्तवं \msNa, चार्द्रवंम्वै \Ed\oo 
\textbf{दृश्येत}\lem \msCb\msNa\Ed, दृ\lk\lk\ \msCa}}% 

%Verse 14:43

{\devanagarifont सद्योगृहीतगर्भाया लिङ्गान्येतानि तत्त्वतः {॥ १४:४३॥} \veg\dontdisplaylinenum }%
 
{\devanagarifont देव्युवाच {\dandab}\dontdisplaylinenum  }%
 
{\devanagarifont केन लिङ्गेन विज्ञेयं पुत्रजन्म महेश्वर \thinspace{\danda} \dontdisplaylinenum }%
 
%Verse 14:44

{\devanagarifont कन्यका केन लिङ्गेन जायते कथयस्व मे {॥ १४:४४॥} \veg\dontdisplaylinenum }%
     \var{{\devanagarifontvar \numemph\vd\textbf{जायते}\lem \msCa\msCb\msNa, ज्ञायते \Ed}}% 

{\devanagarifont भगवानुवाच {\dandab}\dontdisplaylinenum  }%
 
{\devanagarifont यदोरुजङ्घपार्श्वं च दक्षिणं यदि ह्युन्नतम् \thinspace{\danda} \dontdisplaylinenum }%
     \var{{\devanagarifontvar \numemph\va\textbf{यदोरुजंघ॰}\lem \msCb\msNa, यदोरुजर्घ्य॰ \msCa, पादोरुजङ्घ॰ \Ed\oo 
\textbf{॰पार्श्वं च}\lem \msCa\msCb\msNa, ॰पार्श्वश्च \Ed}}% 
    \var{{\devanagarifontvar \numnoemph\vb\textbf{॰न्नतम्}\lem \msCa\msCb\msNa, ॰न्नतः \Ed}}% 

%Verse 14:45

{\devanagarifont दक्षिणं विपुलं नेत्रं तदा पुत्रः प्रजायते {॥ १४:४५॥} \veg\dontdisplaylinenum }%
     \var{{\devanagarifontvar \numnoemph\vc\textbf{नेत्रं}\lem \msCa\msCb\msNa, तत्र \Ed}}% 
    \var{{\devanagarifontvar \numnoemph\vd\textbf{पुत्रः}\lem \msCa\msCb\Ed, पुत्रं \msNa\oo 
\textbf{प्रजायते}\lem \msCb\msNa\Ed, प्रजा\lk ते \msCa}}% 

{\devanagarifont वामं चैव यदा पश्येत्तदा जायेत कन्यका \thinspace{\dandab} \dontdisplaylinenum }%
     \var{{\devanagarifontvar \numemph\va\textbf{वामं चैव}\lem \msCa\msCb\msNa, वामश्चैव \Ed}}% 
    \var{{\devanagarifontvar \numnoemph\vb\textbf{जायेत क॰}\lem \msCb\msNa\Ed, जायेत्क॰ \msCa}}% 

%Verse 14:46

{\devanagarifont उन्नतं मध्यमस्थानं तदा जायेन्नपुंसकः {॥ १४:४६॥} \veg\dontdisplaylinenum }%
     \var{{\devanagarifontvar \numnoemph\vd\textbf{मध्यम॰}\lem \msCa\msCb\Ed, मध्य\uncl{मं} \msNa\oo 
\textbf{॰स्थानं}\lem \msCa\msCb\msNa, स्थाश्च \Ed}}% 
    \var{{\devanagarifontvar \numnoemph\vd\textbf{जायेन्न}\lem \msCa\msNa\Ed, जायेत न \msCb\oo 
\textbf{॰पुंसकः}\lem \msCa\msCb\msNa, ॰पुंसकम् \Ed}}% 

{\devanagarifont देव्युवाच {\dandab}\dontdisplaylinenum  }%
 
{\devanagarifont पुंसां कपोलरोमानि खलितं केन जायते \thinspace{\danda} \dontdisplaylinenum }%
     \var{{\devanagarifontvar \numemph\va\textbf{पुंसां}\lem \msCa\msCb\msNa, पुंसा \Ed\oo 
\textbf{॰रोमानि}\lem \msCa\msNa\Ed, ॰रोगानि \msCb}}% 

%Verse 14:47

{\devanagarifont कथं स्त्रीणां न जायेत रोमाणि खलितं तथा {॥ १४:४७॥} \veg\dontdisplaylinenum }%
 
{\devanagarifont भगवानुवाच {\dandab}\dontdisplaylinenum  }%
 
{\devanagarifont तथा वृषणगा जन्तोर् यस्य रेतोवहा सिरा \thinspace{\danda} \dontdisplaylinenum }%
     \var{{\devanagarifontvar \numemph\vb\textbf{सिरा}\lem \msCa\msCb\msNa, शिरः \Ed}}% 

%Verse 14:48

{\devanagarifont निबद्धा मस्तके तास्तु कपोलास्तु समाश्रिताः {॥ १४:४८॥} \veg\dontdisplaylinenum }%
     \var{{\devanagarifontvar \numnoemph\vc\textbf{तास्तु}\lem \msCa\msCb\msNa, तालु \Ed}}% 

{\devanagarifont तैः कपोलेषु रोमाणि जायन्ते अन्तरेतसः \thinspace{\dandab} \dontdisplaylinenum }%
     \var{{\devanagarifontvar \numemph\va\textbf{रोमाणि}\lem \msCa\msCb\msNapcorr\Ed, \om\ \msNaacorr}}% 

%Verse 14:49

{\devanagarifont खलितं शुक्रदोषेण नराणामुपजायते {॥ १४:४९॥} \veg\dontdisplaylinenum }%
 
{\devanagarifont सिरा शुक्रवहा स्त्रीणां न स्याद्यस्मान्न जायते \thinspace{\dandab} \dontdisplaylinenum }%
     \var{{\devanagarifontvar \numemph\vb\textbf{स्याद्यस्मान्न}\lem \conj, शून्यस्मान्न \msCa\msCb, शून्यस्यान्न \Ed}}% 

%Verse 14:50

{\devanagarifont यो त्माषालो च कस्त्वग्निर्दृष्टिमण्डलसंश्रितः {॥ १४:५०॥} \veg\dontdisplaylinenum }%
     \var{{\devanagarifontvar \numnoemph\vc\textbf{योत्माषालो च}\lem \msCa, योत्मा वालो च \msCb, योन्मयालोम \msNa, यात्मापालो च \Ed}}% 
    \var{{\devanagarifontvar \numnoemph\vd\textbf{कस्त्वग्निर्दृ॰}\lem \msCa\msCb\msNa, कास्त्वग्नि दृ॰ \Ed}}% 

{\devanagarifont शोणितं सोक्तिकोष्टस्थन्निशोषयति तत्त्वतः \thinspace{\dandab} \dontdisplaylinenum }%
     \var{{\devanagarifontvar \numemph\va\textbf{शोणितं}\lem \msCa\msCb\msNa, शोणितै \Ed}}% 
    \var{{\devanagarifontvar \numnoemph\vab\textbf{सोक्तिकोष्टस्थन्नि॰}\lem \Ed, सो\lk\lk ष्टस्थ नि॰ \msCa, सोक्षिकोष्ठस्थन्नि॰ \msCb, 
सोक्षिकोष्ठस्थं नि॰ \msNa}}% 

%Verse 14:51

{\devanagarifont न वर्धन्ते ऽक्षिपक्ष्माणि तेन रोमाणि च भ्रुवोः {॥ १४:५१॥} \veg\dontdisplaylinenum }%
     \var{{\devanagarifontvar \numnoemph\vc\textbf{न वर्धन्ते ऽक्षि}\lem \msCa\msCb\msNa, निबद्धन्त्यक्षि॰ \Ed}}% 
    \var{{\devanagarifontvar \numnoemph\vd\textbf{भ्रुवोः}\lem \msCa\msCb\Ed, भ्रुवो \msNa}}% 

{\devanagarifont अशुक्रत्वाच्च नारीणां खलितं नोपजायते \thinspace{\dandab} \dontdisplaylinenum }%
 
{\devanagarifont छायाव्यपगतस्नेहा रूक्षागात्रशिरोरुहा  \danda\dontdisplaylinenum }%
     \var{{\devanagarifontvar \numemph\vd\textbf{रूक्षा॰}\lem \msCa\msCb\msNa, रुक्षा \Ed}}% 

%Verse 14:52

{\devanagarifont उद्भूतोस्माभजठरा मृतगर्भः प्रजायते {॥ १४:५२॥} \veg\dontdisplaylinenum }%
     \var{{\devanagarifontvar \numnoemph\ve\textbf{उद्भूतोस्माभ॰}\lem \msCa\msNa, उद्भूतोम्मभ्॰ \msCb, ग्रसतोस्माभ \Ed\oo 
\textbf{॰जठरा}\lem \msCa\msCb\Ed, ॰जठरो \msNa}}% 

{\devanagarifont देव्युवाच {\dandab}\dontdisplaylinenum  }%
 
{\devanagarifont सोमधातु कति ज्ञेया अग्निधातुस्तथेश्वर \thinspace{\danda} \dontdisplaylinenum }%
     \var{{\devanagarifontvar \numemph\va\textbf{कति}\lem \msCa\msNa, ॰क वि॰ \msCb, कथंर \Ed}}% 
    \var{{\devanagarifontvar \numnoemph\vb\textbf{॰थेश्वर}\lem \msCa\msNa\Ed, ॰थैव च \msCb}}% 

%Verse 14:53

{\devanagarifont पृथग्भागविशेषेण कथयस्व महेश्वर {॥ १४:५३॥} \veg\dontdisplaylinenum }%
 
{\devanagarifont महेश्वरउवाच {\dandab}\dontdisplaylinenum  }%
     \var{{\devanagarifontvar \numemph\vo\textbf{महेश्वर}\lem \msCa\msCb\msNa, भगवान् \Ed}}% 

{\devanagarifont श्लेष्मा मेदस्तथा स्नायु अस्थि दन्त नखानि च \thinspace{\danda} \dontdisplaylinenum }%
     \var{{\devanagarifontvar \numnoemph\va\textbf{श्लेष्मा}\lem \msCa\msCb\msNa, श्लेष्म \Ed\oo 
\textbf{स्नायु}\lem \msCa\msCb\msNa, स्नायुः \Ed}}% 

{\devanagarifont स्त्रियाः स्तन्यं च शुक्रं च यच्च श्वेतं तथाक्षिषु  \danda\dontdisplaylinenum }%
     \var{{\devanagarifontvar \numnoemph\vc\textbf{स्त्रियाः स्तन्यं च}\lem \conj, स्त्रिया स्तन्यं च \msCa\msCb\msNa, 
स्त्रियास्तन्यश्च \Ed\oo 
\textbf{शुक्रं च}\lem \msCa\msCb\msNa, शुक्रश्च \Ed}}% 

%Verse 14:54

{\devanagarifont एतेषां सौम्यभावत्वाच्छ्वेतत्वमुपजायते {॥ १४:५४॥} \veg\dontdisplaylinenum }%
     \var{{\devanagarifontvar \numnoemph\ve\textbf{॰भावत्वा॰}\lem \msCa\msCb\msNa, ॰भागत्वा \Ed}}% 

{\devanagarifont आग्नेयभावाद्रक्तत्वं कृष्णत्वं चापि गच्छति \thinspace{\dandab} \dontdisplaylinenum }%
     \var{{\devanagarifontvar \numemph\va\textbf{॰क्तत्वं}\lem \msCa\msNa\Ed, ॰क्तत्व \msCb}}% 

%Verse 14:55

{\devanagarifont त्वग्मांस रुधिर मज्जा दृष्टिरोम तथैव च {॥ १४:५५॥} \veg\dontdisplaylinenum }%
     \var{{\devanagarifontvar \numnoemph\vc\textbf{रुधिर}\lem \msCa\msCb, रुधिरं \msNa\Ed}}% 
    \var{{\devanagarifontvar \numnoemph\vd\textbf{॰रोम}\lem \msCa\msCb\Ed, ॰रोमं \msNa}}% 

{\devanagarifont आग्नेयधातुं सोमं च कथितो ऽस्मि वरानने \thinspace{\dandab} \dontdisplaylinenum }%
     \var{{\devanagarifontvar \numemph\va\textbf{आग्नेयधातुं सोमं च}\lem \msCb, धातुं सोतुञ्च \msCa, 
आग्नेयधातु सोमं च \msNa, धातु सोमश्च \Ed}}% 

%Verse 14:56

{\devanagarifont ब्रूहि ब्रूहि विशालाक्षि यद्यस्ति तव संशयः {॥ १४:५६॥} \veg\dontdisplaylinenum }%
 
{\devanagarifont 
\jump
\begin{center}
\ketdanda~इति वृषसारसंग्रहे प्रश्नव्याकरणो नामाध्यायश्चतुर्दशमः~\ketdanda
\end{center}
\dontdisplaylinenum\vers  }%
     \var{{\devanagarifontvar \numnoemph{\englishfont \Colo:}\textbf{नामाध्यायश्चतुर्दशमः}\lem \msCa\msCb\msNa, नामश्चतुर्दशो ऽध्यायः \Ed}}% 
\bekveg\szamveg
\vfill
\phpspagebreak

\versno=0\fejno=15
\thispagestyle{empty}

\centerline{\Large\devanagarifontbold [   पञ्चदशमो ऽध्यायः  ]}{\vrule depth10pt width0pt} \fancyhead[CE]{{\footnotesize\devanagarifont वृषसारसंग्रहे  }}
\fancyhead[CO]{{\footnotesize\devanagarifont पञ्चदशमो ऽध्यायः  }}
\fancyhead[LE]{}
\fancyhead[RE]{}
\fancyhead[LO]{}
\fancyhead[RO]{}
\szam\bek



\alalfejezet{जीववर्णनम्}
{\devanagarifont देव्युवाच {\dandab}\dontdisplaylinenum  }%
 
{\devanagarifont जीवभूतेति यत्प्रोक्तं लक्षणं कीदृशं भवेत् \thinspace{\danda} \dontdisplaylinenum }%
     \var{{\devanagarifontvar \numemph\vb\textbf{लक्षणं की॰}\lem \msNa\msNb\msNc\Ed, लक्षणाङ्की॰ \msCa, लणं की॰ \msCb}}% 

%Verse 15:1

{\devanagarifont स्थानमस्य न जानामि रूपं वर्णं च ईश्वर {॥ १५:१॥} \veg\dontdisplaylinenum }%
     \var{{\devanagarifontvar \numnoemph\vc\textbf{स्थानमस्य}\lem \msCb\msNa\msNb\msNc\Ed, {\il}\uncl{न}मस्य \msCa}}% 
    \var{{\devanagarifontvar \numnoemph\vd\textbf{रूपं वर्णं}\lem \msCa\msCb\msNa\Ed, रूपवर्णं \msNb\msNc}}% 
    \paral{{\devanagarifontsmall {\englishfont Witnesses used for this chapter: \msCa\ ff.\thinspace 219r--220r, 
                                              \msCb\ ff.\thinspace 222v--223v, 
                                              \msCc\ is not available for this chapter,
                                              \msNa\ ff.\thinspace 26r--27r, 
                                              \msNb\ ff.\thinspace 230v--231r, 
                                              \msNc\ ff.\thinspace 234r--235r} }}

{\devanagarifont एतत्कौतूहलं छिन्धि संशयं परमेश्वर \thinspace{\dandab} \dontdisplaylinenum }%
     \var{{\devanagarifontvar \numemph\va\textbf{एतत्कौतूहलं}\lem \msCa\msCb\msNa\msNb\Ed, एतत्कौतूलं \msNc\oo 
\textbf{छिन्धि}\lem \msCa\msCb\msNa\msNb\Ed, छित्वान्धि \msNc}}% 
    \var{{\devanagarifontvar \numnoemph\vb\textbf{संशयं}\lem \msCa\msCb\msNa\msNc\Ed, संशय \msNb}}% 

%Verse 15:2

{\devanagarifont न चान्यदेव पश्यामि जीवनिर्णय कीर्तय {॥ १५:२॥} \veg\dontdisplaylinenum }%
 
{\devanagarifont ईश्वर उवाच {\dandab}\dontdisplaylinenum  }%
     \var{{\devanagarifontvar \numemph\vo\textbf{ईश्वर}\lem \msCa\msCb\msNa\msNb\msNc, भगवान् \Ed}}% 

{\devanagarifont जीवस्य लक्षणं देवि कथितुं केन शक्यते \thinspace{\danda} \dontdisplaylinenum }%
     \var{{\devanagarifontvar \numnoemph\va\textbf{लक्षणं}\lem \msCb\msNa\msNb\msNc\Ed, कथितं \msCa}}% 

%Verse 15:3

{\devanagarifont न रूपवर्णं जीवस्य विद्यते स्थानमेव च {॥ १५:३॥} \veg\dontdisplaylinenum }%
     \var{{\devanagarifontvar \numnoemph\vc\textbf{॰वर्णं}\lem \msCb\msNa\msNc, ॰वर्ण \msCa\msNb\Ed}}% 

{\devanagarifont व्यापि सर्वगतं सूक्ष्मं सर्वमाश्रित्य तिष्ठति \thinspace{\dandab} \dontdisplaylinenum }%
     \var{{\devanagarifontvar \numemph\va\textbf{व्यापि}\lem \msCb\msNa\msNb\msNc, व्या\uncl{पि} \msCa, व्यापी \Ed}}% 
    \var{{\devanagarifontvar \numnoemph\va\textbf{॰श्रित्य}\lem \msCb, ॰शृत्य \msCa\msNa\msNb, ॰श्रुत्य \msNc, ॰वृत्य \Ed}}% 

%Verse 15:4

{\devanagarifont निरालम्बमनाधारमनौपम्यं निरञ्जनम् {॥ १५:४॥} \veg\dontdisplaylinenum }%
     \var{{\devanagarifontvar \numnoemph\vd\textbf{॰पम्यं}\lem \msCa\msCb\msNa\msNc\Ed, ॰पम्य \msNb}}% 

{\devanagarifont अरणिस्थो यथा वह्निः काष्ठेषु नोपलभ्यते \thinspace{\dandab} \dontdisplaylinenum }%
 
%Verse 15:5

{\devanagarifont तद्वज्जीवो न पश्येत शरीरस्थो ऽपि सुन्दरि {॥ १५:५॥} \veg\dontdisplaylinenum }%
     \var{{\devanagarifontvar \numemph\vc\textbf{जीवो न}\lem \msCb\msNa\msNb\msNc, जीवोन्न \msCa, जीवं न \Ed}}% 
    \var{{\devanagarifontvar \numnoemph\vd\textbf{ऽपि}\lem \msCa\msCb\msNa\msNc\Ed, हि \msNb}}% 

{\devanagarifont दधिवच्च यथा सर्पिर्दृश्यते न च दृश्यते \thinspace{\dandab} \dontdisplaylinenum }%
 
%Verse 15:6

{\devanagarifont तद्वज्जीवः शरीरस्थो दृश्यते न च दृश्यते {॥ १५:६॥} \veg\dontdisplaylinenum }%
     \var{{\devanagarifontvar \numemph\vc\textbf{तद्वज्जीवः}\lem \msCa\msCb\msNa\msNb, तद्व जीवः \msNc, तद्वज्जीव \Ed}}% 

{\devanagarifont देव्युवाच {\dandab}\dontdisplaylinenum  }%
 
{\devanagarifont अदृष्टप्रत्ययो ह्यस्ति नास्ति प्रत्ययदर्शनम् \thinspace{\danda} \dontdisplaylinenum }%
 
%Verse 15:7

{\devanagarifont व्यापी कथं महादेव सर्वत्रावस्थितः कथम् {॥ १५:७॥} \veg\dontdisplaylinenum }%
     \var{{\devanagarifontvar \numemph\vd\textbf{॰स्थितः}\lem \msCb\msNc\Ed, ॰स्थितं \msCa\msNa, ॰स्थित \msNb}}% 

{\devanagarifont महेश्वरउवाच {\dandab}\dontdisplaylinenum  }%
     \var{{\devanagarifontvar \numemph\vo\textbf{महेश्वर}\lem \msCa\msCb\msNb\msNc, महादेव \msNa, भगवान् \Ed}}% 

{\devanagarifont असंशयो महादेवि व्यापी सर्वगतः शिवः \thinspace{\danda} \dontdisplaylinenum }%
 
%Verse 15:8

{\devanagarifont दृश्यतेन्द्रियसंयोगाज्जीवप्रत्ययदर्शनम् {॥ १५:८॥} \veg\dontdisplaylinenum }%
     \var{{\devanagarifontvar \numnoemph\vc\textbf{दृश्यते॰}\lem \msCb\msNa\msNb\msNc, दृश्येते॰ \msCa, दृश्यन्ते \Ed}}% 
    \var{{\devanagarifontvar \numnoemph\vd\textbf{॰जीव॰}\lem \msCa\msCb\msNa\msNb\Ed, ॰जी॰ \msNc}}% 

{\devanagarifont यथाकाशस्थितो वायुः शब्दस्पर्शगुणान्वितः \thinspace{\dandab} \dontdisplaylinenum }%
     \var{{\devanagarifontvar \numemph\vab\textbf{वायुः शब्द॰}\lem \msCb\msNa\msNb\msNc\Ed, वायु\uncl{श्श}\lk\ \msCa}}% 
    \var{{\devanagarifontvar \numnoemph\vb\textbf{॰न्वितः}\lem \msCa\msNa\msNb\msNc\Ed, ॰न्वितम् \msCb}}% 

%Verse 15:9

{\devanagarifont तद्वद्देही विजानीयाद्गुणचेष्टेन नान्यथा {॥ १५:९॥} \veg\dontdisplaylinenum }%
     \var{{\devanagarifontvar \numnoemph\vd\textbf{॰चेष्टेन}\lem \msCa\msCb\msNa\msNb, ॰वेष्टन \msNc, ॰वेष्टेन \Ed}}% 

{\devanagarifont देव्युवाच {\dandab}\dontdisplaylinenum  }%
 
{\devanagarifont व्यापीति कथितः पूर्वं जीवः सर्वगतो ऽपि च \thinspace{\danda} \dontdisplaylinenum }%
     \var{{\devanagarifontvar \numemph\va\textbf{कथितः}\lem \msCa\msNa\msNcpcorr\Ed, कथितं \msCb\msNb, कथतिः \msNcacorr}}% 

%Verse 15:10

{\devanagarifont तं वृथा कथितो ऽस्यद्य म्रियते केन हेतुना {॥ १५:१०॥} \veg\dontdisplaylinenum }%
     \var{{\devanagarifontvar \numnoemph\vc\textbf{वृथा}\lem \msCa\msCb\msNa\msNb\Ed, व्यथा \msNc\oo 
\textbf{ऽस्यद्य}\lem \msCa\msCb\msNc, स्म्यद्य \msNa\Ed, स्य{\il} \msNb}}% 

{\devanagarifont ईश्वर उवाच {\dandab}\dontdisplaylinenum  }%
     \var{{\devanagarifontvar \numemph\vo\textbf{ईश्वर}\lem \msCa\msCb\msNb\msNc, भगवान् \msNa\Ed}}% 

{\devanagarifont न जीवो म्रियते देवि सर्वेषां सुरसुन्दरि \thinspace{\danda} \dontdisplaylinenum }%
 
%Verse 15:11

{\devanagarifont घटान्तस्थो यथाकाशो बहिराकाशवद्यथा {॥ १५:११॥} \veg\dontdisplaylinenum }%
 
{\devanagarifont घटभिन्ने विशालाक्षि विशेषो नोपलक्ष्यते \thinspace{\dandab} \dontdisplaylinenum }%
     \var{{\devanagarifontvar \numemph\vb\textbf{नोपलक्ष्यते}\lem \msCa\msCb\msNb\msNc\Ed, नोपलभ्यते \msNa}}% 

%Verse 15:12

{\devanagarifont देहभिन्ने यदा देवि विनाशो नोपलभ्यते {॥ १५:१२॥} \veg\dontdisplaylinenum }%
     \var{{\devanagarifontvar \numnoemph\vc\textbf{देह॰}\lem \msCa\msNa\msNb\msNc\Ed, देहे \msCb\oo 
\textbf{यदा देवि}\lem \msCa\msCb\msNa\msNb\msNc, तथा देही \Ed}}% 
    \paral{{\devanagarifontsmall \vo {\englishfont cf.\ Bhāgavatapurāṇa 12.5.5:}
                          घटे भिन्ने घटाकाश आकाशः स्याद् यथा पुरा\thinspace{\devanagarifontsmall ।}
                          एवं देहे मृते जीवो ब्रह्म सम्पद्यते पुनः\thinspace{\devanagarifontsmall ॥} }}

{\devanagarifont सुसूक्ष्मः सर्वगो व्यापी परमात्मानमव्ययः \thinspace{\dandab} \dontdisplaylinenum }%
     \var{{\devanagarifontvar \numemph\va\textbf{सुसूक्ष्मः}\lem \msCa\msCb\msNa\msNb, सुसूक्ष्म \msNc, स सूक्ष्मः \Ed}}% 

%Verse 15:13

{\devanagarifont बहिरन्तश्च भूतानामचरश्चर एव सः {॥ १५:१३॥} \veg\dontdisplaylinenum }%
     \var{{\devanagarifontvar \numnoemph\vd\textbf{॰चरश्च॰}\lem \msCa\msCb\msNa\msNb\msNc, ॰चरन्च॰ \Ed\oo 
\textbf{सः}\lem \msCa\msCb\msNa\msNb\msNc, स \Ed}}% 
    \paral{{\devanagarifontsmall \vcd {\englishfont \similar\ MBh 6.35.15ab:} बहिर् अन्तश्च भूतानाम् अचरं चरमेव च }}

{\devanagarifont अप्रमेयो ऽविनाशी च अप्रपञ्चः प्रपञ्चकः \thinspace{\dandab} \dontdisplaylinenum }%
     \var{{\devanagarifontvar \numemph\vab\textbf{(अप्रमेयो{\englishfont ...} प्रपञ्चकः)}\lem \msCa\msCb\msNa\msNc\Ed, \om\ \msNb}}% 

%Verse 15:14

{\devanagarifont सर्वेन्द्रियगुणाभासः सर्वेन्द्रियविवर्जितः {॥ १५:१४॥} \veg\dontdisplaylinenum }%
     \paral{{\devanagarifontsmall \vcd {\englishfont \similar\ MBh 6.35.14ab:} सर्वेन्द्रियगुणाभासं सर्वेन्द्रियविवर्जितम् }}

{\devanagarifont एवमेष महादेवि जीवस्य वरवर्णिनि \thinspace{\dandab} \dontdisplaylinenum }%
 
%Verse 15:15

{\devanagarifont कथितो ऽस्मि समासेन किमन्यच्छ्रोतुमिच्छसि {॥ १५:१५॥} \veg\dontdisplaylinenum }%
     \var{{\devanagarifontvar \numemph\vd\textbf{इच्छसि}\lem \msCb\msNa\msNb\msNc\Ed, इच्छति \msCa}}% 


\alalfejezet{सारश्रेष्ठम्}
{\devanagarifont देव्युवाच {\dandab}\dontdisplaylinenum  }%
 
{\devanagarifont सारश्रेष्ठं महादेव कथयेशान ईश्वर \thinspace{\danda} \dontdisplaylinenum }%
     \var{{\devanagarifontvar \numemph\va\textbf{॰श्रेष्ठं}\lem \msCb\msNa\Ed, ॰श्रेष्ठ \msCa\msNb\msNc}}% 

%Verse 15:16

{\devanagarifont श्रोतुमिच्छामि देवेश मानुषाणां हितं वद {॥ १५:१६॥} \veg\dontdisplaylinenum }%
     \var{{\devanagarifontvar \numnoemph\vd\textbf{वद}\lem \msCa\msCb\msNa\msNb, वदः \msNc\Ed}}% 

{\devanagarifont ईश्वरउवाच {\dandab}\dontdisplaylinenum  }%
     \var{{\devanagarifontvar \numemph\vo\textbf{ईश्वर}\lem \msCa\msCb\msNa\msNb\msNc, भगवान् \Ed}}% 

{\devanagarifont आश्रमाणां गृही श्रेष्ठो वर्णश्रेष्ठा द्विजातयः \thinspace{\danda} \dontdisplaylinenum }%
     \var{{\devanagarifontvar \numnoemph\va\textbf{आश्रमाणां}\lem \msCa\msCb\msNa\msNc\Ed, आश्रमाणा \msNb\oo 
\textbf{गृही}\lem \msCb\msNa\msNb\msNc\Ed, गृ\uncl{ही} \msCa}}% 
    \var{{\devanagarifontvar \numnoemph\vb\textbf{॰श्रेष्ठा}\lem \msCa\msCb\msNc, ॰श्रेष्टो \msNa\msNb\Ed}}% 

%Verse 15:17

{\devanagarifont अश्वमेधः क्रतुश्रेष्ठो जपश्रेष्ठो ऽघमर्षणः {॥ १५:१७॥} \veg\dontdisplaylinenum }%
     \var{{\devanagarifontvar \numnoemph\vd\textbf{जप॰}\lem \msCapcorr\msNa\msNb\msNc\Ed, ज॰ \msCaacorr, ऽजप॰ \msCb\oo 
\textbf{ऽघमर्षणः}\lem \msCb\msNa\msNb\msNc\Ed, र्घमर्षणः \msCa}}% 

{\devanagarifont देवतानां हरिः श्रेष्ठः श्रेष्ठा गङ्गा नदीषु च \thinspace{\dandab} \dontdisplaylinenum }%
     \var{{\devanagarifontvar \numemph\vab\textbf{श्रेष्ठः श्रेष्ठा गङ्गा}\lem \msCa\msNa\msNb\msNc\Ed, श्रेष्ठा गङ्गाणाञ्च \msCb}}% 

%Verse 15:18

{\devanagarifont अनाशनस्तपःश्रेष्ठस्तीर्थश्रेष्ठः सुरद्रहः {॥ १५:१८॥} \veg\dontdisplaylinenum }%
     \var{{\devanagarifontvar \numnoemph\vc\textbf{अनाशन॰}\lem \msCa\msCb\msNa\msNb\Ed, अनशन॰ \msNc}}% 
    \var{{\devanagarifontvar \numnoemph\vd\textbf{॰र्थश्रेष्ठः}\lem \msCa\msCb\msNa\msNb\Ed, ॰र्थश्रेष्ठ \msNc\oo 
\textbf{॰द्रहः}\lem \msCa\msCb\msNa\msNb\msNc, ॰ह्रदः \Ed}}% 

{\devanagarifont क्षोमं वस्त्रेषु च श्रेष्ठं यशः श्रेष्ठं विभूषणम् \thinspace{\dandab} \dontdisplaylinenum }%
     \var{{\devanagarifontvar \numemph\va\textbf{क्षौमं}\lem \msNc\Ed, क्षोमं \msCa\msCb\msNa, क्षोम \msNb}}% 
    \var{{\devanagarifontvar \numnoemph\vb\textbf{श्रेष्ठं}\lem \msCa\msCb\msNa\msNc\Ed, श्रेष्ठ \msNb\oo 
\textbf{॰भूषणम्}\lem \msCa\msNa\msNb\msNc\Ed, ॰भूषिणम् \msCb}}% 

%Verse 15:19

{\devanagarifont भारतं श्रुतिषु श्रेष्ठं व्रतश्रेष्ठो दयापरः {॥ १५:१९॥} \veg\dontdisplaylinenum }%
     \var{{\devanagarifontvar \numnoemph\vd\textbf{॰श्रेष्ठो}\lem \msCa\msCb\msNa\msNc\Ed, श्रेष्ठं \msNb\oo 
\textbf{दयापरः}\lem \msCb\msNa\msNb\msNc\Ed, \uncl{दयाप}रः \msCa}}% 

{\devanagarifont दानेषु चाभयं श्रेष्ठं मनः श्रेष्ठेन्द्रियेषु च \thinspace{\dandab} \dontdisplaylinenum }%
 
%Verse 15:20

{\devanagarifont विद्या संग्रहषु श्रेष्ठा सत्यं श्रेष्ठं वचःसु च {॥ १५:२०॥} \veg\dontdisplaylinenum }%
     \var{{\devanagarifontvar \numemph\vc\textbf{संग्रहषु}\lem \msCa\msCb\msNa\msNb\Ed, संग्रहेषु \msNc\ \unmetr\oo 
\textbf{श्रेष्ठा}\lem \msCa\msCb\msNa\msNb\msNc, श्रेष्ठो \Ed}}% 

{\devanagarifont आयुधानां धनुः श्रेष्ठं बान्धवेषु च मातरः \thinspace{\dandab} \dontdisplaylinenum }%
     \var{{\devanagarifontvar \numemph\va\textbf{श्रेष्ठं}\lem \msCa\msCb\msNa\msNc\Ed, श्रेष्ठ \msNb}}% 
    \var{{\devanagarifontvar \numnoemph\vb\textbf{बान्धवेषु च मातरः}\lem \msCa\msCb\msNaacorr\msNc\Ed, बान्धवेषु च मातरं \msNapcorr, 
ग्रहश्रेष्ठो दिवाकरः \eyeskip{\englishfont तो १५.२४ब्} \msNb}}% 

%Verse 15:21

{\devanagarifont ज्ञानमौषधिषु श्रेष्ठं वैद्यश्रेष्ठः शिवाक्षरः {॥ १५:२१॥} \veg\dontdisplaylinenum }%
     \var{{\devanagarifontvar \numnoemph\vcd\textbf{(ज्ञान॰{\englishfont ...} शिवाक्षरः)}\lem \msCa\msCb\msNa\msNc\Ed, \om\ \msNb}}% 
    \var{{\devanagarifontvar \numnoemph\vc\textbf{ज्ञानमोषधिषु}\lem \msNc, ज्ञानमौषधिषु \msCa\msCb\msNa\msNb\Ed}}% 
    \var{{\devanagarifontvar \numnoemph\vd\textbf{वैद्य॰}\lem \msCa\msCb\msNa, \om\ \msNb, वैद्यः \msNc, वैद्यो \Ed\oo 
\textbf{॰श्रेष्ठः}\lem \msCb\msNa\msNc\Ed, ॰श्रेष्ठ \msCa, \om\ \msNb}}% 

{\devanagarifont अकारश्चाक्षरः श्रेष्ठो धर्मश्रेष्ठो ह्यहिंसकः \thinspace{\dandab} \dontdisplaylinenum }%
 
%Verse 15:22

{\devanagarifont पशुषु सौरभी श्रेष्ठा नरेषु च नराधिपः {॥ १५:२२॥} \veg\dontdisplaylinenum }%
     \var{{\devanagarifontvar \numemph\vo\textbf{(अकार॰{\englishfont ...} नराधिपः)}\lem \msCa\msCb\msNa\msNc\Ed, \om\ \msNb}}% 

{\devanagarifont मासि मार्गशिरः श्रेष्ठं कृतः श्रेष्ठश्चतुर्युगे \thinspace{\dandab} \dontdisplaylinenum }%
     \var{{\devanagarifontvar \numemph\vo\textbf{(मासि{\englishfont ...} चायनमुत्तरम)}\lem \msCa\msCb\msNa\msNc\Ed, \om\ \msNb}}% 
    \var{{\devanagarifontvar \numnoemph\va\textbf{मासि}\lem \msCa\msCb\msNa\msNc, \om\ \msNb, मासी \Ed\oo 
\textbf{॰शिरः}\lem \msCa\msCb\msNa\msNb\Ed, ॰शिर \msNc}}% 
    \var{{\devanagarifontvar \numnoemph\vb\textbf{श्रेष्ठश्चतुर्युगे}\lem \msCa\msNa\Ed, श्रेष्ठं चतुर्युगे \msCb, 
\om\ \msNb, श्रेष्ठश्चर्युगे \msNc}}% 

%Verse 15:23

{\devanagarifont वसन्त ऋतुषु श्रेष्ठः श्रेष्ठं चायनमुत्तरम् {॥ १५:२३॥} \veg\dontdisplaylinenum }%
     \var{{\devanagarifontvar \numnoemph\vd\textbf{श्रेष्ठं चा॰}\lem \msCa\msCb\msNc\Ed, श्रेष्ठश्चा॰ \msNa, \om\ \msNb\oo 
\textbf{॰त्तरम्}\lem \msCa\msNa\msNc\Ed, ॰त्त\uncl{मे}म् \msCb, \om\ \msNb}}% 

{\devanagarifont अमावास्या दिनश्रेष्ठा ग्रहश्रेष्ठो दिवाकरः \thinspace{\dandab} \dontdisplaylinenum }%
     \var{{\devanagarifontvar \numemph\va\textbf{अमावास्या दिनश्रेष्ठा}\lem \msCa\msCb\msNc\Ed, \om\ \msNb, अमावास्या दिनश्रेष्ठो \msNa}}% 
    \var{{\devanagarifontvar \numnoemph\vb\textbf{ग्रहश्रेष्ठो दिवाकरः}\lem \msCa\msCb\msNa\msNb, ग्रहः श्रेष्ठो दिवाकरः \msNc, 
वसुश्रेष्ठो हुताशनः \Ed}}% 

%Verse 15:24

{\devanagarifont स्त्रीषु लक्ष्मीर्धृतिः श्रेष्ठा वसुश्रेष्ठो हुताशनः {॥ १५:२४॥} \veg\dontdisplaylinenum }%
     \var{{\devanagarifontvar \numnoemph\vcd\textbf{(स्त्रीषु{\englishfont ...} हुताशनः)}\lem \msCa\msCb\msNa\msNb\msNc, \om\ \Ed}}% 
    \var{{\devanagarifontvar \numnoemph\vc\textbf{स्त्रीषु}\lem \msCa\msNa\msNb\msNc, स्त्री \msCb, \om\ \Ed\oo 
\textbf{लक्ष्मीर्धृतिः}\lem \msCa, लक्ष्मीधृतिः \msCb\msNa\msNb\msNc, \om\ \Ed}}% 

{\devanagarifont ऋषिषु उशणा श्रेष्ठः कान्तिश्रेष्ठो निशाकरः \thinspace{\dandab} \dontdisplaylinenum }%
     \var{{\devanagarifontvar \numemph\va\textbf{उशणा}\lem \corr, उशनाः \msCa\msCb\msNa\msNb\msNc, उशनः \Ed}}% 
    \var{{\devanagarifontvar \numnoemph\vb\textbf{कान्ति॰}\lem \msCb\msNa\msNb\Ed, कान्तिः \msNc, का\lk\  \msCa}}% 

%Verse 15:25

{\devanagarifont नक्षत्रेष्वभिजित् श्रेष्ठः कालः श्रेष्ठः कलेषु च  {॥ १५:२५॥} \veg\dontdisplaylinenum }%
     \var{{\devanagarifontvar \numnoemph\vc\textbf{॰भिजित् श्रे॰}\lem \Ed, ॰भिजिः श्रे॰ \msCa\msCb\msNa\msNbpcorr\msNc, ॰भिजि \msNbacorr}}% 
    \var{{\devanagarifontvar \numnoemph\vd\textbf{कालः}\lem \msCa\msCb\msNa\msNb\msNc, कलिः \Ed}}% 

{\devanagarifont वेदेषु च वरं साम स्थावरेषु हिमालयः \thinspace{\dandab} \dontdisplaylinenum }%
 
%Verse 15:26

{\devanagarifont अश्वत्थो वट वृक्षेषु भूतेषु वर चेतनः {॥ १५:२६॥} \veg\dontdisplaylinenum }%
     \var{{\devanagarifontvar \numemph\vc\textbf{वट}\lem \msCa\msCb\msNa\msNb, वर \msNc\Ed}}% 
    \var{{\devanagarifontvar \numnoemph\vd\textbf{वर चेतनः}\lem \msCb\Ed, वरश्चेतनः \msCa\msNa\msNc\ \unmetr, वश्चेतनः \msNb}}% 

{\devanagarifont अध्यात्म सर्वविद्यासु वाक्य सत्य वर स्मृतः \thinspace{\dandab} \dontdisplaylinenum }%
     \var{{\devanagarifontvar \numemph\va\textbf{अध्यात्म}\lem \msCb\msNb\Ed, अध्यात्मा \msCa\msNc, आध्यात्मं \msNa\oo 
\textbf{सर्वविद्यासु}\lem \msCa\msNa\msNb\msNc, सर्वविद्यानां \msCb, वरविद्यासु \Ed}}% 
    \var{{\devanagarifontvar \numnoemph\vb\textbf{वाक्य}\lem \msCb, वाहु \msCa\msNa\msNb\msNc, वाचः \Ed\oo 
\textbf{वर}\lem \msCa\msCb\Ed, व\uncl{र}ः \msNa, वरः \msNb\msNc}}% 

%Verse 15:27

{\devanagarifont प्रह्लादो वर दैत्येषु यक्षरक्षो धनेश्वरः {॥ १५:२७॥} \veg\dontdisplaylinenum }%
     \var{{\devanagarifontvar \numnoemph\vc\textbf{प्रह्लादो}\lem \msCa\msCb\msNa\Ed, प्रह्रादो \msNb\msNc}}% 
    \var{{\devanagarifontvar \numnoemph\vd\textbf{॰श्वरः}\lem \msCa\msCb\msNa\msNc\Ed, ॰श्वर \msNb}}% 

{\devanagarifont मरीचिर्वर वातेषु हरिः श्रेष्ठो मृगेषु च \thinspace{\dandab} \dontdisplaylinenum }%
     \var{{\devanagarifontvar \numemph\va\textbf{मरीचिर्वर}\lem \msNc, मरीचि वर \msCb\msNa\msNb\Ed, म\lk\lk \lk\lk\ \msCa}}% 
    \var{{\devanagarifontvar \numnoemph\vb\textbf{हरिः}\lem \msCa\msCb\msNb\msNc\Ed, हरि \msNa}}% 

%Verse 15:28

{\devanagarifont साध्य नारायणः श्रेष्ठः पितॄणां च पितामहः {॥ १५:२८॥} \veg\dontdisplaylinenum }%
 
{\devanagarifont एतत्समासतो देवि कथितो ऽसि वरानने \thinspace{\dandab} \dontdisplaylinenum }%
     \var{{\devanagarifontvar \numemph\vb\textbf{ऽसि}\lem \msCa\msCb\msNa\msNb, स्मि \msNc\Ed}}% 

%Verse 15:29

{\devanagarifont सर्वसारं समुद्धृत्य किं भूयः कथयाम्यहम् {॥ १५:२९॥} \veg\dontdisplaylinenum }%
     \var{{\devanagarifontvar \numnoemph\vd\textbf{किं}\lem \msCb\msNa\msNb\msNc\Ed, कि \msCa}}% 

{\devanagarifont 
\jump
\begin{center}
\ketdanda~इति वृषसारसंग्रहे जीवनिर्णयो नामाध्यायः पञ्चदशमः~\ketdanda
\end{center}
\dontdisplaylinenum\vers  }%
     \var{{\devanagarifontvar \numnoemph{\englishfont \Colo:}\textbf{नामाध्यायः पञ्चदशमः}\lem \msCa\msCb\msNa, नामाध्यायः पञ्चमः \msNb, 
नामाध्यायः पञ्चदशम \msNc, नाम पञ्चदशो ऽध्यायः \Ed}}% 
\bekveg\szamveg
\vfill
\phpspagebreak

\versno=0\fejno=16
\thispagestyle{empty}

\centerline{\Large\devanagarifontbold [   षोडशमो ऽध्यायः  ]}{\vrule depth10pt width0pt} \fancyhead[CE]{{\footnotesize\devanagarifont वृषसारसंग्रहे  }}
\fancyhead[CO]{{\footnotesize\devanagarifont षोडशमो ऽध्यायः  }}
\fancyhead[LE]{}
\fancyhead[RE]{}
\fancyhead[LO]{}
\fancyhead[RO]{}
\szam\bek



\alalfejezet{योगसद्भावनिर्णयः}
\vers


{\devanagarifont देव्युवाच {\dandab}\dontdisplaylinenum  }%
 
{\devanagarifont अधुना श्रोतुमिच्छामि योगसद्भावनिर्णयम् \thinspace{\danda} \dontdisplaylinenum }%
     \var{{\devanagarifontvar \numemph\vb\textbf{॰सद्भाव॰}\lem \msCa\msCb\msNa\msNb\Ed, ॰संद्भाव॰ \msNc\oo 
\textbf{॰निर्णयम्}\lem \msCa\msCb\msNa\msNb\msNc, ॰निर्णयः \Ed}}% 
    \paral{{\devanagarifontsmall \vo {\englishfont   \msCa\ 435.jpg line 2; 
                       \msCb\ 448.jpg line 2;
                       This chapter is missing in \msCc.
                       \msNa\ 220.jpg lower image line 5; 
                       \msNb\ 65.jpg upper image line 6;
                       \msNc\ f. 235r line 3} }}

%Verse 16:1

{\devanagarifont करणं च यथान्यायं कथयस्व सुरेश्वर {॥ १६:१॥} \veg\dontdisplaylinenum }%
     \var{{\devanagarifontvar \numnoemph\vc\textbf{करणं च}\lem \msCa\msCb\msNa\msNb\msNc, करणश्च \Ed}}% 
    \var{{\devanagarifontvar \numnoemph\vd\textbf{सुरेश्वर}\lem \msCa\msCb\msNa\msNb, सुरेश्वरे \msNc, सुरेश्वरः \Ed}}% 

{\devanagarifont ईश्वर उवाच {\dandab}\dontdisplaylinenum  }%
     \var{{\devanagarifontvar \numemph\vo\textbf{ईश्वर}\lem \msCa\msCb\msNb\msNc, सुरेश \msNa, भगवान् \Ed}}% 

{\devanagarifont शृणु देवि प्रवक्ष्यामि योगसद्भावमुत्तमम् \thinspace{\danda} \dontdisplaylinenum }%
     \var{{\devanagarifontvar \numnoemph\vb\textbf{॰मुत्तमम्}\lem \msCa\msCb\msNa\msNc\Ed, ॰निर्णयम् \msNb}}% 
    \paral{{\devanagarifontsmall \vab {\englishfont \similar\ \NISVK\ 33.6ab:}
                 शृणु देवि परम् गूह्यं योगसद्भावमुत्तमम् }}

%Verse 16:2

{\devanagarifont यं विदित्वा न पश्यन्ति जनाः संसारबन्धनम् {॥ १६:२॥} \veg\dontdisplaylinenum }%
 
{\devanagarifont ब्रह्महा गुरुतल्पी वा सुरापस्तेय एव वा \thinspace{\dandab} \dontdisplaylinenum }%
     \var{{\devanagarifontvar \numemph\vb\textbf{वा}\lem \msCb\msNa\msNb\msNc, \uncl{वा} \msCa, च \Ed}}% 

%Verse 16:3

{\devanagarifont अथवा संकरे जातस्तत्सर्वमपनोदति {॥ १६:३॥} \veg\dontdisplaylinenum }%
     \var{{\devanagarifontvar \numnoemph\vc\textbf{संकरे}\lem \msNa\msNc, शृङ्करे \msCa, शङ्करे \msCb\Ed, \uncl{शं}करे \msNb}}% 
    \var{{\devanagarifontvar \numnoemph\vd\textbf{तत्सर्व॰}\lem \msCa\msCb\msNa\msNb\msNc, तसर्व॰ \Ed}}% 

{\devanagarifont मुहूर्तार्धे मुहूर्ते वा प्राणायामपरायणः \thinspace{\dandab} \dontdisplaylinenum }%
     \var{{\devanagarifontvar \numemph\va\textbf{मुहूर्तार्धे मुहूर्ते वा}\lem \msCa\msNb\msNc, मुहूर्तार्धे वा \msCb, 
मुहूर्त्तार्द्ध मुहूर्ते वा \msNa, मुहूर्तार्धमुहूर्तं च \Ed}}% 
    \paral{{\devanagarifontsmall \vo {\englishfont  cf.\ 16.10. } }}

%Verse 16:4

{\devanagarifont ध्येयं चिन्तयमानस्य तत्पापं क्षीयते नरात् {॥ १६:४॥} \veg\dontdisplaylinenum }%
     \var{{\devanagarifontvar \numnoemph\vc\textbf{ध्येयं चि॰}\lem \msCa\msNb\msNc, धेयञ्चि॰ \msCb, ध्येय चि॰ \msNa\Ed}}% 
    \var{{\devanagarifontvar \numnoemph\vd\textbf{नरात्}\lem \msCb\msNa\msNc, नरान् \msCa\msNb\Ed}}% 
    \paral{{\devanagarifontsmall \vo {\englishfont  \similar\ a citation in Kauṇḍinya's commentary ad \PS\ 5.24:}
                 मुहूर्तार्धं मुहूर्तं वा प्राणायामान्तरे ऽपि वा\thinspace{\devanagarifontsmall ।} 
                 ध्येयं चिन्तयमानस्तु पापं क्षपयते नरः\thinspace{\devanagarifontsmall ॥}  }}

{\devanagarifont न यमो नान्तकः क्रुद्धो न मृत्युर्भीमविग्रहः \thinspace{\dandab} \dontdisplaylinenum }%
     \var{{\devanagarifontvar \numemph\vb\textbf{मृत्युर्भी॰}\lem \msCa\msCb\Ed, मृत्यु भी॰ \msNa\msNb\msNc\oo 
\textbf{भीमविग्रहः}\lem \msCa\msCb\msNa\msNb\msNc, नापविग्रहः \Ed}}% 
    \paral{{\devanagarifontsmall \vab \similar\ {\englishfont  MBh 12.289.25ab:} न यमो नान्तकः क्रुद्धो न मृत्युर्भीमविक्रमः }}

%Verse 16:5

{\devanagarifont नाविशन्ति महात्मानो योगिनो बलवत्तराः {॥ १६:५॥} \veg\dontdisplaylinenum }%
     \var{{\devanagarifontvar \numnoemph\vc\textbf{नाविशन्ति}\lem \msCa\msCb\msNa\msNb\msNc, विशन्ति स्म \Ed}}% 
    \var{{\devanagarifontvar \numnoemph\vd\textbf{बलवत्तराः}\lem \msCa\msCb\msNa\msNb\msNc, वरवत्तरा \Ed}}% 

{\devanagarifont यथा वै सर्वधातूनां दोषा दह्यन्ति धाम्यताम् \thinspace{\dandab} \dontdisplaylinenum }%
     \var{{\devanagarifontvar \numemph\va\textbf{॰धातूनां}\lem \msCa\msCb\msNa\msNb\Ed, ॰धातीनां \msNcacorr, ॰धातृनां \msNcpcorr}}% 
    \var{{\devanagarifontvar \numnoemph\vb\textbf{दोषा दह्यन्ति}\lem \msNb\msNc, \uncl{दोषां दह्य}न्ति \msCa, दोषां दह्यन्ति \msCb\msNa\Ed}}% 

%Verse 16:6

{\devanagarifont तथा पापाः प्रदह्यन्ते ध्रुवं प्राणस्य निग्रहात् {॥ १६:६॥} \veg\dontdisplaylinenum }%
     \var{{\devanagarifontvar \numnoemph\vc\textbf{पापाः}\lem \msCa\msCb\msNa\msNb\msNc, पापः \Ed}}% 
    \var{{\devanagarifontvar \numnoemph\vd\textbf{निग्रहात्}\lem \msCa\msCb\msNa\msNc\Ed, निग्रहान् \msNb}}% 
    \paral{{\devanagarifontsmall \vo  {\englishfont  \similar\ Bhaviṣyapurāṇa 1.145.9:}
                                 ध्यायमानस्य दह्यन्ते चान्ते दोषा यथाग्निना\thinspace{\devanagarifontsmall ।}
                                 तथेन्द्रियकृता दोषा दह्यन्ते प्राणनिग्रहात्\thinspace{\devanagarifontsmall ॥}
                     {\englishfont  \similar\ Gheraṇḍasaṃhitā (ed.\ Thomi) 4.11:}
                                 यथा पर्वतधातूनां दोषा दह्यन्ति धाम्यताम्\thinspace{\devanagarifontsmall ।}
                                 तथेन्द्रियकृता दोषा दह्यन्ते प्राणनिग्रहात्\thinspace{\devanagarifontsmall ॥} }}

{\devanagarifont अश्वमेधसहस्रं च राजसूयशतं तथा \thinspace{\dandab} \dontdisplaylinenum }%
 
%Verse 16:7

{\devanagarifont प्राणायामशतं चैव न तत्तुल्यं कदाचन {॥ १६:७॥} \veg\dontdisplaylinenum }%
     \var{{\devanagarifontvar \numemph\vd\textbf{कदाचन}\lem \msCa\msCb\msNa\msNbpcorr\msNc\Ed, कदाच \msNbacorr}}% 

{\devanagarifont यज्ञेन देवानाप्नोति राज्यं वै तपसः फलम् \thinspace{\dandab} \dontdisplaylinenum }%
     \var{{\devanagarifontvar \numemph\va\textbf{देवानाप्नोति}\lem \msCa\msCb\msNa\msNbpcorr\msNc\Ed, देवाप्नोति \msNbacorr}}% 
    \paral{{\devanagarifontsmall \vo \similar\ {\englishfont \AGNIP\ 378.1:} 
                 यज्ञैश्च देवानाप्नोति वैराजं(?)\ तपसा पदम्\thinspace{\devanagarifontsmall ।}
                 ब्रह्मणः कर्मसन्न्यासाद्वैराग्यात्प्रकृतौ लयम्\thinspace{\devanagarifontsmall ॥}
                 \similar\ {\englishfont  Maskarin's commentary CHECK ad \GautDhS\ 3.1:}
                         यज्ञेन देवानाप्नोति वैराजं(!)\ तपसा पुनः\thinspace{\devanagarifontsmall ।}
                         संन्यासाद्ब्रह्मणः स्थानं वैराग्यात्प्रकृतौ लयम्\thinspace{\devanagarifontsmall ॥}
                 {\englishfont cf.\ ŚDhU 3.40:}
                 यज्ञेन देवानाप्नोति तपोर्भिर्ब्रह्मणः पदम्\thinspace{\devanagarifontsmall ।}
                 दानेन विविधान्भोगान् ज्ञानान्मोक्षमवाप्नुयात्\thinspace{\devanagarifontsmall ॥} }}

%Verse 16:8

{\devanagarifont संन्यासाद्ब्रह्मणः स्थानं वैराग्यात्प्रकृतालयम् {॥ १६:८॥} \veg\dontdisplaylinenum }%
     \var{{\devanagarifontvar \numnoemph\vc\textbf{॰द्ब्रह्मणः}\lem \msCa\msCb\msNa\msNc\Ed, ॰द्ब्राह्मणः \msNb}}% 
    \var{{\devanagarifontvar \numnoemph\vd\textbf{वैराग्यात्प्रकृतालयम्}\lem \msCa\msNa\msNb\msNc\Ed, 
महात्मानो प्रकृतालयम् \msCb\ ({\englishfont eyeskip to 16.5c?})}}% 
    \paral{{\devanagarifontsmall \vcd {\englishfont cf.\ 11.27ab above} }}

{\devanagarifont ज्ञानात्प्राप्नोति कैवल्यं परं ब्रह्म सनातनम् \thinspace{\dandab} \dontdisplaylinenum }%
     \var{{\devanagarifontvar \numemph\vb\textbf{परं}\lem \msCa\msCb\msNa\msNc\Ed, पर॰ \msNb}}% 

%Verse 16:9

{\devanagarifont इत्येता गतयः पञ्च विधिवत्परिकीर्तिताः {॥ १६:९॥} \veg\dontdisplaylinenum }%
 
{\devanagarifont मुहूर्तार्धं मुहूर्तं वा योगं युञ्जीत योगवित् \thinspace{\dandab} \dontdisplaylinenum }%
     \var{{\devanagarifontvar \numemph\va\textbf{मुहूर्तार्धं मुहूर्तं वा}\lem \msCa\msCb\msNb, 
मुहूर्तार्द्ध मुहू\uncl{र्त्तं} वा \msNa, 
मुहूतार्थं मुहूर्त्तम् वा \msNc, 
मुहूर्तार्ध मुहूर्तं वा \Ed}}% 
    \var{{\devanagarifontvar \numnoemph\vb\textbf{योगं}\lem \msCa\msCb\msNa\msNc\Ed, योग \msNb\oo 
\textbf{योगवित्}\lem \msCa\msCb\msNa\msNb\Ed, योवित् \msNc}}% 
    \paral{{\devanagarifontsmall \vo {\englishfont  cf.\ 16.4. } }}

%Verse 16:10

{\devanagarifont निस्तरेत्सर्वपापानि अमृतत्वं च गच्छति {॥ १६:१०॥} \veg\dontdisplaylinenum }%
     \var{{\devanagarifontvar \numnoemph\vc\textbf{निस्तरेत्स॰}\lem \msCb\msNa\msNb\Ed, विस्तरेत्स॰ \msCa, \uncl{नि}स्तरेणत्स॰ \msNc}}% 
    \var{{\devanagarifontvar \numnoemph\vd\textbf{अमृतत्वं}\lem \msCa\msCb\msNa\msNc\Ed, अमृतत्व \msNb}}% 

{\devanagarifont युञ्जानो ऽपि प्रयत्नेन यावत्तत्त्वं न विन्दति \thinspace{\dandab} \dontdisplaylinenum }%
     \var{{\devanagarifontvar \numemph\vb\textbf{यावत्तत्त्वं न विन्दति}\lem \msNa\msNc\Ed, 
यावन्तन्न विन्दति \unmetr\ \msCa, याव तत्वं न विन्दति \msCb, यावत्तंन्न विन्दति \msNb}}% 

%Verse 16:11

{\devanagarifont ब्रह्मलोके ध्रुवं वासो विष्णुलोके च सुन्दरि {॥ १६:११॥} \veg\dontdisplaylinenum }%
     \var{{\devanagarifontvar \numnoemph\vc\textbf{ब्रह्मलोके}\lem \msCa\msCb\msNa\msNc\Ed, ब्रह्मलोको \msNb\oo 
\textbf{वासो}\lem \msCa\msNa\msNb\msNc\Ed, वास्वा \msCb}}% 

{\devanagarifont भुक्त्वा कर्मसहस्राणि सर्वकामसमन्वितः \thinspace{\dandab} \dontdisplaylinenum }%
 
%Verse 16:12

{\devanagarifont क्षीणपुण्यस्ततो मर्त्ये जायते विपुले कुले {॥ १६:१२॥} \veg\dontdisplaylinenum }%
     \var{{\devanagarifontvar \numemph\vc\textbf{॰पुण्यस्त॰}\lem \msNa\msNb\msNc\Ed, ॰पुण्ये त॰ \msCa\msCb\oo 
\textbf{मर्त्ये}\lem \msCa\msCb\msNa\msNb\msNc, मर्त्यां \Ed}}% 

{\devanagarifont योगमेवाभिसेवेत पूर्वजातिस्मरो नरः \thinspace{\dandab} \dontdisplaylinenum }%
 
%Verse 16:13

{\devanagarifont संसारार्णवमुत्तीर्य स शिवत्वमवाप्नुयात् {॥ १६:१३॥} \veg\dontdisplaylinenum }%
 

\alalfejezet{योगविधिः}
{\devanagarifont देव्युवाच {\dandab}\dontdisplaylinenum  }%
 
{\devanagarifont योगस्य विधिमिच्छामि श्रोतुं मे पुरुषोत्तम \thinspace{\danda} \dontdisplaylinenum }%
     \var{{\devanagarifontvar \numemph\vb\textbf{श्रोतुं मे}\lem \msCa\msCb\msNc\Ed, श्रोतुं वै \msNa, श्रोतु मे \msNb\oo 
\textbf{॰त्तम}\lem \msCa\msCb\msNa\msNb\msNc, ॰त्तमः \Ed}}% 

%Verse 16:14

{\devanagarifont ध्यानधारणसिद्धीनां कथयस्व सुरेश्वर {॥ १६:१४॥} \veg\dontdisplaylinenum }%
     \var{{\devanagarifontvar \numnoemph\vc\textbf{॰सिद्धीनां}\lem \msCa\msCb\msNa\msNb\msNc, ॰सिद्धानां \Ed}}% 
    \var{{\devanagarifontvar \numnoemph\vd\textbf{सुरेश्वर}\lem \msCa\msCb\msNa\msNb\msNc, सुरेश्वरः \Ed}}% 

{\devanagarifont महेश्वर उवाच {\dandab}\dontdisplaylinenum  }%
     \var{{\devanagarifontvar \numemph\vo\textbf{महेश्वर}\lem \msCa\msCb\msNa\msNb\msNc, भगवान् \Ed}}% 

{\devanagarifont शृणु योगविधिं वक्ष्ये भवपाशनिकृन्तनम् \thinspace{\danda} \dontdisplaylinenum }%
 
{\devanagarifont शुचिरेकाग्रचित्तस्तु जनशब्दविवर्जिते  \danda\dontdisplaylinenum }%
     \var{{\devanagarifontvar \numnoemph\vd\textbf{॰चित्तस्तु}\lem \msCa\msNa\msNb\msNc\Ed, ॰चित्तस्यस्तु \msCb\oo 
\textbf{जन॰}\lem \msCa\msCb\msNa\msNb\msNc, ध्यान॰ \Ed\oo 
\textbf{॰विवर्जिते}\lem \msNa, ॰विवर्जितः \msCa\msCb\msNb\msNc, ॰विवर्जितम् \Ed}}% 

%Verse 16:15

{\devanagarifont तत्रासीनासने योगी परमात्मान चिन्तयेत् {॥ १६:१५॥} \veg\dontdisplaylinenum }%
     \var{{\devanagarifontvar \numnoemph\vf\textbf{॰त्मान चिन्तयेत्}\lem \msCb\msNa\msNcpcorr, ॰त्मानं चिन्तयेत् \msCa\Ed\ \unmetr, ॰त्माना विचिन्तयेत् \msNb, 
॰त्मान चिन्तया \msNcacorr}}% 

{\devanagarifont पद्मकं स्वस्तिकं चैव निष्कलमञ्जलिस्तथा \thinspace{\dandab} \dontdisplaylinenum }%
     \var{{\devanagarifontvar \numemph\va\textbf{पद्मकं}\lem \msCb\msNa\msNb\msNc\Ed, पद्मक \msCa}}% 
    \var{{\devanagarifontvar \numnoemph\vb\textbf{निष्कलमञ्जलिस्तथा}\lem \msCa\msCb\msNb\msNc, निष्कलंमञ्जलिस्तथा \msNa, 
निष्कलमकञ्जलिन्तथा \Ed}}% 

%Verse 16:16

{\devanagarifont अर्धचन्द्रं च दण्डं च पर्यङ्कं भद्रमेव च {॥ १६:१६॥} \veg\dontdisplaylinenum }%
     \var{{\devanagarifontvar \numnoemph\vd\textbf{पर्यङ्कं}\lem \msCb\msNa\msNb\msNc\Ed, प\lk ङ्कं \msCa}}% 
    \paral{{\devanagarifontsmall \vo {\englishfont  cf.\ Sarvajñānottara 27:9cd--10ab:} 
                 पद्मकं स्वस्तिकं वापि उपस्थाञ्जलिकं तथा\thinspace{\devanagarifontsmall ॥} 
                 पीठार्धमर्धचन्द्रं वा सर्वतोभद्रमेव वा\thinspace{\devanagarifontsmall ।} }}

{\devanagarifont एतदासनबन्धेन बद्ध्वा योगं समभ्यसेत् \thinspace{\dandab} \dontdisplaylinenum }%
     \var{{\devanagarifontvar \numemph\vb\textbf{बद्ध्वा योगं}\lem \msCa\msCb\msNa\msNb\msNc, बद्धा योग \Ed}}% 

%Verse 16:17

{\devanagarifont समं कायशिरोग्रीवं धारयन्नचलस्थितः {॥ १६:१७॥} \veg\dontdisplaylinenum }%
     \var{{\devanagarifontvar \numnoemph\vc\textbf{समं}\lem \msCa\msCb\msNa\Ed, सम॰ \msNb\msNc}}% 
    \paral{{\devanagarifontsmall \vcd \similar\ {\englishfont  MBh 6.28.13ab (BhG 6.13ab):} समं कायशिरोग्रीवं धारयन्न्  अचलं स्थिरः }}

{\devanagarifont प्रत्याहारस्तथा ध्यानं प्राणायामश्च धारणा \thinspace{\dandab} \dontdisplaylinenum }%
     \var{{\devanagarifontvar \numemph\va\textbf{प्रत्याहारस्त॰}\lem \msCa\msCb\msNa\msNb\msNc, प्रत्यहारस्त॰ \Ed\oo 
\textbf{ध्यानं}\lem \msCa\msCb\msNb\msNc\Ed, ध्यान \msNa}}% 
    \var{{\devanagarifontvar \numnoemph\vb\textbf{प्राणायामश्च}\lem \msCa\msCb\msNa\msNb\msNc, प्राणायामञ्च \Ed}}% 

%Verse 16:18

{\devanagarifont तर्कश्चैव समाधिश्च षडङ्गो योग उच्यते {॥ १६:१८॥} \veg\dontdisplaylinenum }%
     \paral{{\devanagarifontsmall \vo {\englishfont  = Dharmaputrikā 1.13 (with } प्राणायामो ऽथ {\englishfont  ) \similar\ \NISVK\ 33.7}
              {\englishfont  cf.\ Sarvajñānottaravṛtti ad Yogapāda 27(?).1:} 
              यदुक्तं श्रीमन्मतङ्गे\thinspace{\devanagarifontsmall ।}
                 प्राणायामस्तथा ध्यानं प्रत्याहारो ऽथ धारणम्\thinspace{\devanagarifontsmall ।}
                 तर्कश्चैव समाधिश्च षडङ्गो योग उच्यते इति\thinspace{\devanagarifontsmall ॥} }}

{\devanagarifont विषयासक्तचित्तानामिन्द्रियाणां प्रति प्रति \thinspace{\dandab} \dontdisplaylinenum }%
     \var{{\devanagarifontvar \numemph\vb\textbf{प्रति प्रति}\lem \msCb\msNa\msNb, प्रतिस्रति \msCa, प्रति पति \msNc, प्रतिष्ठति \Ed}}% 

%Verse 16:19

{\devanagarifont मनसाकर्षयेद्यस्तु प्रत्याहारः स उच्यते {॥ १६:१९॥} \veg\dontdisplaylinenum }%
     \var{{\devanagarifontvar \numnoemph\vc\textbf{मनसा॰}\lem \msCa\msCb\msNa\msNb\msNc, मनमा॰ \Ed}}% 
    \var{{\devanagarifontvar \numnoemph\vd\textbf{प्रत्याहारः स}\lem \msCa\msCb\msNa\msNb\msNc, प्रत्यहारस्तद् \Ed}}% 
    \paral{{\devanagarifontsmall \vo \similar\ {\englishfont  Dharmaputrikā 1.14:} 
                 विषयेष्वतिसक्तानि इन्द्रियाणि प्रति प्रति\thinspace{\devanagarifontsmall ।}
                 चित्तेनाकर्षणं यत्र प्रत्याहारः स उच्यते\thinspace{\devanagarifontsmall ॥} }}

{\devanagarifont शब्दादिविषयान्देवि वर्तुलीकृत्य धारयेत् \thinspace{\dandab} \dontdisplaylinenum }%
     \var{{\devanagarifontvar \numemph\va\textbf{॰विषयान्दे॰}\lem \msCa\msNa\msNb\msNc\Ed, विषया दे॰ \msCb}}% 
    \paral{{\devanagarifontsmall \vo {\englishfont  cf.\ Dharmaputrikā 1.16cd: 
                 } एकत्र वर्तुलीकृत्य ध्येये वस्तुनि योजयेत्  }}

%Verse 16:20

{\devanagarifont वीतरागः समाधिस्थो ध्येये वस्तुनि योजयेत् {॥ १६:२०॥} \veg\dontdisplaylinenum }%
     \var{{\devanagarifontvar \numnoemph\vc\textbf{वीतरागः}\lem \msCa\msNb\msNc, वीतराग॰ \msCb\msNa\Ed}}% 
    \var{{\devanagarifontvar \numnoemph\vd\textbf{ध्येये वस्तुनि}\lem \msCb\msNa, ध्येयस्तुनि \msCa, ध्येयवस्तुनि \msNb\msNc\Ed}}% 

{\devanagarifont आत्मा ध्याता मनो ध्यानं ध्येयः शुद्धः परः शिवः \thinspace{\dandab} \dontdisplaylinenum }%
     \var{{\devanagarifontvar \numemph\va\textbf{आत्मा}\lem \msCa\msCb\msNa\msNb\msNc, आत्म \Ed\oo 
\textbf{ध्याता}\lem \msCa\msCb\msNa\msNc\Ed, ध्यातं \msNb}}% 
    \var{{\devanagarifontvar \numnoemph\vb\textbf{परः शिवः}\lem \msCb\msNa\msNb\msNc, परशिवः \msCa\Ed\ \unmetr}}% 
    \paral{{\devanagarifontsmall \vo {\englishfont  \similar\ Śivadharmottara 10.49:}
                 ध्याता ध्यानं तथा ध्येयं यच्च ध्याने प्रयोजनम्\thinspace{\devanagarifontsmall ।}
                 एतच्चतुष्टयं ज्ञात्वा योगं युञ्जीत योगवित्\thinspace{\devanagarifontsmall ॥}
                 {\englishfont  \similar\ Dharmaputrikā 1.18:}
                 ध्येयः शिवो ध्यातृ मनो ध्यानमेकाग्रचित्तता\thinspace{\devanagarifontsmall ।}
                 दुःखहानिर्गुणैश्वर्यं स्वातन्त्र्यञ्च प्रयोजनम्\thinspace{\devanagarifontsmall ॥}
                 {\englishfont  \similar\ Sarvajñānottara Yogapāda 27(?):4:}
                 आत्मा ध्याता मनो ध्यानं ध्येयः सूक्ष्मो महेश्वरः\thinspace{\devanagarifontsmall ।}
                 यत्परं परमैश्वर्यम् एतद्ध्यानप्रयोजनम्\thinspace{\devanagarifontsmall ॥}
                 \similar\ {\englishfont \NISVK\ 12.39cd:}
                 आत्मा ध्याता मनो ध्यानं ध्येयश्शुद्धो महेश्वरः
                 \similar\ {\englishfont  Agnipurāṇa 165.22cd:}
                 आत्मा ध्याता मनो ध्यानं ध्येयो विष्णुः फलं हरिः }}

%Verse 16:21

{\devanagarifont यत्परं परमैश्वर्यमेकं तत्र प्रयोजनम् {॥ १६:२१॥} \veg\dontdisplaylinenum }%
     \var{{\devanagarifontvar \numnoemph\vc\textbf{परमै॰}\lem \msCa\msCb\msNa\msNb\msNcpcorr\Ed, परमे॰ \msNcacorr}}% 
    \var{{\devanagarifontvar \numnoemph\vcd\textbf{॰मेकं तत्र}\lem \msCa\msCb\msNa\msNc\Ed, ॰मेतत्तत्र \msNb}}% 

{\devanagarifont पूरकः कुम्भकश्चैव रेचकस्तदनन्तरम् \thinspace{\dandab} \dontdisplaylinenum }%
     \paral{{\devanagarifontsmall \vo = {\englishfont  Dharmaputrikā 1.19ab (with } चैव {\englishfont for } चेति{\englishfont  )}  }}

%Verse 16:22

{\devanagarifont प्रशान्तश्चेति विख्यातः प्राणायामश्चतुर्विधः {॥ १६:२२॥} \veg\dontdisplaylinenum }%
     \var{{\devanagarifontvar \numemph\vc\textbf{प्रशान्त॰}\lem \msCb\msNa\msNb\msNc\Ed, \uncl{प्र}शान्त॰ \msCa\oo 
\textbf{विख्यातः}\lem \msCa\msCb\msNb\msNc\Ed, विख्याताः \msNa}}% 
    \var{{\devanagarifontvar \numnoemph\vd\textbf{॰विधः}\lem \msCa\msCb\msNa\msNb\msNc, ॰विधाः \Ed}}% 
    \paral{{\devanagarifontsmall \vcd {\englishfont  See NiśvāsaNaya 4:113:}
                 नाभ्यां हृदयसंचारान्मनश्चेन्द्रियगोचरात्\thinspace{\devanagarifontsmall ।}
                 प्राणायामश्चतुर्थस्तु सुप्रशान्तस्तु विश्रुतः\thinspace{\devanagarifontsmall ॥} 
                 {\englishfont  See also Svaccandatantra 7.298ab:}
                 प्राणायामश्चतुर्थस्तु सुप्रशान्त इति श्रुतः }}

{\devanagarifont पूरके स्थापयेद्वह्निं पादाङ्गुष्ठेन बुद्धिमान् \thinspace{\dandab} \dontdisplaylinenum }%
     \var{{\devanagarifontvar \numemph\va\textbf{पूरके}\lem \eme, पूरकः \msCa\msCb\msNa\msNb\msNc\Ed\oo 
\textbf{वह्निं}\lem \msCa\msCb\msNa\msNc,  वह्नि \msNb\Ed}}% 
    \var{{\devanagarifontvar \numnoemph\vb\textbf{॰ष्ठेन}\lem \msCa\msCb\msNb\msNc\Ed, ॰ष्ठेषु \msNa}}% 

%Verse 16:23

{\devanagarifont कुम्भकेन विरुध्येत दह्यमानं विचिन्तयेत् {॥ १६:२३॥} \veg\dontdisplaylinenum }%
     \var{{\devanagarifontvar \numnoemph\vcd\textbf{विरुध्येत दह्यमानं}\lem \msCa\msCb\msNa\msNc, निरुध्येत दह्यमानम् \msNb, 
निरुध्येत दैह्यमान \Ed}}% 

{\devanagarifont भस्मीभूतं तथात्मानं रेचकेन विचिन्तयेत् \thinspace{\dandab} \dontdisplaylinenum }%
     \var{{\devanagarifontvar \numemph\va\textbf{॰भूतं}\lem \msCa\msCb\msNa\msNb\Ed, ॰भूत \msNc}}% 

%Verse 16:24

{\devanagarifont शुद्धदेहस्ततश्चात्मा शुद्धस्फटिकनिर्मलः {॥ १६:२४॥} \veg\dontdisplaylinenum }%
     \var{{\devanagarifontvar \numnoemph\vc\textbf{॰देहस्ततश्चात्मा}\lem \msCa\msCb\msNa\msNb\Ed, ॰देहश्चतश्चात्मा \msNc}}% 

{\devanagarifont तालशब्दस्तु निर्वाणं दश द्वे च प्रकीर्तितः \thinspace{\dandab} \dontdisplaylinenum }%
     \var{{\devanagarifontvar \numemph\va\textbf{तालाशब्दस्तु}\lem \msCa\msCb\msNa\msNb\msNcacorr\Ed, 
तालशब्दस्तुस्तु \msNcpcorr\oo 
\textbf{निर्वाणं}\lem \msCa\msCb\msNc\Ed, निर्वाण \msNa, \uncl{निर्व्वा}णं \msNb}}% 

%Verse 16:25

{\devanagarifont प्राणायामान्न संदेहो द्विगुणा धारणा स्मृता {॥ १६:२५॥} \veg\dontdisplaylinenum }%
     \var{{\devanagarifontvar \numnoemph\vc\textbf{प्राणायामान्न}\lem \msCa\msNa\msNb\msNc\Ed, प्राणायान्न \msCb}}% 
    \var{{\devanagarifontvar \numnoemph\vd\textbf{स्मृता}\lem \msCa\msCb\msNa\msNb\msNc, स्मृताः \Ed}}% 

{\devanagarifont योगे तु त्रिगुणा प्रोक्ता संक्रमे च चतुर्गुणा \thinspace{\dandab} \dontdisplaylinenum }%
     \var{{\devanagarifontvar \numemph\va\textbf{॰गुणा}\lem \msCa\msCb\msNb\msNc\Ed, ॰गुणाः \msNa}}% 
    \var{{\devanagarifontvar \numnoemph\vab\textbf{प्रोक्ता संक्रमे च चतुर्गुणा}\lem \msCa\msCb, 
प्रोक्ताः संक्रमे च चतुर्गुणा \msNa, 
प्रोक्तां संक्रमे च चतुर्गुणा \msNb, 
प्रोक्ता सक्रमे च चर्तुगुणा \msNc, 
प्रोक्ताः संक्रमेण चतुर्गुणाः \Ed}}% 

%Verse 16:26

{\devanagarifont तथोत्क्रान्तौ पञ्चगुणा योगसिद्धिस्तु षड्गुणा {॥ १६:२६॥} \veg\dontdisplaylinenum }%
     \var{{\devanagarifontvar \numnoemph\vc\textbf{तथोत्क्रान्तौ}\lem \msCa\msCb\msNa\msNb\msNc, तथाक्रतौ \Ed}}% 
    \var{{\devanagarifontvar \numnoemph\vd\textbf{षड्गुणा}\lem \eme, षड्गुणाः \msCa\msCb\msNa\msNb\Ed, षडङ्गुणाः \msNc}}% 

{\devanagarifont षडङ्गेन समायुक्तो योगयुक्तस्तु नित्यशः \thinspace{\dandab} \dontdisplaylinenum }%
     \var{{\devanagarifontvar \numemph\va\textbf{षडङ्गेन}\lem \msCa\msCb\msNa\msNb\Ed, सदङ्गेन \msNc}}% 
    \var{{\devanagarifontvar \numnoemph\vb\textbf{योगयुक्तस्तु}\lem \msCa\msCb\msNa\msNb\msNc, योगमुक्तस्तु \Ed}}% 

%Verse 16:27

{\devanagarifont मानसो यौगपद्यश्च द्विरूपो योग उच्यते {॥ १६:२७॥} \veg\dontdisplaylinenum }%
     \var{{\devanagarifontvar \numnoemph\vcd\textbf{यौगपद्यश्च द्विरूपो}\lem \msNa\msNc, 
यौगपद्य\uncl{श्च} \lk ि\lk\lk\  \msCa, 
योगपद्यश्च द्विरूपो \msCb\msNb, 
योगपद्यञ्च द्विरूपो \Ed}}% 
    \paral{{\devanagarifontsmall \vcd = {\englishfont  Dharmaputrikā 1.54ab.}  }}

{\devanagarifont अकृत्वा प्राणसंरोधं मनसैकेन केवलम् \thinspace{\dandab} \dontdisplaylinenum }%
     \var{{\devanagarifontvar \numemph\va\textbf{॰संरोधं}\lem \msCa\msNb\msNc\Ed, ॰संरोध \msCb\msNa}}% 
    \var{{\devanagarifontvar \numnoemph\vb\textbf{मनसैकेन}\lem \msCb\msNb\msNc\Ed, मनसेकेन \msCa\msNa}}% 

%Verse 16:28

{\devanagarifont ध्यायेत परमं सूक्ष्मं स योगो मानसः स्मृतः {॥ १६:२८॥} \veg\dontdisplaylinenum }%
     \var{{\devanagarifontvar \numnoemph\vc\textbf{ध्यायेत प॰}\lem \msCa\msCb\msNa\msNc\Ed, ध्यायेतत्प \msNb\oo 
\textbf{परमं}\lem \msCa\msCb\msNa\msNb\Ed, परम \msNc}}% 
    \var{{\devanagarifontvar \numnoemph\vd\textbf{स योगो}\lem \msCa\msNa\msNb\msNc\Ed, संयोगो \msCb\oo 
\textbf{मानसः}\lem \msCapcorr\msCb\msNb\msNc\Ed, मानस \msCaacorr\msNa\oo 
\textbf{स्मृतः}\lem \msCa\msCb\msNa\msNb\msNc, स्मृतम् \Ed}}% 
    \paral{{\devanagarifontsmall \vo = {\englishfont  Dharmaputrikā 1.54cd--55ab.}  }}

{\devanagarifont संयम्य मनसा प्राणं प्राणायामान्मनस्तथा \thinspace{\dandab} \dontdisplaylinenum }%
     \var{{\devanagarifontvar \numemph\va\textbf{संयम्य}\lem \msCa\msCb\msNa\msNc\Ed, सयम्य \msNb\oo 
\textbf{प्राणं}\lem \msCa\msNa\msNb\msNc\Ed, \om\ \msCb}}% 
    \var{{\devanagarifontvar \numnoemph\vb\textbf{प्राणायामान्म॰}\lem \eme, प्राणायामाम्म॰ \msCa\msNb, प्राणायामा म॰ \msCb, 
प्राणायामं म॰ \msNa, प्राणायामां म॰ \msNc, प्राणायामात्म॰ \Ed}}% 

%Verse 16:29

{\devanagarifont एवं ध्यायेत्परं सूक्ष्मं यौगपद्यः स उच्यते {॥ १६:२९॥} \veg\dontdisplaylinenum }%
     \var{{\devanagarifontvar \numnoemph\vd\textbf{यौगपद्यः}\lem \msCa\msCb\msNc\Ed, योगपद्यः \msNa, योगपद्य \msNb}}% 
    \paral{{\devanagarifontsmall \vo \similar\ {\englishfont  Dharmaputrikā 1.55cd--56ab:}
                 संयम्य मनसा प्राणं प्राणायामैर्मनस्तथा\thinspace{\devanagarifontsmall ।}
                 एवं ध्यायेत्परं सूक्ष्मं यौगपद्यः स उच्यते\thinspace{\devanagarifontsmall ॥} }}


\alalfejezet{सिद्धिलक्षणम्}
{\devanagarifont सिद्धिलक्षण योगस्य शृणु वक्ष्यामि सुन्दरि \thinspace{\dandab} \dontdisplaylinenum }%
     \var{{\devanagarifontvar \numemph\va\textbf{सिद्धि॰}\lem \msCa\msCb\msNa\msNb\msNc, सिद्धिर् \Ed}}% 

{\devanagarifont शङ्खभेरीमृदङ्गं च वेणुदुन्दुभिमेव च  \danda\dontdisplaylinenum }%
     \var{{\devanagarifontvar \numnoemph\vc\textbf{शङ्खभेरीमृदङ्गं च}\lem \msNb, शङ्ख\lk\lk \lk\lk \lk श्च \msCa, शङ्खभेरीमृदङ्गश्च \msCb\msNa\msNc\Ed}}% 
    \var{{\devanagarifontvar \numnoemph\vd\textbf{॰दुन्दुभिमेव}\lem \msCa\msCb\msNa\msNb\msNc, ॰दुन्दुभिरेव \Ed}}% 

%Verse 16:30

{\devanagarifont ताडितं न च विन्देत यदा तन्मयतां गतः {॥ १६:३०॥} \veg\dontdisplaylinenum }%
     \paral{{\devanagarifontsmall \vo \similar\ {\englishfont  Kulasāra f.\ 38r:}
                  शंखभेरीमृदंगैश्च वीणावेणुशतैरपि\thinspace{\devanagarifontsmall ।} 
                  ताड्यमानैर्न विन्देत यदा तन्मयतां गतः\thinspace{\devanagarifontsmall ॥} }}
    \paral{{\devanagarifontsmall \vef {\englishfont  cf. NiśvāsaMuKa 4:65:}
                 ताडितञ्च न विन्देत चक्षुषा न च पश्यति\thinspace{\devanagarifontsmall ।}
                 दिव्यदृष्टिः प्रजायेत यदा तन्मयताङ्गतः\thinspace{\devanagarifontsmall ॥} }}

{\devanagarifont शीतोष्णं सुखदुःखं च तृष्णाभुक्षं तथैव च \thinspace{\dandab} \dontdisplaylinenum }%
     \var{{\devanagarifontvar \numemph\vb\textbf{तृष्णाभुक्षं}\lem \msCa\msCb\msNa\msNb\msNc, तृड्बुभुक्षां \Ed}}% 

%Verse 16:31

{\devanagarifont वेदनां नैव जानाति योगसिद्धस्तु सुन्दरि {॥ १६:३१॥} \veg\dontdisplaylinenum }%
     \var{{\devanagarifontvar \numnoemph\vc\textbf{वेदनां}\lem \msNa, वेदनान् \msCa\msCb, वेदना \msNb\Ed, वैदना \msNc}}% 
    \var{{\devanagarifontvar \numnoemph\vd\textbf{॰सिद्ध॰}\lem \msCa\msNa\msNc, ॰सिद्धि॰ \msCb\msNb, ॰युक्त॰ \Ed}}% 

{\devanagarifont एष योगविधिर्देवि तव पृष्टेन सुन्दरि \thinspace{\dandab} \dontdisplaylinenum }%
     \var{{\devanagarifontvar \numemph\va\textbf{॰विधिर्देवि}\lem \msCa\msCb\msNa\msNb\Ed, ॰विधिन्देवि \msNc}}% 

%Verse 16:32

{\devanagarifont कथितो ऽस्मि समासेन किमन्यत्कथयाम्यहम् {॥ १६:३२॥} \veg\dontdisplaylinenum }%
 
{\devanagarifont देव्युवाच {\dandab}\dontdisplaylinenum  }%
 
{\devanagarifont विना योगेन देवेश संसारतारणं मम \thinspace{\danda} \dontdisplaylinenum }%
     \var{{\devanagarifontvar \numemph\va\textbf{देवेश}\lem \msCb\msNa\msNb\msNc\Ed, वेश \msCa}}% 
    \var{{\devanagarifontvar \numnoemph\vb\textbf{संसारतारणं मम}\lem \msCa\msCb\msNc\Ed, संसारात्तारणं मम \msNa, संसारार्ण्णवतारण \msNb}}% 

%Verse 16:33

{\devanagarifont कथयस्व महादेव निर्विकल्पकरं मनः {॥ १६:३३॥} \veg\dontdisplaylinenum }%
     \var{{\devanagarifontvar \numnoemph\vc\textbf{महादेव}\lem \msCa\msCb\msNa\msNb\Ed, सुरेशान \msNc}}% 

{\devanagarifont महेश्वर उवाच {\dandab}\dontdisplaylinenum  }%
     \var{{\devanagarifontvar \numemph\vo\textbf{महेश्वर}\lem \msCa\msCb\msNb\msNc, देवेश \msNa, भगवान् \Ed}}% 

{\devanagarifont सदाशिवस्तु निश्वास ऊर्ध्वश्वासः परः शिवः \thinspace{\danda} \dontdisplaylinenum }%
     \var{{\devanagarifontvar \numnoemph\vb\textbf{ऊर्ध्वश्वासः}\lem \msCa\msCb\msNa\msNc, ऊर्ध्वश्वास \msNb, अर्द्धश्वासः \Ed}}% 

%Verse 16:34

{\devanagarifont तयोर्मध्ये तु विज्ञेयः परमात्मा शिवो ऽव्ययः {॥ १६:३४॥} \veg\dontdisplaylinenum }%
 
{\devanagarifont ध्यानयोगं न तस्यास्ति करणं च न विद्यते \thinspace{\dandab} \dontdisplaylinenum }%
 
%Verse 16:35

{\devanagarifont ज्ञातमात्रेण मुच्यन्ते किमन्यत्परिपृच्छसि {॥ १६:३५॥} \veg\dontdisplaylinenum }%
     \var{{\devanagarifontvar \numemph\vc\textbf{ज्ञात॰}\lem \msCa\msCb\msNa\msNb, ज्ञान॰ \msNc\Ed}}% 
    \var{{\devanagarifontvar \numnoemph\vcd\textbf{मुच्यन्ते किमन्यत्प॰}\lem \msCb\msNa\msNc\Ed, मुच्य\uncl{न्ते}\lk मन्यत्प॰ \msCa, 
\uncl{मुच्य}न्ते किमत्प॰ \msNb}}% 


\alalfejezet{पञ्च शास्त्राणि}
{\devanagarifont ज्ञानमन्यत्प्रवक्ष्यामि शृणु देवि निबोध मे \thinspace{\dandab} \dontdisplaylinenum }%
 
{\devanagarifont शास्त्रपञ्चसु यत्प्रोक्तं शृणु संक्षेप निर्णयम्  \danda\dontdisplaylinenum }%
     \var{{\devanagarifontvar \numemph\vd\textbf{संक्षेप}\lem \msCb\msNa\msNb\msNc\Ed, संक्षेपे \msCa\ \unmetr}}% 

%Verse 16:36

{\devanagarifont सांख्ये योगे पञ्चरात्रे शैवे वेदे च निर्मितम् {॥ १६:३६॥} \veg\dontdisplaylinenum }%
     \paral{{\devanagarifontsmall \vo {\englishfont \compare\ \MBH\ 12.337.1:}
                         सांख्यं योगं पञ्चरात्रं वेदारण्यकम् एव च\thinspace{\devanagarifontsmall ।} 
 		        ज्ञानान्य्  एतानि ब्रह्मर्षे लोकेषु प्रचरन्ति ह\thinspace{\devanagarifontsmall ॥} }}
    \var{{\devanagarifontvar \numnoemph\ve\textbf{सांख्ये}\lem \msCa\msCb\msNa\msNb\msNc, सांख्य॰ \Ed\oo 
\textbf{पञ्च॰}\lem \msCa\msCb\msNb\msNc\Ed, पच॰ \msNa}}% 
    \var{{\devanagarifontvar \numnoemph\vf\textbf{शैवे}\lem \msCa\msCb\msNa\msNb\msNc, शैव॰ \Ed}}% 

\ujvers\nemsloka {
{\devanagarifont यत्सांख्यसिद्धं कथयाम्यहं ते }%
  \dontdisplaylinenum}    \var{{\devanagarifontvar \numemph\va\textbf{॰सिद्धं}\lem \msCa\msCb\msNa\msNc\Ed, ॰सिद्धिं \msNb\oo 
\textbf{ते}\lem \msCa\msCb\msNapcorr\msNb\msNc\Ed, \om\ \msNaacorr}}% 


\nemslokab

{\devanagarifont संसारघोरार्णवयोगसारम्  \danda\dontdisplaylinenum }%
     \var{{\devanagarifontvar \numnoemph\vb\textbf{॰र्णव॰}\lem \msCa\msCb\msNb\msNc\Ed, ॰ण्ण॰ \msNaacorr, ॰ण्णव॰ \msNapcorr\oo 
\textbf{॰सारम्}\lem \msCa\msNa\msNb\msNc\Ed, सागरम् \msCb}}% 

\nemslokac

{\devanagarifont योगेषु सारेष्वथ पञ्चरात्रे }%
  \dontdisplaylinenum    \var{{\devanagarifontvar \numnoemph\vc\textbf{॰ष्वथ}\lem \msCa\msNa\msNb\msNc\Ed, ॰ष्वेथ \msCb\oo 
\textbf{पञ्चरात्रे}\lem \msCb\msNa\msNb\msNc\Ed, पञ्च\uncl{रात्रे} \msCa}}% 

%Verse 16:37


\nemslokad

{\devanagarifont वेदेषु शैवेषु च निश्चयस्ते {॥ १६:३७॥} \veg\dontdisplaylinenum }%
     \var{{\devanagarifontvar \numnoemph\vd\textbf{वेदेषु}\lem \msCb\msNa\msNc\Ed, \lk देषु \msCa, देवेषु \msNb\oo 
\textbf{निश्चयस्ते}\lem \msCa\msNc, निश्चयन्ते \msCb\Ed, 
निश्चयास्ते \msNa, निश्चय\uncl{स्वे} \msNb}}% 

\ujvers\nemsloka {
{\devanagarifont घ्राणेन्द्रियाद्येषु च यत्समस्तम् }%
  \dontdisplaylinenum}

\nemslokab

{\devanagarifont मनश्च लीनं भवतीव यस्य  \danda\dontdisplaylinenum }%
     \var{{\devanagarifontvar \numemph\vb\textbf{मनश्च}\lem \msCa\msCb\msNa\msNb\msNc, नभश्च \Ed}}% 

\nemslokac

{\devanagarifont बुद्ध्या नियम्य सकलान्हि भावान् }%
  \dontdisplaylinenum    \var{{\devanagarifontvar \numnoemph\vc\textbf{सकलान्हि}\lem \corr, सकलां हि \msCa\msNa\msNb\msNc, सकला हि \msCb, शकलां हि \Ed}}% 

%Verse 16:38


\nemslokad

{\devanagarifont स लब्धलक्ष्यः शिवमभ्युपैति {॥ १६:३८॥} \veg\dontdisplaylinenum }%
     \var{{\devanagarifontvar \numnoemph\vd\textbf{॰लक्ष्यः}\lem \msCa\msCb\msNb, ॰लक्ष्य॰ \msNa\Ed, ॰लक्ष॰ \msNc\oo 
\textbf{॰पैति}\lem \msCa\msNa\msNb\Ed, ॰पेति \msCb\msNc}}% 

\ujvers\nemsloka {
{\devanagarifont श्रोत्रादिसर्वेन्द्रियनिश्चलत्वे }%
  \dontdisplaylinenum}    \var{{\devanagarifontvar \numemph\va\textbf{श्रोत्रा॰}\lem \msCa\msCb\msNa\msNb\Ed, श्रोता॰ \msNc\oo 
\textbf{॰चलत्वे}\lem \emeHaru, ॰चलत्वम् \msCa\msCb\msNa\msNb\msNc\Ed}}% 


\nemslokab

{\devanagarifont एकाग्रचित्तं मनसा नियम्य  \danda\dontdisplaylinenum }%
 
\nemslokac

{\devanagarifont स्वदेहशून्यः स भवेच्चिरेण }%
  \dontdisplaylinenum    \var{{\devanagarifontvar \numnoemph\vc\textbf{॰शून्यः}\lem \msCa\msNa\msNb\msNc\Ed, ॰शून्यं \msCb}}% 

%Verse 16:39


\nemslokad

{\devanagarifont संयोगसिद्धिं प्रवदन्ति तज्ज्ञाः {॥ १६:३९॥} \veg\dontdisplaylinenum }%
     \var{{\devanagarifontvar \numnoemph\vd\textbf{संयोगसिद्धिं}\lem \msNa, संयोगसि\lk\  \msCa, संगसिद्धिं \msCb, स योगसिद्धिं \msNb\Ed, 
संयोगसिद्धं \msNc}}% 

\nemslokalong


\ujvers\nemsloka {
{\devanagarifont आदावेव मनः शनैरुपरमेत्कृत्वा च वश्येन्द्रियं }%
  \dontdisplaylinenum}    \var{{\devanagarifontvar \numemph\va\textbf{उपरमेत्कृ॰}\lem \msCa\msCb\msNb\msNc\Ed, उपरमे कृ॰ \msNa\oo 
\textbf{॰न्द्रियम्}\lem \msCa\msNa\msNb\msNc\Ed, ॰न्द्रियः \msCb}}% 


\nemslokab

{\devanagarifont यावत्तल्लयतां व्रजेत मनसा निःसंज्ञदेहस्तथा  \danda\dontdisplaylinenum }%
     \var{{\devanagarifontvar \numnoemph\vb\textbf{तल्लयतां}\lem \msCa\msCb\msNa\msNb\msNc, तत्तपतां \Ed\oo 
\textbf{मनसा निःसंज्ञ॰}\lem \Ed, मनसान्निस्संज्ञ॰ \msCa, मनसां निःसंज्ञ॰ \msCb, मनसान्निसंज्ञ॰ \msNa, 
मनसान्निस्सज्ञ॰ \msNb, मनसान्निःसज्ञ॰ \msNc}}% 

\nemslokac

{\devanagarifont एतद्ध्यानसमाधियोगसकलं प्राप्नोति निःसंशयं }%
  \dontdisplaylinenum    \var{{\devanagarifontvar \numnoemph\vc\textbf{॰सकलं}\lem \msCa\msCb\msNa\msNb\Ed, ॰सकल \msNc\oo 
\textbf{निःसंशयम्}\lem \msCa\msNb\Ed, निःसंशयः \msCb\msNa, निसंशयं \msNc}}% 

%Verse 16:40


\nemslokad

{\devanagarifont किं तच्छास्त्रसहस्रकोटिपठितं सारं न यो ऽन्विष्यति {॥ १६:४०॥} \veg\dontdisplaylinenum }%
     \var{{\devanagarifontvar \numnoemph\vd\textbf{किं त॰}\lem \msCa\msCb\msNa\msNb\msNc, चित्स॰ \Ed\oo 
\textbf{॰कोटि॰}\lem \msCb\msNa\msNb\msNc\Ed, ॰टोकि॰ \msCa\oo 
\textbf{॰पठितं}\lem \msCa\msCb\msNa\msNb\msNc, ॰मथितं \Ed\oo 
\textbf{न यो ऽन्विष्यति}\lem \msCa\msCb\msNc, न यो ऽन्विष्यते \msNa\msNb, तयेरिष्यति \Ed}}% 

\ujvers\nemsloka {
{\devanagarifont आत्मारामजितः समाधिनिरतो वैराग्यमप्याश्रितः }%
  \dontdisplaylinenum}    \var{{\devanagarifontvar \numemph\va\textbf{आत्मारामजितः}\lem \msCb\msNa\msNb\msNc, आत्मारा\uncl{म}\lk\lk\ \msCa, आत्मारामः जितः \Ed\oo 
\textbf{वैराग्यमप्याश्रितः}\lem \msCa\msCb\msNa\msNb, वैराग्य यस्याश्रितः \msNc, 
वैरागशय्याश्रितः \Ed}}% 


\nemslokab

{\devanagarifont चित्तं यस्य परिक्षयो यदि भवेत्तिष्ठेत्तनुत्वं यथा  \danda\dontdisplaylinenum }%
     \var{{\devanagarifontvar \numnoemph\vb\textbf{परि॰}\lem \msCa\msCb\msNb\msNc\Ed, परी॰ \msNa\oo 
\textbf{॰ष्ठेत्त॰}\lem \msCa\msCb\msNa\msNb\Ed, ॰ष्ठन्त॰ \msNc}}% 

\nemslokac

{\devanagarifont तज्ज्ञेयं गतिमुत्तमं शिवपदं संसारदुःखच्छिदं }%
  \dontdisplaylinenum    \paral{{\devanagarifontsmall \vc {\englishfont  cf.\ 22.41d:} उत्तमां गतिमाप्नुयात् }}

%Verse 16:41


\nemslokad

{\devanagarifont वेदान्तेषु च निष्ठ एष कथितः किं शास्त्रमन्यद्विशेत् {॥ १६:४१॥} \veg\dontdisplaylinenum }%
     \var{{\devanagarifontvar \numnoemph\vd\textbf{अन्यद्वि॰}\lem \msCa\msCb\msNb\msNc\Ed, अन्यं वि॰ \msNa}}% 

\ujvers\nemsloka {
{\devanagarifont हृत्पद्मे कर्णिकायामुपरि रविरवद्योतयन्तो ऽन्तरालम् }%
  \dontdisplaylinenum}    \var{{\devanagarifontvar \numemph\va\textbf{॰पद्मे}\lem \conj, ॰पद्म॰ \msCa\msCb\msNa\msNb\msNc\Ed\unmetr\oo 
\textbf{रविरवद्योतयन्तो}\lem \eme, रविरवंद्योतयन्तो \msCa\msCb\msNa\msNb, 
रविरिव द्योन्तयन्तो \msNc, रविरतद्योतयन्तो \Ed}}% 


\nemslokab

{\devanagarifont यत्तेजस्तेजमार्गैर्बहलतमघनैर्द्योतनाद्दीप्तदीपम्  \danda\dontdisplaylinenum }%
     \var{{\devanagarifontvar \numnoemph\vb\textbf{यत्ते॰}\lem \msCb, यस्ते॰ \msCa\msNa\msNb\Ed, सस्ते॰ \msNc\oo 
\textbf{॰मार्गैर्बहल॰}\lem \msCa\msCb\msNb, ॰मार्गै बहल॰ \msNa, ॰मार्गे बहुल॰ \msNc, ॰मार्गौ बहुल॰ \Ed\oo 
\textbf{॰तमघनैर्द्योतनाद्दीप्तदीपम्}\lem \conj, 
॰तमघनैर्घातनाद्दीप्तदीपम् \msCa, 
॰मघनै घाटनादीप्तदीपम् \msCb, 
॰तमघनैर्घाटनादीप्तदीपम् \msNa, 
॰तमघनै घाटनादीप्तदीपम् \msNb, 
॰तमघनैर्द्योटना दीप्तदीपं \msNc, 
॰तमघनैर्द्योतनाद्दीप्तदीपः \Ed}}% 

\nemslokac

{\devanagarifont भित्त्वा यत्तालुदेशे मुखमुपरिगतं तालुदेशेन मूर्ध्नि }%
  \dontdisplaylinenum    \var{{\devanagarifontvar \numnoemph\vc\textbf{यत्तालु॰}\lem \Ed, घंट्टाल॰ \msCa, घतोल॰ \msCb घण्टाल॰ \msNa\msNb, द्यण्टाल॰ \msNc\oo 
\textbf{॰गतं}\lem \msNc\Ed, ॰गत॰ \msCa\msNa\msNb, ॰गतस्॰ \msCb}}% 

%Verse 16:42


\nemslokad

{\devanagarifont ! मूर्ध्नि द्वारान्तरेण शिवपरमपदं यान्ति योगेन युक्ताः {॥ १६:४२॥} \veg\dontdisplaylinenum }%
     \var{{\devanagarifontvar \numnoemph\vd\textbf{मूर्ध्नि}\lem \msNa, मूर्ध्न \msCa\msCb\msNb, मूर्द्ध॰ \msNc, मूर्ध्न्या \Ed}}% 

\ujvers\nemsloka {
{\devanagarifont कृष्णः कृष्णतमोत्तमो ऽतिमहतो यस्तेजतेजात्मकः }%
  \dontdisplaylinenum}    \var{{\devanagarifontvar \numemph\va\textbf{कृष्णः}\lem \emeKafle, कृष्णं \msCa\msCb\msNa\msNb\msNc, कृत्स्नं \Ed\oo 
\textbf{॰तमोत्तमो}\lem \conj, ॰तमोतमो \msCa\msCb\msNa\msNb\msNc\Ed\oo 
\textbf{ऽति॰}\lem \msCa\msNa\msNb\msNc\Ed, हि \msCb\oo 
\textbf{यस्तेजते॰}\lem \Ed, यस्तेजस्ते॰ \msCa\msCb\msNa\msNb\msNc\ \unmetr}}% 


\nemslokab

{\devanagarifont लोकालोकधराधरः श्रियपतिः प्राणप्रविष्टालयः  \danda\dontdisplaylinenum }%
     \var{{\devanagarifontvar \numnoemph\vb\textbf{॰धराधरः श्रियपतिः}\lem \Ed, 
॰धरो धराधरधरः \msCa\msCb\msNb\msNc, 
॰धरो धरधरधरः \msNa\ \unmetr\oo 
\textbf{श्रियपतिः}\lem \msCa\msCb\msNa\Ed, \om\ \msNb\msNc\oo 
\textbf{प्राण॰}\lem \msCa\msCb\msNa\msNb\Ed, प्राणः \msNc\oo 
\textbf{॰प्रविष्टालयः}\lem \msCb\msNa\msNb\msNc, ॰\uncl{प्र}विष्टो लयः \msCa, प्रतिष्ठालयः \Ed}}% 

\nemslokac

{\devanagarifont कर्ता कारणमव्ययो ऽव्ययमसौ व्यापी विभक्ताविदम् }%
  \dontdisplaylinenum
%Verse 16:43


\nemslokad

{\devanagarifont विष्णुर्भावमयो विभक्तविषयैर्विश्वेश्वरो विश्ववित् {॥ १६:४३॥} \veg\dontdisplaylinenum }%
     \var{{\devanagarifontvar \numnoemph\vd\textbf{भावमयो}\lem \msCa\msCb\msNa\msNb\msNc, भावमयैर् \Ed\oo 
\textbf{विश्ववित्}\lem \msCa\msCb\msNa\msNb\Ed, विश्ववत् \msNc}}% 

\ujvers\nemsloka {
{\devanagarifont ! एष तत्त्ववरः परापरमयस्तेजः परस्थानदः }%
  \dontdisplaylinenum}    \var{{\devanagarifontvar \numemph\va\textbf{परापरमयस्ते॰}\lem \conj, परः परमयस्ते॰ \msCa\msNa\msNb\Ed, परः परमस्ते॰ \msCb, 
परः परमयेस्ते॰ \msNc\oo 
\textbf{॰परस्थानदः}\lem \conj, ॰परः स्थानदः \msCa\msCb\msNa\msNb\msNcpcorr\Ed, ॰परः स्थानद \msNcacorr}}% 


\nemslokab

{\devanagarifont बुद्ध्या भावनभावयेन्द्रियमनो देहान्तरालोकयन्  \danda\dontdisplaylinenum }%
     \var{{\devanagarifontvar \numnoemph\vb\textbf{॰भावयेन्द्रियमनो}\lem \msCa\msNa\msNb, ॰भावयन्द्रियमनो \msCb, 
॰भाव\uncl{व}येन्द्रियमनो \msNc, ॰भावयन्नियमनो \Ed\oo 
\textbf{देहान्तरालोकयन्}\lem \msCa\msNa\msNb\msNc, देहान्तरालोकयत् \msCb, देहान्तरोस्तोकयन् \Ed}}% 

\nemslokac

{\devanagarifont हृत्पद्मायतनस्थितः स पुरुषो निश्वासमुच्छ्वासदः }%
  \dontdisplaylinenum    \var{{\devanagarifontvar \numnoemph\vc\textbf{स पुरुषो नि॰}\lem \msNa\msNb\msNc\Ed, \uncl{स पुरुषो} \lk\ \msCa, पुरुषौ नि॰ \msCb\oo 
\textbf{॰च्छ्वासदः}\lem \msCa\msCb\msNa\msNb\msNc, ॰च्छ्वासदाम् \Ed}}% 

%Verse 16:44


\nemslokad

{\devanagarifont नादस्तस्य सदा सदा नदति तं नादोपरिष्ठा हरः {॥ १६:४४॥} \veg\dontdisplaylinenum }%
     \var{{\devanagarifontvar \numnoemph\vd\textbf{नादस्तस्य}\lem \msCa\msCb\msNa\msNb\msNc, नादन्तस्य \Ed\oo 
\textbf{नदति तं}\lem \msCa\msCb\msNa\msNb\msNc, न पतितं \Ed\oo 
\textbf{॰परिष्ठा हरः}\lem \msCa\msCb\msNa\msNb, ॰परिष्ठारद्धरः \msNc, ॰परिष्टद्वरः \Ed}}% 

\ujvers\nemsloka {
{\devanagarifont यस्तेजस्तेजते ऽजो बहुनिविडघनो ग्रन्थिमालोपगूढः }%
  \dontdisplaylinenum}    \var{{\devanagarifontvar \numemph\va\textbf{यस्तेजस्तेजते ऽजो}\lem \conj, 
यस्तेजस्तेजस्तेजो \msCa\msCb\msNa\msNb\msNc\ \unmetr\ 
यस्तेजस्तेजसो वा \Ed\oo 
\textbf{॰निविड॰}\lem \msCa\msCb\msNa\msNb\msNc, ॰निविदु॰ \Ed\oo 
\textbf{॰घनो}\lem \msNc\msCb, ॰घनः \msCa\msNa\msNb\Ed\oo 
\textbf{ग्रन्थिमालो॰}\lem \msCa\msNa\msNb\msNc, ग्रत्थिमानो॰ \msCb\Ed}}% 


\nemslokab

{\devanagarifont मूर्तिर्मूर्तानुसारी बहुकरणभृतं कारणाद्देहबन्धः  \danda\dontdisplaylinenum }%
     \var{{\devanagarifontvar \numnoemph\vb\textbf{मूर्तिर्मूर्ता॰}\lem \msCa, मूर्तिमूर्ता॰ \msCb\msNa\msNb\msNc, मूर्तिर्मूर्त्य॰ \Ed\oo 
\textbf{बहुकरण॰}\lem \msCa\msNa\msNb\msNcpcorr\Ed, बहुकर॰ \msNcacorr, बह्यकरण॰ \msCb\unmetr\oo 
\textbf{॰भृतं}\lem \msCa\msCb\msNa\msNcpcorr\Ed, ॰वृतं \msNb, ॰भृत \msNcacorr\oo 
\textbf{कारणाद्देहबन्धः}\lem \msCa\msCb\msNa\msNb, कारणाद्देहबन्ध \msNc, कारणं देहबन्धः \Ed}}% 

\nemslokac

{\devanagarifont भित्त्वा ग्रन्थिं सपाशं विषमिव विषयं त्यक्तसङ्गैकभावाः }%
  \dontdisplaylinenum    \var{{\devanagarifontvar \numnoemph\vc\textbf{ग्रन्थिं}\lem \msCa\msCb\msNa\msNb\Ed, ग्रन्थि \msNc\oo 
\textbf{सपाशं}\lem \msNa\Ed, सपाशां \msCa\msCb\msNb\msNcpcorr, सपाशा \msNcacorr\oo 
\textbf{॰सङ्गैक॰}\lem \msCb\msNa\msNb\msNc\Ed, ॰सङ्सैक॰ \msCa}}% 

%Verse 16:45


\nemslokad

{\devanagarifont पश्यन्त्येते तमीशं गुणकलरहितं निर्विकारं प्रकाशम् {॥ १६:४५॥} \veg\dontdisplaylinenum }%
     \var{{\devanagarifontvar \numnoemph\vd\textbf{पश्यन्त्येते तमी॰}\lem \msCb\msNa\msNb, पश्यन्त्ये\lk\lk मी॰ \msCa, 
पश्यन्तेते तमी॰ \msNc, पश्यन्त्येतेनमी॰ \Ed}}% 

\ujvers\nemsloka {
{\devanagarifont यो ऽसौ तेजान्तरात्मा कमलपुटकुटीसंकटस्थानलीनः }%
  \dontdisplaylinenum}    \var{{\devanagarifontvar \numemph\va\textbf{(यो{\englishfont ...} ॰लीनः)}\lem \msCapcorr\msCb\msNa\msNb\msNc\Ed, \om\ \msCaacorr\oo 
\textbf{यो ऽसौ तेजान्तरात्मा}\lem \msCb\msNa\msNc\Ed, \lk\lk \lk \uncl{जान्त}रात्मा \msCapcorr, \om\ \msCaacorr, 
यो सौ तेजान्तराल॰ \msNb\oo 
\textbf{॰कुटी॰}\lem \msCapcorr\msCb\msNa\msNb\msNc, \om\ \msCaacorr, ॰कुटि॰ \Ed}}% 


\nemslokab

{\devanagarifont इन्दोर्भासानुरूपी विमलदलसदाच्छादितः कर्णिकायाम्  \danda\dontdisplaylinenum }%
     \var{{\devanagarifontvar \numnoemph\vb\textbf{इन्दोर्भासानु॰}\lem \msCa\msNb\msNcpcorr\Ed, इन्दो भासानु॰ \msCb\msNa, 
इन्दोर्भासानुरूपी विमलः इन्दोर्भासानु॰ \msNcacorr\oo 
\textbf{॰रूपी}\lem \msCa\msCb\msNa\msNb\msNc, ॰रूपि \Ed\ \unmetr\oo 
\textbf{॰च्छादितः}\lem \msCa\msCb\msNa\msNbpcorr\msNc\Ed, ॰च्छादि \msNbacorr}}% 

\nemslokac

{\devanagarifont तत्र स्थाने स्थितो ऽसौ त्रिभुवननिलयः सर्वभूताधिवासः }%
  \dontdisplaylinenum
%Verse 16:46


\nemslokad

{\devanagarifont आकाशादूर्ध्वतत्त्वस्थितविकसकलासंहतो मुक्तबन्धः {॥ १६:४६॥} \veg\dontdisplaylinenum }%
     \var{{\devanagarifontvar \numnoemph\vd\textbf{आकाशादूर्ध्व॰}\lem \msCa\msCb\msNa\msNb\Ed, आकाशाद्दूर्ध्व॰ \msNc\oo 
\textbf{॰स्थित॰}\lem \conj, ॰सित॰ \msCa\msCb\msNa\msNb\Ed\ \unmetr, ॰सिसित॰ \msNc\ \unmetr\oo 
\textbf{॰कलासंहतो}\lem \Ed, ॰कसासंहतो \msCa\msCb\msNa\msNb\msNc\oo 
\textbf{मुक्त॰}\lem \conj, मुक्ति॰ \msCa\msCb\msNa\msNb\msNc\Ed}}% 

\nemslokanormal


\ujvers\nemsloka {
{\devanagarifont एतानि तत्त्वान्यखिलानि देवि }%
  \dontdisplaylinenum}    \var{{\devanagarifontvar \numemph\va\textbf{॰खिलानि}\lem \msCa\msNa\msNb\msNc\Ed, ॰खिकाति \msCb\oo 
\textbf{देवि}\lem \msCb\msNa\msNb\msNc\Ed, \uncl{दे}\lk\ \msCa}}% 


\nemslokab

{\devanagarifont संक्षेपतः कीर्तितः पञ्चभेदः  \danda\dontdisplaylinenum }%
 
\nemslokac

{\devanagarifont श्रोतुं किमन्यद्विजिगीषितार्थम् }%
  \dontdisplaylinenum    \var{{\devanagarifontvar \numnoemph\vc\textbf{श्रोतुं किम्}\lem \msCa\msCb\msNa\msNb\msNc, श्रोतकिम् \Ed\oo 
\textbf{विजिगीषिता॰}\lem \msCa\msNa\msNb\msNc\Ed, विजिगीषता॰ \msCb}}% 

%Verse 16:47


\nemslokad

{\devanagarifont संसारमोक्षेण च तत्परो ऽस्ति {॥ १६:४७॥} \veg\dontdisplaylinenum }%
 
\vers


{\devanagarifont देव्युवाच {\dandab}\dontdisplaylinenum  }%
 
\nemsloka 
{\devanagarifont तुष्टास्मि देव मम संशयमद्य नष्टम् }%
  \dontdisplaylinenum    \var{{\devanagarifontvar \numemph\va\textbf{तुष्टा॰}\lem \msCa\msNa\msNb\msNc, तु\uncl{ष्टा}॰ \msCb, तुष्टो \Ed}}% 


\nemslokab

{\devanagarifont अद्य प्रसन्नपरमेश्वर ईश्वर त्वम्  \danda\dontdisplaylinenum }%
     \var{{\devanagarifontvar \numnoemph\vb\textbf{॰परमेश्वर}\lem \msCa\msCb\msNa\msNc\Ed, ॰परमेरश्वर \msNb\oo 
\textbf{ईश्वर}\lem \msCa\msCb\msNa\msNb\msNc, ईश्वरम \Ed}}% 

\nemslokac

{\devanagarifont अद्य श्रुतं त्वयि च पुण्यफलप्रभावम् }%
  \dontdisplaylinenum    \var{{\devanagarifontvar \numnoemph\vc\textbf{(अद्य{\englishfont ...} ॰प्रभावम)}\lem \msCa\msNa\msNb\msNc\Ed, \om\ \msCb}}% 

%Verse 16:48


\nemslokad

{\devanagarifont पूर्णानि चाद्य मम इष्टमनोरथानि {॥ १६:४८॥} \veg\dontdisplaylinenum }%
     \var{{\devanagarifontvar \numnoemph\vd\textbf{इष्टमनोरथानि}\lem \msCb\msNa\msNb\msNc\Ed, \uncl{इष्ट}\lk\lk \lk थानि \msCa}}% 

\ujvers\nemsloka {
{\devanagarifont अज्ञानपङ्कघनमध्यनिलीयमानाम् }%
  \dontdisplaylinenum}    \var{{\devanagarifontvar \numemph\va\textbf{॰निलीयमानाम्}\lem \msCa\msNa\msNc, ॰निलीयमानम् \msCb\msNb\Ed}}% 


\nemslokab

{\devanagarifont उत्तारयेश सकलार्तिविनाशनाय  \danda\dontdisplaylinenum }%
     \var{{\devanagarifontvar \numnoemph\vb\textbf{उत्तारयेश}\lem \msCb\msNa\msNb\msNc\Ed, उत्तरायेश \msCaacorr, उत्तरयेश \msCapcorr}}% 

\nemslokac

{\devanagarifont सर्वेश तत्त्वपरमार्थ नमो नमस्ते }%
  \dontdisplaylinenum
%Verse 16:49


\nemslokad

{\devanagarifont अद्यापि तृप्तिरिह नास्ति ममापि शम्भो {॥ १६:४९॥} \veg\dontdisplaylinenum }%
     \var{{\devanagarifontvar \numnoemph\vd\textbf{नास्ति ममापि}\lem \msCa\msCb\msNapcorr\msNb\msNc\Ed, ना पि \msNaacorr}}% 

\ujvers\nemsloka {
{\devanagarifont पीत्वामृतं चोत्तमवक्त्रजातम् }%
  \dontdisplaylinenum}    \var{{\devanagarifontvar \numemph\va\textbf{॰वक्त्र॰}\lem \msCa\msCb\msNb\msNc\Ed, ॰वचक्त्र॰ \msNaacorr, ॰चक्त्र॰ \msNapcorr}}% 


\nemslokab

{\devanagarifont आख्याहि दानं फलधर्मसारम्  \danda\dontdisplaylinenum }%
 
\nemslokac

{\devanagarifont संसारपारं परमं नयस्व }%
  \dontdisplaylinenum    \var{{\devanagarifontvar \numnoemph\vc\textbf{परमं}\lem \msCa\msCb\msNa\msNc\Ed, परम \msNb\oo 
\textbf{नयस्व}\lem \msCb\msNa\msNb\msNc\Ed, नय\lk\ \msCa}}% 

%Verse 16:50


\nemslokad

{\devanagarifont कृपां मयीशान कुरु प्रसीद {॥ १६:५०॥} \veg\dontdisplaylinenum }%
     \var{{\devanagarifontvar \numnoemph\vd\textbf{कृपां मयीशान कुरु प्रसीद}\lem \Ed, \om\ \msCa\msCb\msNa\msNb\msNc}}% 

\vers


{\devanagarifont 
\jump
\begin{center}
\ketdanda~इति वृषसारसंग्रहे ऽध्यात्मनिर्णयो नामाध्यायः षोडशमः~\ketdanda
\end{center}
\dontdisplaylinenum\vers  }%
     \var{{\devanagarifontvar \numnoemph{\englishfont \Colo:}\textbf{ऽध्यात्म॰}\lem \corr, अध्यात्म॰ \msCa\msNa\msNb\msNc\Ed, आत्म॰ \msCb\oo 
\textbf{नामाध्यायः षोडशमः}\lem \msCa\msCb\msNa\msNb\msNc, 
नाम षोडशो ऽध्यायः \Ed}}% 
\bekveg\szamveg
\vfill
\phpspagebreak

\versno=0\fejno=17
\thispagestyle{empty}

\centerline{\Large\devanagarifontbold [   सप्तदशमो ऽध्यायः  ]}{\vrule depth10pt width0pt} \fancyhead[CE]{{\footnotesize\devanagarifont वृषसारसंग्रहे  }}
\fancyhead[CO]{{\footnotesize\devanagarifont सप्तदशमो ऽध्यायः  }}
\fancyhead[LE]{}
\fancyhead[RE]{}
\fancyhead[LO]{}
\fancyhead[RO]{}
\szam\bek



\alalfejezet{दानधर्मविशेषः}
\vers


{\devanagarifont देव्युवाच {\dandab}\dontdisplaylinenum  }%
     \lacuna{\devanagarifontsmall {\englishfont Witnesses used for this chapter---\msCa: f.~222r line~2 -- f.~224r line~4; 
                                               \msCb: f.~225r line~3 -- f.~226v line~6;
                                               \msNa: f.~29r line~4 -- f.~31r;
                                               \msParis:  f.~137 line~4 -- f.~141 line~1;
                                               \msPaperA: f.~229r line~2 -- f.~232v line~7;
                                               \Ed: pp.~646--649        
                                               (it breaks down after 17.38);
                                               this chapter is missing in \msCc.} }%
  
{\devanagarifont पृथग्दानस्य इच्छामि श्रोतुं मां दातुमर्हसि \thinspace{\danda} \dontdisplaylinenum }%
     \var{{\devanagarifontvar \numemph\vb\textbf{श्रोतुं मां दातुमर्हसि}\lem \mssALL, 
माहात्म्यं वक्तुमर्हसि \Ed}}% 

%Verse 17:1

{\devanagarifont अन्नवस्त्रहिरण्यानां गोभूमिकनकस्य च {॥ १७:१॥} \veg\dontdisplaylinenum }%
     \var{{\devanagarifontvar \numnoemph\vc\textbf{अन्न॰}\lem \mssALL, 
अन्य॰ \msParisacorr\ {\englishfont (diff.\ hand)}\oo 
\textbf{॰वस्त्र॰}\lem \mssALL, ॰वस्त्रं \msNa}}% 


\alalalfejezet{अन्नप्रदानम्}

{\devanagarifont भगवानुवाच {\dandab}\dontdisplaylinenum  }%
 
\nemsloka 
{\devanagarifont सुसंस्कृतमन्नमतिप्रदद्याद् }%
  \dontdisplaylinenum

\nemslokab

{\devanagarifont घृतप्रभूतमवदंशयुक्तम्  \danda\dontdisplaylinenum }%
     \var{{\devanagarifontvar \numemph\vb\textbf{॰भूत॰}\lem \mssALL, 
॰सूत॰ \msNa, ॰भूतव॰ \msParisacorr\ \hypermetr}}% 

\nemslokac

{\devanagarifont घृतप्रपक्वं सुकृतं च पूपं }%
  \dontdisplaylinenum    \var{{\devanagarifontvar \numnoemph\vc\textbf{सुकृतं च पूपं}\lem \mssALL, 
सुकृतं पूपं \msNaacorr, सुकृतम्मपूपं \Ed}}% 

%Verse 17:2


\nemslokad

{\devanagarifont सितेन खण्डेन गुडेन युक्तम् {॥ १७:२॥} \veg\dontdisplaylinenum }%
 
\ujvers\nemsloka {
{\devanagarifont मार्गं खगं चोदकजङ्गलं च }%
  \dontdisplaylinenum}    \var{{\devanagarifontvar \numemph\va\textbf{मार्गं}\lem \mssALL, मार्ग॰ \Ed\ \unmetr\oo 
\textbf{खगं चो॰}\lem \mssALL, 
खञ्चो॰ \msCa, खगश्चो॰ \Ed\oo 
\textbf{॰जङ्गलं च}\lem \mssALL, ॰जङ्गमश्च \Ed}}% 


\nemslokab

{\devanagarifont दद्याद्वटं नागरवंशमूलम्  \danda\dontdisplaylinenum }%
     \var{{\devanagarifontvar \numnoemph\vb\textbf{वटं}\lem \mssALL, वट \Ed\ \unmetr}}% 

\nemslokac

{\devanagarifont शाकं फलं चाम्लमधूरतिक्तं }%
  \dontdisplaylinenum
%Verse 17:3


\nemslokad

{\devanagarifont पानं पयः शीतसुगन्धतोयम् {॥ १७:३॥} \veg\dontdisplaylinenum }%
 
\ujvers\nemsloka {
{\devanagarifont दधि प्रदद्याद्गुडमिश्रितं च }%
  \dontdisplaylinenum}

\nemslokab

{\devanagarifont मृणाल शालूक च नालका च  \danda\dontdisplaylinenum }%
     \var{{\devanagarifontvar \numemph\vb\textbf{॰शालूक च}\lem \conj, ॰शालूक व \msCa\msNa\msParis\msPaperA\Ed, ॰क व \msCb}}% 

\nemslokac

{\devanagarifont सदक्षिणालेपपवित्रपुष्पं }%
  \dontdisplaylinenum    \var{{\devanagarifontvar \numnoemph\vc\textbf{सदक्षिणा॰}\lem \mssALL, सक्षिणा॰ \msParisacorr, 
स\uncl{द}क्षिणा॰ \msParispcorr\ {\englishfont (diff.\ hand)}}}% 

%Verse 17:4


\nemslokad

{\devanagarifont श्रद्धान्वितः सत्कृतया प्रणम्य {॥ १७:४॥} \veg\dontdisplaylinenum }%
     \var{{\devanagarifontvar \numnoemph\vd\textbf{॰न्वितः}\lem \mssALL, ॰न्वित \msParis\oo 
\textbf{सत्कृतया}\lem \mssALL, सक्ततया \Ed}}% 

\ujvers\nemsloka {
{\devanagarifont प्रयाति लोकं जगदीश्वरस्य }%
  \dontdisplaylinenum}    \var{{\devanagarifontvar \numemph\va\textbf{प्रयाति}\lem \msParispcorr\Ed, प्रयान्ति \msCa\msCb\msNa\msParisacorr, 
प्रया ि{}\lk\ \msPaperA}}% 


\nemslokab

{\devanagarifont विमानयानैः सहितो ऽप्सरोभिः  \danda\dontdisplaylinenum }%
 
\nemslokac

{\devanagarifont एकैकसिक्थस्य सहस्रवर्षम् }%
  \dontdisplaylinenum    \var{{\devanagarifontvar \numnoemph\vc\textbf{॰सिक्थस्य}\lem \mssALL, ॰सिष्टस्य \Ed}}% 

%Verse 17:5


\nemslokad

{\devanagarifont अन्नप्रदो मोदति देवलोके {॥ १७:५॥} \veg\dontdisplaylinenum }%
 
\ujvers\nemsloka {
{\devanagarifont च्युतश्च मर्त्ये स भवेद्धनाढ्यः }%
  \dontdisplaylinenum}

\nemslokab

{\devanagarifont कुलोद्गतः सर्वगुणोपपन्नः  \danda\dontdisplaylinenum }%
 
\nemslokac

{\devanagarifont यशः श्रियं सर्वकलाज्ञता च }%
  \dontdisplaylinenum    \var{{\devanagarifontvar \numemph\vc\textbf{॰कला॰}\lem \eme, ॰कल॰ \msCa\msCb\msNa\msParis\msPaperA\Ed}}% 

%Verse 17:6


\nemslokad

{\devanagarifont भवेत्स भोगी सकलत्रपुत्रः {॥ १७:६॥} \veg\dontdisplaylinenum }%
     \var{{\devanagarifontvar \numnoemph\vd\textbf{॰कलत्रपुत्रः}\lem \msCa\msNa\msParis\msPaperA\Ed, ॰कलत्रः \msCb}}% 

\ujvers\nemsloka {
{\devanagarifont दद्याद्दरिद्रकृपणार्तदीनां }%
  \dontdisplaylinenum}    \var{{\devanagarifontvar \numemph\va\textbf{॰रिद्र}\lem \msCb\msNa\msParis, ॰रिद्रः \msCa\msPaperA\Ed\oo 
\textbf{॰दीनां}\lem \eme, ॰दीना \msCa\msCb\msNa\msParis\msPaperA, ॰दीनो \Ed}}% 


\nemslokab

{\devanagarifont कालागतत्वातुरमागतानाम्  \danda\dontdisplaylinenum }%
     \var{{\devanagarifontvar \numnoemph\vb\textbf{कालागतत्वा॰}\lem \msCa\msCb\msNa\msParispcorr, 
कलागतत्वा॰ \msParisacorr, कालागदत्वा॰ \msPaperA, वालागदत्वा॰ \Ed\oo 
\textbf{॰तुरमा}\lem \mssALL, ॰तुमा॰ \msParis}}% 

\nemslokac

{\devanagarifont तृष्णाबुभुक्षागतिकागतानां }%
  \dontdisplaylinenum    \var{{\devanagarifontvar \numnoemph\vc\textbf{तृष्णा॰}\lem \msNa\msPaperA\Ed, तृष्णां \msCa\msCb, कृष्णाम् \msParis\oo 
\textbf{॰बुभुक्षा॰}\lem \mssALL, भुभुक्ता॰ \msCa}}% 

%Verse 17:7


\nemslokad

{\devanagarifont दत्त्वा स धर्मफलमाश्रयेत {॥ १७:७॥} \veg\dontdisplaylinenum }%
     \var{{\devanagarifontvar \numnoemph\vd\textbf{॰श्रयेत}\lem \msCa\msNa\msPaperA\Ed, ॰श्रयेत् \msCb\msParis}}% 

\ujvers\nemsloka {
{\devanagarifont देशे च काले च तथा च पात्रे }%
  \dontdisplaylinenum}    \var{{\devanagarifontvar \numemph\va\textbf{पात्रे}\lem \mssALL, यात्रे \msNa}}% 


\nemslokab

{\devanagarifont दानादिधर्मस्य फलं कनिष्ठम्  \danda\dontdisplaylinenum }%
     \var{{\devanagarifontvar \numnoemph\vb\textbf{दानादि॰}\lem \msCa\msNa\msPaperA, दानानि \msCb\msParis}}% 
    \lacuna{\devanagarifontsmall \vab {\englishfont missing in \Ed.} }%
      \paral{{\devanagarifontsmall \vab {\englishfont \compare\ \BhG\ 17.20 = \Hitop\ 1.16:}
                         दातव्यमिति यद्दानं दीयते ऽनुपकारिणे\thinspace{\devanagarifontsmall ।}
                         देशे काले च पात्रे च तद्दानं सात्त्विकं विदुः\thinspace{\devanagarifontsmall ॥} }}

\nemslokac

{\devanagarifont वाणिज्यधर्मा हि फलाश्रितानां }%
  \dontdisplaylinenum    \var{{\devanagarifontvar \numnoemph\vc\textbf{वाणिज्य॰}\lem \mssALL, 
वाणि \msNaacorr, वणिज्यं \msNapcorr\oo 
\textbf{॰धर्मा हि॰}\lem \mssALL, 
॰धर्म्मो \msParispcorr, ॰धर्मादि \Ed}}% 

%Verse 17:8


\nemslokad

{\devanagarifont धर्मो हि तस्य न च निर्मलो ऽस्ति {॥ १७:८॥} \veg\dontdisplaylinenum }%
     \var{{\devanagarifontvar \numnoemph\vd\textbf{हि}\lem \mssALL, स्ति \msNa}}% 

\ujvers\nemsloka {
{\devanagarifont तोयं च दद्याल्लघुपूर्णकुम्भं }%
  \dontdisplaylinenum}

\nemslokab

{\devanagarifont शीतं सुगन्धं परिवासितं च  \danda\dontdisplaylinenum }%
 
\nemslokac

{\devanagarifont स याति लोकं सलिलेश्वरस्य }%
  \dontdisplaylinenum
%Verse 17:9


\nemslokad

{\devanagarifont न सप्तजन्मानि तृषाभिभूतः {॥ १७:९॥} \veg\dontdisplaylinenum }%
     \var{{\devanagarifontvar \numemph\vd\textbf{सप्त॰}\lem \mssALL, तस्य \Ed}}% 


\alalfejezet{वस्त्रादिप्रदानम्}
\ujvers\nemsloka {
{\devanagarifont उपानहं यो ददति द्विजाय }%
  \dontdisplaylinenum}    \var{{\devanagarifontvar \numemph\va\textbf{यो}\lem \msCa\msNa\Ed, ये \msCb\msParis, य \msPaperA}}% 


\nemslokab

{\devanagarifont सुशोभनं तैलसुदीपितं च  \danda\dontdisplaylinenum }%
     \var{{\devanagarifontvar \numnoemph\vb\textbf{॰दीपितं च}\lem \mssALL, 
॰दीपितं \msNaacorr, 
॰दीसुरपितञ्च \Ed\ \hypermetr}}% 

\nemslokac

{\devanagarifont ते यान्ति लोकममराधिपस्य }%
  \dontdisplaylinenum    \var{{\devanagarifontvar \numnoemph\vc\textbf{लोकममरा॰}\lem \mssALL, लोकं समरा॰ \msNa}}% 

%Verse 17:10


\nemslokad

{\devanagarifont यमालयं कष्टपथा न यान्ति {॥ १७:१०॥} \veg\dontdisplaylinenum }%
 
\ujvers\nemsloka {
{\devanagarifont प्रक्षीणपुण्यः पुनरत्र लोके }%
  \dontdisplaylinenum}    \var{{\devanagarifontvar \numemph\va\textbf{॰पुण्यः}\lem \msCa\msCb\msParis\msPaperA, 
\om\ \msNaacorr, ॰पुण्य \msNapcorr, ॰पुण्या \Ed\oo 
\textbf{पुनरत्र लोके}\lem \msCa\msNa\msParis\msPaperA\Ed, 
पुनरभ्युपेति \msCb}}% 


\nemslokab

{\devanagarifont जातो भवेद्दिव्यकुलोपपन्नः  \danda\dontdisplaylinenum }%
     \var{{\devanagarifontvar \numnoemph\vb\textbf{॰कुलोप॰}\lem \mssALL, ॰कुलोत्प॰ \msPaperA}}% 

\nemslokac

{\devanagarifont धनैः समृद्धो ऽधिपतित्वतां च }%
  \dontdisplaylinenum    \var{{\devanagarifontvar \numnoemph\vc\textbf{समृद्धो}\lem \mssALL, समुद्धृतो \msParis\ \hypermetr\oo 
\textbf{॰पतित्वतां च}\lem \mssALL, 
॰पतित्वताश्च \Ed}}% 

%Verse 17:11


\nemslokad

{\devanagarifont रथाश्वनागासनगा भवन्ति {॥ १७:११॥} \veg\dontdisplaylinenum }%
     \var{{\devanagarifontvar \numnoemph\vd\textbf{रथाश्वनागासनगा भवन्ति}\lem \mssALL, 
रथाश्वनाथगासनगा भवन्ति \msParisacorr, 
रथाश्च नागा प्रभवन्ति तस्य \Ed}}% 

\ujvers\nemsloka {
{\devanagarifont वस्त्रप्रदानेन भवन्ति देवि }%
  \dontdisplaylinenum}

\nemslokab

{\devanagarifont रूपोत्तमाः सर्वकलाज्ञता च  \danda\dontdisplaylinenum }%
     \var{{\devanagarifontvar \numemph\vb\textbf{॰त्तमाः}\lem \mssALL, ॰त्तमा \msCb, ॰त्तम॰ \Ed\oo 
\textbf{॰कलाज्ञता च}\lem \eme, ॰कलज्ञताश्च \msCa\msNa\msParis\msPaperA, 
॰कलज्ञता च \msCb, ॰कलज्ञताञ्च \Ed}}% 

\nemslokac

{\devanagarifont समृद्धिसौभाग्यगुणान्विताश्च }%
  \dontdisplaylinenum
%Verse 17:12


\nemslokad

{\devanagarifont स्वर्गच्युतास्ते पुरुषा भवन्ति {॥ १७:१२॥} \veg\dontdisplaylinenum }%
 
\ujvers\nemsloka {
{\devanagarifont वस्त्रप्रदानाभिरतस्य पुंसः }%
  \dontdisplaylinenum}

\nemslokab

{\devanagarifont अन्यां प्रवक्ष्यामि ततः प्रशंसाम्  \danda\dontdisplaylinenum }%
     \var{{\devanagarifontvar \numemph\vb\textbf{अन्यां प्र॰}\lem \msCa\msNa\msParis, अन्यत्प्र॰ \msCb\Ed, अन्यात्प्र॰ \msPaperA\oo 
\textbf{॰शंसाम्}\lem \mssALL, ॰शसाम् \msCb, ॰शस्तां \Ed}}% 

\nemslokac

{\devanagarifont वस्त्रं तु लोकेष्वभिपूजनीयं }%
  \dontdisplaylinenum    \var{{\devanagarifontvar \numnoemph\vc\textbf{॰भिपूज॰}\lem \msCa\msCb\msNa\msParis\msPaperA, ॰तिपूज॰ \Ed}}% 

%Verse 17:13


\nemslokad

{\devanagarifont वस्त्रं नराणां त्वतिमाननीयम् {॥ १७:१३॥} \veg\dontdisplaylinenum }%
 
\ujvers\nemsloka {
{\devanagarifont वस्त्रं तु भूयो न च मानलाभः }%
  \dontdisplaylinenum}

\nemslokab

{\devanagarifont पराभवश्चातिजुगुप्सनं च  \danda\dontdisplaylinenum }%
     \var{{\devanagarifontvar \numemph\vb\textbf{॰जुगुप्सनं च}\lem \mssALL, 
॰जुप्सितं च \msNa, जुगुप्सनश्च \Ed}}% 
    \lacuna{\devanagarifontsmall \vab {\englishfont missing in \msParis} }%
  
\nemslokac

{\devanagarifont तस्माद्धि वस्त्रं सततं प्रदेयं }%
  \dontdisplaylinenum
%Verse 17:14


\nemslokad

{\devanagarifont यशः श्रियः स्वर्गमनन्तलाभम् {॥ १७:१४॥} \veg\dontdisplaylinenum }%
     \var{{\devanagarifontvar \numnoemph\vd\textbf{श्रियः}\lem \mssALL, श्रियंः \msNa\oo 
\textbf{स्वर्गमनन्त॰}\lem \mssALL, 
॰स्वर्गसमन्त॰ \Ed}}% 

\ujvers\nemsloka {
{\devanagarifont यावन्ति सूत्राणि भवन्ति वस्त्रे }%
  \dontdisplaylinenum}

\nemslokab

{\devanagarifont तावद्युगं गच्छति सोमलोकम्  \danda\dontdisplaylinenum }%
     \var{{\devanagarifontvar \numemph\vb\textbf{तावद्यु॰}\lem \mssALL, ताव यु॰ \msCb\oo 
\textbf{गच्छति}\lem \mssALL, गति \msCaacorr}}% 

\nemslokac

{\devanagarifont पुण्यक्षयाज्जायति मर्त्यलोके }%
  \dontdisplaylinenum    \var{{\devanagarifontvar \numnoemph\vc\textbf{मर्त्य॰}\lem \mssALL, मार्त्य॰ \msNa, मृत्यु॰ \Ed}}% 


\nemslokab

{\devanagarifont वस्त्रप्रभूते धनधान्यकीर्णे  \danda\dontdisplaylinenum }%
     \var{{\devanagarifontvar \numnoemph\vd\textbf{॰कीर्णे}\lem \mssALL, ॰कीर्णो \Ed}}% 

\nemslokae

{\devanagarifont सुरूपसौभाग्ययशस्विनश्च }%
  \dontdisplaylinenum
%Verse 17:15


\nemslokad

{\devanagarifont विद्याधरा लोकप्रभुत्वतां च {॥ १७:१५॥} \veg\dontdisplaylinenum }%
     \var{{\devanagarifontvar \numnoemph\vf\textbf{॰धरा}\lem \mssALL, ॰धरो \Ed\oo 
\textbf{॰त्वतां च}\lem \mssALL, ॰त्वताश्च \Ed}}% 

\ujvers\nemsloka {
{\devanagarifont द्विजेभ्यश्छत्रं सुकृतं प्रदद्याद् }%
  \dontdisplaylinenum}

\nemslokab

{\devanagarifont वर्षातपत्रं दृढशोभनं च  \danda\dontdisplaylinenum }%
 
\nemslokac

{\devanagarifont अङ्गारवर्षं त्रपुखड्गमाद्यम् }%
  \dontdisplaylinenum    \var{{\devanagarifontvar \numemph\vc\textbf{॰वर्षं त्रपु॰}\lem \msNa, ॰वर्षत्रपु॰ \msCa\msCb\msParis\msPaperA, 
॰वर्षत्रषु॰ \Ed\oo 
\textbf{॰ड्गमाद्यम्}\lem \mssALL, ॰ड्गमद्यम् \msCb}}% 
    \paral{{\devanagarifontsmall \vc {\englishfont \compare\ \SDHU\ 7.49cd:}
                 प्रदीप्ताङ्गारवर्षेण दह्यमाना व्रजन्ति च }}

%Verse 17:16


\nemslokad

{\devanagarifont असंशयं त्रायति याम्यमार्गे {॥ १७:१६॥} \veg\dontdisplaylinenum }%
 
\ujvers\nemsloka {
{\devanagarifont स्वर्गं च याति ग्रहनायकस्य }%
  \dontdisplaylinenum}    \var{{\devanagarifontvar \numemph\va\textbf{याति}\lem \mssALL, यान्ति \Ed\oo 
\textbf{॰कस्य}\lem \mssALL, ॰कश्च \Ed}}% 


\nemslokab

{\devanagarifont स वर्षकोट्यायुतमन्तकाले  \danda\dontdisplaylinenum }%
 
\nemslokac

{\devanagarifont जायन्ति ते मानुष मर्त्यलोके }%
  \dontdisplaylinenum    \var{{\devanagarifontvar \numnoemph\vc\textbf{जायन्ति}\lem \mssALL, जान्ति \msCbacorr}}% 

%Verse 17:17


\nemslokad

{\devanagarifont गृहोत्तमे भोगपतिर्भवन्ति {॥ १७:१७॥} \veg\dontdisplaylinenum }%
     \var{{\devanagarifontvar \numnoemph\vd\textbf{॰पतिर्भ॰}\lem \mssALL, 
॰पति भ॰ \msNa\ \unmetr}}% 

\ujvers\nemsloka {
{\devanagarifont कृत्वा मठं शोभन विप्रदाता }%
  \dontdisplaylinenum}

\nemslokab

{\devanagarifont द्रव्येण शुद्धेन तु पूरयित्वा  \danda\dontdisplaylinenum }%
 
\nemslokac

{\devanagarifont स याति देवेन्द्रसदो यथेष्टं }%
  \dontdisplaylinenum    \var{{\devanagarifontvar \numemph\vc\textbf{॰सदो}\lem \mssALL, ॰सदं \Ed}}% 

%Verse 17:18


\nemslokad

{\devanagarifont स वर्षकोटीशत दिव्यसंख्ये {॥ १७:१८॥} \veg\dontdisplaylinenum }%
     \var{{\devanagarifontvar \numnoemph\vd\textbf{॰कोटी॰}\lem \mssALL, ॰कोटि॰ \Ed\ \unmetr\oo 
\textbf{॰संख्ये}\lem \msCa\msCb\msPaperA, ॰संख्यै \msNa, ॰सं\uncl{ख्यं} \msParis, 
॰संख्यैः \Ed}}% 

\ujvers\nemsloka {
{\devanagarifont तदन्तकाले यदि मानुषत्वं }%
  \dontdisplaylinenum}

\nemslokab

{\devanagarifont जायन्ति ते सप्तमहीप्रभोक्ता  \danda\dontdisplaylinenum }%
 
\nemslokac

{\devanagarifont स सप्तरत्नत्रयसम्प्रयुक्तो }%
  \dontdisplaylinenum    \var{{\devanagarifontvar \numemph\vc\textbf{॰रत्न॰}\lem \mssALL, ॰रथ्य॰ \Ed\oo 
\textbf{॰युक्तो}\lem \msCb, ॰युक्ता \msCa\msNa\msParis\msPaperA\Ed}}% 

%Verse 17:19


\nemslokad

{\devanagarifont बलाधिको यज्ञसहस्रकर्ता {॥ १७:१९॥} \veg\dontdisplaylinenum }%
 

\alalfejezet{भूमिप्रदानम्}
\ujvers\nemsloka {
{\devanagarifont भूमिप्रदाता द्विज हीनदीनः }%
  \dontdisplaylinenum}    \var{{\devanagarifontvar \numemph\va\textbf{॰दीनः}\lem \mssALL, ॰दीनं \Ed}}% 


\nemslokab

{\devanagarifont समृद्धसस्यो जलसंनिकृष्टः  \danda\dontdisplaylinenum }%
     \var{{\devanagarifontvar \numnoemph\vb\textbf{समृद्ध॰}\lem \mssALL, 
समृद्धस् \msParis\ \unmetr, संमृद्ध॰ \Ed\oo 
\textbf{॰कृष्टः}\lem \mssALL, ॰कृष्ट \Ed}}% 

\nemslokac

{\devanagarifont स याति लोकममराधिपस्य }%
  \dontdisplaylinenum    \var{{\devanagarifontvar \numnoemph\vc\textbf{याति}\lem \mssALL, जाति \msPaperA}}% 

%Verse 17:20


\nemslokad

{\devanagarifont विमानयानेन मनोहरेण {॥ १७:२०॥} \veg\dontdisplaylinenum }%
 
\ujvers\nemsloka {
{\devanagarifont मन्वन्तरं यावदभुक्तभोगान् }%
  \dontdisplaylinenum}

\nemslokab

{\devanagarifont तदन्तकाले च्युत मर्त्यलोके  \danda\dontdisplaylinenum }%
 
\nemslokac

{\devanagarifont स जम्बुखण्डाधिपतिर्भवेत }%
  \dontdisplaylinenum    \var{{\devanagarifontvar \numemph\vc\textbf{स जम्बुखण्डाधिपतिर्भवेत}\lem \mssALL, 
स जबमुखण्डाधिपतिर्भवेत \Ed\ \hypermetr}}% 

%Verse 17:21


\nemslokad

{\devanagarifont वीर्यान्वितो राजसहस्रनाथः {॥ १७:२१॥} \veg\dontdisplaylinenum }%
     \var{{\devanagarifontvar \numnoemph\vd\textbf{राज॰}\lem \msCa\msCb\msNa\msParis\Ed, रा \msPaperA}}% 


\alalfejezet{गोप्रदानम्}
\ujvers\nemsloka {
{\devanagarifont सचैलघण्टां कनकाग्रशृङ्गां }%
  \dontdisplaylinenum}    \var{{\devanagarifontvar \numemph\va\textbf{॰घण्टां}\lem \msCa\msCb\msNa\msParis\Ed, ॰घण्टा॰ \msPaperA}}% 


\nemslokab

{\devanagarifont दोग्ध्रीं सवत्सां पयसा द्विजानाम्  \danda\dontdisplaylinenum }%
     \var{{\devanagarifontvar \numnoemph\vb\textbf{दोग्ध्रीं}\lem \Ed, द्रोग्ध्रीं \msCa, द्रोग्धी \msCb, दोग्ध्री \msNa, 
द्रोग्ध्री \msParis, दो\lk\ \msPaperA\oo 
\textbf{पयसा द्विजानाम्}\lem \msCa\msPaperA, पयसा \msCb, पयसां द्विजानाम् \msNa\Ed, 
पयसान्वितानां \msParis}}% 
    \paral{{\devanagarifontsmall \vo {\englishfont \compare\ \SDHSAMGR\ 6.89:}
                         हेमशृंगां रौप्यक्षुरां रत्नाङ्गीं कांस्यदोहिनीम्\thinspace{\devanagarifontsmall ।}
                         सचैलघण्टाङ्गान्दद्याच्छिवभक्तद्विजन्मने\thinspace{\devanagarifontsmall ॥} }}

\nemslokac

{\devanagarifont दत्त्वा द्विजेभ्यः समलङ्कृतां गां }%
  \dontdisplaylinenum    \var{{\devanagarifontvar \numnoemph\vc\textbf{दत्त्वा}\lem \mssALL, \om\ \msCb\oo 
\textbf{समलङ्कृतां गां}\lem \msParispcorr, 
समलङ्कृतानाम् \msCa\msCb\msNa\msParisacorr\msPaperA\Ed}}% 

%Verse 17:22


\nemslokad

{\devanagarifont प्रयान्ति लोकं सुरभीसुतानाम् {॥ १७:२२॥} \veg\dontdisplaylinenum }%
     \var{{\devanagarifontvar \numnoemph\vd\textbf{लोकं सुरभीसुतानाम्}\lem \mssALL, \om\ \msCa}}% 

\ujvers\nemsloka {
{\devanagarifont यावन्ति रोमाणि भवन्ति गावस् }%
  \dontdisplaylinenum}    \var{{\devanagarifontvar \numemph\va\textbf{यावन्ति}\lem \mssALL, \om\ \msCa}}% 


\nemslokab

{\devanagarifont तावद्युगानामनुभूय भोगान्  \danda\dontdisplaylinenum }%
     \var{{\devanagarifontvar \numnoemph\vb\textbf{॰गानामनु॰}\lem \mssALL, 
॰गानानु॰ \msParisacorr}}% 

\nemslokac

{\devanagarifont तस्माच्च्युता मर्त्य महीभुजास्ते }%
  \dontdisplaylinenum    \var{{\devanagarifontvar \numnoemph\vc\textbf{॰भुजास्ते}\lem \msCa\msPaperA\Ed, 
॰भुजस्ते \msCb\msParis, ॰भुजानां \msNa}}% 

%Verse 17:23


\nemslokad

{\devanagarifont सहस्रराजानुगतो महात्मा {॥ १७:२३॥} \veg\dontdisplaylinenum }%
 

\alalfejezet{सुवर्णादिप्रदानम्}
\ujvers\nemsloka {
{\devanagarifont सुवर्णकांस्यायसरौप्यदाता }%
  \dontdisplaylinenum}

\nemslokab

{\devanagarifont ताम्रप्रवालान्मणिमौक्तिकाद्यान्  \danda\dontdisplaylinenum }%
     \var{{\devanagarifontvar \numemph\vb\textbf{॰न्मणि॰}\lem \msNa\msParis\msPaperA, ॰त्मणि॰ \msCa\msCb, ॰मणि॰ \Ed\oo 
\textbf{॰मौक्तिका॰}\lem \mssALL, ॰मुक्तका॰ \msParis}}% 

\nemslokac

{\devanagarifont दत्त्वा द्विजेभ्यो वसुसाध्यलोके }%
  \dontdisplaylinenum    \paral{{\devanagarifontsmall \vo {\englishfont \compare\ \NISVMUKH\ 2.56cd:}
                            कांसताम्रप्रवालानि दत्त्वा याति वसोः पुरम;
                    {\englishfont and \SDHSAMGR\ 6.125cd:}        
                            कांस्यताम्रप्रवालानि दत्त्वैति वसुमन्दिरम् }}

%Verse 17:24


\nemslokad

{\devanagarifont प्राप्नोति वर्षं दशपञ्चकोट्यः {॥ १७:२४॥} \veg\dontdisplaylinenum }%
     \var{{\devanagarifontvar \numnoemph\vd\textbf{॰कोट्यः}\lem \corr, ॰कोट्यो \msCa\msCb\msNa\msParis\msPaperA\Ed}}% 

\ujvers\nemsloka {
{\devanagarifont भुक्त्वा यथेष्टं क्रम देवलोकान् }%
  \dontdisplaylinenum}    \var{{\devanagarifontvar \numemph\va\textbf{॰लोकान्}\lem \msCa\msPaperA\Ed, ॰लोकाच् \msCb\msNa, ॰लोकात् \msParis}}% 


\nemslokab

{\devanagarifont च्युतश्च मर्त्ये स भवेन्नरेन्द्रः  \danda\dontdisplaylinenum }%
     \var{{\devanagarifontvar \numnoemph\vb\textbf{च्युतश्च}\lem \mssALL, 
च्युताश्च \msCb, च्युतञ्च \Ed}}% 

\nemslokac

{\devanagarifont सुदुर्जयः शक्रसहस्रजेता }%
  \dontdisplaylinenum    \var{{\devanagarifontvar \numnoemph\vc\textbf{सुदुर्जयः}\lem \mssALL, सुदु\lac\ \msCa}}% 

%Verse 17:25


\nemslokad

{\devanagarifont सुदीर्घमायुश्च पराक्रमश्च {॥ १७:२५॥} \veg\dontdisplaylinenum }%
 

\alalfejezet{विमिश्रविषयाणि}
\ujvers\nemsloka {
{\devanagarifont यत्प्रेक्षणं दर्शयितुं प्रदाता }%
  \dontdisplaylinenum}

\nemslokab

{\devanagarifont सुरूपसौभाग्यफलं लभेत  \danda\dontdisplaylinenum }%
     \var{{\devanagarifontvar \numemph\vb\textbf{लभेत}\lem \msPaperA\Ed, लभेत् \msCa\msCb\msNa\msParis\ \hypometr}}% 

\nemslokac

{\devanagarifont तृणाशनामूलफलाशनेन }%
  \dontdisplaylinenum    \var{{\devanagarifontvar \numnoemph\vc\textbf{तृणा॰}\lem \mssALL, तृण॰ \msPaperA}}% 
    \paral{{\devanagarifontsmall \vab {\englishfont \compare\ \MBH\ 13.7.17b:} स्वर्गगामी तृणाशनः }}

%Verse 17:26


\nemslokad

{\devanagarifont लभेत राज्यानि अकण्टकानि {॥ १७:२६॥} \veg\dontdisplaylinenum }%
     \var{{\devanagarifontvar \numnoemph\vd\textbf{अकण्टकानि}\lem \mssALL, 
कण्टकानि \Ed\ \hypometr}}% 
    \paral{{\devanagarifontsmall \vcd {\englishfont \compare\ \MBH\ 13.7.15c:} फलमूलाशिनां राज्यं }}

\ujvers\nemsloka {
{\devanagarifont लभेत पर्णाशन स्वर्गवासं }%
  \dontdisplaylinenum}    \paral{{\devanagarifontsmall \vab {\englishfont \compare\ \MBH\ 13.7.15d:} स्वर्गः पर्णाशिनां तथा }}


\nemslokab

{\devanagarifont पयःप्रयोगेन च देवलोकम्  \danda\dontdisplaylinenum }%
     \var{{\devanagarifontvar \numemph\vb\textbf{॰लोकम्}\lem \mssALL, ॰लोके \msPaperA\Ed}}% 

\nemslokac

{\devanagarifont शुश्रूषणे यो गुरवे च नित्यं }%
  \dontdisplaylinenum    \var{{\devanagarifontvar \numnoemph\vc\textbf{शुश्रूषणे}\lem \mssALL, शुश्रूषणो \Ed}}% 

%Verse 17:27


\nemslokad

{\devanagarifont विद्याधरो जायति मर्त्यलोके {॥ १७:२७॥} \veg\dontdisplaylinenum }%
 
\ujvers\nemsloka {
{\devanagarifont दद्याद्गवां ग्रास तृणस्य मुष्टिं }%
  \dontdisplaylinenum}    \var{{\devanagarifontvar \numemph\va\textbf{ग्रास॰}\lem \mssALL, धास॰ \Ed\oo 
\textbf{तृणस्य}\lem \mssALL, तृणास्य \msPaperA\oo 
\textbf{मुष्टिं}\lem \mssALL, मुष्टि \msPaperA, मुष्टिः \Ed}}% 
    \paral{{\devanagarifontsmall \vab {\englishfont \compare\ \MBH\ 13.7.17a:} गवाढ्यः शाकदीक्षायां }}


\nemslokab

{\devanagarifont गवाढ्यतां जायति मर्त्यलोके  \danda\dontdisplaylinenum }%
     \var{{\devanagarifontvar \numnoemph\vb\textbf{गवाढ्यतां}\lem \mssALL, सर्वाद्यतां \msPaperA}}% 

\nemslokac

{\devanagarifont श्राद्धं च दत्त्वा प्रयतो द्विजाय }%
  \dontdisplaylinenum
%Verse 17:28


\nemslokad

{\devanagarifont समृद्धसन्तान भवेद्युगान्ते {॥ १७:२८॥} \veg\dontdisplaylinenum }%
 
\ujvers\nemsloka {
{\devanagarifont अहिंसको जायति दीर्घमायुः }%
  \dontdisplaylinenum}    \var{{\devanagarifontvar \numemph\va\textbf{जायति}\lem \mssALL, जाति \msNaacorr}}% 


\nemslokab

{\devanagarifont कुलोत्तमो जायति दीक्षितेन  \danda\dontdisplaylinenum }%
     \var{{\devanagarifontvar \numnoemph\vb\textbf{॰त्तमो}\lem \msCa\msCb\msParis, ॰त्तमे \msNa, ॰हमा \msPaperA, ॰त्तमं \Ed}}% 
    \paral{{\devanagarifontsmall \vb {\englishfont \compare\ \MBH\ 13.7.16d:} दीक्षया कुलमुत्तमम् }}

\nemslokac

{\devanagarifont कालत्रयं स्नानकृतेन राज्यं }%
  \dontdisplaylinenum
%Verse 17:29


\nemslokad

{\devanagarifont पीत्वा च वायुं त्रिदशाधिपत्वम् {॥ १७:२९॥} \veg\dontdisplaylinenum }%
     \var{{\devanagarifontvar \numnoemph\vd\textbf{वायुं त्रि॰}\lem \mssALL, 
वायु\uncl{म्त्रि}॰ \msPaperA, 
वायुस्त्रि॰ \Ed}}% 
    \paral{{\devanagarifontsmall \vcd {\englishfont \compare\ \MBH\ 13.7.17cd:}
                  स्त्रियस्त्रिषवणं स्नात्वा वायुं पीत्वा क्रतुं लभेत् }}

\ujvers\nemsloka {
{\devanagarifont अनश्नतायाः फलमीशलोके }%
  \dontdisplaylinenum}    \var{{\devanagarifontvar \numemph\va\textbf{फलमी॰}\lem \mssALL, फलशी॰ \msNa}}% 
    \paral{{\devanagarifontsmall \va {\englishfont \compare\ \MBH\ 13.7.16ab:}
                  प्रायोपवेशनाद्राज्यं सर्वत्र सुखमुच्यते }}


\nemslokab

{\devanagarifont तृप्तिर्भवेत्तोयप्रदानशीलः  \danda\dontdisplaylinenum }%
     \var{{\devanagarifontvar \numnoemph\vb\textbf{तृप्तिर्भ॰}\lem \mssALL, तृप्ति भ॰ \msNa}}% 

\nemslokac

{\devanagarifont अन्नप्रदाता पुरुषः समृद्धः }%
  \dontdisplaylinenum
%Verse 17:30


\nemslokad

{\devanagarifont स सर्वकामान्लभतीह लोके {॥ १७:३०॥} \veg\dontdisplaylinenum }%
     \var{{\devanagarifontvar \numnoemph\vd\textbf{॰कामान्ल॰}\lem \mssALL, ॰कामा ल॰ \Ed}}% 

\ujvers\nemsloka {
{\devanagarifont श्रद्धामतिर्यः प्रविशेद्धुताशं }%
  \dontdisplaylinenum}    \var{{\devanagarifontvar \numemph\va\textbf{॰द्धुताशं}\lem \msParispcorr, 
॰द्धुताशनं \msCa\msCb\msNa\msParisacorr\msPaperA\Ed\ \hypermetr}}% 


\nemslokab

{\devanagarifont स याति लोकं प्रपितामहस्य  \danda\dontdisplaylinenum }%
     \var{{\devanagarifontvar \numnoemph\vb\textbf{याति}\lem \mssALL, जाति \msPaperA}}% 

\nemslokac

{\devanagarifont सत्यं वदेद्यो ऽपि च धर्मशीलो }%
  \dontdisplaylinenum
%Verse 17:31


\nemslokad

{\devanagarifont मोदत्यसौ देवि सहाप्सरोभिः {॥ १७:३१॥} \veg\dontdisplaylinenum }%
     \paral{{\devanagarifontsmall \vcd {\englishfont \compare\ \MBH\ 13.7.16c:} स्वर्गं सत्येन लभते }}

\ujvers\nemsloka {
{\devanagarifont रसांस्तु षड्ये परिवर्जयन्ति }%
  \dontdisplaylinenum}    \var{{\devanagarifontvar \numemph\va\textbf{रसांस्तु षड्ये}\lem \msCa\msCb\msNa\msPaperA, 
रसास्तु षड्यो \msParis\Ed}}% 


\nemslokab

{\devanagarifont अतीव सौभाग्य लभेत साध्वी  \danda\dontdisplaylinenum }%
     \var{{\devanagarifontvar \numnoemph\vb\textbf{लभेत}\lem \mssALL, लभे \msCaacorr}}% 
    \paral{{\devanagarifontsmall \vab {\englishfont \compare\ \MBH\ 13.7.10ab:}
                  रसानां प्रतिसंहारे सौभाग्यम् अनुगच्छति }}

\nemslokac

{\devanagarifont दानेन भोगानतुलान्लभेत }%
  \dontdisplaylinenum    \var{{\devanagarifontvar \numnoemph\vc\textbf{॰तुलान्ल॰}\lem \mssALL, ॰तुलं ल॰ \Ed}}% 

%Verse 17:32


\nemslokad

{\devanagarifont चिरायुतां याति हि ब्रह्मचर्यात् {॥ १७:३२॥} \veg\dontdisplaylinenum }%
     \paral{{\devanagarifontsmall \vcd {\englishfont \compare\ \MBH\ 13.7.14:}
                  धनं लभेत दानेन मौनेनाज्ञां विशां पते\thinspace{\devanagarifontsmall ।}
                  उपभोगांश्च तपसा ब्रह्मचर्येण जीवितम्\thinspace{\devanagarifontsmall ॥} }}

\ujvers\nemsloka {
{\devanagarifont धनाढ्यतां याति हि पुण्यकर्मा }%
  \dontdisplaylinenum}    \var{{\devanagarifontvar \numemph\va\textbf{धना॰}\lem \mssALL, धन॰ \msNa\oo 
\textbf{याति}\lem \mssALL, यान्ति \Ed\oo 
\textbf{॰कर्मा}\lem \mssALL, ॰कर्मान् \Ed}}% 


\nemslokab

{\devanagarifont मौनेन आज्ञां लभते अलङ्घ्याम्  \danda\dontdisplaylinenum }%
     \var{{\devanagarifontvar \numnoemph\vb\textbf{आज्ञां}\lem \mssALL, आज्ञा \Ed\oo 
\textbf{लभते}\lem \mssALL, लभेत \msCb\ \unmetr}}% 
    \paral{{\devanagarifontsmall \vb {\englishfont \compare\ \MBH\ 13.7.14b:} मौनेनाज्ञां विशां पते }}

\nemslokac

{\devanagarifont प्राप्नोति कामं तपसः सुतप्तं }%
  \dontdisplaylinenum

\nemslokab

{\devanagarifont कीर्तिं यशः स्वर्गमनन्तभोगम्  \danda\dontdisplaylinenum }%
     \var{{\devanagarifontvar \numnoemph\vd\textbf{कीर्तिं य॰}\lem \msCa\msCb\msParis, कीर्तिर्य॰ \msNa\Ed, 
कीर्त्तियशः \msPaperA}}% 

\nemslokae

{\devanagarifont आयुःश्रियारोग्यधनप्रभुत्वं }%
  \dontdisplaylinenum
%Verse 17:33


\nemslokad

{\devanagarifont ज्ञानादिलाभं तपसा लभेत {॥ १७:३३॥} \veg\dontdisplaylinenum }%
     \var{{\devanagarifontvar \numnoemph\vf\textbf{ज्ञानादिलाभं तपसा}\lem \msCa\msNapcorr\msParis\msPaperA\Ed, 
आज्ञादिलाभं तपसा \msCb, \om\ \msNaacorr}}% 

\ujvers\nemsloka {
{\devanagarifont त्रैलोक्याधिपतित्व शक्रमगमत्कृत्वा तपो दुष्करं }%
  \dontdisplaylinenum}    \var{{\devanagarifontvar \numemph\va\textbf{॰गमत्कृ॰}\lem \mssALL, ॰गमं कृ॰ \msCb}}% 


\nemslokab

{\devanagarifont यक्षेशो ऽपि तपःप्रभावमभवद्गुह्याधिपत्यं महत्  \danda\dontdisplaylinenum }%
     \var{{\devanagarifontvar \numnoemph\vb\textbf{॰मभवद्गु॰}\lem \mssALL, ॰गुरुणा गु॰ \Ed\oo 
\textbf{॰पत्यं}\lem \msCb\msNa\msParis, ॰पत्वं \msCa\msPaperA\Ed}}% 

\nemslokac

{\devanagarifont रक्षेशो ऽपि विभीषणस्त्वमरतां प्राप्तस्तपसैव तु }%
  \dontdisplaylinenum    \var{{\devanagarifontvar \numnoemph\vc\textbf{॰पसैव}\lem \mssALL, ॰पस्यैव \msPaperA\Ed\oo 
\textbf{तु}\lem \mssALL, तुं \msParis\msPaperA}}% 

%Verse 17:34


\nemslokad

{\devanagarifont रुद्राराधनतत्परस्तपफलान्नन्दी गणत्वं गतः {॥ १७:३४॥} \veg\dontdisplaylinenum }%
     \var{{\devanagarifontvar \numnoemph\vd\textbf{॰तत्पर॰}\lem \mssALL, ॰तत्परा॰ \msPaperA\Ed\oo 
\textbf{॰पफलान्न॰}\lem \corr, ॰पःफलान्न॰ \msCa\msNa\msParis\msPaperA\unmetr, 
॰पःफलन्न॰ \msCb, 
॰पफलात् न \Ed}}% 

\ujvers\nemsloka {
{\devanagarifont ज्ञानं द्विजानां तप आह विष्णुः }%
  \dontdisplaylinenum}    \var{{\devanagarifontvar \numemph\va\textbf{द्विजानां तप}\lem \mssALL, 
द्विजान्तपसो \Ed\unmetr}}% 


\nemslokab

{\devanagarifont क्षत्रं तपो रक्षणमाह सूर्यः  \danda\dontdisplaylinenum }%
     \var{{\devanagarifontvar \numnoemph\vb\textbf{क्षत्रं त॰}\lem \mssALL, 
क्षेत्रन्त॰ \msPaperA\oo 
\textbf{सूर्यः}\lem \msCb\msParis\msPaperA, सूर्य \msCa\msNa\Ed}}% 

\nemslokac

{\devanagarifont वैश्यं तपश्चार्जनमाह वायुः }%
  \dontdisplaylinenum    \var{{\devanagarifontvar \numnoemph\vc\textbf{॰र्जनमाह वायुः}\lem \msCa\msParis\msPaperA, 
॰र्जनमाहमायुः \msCb, 
॰र्जानमाह वायुः \msNa, ॰ञ्जनमाह वायुः \Ed}}% 

%Verse 17:35


\nemslokad

{\devanagarifont शूद्रं हि शिल्पं तप आह इन्द्रः {॥ १७:३५॥} \veg\dontdisplaylinenum }%
     \var{{\devanagarifontvar \numnoemph\vd\textbf{इन्द्रः}\lem \mssALL, इन्दुः \msCb}}% 

\ujvers\nemsloka {
{\devanagarifont रणोत्सहं क्षत्रिययज्ञमिष्टं }%
  \dontdisplaylinenum}

\nemslokab

{\devanagarifont वैश्ये हविर्यज्ञमुदाहरन्ति  \danda\dontdisplaylinenum }%
     \var{{\devanagarifontvar \numemph\vb\textbf{वैश्ये हविर्य॰}\lem \msCa\msCb\msParis, 
वैश्यं हवि य॰ \msNa, वैश्यं हविर्य्य॰ \msPaperA\Ed}}% 

\nemslokac

{\devanagarifont शूद्रस्य यज्ञः परिचर्यमिष्टं }%
  \dontdisplaylinenum
%Verse 17:36


\nemslokad

{\devanagarifont यज्ञं द्विजानां जपमुक्त†मोक्षः† {॥ १७:३६॥} \veg\dontdisplaylinenum }%
 

\alalfejezet{स्वमांसरुधिरपुत्रकलत्रदानम्}
\vers


{\devanagarifont देव्युवाच {\dandab}\dontdisplaylinenum  }%
 
{\devanagarifont स्वमांसरुधिरं दानं दानं पुत्रकलत्रयोः \thinspace{\danda} \dontdisplaylinenum }%
 
%Verse 17:37

{\devanagarifont किं प्रशस्यं महादेव श्रोतुमिच्छामि तत्त्वतः {॥ १७:३७॥} \veg\dontdisplaylinenum }%
     \var{{\devanagarifontvar \numemph\vd\textbf{श्रोतुमिच्छामि तत्त्वतः}\lem \mssALL, 
तत्त्वं वक्तुमिहार्हसि \Ed}}% 
    \lacuna{\devanagarifontsmall \vo {\englishfont \Ed\ breaks down after 17.37, and resumes only at 18.16c.} }%
  
{\devanagarifont महेश्वर उवाच {\dandab}\dontdisplaylinenum  }%
     \var{{\devanagarifontvar \numemph\vo\textbf{महेश्वर}\lem \msCa\msCb\msNa\msParis, भगवान् \msPaperA}}% 

{\devanagarifont स्वमांसरुधिरं दानं प्रशंसन्ति मनीषिणः \thinspace{\danda} \dontdisplaylinenum }%
 
%Verse 17:38

{\devanagarifont श्रूयतां पूर्ववृत्तानि संक्षिप्य कथयाम्यहम् {॥ १७:३८॥} \veg\dontdisplaylinenum }%
 
{\devanagarifont उशीनरस्तु राजर्षिः कपोतार्थे स्वकां तनुम् \thinspace{\dandab} \dontdisplaylinenum }%
     \var{{\devanagarifontvar \numemph\va\textbf{॰नरस्तु}\lem \mssALL, ॰नरस्य \msPaperA\oo 
\textbf{राजर्षिः}\lem \mssALL, 
राजर्षि \msCb, रार्षिः \msParisacorr}}% 

%Verse 17:39

{\devanagarifont त्यक्त्वा स्वर्गमनुप्राप्तः परार्थे परतत्परः {॥ १७:३९॥} \veg\dontdisplaylinenum }%
 
{\devanagarifont पुत्रमांसं स्वयं छित्त्वा अग्निदत्तं पुरानघे \thinspace{\dandab} \dontdisplaylinenum }%
     \var{{\devanagarifontvar \numemph\vb\textbf{अग्निदत्तं}\lem \msCa\msNa\msParis, 
अग्निर्दत्तं \msCb, अग्निं दत्तं \msPaperA\oo 
\textbf{॰नघे}\lem \msCa\msCb\msNa\msParis, ॰\uncl{नघा} \msPaperA}}% 

%Verse 17:40

{\devanagarifont तेन दानप्रभावेन अलर्कस्त्रिदिवं गतः {॥ १७:४०॥} \veg\dontdisplaylinenum }%
     \var{{\devanagarifontvar \numnoemph\vd\textbf{अलर्क॰}\lem \mssALL, अर्लर्क॰ \msParis}}% 

\ujvers\nemsloka {
{\devanagarifont स्वदारदानेन सुदासपुत्र }%
  \dontdisplaylinenum}

\nemslokab

{\devanagarifont अपुत्रभूतस्य च पुत्र जातः  \danda\dontdisplaylinenum }%
 
\nemslokac

{\devanagarifont स्वर्गे स्वयं चाक्षयभोगलाभं }%
  \dontdisplaylinenum    \var{{\devanagarifontvar \numemph\vc\textbf{चाक्षय॰}\lem \mssALL, चोक्वय॰ \msCa}}% 

%Verse 17:41


\nemslokad

{\devanagarifont प्राप्तो महद्दानफलप्रभावात् {॥ १७:४१॥} \veg\dontdisplaylinenum }%
 
\vers


{\devanagarifont यादवश्चार्जुनो देवि दत्त्वा खाण्डवभोजनम् \thinspace{\dandab} \dontdisplaylinenum }%
     \var{{\devanagarifontvar \numemph\va\textbf{यादव॰}\lem \mssALL, याद्रव॰ \msNa}}% 
    \var{{\devanagarifontvar \numnoemph\vb\textbf{खाण्डव॰}\lem \msPaperA, खण्डव॰ \msCa\msCb\msNa\msParis\oo 
\textbf{॰भोजनम्}\lem \conjKafle, ॰भाजनं \msCa\msCb\msNa\msParis\msPaperA}}% 

%Verse 17:42

{\devanagarifont तपनस्य प्रसादेन सप्तद्वीपेश्वरो भवेत् {॥ १७:४२॥} \veg\dontdisplaylinenum }%
     \var{{\devanagarifontvar \numnoemph\vc\textbf{प्रसादेन}\lem \msCa\msNa\msPaperA, प्रभावेन \msCb, प्रतापेन \msParis}}% 

{\devanagarifont हरिणा च शिरां भित्त्वा दत्तं मे रुधिरं पुरा \thinspace{\dandab} \dontdisplaylinenum }%
     \var{{\devanagarifontvar \numemph\va\textbf{शिरां}\lem \emeYokochi, शिरो \msCa\msCb\msNa\msParis\msPaperA\Ed}}% 
    \paral{{\devanagarifontsmall \vab {\englishfont \compare\ \SKANDAP\ 6.5ab:}
                 शिरां ललाटात्सम्भिद्य रक्तधारामपातयत् }}

%Verse 17:43

{\devanagarifont प्रतीच्छितं कपालेन ब्रह्मसम्भवजेन मे {॥ १७:४३॥} \veg\dontdisplaylinenum }%
     \var{{\devanagarifontvar \numnoemph\vd\textbf{॰सम्भव॰}\lem \mssALL, ॰स्वम्भव॰ \msParis}}% 

{\devanagarifont दिव्यवर्षसहस्राणि धारा तस्य न छिद्यते \thinspace{\dandab} \dontdisplaylinenum }%
     \var{{\devanagarifontvar \numemph\va\textbf{दिव्य॰}\lem \msCa\msNa\msPaperA, दिव्यं \msCb\msParis}}% 
    \var{{\devanagarifontvar \numnoemph\vb\textbf{छिद्यते}\lem \msCa\msCb\msNa\msParis, विद्यते \msPaperA}}% 

%Verse 17:44

{\devanagarifont परितुष्टो ऽस्मि तेनाहं कर्मणानेन सुन्दरि {॥ १७:४४॥} \veg\dontdisplaylinenum }%
     \paral{{\devanagarifontsmall \vcd {\englishfont \compare\ \SKANDAP\ 6.8ab:}
                 तुष्टो ऽस्मि तव दानेन युक्तेनानेन मानद }}

{\devanagarifont वरं दत्तं मया देवि पुराणपुरुषो ऽव्ययः \thinspace{\dandab} \dontdisplaylinenum }%
     \paral{{\devanagarifontsmall \vcd {\englishfont \compare\ \SKANDAP\ 6.8cd:}
                 वरं वरय भद्रं ते वरदो ऽस्मि तवाद्य वै }}

%Verse 17:45

{\devanagarifont अक्षयं बलमूर्जं च अजरामरमेव च {॥ १७:४५॥} \veg\dontdisplaylinenum }%
 
{\devanagarifont ममाधिकं भवेद्विष्णुर्मामपि त्वं विजेष्यसि \thinspace{\dandab} \dontdisplaylinenum }%
     \var{{\devanagarifontvar \numemph\vab\textbf{॰ष्णुर्मा॰ }\lem \msCa\msNa\msParis\msPaperA, ॰ष्णु मा॰ \msCb}}% 
    \var{{\devanagarifontvar \numnoemph\vb\textbf{त्वं}\lem \msCa\msCb\msPaperA, \om\ \msNa, त्व \msParis}}% 

%Verse 17:46

{\devanagarifont एवमादीन्यनेकानि मयोक्तानि जनार्दने {॥ १७:४६॥} \veg\dontdisplaylinenum }%
 
{\devanagarifont निष्कम्पनिश्चलमनाः स्थाणुभूत इव स्थितः \thinspace{\dandab} \dontdisplaylinenum }%
     \var{{\devanagarifontvar \numemph\va\textbf{॰मनाः}\lem \eme, ॰मनः \msCa\msCb\msNa\msParis\msPaperA}}% 
    \var{{\devanagarifontvar \numnoemph\vb\textbf{स्थाणु॰}\lem \mssALL, स्थानु॰ \msParis\msPaperA}}% 

{\devanagarifont दधीचिः स्वतनुं दत्त्वा विबुधानां वरानने  \danda\dontdisplaylinenum }%
     \var{{\devanagarifontvar \numnoemph\vc\textbf{दधीचिः}\lem \msCa\msParis, दधचि \msCb, दधीचि \msNa\msPaperA}}% 
    \var{{\devanagarifontvar \numnoemph\vd\textbf{विबुधानां}\lem \mssALL, विधूनानां \msNa}}% 

%Verse 17:47

{\devanagarifont भुक्त्वा लोकान् क्रमात्सर्वान् शिवलोके प्रतिष्ठितः {॥ १७:४७॥} \veg\dontdisplaylinenum }%
 

\alalfejezet{अन्यानि दानानि}
{\devanagarifont जामदग्निर्महीं दत्त्वा काश्यपाय महात्मने \thinspace{\dandab} \dontdisplaylinenum }%
 
%Verse 17:48

{\devanagarifont इहैव स फलं भोक्ता देवराज्यमवाप्स्यति {॥ १७:४८॥} \veg\dontdisplaylinenum }%
 
{\devanagarifont दत्त्वा गोसकलं देवि व्यासस्यामिततेजसः \thinspace{\dandab} \dontdisplaylinenum }%
     \var{{\devanagarifontvar \numemph\va\textbf{सकलं}\lem \mssALL, सफलं \msParis}}% 
    \var{{\devanagarifontvar \numnoemph\vb\textbf{॰तेजसः}\lem \mssALL, तेतजसः \msCb}}% 

%Verse 17:49

{\devanagarifont युधिष्ठिरो महीपालः सदेहस्त्रिदिवं गतः {॥ १७:४९॥} \veg\dontdisplaylinenum }%
     \var{{\devanagarifontvar \numnoemph\vc\textbf{॰ष्ठिरो}\lem \msPaperA, ॰ष्ठिर॰ \msCa\msCb\msNa\msParis\oo 
\textbf{॰पालः}\lem \mssALL, ॰या\lk\ \msCa}}% 
\versno=49

{\devanagarifont सत्यभामा स्वकं भर्त्रा दत्त्वा नारदसत्कृतम् \thinspace{\dandab} \dontdisplaylinenum }%
     \var{{\devanagarifontvar \numemph\va\textbf{॰भामा}\lem \eme, ॰भामः \msCa\msCb\msParis\msPaperA, ॰भासः \msNa\oo 
\textbf{भर्त्रा}\lem \eme, भर्ता \msCa\msCb\msNa\msParis\msPaperA}}% 
    \var{{\devanagarifontvar \numnoemph\vb\textbf{नारद॰}\lem \mssALL, नार॰ \msCb}}% 

%Verse 17:50

{\devanagarifont दानस्यास्य प्रभावेन अक्षयं त्रिदिवं गतः {॥ १७:५०॥} \veg\dontdisplaylinenum }%
     \var{{\devanagarifontvar \numnoemph\vc\textbf{दानस्यास्य}\lem \mssALL, दानास्यास्य \msPaperA}}% 

{\devanagarifont चतुःषष्टिसहस्राणि गवां दत्त्वा द्विजन्मने \thinspace{\dandab} \dontdisplaylinenum }%
     \var{{\devanagarifontvar \numemph\vb\textbf{दत्त्वा}\lem \mssALL, दत्वां \msNa}}% 
    \lacuna{\devanagarifontsmall \vo {\englishfont The folios that may have contained 17.51--18.20ab are missing in \msParis.} }%
  
%Verse 17:51

{\devanagarifont दुर्योधनो महीपालो गतः स्वर्गमनन्तकम् {॥ १७:५१॥} \veg\dontdisplaylinenum }%
     \var{{\devanagarifontvar \numnoemph\vc\textbf{दुर्योधनो}\lem \msPaperA, दुर्योधन॰ \msCa\msCb\msNa}}% 

{\devanagarifont वासुकिः सर्पराजेन्द्रो दत्त्वा विप्र सुसंस्कृताम् \thinspace{\dandab} \dontdisplaylinenum }%
     \var{{\devanagarifontvar \numemph\vb\textbf{॰स्कृताम्}\lem \emeKafle, ॰स्कृतम् \msCa\msCb\msNa\msPaperA}}% 

%Verse 17:52

{\devanagarifont जरत्कारोश्च सा भार्या सर्वे नाग विमोक्षिताः {॥ १७:५२॥} \veg\dontdisplaylinenum }%
     \var{{\devanagarifontvar \numnoemph\vc\textbf{जरत्कारो॰}\lem \eme, जरत्कारु॰ \msCb\msNa\msPaperA, \lk रत्कारु॰ \msCa}}% 
    \var{{\devanagarifontvar \numnoemph\vd\textbf{नाग}\lem \msCa\msCb\msNa, नग \msPaperA\oo 
\textbf{॰मोक्षिताः}\lem \msCa\msNa\msPaperA, ॰मोहिताः \msCb}}% 


\alalfejezet{दानभूमयः}
{\devanagarifont गोभूमिकनकादीनां दानं कन्यसमुच्यते \thinspace{\dandab} \dontdisplaylinenum }%
     \var{{\devanagarifontvar \numemph\vb\textbf{दानं}\lem \msCa\msNa\msPaperA, दान \msCb}}% 

%Verse 17:53

{\devanagarifont भृत्यपुत्रकलत्राणां दानं मध्यममुच्यते {॥ १७:५३॥} \veg\dontdisplaylinenum }%
     \var{{\devanagarifontvar \numnoemph\vd\textbf{मध्यममु॰}\lem \msCa\msCb\msNa, मध्यम उ॰ \msPaperA}}% 

{\devanagarifont स्वदेहपिशितादीनां दानमुत्तममुच्यते \thinspace{\dandab} \dontdisplaylinenum }%
     \var{{\devanagarifontvar \numemph\va\textbf{॰देह॰}\lem \msPaperA, ॰देहं \msCa\msCb\msNa}}% 

%Verse 17:54

{\devanagarifont एतत्सर्वं यदा दानं तद्दानमुत्तमोत्तमम् {॥ १७:५४॥} \veg\dontdisplaylinenum }%
     \var{{\devanagarifontvar \numnoemph\vd\textbf{॰त्तमम्}\lem \msCapcorr\msCb\msNa, ॰त्त \msCaacorr, ॰त्तम \msPaperA}}% 

{\devanagarifont यावज्जन्मसहस्राणि भोक्ता भवति कन्यसः \thinspace{\dandab} \dontdisplaylinenum }%
 
%Verse 17:55

{\devanagarifont शतजन्मसहस्राणि भोक्ता भवति मध्यमः {॥ १७:५५॥} \veg\dontdisplaylinenum }%
 
{\devanagarifont उत्तमः फलभोक्ता च जन्मकोटिशतत्रयम् \thinspace{\dandab} \dontdisplaylinenum  }%
     \var{{\devanagarifontvar \numemph\va\textbf{च}\lem \msCapcorr\msCb\msNa\msPaperA, वि \msCaacorr}}% 

%Verse 17:56

{\devanagarifont परार्धद्वयजन्मानां भोक्ता वै चोत्तमोत्तमः {॥ १७:५६॥} \veg\dontdisplaylinenum }%
 
\ujvers\nemsloka {
{\devanagarifont भूतानामनुकम्पया यदि धनं दाता सदा त्वर्थिने }%
  \dontdisplaylinenum}    \var{{\devanagarifontvar \numemph\va\textbf{॰कम्पया}\lem \msCa\msPaperA, ॰कम्पाया \msCb\msNa}}% 


\nemslokab

{\devanagarifont दीनान्धकृपणेष्वनाथमलिने श्वानादितिर्यक्षु च  \danda\dontdisplaylinenum }%
     \var{{\devanagarifontvar \numnoemph\vb\textbf{॰र्यक्षु}\lem \mssALL, ॰\uncl{र्यक्षु} \msCa}}% 

\nemslokac

{\devanagarifont यद्येवं कुरुते सदार्तिहरणं श्रद्धान्वितो भक्तिमान् }%
  \dontdisplaylinenum
%Verse 17:57


\nemslokad

{\devanagarifont तस्यानन्तफलं वदन्ति विबुधाः संयम्य संदर्शनात् {॥ १७:५७॥} \veg\dontdisplaylinenum }%
 
\vers


{\devanagarifont 
\jump
\begin{center}
\ketdanda~इति वृषसारसंग्रहे दानधर्मविशेषं नाम सप्तादशमो ऽध्यायः~\ketdanda
\end{center}
\dontdisplaylinenum\vers  }%
 \bekveg\szamveg
\vfill
\phpspagebreak

\versno=0\fejno=18
\thispagestyle{empty}

\centerline{\Large\devanagarifontbold [   अष्टादशमो ऽध्यायः  ]}{\vrule depth10pt width0pt} \fancyhead[CE]{{\footnotesize\devanagarifont वृषसारसंग्रहे  }}
\fancyhead[CO]{{\footnotesize\devanagarifont अष्टादशमो ऽध्यायः  }}
\fancyhead[LE]{}
\fancyhead[RE]{}
\fancyhead[LO]{}
\fancyhead[RO]{}
\szam\bek



\alalfejezet{स्वर्गान्मर्त्यमुपागतानां चिह्नानि}
\vers


{\devanagarifont देव्युवाच {\dandab}\dontdisplaylinenum  }%
     \lacuna{\devanagarifontsmall {\englishfont Witnesses used for this chapter---\msCa: f.~224r line 4 -- f.~226r line 4; 
                              \msCb: f.~226v line 6 -- f.~228r line 6;
                              \msCc: f.~306r line 1 -- f.~306v line 5;
                              \msNa: f.~31v line 1 --  f.~33r line 6;
                              \msM:  f.~33v line 2 -- f.~35v line 4;
                              \msPaperA: f.~231r line 2 -- f.~232v line 7;
                              \Ed: pp.~649--651} }%
  
\nemsloka 
{\devanagarifont भुक्त्वा तु भोगान्सुचिरं यथेष्टं }%
  \dontdisplaylinenum    \lacuna{\devanagarifontsmall \vo {\englishfont \msCc\ broke off in chapter 14 and resumes below at 18.28b.}
                    {\englishfont The gap in \Ed\ that started at 17.39 continues up to 18.16c.} }%
      \var{{\devanagarifontvar \numemph\va\textbf{तु}\lem \mssALL, \om\ \msNa, च \msM}}% 


\nemslokab

{\devanagarifont पुण्यक्षयान्मर्त्यमुपागतानाम्  \danda\dontdisplaylinenum }%
     \var{{\devanagarifontvar \numnoemph\vb\textbf{॰क्षयान्म॰}\lem \mssALL, ॰क्षया म॰ \msM}}% 

\nemslokac

{\devanagarifont चिह्नानि तेषां कथयस्व मे ऽद्य }%
  \dontdisplaylinenum    \var{{\devanagarifontvar \numnoemph\vc\textbf{चिह्नानि}\lem \mssALL, किञ्चिह्न \msM}}% 

%Verse 18:1


\nemslokad

{\devanagarifont यथाक्रमं कर्मफलं विशेषात् {॥ १८:१॥} \veg\dontdisplaylinenum }%
     \var{{\devanagarifontvar \numnoemph\vd\textbf{विशेषात्}\lem \mssALL, विशेषः \msM}}% 


\alalfejezet{दानाष्टकम्}
\vers


{\devanagarifont महेश्वर उवाच {\dandab}\dontdisplaylinenum  }%
     \var{{\devanagarifontvar \numemph\vo\textbf{महेश्वर}\lem \msCa\msCb, भगवान् \msNa\msM\msPaperA}}% 

\nemsloka 
{\devanagarifont सदान्नदाता कृपणार्तिदीनां }%
  \dontdisplaylinenum    \var{{\devanagarifontvar \numnoemph\va\textbf{॰दीनां}\lem \mssALL, ॰दीना \msM}}% 


\nemslokab

{\devanagarifont स वर्षकोट्यायुतमीशलोके  \danda\dontdisplaylinenum }%
     \var{{\devanagarifontvar \numnoemph\vb\textbf{॰युतमीशलोके}\lem \mssALL, 
॰युतमीनशलोके \msCaacorr, ॰युतस्वर्गलोके \msM}}% 

\nemslokac

{\devanagarifont भुक्त्वा च भोगान्सममप्सरोभिः }%
  \dontdisplaylinenum    \var{{\devanagarifontvar \numnoemph\vc\textbf{॰न्सममप्सरोभिः}\lem \mssALL, ॰न्सहरप्सरोभिः \msM, ॰न्सरोभिः \msPaperA}}% 

%Verse 18:2


\nemslokad

{\devanagarifont प्रक्षीणपुण्यः पुनरेति मर्त्यम् {॥ १८:२॥} \veg\dontdisplaylinenum }%
     \var{{\devanagarifontvar \numnoemph\vd\textbf{॰पुण्यः पुनरेति मर्त्यम्}\lem \mssALL, ॰पुण्यः पुनरेति मर्त्ये \msNa, 
॰पुण्य पुनः मर्त्यलोके \msM}}% 

\ujvers\nemsloka {
{\devanagarifont जायन्ति दिव्येषु कुलेषु पुंसः }%
  \dontdisplaylinenum}    \var{{\devanagarifontvar \numemph\va\textbf{जायन्ति दिव्येषु कुलेषु पुंसः}\lem \msCa\msNa, 
\lac\ \msCb, 
जायन्ति ते दिव्यकुलेषु पुन्साम् \msM, 
जायन्ति ते दिव्ये कुलेषु पुंसः \msPaperA\ \unmetr}}% 


\nemslokab

{\devanagarifont सस्त्रीसमृद्धे बहुभृत्यपूर्णे  \danda\dontdisplaylinenum }%
 
\nemslokac

{\devanagarifont गौरश्वरत्नादिधनाकुलेषु }%
  \dontdisplaylinenum    \var{{\devanagarifontvar \numnoemph\vc\textbf{गौर॰}\lem \msCapcorr\msNa\msPaperA, गौरव॰ \msCaacorr, गोर॰ \msCb\msM\oo 
\textbf{॰रत्नादि॰}\lem \msNa\msM, ॰रन्नादि॰ \msCa\msPaperA, ॰रत्नाति॰ \msCb\oo 
\textbf{॰धना॰}\lem \mssALL, ॰समा॰ \msM}}% 

%Verse 18:3


\nemslokad

{\devanagarifont रूपोज्ज्वलः कान्तिसमायुतश्च {॥ १८:३॥} \veg\dontdisplaylinenum  }%
     \var{{\devanagarifontvar \numnoemph\vd\textbf{रूपोज्ज्वलः}\lem \eme, रूपोज्ज्वल॰ \msCa\msCb\msNa, रूपोज्वलः \msM, रूपर्ज्वल॰ \msPaperA\oo 
\textbf{॰समायुतश्च}\lem \msCapcorr\msNa, ॰समायुतञ्च \msCaacorr\msCb\msM}}% 

\ujvers\nemsloka {
{\devanagarifont वस्त्रं सुसत्कृत्य द्विजस्य दानात् }%
  \dontdisplaylinenum}    \var{{\devanagarifontvar \numemph\va\textbf{वस्त्रं सुसत्कृत्य}\lem \msCa\msCb, वस्त्रं सुसंस्कृत्य \msNa, 
सुवस्त्र सत्कृत्य \msM, 
वस्त्रं सुसंकृत्य \msPaperA}}% 


\nemslokab

{\devanagarifont स्वर्गेषु मोदन्ति स वर्षकोट्यः  \danda\dontdisplaylinenum }%
     \var{{\devanagarifontvar \numnoemph\vb\textbf{॰कोट्यः}\lem \mssALL, ॰कोट्या \msM}}% 

\nemslokac

{\devanagarifont पुनश्च ते मर्त्यमुपागताश्च }%
  \dontdisplaylinenum    \var{{\devanagarifontvar \numnoemph\vc\textbf{पुनश्च ते मर्त्यमुपागताश्च}\lem \msCa\msCb\msNa\msPaperA, 
पुनश्च्युता मर्त्यमुपा$\-$गतानां \msM}}% 

%Verse 18:4


\nemslokad

{\devanagarifont चिह्नं महच्छ्रीपदमाप्नुवन्ति {॥ १८:४॥} \veg\dontdisplaylinenum }%
     \var{{\devanagarifontvar \numnoemph\vd\textbf{चिह्नं म॰}\lem \mssALL, चिह्न\uncl{म्म}॰ \msCa, चिह्नं न॰ \msM}}% 

\ujvers\nemsloka {
{\devanagarifont कूपप्रपापुष्करिणीप्रदाता }%
  \dontdisplaylinenum}    \var{{\devanagarifontvar \numemph\va\textbf{कूप॰}\lem \mssALL, कूपं \msM\oo 
\textbf{॰पुष्करिणी॰}\lem \msNa\msPaperA, ॰पुष्करणी॰ \msCa, ॰पुष्किरणी॰ \msCb, ॰पुष्किरिणी॰ \msM}}% 


\nemslokab

{\devanagarifont स लोकमाप्नोति जलेश्वरस्य  \danda\dontdisplaylinenum }%
     \var{{\devanagarifontvar \numnoemph\vb\textbf{जलेश्वरस्य}\lem \msCa\msCb\msNa\msPaperA, जलेःस्वरस्य \msM}}% 

\nemslokac

{\devanagarifont ततः स तस्माच्च्युतिमाप्य लोकात् }%
  \dontdisplaylinenum    \var{{\devanagarifontvar \numnoemph\vc\textbf{ततः स तस्माच्च्युतिमाप्य लोकात्}\lem \msCa\msPaperA, 
ततः स तस्माच्च्युतिमप्य लोकात् \msCb, 
ततः स तस्माच्च्युतिमाप्य लोके \msNa, 
तस्मात्स लोकाश्च्युत मर्त्यलोके \msM}}% 

%Verse 18:5


\nemslokad

{\devanagarifont सुखी सुतृप्तेषु कुलेषु जायेत् {॥ १८:५॥} \veg\dontdisplaylinenum }%
     \var{{\devanagarifontvar \numnoemph\vd\textbf{॰तृप्तेषु}\lem \mssALL, ॰तृ\lac\ \msCb, ॰तप्तेषु \msNa\oo 
\textbf{कुलेषु जायेत्}\lem \mssALL, च जायते सः \msM}}% 

\ujvers\nemsloka {
{\devanagarifont रत्निप्रमाणादपि हेमदानात् }%
  \dontdisplaylinenum}    \var{{\devanagarifontvar \numemph\va\textbf{रत्नि॰}\lem \msCa\msNa, रति॰ \msCb, रत्ति॰ \msM, रत्न॰ \msPaperA\oo 
\textbf{॰प्रमाणा॰}\lem \mssALL, ॰प्रमाना॰ \msM\oo 
\textbf{॰दानात्}\lem \mssALL, ॰दाता \msM}}% 


\nemslokab

{\devanagarifont सुरेन्द्रलोकं समवाप्नुवन्ति  \danda\dontdisplaylinenum }%
     \var{{\devanagarifontvar \numnoemph\vb\textbf{समवाप्नुवन्ति}\lem \mssALL, समुवाप्नुवन्ति \msCb, चमवाप्नुवन्ति \msM}}% 

\nemslokac

{\devanagarifont तस्माच्च्युतो मर्त्यमुपागतानां }%
  \dontdisplaylinenum    \var{{\devanagarifontvar \numnoemph\vc\textbf{तस्माच्च्युतो}\lem \mssALL, तस्माच्च्युते \msM}}% 

%Verse 18:6


\nemslokad

{\devanagarifont चिह्नं समृद्धिर्धनधान्यलक्ष्म्याः {॥ १८:६॥} \veg\dontdisplaylinenum }%
     \var{{\devanagarifontvar \numnoemph\vd\textbf{चिह्नं}\lem \mssALL, चिह्न \msCa\oo 
\textbf{॰लक्ष्म्याः}\lem \msNa, लक्ष्याः \msCa\msM, लक्ष्मः \msCb, ल\uncl{क्ष्या} \msPaperA\oo 
\textbf{समृद्धिर्ध॰}\lem \mssALL, समृद्ध्यध॰ \msM}}% 

\ujvers\nemsloka {
{\devanagarifont अदूष्यभूमीवरविप्रदानात् }%
  \dontdisplaylinenum}    \var{{\devanagarifontvar \numemph\va\textbf{॰दानात्}\lem \mssALL, ॰दाता \msM}}% 


\nemslokab

{\devanagarifont स लोकमाप्नोति सुरेश्वरस्य  \danda\dontdisplaylinenum }%
     \var{{\devanagarifontvar \numnoemph\vb\textbf{लोकमाप्नोति}\lem \mssALL, लोक प्राप्नोति \msPaperA\oo 
\textbf{सुरे॰}\lem \mssALL, स्वरे॰ \msM}}% 

\nemslokac

{\devanagarifont भुक्त्वा तु भोगान्च्युत मर्त्यलोके }%
  \dontdisplaylinenum    \var{{\devanagarifontvar \numnoemph\vc\textbf{भुक्त्वा भोगान्च्युत}\lem \msCa\msCb\msNa, 
स भुक्तभोगां च्युत \msM, 
भुक्त्वा भोगानच्युत \msPaperA}}% 

%Verse 18:7


\nemslokad

{\devanagarifont चिह्नं लभेद्वै विषयाधिपत्वम् {॥ १८:७॥} \veg\dontdisplaylinenum }%
     \var{{\devanagarifontvar \numnoemph\vd\textbf{लभेद्वै}\lem \msCa\msPaperA, लभेद्वैद् \msCb, भवेद्वै \msNa, भवेतद् \msM}}% 

\ujvers\nemsloka {
{\devanagarifont द्विजस्य सत्कृत्य तिलप्रदाता }%
  \dontdisplaylinenum}    \var{{\devanagarifontvar \numemph\va\textbf{तिल॰}\lem \mssALL, तिलः \msM}}% 


\nemslokab

{\devanagarifont स लोकमाप्नोति च केशवस्य  \danda\dontdisplaylinenum }%
     \var{{\devanagarifontvar \numnoemph\vb\textbf{केशवस्य}\lem \mssALL, वासवेभ्यः \msM}}% 

\nemslokac

{\devanagarifont भ्रष्टस्ततो मर्त्यमुपागतस्तु }%
  \dontdisplaylinenum    \var{{\devanagarifontvar \numnoemph\vc\textbf{॰गतस्तु}\lem \mssALL, ॰गतस्य \msM}}% 

%Verse 18:8


\nemslokad

{\devanagarifont चिह्नं लभेदक्षयमर्थलाभम् {॥ १८:८॥} \veg\dontdisplaylinenum }%
     \var{{\devanagarifontvar \numnoemph\vd\textbf{लभेद॰}\lem \mssALL, भवेद॰ \msNa, नृणाञ्चो \msM}}% 

\ujvers\nemsloka {
{\devanagarifont गवां सुरूपां विधिवद्द्विजानां }%
  \dontdisplaylinenum}    \var{{\devanagarifontvar \numemph\va\textbf{गवां सुरूपां}\lem \msPaperA, गवां स्वरूपां \msCa\msNa, गवां स्वरूपं \msCb, गवा सुरूपा \msM}}% 


\nemslokab

{\devanagarifont दत्त्वा च गोलोकमवाप्नुवन्ति  \danda\dontdisplaylinenum }%
     \var{{\devanagarifontvar \numnoemph\vb\textbf{च}\lem \mssALL, स \msM}}% 

\nemslokac

{\devanagarifont कल्पावसाने समुपेत्य मर्त्ये }%
  \dontdisplaylinenum    \var{{\devanagarifontvar \numnoemph\vc\textbf{समुपेत्य मर्त्ये}\lem \mssALL, पुनः मर्त्यलोके \msM}}% 

%Verse 18:9


\nemslokad

{\devanagarifont चिह्नं गवाढ्यं शतगोयुतं च {॥ १८:९॥} \veg\dontdisplaylinenum }%
     \var{{\devanagarifontvar \numnoemph\vd\textbf{चिह्नं}\lem \mssALL, चिह्न \msPaperA}}% 

\ujvers\nemsloka {
{\devanagarifont स्वर्गं गतानां पुरुषस्य चिह्नं }%
  \dontdisplaylinenum}    \var{{\devanagarifontvar \numemph\va\textbf{स्वर्गं गतानां}\lem \mssALL, स्वर्गागतानां \msM}}% 


\nemslokab

{\devanagarifont धनाढ्यता श्री सुखभोगलाभम्  \danda\dontdisplaylinenum }%
     \var{{\devanagarifontvar \numnoemph\vb\textbf{॰ढ्यता}\lem \mssALL, ॰ढ्यतां \msM}}% 

\nemslokac

{\devanagarifont आयुर्यशोरूपकलत्रपुत्रं }%
  \dontdisplaylinenum    \var{{\devanagarifontvar \numnoemph\vc\textbf{॰र्यशो॰}\lem \mssALL, ॰र्यषे॰ \msM}}% 

%Verse 18:10


\nemslokad

{\devanagarifont सम्पद्विभूतिकुलकीर्तिमर्थम् {॥ १८:१०॥} \veg\dontdisplaylinenum }%
     \var{{\devanagarifontvar \numnoemph\vd\textbf{॰मर्थम्}\lem \mssALL, ॰मत्वम् \msM}}% 


\alalfejezet{निरयान्मर्त्यमुपागतानां चिह्नानि}
\ujvers\nemsloka {
{\devanagarifont दानाष्टकं चोत्तम कीर्तितं ते }%
  \dontdisplaylinenum}    \var{{\devanagarifontvar \numemph\va\textbf{चोत्तम}\lem \mssALL, चोतम \msM\oo 
\textbf{कीर्तितं ते}\lem \eme, कीर्तनं ते \msCa\msNa, \uncl{कीर्तितन्ते} \msCb, कीर्त्तितो यम् \msM, 
कीर्त्तिनन्ते \msPaperA}}% 


\nemslokab

{\devanagarifont चिह्नं च लोकं च समासतो मे  \danda\dontdisplaylinenum }%
     \var{{\devanagarifontvar \numnoemph\vb\textbf{चिह्नं च}\lem \mssALL, चिंह्नं स \msM\oo 
\textbf{समासतो}\lem \mssALL, समागतो \msM}}% 

\nemslokac

{\devanagarifont शृणोतु देवी निरयागतानां }%
  \dontdisplaylinenum    \var{{\devanagarifontvar \numnoemph\vc\textbf{शृणोतु देवी}\lem \mssALL, \uncl{शृण्वन्तु देवो} \msM}}% 

%Verse 18:11


\nemslokad

{\devanagarifont चिह्नं च कर्मं च विपाकतां च {॥ १८:११॥} \veg\dontdisplaylinenum }%
     \var{{\devanagarifontvar \numnoemph\vd\textbf{चिह्नं च}\lem \mssALL, चिंह्नं स्व॰ \msM\oo 
\textbf{विपाकतां च}\lem \mssALL, विपाकतानाम् \msM}}% 

\ujvers\nemsloka {
{\devanagarifont हत्वा च विप्रं मनसा च वाचा }%
  \dontdisplaylinenum}    \var{{\devanagarifontvar \numemph\va\textbf{विप्रं}\lem \mssALL, विप्र \msM\msPaperA}}% 


\nemslokab

{\devanagarifont स याति पारं निरयस्य घोरम्  \danda\dontdisplaylinenum }%
     \var{{\devanagarifontvar \numnoemph\vb\textbf{स याति पारं निरयस्य घोरम्}\lem \msCa\msCb\msNa, 
स यान्ति पारं णिरयं सुघोरम् \msM, 
स याति पारं निरयश्च घोरम् \msPaperA}}% 

\nemslokac

{\devanagarifont अशीतिकल्पं निरये क्रमेण }%
  \dontdisplaylinenum    \var{{\devanagarifontvar \numnoemph\vc\textbf{निरये}\lem \mssALL, निरयः \msM}}% 

%Verse 18:12


\nemslokad

{\devanagarifont भुक्त्वा पुनस्तिर्य शतायुतानाम् {॥ १८:१२॥} \veg\dontdisplaylinenum }%
     \var{{\devanagarifontvar \numnoemph\vd\textbf{पुनस्तिर्य}\lem \mssALL, पुनः तिय \msM}}% 

\ujvers\nemsloka {
{\devanagarifont जायन्ति ते मानुष हीनविद्याः }%
  \dontdisplaylinenum}    \var{{\devanagarifontvar \numemph\va\textbf{॰विद्याः}\lem \msCb\msM, ॰विद्या \msCa\msNa\msPaperA}}% 


\nemslokab

{\devanagarifont प्रत्यन्तवासाः कुलवित्तहीनाः  \danda\dontdisplaylinenum }%
     \var{{\devanagarifontvar \numnoemph\vb\textbf{॰वासाः}\lem \mssALL, ॰वासा \msNa, ॰वासी \msM\oo 
\textbf{॰हीनाः}\lem \mssALL, ॰हीना \msM}}% 

\nemslokac

{\devanagarifont नित्यं च तस्या क्षयरोगपीडा }%
  \dontdisplaylinenum    \paral{{\devanagarifontsmall \vb {\englishfont \compare\ \MANU\ 11.49:}
                         ब्रह्महा क्षयरोगित्वं;
                 {\englishfont \compare\ \YAJNS\ 3.210a:}
                         ब्रह्महा क्षयरोगी स्यात् }}

%Verse 18:13


\nemslokad

{\devanagarifont इदं तु चिह्नं द्विजजीवहर्तुः {॥ १८:१३॥} \veg\dontdisplaylinenum }%
     \var{{\devanagarifontvar \numnoemph\vd\textbf{इदं तु चिह्नं}\lem \mssALL, चिंह्नञ्च मे त \msM}}% 

\ujvers\nemsloka {
{\devanagarifont पीत्वा च मद्यं द्विज कामतो वा }%
  \dontdisplaylinenum}    \var{{\devanagarifontvar \numemph\va\textbf{मद्यं}\lem \mssALL, मद्य \msM\oo 
\textbf{द्विज}\lem \msNa\msM, द्विजः \msCa\msCb\msPaperA\ \unmetr\oo 
\textbf{वा}\lem \mssALL, वे \msPaperA}}% 


\nemslokab

{\devanagarifont आघ्राति गन्धं स्वमनीषिकेण  \danda\dontdisplaylinenum }%
     \var{{\devanagarifontvar \numnoemph\vb\textbf{आघ्राति}\lem \mssALL, माघ्राति \msM\oo 
\textbf{॰मनीषिकेण}\lem \mssALL, ॰मनीशिकेन \msM, ॰मणीषकेण \msPaperA}}% 

\nemslokac

{\devanagarifont स याति घोरं नरकमसह्यं }%
  \dontdisplaylinenum    \var{{\devanagarifontvar \numnoemph\vc\textbf{स याति घोरं नरकमसह्यं}\lem \mssALL, 
स यान्ति घोरा नरकर्मसह्यं \msM}}% 

%Verse 18:14


\nemslokad

{\devanagarifont यावच्च कल्पं दश अत्र भुक्त्वा {॥ १८:१४॥} \veg\dontdisplaylinenum }%
     \var{{\devanagarifontvar \numnoemph\vd\textbf{कल्पं दश अत्र}\lem \mssALL, कल्पा दषमन्त्र \msM}}% 

\ujvers\nemsloka {
{\devanagarifont तिर्यं च सर्वमनुभूय दुःखं }%
  \dontdisplaylinenum}    \var{{\devanagarifontvar \numemph\va\textbf{सर्वमनुभूय दुःखं}\lem \mssALL, सर्वमनुभूय दुःख \msNa, 
सर्वं मनुभूय दुःखा \msM}}% 


\nemslokab

{\devanagarifont स कष्टकष्टेन मनुष्यजन्म  \danda\dontdisplaylinenum }%
     \var{{\devanagarifontvar \numnoemph\vb\textbf{कष्टकष्टेन}\lem \mssALL, ष्टकष्टेन \msCb, कष्टकष्टेण \msM\oo 
\textbf{॰जन्म}\lem \mssALL, ॰जन्मम् \msM}}% 

\nemslokac

{\devanagarifont चण्डालशौनश्वपचत्वमेति }%
  \dontdisplaylinenum    \var{{\devanagarifontvar \numnoemph\vc\textbf{चण्डाल॰}\lem \mssALL, चाण्डाल॰ \msM\msPaperA\oo 
\textbf{॰त्वमेति}\lem \mssALL, ॰त्वमे \msCb}}% 

%Verse 18:15


\nemslokad

{\devanagarifont श्यामं च तालु भवतीह चिह्नम् {॥ १८:१५॥} \veg\dontdisplaylinenum }%
     \var{{\devanagarifontvar \numnoemph\vd\textbf{श्यामं}\lem \mssALL, श्यामः \msPaperA\oo 
\textbf{भवतीह}\lem \mssALL, भवतीति ह \msPaperA\oo 
\textbf{चिह्नम्}\lem \mssALL, चिंह्नं \msM}}% 
    \paral{{\devanagarifontsmall \vb {\englishfont \compare\ \MANU\ 11.49b:} 
                         सुरापः श्यावदन्तताम;
                    {\englishfont \compare\ \YAJNS\ 3.210b:} 
                         सुरापः श्यावदन्तकः }}

\ujvers\nemsloka {
{\devanagarifont निन्दन्ति ये वेद †सम्भूय† जिह्वा }%
  \dontdisplaylinenum}    \var{{\devanagarifontvar \numemph\va\textbf{निन्दन्ति ये वेद सम्भूय जिह्वा}\lem \msCa\msNa\msPaperA, 
निन्दन्ति ये वेद \uncl{सम्भूय जिह्वा} \msCb, 
निन्दन्ति ये वेद सस्त्रपजिह्वा \msNb, 
निन्दन्ति ये वद शस्त्राय जिह्वा \msNc, 
निन्दन्ति यो वेद स भूय जिह्वा \msM}}% 
    \paral{{\devanagarifontsmall \vo {\englishfont \compare\ \MANU\ 11.57:}
                 ब्रह्मोज्झता वेदनिन्दा कौटसाक्ष्यं सुहृद्वधः\thinspace{\devanagarifontsmall ।}
                 गर्हितानाद्ययोर्जग्धिः सुरापान$\-$समानि षट्\thinspace{\devanagarifontsmall ॥} }}


\nemslokab

{\devanagarifont यः कूटसाक्षी स च खल्वलान्धौ  \danda\dontdisplaylinenum }%
     \var{{\devanagarifontvar \numnoemph\vb\textbf{यः कूटसाक्षी}\lem \mssALL, यः कूटसाक्ष \msCb, यो कूटसा\lk क्षी \msM\oo 
\textbf{खल्वलान्धौ}\lem \msCa\msCb\msNa\msPaperA, ख\uncl{स्पृ}लत्वौ \msM}}% 

\nemslokac

{\devanagarifont सुहृद्वधा मृत्युशतं हि गर्भे }%
  \dontdisplaylinenum    \var{{\devanagarifontvar \numnoemph\vc\textbf{मृत्यु॰}\lem \mssALL, महो॰ \Ed}}% 
    \lacuna{\devanagarifontsmall \vc {\englishfont \Ed\ resumes here on p.~649 with} महो {\englishfont (for} मृत्यु{\englishfont ; about two folio sides seem to be missing)}. }%
  
%Verse 18:16


\nemslokad

{\devanagarifont गर्हाशनोच्छिष्टभुजो भवन्ति {॥ १८:१६॥} \veg\dontdisplaylinenum }%
     \var{{\devanagarifontvar \numnoemph\vd\textbf{गर्हा॰}\lem \mssALL, गर्भा॰ \msM\oo 
\textbf{भवन्ति}\lem \mssALL, भ्वन्ति भुङ्क्त्वा महद्दुःख \Ed}}% 

\ujvers\nemsloka {
{\devanagarifont स्तैन्यं तु यः कुर्वति पापसत्त्वं }%
  \dontdisplaylinenum}    \var{{\devanagarifontvar \numemph\va\textbf{स्तैन्यं तु}\lem \corr, स्तैन्यस्तु \msCa\msNa\msPaperA, 
\gap न्य\uncl{स्तु} \msCb, 
स्तैन्यञ्च \msM, 
दैत्यस्तु \Ed\oo 
\textbf{यः}\lem \msCb\msM\Ed, ये \msCapcorr\msNa\msPaperA, यैः \msCaacorr\oo 
\textbf{॰सत्त्वम्}\lem \mssALL, ॰स\uncl{त्वन्} \msCa}}% 


\nemslokab

{\devanagarifont ते पापदोषान्नरकं व्रजन्ति  \danda\dontdisplaylinenum }%
     \var{{\devanagarifontvar \numnoemph\vb\textbf{ते}\lem \mssALL, \lac\  \msCa, स \msM\oo 
\textbf{॰दोषान्न॰}\lem \mssALL, ॰दोषा न॰ \msCb\msM\oo 
\textbf{व्रजन्ति}\lem \mssALL, प्रयान्ति \msM}}% 

\nemslokac

{\devanagarifont मन्वन्तरादीन्यनुभूय दुःखं }%
  \dontdisplaylinenum    \var{{\devanagarifontvar \numnoemph\vc\textbf{॰न्तरादीन्य॰}\lem \mssALL, ॰न्तराद्धीम॰ \msM\oo 
\textbf{दुःखं}\lem \mssALL, दुःख \msPaperA}}% 

%Verse 18:17


\nemslokad

{\devanagarifont पुनश्च तिर्यं शतशो ऽनुभूयात् {॥ १८:१७॥} \veg\dontdisplaylinenum }%
     \var{{\devanagarifontvar \numnoemph\vd\textbf{तिर्यं श॰}\lem \mssALL, तिय स॰ \msM, तिर्यक् श॰ \msPaperA\Ed\oo 
\textbf{नुभूयात्}\lem \mssALL, नुभुक्त्वा \msM}}% 

\ujvers\nemsloka {
{\devanagarifont मानुष्यजन्मेषु च दुःखभागी }%
  \dontdisplaylinenum}    \var{{\devanagarifontvar \numemph\va\textbf{मानुष्य॰}\lem \mssALL, मनुष्य॰ \msNa\msM}}% 


\nemslokab

{\devanagarifont स्तेनत्वमायाति पुनश्च मूढः  \danda\dontdisplaylinenum }%
     \var{{\devanagarifontvar \numnoemph\vb\textbf{स्तेनत्वम आयाति पुनश्च मूढः}\lem \msCa\msCb\msNa\msPaperA, 
स्ते\uncl{यु}त्वमायांति पुनश्च मूढाः \msM, 
स्तेने ऽयम आयाति पुनश्च मूढाः \Ed}}% 

\nemslokac

{\devanagarifont सुवर्णचोरी कुनखत्व चिह्नं }%
  \dontdisplaylinenum    \var{{\devanagarifontvar \numnoemph\vc\textbf{॰चोरी}\lem \msCa, ॰चौ\lk\ \msCb, ॰चौरी \msNa\msM\msPaperA, ॰चौर \Ed\oo 
\textbf{कुनखत्व चिह्नम्}\lem \msCa\msNa\msPaperA\Ed, \uncl{कनखत्व चिह्न} \msCb, 
कुनत्व चिंह्नं \msMacorr, 
कुनक्षत्व चिंह्नं \msMpcorr}}% 
    \paral{{\devanagarifontsmall \vb {\englishfont \compare\ \MANU\ 11.49a:}
                         सुवर्णचौरः कौनख्यं;
                    {\englishfont \compare\ \YAJNS\ 3.210c:}
                         हेमहारी तु कुनखी }}

%Verse 18:18


\nemslokad

{\devanagarifont विशीर्णगात्रो रजतापहारी {॥ १८:१८॥} \veg\dontdisplaylinenum }%
     \var{{\devanagarifontvar \numnoemph\vd\textbf{विशीर्ण॰}\lem \mssALL, \uncl{विस्तीर्ण}॰ \msCb\oo 
\textbf{॰गात्रो}\lem \mssALL, ॰सूत्रद् \msM}}% 

\ujvers\nemsloka {
{\devanagarifont ताम्रापहारी स्फुटिताग्रपाणिर् }%
  \dontdisplaylinenum}    \var{{\devanagarifontvar \numemph\va\textbf{ताम्रापहारी}\lem \mssALL, त्रांम्रापहारी \msM, ताम्रापहारि \Ed\oo 
\textbf{स्फुटिता॰}\lem \mssALL, स्फटिता॰ \Ed\oo 
\textbf{॰पाणिर्}\lem \mssALL, ॰पाणि \msM\msPaperA, ॰पाणीर् \Ed}}% 


\nemslokab

{\devanagarifont लोहापहारी भुजछेद चिह्नम्  \danda\dontdisplaylinenum }%
     \var{{\devanagarifontvar \numnoemph\vb\textbf{लोहापहारी}\lem \mssALL, लोह\lac\ हारी \msCa\oo 
\textbf{चिह्नम्}\lem \mssALL, चिंह्नम् \msM}}% 

\nemslokac

{\devanagarifont कांसापहारी करभग्न चिह्नं }%
  \dontdisplaylinenum    \var{{\devanagarifontvar \numnoemph\vc\textbf{कांसा॰}\lem \mssALL, कांसो॰ \msPaperA\oo 
\textbf{कर॰}\lem \mssALL, क॰ \msCaacorr\oo 
\textbf{चिह्नं}\lem \mssALL, चिह्णम् \msMpcorr, ह्णम् \msMacorr}}% 

%Verse 18:19


\nemslokad

{\devanagarifont हृत्वा च रीतित्रपुसीसकानाम् {॥ १८:१९॥} \veg\dontdisplaylinenum }%
     \var{{\devanagarifontvar \numnoemph\vd\textbf{हृत्वा च रीति॰}\lem \msCa\msPaperA\Ed, हृत्वापचारीति \msCb, हृत्वा च रीतिं \msNa, 
हृत्वा च रीती॰ \msM}}% 

\ujvers\nemsloka {
{\devanagarifont नासोष्ठकर्णश्रवणस्य छेदश् }%
  \dontdisplaylinenum}    \var{{\devanagarifontvar \numemph\va\textbf{नासो॰}\lem \mssALL, नासौ॰ \Ed}}% 


\nemslokab

{\devanagarifont चिह्नं नृणां वस्त्रहरः कुचैलः  \danda\dontdisplaylinenum }%
     \var{{\devanagarifontvar \numnoemph\vb\textbf{चिह्नं}\lem \mssALL, चिंह्नं \msM\oo 
\textbf{॰हरः}\lem \mssALL, ॰हरं \Ed\oo 
\textbf{कुचैलः}\lem \mssALL, कुचेलः \msPaperA\Ed}}% 

\nemslokac

{\devanagarifont धान्यापहारी भवते ऽङ्गहीनो }%
  \dontdisplaylinenum    \var{{\devanagarifontvar \numnoemph\vc\textbf{भवते ऽङ्ग॰}\lem \msNa\msM\msPaperA, 
भवदेङ्ग॰ \msCa, 
\uncl{भवत्यङ्ग}॰ \msCb, 
भवत्येङ्ग॰ \Ed}}% 

%Verse 18:20


\nemslokad

{\devanagarifont दीपापहारी भवते ऽन्ध चिह्नम् {॥ १८:२०॥} \veg\dontdisplaylinenum }%
     \var{{\devanagarifontvar \numnoemph\vd\textbf{दीपा॰}\lem \mssALL, दिपो \Ed\oo 
\textbf{भवते ऽन्ध}\lem \msCa\msCb\msNa\msM\msPaperA, भवत्यन्ध \Ed}}% 

\ujvers\nemsloka {
{\devanagarifont निर्वापहा काण भवेत चिह्नं }%
  \dontdisplaylinenum}    \var{{\devanagarifontvar \numemph\va\textbf{चिह्नं}\lem \mssALL, \uncl{चिह्नं} \msCb, चिंह्नं \msM}}% 


\nemslokab

{\devanagarifont यः स्त्रीं हरेत्सो ऽपि जितः स्त्रिया स्यात्  \danda\dontdisplaylinenum }%
     \var{{\devanagarifontvar \numnoemph\vb\textbf{यः स्त्रीं हरेत्सो ऽपि जितः स्त्रिया स्यात्}\lem \corr, 
यः स्त्री हरेत्सो ऽपि जितस्स्त्रिया स्यात् \msCa, 
यः स्त्री हरे सो ऽपि जितः स्त्रियायात् \msCb, 
यः स्त्री हरेत्सो ऽपि जित स्त्रिया स्यात् \msNa, 
यः स्त्री हरे स्त्रीभिः जिता भवन्ति \msM, 
यः स्त्री हरेत्सो ऽपि जितः स्त्रिया स्यात् \msPaperA\Ed}}% 
    \paral{{\devanagarifontsmall \vo {\englishfont \compare\ \MITAKSARA\ ad \YAJNS\ 3.216cd:}
          ... न्यासापहारी च काणः, स्त्रीपण्योपजीवी षण्ढः, कौमारदारत्यागी दुर्भगः... }}

\nemslokac

{\devanagarifont सस्यापहारी भवते ऽन्नहीनो }%
  \dontdisplaylinenum    \var{{\devanagarifontvar \numnoemph\vc\textbf{भवते ऽन्नहीनो}\lem \mssALL, भवते ऽन्नहीवः \msCb, 
भवेतन्नहीनः \msPaperA}}% 

%Verse 18:21


\nemslokad

{\devanagarifont हृत्वायुधमस्त्रहतत्व चिह्नम् {॥ १८:२१॥} \veg\dontdisplaylinenum }%
     \var{{\devanagarifontvar \numnoemph\vd\textbf{॰युधमस्त्र॰}\lem \mssALL, ॰युधयन्त्र \Ed}}% 

\ujvers\nemsloka {
{\devanagarifont अन्नापहारी परदत्तभोक्ता }%
  \dontdisplaylinenum}

\nemslokab

{\devanagarifont हृत्वा तु गावः स भवेद्दरिद्रः  \danda\dontdisplaylinenum }%
     \var{{\devanagarifontvar \numemph\vb\textbf{भवेद्दरिद्रः}\lem \mssALL, भवे दरिद्रंः \msM}}% 

\nemslokac

{\devanagarifont हरिं हरेत्तद्धरिणा दहन्ति }%
  \dontdisplaylinenum    \var{{\devanagarifontvar \numnoemph\vc\textbf{हरिं हरेत्तद्धरिणा}\lem \mssALL, हरिन्भवे त हरिणा \msM, 
हरिहरेत्तद्धरिणा \Ed}}% 

%Verse 18:22


\nemslokad

{\devanagarifont हृत्वा तु मेषान् अजगर्दभं वा {॥ १८:२२॥} \veg\dontdisplaylinenum }%
     \var{{\devanagarifontvar \numnoemph\vd\textbf{हृत्वा तु मेषानजगर्दभं वा}\lem \msCa\msCb\msPaperA, 
हृत्वा च मेषानजगर्दभम्च \msNa, 
हृत्वा च मेषामजगर्दभञ्च \msM, 
हृत्वा तु मेषानजगर्दभश्च \Ed}}% 

\ujvers\nemsloka {
{\devanagarifont स भारभृज्जीव्यमुदाहरन्ति }%
  \dontdisplaylinenum}    \var{{\devanagarifontvar \numemph\va\textbf{॰ज्जीव्य॰}\lem \mssALL, ॰ज्जीवा॰ \msM, ॰ज्जीव॰ \Ed}}% 


\nemslokab

{\devanagarifont रत्नापहारी अनपत्यता च  \danda\dontdisplaylinenum }%
     \var{{\devanagarifontvar \numnoemph\vb\textbf{अनपत्यता}\lem \mssALL, \lac\ त्यता \msCa}}% 
    \paral{{\devanagarifontsmall \vb {\englishfont \compare\ \MITAKSARA\ ad Yājñavalkya 3.216cd:}
                        ... गौतमो ऽपि क्वचिद्विशेषमाह\thinspace{\devanagarifontsmall ।} ... 
                        न्यासापहार्यनपत्यः, रत्नापहार्यत्यन्तदरिद्रः...  }}

\nemslokac

{\devanagarifont छत्रापहारी अपवित्रता च }%
  \dontdisplaylinenum    \var{{\devanagarifontvar \numnoemph\vc\textbf{अपवित्रता}\lem \msCb\msPaperA\Ed, अपरित्रता \msCa\msNa\msM}}% 

%Verse 18:23


\nemslokad

{\devanagarifont हृत्वा च बीजं स भवेदबीजः {॥ १८:२३॥} \veg\dontdisplaylinenum }%
     \var{{\devanagarifontvar \numnoemph\vd\textbf{हृत्वा च बीजं स भवेदबीजः}\lem \msCa\msNa\msM\Ed, 
\uncl{हृत्वा नृजीवः स भवेदजीवः} \msCb, 
हृत्वा च बीजं स भवेदजीवः \msPaperA}}% 

\ujvers\nemsloka {
{\devanagarifont गोधूमशालियवमुद्गमाषान् }%
  \dontdisplaylinenum}    \var{{\devanagarifontvar \numemph\va\textbf{॰मुद्गमाषान्}\lem \msCa\msNa\Ed, ॰मुद्गमाषा \msCb, ॰माषमुजान् \msM, ॰मुद्गमाषां \msPaperA}}% 


\nemslokab

{\devanagarifont हृत्वा मसूरं विलयं व्रजन्ति  \danda\dontdisplaylinenum }%
     \var{{\devanagarifontvar \numnoemph\vb\textbf{मसूरं}\lem \mssALL, मूत्रं \msMacorr, मूरं \msMpcorr}}% 

\nemslokac

{\devanagarifont कामातुरो मातर मातृपुत्रीं }%
  \dontdisplaylinenum    \var{{\devanagarifontvar \numnoemph\vc\textbf{मातृपुत्रीं}\lem \msCa\msCb\msNa, मात्रपुत्री \msM\msPaperA, मातृपुत्री \Ed}}% 

%Verse 18:24


\nemslokad

{\devanagarifont मातृस्वसां गच्छति मातुलानीम् {॥ १८:२४॥} \veg\dontdisplaylinenum }%
     \var{{\devanagarifontvar \numnoemph\vd\textbf{॰स्वसां}\lem \mssALL, ॰स्वसा \msM\oo 
\textbf{मातुलानीम्}\lem \mssALL, मातुलानी \msNa, मातुलानीः \msM}}% 

\ujvers\nemsloka {
{\devanagarifont राजाङ्गनां पुत्रसुतां स्नुषां च }%
  \dontdisplaylinenum}    \var{{\devanagarifontvar \numemph\va\textbf{राजाङ्गनां}\lem \mssALL, राजाङ्गणा \msCb\msM\oo 
\textbf{॰सुतां}\lem \msNa\Ed, ॰सुत \msCa\msCb\msM, ॰सुता \msPaperA\oo 
\textbf{स्नुषां}\lem \mssALL, स्नुसा \msM}}% 


\nemslokab

{\devanagarifont प्रव्राजिनीं ब्राह्मणिमन्त्यजां च  \danda\dontdisplaylinenum  }%
     \var{{\devanagarifontvar \numnoemph\vb\textbf{॰व्राजिनीं}\lem \msCa\msCb\Ed, ॰व्राजिनी \msNa, ॰व्रजनी \msM, ॰व्रजिनां \msPaperA\oo 
\textbf{ब्राह्मणिमन्त्यजां च}\lem \msCb\msNa, ब्राह्मणिमन्त्य\lk ञ्च \msCa, 
ब्राह्मणि चान्त्यजा च \msM, 
ब्रह्मणिमन्त्यजाञ्च \msPaperA, 
ब्राह्मणीमन्त्यजां च \Ed}}% 

\nemslokac

{\devanagarifont अजाश्वमेषं सुरभीसुतां च }%
  \dontdisplaylinenum    \var{{\devanagarifontvar \numnoemph\vc\textbf{॰मेषं}\lem \msCa\msCb\msNa, ॰मेष॰ \msM\msPaperA\Ed\oo 
\textbf{॰सुतां च}\lem \msPaperA, ॰सुतं च \msCa\msCb\msNa, ॰सुतश्च \msM, ॰सुताश्च \Ed}}% 

%Verse 18:25


\nemslokad

{\devanagarifont यत्कामयेत्तेषु विमूढचेताः {॥ १८:२५॥} \veg\dontdisplaylinenum }%
     \var{{\devanagarifontvar \numnoemph\vd\textbf{यत्का॰}\lem \mssALL, यः का॰ \msM\oo 
\textbf{॰चेताः}\lem \mssALL, ॰चेतः \Ed}}% 

\ujvers\nemsloka {
{\devanagarifont स याति कृच्छ्रं नरकं सुघोरं }%
  \dontdisplaylinenum}    \var{{\devanagarifontvar \numemph\va\textbf{याति}\lem \mssALL, यान्ति \msM}}% 


\nemslokab

{\devanagarifont स वर्षकोटीशतशो भ्रमित्वा  \danda\dontdisplaylinenum }%
 
\nemslokac

{\devanagarifont तिर्यं च भूयः शतशो व्यतीत्य }%
  \dontdisplaylinenum    \var{{\devanagarifontvar \numnoemph\vc\textbf{तिर्यं च}\lem \mssALL, तियञ्च \msM, तीर्यञ्च \Ed\oo 
\textbf{व्यतीत्य}\lem \mssALL, व्यतित्य \msPaperA}}% 

%Verse 18:26


\nemslokad

{\devanagarifont कष्टेन वै जायति मानुषत्वम् {॥ १८:२६॥} \veg\dontdisplaylinenum }%
     \var{{\devanagarifontvar \numnoemph\vd\textbf{वै जायति}\lem \mssALL, 
वै \uncl{नेत्यद्मि} \msCb, प्राप्नोति स \msM}}% 

\ujvers\nemsloka {
{\devanagarifont हीनाङ्गतां दीनशरीरतां च }%
  \dontdisplaylinenum}    \var{{\devanagarifontvar \numemph\va\textbf{हीनाङ्गतां दीनशरीरतां च}\lem \msCb, हीनाङ्गता दीनशरीरताश्च \msNa\Ed, 
हीनाङ्गतान्दीनशरीरताञ्च \msCa, 
हीनाङ्गता दीनशरीरताञ्च \msPaperA, 
हीनाङ्ग दीनकुशरीरता च \msM}}% 


\nemslokab

{\devanagarifont यो मातृगामी स भवेदलिङ्गः  \danda\dontdisplaylinenum }%
     \var{{\devanagarifontvar \numnoemph\vb\textbf{॰लिङ्गः}\lem \mssALL, ॰लि\lk\ \msCa}}% 

\nemslokac

{\devanagarifont मातृस्वसातल्पग वातलिङ्गो }%
  \dontdisplaylinenum    \var{{\devanagarifontvar \numnoemph\vc\textbf{॰वातलिङ्गो}\lem \corr, ॰वातलिङ्गा \msCa\msNa, ॰वात॰ \msCb, 
॰वातलिङ्गः \msM, ॰वानलि\lk\ \msPaperA, ॰वानलिङ्गा \Ed}}% 

%Verse 18:27


\nemslokad

{\devanagarifont लिङ्गोपरोधः सुतपुत्रिकामः {॥ १८:२७॥} \veg\dontdisplaylinenum }%
     \var{{\devanagarifontvar \numnoemph\vd\textbf{लिङ्गोपरोधः सुतपुत्रिकामः}\lem \eme, 
लिङ्गापरोधः सुतपुत्रिकाम \msCa, 
लिङ्गापरोधः सुतपुत्रिकामा \msCb, 
लिङ्गापरोधः सुतपुत्रिकामः \msNa\msPaperA, 
लिङ्गापरोधः सुत्रपुत्रिकामा \msM, 
लिङ्गेपरोधः सुतपुत्रिकामः \Ed}}% 

\ujvers\nemsloka {
{\devanagarifont स्नुषां च यः सेवति रक्तमेही }%
  \dontdisplaylinenum}

\nemslokab

{\devanagarifont दौश्चर्मतां च द्विजसुन्दरीषु  \danda\dontdisplaylinenum }%
     \var{{\devanagarifontvar \numemph\vb\textbf{दौश्चर्मतां च द्विजसुन्दरीषु}\lem \msCa\msCc, 
\lac\ ञ्च  द्विजसुन्दरीषु \msCb, 
दौचर्मतां च द्विजसुन्दरीषु \msNa, 
दौचर्म्मता याति द्विजाङ्गनाम्च \msM, 
दौःचर्म्मतांच द्विजसुन्दरीषु \msPaperA, 
दौः चर्मराश् च द्विजसुन्दरीषु \Ed}}% 
    \lacuna{\devanagarifontsmall \vb {\englishfont \msCc\ resumes here in f.~306r with} चर्मताश्च द्विजसुन्दरीषु }%
      \paral{{\devanagarifontsmall \vb  {\englishfont \compare\ \MANU\ 11.49d:}
                         दौश्चर्म्यं गुरुतल्पगः;
                     {\englishfont \compare\ \YAJNS\ 3.210d:}
                         दुश्चर्मा गुरुतल्पगः }}

\nemslokac

{\devanagarifont राजाङ्गनायासु च लिङ्गच्छेदः }%
  \dontdisplaylinenum    \var{{\devanagarifontvar \numnoemph\vc\textbf{लिङ्ग॰}\lem \mssALL, लिङ्गः \msNa}}% 

%Verse 18:28


\nemslokad

{\devanagarifont प्रव्राजिनीकामुक मूत्रकृच्छ्रम् {॥ १८:२८॥} \veg\dontdisplaylinenum }%
     \var{{\devanagarifontvar \numnoemph\vd\textbf{॰व्राजिनी॰}\lem \mssALL, ॰वाजिनी॰ \msCc, ॰व्रजिनी॰ \msPaperA\oo 
\textbf{॰कृच्छ्रम्}\lem \mssALL, ॰कृच्छ्रः \msM, ॰कृच्छ्र \msPaperA}}% 

\ujvers\nemsloka {
{\devanagarifont सव्याधिलिङ्गं लभते ऽन्त्यजासु }%
  \dontdisplaylinenum}    \var{{\devanagarifontvar \numemph\va\textbf{सव्याधिलिङ्गं लभते}\lem \mssALL, 
अत्याधिलिङ्गा भवते \msM, सव्याधिलिङ्ग लभते \msPaperA\Ed}}% 


\nemslokab

{\devanagarifont विलीनलिङ्गः पशुयोनिगामी  \danda\dontdisplaylinenum }%
     \var{{\devanagarifontvar \numnoemph\vb\textbf{विलीनलिङ्गः}\lem \mssALL, विलीनः \msCb\oo 
\textbf{॰योनिगामी}\lem \mssALL, ॰यो\lac\ मी \msCa}}% 

\nemslokac

{\devanagarifont जायन्ति ते मूषिक धान्यचौरी }%
  \dontdisplaylinenum    \var{{\devanagarifontvar \numnoemph\vc\textbf{॰चौरी}\lem \msCb\msNa\Ed, ॰चोरी \msCa\msCc\msM\msPaperA}}% 
    \paral{{\devanagarifontsmall \vcd {\englishfont for these pādas and the next verse, \compare\ \MANU\ 12.62:}
                         धान्यं हृत्वा भवत्याखुः कांस्यं हंसो जलं प्लवः\thinspace{\devanagarifontsmall ।}
                         मधु दंशः पयः काको रसं श्वा नकुलो घृतम्\thinspace{\devanagarifontsmall ॥};
                 {\englishfont \compare\ also \YAJNS\ 3.215:}
                         मूषको धान्यहारी स्याद्यानमुष्ट्रः फलं कपिः\thinspace{\devanagarifontsmall ।}
                         अजः पशुं पयः काको गृहकार उपस्करम्\thinspace{\devanagarifontsmall ॥} }}

%Verse 18:29


\nemslokad

{\devanagarifont क्षीरं हरेद्वायसतां प्रयाति {॥ १८:२९॥} \veg\dontdisplaylinenum }%
     \var{{\devanagarifontvar \numnoemph\vd\textbf{॰यसतां}\lem \mssALL, ॰यता \msNaacorr, ॰यसता \msNapcorr}}% 

\ujvers\nemsloka {
{\devanagarifont कांसापहारी स भवेत्तु हंसः }%
  \dontdisplaylinenum}    \var{{\devanagarifontvar \numemph\va\textbf{कांसा॰}\lem \eme, हंसा॰ \mssCaCbCc\msNa\msPaperA\Ed, हान्सा॰ \msM\oo 
\textbf{भवेत्तु}\lem \conj, भवेन्नि॰ \mssCaCbCc\msNa\msPaperA\Ed, भवेत \msM}}% 
    \paral{{\devanagarifontsmall \va {\englishfont \compare\ \MITAKSARA\ ad \YAJNS\ 3.316:}
                         यथा कांस्यहारी हंस इति }}


\nemslokab

{\devanagarifont श्वानत्वमायाति रसापहारी  \danda\dontdisplaylinenum }%
     \var{{\devanagarifontvar \numnoemph\vb\textbf{श्वानत्व॰}\lem \mssALL, श्वातत्व॰ \msCb, श्वनत्व॰ \msPaperA\oo 
\textbf{रसा॰}\lem \mssALL, रषा॰ \msNa}}% 

\nemslokac

{\devanagarifont हृत्वा च सूचीं तु भवेत्स दंशः }%
  \dontdisplaylinenum    \var{{\devanagarifontvar \numnoemph\vc\textbf{सूचीं तु भवेत्स}\lem \msCa\msCb, सूची तु भवेत्स \msNa\msPaperA, माध्वीकरसं स \msM, 
सूचीन्तु भवेत्स \msCc\Ed}}% 

%Verse 18:30


\nemslokad

{\devanagarifont हृत्वा तु सर्पिर्वृकतां प्रयाति {॥ १८:३०॥} \veg\dontdisplaylinenum }%
     \var{{\devanagarifontvar \numnoemph\vd\textbf{सर्पिर्वृकतां प्रयाति}\lem \msCa, 
सर्प्पि \uncl{वृ}कता प्रयाति \msCb, 
सर्प्पिर्वृकृतां प्रयान्ति \msCc, 
सर्पि वृ\uncl{क}तां प्रयाति \msNa, 
सर्प्पिर्वृकृतां प्रयाति \msM, 
सर्प्पि वृषतां प्रयाति \msPaperA, 
सर्पिर्वृषतां प्रयाति \Ed}}% 
    \paral{{\devanagarifontsmall \vo {\englishfont \compare\ \YAJNS\ 3.216:}
                         मधु दंशः पलं गृध्रो गां गोधाग्निं बकस्तथा\thinspace{\devanagarifontsmall ।}
                         श्वित्री वस्त्रं श्वा रसं तु चीरी लवणमेव च\thinspace{\devanagarifontsmall ॥} }}

\ujvers\nemsloka {
{\devanagarifont मांसं तु हृत्वा स भवेत गृध्रस् }%
  \dontdisplaylinenum}    \var{{\devanagarifontvar \numemph\va\textbf{मांसं}\lem \mssALL, मान्सा \msM}}% 
    \paral{{\devanagarifontsmall \va {\englishfont \compare\ \MANU\ 12.63a:}
                      मांसं गृध्रो वसां मद्गुः }}


\nemslokab

{\devanagarifont तैलापहारी खगतां प्रयाति  \danda\dontdisplaylinenum }%
     \var{{\devanagarifontvar \numnoemph\vb\textbf{॰हारी}\lem \mssALL, ॰हारा \msM\oo 
\textbf{खगतां}\lem \mssALL, खशतां \msPaperA\oo 
\textbf{॰याति}\lem \mssALL, ॰या\lk\ \msCa}}% 
    \paral{{\devanagarifontsmall \va {\englishfont \compare\ \MANU\ 12.63b:}
                         तैलं तैलपकः खगः;
                    {\englishfont \compare\ \YAJNS\ 3.212c:}
                         तैलहृत् तैलपायी }}

\nemslokac

{\devanagarifont गुडं च हृत्वा गुडिका भवन्ति }%
  \dontdisplaylinenum    \var{{\devanagarifontvar \numnoemph\vc\textbf{गुडं च}\lem \mssALL, \lk डञ्च \msCa, गुडन्तु \msM\oo 
\textbf{भवन्ति}\lem \mssALL, \uncl{भवन्तिम्} \msCb}}% 
    \paral{{\devanagarifontsmall \vc {\englishfont \compare\ \MANU\ 12.64d:} गोधा गां वाग्गुदो गुडम् }}

%Verse 18:31


\nemslokad

{\devanagarifont शाकापहारी स भवेन्मयूरः {॥ १८:३१॥} \veg\dontdisplaylinenum }%
     \var{{\devanagarifontvar \numnoemph\vd\textbf{॰पहारी}\lem \mssALL, ॰प्रहारी \msM\oo 
\textbf{भवेन्मयूरः}\lem \msCa\msNa, भवेत्मयूरः \msCb\msCc\msPaperA, 
भवे मयूरः \msM, 
भवेन्मयूरम् \Ed}}% 
    \paral{{\devanagarifontsmall \vd {\englishfont \compare\ \MANU\ 12.65b:} पत्रशाकं तु बर्हिणः;
                    {\englishfont \compare\ \YAJNS\ 3.214c:} पत्रशाकं शिखी हत्वा }}

\ujvers\nemsloka {
{\devanagarifont हृत्वा पशुं पङ्गुर जायते ह }%
  \dontdisplaylinenum}    \var{{\devanagarifontvar \numemph\va\textbf{हृत्वा पशुं पङ्गुर जायते ह}\lem \msCc\msPaperA, हृत्वा य पशुं पङ्गुर जायते ह \msCa, 
\uncl{हृत्वा पशुं पङ्गुनु} जायते हः \msCb, 
हृत्वा पशु पङ्गुर जायते ह \msNa, 
हृत्वा पशु पंगुक जायतीहः \msM, 
हृत्वा पशुं पङ्गुर जायते हः \Ed}}% 


\nemslokab

{\devanagarifont श्वित्रत्वमायाति सुवस्त्रहारी  \danda\dontdisplaylinenum }%
     \var{{\devanagarifontvar \numnoemph\vb\textbf{श्वित्रत्व॰}\lem \mssALL, श्वैत्रत्व॰ \msM, चित्रत्व॰ \Ed\oo 
\textbf{॰वस्त्र॰}\lem \mssALL, ॰व॰ \msNaacorr\oo 
\textbf{॰हारी}\lem \mssALL, ॰हर्त्ता \msM}}% 

\nemslokac

{\devanagarifont हृत्वा दुकूलं स च सारसत्वं }%
  \dontdisplaylinenum    \var{{\devanagarifontvar \numnoemph\vc\textbf{दुकूलं}\lem \mssALL, ऽकूलं \msPaperA}}% 

%Verse 18:32


\nemslokad

{\devanagarifont क्षौमं च हृत्वा स च दर्दुरत्वम् {॥ १८:३२॥} \veg\dontdisplaylinenum }%
     \var{{\devanagarifontvar \numnoemph\vd\textbf{क्षौमं च}\lem \msCa\msM\msPaperA\Ed, क्षोमं च \msCb\msCc\msNa\oo 
\textbf{दर्दुरत्वम्}\lem \msCa\msCb\msNa\msM, दुर्दुरत्वम् \msCc, दुर्द्दलत्वं \msPaperA, दुर्व्वलत्वम् \Ed}}% 

\ujvers\nemsloka {
{\devanagarifont और्णानि वस्त्राण्यपहृत्य मेषः }%
  \dontdisplaylinenum}    \var{{\devanagarifontvar \numemph\va\textbf{और्णानि वस्त्राण्य॰}\lem \msCc\msNa\msPaperA, 
ओर्णानि वस्त्राण्य॰ \msCa\msCb, 
उर्ण्णञ्च वस्त्रम॰  \msM, 
ऊर्णानि वस्त्राण्य॰ \Ed\oo 
\textbf{मेषः}\lem \mssALL, मेसं \msM}}% 


\nemslokab

{\devanagarifont छुच्छुन्दरी जायति गन्धहारी  \danda\dontdisplaylinenum }%
     \var{{\devanagarifontvar \numnoemph\vb\textbf{छुच्छुन्दरी}\lem \mssALL, छुंछुन्दरी \msNa}}% 
    \paral{{\devanagarifontsmall \vb {\englishfont \compare\ \YAJNS\ 3.214d:}
                         गन्धांश्छुच्छुन्दरी शुभान् }}

\nemslokac

{\devanagarifont ब्रह्मस्वमल्पमपि हृत्य भोक्ता }%
  \dontdisplaylinenum    \var{{\devanagarifontvar \numnoemph\vc\textbf{ब्रह्मस्वमल्प॰}\lem \msCb\msCc\msNa\msPaperA\Ed, \uncl{ब्रह्म}\lac\ मल्प॰ \msCa, 
ब्रह्मश्वद्रव्य॰ \msM\oo 
\textbf{॰पि हृत्य}\lem \eme\ \Torzsok, ॰पहृत्य \mssCaCbCc\msNa\msM\msPaperA\Ed}}% 

%Verse 18:33


\nemslokad

{\devanagarifont स गृध्र उच्छिष्टभुजो भवन्ति {॥ १८:३३॥} \veg\dontdisplaylinenum }%
     \var{{\devanagarifontvar \numnoemph\vd\textbf{॰भुजो}\lem \msCc\msNa\msM\Ed, ॰भुजे \msCa, ॰\uncl{भुजा} \msCb, ॰भुजा \msPaperA}}% 

\ujvers\nemsloka {
{\devanagarifont पादेन यः स्पर्शयते द्विजाङ्घ्रिं }%
  \dontdisplaylinenum}    \var{{\devanagarifontvar \numemph\va\textbf{पादेन यः स्पर्शयते द्विजाङ्घ्रिं}\lem \mssALL, 
\lac\ \msCb, 
पादेन य स्पर्शयते द्विजानां \msM}}% 


\nemslokab

{\devanagarifont तद्वातरक्तं चरणे भवेत  \danda\dontdisplaylinenum }%
     \var{{\devanagarifontvar \numnoemph\vb\textbf{तद्वातरक्तं चरणे भवेत}\lem \msCa\msCc\msNa\msPaperA, 
\uncl{तद्वात}रक्तञ्चरणे भवेत् \msCb\ \unmetr, 
स वातरक्त चरणा भवन्ति \msM, 
तच्छीतरक्तं चरणौ भवेत \Ed}}% 

\nemslokac

{\devanagarifont पादेन यः स्पर्शयते च गावः }%
  \dontdisplaylinenum    \var{{\devanagarifontvar \numnoemph\vc\textbf{पादेन यः}\lem \mssALL, पादा हि सं॰ \msM}}% 

%Verse 18:34


\nemslokad

{\devanagarifont स पादरोगान्विविधान्लभेत {॥ १८:३४॥} \veg\dontdisplaylinenum }%
     \var{{\devanagarifontvar \numnoemph\vd\textbf{स पादरोगान्विविधान्लभेत}\lem \msCa\msCc\Ed, स पादरोगान्विविधान्लभेत् \msCb\msNa\ \unmetr, 
स पादरोगा विविधा भवन्ति \msM, 
स पादरोगा विधा लभेत \msPaperA}}% 

\ujvers\nemsloka {
{\devanagarifont यो मातरं ताडयते पदेन }%
  \dontdisplaylinenum}    \var{{\devanagarifontvar \numemph\va\textbf{यो मातरं ताडयते पदेन}\lem \mssALL, 
पादेन यो ताडयतीह माताः \msM, 
यो मातरः ताडयते पदेन \Ed}}% 


\nemslokab

{\devanagarifont पादे तदीये कृमयः पतन्ति  \danda\dontdisplaylinenum }%
     \var{{\devanagarifontvar \numnoemph\vb\textbf{पादे तदीये कृमयः}\lem \mssALL, पादेषु तस्य तृमियः \msM}}% 

\nemslokac

{\devanagarifont पदा स्पृशेद्यः पितरं दुरात्मा }%
  \dontdisplaylinenum    \var{{\devanagarifontvar \numnoemph\vc\textbf{पदा स्पृशेद्यः}\lem \msCa\msCb\msNa, पादा शेद्यः \msCc, पादेन पृष्त \msM, 
पादात्पृशेद्यः \msPaperA\Ed\oo 
\textbf{पितरं}\lem \mssALL, पित\lk\ \msCa}}% 

%Verse 18:35


\nemslokad

{\devanagarifont शूनोन्नपादः स भवेत्परत्र {॥ १८:३५॥} \veg\dontdisplaylinenum }%
     \var{{\devanagarifontvar \numnoemph\vd\textbf{शूनोन्नपादः}\lem \msNa\msPaperA\Ed, सूनोन्नपाद \mssCaCbCc\ \unmetr, सूनोनपाद \msM\oo 
\textbf{॰त्र}\lem \mssALL, ॰त्रः \msM}}% 

\ujvers\nemsloka {
{\devanagarifont पदा स्पृशेत्तोयमनादरेण }%
  \dontdisplaylinenum}    \var{{\devanagarifontvar \numemph\va\textbf{पदा स्पृशेत्तो॰}\lem \msCa\msCb\msNa\msPaperA, पादा स्पृशे तो॰ \msCc, 
\uncl{पादे} स्पृशे तो॰ \msM, 
पदात्पृशेत्तो॰ \Ed}}% 


\nemslokab

{\devanagarifont स श्लीपदी पादयुगे भवेत  \danda\dontdisplaylinenum }%
     \var{{\devanagarifontvar \numnoemph\vb\textbf{पादयुगे भवेत}\lem \msCa\msCb\msPaperA\Ed, 
पादयुगे भवेत् \msCc\msNa\ \unmetr, पाद महद्भवन्ति \msM}}% 

\nemslokac

{\devanagarifont पादेन यः स्पर्शयते हुताशं }%
  \dontdisplaylinenum    \var{{\devanagarifontvar \numnoemph\vc\textbf{पादेन यः स्पर्शयते हुताशं}\lem \msCa\msCc\msPaperA, \lac\ \msCb, 
पादेन यः स्पर्शयते हुताशनं \msNa\ \unmetr, 
\lac\ \msM, 
पादेन य स्पर्शयते हुताशं \Ed}}% 

%Verse 18:36


\nemslokad

{\devanagarifont स चाग्निपादः सततं भवेत {॥ १८:३६॥} \veg\dontdisplaylinenum }%
     \var{{\devanagarifontvar \numnoemph\vd\textbf{स चाग्निपादः सततं भवेत}\lem \msCa\msNa\msPaperA\Ed, 
स चाग्निपादः सततं भवेत् \lk\ \msCb\msCc, 
तथाग्निपादा सततम्भवन्ति \msM}}% 

\ujvers\nemsloka {
{\devanagarifont पादेन यश्चार्यमुपस्पृशेत }%
  \dontdisplaylinenum}    \var{{\devanagarifontvar \numemph\va\textbf{पादेन यश्चार्यमुपस्पृशेत}\lem \mssALL, 
\lac\ य\lk श्चार्यमु\uncl{पस्पृशेत} \msCb, 
पादेन ये चायम्मुपस्पृशन्ति \msM}}% 


\nemslokab

{\devanagarifont स पादछेदं बहुशो लभेत  \danda\dontdisplaylinenum }%
     \var{{\devanagarifontvar \numnoemph\vb\textbf{स}\lem \mssALL, ते \msM\oo 
\textbf{॰छेदं}\lem \mssALL, ॰च्छेद \msM\oo 
\textbf{लभेत}\lem \msCa\msNa\msPaperA\Ed, लभेत् \msCb\msCc\ \unmetr, भवन्ति \msM}}% 

\nemslokac

{\devanagarifont ग्रन्थापहारी स भवेत मूकः }%
  \dontdisplaylinenum    \var{{\devanagarifontvar \numnoemph\vc\textbf{भवेत}\lem \mssALL, भवे ह \msM}}% 
    \paral{{\devanagarifontsmall \vc {\englishfont \compare\ \YAJNS\ 3.211d:} मूको वागपहारकः }}

%Verse 18:37


\nemslokad

{\devanagarifont दुर्गन्धवक्त्रः परछिद्रवादी {॥ १८:३७॥} \veg\dontdisplaylinenum }%
     \var{{\devanagarifontvar \numnoemph\vd\textbf{दुर्गन्धवक्त्रः}\lem \mssALL, 
\lk र्ग्गन्धवक्त्रः \msCa, दुर्गन्धवक्त्र \msM}}% 
    \paral{{\devanagarifontsmall \vd {\englishfont \compare\ \YAJNS\ 3.212d:}
                         पूतिवक्त्रस्तु सूचकः }}

\ujvers\nemsloka {
{\devanagarifont पैशुन्यवादी स च पूतिनासो }%
  \dontdisplaylinenum}    \var{{\devanagarifontvar \numemph\va\textbf{पैशु॰}\lem \msM, पैशू॰ \mssCaCbCc\msNa\msPaperA\Ed\oo 
\textbf{॰नासो}\lem \mssCaCbCc\msNa, ॰नासा \msPaperA\Ed}}% 
    \paral{{\devanagarifontsmall \va {\englishfont \compare\ \YAJNS\ 3.212b:}
                         पिशुनः पूतिनासिकः }}


\nemslokab

{\devanagarifont नृ नम्रवक्त्रस्त्वनृतापवादी  \danda\dontdisplaylinenum }%
     \var{{\devanagarifontvar \numnoemph\vb\textbf{नृ नम्र॰}\lem \msCa\msCb\msNa, ननम्र॰ \msCc, तृ नम्र \msM, ननृम्र॰ \msPaperA, मनम्र॰ \Ed\oo 
\textbf{॰पवादी}\lem \mssALL, ॰प्रवादी \msM}}% 

\nemslokac

{\devanagarifont पारुष्यवक्ता मुखपाकरोगी }%
  \dontdisplaylinenum    \var{{\devanagarifontvar \numnoemph\vc\textbf{॰वक्ता}\lem \mssALL, ॰वक्त्रा \msM}}% 

%Verse 18:38


\nemslokad

{\devanagarifont असत्प्रलापी स च दन्तरोगः {॥ १८:३८॥} \veg\dontdisplaylinenum }%
     \var{{\devanagarifontvar \numnoemph\vd\textbf{॰रोगः}\lem \mssALL, ॰रोगीः \msCc}}% 

\ujvers\nemsloka {
{\devanagarifont तीक्ष्णप्रदायी स च वक्रनासः }%
  \dontdisplaylinenum}    \var{{\devanagarifontvar \numemph\va\textbf{तीक्ष्ण॰}\lem \mssALL, क्ष्ण॰ \msPaperA, स्तीक्ष्ण॰ \Ed\oo 
\textbf{स च}\lem \mssALL, भव \msM\oo 
\textbf{॰नासः}\lem \mssALL, ॰नास \Ed}}% 


\nemslokab

{\devanagarifont सम्भिन्नवक्ता स च कण्ठरोगी  \danda\dontdisplaylinenum }%
     \var{{\devanagarifontvar \numnoemph\vb\textbf{सम्भिन्नवक्ता स च कण्ठरोगी}\lem \mssALL, 
संभिनं वक्ता सद कण्ठरोगः \msM}}% 

\nemslokac

{\devanagarifont क्रुद्धेक्षणः पश्यति यस्तु विप्रं }%
  \dontdisplaylinenum    \var{{\devanagarifontvar \numnoemph\vc\textbf{क्रुद्धेक्षणः पश्यति यस्तु विप्रं}\lem \mssALL, 
क्रुद्धेक्षणः पश्यति यस्तु विप्रः \msCc, 
क्रोधेन यः पश्यति विप्र मूढा \msM}}% 

%Verse 18:39


\nemslokad

{\devanagarifont तीव्राक्षिरोगी स तु जायते हि {॥ १८:३९॥} \veg\dontdisplaylinenum }%
     \var{{\devanagarifontvar \numnoemph\vd\textbf{॰रोगी स तु जायते हि}\lem \mssALL, ॰रोगातुर जायतीहः \msM}}% 

\ujvers\nemsloka {
{\devanagarifont प्रद्वेषयालोकयते ऽतिथीन्य }%
  \dontdisplaylinenum}    \var{{\devanagarifontvar \numemph\va\textbf{तिथीन्य}\lem \mssALL, तिथिश्च \msM, तिथीन्य \msPaperA}}% 


\nemslokab

{\devanagarifont उत्पाटिताक्षिः स भवेत्परत्र  \danda\dontdisplaylinenum }%
     \var{{\devanagarifontvar \numnoemph\vb\textbf{उत्पाटिताक्षिः स भवेत्परत्र}\lem \mssALL, 
स चाक्षिमुत्पाटयते परत्र \msM, 
उत्पादिताक्षिः स भवेत्परत्र \msPaperA\Ed}}% 

\nemslokac

{\devanagarifont वैरूप्यचक्षुस्त्वतिसूक्ष्मचक्षुः }%
  \dontdisplaylinenum    \var{{\devanagarifontvar \numnoemph\vc\textbf{॰तिसूक्ष्म॰}\lem \mssALL, ॰निमृ\lk\ \msM\oo 
\textbf{॰चक्षुः}\lem \mssALL, ॰चक्षु \msCc}}% 

%Verse 18:40


\nemslokad

{\devanagarifont स जायते केकरपिङ्गचक्षुः {॥ १८:४०॥} \veg\dontdisplaylinenum }%
     \var{{\devanagarifontvar \numnoemph\vd\textbf{स जायते}\lem \mssALL, जायन्ति ते \msM\oo 
\textbf{केकर॰}\lem \mssALL, केककर॰ \msCc}}% 

\ujvers\nemsloka {
{\devanagarifont गर्ताक्षिकादीनि विपण्डुलानि }%
  \dontdisplaylinenum}    \var{{\devanagarifontvar \numemph\va\textbf{॰क्षिकादीनि}\lem \mssALL, ॰क्षि\uncl{सासाभि} \msM\oo 
\textbf{विपण्डुलानि}\lem \mssALL, विषण्डुलानि \msPaperA, विपाण्डुरानि \Ed}}% 


\nemslokab

{\devanagarifont नेत्रामयान्येव च पापदोषात्  \danda\dontdisplaylinenum }%
     \var{{\devanagarifontvar \numnoemph\vb\textbf{नेत्रामयान्येव च पापदोषात्}\lem \mssALL, 
नेत्रामयानेव च पापदोषात् \msCc, 
भवन्ति नेत्रामय पापदोसा \msM}}% 

\nemslokac

{\devanagarifont शृण्वन्ति ये पापकथां प्रशस्तां }%
  \dontdisplaylinenum    \var{{\devanagarifontvar \numnoemph\vc\textbf{ये}\lem \mssALL, यो \msPaperA\oo 
\textbf{॰कथां प्रशस्तां}\lem \msCb\msCc\msNa\msPaperA\Ed, ॰कथां \lac\ स्तान् \msCa, ॰कथा प्रस्तं \msM}}% 

%Verse 18:41


\nemslokad

{\devanagarifont तान्कर्णसर्पिः परिपीडयेत {॥ १८:४१॥} \veg\dontdisplaylinenum }%
     \var{{\devanagarifontvar \numnoemph\vd\textbf{तान्कर्णसर्पिः}\lem \mssALL, ता कर्णसर्पिः \msCb, स कर्ण्णसर्प्प \msM\oo 
\textbf{॰पीडयेत}\lem \msCapcorr\msCc\msNa, ॰पीडियेत \msCaacorr\Ed, 
॰पीडयेत् \msCb\ \unmetr, ॰पीडयन्ति \msM, ॰पीडयोत \msPaperA}}% 

\ujvers\nemsloka {
{\devanagarifont शृणोति निन्दां हरिशर्वयोर्यः }%
  \dontdisplaylinenum}    \var{{\devanagarifontvar \numemph\va\textbf{शृणोति निन्दां हरिशर्वयोर्यः}\lem \mssALL, 
शृण्वंति ये निन्द हरीश्वराभ्याम् \msM, 
शृण्वन्ति निन्द्रा हरिशर्व्वयोर्य्यः \msPaperA, 
शृण्वन्ति निन्दां हरिशर्वयोर्यः \Ed}}% 


\nemslokab

{\devanagarifont स कर्णशूलेन तु जीवतीव  \danda\dontdisplaylinenum }%
     \var{{\devanagarifontvar \numnoemph\vb\textbf{जीवतीव}\lem \mssALL, जीतीव \msCb, जीवनिष्टं \msM, जीवतीं वा \Ed}}% 

\nemslokac

{\devanagarifont मातापितॄणां शृणुते ऽपवादं }%
  \dontdisplaylinenum    \var{{\devanagarifontvar \numnoemph\vc\textbf{॰पितॄणां}\lem \mssALL, ॰पितॄणा \msCb, ॰पितृभ्यां \msM\oo 
\textbf{॰वादं}\lem \mssALL, ॰वादः \msM}}% 

%Verse 18:42


\nemslokad

{\devanagarifont स कर्णशोफेन विनाशमेति {॥ १८:४२॥} \veg\dontdisplaylinenum }%
     \var{{\devanagarifontvar \numnoemph\vd\textbf{शोफेन}\lem \msMacorr\msPaperA, ॰सोहेन \mssCaCbCc\msNa, शोलेन \msMpcorr, ॰साफेन \Ed\oo 
\textbf{विनाशमेति}\lem \mssALL, विनासयन्ति \msM}}% 

\ujvers\nemsloka {
{\devanagarifont शृणोति निन्दां गुरुविप्रजां यः }%
  \dontdisplaylinenum}    \var{{\devanagarifontvar \numemph\va\textbf{शृणोति निन्दां गुरुविप्रजां यः}\lem \msCc\msNa\msPaperA, 
शृणोति निन्दां गुरुविप्रजा यः \msCa\Ed, 
शृणोति निन्दा गुरुविप्रजां यः \msCb, 
शृण्वन्ति ये निन्द गुरुः द्विजा वां \msM}}% 


\nemslokab

{\devanagarifont स कर्णपूयं स्रवते सरक्तम्  \danda\dontdisplaylinenum }%
     \var{{\devanagarifontvar \numnoemph\vb\textbf{॰पूयं}\lem \mssALL, ॰पूय \msCc\msM}}% 

\nemslokac

{\devanagarifont विरूपदारिद्र्यकुलाधमेषु }%
  \dontdisplaylinenum    \var{{\devanagarifontvar \numnoemph\vc\textbf{विरूपदारिद्र्य॰}\lem \msCb, विरूपदारि\uncl{द्र्य}॰ \msCa, 
विरूप्यदारिद्र्य \msCc\msPaperA, 
विरूपदारिद्र॰ \msNa\msM, 
विरूप्यदारिध्र॰ \Ed\oo 
\textbf{॰कुलाधमेषु}\lem \mssALL, \lac\ मेषु \msCa}}% 


\nemslokab

{\devanagarifont अनिष्टकर्मभृतिजीवनं च  \danda\dontdisplaylinenum }%
     \var{{\devanagarifontvar \numnoemph\vd\textbf{॰भृतिजीवनं च}\lem \mssALL, 
॰भृतजीवनञ्च \msCc, ॰भृतिजीवनाश्च \Ed}}% 

\nemslokae

{\devanagarifont अकीर्तनं दर्शनवर्जनं च }%
  \dontdisplaylinenum    \var{{\devanagarifontvar \numnoemph\ve\textbf{अकीर्तनं}\lem \mssALL, अकीर्त्तनी \msM\oo 
\textbf{॰वर्जनं च}\lem \mssALL, ॰वर्जितञ्च \msM}}% 

%Verse 18:43


\nemslokad

{\devanagarifont श्वपाकडोम्बादिषु जायते सः {॥ १८:४३॥} \veg\dontdisplaylinenum }%
     \var{{\devanagarifontvar \numnoemph\vf\textbf{श्वपाकडोम्बादिषु}\lem \mssALL, श्वापाकतोम्बादिषु \msPaperA, श्वापाकतोश्वादिषु \Ed}}% 

\ujvers\nemsloka {
{\devanagarifont एतानि चिह्नं निरयागतानां }%
  \dontdisplaylinenum}    \var{{\devanagarifontvar \numemph\va\textbf{चिह्नं}\lem \mssALL, चिह्ना \msCb, चिंह्ना \msM\oo 
\textbf{निरयागतानां}\lem \mssALL, निररागताना \msPaperA}}% 


\nemslokab

{\devanagarifont मानुष्यलोके कुकृतस्य दृष्टम्  \danda\dontdisplaylinenum }%
     \var{{\devanagarifontvar \numnoemph\vb\textbf{मानुष्य॰}\lem \mssALL, मनुष्य॰ \msM\oo 
\textbf{दृष्टम्}\lem \mssALL, निष्ठे \msMacorr, 
दृष्ठे \msMpcorr}}% 

\nemslokac

{\devanagarifont समासतः कीर्तित एव देवि }%
  \dontdisplaylinenum    \var{{\devanagarifontvar \numnoemph\vc\textbf{कीर्तित एव}\lem \mssALL, कीर्तित एष \msNa, कीर्तितमेष \msM}}% 

%Verse 18:44


\nemslokad

{\devanagarifont यथैव मुक्तस्त्विह कर्मभङ्गः {॥ १८:४४॥} \veg\dontdisplaylinenum }%
     \var{{\devanagarifontvar \numnoemph\vd\textbf{यथैव}\lem \mssALL, यथाव \msM}}% 

\nemslokalong


\ujvers\nemsloka {
{\devanagarifont मातापित्रोघतोया सुतदुहितृवहा भ्रातृगम्भीरवेगा }%
  \dontdisplaylinenum}    \var{{\devanagarifontvar \numemph\va\textbf{घतो या}\lem \mssALL, पघातो \msNa, घतो याः \msM\oo 
\textbf{सुत॰}\lem \mssALL, सुतृ॰ \msCc\oo 
\textbf{॰वहा}\lem \mssALL, ॰वही \msM\oo 
\textbf{॰वेगा}\lem \mssALL, ॰वेगात् \msM}}% 


\nemslokab

{\devanagarifont भार्यावर्ता विवर्ता कुटिलगतिवधू बान्धवोर्मीतरङ्गा  \danda\dontdisplaylinenum }%
     \var{{\devanagarifontvar \numnoemph\vb\textbf{॰वधू बा॰}\lem \mssALL, ॰वधुर्बा॰ \msPaperA\Ed\oo 
\textbf{॰तरङ्गा}\lem \mssALL, ॰तरङ्गाः \msM}}% 

\nemslokac

{\devanagarifont कामक्रोधोभकूला करिमकरझषाग्राहकामा भयन्ते }%
  \dontdisplaylinenum    \var{{\devanagarifontvar \numnoemph\vc\textbf{॰कूला क॰}\lem \mssALL, ॰कूलात्क॰ \msM\oo 
\textbf{॰मकर॰}\lem \mssALL, ॰मरण॰ \msCa\oo 
\textbf{भयन्ते}\lem \mssALL, भषन्तः \msMacorr, हयन्ते \msPaperA}}% 

%Verse 18:45


\nemslokad

{\devanagarifont मृत्योराख्यार्णवे ऽस्मिन्न शरण विवशा कालदष्टा प्रयान्ति {॥ १८:४५॥} \veg\dontdisplaylinenum }%
     \var{{\devanagarifontvar \numnoemph\vd\textbf{स्मिन्न}\lem \mssALL, स्मि\uncl{न्सु}॰ \msM\oo 
\textbf{॰दष्टा}\lem \msCa, ॰दुष्टा \msCb, ॰दुष्ट \msCc, ॰द्रष्ट्रा \msNa, ॰दष्ट \msM, ॰दृष्टा \msPaperA, ॰दृष्टो \Ed\oo 
\textbf{॰यान्ति}\lem \eme, ॰याति \mssCaCbCc\msNa\msM\msPaperA\Ed}}% 

\ujvers\nemsloka {
{\devanagarifont नित्यं येन विनाश याति दिवसं पञ्चत्वमापद्यते }%
  \dontdisplaylinenum}    \var{{\devanagarifontvar \numemph\va\textbf{विनाश याति}\lem \msM, विना न याति \msCa\msCc\msNa\msPaperA\Ed, विना\uncl{स याति} \msCb}}% 


\nemslokab

{\devanagarifont त्यक्त्वा देह वनान्तरेषु विषमे श्वानशृगालाकुले  \danda\dontdisplaylinenum }%
     \var{{\devanagarifontvar \numnoemph\vb\textbf{श्वान॰}\lem \mssALL, श्वानः \msM}}% 

\nemslokac

{\devanagarifont बन्धुः सर्व निवर्तते गतदया धर्मैक तत्र स्थितस् }%
  \dontdisplaylinenum    \var{{\devanagarifontvar \numnoemph\vc\textbf{बन्धुः}\lem \mssALL, बन्धु \msM\ \unmetr\oo 
\textbf{॰दया धर्मैक}\lem \msM\msPaperA\Ed, ॰दया धर्मैकस् \msCa\ \unmetr, 
॰दयो धर्मैकस् \msCb\msNa\ \unmetr}}% 
    \lacuna{\devanagarifontsmall \vc {\englishfont \msCc\ breaks down after reading} सर्व्व {\englishfont and resumes with}
                णेषु च सर्वेषु {\englishfont at 19.53a} }%
  
%Verse 18:46


\nemslokad

{\devanagarifont तस्माद्धर्मपरो न चान्यसुहृदः सेवेत्परत्रार्थिनः {॥ १८:४६॥} \veg\dontdisplaylinenum }%
     \var{{\devanagarifontvar \numnoemph\vd\textbf{चान्य॰}\lem \eme, चान्यः \msCa\msCb\msNa\msM\msPaperA\Ed\ \unmetr\oo 
\textbf{सुहृदः सेवेत्प॰}\lem \mssALL, सुहृतः सेवे प॰ \msM, सुहृदः सवस्प॰ \msPaperA}}% 

\vers


{\devanagarifont 
\jump
\begin{center}
\ketdanda~इति वृषसारसंग्रहे पूर्वकर्मविपाकचिह्नाष्टादशमो ऽध्यायः~\ketdanda
\end{center}
\dontdisplaylinenum\vers  }%
     \var{{\devanagarifontvar \numnoemph{\englishfont \Colo:}\textbf{॰विपाकचिह्नाष्टादशमो ऽध्यायः}\lem \msCa\msCb\msNa\msPaperA, 
॰चिंह्नाध्यायः अष्टादशमः \msM, 
॰विपाकचिह्नाष्टादशो ऽध्यायः \Ed}}% 
\bekveg\szamveg
\vfill
\phpspagebreak

\versno=0\fejno=19
\thispagestyle{empty}

\centerline{\Large\devanagarifontbold [   एकोनविंशतिमो ऽध्यायः  ]}{\vrule depth10pt width0pt} \fancyhead[CE]{{\footnotesize\devanagarifont वृषसारसंग्रहे  }}
\fancyhead[CO]{{\footnotesize\devanagarifont एकोनविंशतिमो ऽध्यायः  }}
\fancyhead[LE]{}
\fancyhead[RE]{}
\fancyhead[LO]{}
\fancyhead[RO]{}
\szam\bek



\alalfejezet{गावः}
\vers


{\devanagarifont विगतराग उवाच {\dandab}\dontdisplaylinenum  }%
 
{\devanagarifont क्रियासूक्ष्मो महाधर्मः कर्मणा केन प्राप्यते \thinspace{\danda} \dontdisplaylinenum }%
 
%Verse 19:1

{\devanagarifont अल्पोपायं नरार्थाय पृच्छामि कथयस्व मे {॥ १९:१॥} \veg\dontdisplaylinenum }%
 
{\devanagarifont अनर्थयज्ञ उवाच {\dandab}\dontdisplaylinenum  }%
 
{\devanagarifont अल्पोपायं महाधर्मं कथयामि द्विजोत्तम \thinspace{\danda} \dontdisplaylinenum }%
     \var{{\devanagarifontvar \numemph\va\textbf{॰धर्मं}\lem \msCa\msCb\Ed, ॰धर्म \msNa}}% 

%Verse 19:2

{\devanagarifont सुखेन लभते स्वर्गं कर्मणा येन तच्छृणु {॥ १९:२॥} \veg\dontdisplaylinenum }%
 
{\devanagarifont लोकानां मातरो गावो गोभिः सर्वं जगद्धृतम् \thinspace{\dandab} \dontdisplaylinenum }%
     \var{{\devanagarifontvar \numemph\vb\textbf{गोभिः}\lem \msCa\msNa\Ed, भिः \msCb}}% 

%Verse 19:3

{\devanagarifont गोमयममृतं सर्वं जातं सर्वं शिवेच्छया {॥ १९:३॥} \veg\dontdisplaylinenum }%
     \var{{\devanagarifontvar \numnoemph\vd\textbf{सर्वं}\lem \msCa\msCb, सर्व॰ \msNa\Ed}}% 

{\devanagarifont सर्वदेवमया गावः सर्वदेवमयो द्विजः \thinspace{\dandab} \dontdisplaylinenum }%
     \var{{\devanagarifontvar \numemph\va\textbf{॰मया}\lem \msCa\msCb\msNa, ॰मयी \Ed}}% 

%Verse 19:4

{\devanagarifont सर्वदेवमयी भूमिः सर्वदेवमयः शिवः {॥ १९:४॥} \veg\dontdisplaylinenum }%
     \var{{\devanagarifontvar \numnoemph\vc\textbf{॰मयी भूमिः}\lem \msCa, ॰मयो भूमि \msCb, ॰मयी भूमि \msNa, ॰मयो भूमिः \Ed}}% 
    \var{{\devanagarifontvar \numnoemph\vd\textbf{॰मयः}\lem \msCa\msNa\Ed, ॰मय \msCb}}% 

{\devanagarifont तस्माद्गावः सदा सेव्या धर्ममोक्षार्थसिद्धिदाः \thinspace{\dandab} \dontdisplaylinenum }%
     \var{{\devanagarifontvar \numemph\vb\textbf{॰दाः}\lem \msCa\msCb\msNa, ॰दा \Ed}}% 

%Verse 19:5

{\devanagarifont परिचर्या यथाशक्त्या ग्रासवासजलादिभिः {॥ १९:५॥} \veg\dontdisplaylinenum }%
 
{\devanagarifont ताडयेन्नातिवेगेन वाचयेन्मृदुनाचरेत् \thinspace{\dandab} \dontdisplaylinenum }%
 
%Verse 19:6

{\devanagarifont पालयेत घनाढ्येषु भग्नोद्विग्नेषु यत्नतः {॥ १९:६॥} \veg\dontdisplaylinenum }%
     \var{{\devanagarifontvar \numemph\vc\textbf{पालयेत घनाढ्येषु}\lem \msCa\msCb\msNa, पालयन्तर्प्पनाद्येषु \Ed}}% 
    \var{{\devanagarifontvar \numnoemph\vd\textbf{॰द्विग्नेषु}\lem \msCa\msNa\Ed, ॰द्विग्नोद्विग्नेषु \msCb}}% 

{\devanagarifont व्याधिव्रणपरिक्लेश ओषधोपक्रमं चरेत् \thinspace{\dandab} \dontdisplaylinenum }%
     \var{{\devanagarifontvar \numemph\va\textbf{॰व्रण॰}\lem \msCa\msCb\msNa, ॰वन॰ \Ed\oo 
\textbf{॰क्लेश}\lem \msCa\msNa\Ed, ॰क्लेशे \msCb}}% 
    \var{{\devanagarifontvar \numnoemph\vb\textbf{॰क्रमं च}\lem \msCa\msCb\msNa, ॰क्रमश्च॰ \Ed}}% 

%Verse 19:7

{\devanagarifont कण्डूयनं च कर्तव्यं यथासौख्यं भवेद्गवाम् {॥ १९:७॥} \veg\dontdisplaylinenum }%
 
{\devanagarifont गवां प्रदक्षिणं कृत्वा श्रद्धाभक्तिसमन्वितः \thinspace{\dandab} \dontdisplaylinenum }%
     \var{{\devanagarifontvar \numemph\vb\textbf{॰न्वितः}\lem \msNa\Ed, ॰न्वित\lk\ \msCa, ॰न्वितं \msCb}}% 

%Verse 19:8

{\devanagarifont सागरान्ता मही सर्वा प्रदक्षिणीकृता भवेत् {॥ १९:८॥} \veg\dontdisplaylinenum }%
     \var{{\devanagarifontvar \numnoemph\vcd\textbf{सर्वा प्र}\lem \msCa\msCb\msNa, सर्वान्प्र \Ed}}% 

{\devanagarifont स्पृष्टसंस्पर्शनाद्ये च श्रद्धया यदि मानवः \thinspace{\dandab} \dontdisplaylinenum }%
     \var{{\devanagarifontvar \numemph\va\textbf{स्पृष्टसंस्पर्शनाद्ये च}\lem \msCa\msCb\msNa, पृष्टसंस्पर्शनाद्यञ्च \Ed}}% 

%Verse 19:9

{\devanagarifont अहोरात्रकृतं पापं नश्यते नात्र संशयः {॥ १९:९॥} \veg\dontdisplaylinenum }%
 
{\devanagarifont लाङ्गूलेनोद्धृतं तोयं मूर्ध्ना गृह्णाति यो नरः \thinspace{\dandab} \dontdisplaylinenum }%
     \var{{\devanagarifontvar \numemph\va\textbf{लाङ्गू॰}\lem \msCa\msNa\Ed, लाङ्गु॰ \msCb\oo 
\textbf{तोयं}\lem \msCa\msCbpcorr\msNa\Ed, तोद्धृतं \msCbacorr}}% 
    \var{{\devanagarifontvar \numnoemph\vb\textbf{गृह्णाति यो नरः}\lem \msCa\msNa\Ed, गृह्णन्ति यो नराः \msCb}}% 

%Verse 19:10

{\devanagarifont यावज्जीवकृतं पापं नश्यते नात्र संशयः {॥ १९:१०॥} \veg\dontdisplaylinenum }%
 
{\devanagarifont विधिवत्स्नापयेद्गांश्च मन्त्रयुक्तेन वारिणा \thinspace{\dandab} \dontdisplaylinenum }%
     \var{{\devanagarifontvar \numemph\va\textbf{विधिवत्स्नापयेद्गां च}\lem \msNa, विधिवच्छापयेद्गां च \msCa, 
विधिवत्स्हापये\uncl{द्गाञ्च} \msCb, 
विधिवत्स्नापयेद्गांश्च \Ed}}% 

%Verse 19:11

{\devanagarifont तेनाम्भसा स्वयं स्नात्वा सर्वपापक्षयो भवेत् {॥ १९:११॥} \veg\dontdisplaylinenum }%
 
{\devanagarifont व्याधिर्विघ्नमलक्ष्मीत्वं नश्यते सद्य एव च \thinspace{\dandab} \dontdisplaylinenum }%
     \var{{\devanagarifontvar \numemph\va\textbf{व्याधिर्वि॰}\lem \msCa\msCb, व्याधिवि॰ \msNa\Ed}}% 

%Verse 19:12

{\devanagarifont मृतापत्यानपत्याश्च स्नानमेव प्रशस्यते {॥ १९:१२॥} \veg\dontdisplaylinenum }%
     \var{{\devanagarifontvar \numnoemph\vc\textbf{मृतापत्यानपत्याश्च}\lem \msCa\msCb\msNapcorr, मृत्यपत्यानपत्याश्च \msNaacorr, 
मृतापत्याश्च गावाश्च \Ed}}% 
    \var{{\devanagarifontvar \numnoemph\vd\textbf{॰व प्र॰}\lem \msCa\msNa\Ed, ॰वम्प्र॰ \msCb}}% 

{\devanagarifont गवां शृङ्गोदकं गृह्य मूर्ध्नि यो धारयेन्नरः \thinspace{\dandab} \dontdisplaylinenum }%
     \var{{\devanagarifontvar \numemph\va\textbf{॰दकं}\lem \msCa\msCb\Ed, ॰दक \msNa}}% 

%Verse 19:13

{\devanagarifont स सर्वतीर्थस्नानस्य फलं प्राप्नोति मानवः {॥ १९:१३॥} \veg\dontdisplaylinenum }%
     \var{{\devanagarifontvar \numnoemph\vc\textbf{॰स्नानस्य}\lem \msCa\msNapcorr\Ed, ॰स्थानस्य \msCb, ॰स्नान \msNaacorr}}% 
    \var{{\devanagarifontvar \numnoemph\vd\textbf{फलं}\lem \msCa\msNa\Ed, फलाम् \msCb}}% 

{\devanagarifont ग्रासमुष्टिप्रदानेन गोषु भक्तिसमन्वितः \thinspace{\dandab} \dontdisplaylinenum }%
     \var{{\devanagarifontvar \numemph\vb\textbf{भक्तिसमन्वितः}\lem \msCa\msNa\Ed, भक्तितः \msCb}}% 

%Verse 19:14

{\devanagarifont अग्निहोत्रं हुतं तेन सर्वदेवाः सुतर्पिताः {॥ १९:१४॥} \veg\dontdisplaylinenum }%
     \var{{\devanagarifontvar \numnoemph\vc\textbf{हुतं}\lem \msCa\msCb\Ed, फलं \msNa}}% 

{\devanagarifont चत्वारः स्तनधारास्तु यस्तु मूर्ध्ना प्रतीच्छति \thinspace{\dandab} \dontdisplaylinenum }%
     \var{{\devanagarifontvar \numemph\va\textbf{स्तन॰}\lem \msCa\msNa\Ed, सूत॰ \msCb}}% 

%Verse 19:15

{\devanagarifont स चतुःसागरं गत्वा स्नानपुण्यफलं लभेत् {॥ १९:१५॥} \veg\dontdisplaylinenum }%
 
{\devanagarifont गवार्थं यस्त्यजेत्प्राणान्गोग्रहेषु द्विजोत्तम \thinspace{\dandab} \dontdisplaylinenum }%
 
%Verse 19:16

{\devanagarifont कल्पकोटिशतं दिव्यं शिवलोके महीयते {॥ १९:१६॥} \veg\dontdisplaylinenum }%
 
{\devanagarifont च्युतभग्नादिसंस्कारं सर्वं यः कुरुते नरः \thinspace{\dandab} \dontdisplaylinenum }%
 
%Verse 19:17

{\devanagarifont भार्याकोटिशतं दानं यत्फलं परिकीर्तितम् {॥ १९:१७॥} \veg\dontdisplaylinenum  }%
     \var{{\devanagarifontvar \numemph\vc\textbf{भार्या॰}\lem \msNa\Ed, आर्या \msCa\msCb}}% 

{\devanagarifont तत्फलं लभते मर्त्यः शिवलोकं च गच्छति \thinspace{\dandab} \dontdisplaylinenum }%
     \var{{\devanagarifontvar \numemph\va\textbf{तत्फलं लभते मर्त्यः}\lem \msCa\msNa\Ed, 
\uncl{तत्फलम्परिकीर्त्तिं त्यंः} \msCb}}% 
    \var{{\devanagarifontvar \numnoemph\vb\textbf{॰लोकं च गच्छति}\lem \msCa\Ed, ॰लो\lk\lk\lk च्छति \msCb, ॰लोके वगच्छति \msNa}}% 

%Verse 19:18

{\devanagarifont शिवलोकपरिभ्रष्टः पृथिव्यामेकराड्भवेत् {॥ १९:१८॥} \veg\dontdisplaylinenum }%
 
{\devanagarifont समासतः समाख्यातं यथातत्त्वं द्विजोत्तम \thinspace{\dandab} \dontdisplaylinenum }%
     \var{{\devanagarifontvar \numemph\vb\textbf{॰तत्त्वं}\lem \msCa\msNa\Ed, ॰तत्व \msCb}}% 

%Verse 19:19

{\devanagarifont न शक्यं विस्तराद्वक्तुं गोमहाभाग्यमुत्तमम् {॥ १९:१९॥} \veg\dontdisplaylinenum }%
     \var{{\devanagarifontvar \numnoemph\vc\textbf{विस्तराद्व॰}\lem \Ed, विस्तरान्व॰ \msCa, विस्तरात्व॰ \msCb,  विस्तरां व॰ \msNa}}% 
    \var{{\devanagarifontvar \numnoemph\vd\textbf{गोमहाभाग्य॰}\lem \msCa\msNa, गोमहाग्य॰ \msCb\ \hypometr, गोमहात्म्यस॰ \Ed}}% 


\alalfejezet{चातुर्वर्ण्यम्}
{\devanagarifont विगतराग उवाच {\dandab}\dontdisplaylinenum  }%
 
{\devanagarifont देवा अष्टविधाः प्रोक्तास्तिर्यक्पञ्चविधः स्मृतः \thinspace{\danda} \dontdisplaylinenum }%
     \var{{\devanagarifontvar \numemph\va\textbf{देवा अष्ट॰}\lem \mssCaCbCc\msNa\msNb\msNc, देवाःरष्ट॰ \Ed}}% 
    \var{{\devanagarifontvar \numnoemph\vb\textbf{॰र्यक्प॰}\lem \msCapcorr\msCb\msNa\msNb\msNc\Ed, ॰क्प॰ \msCaacorr\oo 
\textbf{स्मृतः}\lem \msCa\msNa\msNb\msNc\Ed, स्मृताः \msCb}}% 

%Verse 19:20

{\devanagarifont मानुषमेकमेवाहुश्चातुर्वर्णः कथं भवेत् {॥ १९:२०॥} \veg\dontdisplaylinenum }%
     \var{{\devanagarifontvar \numnoemph\vc\textbf{मानुष॰}\lem \mssCaCbCc\msNa\msNb\msNc, मानुष्य॰ \Ed}}% 
    \var{{\devanagarifontvar \numnoemph\vd\textbf{॰वर्णः}\lem \msCa\msCb\msNa\msNb, ॰वर्ण्ण \msNc, ॰व्वर्ण्यः \Ed}}% 

{\devanagarifont अनर्थयज्ञ उवाच {\dandab}\dontdisplaylinenum  }%
 
{\devanagarifont पूर्वकल्पसृजस्त्वेष विष्णुना प्रभविष्णुना \thinspace{\danda} \dontdisplaylinenum }%
     \var{{\devanagarifontvar \numemph\va\textbf{॰सृजस्त्वेष}\lem \msCa\msNa\msNb\msNc, ॰सृज\lk\lk\ \msCb, ॰सृजत्येष \Ed}}% 

%Verse 19:21

{\devanagarifont एकवर्णो द्विजश्चासीत्सर्वकल्पाग्रमग्रतः {॥ १९:२१॥} \veg\dontdisplaylinenum }%
     \var{{\devanagarifontvar \numnoemph\vc\textbf{एकवर्णो द्विजश्चासी॰}\lem \msNc, ए{\il}{\il}\uncl{र्ण्णो}{\il}{\il}{\il}श्चासी॰ \msCa, 
एकवर्ण्णा द्विजश्चासी॰ \msCb, 
एकवर्ण्ण द्विजश्चासी॰ \msNb, 
एकव\uncl{र्ण्णो} द्विजश्चासी॰ \msNa, एवं वर्णा द्विजश्चासी॰ \Ed}}% 

{\devanagarifont सर्ववेदविदो विप्राः सर्वयज्ञविदस्तथा \thinspace{\dandab} \dontdisplaylinenum }%
     \var{{\devanagarifontvar \numemph\vb\textbf{॰यज्ञ॰}\lem \msCa\msCb\msNa\msNb\msNc, ॰वेद॰ \Ed}}% 

%Verse 19:22

{\devanagarifont तेषां विप्रसहस्राणां यज्ञोत्साहमनो भवेत् {॥ १९:२२॥} \veg\dontdisplaylinenum }%
     \var{{\devanagarifontvar \numnoemph\vc\textbf{तेषां}\lem \msCa\msCb\msNa\msNc, तथा \Ed}}% 
    \var{{\devanagarifontvar \numnoemph\vd\textbf{यज्ञोत्सा॰}\lem \msCa\msCb\msNa\Ed, यज्ञोच्छाह॰ \msNb, यज्ञोतसाह॰ \msNc\ \unmetr}}% 

{\devanagarifont वृद्धविप्रसहस्राणां मतमाज्ञाय ब्राह्मणैः \thinspace{\dandab} \dontdisplaylinenum }%
     \var{{\devanagarifontvar \numemph\vb\textbf{॰ज्ञाय}\lem \msCb\msNa\msNb\msNc, ॰\uncl{ज्ञाय} \msCa, ॰श्रित्य \Ed\oo 
\textbf{ब्राह्मणैः}\lem \msCa\msNa\msNb\msNc\Ed, ब्राह्मणेः \msCb}}% 

%Verse 19:23

{\devanagarifont कर्तुं कर्म समारब्धं कर्म चापि विभज्यते {॥ १९:२३॥} \veg\dontdisplaylinenum }%
     \var{{\devanagarifontvar \numnoemph\vc\textbf{कर्तुं}\lem \msCa\msCb\msNc\Ed, कर्तु \msNa\msNb\oo 
\textbf{समारब्धं}\lem \msCa\msNa\msNb\msNc, समारन्धं \msCb, समारब्ध \Ed}}% 
    \var{{\devanagarifontvar \numnoemph\vd\textbf{कर्म चापि}\lem \msCa\msCb\msNb\msNc, कर्मं चापि \msNa, कर्मश्चापि \Ed}}% 

{\devanagarifont ऋत्विजत्वे स्थिताः केचित्केचित्संरक्षणे स्थिताः \thinspace{\dandab} \dontdisplaylinenum }%
     \var{{\devanagarifontvar \numemph\vb\textbf{॰रक्षणे स्थिताः}\lem \msNa\msNb\msNc\Ed, ॰रक्ष\lac\  \msCa, \uncl{रक्षणे स्थि}\lk\ \msCb}}% 

%Verse 19:24

{\devanagarifont अर्थोपार्जनयुक्तान्ये अन्ये शिल्पे नियोजिताः {॥ १९:२४॥} \veg\dontdisplaylinenum }%
     \var{{\devanagarifontvar \numnoemph\vc\textbf{॰युक्तान्ये}\lem \msCa\msNb\msNc\Ed, ॰युक्ता\uncl{न्ये} \msCb, ॰युक्तात्मे \msNa}}% 
    \var{{\devanagarifontvar \numnoemph\vd\textbf{अन्ये}\lem \msCapcorr\msCb\msNa\msNb\msNc\Ed, \om\ \msCaacorr}}% 

{\devanagarifont एवं यज्ञविधानेन कर्तुमारेभिरे पुरा \thinspace{\dandab} \dontdisplaylinenum }%
     \var{{\devanagarifontvar \numemph\vb\textbf{॰मारेभिरे}\lem \msCa\msCb\msNb\msNc\Ed, ॰म आरेमिरे \msNa}}% 

%Verse 19:25

{\devanagarifont यथोद्दिष्टेन कर्मेण यज्ञोत्साहमवर्तत {॥ १९:२५॥} \veg\dontdisplaylinenum }%
     \var{{\devanagarifontvar \numnoemph\vc\textbf{यथोद्दि॰}\lem \Ed, यथोदि॰ \msCa\msCb\msNa\msNb\msNc}}% 
    \var{{\devanagarifontvar \numnoemph\vd\textbf{॰वर्तत}\lem \msCa\msCb\msNa\msNb\Ed, ॰वर्त्ततः \msNc}}% 

{\devanagarifont आगता ऋषयः सर्वे देवताः पितरस्तथा \thinspace{\dandab} \dontdisplaylinenum }%
 
%Verse 19:26

{\devanagarifont अन्योन्यमब्रुवन्तत्र देवर्षिपितृदेवताः {॥ १९:२६॥} \veg\dontdisplaylinenum }%
 
{\devanagarifont यज्ञार्थमसृजद्वर्णं विधिना क्रतुहेतवः \thinspace{\dandab} \dontdisplaylinenum }%
     \var{{\devanagarifontvar \numemph\vab\textbf{॰सृजद्वर्णं वि॰}\lem \msCa\msNb\msNc\Ed, ॰सृजद्वर्ण्णान्वि॰ \msCb, ॰सृद्वण्णन्विवि॰ \msNa}}% 
    \var{{\devanagarifontvar \numnoemph\vb\textbf{॰धिना}\lem \msCa\msCb\msNa\msNc\Ed, ॰धानां \msNb\oo 
\textbf{क्रतुहेतवः}\lem \msNa\msNb\msNc, \uncl{क्रतु}\lac  तवः \msCa, क्रतुहेतु\uncl{त} \msCb, 
पातुहेतवः \Ed}}% 

%Verse 19:27

{\devanagarifont एवमेव प्रवर्तन्तु भवद्भिर्द्विजसत्तमाः {॥ १९:२७॥} \veg\dontdisplaylinenum }%
     \var{{\devanagarifontvar \numnoemph\vd\textbf{भवद्भिर्द्वि॰}\lem \msCapcorr\msCb, भगवद्भिर्द्वि॰ \msCaacorr, भवद्भि द्वि॰ \msNa\msNb\msNc, 
भवतिद्वि॰ \Ed\oo 
\textbf{॰सत्तमाः}\lem \msCa\msCb\msNa\msNc\Ed, ॰सत्तमः \msNb}}% 

{\devanagarifont इज्याध्ययनसम्पन्ना ब्राह्मणा ये ऽत्र कल्पिताः \thinspace{\dandab} \dontdisplaylinenum }%
     \var{{\devanagarifontvar \numemph\vb\textbf{ये ऽत्र}\lem \msCa\msCb\msNa\msNb\msNc, यत्र \Ed\oo 
\textbf{कल्पिताः}\lem \msCa\msCb\msNa\msNc\Ed, कल्पिता \msNb}}% 

%Verse 19:28

{\devanagarifont सुविप्रा विप्रतां यान्तु षट्कर्मनिरताः सदा {॥ १९:२८॥} \veg\dontdisplaylinenum }%
     \var{{\devanagarifontvar \numnoemph\vc\textbf{सुविप्रा}\lem \msCa\msCb\msNa\msNc\Ed, सुविप्र \msNb\oo 
\textbf{यान्तु}\lem \msCa\msCb\msNa\msNc\Ed, यातु \msNb}}% 
    \var{{\devanagarifontvar \numnoemph\vd\textbf{षट्कर्म॰}\lem \msCa\msCb\msNa\msNb\msNc, षड्कर्मा॰ \Ed\oo 
\textbf{सदा}\lem \msCa\msCb\msNa\Ed, सजा \msNb, सदाः \msNc}}% 

{\devanagarifont रक्षणार्थं तु ये विप्राः कल्पिताः शस्त्रपाणयः \thinspace{\dandab} \dontdisplaylinenum }%
     \var{{\devanagarifontvar \numemph\va\textbf{विप्राः}\lem \msCa\msNa\msNb\msNc\Ed, विप्रा \msCb}}% 
    \var{{\devanagarifontvar \numnoemph\vb\textbf{शस्त्र॰}\lem \msCa\msNa\msNc\Ed, शास्त्र॰ \msCb\msNb}}% 

%Verse 19:29

{\devanagarifont क्षतत्राणाय विप्राणां नित्यक्षत्रव्रतोद्भवाः {॥ १९:२९॥} \veg\dontdisplaylinenum }%
     \var{{\devanagarifontvar \numnoemph\vc\textbf{क्षत॰}\lem \msCa\msNa\msNb, क्षत्र॰ \msCb\msNc, कृत॰ \Ed\oo 
\textbf{विप्राणां}\lem \msCa\msCb\msNa\Ed, विप्राणा \msNb}}% 
    \var{{\devanagarifontvar \numnoemph\vd\textbf{नित्यक्षत्र॰}\lem \msCa\msCb\msNc, नित्यं क्षत्र॰ \msNa\msNb, नित्यं क्षात्र॰ \Ed\oo 
\textbf{॰व्रतोद्भवाः}\lem \msCa\msCb\msNa\msNc\Ed, ॰व्रतोत्तमः \msNb}}% 
    \paral{{\devanagarifontsmall \vcd {\englishfont cf.\ MBh 12.59.128ab:}
                  ब्राह्मणानां क्षतत्राणात्ततः क्षत्रिय उच्यते }}

{\devanagarifont अर्थोपार्जनमुद्दिश्य कल्पिता ये द्विजातयः \thinspace{\dandab} \dontdisplaylinenum }%
     \var{{\devanagarifontvar \numemph\vb\textbf{ये}\lem \msCa\msCb\msNa\msNc\Ed, यो \msNb}}% 

%Verse 19:30

{\devanagarifont ते तु वैश्यत्वमायान्तु वार्त्तोपायरतोद्भवाः {॥ १९:३०॥} \veg\dontdisplaylinenum }%
     \var{{\devanagarifontvar \numnoemph\vd\textbf{वार्त्तोपायरतो॰}\lem \msCa\msCb\msNa\msNb\msNc, वार्त्तो आपणतोद्भवाः \Ed}}% 

{\devanagarifont वधबन्धनकर्मसु शिल्पस्थानविधेषु च \thinspace{\dandab} \dontdisplaylinenum }%
     \var{{\devanagarifontvar \numemph\va\textbf{वधबन्धनकर्मसु}\lem \msCa\msNa, वधवन्धनकर्मेषु \msCb\msNb\msNc\Ed}}% 
    \var{{\devanagarifontvar \numnoemph\vb\textbf{॰विधेषु}\lem \msCa\msCb\msNa\msNb\msNc, ॰वधेषु \Ed}}% 

%Verse 19:31

{\devanagarifont कल्पिता ये द्विजातीनां सर्वे शूद्रा भवन्तु ते {॥ १९:३१॥} \veg\dontdisplaylinenum }%
     \var{{\devanagarifontvar \numnoemph\vc\textbf{॰जातीनां}\lem \msCa\msCb\msNa\msNb\Ed, ॰जातीना \msNc}}% 

{\devanagarifont प्राजापत्यं ब्राह्मणानामिज्याध्ययनतत्परात् \thinspace{\dandab} \dontdisplaylinenum }%
     \var{{\devanagarifontvar \numemph\vab\textbf{प्राजापत्यं ब्राह्मणानामिज्याध्ययनतत्परात्}\lem \msCa\msCb, 
प्राजापत्यं ब्राह्मणानांमिज्याध्ययनतत्परात् \msNa, 
प्राजापत्य  ब्राह्मणामीज्याध्ययनतत्परात् \msNb, 
प्राजापत्यं ब्राह्मणामीज्याध्ययनतत्परात् \msNc, 
प्राजापत्यं ब्राह्मणानामीज्याध्ययनतत्परां \Ed}}% 

%Verse 19:32

{\devanagarifont स्थानमैन्द्रं क्षत्रियाणां प्रजापालनतत्परात् {॥ १९:३२॥} \veg\dontdisplaylinenum }%
     \var{{\devanagarifontvar \numnoemph\vc\textbf{॰न्द्रं}\lem \msCa\msCb\msNa\Ed, ॰न्द्र \msNb\msNc}}% 
    \var{{\devanagarifontvar \numnoemph\vd\textbf{॰त्परात्}\lem \msCb\msNa\msNb\msNc, ॰\uncl{त्परात्} \msCa, ॰त्परं \Ed}}% 
    \paral{{\devanagarifontsmall \vo {\englishfont cf.\ Vāyupurāṇa 1.8.166:}
                 प्राजापत्यं ब्राह्मणानां स्मृतं स्थानं क्रियावताम्\thinspace{\devanagarifontsmall ।}
                 स्थानम् ऐन्द्रं क्षत्रियाणां संग्रामेष्वपलायिनाम्\thinspace{\devanagarifontsmall ॥}
                    {\englishfont \similar\ Bhaviṣyapurāṇa 2.1.34, etc.} }}

{\devanagarifont वैश्यानां वासवस्थानं वाणिज्यकृषिजीविनाम् \thinspace{\dandab} \dontdisplaylinenum }%
     \var{{\devanagarifontvar \numemph\vb\textbf{वाणिज्य॰}\lem \msCa\msCb\msNb, वाणिज्यं \msNa\Ed\oo 
\textbf{॰जीविनाम्}\lem \msCa\msCb\msNa\msNc\Ed, ॰जीविनम् \msNb}}% 

%Verse 19:33

{\devanagarifont शूद्राणां मरुतः स्थानं शुश्रूषानिरतात्मनाम् {॥ १९:३३॥} \veg\dontdisplaylinenum }%
     \var{{\devanagarifontvar \numnoemph\vcd\textbf{शूद्राणां मरुतः स्थानं शुश्रूषानिरतात्मनाम्}\lem \msCa\msCb\msNc\Ed, \om\ \msNaacorr, 
शूद्राणां मरुतस्थानं शुश्रूषानिरतात्मनाम् \msNapcorr 
शूद्राणा मरुतः स्थानं शुश्रूषानिरतात्मनाम् \msNb}}% 
    \paral{{\devanagarifontsmall \vo {\englishfont cf.\ Vāyupurāṇa 1.8.167--168ab:}
                 वैश्यानां मारुतं स्थानं स्वधर्ममुपजीविनाम्\thinspace{\devanagarifontsmall ।} 
                 गान्धर्वं शूद्रजातीनां प्रतिचारेण तिष्ठताम्\thinspace{\devanagarifontsmall ॥}
                 स्थानान्येतानि वर्णानां व्यत्याचारवतां स्वयम्\thinspace{\devanagarifontsmall ।} }}

{\devanagarifont महर्षिपितृदेवानां मतमाज्ञाय निश्चितः \thinspace{\dandab} \dontdisplaylinenum }%
     \var{{\devanagarifontvar \numemph\va\textbf{॰देवानां म॰}\lem \msCa\msNa\msNb\Ed, ॰देवाना म॰ \msCb}}% 
    \var{{\devanagarifontvar \numnoemph\vb\textbf{मत॰}\lem \msCa\msCb\msNa\msNb\Ed, मन॰ \msNc\oo 
\textbf{निश्चितः}\lem \msCa\msCb\msNb\msNc\Ed, निश्चिताः \msNa}}% 

%Verse 19:34

{\devanagarifont एष संकल्पितो ब्रह्मा पद्मयोनिः पितामहः {॥ १९:३४॥} \veg\dontdisplaylinenum }%
     \var{{\devanagarifontvar \numnoemph\vc\textbf{॰कल्पितो ब्रह्मा}\lem \msCa\msCb\msNa\Ed, ॰कल्पिता ब्राह्मा \msNb}}% 

{\devanagarifont संकल्पप्रभवाः सर्वे देवदानवमानवाः \thinspace{\dandab} \dontdisplaylinenum }%
     \var{{\devanagarifontvar \numemph\vb\textbf{देवदानवमानवाः}\lem \msCa\msCb\msNa\msNc\Ed, देवदेदानमानवः \msNbacorr, 
देवदानमानवः \msNbpcorr}}% 

%Verse 19:35

{\devanagarifont पशुपक्षिमृगा मुख्या यावन्ति जगसम्भवाः {॥ १९:३५॥} \veg\dontdisplaylinenum }%
     \var{{\devanagarifontvar \numnoemph\vc\textbf{॰मृगा}\lem \msCb\msNa\msNb\Ed, \lk गा \msCa}}% 
    \var{{\devanagarifontvar \numnoemph\vd\textbf{जग॰}\lem \msCa\msCb\Ed, जंगम॰ \msNa, जगे \msNb}}% 

{\devanagarifont भूतसंकल्पकं नाम कल्पमासीद्द्विजोत्तम \thinspace{\dandab} \dontdisplaylinenum }%
     \var{{\devanagarifontvar \numemph\va\textbf{भूतसंकल्पकं नाम}\lem \msCa\msCb\msNa\msNb\msNc, भूतसंकल्पकर्ता य \Ed}}% 

%Verse 19:36

{\devanagarifont कीर्तितानि समासेन किमन्यच्छ्रोतुमिच्छसि {॥ १९:३६॥} \veg\dontdisplaylinenum }%
 
{\devanagarifont विगतराग उवाच {\dandab}\dontdisplaylinenum  }%
 
{\devanagarifont किं तपः सर्ववर्णानां वृत्तिं वापि तपोधन \thinspace{\danda} \dontdisplaylinenum }%
     \var{{\devanagarifontvar \numemph\vb\textbf{वृत्तिं वापि}\lem \msCa\msCb\msNa, वृत्तिर्व्वापि \Ed}}% 

%Verse 19:37

{\devanagarifont यज्ञांश्चैव पृथक्त्वेन श्रोतुमिच्छामि तत्त्वतः {॥ १९:३७॥} \veg\dontdisplaylinenum }%
     \var{{\devanagarifontvar \numnoemph\vc\textbf{यज्ञांश्चैव}\lem \msCa\msNa, यज्ञाश्चैव \msCb\Ed}}% 

{\devanagarifont अनर्थयज्ञ उवाच {\dandab}\dontdisplaylinenum  }%
     \var{{\devanagarifontvar \numemph\vo\textbf{अनर्थ॰}\lem \msCb\msNa\Ed, \lac र्थ॰ \msCa}}% 

{\devanagarifont ब्राह्मणस्य तपो ज्ञानं तपः क्षत्रस्य रक्षणम् \thinspace{\danda} \dontdisplaylinenum }%
     \var{{\devanagarifontvar \numnoemph\va\textbf{ज्ञानं}\lem \msCa\msCb\msNa, यज्ञाः \Ed}}% 
    \var{{\devanagarifontvar \numnoemph\vb\textbf{क्षत्रस्य}\lem \msCa\msCb\msNa, क्षात्रस्य \Ed}}% 

%Verse 19:38

{\devanagarifont वैश्यस्य च तपो वार्त्ता तपः शूद्रस्य सेवनम् {॥ १९:३८॥} \veg\dontdisplaylinenum }%
     \var{{\devanagarifontvar \numnoemph\vc\textbf{वैश्यस्य च तपो वार्त्ता}\lem \msCa\msCb\msNa, वैश्यश्च तप वाणिज्य \Ed}}% 

{\devanagarifont प्रतिग्रह धनं विप्रः क्षत्रियस्य धनुर्धनम् \thinspace{\dandab} \dontdisplaylinenum }%
     \var{{\devanagarifontvar \numemph\va\textbf{धनं}\lem \msCa\msCb\msNa, धनो \Ed}}% 

%Verse 19:39

{\devanagarifont कृषिर्धनं तथा वैश्यः शूद्रः शुश्रूषणं धनम् {॥ १९:३९॥} \veg\dontdisplaylinenum }%
     \var{{\devanagarifontvar \numnoemph\vc\textbf{॰र्धनं}\lem \msCa\msCb\Ed, ॰र्धन \msNa}}% 
    \var{{\devanagarifontvar \numnoemph\vd\textbf{शूद्रः}\lem \msCa\msCb\Ed, शूद्र \msNa}}% 

{\devanagarifont आरम्भयज्ञः क्षत्रस्य हविर्यज्ञा विशस्तथा \thinspace{\dandab} \dontdisplaylinenum }%
     \var{{\devanagarifontvar \numemph\vb\textbf{॰र्यज्ञा}\lem \msNa, ॰र्यज्ञ \msCa, ॰\uncl{र्य}\lk\ \msCb, ॰र्यज्ञो \Ed\oo 
\textbf{विशस्तथा}\lem \msNa\Ed, विश\uncl{स्त}\lac\ \msCa, 
\uncl{विसस्तथा} \msCb}}% 
    \paral{{\devanagarifontsmall \vo {\englishfont \similar\ \MBH\ 12.224.61:}
                         आरम्भयज्ञाः क्षत्रस्य हविर्यज्ञा विशस् तथा\thinspace{\devanagarifontsmall ।}
                         परिचारयज्ञाः शूद्रास् तु तपोयज्ञा द्विजातयः\thinspace{\devanagarifontsmall ॥}
                  {\englishfont \similar\ \MBH\ 12.230.12:}
                         आरम्भयज्ञाः क्षत्रस्य हविर्यज्ञा विशः स्मृताः\thinspace{\devanagarifontsmall ।}
                         परिचारयज्ञाः शूद्राश् च जपयज्ञा द्विजातयः\thinspace{\devanagarifontsmall ॥} }}

%Verse 19:40

{\devanagarifont शूद्राः परिचरायज्ञा जपयज्ञा द्विजातयः {॥ १९:४०॥} \veg\dontdisplaylinenum }%
     \var{{\devanagarifontvar \numnoemph\vc\textbf{शूद्राः}\lem \eme, शूद्रः \msCa\msCb\msNa\Ed\oo 
\textbf{परिचरायज्ञा}\lem \msCa\msCb\msNa, परिचरो यज्ञो \Ed}}% 
    \var{{\devanagarifontvar \numnoemph\vd\textbf{॰यज्ञा}\lem \msCa\msCb\msNa, ॰यज्ञो \Ed}}% 

{\devanagarifont सत्य तीर्थं द्विजातीनां रण तीर्थं तु क्षत्रियाः \thinspace{\dandab} \dontdisplaylinenum }%
     \var{{\devanagarifontvar \numemph\va\textbf{तीर्थं}\lem \msCa\msCb\msNa, तीर्थ \Ed}}% 

%Verse 19:41

{\devanagarifont आर्या तीर्थं तु वैश्यानां शूद्रतीर्थं च वै द्विजाः {॥ १९:४१॥} \veg\dontdisplaylinenum }%
     \var{{\devanagarifontvar \numnoemph\vc\textbf{वैश्यानां}\lem \msCa\msCb\msNa, वैशानां \Ed}}% 
    \var{{\devanagarifontvar \numnoemph\vd\textbf{च}\lem \msCa\msCb, तु \msNa\Ed}}% 

{\devanagarifont नास्ति विद्यासमो मित्रो नास्ति दानसमः सखा \thinspace{\dandab} \dontdisplaylinenum }%
 
%Verse 19:42

{\devanagarifont नास्ति ज्ञानसमो बन्धुर्नास्ति यज्ञो जपसमः {॥ १९:४२॥} \veg\dontdisplaylinenum }%
     \var{{\devanagarifontvar \numemph\vcd\textbf{बन्धुर्ना॰}\lem \msCa\msNa\Ed, बन्धु ना॰ \msCb}}% 
    \var{{\devanagarifontvar \numnoemph\vd\textbf{जप॰}\lem \msCa\msCb\msNa\ \unmetr, जपः॰ \Ed}}% 

{\devanagarifont धर्महीनो मृतैस्तुल्यो देवतुल्यो जितेन्द्रियः \thinspace{\dandab} \dontdisplaylinenum }%
     \var{{\devanagarifontvar \numemph\va\textbf{मृतैस्तु॰}\lem \msCb\msNa, \lk तेस्तु॰ \msCa, मृतस्तु॰ \Ed}}% 

%Verse 19:43

{\devanagarifont यज्ञतुल्यो ऽभयं दाता शिवतुल्यो मनोन्मनः {॥ १९:४३॥} \veg\dontdisplaylinenum }%
 
{\devanagarifont विगतराग उवाच {\dandab}\dontdisplaylinenum  }%
 
{\devanagarifont दानं यज्ञस्तपस्तीर्थं संन्यासं योग एव च \thinspace{\danda} \dontdisplaylinenum }%
     \var{{\devanagarifontvar \numemph\va\textbf{दानं}\lem \corr, दान \msCa\msCb\msNa\oo 
\textbf{॰पस्तीर्थं}\lem \msCa\msNa\Ed, ॰पतीर्थं \msCb}}% 

%Verse 19:44

{\devanagarifont एतेषु कतमः श्रेष्ठः श्रोतुमिच्छामि कीर्तय {॥ १९:४४॥} \veg\dontdisplaylinenum }%
 
{\devanagarifont अनर्थयज्ञ उवाच {\dandab}\dontdisplaylinenum  }%
     \var{{\devanagarifontvar \numemph\vo\textbf{अनर्थयज्ञ उवाच}\lem \msCapcorr\msCb\msNa\Ed, \om\ \msCaacorr}}% 

{\devanagarifont दानधर्मसहस्रेभ्यो यज्ञयाजी विशिष्यते \thinspace{\danda} \dontdisplaylinenum }%
 
%Verse 19:45

{\devanagarifont यज्ञयाजिसहस्रेभ्यस्तीर्थयात्री विशिष्यते {॥ १९:४५॥} \veg\dontdisplaylinenum }%
     \var{{\devanagarifontvar \numnoemph\vc\textbf{॰याजि॰}\lem \msCa\msCb\msNa, ॰याजी॰ \Ed}}% 

{\devanagarifont तीर्थयात्रिसहस्रेभ्यस्तपोनिष्ठो विशिष्यते \thinspace{\dandab} \dontdisplaylinenum }%
     \var{{\devanagarifontvar \numemph\vb\textbf{॰पोनिष्ठो}\lem \msCa\msCb\msNa, ॰पनिष्ठो \Ed}}% 

%Verse 19:46

{\devanagarifont तपोनिष्ठसहस्रेभ्यः श्रेष्ठः संन्यासिकः स्मृतः {॥ १९:४६॥} \veg\dontdisplaylinenum }%
     \var{{\devanagarifontvar \numnoemph\vc\textbf{तपो॰}\lem \msCa\msCb\msNa, तप॰ \Ed}}% 

{\devanagarifont संन्यासीनां सहस्रेभ्यः श्रेष्ठो यस्तु जितेन्द्रियः \thinspace{\dandab} \dontdisplaylinenum }%
     \var{{\devanagarifontvar \numemph\vb\textbf{यस्तु}\lem \msCa\msCb\msNa, यच्य \Ed}}% 

%Verse 19:47

{\devanagarifont जितेन्द्रियसहस्रेभ्यो योगयुक्तो विशिष्यते {॥ १९:४७॥} \veg\dontdisplaylinenum }%
 
{\devanagarifont योगयुक्तसहस्रेभ्यः श्रेष्ठो लीनमनाः स्मृतः \thinspace{\dandab} \dontdisplaylinenum }%
     \var{{\devanagarifontvar \numemph\va\textbf{॰युक्तसहस्रे॰}\lem \msCb\msNa\Ed, ॰यु\uncl{क्त}\lk हस्रे॰ \msCa}}% 
    \var{{\devanagarifontvar \numnoemph\vb\textbf{॰मनाः}\lem \msCa\msCb\msNa, ॰मनः \Ed}}% 

%Verse 19:48

{\devanagarifont तस्मात्सर्वप्रयत्नेन मन आदौ विशोधयेत् {॥ १९:४८॥} \veg\dontdisplaylinenum }%
     \var{{\devanagarifontvar \numnoemph\vd\textbf{मन आदौ}\lem \msCa\msCb\msNa, आदौ मन \Ed}}% 
    \paral{{\devanagarifontsmall \vcd {\englishfont \similar\ \DHARMP\ 16.19ab:}
                       तस्मात्सर्वप्रयत्नेन चित्तमादौ विशोधयेत् }}

{\devanagarifont निगृहीतेन्द्रियग्रामः स्वर्गमोक्षौ तु साधयेत् \thinspace{\dandab} \dontdisplaylinenum }%
     \var{{\devanagarifontvar \numemph\vb\textbf{साधयेत्}\lem \msCa\msCb\msNa, साधनम् \Ed}}% 

%Verse 19:49

{\devanagarifont विसृष्टे त्विन्द्रियग्रामे तिर्यक्नरकसाधनम् {॥ १९:४९॥} \veg\dontdisplaylinenum }%
     \var{{\devanagarifontvar \numnoemph\vc\textbf{विसृष्टे}\lem \msCa\msNa, विशिष्टे \msCb, विशिष्ठे \Ed}}% 
    \var{{\devanagarifontvar \numnoemph\vd\textbf{तिर्यक्नरक॰}\lem \msCa\msCb\msNa, तिर्यन्नरक॰ \Ed}}% 

{\devanagarifont विगतराग उवाच {\dandab}\dontdisplaylinenum  }%
 
{\devanagarifont चराचराणां भूतानां श्रेष्ठः कतम उच्यते \thinspace{\danda} \dontdisplaylinenum }%
     \var{{\devanagarifontvar \numemph\vb\textbf{श्रेष्ठः कतम}\lem \msCa\msCb\msNa, कतमः श्रेष्ठ \Ed}}% 

%Verse 19:50

{\devanagarifont कथयस्व ममाद्य त्वं छेत्तुमर्हसि संशयम् {॥ १९:५०॥} \veg\dontdisplaylinenum }%
     \var{{\devanagarifontvar \numnoemph\vc\textbf{कथयस्व}\lem \msCb\msNa\Ed, कथ\lac स्व \msCa}}% 

{\devanagarifont अनर्थयज्ञ उवाच {\dandab}\dontdisplaylinenum  }%
 
{\devanagarifont चराचराणां भूतानां तत्र श्रेष्ठाश्चराः स्मृताः \thinspace{\danda} \dontdisplaylinenum }%
     \var{{\devanagarifontvar \numemph\va\textbf{॰चराणां}\lem \msCa\msNa\Ed, ॰चराणा \msCb}}% 
    \var{{\devanagarifontvar \numnoemph\vb\textbf{श्रेष्ठाश्चराः}\lem \msCa, श्रेष्ठोःश्चराः \msCb, श्रेष्ठ चराः \msNa, श्रेष्ठो चराः \Ed}}% 

%Verse 19:51

{\devanagarifont चराणां चैव सर्वेषां बुद्धिमान्श्रेष्ठ उच्यते {॥ १९:५१॥} \veg\dontdisplaylinenum }%
     \var{{\devanagarifontvar \numnoemph\vc\textbf{॰चराणां}\lem \msCa\msNa\Ed, ॰चराणा \msCb}}% 

{\devanagarifont बुद्धिमत्सु च सर्वेषु ततः श्रेष्ठा नराः स्मृताः \thinspace{\dandab} \dontdisplaylinenum }%
     \var{{\devanagarifontvar \numemph\va\textbf{बुद्धिमात्सु}\lem \msCa\msCb\msNa, बुद्धिमान्षु \Ed}}% 
    \var{{\devanagarifontvar \numnoemph\vb\textbf{श्रेष्ठा}\lem \msCa, श्रेष्ठो \msCb\msNa, श्रेष्ठ॰ \Ed}}% 

%Verse 19:52

{\devanagarifont नराणां चैव सर्वेषां ब्राह्मणः श्रेष्ठ उच्यते {॥ १९:५२॥} \veg\dontdisplaylinenum }%
     \var{{\devanagarifontvar \numnoemph\vd\textbf{श्रेष्ठ उच्यते}\lem \msCb\msNa\Ed, श्रे\lac च्यते \msCa}}% 

{\devanagarifont ब्राह्मणेषु च सर्वेषु विद्वान् श्रेष्ठः स उच्यते  \thinspace{\dandab} \dontdisplaylinenum }%
     \lacuna{\devanagarifontsmall \vab {\englishfont Missing in \Ed.} }%
      \lacuna{\devanagarifontsmall \va {\englishfont \msCc\ resumes here with} णेषु च सर्व्वेषु }%
      \var{{\devanagarifontvar \numemph\va\textbf{च}\lem \msCb\msCc\msNa, \om\ \msCa}}% 
    \var{{\devanagarifontvar \numnoemph\vb\textbf{श्रेष्ठः}\lem \msCa\msCb\msNa, श्रेष्ठ \msCc\oo 
\textbf{स उच्यते}\lem \msCa\msNa\Ed, समुच्यते \msCb}}% 

%Verse 19:53

{\devanagarifont विद्वत्स्वपि च सर्वेषु कृतबुद्धिर्विशिष्यते {॥ १९:५३॥} \veg\dontdisplaylinenum }%
     \var{{\devanagarifontvar \numnoemph\vc\textbf{च}\lem \mssALL, \om\ \msNa}}% 
    \var{{\devanagarifontvar \numnoemph\vd\textbf{॰बुद्धिर्वि॰}\lem \msCa\msCb\msNa\Ed, ॰बुद्धि वि॰ \msCc}}% 

{\devanagarifont कृतबुद्धिषु सर्वेषु श्रेष्ठः कर्ता समुच्यते \thinspace{\dandab} \dontdisplaylinenum }%
     \var{{\devanagarifontvar \numemph\vb\textbf{समुच्यते}\lem \mssCaCbCc\msNa, स उच्यते \Ed}}% 

%Verse 19:54

{\devanagarifont कर्तृष्वपि च सर्वेषु ब्रह्मवेदी विशिष्यते {॥ १९:५४॥} \veg\dontdisplaylinenum }%
 
{\devanagarifont ब्रह्मवेदि परं विप्रः नान्यं वेद्मि परं तपः \thinspace{\dandab} \dontdisplaylinenum }%
     \var{{\devanagarifontvar \numemph\va\textbf{विप्रः}\lem \msNa\Ed, विप्र \msCb\msCc}}% 
    \lacuna{\devanagarifontsmall \vab {\englishfont Missing in \msCa.} }%
  
%Verse 19:55

{\devanagarifont स विप्रः स तपस्वी च स योगी स शिवः स्मृतः {॥ १९:५५॥} \veg\dontdisplaylinenum }%
     \var{{\devanagarifontvar \numnoemph\vc\textbf{विप्रः}\lem \msCa\msCb\msNa\Ed, विप्र \msCc}}% 

{\devanagarifont 
\jump
\begin{center}
\ketdanda~इति वृषसारसंग्रहे दानयज्ञविशेषो नाम ऊनविंशतितमो ऽध्यायः~\ketdanda
\end{center}
\dontdisplaylinenum\vers  }%
     \var{{\devanagarifontvar \numnoemph{\englishfont \Colo:}\textbf{वृषसारसंग्रहे}\lem \msCb\msCc\msNa\Ed, वृष\lac हे \msCa\oo 
\textbf{नाम ऊनविंश॰}\lem \msCa\msCc\msNa, ना ऊनविश॰ \msCb, नाम उनविंश॰ \Ed}}% 
\bekveg\szamveg
\vfill
\phpspagebreak

\versno=0\fejno=20
\thispagestyle{empty}

\centerline{\Large\devanagarifontbold [   विंशतिमो ऽध्यायः  ]}{\vrule depth10pt width0pt} \fancyhead[CE]{{\footnotesize\devanagarifont वृषसारसंग्रहे  }}
\fancyhead[CO]{{\footnotesize\devanagarifont विंशतिमो ऽध्यायः  }}
\fancyhead[LE]{}
\fancyhead[RE]{}
\fancyhead[LO]{}
\fancyhead[RO]{}
\szam\bek


\vers


{\devanagarifont विगतराग उवाच {\dandab}\dontdisplaylinenum  }%
 
{\devanagarifont पञ्चविंशति यत्तत्त्वं ज्ञातुमिच्छामि तत्त्वतः \thinspace{\danda} \dontdisplaylinenum }%
 
%Verse 20:1

{\devanagarifont कथयस्व ममाद्य त्वं छिद्यते येन संशयः {॥ २०:१॥} \veg\dontdisplaylinenum }%
 

\alalfejezet{तत्त्वनिर्णयम्}
{\devanagarifont अनर्थयज्ञ उवाच {\dandab}\dontdisplaylinenum  }%
 
{\devanagarifont सर्वप्रत्यक्षदर्शित्वं कथं मां प्रष्टुमर्हसि \thinspace{\danda} \dontdisplaylinenum }%
     \var{{\devanagarifontvar \numemph\va\textbf{सर्व॰}\lem \mssALL, सर्वं \msNa\Ed\oo 
\textbf{॰दर्शित्वं}\lem \mssALL, ॰दर्शीत्वं \msCb}}% 
    \var{{\devanagarifontvar \numnoemph\vb\textbf{मां}\lem \mssALL, मं \msNa}}% 

{\devanagarifont पृष्टेन कथनीयो ऽस्मि एष मे कृतनिश्चयः  \danda\dontdisplaylinenum }%
     \var{{\devanagarifontvar \numnoemph\vc\textbf{ऽस्मि}\lem \mssALL, स्मी \msCb}}% 

%Verse 20:2

{\devanagarifont शृणु ते सम्प्रवक्ष्यामि तत्त्वसद्भावमुत्तमम् {॥ २०:२॥} \veg\dontdisplaylinenum }%
 

\alalfejezet{पुरुषशिवब्रह्मा (२५)}
{\devanagarifont नादिमध्यं न चान्तं च यन्न वेद्यं सुरैरपि \thinspace{\dandab} \dontdisplaylinenum }%
     \var{{\devanagarifontvar \numemph\va\textbf{॰मध्यं}\lem \mssALL, ॰मद्य \msNb\oo 
\textbf{चान्तं च}\lem \mssALL, चान्तश्च \Ed}}% 
    \var{{\devanagarifontvar \numnoemph\vb\textbf{सुरैरपि}\lem \mssALL, सुरेरपि \msCb}}% 

%Verse 20:3

{\devanagarifont अतिसूक्ष्मो ह्यतिस्थूलो निरालम्बो निरञ्जनः {॥ २०:३॥} \veg\dontdisplaylinenum }%
     \var{{\devanagarifontvar \numnoemph\vc\textbf{ह्यति॰}\lem \mssALL, ह्यदि॰ \msCc}}% 

{\devanagarifont अचिन्त्यश्चाप्रमेयश्च अक्षराक्षरवर्जितः \thinspace{\dandab} \dontdisplaylinenum }%
 
%Verse 20:4

{\devanagarifont सर्वः सर्वगतो व्यापी सर्वमावृत्य तिष्ठति {॥ २०:४॥} \veg\dontdisplaylinenum }%
     \var{{\devanagarifontvar \numemph\vcd\textbf{(सर्वः{\englishfont ...} तिष्ठति)}\lem \mssCaCbCc\msNa\msNb, \om\ \Ed}}% 
    \paral{{\devanagarifontsmall \vcd {\englishfont \similar\ \NISVK\ 5.48cd:}
                         सर्वगः सर्वतो व्यापि सर्वमापूर्य तिष्ठति }}

{\devanagarifont सर्वेन्द्रियगुणाभासः सर्वेन्द्रियविवर्जितः \thinspace{\dandab} \dontdisplaylinenum }%
     \var{{\devanagarifontvar \numemph\vab\textbf{सर्वे॰{\englishfont ...} वर्जितः}\lem \mssCaCbCc\msNa\msNb, \om\ \Ed}}% 
    \paral{{\devanagarifontsmall \vab {\englishfont \similar\ \MBH\ 6.35.14ab:} सर्वेन्द्रियगुणाभासं सर्वेन्द्रियविवर्जितम्  }}

%Verse 20:5

{\devanagarifont अजरामरजः शान्तः परमात्मा शिवो ऽव्ययः {॥ २०:५॥} \veg\dontdisplaylinenum }%
     \var{{\devanagarifontvar \numnoemph\vc\textbf{॰जः}\lem \mssCaCbCc\msNa\msNb, यः \Ed}}% 

{\devanagarifont अलक्ष्यलक्षणः स्वस्थो ब्रह्मा पुरुषसंज्ञितः \thinspace{\dandab} \dontdisplaylinenum }%
     \var{{\devanagarifontvar \numemph\vb\textbf{ब्रह्मा}\lem \msCa\msCb\Ed, ब्रह्म \msCc\msNa\msNb}}% 

%Verse 20:6

{\devanagarifont पञ्चविंशः स विज्ञेयो जन्ममृत्युहरः प्रभुः {॥ २०:६॥} \veg\dontdisplaylinenum }%
     \var{{\devanagarifontvar \numnoemph\vc\textbf{॰विंशः}\lem \mssCaCbCc\msNb\Ed, ॰विंशत् \msNaacorr, ॰विंश \msNapcorr\oo 
\textbf{स विज्ञेयो}\lem \mssCaCbCc\msNaacorr\msNb\Ed, सर्वज्ञेयो \msNapcorr}}% 

{\devanagarifont कलाकलङ्कनिर्मुक्तो व्योमपञ्चाशवर्जितः \thinspace{\dandab} \dontdisplaylinenum }%
     \var{{\devanagarifontvar \numemph\va\textbf{॰निर्मुक्तो}\lem \mssCaCbCc\msNa\Ed, ॰लिर्मुक्तो \msNb}}% 
    \var{{\devanagarifontvar \numnoemph\vb\textbf{॰पञ्चाश॰}\lem \mssCaCbCc\msNa\Ed, ॰पञ्चस॰ \msNb}}% 

{\devanagarifont जलपक्षी यथा तोयैर्न लिप्येत जले चरन्  \danda\dontdisplaylinenum }%
     \var{{\devanagarifontvar \numnoemph\vcd\textbf{यथा तोयैर्न}\lem \msCa\Ed, यथा तोयी न \msCbacorr, 
यथा तोयेर्न \msCbpcorr, यथा तोयै न \msCc\msNa, यदा तोयै न्न \msNb}}% 
    \var{{\devanagarifontvar \numnoemph\vd\textbf{लिप्येत}\lem \mssALL, लिप्यते \Ed\oo 
\textbf{जले}\lem \mssALL, जलै \msCb}}% 

%Verse 20:7

{\devanagarifont तद्वद्दोषैर्न लिप्येत पापकर्मशतैरपि {॥ २०:७॥} \veg\dontdisplaylinenum }%
     \var{{\devanagarifontvar \numnoemph\ve\textbf{॰षैर्न}\lem \mssALL, ॰षै न \msCc, ॰षै न्न \msNb}}% 


\alalfejezet{प्रकृतिः (२४)}
{\devanagarifont चतुर्विंशति यत्तत्त्वं प्रकृतिं विद्धि निश्चयम् \thinspace{\dandab} \dontdisplaylinenum }%
     \var{{\devanagarifontvar \numemph\va\textbf{यत्तत्त्वं}\lem \mssALL, य तत्वं \msCc, यन्तत्वं \Ed}}% 
    \var{{\devanagarifontvar \numnoemph\vb\textbf{प्रकृतिं विद्धि निश्चयम्}\lem \conj, प्रकृतिर्विधिनिश्चयः \msCa\msCb\msNa\msNb\Ed, 
प्रकृति विधिनिश्चयः \msCc}}% 

%Verse 20:8

{\devanagarifont विकृतिश्च स विज्ञेयस्तत्त्वतः स मनीषिभिः {॥ २०:८॥} \veg\dontdisplaylinenum }%
     \var{{\devanagarifontvar \numnoemph\vc\textbf{विकृतिश्च}\lem \mssALL, विकृतिञ्च \msCc}}% 
    \var{{\devanagarifontvar \numnoemph\vd\textbf{॰ज्ञेयस्तत्त्व॰}\lem \mssALL, ॰ज्ञेयोस्तत्व॰ \msNb}}% 

{\devanagarifont प्रकृतिप्रभवाः सर्वे बुद्ध्यहंकार-आदयः \thinspace{\dandab} \dontdisplaylinenum }%
     \var{{\devanagarifontvar \numemph\va\textbf{॰भवाः}\lem \mssALL, ॰भावः \msNb}}% 
    \var{{\devanagarifontvar \numnoemph\vb\textbf{बुद्ध्यहंकार-आदयः}\lem \mssALL, बुबुद्ध्यहंकार आदयः \msCc, 
बुद्ध्याहंकारकादयः \Ed}}% 

%Verse 20:9

{\devanagarifont विकृतिं प्रतिलीयन्ते भूम्यादि क्रमशस्तु वै {॥ २०:९॥} \veg\dontdisplaylinenum }%
     \var{{\devanagarifontvar \numnoemph\vc\textbf{विकृतिं}\lem \mssCaCbCc\msNa\Ed, विकृति \msNb}}% 
    \var{{\devanagarifontvar \numnoemph\vd\textbf{क्रमशस्तु वै}\lem \msCa\msCb\msNb\Ed, यः क्रमस्तु वै \msNa, क्रमसंस्तु वैः \msCc}}% 


\alalfejezet{मतिः/बुद्धिः (२३)}
{\devanagarifont मतितत्त्व त्रयोविंश धर्मादिगुणसंयुतः \thinspace{\dandab} \dontdisplaylinenum }%
     \var{{\devanagarifontvar \numemph\vb\textbf{॰युतः}\lem \msCa\msCc\msNa\msNb\Ed, ॰युतम् \msCb}}% 

%Verse 20:10

{\devanagarifont सत्त्वाधिकसमुत्पन्नबोद्धारं विद्धि देहिनः {॥ २०:१०॥} \veg\dontdisplaylinenum }%
     \var{{\devanagarifontvar \numnoemph\vc\textbf{॰समुत्पन्न॰}\lem \msCa\msCc\msNa\msNb\Ed, ॰समुत्पन्नो \msCb}}% 
    \var{{\devanagarifontvar \numnoemph\vd\textbf{॰बोद्धारं}\lem \eme, ॰बोधात \mssCaCbCc\msNb, ॰बोद्धातं \msNa, ॰बोद्धात \Ed\oo 
\textbf{विद्धि}\lem \eme, विधि \mssCaCbCc\msNa\msNb\Ed}}% 


\alalfejezet{अहंकारः (२२)}
{\devanagarifont द्वाविंशति अहंकारस्तत्त्वमुक्तं मनीषिभिः \thinspace{\dandab} \dontdisplaylinenum }%
     \var{{\devanagarifontvar \numemph\vb\textbf{उक्तं}\lem \msCa\msCb\msNa, उक्त \msCc\msNb\Ed}}% 

%Verse 20:11

{\devanagarifont भूतादि मम पञ्चाह रजाधिकसमुद्भवम् {॥ २०:११॥} \veg\dontdisplaylinenum }%
     \var{{\devanagarifontvar \numnoemph\vc\textbf{भूतादि मम पञ्चाह}\lem \mssALL, भूतादिर्नाम पञ्चाह \Ed}}% 
    \var{{\devanagarifontvar \numnoemph\vd\textbf{रजा॰}\lem \mssALL, रजो॰ \Ed\oo 
\textbf{॰द्भवम्}\lem \mssALL, ॰द्भवः \msCb}}% 


\alalfejezet{आकाशः (सुषिरत्वं) शब्दश्च (२१-२०)}
{\devanagarifont एकविंशति यत्तत्त्वं सुषिरं विद्धि भो द्विज \thinspace{\dandab} \dontdisplaylinenum }%
     \var{{\devanagarifontvar \numemph\va\textbf{यत्तत्त्वं}\lem \mssALL, य तत्वं \msCc}}% 
    \var{{\devanagarifontvar \numnoemph\vb\textbf{सुषिरं विद्धि}\lem \eme, सुशिरं विद्धि \msCa\msCb\msNa\Ed, सुसिर वृद्धि \msCc, 
सुसिरं वृद्धि \msNb\oo 
\textbf{द्विज}\lem \mssALL, द्विजः \msNb}}% 

%Verse 20:12

{\devanagarifont शब्दातीतं सुषिरत्वं सशब्दगुणलक्षणम् {॥ २०:१२॥} \veg\dontdisplaylinenum }%
     \var{{\devanagarifontvar \numnoemph\vc\textbf{सुषिरत्वं}\lem \eme, सुशिरत्वं \mssCaCbCc\msNa\msNb\Ed}}% 
    \var{{\devanagarifontvar \numnoemph\vd\textbf{॰लक्षणम्}\lem \mssALL, ॰\uncl{ल}\lac  णम् \msCa}}% 


\alalalfejezet{शब्दः}

{\devanagarifont सप्तस्वरास्त्रयो ग्रामा मूर्छनास्त्वेकविंशतिः \thinspace{\dandab} \dontdisplaylinenum }%
     \var{{\devanagarifontvar \numemph\va\textbf{ग्रामा}\lem \mssALL, ग्रामाः \Ed}}% 
    \var{{\devanagarifontvar \numnoemph\vb\textbf{मूर्छना॰}\lem \mssALL, मूर्च्छाना॰ \msNb\oo 
\textbf{॰विंशतिः}\lem \msCc\Ed, ॰विंशति \msCa\msCb\msNa\msNb}}% 

%Verse 20:13

{\devanagarifont तानामेकोनपञ्चाशच्छब्दभेदस्तदादयः {॥ २०:१३॥} \veg\dontdisplaylinenum }%
     \var{{\devanagarifontvar \numnoemph\vc\textbf{॰कोन॰}\lem \msCa\Ed, ॰कून॰ \msCb\msCc\msNa\msNb}}% 

{\devanagarifont एवमादीन्यनेकानि स्वरभेदा द्विजोत्तम \thinspace{\dandab} \dontdisplaylinenum }%
     \var{{\devanagarifontvar \numemph\vb\textbf{॰भेदा}\lem \mssALL, ॰भेदान् \Ed\oo 
\textbf{॰त्तम}\lem \mssALL, ॰त्तमः \msCc}}% 

%Verse 20:14

{\devanagarifont गान्धर्वस्वरतत्त्वज्ञैर्मुनिभिः समुदाहृतम् {॥ २०:१४॥} \veg\dontdisplaylinenum }%
     \var{{\devanagarifontvar \numnoemph\vc\textbf{गान्धर्वस्वरतत्त्व॰}\lem \mssALL, गान्धर्व्वासुरतत्व॰ \msCa, 
गन्धर्व्वासुरस्तत्व॰ \msCc}}% 
    \var{{\devanagarifontvar \numnoemph\vcd\textbf{॰ज्ञैर्मुनिनिभिः}\lem \mssALL, 
॰ज्ञैर्मुनिभि \msCc, ॰ज्ञै मुनिभिः \msNb}}% 

{\devanagarifont वेणुमुरजतन्त्रीणां दुन्दुभीनां स्वनानि च \thinspace{\dandab} \dontdisplaylinenum }%
     \var{{\devanagarifontvar \numemph\va\textbf{॰तन्त्रीणां}\lem \mssALL, ॰तन्तीनां \msNb}}% 
    \var{{\devanagarifontvar \numnoemph\vb\textbf{दुन्दुभीनां}\lem \mssALL, दुन्दुभीना \msNb\oo 
\textbf{स्वनानि}\lem \mssALL, स्तनानि \msCa}}% 

%Verse 20:15

{\devanagarifont शङ्खकाहलकांस्यानां शब्दानि विविधानि च {॥ २०:१५॥} \veg\dontdisplaylinenum }%
     \var{{\devanagarifontvar \numnoemph\vcd\textbf{॰काहलकांस्यानां शब्दानि}\lem \msNa\msNb\Ed, ॰काहलकांस्या\uncl{नां} \lac  नि \msCa, 
॰काहलकास्यानां शब्दानि \msCb, ॰कांस्यानां शब्दानि \msCc}}% 


\alalalfejezet{आकाशः}

{\devanagarifont आकाशधातु विप्रेन्द्र शृणु वक्ष्यामि ते दश \thinspace{\dandab} \dontdisplaylinenum }%
     \var{{\devanagarifontvar \numemph\va\textbf{॰धातु}\lem \mssALL, ॰धातुं \msCa}}% 

%Verse 20:16

{\devanagarifont पायूपस्थोदर कण्ठ शङ्खलौ मुख नासिकौ {॥ २०:१६॥} \veg\dontdisplaylinenum }%
     \var{{\devanagarifontvar \numnoemph\vc\textbf{॰दर}\lem \mssALL, ॰दरः \Ed}}% 
    \var{{\devanagarifontvar \numnoemph\vd\textbf{शङ्खलौ}\lem \mssALL, श्रोतौ च \Ed}}% 

{\devanagarifont हृदिं च दशमं ज्ञेयं देह आकाशसम्भवः \thinspace{\dandab} \dontdisplaylinenum }%
     \var{{\devanagarifontvar \numemph\va\textbf{हृदिं}\lem \mssALL, हृदिश् \Ed\oo 
\textbf{दशमं}\lem \mssALL, दशम \msCc}}% 

%Verse 20:17

{\devanagarifont पुनरन्यत्प्रवक्ष्यामि तच्छृणुष्व द्विजोत्तम {॥ २०:१७॥} \veg\dontdisplaylinenum }%
     \var{{\devanagarifontvar \numnoemph\vc\textbf{अन्यत्प्र॰}\lem \mssCaCbCc\Ed, अन्यं प्र॰ \msNa, अन्य प्र॰ \msNb}}% 
    \var{{\devanagarifontvar \numnoemph\vd\textbf{द्विजोत्तम}\lem \mssCaCbCc\msNa\Ed, जिजोत्तम \msNb}}% 

{\devanagarifont दश धातुगुणा ज्ञेयाः पञ्चभूतः पृथक्पृथक् \thinspace{\dandab} \dontdisplaylinenum }%
     \var{{\devanagarifontvar \numemph\vb\textbf{॰भूतः}\lem \msCa\msCc\msNa\msNb\Ed, ॰भूत \msCb}}% 

%Verse 20:18

{\devanagarifont आकाशस्य गुणाः शब्दो व्यापित्वं छिद्रतापि च {॥ २०:१८॥} \veg\dontdisplaylinenum }%
     \var{{\devanagarifontvar \numnoemph\vc\textbf{आकाशस्य}\lem \msCb\msCc\msNa\msNb\Ed, आकाश\lac\  \msCa}}% 
    \var{{\devanagarifontvar \numnoemph\vd\textbf{व्यापित्वं}\lem \msCa\msCc\msNa\msNb\Ed, व्यापित्वां \msCb}}% 
    \paral{{\devanagarifontsmall \vcd {\englishfont  \similar\ MBh 12.247.7ab:}आकाशस्य गुणः शब्दो व्यापित्वं छिद्रतापि च }}

{\devanagarifont अनाश्रयनिरालम्बमव्यक्तमविकारिता \thinspace{\dandab} \dontdisplaylinenum }%
 
%Verse 20:19

{\devanagarifont अप्रतीघातिता चैव भूतत्वं प्रकृतानि च {॥ २०:१९॥} \veg\dontdisplaylinenum }%
     \var{{\devanagarifontvar \numemph\vc\textbf{अप्रतीघातिता}\lem \msCa\msNa\msNb\Ed, अप्रतीघातता \msCb\msCc}}% 
    \paral{{\devanagarifontsmall \vo {\englishfont  \similar\ MBh 12.247.7cd--8ab:}
                          अनाश्रयमनालम्बमव्यक्तमविकारिता\thinspace{\devanagarifontsmall ॥}
                          अप्रतीघातता चैव भूतत्वं विकृतानि च\thinspace{\devanagarifontsmall ।} }}


\alalfejezet{वायुः स्पर्शश्च (१९-१८)}
{\devanagarifont आकाशधातोर्विप्रेन्द्र ततो वायुसमुद्भवः \thinspace{\dandab} \dontdisplaylinenum }%
     \var{{\devanagarifontvar \numemph\va\textbf{॰धातोर्वि॰}\lem \msCb, ॰धातो वि॰ \msCa\msCc\msNa\msNb\Ed}}% 

%Verse 20:20

{\devanagarifont शब्दपूर्वगुणं गृह्य वायोः स्पर्शगुणः स्मृतः {॥ २०:२०॥} \veg\dontdisplaylinenum }%
     \var{{\devanagarifontvar \numnoemph\vc\textbf{शब्द॰}\lem \mssALL, शब्दः \msNb\oo 
\textbf{॰पूर्व॰}\lem \mssALL, ॰पूर्वं \msCb}}% 

{\devanagarifont शब्द पूर्वं मयाख्यातं शृणु स्पर्शं द्विजोत्तम \thinspace{\dandab} \dontdisplaylinenum }%
     \var{{\devanagarifontvar \numemph\va\textbf{पूर्वं}\lem \mssALL, पूर्व \msCc}}% 
    \var{{\devanagarifontvar \numnoemph\vb\textbf{स्पर्शं द्विजोत्तम}\lem \msCc\msNa, स्पर्श\lac  जोत्तम \msCa, स्पर्शं द्विजोत्तमः \msCb, 
स्पर्श द्विजोत्तम \msNb\Ed}}% 

%Verse 20:21

{\devanagarifont कठिनश्चिक्कणः श्लक्ष्णो मृदुस्निग्धखरद्रवाः {॥ २०:२१॥} \veg\dontdisplaylinenum }%
     \var{{\devanagarifontvar \numnoemph\vc\textbf{चिक्कणः}\lem \corr, चिक्कनः \msCa\msCb\msNa\msNb, चिक्कलः \msCc, चिक्करः \Ed}}% 
    \var{{\devanagarifontvar \numnoemph\vd\textbf{॰स्निग्ध॰}\lem \mssALL, ॰श्निध॰ \msCb}}% 
    \paral{{\devanagarifontsmall \vcd {\englishfont \similar\ \MBH\ 12.177.34ab:} कठिनश्चिक्कणः श्लक्ष्णः पिच्छलो मृदुदारुणः }}

{\devanagarifont कर्कशः परुषस्तीक्ष्णः शीतोष्ण दश च द्वयम् \thinspace{\dandab} \dontdisplaylinenum }%
     \var{{\devanagarifontvar \numemph\va\textbf{परुषस्तीक्ष्णः}\lem \msCb\msNb, परुषस्त्रीक्ष्णश् \msCa, 
तीक्ष्णः \msNaacorr, परुषा तीक्ष्णः \msNapcorr, परुषस्तीक्ष्ण \msCc\Ed}}% 
    \var{{\devanagarifontvar \numnoemph\vb\textbf{द्वयम्}\lem \msNa, द्वय \msCa\msCc\msNb\Ed, द्वयः \msCb}}% 

%Verse 20:22

{\devanagarifont इष्टानिष्टद्वयस्पर्श वपुषा परिगृह्यते {॥ २०:२२॥} \veg\dontdisplaylinenum }%
     \var{{\devanagarifontvar \numnoemph\vc\textbf{॰द्वय॰}\lem \mssCaCbCc\msNa\Ed, ॰द्वयो \msNb}}% 
    \var{{\devanagarifontvar \numnoemph\vd\textbf{॰गृह्यते}\lem \msCa\msNa\msNb\Ed, ॰गृहते \msCb}}% 
    \paral{{\devanagarifontsmall \vc {\englishfont Folio 309v in \msCc\ ends with} इष्टानिष्टद्वय {\englishfont and the next folio is missing.
                   \msCc\ resumes on folio 311r with 20.50c }(मान्सञ्च मेदञ्च) }}


\alalalfejezet{प्राणाः}

{\devanagarifont प्राणो ऽपानः समानश्च उदानो व्यान एव च \thinspace{\dandab} \dontdisplaylinenum }%
     \var{{\devanagarifontvar \numemph\va\textbf{ऽपानः}\lem \mssALL, पान॰ \msCb\msNb}}% 
    \paral{{\devanagarifontsmall \vab {\englishfont = Dharmaputrikā 4.16ab} }}

%Verse 20:23

{\devanagarifont नागकूर्मो ऽथ कृकरो देवदत्तो धनंजयः {॥ २०:२३॥} \veg\dontdisplaylinenum }%
     \var{{\devanagarifontvar \numnoemph\vc\textbf{नाग॰}\lem \mssALL, नाम॰ \msCa\oo 
\textbf{कृकरो}\lem \mssALL, कृकलो \Ed}}% 
    \paral{{\devanagarifontsmall \vo {\englishfont  The next XX verses are parallel to a
                          passage in the Bṛhatkālottara (NGMPP Reel No.\ B 29/59 Manuscript No.\ pra - 89):}
         प्राणोपानः समानश्च उदानो व्यान एव च\thinspace{\devanagarifontsmall ॥}
         नागः कुर्मोध्व कृकरो देवदत्तधनंययौ\thinspace{\devanagarifontsmall ।}
         प्राणस्तु प्रथमो वायुर्दशानामपि स प्रभुः\thinspace{\devanagarifontsmall ॥}
         प्राणः प्राणमयः प्राण विसर्गापूरणं प्रति\thinspace{\devanagarifontsmall ।}
         नित्यमापूरयत्येष प्राणिनामुरसि स्थितः\thinspace{\devanagarifontsmall ॥}
         निश्वासोच्छ्वासकामैस्तु प्राणो जीवसमाश्रितः\thinspace{\devanagarifontsmall ।}
         प्रयाणं कुरुते यस्मात्तस्मात्प्राण प्रकीर्तितः\thinspace{\devanagarifontsmall ॥}
         अपानसहापानस्तु आहारं च नृणामधः\thinspace{\devanagarifontsmall ।}
         मूत्रशुक्रवहोवायुरपानस्तेन कीर्तितः\thinspace{\devanagarifontsmall ॥}
         पीतं भक्षितमाघ्रातं रक्तपितकफानिलं\thinspace{\devanagarifontsmall ।}
         समं नयति मात्रेषु समानो नाम मारुतः\thinspace{\devanagarifontsmall ॥}
         स्पदंयभ्यधरं वक्त्रं नेत्रगात्र प्रकोपनः\thinspace{\devanagarifontsmall ।}
         उद्वेजयति मर्माणि उदातो नाम मारुतः\thinspace{\devanagarifontsmall ॥}
         व्यानो विनामयत्यंगं व्यानो व्याधिप्रकोपकः\thinspace{\devanagarifontsmall ।}
         प्रीतेचिनासी कथितो वाद्धिक्यात् व्यान उच्यते\thinspace{\devanagarifontsmall ॥}
         {\englishfont ...
          cf.\ also Sārdhatriśatikālottara, Agnipurāṇa and Dīpikā by AGoraśivācārya on the Mṛgendra }  }}

{\devanagarifont दश वायुप्रधानैते कीर्तिता द्विजसत्तम \thinspace{\dandab} \dontdisplaylinenum }%
     \var{{\devanagarifontvar \numemph\vb\textbf{कीर्तिता}\lem \mssALL, \uncl{कीर्त्ति}ता \msCa, कीर्तिताः \Ed}}% 

%Verse 20:24

{\devanagarifont धनंजयो भवेद्घोषो देवदत्तो विजृम्भकः {॥ २०:२४॥} \veg\dontdisplaylinenum }%
     \var{{\devanagarifontvar \numnoemph\vc\textbf{भवेद्घोषो}\lem \mssALL, भवेद्योषो \msNa}}% 

{\devanagarifont कृकरः क्षुधकृन्नित्यं कूर्मोन्मीलितलोचनः \thinspace{\dandab} \dontdisplaylinenum }%
     \var{{\devanagarifontvar \numemph\va\textbf{कृकरः}\lem \mssALL, कृकर \Ed\oo 
\textbf{॰कृन्नित्यं}\lem \mssALL, कृन्नित्य \msCb}}% 
    \var{{\devanagarifontvar \numnoemph\vb\textbf{कूर्मोन्मीलितलोचनः}\lem \mssALL, कर्मोल्मीनलोचनः \msCb, कूर्मोन्मीनलोचनः \msNb}}% 

%Verse 20:25

{\devanagarifont नाग उद्घाटनं पुष्यं करोति सततं द्विज {॥ २०:२५॥} \veg\dontdisplaylinenum }%
     \var{{\devanagarifontvar \numnoemph\vc\textbf{पुष्यं}\lem \mssALL, पुन्सां \msNb}}% 
    \var{{\devanagarifontvar \numnoemph\vd\textbf{द्विज}\lem \mssALL, द्विजः \msNb}}% 

{\devanagarifont प्राणः श्वसति भूतानां निश्वसन्ति च नित्यशः \thinspace{\dandab} \dontdisplaylinenum }%
     \var{{\devanagarifontvar \numemph\va\textbf{प्राणः}\lem \mssALL, प्राणाः \Ed}}% 
    \var{{\devanagarifontvar \numnoemph\vb\textbf{नित्यशः}\lem \mssALL, नित्य यः \Ed}}% 

%Verse 20:26

{\devanagarifont प्रयाणं कुरुते यस्मात्तस्मात्प्राण इति स्मृतः {॥ २०:२६॥} \veg\dontdisplaylinenum }%
     \var{{\devanagarifontvar \numnoemph\vc\textbf{प्रयाणं}\lem \mssALL, प्रयाणा \Ed}}% 

{\devanagarifont अपनयत्यपानस्तु आहारं मनुजामधः \thinspace{\dandab} \dontdisplaylinenum }%
     \var{{\devanagarifontvar \numemph\va\textbf{अपनय॰}\lem \msCb\msNa\msNb\Ed, अ\uncl{प} \lac\  य॰ \msCa}}% 
    \var{{\devanagarifontvar \numnoemph\vb\textbf{आहारं मनुजामधः}\lem \msCa\msCb, आहारं मनुजाधमः \msNa, 
आहार मनुजाधमः \msNb, आहारं मनुजापवः \Ed}}% 

%Verse 20:27

{\devanagarifont शुक्रमूत्रवहो वायुरपानस्तेन कीर्तितः {॥ २०:२७॥} \veg\dontdisplaylinenum }%
     \var{{\devanagarifontvar \numnoemph\vd\textbf{॰पानस्तेन}\lem \msCa\msNa\msNb\Ed, ॰वानस्तेन \msCb}}% 

{\devanagarifont पीतभक्षितमाघ्रातं रक्तपित्तकफानिलम् \thinspace{\dandab} \dontdisplaylinenum }%
     \var{{\devanagarifontvar \numemph\va\textbf{॰घ्रातं}\lem \msCa\msCb\msNb\Ed, ॰घ्राति \msNa}}% 
    \var{{\devanagarifontvar \numnoemph\vb\textbf{रक्तपित्त॰}\lem \mssALL, रक्तः पित्तः \msNb}}% 

%Verse 20:28

{\devanagarifont समं नयति गात्रेषु समानो नाम मारुतः {॥ २०:२८॥} \veg\dontdisplaylinenum }%
 
{\devanagarifont स्पन्दयत्यधरं वक्त्रं नेत्रगात्रप्रकोपनम् \thinspace{\dandab} \dontdisplaylinenum }%
     \var{{\devanagarifontvar \numemph\vo\textbf{(स्पन्दयत्यधरं{\englishfont ...} मारुतः)}\lem \mssALL, \om\ \msNb}}% 
    \var{{\devanagarifontvar \numnoemph\va\textbf{॰धरं}\lem \mssALL, \om\ \msNb, ॰धर॰ \Ed\ \unmetr}}% 
    \var{{\devanagarifontvar \numnoemph\vb\textbf{॰गात्रप्र॰}\lem \mssALL, ॰गात्रम्प्र॰ \msCb, \om\ \msNb}}% 

%Verse 20:29

{\devanagarifont उद्वेजयति मर्माणि उदानो नाम मारुतः {॥ २०:२९॥} \veg\dontdisplaylinenum }%
     \var{{\devanagarifontvar \numnoemph\vc\textbf{मर्माणि}\lem \mssALL, \om\ \msNb, कर्माणि \Ed}}% 
    \var{{\devanagarifontvar \numnoemph\vd\textbf{उदानो नाम}\lem \mssALL, \uncl{उ}\lac\  \msCa, \om\ \msNb}}% 

{\devanagarifont व्यानो विनामयत्यङ्गं व्यङ्गो व्याधिप्रकोपनः \thinspace{\dandab} \dontdisplaylinenum }%
     \var{{\devanagarifontvar \numemph\va\textbf{व्यानो वि॰}\lem \mssALL, व्यानो पि \msNb}}% 
    \var{{\devanagarifontvar \numnoemph\vb\textbf{॰कोपनः}\lem \mssALL, ॰कोपमः \msNb}}% 

%Verse 20:30

{\devanagarifont प्रीतिविनाशकथितं वार्धिक्यं व्यान उच्यते {॥ २०:३०॥} \veg\dontdisplaylinenum }%
     \var{{\devanagarifontvar \numnoemph\vc\textbf{प्रीतिवि॰}\lem \mssALL, प्रीतिर्वि॰ \msNb\Ed}}% 

{\devanagarifont दशवायुविभागे च कीर्तितो मे द्विजोत्तम \thinspace{\dandab} \dontdisplaylinenum }%
     \var{{\devanagarifontvar \numemph\vb\textbf{मे}\lem \mssALL, ये \Ed}}% 

%Verse 20:31

{\devanagarifont दशवायुगुणांश्चान्यां छृणु कीर्तयतो मम {॥ २०:३१॥} \veg\dontdisplaylinenum }%
     \var{{\devanagarifontvar \numnoemph\vcd\textbf{॰वायुगुणांश्चान्यां शृणु}\lem \mssALL, ॰वायुगुणाश्चान्यं शृणु \msNb, 
॰धातुगुणाश्चान्यच्छृणु \Ed}}% 
    \var{{\devanagarifontvar \numnoemph\vd\textbf{कीर्तयतो मम}\lem \mssALL, कीर्त्तियतो मम \msCb, कीर्तय मे द्विज \Ed}}% 

{\devanagarifont वायोरनियम स्पर्शो वादस्थानं स्वतन्त्रता \thinspace{\dandab} \dontdisplaylinenum }%
     \var{{\devanagarifontvar \numemph\va\textbf{॰नियमः}\lem \corr, ॰नियम \msCa\msCb\msNa\msNb\Ed}}% 
    \var{{\devanagarifontvar \numnoemph\vb\textbf{वादस्थानं}\lem \eme, वात\lac  ने \msCa, वातस्थाने \msCb\msNa\msNb\Ed}}% 

%Verse 20:32

{\devanagarifont बलं शीघ्रं च मोक्षं च चेष्टा कर्मात्मना भवः {॥ २०:३२॥} \veg\dontdisplaylinenum }%
     \paral{{\devanagarifontsmall \vo {\englishfont \similar\ MBh 12.247.6:}
                         वायोरनियमः स्पर्शो वादस्थानं स्वतन्त्रता\thinspace{\devanagarifontsmall ।}
                         बलं शैघ्र्यं च मोहश्च चेष्टा कर्मकृता भवः\thinspace{\devanagarifontsmall ॥} }}


\alalfejezet{तेजो रूपश्च (१७-१६)}
{\devanagarifont वायुनापि सृजस्तेजस्तद्रूपं गुणमुच्यते \thinspace{\dandab} \dontdisplaylinenum }%
     \var{{\devanagarifontvar \numemph\vab\textbf{सृजस्तेजस्त॰}\lem \mssALL, सृजत्वेजत्त॰ \msNb}}% 
    \var{{\devanagarifontvar \numnoemph\vb\textbf{॰रूपं गुण॰}\lem \mssALL, ॰रूपगुण॰ \Ed}}% 

%Verse 20:33

{\devanagarifont शब्दस्पर्शसम ज्योतिस्त्रिगुणं समुदाहृतम् {॥ २०:३३॥} \veg\dontdisplaylinenum }%
     \var{{\devanagarifontvar \numnoemph\vcd\textbf{॰ज्योतिस्त्रि॰}\lem \mssALL, ॰ज्योतित्रि॰ \msCb}}% 

{\devanagarifont शब्दः स्पर्शः पुरा प्रोक्तः शृणु रूपगुणं ततः \thinspace{\dandab} \dontdisplaylinenum }%
     \var{{\devanagarifontvar \numemph\va\textbf{शब्दः}\lem \msNa\Ed, शब्द॰ \msCa\msCb\msNb\oo 
\textbf{प्रोक्तः}\lem \mssALL, प्रोक्ताः \msCb\Ed}}% 
    \var{{\devanagarifontvar \numnoemph\vb\textbf{रूपगुणं}\lem \mssALL, रूपं गुणं \msNa}}% 

%Verse 20:34

{\devanagarifont ह्रस्वं दीर्घमणु स्थूलं वृत्तमण्डलमेव च {॥ २०:३४॥} \veg\dontdisplaylinenum }%
     \var{{\devanagarifontvar \numnoemph\vc\textbf{ह्रस्वं}\lem \msCa\msNa, ह्रस्व॰ \msCb\msNb\Ed\oo 
\textbf{दीर्घमणु}\lem \mssALL, ॰दीर्घमनु \msNb, ॰दीर्घलघु \Ed\oo 
\textbf{स्थूलं}\lem \mssALL, स्थूल \msNb}}% 
    \var{{\devanagarifontvar \numnoemph\vd\textbf{॰मण्डलमेव}\lem \msCa\msNa\Ed, ॰मण्डमेव \msCb}}% 

{\devanagarifont चतुरस्रं द्विरस्रं च त्र्यस्रं चैव षडस्रकम् \thinspace{\dandab} \dontdisplaylinenum }%
     \var{{\devanagarifontvar \numemph\va\textbf{चतुरस्रं द्विरस्रं}\lem \msCb\msNa, 
चतुरश्रन्द्वि\lac  श्रं \msCa, 
चतुरस्रद्विरस्रश् \Ed}}% 
    \var{{\devanagarifontvar \numnoemph\vb\textbf{त्र्यस्रं}\lem \mssALL, तिस्रश् \Ed}}% 

%Verse 20:35

{\devanagarifont शुक्लः कृष्णस्तथा रक्तो नीलः पीतो ऽरुणस्तथा {॥ २०:३५॥} \veg\dontdisplaylinenum }%
     \var{{\devanagarifontvar \numnoemph\vc\textbf{शुक्लः}\lem \mssALL, शुक्लं \Ed}}% 
    \var{{\devanagarifontvar \numnoemph\vd\textbf{नीलः}\lem \mssALL, नील॰ \Ed}}% 
    \paral{{\devanagarifontsmall \vcd {\englishfont = \MBH\ 12.177.32cd}  }}

{\devanagarifont श्यामः पिङ्गल बभ्रुश्च नव रङ्गाः प्रकीर्तिताः \thinspace{\dandab} \dontdisplaylinenum }%
     \var{{\devanagarifontvar \numemph\va\textbf{श्यामः पिङ्गल बभ्रुश्च}\lem \Ed, 
श्यामः पिङ्गलो बभ्रुश्च \msCa\msCb, 
श्यामश्च पिङ्गलो बभ्रुश्च \msNaacorr, 
श्याम पिङ्गलो भ्रुश्च \msNapcorr}}% 
    \var{{\devanagarifontvar \numnoemph\vb\textbf{रङ्गाः}\lem \mssALL, रङ्गः \Ed}}% 

%Verse 20:36

{\devanagarifont नवधा नवरङ्गानामेकाशीति गुणाः स्मृताः {॥ २०:३६॥} \veg\dontdisplaylinenum }%
     \var{{\devanagarifontvar \numnoemph\vd\textbf{स्मृताः}\lem \mssALL, स्मृतं \Ed}}% 

{\devanagarifont तेजोधातु दश ब्रूमः शृणुष्वावहितो भव \thinspace{\dandab} \dontdisplaylinenum }%
     \var{{\devanagarifontvar \numemph\va\textbf{तेजोधातु दश}\lem \mssALL, 
तेजोधातुर्दशं \Ed}}% 

%Verse 20:37

{\devanagarifont कामस्तेजो क्षणः क्रोधो जठराग्निश्च पञ्चमः {॥ २०:३७॥} \veg\dontdisplaylinenum }%
     \var{{\devanagarifontvar \numnoemph\vc\textbf{तेजो क्षणः}\lem \msCa, तेजः क्षणः \msCb, तेज क्षणः \msNa, तेजेक्षणः \Ed}}% 
    \var{{\devanagarifontvar \numnoemph\vd\textbf{जठराग्निश्च}\lem \msCb\msNa\Ed, जठ\lac  ग्निश्च \msCa}}% 

{\devanagarifont ज्ञानं योगस्तपो ध्यानं विश्वाग्निर्दशमः स्मृतः \thinspace{\dandab} \dontdisplaylinenum }%
     \var{{\devanagarifontvar \numemph\vb\textbf{विश्वाग्निर्द॰}\lem \msCa\msCb\Ed, विश्वाग्नि द॰ \msNa}}% 

%Verse 20:38

{\devanagarifont दश तेजोगुणांश्चान्यान्प्रवक्ष्यामि द्विजोत्तम {॥ २०:३८॥} \veg\dontdisplaylinenum }%
     \var{{\devanagarifontvar \numnoemph\vc\textbf{दश तेजोगुणांश्चा॰}\lem \msCa\msNa, दश तेजोगुणाश्चा॰ \msCb, दंशतेजोगुणाश्चा॰ \Ed}}% 

{\devanagarifont अग्नेर्दुर्धर्षताप्नोति तापपाकप्रकाशनः \thinspace{\dandab} \dontdisplaylinenum }%
     \var{{\devanagarifontvar \numemph\va\textbf{अग्नेर्दुर्धर्षताप्नोति}\lem \conj, अग्नेर्दुर्द्धषताप्नोति \msCa, 
अग्ने दुर्द्धषताप्नोति \msCb\msNa, अग्नेर्दुर्धर्षवाप्नोति \Ed}}% 

%Verse 20:39

{\devanagarifont शौचं रागो लघुस्तैक्ष्ण्यं दशमं चोर्ध्वभागिता {॥ २०:३९॥} \veg\dontdisplaylinenum }%
     \var{{\devanagarifontvar \numnoemph\vc\textbf{रागो}\lem \mssALL, गङ्गा \Ed\oo 
\textbf{लघुस्तैक्ष्ण्यं}\lem \corr, लघुस्तैक्ष्णं \msCa\msNa\Ed, लघुस्तीक्ष्णं \msCb}}% 
    \var{{\devanagarifontvar \numnoemph\vd\textbf{दशमं चोर्ध्वभागिता}\lem \eme, दशपञ्चोर्द्धभाषितम् \msCa, दशमं चोर्धभाषितम् \msCb\msNa, 
दशमश्चोर्धभाषितम् \Ed}}% 
    \paral{{\devanagarifontsmall \vcd {\englishfont  \similar\ MBh 12.247.5cd:} 
                 अग्नेर्दुर्धर्षता तेजस्तापः पाकः प्रकाशनम्\thinspace{\devanagarifontsmall ।}
                 शौचं रागो लघुस्तैक्ष्ण्यं दशमं चोर्ध्वभागिता\thinspace{\devanagarifontsmall ॥} }}


\alalfejezet{आपो रसश्च (१५-१४)}
{\devanagarifont ज्योतिसो ऽपि सृजश्चापः सरसो गुणसंयुतः \thinspace{\dandab} \dontdisplaylinenum }%
     \var{{\devanagarifontvar \numemph\va\textbf{ज्योतिसो}\lem \msCb, ज्योतिः सो \msCa\msNa\Ed\oo 
\textbf{सृजश्चापः}\lem \msCb, सृजश्चापि \msCa\msNa\Ed}}% 

%Verse 20:40

{\devanagarifont चतुर्गुणाः स्मृता आपः विज्ञेया च मनीषिभिः {॥ २०:४०॥} \veg\dontdisplaylinenum }%
     \var{{\devanagarifontvar \numnoemph\vd\textbf{विज्ञेया च मनीषिभिः}\lem \Ed, \om\ \msCa\msCb\msNa\msNb, तान्निबोध द्विजोत्तमः \msNc}}% 

{\devanagarifont शब्दः स्पर्शश्च रूपं च रसश्च स चतुर्गुणः \thinspace{\dandab} \dontdisplaylinenum }%
     \var{{\devanagarifontvar \numemph\va\textbf{रूपं}\lem \mssALL, रूपश् \Ed}}% 
    \paral{{\devanagarifontsmall \vab \similar\ {\englishfont  MBh 12.299.11ab (= 3.202.5ab = 6.6.5ab = 12.299.11ab):} 
                 शब्दः स्पर्शश्च रूपं च रसो गन्धश्च पञ्चमः }}

%Verse 20:41

{\devanagarifont रूपादिगुण पूर्वोक्त अधुनाथ रसं शृणु {॥ २०:४१॥} \veg\dontdisplaylinenum }%
     \var{{\devanagarifontvar \numnoemph\vc\textbf{पूर्वोक्त}\lem \mssALL, पूर्वोक्तं \Ed}}% 

{\devanagarifont कटुतिक्तकषायाश्च लवणाम्लस्तथैव च \thinspace{\dandab} \dontdisplaylinenum }%
     \var{{\devanagarifontvar \numemph\va\textbf{लवणाम्लस्त॰}\lem \mssALL, लवणान्तस्त॰ \Ed}}% 

%Verse 20:42

{\devanagarifont मधुरश्च रसान्षड्वै प्रवदन्ति मनीषिणः {॥ २०:४२॥} \veg\dontdisplaylinenum }%
     \var{{\devanagarifontvar \numnoemph\vc\textbf{रसान्षड्वै}\lem \corr, रसां षड्वै \msCa, रसा षड्वै \msCb\msNa\Ed}}% 

{\devanagarifont षड्रसाः षड्विभेदेन षट्त्रिंशगुण उच्यते \thinspace{\dandab} \dontdisplaylinenum }%
     \var{{\devanagarifontvar \numemph\va\textbf{॰रसाः}\lem \mssALL, ॰रसा \Ed\oo 
\textbf{षड्विभेदेन}\lem \mssALL, षड्भिर्भेदेन \msNa}}% 

%Verse 20:43

{\devanagarifont आपधातु दश त्वन्यान्शृणु कीर्तयतो मम {॥ २०:४३॥} \veg\dontdisplaylinenum }%
     \var{{\devanagarifontvar \numnoemph\vc\textbf{आप॰}\lem \mssALL, \uncl{आ}प॰ \msCa\oo 
\textbf{दश त्वन्यान्}\lem \corr, दश त्वन्यां \msCa\msNa, दशत्वंन्यां \msCb, दश त्वन्या \Ed}}% 
    \var{{\devanagarifontvar \numnoemph\vd\textbf{कीर्तयतो}\lem \mssALL, कीर्त्तियतो \msCb}}% 

{\devanagarifont लाला सिङ्घाणिका श्लेष्मा रक्तः पित्तः कफस्तथा \thinspace{\dandab} \dontdisplaylinenum }%
     \var{{\devanagarifontvar \numemph\va\textbf{लाला}\lem \mssALL, ललां \msCb\oo 
\textbf{सिङ्घाणिका}\lem \corr, सिघानिका \msCa, सि\uncl{घा}निका \msCb, सिंघानिका \msNa\Ed\oo 
\textbf{श्लेष्मा}\lem \mssALL, शोष्मा \Ed}}% 
    \var{{\devanagarifontvar \numnoemph\vb\textbf{रक्तः}\lem \msCa\msNa, रक्त॰ \msCb\Ed}}% 

%Verse 20:44

{\devanagarifont स्वेदमश्रु रसश्चैव मेदश्च दशमः स्मृतः {॥ २०:४४॥} \veg\dontdisplaylinenum }%
     \var{{\devanagarifontvar \numnoemph\vc\textbf{रसश्चैव}\lem \mssALL, रसंश्चैव \msNa}}% 
    \var{{\devanagarifontvar \numnoemph\vd\textbf{मेदश्च}\lem \mssALL, मेदं च \Ed\oo 
\textbf{दशमः}\lem \mssALL, मदः \msCbacorr, मदनः \msCbpcorr}}% 

{\devanagarifont दश आपगुणाश्चान्ये कीर्तयिष्यामि तान्शृणु \thinspace{\dandab} \dontdisplaylinenum }%
     \var{{\devanagarifontvar \numemph\va\textbf{दश आप॰}\lem \mssALL, दशश्चाप॰ \Ed\oo 
\textbf{चान्ये}\lem \msCa\msCb, चान्या \msNa\Ed}}% 
    \var{{\devanagarifontvar \numnoemph\vb\textbf{तान्}\lem \Ed, तां \msCa\msCb\msNa}}% 

{\devanagarifont अद्भ्यः शैत्यं रस क्लेदो द्रवत्वं स्नेहसौम्यता  \danda\dontdisplaylinenum }%
     \var{{\devanagarifontvar \numnoemph\vc\textbf{अद्भ्यः शैत्यं}\lem \conj, अद्भ्य शैत्यं \msCa, 
अङ्गशैत्यं \msCb, अङ्ग्यशैत्यं \msNa, अग्न्यशैत्य॰ \Ed}}% 

%Verse 20:45

{\devanagarifont जिह्वा विष्यन्दिनी चैव भौमान्यश्रवणाधमः {॥ २०:४५॥} \veg\dontdisplaylinenum }%
     \var{{\devanagarifontvar \numnoemph\ve\textbf{विष्यन्दिनी}\lem \msCb\msNa, ॰वि\uncl{ष्}\lac  नी \msCa, ॰निष्पन्दिनी \Ed}}% 
    \var{{\devanagarifontvar \numnoemph\vf\textbf{भौमान्यश्रवणाधमः}\lem \mssALL, भौमान्दशगुणाञ्शृणु \Ed}}% 
    \paral{{\devanagarifontsmall \vo {\englishfont \similar\ \MBH\ 12.247.4 (with} अद्भ्यः {\englishfont as a variant in the critical edition):}
                 अपां शैत्यं रसः क्लेदो द्रवत्वं स्नेहसौम्यता\thinspace{\devanagarifontsmall ।}
                 जिह्वा विष्यन्दिनी चैव भौमाप्यास्रवणं तथा\thinspace{\devanagarifontsmall ॥} }}


\alalfejezet{भूमिर्गन्धश्च (१३-१२)}
{\devanagarifont आपश्चाप्यसृजद्भूमिस्तस्या गन्धगुणः स्मृतः \thinspace{\dandab} \dontdisplaylinenum }%
     \var{{\devanagarifontvar \numemph\va\textbf{आपश्चाप्यसृजद्भू॰}\lem \msCb, 
आपश्चापीज्यजा भू॰ \msCa\msNa, आपश्च बीज्यजा भू॰ \Ed}}% 

%Verse 20:46

{\devanagarifont चतुरापगुणान्गृह्य भूमेर्गन्धगुणः स्मृतः {॥ २०:४६॥} \veg\dontdisplaylinenum }%
     \var{{\devanagarifontvar \numnoemph\vc\textbf{॰गुणान्गृ॰}\lem \mssALL, ॰गुणं गृ॰ \msCb, ॰गुणा गृ॰ \msNa}}% 

{\devanagarifont शब्दः स्पर्शश्च रूपं च रसो गन्धश्च पञ्चमः \thinspace{\dandab} \dontdisplaylinenum }%
     \var{{\devanagarifontvar \numemph\va\textbf{रूपं च}\lem \mssALL, रूपश्च \msCb\Ed}}% 
    \var{{\devanagarifontvar \numnoemph\vb\textbf{पञ्चमः}\lem \mssALL, पञ्चम \Ed}}% 

%Verse 20:47

{\devanagarifont आपःपूर्वगुणाः प्रोक्ता भूमेर्गन्धगुणं शृणु {॥ २०:४७॥} \veg\dontdisplaylinenum }%
     \var{{\devanagarifontvar \numnoemph\vc\textbf{आपः॰}\lem \mssALL, आप॰ \Ed\oo 
\textbf{प्रोक्ता}\lem \mssALL, प्रोक्तो \Ed}}% 
    \var{{\devanagarifontvar \numnoemph\vd\textbf{भूमेर्ग॰}\lem \mssALL, भूमे ग॰ \msNa, भूमिर्ग॰ \Ed\oo 
\textbf{शृणु}\lem \mssALL, स्मृत \Ed}}% 

{\devanagarifont इष्टानिष्टद्वयोर्गन्धः सुरभिर्दुरभिस्तथा \thinspace{\dandab} \dontdisplaylinenum }%
     \var{{\devanagarifontvar \numemph\va\textbf{द्वयोर्गन्धः}\lem \mssALL, द्वयो\lac\  \msCa}}% 

%Verse 20:48

{\devanagarifont कर्पूरः कस्तुरीकं च चन्दनागरुमेव च {॥ २०:४८॥} \veg\dontdisplaylinenum }%
     \var{{\devanagarifontvar \numnoemph\vc\textbf{कस्तुरीकं च}\lem \mssALL, कस्तूरीकश्च \Ed\ \unmetr}}% 
    \var{{\devanagarifontvar \numnoemph\vd\textbf{॰गरु॰}\lem \mssALL, ॰गुरु॰ \Ed}}% 

{\devanagarifont कुङ्कुमादिसुगन्धानि घ्राणमिष्टं प्रकीर्तितम् \thinspace{\dandab} \dontdisplaylinenum }%
     \var{{\devanagarifontvar \numemph\vb\textbf{घ्राणमिष्टं कीर्तितम्}\lem \mssALL, 
\om\ \msNaacorr, घ्राणमिष्टं कीर्तितः \Ed}}% 

{\devanagarifont विङ्मूत्रस्वेदगन्धानि वक्त्रगन्धं च दुःसहम्  \danda\dontdisplaylinenum }%
     \var{{\devanagarifontvar \numnoemph\vc\textbf{विङ्मूत्रस्वेदगन्धानि}\lem \mssALL, \om\ \msNaacorr}}% 
    \var{{\devanagarifontvar \numnoemph\vd\textbf{॰गन्धं च}\lem \mssALL, ॰गन्धश्च \Ed}}% 

%Verse 20:49

{\devanagarifont जीर्णस्फोटितगन्धानि अनिष्टानीति कीर्तितम् {॥ २०:४९॥} \veg\dontdisplaylinenum }%
     \var{{\devanagarifontvar \numnoemph\ve\textbf{॰स्फोटित॰}\lem \mssALL, ॰स्फुटित॰ \msCb, ॰स्फोटक॰ \Ed}}% 
    \var{{\devanagarifontvar \numnoemph\vf\textbf{कीर्तितम्}\lem \mssALL, कीतम् \msCbacorr}}% 

{\devanagarifont भूमेर्धातु दश त्वन्यान्कथयिष्यामि तच्छृणु \thinspace{\dandab} \dontdisplaylinenum }%
     \var{{\devanagarifontvar \numemph\va\textbf{भूमेर्धा॰}\lem \mssALL, भूमे धा॰ \Ed}}% 
    \var{{\devanagarifontvar \numnoemph\vab\textbf{त्वन्यान्क॰}\lem \msCa, त्वन्यां \msCb\msNa, त्वन्या क॰ \Ed}}% 
    \var{{\devanagarifontvar \numnoemph\vb\textbf{तच्छृणु}\lem \mssALL, \uncl{त}\lac  णु \msCa}}% 

{\devanagarifont त्वचं मांसं च मेदं च स्नायु मज्जा सिरा तथा  \danda\dontdisplaylinenum }%
     \var{{\devanagarifontvar \numnoemph\vc\textbf{त्वचं मांसं च मेदं च}\lem \msCa, त्वचं मांसञ्च \msCb, \lac\  मान्सञ्च मेदञ्च \msCc, 
त्वचं मासं च मेदं च \msNa, त्वचा मांसश्च मेदश्च \Ed}}% 
    \var{{\devanagarifontvar \numnoemph\vd\textbf{स्नायु}\lem \mssALL, श्नायुं \msCa\msCb\oo 
\textbf{सिरा तथा}\lem \eme,  शिरास्तथा \msCa\msNa, शिरस्तथा \msCb\msCc\Ed}}% 
    \lacuna{\devanagarifontsmall \vc {\englishfont \msCc\ resumes here, on folio 311r, with} मान्सञ्च मेदञ्च }%
  
%Verse 20:50

{\devanagarifont नखदन्तरुहाश्चैव केशश्च दशमस्तथा {॥ २०:५०॥} \veg\dontdisplaylinenum }%
     \var{{\devanagarifontvar \numnoemph\vf\textbf{केश॰}\lem \mssALL, केशा॰ \Ed}}% 

{\devanagarifont दश त्वन्यान्प्रवक्ष्यामि शृणु भूमिगुणान्द्विज \thinspace{\dandab} \dontdisplaylinenum }%
     \var{{\devanagarifontvar \numemph\va\textbf{त्वन्यान्प्र॰}\lem \Ed, त्वन्याम्प्र॰ \msCa, त्वन्यां प्र॰ \msCb\msCc\msNa}}% 
    \var{{\devanagarifontvar \numnoemph\vb\textbf{॰गुणान्द्वि॰}\lem \mssALL, ॰गुणा द्वि॰ \msCb, ॰गुणां दिव्॰ \msCc}}% 

{\devanagarifont भूमेः स्थैर्यं रजस्त्वं च काठिन्यं प्रसवात्मकम्  \danda\dontdisplaylinenum }%
     \var{{\devanagarifontvar \numnoemph\vc\textbf{भूमेः}\lem \mssALL, भूमिः \msCb\oo 
\textbf{स्थैर्यं}\lem \mssALL, स्थैर्य \msCc\oo 
\textbf{रजस्त्वं च}\lem \mssALL, रजत्वश् च \Ed}}% 
    \var{{\devanagarifontvar \numnoemph\vd\textbf{काठिन्यं}\lem \mssALL, कठिन्यं \Ed}}% 

%Verse 20:51

{\devanagarifont गन्धो गुरुश्च शक्तिश्च नीहारस्थापनाकृतिः {॥ २०:५१॥} \veg\dontdisplaylinenum }%
     \var{{\devanagarifontvar \numnoemph\vf\textbf{॰कृतिः}\lem \mssALL, ॰कृति \msCc}}% 
    \paral{{\devanagarifontsmall \vo {\englishfont \similar\ MBh 12.247.3:}
                         भूमेः स्थैर्यं पृथुत्वं च काठिन्यं प्रसवात्मता\thinspace{\devanagarifontsmall ।}
                         गन्धो गुरुत्वं शक्तिश् च संघातः स्थापना धृतिः\thinspace{\devanagarifontsmall ॥} }}

{\devanagarifont गुणधातुविशेषश्च उत्पत्तिश्च द्विजोत्तम \thinspace{\dandab} \dontdisplaylinenum }%
     \var{{\devanagarifontvar \numemph\va\textbf{गुणधातु॰}\lem \msNa, \uncl{गुणन्धा}तु॰ \msCa, गुणात्वातु॰ \msCb, गुणं धातु॰ \msCc\Ed}}% 

%Verse 20:52

{\devanagarifont यथा श्रुतं मया पूर्वं कीर्तितं निखिलेन तु {॥ २०:५२॥} \veg\dontdisplaylinenum }%
     \var{{\devanagarifontvar \numnoemph\vcd\textbf{पूर्वं कीर्तितं}\lem \msCa\msCc\msNa\Ed, पूर्व कीर्तित \msCb}}% 


\alalfejezet{बुद्धीन्द्रियाणि कर्मेन्द्रियाणि च (११-२)}
{\devanagarifont वैकारिकमहंकारं सत्त्वोद्रिक्तात्तु सात्त्विकः \thinspace{\dandab} \dontdisplaylinenum }%
     \var{{\devanagarifontvar \numemph\vab\textbf{सत्त्वोद्रिक्तात्तु}\lem \corr, 
सत्वोदृक्तात्तु \msCa\msNa, 
सत्वोदृक्तान्तु \msCb, 
सत्त्वोनुक्तानु \Ed}}% 
    \var{{\devanagarifontvar \numnoemph\vb\textbf{सात्त्विकः}\lem \msCa\msCb\Ed, सात्विकाः \msNa}}% 
    \paral{{\devanagarifontsmall \vab {\englishfont \similar\ \LINPU\ 1.70.38cd (= \SIVP\ 7.1.10.14cd):}
                     वैकारिकादहंकारात्सत्त्वोद्रिक्तात्तु सात्त्विकात् }}

%Verse 20:53

{\devanagarifont श्रोत्रं त्वक्चक्षुषी जिह्वा नासिका चैव पञ्चमी {॥ २०:५३॥} \veg\dontdisplaylinenum }%
     \var{{\devanagarifontvar \numnoemph\vc\textbf{श्रोत्रं}\lem \mssALL, श्रोत्र \msCb}}% 
    \paral{{\devanagarifontsmall \vcd {\englishfont = \MBH\ 14.42.13ab = \LINPU\ 1.70.41ab} }}

{\devanagarifont बुद्धीन्द्रियाणि पञ्चैव कीर्तितानि द्विजोत्तम \thinspace{\dandab} \dontdisplaylinenum }%
 
%Verse 20:54

{\devanagarifont हस्तपादस्तथा पायुरुपस्थो वाक्च पञ्चमः {॥ २०:५४॥} \veg\dontdisplaylinenum }%
     \var{{\devanagarifontvar \numemph\vc\textbf{पायु॰}\lem \mssALL, स्नायु॰ \Ed}}% 
    \var{{\devanagarifontvar \numnoemph\vd\textbf{॰पस्थो वाक्च}\lem \mssALL, ॰प\uncl{स्थो वा}\lac\  \msCa\oo 
\textbf{पञ्चमः}\lem \mssALL, पञ्चमम् \Ed}}% 


\alalalfejezet{श्रोत्रम् (११)}

{\devanagarifont श्रोत्रेण गृह्यते शब्दो विविधस्तु द्विजोत्तम \thinspace{\dandab} \dontdisplaylinenum }%
 
%Verse 20:55

{\devanagarifont वेणुवीणास्वनानां च तन्त्रीशब्दमनेकधा {॥ २०:५५॥} \veg\dontdisplaylinenum }%
 
{\devanagarifont मुरज†मौन्द†पणवभेरीपटहनिस्वनम् \thinspace{\dandab} \dontdisplaylinenum }%
     \var{{\devanagarifontvar \numemph\va\textbf{मुरज॰}\lem \Ed, मुरव॰ \msCa\msNa, सुरव॰ \msCb\oo 
\textbf{॰मौन्द॰}\lem \msCa\msNa, ॰मोन्द॰ \msCb, ॰सौन्द॰ \Ed}}% 

{\devanagarifont शङ्खकाहलशब्दं च शब्दं डिण्डिमगोमुखम्  \danda\dontdisplaylinenum }%
 
%Verse 20:56

{\devanagarifont कांसिकातालमिश्रं च गीतानि विविधानि च {॥ २०:५६॥} \veg\dontdisplaylinenum }%
     \var{{\devanagarifontvar \numnoemph\ve\textbf{॰काताल॰}\lem \mssALL, ॰काहल॰ \Ed}}% 


\alalalfejezet{त्वक् (१०)}

{\devanagarifont त्वचया गृह्यते स्पर्शः सुखदुःखसमन्वितः \thinspace{\dandab} \dontdisplaylinenum }%
     \var{{\devanagarifontvar \numemph\va\textbf{गृह्यते}\lem \mssALL, गृह्य\lac\  \msCa}}% 

%Verse 20:57

{\devanagarifont मृदुसूक्ष्म सुखस्पर्शः वस्त्रशय्यासनादयः {॥ २०:५७॥} \veg\dontdisplaylinenum }%
     \var{{\devanagarifontvar \numnoemph\vc\textbf{॰सुख}\lem \mssALL, ॰सुखं \Ed}}% 

{\devanagarifont तीक्ष्णशस्त्रजलशैत्यं उष्णे तप्ते क्षते क्षरः \thinspace{\dandab} \dontdisplaylinenum }%
     \var{{\devanagarifontvar \numemph\va\textbf{॰जलशैत्यं}\lem \msNa, ॰जलं शैत्ये \msCa, ॰जल शैत्य \msCb, ॰जलं शैत्य \Ed}}% 
    \var{{\devanagarifontvar \numnoemph\vb\textbf{उष्णे तप्ते}\lem \msCa\msNa, उष्णे तप्त \msCb, उष्णतप्त \Ed}}% 

%Verse 20:58

{\devanagarifont एवमादीन्यनेकानि ज्ञेयानीष्टं द्विजोत्तम {॥ २०:५८॥} \veg\dontdisplaylinenum }%
 

\alalalfejezet{चक्षुः (९)}

{\devanagarifont चक्षुषा गृह्यते रूपं सहस्राणि शतानि च \thinspace{\dandab} \dontdisplaylinenum }%
 
%Verse 20:59

{\devanagarifont देवरूपविकाराणि नक्षत्रग्रहतारकाः {॥ २०:५९॥} \veg\dontdisplaylinenum }%
 
{\devanagarifont मानुषानां विकाराणि ग्रामं नगरपत्तनम् \thinspace{\dandab} \dontdisplaylinenum }%
 
%Verse 20:60

{\devanagarifont वृक्षगुल्मलतानां च पशुपक्षिशरीसृपाम् {॥ २०:६०॥} \veg\dontdisplaylinenum }%
 
{\devanagarifont कृमिकीटपतङ्गानां जलजानामनेकधा \thinspace{\dandab} \dontdisplaylinenum }%
 
{\devanagarifont शैलदारवहेमानि रूपाणि विविधानि च  \danda\dontdisplaylinenum }%
     \var{{\devanagarifontvar \numemph\vc\textbf{हेमानि}\lem \msNa, ॰होमानि \msCa, ॰भोमानि \msCb, ॰रोमाणि \Ed}}% 

%Verse 20:61

{\devanagarifont धातुद्रव्यविकाराणि रूपाणि द्विजसत्तम {॥ २०:६१॥} \veg\dontdisplaylinenum }%
     \var{{\devanagarifontvar \numnoemph\vf\textbf{द्विजसत्तम}\lem \msCb\msNa\Ed, द्विज\uncl{स}\lac\  \msCa}}% 


\alalalfejezet{जिह्वा (८)}

{\devanagarifont जिह्वया गृह्यते स्वादो हृद्याहृद्यो द्विजोत्तम \thinspace{\dandab} \dontdisplaylinenum }%
     \var{{\devanagarifontvar \numemph\va\textbf{जिह्वया}\lem \msCb\msNa\Ed, \lac  या \msCa\oo 
\textbf{गृह्यते}\lem \msCa\msNa\Ed, गृहत्वे \msCb}}% 

%Verse 20:62

{\devanagarifont फलमूलानि शाकानि कन्दानि पिशितानि च {॥ २०:६२॥} \veg\dontdisplaylinenum }%
 
{\devanagarifont पक्वापक्वविशेषाणि दधिक्षीरघृतानि च \thinspace{\dandab} \dontdisplaylinenum }%
 
{\devanagarifont व्रीह्यौषधरसानां च मिश्रामिश्रमनेकधा  \danda\dontdisplaylinenum }%
     \var{{\devanagarifontvar \numemph\vc\textbf{व्रीह्यौषध॰}\lem \mssALL, व्रीह्योषधि॰ \Ed}}% 

%Verse 20:63

{\devanagarifont षट्कर्मप्रतिभेदेन रसभेदशतं स्मृतम् {॥ २०:६३॥} \veg\dontdisplaylinenum }%
     \var{{\devanagarifontvar \numnoemph\vf\textbf{॰शतं}\lem \mssALL, ॰शत \Ed}}% 


\alalalfejezet{घ्राणम् (७)}

{\devanagarifont घ्राणेन गृह्यते गन्ध इष्टानिष्टो द्विजर्षभ \thinspace{\dandab} \dontdisplaylinenum }%
     \var{{\devanagarifontvar \numemph\vab\textbf{गृह्यते गन्ध इष्टा॰}\lem \msCb\msNa\Ed, 
गृ\uncl{ह्यते ग}\lac  ष्टा॰ \msCa}}% 
    \var{{\devanagarifontvar \numnoemph\vb\textbf{॰निष्टो}\lem \mssALL, ॰निष्टा \Ed\oo 
\textbf{॰र्षभ}\lem \mssALL, ॰र्षभः \Ed}}% 

{\devanagarifont गुडाज्यं गुग्गुलुर्भस्मचन्दनागरुकं तथा  \danda\dontdisplaylinenum }%
     \var{{\devanagarifontvar \numnoemph\vc\textbf{॰ज्यं गुग्गुलुर्भ॰}\lem \msCa\msNa, 
॰ज्यं गुग्गुलं भ॰ \msCb, 
॰ज्यगुग्गुलुभ॰ \Ed}}% 
    \var{{\devanagarifontvar \numnoemph\vd\textbf{॰गरुकं}\lem \mssALL, ॰गुरुकस् \Ed}}% 

%Verse 20:64

{\devanagarifont कस्तूरिकुङ्कुमादीनामिष्टो गन्धो मनोहरः {॥ २०:६४॥} \veg\dontdisplaylinenum }%
     \var{{\devanagarifontvar \numnoemph\vf\textbf{गन्धो}\lem \mssALL, गन्ध \Ed}}% 

{\devanagarifont व्रणमूत्रपुरीषाणां मांसपर्युषितानि च \thinspace{\dandab} \dontdisplaylinenum }%
     \var{{\devanagarifontvar \numemph\vb\textbf{मांस॰}\lem \msCb\msNa\Ed, मास॰ \msCa}}% 

%Verse 20:65

{\devanagarifont वातकर्मादिदुर्गन्ध अनिष्टः समुदाहृतः {॥ २०:६५॥} \veg\dontdisplaylinenum }%
     \var{{\devanagarifontvar \numnoemph\vc\textbf{॰न्ध}\lem \msCa\msCb\Ed, ॰न्धो \msNa}}% 


\alalalfejezet{हस्तकर्म (६)}

{\devanagarifont हस्तेन कुरुते कर्म विविधानि द्विजोत्तम \thinspace{\dandab} \dontdisplaylinenum }%
     \var{{\devanagarifontvar \numemph\va\textbf{हस्तेन}\lem \mssALL, हस्ताभ्यां \Ed}}% 

%Verse 20:66

{\devanagarifont माहेन्द्रं वारुणं चैव वायव्याग्नेयमेव च {॥ २०:६६॥} \veg\dontdisplaylinenum }%
     \var{{\devanagarifontvar \numnoemph\vc\textbf{माहेन्द्रं वारुणं}\lem \msCb\msNa, 
\lac  न्द्रम्वारुणञ् \msCa, मोहेन्द्रवारुणं \Ed}}% 

{\devanagarifont आग्नेय पचनादीनि कांस्यो लोहस्त्रपुस्तथा \thinspace{\dandab} \dontdisplaylinenum }%
     \var{{\devanagarifontvar \numemph\va\textbf{॰पचना॰}\lem \msCa\msCb, ॰पवना॰ \msNa\Ed}}% 

%Verse 20:67

{\devanagarifont अग्निकर्माण्यनेकानि यज्ञहोमक्रियास्तथा {॥ २०:६७॥} \veg\dontdisplaylinenum }%
 
{\devanagarifont सूर्पव्यजनवातेन मुखवातेन वै तथा \thinspace{\dandab} \dontdisplaylinenum }%
     \var{{\devanagarifontvar \numemph\va\textbf{॰व्यजन॰}\lem \msCa\msNa\Ed, ॰व्यज॰ \msCb}}% 
    \var{{\devanagarifontvar \numnoemph\vb\textbf{मुख॰}\lem \msCa\msNa\Ed, सुख॰ \msCb}}% 

%Verse 20:68

{\devanagarifont चमरचर्मवातेन वातयन्त्रं च वायवम् {॥ २०:६८॥} \veg\dontdisplaylinenum }%
 
{\devanagarifont वारुणं तोयकर्माणि कुरुते विविधानि च \thinspace{\dandab} \dontdisplaylinenum }%
     \var{{\devanagarifontvar \numemph\vb\textbf{कुरुते}\lem \msCb\msNa\Ed, कुरु\lac\  \msCa}}% 

%Verse 20:69

{\devanagarifont रसोपरसकर्माणि तस्य पोषणकर्म च {॥ २०:६९॥} \veg\dontdisplaylinenum }%
 
{\devanagarifont स्नानाचमनकर्माणि वस्त्रशौचादयस्तथा \thinspace{\dandab} \dontdisplaylinenum }%
 
%Verse 20:70

{\devanagarifont कायशौचं च कुरुते तृषानाशनमेव च {॥ २०:७०॥} \veg\dontdisplaylinenum }%
     \var{{\devanagarifontvar \numemph\vd\textbf{तृषानाशन॰}\lem \Ed, तृषनाशन॰ \msCa\msNa, तृषणाशत॰ \msCb}}% 

{\devanagarifont वमनानि ह्यनेकानि वारुणं कर्म उच्यते \thinspace{\dandab} \dontdisplaylinenum }%
     \var{{\devanagarifontvar \numemph\va\textbf{वमनानि}\lem \conj, नवमानि \msCa\msNa\Ed, एवमादि \msCb}}% 
    \var{{\devanagarifontvar \numnoemph\vb\textbf{वारुणं}\lem \msCa\msNa\Ed, वारुण॰ \msCb}}% 

%Verse 20:71

{\devanagarifont माहेन्द्रं पार्थिवं कर्म अनेकानि द्विजोत्तम {॥ २०:७१॥} \veg\dontdisplaylinenum }%
 
{\devanagarifont कुलालकर्म भूकर्म कर्म पाषाणमेव च \thinspace{\dandab} \dontdisplaylinenum }%
     \var{{\devanagarifontvar \numemph\va\textbf{कुलालकर्म॰}\lem \msCb\msNa\Ed, कु\uncl{ल}\lac  र्म्म॰ \msCa}}% 
    \var{{\devanagarifontvar \numnoemph\vb\textbf{कर्म}\lem \mssALL, \om\ \msNaacorr, कर्मं \Ed}}% 

{\devanagarifont दारुदन्तिमशृङ्गादिकर्म पार्थिवमुच्यते  \danda\dontdisplaylinenum }%
     \var{{\devanagarifontvar \numnoemph\vc\textbf{॰शृङ्गादि॰}\lem \msCa\msCb\Ed, ॰शृङ्गानि \msNa}}% 
    \var{{\devanagarifontvar \numnoemph\vd\textbf{पार्थिवमु॰}\lem \msCa\msNa\Ed, पार्थिव उ॰ \msCb}}% 

%Verse 20:72

{\devanagarifont चतुष्कर्म समासेन हस्ततः परिकीर्तितम् {॥ २०:७२॥} \veg\dontdisplaylinenum }%
     \var{{\devanagarifontvar \numnoemph\ve\textbf{॰ष्कर्म}\lem \msCb\msNa\Ed, ॰ष्क \msCa}}% 


\alalalfejezet{पादकर्म (५)}

{\devanagarifont पादाभ्यां गमनं कर्म दिशश्च विदिशस्तथा \thinspace{\dandab} \dontdisplaylinenum }%
     \var{{\devanagarifontvar \numemph\va\textbf{गमनं}\lem \msCa\Ed, गमन \msCb, गमनः \msNa}}% 
    \var{{\devanagarifontvar \numnoemph\vb\textbf{दिशश्च विदिशस्}\lem \msCa\msCb\msNa, दिशञ्च विदिशन् \Ed}}% 

{\devanagarifont निम्नोन्नतसमे देशे शिलासंकटकोटरे  \danda\dontdisplaylinenum }%
     \var{{\devanagarifontvar \numnoemph\vd\textbf{शिला॰}\lem \msCa\msCb\Ed, तिल॰ \msNa}}% 

%Verse 20:73

{\devanagarifont तोयकर्दमसंघाते बहुकण्टकसंकुले {॥ २०:७३॥} \veg\dontdisplaylinenum }%
     \var{{\devanagarifontvar \numnoemph\vf\textbf{बहुकण्टक॰}\lem \msCb\Ed, \uncl{बहु}\lac  क॰ \msCa, 
बहुसंकट॰ \msNa\oo 
\textbf{॰कुले}\lem \mssALL, ॰युते \Ed}}% 


\alalalfejezet{पायुकर्म (४)}

{\devanagarifont पायुकर्म विसर्गं तु कठिनद्रवपिच्छलम् \thinspace{\dandab} \dontdisplaylinenum }%
     \var{{\devanagarifontvar \numemph\va\textbf{पायु॰}\lem \msCa\msCb\msNa, पाप॰ \Ed}}% 
    \var{{\devanagarifontvar \numnoemph\vb\textbf{॰पिच्छलम्}\lem \msCa\msNa\Ed, ॰पिच्छिलम् \msCb}}% 

%Verse 20:74

{\devanagarifont सरक्तफेनिलादीनि पायुशक्ति प्रमुञ्चति {॥ २०:७४॥} \veg\dontdisplaylinenum }%
     \var{{\devanagarifontvar \numnoemph\vc\textbf{सरक्त॰}\lem \msCa\msNa\Ed, सक्त॰ \msCb}}% 
    \var{{\devanagarifontvar \numnoemph\vd\textbf{पायुशक्ति}\lem \Ed,  पायुच्छक्ति \msCa\msNa, पायुश्छक्ति \msCb\oo 
\textbf{॰मुञ्चति}\lem \msCa\msCb\msNa, ॰मुञ्चते \Ed}}% 


\alalalfejezet{उपस्थकर्म (३)}

{\devanagarifont उपस्थकर्म आनन्दं करोति जननं प्रजा \thinspace{\dandab} \dontdisplaylinenum }%
     \var{{\devanagarifontvar \numemph\va\textbf{आनन्दं}\lem \msCa\msCb\msNa, आनन्द \Ed}}% 

%Verse 20:75

{\devanagarifont स्त्रीपुंनपुंसकं चैव उपस्थं कुरुते द्विज {॥ २०:७५॥} \veg\dontdisplaylinenum }%
 

\alalalfejezet{वाक्कर्म (२)}

{\devanagarifont वाचा तु कुरुते कर्म नवधा द्विजपुङ्गव \thinspace{\dandab} \dontdisplaylinenum }%
     \var{{\devanagarifontvar \numemph\vb\textbf{॰पुङ्गव}\lem \msCa\msCb\Ed, ॰पुङ्गवः \msNa}}% 

%Verse 20:76

{\devanagarifont स्तुति निन्दा प्रशंसा च आक्रोशः प्रिय एव सः {॥ २०:७६॥} \veg\dontdisplaylinenum }%
     \var{{\devanagarifontvar \numnoemph\vd\textbf{आक्रोशः}\lem \msCb\msNa\Ed, \lac  क्रोशः \msCa}}% 

{\devanagarifont प्रश्नो ऽनुज्ञा तथाख्यानमाशीश्च विधयो नव \thinspace{\dandab} \dontdisplaylinenum }%
     \var{{\devanagarifontvar \numemph\vb\textbf{˚श्च विधयो नव}\lem \msCb, 
˚श्च विधयोनय \msCa, 
˚श्च विधयोनयः \msNa, 
˚श्चाविधियोनयः \Ed}}% 

%Verse 20:77

{\devanagarifont एता नवविधा वाणी कीर्तिता मे द्विजोत्तम {॥ २०:७७॥} \veg\dontdisplaylinenum }%
     \var{{\devanagarifontvar \numnoemph\vd\textbf{कीर्तिता}\lem \msCa\msCb\msNa, कीर्तितो \Ed\oo 
\textbf{॰त्तम}\lem \msCa\msCb\msNa, ॰त्तमः \Ed}}% 


\alalfejezet{मनश्चोन्मनश्च (१)}
{\devanagarifont अधुना कथयिष्यामि मनसो नव वै गुणान् \thinspace{\dandab} \dontdisplaylinenum }%
 
{\devanagarifont चलोपपत्तिः स्थैरं च विसर्ग कल्पना क्षमा  \danda\dontdisplaylinenum }%
     \var{{\devanagarifontvar \numemph\vd\textbf{विसर्ग॰}\lem \msCb\msNa\Ed, विसर्गे \msCa\oo 
\textbf{॰क्षमा}\lem \msCa\msCb\msNa, ॰समा \Ed}}% 

%Verse 20:78

{\devanagarifont सदसच्चाशुता चैव मनसो नव वै गुणाः {॥ २०:७८॥} \veg\dontdisplaylinenum }%
     \var{{\devanagarifontvar \numnoemph\ve\textbf{सदसच्चाशुता चैव}\lem \eme, 
सदसब्बांशुताञ्चैव \msCa, 
सदसताञ्चैव \msCb, 
सदशब्दांशुतां चैव \msNa, 
सदसच्चाशुताश्चैव \Ed}}% 
    \paral{{\devanagarifontsmall \vcdef {\englishfont \similar\ \MBH\ 12.247.9:}
                         चलोपपत्तिर्व्यक्तिश्च विसर्गः कल्पना क्षमा\thinspace{\devanagarifontsmall ।}
                         सदसच्चाशुता चैव मनसो नव वै गुणाः\thinspace{\devanagarifontsmall ॥} }}

{\devanagarifont इष्टानिष्टविकल्पश्च व्यवसायः समाधिता \thinspace{\dandab} \dontdisplaylinenum }%
     \var{{\devanagarifontvar \numemph\va\textbf{इष्टा॰}\lem \msCb\msNa\Ed, \lac  ष्टा॰ \msCa}}% 
    \var{{\devanagarifontvar \numnoemph\vb\textbf{॰सायः}\lem \msCa\msNa\Ed, ॰साय \msCb\oo 
\textbf{समाधिता}\lem  \msCa\msCb\msNa, समाधिना \Ed}}% 
    \paral{{\devanagarifontsmall \vab {\englishfont =\MBH\ 12.247.10ab} }}

%Verse 20:79

{\devanagarifont मनसो द्विविधं रूपं मनश्चोन्मन एव च {॥ २०:७९॥} \veg\dontdisplaylinenum }%
     \var{{\devanagarifontvar \numnoemph\vd\textbf{॰न्मन}\lem \msCb\msNa\Ed, ॰त्मन \msCa}}% 

{\devanagarifont मनस्त्विन्द्रियभावत्वे उन्मनस्त्वमनिन्द्रिये \thinspace{\dandab} \dontdisplaylinenum }%
     \var{{\devanagarifontvar \numemph\va\textbf{॰भावत्वे}\lem \msCa\msNa\Ed, ॰भावेत्वे \msCb}}% 
    \var{{\devanagarifontvar \numnoemph\vb\textbf{॰निन्द्रिये}\lem \corr, ॰नीन्द्रिये \msCa\msCb\msNa, ॰तीन्द्रिय \Ed}}% 

%Verse 20:80

{\devanagarifont निगृहीता विसृष्तं च बन्धमोक्षौ तु साधनम् {॥ २०:८०॥} \veg\dontdisplaylinenum }%
     \var{{\devanagarifontvar \numnoemph\vcd\textbf{निगृहीता विसृष्तं च बन्धमोक्षौ तु साधनम्}\lem \msCa\msCb\msNa, \om\ \Ed}}% 

{\devanagarifont निगृहीतेन्द्रियग्रामः स्वर्गमोक्षौ तु साधनम् \thinspace{\dandab} \dontdisplaylinenum }%
 
%Verse 20:81

{\devanagarifont विसृष्ट इन्द्रियग्रामे दुःखसंसारसाधनम् {॥ २०:८१॥} \veg\dontdisplaylinenum }%
     \var{{\devanagarifontvar \numemph\vc\textbf{॰सृष्ट}\lem \msNa, ॰सृष्टे \msCa\msCb\Ed}}% 
    \var{{\devanagarifontvar \numnoemph\vd\textbf{दुःख॰}\lem \msNa, \lac  ख॰ \msCa, दुःखं \msCb\Ed}}% 

{\devanagarifont सकलं निष्कलं चैव मन एव विदुर्बुधाः \thinspace{\dandab} \dontdisplaylinenum }%
     \var{{\devanagarifontvar \numemph\vb\textbf{मन एव}\lem \msCa\msCb\Ed, मनरेव \msNa}}% 

%Verse 20:82

{\devanagarifont सकलं मन नानात्वे एकत्वे मन निष्कलम् {॥ २०:८२॥} \veg\dontdisplaylinenum }%
 
{\devanagarifont विगतराग उवाच {\dandab}\dontdisplaylinenum  }%
 
{\devanagarifont मनः स्ववेद्यं लोकानामुन्मनस्तु न विद्यते \thinspace{\danda} \dontdisplaylinenum }%
     \var{{\devanagarifontvar \numemph\va\textbf{मनः}\lem \msCa\msNa\Ed, मन \msCb}}% 

%Verse 20:83

{\devanagarifont उन्मनः कथयास्माकं कीदृशं लक्षणं भवेत् {॥ २०:८३॥} \veg\dontdisplaylinenum }%
 
{\devanagarifont अनर्थयज्ञ उवाच {\dandab}\dontdisplaylinenum  }%
 
{\devanagarifont उन्मनस्त्वं गते विप्र निबोध दशलक्षणम् \thinspace{\danda} \dontdisplaylinenum }%
     \var{{\devanagarifontvar \numemph\vb\textbf{निबोध}\lem \msCb\msNa\Ed, \lac  बोध \msCa}}% 

%Verse 20:84

{\devanagarifont न शब्दं शृणुते श्रोत्रं शङ्खभेरीस्वनादपि {॥ २०:८४॥} \veg\dontdisplaylinenum }%
     \var{{\devanagarifontvar \numnoemph\vc\textbf{शब्दं}\lem \msCa\msNa\Ed, शब्द \msCb\oo 
\textbf{श्रोत्रं}\lem \msCa\msCb\msNa, श्रोत्रे \Ed}}% 

{\devanagarifont त्वचः स्पर्शं न जानाति शीतोष्णमपि दुःसहम् \thinspace{\dandab} \dontdisplaylinenum }%
 
%Verse 20:85

{\devanagarifont रूपं पश्यति नो चक्षुः पर्वताभ्यधिको ऽपि वा {॥ २०:८५॥} \veg\dontdisplaylinenum }%
 
{\devanagarifont जिह्वा रसं न विन्देत मधुराम्लवणो ऽपि वा \thinspace{\dandab} \dontdisplaylinenum }%
     \var{{\devanagarifontvar \numemph\vb\textbf{॰राम्लवणो}\lem \corr, ॰राम्लवनो \msCa\msNa, 
॰रो लवनो \msCb, ॰राम्लवतो \Ed}}% 

%Verse 20:86

{\devanagarifont गन्धं जिघ्रति न घ्राणा तीक्ष्णं वाप्यशुचीन्यपि {॥ २०:८६॥} \veg\dontdisplaylinenum }%
     \var{{\devanagarifontvar \numnoemph\vc\textbf{घ्राणा}\lem \mssALL, घ्राणो \Ed}}% 
    \var{{\devanagarifontvar \numnoemph\vd\textbf{वाप्यशुची॰}\lem \mssALL, वापि शुची॰ \msCb}}% 

{\devanagarifont उन्मनस्त्वेष मे ख्यातं सर्वद्वैतविनाशनम् \thinspace{\dandab} \dontdisplaylinenum }%
     \var{{\devanagarifontvar \numemph\va\textbf{उन्मनस्त्वेष मे}\lem \msCb\msNa, 
\lac\  \msCa, 
उन्मनस्तव मे \Ed}}% 

%Verse 20:87

{\devanagarifont भवपारगसुव्यक्तं निष्कलं शिवमव्ययम् {॥ २०:८७॥} \veg\dontdisplaylinenum }%
 
{\devanagarifont स शिवः स परो ब्रह्मा स विष्णुः स परो ऽक्षरः \thinspace{\dandab} \dontdisplaylinenum }%
 
%Verse 20:88

{\devanagarifont स सूक्ष्मः स परो हंसः सो ऽक्षरः क्षरवर्जितः {॥ २०:८८॥} \veg\dontdisplaylinenum }%
 
{\devanagarifont एष उन्मन जानीहि शिवश्च द्विजपुङ्गव \thinspace{\dandab} \dontdisplaylinenum }%
     \var{{\devanagarifontvar \numemph\vb\textbf{॰पुङ्गव}\lem \msCa\msCb\Ed, ॰पुङ्गवः \msNa}}% 

%Verse 20:89

{\devanagarifont कीर्तितो ऽस्मि समासेन किमन्यत्परिपृच्छसि {॥ २०:८९॥} \veg\dontdisplaylinenum }%
     \var{{\devanagarifontvar \numnoemph\vd\textbf{परिपृच्छसि}\lem \msCb\msNa\Ed, प\uncl{रि}\lac\  \msCa}}% 

{\devanagarifont 
\jump
\begin{center}
\ketdanda~इति वृषसारसंग्रहे पञ्चविंशतितत्त्वनिर्णयो नाम विंशतिमो ऽध्यायः~\ketdanda
\end{center}
\dontdisplaylinenum\vers  }%
     \var{{\devanagarifontvar \numnoemph{\englishfont \Colo:}\textbf{॰विंशतितत्त्वनिर्णयो नाम विंशतिमो}\lem \msCa\msNa, 
॰विंशतिमो \msCb, 
॰विंशतितत्त्वनिर्णयो नाम विंशतितमो \Ed}}% 
\bekveg\szamveg
\vfill
\phpspagebreak

\versno=0\fejno=21
\thispagestyle{empty}


\vers

\centerline{\Large\devanagarifontbold [   एकविंशतिमो ऽध्यायः  ]}{\vrule depth10pt width0pt} \fancyhead[CE]{{\footnotesize\devanagarifont वृषसारसंग्रहे  }}
\fancyhead[CO]{{\footnotesize\devanagarifont एकविंशतिमो ऽध्यायः  }}
\fancyhead[LE]{}
\fancyhead[RE]{}
\fancyhead[LO]{}
\fancyhead[RO]{}
\szam\bek



\alalfejezet{विष्णुः स्वरूपं दर्शयति}
{\devanagarifont विगतराग उवाच {\dandab}\dontdisplaylinenum  }%
 
{\devanagarifont अहो मतिमतां श्रेष्ठ अहो धर्मभृतां वर \thinspace{\danda} \dontdisplaylinenum }%
     \var{{\devanagarifontvar \numemph\va\textbf{मतिमतां}\lem \mssCaCbCc\msNa\msNb\msNc, मतिमना \Ed}}% 
    \var{{\devanagarifontvar \numnoemph\vb\textbf{वर}\lem \msCa\msCc\msNa\msNb\msNc, वरः \msCb\Ed}}% 

%Verse 21:1

{\devanagarifont अहो दम शमः सत्य अहो यज्ञ अहो तपः {॥ २१:१॥} \veg\dontdisplaylinenum }%
     \var{{\devanagarifontvar \numnoemph\vc\textbf{दम शमः}\lem \msCa\msCb\msNa\msNb, दमः शमः \msCc\msNc\Ed}}% 

{\devanagarifont अनेनामृतवाक्येन विस्मयो मे परो गतः \thinspace{\dandab} \dontdisplaylinenum }%
     \var{{\devanagarifontvar \numemph\vb\textbf{मे परो गतः}\lem \mssCaCbCc\msNa\msNc\Ed, \lac\  \msNb}}% 

%Verse 21:2

{\devanagarifont प्रीतो ऽस्मि च तपाधारज्ञानाद्भुतरसेन च {॥ २१:२॥} \veg\dontdisplaylinenum }%
     \var{{\devanagarifontvar \numnoemph\vc\textbf{प्रीतो ऽस्मि च}\lem \msCb\msCc\msNa\msNc\Ed, \uncl{प्र् }\lac  च \msCa, \lac\  \msNb}}% 
    \var{{\devanagarifontvar \numnoemph\vd\textbf{तपाधारज्ञानाद्भुतरसेन च}\lem \mssCaCbCc\msNa\msNc, 
\lac\  \msNb, तपाधारज्ञानाद्भूतरसेन च \Ed}}% 

{\devanagarifont किं ददामि वरं ब्रूहि दातास्मि तव चेप्सितम् \thinspace{\dandab} \dontdisplaylinenum }%
     \var{{\devanagarifontvar \numemph\va\textbf{किं ददामि वरं ब्रूहि}\lem \mssCaCbCc\msNa\msNc\Ed, \lac  हि \msNb}}% 
    \var{{\devanagarifontvar \numnoemph\vb\textbf{चेप्सितम्}\lem \msCa\msCc\msNa\msNb\msNc\Ed, चेस्मितम् \msCb}}% 

%Verse 21:3

{\devanagarifont एतच्छ्रुत्वा ततस्तेन प्रत्युवाच शुभां गिरम् {॥ २१:३॥} \veg\dontdisplaylinenum }%
     \var{{\devanagarifontvar \numnoemph\vd\textbf{शुभां गिरम्}\lem \mssCaCbCc\msNa\msNb\msNc, शुभाङ्गिराम् \Ed}}% 

{\devanagarifont [अनर्थयज्ञ उवाच {\dandab}\dontdisplaylinenum ] }%
 
{\devanagarifont को भवान् वरदश्रेष्ठ देवदानवराक्षसाः \thinspace{\danda} \dontdisplaylinenum }%
     \var{{\devanagarifontvar \numemph\va\textbf{भवान्}\lem \msCapcorr\msCb\msCc\msNa\msNb\msNc\Ed, भगवान् \msCaacorr\oo 
\textbf{वरद श्रेष्ठ}\lem \mssCaCbCc\msNa\msNb\msNc, वरदः श्रेष्ठः \Ed}}% 
    \var{{\devanagarifontvar \numnoemph\vb\textbf{॰राक्षसाः}\lem \mssCaCbCc\msNa\msNb\msNc, ॰राक्षसः \Ed}}% 

%Verse 21:4

{\devanagarifont अथवा भगवान्विष्णुर्मम जिज्ञासुरागतः {॥ २१:४॥} \veg\dontdisplaylinenum }%
     \var{{\devanagarifontvar \numnoemph\vd\textbf{॰गतः}\lem \mssCaCbCc\msNa\msNc\Ed, ॰ग्रतः \msNb}}% 

{\devanagarifont व्यक्तं त्वां पुरुषश्रेष्ठ जानामि पुरुषोत्तम \thinspace{\dandab} \dontdisplaylinenum }%
     \var{{\devanagarifontvar \numemph\va\textbf{व्यक्तं त्वां}\lem \msCa\msCb\msNa, व्यक्तत्वं \msCc\Ed, व्यक्तत्व \msNb, व्यक्तं त्वं \msNc\oo 
\textbf{॰श्रेष्ठ}\lem \mssCaCbCc\msNa\msNb\msNc, ॰श्रेष्ठः \Ed}}% 
    \var{{\devanagarifontvar \numnoemph\vb\textbf{पुरुषोत्तम}\lem \msCb\msNa\msNb\msNc, \uncl{पु}\lac  त्तम \msCa, 
पुरुषोत्त\lac\  \msCc, पुरुषोत्तमः \Ed}}% 

%Verse 21:5

{\devanagarifont रूपं दर्शय गोविन्द यद्यस्ति तपसः फलम् {॥ २१:५॥} \veg\dontdisplaylinenum }%
     \var{{\devanagarifontvar \numnoemph\vc\textbf{रूपं दर्शय गोविन्द}\lem \msCa\msCb\msNa\msNb\msNc\Ed, \lac  विन्द \msCc}}% 
    \var{{\devanagarifontvar \numnoemph\vd\textbf{तपसः फलम्}\lem \mssCaCbCc\msNa\msNc\Ed, त\lac\  \msNb}}% 

{\devanagarifont [वैशम्पायन उवाच] }%
 
{\devanagarifont ततस्तु पुण्डरीकाक्षो दर्शयामास स्वां तनुम् \thinspace{\dandab} \dontdisplaylinenum }%
     \var{{\devanagarifontvar \numemph\vab\textbf{ततस्तु पुण्डरीकाक्षो दर्शयामास स्वां तनुम्}\lem \mssCaCbCc\msNa\msNc\Ed, \lac\  \msNb}}% 

%Verse 21:6

{\devanagarifont शङ्खचक्रगदापाणिः पीताम्बरधरो हरिः {॥ २१:६॥} \veg\dontdisplaylinenum }%
     \var{{\devanagarifontvar \numnoemph\vc\textbf{शङ्खचक्रगदापाणिः}\lem \mssCaCbCc\msNa\msNc\Ed, \lac\  \msNb}}% 

{\devanagarifont अनर्थयज्ञस्तं दृष्ट्वा विस्मयं परमं गतः \thinspace{\dandab} \dontdisplaylinenum }%
     \var{{\devanagarifontvar \numemph\vb\textbf{विस्मयं}\lem \mssCaCbCc\msNa\msNb\Ed, विस्मसं \msNc}}% 

%Verse 21:7

{\devanagarifont प्रहर्षमतुलं लब्ध्वा अश्रुपूर्णाकुलेक्षणः {॥ २१:७॥} \veg\dontdisplaylinenum }%
     \var{{\devanagarifontvar \numnoemph\vc\textbf{लब्ध्वा}\lem \msCa\msCb\msNa\msNc\Ed, लब्ब \msCc, \uncl{लद्या} \msNb}}% 

{\devanagarifont वेपमानस्वरेणात्र उवाच च जनार्दनम् \thinspace{\dandab} \dontdisplaylinenum }%
     \var{{\devanagarifontvar \numemph\vab\textbf{वेपमानस्वरेणात्र उवाच च जनार्दनम्}\lem \msCb\msNa\msNb\msNc, 
वेपमान\lac  च च जनार्दनम् \msCa, 
वेपमान\lac  त्र उ\lac\  \msCc, 
वेपमानस्वरेणार्त उवाच च जनार्दनम् \Ed}}% 

%Verse 21:8

{\devanagarifont अद्य मे सफलं जन्म अद्य मे सफलं तपः {॥ २१:८॥} \veg\dontdisplaylinenum }%
     \var{{\devanagarifontvar \numnoemph\vc\textbf{अद्य मे सफलं जन्म}\lem \msCa\msCb\msNa\msNb\msNc\Ed, 
\uncl{अद्य}\lac  \uncl{जन्म} \msCc}}% 
    \paral{{\devanagarifontsmall \vcd {\englishfont = Kūrmapurāṇa 1.11.219 \similar\ MBh 5.113.5ab:}
                                  अद्य मे सफलं जन्म तारितं चाद्य मे कुलम
                          \similar\ {\englishfont MBh 13.14.179a:} 
                                 अद्य जातो ह्य्  अहं देव अद्य मे सफलं तपः }}

\ujvers\nemsloka {
{\devanagarifont नमो नमस्ते ऽस्तु जनादिसम्भवे }%
  \dontdisplaylinenum}    \lacuna{\devanagarifontsmall \vo {\englishfont This verse is omitted in \msCb.} }%
  

\nemslokab

{\devanagarifont नमो नमस्ते ऽस्तु च विश्वरूपिणे  \danda\dontdisplaylinenum }%
     \var{{\devanagarifontvar \numemph\vb\textbf{नमस्ते}\lem \mssCaCbCc\msNapcorr\msNb\msNc\Ed, नमस्तु \msNaacorr\oo 
\textbf{ऽस्तु च विश्वरूपिणे}\lem \mssCaCbCc\msNa\msNc\Ed, \lac\  \msNb}}% 

\nemslokac

{\devanagarifont नमो नमस्ते ऽस्तु जनाभिसम्भवे }%
  \dontdisplaylinenum    \var{{\devanagarifontvar \numnoemph\vc\textbf{नमो नमस्ते ऽस्तु जनाभिसम्भवे}\lem \mssCaCbCc\msNc\Ed, \om\ \msNa, \lac\  \msNb}}% 

%Verse 21:9


\nemslokad

{\devanagarifont नमो नमस्ते ऽस्तु पितामहोद्भवे {॥ २१:९॥} \veg\dontdisplaylinenum }%
     \var{{\devanagarifontvar \numnoemph\vd\textbf{नमो नमस्ते ऽस्तु पितामहोद्भवे}\lem \mssCaCbCc\msNa\msNc\Ed, \lac  होत्तवे \msNb}}% 

\ujvers\nemsloka {
{\devanagarifont नमो नमस्ते ऽस्तु सहस्रशीर्षिणे }%
  \dontdisplaylinenum}    \var{{\devanagarifontvar \numemph\va\textbf{॰शीर्षिणे}\lem \msNa\msNb\msNc\Ed, \om\ \msCa\msCb, ॰शीर्षणे \msCc}}% 
    \lacuna{\devanagarifontsmall \vo {\englishfont This verse is omitted in \msCa\msCb.} }%
  

\nemslokab

{\devanagarifont नमो नमस्ते ऽस्तु सहस्रचक्षुषे  \danda\dontdisplaylinenum }%
 
\nemslokac

{\devanagarifont नमो नमस्ते ऽस्तु सहस्रलिङ्गिने }%
  \dontdisplaylinenum
%Verse 21:10


\nemslokad

{\devanagarifont नमो नमस्ते ऽस्तु सहस्रवक्षसे {॥ २१:१०॥} \veg\dontdisplaylinenum }%
 
\ujvers\nemsloka {
{\devanagarifont नमो नमस्ते ऽस्तु सहस्रमूर्तये }%
  \dontdisplaylinenum}    \lacuna{\devanagarifontsmall \vo {\englishfont This verse is omitted in \msCa.} }%
  

\nemslokab

{\devanagarifont नमो नमस्ते ऽस्तु सहस्रबाहवे  \danda\dontdisplaylinenum }%
 
\nemslokac

{\devanagarifont नमो नमस्ते ऽस्तु सहस्रवक्त्रिणे }%
  \dontdisplaylinenum    \var{{\devanagarifontvar \numemph\vc\textbf{॰वक्त्रिणे}\lem \Ed, \om\ \msCa, ॰चक्रिणे \msCb\msCc\msNc, ॰वक्रिणे \msNa\msNb}}% 

%Verse 21:11


\nemslokad

{\devanagarifont नमो नमस्ते ऽस्तु सहस्रमायिने {॥ २१:११॥} \veg\dontdisplaylinenum }%
 
\ujvers\nemsloka {
{\devanagarifont नमो नमस्ते ऽस्तु वराहरूपिणे }%
  \dontdisplaylinenum}    \lacuna{\devanagarifontsmall \va {\englishfont This pāda is omitted in \msCa.} }%
      \var{{\devanagarifontvar \numemph\va\textbf{नमो नमस्ते ऽस्तु वराहरूपिणे}\lem \msCb\msCc\msNa\msNb\msNc\Ed, \om\ \msCa}}% 


\nemslokab

{\devanagarifont नमो नमस्ते ऽस्तु महीसमुद्धृते  \danda\dontdisplaylinenum }%
 
\nemslokac

{\devanagarifont नमो नमस्ते ऽस्तु च भूतसृष्टिने }%
  \dontdisplaylinenum    \var{{\devanagarifontvar \numnoemph\vc\textbf{॰सृष्टिने}\lem \msCb\msCc\msNa\msNb\msNc\Ed, ॰सृ\lac\  \msCa}}% 

%Verse 21:12


\nemslokad

{\devanagarifont नमो नमस्ते चतुराश्रमाश्रये {॥ २१:१२॥} \veg\dontdisplaylinenum }%
     \var{{\devanagarifontvar \numnoemph\vd\textbf{नमस्ते}\lem \msCa\msCb\msNapcorr\msNb\msNc\Ed, नमस्ते स्तु \msCc\msNaacorr\oo 
\textbf{॰श्रये}\lem \msCa\msCc\msNa\msNb\msNc\Ed, ॰श्रमे \msCb}}% 

\ujvers\nemsloka {
{\devanagarifont नमो नमस्ते नरसिंहरूपिणे }%
  \dontdisplaylinenum}

\nemslokab

{\devanagarifont नमो नमस्ते दितिजोरदारिणे  \danda\dontdisplaylinenum }%
     \var{{\devanagarifontvar \numemph\vb\textbf{नमो नमस्ते दितिजोरदारिणे}\lem \mssCaCbCc\Ed, \om\ \msNa, नमो नमस्ते दितिजोरदारुणे \msNb, 
नमो नमस्ते स्तु दितिजोरदारिणे \msNc, नमो नमस्ते ऽदितिजोरदारणे \Ed}}% 

\nemslokac

{\devanagarifont नमो नमस्ते ऽसुरचक्रसूदने }%
  \dontdisplaylinenum    \var{{\devanagarifontvar \numnoemph\vc\textbf{॰चक्र॰}\lem \conj, ॰शक्र॰ \mssCaCbCc\msNa\msNb\msNc\Ed\oo 
\textbf{॰सूदने}\lem \mssCaCbCc\msNa\msNb\Ed, ॰सूदेने \msNc}}% 

%Verse 21:13


\nemslokad

{\devanagarifont नमो नमस्ते ऽसुरदर्पनाशने {॥ २१:१३॥} \veg\dontdisplaylinenum }%
 
\ujvers\nemsloka {
{\devanagarifont नमो नमस्ते दितिपुत्रदामने }%
  \dontdisplaylinenum}    \var{{\devanagarifontvar \numemph\va\textbf{॰दामने}\lem \msCa\msCb\msNa, ॰वामने \msCc\Ed, ॰नासने \msNb, ॰\uncl{वा}मने \msNc}}% 


\nemslokab

{\devanagarifont नमो नमस्ते बलियज्ञसूदने  \danda\dontdisplaylinenum }%
 
\nemslokac

{\devanagarifont नमो नमस्ते ऽस्तु षडर्धविक्रमे }%
  \dontdisplaylinenum    \var{{\devanagarifontvar \numnoemph\vc\textbf{षडर्धविक्रमे}\lem \msCb\msCc\msNa\msNb\msNc\Ed, \uncl{ष}\lac  क्रमे \msCa}}% 

%Verse 21:14


\nemslokad

{\devanagarifont नमो नमस्ते त्रिदशार्तिनाशने {॥ २१:१४॥} \veg\dontdisplaylinenum }%
 
\ujvers\nemsloka {
{\devanagarifont नमो नमस्ते ऽस्तु अनन्त अच्युते }%
  \dontdisplaylinenum}

\nemslokab

{\devanagarifont नमो नमस्ते जगदर्तिनाशने  \danda\dontdisplaylinenum }%
     \var{{\devanagarifontvar \numemph\vb\textbf{जगद॰}\lem \mssCaCbCc\msNa\msNb\msNc, जगदा॰ \Ed}}% 

\nemslokac

{\devanagarifont नमो नमस्ते मधुकैटनाशने }%
  \dontdisplaylinenum    \var{{\devanagarifontvar \numnoemph\vc\textbf{॰कैट॰}\lem \msCa\msCb\msNa\msNc, ॰कीट॰ \msCc\msNb\Ed}}% 

%Verse 21:15


\nemslokad

{\devanagarifont नमो नमस्ते ऽस्तु त्रिलोकबान्धवे {॥ २१:१५॥} \veg\dontdisplaylinenum }%
 
\ujvers\nemsloka {
{\devanagarifont नमो नमस्ते त्रिदशाभिनन्दने }%
  \dontdisplaylinenum}

\nemslokab

{\devanagarifont नमो नमस्ते ऽस्तु च दिव्यचक्षुषे  \danda\dontdisplaylinenum }%
 
\nemslokac

{\devanagarifont नमो नमस्ते ऽस्तु भवान्तपारगे }%
  \dontdisplaylinenum
%Verse 21:16


\nemslokad

{\devanagarifont नमो नमस्ते ऽस्तु त्रिलोकपूजिते {॥ २१:१६॥} \veg\dontdisplaylinenum }%
 
\ujvers\nemsloka {
{\devanagarifont नमो नमस्ते ऽस्तु गदाग्रपाणये }%
  \dontdisplaylinenum}

\nemslokab

{\devanagarifont नमो नमस्ते वरचक्रपाणये  \danda\dontdisplaylinenum }%
     \var{{\devanagarifontvar \numemph\vb\textbf{वरचक्रपाणये}\lem \mssCaCbCc\msNa\msNb\Ed, वरक्रपाणसे \msNc}}% 

\nemslokac

{\devanagarifont नमो नमस्ते ऽस्तु च शङ्खपाणये }%
  \dontdisplaylinenum
%Verse 21:17


\nemslokad

{\devanagarifont नमो नमस्ते ऽस्तु च कम्बुपाणये {॥ २१:१७॥} \veg\dontdisplaylinenum }%
 
\ujvers\nemsloka {
{\devanagarifont नमो नमस्ते ऽस्तु जलौघशायिने }%
  \dontdisplaylinenum}    \var{{\devanagarifontvar \numemph\va\textbf{जलौघ॰}\lem \msCa\msCb\Ed, जलोघ॰ \msCc\msNa\msNb\msNc}}% 


\nemslokab

{\devanagarifont नमो नमस्ते हरमर्दरूपिणे  \danda\dontdisplaylinenum }%
     \var{{\devanagarifontvar \numnoemph\vb\textbf{नमस्ते हरमर्दरूपिणे}\lem \msCb\msNa\msNb\msNc\Ed, 
नम\lac  र्द्दरूपिणे \msCa, 
\lac  मर्द्दरूपिणे \msCc}}% 

\nemslokac

{\devanagarifont नमो नमस्ते खगराजकेतवे }%
  \dontdisplaylinenum    \var{{\devanagarifontvar \numnoemph\vc\textbf{॰केतवे}\lem \mssCaCbCc\msNa\msNb\msNc, ॰केतने \Ed}}% 

%Verse 21:18


\nemslokad

{\devanagarifont नमो नमस्ते शशिसूर्यलोचने {॥ २१:१८॥} \veg\dontdisplaylinenum }%
 
\ujvers\nemsloka {
{\devanagarifont नमो नमस्ते उरगारिवाहने }%
  \dontdisplaylinenum}

\nemslokab

{\devanagarifont नमो नमस्ते ऽद्भुतरूपदर्शिने  \danda\dontdisplaylinenum }%
     \var{{\devanagarifontvar \numemph\vb\textbf{॰दर्शिने}\lem \mssCaCbCc\msNa\msNc\Ed, ॰दर्शने \msNb}}% 

\nemslokac

{\devanagarifont नमो नमस्ते ऽयुतसूर्यतेजसे }%
  \dontdisplaylinenum    \var{{\devanagarifontvar \numnoemph\vc\textbf{ऽयुत॰}\lem \mssCaCbCc\msNa\msNb\msNc, ऽस्तु च \Ed\oo 
\textbf{॰तेजसे}\lem \msCa\msCb\msNa\msNc\Ed, ॰ते \msCc, ॰लोचने \msNb}}% 

%Verse 21:19


\nemslokad

{\devanagarifont नमो नमस्ते ऽमृतमन्थनध्रुवे {॥ २१:१९॥} \veg\dontdisplaylinenum }%
 
\ujvers\nemsloka {
{\devanagarifont नमो नमस्ते ऽमरलोकसंस्तुते }%
  \dontdisplaylinenum}    \var{{\devanagarifontvar \numemph\va\textbf{ऽमरलोकसंस्तुते}\lem \mssCaCbCc\msNb\msNc, मरलोकवन्दिते \msNa, मललोकसंस्तुते \Ed}}% 


\nemslokab

{\devanagarifont नमो नमस्ते जगमण्डपाश्रये  \danda\dontdisplaylinenum }%
     \var{{\devanagarifontvar \numnoemph\vb\textbf{नमो नमस्ते जगमण्डपाश्रये}\lem \msCa\msCb\msNc\Ed, 
\lac  \uncl{श्रये} \msCc, \om\ \msNa, नमो नमस्ते जगमण्डलाश्रये \msNb}}% 

\nemslokac

{\devanagarifont नमो नमस्ते जगदेकवत्सले }%
  \dontdisplaylinenum    \var{{\devanagarifontvar \numnoemph\vc\textbf{जगदेक॰}\lem  \msCa\msCb\msNa\msNb\msNc\Ed, जग\uncl{दे}क॰ \msCc\oo 
\textbf{॰वत्सले}\lem \mssCaCbCc\msNa\msNb\msNc, ॰वत्सरे \Ed}}% 

%Verse 21:20


\nemslokad

{\devanagarifont नमो नमस्ते शिवसर्वदे नमः {॥ २१:२०॥} \veg\dontdisplaylinenum }%
     \var{{\devanagarifontvar \numnoemph\vd\textbf{॰सर्वदे}\lem \mssCaCbCc\msNa\msNc\Ed, ॰सर्वदो \msNb}}% 
    \paral{{\devanagarifontsmall \vd {\englishfont Cf.\ Bṛhatkālottara (NGMPP B 29/59) f.\ 87a:} 
                 ज्ञान २ शब्द २ सूक्ष्म २ शिवसर्वद ओं नमः शिवाय\thinspace{\devanagarifontsmall ।} }}

\ujvers\nemsloka {
{\devanagarifont क्षमस्व गोविन्द ममापराधम् }%
  \dontdisplaylinenum}    \var{{\devanagarifontvar \numemph\va\textbf{ममा॰}\lem \msCa\msCb\msNa\msNb\msNc\Ed, मम \msCc}}% 


\nemslokab

{\devanagarifont अतीव पृष्टेन दुरात्मनेन  \danda\dontdisplaylinenum }%
     \var{{\devanagarifontvar \numnoemph\vb\textbf{॰त्मनेन}\lem \msCa\msCc\msNa\msNb\msNc\Ed, ॰त्मने \msCb}}% 

\nemslokac

{\devanagarifont मयेद सर्वं कथितं स्मयेन }%
  \dontdisplaylinenum    \var{{\devanagarifontvar \numnoemph\vc\textbf{मयेद}\lem \mssCaCbCc\msNa\msNb\msNc, मयेदं \Ed\ \unmetr}}% 

%Verse 21:21


\nemslokad

{\devanagarifont दयां कुरु त्वं त्रिदशेश्वरेण {॥ २१:२१॥} \veg\dontdisplaylinenum }%
     \var{{\devanagarifontvar \numnoemph\vd\textbf{॰शेश्वरेण}\lem \msCa\msCc\msNa\msNb\msNc\Ed, ॰शैश्वरेण \msCb}}% 

\vers


{\devanagarifont वैशम्पायन उवाच {\dandab}\dontdisplaylinenum  }%
 
{\devanagarifont स्तोत्रेणानेन संतुष्टः केशवः परवीरहा \thinspace{\danda} \dontdisplaylinenum }%
     \var{{\devanagarifontvar \numemph\va\textbf{स्तोत्रे॰}\lem \mssCaCbCc\msNa\Ed, स्त्रोत्रे॰ \msNb, \uncl{स्तो}त्रे॰ \msNc}}% 
    \var{{\devanagarifontvar \numnoemph\vb\textbf{केशवः परवीरहा}\lem \msCa\msCb\msNa\msNb\msNc\Ed, 
केशव\uncl{ः परवीरहा} \msCc\ \toplost}}% 

%Verse 21:22

{\devanagarifont प्रत्युवाच महासेनो गिरया निरुपस्पृहा {॥ २१:२२॥} \veg\dontdisplaylinenum }%
     \var{{\devanagarifontvar \numnoemph\vc\textbf{प्रत्युवाच}\lem \msCa\msCb\msNa\msNb\msNc\Ed, \uncl{प्रत्युवाच} \msCc\ \toplost\oo 
\textbf{महासेनो}\lem \msCb\msCc\msNa\msNb\msNc\Ed, म\lac  \msCa}}% 
    \var{{\devanagarifontvar \numnoemph\vd\textbf{गिरया}\lem \msCa\msCb\msNa\msNb\msNc\Ed, गिरिया \msCc\oo 
\textbf{निरुप॰}\lem \mssCaCbCc\msNa\msNb, निरूप॰ \msNc\Ed}}% 

{\devanagarifont स्तोत्रेणानेन मे तात तुष्टो ऽस्मि भृशमेजितः \thinspace{\dandab} \dontdisplaylinenum }%
     \var{{\devanagarifontvar \numemph\va\textbf{स्तोत्रे॰}\lem \mssCaCbCc\msNa\msNc\Ed, स्त्रोत्रे॰ \msNb\oo 
\textbf{मे तात}\lem \msCa\msCc\msNb\msNc\Ed, मत्तात \msCb, संतात \msNa}}% 

%Verse 21:23

{\devanagarifont दुर्लभान्यपि त्रैलोक्ये ददामि वरमीप्सितम् {॥ २१:२३॥} \veg\dontdisplaylinenum }%
     \var{{\devanagarifontvar \numnoemph\vc\textbf{त्रैलोक्ये}\lem \msCa\msCb\msNa\msNb\msNc\Ed, त्रैलोक्य \msCc}}% 

\ujvers\nemsloka {
{\devanagarifont अनेन मां स्तौति निराश्रितेन }%
  \dontdisplaylinenum}    \var{{\devanagarifontvar \numemph\va\textbf{स्तौति}\lem \msCa\msCb\msNa\msNc\Ed, स्तोति \msCc\msNb}}% 


\nemslokab

{\devanagarifont त्वयोक्तवेदार्थमनोहरेण  \danda\dontdisplaylinenum }%
     \var{{\devanagarifontvar \numnoemph\vb\textbf{॰वेदार्थ॰}\lem \mssCaCbCc\msNa\msNc\Ed, ॰वेदार्थि॰ \msNb}}% 

\nemslokac

{\devanagarifont यावन्ति तत्राक्षरसंख्यमस्ति }%
  \dontdisplaylinenum
%Verse 21:24


\nemslokad

{\devanagarifont तावन्ति कल्पान्दिवि ते वसन्ति {॥ २१:२४॥} \veg\dontdisplaylinenum }%
     \var{{\devanagarifontvar \numnoemph\vd\textbf{कल्पान्}\lem \msCa\msNa, कल्पं \msCb\msNb\msNc\Ed, कल्प \msCc}}% 

\ujvers\nemsloka {
{\devanagarifont त्वं चापि मे ब्रूहि वरं यथेष्टं }%
  \dontdisplaylinenum}    \var{{\devanagarifontvar \numemph\va\textbf{त्वं चापि मे ब्रूहि}\lem \msCb\msCc\msNa\msNb\msNc\Ed, 
त्वञ्च् \lac  हि \msCa}}% 


\nemslokab

{\devanagarifont त्रैलोक्यराज्यादपि निर्विशङ्कम्  \danda\dontdisplaylinenum }%
     \var{{\devanagarifontvar \numnoemph\vb\textbf{॰राज्या॰}\lem \mssCaCbCc\msNb\msNc\Ed, ॰रा॰ \msNaacorr, ॰राजा॰ \msNapcorr\oo 
\textbf{॰शङ्कम्}\lem \mssCaCbCc\msNa\msNb\msNc, ॰शङ्क \Ed}}% 

\nemslokac

{\devanagarifont ददामि किं सप्तमहीश्वरत्वम् }%
  \dontdisplaylinenum    \var{{\devanagarifontvar \numnoemph\vc\textbf{किं}\lem \msCa\msCc\msNa\msNb\msNc\Ed, कि \msCb\oo 
\textbf{॰त्वम्}\lem \msCa\msCb\msNa\msNb\msNc\Ed, ॰त्वंम् \msCc}}% 

%Verse 21:25


\nemslokad

{\devanagarifont अथार्थराशिं बहुकन्यकां वा {॥ २१:२५॥} \veg\dontdisplaylinenum }%
     \var{{\devanagarifontvar \numnoemph\vd\textbf{अथार्थराशिं}\lem \msCa\msCb\msNa\msNc, अथार्थराशि \msCc, अर्थार्थरासि \msNb, 
अथार्थं राशीं \Ed\oo 
\textbf{॰कन्यकां वा}\lem \msNa\msNc\Ed, ॰कन्यका वा \msCa\msCc\msNb, ॰कन्यका\lac   \ \msCb}}% 

\vers


{\devanagarifont वैशम्पायन उवाच {\dandab}\dontdisplaylinenum  }%
     \var{{\devanagarifontvar \numemph\vo\textbf{वैशम्पायन उवाच}\lem \eme, अनर्थयज्ञ उवाच \msCa\msCb\msNa\msNc, 
विगतराग उवाच \msCc\msNb, \om\ \Ed}}% 

\nemsloka 
{\devanagarifont श्रुत्वैव दिव्यं वरमच्युतस्य }%
  \dontdisplaylinenum    \var{{\devanagarifontvar \numnoemph\va\textbf{श्रुत्वैव}\lem \mssCaCbCc\msNa\msNc\Ed, श्रुतैव \msNb\oo 
\textbf{वरमच्युतस्य}\lem \mssCaCbCc\msNa\msNb\Ed, वरमुच्युतस्य \msNc}}% 


\nemslokab

{\devanagarifont प्रणम्य पादद्वयपङ्कजे तु  \danda\dontdisplaylinenum }%
     \var{{\devanagarifontvar \numnoemph\vb\textbf{॰जे तु}\lem \msCa\msCc\msNb\msNc\Ed, ॰हेतु \msCb, ॰जे नु \msNa}}% 

\nemslokac

{\devanagarifont विज्ञाय विष्णुं वरदं वरेण्यं }%
  \dontdisplaylinenum    \var{{\devanagarifontvar \numnoemph\vcd\textbf{(विज्ञाय{\englishfont ...} ऽब्रवीत)}\lem \Ed, \om\ \mssCaCbCc\msNa\msNb\msNc}}% 

%Verse 21:26


\nemslokad

{\devanagarifont ? प्रहृ चेतः पुकान्चितो ऽतो ऽब्रवीत् {॥ २१:२६॥} \veg\dontdisplaylinenum }%
 
\ujvers\nemsloka {
{\devanagarifont न कामये ऽन्यप्रवरं तु देव }%
  \dontdisplaylinenum}    \var{{\devanagarifontvar \numemph\va\textbf{न कामये}\lem \mssCaCbCc\msNa\msNb\msNc, अनर्थयज्ञ उवाच न कामये \Ed\oo 
\textbf{ऽन्यप्रवरं तु}\lem \msCa\msCc\msNa\msNb\msNc, न्यप्रभवन्तु \msCb, ऽन्यं प्रवरं तु \Ed\oo 
\textbf{देव}\lem \mssCaCbCc\msNa\msNc\Ed, देदेव \msNb}}% 


\nemslokab

{\devanagarifont असंशयं बन्धनसारमेकम्  \danda\dontdisplaylinenum }%
     \var{{\devanagarifontvar \numnoemph\vb\textbf{असंशयं}\lem \mssCaCbCc\msNa\msNc, असंशय \msNb\Ed\oo 
\textbf{॰सारमेकम्}\lem \msCb\msCc\msNa\msNb\msNc\Ed, ॰सारमे\lac\  \msCa}}% 

\nemslokac

{\devanagarifont विमुक्तबन्धो भवतः प्रसादाद् }%
  \dontdisplaylinenum    \var{{\devanagarifontvar \numnoemph\vc\textbf{विमुक्तबन्धो}\lem \msCb\msCc\msNa\msNb\msNc\Ed, \lac\  \msCa\oo 
\textbf{प्रसादाद्}\lem \mssCaCbCc\msNa\msNb\msNc, प्रमादाद् \Ed}}% 

%Verse 21:27


\nemslokad

{\devanagarifont भवामि गोविन्द रतश्च धर्मे {॥ २१:२७॥} \veg\dontdisplaylinenum }%
     \var{{\devanagarifontvar \numnoemph\vd\textbf{रतश्च}\lem \mssCaCbCc\msNa\msNb\msNc, रतञ्च \Ed}}% 

\vers


{\devanagarifont भगवानुवाच {\dandab}\dontdisplaylinenum  }%
 
\nemsloka 
{\devanagarifont यथैव चित्तं तव सुप्रसन्नं }%
  \dontdisplaylinenum

\nemslokab

{\devanagarifont महर्षिदेवैरपि नैव दृष्टम्  \danda\dontdisplaylinenum }%
 
\nemslokac

{\devanagarifont अकल्मषं दुःखविवर्जितत्वम् }%
  \dontdisplaylinenum    \var{{\devanagarifontvar \numemph\vc\textbf{अकल्मषं}\lem \msNb\Ed, अकल्मषस्त्वं \msCa\msNa\ \unmetr, अकल्मषत्वं \msCb\msNc\ \unmetr, 
अकल्मत्वं \msCc\ \unmetr\oo 
\textbf{दुःख॰}\lem \msCapcorr\msCb\msCc\msNa\msNb\msNc\Ed, दुः॰ \msCaacorr}}% 

%Verse 21:28


\nemslokad

{\devanagarifont भवार्णवस्तीर्णमसंशयेन {॥ २१:२८॥} \veg\dontdisplaylinenum }%
 
\ujvers\nemsloka {
{\devanagarifont गच्छाम भो साम्प्रत श्वेतद्वीपम् }%
  \dontdisplaylinenum}    \var{{\devanagarifontvar \numemph\va\textbf{गच्छाम भो}\lem \mssCaCbCc\msNa\msNb\msNc, गच्छामतो \Ed\oo 
\textbf{साम्प्रत}\lem \mssCaCbCc\msNa\msNb\msNc, सम्प्रति \Ed\oo 
\textbf{॰द्वीपम्}\lem \mssCaCbCc\msNa\msNc\Ed, ॰द्वीप \msNb}}% 


\nemslokab

{\devanagarifont अगम्य देवैरपि दुर्निरीक्ष्यम्  \danda\dontdisplaylinenum }%
     \var{{\devanagarifontvar \numnoemph\vb\textbf{दुर्निरीक्ष्यम्}\lem \msCb\msCc\msNa\msNb\Ed, 
दुर्निरी\uncl{क्ष् }\lac\  \msCa, दुर्निरीक्षं \msNc}}% 

\nemslokac

{\devanagarifont मद्भक्तिपूतमनसा प्रयाति }%
  \dontdisplaylinenum    \var{{\devanagarifontvar \numnoemph\vc\textbf{मद्भक्ति॰}\lem \msCb\msCc\msNa\msNb\msNc\Ed, \lac  क्ति \msCa\oo 
\textbf{॰पूत}\lem \mssCaCbCc\msNa, ॰पूतं \msNb\msNc\Ed}}% 

%Verse 21:29


\nemslokad

{\devanagarifont घोरार्णवे नैव पुनश्चरन्ति {॥ २१:२९॥} \veg\dontdisplaylinenum }%
 
\vers


{\devanagarifont वैशम्पायन उवाच {\dandab}\dontdisplaylinenum  }%
     \var{{\devanagarifontvar \numemph\vo\textbf{वैशम्पायन उवाच}\lem \msCa\Ed, \om\ \msCb\msCc\msNaacorr\msNb\msNc, 
वैशं उ \msNapcorr}}% 

{\devanagarifont एवमुक्त्वा हरिस्तत्र करे गृह्य तपोधनम् \thinspace{\danda} \dontdisplaylinenum }%
     \var{{\devanagarifontvar \numnoemph\vb\textbf{गृह्य तपोधनम्}\lem \mssCaCbCc\msNb\msNc\Ed, \lac  ध\uncl{न} \msNa}}% 

%Verse 21:30

{\devanagarifont ततः सो ऽन्तर्हितस्तत्र तेनैव सह केशवः {॥ २१:३०॥} \veg\dontdisplaylinenum }%
     \var{{\devanagarifontvar \numnoemph\vc\textbf{ततस्सो ऽन्तर्हितस्त॰}\lem \msCa\msCb, 
त\uncl{तः सो न्त}र्हितस्त॰ \msNa, 
ततस्ते न्तर्हितास्त॰ \msCc, 
ततस्ते त्तर्हितास्त॰ \msNb, 
ततेंस्ते तर्हितस्त॰ \msNc, 
ततस्ते कर्हितास्त॰ \Ed}}% 
    \var{{\devanagarifontvar \numnoemph\vd\textbf{केशवः}\lem \mssCaCbCc\msNa\msNb\msNc, केशव \Ed}}% 

\ujvers\nemsloka {
{\devanagarifont एवं हि धर्मस्त्वधिकप्रभावाद् }%
  \dontdisplaylinenum}    \var{{\devanagarifontvar \numemph\va\textbf{अधिक॰}\lem \msCa\msCb\msNa\msNc, अधिकं \msCc\msNb\Ed}}% 


\nemslokab

{\devanagarifont गतः स लोकं पुरुषोत्तमस्य  \danda\dontdisplaylinenum }%
     \var{{\devanagarifontvar \numnoemph\vb\textbf{गतः}\lem \mssCaCbCc\msNa\msNb\Ed, गता \msNc\oo 
\textbf{लोकं}\lem \mssCaCbCc\msNa\msNc\Ed, लोक \msNb}}% 

\nemslokac

{\devanagarifont अशेषभूतप्रभवाव्ययस्य }%
  \dontdisplaylinenum
%Verse 21:31


\nemslokad

{\devanagarifont सनातनं शाश्वतमक्षरस्य {॥ २१:३१॥} \veg\dontdisplaylinenum }%
     \var{{\devanagarifontvar \numnoemph\vd\textbf{सनातनं}\lem \Ed, सनातन \msCa\ \toplost\ 
\msCb\msCc\msNa\msNb\msNc\oo 
\textbf{॰क्षरस्य}\lem \msCb\msCc\msNa\msNb\msNc\Ed, \lac\  \msCa}}% 

\ujvers\nemsloka {
{\devanagarifont त्वमेव भक्तिं कुरु केशवस्य }%
  \dontdisplaylinenum}    \var{{\devanagarifontvar \numemph\va\textbf{त्वमेव}\lem \msCb\msCc\msNa\msNb\msNc\Ed, \lac  मेव \msCa}}% 


\nemslokab

{\devanagarifont जनार्दनस्यामितविक्रमस्य  \danda\dontdisplaylinenum }%
 
\nemslokac

{\devanagarifont यथा हि तस्यैव द्विजर्षभस्य }%
  \dontdisplaylinenum    \var{{\devanagarifontvar \numnoemph\vc\textbf{हि तस्यैव}\lem \msCa\msCb\msNc\Ed, जितस्यैव \msCc\msNb, \uncl{हि तस्यव} \toplost\ \msNa}}% 

%Verse 21:32


\nemslokad

{\devanagarifont गतिं लभस्व पुरुषोत्तमस्य {॥ २१:३२॥} \veg\dontdisplaylinenum }%
     \var{{\devanagarifontvar \numnoemph\vd\textbf{लभस्व}\lem \mssCaCbCc\msNa\msNb\Ed, लभत्वं \msNc}}% 

\ujvers\nemsloka {
{\devanagarifont किमन्य भूयः कथयामि राजन् }%
  \dontdisplaylinenum}    \var{{\devanagarifontvar \numemph\va\textbf{किमन्य भू॰}\lem \msCc\Ed\msNb\msNc, किमन्यद्भू॰ \msCa\msCb\msNa\ \unmetr\oo 
\textbf{राजन्}\lem \mssCaCbCc\msNa\msNb\Ed, राजद् \msNc}}% 


\nemslokab

{\devanagarifont यदस्ति कौतूहलमन्यशेषम्  \danda\dontdisplaylinenum }%
 
\nemslokac

{\devanagarifont पृच्छस्व मां तात यथेप्सितं ते }%
  \dontdisplaylinenum
%Verse 21:33


\nemslokad

{\devanagarifont भविष्यभूतं भवतो यथेष्टम् {॥ २१:३३॥} \veg\dontdisplaylinenum }%
     \var{{\devanagarifontvar \numnoemph\vd\textbf{भविष्य॰}\lem \mssCaCbCc\msNa\Ed, भवस्व॰ \msNb, भवस्य \msNc\oo 
\textbf{भवतो}\lem \mssCaCbCc\msNa\msNb\Ed, भवते \msNc\oo 
\textbf{यथेष्टम्}\lem \msCa\msCb\msNa\msNb\msNc\Ed, यथेष्ट \msCc}}% 

\vers


{\devanagarifont जनमेजय उवाच {\dandab}\dontdisplaylinenum  }%
     \var{{\devanagarifontvar \numemph\vo\textbf{जनमेजय उवाच}\lem \mssCaCbCc\msNapcorr\msNb\Ed, \om\ \msNaacorr, जयमेजय उवाच \msNc}}% 

\nemsloka 
{\devanagarifont कियन्ति कल्पानि गतानि पूर्वम् }%
  \dontdisplaylinenum    \var{{\devanagarifontvar \numnoemph\va\textbf{कियन्ति}\lem \mssCaCbCc\msNa\msNb\msNc, कियन्त \Ed}}% 


\nemslokab

{\devanagarifont भविष्यकल्पानि कियन्ति विप्र  \danda\dontdisplaylinenum }%
     \var{{\devanagarifontvar \numnoemph\vb\textbf{कियन्ति}\lem \mssCaCbCc\msNa\msNb\Ed, कियन्त \msNc}}% 

\nemslokac

{\devanagarifont एकैककल्पं कियदिन्द्रमुक्तम् }%
  \dontdisplaylinenum    \var{{\devanagarifontvar \numnoemph\vc\textbf{॰कल्पं}\lem \msCa\msCc\msNa\msNb\msNc\Ed, ॰कल्प \msCb\msNa}}% 

%Verse 21:34


\nemslokad

{\devanagarifont प्रवर्तमानादपि कीर्तयस्व {॥ २१:३४॥} \veg\dontdisplaylinenum }%
 
\vers


{\devanagarifont वैशम्पायन उवाच {\dandab}\dontdisplaylinenum  }%
     \var{{\devanagarifontvar \numemph\vo\textbf{वैशम्पायन उवाच}\lem \msCa\msCc\msNa\Ed, वेशनम्पायन उवाच \msCb, \lk\lk \lk\lk \lk\lk \lk\lk\ \msNb, वैशंपाराय उवाच \msNc}}% 

\nemsloka 
{\devanagarifont परार्धकल्पं गत पूर्व राज्यम् }%
  \dontdisplaylinenum

\nemslokab

{\devanagarifont चतुर्दशैवेन्द्र नरेन्द्र कल्पम्  \danda\dontdisplaylinenum }%
 
\nemslokac

{\devanagarifont तथैव मन्वन्तर कल्पमेकम् }%
  \dontdisplaylinenum    \var{{\devanagarifontvar \numnoemph\vc\textbf{मन्वन्तरकल्पमेकम्}\lem \mssCaCbCc\msNapcorr\msNb, मन्वरकल्पमेकम् \msNaacorr, 
मण्वन्तरकल्पमेकं \msNc, मन्वन्तरमेककल्पम् \Ed}}% 

%Verse 21:35


\nemslokad

{\devanagarifont भविष्यकल्पं च परार्धमेव {॥ २१:३५॥} \veg\dontdisplaylinenum }%
     \var{{\devanagarifontvar \numnoemph\vd\textbf{कल्पं च परार्धमेव}\lem \msCb\msCc\msNa\msNc\Ed, 
क\lac\  \msCa, भूतञ्च परार्द्धतञ्च \eyeskip{\englishfont तो २१.३३द्} \msNb}}% 

\ujvers\nemsloka {
{\devanagarifont वराहकल्पः प्रथमो बभूव }%
  \dontdisplaylinenum}    \var{{\devanagarifontvar \numemph\va\textbf{बभूव}\lem \msCa\msCc\msNa\msNb\msNc\Ed, बभू \msCb}}% 


\nemslokab

{\devanagarifont गताश्च मन्वन्तर षड् नरेन्द्र  \danda\dontdisplaylinenum }%
     \var{{\devanagarifontvar \numnoemph\vb\textbf{मन्वन्तर षड् नरेन्द्र}\lem \msCa\msCb\msNapcorr\msNb\msNc, 
मन्वरषट्नरेन्द्र \msCc, 
मन्वन्तषट्नरेन्द्र \msNaacorr, 
मन्वन्तरषट्नरेन्द्रः \Ed}}% 

\nemslokac

{\devanagarifont चतुर्युगं सप्तति एकयुक्तं }%
  \dontdisplaylinenum    \var{{\devanagarifontvar \numnoemph\vc\textbf{॰युगं}\lem \mssCaCbCc\msNa\msNb\msNc, ॰युग॰ \Ed\oo 
\textbf{एकयुक्तं}\lem \msCa\msCc\msNa\msNb\msNc\Ed, एकमुक्तं \msCb}}% 

%Verse 21:36


\nemslokad

{\devanagarifont मन्वन्तरा संख्यमुदाहरन्ति {॥ २१:३६॥} \veg\dontdisplaylinenum }%
 
\ujvers\nemsloka {
{\devanagarifont मन्वन्तराणां च चतुर्दशैव }%
  \dontdisplaylinenum}    \var{{\devanagarifontvar \numemph\va\textbf{मन्वन्तराणां च चतुर्दशैव}\lem \msCa\msCb\msNa\msNc, मन्वन्तराणान्तु चतुर्दशैव \msCc\msNb, \om\ \Ed}}% 


\nemslokab

{\devanagarifont कल्पस्य संख्या मुनयो वदन्ति  \danda\dontdisplaylinenum }%
 
\nemslokac

{\devanagarifont कल्पायुतश्चाह पितामहस्य }%
  \dontdisplaylinenum    \var{{\devanagarifontvar \numnoemph\vc\textbf{कल्पायुतश्चाह}\lem \mssCaCbCc\msNa\Ed, \lk\lk\lk\lk\lk ह \msNb, कल्पायुतश्चाह \msNc}}% 

%Verse 21:37


\nemslokad

{\devanagarifont तथा च रात्रिं प्रवदन्ति तज्ज्ञाः {॥ २१:३७॥} \veg\dontdisplaylinenum }%
     \var{{\devanagarifontvar \numnoemph\vd\textbf{रात्रिं}\lem \mssCaCbCc\msNa\msNc\Ed, रात्रि \msNb}}% 

\ujvers\nemsloka {
{\devanagarifont षड्लक्षकल्पेन तु मासमाहुस् }%
  \dontdisplaylinenum}    \var{{\devanagarifontvar \numemph\va\textbf{लक्षकल्पेन तु मासमाहुस्}\lem \msCb\msCc\msNa\msNb\msNc\Ed, 
लक्षक\lac  माहुस् \msCa}}% 

%Verse 21:38


\nemslokad

{\devanagarifont तद्द्वादशा वर्षमुदाहरन्ति {॥ २१:३८॥} \veg\dontdisplaylinenum }%
     \var{{\devanagarifontvar \numnoemph\vb\textbf{तद्द्वादशा व॰}\lem \corr, तद्वादशा व॰ \msCa\msCb\msNb, ततद्वादशा व॰ \msCc, 
तद्वादशाद्व॰ \msNa, तद्द्वादशाद्व॰ \msNc, त्वद्द्वादशव॰ \Ed}}% 

\ujvers\nemsloka {
{\devanagarifont तेनाब्देन परार्धकल्पगुणितं ब्रह्मायुरित्युच्यते }%
  \dontdisplaylinenum}    \var{{\devanagarifontvar \numemph\va\textbf{॰ब्देन}\lem \mssCaCbCc\msNa\msNb\msNc, ॰र्धेन \Ed}}% 


\nemslokab

{\devanagarifont त्रैलोक्याधिपतिः प्रधानपुरुषो ब्रह्माप्यनित्यः स्मृतः  \danda\dontdisplaylinenum }%
     \var{{\devanagarifontvar \numnoemph\vb\textbf{॰पुरुषो}\lem \msCa\msCb\msNa\msNb\msNc\Ed, ॰पुरुषा \msCc\oo 
\textbf{॰प्यनित्यः}\lem \mssCaCbCc\msNa\msNb\Ed, ॰पि नित्यः \msNc}}% 

\nemslokac

{\devanagarifont शेषं भूतचतुर्विधस्य नियतं जीवस्य किं शोच्यते }%
  \dontdisplaylinenum    \var{{\devanagarifontvar \numnoemph\vc\textbf{नियतं}\lem \mssCaCbCc\msNa\msNb\Ed, नियितं \msNc\oo 
\textbf{किं}\lem \mssCaCbCc\msNa\msNb\Ed, कि \msNc}}% 

%Verse 21:39


\nemslokad

{\devanagarifont तस्मान्नास्ति जगत्सुसारविमलं मुक्त्वा शिवं शाश्वतम् {॥ २१:३९॥} \veg\dontdisplaylinenum }%
     \var{{\devanagarifontvar \numnoemph\vd\textbf{॰विमलं मुक्त्वा}\lem \msCc, ॰विरलं मुक्त्वा \msCa\msCb\msNa\msNb\msNc, ॰विरलमुक्ता \Ed}}% 
    \paral{{\devanagarifontsmall \vd {\englishfont See the expression } जगत्सुसार {\englishfont also in 1.1b} }}

\vers


{\devanagarifont 
\jump
\begin{center}
\ketdanda~इति वृषसारसंग्रहे कल्पनिर्णयो नामैकविंशतिमो ऽध्यायः~\ketdanda
\end{center}
\dontdisplaylinenum\vers  }%
     \var{{\devanagarifontvar \numnoemph{\englishfont \Colo:}\textbf{॰विंशतिमो}\lem \mssCaCbCc\msNa\msNb\msNc, ॰विंशतितमो \Ed\oo 
\textbf{ऽध्यायः}\lem \mssCaCbCc\msNa\msNb\Ed, ध्याय \msNc}}% 
\bekveg\szamveg
\vfill
\phpspagebreak

\versno=0\fejno=22
\thispagestyle{empty}

\centerline{\Large\devanagarifontbold [   द्वाविंशो ऽध्यायः  ]}{\vrule depth10pt width0pt} \fancyhead[CE]{{\footnotesize\devanagarifont वृषसारसंग्रहे  }}
\fancyhead[CO]{{\footnotesize\devanagarifont द्वाविंशो ऽध्यायः  }}
\fancyhead[LE]{}
\fancyhead[RE]{}
\fancyhead[LO]{}
\fancyhead[RO]{}
\szam\bek


\vers


{\devanagarifont जनमेजय उवाच {\dandab}\dontdisplaylinenum  }%
 
{\devanagarifont श्रुतो ऽथाब्जमुखाद्धर्मसारसंग्रहमुत्तमम् \thinspace{\danda} \dontdisplaylinenum }%
     \var{{\devanagarifontvar \numemph\va\textbf{श्रुतो ऽथाब्जमुखाद्धर्म॰}\lem \eme, 
श्रुतो वाब्जमुखाद्धर्मः \msCa, श्रुतो वाब्जमुखोद्धर्मः \msCb, श्रुतो वाब्जमुखा धर्मः \msCc, 
श्रुतो चाब्जमुखाद्धर्मः \msNa\msL, श्रुतो चाब्दमुखा धर्मः \msNb, श्रुत्वा वाब्जमुखाद्धर्मः \msNc, 
श्रुतो वा त्वन्मुखाद्धर्मः \Ed}}% 
    \paral{{\devanagarifontsmall {\englishfont Witnesses used for this chapter: \msCa\ ff.\thinspace 232r--234v, 
                                              \msCb\ ff.\thinspace 233v--235r, 
                                              \msCc\ ff.\thinspace 314r--317r,
                                              \msNa\ ff.\thinspace 39r--41v,
                                              \msNb\ ff.\thinspace 241v--243v, 
                                              \msNc\ ff.\thinspace 247v--250r;
                                                 \mssCaCbCc\ = \msCa + \msCb + \msCc } }}

%Verse 22:1

{\devanagarifont मधुरश्लक्ष्णवाणीभिः सम्यग्वेदार्थसंयुतम् {॥ २२:१॥} \veg\dontdisplaylinenum }%
     \var{{\devanagarifontvar \numnoemph\vc\textbf{॰श्लक्ष्णवाणी॰}\lem \msCb\msCc\msNa\msNb\msNc, 
श्लक्ष्णणी॰ \msCa, ॰श्लक्ष्यवानी॰ \msL, ॰श्लक्ष्णावाणी॰ \Ed}}% 

{\devanagarifont न्याययुक्तं महासारं गुह्यज्ञानमनुत्तरम् \thinspace{\dandab} \dontdisplaylinenum }%
     \var{{\devanagarifontvar \numemph\va\textbf{न्याययुक्तं महासारं}\lem \msCa\msCc\msNb\msNc\Ed, न्यायमुक्तं महत्सारं \msCb, 
न्याययुक्तं महत्सारं \msNa\msL}}% 
    \var{{\devanagarifontvar \numnoemph\vb\textbf{गुह्य॰}\lem \mssCaCbCc\msNa\msNb\msNc\msL, गुह्यं \Ed\oo 
\textbf{॰नुत्तरम्}\lem \msCa\msNa\msNb\msL, ॰नुत्तमम् \msCb\msCc\msNc, ॰नन्तरम् \Ed}}% 

%Verse 22:2

{\devanagarifont तृप्तो ऽस्मीहामृतं पीत्वा जन्ममृत्युरुजापहम् {॥ २२:२॥} \veg\dontdisplaylinenum }%
     \var{{\devanagarifontvar \numnoemph\vcd\textbf{पीत्वा जन्म॰}\lem \msCb\msCc\msNa\msNb\msNc\msL\Ed, \uncl{पी}\lac  न्म \msCa}}% 
    \var{{\devanagarifontvar \numnoemph\vd\textbf{॰रुजा॰}\lem \msCa\msCc\msNa\msNb\msNc\msL\Ed, ॰मुजा॰ \msCb}}% 

{\devanagarifont प्रश्नमेकान्य पृच्छामि नामहेतुं तपोधन \thinspace{\dandab} \dontdisplaylinenum }%
     \var{{\devanagarifontvar \numemph\va\textbf{प्रश्न॰}\lem \mssCaCbCc\msNa\msNc\msL\Ed, प्रस्त॰ \msNb\oo 
\textbf{॰कान्य}\lem \mssCaCbCc\msNb\msNc, ॰कान्यत् \msNa\ \unmetr, 
॰कांन्यत् \msL\ \unmetr, ॰कोन्य \Ed}}% 
    \var{{\devanagarifontvar \numnoemph\vb\textbf{नाम॰}\lem \mssCaCbCc\msNa\msNb\msL\Ed, नाय॰ \msNc\oo 
\textbf{॰हेतुं}\lem \msCa\msCb\msNa\msL, ॰हेतु \msCc\msNb\msNc\Ed\oo 
\textbf{॰धन}\lem \mssCaCbCc\msNb\msNc\Ed, ॰धनम् \msNa\msL}}% 

%Verse 22:3

{\devanagarifont वर्णगोत्राश्रमं तस्माच्छ्रोतुमिच्छामि ते पुनः {॥ २२:३॥} \veg\dontdisplaylinenum }%
     \var{{\devanagarifontvar \numnoemph\vc\textbf{वर्ण॰}\lem \mssCaCbCc\msNa\msNb\msNc\msBod\msL, वर्णं \Ed}}% 

{\devanagarifont वैशम्पायन उवाच {\dandab}\dontdisplaylinenum  }%
     \var{{\devanagarifontvar \numemph\vo\textbf{उवाच}\lem \mssCaCbCc\msNa\msNb\msL\Ed, \lac\  \msNc}}% 

{\devanagarifont शृणु राजन्नवहितो योगेन्द्रस्य महात्मनः \thinspace{\danda} \dontdisplaylinenum }%
     \var{{\devanagarifontvar \numnoemph\va\textbf{राजन्न॰}\lem \msCb\msCc\msNa\msNc\msL\Ed, राजन॰ \msCa\msNb}}% 
    \var{{\devanagarifontvar \numnoemph\vab\textbf{॰वहितो योगेन्द्रस्य}\lem \mssCaCbCc\msNapcorr\msNb\msNc\Ed, ॰वहितो योगेन्द्र \msNaacorr, 
॰हितो योगन्द्रस्य \msL}}% 

%Verse 22:4

{\devanagarifont आश्रमं वर्णजातीनां वक्ष्याम्येव नराधिप {॥ २२:४॥} \veg\dontdisplaylinenum }%
     \var{{\devanagarifontvar \numnoemph\vd\textbf{वक्ष्याम्येव}\lem \msCa\msCc\msNa\msNb\Ed, वक्ष्यामेव \msCb\msNc\msL\oo 
\textbf{॰प}\lem \mssCaCbCc\msNa\msNb\msNc\msL, ॰पः \Ed}}% 

{\devanagarifont हिमवद्दक्षिणे पार्श्वे मृगेन्द्रशिखरे नृप \thinspace{\dandab} \dontdisplaylinenum }%
     \var{{\devanagarifontvar \numemph\vb\textbf{मृगेन्द्र॰}\lem \msCb\msCc\msNa\msNb\msNc\msL\Ed, \uncl{मृ}\lac  न्द्र॰ \msCa\oo 
\textbf{नृप}\lem \mssCaCbCc\msNa\msNb, नृपः \msNc\msL\Ed}}% 

%Verse 22:5

{\devanagarifont महेन्द्रपथगानामनदीतीरे नराधिप {॥ २२:५॥} \veg\dontdisplaylinenum }%
     \var{{\devanagarifontvar \numnoemph\vc\textbf{महेन्द्र॰}\lem \mssCaCbCc\msNa\msNc\Ed, मृगेन्द्र॰ \msNb, महिन्द्र॰ \msL}}% 
    \var{{\devanagarifontvar \numnoemph\vd\textbf{॰प}\lem \mssCaCbCc\msNa\msNb\msNc\msL, ॰पः \Ed}}% 

{\devanagarifont तत्राश्रमपदं तस्य पुलिने सुमनोरमे \thinspace{\dandab} \dontdisplaylinenum }%
     \var{{\devanagarifontvar \numemph\vb\textbf{पुलिने सु॰}\lem \msCa\msCb\msNa, पुलिनेषु \msCc\msNb\msNc\Ed, पुलिने पु॰ \msL}}% 

%Verse 22:6

{\devanagarifont वसति स्म महाभागस्तत्त्वपारगनिस्पृहः {॥ २२:६॥} \veg\dontdisplaylinenum }%
     \var{{\devanagarifontvar \numnoemph\vc\textbf{वसति}\lem \mssCaCbCc\msNa\msNb\msNc\Ed, वसन्ति \msL}}% 
    \var{{\devanagarifontvar \numnoemph\vd\textbf{॰पारग॰}\lem \msCa\msCc\msNa\msNb\msNc\msL\Ed, ॰पार॰ \msCb\oo 
\textbf{॰स्पृहः}\lem \mssCaCbCc\msNa\msNb\msNc\msL, ॰स्पृहाः \Ed}}% 

{\devanagarifont शीलशौचसमाचारो जितद्वन्द्वो जितश्रमः \thinspace{\dandab} \dontdisplaylinenum }%
 
%Verse 22:7

{\devanagarifont जितमानभयक्रोधो जितसर्वपरिग्रहः {॥ २२:७॥} \veg\dontdisplaylinenum }%
     \var{{\devanagarifontvar \numemph\vd\textbf{जित॰}\lem \msCa\msCc\msNa\msNb\msNc\msL\Ed, जिज॰ \msCb}}% 

{\devanagarifont सोमवंशप्रसूतास्ते क्षत्रिया द्विजतां गताः \thinspace{\dandab} \dontdisplaylinenum }%
     \var{{\devanagarifontvar \numemph\va\textbf{सोम॰}\lem \mssCaCbCc\msNa\msNb\msNc\Ed, सोय॰ \msL\oo 
\textbf{प्रसूतास्ते}\lem \msCb\msCc\msNb\msNc\Ed, प्र\lac\  \msCa, प्रसूतस्ते \msNa\msL}}% 
    \var{{\devanagarifontvar \numnoemph\vb\textbf{क्षत्रिया}\lem \mssCaCbCc\msNb, क्षत्रियो \msNa\msNc\msL\Ed\oo 
\textbf{गताः}\lem \mssCaCbCc\msNb\Ed, गतः \msNa\msNc\msL}}% 

%Verse 22:8

{\devanagarifont तपसा विनयाचारैर्विष्णुना द्विजकल्पिताः {॥ २२:८॥} \veg\dontdisplaylinenum }%
     \var{{\devanagarifontvar \numnoemph\vc\textbf{॰चारैर्वि॰}\lem \msCa\msCb\msNa\msNb\msNc\msL\Ed, ॰चारै वि॰ \msCc}}% 
    \var{{\devanagarifontvar \numnoemph\vd\textbf{द्विजकल्पिताः}\lem \Ed, द्विजः कल्पितः \mssCaCbCc\msNc\ \unmetr, 
द्विजकल्पितः \msNa\msNb\msL}}% 

{\devanagarifont अजिता नाम तत्पूर्वं कामक्रोधजितेन तु \thinspace{\dandab} \dontdisplaylinenum }%
     \var{{\devanagarifontvar \numemph\va\textbf{पूर्वं}\lem \mssCaCbCc\msNb\msNc\Ed, पूर्व \msNa\msL}}% 

%Verse 22:9

{\devanagarifont संकल्पस्तस्य राजेन्द्र कथयिष्यामि तच्छृणु {॥ २२:९॥} \veg\dontdisplaylinenum }%
     \var{{\devanagarifontvar \numnoemph\vc\textbf{संकल्पस्त}\lem \mssCaCbCc\msNa\msNb\msNc\Ed, संकल्प त \msL}}% 

{\devanagarifont अध्यात्मनगरस्फीतः अधिभूतजनाकुलः \thinspace{\dandab} \dontdisplaylinenum }%
     \var{{\devanagarifontvar \numemph\vab\textbf{॰स्फीतः अधि॰}\lem \msCb\msCc\msNa\msNb\msNc\msL\Ed, ॰स्फीतरधि॰ \msCa}}% 

%Verse 22:10

{\devanagarifont अधिदैवतसांनिध्यं दशायतन पञ्च च {॥ २२:१०॥} \veg\dontdisplaylinenum }%
     \var{{\devanagarifontvar \numnoemph\vc\textbf{॰सांनिध्यं}\lem \msCa\Ed, सानैध्यं \msCb\msCc\msNa\msNb\msL, सान्नैध्यं \msNc}}% 
    \var{{\devanagarifontvar \numnoemph\vd\textbf{दशा॰}\lem \mssCaCbCc\msNa\msNb\msNc\msL, देशा॰ \Ed}}% 
    \paral{{\devanagarifontsmall \vo {\englishfont Cf.\ 4.72:} चतुरायतनं विप्र कथयिष्यामि तच्छृणु\thinspace{\devanagarifontsmall ।}
                                  करुणामुदितोपेक्षामैत्री चायातनं स्मृतम्\thinspace{\devanagarifontsmall ॥} }}

{\devanagarifont दशयज्ञव्रतं चीर्णं दशकामपराजितः \thinspace{\dandab} \dontdisplaylinenum }%
     \var{{\devanagarifontvar \numemph\va\textbf{दशयज्ञव्रतं चीर्णं}\lem \msNa\msNb\msNc\msL, द\uncl{शयज्ञं}\lac  ञ्चीर्णन् \msCa, 
दशयज्ञव्रतचीर्णन् \msCb\msCc, दशयज्ञं व्रतं चीर्ण॰ \Ed}}% 
    \var{{\devanagarifontvar \numnoemph\vb\textbf{॰पराजितः}\lem \msCa\msCc\msNa\msNb\msNc\msL\Ed, ॰पपराजितः \msCb}}% 

%Verse 22:11

{\devanagarifont नियमान्दश संश्रित्य दश वायव ऋत्विजः {॥ २२:११॥} \veg\dontdisplaylinenum }%
     \var{{\devanagarifontvar \numnoemph\vc\textbf{नियमान्दश}\lem \mssCaCbCc\msNa\msNb\msNc\Ed, निमाया दश \msLacorr, नियमा दश \msLpcorr}}% 
    \paral{{\devanagarifontsmall \vd {\englishfont cf.\ 11.17ab:} धारणाध्वर्युवत्कृत्वा प्राणायामश्च ऋत्विजः }}

{\devanagarifont दशाक्षरेण मन्त्रेण दशधर्मक्रियापदः \thinspace{\dandab} \dontdisplaylinenum }%
     \var{{\devanagarifontvar \numemph\vb\textbf{॰धर्मक्रियापदः}\lem \msCa\msCb\msNa\msNb\msNc\msL\Ed, ॰धर्मः क्रिपदः \msCc}}% 

%Verse 22:12

{\devanagarifont दशसंयमदीप्ताग्नौ जिह्वातेजोदशेन्द्रियः {॥ २२:१२॥} \veg\dontdisplaylinenum }%
     \var{{\devanagarifontvar \numnoemph\vc\textbf{॰संयम॰}\lem \mssCaCbCc\msNa\msNb\msNc\Ed, ॰संशय॰ \msL\oo 
\textbf{॰दीप्ता॰}\lem \mssCaCbCc\msNa\msNc\msL, ॰दीप्तो \msNb, ॰दीपा॰ \Ed}}% 
    \var{{\devanagarifontvar \numnoemph\vd\textbf{॰दशे॰}\lem \mssCaCbCc\msNa\msNb\msL, ॰जिते॰ \msNc\Ed}}% 

{\devanagarifont दशयोगासनासीनो दशध्यानपरायणः \thinspace{\dandab} \dontdisplaylinenum }%
     \var{{\devanagarifontvar \numemph\va\textbf{॰सनासीनो}\lem \mssCaCbCc\msNa\msNb\msNc\msL, समासीना \Ed}}% 
    \var{{\devanagarifontvar \numnoemph\vb\textbf{॰यणः}\lem \mssCaCbCc\msNb\msNc\Ed, ॰यणाः \msNa\msL}}% 

%Verse 22:13

{\devanagarifont बुद्धिर्वेदी मनो यूपः सोमपानो ऽमृताक्षरः {॥ २२:१३॥} \veg\dontdisplaylinenum }%
     \var{{\devanagarifontvar \numnoemph\vc\textbf{बुद्धिर्वेदी}\lem \mssCaCbCc\msNa\msNb\msL, बुद्धि वेदी \msNc, बुद्धिर्वेदि \Ed}}% 
    \var{{\devanagarifontvar \numnoemph\vd\textbf{॰पानो ऽमृताक्षरः}\lem \msCb\msNa\msNb\msNc\msL, \lac\  \msCa, 
॰पानमृताक्षरः \msCc, ॰दानमृताक्षरः \Ed}}% 

{\devanagarifont दक्षिणाभय भूतेभ्यः पशुबन्ध स्वयंकृतः \thinspace{\dandab} \dontdisplaylinenum }%
     \var{{\devanagarifontvar \numemph\va\textbf{॰भय}\lem \mssCaCbCc\msNa\msNb\msNc\msL, ॰ग्नय \Ed}}% 

{\devanagarifont विनार्थं यज्ञमिष्ट्वा तु कालं च क्षपयत्यसौ  \danda\dontdisplaylinenum }%
     \var{{\devanagarifontvar \numnoemph\va\textbf{॰र्थं}\lem \msCa\msCb\Ed, ॰र्थ॰ \msCc\msNa\msNb\msNc\msL}}% 
    \var{{\devanagarifontvar \numnoemph\vb\textbf{कालं}\lem \mssCaCbCc\msNa\msNb\msNc\msL, कालाञ् \Ed\oo 
\textbf{क्षपयत्यसौ}\lem \mssCaCbCc\msNa\msNc\msL, 
\uncl{क्षपयत्यसौ} \msNb, क्षपयत्यसौः \Ed}}% 

%Verse 22:14

{\devanagarifont अनर्थयज्ञं तं प्राहुर्मुनयस्तत्त्वदर्शिनः {॥ २२:१४॥} \veg\dontdisplaylinenum }%
     \var{{\devanagarifontvar \numnoemph\vcd\textbf{॰यज्ञं तं प्राहुर्मुनयस्त॰}\lem \msCa\msCb\msNb\msNc\Ed, 
॰यज्ञ तं प्राहु मुनयस्त॰ \msCc, ॰यज्ञन्तं प्राहुर्मुनय त॰ \msNa, 
॰यज्ञं प्राहुर्मुनय त॰ \msL}}% 

{\devanagarifont जनमेजय उवाच {\dandab}\dontdisplaylinenum  }%
 
{\devanagarifont दशयज्ञमहं श्रोतुं देहि मां द्विजसत्तम \thinspace{\danda} \dontdisplaylinenum }%
     \var{{\devanagarifontvar \numemph\va\textbf{॰यज्ञमहं}\lem \mssCaCbCc\msNa\msNb\msNc\msLpcorr, ॰यज्ञमिदं \Ed}}% 
    \var{{\devanagarifontvar \numnoemph\vb\textbf{मां}\lem \msCa\msCb\msNa\msNb\msNc\msL\Ed, मा \msCc\oo 
\textbf{॰त्तम}\lem \mssCaCbCc\msNb\msNc\Ed, ॰त्तमः \msNa\msL}}% 

%Verse 22:15

{\devanagarifont दशकामदशध्यानं दशयोगदशाक्षरम् {॥ २२:१५॥} \veg\dontdisplaylinenum }%
     \var{{\devanagarifontvar \numnoemph\vc\textbf{॰दशध्यानं}\lem \msCa\msCb\msNa\msNb\msNc, ॰दशध्यान॰ \msCc\Ed, ॰दतध्यानन् \msL}}% 
    \var{{\devanagarifontvar \numnoemph\vd\textbf{॰क्षरम्}\lem \msCb\msNb\msNc, ॰क्षर\lac\  \msCa, ॰क्षरः \msCc\msNa\msL\Ed}}% 

{\devanagarifont वैशम्पायन उवाच {\dandab}\dontdisplaylinenum  }%
     \var{{\devanagarifontvar \numemph\vo\textbf{वैशम्पायन उवाच}\lem \msCb\msCc\msNa\msNb\msNc\msL\Ed, \lac  वाच \msCa}}% 

{\devanagarifont ब्रह्मदेवपितृयज्ञो यज्ञो भूतातिथेश्च ह \thinspace{\danda} \dontdisplaylinenum }%
     \var{{\devanagarifontvar \numnoemph\va\textbf{॰देव॰}\lem \msCa\msCc\msNa\msNb\msNc\msL\Ed, ॰दैव॰ \msCb\oo 
\textbf{॰यज्ञो}\lem \msCa\msCb\msNa\msNb\Ed, ॰योज्ञो \msNc, ॰यज्ञ \msCc\msL}}% 
    \var{{\devanagarifontvar \numnoemph\vb\textbf{यज्ञो}\lem \msCa\msCb\msNa\msL, यज्ञ॰ \msCc\msNb\msNc\Ed\oo 
\textbf{॰तिथेश्च ह}\lem \msCb, ॰तिथिश्च ह \msCa\msCc\msNa\msNb\msNc\msL, ॰तिथिञ्च यः \Ed}}% 
    \paral{{\devanagarifontsmall \vb {\englishfont \compare\ Garuḍapurāṇa 1.50.71cd:} भूतयज्ञः स वै ज्ञेयो भूतेभ्यो यस्त्वयं बलिः;
                    {\englishfont \compare\ Śatapathabrāhmana 11.5.6:} अहरहर्भूतेभ्यो बलिं हरेत् तथैतम् भूतयज्ञं }}

%Verse 22:16

{\devanagarifont जपो योगस्तपो ध्यानं स्वाध्यायश्च दश स्मृतः {॥ २२:१६॥} \veg\dontdisplaylinenum }%
     \var{{\devanagarifontvar \numnoemph\vc\textbf{योगस्तपो ध्यानं}\lem \mssCaCbCc\msNb\msNc\Ed, योग\lac \uncl{धानं} \msNa, 
योग \gap\gap\ पानं \msL}}% 
    \var{{\devanagarifontvar \numnoemph\vd\textbf{स्वाध्यायश्च}\lem \mssCaCbCc\msNb\msNc\Ed, \uncl{साध्या}यश्च \msNa, 
साधुतपश्च \msL}}% 

{\devanagarifont पत्नीपुत्रपशुभृत्यधनधान्ययशःश्रियः \thinspace{\dandab} \dontdisplaylinenum }%
     \var{{\devanagarifontvar \numemph\va\textbf{॰यशः॰}\lem \msCa\msCb\msNa\msNb\msNc\msL, ॰यश॰ \msCc\Ed}}% 

%Verse 22:17

{\devanagarifont मान भोग दश राजन्दशकाम उदाहृतः {॥ २२:१७॥} \veg\dontdisplaylinenum }%
     \var{{\devanagarifontvar \numnoemph\vc\textbf{॰भोग}\lem \mssCaCbCc\msNa\msNb\msNc\msL, ॰भोगं \Ed}}% 
    \var{{\devanagarifontvar \numnoemph\vd\textbf{॰हृतः}\lem \msCa\msCc\msNa\msNb\msNc\msL\Ed, ॰हृतम् \msCb}}% 

{\devanagarifont मानसो यौगपद्यश्च संक्षिप्तश्च विशाम्पते \thinspace{\dandab} \dontdisplaylinenum }%
     \var{{\devanagarifontvar \numemph\va\textbf{यौगपद्यश्च}\lem \corr, यौगपद्यञ्च \msCa\msCb\msNb, 
योगपद्यं च \msCc\msNa\msNc\msL, योगपद्यश्च \Ed}}% 
    \var{{\devanagarifontvar \numnoemph\vb\textbf{॰क्षिप्तश्च}\lem \Ed, ॰क्सिप्तं च \mssCaCbCc\msNa\msNb\msNc\msL}}% 
    \paral{{\devanagarifontsmall \vo {\englishfont cf.\ Dharmaputrikā 1.56:} संक्षिप्ता प्रथमा ज्ञेया विशाला समनन्तरम्\thinspace{\devanagarifontsmall ॥}
                                         ततो द्विकरणी चेति त्रिविधो योग उच्यते\thinspace{\devanagarifontsmall ।} }}

%Verse 22:18

{\devanagarifont विशाला नाम योगश्च ततो द्विकरणः स्मृतः {॥ २२:१८॥} \veg\dontdisplaylinenum }%
     \var{{\devanagarifontvar \numnoemph\vc\textbf{विशाला नाम योगश्च}\lem \Ed, वि\lac  योगञ्च \msCa, 
विशाला नाम योगं च \msCb\msCc\msNa\msNb\msNc\msL}}% 
    \var{{\devanagarifontvar \numnoemph\vd\textbf{द्विकरणः}\lem \msCa\msCb\msNa\msL, विकरणः \msCc\Ed, द्विकरणी \msNb, द्विकरण \msNc}}% 

{\devanagarifont रविः सोमो हुताशश्च स्फटिकाम्बरमेव च \thinspace{\dandab} \dontdisplaylinenum }%
     \var{{\devanagarifontvar \numemph\va\textbf{रविः}\lem \msCa, रवि॰ \msCb\msCc\msNa\msNb\msNc\msL\Ed}}% 
    \var{{\devanagarifontvar \numnoemph\vb\textbf{स्फटिकाम्बर॰}\lem \mssCaCbCc\msNb\msNc\Ed, स्फटिकां\lac  र॰ \msNa, स्फटिकांसत॰ \msL}}% 
    \paral{{\devanagarifontsmall \vab {\englishfont cf. Dharmaputrikā 4:5cd:} सूर्यचन्द्रहुताशार्चिःस्फाटिकाम्बरसन्निभाः }}

%Verse 22:19

{\devanagarifont दशयोगासनासीनो नित्यमेव तपोधनः {॥ २२:१९॥} \veg\dontdisplaylinenum }%
     \var{{\devanagarifontvar \numnoemph\vc\textbf{दशयोगासनासीनो}\lem \msCa\msCc\msNa\msNb\msNc, दशयोगसमासीनो \msCb, 
देवयोगासतासीनो \msL, दशयोगासनासीनौ \Ed}}% 
    \var{{\devanagarifontvar \numnoemph\vd\textbf{॰धनः}\lem \msCa\msCb\msNa\msL, ॰धन \msCc\msNb\msNc\Ed}}% 

{\devanagarifont अनिरोधमनाः सूक्ष्मं ध्यायेद्योगः स मानसः \thinspace{\dandab} \dontdisplaylinenum }%
     \var{{\devanagarifontvar \numemph\va\textbf{अनिरोध॰}\lem \mssCaCbCc\msNa\msNb\msNc\msL, अनिलाध॰ \Ed\oo 
\textbf{॰मनाः}\lem \mssCaCbCc\msNa\msNc\msL\Ed, ॰मना \msNb}}% 
    \var{{\devanagarifontvar \numnoemph\vb\textbf{ध्यायेद्यो॰}\lem \msCa\msCb\msNa\msNb\msNc\msL, ध्यायो॰ \msCc, ध्यानं यो॰ \Ed}}% 
    \paral{{\devanagarifontsmall \vab {\englishfont cf.\ Dharmaputrikā 1.54:} अकृत्वा प्राणसंरोधं मनसैकेन केवलम्\thinspace{\devanagarifontsmall ।}
                                 ध्यायेत परमं सूक्ष्मं स योगो मानसः स्मृतः\thinspace{\devanagarifontsmall ॥} }}

%Verse 22:20

{\devanagarifont प्राणायामैर्मनो रुद्ध्वा यौगपद्यः स उच्यते {॥ २२:२०॥} \veg\dontdisplaylinenum }%
     \var{{\devanagarifontvar \numnoemph\vc\textbf{॰यामैर्म॰}\lem \msCa\msNa\msNb\msNcpcorr\msL\Ed, 
॰यामै म॰ \msCb, ॰यामै म्म॰ \msCc, ॰यामेर्म॰ \msNcacorr\oo 
\textbf{रुद्ध्वा}\lem \mssCaCbCc\msNa\msNb\msNc\msL, रुद्धा \Ed}}% 
    \var{{\devanagarifontvar \numnoemph\vd\textbf{यौग॰}\lem \msCa\msCb\msNa\msNcpcorr\msL, योग॰ \msCc\msNb\msNcacorr\Ed}}% 
    \paral{{\devanagarifontsmall \vcd {\englishfont cf.\ Dharmaputrikā 1.55:} संयम्य मनसा प्राणं प्राणायामैर्मनस्तथा\thinspace{\devanagarifontsmall ।}
                                         एवं ध्यायेत्परं सूक्ष्मं यौगपद्यः स उच्यते\thinspace{\devanagarifontsmall ॥} }}

{\devanagarifont ब्रह्मादिस्तम्बपर्यन्तं सर्वं स्थावरजङ्गमम् \thinspace{\dandab} \dontdisplaylinenum }%
     \var{{\devanagarifontvar \numemph\vo\textbf{(ब्रह्मादि॰{\englishfont ...} विचिन्तयेत)}\lem \mssCaCbCc\msNa\msNc\msL\Ed, \om\ \msNb}}% 
    \var{{\devanagarifontvar \numnoemph\va\textbf{॰स्तम्ब॰}\lem \mssCaCbCc\msNa\msNc\Ed, \om\ \msNb, ॰स्तंभ॰ \msL\oo 
\textbf{॰पर्यन्तं}\lem \msCb\msCc\msNa\msL, ॰\uncl{द्विय}\lac\  \msCa, \om\ \msNb, ॰पर्यन्त॰ \msNc\Ed}}% 
    \var{{\devanagarifontvar \numnoemph\vb\textbf{सर्वं}\lem \msCb\msNa, \lac\  \msCa, सर्व॰ \msCc\msNc\msL\Ed, \om\ \msNb}}% 
    \paral{{\devanagarifontsmall \vab {\englishfont \similar\ Dharmaputrikā 1.57cd:} ब्रह्मादिस्तम्भपर्यन्ताः सर्वे स्थावरजङ्गमाः }}

%Verse 22:21

{\devanagarifont प्रलीयमानं ध्यायेत क्रमात्सूक्ष्मं विचिन्तयेत् {॥ २२:२१॥} \veg\dontdisplaylinenum }%
     \var{{\devanagarifontvar \numnoemph\vc\textbf{प्रलीय॰}\lem \mssCaCbCc\msNa\msNc\Ed, \om\ \msNb, प्रणीय॰ \msL}}% 
    \var{{\devanagarifontvar \numnoemph\vd\textbf{क्रमात्सू॰}\lem \msCa\msCb\msNa\msNc\msL\Ed, क्रमा सू॰ \msCc, \om\ \msNb}}% 
    \paral{{\devanagarifontsmall \vcd {\englishfont \similar\ Dharmaputrikā 1.59ab:} प्रलीयमानन्ध्यायेत क्रमाच्छून्यं भवेज्जगत् }}

{\devanagarifont संक्षिप्त एष आख्यातो विशालां छृणु तत्त्वतः \thinspace{\dandab} \dontdisplaylinenum }%
     \var{{\devanagarifontvar \numemph\vo\textbf{(संक्षिप्त{\englishfont ...} विचक्षणः)}\lem \mssCaCbCc\msNa\msNc\msL\Ed, \om\ \msNb}}% 
    \var{{\devanagarifontvar \numnoemph\va\textbf{संक्षिप्त}\lem \mssCaCbCc\msNa\msNc\Ed, \om\ \msNb, संक्षिप्तः \msL\oo 
\textbf{एष}\lem \mssCaCbCc\msNa\msNc\msL, \om\ \msNb, एव \Ed\oo 
\textbf{आख्यातो}\lem \msCb\msNc, आख्यातः \msCa\msCc\msNa\msL\Ed, \om\ \msNb}}% 
    \paral{{\devanagarifontsmall \vab {\englishfont cf.\ Dharmaputrikā 1.60ab:} एष योगविधिः प्रोक्तः संक्षिप्तो नाम नामतः }}

%Verse 22:22

{\devanagarifont ब्रह्मादिसूक्ष्मपर्यन्तं चिन्तयीत विचक्षणः {॥ २२:२२॥} \veg\dontdisplaylinenum }%
     \var{{\devanagarifontvar \numnoemph\vc\textbf{॰सूक्ष्म॰}\lem \mssCaCbCc\msNc\Ed, ॰स्तंब॰ \msNa, \om\ \msNb, तव \msL\oo 
\textbf{॰पर्यन्तं}\lem \mssCaCbCc\msNa\msL, \om\ \msNb, ॰पर्यन्त \msNc\Ed}}% 
    \var{{\devanagarifontvar \numnoemph\vd\textbf{चिन्तयीत}\lem \msCa\msCbpcorr\msCc\msNa\msNc\msL\Ed, \om\ \msNb, चियीत \msCbacorr}}% 

{\devanagarifont संक्षिप्तां च विशालां च चिन्तयीत परस्परम् \thinspace{\dandab} \dontdisplaylinenum }%
     \var{{\devanagarifontvar \numemph\vo\textbf{(संक्षिप्तां{\englishfont ...} विधिरुच्यते)}\lem \mssCaCbCc\msNa\msNc\msL\Ed, \om\ \msNb}}% 
    \var{{\devanagarifontvar \numnoemph\va\textbf{संक्षिप्तां}\lem \msCb\msNc, संक्षिप्ता \msCapcorr\msCc\msNa\msL\Ed, \om\ \msCaacorr\msNb\oo 
\textbf{विशालां}\lem \msCapcorr\msCb\msNc, \om\ \msCaacorr, विशाला \msCc\msNa\msL\Ed, \om\ \msNb}}% 

%Verse 22:23

{\devanagarifont एषा द्विकरणी नाम योगस्य विधिरुच्यते {॥ २२:२३॥} \veg\dontdisplaylinenum }%
     \var{{\devanagarifontvar \numnoemph\vc\textbf{द्वि॰}\lem \msCa\msCb\msNa\msNc\msL, वि॰ \msCc\Ed, \om\ \msNb}}% 
    \paral{{\devanagarifontsmall \vo {\englishfont \similar\ Dharmaputrikā 1.62cd--63ab:} एतौ संहारसर्गौ द्वौ पारम्पर्येण चिन्तयेत्\thinspace{\devanagarifontsmall ॥}
                                                        एषा द्विकरणी नाम योगस्य विधिरिष्यते\thinspace{\devanagarifontsmall ।} }}

{\devanagarifont देहमध्ये हृदि ज्ञेयं हृदिमध्ये तु पङ्कजम् \thinspace{\dandab} \dontdisplaylinenum }%
     \var{{\devanagarifontvar \numemph\va\textbf{ज्ञेयं}\lem \msCa\msCb\msNa\msNc\Ed, ज्ञेय \msCc\msL, ज्ञे \msNbacorr, ज्ञे\lac\  \msNbpcorr}}% 
    \var{{\devanagarifontvar \numnoemph\vb\textbf{तु पङ्कजम्}\lem \msCb\msCc\msNa\msNb\msNc\msL\Ed, \uncl{तु} प\lac\  \msCa}}% 

%Verse 22:24

{\devanagarifont पङ्कजस्य च मध्ये तु कर्णिकां विद्धि गोपते {॥ २२:२४॥} \veg\dontdisplaylinenum }%
     \var{{\devanagarifontvar \numnoemph\vc\textbf{पङ्कजस्य च}\lem \msCb\msCc\msNa\msNc\Ed, \lac  ङ्कजस्य च \msCa, 
कङ्कस्य तु \msNb, पन्कजंस्य च \msL}}% 
    \var{{\devanagarifontvar \numnoemph\vd\textbf{कर्णिकां विद्धि गोपते}\lem \msCa\msCb\msNa\msNb\msNc\msL, कर्णिद्धिद्धि गोपते \msCc, 
कर्णिकां च विंशापते \Ed}}% 

{\devanagarifont कर्णिकायास्तु मध्ये तु पञ्चबिन्दुं विदुर्बुधाः \thinspace{\dandab} \dontdisplaylinenum }%
     \var{{\devanagarifontvar \numemph\vb\textbf{॰बिन्दुं}\lem \msCa\msNc, ॰बिन्दु \msCb\msCc\msNa\msNb\msL\Ed}}% 

%Verse 22:25

{\devanagarifont रविसोमशिखां चैव स्फटिकाम्बरमेव च {॥ २२:२५॥} \veg\dontdisplaylinenum }%
     \var{{\devanagarifontvar \numnoemph\vc\textbf{॰शिखां}\lem \msCa\msNa\msL, ॰शिखा \msCb\msCc\msNb\msNc\Ed}}% 
    \var{{\devanagarifontvar \numnoemph\vd\textbf{स्फटि॰}\lem \msCa\msCc\msNa\msNb\msNc\msL\Ed, स्फाटि॰ \msCb}}% 
    \paral{{\devanagarifontsmall \vcd {\englishfont cf.\ Dharmaputrikā 4.5cd:} सूर्यचन्द्रप्रका$\-$शार्चिस्फाटिकाम्बरसन्निभाः }}

{\devanagarifont रविमण्डलमध्ये तु भावयेच्चन्द्रमण्डलम् \thinspace{\dandab} \dontdisplaylinenum }%
     \var{{\devanagarifontvar \numemph\vb\textbf{भावयेच्चन्द्रमण्डलम्}\lem \msCa\msCb\msNa\msNb\msNc\msL\Ed, भावये चन्द्रमण्डलः \msCc}}% 

%Verse 22:26

{\devanagarifont तस्य मध्ये शिखां ध्यायेन्निर्धूमज्वलनप्रभाम् {॥ २२:२६॥} \veg\dontdisplaylinenum }%
     \var{{\devanagarifontvar \numnoemph\vc\textbf{॰शिखां}\lem \msCa\msCb\msNa\msNb\msNc\msL, ॰शिखा \msCc\Ed}}% 

{\devanagarifont अग्निमध्ये मणिं ध्यायेच्छुद्धधाराजलप्रभम् \thinspace{\dandab} \dontdisplaylinenum }%
     \var{{\devanagarifontvar \numemph\vab\textbf{मणिं ध्यायेच्छुद्ध॰}\lem \msCb\msNa\msNb\msNc\msL\Ed, \lac\  \msCa, 
मनिं ध्यायेच्छुद्ध॰ \msCc}}% 
    \var{{\devanagarifontvar \numnoemph\vb\textbf{॰धारा॰}\lem \msCa\msCb\msNa\msNb\msNc\msL, ॰धार॰ \msCc\Ed\oo 
\textbf{॰प्रभम्}\lem \msCc\msNa\msNb\msNc\msL\Ed, ॰प्रभाम् \msCa\msCb}}% 

%Verse 22:27

{\devanagarifont तस्य मध्ये ऽम्बरं ध्यायेत्सुसूक्ष्मं शिवमव्ययम् {॥ २२:२७॥} \veg\dontdisplaylinenum }%
     \var{{\devanagarifontvar \numnoemph\vc\textbf{ऽम्बरं}\lem \msCa\msCb\msNa\msNb\msNc, ऽम्बर \msCc, बरं \msL, ऽक्षरं \Ed}}% 
    \var{{\devanagarifontvar \numnoemph\vd\textbf{सुसूक्ष्मं}\lem \msCc\msNa\msNc\msL, सूक्ष्मं \msCa, सुसूक्ष्म॰ \msCb, 
\uncl{स्व}सूक्ष्म॰ \msNb, ससूक्ष्मं \Ed}}% 

{\devanagarifont दशयोगमिदं राजन्कथितं च मया तव \thinspace{\dandab} \dontdisplaylinenum }%
 
%Verse 22:28

{\devanagarifont दशध्यानं समासेन कीर्तितं शृणु तद्यथा {॥ २२:२८॥} \veg\dontdisplaylinenum }%
     \var{{\devanagarifontvar \numemph\vc\textbf{॰ध्यानं}\lem \mssCaCbCc\msNa\msNc, ॰ध्यान \msNb\msL\Ed}}% 

{\devanagarifont घोषणी पिङ्गला चैव वैद्युती चन्द्रमालिनी \thinspace{\dandab} \dontdisplaylinenum }%
     \var{{\devanagarifontvar \numemph\va\textbf{घोषणी}\lem \mssCaCbCc\msNa\msNb\msNc\msL, घोषणा \Ed}}% 
    \var{{\devanagarifontvar \numnoemph\vb\textbf{वैद्युती}\lem \msCa\msCb\msNa\msNb\msNc\msL\Ed, विद्युत \msCc, विद्युती \Ed}}% 
    \paral{{\devanagarifontsmall \vo {\englishfont \NISVK\ 33.27cd--28ab:}
                 घोषिणि पिङ्गला चैव वैद्युती बिन्दुमालिनी\thinspace{\devanagarifontsmall ॥}
                 चान्द्री मनोनुगा चैव सुकृता च तथा परा\thinspace{\devanagarifontsmall ।} }}

%Verse 22:29

{\devanagarifont चन्द्रा मनोऽनुगा चैव सुकृता च तथापरा {॥ २२:२९॥} \veg\dontdisplaylinenum }%
     \var{{\devanagarifontvar \numnoemph\vc\textbf{चन्द्रा मनोऽनुगा}\lem \msCb\msNa\msNb\msNc\msL, 
चन्द्रा मनानुगा \msCa, चन्द्रमनोनुगा \msCc, चन्द्रो मनोऽनुगा \Ed}}% 
    \var{{\devanagarifontvar \numnoemph\vd\textbf{सुकृता च तथापरा}\lem \msCa\msCc\msNa\msNc\msL, सुकृता तथापरा \msCb, \om\ \msNb, 
सुकृता च तथापर \Ed}}% 

{\devanagarifont सौम्या निरञ्जना चैव निरालम्बा च कीर्तिता \thinspace{\dandab} \dontdisplaylinenum }%
     \var{{\devanagarifontvar \numemph\va\textbf{सौम्या निरञ्जना चैव}\lem \msCb\msCc\msNa\msL\Ed, सौम्या निरञ्जना \lac\  \msCa, \om\ \msNb, 
सौम्या णिरञ्जना चैव \msNc}}% 
    \var{{\devanagarifontvar \numnoemph\vb\textbf{कीर्तिता}\lem \mssCaCbCc\msNa\msNb\msNc\Ed, कीर्तिताः \msL}}% 
    \paral{{\devanagarifontsmall \vo {\englishfont \NISVK\ 33.28cd:}
                 सौम्या निरञ्जना चैव निरालम्बा च कथ्यते }}

%Verse 22:30

{\devanagarifont सुपिषित्वाङ्गुलौ श्रोत्रे ध्वनिमाकर्णयेन्नरः {॥ २२:३०॥} \veg\dontdisplaylinenum }%
     \var{{\devanagarifontvar \numnoemph\vc\textbf{सुपिषित्वाङ्गुलौ}\lem \msCa\msCb\msNa\msNb\msNc, सु\lac ि{}षिचाङ्गुलौ \msCc, 
सुपिथित्वाङ्गुलौ \msL, सुशिषि चाङ्गुलौ \Ed}}% 
    \var{{\devanagarifontvar \numnoemph\vd\textbf{॰कर्णये॰}\lem \msNb, ॰कर्षये॰ \mssCaCbCc\msNa\msNc\Ed, ॰कर्षय॰ \msL}}% 

{\devanagarifont तत्तदक्षरमाकर्ण्य अमृतत्वाय कल्प्यते \thinspace{\dandab} \dontdisplaylinenum }%
     \var{{\devanagarifontvar \numemph\va\textbf{॰कर्ण्य}\lem \mssCaCbCc\msNb\msNc\msL\Ed, ॰कण्ण्य \msNa}}% 
    \paral{{\devanagarifontsmall \vab {\englishfont \similar\ \NISVK\ 33.112cd:}
                 सदृशं शब्द आकर्ण्य अमृतत्त्वाय कल्पते }}

%Verse 22:31

{\devanagarifont पिङ्गलां तु शिखाधूमां ध्यायेन्नित्यमतन्द्रितः {॥ २२:३१॥} \veg\dontdisplaylinenum }%
     \var{{\devanagarifontvar \numnoemph\vc\textbf{पिङ्गलांतु शिखाधूमां}\lem \msCa\msCb\msNb\msL, पिङ्गला तु शिखाधूमं \msCc\Ed, 
पिङ्गलांन्तु शिखाधूमां \msNa, पिङ्गलान्तु शिखाधूमा \msNc}}% 
    \var{{\devanagarifontvar \numnoemph\vd\textbf{॰तन्द्रितः}\lem \mssCaCbCc\msNa\msNb\msNc\Ed, ॰तेन्द्रितः \msL}}% 

{\devanagarifont विमुक्तः सर्वपापेभ्यो निर्द्वन्द्वपदमाप्नुयात् \thinspace{\dandab} \dontdisplaylinenum }%
     \var{{\devanagarifontvar \numemph\va\textbf{विमुक्तः}\lem \msCa\msCb\msNa\msNb\msNc\msL\Ed, विमुक्त \msCc}}% 
    \var{{\devanagarifontvar \numnoemph\vb\textbf{निर्द्वन्द्व॰}\lem \mssCaCbCc\msNc, निद्वन्द॰ \msNa\msNb\msL, निर्द्वन्द॰ \Ed}}% 
    \paral{{\devanagarifontsmall \vab {\englishfont \similar\ \NISVK\ 33.56ab:}
                         विमुक्तस्सर्वपापेभ्यो निर्द्वन्द्वं पदमाप्नुयात् }}

%Verse 22:32

{\devanagarifont वैद्युती तु निशामध्ये लक्षते ऽजमनामयम् {॥ २२:३२॥} \veg\dontdisplaylinenum }%
     \var{{\devanagarifontvar \numnoemph\vc\textbf{वैद्युती तु}\lem \mssCaCbCc\msNa\msNb\msNc\Ed, वैद्युतीन्त \msL}}% 
    \var{{\devanagarifontvar \numnoemph\vd\textbf{लक्षते ऽजम॰}\lem \msCc\Ed, लक्ष्ये तेजअ॰ \msCa\msCb, लक्ष्यतेजअ॰ \msNa\msNb\msL, 
लक्षतेज अ॰ \msNc}}% 

{\devanagarifont पञ्चमाससदाभ्यासाद्दिव्यचक्षुर्भवेन्नरः \thinspace{\dandab} \dontdisplaylinenum }%
     \var{{\devanagarifontvar \numemph\va\textbf{पञ्चमाससदा॰}\lem \msCb\msNa\msNb\msL, \uncl{प}\lac  ससदा॰ \msCa, पञ्चमासस्सदा॰ \msCc, 
पञ्चमाससमा॰ \Ed, पञ्चमासं सदा॰ \msNc}}% 
    \var{{\devanagarifontvar \numnoemph\vab\textbf{॰साद्दि}\lem \mssCaCbCc\msNa\msNb\msL\Ed, ॰सा दि॰ \msNc}}% 
    \var{{\devanagarifontvar \numnoemph\vb\textbf{॰क्षुर्भवेन्न॰}\lem \msCa\msCb\msNa\Ed, ॰क्षुर्भवे न॰ \msCc, 
॰क्षु भवेन्न॰ \msNb\msL, ॰र्क्षु भवेन्न \msNc}}% 

%Verse 22:33

{\devanagarifont बिन्दुमालां ततः पश्येत्तरुच्छायासमाश्रिताम् {॥ २२:३३॥} \veg\dontdisplaylinenum }%
     \var{{\devanagarifontvar \numnoemph\vc\textbf{ततः पश्येत्}\lem \mssCaCbCc\msNa\msNb\msNc\msL, तु यः पश्येन् \Ed}}% 
    \var{{\devanagarifontvar \numnoemph\vd\textbf{तरुच्छाया॰}\lem \mssCaCbCc\msNa\msNb\msNc\msL, नरच्छायां \Ed\oo 
\textbf{॰श्रिताम्}\lem \mssCaCbCc\msNb, ॰श्रिताः \msNa\msL, ॰श्रितम् \msNc\Ed}}% 

{\devanagarifont जात्यस्फटिकसंकाशं दृष्ट्वा मुच्यति बन्धनैः \thinspace{\dandab} \dontdisplaylinenum }%
     \var{{\devanagarifontvar \numemph\va\textbf{॰कसंकाशं}\lem \mssCaCbCc\msNa\msNb\msNc\msLpcorr\Ed, ॰संककाशं \msLpcorr}}% 
    \var{{\devanagarifontvar \numnoemph\vb\textbf{बन्धनैः}\lem \msCa\msNa\msNc, बन्धवैः \msCb, बन्धनात् \msCc\msNb\Ed, 
वंचनैः \msL}}% 

%Verse 22:34

{\devanagarifont ध्यायेन्मनोऽनुगा नाम पक्ष्मीरापीड्य लोचने {॥ २२:३४॥} \veg\dontdisplaylinenum }%
     \var{{\devanagarifontvar \numnoemph\vd\textbf{पक्ष्मी॰}\lem \mssCaCbCc\msNa\msL, यक्ष्मी \msNb, यक्ष्मो॰ \msNc, पक्षी॰ \Ed\oo 
\textbf{लोचने}\lem \msCa\msCb\msNa\msL, लोचनः \msNb, लोचनैः \msCc\Ed, लोचनै \msNc}}% 

{\devanagarifont श्वेतपीतारुणं बिन्दुं दृष्ट्वा भूयो न जायते \thinspace{\dandab} \dontdisplaylinenum }%
 
%Verse 22:35

{\devanagarifont मनोऽनुगादि षट्त्वेते ध्यानमुक्तं मया तव {॥ २२:३५॥} \veg\dontdisplaylinenum }%
     \var{{\devanagarifontvar \numemph\vc\textbf{॰षट्त्वेते}\lem \msCa\msNa\msNb\msNc\msL, ॰षट्त्वेता \msCb, ॰षट्केन \msCc\Ed}}% 
    \var{{\devanagarifontvar \numnoemph\vd\textbf{॰क्तं मया तव}\lem \msCc\msNa\msNc\msL\Ed, \uncl{क}\lac  तव \msCa, ॰क्तं समासतः \msCb, 
॰क्त मया तव \msNb}}% 


\alalfejezet{परमाणुः}
{\devanagarifont अधुनान्यत्प्रवक्ष्यामि परमाणु चतुर्विधम् \thinspace{\dandab} \dontdisplaylinenum }%
     \var{{\devanagarifontvar \numemph\vb\textbf{॰विधम्}\lem \msCa\msCc\msNa\msNb\msNc\Ed, ॰विधः \msCb}}% 

{\devanagarifont पार्थिवादिचतुर्भूतं यैर्व्याप्तं निखिलं जगत्  \danda\dontdisplaylinenum }%
     \var{{\devanagarifontvar \numnoemph\vcd\textbf{॰भूतं यैर्व्याप्तं}\lem \msNa, ॰भूतं यैर्व्याप्तिन् \msCa, 
॰भूतं यै व्याप्तं \msCb\msCc\msNb, ॰भूतं यै व्याप्त \msNc, ॰भूतैर्यैर्व्याप्तं \Ed}}% 

%Verse 22:36

{\devanagarifont लक्षणं तस्य राजेन्द्र शृणु वक्ष्यामि साम्प्रतम् {॥ २२:३६॥} \veg\dontdisplaylinenum }%
 
{\devanagarifont पार्थिवोर्ध्वगतिः सूक्ष्मः परमाणु नराधिप \thinspace{\dandab} \dontdisplaylinenum }%
     \var{{\devanagarifontvar \numemph\va\textbf{पार्थिवोर्ध्व॰}\lem \mssCaCbCc\msNa\msNb\msNc, पार्थिवोर्द्ध॰ \Ed}}% 
    \var{{\devanagarifontvar \numnoemph\vb\textbf{परमाणु नराधिप}\lem \msCa\msCb\msNapcorr, परमाणु नराधिपः \msCc, 
परमाणु नराधिनराधिप \msNaacorr, परमानु नराधिप \msNb, 
परमाणुर्नराधिप \msNc\Ed}}% 

%Verse 22:37

{\devanagarifont प्रत्यक्षदर्शनं ध्यानं लक्षयेन्नियतं शुचिः {॥ २२:३७॥} \veg\dontdisplaylinenum }%
     \var{{\devanagarifontvar \numnoemph\vc\textbf{प्रत्यक्षदर्शनं}\lem \mssCaCbCc\msNb\Ed, प्रत्यक्षं दर्शनं \msNa\msNc}}% 
    \var{{\devanagarifontvar \numnoemph\vd\textbf{लक्षयेन्नियतं}\lem \msCa\msNa\msNb\msNc, लक्षयेन्नियतः \msCb, 
लक्षयेन्नियत \msCc, लक्षयन्नियतः \Ed}}% 

{\devanagarifont मुच्यते सर्वपापेभ्यो राहुना चन्द्रमा यथा \thinspace{\dandab} \dontdisplaylinenum }%
     \var{{\devanagarifontvar \numemph\va\textbf{सर्वपापेभ्यो}\lem \msCb\msCc\msNa\msNb\msNc\Ed, \uncl{सर्वपापेभ्यो} \msCa}}% 
    \var{{\devanagarifontvar \numnoemph\vb\textbf{राहुना}\lem \msCb\msCc\msNa\msNb\msNc\Ed, \lk\lk ना \msCa}}% 

%Verse 22:38

{\devanagarifont तेन यो ऽभ्यसते नित्यं स योगी भुवनेश्वरः {॥ २२:३८॥} \veg\dontdisplaylinenum }%
     \var{{\devanagarifontvar \numnoemph\vc\textbf{ऽभ्यसते}\lem \msCa\msCc\msNa\msNb\msNc\Ed, लभ्यते \msCb}}% 
    \var{{\devanagarifontvar \numnoemph\vd\textbf{॰श्वरः}\lem \mssCaCbCc\msNa\msNb\msNc, ॰श्वर \Ed}}% 

{\devanagarifont अधोगति महाराज परमाणु जलोद्भवः \thinspace{\dandab} \dontdisplaylinenum }%
     \var{{\devanagarifontvar \numemph\vb\textbf{परमाणु ज॰}\lem \mssCaCbCc\msNa\msNb\Ed, परमानुर्ज॰ \msNc}}% 

%Verse 22:39

{\devanagarifont अभ्यसेद्यदिदं राजन्सर्वपातकनाशनम् {॥ २२:३९॥} \veg\dontdisplaylinenum }%
     \var{{\devanagarifontvar \numnoemph\vc\textbf{अभ्यसेद्यदिदं}\lem \msCa\msCb\msNa\msNb\msNc\Ed, अभ्यसेदिदं \msCc}}% 

{\devanagarifont आग्नेयपरमाणूनि तिर्यगूर्ध्वगतिः स्मृता \thinspace{\dandab} \dontdisplaylinenum }%
     \var{{\devanagarifontvar \numemph\va\textbf{आग्नेय॰}\lem \mssCaCbCc\msNa\msNc\Ed, अग्नेय॰ \msNb\oo 
\textbf{॰परमाणूनि}\lem \mssCaCbCc\msNa\msNc, ॰परमानूनि \msNb, परमाणुश्च \Ed}}% 
    \var{{\devanagarifontvar \numnoemph\vb\textbf{तिर्यगूर्ध्व॰}\lem \mssCaCbCc\msNa\msNb\msNc, तिर्यगूर्द्ध॰ \Ed\oo 
\textbf{॰गतिः}\lem \mssCaCbCc\msNa\msNc\Ed, ॰मितिः \msNb\oo 
\textbf{स्मृता}\lem \msCa\msNa, स्मृताः \msCb\msCc\msNb\msNc\Ed}}% 

%Verse 22:40

{\devanagarifont य इदं ध्यायते नित्यमुत्तमां गतिमाप्नुयात् {॥ २२:४०॥} \veg\dontdisplaylinenum }%
     \var{{\devanagarifontvar \numnoemph\vd\textbf{गतिमाप्नु॰}\lem \msCa\msCc\msNa\msNb\msNc\Ed, फलमाप्नु॰ \msCb}}% 

{\devanagarifont वायव्यपरमाणूनि अधोर्ध्वतिर्यगास्मृता \thinspace{\dandab} \dontdisplaylinenum }%
     \var{{\devanagarifontvar \numemph\va\textbf{वायव्यपरमाणूनि}\lem \msCb\msNc, वाय\lk\lk रमाणूनि \msCa, वायव्यं परमाणूनि \msCc\msNa, 
वायव्या परमाणूनि \msNb, वायव्यं परमाणुश्च \Ed}}% 
    \var{{\devanagarifontvar \numnoemph\vb\textbf{॰र्ध्वतिर्य॰}\lem \mssCaCbCc\msNb\msNc\Ed, ॰र्ध्वन्तिर्य॰ \msNa}}% 

%Verse 22:41

{\devanagarifont न स मुह्यति तं दृष्ट्वा वायुसम्भव भूपते {॥ २२:४१॥} \veg\dontdisplaylinenum }%
 
{\devanagarifont चत्वार एते राजेन्द्र परमाणु निरीक्षते \thinspace{\dandab} \dontdisplaylinenum }%
     \var{{\devanagarifontvar \numemph\vb\textbf{परमाणु निरीक्षते}\lem \msCa\msCc\msNa\msNb, परमाणुर् रीक्षते \msCb, परमाणुं निरीक्षते \msNc, 
परमाणुर्निरीक्षते \Ed}}% 

%Verse 22:42

{\devanagarifont तेन सर्वमखैरिष्टं तेन तप्तं तपस्तथा {॥ २२:४२॥} \veg\dontdisplaylinenum }%
     \var{{\devanagarifontvar \numnoemph\vc\textbf{॰मखैरि॰}\lem \msCa\msCb\msNa\msNb\Ed, ॰मयैरि॰ \msCc, ॰मखेरि॰ \msNc}}% 
    \var{{\devanagarifontvar \numnoemph\vd\textbf{तप्तं तपस्तथा}\lem \msCa\msCb\msNa\msNb, तप्तं तपन्तथा \msCc, 
सप्तन्तपस्तथा \msNc, तप्तन्तप्तं तथा \Ed}}% 

{\devanagarifont तेन दत्ता मही कृत्स्ना सप्तसागरसंवृता \thinspace{\dandab} \dontdisplaylinenum }%
 
%Verse 22:43

{\devanagarifont सर्वतीर्थाभिषेकश्च सर्वव्रतक्रिया तथा {॥ २२:४३॥} \veg\dontdisplaylinenum }%
     \var{{\devanagarifontvar \numemph\vc\textbf{॰भिषेकश्च}\lem \mssCaCbCc\msNb\msNc\Ed, ॰भिषेक \msNaacorr, ॰भिषेकं च \msNapcorr}}% 

{\devanagarifont अनेनैव विधानेन दशध्यानं नराधिप \thinspace{\dandab} \dontdisplaylinenum }%
     \var{{\devanagarifontvar \numemph\va\textbf{अनेनैव विधानेन}\lem \msCb\msCc\msNa\msNb\msNc\Ed, अ\lk\lac  धानेन \msCa}}% 

%Verse 22:44

{\devanagarifont कुरुते अव्यवच्छिन्नं सर्वकामफलप्रदम् {॥ २२:४४॥} \veg\dontdisplaylinenum }%
     \var{{\devanagarifontvar \numnoemph\vc\textbf{॰च्छिन्नं}\lem \mssCaCbCc\msNa\msNc\Ed, ॰च्छिन्न \msNb}}% 


\alalfejezet{दशाक्षरमन्त्रः}
{\devanagarifont दशाक्षरं महाराज योगीन्द्रस्य महात्मनः \thinspace{\dandab} \dontdisplaylinenum }%
     \var{{\devanagarifontvar \numemph\va\textbf{दशाक्षरं}\lem \msCa\msCb\msNa\msNb\msNc, दशाक्षर॰ \msCc\Ed}}% 

%Verse 22:45

{\devanagarifont कथयामि समासेन शृणुष्वावहितो भव {॥ २२:४५॥} \veg\dontdisplaylinenum }%
 
{\devanagarifont प्रणवादिस्वरा त्रीणि दीर्घबिन्दुसमायुतम् \thinspace{\dandab} \dontdisplaylinenum }%
     \var{{\devanagarifontvar \numemph\va\textbf{त्रीणि}\lem \mssCaCbCc\msNa\msNb\Ed, त्रीनि \msNc}}% 

%Verse 22:46

{\devanagarifont पञ्च पञ्च चवर्गे तु वायुबीजमधःस्थितम् {॥ २२:४६॥} \veg\dontdisplaylinenum }%
     \var{{\devanagarifontvar \numnoemph\vc\textbf{पञ्च पञ्च}\lem \mssCaCbCc\msNa\msNb\msNc\Ed, पञ्च\lk\lk  \Ed\oo 
\textbf{तु}\lem \mssCaCbCc\msNa\msNc\Ed, च \msNb}}% 
    \var{{\devanagarifontvar \numnoemph\vd\textbf{॰धःस्थितम्}\lem \mssCaCbCc\msNb\msNc, ॰धस्थितं \msNa\Ed}}% 

{\devanagarifont त्रयोदशस्वरायुक्तं पञ्चमे परिकीर्तितम् \thinspace{\dandab} \dontdisplaylinenum }%
     \var{{\devanagarifontvar \numemph\va\textbf{त्रयोदश॰}\lem \msCb\msCc\msNa\msNb\Ed, \lac  योदश॰ \msCa\msNc\oo 
\textbf{॰युक्तं}\lem \mssCaCbCc\msNa\msNb\Ed, ॰युक्त \msNc}}% 
    \var{{\devanagarifontvar \numnoemph\vb\textbf{पञ्चमे }\lem\mssCaCbCc\msNa\msNb\msNc, पञ्चम \Ed}}% 

%Verse 22:47

{\devanagarifont पञ्चविंशतिमः षष्ठः अक्षरः परिकीर्तितः {॥ २२:४७॥} \veg\dontdisplaylinenum }%
     \var{{\devanagarifontvar \numnoemph\vc\textbf{षष्ठः}\lem \mssCaCbCc\msNa\msNb\msNc, षष्ठ \Ed}}% 

{\devanagarifont यादृशं पञ्चमे प्रोक्तं सप्तमे च प्रयोजयेत् \thinspace{\dandab} \dontdisplaylinenum }%
     \var{{\devanagarifontvar \numemph\va\textbf{पञ्चमे}\lem \msCb\msNa\msNc, पञ्चमेः \msCa\msCc\msNb, पञ्चमः \Ed}}% 

%Verse 22:48

{\devanagarifont आकारस्वरसंयुक्तं सर्वपातकनाशनम् {॥ २२:४८॥} \veg\dontdisplaylinenum }%
     \var{{\devanagarifontvar \numnoemph\vc\textbf{आकार॰}\lem \msCa\msCb\msNa\msNb\msNc, अकार॰ \msCc\Ed}}% 

{\devanagarifont प्रथमं पञ्चमे वर्गे तृतीयस्वरयोजितम् \thinspace{\dandab} \dontdisplaylinenum }%
 
%Verse 22:49

{\devanagarifont उत्तरेकारसंयुक्तं नवमं परिकीर्तितम् {॥ २२:४९॥} \veg\dontdisplaylinenum }%
     \var{{\devanagarifontvar \numemph\vc\textbf{उत्तरेकारसंयुक्तं}\lem \msCb\msCc\msNa\msNb\msNc, उत्तरेका\lac \lk\lk\ \msCa, उक्तरेकारसंयुक्तं \Ed}}% 
    \var{{\devanagarifontvar \numnoemph\vb\textbf{नवमं परिकीर्तितम्}\lem \msCb\msCc\msNa\msNb\msNc\Ed, \lk\lk\lac  रिकीर्तितम् \msCa}}% 

{\devanagarifont दशमः पुनरोंकारः मन्त्रश्रेष्ठो दशाक्षरः \thinspace{\dandab} \dontdisplaylinenum }%
     \var{{\devanagarifontvar \numemph\va\textbf{॰कारः}\lem \msCa\msCc\msNa\msNb\msNc\Ed, ॰कारौ \msCb}}% 

{\devanagarifont जपतो ध्यायतो वापि पार्थिवादिक्रमेण तु  \danda\dontdisplaylinenum }%
 
%Verse 22:50

{\devanagarifont मुच्यते सो ऽपि संसारे संशयो नास्ति भूपते {॥ २२:५०॥} \veg\dontdisplaylinenum }%
     \var{{\devanagarifontvar \numnoemph\va\textbf{संसारे}\lem \mssCaCbCc\msNb\msNc\Ed, संसार \msNa}}% 


\alalfejezet{आचारविधिः}
{\devanagarifont आचारमूलो धर्मस्तु धर्ममूलो जनार्दनः \thinspace{\dandab} \dontdisplaylinenum }%
 
%Verse 22:51

{\devanagarifont तेन सर्वजगद्व्याप्तं त्रैलोक्यं सचराचरं {॥ २२:५१॥} \veg\dontdisplaylinenum }%
 
\ujvers\nemsloka {
{\devanagarifont आचाराल्लभतीह आयुरतुलमक्षप्यवित्तं तथा }%
  \dontdisplaylinenum}    \var{{\devanagarifontvar \numemph\va\textbf{आचाराल्लभतीह}\lem \msCb\msNa\msNb\msNc\Ed, \lac  भतीह \msCa, आचारा लभतीह \msCc\oo 
\textbf{॰मक्षप्यवित्तं तथा}\lem \mssCaCbCc\msNa\msNb\msNc, ॰मैश्वर्य्यवित्तन्तथा \Ed}}% 


\nemslokab

{\devanagarifont आचारात्सुतमीप्सितं च लभते श्रीकीर्तिप्रज्ञायशः  \danda\dontdisplaylinenum }%
     \var{{\devanagarifontvar \numnoemph\vb\textbf{आचारात्सुतमीप्सितं च}\lem \msCb\msCc\msNa\msNb\msNc\Ed, 
आचारात्सु\uncl{तमीप्सित}\lac\  \msCa\oo 
\textbf{श्रीकीर्तिप्रज्ञायशः}\lem \msCa\msCb\msNapcorr\msNb\msNc\Ed, 
श्रीकीर्तिप्रज्ञां यशः \msCc, \om\ \msNaacorr}}% 

\nemslokac

{\devanagarifont आचाराल्लभते च लक्ष्मिमतुलां ख्यातिं तथैवोत्तमाम् }%
  \dontdisplaylinenum    \var{{\devanagarifontvar \numnoemph\vc\textbf{आचाराल्लभते}\lem \mssCaCbCc\msNapcorr\msNb\msNc\Ed, \om\ \msNaacorr\oo 
\textbf{लक्ष्मिमतुलां}\lem \msCb\msCc, लतुलं \msCa, लक्ष्मिमतुलं \msNa\msNb\Ed, \lk क्ष्मिमतुलं \msNc\oo 
\textbf{॰त्तमाम्}\lem \msCb\msCc\msNa, ॰त्तमम् \msCa\msNb\msNc\Ed}}% 

%Verse 22:52


\nemslokad

{\devanagarifont आचारादिह मन्त्रधर्मपरमं प्राप्नोति निःसंशयम् {॥ २२:५२॥} \veg\dontdisplaylinenum }%
     \var{{\devanagarifontvar \numnoemph\vd\textbf{आचारादिह}\lem \mssCaCbCc\msNa\msNc\Ed, आचारादिषु \msNb\oo 
\textbf{॰परमं}\lem \mssCaCbCc\msNa\msNb\Ed, ॰परम \msNc\oo 
\textbf{॰संशयम्}\lem \mssCaCbCc\msNc\Ed, ॰संशयः \msNa\msNb}}% 

\vers


{\devanagarifont जनमेजय उवाच {\dandab}\dontdisplaylinenum  }%
 
\nemsloka 
{\devanagarifont आचारात्प्रभवानुसंशकथितं सुश्लिष्टधर्माकरम् }%
  \dontdisplaylinenum    \var{{\devanagarifontvar \numemph\va\textbf{॰संशकथितं सुश्लिष्ट॰}\lem \msCb, ॰संश\lac  श्लिष्ट \msCa, ॰संसयथितं सुश्लिष्व॰ \msCc, ॰सङ्गकथितं सुश्लिष्ट॰ \Ed}}% 


\nemslokab

{\devanagarifont आचारात्कतिवंश कीर्तय पुनस्तृप्तिर्न मे जायते  \danda\dontdisplaylinenum }%
     \var{{\devanagarifontvar \numnoemph\vb\textbf{॰तिवंश}\lem \msCa\msCb, ॰थिवंस \msCc, ॰तिधाऽङ्ग \Ed\oo 
\textbf{पुनस्तृप्तिर्न}\lem \msCa\msCb\Ed, पुनस्तृप्ति न्न \msCc}}% 

\nemslokac

{\devanagarifont सर्वज्ञः त्वमहं शृणोमि वरदं किञ्चिन्न मे शाश्वतम् }%
  \dontdisplaylinenum    \var{{\devanagarifontvar \numnoemph\vc\textbf{॰ज्ञः}\lem \msCa\msCb\Ed, ॰ज्ञ \msCc\oo 
\textbf{किञ्चिन्न}\lem \msCc\Ed, किञ्चन \msCa, किञ्चान् \msCb}}% 

%Verse 22:53


\nemslokad

{\devanagarifont तन्मे कीर्तय धर्मसारशुभदमाचारमूलाश्रयम् {॥ २२:५३॥} \veg\dontdisplaylinenum }%
     \var{{\devanagarifontvar \numnoemph\vd\textbf{॰सारशुभदमा॰}\lem \msCa\msCb\Ed, ॰साभदं \msCc}}% 

\vers


{\devanagarifont वैशम्पायन उवाच {\dandab}\dontdisplaylinenum  }%
 
\nemsloka 
{\devanagarifont नित्यं नम्रशिरोद्विजातिगुरुषु शुश्रूषणं देवता }%
  \dontdisplaylinenum    \var{{\devanagarifontvar \numemph\va\textbf{देवता}\lem \msCb, देव\lk\ \msCa, दैवता \msCc, दैवतम् \Ed}}% 


\nemslokab

{\devanagarifont तिष्ठेताचमनेन चाशनकरं वामास्थिनानोददे  \danda\dontdisplaylinenum }%
     \var{{\devanagarifontvar \numnoemph\vb\textbf{तिष्ठेताचमनेन}\lem \msCb\msCc, \lac  ताचमनेन \msCa, तिष्ठेनाचमनेन \Ed\oo 
\textbf{वामास्थिनानोददे}\lem \msCa\msCb, वामास्थिनानोओदरे \msCc, वामास्थि मानादरम् \Ed}}% 

\nemslokac

{\devanagarifont सूर्याग्निशशिबन्धुरार्यपुरतः कुर्यान्न चावश्यकम् }%
  \dontdisplaylinenum    \var{{\devanagarifontvar \numnoemph\vc\textbf{बन्धुरार्यपुरतः}\lem \Ed, बन्धु आर्यपुरुतः \msCa, बन्धु आर्यपुरतः \msCb\msCc}}% 

%Verse 22:54


\nemslokad

{\devanagarifont शस्ये भस्मनि गोव्रजे द्विज जलं कुर्यान्न चार्कं नरः {॥ २२:५४॥} \veg\dontdisplaylinenum }%
     \var{{\devanagarifontvar \numnoemph\vd\textbf{भस्मनि गोव्रजे}\lem \msCa\msCc\Ed, \uncl{भस्मनि गोव्र}जे \msCb}}% 

\ujvers\nemsloka {
{\devanagarifont पादेनाग्निजलं स्पृशेन्न च गुरुं पादेन पादं तथा }%
  \dontdisplaylinenum}    \var{{\devanagarifontvar \numemph\va\textbf{॰ग्निजलं}\lem \msCa\msCc\Ed, ॰ग्निं जलं \msCb}}% 


\nemslokab

{\devanagarifont शौचं कार्य जलादिना च नियतं नाधो जलं कारयेत्  \danda\dontdisplaylinenum }%
     \var{{\devanagarifontvar \numnoemph\vb\textbf{जलं कारयेत्}\lem \msCb\msCc\Ed, \lac\  \msCa\oo 
\textbf{नियतं}\lem \msCa\msCb\Ed, निनियतं \msCc}}% 

\nemslokac

{\devanagarifont कुर्यान्नित्यभिवादनं द्विजगुरोर्मातापितृदेवताम् }%
  \dontdisplaylinenum    \var{{\devanagarifontvar \numnoemph\vc\textbf{॰वादनं}\lem \msCa\msCc\Ed, ॰वादं न \msCb\oo 
\textbf{॰पितृदेवताम्}\lem \mssCaCbCc, ॰पित्\char"0930\char"094D\char"090Bद्दैवतम् \Ed}}% 

%Verse 22:55


\nemslokad

{\devanagarifont एताचारविधिः समासनियमस्तुभ्यं मया कीर्तितम् {॥ २२:५५॥} \veg\dontdisplaylinenum }%
     \var{{\devanagarifontvar \numnoemph\vd\textbf{समास॰}\lem \msCa\msCc\Ed, समा॰ \msCb}}% 


\alalfejezet{स्त्रियः}
\vers


{\devanagarifont जनमेजय उवाच {\dandab}\dontdisplaylinenum  }%
 
\nemsloka 
{\devanagarifont स्त्रीणां किं प्रियमस्ति तद्वद विभो संसारसारस्त्रियाम् }%
  \dontdisplaylinenum

\nemslokab

{\devanagarifont किं सद्भाव न वेद्मि तस्य विषये किं द्वेष्य किं तात्प्रियम्  \danda\dontdisplaylinenum }%
 
\nemslokac

{\devanagarifont पश्यामि न च तस्य गर्भकलया प्राप्नोति निःसंशयम् }%
  \dontdisplaylinenum
%Verse 22:56


\nemslokad

{\devanagarifont मायाजालसहस्रगापि युवती कुर्वन्ति भर्ता प्रियम् {॥ २२:५६॥} \veg\dontdisplaylinenum }%
 
\vers


{\devanagarifont वैशम्पायन उवाच {\dandab}\dontdisplaylinenum  }%
 
\nemsloka 
{\devanagarifont राजन्किं प्रियमस्ति अर्थपरतः पश्यामि नान्यन्नृपे }%
  \dontdisplaylinenum

\nemslokab

{\devanagarifont पुत्रार्थैकप्रयोजनं युवतयः स्वायम्भुवोक्तामरैः  \danda\dontdisplaylinenum }%
 
\nemslokac

{\devanagarifont कान्ता नित्यकला प्रवर्तनकरी धर्मसखाया सती }%
  \dontdisplaylinenum
%Verse 22:57


\nemslokad

{\devanagarifont माया वापि करोति सद्य मनुजात्यक्तान्य वा सेवते {॥ २२:५७॥} \veg\dontdisplaylinenum }%
 
\ujvers\nemsloka {
{\devanagarifont स्त्रीसङ्गं परिवर्जयेन्नरपते आयासदं दुःखदम् }%
  \dontdisplaylinenum}

\nemslokab

{\devanagarifont मृत्युद्वारभयाकरं विषगृहमापत्सुघोरालयम्  \danda\dontdisplaylinenum }%
 
\nemslokac

{\devanagarifont अग्निं मारुत मत्तवारणसमं तस्यानुगामी सदा }%
  \dontdisplaylinenum
%Verse 22:58


\nemslokad

{\devanagarifont स्त्रीहेतोर्हत रावणस्त्रिदशपैन्द्रो ऽप्यवस्थाकृतः {॥ २२:५८॥} \veg\dontdisplaylinenum }%
 
%Verse 22:58


\nemslokad

{\devanagarifont दण्डक्यो हतराष्ट्रपौरसहितः किं भूय वक्ष्याम्यहम् {॥ २२:५८॥} \veg\dontdisplaylinenum }%
 

\alalfejezet{विप्र-मुनि-भिक्षु-निर्ग्रन्थि-परिव्राजक-र्ष्यादयः}
\vers


{\devanagarifont जनमेजय उवाच {\dandab}\dontdisplaylinenum  }%
 
\nemsloka 
{\devanagarifont विप्रे कीदृशलक्षणं भवति भो कीदृग्मुनिश्चोच्यते }%
  \dontdisplaylinenum

\nemslokab

{\devanagarifont तेनार्थेन भवेत भिक्षु भगवन्निग्रन्थि को वा द्विज  \danda\dontdisplaylinenum }%
 
\nemslokac

{\devanagarifont केनार्थेन भवेद्द्विजेन्द्र भगवन्ज्ञेयः परिव्राजकः }%
  \dontdisplaylinenum
%Verse 22:59


\nemslokad

{\devanagarifont ! ज्ञेयाः किमृषयश्च लक्षण मुनेरिच्छामि ज्ञातुं पुनः {॥ २२:५९॥} \veg\dontdisplaylinenum }%
 
\vers


{\devanagarifont वैशम्पायन उवाच {\dandab}\dontdisplaylinenum  }%
 
\nemsloka 
{\devanagarifont सत्यं शौचमहिंसता दमशमौ भूतानुकम्पी सदा }%
  \dontdisplaylinenum

\nemslokab

{\devanagarifont आत्मारामजितो स्वधर्मनिरतः सत्त्वस्थ नित्यं मनः  \danda\dontdisplaylinenum }%
 
\nemslokac

{\devanagarifont कामक्रोधयमस्वदारनिरतः संत्यज्य लोभः शनैः }%
  \dontdisplaylinenum
%Verse 22:60


\nemslokad

{\devanagarifont एवं यः कुरुते द्विजातिसुवरः शूद्रो ऽपि यः संयमी {॥ २२:६०॥} \veg\dontdisplaylinenum }%
 
\ujvers\nemsloka {
{\devanagarifont तस्माच्छद्मकवर्जितः स भगवान्संसारभीभिद्यकः }%
  \dontdisplaylinenum}

\nemslokab

{\devanagarifont यत्तत्स्थानपरं व्रजन्ति पुरुषाः तस्मात्परिव्राजकः  \danda\dontdisplaylinenum }%
 
\nemslokac

{\devanagarifont ग्रन्थिदारसुतं धनंश्च विरति निर्ग्रन्थिक सोच्यते }%
  \dontdisplaylinenum
%Verse 22:61


\nemslokad

{\devanagarifont रम्यन्ते ऋषिराश्रमे धृतिमनस्तस्मादृषिः सोच्यते {॥ २२:६१॥} \veg\dontdisplaylinenum }%
 
\ujvers\nemsloka {
{\devanagarifont कायवाङ्मनदण्डतत्परतरस्ते दण्डिकरूच्यते }%
  \dontdisplaylinenum}

\nemslokab

{\devanagarifont सद्धर्मश्रवणं वदन्ति श्रवणः सद्धर्मब्रह्माक्षरः  \danda\dontdisplaylinenum }%
 
\nemslokac

{\devanagarifont पाशप्रक्षिपतं पशुत्वसकलं पाशूपतास्ते स्मृताः }%
  \dontdisplaylinenum
%Verse 22:62


\nemslokad

{\devanagarifont विप्रे पाशुपतादिभिक्षुसकलं पृष्टो ऽस्म्यहं लक्षणम् {॥ २२:६२॥} \veg\dontdisplaylinenum }%
 
\ujvers\nemsloka {
{\devanagarifont सर्वं तत्कथितो ऽसि लक्षण मया सन्धिश्वनिर्नाशनम् }%
  \dontdisplaylinenum}

\nemslokab

{\devanagarifont प्रज्ञासंग्रहशीतवर्धनपरं संसारनिर्मूलनम्  \danda\dontdisplaylinenum }%
 
\nemslokac

{\devanagarifont   }%
  \dontdisplaylinenum
%Verse 22:63


\nemslokad

{\devanagarifont एतज्ज्ञानपरं प्रबोधमतुलं नित्यं शिवं धार्यते {॥ २२:६३॥} \veg\dontdisplaylinenum }%
 
\vers


{\devanagarifont 
\jump
\begin{center}
\ketdanda~इति वृषसारसंग्रहे द्वाविंशतितमो ऽध्यायः~\ketdanda
\end{center}
\dontdisplaylinenum\vers  }%
 \bekveg\szamveg
\vfill
\phpspagebreak

\versno=0\fejno=23
\thispagestyle{empty}

\centerline{\Large\devanagarifontbold [   त्रयोविंशतितमो ऽध्यायः  ]}{\vrule depth10pt width0pt} \fancyhead[CE]{{\footnotesize\devanagarifont वृषसारसंग्रहे  }}
\fancyhead[CO]{{\footnotesize\devanagarifont त्रयोविंशतितमो ऽध्यायः  }}
\fancyhead[LE]{}
\fancyhead[RE]{}
\fancyhead[LO]{}
\fancyhead[RO]{}
\szam\bek


\vers


{\devanagarifont जनमेजय उवाच {\dandab}\dontdisplaylinenum  }%
 
{\devanagarifont देवानां दानवानां च उत्तरारणिमेव च \thinspace{\danda} \dontdisplaylinenum }%
     \var{{\devanagarifontvar \numemph\vab\textbf{दानवानां च उत्तरारणिमेव}\lem \msNa\Ed, 
दा\lac  णिमेव \msCa}}% 

%Verse 23:1

{\devanagarifont विद्विषन्ति च ते ऽन्योन्यं कारणं तस्य कीर्तय {॥ २३:१॥} \veg\dontdisplaylinenum }%
     \var{{\devanagarifontvar \numnoemph\vd\textbf{तस्य}\lem \msCa\Ed, त\uncl{स्य} \msNa}}% 

{\devanagarifont वैशम्पायन उवाच {\dandab}\dontdisplaylinenum  }%
 
{\devanagarifont पापपुण्यस्वभावाभ्यां देवदैत्यस्य भूपते \thinspace{\danda} \dontdisplaylinenum }%
 
%Verse 23:2

{\devanagarifont धर्मपक्षस्मृतो देवो दानवो ऽधर्मपक्षतः {॥ २३:२॥} \veg\dontdisplaylinenum }%
     \var{{\devanagarifontvar \numemph\vc\textbf{धर्मपक्ष॰}\lem \msNa, धर्मे पक्षः \msCa, 
धर्मपक्षः \Ed\oo 
\textbf{देवो}\lem \msCa\msNa, देवा \Ed}}% 
    \var{{\devanagarifontvar \numnoemph\vd\textbf{ऽधर्म॰}\lem \Ed, दर्प्प॰ \msCa, दर्प॰ \msNa}}% 

{\devanagarifont हेतुना तेन राजेन्द्र अन्योन्यं विद्विषन्ति ते \thinspace{\dandab} \dontdisplaylinenum }%
 
%Verse 23:3

{\devanagarifont देवद्वेष्टासुराः सर्वे विबुधाश्चासुरद्विषः {॥ २३:३॥} \veg\dontdisplaylinenum }%
     \var{{\devanagarifontvar \numemph\vc\textbf{देवद्वेष्टासुराः सर्वे}\lem \eme, 
देवद्वेष्टासुरः सर्वे \msNa\Ed, 
\uncl{दे}वद्वे\uncl{ष्टा}सुरस् \lac\  \msCa}}% 
    \var{{\devanagarifontvar \numnoemph\vd\textbf{विबुधाश्}\lem \msNa\Ed, \lac  धाश् \msCa}}% 


\alalfejezet{धर्माधर्मविपक्षता}
\ujvers\nemsloka {
{\devanagarifont धर्माधर्मविपक्षतां शृणु परां भूतानुकम्पोदयाम् }%
  \dontdisplaylinenum}    \var{{\devanagarifontvar \numemph\va\textbf{॰विपक्षतां}\lem \Ed, ॰विवक्षतां \msCa\msNa\oo 
\textbf{॰कम्पोदयाम्}\lem \msCa\msNa, ॰कम्पादयाम् \Ed}}% 


\nemslokab

{\devanagarifont सत्यं शौचमहिंसता दमशमो निर्मानमीर्ष्यारुषा  \danda\dontdisplaylinenum }%
     \var{{\devanagarifontvar \numnoemph\vb\textbf{ईर्षा॰}\lem \msCa\msNa, ईर्ष्या॰ \Ed}}% 

\nemslokac

{\devanagarifont तृष्णालोभरतस्य कामविषयः सर्वेन्द्रियाणां जयः }%
  \dontdisplaylinenum
%Verse 23:4


\nemslokad

{\devanagarifont आध्यात्मेषु रतिः प्रसन्नमनसो निर्द्वन्द्वसर्वालयः {॥ २३:४॥} \veg\dontdisplaylinenum }%
     \var{{\devanagarifontvar \numnoemph\vd\textbf{प्रसन्नमनसो निर्द्वन्द्व॰}\lem \msNa\Ed, 
प्रसन्न\lac\  \msCa}}% 

\ujvers\nemsloka {
{\devanagarifont पापोपेक्षणशश्वपुण्यमुदितो दीनेषु कारुण्यता }%
  \dontdisplaylinenum}    \var{{\devanagarifontvar \numemph\va\textbf{पापो॰}\lem \msCa\msNa, पापा॰ \Ed\oo 
\textbf{॰शश्व॰}\lem \msCa\msNa, ॰शश्च॰ \Ed}}% 


\nemslokab

{\devanagarifont दानं शीलधृतिक्षमाजपतपः स्वाध्यायमौने रतिः  \danda\dontdisplaylinenum }%
 
\nemslokac

{\devanagarifont योगाभ्यासरतिर्दिवौकसगणे ज्ञाने च सांख्ये तथा }%
  \dontdisplaylinenum    \var{{\devanagarifontvar \numnoemph\vc\textbf{योगाभ्यासरतिर्दिवौकस॰}\lem \msCa, 
योगाभ्यासरतिदिवौकस॰ \msNa\ \unmetr, 
योगभ्यासरतिदिवैकस॰ \Ed\ \unmetr}}% 

%Verse 23:5


\nemslokad

{\devanagarifont अक्रोधार्जवतेजयज्ञमभयं संतोष अद्रोहता {॥ २३:५॥} \veg\dontdisplaylinenum }%
     \var{{\devanagarifontvar \numnoemph\vd\textbf{॰भयं}\lem \Ed, ॰भयस् \msCa, ॰भयः \msNa}}% 

\ujvers\nemsloka {
{\devanagarifont त्यागो मार्दवह्रीरचापलरतिर्न्यासाभिमानो ग्रहात् }%
  \dontdisplaylinenum}    \var{{\devanagarifontvar \numemph\va\textbf{॰ह्रीरचापलरतिन्यासा॰}\lem \msNa\Ed, 
\uncl{ह्री}\lac  रतिर्न्यासा॰ \msCa}}% 


\nemslokab

{\devanagarifont मैत्रीभावसदारपैशुनमतिर्ब्राह्मण्यश्रद्धान्वितः  \danda\dontdisplaylinenum }%
     \var{{\devanagarifontvar \numnoemph\vb\textbf{॰न्वितः}\lem \msNa, ॰न्विता \msCa\Ed}}% 

\nemslokac

{\devanagarifont एताचार सदा नरेन्द्र विबुधाः संक्षेपतः कीर्तिताः }%
  \dontdisplaylinenum    \var{{\devanagarifontvar \numnoemph\vc\textbf{कीर्तिताः}\lem \msCa\msNa, कीर्तितः \Ed}}% 

%Verse 23:6


\nemslokad

{\devanagarifont दैत्यानां शृणु कीर्तये स्ववहितो ऽसम्भाव्य तेषां निजम् {॥ २३:६॥} \veg\dontdisplaylinenum }%
     \var{{\devanagarifontvar \numnoemph\vd\textbf{दैत्यानां}\lem \msNa\Ed, दैत्याना \msCa\oo 
\textbf{कीर्तये}\lem \msCa\Ed, कीर्तय \msNa\oo 
\textbf{स्ववहितो}\lem \msCa, स्ववहिसं \msNa, 
त्ववहितो \Ed\oo 
\textbf{निजम्}\lem \msCa\Ed, निजः \msNa}}% 

\ujvers\nemsloka {
{\devanagarifont दैत्याः पापरतिस्वभावचपला निर्लज्जदर्पालसाः }%
  \dontdisplaylinenum}    \var{{\devanagarifontvar \numemph\va\textbf{दैत्याः}\lem \msCa, दैत्या \msNa\Ed}}% 


\nemslokab

{\devanagarifont कामक्रोधवशाः सुदुष्टमनसस्तृष्णाधिका निर्दयाः  \danda\dontdisplaylinenum }%
     \var{{\devanagarifontvar \numnoemph\vb\textbf{कामक्रोधवशाः}\lem \msNa\Ed, \lk\lac  शास् \msCa}}% 

\nemslokac

{\devanagarifont शौचाचारविवर्जिता गुरुगिरान्नानित्य कुर्युः क्रियाः }%
  \dontdisplaylinenum
%Verse 23:7


\nemslokad

{\devanagarifont जीवाकर्षणजीवनः प्रतिदिनं मोहान्धरागान्विताः {॥ २३:७॥} \veg\dontdisplaylinenum }%
     \var{{\devanagarifontvar \numnoemph\vd\textbf{जीवाकर्षण॰}\lem \msCa\msNa, नैवाकर्षण॰ \Ed}}% 

\ujvers\nemsloka {
{\devanagarifont निद्रा नित्य दिवा प्रसक्तमशुचिः सूर्योदये सुप्यते }%
  \dontdisplaylinenum}

\nemslokab

{\devanagarifont आशापाशशतैर्निबद्धहृदयो हृत्वा परस्वं पुनः  \danda\dontdisplaylinenum }%
     \var{{\devanagarifontvar \numemph\vb\textbf{हृत्वा परस्वं पुनः}\lem \msNa\Ed, 
\uncl{हृ}\lac  नः \msCa}}% 

\nemslokac

{\devanagarifont मात्सर्यात्परपाकभेदनिरतो मूलस्य दुष्पूरता }%
  \dontdisplaylinenum    \var{{\devanagarifontvar \numnoemph\vc\textbf{मात्सर्या}\lem \msCa\msNa, मांसर्या॰ \Ed}}% 

%Verse 23:8


\nemslokad

{\devanagarifont ! नास्तीकत्वपराङ्गनास्वभिरत उत्कोचकामः सदा {॥ २३:८॥} \veg\dontdisplaylinenum }%
     \var{{\devanagarifontvar \numnoemph\vd\textbf{॰पराङ्गनास्वभिरत}\lem \msCa, 
॰पराङ्गनास्त्वभिरत \msNa, 
॰पराङ्गनाप्यभिरतो \Ed\oo 
\textbf{उत्कोच॰}\lem \msCa\msNa, उक्ता च \Ed}}% 

\ujvers\nemsloka {
{\devanagarifont देवब्राह्मण विद्विषन्ति सततं लोभाच्च कार्यक्रिया }%
  \dontdisplaylinenum}

\nemslokab

{\devanagarifont धर्मं दूषयते च मूढमनसा आर्यं च तीर्थं तथा  \danda\dontdisplaylinenum }%
 
\nemslokac

{\devanagarifont हन्तव्यान्यहताश्च मन्यबहवो विस्फूर्जितमद्रुवन् }%
  \dontdisplaylinenum    \var{{\devanagarifontvar \numemph\vc\textbf{॰हताश्}\lem \msCa, ॰हतांश् \msNa, ॰हतां \Ed\oo 
\textbf{मन्य॰}\lem \msCa\msNa, यन्य \Ed\oo 
\textbf{विस्फूर्जितमद्रुवन्}\lem \msNa, 
वि\lac  द्रुवन् \msCa, 
विस्फुर्ज्जिते नक्रवत् \Ed}}% 

%Verse 23:9


\nemslokad

{\devanagarifont दैत्यानां कथितं च चिह्न कतिचित्सद्भावतः कीर्तितम् {॥ २३:९॥} \veg\dontdisplaylinenum }%
     \var{{\devanagarifontvar \numnoemph\vd\textbf{कथितं}\lem \msCa\msNa, कथितश् \Ed}}% 

\ujvers\nemsloka {
{\devanagarifont मर्त्येष्वेव नरेन्द्र मानुषमभूद्देवासुराणां नृपः }%
  \dontdisplaylinenum}

\nemslokab

{\devanagarifont यो यं प्रोक्तः स्वभावतामुभयतो मानुष्यलोके तथा  \danda\dontdisplaylinenum }%
     \var{{\devanagarifontvar \numemph\vb\textbf{॰लोके}\lem \msCa\msNa, ॰लोकन् \Ed}}% 

\nemslokac

{\devanagarifont यन्मे पृच्छितवान्नरेन्द्र कथितं यत्त्वं पुरा गोपितम् }%
  \dontdisplaylinenum    \var{{\devanagarifontvar \numnoemph\vc\textbf{पृच्छितवान्}\lem \msNa\Ed, पृच्छितवा \msCa}}% 

%Verse 23:10


\nemslokad

{\devanagarifont विद्वेषोभयकारणं नरपते किं भूय वक्ष्याम्यहम् {॥ २३:१०॥} \veg\dontdisplaylinenum }%
     \var{{\devanagarifontvar \numnoemph\vd\textbf{विद्वेषोभयकारणं नरपते किं}\lem \msNa\Ed, 
वि\uncl{द्वेषोभय}\lac  पते कि \msCa}}% 


\alalfejezet{निद्रोत्त्पत्तिः}
\vers


{\devanagarifont जनमेजय उवाच {\dandab}\dontdisplaylinenum  }%
 
{\devanagarifont अस्ति कौतूहलं चान्यं पृच्छामि त्वां द्विजोत्तम \thinspace{\danda} \dontdisplaylinenum }%
     \var{{\devanagarifontvar \numemph\va\textbf{कौतूहलं}\lem \msCa\msNa, कौतुहलंश् \Ed}}% 

%Verse 23:11

{\devanagarifont कथं निद्रा समुत्पन्ना सर्वभूतविमोहनी {॥ २३:११॥} \veg\dontdisplaylinenum }%
     \var{{\devanagarifontvar \numnoemph\vd\textbf{॰मोहनी}\lem \msCapcorr\msNa\Ed, ॰मोहिनी \msCaacorr}}% 

{\devanagarifont रात्रौ प्रजायते कस्माद्दिवा कस्मान्न जायते \thinspace{\dandab} \dontdisplaylinenum }%
 
{\devanagarifont कस्माच्च कुरुते जन्तोर्निद्रा नेत्रप्रमीलनम्  \danda\dontdisplaylinenum }%
     \var{{\devanagarifontvar \numemph\vc\textbf{जन्तोर्}\lem \msCa\msNa, जन्तो \Ed}}% 

%Verse 23:12

{\devanagarifont एतन्मे संशयं छिन्धि सर्वज्ञो ऽसि द्विजोत्तम {॥ २३:१२॥} \veg\dontdisplaylinenum }%
     \var{{\devanagarifontvar \numnoemph\vf\textbf{सर्वज्ञो ऽसि}\lem \msNa\Ed, \lac\  \msCa}}% 

{\devanagarifont वैशम्पायन उवाच {\dandab}\dontdisplaylinenum  }%
 
{\devanagarifont देवी ह्येषा महाभागा निद्रा नेत्राश्रया नृणाम् \thinspace{\danda} \dontdisplaylinenum }%
     \var{{\devanagarifontvar \numemph\vb\textbf{॰श्रया}\lem \msCa\msNa, ॰श्रयो \Ed}}% 

%Verse 23:13

{\devanagarifont तस्या वशं गतं सर्वं जगत्स्थावरजङ्गमम् {॥ २३:१३॥} \veg\dontdisplaylinenum }%
 
{\devanagarifont सदेवदानवगणा गन्धर्वोरगराक्षसाः \thinspace{\dandab} \dontdisplaylinenum }%
     \var{{\devanagarifontvar \numemph\va\textbf{॰दानव॰}\lem \msCa\Ed, ॰दानवा॰ \msNa}}% 

%Verse 23:14

{\devanagarifont यक्षभूतपिशाचाश्च पशुपक्षिसरीसृपाः {॥ २३:१४॥} \veg\dontdisplaylinenum }%
     \var{{\devanagarifontvar \numnoemph\vd\textbf{॰सरीसृपाः}\lem \msCa\msNa, ॰शरीसृपः \Ed}}% 

{\devanagarifont गुह्यकाश्च मृगा नागा किंनरा जलजोरगाः \thinspace{\dandab} \dontdisplaylinenum }%
     \var{{\devanagarifontvar \numemph\va\textbf{गुह्यकाश्च}\lem \eme, 
गुह्यकश्च \Ed, 
गुह्यवस्त्र॰ \msCa\msNa\oo 
\textbf{नागाः}\lem \msCa\msNa, नागा \Ed}}% 
    \var{{\devanagarifontvar \numnoemph\vb\textbf{किंनरा जलजोरगाः}\lem \eme, 
किंनरा जलजा नगाः \msNa\Ed, 
किन्न\lac\  गाः \msCa}}% 

%Verse 23:15

{\devanagarifont निद्रावशगताः सर्वे पाप्मना त्वभिलङ्घिताः {॥ २३:१५॥} \veg\dontdisplaylinenum }%
 
{\devanagarifont देवदानवकर्मान्ते तस्मिन्नमृतसम्भवे \thinspace{\dandab} \dontdisplaylinenum }%
     \var{{\devanagarifontvar \numemph\va\textbf{॰कर्मान्ते}\lem \msCa\msNa, ॰कर्मात्ते \Ed}}% 
    \var{{\devanagarifontvar \numnoemph\vb\textbf{॰मृत॰}\lem \msCa\msNa, ॰नृत॰ \Ed}}% 

%Verse 23:16

{\devanagarifont मन्दरोत्थापने विष्णुर्देवासुरसमागमे {॥ २३:१६॥} \veg\dontdisplaylinenum }%
     \var{{\devanagarifontvar \numnoemph\vc\textbf{॰त्थापने}\lem \Ed, ॰त्पादने \msCa\msNa}}% 

{\devanagarifont जायते विग्रहे त्वेषा कृते ह्यमृतमन्थने \thinspace{\dandab} \dontdisplaylinenum }%
 
%Verse 23:17

{\devanagarifont रजस्तमश्चासुरं वै सत्त्वं देवकृतैः शुभैः {॥ २३:१७॥} \veg\dontdisplaylinenum }%
 
{\devanagarifont ततः सत्त्वमयी देवी रजस्तमनिवासिनी \thinspace{\dandab} \dontdisplaylinenum }%
     \var{{\devanagarifontvar \numemph\vab\textbf{सत्त्वमयी देवी रजस्तमसि वासिनी}\lem \msNa, 
सत्त्वमयी \uncl{दे}\lac\  मसि वासिनी \msCa, 
सत्त्वमयी देवी रजस्तमनिवासिनी \Ed}}% 

%Verse 23:18

{\devanagarifont क्रोधजा वै स्थिता मध्ये देवदानवपक्षयोः {॥ २३:१८॥} \veg\dontdisplaylinenum }%
 
{\devanagarifont तामद्भुतमयीं दृष्ट्वा विस्मिता देवदानवाः \thinspace{\dandab} \dontdisplaylinenum }%
     \var{{\devanagarifontvar \numemph\va\textbf{॰भुत॰}\lem \msCa\msNa, ॰भूत॰ \Ed}}% 

%Verse 23:19

{\devanagarifont तस्याः प्रभावाभिहता दुद्रुवस्ते दिशो दश {॥ २३:१९॥} \veg\dontdisplaylinenum }%
     \var{{\devanagarifontvar \numnoemph\vd\textbf{दश}\lem \msCa\msNa, दशः \Ed}}% 

{\devanagarifont तत्र पीताम्बरधरो विष्णुरेकस्तु तिष्ठति \thinspace{\dandab} \dontdisplaylinenum }%
     \var{{\devanagarifontvar \numemph\va\textbf{पीता॰}\lem \msCa\msNa, पिता॰ \Ed}}% 

%Verse 23:20

{\devanagarifont साभिगत्वा विशालाक्षी नारायणमथाब्रवीत् {॥ २३:२०॥} \veg\dontdisplaylinenum }%
     \var{{\devanagarifontvar \numnoemph\vc\textbf{साभि॰}\lem \msCa\msNa, सोभि॰ \Ed}}% 
    \var{{\devanagarifontvar \numnoemph\vd\textbf{॰ब्रवीत्}\lem \msNa\Ed, ॰\uncl{ब्र}\lac\  \msCa}}% 

{\devanagarifont देवदानवनाथस्त्वं त्वयि सर्वं प्रतिष्ठितम् \thinspace{\dandab} \dontdisplaylinenum }%
     \var{{\devanagarifontvar \numemph\va\textbf{देव॰}\lem \msNa\Ed, \lac\  \msCa}}% 
    \var{{\devanagarifontvar \numnoemph\vb\textbf{सर्वं}\lem \msCa\msNa, सर्व॰ \Ed}}% 

%Verse 23:21

{\devanagarifont देहि देव ममावासं यत्राहं निवसे सुखम् {॥ २३:२१॥} \veg\dontdisplaylinenum }%
 
{\devanagarifont ततो नारायणस्तुष्टस्तां देवीं प्रत्यभाषत \thinspace{\dandab} \dontdisplaylinenum }%
 
%Verse 23:22

{\devanagarifont शरीरे मम वस्तव्यं विष्णुरेनामथाब्रवीत् {॥ २३:२२॥} \veg\dontdisplaylinenum }%
     \var{{\devanagarifontvar \numemph\vc\textbf{वस्तव्यं}\lem \Ed, वास्तव्यम् \msCa\msNa}}% 

{\devanagarifont ततस्तां वैष्णवं तेजः पाप्मना समतिष्ठत \thinspace{\dandab} \dontdisplaylinenum }%
     \var{{\devanagarifontvar \numemph\va\textbf{वैष्णवं}\lem \msCa\msNa, विष्णुवत् \Ed}}% 

%Verse 23:23

{\devanagarifont ततः शेते स वैकुण्ठः पाप्मना त्वभिलङ्घितः {॥ २३:२३॥} \veg\dontdisplaylinenum }%
     \var{{\devanagarifontvar \numnoemph\vd\textbf{पाप्मना त्वभिलङ्घितः}\lem \msNa, 
पाप्म\lac  घितः \msCa, 
पाप्मना त्वभिलङ्घिताः \Ed}}% 

{\devanagarifont तस्मिन्शयाने वित्रस्ता देवासुरगणास्तथा \thinspace{\dandab} \dontdisplaylinenum }%
     \var{{\devanagarifontvar \numemph\va\textbf{तस्मिन्}\lem \msCa\Ed, तस्मि \msNa}}% 

%Verse 23:24

{\devanagarifont ऊचुस्ते परमोद्विग्नाः शयानं विष्णुमच्युतम् {॥ २३:२४॥} \veg\dontdisplaylinenum }%
 
{\devanagarifont त्रातारं नाभिगच्छाम उत्तिष्ठोत्तिष्ठ केशव \thinspace{\dandab} \dontdisplaylinenum }%
 
%Verse 23:25

{\devanagarifont ततः शङ्खगदापाणिरुत्तिष्ठत महाभुजः {॥ २३:२५॥} \veg\dontdisplaylinenum }%
 
{\devanagarifont उत्थितश्च विशालाक्षः पाप्मना तस्य पृष्ठतः \thinspace{\dandab} \dontdisplaylinenum }%
     \var{{\devanagarifontvar \numemph\va\textbf{उत्थित॰}\lem \msCa\msNa, उत्तिष्ठ॰ \Ed\oo 
\textbf{विशालाक्षः}\lem \msCa\msNa, विशालाक्षिः \Ed}}% 

%Verse 23:26

{\devanagarifont ततः सा विग्रहवती स्थिता नारायणालये {॥ २३:२६॥} \veg\dontdisplaylinenum }%
     \var{{\devanagarifontvar \numnoemph\vc\textbf{ततः सा विग्रहवती}\lem \msNa\Ed, 
तत\lac  \uncl{व}ती \msCa}}% 

{\devanagarifont विष्णुर्देवासुरगणानिदं वचनमब्रवीत् \thinspace{\dandab} \dontdisplaylinenum }%
     \var{{\devanagarifontvar \numemph\va\textbf{विष्णुर्}\lem \msCa\Ed, विष्णु \msNa\oo 
\textbf{॰गणान्}\lem \msCa\Ed, ॰गणा \msNa}}% 

%Verse 23:27

{\devanagarifont अस्माकं वै शरीरेषु इयं पाप्मा विनिःसृता {॥ २३:२७॥} \veg\dontdisplaylinenum }%
     \var{{\devanagarifontvar \numnoemph\vd\textbf{विनिःसृता}\lem \eme, विनिसृता \msCa\msNa\Ed\ \unmetr}}% 

{\devanagarifont एषाभिसत्त्वारसता सत्येन भगिनी मम \thinspace{\dandab} \dontdisplaylinenum }%
     \var{{\devanagarifontvar \numemph\va\textbf{एषाभिसत्त्वारसता}\lem \msCa, 
एषातिसत्वानसता \msNa, 
एषातिसत्त्वामसती \Ed}}% 

%Verse 23:28

{\devanagarifont विश्रुतां त्रिषु लोकेषु तां पूजयथ मां यथा {॥ २३:२८॥} \veg\dontdisplaylinenum }%
     \var{{\devanagarifontvar \numnoemph\vc\textbf{॰श्रुतां}\lem \msCa, ॰श्रुता \msNa, ॰श्रुतो \Ed}}% 

{\devanagarifont ततो देवासुरगणाः सप्तलोकाः समानुषाः \thinspace{\dandab} \dontdisplaylinenum }%
     \var{{\devanagarifontvar \numemph\vb\textbf{॰लोकाः समानुषाः}\lem \msNa\Ed, ॰\uncl{लो}\lac  नुषाः \msCa}}% 

%Verse 23:29

{\devanagarifont विभक्ता वैष्णवी पाप्मा तेषु सर्वेषु देवता {॥ २३:२९॥} \veg\dontdisplaylinenum }%
 
{\devanagarifont पर्वतेष्वथ वृक्षेषु सागरेषु सरित्सु च \thinspace{\dandab} \dontdisplaylinenum }%
 
%Verse 23:30

{\devanagarifont ततो निद्रावशगतं जगत्स्थावरजङ्गमम् {॥ २३:३०॥} \veg\dontdisplaylinenum }%
 
{\devanagarifont एषोत्पत्तिश्च निद्राया यथा वसति तच्छृणु \thinspace{\dandab} \dontdisplaylinenum }%
 
%Verse 23:31

{\devanagarifont त्रीणि स्थानानि यस्या वै शरीरेषु शरीरिणाम् {॥ २३:३१॥} \veg\dontdisplaylinenum }%
 
{\devanagarifont श्लेष्मपित्तानिलस्थाने त्रीणि पक्षाणि वासिनः \thinspace{\dandab} \dontdisplaylinenum }%
     \var{{\devanagarifontvar \numemph\vab\textbf{॰निलस्थाने त्रीणि}\lem \Ed, 
नि\lac  णि \msCa, 
॰निलस्थान त्रीणि \msNa}}% 
    \var{{\devanagarifontvar \numnoemph\vb\textbf{पक्षाणि}\lem \msCa, पक्षा नि॰ \msNa\Ed}}% 

%Verse 23:32

{\devanagarifont तमः श्लेष्माश्रया निद्रा रजोनिद्रा तु वातिका {॥ २३:३२॥} \veg\dontdisplaylinenum }%
     \var{{\devanagarifontvar \numnoemph\vc\textbf{तमः}\lem \msCa\msNa, तम॰ \Ed}}% 
    \var{{\devanagarifontvar \numnoemph\vd\textbf{निद्रा तु}\lem \msCa\msNa, निद्राति॰ \Ed}}% 

{\devanagarifont पित्ताश्रयां स्मृतां निद्रां सात्त्विकां विद्धि भूपते \thinspace{\dandab} \dontdisplaylinenum }%
     \var{{\devanagarifontvar \numemph\va\textbf{स्मृतां}\lem \msCa\Ed, स्मृता \msNa}}% 

%Verse 23:33

{\devanagarifont आदित्यप्रभवं तेजस्तस्मिन्सत्त्वं प्रतिष्ठति {॥ २३:३३॥} \veg\dontdisplaylinenum }%
     \var{{\devanagarifontvar \numnoemph\vd\textbf{सत्त्वं प्रतिष्ठति}\lem \msCa\msNa, सर्व प्रतिष्ठितं \Ed}}% 

{\devanagarifont निद्रा दिवा न भवति तस्मात्सत्त्वगुणात्मिका \thinspace{\dandab} \dontdisplaylinenum }%
 
%Verse 23:34

{\devanagarifont यस्मात्सोमोद्भवा निद्रा तमांसि च रजांसि च {॥ २३:३४॥} \veg\dontdisplaylinenum }%
     \var{{\devanagarifontvar \numemph\vc\textbf{यस्मा॰}\lem \msCa\msNa, तस्मा॰ \Ed}}% 
    \var{{\devanagarifontvar \numnoemph\vd\textbf{तमांसि च रजांसि च}\lem \msNa\Ed, 
त\uncl{मां}सि च \uncl{र}\lac\  \msCa}}% 

{\devanagarifont तस्माद्रात्रौ भवेन्निद्रा तामसी हरजात्मिका \thinspace{\dandab} \dontdisplaylinenum }%
     \var{{\devanagarifontvar \numemph\va\textbf{भवेन्}\lem \msCa\Ed, भवन् \msNa}}% 

%Verse 23:35

{\devanagarifont यदा हि सर्वाङ्गगतौ श्रोतांसि प्रतिपद्यते {॥ २३:३५॥} \veg\dontdisplaylinenum }%
     \var{{\devanagarifontvar \numnoemph\vc\textbf{सर्वा॰}\lem \msNa\Ed, सत्वा॰ \msCa}}% 

{\devanagarifont रजस्तमश्च नियतस्तदा निद्रा प्रवर्तते \thinspace{\dandab} \dontdisplaylinenum }%
     \var{{\devanagarifontvar \numemph\va\textbf{नियतस्}\lem \msCa\msNa, नियतंस् \Ed}}% 

%Verse 23:36

{\devanagarifont तमस्यूर्ध्वगतश्रोतो ह्यक्षिपक्ष्मासमाश्रिता {॥ २३:३६॥} \veg\dontdisplaylinenum }%
     \var{{\devanagarifontvar \numnoemph\vc\textbf{॰गतश्रोतो}\lem \msCa\msNa, ॰गते श्रोत्रो \Ed}}% 
    \var{{\devanagarifontvar \numnoemph\vd\textbf{ह्यक्षि॰}\lem \msCa\msNa, ह्याक्षि॰ \Ed}}% 

{\devanagarifont तमः प्रवर्तते जन्तोस्ततस्त्वक्ष्नोर्निमीलनम् \thinspace{\dandab} \dontdisplaylinenum }%
     \var{{\devanagarifontvar \numemph\vab\textbf{जन्तोस्तत॰}\lem \msCa\msNa, जन्तो तम॰ \Ed}}% 
    \var{{\devanagarifontvar \numnoemph\vb\textbf{त्वक्ष्णोर्नि॰}\lem \msCa, त्वक्ष्णो नि॰ \msNa\Ed}}% 

%Verse 23:37

{\devanagarifont नासाक्षिकर्णश्रोतांसि प्रयुज्यन्ते कफेन तु {॥ २३:३७॥} \veg\dontdisplaylinenum }%
     \var{{\devanagarifontvar \numnoemph\vc\textbf{॰श्रोतांसि}\lem \msNa\Ed, श्रो\uncl{ता}\lac\  \msCa}}% 
    \var{{\devanagarifontvar \numnoemph\vd\textbf{प्रयुज्यन्ते कफेन}\lem \msNa\Ed, \lac  फेन \msCa}}% 

{\devanagarifont हृदयं मुह्यते चापि तमसा चावृतं मनः \thinspace{\dandab} \dontdisplaylinenum }%
 
%Verse 23:38

{\devanagarifont स्पर्शं न वेदयत्येव न शृणोति न पश्यति {॥ २३:३८॥} \veg\dontdisplaylinenum }%
 
{\devanagarifont नोच्छ्वासयति नासाभ्यां विवृताक्षिमुखो नरः \thinspace{\dandab} \dontdisplaylinenum }%
     \var{{\devanagarifontvar \numemph\vb\textbf{॰मुखो नरः}\lem \msCa\msNa, ॰मुखेन च \Ed}}% 

%Verse 23:39

{\devanagarifont एषा नृणामन्तकरी निद्रा वै तामसी स्मृता {॥ २३:३९॥} \veg\dontdisplaylinenum }%
     \var{{\devanagarifontvar \numnoemph\vc\textbf{॰न्तकरी}\lem \msNa\Ed, ॰नकरी \msCa}}% 

{\devanagarifont अकर्मण्यप्रवृत्तिश्च मृतवत्स्वपते क्षितौ \thinspace{\dandab} \dontdisplaylinenum }%
 
{\devanagarifont निद्रोत्पत्तिं विकारं च कथितो ऽस्मि नराधिप  \danda\dontdisplaylinenum }%
     \var{{\devanagarifontvar \numemph\va\textbf{॰त्पत्तिं विकारं च}\lem \msNa, 
॰त्प\uncl{त्ति}\lac\  \msCa, 
॰त्पत्तिं विकारश्च \Ed}}% 

%Verse 23:40

{\devanagarifont तस्मान्निद्रां न सेवेत तमोमोहप्रवर्धनीम् {॥ २३:४०॥} \veg\dontdisplaylinenum }%
     \var{{\devanagarifontvar \numnoemph\vd\textbf{॰वर्धनीम्}\lem \msCa, ॰वर्धनी \msNa\Ed}}% 

{\devanagarifont 
\jump
\begin{center}
\ketdanda~इति वृषसारसंग्रहे निद्रोत्पत्तिस्त्रयोविंशतिमो ऽध्यायः~\ketdanda
\end{center}
\dontdisplaylinenum\vers  }%
     \var{{\devanagarifontvar \numnoemph{\englishfont \Colo:}\textbf{॰विंशतिमो}\lem \msCa\msNa, ॰विंशतितमो \Ed}}% 
\bekveg\szamveg
\vfill
\phpspagebreak

\versno=0\fejno=24
\thispagestyle{empty}

\centerline{\Large\devanagarifontbold [   चतुर्विंशतिमो ऽध्यायः  ]}{\vrule depth10pt width0pt} \fancyhead[CE]{{\footnotesize\devanagarifont वृषसारसंग्रहे  }}
\fancyhead[CO]{{\footnotesize\devanagarifont चतुर्विंशतिमो ऽध्यायः  }}
\fancyhead[LE]{}
\fancyhead[RE]{}
\fancyhead[LO]{}
\fancyhead[RO]{}
\szam\bek


\vers


{\devanagarifont जनमेजय उवाच {\dandab}\dontdisplaylinenum  }%
 
{\devanagarifont देवानां दानवानां च वैषम्यानि श्रुतानि मे \thinspace{\danda} \dontdisplaylinenum }%
     \var{{\devanagarifontvar \numemph\vb\textbf{वैषम्यानि}\lem \eme,  वैशम्यानि \msCa\msCb\Ed\oo 
\textbf{मे}\lem \msCa\msCb, वै \Ed}}% 

%Verse 24:1

{\devanagarifont निद्रासम्भवमाश्चर्यं त्वत्प्रसादेन वेदितम् {॥ २४:१॥} \veg\dontdisplaylinenum }%
     \var{{\devanagarifontvar \numnoemph\vd\textbf{त्वत्प्रसादेन वेदितम्}\lem \msCb\Ed, त्वत्प्र\lac  तम् \msCa}}% 

{\devanagarifont त्रैलोक्यविस्तरायामं श्रोतुमिच्छामि भो द्विज \thinspace{\dandab} \dontdisplaylinenum }%
     \var{{\devanagarifontvar \numemph\va\textbf{॰लोक्य॰}\lem \msCa\msCb, ॰लोक्या॰ \Ed}}% 
    \var{{\devanagarifontvar \numnoemph\vb\textbf{भो}\lem \msCa\msCb, वै \Ed}}% 

%Verse 24:2

{\devanagarifont कस्मिंश्चिन्नरकं ज्ञेयं पातालं च द्विजोत्तम {॥ २४:२॥} \veg\dontdisplaylinenum }%
     \var{{\devanagarifontvar \numnoemph\vc\textbf{कस्मिंश्चिन्नरकं}\lem \eme, कस्मिंश्चिन्नरके \msCa\msCb, कस्मिश्चिन्नरकं \Ed}}% 

{\devanagarifont सप्तद्वीपं समिच्छामि सप्तसागरमेव च \thinspace{\dandab} \dontdisplaylinenum }%
 
%Verse 24:3

{\devanagarifont मेरुमूर्धं च विप्रेन्द्र देवालयं निबोध माम् {॥ २४:३॥} \veg\dontdisplaylinenum }%
     \var{{\devanagarifontvar \numemph\vc\textbf{॰मूर्धं}\lem \msCa\msCb, ॰मूर्धश् \Ed}}% 
    \var{{\devanagarifontvar \numnoemph\vd\textbf{देवालयं}\lem \corr, देवालय \msCa\msCb\Ed}}% 


\alalfejezet{त्रैलोक्यं नरकाणि च}
{\devanagarifont वैशम्पायन उवाच {\dandab}\dontdisplaylinenum  }%
 
{\devanagarifont शृणु संक्षेपतो राजन्त्रैलोक्यायामविस्तरम् \thinspace{\danda} \dontdisplaylinenum }%
     \var{{\devanagarifontvar \numemph\vb\textbf{॰विस्तरम्}\lem \msCb\Ed, \lac\  \msCa}}% 

%Verse 24:4

{\devanagarifont कालाग्निः प्रथमो ज्ञेयः सर्वाधस्तान्नरेश्वर {॥ २४:४॥} \veg\dontdisplaylinenum }%
 
{\devanagarifont तस्योपरि नृपश्रेष्ठ ज्ञेया नरककोटयः \thinspace{\dandab} \dontdisplaylinenum }%
     \var{{\devanagarifontvar \numemph\va\textbf{नृप॰}\lem \msCa\Ed, नृ॰ \msCb}}% 

%Verse 24:5

{\devanagarifont रौरवादि अवीच्यन्तं यातनास्थानमुच्यते {॥ २४:५॥} \veg\dontdisplaylinenum }%
 

\alalfejezet{सप्त पातालाः}
{\devanagarifont उपरिष्टात्तु विज्ञेयाः पातालाः सप्त एव तु \thinspace{\dandab} \dontdisplaylinenum }%
     \paral{{\devanagarifontsmall {\englishfont Niśv Kārikā 149:} उपरिष्टात् तु देवेशि पातालास्सप्त एव तु }}

%Verse 24:6

{\devanagarifont आभासतालः प्रथमः स्वतालश्च ततः परम् {॥ २४:६॥} \veg\dontdisplaylinenum }%
     \var{{\devanagarifontvar \numemph\vd\textbf{स्वतालश्च}\lem \Ed, स्वलालञ्च \msCaacorr, स्वतालञ्च \msCapcorr, 
सुतालञ्च \msCb}}% 

{\devanagarifont शीतलश्च गभस्तिश्च शर्करश्च शिलातलम् \thinspace{\dandab} \dontdisplaylinenum }%
     \var{{\devanagarifontvar \numemph\va\textbf{शीतलश्च}\lem \msCa\Ed, श्रीतलश्च \msCb}}% 
    \var{{\devanagarifontvar \numnoemph\vb\textbf{शर्करश्च शिलातलम्}\lem \eme, \lac  लातलम् \msCa, 
शिलातलम् \msCb, 
शर्करश्च शिलावृतम् \Ed}}% 

%Verse 24:7

{\devanagarifont सप्तमं तु महातालं शेषनागकृतालयः {॥ २४:७॥} \veg\dontdisplaylinenum }%
     \var{{\devanagarifontvar \numnoemph\vc\textbf{सप्तमं}\lem \msCa\msCb, सप्तमस् \Ed}}% 
    \var{{\devanagarifontvar \numnoemph\vd\textbf{॰लयः}\lem \msCa\Ed, ॰लयम् \msCb}}% 

{\devanagarifont बलिश्च दैत्यराजेन्द्रो राक्षसश्च विशंखणः \thinspace{\dandab} \dontdisplaylinenum }%
     \var{{\devanagarifontvar \numemph\vb\textbf{विशंखणः}\lem \Ed, विसंशनः \msCa, विसंशयः \msCb}}% 

%Verse 24:8

{\devanagarifont इत्येवमादयः सर्वे नागदानवराक्षसाः {॥ २४:८॥} \veg\dontdisplaylinenum }%
 

\alalfejezet{सप्त द्वीपाः प्रियव्रतसुताश्च}
{\devanagarifont सप्त द्वीपास्ततो ज्ञेयाः सप्तसागरसंवृताः \thinspace{\dandab} \dontdisplaylinenum }%
 
%Verse 24:9

{\devanagarifont प्रियव्रतस्य पुत्रो ऽभूद्दश राजपराक्रमः {॥ २४:९॥} \veg\dontdisplaylinenum  }%
     \var{{\devanagarifontvar \numemph\vo\textbf{(सप्त{\englishfont ...}॰पराक्रमः)}\lem \msCa\msCb, \om\ \Ed}}% 
    \paral{{\devanagarifontsmall {\englishfont For a similar enumeration of Priyavrata's ten sons and the seven islands,
                 see, e.g., Vāyupurāṇa 33.1 ff.} }}

{\devanagarifont अग्नीध्रश्चाग्निबाहुश्च मेधा मेधातिथिर्वसुः \thinspace{\dandab} \dontdisplaylinenum }%
     \var{{\devanagarifontvar \numemph\vab\textbf{अग्नीध्रश्चाग्निबाहुश्च मेधा मेधातिथिर्वसुः}\lem \corr, 
अग्नीन्ध्रश्चाग्निबाहुश्च मेधा मेधातिथिर्वसुः \msCb, 
अग्निन्व्रश्चाग्निवा\lk \lac  धातिथिर्व्वसुः \msCa, \om\ \Ed}}% 

%Verse 24:10

{\devanagarifont ज्योतिष्मान्द्युतिमान्हव्यः सवनः पत्र एव च {॥ २४:१०॥} \veg\dontdisplaylinenum }%
     \var{{\devanagarifontvar \numnoemph\vcd\textbf{हव्यः सवनः पत्र एव च}\lem \msCa\Ed, 
\uncl{हव्यः सवनः पत्र एव च} \msCb}}% 
    \paral{{\devanagarifontsmall \vo {\englishfont \similar\ Liṅgapurāṇa 1.46.17:}
                 आग्नीध्रश्चाग्निबाहुश्च मेधा मेधातिथिर्वसुः\thinspace{\devanagarifontsmall ।}
                 ज्योतिष्मान्द्युतिमान् हव्यः सवनः पुत्र एव च\thinspace{\devanagarifontsmall ॥}
                 {\englishfont \similar\ Brahmapurāṇa 5.9:}
                 आग्नीध्रश् चाग्निबाहुश् च मेध्यो मेधातिथिर् वसुः\thinspace{\devanagarifontsmall ।}
                 ज्योतिष्मान् द्युतिमान् हव्यः सवलः पुत्रसंज्ञकः\thinspace{\devanagarifontsmall ॥}
                 {\englishfont \similar\ Brahmāṇḍapurāṇa 1.13.104 and 1.14.9 
                 \similar\ \PADMAP\ 1.7.83 etc.} }}

{\devanagarifont अग्निबाहुश्च मेधा च पत्रश्चैव त्रयो जनाः \thinspace{\dandab} \dontdisplaylinenum }%
     \var{{\devanagarifontvar \numemph\va\textbf{मेधा च}\lem \msCa\Ed, मेधाश्च \msCb}}% 

%Verse 24:11

{\devanagarifont संसारभयभीतेन मोक्षमार्गसमाश्रिताः {॥ २४:११॥} \veg\dontdisplaylinenum }%
     \var{{\devanagarifontvar \numnoemph\vd\textbf{॰मार्ग॰}\lem \msCb\Ed, ॰मार्गं \msCa}}% 

{\devanagarifont अग्नीध्रं प्रथमद्वीपे अभ्यषिञ्चत्प्रियव्रतः \thinspace{\dandab} \dontdisplaylinenum }%
     \var{{\devanagarifontvar \numemph\va\textbf{अग्नीध्रं}\lem \eme, अग्निन्ध्रं \msCa, अग्नीन्ध्र \msCb, अग्निन्धं \Ed\oo 
\textbf{प्रथम॰}\lem \Ed, प्रथमं \msCa\msCb}}% 
    \var{{\devanagarifontvar \numnoemph\vb\textbf{अभ्यषिञ्चत्}\lem \msCa\msCb, अभ्यषिञ्चत \Ed}}% 

%Verse 24:12

{\devanagarifont प्लक्षद्वीपेश्वरं चक्रे नाम्ना मेधातिथिं तथा {॥ २४:१२॥} \veg\dontdisplaylinenum }%
     \var{{\devanagarifontvar \numnoemph\vd\textbf{मेधातिथिं तथा}\lem \msCa\Ed, मेधातिथितन्तथा \msCb}}% 

{\devanagarifont वसुश्च शाल्मलीद्वीपे अभिषिक्तो महीपतिः \thinspace{\dandab} \dontdisplaylinenum }%
     \var{{\devanagarifontvar \numemph\va\textbf{वसुश्च शाल्मली॰}\lem \msCb, \lac\  \msCa, वसुञ्च शाल्मली \Ed}}% 

%Verse 24:13

{\devanagarifont ज्योतिष्मन्तं कुशद्वीपे राजानमभिषेचयेत् {॥ २४:१३॥} \veg\dontdisplaylinenum }%
 
{\devanagarifont क्रौञ्चद्वीपेश्वरं चक्रे द्युतिमन्तं नरेश्वर \thinspace{\dandab} \dontdisplaylinenum }%
     \var{{\devanagarifontvar \numemph\vb\textbf{द्युतिमन्तं नरेश्वर}\lem \msCa, द्युतिमन्तन्नरेश्वरम् \msCbpcorr, 
श्वरञ्चक्रे द्युतिमन्तन्नरेश्वरम् \msCbacorr, 
द्युतिमन्तं नरेश्वरः \Ed}}% 

%Verse 24:14

{\devanagarifont शाकद्वीपेश्वरं हव्यं पुष्करे सवनः स्मृतः {॥ २४:१४॥} \veg\dontdisplaylinenum }%
     \var{{\devanagarifontvar \numnoemph\vd\textbf{सवनः}\lem \msCa\msCb, सवन \Ed}}% 

{\devanagarifont मध्ये पुष्करद्वीपस्य पर्वतो मानसोत्तरः \thinspace{\dandab} \dontdisplaylinenum }%
 
%Verse 24:15

{\devanagarifont लोकपालाः स्थितास्तत्र चतुर्भिश्चतुरो दिशः {॥ २४:१५॥} \veg\dontdisplaylinenum }%
     \var{{\devanagarifontvar \numemph\vd\textbf{चतुरो दिशः}\lem \msCb\Ed, \lac\  \msCa}}% 

{\devanagarifont महावीतः स्मृतो वर्षो धातकी च नराधिप \thinspace{\dandab} \dontdisplaylinenum }%
     \var{{\devanagarifontvar \numemph\vo\textbf{(महावीतः{\englishfont ...}स्मृतः)}\lem \msCa\msCb, \om\ \Ed}}% 
    \var{{\devanagarifontvar \numnoemph\va\textbf{महावीतः}\lem \msCa\msCb, महानीतः \Ed\oo 
\textbf{स्मृतो}\lem \msCa\Ed, स्मृता \msCb}}% 

%Verse 24:16

{\devanagarifont तस्य बाह्यः समुद्रो ऽभूत्स्वादूदक इति स्मृतः {॥ २४:१६॥} \veg\dontdisplaylinenum }%
     \var{{\devanagarifontvar \numnoemph\vc\textbf{बाह्यः}\lem \msCa\Ed, बाह्य \msCb}}% 
    \var{{\devanagarifontvar \numnoemph\vd\textbf{॰दूदक}\lem \msCa\Ed, ॰दूक \msCb}}% 

{\devanagarifont चतुःषष्टि स्मृतो लक्षो योजनानां नराधिप \thinspace{\dandab} \dontdisplaylinenum }%
     \var{{\devanagarifontvar \numemph\va\textbf{चतुः॰}\lem \msCb, चतु॰ \msCa\oo 
\textbf{लक्षो}\lem \msCa, लक्षा \msCb, \om\ \Ed}}% 
    \var{{\devanagarifontvar \numnoemph\vb\textbf{नराधिप}\lem \msCa, नराधिपः \msCb, \om\ \Ed}}% 

%Verse 24:17

{\devanagarifont पुष्करद्वीपमन्तश्च क्षीरोदो नाम सागरः {॥ २४:१७॥} \veg\dontdisplaylinenum }%
 
{\devanagarifont द्वात्रिंशल्लक्षविस्तारः शाकद्वीपबहिर्वृतः \thinspace{\dandab} \dontdisplaylinenum }%
     \var{{\devanagarifontvar \numemph\va\textbf{॰विस्तारः}\lem \msCa\msCb, ॰विस्तारैः \Ed}}% 
    \var{{\devanagarifontvar \numnoemph\vb\textbf{॰बहिर्वृतः}\lem \conj, ॰वहवृणः \msCa, ॰बहुवृतः \msCb, ॰वहवृणे \Ed}}% 

%Verse 24:18

{\devanagarifont जलदश्च कुमारश्च सुकुमारमणीचकः {॥ २४:१८॥} \veg\dontdisplaylinenum }%
     \var{{\devanagarifontvar \numnoemph\vcd\textbf{कुमारश्च सुकुमारमणीचकः}\lem \msCb\Ed, 
कुमा\lk\lac \lk\lk णीचकः \msCa}}% 

{\devanagarifont कुसुमोत्तरमोदश्च सप्तमं च महाद्रुमम् \thinspace{\dandab} \dontdisplaylinenum }%
     \var{{\devanagarifontvar \numemph\vb\textbf{सप्तमं}\lem \msCa\msCb, सप्तमश् \Ed}}% 

%Verse 24:19

{\devanagarifont हव्यपुत्राः स्मृताः सप्त वर्षनाम तथा स्मृतः {॥ २४:१९॥} \veg\dontdisplaylinenum }%
 
{\devanagarifont द्वीपान्तं दधिमण्डोदक्षीरोदार्धं विनिर्दिशेत् \thinspace{\dandab} \dontdisplaylinenum }%
     \var{{\devanagarifontvar \numemph\va\textbf{॰मण्डोद॰}\lem \msCb, ॰मण्डादि॰ \msCa\Ed}}% 
    \var{{\devanagarifontvar \numnoemph\vb\textbf{विनिर्दिशेत्}\lem \msCa\msCb, निर्दिशेत् \Ed}}% 

%Verse 24:20

{\devanagarifont क्रौञ्चद्वीपसमुद्रान्ते सप्त वर्षास्तु ते स्मृताः {॥ २४:२०॥} \veg\dontdisplaylinenum }%
     \var{{\devanagarifontvar \numnoemph\vc\textbf{॰द्वीप॰}\lem \msCa\msCb, ॰द्वीपे \Ed}}% 
    \var{{\devanagarifontvar \numnoemph\vd\textbf{वर्षास्}\lem \msCa\msCb, वर्षन् \Ed}}% 

{\devanagarifont कुशलो मनोनुगश्चोष्णः यावनश्चान्धकारकः \thinspace{\dandab} \dontdisplaylinenum }%
     \var{{\devanagarifontvar \numemph\va\textbf{कुशलो मनोनुगश्चोष्णः}\lem \msCb, 
कुशलो मनोनुगश्चो\uncl{ष्णः} \msCa, 
कुशलोम्नोनुगश्चोष्णः \Ed}}% 
    \var{{\devanagarifontvar \numnoemph\vb\textbf{यावनश्चान्धकारकः}\lem \msCb, 
\uncl{या}वन\uncl{श्चा}\lk \lac\  \msCa, 
यवनश्चान्धकारकः \Ed}}% 

%Verse 24:21

{\devanagarifont मुनिश्च दुन्दुभिश्चैव सुता द्युतिमतस्तु वै {॥ २४:२१॥} \veg\dontdisplaylinenum }%
     \var{{\devanagarifontvar \numnoemph\vd\textbf{सुता द्युतिमतस्}\lem \msCa\msCb, सुतद्युतिमनस् \Ed}}% 
    \paral{{\devanagarifontsmall \vcd {\englishfont = Liṅgapurāṇa 1.46.31ab} }}

{\devanagarifont दध्यर्धे घृतमण्डोदः कुशद्वीपसमावृतः \thinspace{\dandab} \dontdisplaylinenum }%
     \var{{\devanagarifontvar \numemph\va\textbf{घृत॰}\lem \msCa\msCb, धृत॰ \Ed}}% 
    \var{{\devanagarifontvar \numnoemph\vb\textbf{॰द्वीप॰}\lem \msCa\msCb, ॰द्वीपः \Ed}}% 

%Verse 24:22

{\devanagarifont तत्रापि सप्तवर्षे च नामतः शृणु भारत {॥ २४:२२॥} \veg\dontdisplaylinenum }%
     \var{{\devanagarifontvar \numnoemph\vc\textbf{॰वर्षे}\lem \msCa\msCb, ॰वर्षं \Ed}}% 
    \var{{\devanagarifontvar \numnoemph\vd\textbf{भारत}\lem \msCa\Ed, भारतः \msCb}}% 

{\devanagarifont उद्भिमान्वेणुमांश्चैव स्वैरन्नालम्बनो धृतिः \thinspace{\dandab} \dontdisplaylinenum }%
     \var{{\devanagarifontvar \numemph\va\textbf{वेणुमांश्चैव}\lem \msCa, वेणुमां व \msCb, धेनुसाश्चैव \Ed}}% 
    \var{{\devanagarifontvar \numnoemph\vb\textbf{स्वैर॰}\lem \msCa, स्वैरा॰ \Ed}}% 

%Verse 24:23

{\devanagarifont षष्ठः प्रभाकरश्चैव कपिलः सप्तमः स्मृतः {॥ २४:२३॥} \veg\dontdisplaylinenum }%
     \paral{{\devanagarifontsmall {\englishfont Cf. Brahmapurāṇa 20.36--37ab:}
                  ज्योतिष्मतः कुशद्वीपे शृणुध्वं तस्य पुत्रकान्\thinspace{\devanagarifontsmall ।}
                  उद्भिदो वेणुमांश्चैव स्वैरथो रन्धनो धृतिः\thinspace{\devanagarifontsmall ॥}
                  प्रभाकरो ऽथ कपिलस्तन्नाम्ना वर्षपद्धतिः\thinspace{\devanagarifontsmall ।} }}

{\devanagarifont घृतमण्डस्तदर्धेन तस्यान्ते मदिरोदधिः \thinspace{\dandab} \dontdisplaylinenum }%
     \var{{\devanagarifontvar \numemph\va\textbf{॰मण्डस्तदर्धेन}\lem \msCb, मण्डोतदर्धेन \msCa, मण्डोतर्धेन \Ed}}% 
    \var{{\devanagarifontvar \numnoemph\vb\textbf{तस्यान्ते मदिरो॰}\lem \Ed, 
\uncl{त}\lac  दिरो॰ \msCa, तस्यान्तेमधिरो॰ \msCb}}% 

%Verse 24:24

{\devanagarifont समन्ताच्छाल्मलीद्वीपो वर्षाः सप्तैव कीर्तिताः {॥ २४:२४॥} \veg\dontdisplaylinenum }%
     \var{{\devanagarifontvar \numnoemph\vd\textbf{वर्षाः}\lem \msCb\Ed, वर्षोः \msCa}}% 

{\devanagarifont श्वेतश्च हरितश्चैव जीमूतो रोहितस्तथा \thinspace{\dandab} \dontdisplaylinenum }%
     \var{{\devanagarifontvar \numemph\vb\textbf{रोहित॰}\lem \msCa\msCb, लोहित॰ \Ed}}% 
    \paral{{\devanagarifontsmall \vab {\englishfont = Liṅgapurāṇa 1.46.38cd} }}

%Verse 24:25

{\devanagarifont वैद्युतो मानसश्चैव सुप्रभः सप्तमः स्मृतः {॥ २४:२५॥} \veg\dontdisplaylinenum }%
     \paral{{\devanagarifontsmall \vcd {\englishfont \similar\ Liṅgapurāṇa 1.46.39ab:}
                         वैद्युतो मानसश्चैव सुप्रभः सप्तमस्तथा }}

{\devanagarifont मदिरोदधितो ऽर्धेन ज्ञेयस्त्विक्षुरसोदधिः \thinspace{\dandab} \dontdisplaylinenum }%
     \var{{\devanagarifontvar \numemph\va\textbf{॰दधितो}\lem \msCa\msCb, ॰दधिनो \Ed}}% 
    \var{{\devanagarifontvar \numnoemph\vb\textbf{ज्ञेयस्त्वि॰}\lem \msCa\msCb, ज्ञेय त्वि॰ \Ed}}% 

%Verse 24:26

{\devanagarifont प्लक्षद्वीपो वृतस्तेन सप्तवर्षसमन्वितः {॥ २४:२६॥} \veg\dontdisplaylinenum }%
 
{\devanagarifont शान्तश्च शिशिरश्चैव सुखदानन्द एव च \thinspace{\dandab} \dontdisplaylinenum }%
     \var{{\devanagarifontvar \numemph\va\textbf{शान्तश्च शिशिरश्}\lem \msCb\Ed, \lac  रश् \msCa}}% 

%Verse 24:27

{\devanagarifont शिवक्षेमो ध्रुवश्चैव सप्त मेधातिथेः सुताः {॥ २४:२७॥} \veg\dontdisplaylinenum }%
     \var{{\devanagarifontvar \numnoemph\vc\textbf{शिव॰}\lem \msCa, शिवशिव॰ \Ed}}% 

{\devanagarifont लवणोदस्तु तस्यान्ते जम्बूद्वीपसमावृतः \thinspace{\dandab} \dontdisplaylinenum }%
     \var{{\devanagarifontvar \numemph\va\textbf{॰दस्तु तस्यान्ते}\lem \msCa\msCb, ॰दधिस्यान्ते \Ed}}% 
    \var{{\devanagarifontvar \numnoemph\vb\textbf{जम्बू॰}\lem \msCa\Ed, ज\uncl{म्बु}॰ \msCb\oo 
\textbf{॰द्वीप॰}\lem \msCa\msCb, ॰द्वीपा॰ \Ed\oo 
\textbf{॰वृतः}\lem \msCa\Ed, ॰वृताः \msCb}}% 

%Verse 24:28

{\devanagarifont लक्षयोजनविस्तार उपद्वीपसमन्वितः {॥ २४:२८॥} \veg\dontdisplaylinenum }%
     \var{{\devanagarifontvar \numnoemph\vc\textbf{॰विस्तार}\lem \msCa\msCb, ॰विस्तारो \Ed}}% 
    \var{{\devanagarifontvar \numnoemph\vd\textbf{॰द्वीप॰}\lem \msCa\msCb, ॰द्विप॰ \Ed}}% 

{\devanagarifont अङ्गद्वीपो यवद्वीपो मलयद्वीप एव च \thinspace{\dandab} \dontdisplaylinenum }%
     \var{{\devanagarifontvar \numemph\vo\textbf{(अङ्गद्वीपो{\englishfont ...}एव च)}\lem \msCa\Ed, \om\ \eyeskip{\englishfont तो २४.३२च्द् } \msCb}}% 

%Verse 24:29

{\devanagarifont शङ्खद्वीपकमुद्वीपो वराहद्वीप एव च {॥ २४:२९॥} \veg\dontdisplaylinenum }%
     \var{{\devanagarifontvar \numnoemph\vd\textbf{एव च}\lem \Ed, \lac\  \msCa, \om\ \msCb}}% 

{\devanagarifont सिंह बर्हिणद्वीपं च पद्मश्चक्रस्तथैव च \thinspace{\dandab} \dontdisplaylinenum }%
     \var{{\devanagarifontvar \numemph\vo\textbf{(सिंह{\englishfont ...}तथा)}\lem \msCa\Ed, \om\ \msCb}}% 
    \var{{\devanagarifontvar \numnoemph\va\textbf{सिंह बर्हिण॰}\lem \Ed, \lac  र्हिण॰ \msCa, \om\ \msCb}}% 
    \var{{\devanagarifontvar \numnoemph\vb\textbf{पद्मश्चक्र}\lem \msCa, \om\ \msCb, पद्मचक्र॰ \Ed}}% 

%Verse 24:30

{\devanagarifont वज्ररत्नाकरद्वीपो हंसकः कुमुदस्तथा {॥ २४:३०॥} \veg\dontdisplaylinenum }%
 
{\devanagarifont लाङ्गलो वृषद्वीपश्च द्वीपो भद्राकरस्तथा \thinspace{\dandab} \dontdisplaylinenum }%
     \var{{\devanagarifontvar \numemph\vo\textbf{(लाङ्गलो{\englishfont ...}कीर्तितम)}\lem \msCa\Ed, \om\ \msCb}}% 

{\devanagarifont चन्द्रद्वीपश्च सिन्धुश्च चन्दनद्वीप एव च  \danda\dontdisplaylinenum }%
     \var{{\devanagarifontvar \numnoemph\vd\textbf{चन्दन॰}\lem \msCa, \om\ \msCb, नन्दन॰ \Ed}}% 

%Verse 24:31

{\devanagarifont उपद्वीपसहस्राणि एवमादीनि कीर्तितम् {॥ २४:३१॥} \veg\dontdisplaylinenum }%
     \var{{\devanagarifontvar \numnoemph\vab\textbf{(उपद्वीप॰{\englishfont ...}कीर्तितम)}\lem \msCa\Ed, \om\ \msCb}}% 


\alalfejezet{अग्नीध्रपुत्रा जम्बुद्वीपे}
{\devanagarifont अग्नीध्रो नववर्षेषु नवपुत्रानसिञ्चयत् \thinspace{\dandab} \dontdisplaylinenum }%
     \var{{\devanagarifontvar \numemph\vc\textbf{अग्नीध्रो}\lem \eme, अग्नीन्ध्र \msCa\msCb, अग्नीन्ध्रो \Ed}}% 
    \var{{\devanagarifontvar \numnoemph\vd\textbf{॰सिञ्चयत्}\lem \msCb, सि\lk\lac\  \msCa, ॰भिषिञ्चयत् \Ed}}% 

%Verse 24:32

{\devanagarifont नाभिः किंपुरुषश्चैव हरिवर्ष इलावृतः {॥ २४:३२॥} \veg\dontdisplaylinenum }%
     \var{{\devanagarifontvar \numnoemph\va\textbf{नाभिः}\lem \Ed, \lac\  \msCa, नाभि \msCb}}% 

{\devanagarifont पञ्चमं रम्यकं वर्षं षष्ठं चैव हिरण्मयम् \thinspace{\dandab} \dontdisplaylinenum }%
 
%Verse 24:33

{\devanagarifont कुरवः सप्तमो ज्ञेयो भद्राश्वश्चाष्टमः स्मृतः {॥ २४:३३॥} \veg\dontdisplaylinenum }%
     \var{{\devanagarifontvar \numemph\vo\textbf{(पञ्चमं॰{\englishfont ...}प्रकीर्तिताः)}\lem \msCa\msCb, \om\ \Ed}}% 

{\devanagarifont नवमः केतुमालो ऽभून् नववर्षाः प्रकीर्तिताः \thinspace{\dandab} \dontdisplaylinenum }%
     \var{{\devanagarifontvar \numemph\vc\textbf{॰मालो}\lem \msCa, ॰मानो \msCb, \om\ \Ed}}% 

%Verse 24:34

{\devanagarifont हिमवद्दक्षिणे पार्श्वे वर्षो भारतसंज्ञितः {॥ २४:३४॥} \veg\dontdisplaylinenum }%
     \var{{\devanagarifontvar \numnoemph\vo\textbf{(नवमः{\englishfont ...}॰सम्भवः)}\lem \msCa\msCb, \om\ \Ed}}% 

{\devanagarifont अत्रापि नवभेदो ऽभूद्भारतात्मजसम्भवः \thinspace{\dandab} \dontdisplaylinenum }%
     \var{{\devanagarifontvar \numemph\vcd\textbf{ऽभूद्भारतात्मज॰}\lem \msCb\Ed, \lac  ज॰ \msCa}}% 

%Verse 24:35

{\devanagarifont इन्द्रद्वीपः कशेरुश्च ताम्रवर्णो गभस्तिमान् {॥ २४:३५॥} \veg\dontdisplaylinenum }%
 
{\devanagarifont नागद्वीपस्तथा सौम्यो गान्धर्वश्चाथ वारुणः \thinspace{\dandab} \dontdisplaylinenum }%
     \var{{\devanagarifontvar \numemph\vc\textbf{सौम्यो}\lem \msCb\Ed, सौम्या \msCa}}% 
    \var{{\devanagarifontvar \numnoemph\vd\textbf{गान्धर्व॰}\lem \msCa\msCb, गन्धर्व॰ \Ed}}% 

{\devanagarifont अयं च नवमो द्वीपः कुमारीद्वीपसंज्ञितः  \danda\dontdisplaylinenum }%
 
%Verse 24:36

{\devanagarifont दक्षिणे हेमकूटस्य वर्षः किंपुरुषः स्मृतः {॥ २४:३६॥} \veg\dontdisplaylinenum }%
 
{\devanagarifont निषधो दक्षिणपार्श्वे हरिवर्ष इति स्मृतः \thinspace{\dandab} \dontdisplaylinenum }%
 
%Verse 24:37

{\devanagarifont मेरुमूले तु राजेन्द्र ज्ञेयो वर्ष इलावृतः {॥ २४:३७॥} \veg\dontdisplaylinenum }%
 
{\devanagarifont उत्तरणेण (उत्तरेण?) तु नीलस्य वर्ष रम्यक उच्यते \thinspace{\dandab} \dontdisplaylinenum }%
 
%Verse 24:38

{\devanagarifont श्वेत-उत्तरतो ज्ञेयो वर्षरम्यहिरण्मयः {॥ २४:३८॥} \veg\dontdisplaylinenum }%
 
{\devanagarifont तस्य उत्तरतो ज्ञेयस्त्रिशृङ्गवरपर्वतः \thinspace{\dandab} \dontdisplaylinenum }%
 
%Verse 24:39

{\devanagarifont तस्य चोत्तरपार्श्वे तु वर्षः कुरुवले स्मृतः {॥ २४:३९॥} \veg\dontdisplaylinenum }%
 
{\devanagarifont पूर्वं भद्राश्वतो ज्ञेयः केतुमालस्तु पश्चिमे \thinspace{\dandab} \dontdisplaylinenum }%
 
%Verse 24:40

{\devanagarifont हिमंवान्हेमकूटश्च निषधो नील एव च {॥ २४:४०॥} \veg\dontdisplaylinenum }%
 
{\devanagarifont श्वेतश्च शृङ्गवन्तश्च षडेते वर्षपर्वताः \thinspace{\dandab} \dontdisplaylinenum }%
 
%Verse 24:41

{\devanagarifont अशीतिनवतीलक्षः - वर्षपर्वतमायतम् {॥ २४:४१॥} \veg\dontdisplaylinenum }%
 
{\devanagarifont हिमवान्हेमकूटश्च निषधश्चेति दक्षिण \thinspace{\dandab} \dontdisplaylinenum }%
 
%Verse 24:42

{\devanagarifont श्वेतश्चैवत्रिशृङ्गश्च नीलश्चैव तथोत्तरे {॥ २४:४२॥} \veg\dontdisplaylinenum }%
 
{\devanagarifont निषधो नीलमध्ये तु मेरुः शैलमनोरमः \thinspace{\dandab} \dontdisplaylinenum }%
 
%Verse 24:43

{\devanagarifont प्रविष्टषोडशाधस्तां चतुराशीतिमुच्छृतः {॥ २४:४३॥} \veg\dontdisplaylinenum }%
 
{\devanagarifont योजनानां सहस्राणि द्वात्रिंशदूर्ध ! विस्तृतः \thinspace{\dandab} \dontdisplaylinenum }%
 
%Verse 24:44

{\devanagarifont ब्रह्मामनोवती नाम पुरेव सतिमध्यमे {॥ २४:४४॥} \veg\dontdisplaylinenum }%
 
%Verse 24:44

{\devanagarifont देवराजो ऽमरावत्यामग्निस्तेजोवती पुरे {॥ २४:४४॥} \veg\dontdisplaylinenum }%
 
{\devanagarifont यमः संयमनी नाम नित्यं वसति भूपते \thinspace{\dandab} \dontdisplaylinenum }%
 
%Verse 24:45

{\devanagarifont नैऋतिर्वसति नित्यं रम्ये शुद्धवती पुरे {॥ २४:४५॥} \veg\dontdisplaylinenum }%
 
{\devanagarifont वरुणो भोगवत्यां तु वायोर्गन्धवती पुरी \thinspace{\dandab} \dontdisplaylinenum }%
 
%Verse 24:46

{\devanagarifont महोदयापुरी रम्या सोमस्यालयरं स्मृतम् {॥ २४:४६॥} \veg\dontdisplaylinenum }%
 
{\devanagarifont यशोवती पुरी रम्यान्नित्यमास्ते त्रिशूलिनः \thinspace{\dandab} \dontdisplaylinenum }%
 
%Verse 24:47

{\devanagarifont तत्र गङ्गा चतुःभिन्ना निपतन्ती महीतले {॥ २४:४७॥} \veg\dontdisplaylinenum }%
 
{\devanagarifont उत्तरे पश्चिमे चैव पूर्वदक्षिणतस्तथा \thinspace{\dandab} \dontdisplaylinenum }%
 
%Verse 24:48

{\devanagarifont पूर्वं गङ्गा स्रवत्याच्चालकानन्दा च दक्षिणे {॥ २४:४८॥} \veg\dontdisplaylinenum }%
 
{\devanagarifont शीता पश्चिमगा गङ्गा भद्रसोमा तथोत्तरे \thinspace{\dandab} \dontdisplaylinenum }%
 
%Verse 24:49

{\devanagarifont षष्टियोजनसाहस्रं निरालम्बा निपत्य च {॥ २४:४९॥} \veg\dontdisplaylinenum }%
 
{\devanagarifont भद्राश्वं प्लावयित्वा तु वनान्युपवनानि च \thinspace{\dandab} \dontdisplaylinenum }%
 
%Verse 24:50

{\devanagarifont द्रोणस्थली गिरीणां च अतिक्रम्यार्णवं गता {॥ २४:५०॥} \veg\dontdisplaylinenum }%
 
{\devanagarifont तथैवालकनन्दा च गताशैलेननिम्नगा \thinspace{\dandab} \dontdisplaylinenum }%
 
%Verse 24:51

{\devanagarifont गङ्गा भारतवर्षं च प्रविष्टालवणो दधिम् {॥ २४:५१॥} \veg\dontdisplaylinenum }%
 
{\devanagarifont प्लावयित्वा स्थलीन्सर्वान्मानुषाकलुषापहा \thinspace{\dandab} \dontdisplaylinenum }%
 
%Verse 24:52

{\devanagarifont पश्चिमेन गतागङ्गा सीतानामा च भारतः {॥ २४:५२॥} \veg\dontdisplaylinenum }%
 
{\devanagarifont प्लावयेत्केतुमालां च क्षेत्रशैववनस्थलीम् \thinspace{\dandab} \dontdisplaylinenum }%
 
%Verse 24:53

{\devanagarifont अतिक्रम्यार्णवगता स्थलीद्रोणी च निम्नगा {॥ २४:५३॥} \veg\dontdisplaylinenum }%
 
{\devanagarifont भद्रसोमनदीत्येवं प्लावयित्वोत्तरं कुरुन् \thinspace{\dandab} \dontdisplaylinenum }%
 
%Verse 24:54

{\devanagarifont स्थली प्रस्रवणद्रोणीमतिक्रम्यार्णवं गता {॥ २४:५४॥} \veg\dontdisplaylinenum }%
 
{\devanagarifont मेरो वै दक्षिणे पार्श्वे जम्बूवृक्षः सनातनः \thinspace{\dandab} \dontdisplaylinenum }%
 
%Verse 24:55

{\devanagarifont तेन नामाङ्कितो राजन्जम्बूद्वीप इति श्रुतम् {॥ २४:५५॥} \veg\dontdisplaylinenum }%
 
{\devanagarifont कोटीषोडशभिश्चैव अयुतानि त्रयोदश \thinspace{\dandab} \dontdisplaylinenum }%
 
%Verse 24:56

{\devanagarifont अधोर्धयाम राजेन्द्र क्षित्यावरणमन्ततः {॥ २४:५६॥} \veg\dontdisplaylinenum }%
 
{\devanagarifont नवलक्षाधिकं राजन्पञ्चकोटी मही स्मृता \thinspace{\dandab} \dontdisplaylinenum }%
 
%Verse 24:57

{\devanagarifont योजनानां तु विज्ञेयः पृथिव्यायामविस्तरात् {॥ २४:५७॥} \veg\dontdisplaylinenum }%
 
{\devanagarifont स्वादूदकस्य च बहिर्लोकालोको महागिरिः \thinspace{\dandab} \dontdisplaylinenum }%
 
%Verse 24:58

{\devanagarifont कञ्चनिद्विगुणाभूमि तस्माद्गिरिबहि स्मृतः {॥ २४:५८॥} \veg\dontdisplaylinenum }%
 
{\devanagarifont तस्माद्बाह्यः समुद्रो भूद्गर्भादेति समुद्रराट् \thinspace{\dandab} \dontdisplaylinenum }%
 
%Verse 24:59

{\devanagarifont अष्टाविंशतिकं लक्षं शतलक्षाणि विस्तरम् {॥ २४:५९॥} \veg\dontdisplaylinenum }%
 \versno=60

\vers


{\devanagarifont एतद्भूर्लोकविस्तारो ह्यत ऊर्ध्वं भुवः स्मृतः \thinspace{\dandab} \dontdisplaylinenum }%
     \var{{\devanagarifontvar \numemph\vb\textbf{ह्यत ऊर्ध्वं}\lem \mssCaCbCc, ह्यतद्र्ध्व \Ed}}% 

%Verse 24:61

{\devanagarifont स्वर्लोकस्य परेणैव महर्लोकमतः परम् {॥ २४:६१॥} \veg\dontdisplaylinenum }%
     \var{{\devanagarifontvar \numnoemph\vc\textbf{स्वर्लोकस्य}\lem \mssCaCbCc, स्वर्ल्लोकास्य \Ed}}% 

{\devanagarifont जनर्लोकस्तपः सत्यं क्रमशः परिकीर्तितम् \thinspace{\dandab} \dontdisplaylinenum }%
     \var{{\devanagarifontvar \numemph\va\textbf{जनर्लोकस्त॰}\lem \msCa\msCb, जनलोक त॰ \msCc, जनलोकस त॰ \Ed}}% 
    \var{{\devanagarifontvar \numnoemph\vb\textbf{॰कीर्तितम्}\lem \msCa\msCb\Ed, ॰कीर्तितः \msCc}}% 

%Verse 24:62

{\devanagarifont ब्रह्मलोकः स्मृतः सत्यं विष्णुलोकमतः परम् {॥ २४:६२॥} \veg\dontdisplaylinenum }%
     \var{{\devanagarifontvar \numnoemph\vc\textbf{॰लोकः}\lem \msCa\msCb\Ed, ॰लोक \msCc}}% 


\alalfejezet{शिवलोकः}
{\devanagarifont तस्मात्परेण बोधव्यं दिव्यध्यानपुरं महत् \thinspace{\dandab} \dontdisplaylinenum }%
     \var{{\devanagarifontvar \numemph\va\textbf{तस्मात्परेण बोधव्यं}\lem \msCb\msCc\Ed, त\lac  धव्यन् \msCa}}% 
    \var{{\devanagarifontvar \numnoemph\vb\textbf{दिव्य॰}\lem \msCa\Ed, दिव्यं \msCb}}% 

%Verse 24:63

{\devanagarifont सहस्रभौमप्रासादं वैडूर्यमणितोरणम् {॥ २४:६३॥} \veg\dontdisplaylinenum }%
     \var{{\devanagarifontvar \numnoemph\vd\textbf{वैडूर्य॰}\lem \msCa\msCb, वैदूर्य॰ \Ed}}% 

{\devanagarifont नानारत्नविचित्राणि नानाभूतगणाकुलम् \thinspace{\dandab} \dontdisplaylinenum }%
 
%Verse 24:64

{\devanagarifont सर्वकामसमृद्धानि पूर्णं तत्र मनोहरैः {॥ २४:६४॥} \veg\dontdisplaylinenum }%
 
{\devanagarifont तत्र सिंहासने दिव्ये सर्वरत्नविभूषिते \thinspace{\dandab} \dontdisplaylinenum }%
 
%Verse 24:65

{\devanagarifont तत्रास्ते भगवान्रुद्रः सोमाङ्कितजटाधरः {॥ २४:६५॥} \veg\dontdisplaylinenum }%
     \var{{\devanagarifontvar \numemph\vd\textbf{॰धरः}\lem \msCb\Ed, \uncl{ध}\lac\  \msCa}}% 

{\devanagarifont त्र्यक्षस्त्रिभुवनश्रेष्ठस्त्रिशूली त्रिदशाधिपः \thinspace{\dandab} \dontdisplaylinenum }%
     \var{{\devanagarifontvar \numemph\va\textbf{त्र्यक्षस्त्रि॰}\lem \corr, \lac  स्त्रि॰ \msCa, त्र्यक्षर॰ \msCb\ \unmetr, त्र्यक्षत्रि॰ \Ed}}% 

%Verse 24:66

{\devanagarifont देव्या सह महाभागो गणैश्च परिवारितः {॥ २४:६६॥} \veg\dontdisplaylinenum }%
 
{\devanagarifont स्कन्दनन्दिपुरोगश्च गणकोटीशताकुलः \thinspace{\dandab} \dontdisplaylinenum }%
     \var{{\devanagarifontvar \numemph\vb\textbf{॰कोटी॰}\lem \msCa, ॰कोटि॰ \msCb\Ed\oo 
\textbf{॰कुलः}\lem \msCa\Ed, ॰कुलम् \msCb}}% 

%Verse 24:67

{\devanagarifont अनेकरुद्रकन्याभी रूपिणीभिरलङ्कृतः {॥ २४:६७॥} \veg\dontdisplaylinenum }%
     \var{{\devanagarifontvar \numnoemph\vc\textbf{अनेकरुद्रकन्याभी॰}\lem \msCb, \lac  न्याभी \msCa, अनेकरुद्रकन्यभि॰ \Ed}}% 

{\devanagarifont तत्र पुण्यनदी सप्त सर्वपापापनोदनी \thinspace{\dandab} \dontdisplaylinenum }%
 
%Verse 24:68

{\devanagarifont सुवर्णवालुका दिव्या रत्नपाषाणशोभिता {॥ २४:६८॥} \veg\dontdisplaylinenum }%
     \var{{\devanagarifontvar \numemph\vcd\textbf{दिव्या रत्नपाषाणशोभिता}\lem \msCb\Ed, दि\lac  णशोभिता \msCa}}% 

{\devanagarifont पावनी च वरेण्या च वरार्हा वरदा वरा \thinspace{\dandab} \dontdisplaylinenum }%
     \var{{\devanagarifontvar \numemph\va\textbf{पावनी}\lem \msCa\Ed, पावणी \msCb}}% 
    \var{{\devanagarifontvar \numnoemph\vb\textbf{वरार्हा}\lem \msCa\Ed, वराहा \msCb}}% 

%Verse 24:69

{\devanagarifont वरेशा वरभद्रा च सुप्रसन्नजला शिवा {॥ २४:६९॥} \veg\dontdisplaylinenum }%
     \var{{\devanagarifontvar \numnoemph\vd\textbf{॰प्रसन्न॰}\lem \msCa\msCb, ॰प्रसन्ना॰ \Ed}}% 

{\devanagarifont अनेककुसुमारामा रत्नपुष्पफलद्रुमाः \thinspace{\dandab} \dontdisplaylinenum }%
     \var{{\devanagarifontvar \numemph\vb\textbf{॰द्रुमाः}\lem \msCb\Ed, ॰\uncl{द्रुमाः} \msCa}}% 

%Verse 24:70

{\devanagarifont अनेकरत्नप्राकारा योजनायुतमुच्छ्रिताः {॥ २४:७०॥} \veg\dontdisplaylinenum }%
     \var{{\devanagarifontvar \numnoemph\vcd\textbf{अनेकरत्नप्राकारा योजनायुतमुच्छ्रिताः}\lem \Ed, 
\om\ \mssCaCbCc\msNa\msNb\msNc\msM\ ({\englishfont \msNa\ marks the omission})}}% 

{\devanagarifont अहिंसासत्यनिरताः कामक्रोधविवर्जिताः \thinspace{\dandab} \dontdisplaylinenum }%
     \var{{\devanagarifontvar \numemph\va\textbf{अहिंसासत्य॰}\lem \msCb\Ed, \lac  त्य॰ \msCa\oo 
\textbf{॰रताः}\lem \msCa\Ed, ॰रता \msCb}}% 

%Verse 24:71

{\devanagarifont ध्यानयोगरता नित्यं तत्र मोदन्ति ते नराः {॥ २४:७१॥} \veg\dontdisplaylinenum  }%
 
{\devanagarifont तत्र गोमातरस् सर्वा निवसन्ति यतव्रताः \thinspace{\dandab} \dontdisplaylinenum }%
     \var{{\devanagarifontvar \numemph\vab\textbf{गोमातरः सर्वा निवसन्ति}\lem \msCb\msNa\Ed, \lac  सन्ति \msCa}}% 

%Verse 24:72

{\devanagarifont गोलोकः शिवलोकश्च एक एव विधीयते {॥ २४:७२॥} \veg\dontdisplaylinenum }%
     \paral{{\devanagarifontsmall \vcd {\englishfont \similar\ \SDhU\ 12.88ab:}
                     गोलोकः शिवलोकश्च एक एव ततः स्मृतः }}
    \lacuna{\devanagarifontsmall \vcd {\englishfont \Ed\ and at least two paper NGMCP MSS (A1341-6 and C107-7) 
                 add two anuṣṭubh verses after this line (minor variations ignored here;
                 in \msNa, there is an omission mark at this point):}
                 तस्मादूर्ध्वं परं ज्ञेयं स्थानत्रयमनुत्तमम्\,{\devanagarifontsmall ।}
                 स्कन्दगौरीमहेशानां नित्यशुद्धं परं शिवम्\,{\devanagarifontsmall ॥}
                 दिनकृत्कोटिसङ्कासमनोपम्यं सनातनम्\,{\devanagarifontsmall ।}
                 आदित्यादिशिवान्तश्च द्विस्थानोर्ध्वक्रमः स्मृतः\,{\devanagarifontsmall ॥} }%
  

\alalfejezet{शास्त्रवर्णना}
\ujvers\nemsloka {
{\devanagarifont अभ्यन्तरे तत्कथितो ऽद्य सारं }%
  \dontdisplaylinenum}    \var{{\devanagarifontvar \numemph\va\textbf{अभ्यन्तरे तत्क॰}\lem \msCa\msCb, अत्यन्तरेत्क॰ \Ed}}% 


\nemslokab

{\devanagarifont किमन्य राजन् कथयामि सारम्  \danda\dontdisplaylinenum }%
     \var{{\devanagarifontvar \numnoemph\vb\textbf{किमन्य रा॰}\lem \msCa\Ed, किमन्यद्रा॰ \msCb}}% 

\nemslokac

{\devanagarifont ज्ञानार्णवं कीर्तित धर्मसारम् }%
  \dontdisplaylinenum    \var{{\devanagarifontvar \numnoemph\vc\textbf{ज्ञानार्णवं कीर्तित धर्मसारम्}\lem \msCb\Ed, 
ज्ञानार्ण्ण\uncl{ङ्कीर्ति}\lac\  \msCa}}% 

%Verse 24:73


\nemslokad

{\devanagarifont पुराणवेदोपनिषत्सुसारम् {॥ २४:७३॥} \veg\dontdisplaylinenum }%
 
\ujvers\nemsloka {
{\devanagarifont यथा हि राजा परिवारमध्ये }%
  \dontdisplaylinenum}    \var{{\devanagarifontvar \numemph\va\textbf{॰वारमध्ये}\lem \msCa, ॰वारणै \msCb, ॰चारमध्ये \Ed}}% 


\nemslokab

{\devanagarifont यथान्तवर्ती बहिवर्तिनेव  \danda\dontdisplaylinenum }%
     \var{{\devanagarifontvar \numnoemph\vb\textbf{यथान्तव॰}\lem \msCb\Ed, यथान्तर्व्व \msCa\oo 
\textbf{॰वर्तिनेव}\lem \msCb\Ed, वर्त्ति\uncl{ने}व \msCa}}% 

\nemslokac

{\devanagarifont भुञ्जन्ति भोगान्सततान्तवर्ती }%
  \dontdisplaylinenum    \var{{\devanagarifontvar \numnoemph\vc\textbf{भुञ्जन्ति भोगान्}\lem \msCb\Ed, \uncl{भुञ्ज} \lac\  \msCa\oo 
\textbf{सततान्तवर्ती}\lem \msCa\Ed, सततान्नवर्ती \msCb}}% 

%Verse 24:74


\nemslokad

{\devanagarifont क्लेशाधिकं नित्य बहिःस्थितानाम् {॥ २४:७४॥} \veg\dontdisplaylinenum }%
     \var{{\devanagarifontvar \numnoemph\vd\textbf{बहिः॰}\lem \msCa\Ed, बहि॰ \Ed}}% 

\ujvers\nemsloka {
{\devanagarifont यथैव राजा करिणो ऽन्तदन्तम् }%
  \dontdisplaylinenum}    \var{{\devanagarifontvar \numemph\va\textbf{करिणो ऽन्तदन्तम्}\lem \msCb, करिणो ऽन्तर्दन्तम् \msCa, करिणान्तदन्तदत्तम् \Ed}}% 
    \paral{{\devanagarifontsmall \vo {\englishfont \SDHU\ 3.83} }}


\nemslokab

{\devanagarifont भुञ्जन्ति भोगान्सततं नरेन्द्र  \danda\dontdisplaylinenum }%
     \var{{\devanagarifontvar \numnoemph\vb\textbf{भुञ्जन्ति}\lem \msCb\Ed, भुजन्ति \msCa}}% 

\nemslokac

{\devanagarifont युध्येत राजा बहिर्दन्तभोगैर् }%
  \dontdisplaylinenum    \var{{\devanagarifontvar \numnoemph\vc\textbf{राजा}\lem \msCa\Ed, राज \msCb\oo 
\textbf{बहिर्दन्तभोगैर्}\lem \msCa\msCb, बहिदत्तभोगैर् \Ed}}% 

%Verse 24:75


\nemslokad

{\devanagarifont यदन्तरं पश्य समानजातम् {॥ २४:७५॥} \veg\dontdisplaylinenum }%
     \var{{\devanagarifontvar \numnoemph\vd\textbf{यदन्तरं पश्य समानजातम्}\lem \msCb, यदन्तरे पश्य समानजातम् \Ed, 
यदन्त\uncl{रे}\lac  नजातम् \msCa}}% 

\ujvers\nemsloka {
{\devanagarifont न दानतुल्यं त्वभयप्रदस्य }%
  \dontdisplaylinenum}

\nemslokab

{\devanagarifont न यज्ञतुल्यं जित-इन्द्रियस्य  \danda\dontdisplaylinenum }%
 
\nemslokac

{\devanagarifont न चार्थतुल्यं जितकामिनश्च }%
  \dontdisplaylinenum    \var{{\devanagarifontvar \numemph\vc\textbf{॰कामिनश्च}\lem \Ed, कामिन\lac\  \msCa}}% 

%Verse 24:76


\nemslokad

{\devanagarifont न धर्मतुल्यं दमकामितस्य {॥ २४:७६॥} \veg\dontdisplaylinenum }%
     \var{{\devanagarifontvar \numnoemph\vd\textbf{न धर्मतुल्यं}\lem \Ed, \lac\  \msCa, \om\ \msCb\oo 
\textbf{दमकामितस्य}\lem \msCa, \om\ \msCb, दमकामिनस्य \Ed}}% 

\ujvers\nemsloka {
{\devanagarifont बह्वन्तरं नैव हि धर्मयोश्च }%
  \dontdisplaylinenum}

\nemslokab

{\devanagarifont क्लेशाधिकं बाह्यफलाल्पसारम्  \danda\dontdisplaylinenum }%
 
\nemslokac

{\devanagarifont यदत्र धर्मं फलनैष्ठिकस्य }%
  \dontdisplaylinenum    \var{{\devanagarifontvar \numemph\vc\textbf{॰नैष्ठिकस्य}\lem \Ed, ॰नैशिकस्य \msCb}}% 

%Verse 24:77


\nemslokad

{\devanagarifont न तुल्य कोटीशतयाजिनापि {॥ २४:७७॥} \veg\dontdisplaylinenum }%
 
\ujvers\nemsloka {
{\devanagarifont एतत्पवित्रं परमं सधर्मम् }%
  \dontdisplaylinenum}

\nemslokab

{\devanagarifont पुरा यथोक्तं परमेश्वरेण  \danda\dontdisplaylinenum }%
 
\nemslokac

{\devanagarifont मयापि तुल्यं कथितं यथावत् }%
  \dontdisplaylinenum
%Verse 24:78


\nemslokad

{\devanagarifont पुराणवेदोपनिषत्सुसारम् {॥ २४:७८॥} \veg\dontdisplaylinenum }%
 
\ujvers\nemsloka {
{\devanagarifont सदोजसौभाग्यमतीव मेधा }%
  \dontdisplaylinenum}    \var{{\devanagarifontvar \numemph\va\textbf{सदोज॰}\lem \Ed, सदोजः \msCb}}% 


\nemslokab

{\devanagarifont निरुत्सुकः सौम्यमनुत्तमं च  \danda\dontdisplaylinenum }%
     \var{{\devanagarifontvar \numnoemph\vb\textbf{निरुत्सुकः}\lem \Ed, निरुत्सुक॰ \msCb}}% 

\nemslokac

{\devanagarifont सुपुत्रपौत्रं न विछिन्नगोत्रम् }%
  \dontdisplaylinenum
%Verse 24:79


\nemslokad

{\devanagarifont भवन्ति विद्याधरलोकपूज्यम् {॥ २४:७९॥} \veg\dontdisplaylinenum }%
 
\ujvers\nemsloka {
{\devanagarifont यशश्रियं कीर्तिरतीव तेजो }%
  \dontdisplaylinenum}    \var{{\devanagarifontvar \numemph\va\textbf{यशः॰}\lem \msCb, यश॰ \Ed}}% 


\nemslokab

{\devanagarifont जनप्रियो धान्यधनायुवृद्धिम्  \danda\dontdisplaylinenum }%
     \var{{\devanagarifontvar \numnoemph\vb\textbf{॰वृद्धिम्}\lem \msCb, ॰वृद्धिः \Ed}}% 

\nemslokac

{\devanagarifont प्रबोधप्रज्ञारुजधर्मवृद्धिम् }%
  \dontdisplaylinenum    \var{{\devanagarifontvar \numnoemph\vc\textbf{॰वृद्धिम्}\lem \Ed, ॰वृद्धि \msCb}}% 

%Verse 24:80


\nemslokad

{\devanagarifont भवन्ति तं शास्त्रसदाभियोगी {॥ २४:८०॥} \veg\dontdisplaylinenum }%
     \var{{\devanagarifontvar \numnoemph\vd\textbf{तं}\lem \msCb, ते \Ed}}% 

\ujvers\nemsloka {
{\devanagarifont यशस्विनी आर्यसुवर्णशृङ्गी }%
  \dontdisplaylinenum}

\nemslokab

{\devanagarifont वेदान्तविप्रद्विजगायनेषु  \danda\dontdisplaylinenum }%
 
\nemslokac

{\devanagarifont दत्त्वा फलं तीर्थमनुत्तमेषु }%
  \dontdisplaylinenum    \var{{\devanagarifontvar \numemph\vc\textbf{॰नुत्तमेषु}\lem \Ed, ॰नुमेषु \msCb}}% 

%Verse 24:81


\nemslokad

{\devanagarifont शृण्वन्ति ये तस्य भवेत्सपुण्यम् {॥ २४:८१॥} \veg\dontdisplaylinenum }%
 
\ujvers\nemsloka {
{\devanagarifont दशाधिकं वाचयितुश्च पुण्यम् }%
  \dontdisplaylinenum}    \var{{\devanagarifontvar \numemph\va\textbf{वाचयितुश्च}\lem \msCb, वाच चतुश्च \Ed}}% 


\nemslokab

{\devanagarifont शताधिकं यः पठति प्रभाते  \danda\dontdisplaylinenum }%
 
\nemslokac

{\devanagarifont सहस्रशः पुस्तकृतस्य पुण्यम् }%
  \dontdisplaylinenum
%Verse 24:82


\nemslokad

{\devanagarifont परे ऽभ्यस्ते कीर्तयते ऽयुतानि {॥ २४:८२॥} \veg\dontdisplaylinenum }%
     \var{{\devanagarifontvar \numnoemph\vd\textbf{परे}\lem \msCb, परो \Ed\oo 
\textbf{कीर्त॰}\lem \msCb\pcorr, कीर्ति॰ \msCb\acorr}}% 

\ujvers\nemsloka {
{\devanagarifont अधीत्य यस्योरगतं सुशास्त्रम् }%
  \dontdisplaylinenum}

\nemslokab

{\devanagarifont समस्तमध्यायमनुक्रमेन  \danda\dontdisplaylinenum }%
 
\nemslokac

{\devanagarifont दशायुताङ्गो ददतुश्च पुण्यम् }%
  \dontdisplaylinenum    \var{{\devanagarifontvar \numemph\vc\textbf{दशायुताङ्गो ददतु॰}\lem \Ed, दशायु\uncl{तङ्ग दे}दतु॰ \msCb}}% 

%Verse 24:83


\nemslokad

{\devanagarifont लभत्यसंदिग्धयथादिनैकं {॥ २४:८३॥} \veg\dontdisplaylinenum }%
     \var{{\devanagarifontvar \numnoemph\vd\textbf{॰यथादिनैकं}\lem \Ed, ॰य\uncl{थादि}नैकं \msCb}}% 

\ujvers\nemsloka {
{\devanagarifont येनेदं शास्त्रसारमविकलमनसा यो ऽभ्यसेत्तत्प्रयत्नात् }%
  \dontdisplaylinenum}    \var{{\devanagarifontvar \numemph\va\textbf{॰सारम॰}\lem \msCa\msCb\msNa\Ed, ॰सारंम॰ \msCc\oo 
\textbf{॰विकल॰}\lem \msCa\msNa\Ed, ॰किल॰ \msCb\oo 
\textbf{भ्यसेत्तत्प्र॰}\lem \mssCaCbCc\msNa,  भ्यसेत प्र॰ \Ed}}% 


\nemslokab

{\devanagarifont व्यक्तो ऽसौ सिद्धयोगी भवति च नियतं यस्तु चित्तप्रसन्नः  \danda\dontdisplaylinenum }%
     \var{{\devanagarifontvar \numnoemph\vb\textbf{सौ}\lem \msCb\msCc\msNa\Ed, सो \msCa\oo 
\textbf{चित्त॰}\lem \mssCaCbCc\Ed, विन्न॰ \msNa\oo 
\textbf{॰प्रसन्नः}\lem \msCa\msCc\msNa\Ed, ॰प्रयान्ना \msCb}}% 

\nemslokac

{\devanagarifont पित्र्यं यो गीतपूर्वं प्रतिदिन शतश उद्ध्रियन्ते च सर्वे }%
  \dontdisplaylinenum    \var{{\devanagarifontvar \numnoemph\vc\textbf{पित्र्यं यो गीतपूर्वं}\lem \mssCaCbCc\msNa, नित्य यो धीतयोत पूर्व्वं \Ed\oo 
\textbf{॰दिनशतश उद्ध्रियन्ते च सर्वे}\lem \msNa, 
\uncl{दिन}\lac\  \msCa, 
॰दिनशतस उद्वियन्ते च सर्वे \msCb, 
॰दिनशतस उद्रियन्ते च सर्वे \msCc, 
॰दिनशतशो उर्द्धि यन्ते च सर्वे \Ed}}% 

%Verse 24:84


\nemslokad

{\devanagarifont आत्मानं निर्विकल्पं शिवपदमसमं प्राप्नुवन्तीह सर्वे {॥ २४:८४॥} \veg\dontdisplaylinenum }%
 
\vers


{\devanagarifont 
\jump
\begin{center}
\ketdanda~इति वृषसारसंग्रहे शास्त्रवर्णना नाम चतुर्विंशतितमो ऽध्यायः समाप्तः~\ketdanda
\end{center}
\dontdisplaylinenum\vers  }%
     \var{{\devanagarifontvar \numnoemph{\englishfont \Colo:}\textbf{॰वर्णना}\lem \msCa\msCb, ॰वर्णनो \Ed\oo 
\textbf{ध्यायः समाप्तः}\lem \msCa\msCb, ध्यायः \Ed}}% 

{\devanagarifont 
\jump
\begin{center}
\ketdanda~वृषसारसंग्रहः समाप्त इति~\ketdanda
\end{center}
\dontdisplaylinenum\vers  }%
     \var{{\devanagarifontvar \numnoemph{\englishfont \Colo:} {\englishfont CHECK DELETE Colo}\textbf{वृषसारसंग्रहः समाप्त इति}\lem \msCa, वृषसारसंग्रहं समाप्त इति \msCb, 
\om\ \Ed}}% 
