
\section{Editorial conventions}
\label{editorial}
\bigskip


\textit{The critically edited Sanskrit text} is to be found at the top of each page:

\begin{itemize}

%\item All Sanskrit words given in \suppl{angle brackets} have been added to the text by me as diagnostic conjectures.  These may or may not have been part of the `original' Sanskrit text.


\item Verse numbering has been supplied by the editor; none of the witnesses had any verse numbering.
\item \skt{avagraha}s that mark elision are mostly supplied but they are sometimes found in the MSS.
      I have not used them to mark crasis, although they sometimes appear in this role.
        See, e.g., \msKOb\ f.~\thinspace 215v line 1, verse 4.25cd, where the \skt{avagraha} 
        appears in both roles: {\devanagarifont नास्ति स्तेनसमाऽकीर्तिर्नास्ति स्तेनसमो ऽनयः~।।}
\item Singe and double \skt{daṇḍa}s have been supplied by the editor.
        There are usually four \skt{pāda}s to a verse, but I have made arbitrary decisions
        based on sense-units, and occasionally grouped six \skt{pāda}s together as one stanza;
        few of the sources clearly indicate where a stanza ends.

\item  Headings given in [square brackets] in the critical edition and the translation 
        have been supplied to clarify the context. 
        These are not supposed to be part of the original Sanskrit text.
\end{itemize}

\bigskip
\noindent
The \textit{apparatus} is fully positive and contains a maximum of three registers. 
When all three registers are present, they contain information as follows:

\begin{itemize}
\item
The bottom register reports the variants found in the manuscripts.
Each entry starts with a verse number which is usually followed by a \skt{pāda} sign.
Both are given in boldface (e.g., \textbf{25b}). The next element is the lemma, a word, phrase, or fragment from the main text,
followed by a lemma sign ( ] ). The lemma sign is followed
by a list of the sigla of the MSS that read the same as the lemma, up to a comma. 
Next, the rejected variants are listed, each followed by the sigla of the MSS that read the given variant.
A sigma sign ($\Sigma$) stands for all available witnesses used for the given chapter,
except for one or two variants in a maximum of two witnesses.
\mssCaCbCc\ signifies all available Cambridge MSS.
A siglum followed by superscript \acorr\ marks the reading of a MS before a scribal
alteration/correction (\textit{ante correctionem}).
A siglum followed by superscript \pcorr\ marks the reading of a MS after
a scribal alteration/correction (\textit{post correctionem}). 
Corrections by the editor are marked by
`\corr' after the lemma sign (~]~\corr~), emendations by `\eme', and conjectures by `\conj'
Whenever these alterations to the text were suggested by others, I give their last names after \corr,
\eme, or \conj\ (e.g., \conj\ \textsc{Devadatta}).
%I use the expression `\textit{diagnostic conj}[\textit{ecture}]' to indicate that a conjecture is highly tentative.
The difference between corrections, emendations and conjectures is somewhat subjective in nature.
Corrections are applied in cases where the editor considers the reasons for his alteration of the text self-evident and
in little need of explanation.
In the case of an emendation, % are usually highlighted in the text by typesetting the given phrase in \textit{italics} and some explanation,
one or more parallel passages in support of the alteration, or a description of the
pal\ae ographical phenomena that resulted in the corruption,
is usually given in the footnotes to the translation of the given passage.
Effort has also been made to support conjectures
%(also typeset in \textit{italics})
with evidence, but conjectures are considered more tentative than emendations.%
        \footnote{See a more detailed discussion on emendations and
                conjectures in \Torzsok\ 1999:lxxv--lxxvi{i}i.}
%An alteration for which I have not been able to provide any supporting evidence but which I consider an improvement upon a corrupt passage is labelled a `diagnostic conj[ecture]'.%

A bullet ($\bullet$) in the apparatus separates different entries that correspond to
the same \skt{pāda}.
% \gap\ stands for a gap in the text of the MS indicated by the scribe (usually by a horizontal line).
{\devanagarifont ॰} indicates that the lemma or variant is part of a longer compound or word. 
The sign \lk\ (anceps) indicates an \skt{akṣara} illegible to me.
\lac\ indicates a complete loss of a number of \skt{akṣara}s, usually due to damage.
The number that is often placed on \lac\ (e.g., \lacwithnum{3}) indicates the approximate
number of lost \skt{akṣara}s.
Letters enclosed by (parentheses) indicate that their reading is uncertain. 
Unmetrical \skt{pāda}s are marked by `\unmetr' only when it is not fully obvious,
i.e., they are usually not marked when there is one or more syllables more 
or less than required in an \skt{anuṣṭubh} in a variant.
Sometimes `\hypometr' or `\hypermetr' are also used for hypometrical and hypermetrical verses, respectively.
%${\circ}$ stands for small circles in the MSS used as abbreviations by the scribe (e.g.\ {\dn u}${\circ}$ for {\dn uvAc}).


\item The middle register contains testimonia, i.e., passages from other sources
or from elsewhere in the \VSS\ that are parallel or similar to the
corresponding verse in the \VSS\ and that can explain, support, or
contextualise the passage or stanza in question.
An entry starts with the verse number and \skt{pāda} sign of the \VSS\ stanza in question. % and in some cases with a lemma.
I then give the title of the source from which the passage has been drawn
and the exact verse number preceded by `=' if the parallel
passage is identical with the reading of the \VSS. 
`\similar' is supplied instead of `=' if the parallel passage
is similar but not identical with the reading of the \VSS.
Testimonia are preceded by `\compare' if the passage is somewhat similar to the \textit{textus criticus} of the
\VSS, or can throw some light on it because it treats a similar subject. 
%This register of the apparatus is also the location of miscellaneous remarks on the Sanskrit text.  These remarks provide information on stanza forms, on instances of double \skt{sandhi} (e.g.\ {\dn vrd\? iEt} \jobbranyil\ {\dn vrd iEt} \jobbranyil\ {\dn vrd\?Et}% %\skt{varade} + \skt{iti} ${\rightarrow}$\ \skt{varada iti} ${\rightarrow}$\ \skt{varadeti}), on alterations to the verse numbering of the MSS, on MSS breaking off, and so on.

\item The top register reports lacun\ae{}, and missing passages, in the MSS, and also, at the beginning of
chapters, provides list of witnesses used for the given chapter.


\end{itemize}

\noindent
\textit{The transcription} of the MSS, both in the critically edited version and
in reporting variants, involves some inevitable falsification:

\begin{itemize}

\item I have not attempted to always report differences in readings between \skt{akṣara}s that are usually
        interchangeable in the Nepālākṣara MSS ({\devanagarifont ब-व} , {\devanagarifont  व-च} , {\devanagarifont त-न} ,  
                {\devanagarifont य-प} , {\devanagarifont ष-स},
                but I always report them when both readings are
                theoretically possible (e.g.\ {\devanagarifont चन्दन}-{\devanagarifont वन्दन} , {\devanagarifont जय-जप}).

%\item There is some confusion between {\dn C} and {\dn \3E3w}, {\dn X} and {\dn P}, {\dn q} and {\dn K}, {\dn G} and {\dn \38Dw}; this I always report.

\item I have ignored all instances of gemination of consonants in ligature with semivowels in the main text 
and when reporting lemmata (e.g.\ {\devanagarifont कर्म} rather than {\devanagarifont कर्म्म}), 
but I always report rejected variants as they appear in the source whenever possible. If the same rejected reading
appears with different orthography in different sources, I usually report it as it appeared in the source collated first;
thus rejected variants are also often slightly falsified.

%\item I always report my corrections of {\dn C} to {\dn QC} within words (e.g.\ {\dn e\?QC\qq{n}}) and in external \skt{sandhi} (e.g.\ {\dn n QC\38DwA{\rdt}}).

\item I have altered \skt{anusvāra}s and homorganic nasals, including \skt{m}, in the main text, as
       required by standard orthography.

\item \skt{Avagraha}s are largely missing in the MSS. I have always silently supplied them
        in the \textit{textus criticus} and in the lemmata, but I have not supplied them when reporting variants.


\end{itemize}
