
  \chptr{navamo 'dhyāyaḥ}
\addcontentsline{toc}{section}{Chapter 9}
\fancyhead[CO]{{\footnotesize\textit{Translation of chapter 9}}}%

  \trchptr{ Chapter Nine}%

  \subchptr{traiguṇyam}%

  \trsubchptr{System of three qualities}%

  \maintext{{\rm [}anarthayajña uvāca | {\rm ]}}%

  \maintext{trikālaguṇabhedena bhinnaṃ sarvacarācaram |}%

  \maintext{tasmāt triguṇabandhena veṣṭitaṃ nikhilaṃ jagat }||\thinspace9:1\thinspace||%
\translation{All that move or do not move are divided by the three subdivisions {\rm (}\textit{guṇa}{\rm )} of time. Therefore the whole world is bound by the ties of the three qualities {\rm (}\textit{guṇa}{\rm )}. \blankfootnote{9.1 It is only \msM, a MS not collated for this chapter, that inserts, post correctionem,
  \textit{anarthayajña uvāca} at the beginning of this chapter. It is not really needed:
  Anarthayajña's teaching continues without interruption here. 
  Another possibility is that this verse was originally the continuation
  of the end of chapter two {\rm (}2:40ef: \textit{traikālyakalanāt kālas tena kālaḥ prakīrtitaḥ}{\rm )}.
  At least it seems to directly connect there topic-wise.
  My translation of \textit{guṇa} in \textit{pāda} a is tentative.
 }}

  \maintext{vigatarāga uvāca |}%

  \maintext{traikālyam iti kiṃ jñeyaṃ traidhātukaśarīriṇaḥ |}%

  \maintext{kiṃcid vistaram eveha kathayasva tapodhana }||\thinspace9:2\thinspace||%
\translation{Vigatarāga spoke: What does the term `the three times' mean for an embodied creature that is made up of the three constituents {\rm (}\textit{dhātuka}{\rm )}? Teach me about this in a somewhat more extended manner, O great ascetic. \blankfootnote{9.2 I have included the element \textit{trai}° in the lemma from \textit{pāda} b
  only because \msCc\ has a slightly unusual ligature there {\rm (}\textit{mtrai}{\rm )}. 
  
  As for the interpretation of
  \textit{traidhātuka} in \textit{pāda} b, an intelligent guess would be a reference to the three so-called
  `humours' of the body, namely \textit{pitta, vāyu/anila/vāta}, and \textit{śleṣman}.
  They are discussed later in \VSS\ chapter 23 in the context of types of sleep.
  \MBH\ 12.330.21--22ab clearly states that the three \textit{dhātu}s, \textit{pitta, śleṣma}, and \textit{vāyu}
  keep the body alive:
  
 
  \textit{trayo hi dhātavaḥ khyātāḥ karmajā iti ca smṛtāḥ}\thinspace |
 
  \textit{pittaṃ śleṣmā ca vāyuś ca eṣa saṃghāta ucyate}\thinspace ||
 
  \textit{etaiś ca dhāryate jantur etaiḥ kṣīṇaiś ca kṣīyate}\thinspace |
  
 
  See also \UUMS\ {\rm (}\msCa\ f.~179r line 4{\rm )}:
  
 
  \textit{tridhātukaṃ śarīram vai manujasya ca dehinaḥ}\thinspace | 
 
  \textit{śleṣmā pittañ ca vāyuś ca śarīraṃ tena vyāpitam}\thinspace ||
  
 
  The present verse in the \VSS\ contains the only occurrence 
  of the term \textit{traidhātuka} in the text.
  In 5.11cd, \textit{dhātu} is probably used in the same Ayurvedic sense 
  that I am proposing here {\rm (}\textit{dhātuvaiṣamyanāśo 'sti na ca rogāḥ sudāruṇāḥ}{\rm )}.
  Elsewhere \textit{dhātu} means `verbal root' {\rm (}3.3{\rm )}, `metal' {\rm (}16.6: 
  \textit{yathā vai sarvadhātūnāṃ doṣā dahyanti dhāmyatām}\thinspace | 
  \textit{tathā pāpāḥ pradahyante dhruvaṃ prāṇasya nigrahāt}\thinspace ||{\rm )},
  and `gross element' {\rm (}for Sāṃkhya-style \textit{mahābhūta}s in chapter 20{\rm )}.
  To slightly complicate things, chapter thirteen claims that the human body is made up
  of two \textit{dhātu}s, \textit{somadhātu} and \textit{agnidhātu}. Semen contains \textit{somadhātu},
  menstrual blood \textit{agnidhātu}, and the new-born baby is thus made up of both. 
  See e.g. 13.21--22:
  
 
  \textit{śukraśoṇitasaṃyogād garbhotpattis tataḥ smṛtā}\thinspace || 
 
  \textit{agnisomātmakaṃ devi śarīraṃ dvayadhātutaḥ}\thinspace | 
 
  \textit{somadhātu smṛtaṃ śukram agnidhātu rajaḥ smṛtam}\thinspace | 
 
  \textit{agnisomāśrayaṃ devi śarīram iti saṃjñitam}\thinspace ||
 }}

  \maintext{anarthayajña uvāca |}%

  \maintext{traikālyaṃ triguṇaṃ jñeyaṃ vyāpī prakṛtisambhavaḥ |}%

  \maintext{anyonyam upajīvanti anyonyam anuvartinaḥ }||\thinspace9:3\thinspace||%
\translation{Anarthayajña spoke: The three times are the three qualities {\rm (}\textit{guṇa}{\rm )}. They are [all-]pervading and are born from Prakṛti. They support each other, they follow each other. \blankfootnote{9.3 Understand \textit{pāda} b as referring to the neuter \textit{traikālyaṃ} or rather 
  \textit{triguṇaṃ} {\rm (}gender confusion{\rm )}.
 }}

  \maintext{sattvaṃ rajas tamaś caiva rajaḥ sattvaṃ tamas tathā |}%

  \maintext{tamaḥ sattvaṃ rajaś caiva anyonyamithunāḥ smṛtāḥ }||\thinspace9:4\thinspace||%
\translation{Sattva, Rajas and Tamas; Rajas, Sattva and Tamas; Tamas, Sattva and Rajas; they are mutually each other's pairs. }

  \maintext{sāttviko bhagavān viṣṇū rājasaḥ kamalodbhavaḥ |}%

  \maintext{tāmaso bhagavān īśaḥ sakalaṃvikaleśvaraḥ }||\thinspace9:5\thinspace||%
\translation{Lord Viṣṇu is Sattvic. [Brahmā], the one who was born on a lotus, is Rājasa. Lord Īśa is Tāmasa, [both in his] complete {\rm (}\textit{sakala}{\rm )} [form] and [as] formless {\rm (}\textit{vikala}{\rm )} Īśvara. \blankfootnote{9.5 My altering the reading \textit{viṣṇu} to \textit{viṣṇū} in \textit{pāda} a against all witnesses may
  be regarded as an overcorrection and the stem form could be original, but
  compare \BRAHMANDAPUR\ 1.4.6cd {\rm (}in the apparatus{\rm )}.
 My translation of \textit{pāda}s cd is tentative. I suspect that \textit{pāda} d is one single compound,
  the \textit{anusvāra} is only inserted to avoid the metric fault of two \textit{laghu} syllables 
  at the second and third position.
  I understand \textit{vikala} as a synonym of \textit{niṣkala}. For the tantric connotations of the pair
  \textit{sakala-niṣkala} see, e.g., \TAKIII\ s.v. \textit{niṣkala}.
 }}

  \maintext{sattvaṃ kundenduvarṇābhaṃ padmarāganibhaṃ rajaḥ |}%

  \maintext{tamaś cāñjanaśailābhaṃ kīrtitāni manīṣibhiḥ }||\thinspace9:6\thinspace||%
\translation{Sattva is of the colour of jasmine and the moon. Rajas is of the colour of ruby. Tamas is of the colour of lamp-black and colliryum. [This is how the colours of the qualities] are taught by the wise. }

  \maintext{sattvaṃ jalaṃ rajo 'ṅgāraṃ tamo dhūmasamākulam |}%

  \maintext{etadguṇamayair baddhāḥ pacyante sarvadehinaḥ }||\thinspace9:7\thinspace||%
\translation{Sattva is water, Rajas is charcoal, Tamas is filled with smoke. All living creature are being cooked by [i.e. on the fire produced from] these qualities {\rm (}\textit{guṇa}{\rm )}. }

  \maintext{vigatarāga uvāca |}%

  \maintext{kena kena prakāreṇa guṇapāśena badhyate |}%

  \maintext{cihnam eṣāṃ pṛthaktvena kathayasva tapodhana }||\thinspace9:8\thinspace||%
\translation{Vigatarāga spoke: By what sort of nooses of the qualities {\rm (}\textit{guṇa}{\rm )} is [a person] bound? Teach me the signs connected to them one by one, O great ascetic. }

  \maintext{anarthayajña uvāca |}%

  \maintext{anekākārabhāvena badhyante guṇabandhanaiḥ |}%

  \maintext{mohitā nābhijānanti jānanti śivayoginaḥ }||\thinspace9:9\thinspace||%
\translation{Anarthayajña spoke: [Living beings] are bound in many ways and by many conditions by the fetters of the qualities {\rm (}\textit{guṇa}{\rm )}. Those who are deluded do not know. The Śivayogins do know. }

  \maintext{ūrdhvaṃgo nityasattvastho madhyago rajasāvṛtaḥ |}%

  \maintext{adhogatis tamo'vasthā bhavanti puruṣādhamāḥ }||\thinspace9:10\thinspace||%
\translation{He who is always established in Sattva goes upwards. He who is covered with Rajas goes in the middle. Those lowest of men in the state of Tamas go downward. \blankfootnote{9.10 Understand \textit{adhogatis} in \textit{pāda} c as a \textit{bahuvrīhi} in the plural {\rm (}\textit{adhogatayas}{\rm )}.
 }}

  \maintext{svarge 'pi hi trayo vaite bhāvanīyās tapodhana |}%

  \maintext{mānuṣeṣu ca tiryeṣu guṇabhedās trayas trayaḥ }||\thinspace9:11\thinspace||%
\translation{These three kinds of [\textit{guṇa}s] are to be acknowledged even in heaven, O great ascetic, and among humans, and also among animals. }

  \subsubchptr{sāttvikottamāḥ}%

  \trsubsubchptr{Superior Sattva-type}%

  \maintext{brahmā viṣṇuś ca rudraś ca dharma indraḥ prajāpatiḥ |}%

  \maintext{somo 'gnir varuṇaḥ sūryo daśa sattvottamāḥ smṛtāḥ }||\thinspace9:12\thinspace||%
\translation{The ten superior Sattva [beings] are: Brahmā, Viṣṇu, Rudra, Dharma, Indra, Prajāpati, Soma, Agni, Varuṇa, and Sūrya. \blankfootnote{9.12 Note that Brahmā was labelled as Rajas-type in 9.5b above.
 }}

  \subsubchptr{sāttvikamadhyamāḥ}%

  \trsubsubchptr{Middle Sattva-type}%

  \maintext{rudrādityā vasusādhyā viśveśamaruto dhruvaḥ |}%

  \maintext{ṛṣayaḥ pitaraś caiva daśaite sattvamadhyamāḥ }||\thinspace9:13\thinspace||%
\translation{The ten middle-ranking Sattva [beings] are: Rudras, Ādityas, Vasus,\linebreak Sādhyas, Viśveśa, the Maruts, Dhruva, the sages, and the ancestors. \blankfootnote{9.13 \textit{Pāda} a is a \textit{sa-vipulā}. Note that there seems to be only nine names/categories listed here unless
  we try to interpret \textit{viśveśa} as \textit{viśvedevāḥ} and \textit{īśaḥ}.
 }}

  \subsubchptr{sāttvikādhamāḥ}%

  \trsubsubchptr{Low Sattva-type}%

  \maintext{tārā grahāḥ surā yakṣā gandharvāḥ kiṃnaroragāḥ |}%

  \maintext{rakṣobhūtapiśācāś ca daśaite sāttvikādhamāḥ }||\thinspace9:14\thinspace||%
\translation{The ten low-ranking Sattva [beings] are the stars, the planets, the Suras, the Yakṣas, the Gandharvas, the Kiṃnaras, the Serpents, the Rakṣases, the Ghosts, and the Piśācas. }

  \subsubchptr{rājasottamāḥ}%

  \trsubsubchptr{Superior Rajas-type}%

  \maintext{ṛtvik purohitācāryayajvāno 'tithi vijñanī |}%

  \maintext{rājā mantrī vratī vedī daśaite rājasottamāḥ }||\thinspace9:15\thinspace||%
\translation{The ten superior Rājasa [categories] are Ṛtvij priests, domestic Purohita chaplains, teachers, sacrificers, guests, the wise, kings, ministers, people engaged in religious observances, and [Brahmins] who know the Vedas. \blankfootnote{9.15 I take \textit{'tithi} as a stem form noun and \textit{vijñanī} as \textit{vijñānī}, both metri causa.
 \textit{rājamantrī} as `minister' makes sense, but by emendading \textit{rāja}° to \textit{rājā} 
  in \textit{pāda} c I aim to arrive at a list of ten categories instead of nine.
 }}

  \subsubchptr{rājasamadhyamāḥ}%

  \trsubsubchptr{Middle Rajas-type}%

  \maintext{sūto 'mbaṣṭhavaṇiś cograḥ śilpikārukamāgadhāḥ |}%

  \maintext{veṇavaidehakāmātyā daśaite rajamadhyamāḥ }||\thinspace9:16\thinspace||%
\translation{The ten middle-ranking Rājasa [categories] are [the following castes and professions]: Sūta [coachman/bard], Ambaṣṭha [doctor], Vaṇij [merchant caste], Ugra [combatant?], Śilpin and Kāruka [both artisans], Māgadha [bard], Veṇa [musician], Vaidehaka [guard], and Āmātya [counsellor]. \blankfootnote{9.16 Since all the wittnesses consulted treat \textit{vaṇi} as an acceptable stem in \textit{pāda} a,
  I have refrained from correcting it to \textit{vaṇij/vaṇik}. The English equivalents that
  I give in square brackets are in some cases not more than traditionally accepted guesses.
 }}

  \subsubchptr{rājasādhamāḥ}%

  \trsubsubchptr{Low Rajas-type}%

  \maintext{carmakṛt kumbhakṛt kolī lohakṛt trapunīlikāḥ |}%

  \maintext{naṭamuṣṭikacaṇḍālā daśaite rajasādhamāḥ }||\thinspace9:17\thinspace||%
\translation{The ten low-ranking Rājasa [professions] are: leathersmith, potter, Kolī, blacksmith, tinsmith, dyer, dancer, goldsmith, Caṇḍāla. \blankfootnote{9.17 Problems with this verse include the following. There are only nine 
  professions/castes listed here instead of the expected ten.
  \textit{kolī} is difficult to interpret; later texts of the Jātiviveka
  genre such as Gopinātha's \Jativiveka\ 
  {\rm (}see \mycite{OHanlonHidasKiss}{\rm )} mention \textit{kolī} as a regional 
  name for the caste Niṣāda {\rm (}sometimes: a falconer{\rm )}. I take \textit{trapu} tentatively as \textit{trapukṛt}
  although I cannot see any attestation of that form. And taking \textit{nīlikā} as a {\rm (}female{\rm )} dyer
  is again tentative.
 }}

  \subsubchptr{tāmasottamāḥ}%

  \trsubsubchptr{Superior Tamas-type}%

  \maintext{gogajagavayā aśvamṛgacāmarakiṃnarāḥ |}%

  \maintext{siṃhavyāghravarāhāś ca daśaite tāmasottamāḥ }||\thinspace9:18\thinspace||%
\translation{These are the ten superior Tāmasa [creatures]: cows, elephants, Gayal oxen, horses, deer, Yaks, Kiṃnaras, lions, tigers, and wild boar. \blankfootnote{9.18 Note that Kiṃnaras have already appeared in another category in 9.14 above.
 }}

  \subsubchptr{tāmasamadhyamāḥ}%

  \trsubsubchptr{Middle Tamas-type}%

  \maintext{ajameṣamahiṣyāś ca mūṣikānakulādayaḥ |}%

  \maintext{uṣṭraraṅkuśaśagaṇḍā daśaite tamamadhyamāḥ }||\thinspace9:19\thinspace||%
\translation{The ten middle-ranking Tāmasa [animals] are: goats, sheep, buffaloes, mice, mongooses etc., camels, Raṅku deer, hares, and rhinoceroses. \blankfootnote{9.19 °\textit{mahiṣyāś} seems to be an equivalent of °\textit{mahiṣāś} metri causa. 
  Again, we expect ten items in this list but we find only nine.
 \textit{Pāda} c is a \textit{sa-vipulā}.
 }}

  \subsubchptr{tāmasādhamāḥ}%

  \trsubsubchptr{Low Tamas-type}%

  \maintext{ṛkṣagodhāmṛgaśṛṅgibakavānaragardabhāḥ |}%

  \maintext{sūkaraśvānagomāyur daśaite tāmasādhamāḥ }||\thinspace9:20\thinspace||%
\translation{The ten low-ranking Tāmasa [beings] are: bears, alligators, deer, horned animals, cranes, apes, donkeys, boar, dogs, and frogs. \blankfootnote{9.20 \textit{Pāda} a is a \textit{sa-vipulā}. Translating \textit{śṛṅgi}, \textit{śṛṅgin}, or perhaps \textit{śṛṅgī} as 
  `horned animals' is not much more than a guess. Other possibilities such as `elephants' 
  or simply `bulls' are less attractive because we have had them above in other categories, 
  although repetitions do occur across, and sometimes within, these lists:
  see, e.g., \textit{mṛga} mentioned both in 9.18 and 20, \textit{śyena} in both 9.21 and 22, 
  and \textit{śuka} repeated in 9.21.
 }}

  \subsubchptr{tamasāttvikāḥ}%

  \trsubsubchptr{The Tamas-Sattva category}%

  \maintext{krauñcahaṃsaśukaśyenabhāsabāruṇḍasārasāḥ |}%

  \maintext{cakrāhvaśukamāyūrā daśaite tamasāttvikāḥ }||\thinspace9:21\thinspace||%
\translation{The ten Tāmasa-Sāttvika [beings] are: curlews, geese, parrots, falcons, vultures, B[h]āruṇḍa birds, cranes, Cakra[vāka] birds, parrots, and peacocks. \blankfootnote{9.21 Although all the manuscripts consulted read \textit{kroñca}° in \textit{pāda} a, I have decided
  to accept \Ed's standard spelling in this case. In \textit{pāda} b, I left °\textit{bāruṇḍa}°
  thus, although what is really meant is probably \textit{bhāraṇḍa}, \textit{bhāruṇḍa} or \textit{bhuruṇḍa}.
 Note the repetition of \textit{śuka} in this stanza.
 }}

  \subsubchptr{tamarājasāḥ}%

  \trsubsubchptr{The Tamas-Rajas category}%

  \maintext{balākāḥ kukkuṭāḥ kākāś cillalāvakatittirāḥ |}%

  \maintext{gṛdhrakaṅkabakaśyena daśaite tamarājasāḥ }||\thinspace9:22\thinspace||%
\translation{The ten Tāmasa-Rājasa [beings] are: Balāka-cranes, cocks, crows, Bengal kites, painted quails, partridges, vultures, herons, Bakas and hawks. \blankfootnote{9.22 It would be easy to correct the stem form °\textit{śyena} in \textit{pāda} c to \textit{śyenā} {\rm (}plural{\rm )} but I suspect
  that the form could be original, possibly because it was confused with an instrumental.
 }}

  \subsubchptr{tāmasādhamādi}%

  \trsubsubchptr{Low Tamas-type etc.}%

  \maintext{kokilolūkakañjalyakapotāḥ pañca eva ca |}%

  \maintext{śārikāś ca kuliṅgāś ca daśaite tamasādhamāḥ }||\thinspace9:23\thinspace||%
\translation{The ten lowest Tāmasa [beings also include]: cuckoos, owls, Kañjala-birds, doves, and the five[?], Śārika birds and sparrows. \blankfootnote{9.23 My impression is that the reading °\textit{kiñjalka}° {\rm (}usually: `the filament of a lotus'{\rm )} in \textit{pāda} a
  is either a mistake for, or rather an altered form metri causa, maybe a regional form, 
  of \textit{kañjala} {\rm (}a kind of bird{\rm )}. \msCa\msCc\msNa\ {\rm (}\textit{kiñjalya}{\rm )} may be slightly closer 
  to the required form {\rm (}\textit{kañjalaka}/\textit{kañjalka}?{\rm )}. My emendation is a compromise.
  Note that there are only six items in this list and that \textit{pāda} b is 
  difficult to make sense of in this context. Something must have gone wrong here.
 }}

  \maintext{makaragohanakrāś ca ṛkṣāś ca tamasāttvikāḥ |}%

  \maintext{kacchapaśiśukumbhīramaṇḍūkās tamarājasāḥ |}%

  \maintext{śaṅkhaśuktikaśambūkāḥ kavayyas tamatāmasāḥ }||\thinspace9:24\thinspace||%
\translation{Makara crocodiles, cow-killing alligators and bears are of Tamas-Sattva. Tortoises, porpoises, crocodiles of the Ganges and frogs are of Tamas-Rajas. Conch-shells, pearl-oysters, shells, and Kavayī fish are Tamas-Tāmasa. \blankfootnote{9.24 Note the two \textit{laghu}s in \textit{pāda} a. 
  The reading that yields `and bears' {\rm (}\textit{ṛkṣāś ca}{\rm )} is my conjecture
  for a problematic \textit{ṛṣā ca}. It is far from satisfactory since bears have already appeared in 
  verse 9.20 above.
 My emendation of the word \textit{śuśu} to \textit{śisu} {\rm (}`porpoise,' for \textit{śiśuka} or \textit{śiśumāra}, lit. 
  `child-killer'{\rm )} in \textit{pāda} c is based on the fact that, most probably,
  we need an aquatic animal here, rather than a hare {\rm (}\textit{śaśa}{\rm )}.
 The readings \textit{kabandhyās} and \textit{kabanas} in \textit{pāda} f make no sense. I conjecture \textit{kavayyas} {\rm (}the plural of
  \textit{kavayī}{\rm )}, which is a type of fish. See them mentioned in \MAHASUBHS\ 388:
  \textit{ajājījambāle rajasi maricānāṃ ca luṭhitāḥ
  kaṭutvād uṣṇatvāj janitarasanauṣṭhavyatikarāḥ\thinspace |
  anirvāṇotthena prabalataratailāktatanavo 
  mayā sadyo bhṛṣṭāḥ katipayakavayyaḥ kavalitāḥ\thinspace ||}.
  See a translation of this verse in the \MAHASUBHS\ {\rm (}ed. Sternbach, vol. 1, p. 67{\rm )}:
  `I rolled them in a cumin swamp / and in a heap of pepper dust / till they were spiced and hot enough /
  to twist your tongue and mouth. / When they were basted well with oil, / I didn't wait to wash or sit ; /
  I gobbled that mess of \textit{koji} fish / as soon as they were fried. {\rm (}D.H.H. Ingalls's translation{\rm )}.'
 }}

  \maintext{candanāgarupadmaṃ ca plakṣodumbarapippalāḥ |}%

  \maintext{vaṭadāruśamībilvā daśaite tamasāttvikāḥ }||\thinspace9:25\thinspace||%
\translation{Sandalwood, aloeswood, lotus, waved-leaf fig-tree, Ficus Glomerata, holy fig-tree, Banyan, Devadāru tree, Śamī tree, wood-apple tree: these ten are Tamas-Sattva. \blankfootnote{9.25 In \textit{pāda} d, \textit{tamas}° or \textit{tamaḥ}° are unmetrical and might be the result
  of scribal correction. The original may have been the metrical \textit{tama}°, here
  transmitted only in \Ed. Cf. 9.27d.
 }}

  \maintext{jāmbīralakucāmrātadāḍimākolavetasāḥ |}%

  \maintext{nimbanīpo {\rm †}dhravāvaś ca{\rm †} daśaite tamarājasāḥ }||\thinspace9:26\thinspace||%
\translation{The ten Tamas-Rajas [trees] are: Citron trees, bread-fruit trees, hog-plum trees, pomegranate trees, jujube trees, rattan trees, Neemb trees, Kadamba trees and ... \blankfootnote{9.26 There seems to be only nine items here instead of the expected ten. I have not been able
  to interpret the last one, \textit{dhravāvaś}.
 }}

  \maintext{vṛkṣavallīlatāveṇutvaksāratṛṇabhūruhāḥ |}%

  \maintext{mīrajāś ca śilāśasyā daśaite tamasāttvikāḥ }||\thinspace9:27\thinspace||%
\translation{Trees, creepers, winding plants, cane, bamboo, grass, plants, seaweed, rocks, grains are the ten Tamas-Sattva ones. }

  \maintext{bhramarādipataṅgāś ca krimikīṭajalaukasaḥ |}%

  \maintext{yūkoddaṃśamaśānāṃ ca viṣṭhājās tamasāttvikāḥ }||\thinspace9:28\thinspace||%
\translation{Bees, butterflies etc., worms, insects, aquatic animals, lice, bugs, mosquitoes, creatures in f\ae ces are Tamas-Sattva ones. \blankfootnote{9.28 \textit{ādi} in \textit{pāda} a is misplaced, in order to avoid the metrical fault of 
  two \textit{laghu} syllables in the second and third syllables; understand \textit{bhramarapataṅgādayaś ca}.
 }}

  \maintext{dayā satyaṃ damaḥ śaucaṃ jñānaṃ maunaṃ tapaḥ kṣamā |}%

  \maintext{śīlaṃ ca nābhimānaṃ ca sāttvikāś cottamā janāḥ }||\thinspace9:29\thinspace||%
\translation{[These ten words describe] people who are the best among the Sāttvika [type]: compassion, truthfulness, self-control, purity, knowledge, observing silence, penance, patience, integrity, lack of self-conceit. }

  \maintext{kāmatṛṣṇāratidyūtamāno yuddhaṃ madaḥ spṛhā |}%

  \maintext{nirghṛṇāḥ kalikartāro rājaseṣūttamā janāḥ }||\thinspace9:30\thinspace||%
\translation{[These ten words describe] the people who are the best among the Rājasa [type]: desire, thirst, pleasure, gambling, arrogance, fight, intoxication, delight, cruel, quarrelling. }

  \maintext{hiṃsāsūyāghṛṇāmūḍhanidrātandrībhayālasāḥ |}%

  \maintext{krodho matsaramāyī ca tāmaseṣūttamā janāḥ }||\thinspace9:31\thinspace||%
\translation{[These words describe] people who are the best among the Tāmasa [type]: violence, envy, incompassionate, stupid, sleepy, lazy, cowardly, idle, anger, greedy, cheating. }

  \maintext{laghuprītiprakāśī ca dhyānayoge sadotsukaḥ |}%

  \maintext{prajñābuddhivirāgī ca sāttvikaṃ guṇalakṣaṇam }||\thinspace9:32\thinspace||%
\translation{The Sāttvika can be characterised as follows: light, joyful, bright, always eager for yoga meditation, wise, intelligent and dispassionate. }

  \maintext{bālako nipuṇo rāgī māno darpaś ca lobhakaḥ |}%

  \maintext{spṛhā īrṣā pralāpī ca rājasaṃ guṇalakṣaṇam }||\thinspace9:33\thinspace||%
\translation{The Rājasa can be characterised as follows: childish, skilful, passionate, proud, arrogant, greedy, desirous, jealous, and chattering. }

  \maintext{udvega ālaso mohaḥ krūras taskaranirdayaḥ |}%

  \maintext{krodhaḥ piśuna nidrā ca tāmasaṃ guṇalakṣaṇam }||\thinspace9:34\thinspace||%
\translation{The Tāmasa can be characterised as follows: anxious, lazy, deluded, cruel, a thief, pitiless, angry, wicked and sleepy. \blankfootnote{9.34 In \textit{pāda} a, \textit{piśuno} could be the right choice: the \textit{pāda} can be a 
  {\rm (}slightly wrong{\rm )} \textit{ra-vipulā} if \textit{dr} in \textit{nidrā} does not make the previous syllable long, 
  a licence often occuring in this text {\rm (}`muta cum liquida'{\rm )}.
 }}

  \subsubchptr{āhāras traiguṇye}%

  \trsubsubchptr{Food and the three qualities}%

  \maintext{vigatarāga uvāca |}%

  \maintext{kena cihnena vijñeya āhāraḥ sarvadehinām |}%

  \maintext{traiguṇyasya pṛthaktvena kathayasva tapodhana }||\thinspace9:35\thinspace||%
\translation{Vigatarāga spoke: By what signs can the food of each [category of] humans be characterised? Teach me with regards to the three qualities {\rm (}\textit{guṇa}{\rm )}, O great ascetic. }

  \maintext{anarthayajña uvāca |}%

  \maintext{āyuḥ kīrtiḥ sukhaṃ prītir balārogyavivardhanam |}%

  \maintext{hṛdyasvādurasaṃ snigdha āhāraḥ sāttvikapriyaḥ }||\thinspace9:36\thinspace||%
\translation{Anarthayajña spoke: The Sāttvikas prefer food that yields [long] life, fame, happiness, joy, which increases strength and health, which is savoury and which tastes nice, and which is soft. }

  \maintext{atyuṣṇam āmlalavaṇaṃ rūkṣaṃ tīkṣṇaṃ vidāhi ca |}%

  \maintext{rājasaśreṣṭha-āhāro duḥkhaśokāmayapradaḥ }||\thinspace9:37\thinspace||%
\translation{The best food for the Rājasas is rather warm, acidic, salty, hard, hot and pungent. It gives you pain, a burning sensation and indigestion. \blankfootnote{9.37 Note the lack of sandhi within what was meant to be a compund in \textit{pāda} c {\rm (}understand
  \textit{rājaśreṣṭhāhāro}{\rm )}, and the total lack of gender agreement between the adjectives in \textit{pāda}s ab, and
  \textit{āhāro} and \textit{pradaḥ}.
 }}

  \maintext{abhakṣyāmedhyapūtī ca pūti paryuṣitaṃ ca yat |}%

  \maintext{āmayārasavisvāda āhāras tāmasapriyaḥ }||\thinspace9:38\thinspace||%
\translation{Tāmasas prefer food that is prohibited, impure and foul-smelling, stinky and stale. It causes indigestion, is sapless and tasteless. \blankfootnote{9.38 Understand °\textit{pūtī} \textit{in} pāda a as standing for °\textit{pūti} metri causa {\rm (}which is oddly repeated in 
  \textit{pāda} b{\rm )}, and note that °āmedhya° in the same \textit{pāda} is an emendation {\rm (}correcting \msNc's reading{\rm )}.
 I have conjectured \textit{āmayārasa}° for \textit{āyāmarasa}° in \textit{pāda} c because the transmitted readings
  make little sense and because \textit{āmaya} appeared in 9.37d above.
 }}

  \subsubchptr{guṇātītam}%

  \trsubsubchptr{Beyond the qualities}%

  \maintext{vigatarāga uvāca |}%

  \maintext{guṇātītaṃ kathaṃ jñeyaṃ saṃsāraparapāragam |}%

  \maintext{guṇapāśanibaddhānāṃ mokṣaṃ kathaya tattvataḥ }||\thinspace9:39\thinspace||%
\translation{Vigatarāga spoke: How can one recognize [the state of getting] beyond the \textit{guṇa}s, which leads one to the other shore of [the ocean] of mundane existence? Tell me truly about the liberation of those who are bound by the noose of the \textit{guṇa}s. }

  \maintext{anarthayajña uvāca |}%

  \maintext{ātmavat sarvabhūtāni samyak paśyeta bho dvija |}%

  \maintext{guṇātītaḥ sa vijñeyaḥ saṃsāraparapāragaḥ }||\thinspace9:40\thinspace||%
\translation{Anarthayajña spoke: Well, he who looks at all living beings in the correct way, as his own Self, O Brahmin, is to be known as one beyond the qualities {\rm (}\textit{guṇa}{\rm )}, as one who has departed to the other shore of [the ocean of] mundane existence. \blankfootnote{9.40 Note passages in the \BHG\ {\rm (}6.32, 12.13, 14.24--25{\rm )}
  quoted in the apparatus to the critical edition, of which \VSS\ 9.40--42 seem
  to be echoes of.
 }}

  \maintext{īrṣādveṣasamo yas tu sukhaduḥkhasamāś ca ye |}%

  \maintext{stutinindāsamā ye ca guṇātītaḥ sa ucyate }||\thinspace9:41\thinspace||%
\translation{He who is indifferent to envy and hate, treats happiness and sorrow as equal, treats praise and reproach as equal, is called `one who is beyond the qualities {\rm (}\textit{guṇa}{\rm )}'. }

  \maintext{tulyapriyāpriyo yaś ca arimitrasamas tathā |}%

  \maintext{mānāpamānayos tulyo guṇātītaḥ sa ucyate  }||\thinspace9:42\thinspace||%
\translation{He who treats pleasant and unpleasant things, enemy and friend, respect and contempt equally, is called `one who is beyond the qualities {\rm (}\textit{guṇa}{\rm )}'. }

  \maintext{eṣa te kathito vipra guṇasadbhāvanirṇayaḥ |}%

  \maintext{guṇayuktas tu saṃsārī guṇātītaḥ parāṅgatiḥ }||\thinspace9:43\thinspace||%
\translation{O Brahmin, thus has the exposition of the essence of the qualities {\rm (}\textit{guṇa}{\rm )} been taught to you. Those who are connected with the qualities {\rm (}\textit{guṇa}{\rm )} are mundane {\rm (}\textit{saṃsārin}{\rm )}, those beyond the qualities {\rm (}\textit{guṇa}{\rm )} are on the supreme path. }

\centerline{\maintext{\dbldanda\thinspace iti vṛṣasārasaṃgrahe traiguṇyaviśeṣaṇīyo nāmādhyāyo navamaḥ\thinspace\dbldanda}}
\translation{Here ends the ninth chapter in the \textit{Vṛṣasārasaṃgraha} called the Particulars of the Three Guṇas.}
