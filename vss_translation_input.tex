
  \chptr{ṣaṣṭho 'dhyāyaḥ}
\addcontentsline{toc}{section}{Chapter 6}
\fancyhead[CO]{{\footnotesize\textit{Translation of chapter 6}}}%

  \trchptr{ Chapter Six }%

  \subchptr{niyameṣv ijyā {\rm {\rm (}2{\rm )}}}%

  \trsubchptr{Second Niyama-rule: sacrifice}%

  \maintext{atha pañcavidhām ijyāṃ pravakṣyāmi dvijottama |}%

  \maintext{dharmamokṣaprasiddhyarthaṃ śṛṇuṣvāvahito dvija }||\thinspace6:1\thinspace||%
\translation{[Anarthayajña continued:] Now I shall teach you the five types of sacrifice {\rm (}\textit{ijyā}{\rm )}, O excellent Brahmin, for success in Dharma and liberation. Listen carefully, O Brahmin. }

  \maintext{arthayajñaḥ kriyāyajño japayajñas tathaiva ca |}%

  \maintext{jñānaṃ dhyānaṃ ca pañcaitat pravakṣyāmi pṛthak pṛthak }||\thinspace6:2\thinspace||%
\translation{Material sacrifice, sacrifice through work, sacrifice through recitation, knowledge and meditation: I shall teach you these five one by one. \blankfootnote{6.2 Note the singular \textit{etat} after a number {\rm (}see Intro \verify{\rm )}.
 
  Compare this list of five to the somewhat similar \BHG\ 4.28:
  \textit{dravyayajñās tapoyajñā yogayajñās tathāpare\thinspace |
  svādhyāyajñānayajñāś ca yatayaḥ saṃśita-vratāḥ\thinspace ||}.
  \SDHU\ chapter 3 can be also relevant since it uses the terms
  \textit{japayajña}, \textit{jñānayajña}, and \textit{dhyānayajña}. See also \SDHU\ 1.10 {\rm (}\msCa\ f.\thinspace 42v l4{\rm )}:
  \textit{karmayajñas tapoyajñaḥ svādhyāyo dhyānam eva ca\thinspace | 
  jñānayajñaś ca pañcaite mahāyajñāḥ prakīrtitāḥ\thinspace ||}.
  Note how this definition of the five \textit{mahāyajña}s in the \SDHU\ 
  is different from the one, e.g., in \MANU\ 3.69--71
  {\rm (}\textit{brahma}°, \textit{pitṛ}°, \textit{daiva}°, \textit{bhauta}°, and \textit{nṛyajña}{\rm )}.
 }}

  \subsubchptr{arthayajñaḥ}%

  \trsubsubchptr{Material sacrifice}%

  \maintext{agnyupāsanakarmādi agnihotrakratukriyā |}%

  \maintext{aṣṭakā pārvaṇī śrāddhaṃ dravyayajñaḥ sa ucyate }||\thinspace6:3\thinspace||%
\translation{Material sacrifice includes the following: the domestic ritual fire worship etc., the public performance of the ritual of Agnihotra, [and the so-called \textit{pākayajña}s such as] the Aṣṭakā oblation, the Pārvaṇī oblation, and the ancestral ritual {\rm (}\textit{śrāddha}{\rm )}. \blankfootnote{6.3 By somewhat overtranslating the items in this list, I want to emphasise that
  the text introduces three categories of sacrifical rituals well-known from
  the time of the Gṛhyasūtras and Śrautasūtras: those of the domestic or \textit{aupāsana} fire {\rm (}\textit{gṛhyakarman}{\rm )},
  the Śrauta rituals such as the Agnihotra, and the Smārta \textit{pākayajña}s, such as the \textit{aṣṭakā}, 
  the \textit{pārvaṇī} and the \textit{śrāddha}. For a mention of the \textit{pākayajña}s in a manner similar to 
  our \textit{pāda}s cd here, see, e.g. the \Diksottara\ quoted in \mycitep{NisvasaGoodall}{275}:
  \textit{aṣṭakāḥ pārvaṇī śrāddhaṃ śrāvaṇy āgrāyaṇī tathā\thinspace |
  caitrī cāśvayujī caiva pākayajñāḥ prakīrtitāḥ\thinspace ||}.
  For an earlier list of \textit{pākayajña}s, see \GAUTDHS\ 1.8.19: 
  \textit{aṣṭakā pārvaṇaḥ śrāddham śrāvaṇy\-āgrahāyaṇī\-caitry\-āśvayujīti sapta pākayajñasamsthāḥ}.
 }}

  \subsubchptr{kriyāyajñaḥ}%

  \trsubsubchptr{Sacrifice through work}%

  \maintext{ārāmodyānavāpīṣu devatāyataneṣu ca |}%

  \maintext{svahastakṛtasaṃskāraḥ kriyāyajña sa ucyate }||\thinspace6:4\thinspace||%
\translation{Sacrifice through work is taking care of/ cleaning/ embellishing {\rm (}\textit{saṃskāra}{\rm )} a grove, a park, a pond or a temple with one's own hands. }

  \subsubchptr{japayajñaḥ}%

  \trsubsubchptr{Sacrifice through recitation}%

  \maintext{japayajñaṃ tato vakṣye svargamokṣaphalapradam |}%

  \maintext{vedādhyayana kartavyaṃ śivasaṃhitam eva ca |}%

  \maintext{itihāsapurāṇaṃ ca japayajñaḥ sa ucyate }||\thinspace6:5\thinspace||%
\translation{Next I shall teach you the sacrifice through recitation, the bestower of the fruits of heaven and liberation. One should recite the Vedas, Śaiva texts or the \textit{Mahābhārata}, the epics and the Purāṇas: this is called sacrifice with recitation. \blankfootnote{6.5 Note the stem form \textit{vedādhyayana} in \textit{pāda} c metri causa. As for the interpretation of
  \textit{śivasaṃhitam} in \textit{pāda} d, see 5.17b above: \textit{śaivabhāratasaṃhite}. 
  The proximity of these two phrases, and the fact that both give instructions
  on using texts, suggest that we should interpret them similarly. 
  It is then a \textit{samāhāra\-dvandva\-samāsa} again, in the neuter.
  Both \textit{śivasaṃhitam} and \textit{itihāsapurāṇaṃ} should be interpreted as
  being part of the compound in \textit{pāda} c: \textit{śiva\-saṃhitādhyayanaṃ} and 
  \textit{itihāsapurāṇādhyayanaṃ}. 
 
  See \textit{japayajña} mentioned, e.g., in \BHG\ 10.25c {\rm (}\textit{yajñānāṃ japayajño 'smi}{\rm )} 
  and \MANU\ 2.86 {\rm (}\textit{vidhiyajñāj japayajño viśiṣṭo daśabhir guṇaiḥ}{\rm )}.
 }}

  \subsubchptr{jñānayajñaḥ}%

  \trsubsubchptr{Sacrifice through knowledge}%

  \maintext{idaṃ karma akarmedam ūhāpohaviśāradaḥ |}%

  \maintext{śāstracakṣuḥ samālokya jñānayajñaḥ sa ucyate }||\thinspace6:6\thinspace||%
\translation{[He who can decide if] `this is [proper] action; the other is improper action' because he is knowledgeable about reasoning pro and contra, and investigates with his eyes on the Śāstras, is called [a person performing] sacrifice through knowledge. \blankfootnote{6.6 For the expression \textit{śāstracakṣuḥ}, see, e.g., \BRAHMAP\ 24.21:
  \textit{tena yajñān yathāproktān mānavāḥ śāstracakṣuṣaḥ\thinspace |
  kurvate 'harahaś caiva devān āpyāyayanti te\thinspace ||}.
  In G. P. Bhatt's translation {\rm (}\mycitep{BrahmapuranaTr1}{126}{\rm )}:
  `Day by day men with the sacred scriptures as their guides
  perform sacrifices in the manner they have been laid down and thereby nourish the gods.'
 }}

  \subsubchptr{dhyānayajñaḥ}%

  \trsubsubchptr{Sacrifice through meditation}%

  \maintext{dhyānayajñaṃ samāsena kathayiṣyāmi te śṛṇu |}%

  \maintext{dhyānaṃ pañcavidhaṃ caiva kīrtitaṃ hariṇā purā |}%

  \maintext{sūryaḥ somo 'gni sphaṭikaḥ sūkṣmaṃ tattvaṃ ca pañcamam }||\thinspace6:7\thinspace||%
\translation{I shall teach you concisely about sacrifice through meditation. Listen to me. Meditation was taught by Hari in the past as of five kinds. [Meditation on] the Sun, the Moon, Fire, Crystal and the subtle \textit{tattva} as fifth. \blankfootnote{6.7 For an analysis of this fivefold method of meditation, and this ancient-looking
  \textit{tattva}-system, see Intro \verify, and for different
  versions of the same teaching of meditation, see \VSS\ 22.19--28 and \DHARMP\ 4.5--14.
 }}

  \maintext{sūryamaṇḍalam ādau tu tattvaṃ prakṛtir ucyate |}%

  \maintext{tasya madhye śaśiṃ dhyāyet tattvaṃ puruṣa ucyate }||\thinspace6:8\thinspace||%
\translation{First it is the Sun [that should be meditated upon], which is said to be \textit{prakṛti-tattva}. He should visualize the Moon in its centre: that \textit{tattva} is said to be \textit{puruṣa}. \blankfootnote{6.8 Note the form \textit{śaśiṃ} for \textit{śaśinaṃ}.
 }}

  \maintext{candramaṇḍalamadhye tu jvālām agniṃ vicintayet |}%

  \maintext{prabhutattvaḥ sa vijñeyo janmamṛtyuvināśanaḥ }||\thinspace6:9\thinspace||%
\translation{In the centre of the Moon's disk, he should visualise a flame, a fire. That is said to be \textit{prabhu}-\textit{tattva}, the destroyer of [the circle of] birth and death. }

  \maintext{agnimaṇḍalamadhye tu dhyāyet sphaṭika nirmalam |}%

  \maintext{vidyātattvaḥ sa vijñeyaḥ kāraṇam ajam avyayam }||\thinspace6:10\thinspace||%
\translation{In the centre of the ring of Fire, he should visualize a spottless crystal. That is said to be \textit{vidyā}-\textit{tattva}, the never-born, imperishable cause. \blankfootnote{6.10 Note the stem form \textit{sphaṭika} in \textit{pāda} b metri causa.
 }}

  \maintext{vidyāmaṇḍalamadhye tu dhyāyet tattvam anuttamam |}%

  \maintext{akīrtitam anaupamyaṃ śivam akṣayam avyayam |}%

  \maintext{pañcamaṃ dhyānayajñasya tattvam uktaṃ samāsataḥ }||\thinspace6:11\thinspace||%
\translation{In the centre of the disk of \textit{vidyā}, he should visualize the highest \textit{tattva}, never-heard, unparalleled, undecaying and imperishable Śiva. The fifth \textit{tattva} of the sacrifice through meditation has been taught in short. }

  \maintext{vigatarāga uvāca |}%

  \maintext{ekaikasya tu tattvasya phalaṃ kīrtaya kīdṛśam |}%

  \maintext{kāni lokāḥ prapadyante kālaṃ vāsya tapodhana }||\thinspace6:12\thinspace||%
\translation{Vigatarāga spoke: Teach me, what are the fruits of [reaching] each \textit{tattva}? Which worlds can be attained and how much time [can one spend there], O great ascetic? \blankfootnote{6.12 The reading \textit{tritattvasya} in \textit{pāda} a in the MSS is a problem 
  because we have just finished a section mentioning five \textit{tattva}s. 
  {\rm (}This was probably noticed by \Ed, hence printing \textit{hi} for \textit{tri}°.{\rm )}
  My conjecture {\rm (}\textit{tu}{\rm )} is based on the assumption that \textit{tri} is ofter written as \textit{tṛ} 
  in Nepalese MSS {\rm (}e.g. in \msM\ at this point{\rm )} and that \textit{tṛ} may then easily get corrupted to \textit{tu}.
 }}

  \maintext{anarthayajña uvāca |}%

  \maintext{brahmalokaṃ tu prathamaṃ tattvaprakṛticintayā |}%

  \maintext{kalpakoṭisahasrāṇi śivavan modate sukhī }||\thinspace6:13\thinspace||%
\translation{Anarthayajña spoke: Through meditation on the first \textit{tattva}, \textit{prakṛti}, [one reaches] Brahmaloka. He will rejoice [there] happily like Śiva for millions of \ae ons. \blankfootnote{6.13 Understand \textit{pāda}s ab as \textit{brahmalokaṃ prathamatattvacintayā prakṛtitattvacintayā}. 
  One might take \textit{prathamaṃ} adverbially {\rm (}`firstly': \textit{prathamaṃ brahmalokaṃ prakṛtitattvacintayā}{\rm )},
  but in the next verses, the ordinal numbers {\rm (}\textit{dvitīyaṃ, tṛtīyaṃ, pañcamaṃ}{\rm )}
  always refer to the \textit{tattva}s.
 }}

  \maintext{dvitīyaṃ tattva puruṣaṃ dhyāyamāno mṛto yadi |}%

  \maintext{viṣṇulokam ito yāti kalpakoṭyayutaṃ sukhī }||\thinspace6:14\thinspace||%
\translation{If one dies while meditating on the second \textit{tattva}, \textit{puruṣa}, one will depart from this world and go to Viṣṇuloka, [and will dwell there] happily for billions of \ae ons. \blankfootnote{6.14 Note the stem form \textit{tattva} in \textit{pāda} a metri causa.
 }}

  \maintext{prabhutattvaṃ tṛtīyaṃ tu dhyāyamāno mariṣyati |}%

  \maintext{śivaloke vasen nityaṃ kalpakoṭyayutaṃ śatam }||\thinspace6:15\thinspace||%
\translation{Should one die while meditating on the third, the \textit{prabhu-tattva}, one can live in Śivaloka continuously for a hundred billion \ae ons. \blankfootnote{6.15 \Ed\ changes \textit{śivaloka} to \textit{rudraloka}, probably for more contrast with
  \textit{sadāśiva} in 6.16 and \textit{śivatattva} in 6.17. \verify
 }}

  \maintext{vidyātattvāmṛtaṃ dhyāyet sadāśivam anāmayam |}%

  \maintext{akṣayaṃ lokam āpnoti kalpānāntaparaṃ tathā  }||\thinspace6:16\thinspace||%
\translation{If he visualizes the nectar of \textit{vidyā-tattva}, [i.e.] Sadāśiva, he can reach [His] diseaseless, imperishable world [and can live there] well beyond endless \ae ons. \blankfootnote{6.16 In \textit{pāda} a, \textit{amṛta} is suspect. It may refer to the world of Sadāśiva and 
  then \textit{vidyātattva} is in stem form. Alternatively, since this verse is the only one in
  this list of worlds {\rm (}6.13--17{\rm )} without an ordinal number, \textit{amṛtaṃ} may mean `four' or possibly `fourth,'
  as suggested by Monier-Williams and Apte in their dictionaries. This meaning would fit in nicely.
  In addition, dying has been mentioned above, thus \textit{amṛtaṃ} might be a corrupted form of 
  a participle from the verbal root \textit{mṛ} {\rm (}\textit{mṛyan} or \textit{maran}?{\rm )}: e.g., 
  \textit{vidyātattvaṃ mṛyan dhyāyet...} {\rm (}`should he meditation upon Vidyātattva while dying...'{\rm )}.
 }}

  \maintext{pañcamaṃ śivatattvaṃ tu sūkṣmaṃ cātmani saṃsthitam |}%

  \maintext{na kālasaṃkhyā tatrāsti śivena saha modate }||\thinspace6:17\thinspace||%
\translation{The fifth one, the subtle \textit{śiva-tattva} dwells in the Self. There is no counting of time there and he will be rejoicing [there] together with Śiva. }

  \maintext{pañcadhyānābhiyukto bhavati ca na punarjanmasaṃskārabandhaḥ}%

 \nonanustubhindent \maintext{jijñāsyantāṃ dvijendra bhavadahanakaraḥ prārthanākalpavṛkṣaḥ |}%

  \maintext{janmenaikena muktir bhavati kimu na vā mānavāḥ sādhayantu}%

 \nonanustubhindent \maintext{pratyakṣān nānumānaṃ sakalamalaharaṃ svātmasaṃvedanīyam }||\thinspace6:18\thinspace||%
\translation{[If] he practises the five meditations, there will be no rebirth and no more fetters of transmigration. O excellent Brahmin, [the Lord] should be seeked, a wishing tree of desires, [as] he burns away existence. Liberation comes within one single birth! People, why should you not strive [for it]! [It is known] as the destroyer of all impurity. [It's ascertainable] by direct perception. It is not inference. It is to be experienced by one's own Self. \blankfootnote{6.18 Note how a plural passive imperative form {\rm (}\textit{jijñāsyantāṃ}{\rm )} stands for the singular
  {\rm (}\textit{jijñāsyatāṃ}{\rm )} metri causa. Note also that the last syllable of
  \textit{dvijendra} {\rm (}at the c\ae sura{\rm )} counts here as long: this phenomenon of a word-ending
  syllable becoming long by position is common in the \VSS.
 The non-standard \textit{janmena} in \textit{pāda} d seems superior to \textit{janmanā} for it
  preserves the metre.
 }}

  \subchptr{niyameṣu tapaḥ {\rm {\rm (}3{\rm )}}}%

  \trsubchptr{Third Niyama-rule: penance}%

  \maintext{mānasaṃ tapa ādau tu dvitīyaṃ vācikaṃ tapaḥ |}%

  \maintext{kāyikaṃ ca tṛtīyaṃ tu manovākkarma tatparam |}%

  \maintext{kāyikaṃ vācikaṃ caiva tapo miśraka pañcamam }||\thinspace6:19\thinspace||%
\translation{The first type of penance is mental penance, the second is verbal penance, the third is the bodily one, the next one is the one which is [characterised by] both mental and verbal action. The fifth type of penance is a mixture of the bodily and the verbal ones. \blankfootnote{6.19 The reading \textit{manovākkāya}° {\rm (}\msNa\msNb{\rm )} in \textit{pāda} d is probably secondary, influenced by
  such common expressions as, e.g., \textit{manovākkāyakarmabhiḥ} in \YAJNS\ 1.27d.
 Note the stem form \textit{miśraka} in \textit{pāda} f metri causa.
 }}

  \maintext{manaḥsaumyaṃ prasādaś ca ātmanigraham eva ca |}%

  \maintext{maunaṃ bhāvaviśuddhiś ca pañcaitat tapa mānasam }||\thinspace6:20\thinspace||%
\translation{Gentleness of the mind, calmness, self-control, observing silence, and the purification of one's state of mind: mental penance comprises these five. \blankfootnote{6.20 Again, we can see the use of the singular {\rm (}\textit{etat}{\rm )} next to numbers; note also 
  the stem form \textit{tapa} in \textit{pāda} d metri causa. This verse is a paraphrase of \MBH\ 3.39.16 {\rm (}\BHG\ 17.16; see text in the
  apparatus{\rm )}.
 }}

  \maintext{anudvegakarā vāṇī priyaṃ satyaṃ hitaṃ ca yat |}%

  \maintext{svādhyāyābhyasanaṃ caiva vācikaṃ tapa ucyate }||\thinspace6:21\thinspace||%
\translation{Verbal penance is taught as speech that causes no anxiety, which is kind, true and useful, and it includes also the practice of recitation. \blankfootnote{6.21 This verse is a variant of \MBH\ 6.39.15 {\rm (}\BHG\ 17.15; see it in the apparatus{\rm )}.
 }}

  \maintext{ārjavaṃ ca ahiṃsā ca brahmacaryaṃ surārcanam |}%

  \maintext{śaucaṃ pañcamam ity etat kāyikaṃ tapa ucyate }||\thinspace6:22\thinspace||%
\translation{Bodily penance is taught as the following: honesty, harmlessness, chastity, the worship of gods, and purity as the fifth. \blankfootnote{6.22 This verse seems to be a paraphrase of \MBH\ 6.39.14 {\rm (}\BHG\ 17.14; see it in the apparatus{\rm )}.
 }}

  \maintext{iṣṭaṃ kalyāṇabhāvaṃ ca dhanyaṃ pathyaṃ hitaṃ vadet |}%

  \maintext{manomiśraka pañcaitat tapa uktaṃ maharṣibhiḥ }||\thinspace6:23\thinspace||%
\translation{[Penance] which is a mixture of the mental [and the verbal] is taught by the great sages to be these five: he should speak [about things that are] agreeable, of a virtuous character, auspicious, salutary and useful. \blankfootnote{6.23 Note the use of the singular {\rm (}\textit{etat}{\rm )} next to a number and the stem form noun in \textit{pāda} c.
 }}

  \maintext{svasti maṅgalam āśīrbhir atithigurupūjanam |}%

  \maintext{kāyamiśraka pañcaitat tapa uktaṃ mahātmabhiḥ }||\thinspace6:24\thinspace||%
\translation{[Penance] in which bodily [and verbal actions] mix is taught by the great-souled ones to be these five: the worship of the guest and the guru, benediction, greetings, and blessings. \blankfootnote{6.24 See \SDHS\ 11.73--79 {\rm (}and \mycitep{SaivaUtopia2021}{91--93 and 120--121}{\rm )} 
  for a somewhat similar discussion on `kind speach.'
 }}

  \maintext{maṇḍūkayogī hemante grīṣme pañcatapās tathā |}%

  \maintext{abhrāvakāśo varṣāsu tapaḥ sādhanam ucyate }||\thinspace6:25\thinspace||%
\translation{[Being] a [so-called] frog-yogin in the winter, or one with the five fires in the summer, or one who has the clouds [i.e. the open sky] for shelter in the rainy season: these kinds of penance is called \textit{sādhana}. \blankfootnote{6.25 \Manu\ 6.23 mentions three kins of penance that corresponds to three seasons:
  \textit{grīṣme pañcatapās tu syād varṣāsv abhrāvakāśikaḥ\thinspace |
  ārdravāsās tu hemante kramaśo vardhayaṃs tapaḥ\thinspace ||}. 
  Translated in \mycitep{OlivelleManu}{149} as:
  `[He should] surround himself with the five fires in the summer; live in the open air during the rainy season;
  and wear wet clothes in the winter---gradually intensifying his ascetic toil.'
  This and \SDHSAMGR\ 9.32ab {\rm (}quoted in the apparatus{\rm )} may suggest that being 
  a `frog-yogin' could be the same as wearing wet clothes or standing in water for a long time.
  A footnote to verse \MBH\ 12.309.9 in the Kumbakonam edition of the \MBH\ {\rm (}\mycite{MBhKumbakonaEd}{\rm )} suggests otherwise:
  \textit{maṇḍūkavat pāṇipādaṃ saṅkocya nyubjaḥ śete iti maṇḍūkaśāyī}. {\rm (}`The word `frog-sleeper' means
  somebody who sleeps like a frog, with his hands and feet withdrawn and with his back humped.'{\rm )} 
 }}

  \maintext{svamāṃsoddhṛtya dānaṃ ca hastapādaśiras tathā |}%

  \maintext{puṣpam utpādya dānaṃ ca sarve te tapasādhanāḥ }||\thinspace6:26\thinspace||%
\translation{Carving out his own flesh as a donation, or [offering his own] hand, feet and head, or drawing [his own] blood {\rm (}\textit{puṣpa}{\rm )} as a donation: all these are \textit{sādhana}-penances, \blankfootnote{6.26 Note the stem form \textit{svamāṃsa} in \textit{pāda} a for the accusative.
 The translation of \textit{pāda} c is tentative, but taking \textit{puṣpa} as `blood' is not only
  normal e.g. in tantric texts {\rm (}see e.g. \verify{\rm )}, but \VSS\ 17.37--38 suggest the same
  in a similar context:
  \textit{devī uvāca\thinspace |
  svamāṃsarudhiraṃ dānaṃ dānaṃ putrakalatrayoḥ\thinspace |
  kiṃ praśasyaṃ mahādeva tattvaṃ vaktum ihārhasi\thinspace ||
  maheśvara uvāca\thinspace |
  svamāṃsarudhiraṃ dānaṃ praśaṃsanti manīṣiṇaḥ\thinspace |
  śrūyatāṃ pūrvavṛttāni saṃkṣipya kathayāmy aham\thinspace ||}.
  {\rm (}`Devī spoke: Why are one's own flesh and blood and one's son and wife praised as donation, O Mahādeva?
  Tell me the truth please. Maheśvara spoke: The wise praise one's own flesh and blood as donation.
  Let's hear the old legends, I shall tell you briefly.'{\rm )}
 }}

  \maintext{kṛcchrātikṛcchraṃ naktaṃ ca taptakṛcchram ayācitam |}%

  \maintext{cāndrāyaṇaṃ parākaṃ ca tapaḥ sāṃtapanādayaḥ }||\thinspace6:27\thinspace||%
\translation{[as also] the `painful penance' and the `extremely paniful one', [eating only] at night, the `hot and painful' and [the one in which only food obtained] without solicitation [can be eaten], the \textit{cāndrāyaṇa} and \textit{parāka} penances, the \textit{sāṃtapana}, etc. \blankfootnote{6.27 For short descriptions and the loci classici of these penances, see, e.g.,
  \mycitep{KaneHistory}{v. 4, 130--152}.
  For \textit{nakta}/\textit{naktānna} see \VSS\ 8.22 below and, e.g., \SDHS\ chapter 10, and for \textit{ayācita}, \VSS\ 8.23 below.
 }}

  \maintext{yenedaṃ tapa tapyate sumanasā saṃsāraduḥkhacchidam}%

 \nonanustubhindent \maintext{āśāpāśa vimucya nirmalamatis tyaktvā jaghanyaṃ phalam |}%

  \maintext{svargākāṅkṣyanṛpatvabhogaviṣayaṃ sarvāntikaṃ tatphalaṃ}%

 \nonanustubhindent \maintext{jantuḥ śāśvatajanmamṛtyubhavane tanniṣṭhasādhyaṃ vahet }||\thinspace6:28\thinspace||%
\translation{He who performs with a well-disposed mind this penance that puts an end to the suffering caused by transmigration {\rm (}\textit{saṃsāra}{\rm )}, abandoning the trap of hope, with a spotless mind, giving up the lowest rewards [such as] wishing for heaven, being a king and having enjoyments for the senses, will have an ultimate {\rm (}\textit{sarvāntika}{\rm )} reward. In this home of eternal births and deaths, man can bring about an accomplishment that puts an end to them. \blankfootnote{6.28 Note my emendation in \textit{pāda} a {\rm (}\textit{sumanasā} from \textit{sumanasaḥ}{\rm )} and that
  in order to restore the metre, I accepted \Ed's stem form \textit{tapa}.
 Note the stem form °\textit{pāśa} in \textit{pāda} b metri causa.
 }}

\centerline{\maintext{\dbldanda\thinspace iti vṛṣasārasaṃgrahe ṣaṣṭho 'dhyāyaḥ\thinspace\dbldanda}}
\translation{Here ends the sixth chapter in the \textit{Vṛṣasārasaṃgraha}.}
