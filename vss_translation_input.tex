
  \chptr{saptadaśamo 'dhyāyaḥ}
\addcontentsline{toc}{section}{Chapter 17}
\fancyhead[CO]{{\footnotesize\textit{Translation of chapter 17}}}%

  \trchptr{Chapter Seventeen}%

  \subchptr{dānadharmaviśeṣaḥ}%

  \trsubchptr{The particulars of the Dharma of donation}%

  \maintext{devy uvāca |}%

  \maintext{pṛthag dānasya icchāmi śrotuṃ māṃ dātum arhasi |}%

  \maintext{annavastrahiraṇyānāṃ gobhūmikanakasya ca }||\thinspace17:1\thinspace||%
\translation{I wish to hear about [the types of] donation one by one: please donate [this knowledge] to me [of] [donating] food, clothes, gold, cows and gold[?!]. }

  \subsubchptr{annapradānam}%

  \trsubsubchptr{Donation of food}%

  \maintext{bhagavān uvāca |}%

  \maintext{susaṃskṛtam annam atipradadyād}%

 \nonanustubhindent \maintext{ghṛtaprabhūtam avadaṃśayuktam |}%

  \maintext{ghṛtaprapakvaṃ sukṛtaṃ ca pūpaṃ}%

 \nonanustubhindent \maintext{sitena khaṇḍena guḍena yuktam }||\thinspace17:2\thinspace||%
\translation{The Lord spoke: One should donate as much as one can food that is well-prepared, rich in ghee and contains pungent ingredients, well-prepared bread baked with ghee, white sugar and molasses. }

  \maintext{mārgaṃ khagaṃ codakajaṅgalaṃ ca}%

 \nonanustubhindent \maintext{dadyād vaṭaṃ nāgaravaṃśamūlam |}%

  \maintext{śākaṃ phalaṃ cāmla madhūratiktaṃ}%

 \nonanustubhindent \maintext{pānaṃ payaḥ śītasugandhatoyam }||\thinspace17:3\thinspace||%
\translation{One should give [animals] that roam paths, the sky and the waters, and [the fruits of the] Banyan-tree, dried ginger {\rm (}\textit{nāgara}{\rm )}, sugarcane, and roots, vegetables, fruits, sweet and pungent tamarind, and for drinks, milk, and cold and perfumed water. \blankfootnote{17.3 Understand \textit{pāda} a as \textit{mārgagaṃ khagam udakajaṅgamaṃ ca}.
 For \textit{nāgara} as `dried ginger' {\rm (}in \textit{pāda} b{\rm )}, see \mycitep{Meulenbeld1974}{567}.
 Note °\textit{madhūra}° for °\textit{madhura}° in \textit{pāda} c metri causa; or read °\textit{madhūka}° {\rm (}Madhuca latifolia{\rm )}.
 }}

  \maintext{dadhi pradadyād guḍamiśritaṃ ca}%

 \nonanustubhindent \maintext{mṛṇāla śālūka ca nālakā ca |}%

  \maintext{sadakṣiṇālepapavitrapuṣpaṃ}%

 \nonanustubhindent \maintext{śraddhānvitaḥ satkṛtayā praṇamya }||\thinspace17:4\thinspace||%
\translation{One should give coagulated milk mixed with molasses, lotus-fibre, lotus-roots, lotus-stalks, ointments accompanied by gifts, ritually pure flowers, with faith and respect, bowing. }

  \maintext{prayānti lokaṃ jagadīśvarasya}%

 \nonanustubhindent \maintext{vimānayānaiḥ sahito 'psarobhiḥ |}%

  \maintext{ekaikasikthasya sahasravarṣam}%

 \nonanustubhindent \maintext{annaprado modati devaloke }||\thinspace17:5\thinspace||%
\translation{They go to the world of Jagadīśvara with \ae rial vehicles, together with Apsarases. He who donates food will have fun in the world of gods for a thousand years for each lump of boiled rice. }

  \maintext{cyutaś ca martye sa bhaved dhanāḍhyaḥ}%

 \nonanustubhindent \maintext{kulodgataḥ sarvaguṇopapannaḥ |}%

  \maintext{yaśaḥ śriyaṃ sarvakalājñatā ca}%

 \nonanustubhindent \maintext{bhavet sa bhogī sakalatraputraḥ }||\thinspace17:6\thinspace||%
\translation{Descending to the human world, he will become a rich man. He will be born in a noble family and will possess all possible virtues, fame, beauty, and knowledge of all the arts. He will be rich and will have a wife and sons. }

  \maintext{dadyād daridraḥ kṛpaṇārtadīnā}%

 \nonanustubhindent \maintext{kālāgatatvāturam āgatānām |}%

  \maintext{tṛṣṇābubhukṣāgatikāgatānām}%

 \nonanustubhindent \maintext{dattvā sa dharmaphalam āśrayeta }||\thinspace17:7\thinspace||%
\translation{One should donate to the poor, the miserable, the oppressed, the wretched, the dying, the suffering, to those whose share is thirst, hunger, and unsuccessfulness. By donating, one will be connected to the fruits of Dharma. }

  \maintext{deśe ca kāle ca tathā ca pātre}%

 \nonanustubhindent \maintext{dānādidharmasya phalaṃ kaniṣṭam |}%

  \maintext{vāṇijyadharmā hi phalāśritānāṃ}%

 \nonanustubhindent \maintext{dharmo hi tasya na ca nirmalo 'sti }||\thinspace17:8\thinspace||%
\translation{[When done] in the right place, right time, and for the right recipient, the fruit of the Dharma of donation etc. is the smallest.[?] He whose Dharma is trade ... does not have a spotless Dharma.[?] }

  \maintext{toyaṃ ca dadyāl laghupūrṇakumbhaṃ}%

 \nonanustubhindent \maintext{śītaṃ sugandhaṃ parivāsitaṃ ca |}%

  \maintext{sa yāti lokaṃ salileśvarasya}%

 \nonanustubhindent \maintext{na saptajanmāni tṛṣābhibhūtaḥ }||\thinspace17:9\thinspace||%
\translation{He should give cool, nice-smelling and scented water [in] a light waterpot fill to the brim. He will go to the world of Salileśvara [i.e.\ Varuṇa] and will not be overcome by thist thoroughout seven births. }

  \subchptr{vastrādipradānam}%

  \trsubchptr{Donation of clothes etc.}%

  \maintext{upānahaṃ yo dadati dvijāya}%

 \nonanustubhindent \maintext{suśobhanaṃ tailasudīpitaṃ ca |}%

  \maintext{te yānti lokam amarādhipasya}%

 \nonanustubhindent \maintext{yamālayaṃ kaṣṭapathā na yānti }||\thinspace17:10\thinspace||%
\translation{He who donates a beautiful pair of sandals, polished with oil, to a Brahmin will go to the world of the King of the Immortal Ones [i.e.\ Indra], and will not approach Yama's abode through a difficult path. }

  \maintext{prakṣīṇapuṇyaḥ punar atra loke}%

 \nonanustubhindent \maintext{jāto bhaved divyakulopapannaḥ |}%

  \maintext{dhanaiḥ samṛddho 'dhipatitvatāṃ ca}%

 \nonanustubhindent \maintext{rathāśvanāgāsanagā bhavanti }||\thinspace17:11\thinspace||%
\translation{When his merits fade away, he will be born again in this world into a divine family. He will be rich with wealth, will be a king, riding on a chariot, on horses and elephants, sitting on a throne. }

  \maintext{vastrapradānena bhavanti devi}%

 \nonanustubhindent \maintext{rūpottamāḥ sarvakalājñatāś ca |}%

  \maintext{samṛddhisaubhāgyaguṇānvitāś ca}%

 \nonanustubhindent \maintext{svargacyutās te puruṣā bhavanti }||\thinspace17:12\thinspace||%
\translation{By donating clothes, O Devī, they will become most beautiful people with knowledge of all the arts, endowed with riches, happiness and virtues, when they descend from heaven. }

  \maintext{vastrapradānābhiratasya puṃsaḥ}%

 \nonanustubhindent \maintext{anyāṃ pravakṣyāmi tataḥ praśaṃsām |}%

  \maintext{vastraṃ tu lokeṣv abhipūjanīyaṃ}%

 \nonanustubhindent \maintext{vastraṃ narāṇāṃ tv atimānanīyam }||\thinspace17:13\thinspace||%
\translation{I shall then praise the man further who engages in the donation of clothes. Clothes are honoured in the worlds, clothes are held in extremely high esteem by people. }

  \maintext{vastraṃ tu bhūyo na ca mānalābhaḥ}%

 \nonanustubhindent \maintext{parābhavaś cātijugupsanaṃ ca |}%

  \maintext{tasmād dhi vastraṃ satataṃ pradeyaṃ}%

 \nonanustubhindent \maintext{yaśaḥ śriyaḥ svargam anantalābham }||\thinspace17:14\thinspace||%
\translation{Furthermore, [if there are] no clothes, there is no respect,[?] but [only] defeat and extreme disgust. Therefore clothes should always be donated, [and by this] fame, fortune, heaven, and endless profit. }

  \maintext{yāvanti sūtrāṇi bhavanti vastre}%

 \nonanustubhindent \maintext{tāvad yugaṃ gacchati somalokam |}%

  \maintext{puṇyakṣayāj jāyati martyaloke}%

 \nonanustubhindent \maintext{vastraprabhūte dhanadhānyakīrṇe }||\thinspace17:15\thinspace||%
\translation{He will stay in Somaloka for as many \ae ons as there are threads in a piece of clothes. Because his merits fade away, he is reborn in the human world, with an abundance of clothes and having a lot of riches and corn. }

  \maintext{surūpasaubhāgyayaśaśivanaś ca}%

 \nonanustubhindent \maintext{vidyādharo lokaprabhutvatāś ca }||\thinspace17:16\thinspace||%


  \maintext{dvijebhyac chatraṃ sukṛtaṃ pradadyāt}%

 \nonanustubhindent \maintext{varṣātapatraṃ dṛḍhaśobhanaṃ ca |}%

  \maintext{aṅgāravarṣatraṣu khaḍgamādyam}%

 \nonanustubhindent \maintext{asaṃśayaṃ trāyati yāmyamārge }||\thinspace17:17\thinspace||%


  \maintext{svargaṃ ca yānti grahanāyakaś ca}%

 \nonanustubhindent \maintext{sa varṣakoṭyāyutam antakāle |}%

  \maintext{jāyanti te mānuṣamartyaloke}%

 \nonanustubhindent \maintext{gṛhottame bhogapatir bhavanti }||\thinspace17:18\thinspace||%


  \maintext{kṛtvā maṭhaṃ śobhanavipradātā}%

 \nonanustubhindent \maintext{dravyeṇa śuddhena tu pūjayitvā |}%

  \maintext{sa yāti devendrasadaṃ yatheṣṭam}%

 \nonanustubhindent \maintext{savarṣakoṭiśatadivyasaṃkhyaiḥ }||\thinspace17:19\thinspace||%


  \maintext{tadantakāle yadi mānuṣatvam}%

 \nonanustubhindent \maintext{jāyanti te saptamahīprabhoktā |}%

  \maintext{sa saptarathyatrayasamprayuktā}%

 \nonanustubhindent \maintext{balādhiko yajñasahasrakartā }||\thinspace17:20\thinspace||%


  \subchptr{bhūmipradānam}%

  \trsubchptr{Donation of land}%

  \maintext{bhūmipradātā dvijahīnadīnam}%

 \nonanustubhindent \maintext{saṃmṛddhasasyo jalasaṃnikṛṣta |}%

  \maintext{sa yāti lokam amarādhipasya !}%

 \nonanustubhindent \maintext{vimānayānena manohareṇa }||\thinspace17:21\thinspace||%


  \maintext{manvantaraṃ yāvad abhuktabhogān}%

 \nonanustubhindent \maintext{tadantakāle cyutamartyaloke |}%

  \maintext{sa javamukhaṇḍādhipatir bhavet}%

 \nonanustubhindent \maintext{vīryānvito rājasahasranāthaḥ }||\thinspace17:22\thinspace||%


  \maintext{sa cailaghaṇṭāṃ kanakāgraśṛṅgām}%

 \nonanustubhindent \maintext{dogdhīṃ savatsāṃ payasāṃ dvijānām |}%

  \maintext{dattvā dvijebhyaḥ samalaṅkṛtānām}%

 \nonanustubhindent \maintext{prayānti lokaṃ surabhīsutānām }||\thinspace17:23\thinspace||%


  \maintext{yāvanti romāṇi bhavanti gāvaḥ}%

 \nonanustubhindent \maintext{tāvad yugānām anubhūyabhogān |}%

  \maintext{tasmāc cyutā martyamahībhujās te}%

 \nonanustubhindent \maintext{sahasrarājānugato mahātmā }||\thinspace17:24\thinspace||%


  \maintext{suvarṇakāṃsyāyasaraupyadātā}%

 \nonanustubhindent \maintext{tāmrapravālāmaṇimauktikādyān |}%

  \maintext{dattvā dvijebhyo vasusādhyaloke}%

 \nonanustubhindent \maintext{prāpnoti varṣaṃ daśapañcakoṭyo !  }||\thinspace17:25\thinspace||%


  \maintext{bhuktvā yatheṣṭaṃ kramadevalokān}%

 \nonanustubhindent \maintext{cyutaṃ ca martye sa bhaven narendraḥ |}%

  \maintext{sudurjayaḥ śakrasahasrajetā}%

 \nonanustubhindent \maintext{sudīrgham āyuś ca parākramaś ca }||\thinspace17:26\thinspace||%


  \maintext{yat prekṣaṇaṃ darśayituṃ pradātā}%

 \nonanustubhindent \maintext{surūpasaubhāgya phalaṃ labheta |}%

  \maintext{tṛṇāśanāmūlaphalāśanena}%

 \nonanustubhindent \maintext{labheta rājyāni kaṇṭakāni }||\thinspace17:27\thinspace||%


  \maintext{labheta parṇāśanasvargavāsam}%

 \nonanustubhindent \maintext{payaḥ prayogena ca devaloke |}%

  \maintext{śuśrūṣaṇo yo gurave ca nityam}%

 \nonanustubhindent \maintext{vidyādharo jāyati martyaloke }||\thinspace17:28\thinspace||%


  \maintext{dadyād gavāṃ dhāsatṛṇasya muṣṭiḥ}%

 \nonanustubhindent \maintext{gavāḍhyatāṃ jāyati martyaloke |}%

  \maintext{śrāddhaṃ ca dattvā prayato dvijāya}%

 \nonanustubhindent \maintext{samṛddhasantāna bhaved yugānte }||\thinspace17:29\thinspace||%


  \maintext{ahiṃsako jāyati dīrgham āyuḥ}%

 \nonanustubhindent \maintext{kulottamaṃ jāyati dīkṣitena |}%

  \maintext{kālatrayaṃ snānakṛtena rājyaṃ}%

 \nonanustubhindent \maintext{pītvā ca vāyus tridaśādhipatvam }||\thinspace17:30\thinspace||%


  \maintext{anaśnatāyāḥ phalam īśaloke}%

 \nonanustubhindent \maintext{tṛptir bhavet toyapradānaśīlaḥ |}%

  \maintext{annapradātā puruṣaḥ samṛddhaḥ}%

 \nonanustubhindent \maintext{sa sarvakāmā labhatīha loke }||\thinspace17:31\thinspace||%


  \maintext{śraddhāmatir yaḥ praviśed dhutāsanaṃ !}%

 \nonanustubhindent \maintext{sa yāti lokaṃ prapitāmahasya |}%

  \maintext{satyaṃ vaded yo 'pi ca dharmaśīlo}%

 \nonanustubhindent \maintext{modaty asau devi sahāpsarobhiḥ }||\thinspace17:32\thinspace||%


  \maintext{rasās tu ṣaḍyo parivarjayanti}%

 \nonanustubhindent \maintext{atīva saubhāgya labheta sādhvī |}%

  \maintext{dānena bhogān atulyaṃ labheta}%

 \nonanustubhindent \maintext{cirāyutāṃ yāti hi brahmacaryāt }||\thinspace17:33\thinspace||%


  \maintext{dhanāḍhyatāṃ yānti hi puṇyakarmān}%

 \nonanustubhindent \maintext{maunena - ājñā labhate alaṅghyām |}%

  \maintext{prāpnoti kāmaṃ tapasaḥ sutaptaṃ}%

 \nonanustubhindent \maintext{kīrtir yaśaḥ svargam anantabhogam |}%

  \maintext{āyuḥ śriyārogyadhanaprabhutvaṃ}%

 \nonanustubhindent \maintext{jñānādilābhaṃ tapasā labheta }||\thinspace17:34\thinspace||%


  \maintext{trailokyādhipatitvaśakram agamat kṛtvā tapo duṣkaram}%

 \nonanustubhindent \maintext{yakṣeśo 'pi tapaḥ prabhāvaguruṇā guhyādhipatvaṃ mahat |}%

  \maintext{rakṣeśo 'pi bibhīṣaṇas tv amaratāṃ prāptas tapasyaiva tu}%

 \nonanustubhindent \maintext{rudrārādhanatatparās tapaphalāt nandīgaṇatvaṃ gataḥ }||\thinspace17:35\thinspace||%


  \maintext{jñānaṃ dvijān tapaso āha viṣṇuḥ}%

 \nonanustubhindent \maintext{kṣatraṃ tapo rakṣaṇam āha sūrya |}%

  \maintext{vaiśyaṃ tapaś cāñjanam āha vāyuḥ}%

 \nonanustubhindent \maintext{śūdraṃ hi śilpaṃ tapa āha indraḥ }||\thinspace17:36\thinspace||%


  \maintext{raṇotsahaṃ kṣatriyayajñam iṣṭaṃ}%

 \nonanustubhindent \maintext{vaiśyaṃ havir yajñam udāharanti |}%

  \maintext{śūdrasya yajñaḥ paricaryam iṣṭaṃ}%

 \nonanustubhindent \maintext{yajñaṃ dvijānāṃ japamuktamokṣam }||\thinspace17:37\thinspace||%


  \subchptr{svamāṃsarudhiradānam}%

  \trsubchptr{Donation of one's own flesh and blood}%

  \maintext{devy uvāca |}%

  \maintext{svamāṃsarudhiraṃ dānaṃ dānaṃ putrakalatrayoḥ |}%

  \maintext{kiṃ praśasyaṃ mahādeva tattvaṃ vaktum ihārhasi }||\thinspace17:38\thinspace||%
\translation{Devī spoke: Why are one's own flesh and blood and one's son and wife praised as donation, O Mahādeva? Tell me the truth please. }

  \maintext{maheśvara uvāca |}%

  \maintext{svamāṃsarudhiraṃ dānaṃ praśaṃsanti manīṣiṇaḥ |}%

  \maintext{śrūyatāṃ pūrvavṛttāni saṃkṣipya kathayāmy aham }||\thinspace17:39\thinspace||%
\translation{Maheśvara spoke: The wise praise one's own flesh and blood as donation. Let's hear the old legends, I shall tell you briefly.'{\rm )} }

  \maintext{uśīnaras tu rājarṣiḥ kayo ?tārthe svakāntantu?  |}%

  \maintext{tyaktvā svargam anuprāptaḥ parārthe paratatparaḥ }||\thinspace17:40\thinspace||%


  \maintext{putramāṃsaṃ svayaṃ chitvā agnidattaṃ purānaghe |}%

  \maintext{tena dānaprabhāvena alarkas tridivaṃ gataḥ }||\thinspace17:41\thinspace||%


  \maintext{svadānadānena mudā sa putra}%

 \nonanustubhindent \maintext{aputrabhūtasya ca putra jātaḥ |}%

  \maintext{svarge svayaṃ cokvaya bhogalābhaṃ}%

 \nonanustubhindent \maintext{prāpto mahaddānay?la prabhāvāt }||\thinspace17:42\thinspace||%


  \maintext{yādavaś cārjano devi dattvā khaṇḍavabhājanam }||\thinspace17:43\thinspace||%


  \maintext{tapanasya prasādena saptadvīpeśvaro bhavet |}%

  \maintext{hariṇā ca śiro bhitvā dattaṃ me rudhiraṃ purā }||\thinspace17:44\thinspace||%


  \maintext{pratīcchitaṃ kapālena brahmasambhavajena me |}%

  \maintext{divyavarṣasahasrāṇi dhārā tasya na chidyate }||\thinspace17:45\thinspace||%


  \maintext{parituṣṭo 'smi tenāhaṃ karmaṇānena sundari |}%

  \maintext{varaṃ dattaṃ mayā devi purāṇapuruṣo 'vyayaḥ }||\thinspace17:46\thinspace||%


  \maintext{akṣayaṃ valamūrjaṃ ca ajarāmaram eva ca |}%

  \maintext{mamādhikaṃ bhaved viṣṇur māma yitvam vijeṣyasi }||\thinspace17:47\thinspace||%


  \maintext{evamādīny anekāni mayoktāni janārdane |}%

  \maintext{niṣkampa niścalamanaḥ sthāṇubhūta iva sthitaḥ }||\thinspace17:48\thinspace||%


  \maintext{da?ciḥ svatanuṃ dattvā vibudhānāṃ varānane |}%

  \maintext{bhuktvā lokān kramāt sarvān śivaloke pratiṣṭhitaḥ }||\thinspace17:49\thinspace||%


  \maintext{jāmadagnir mahīṃ dattvā kāśyapāya mahātmane |}%

  \maintext{ihaiva sa yālaṃ bhoktā devarājyam avāpsyati  }||\thinspace17:50\thinspace||%


  \maintext{dattvā go sakalaṃ devi vyāsasyāmitatejasaḥ |}%

  \maintext{yudhiṣṭhira mahīyāsa dehas tridivadbhataḥ }||\thinspace17:51\thinspace||%


  \maintext{satyanāmaḥ ? {\rm (}bhīmaḥ?{\rm )} svakaṃ bhartā dattvā nārādasatkṛtam |}%

  \maintext{dānasyāsya prabhāvena akṣayaṃ tridivadbhataḥ ? }||\thinspace17:52\thinspace||%


  \maintext{catuḥṣaṣṭhisahastāṇi gavāṃ dattvā dvijanmane |}%

  \maintext{duryodhanamahīyā?o gataḥ svargam anantakam }||\thinspace17:53\thinspace||%


  \maintext{vāsukis sarparājendro dattvā viprasusaṃskṛtam |}%

  \maintext{ratkāruś ca ? sābhānyā sarve nāgavimokṣitāḥ }||\thinspace17:54\thinspace||%


  \maintext{gobhūmikanakādīnāṃ dānaṃ kanyasam ucyate |}%

  \maintext{bhṛtyaputrakalatrāṇāṃ dānaṃ madhyamam ucyate }||\thinspace17:55\thinspace||%


  \maintext{svadehaṃ pisitādīnāṃ dānam uttamam ucyate |}%

  \maintext{etat sarvaṃ yadā dānaṃ tad dānam uttamottamam }||\thinspace17:56\thinspace||%


  \maintext{jāvaj janmasahasrāṇi bhoktā bhavati kanyasaḥ |}%

  \maintext{śatajanmasahasrāṇi bhoktā bhavati madhyamaḥ }||\thinspace17:57\thinspace||%


  \maintext{uttamaḥ palabhoktā {\rm (}phala?{\rm )} vi ? janmakoṭiśatatrayam | }%

  \maintext{parārdhadvayajanmānāṃ bhoktā vai cottamottamaḥ }||\thinspace17:58\thinspace||%


  \maintext{bhūtānām anukampayā yadi dhanaṃ dātā sadānvarṣine |}%

  \maintext{dīnānvakṛyaṇeṣv anāthamalineśvānādini?? ca }||\thinspace17:59\thinspace||%


  \maintext{yady eva kurute sadārtiharaṇaṃ śraddhānvitau bhaktimān |}%

  \maintext{tasyānantayālaṃ vadanti vibudhāṃs sa yasya sandarśanāt }||\thinspace17:60\thinspace||%

\centerline{\maintext{\dbldanda\thinspace iti vṛṣasārasaṃgrahe dānadharmaviśeṣaṃ nāma saptādaśamo 'dhyāyaḥ\thinspace\dbldanda}}
