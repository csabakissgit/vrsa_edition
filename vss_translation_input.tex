
  \chptr{prathamo 'dhyāyaḥ}
\addcontentsline{toc}{section}{Chapter 1}
\fancyhead[CO]{{\footnotesize\textit{Translation of chapter 1}}}%

  \trchptr{Chapter One}%

  \subchptr{stutiḥ}%

  \trsubchptr{Invocation}%

  \maintext{anādimadhyāntam anantapāraṃ}%

 \nonanustubhindent \maintext{susūkṣmam avyaktajagatsusāram |}%

  \maintext{harīndrabrahmādibhir āsamagraṃ}%

 \nonanustubhindent \maintext{praṇamya vakṣye vṛṣasārasaṃgraham }||\thinspace1:1\thinspace||%
\translation{Having bowed to the One who has no beginning, no middle part and no end, whose boundaries are limitless, who is very subtle and who is the unmanifest and fine essence of the world, together with Hari, Indra, Brahmā and the other [gods], I shall recite [the work called] `A Compendium on the Essence of the Bull [of Dharma]'. \blankfootnote{1.1 This verse echoes \VSS\ 20.3:
  \textit{nādimadhyaṃ na cāntaṃ ca yan na vedyaṃ surair api| 
  atisūkṣmo hy atisthūlo nirālambo nirañjanaḥ\thinspace ||}; which could suggest
  that \textit{pāda} c above might be parallel with \textit{na vedyaṃ surair api}. Perhaps
  understand \textit{asamagraṃ} [\textit{vedyaṃ}]; `incompletely [known].
 
  \textit{Pāda} a is also reminiscent of, among other famous passages, \BHG\ 11.19:
  
 
  \textit{anādimadhyāntam anantavīryam 
  anantabāhuṃ śaśisūryanetram\thinspace | 
  paśyāmi tvāṃ dīptahutāśavaktraṃ 
  svatejasā viśvam idaṃ tapantam\thinspace ||}.
  
 
  See also \BHG\ 10.20cd:
  
 
  \textit{aham ādiś ca madhyaṃ ca bhūtānām anta eva ca}\thinspace ||.
  
 
  A faint reference to the \BHG\ seems proper at the
  beginning of a work that claims to deliver a teaching 
  based on, but also to surpass, the \MBH\ {\rm (}see following verses of the \VSS{\rm )}.
  Compare also, e.g., \KURMP\ 1.11.237:
  
 
  \textit{rūpaṃ tavāśeṣakalāvihīnam
  agocaraṃ nirmalam ekarūpam\thinspace | 
  anādimadhyāntam anantam ādyaṃ 
  namāmi satyaṃ tamasaḥ parastāt\thinspace ||.}
  
 
  To say that a god has no beginning and no end in a temporal or spacial sense is natural
  {\rm (}\textit{anādi°...°antam}{\rm )}, but to have no `middle part' {\rm (}\textit{°madhya°}{\rm )} in these senses is slightly less so.
  Thus the rather commonly occuring phrase \textit{anādimadhyāntam} is probably a fixed expression usually 
  referring to a deity that is endless, eternal and immaterial. 
  As to which deity or what form of a deity this stanza refers to, 
  it may be Śiva, his name missing in pāda c, but the phrasing of the verse 
  is vague enough to keep the question somewhat open: the impersonal Brahman 
  might be another option, even more so if we look at verses 1.9--10, whose
  topic is \textit{brahmavidyā}.
 
 
  
 
  In \textit{pāda} b \textit{jagat-susāraṃ} is most probably not 
  to be interpreted as \textit{jagatsu sāraṃ} {\rm (}`the essence in the worlds'{\rm )}.
  Another way to translate \textit{avyaktajagatsusāraṃ} would be: 
  `who is the fine essence of the unmanifest world.'
 
  
 
  Strictly speaking, \textit{pāda} c is unmetrical, but it is better to 
  simply acknowledge here the phenomenon of `muta cum liquida', namely
  that syllables followed by consonant clusters such as 
  \textit{ra, bra, hra, kra, śra, śya, śva, sva, dva} can be treated as short {\rm (}\textit{laghu}{\rm )}.
  {\rm (}See Introduction pp.~\pageref{muta}ff.{\rm )}
  Thus \textit{harīndrabrahmā°} can be treated as a regular beginning
  of an \textit{upajāti} {\rm (}\shortsyllable\ - \shortsyllable\ - -{\rm )}, the syllable 
  \textit{bra} not turning the previous syllable long {\rm (}\textit{guru}{\rm )}.
 
  
 
  The reading \textit{āsamagraṃ} in \textit{pāda} c is suspect 
  {\rm (}see a preliminary comment on this above{\rm )},
  although the initial \textit{ā-} might convey some sort of
  completeness, meaning `all round'
  {\rm (}see e.g. \mycitep{KaleHigherGrammar}{226}{\rm )}.
  The fact that we could percieve the ending of \textit{pāda}s a and b 
  {\rm (}\textit{pāraṃ}--\textit{sāram}{\rm )}, as well as \textit{pāda}s c and d, as {\rm (}in the 
  latter case, oddly{\rm )} rhyming pairs {\rm (}\textit{graṃ}-\textit{graham}{\rm )}
  suggests that accepting the reading \textit{āsamagram} could be 
  the right decision {\rm (}as suggested by Alessandro Battistini{\rm )}.
  I translate this verse accordingly. \msM\ gives an exciting,
  albeit unmetrical, alternative {\rm (}\textit{yat samagraṃ}{\rm )}, but
  this seems more like a guess to me than the correct reading.
  For some time I was considering emending \textit{āsamagraṃ}.
  The most tempting of all the possible options 
  {\rm (}\textit{arcyam/arhyam/arghyam/īḍyam/āḍhyam agraṃ, āsamastaṃ}{\rm )} 
  seemed to be \textit{āptam agraṃ},
  meaning `appointed/received/respected [by Hari, Indra,
  Brahmā etc.] as the foremost one'. The fact that 
  the \textit{akṣara}s \textit{āsam} and \textit{āptam} look similar in most
  of the scripts used in our manuscripts could support this
  conjecture. \textit{āptam} could also
  possibly refer to the text itself, although then the
  syntax becomes slightly confusing: `I shall recite the
  \textit{Vṛṣasārasaṃgraha} that was first received by Hari...' etc.
  Another candidate was \textit{āḍhyam agram}:
  `Having bowed to [Him] who contains/is rich with Hari, Indra, Brahmā
  etc.' I have not emended the text because it is difficult
  to know if there is any need for change and if there is, which reading 
  to chose. There was no consensus when this verse was discussed 
  in our extended Śivadharma reading group.
 
  
 
  Pāda d seems hypermetrical, but it can be interpreted as a \textit{vaṃśastha}
  line, a change from \textit{triṣṭubh} to \textit{jagatī} {\rm (}as suggested by Dominic Goodall{\rm )}.
 }}

  \subchptr{janamejayavaiśampāyanasaṃvādaḥ}%

  \trsubchptr{Dialogue of Janamejaya and Vaiśampāyana}%

  \maintext{śatasāhasrikaṃ granthaṃ sahasrādhyāyam uttamam |}%

  \maintext{parva cāsya śataṃ pūrṇaṃ śrutvā bhāratasaṃhitām }||\thinspace1:2\thinspace||%
\translation{Having listened to the \textit{Bhāratasaṃhitā} [i.e. the \textit{Mahābhārata}], the supreme book of a hundred thousand [verses] and a thousand chapters {\rm (}\textit{adhyāya}{\rm )}, with all its hundred sections {\rm (}\textit{parvan}{\rm )}, \blankfootnote{1.2 The dialogue of Janamejaya and Vaiśampāyana makes up the outermost layer of the \VSS\ 
  {\rm (}except for the introductory stanzas 1.1--3; see Introduction \CHECK{\rm )}, mostly containing
  general \textit{dharmaśāstric} material.
  
 
  That the \MBH\ should contain a hundred thousand verses is hinted at e.g. in line 19 of
  the Khoh Charter 2 of Śarvanātha, year 214 {\rm (}Siddham IN00088: 
  \textit{uktañ ca mahābhārate śatasāhasryaṃ} {\rm (}understand °\textit{ryāṃ}{\rm )} \textit{saṃhitāyāṃ}...{\rm )}.
  The hundred \textit{parvan}s of the \textit{Mahābhārata} are listed in \MBH\ 1.2.33--70.
  Note the use of the singular {\rm (}\textit{parva}{\rm )} in connection with numerals {\rm (}\textit{śataṃ}{\rm )}.
 }}

  \maintext{atṛptaḥ puna papraccha vaiśampāyanam eva hi |}%

  \maintext{janamejayena yat pūrvaṃ tac chṛṇu tvam atandritam }||\thinspace1:3\thinspace||%
\translation{Janamejaya remained unsatisfied. Listen unweariedly to what he asked Vaiśampāyana in the past. \blankfootnote{1.3 My emendation from the unmetrical \textit{punaḥ} to the unusual, or rather, Middle Indic
  {\rm (}\mycitep{EdgertonHybrid}{vol. 2, p. 347}{\rm )},
  and Newari {\rm (}\mycitep{JorgensenGrammar}{113}{\rm )}, \textit{puna} is based
  on the assumption that in the original the metre must have overridden 
  morphology, similarily to what may have happened in 8.44d {\rm (}Mālinī metre{\rm )}:
  \textit{na bhavati punajanma kalpakoṭyāyute 'pi}, and in 12.151c {\rm (}Sragdharā metre{\rm )}:
  \textit{garbhāvāsaṃ na ca tvan na ca punamaraṇaṃ kleśam āyāsapūrṇam}.
 
  
 
  For an unsatisfaction or dissatisfaction {\rm (}\textit{atṛpti}{\rm )} with previous 
  teachings in a somewhat similar manner to what
  Janamejaya experiences here, see e.g. \textit{Niśvāsa} mūla 1.9:
  
 
  \textit{vedāntaṃ viditaṃ deva sāṃkhyaṃ vai pañcaviṃśakam\thinspace |
  na ca tṛptiṃ gamiṣyāmo hy ṛte śaivād anugrahāt\thinspace ||};
  
 
  and the \SDhS\... \CHECK.
 Vaiśampāyana, a Ṛṣi, disciple of Vyāsa, great-grandson to Arjuna,
  recited the Mahābhārata at the snake sacrifice of 
  Janamejaya. This setting is an echo of the starting point of the Mahābhārata, see \MBH\ 1.1.8ff.
  In fact the next few verses in the \VSS\ make it clear that the \VSS\
  picks up where the Mahābhārata left off: Janamejaya has heard the whole Mahābhārata from
  Vaiśampāyana, but he is eager to hear more.
  
 
  It is tempting to emend \textit{pāda} c to contain a stem form proper noun {\rm (}\textit{janamejaya}{\rm )}
  in order to maintain the metre, 
  and note how the manuscripts struggle with this \textit{pāda}. 
  Stem form nouns, \textit{prātipadika}s, abound in the \VSS: see Introduction p. \CHECK.
  On the other hand, the contracted/syncopated form \textit{janmejaya} occurs, 
  e.g., in \BHAGP\ 12.06.16 and \BRAHMAVP\ 4.14.41, 46. 
  The hypermetrical form \textit{janamejayena}, and the construction finite verb + instrumental
  {\rm (}\textit{papraccha... janamejayena}{\rm )}, could be original; compare 1.8 and 4.75 below.
 }}

  \maintext{janamejaya uvāca |}%

  \maintext{bhagavan sarvadharmajña sarvaśāstraviśārada |}%

  \maintext{asti dharmaṃ paraṃ guhyaṃ saṃsārārṇavatāraṇam }||\thinspace1:4\thinspace||%
\translation{Janamejaya spoke: O venerable sir, O knower of the entire Dharma, O you who are well-versed in all the sciences {\rm (}\textit{śāstra}{\rm )}! There is a supreme and secret Dharma [that brings about] liberation from the ocean of mundane existence {\rm (}\textit{saṃsāra}{\rm )}, \blankfootnote{1.4 Note \textit{dharma} as a neuter noun in \textit{pāda} c and in the next verse.
 }}

  \maintext{dvaipāyanamukhodgīrṇaṃ dharmaṃ vā yad dvijottama |}%

  \maintext{kathayasva hi me tṛptiṃ kuru yatnāt tapodhana }||\thinspace1:5\thinspace||%
\translation{that is, the Dharma that emerged from [Vyāsa] Dvaipāyana's mouth, O best of Brahmins. Teach me that and help me find satisfaction at all cost, O great ascetic! \blankfootnote{1.5 The majority of the MSS consulted include a \textit{vā} in \textit{pāda} b, 
  and although \msCb's reading seems a bit smoother, that manuscript rarely gives superior readings.
  Therefore I have chosen \textit{dharmaṃ vā yad}, in which \textit{vā} functions probably in a weak sense.
  That the secret Dharma Janamejaya is seeking is the one taught by Vyāsa Dvaipāyana, and
  thus no real options are involved here, becomes clear in 1.6cd.
  The reading of \msM\ in \textit{pāda} b {\rm (}\textit{dharmavākyaṃ}{\rm )} is tempting but could be a later correction.
 In general, \msM's readings here are unique but probably secondary: 
  \textit{hi me tṛptiṃ} in \textit{pāda} c seems more attractive than \msM's 
  \textit{prasādena} because it echoes \textit{atṛptaḥ} in 1.3a
 }}

  \maintext{vaiśampāyana uvāca |}%

  \maintext{śṛṇu rājann avahito dharmākhyānam anuttamam |}%

  \maintext{vyāsānugrahasamprāptaṃ guhyadharmaṃ śṛṇotu me }||\thinspace1:6\thinspace||%
\translation{Vaiśampāyana spoke: Listen with great attention, O king, to this unsurpassed narration of Dharma. Hear the secret Dharma that I received by Vyāsa's favour. }

  \maintext{anarthayajñakartāraṃ tapovrataparāyaṇam |}%

  \maintext{śīlaśaucasamācāraṃ sarvabhūtadayāparam }||\thinspace1:7\thinspace||%
{\blankfootnote{1.7 On Anarthayajña, the interlocutor of VSS 1.9--10.2 and 19.1--21.22, and
  an important figure discussed in 22.3ff, as well as a concept {\rm (}`nonmaterial sacrifice'{\rm )},
  see \mycite{KissVolume2021} and Introduction p.~\pageref{anarthayajna}.
 }}

  \maintext{jijñāsanārthaṃ praśnaikaṃ viṣṇunā prabhaviṣṇunā |}%

  \maintext{dvijarūpadharo bhūtvā papraccha vinayānvitaḥ }||\thinspace1:8\thinspace||%
\translation{Viṣṇu, the great Lord, assuming the form of a twice-born [Brahmin], wanted to test [Anarthayajña, the ascetic yogin] who performed nonmaterial sacrifices {\rm (}\textit{anarthayajña}{\rm )}, focused on his austerities and observances, whose conduct was virtuous and pure, and who was intent on compassion towards all living beings; therefore he [Viṣṇu] humbly asked him a question. \blankfootnote{1.8 Note the odd syntax here: \textit{viṣṇunā... dvijarūpadharo bhūtvā papraccha}.
  The agent of the active verb is in the instrumental case {\rm (}ergative structure{\rm )}.
 }}

  \subchptr{brahmavidyā}%

  \trsubchptr{Knowledge of Brahman}%

  \maintext{{\rm [}vigatarāga uvāca | {\rm ]}}%

  \maintext{brahmavidyā kathaṃ jñeyā rūpavarṇavivarjitā |}%

  \maintext{svaravyañjananirmuktam akṣaraṃ kimu tat param }||\thinspace1:9\thinspace||%
\translation{[Vigatarāga spoke:] How is the knowledge of the Brahman to be understood if it is devoid of form and colour? Why is that supreme syllable that is devoid of vowels and consonants the supreme one? \blankfootnote{1.9 The translation of this verse, and the reconstruction and interpretation
  of \textit{pāda} d, which is echoed in 1.10d, is slightly tentative.
  I doubt if \textit{kimu} could have the standard {\rm (}Vedic{\rm )} meaning `how much more/less'
  here. Rather \textit{u} is probably just an expletive. In general it seems that
  this verse references the syllable \textit{oṃ}.
 }}

  \maintext{anarthayajña uvāca |}%

  \maintext{anuccāryam asandigdham avicchinnam anākulam |}%

  \maintext{nirmalaṃ sarvagaṃ sūkṣmam akṣaraṃ kim ataḥ param }||\thinspace1:10\thinspace||%
\translation{Anarthayajña replied: That syllable is not to be pronounced, is unquestionable, non-dividable, consistent, spotless, all-pervading and subtle: what could be higher than that? \blankfootnote{1.10 In \textit{pāda} d, I have choosen, somewhat randomly, \textit{kim ataḥ} instead of \textit{kimu tat},
  trying to make sense of 10.9--10.
 }}

  \subchptr{kālapāśaḥ}%

  \trsubchptr{Noose of death and time}%

  \maintext{vigatarāga uvāca |}%

  \maintext{dehī dehe kṣayaṃ yāte bhūjalāgniśivādibhiḥ |}%

  \maintext{yamadūtaiḥ kathaṃ nīto nirālambo nirañjanaḥ }||\thinspace1:11\thinspace||%
\translation{Vigatarāga spoke: When the body disintegrates in the ground, in water, in fire, or [is torn apart] by jackals and other [animals], how is the supportless and spotless soul led [to the netherworld] by Yama's messengers? \blankfootnote{1.11 The word \textit{°śivā°} in \textit{pāda} b is slightly suspect, and could be the result
  of metathesis, from \textit{°viṣā°} {\rm (}`by poison'{\rm )}. Nevertheless, 
  jackals seems appropriate in this context, for they 
  are commonly associated with human corpses, death and the cremation ground
  {\rm (}see e.g. \mycite{Ohnuma2019}{\rm )}. Furthermore, \textit{pāda} b lists phenomena
  that cause the body to disintegrate, and not causes of death; thus the reading \textit{śiva}
  is probably correct.
 }}

  \maintext{kālapāśaiḥ kathaṃ baddho nirdehaś ca kathaṃ vrajet |}%

  \maintext{svargaṃ vā sa kathaṃ yāti nirdeho bahudharmakṛt |}%

  \maintext{etan me saṃśayaṃ brūhi jñātum icchāmi tattvataḥ }||\thinspace1:12\thinspace||%
\translation{How is it bound by the nooses of death/time? And if it is bodiless, how can it move? And how does the [soul of a] virtuous [person] {\rm (}\textit{bahudharmakṛt}{\rm )} reach heaven if it has no body? This is my doubt. Teach me. I want to know the truth. \blankfootnote{1.12 The word \textit{kāla} has, as usual, a double meaning here: \textit{kālapāśa}
  is both Yama's noose, and also the limitations and bondage caused by time, 
  as becomes clear at the discussion on the different time units in verses 1.18--31.
 }}

  \maintext{anarthayajña uvāca |}%

  \maintext{atisaṃśayakaṣṭaṃ te pṛṣṭo 'haṃ dvijasattama |}%

  \maintext{durvijñeyaṃ manuṣyais tu devadānavapannagaiḥ }||\thinspace1:13\thinspace||%
\translation{Anarthayajña spoke: You are asking me about an extremely doubtful and problematic matter, O truest of the twice-born. [This is something that] is difficult to understand by humans, and [even] by gods {\rm (}\textit{deva}{\rm )}, demons {\rm (}\textit{dānava}{\rm )} and serpents {\rm (}\textit{pannaga}{\rm )}. \blankfootnote{1.13 Note \textit{te} used for \textit{tvayā} in \textit{pāda} a. Alternatively, taking \textit{te} as genitive, the line
  could be translatied as: `I am being asked about a great 
  problem of yours that originates in doubts\dots'
 }}

  \maintext{karmahetu śarīrasya utpatti nidhanaṃ ca yat |}%

  \maintext{sukṛtaṃ duṣkṛtaṃ caiva pāśadvayam udāhṛtam }||\thinspace1:14\thinspace||%
\translation{The cause of both the birth and death of the body is karma. Good and bad deeds are called the two nooses. \blankfootnote{1.14 The MSS give \textit{karmahetu} in \textit{pāda} a overwhelmingly, which could work as a neuter
  \textit{bahuvrīhi} compound picking up both a stem-form \textit{utpatti} and \textit{nidhanaṃ}. 
  \textit{karmahetuḥ} {\rm (}\msCb{\rm )} is grammatically more correct, picking up the feminine \textit{utpatti},
  but now I consider a neuter stem-form \textit{utpatti} unsurprising.
 }}

  \maintext{tenaiva saha saṃyāti narakaṃ svargam eva vā |}%

  \maintext{sukhaduḥkhaṃ śarīreṇa bhoktavyaṃ karmasambhavam }||\thinspace1:15\thinspace||%
\translation{[The soul] goes to hell or heaven accordingly. Happiness and suffering, both arising from karma, are to be experienced by the body. }

  \maintext{hetunānena viprendra dehaḥ sambhavate nṛṇām |}%

  \maintext{yaṃ kālapāśam ity āhuḥ śṛṇu vakṣyāmi suvrata }||\thinspace1:16\thinspace||%
\translation{It is for this reason, O great Brahmin, that the human body is born. Now learn about that which they call the noose of time, I shall teach you, O you of great observances. }

  \maintext{na tvayā viditaṃ kiñcij jijñāsyasi kathaṃ dvija |}%

  \maintext{kālapāśaṃ ca viprendra sakalaṃ vettum arhasi }||\thinspace1:17\thinspace||%
\translation{[If] you don't know anything, how could you start your investigation, O twice-born? O great Brahmin, you should know the noose of time in its entirety. \blankfootnote{1.17 The variant \textit{jijñāsyasi} seems to be the lectio difficilior as opposed to
  \textit{vijñāsyasi}, but the latter could also work fine here.
 Note how \msM\ {\rm (}agreeing with \Ed{\rm )} gives a reading {\rm (}\textit{vaktum arhasi}{\rm )} that is clearly out
  of context. This confirms that while \msM\ comes up with interesting readings, 
  they are mostly to be ignored.
 }}

  \maintext{kalākalitakālaṃ ca kālatattvakalāṃ śṛṇu |}%

  \maintext{truṭidvayaṃ nimeṣas tu nimeṣadviguṇā kalā }||\thinspace1:18\thinspace||%
\translation{Learn about time {\rm (}\textit{kāla}{\rm )} which is divided into digits {\rm (}\textit{kalā}{\rm )}, [i.e. about] the division[s] {\rm (}\textit{kalā}{\rm )} of the entity [called] time {\rm (}\textit{kālatattva}{\rm )}. Two atomic units of time {\rm (}\textit{truṭi}{\rm )} is one twinkling {\rm (}\textit{nimeṣa}{\rm )}. One digit {\rm (}\textit{kalā}, cca. 1.6 second{\rm )} is twice a twinkling. \blankfootnote{1.18 1.18d and 1.19a are problematic in the light of 1.19b, which 
  redefines \textit{kalā} in harmony with the traditional
  interpretation, see e.g. \textit{Arthaśāstra} 2.20.33: \textit{trimśatkāṣṭhāḥ kalāḥ}.
  \nocite{Arthasastra1969}
  On divisions of time, see also, e.g., \MANU\ 1.64ff.
  I have calculated 1.6 second for one \textit{kalā} backwards, starting from one day {\rm (}see 1.20ab{\rm )}.
 }}

  \maintext{kalādviguṇitā kāṣṭhā kāṣṭhā vai triṃśatiḥ kalā |}%

  \maintext{triṃśatkalā muhūrtaś ca mānuṣena dvijottama }||\thinspace1:19\thinspace||%
\translation{Two digits {\rm (}\textit{kalā}{\rm )} form one bit {\rm (}\textit{kāṣṭhā}, 3.2 seconds{\rm )}. Thirty bits {\rm (}\textit{kāṣṭhā}{\rm )} is one digit {\rm (}\textit{kalā}?, 1.6 minutes{\rm )}. Thirty digits {\rm (}\textit{kalā}{\rm )} make up one section {\rm (}\textit{muhūrta}, 48 minutes{\rm )} in human terms, O great Brahmin. \blankfootnote{1.19 Underestand \textit{mānuṣena} as \textit{mānuṣasaṃkhyayā} {\rm (}1.21d{\rm )}.
 }}

  \maintext{muhūrtatriṃśakenaiva ahorātraṃ vidur budhāḥ |}%

  \maintext{ahorātraṃ punas triṃśan māsam āhur manīṣiṇaḥ }||\thinspace1:20\thinspace||%
\translation{Thirty sections {\rm (}\textit{muhūrta}{\rm )} are known to the wise as night and day [i.e. a full day]. Thirty days and nights are taught by the wise to be one month. }

  \maintext{samā dvādaśa māsāś ca kālatattvavido janāḥ |}%

  \maintext{śataṃ varṣasahasrāṇi trīṇi mānuṣasaṃkhyayā }||\thinspace1:21\thinspace||%
\translation{One year is twelve months [according to] people who know the entity of time. The time span of three hundred \blankfootnote{1.21 Note how a verb {\rm (}e.g. \textit{iti vadanti, iti prāhur}{\rm )} is missing in the first half-verse.
 }}

  \maintext{ṣaṣṭiṃ caiva sahasrāṇi kālaḥ kaliyugaḥ smṛtaḥ |}%

  \maintext{dviguṇaḥ kalisaṃkhyāto dvāparo yuga saṃjñitaḥ }||\thinspace1:22\thinspace||%
\translation{and sixty thousand years by human terms is said to be the Kali age {\rm (}\textit{yuga}{\rm )}. The Dvāpara age is known to be twice as long as the Kali age. \blankfootnote{1.22 Note the stem form noun \textit{yuga} metri causa, or rather the compound \textit{dvāparo-yuga-saṃjñitaḥ},
  and also \msM's unique but confused readings.
 }}

  \maintext{tretā tu triguṇā jñeyā catuḥ kṛtayugaḥ smṛtaḥ |}%

  \maintext{eṣā caturyugāsaṃkhyā kṛtvā vai hy ekasaptatiḥ }||\thinspace1:23\thinspace||%
\translation{The Tretā age is thrice [as long], the Kṛta age four [times as long as the Kali age]. This is the figure related to the four ages {\rm (}\textit{yuga}{\rm )}. Taking it seventy-one [times], \blankfootnote{1.23 Note the lengthened vowel in °\textit{yugā} {\rm (}metri causa{\rm )}.
 
  The `figure' mentioned in this verse is the sum of the duration of the four \textit{yuga}s, 
  which makes up one \textit{mahāyuga}:
  Kaliyuga = 360,000 years,
  Dvāparayuga = 720,000 years,
  Tretāyuga = 1,080,000 years,
  Dvāparayuga = 1,440,000 years; altogether 3,600,000 years. 71 \textit{mahāyuga}s make up
  a \textit{manvantara} {\rm (}= 255,600,000 years; cf. Manu 1.79{\rm )}. 
  One \textit{kalpa} is 14 \textit{manvantara}s {\rm (}= 3,578,400,000 years{\rm )}. 
  Ten thousand \textit{kalpa}s are one day of Brahmā, and his night is of the same length, which
  makes one full day of Brahmā 71,568,000,000,000 years. See next verses and
  e.g. \mycite{GonzalezCosmic}.
 }}

  \maintext{manvantarasya caikasya jñānam uktaṃ samāsataḥ |}%

  \maintext{kalpo manvantarāṇāṃ tu caturdaśa tu saṃkhyayā }||\thinspace1:24\thinspace||%
\translation{the knowledge about one time-span of a Manu {\rm (}\textit{manvantara}{\rm )} has been taught briefly. One \ae on {\rm (}\textit{kalpa}{\rm )} is fourteen \textit{manvantara}s in total. \blankfootnote{1.24 See 21.34ff on \textit{kalpa} etc.
 The accepted reading \textit{kalpo} in \textit{pāda} c is probably not original.
 }}

  \maintext{daśa kalpasahasrāṇi brahmāhaḥ parikalpitam |}%

  \maintext{rātrir etāvatī proktā munibhis tattvadarśibhiḥ }||\thinspace1:25\thinspace||%
\translation{Brahmā's day {\rm (}\textit{brahmāhar}{\rm )} is made up of ten thousand Kalpas. [Brahmā's] night is of the same [duration] according to the wise who know the truth. \blankfootnote{1.25 \msM\ has a separator sign {\rm (}|o|{\rm )} at the end of \textit{pāda} b, as if a section ended here.
 }}

  \maintext{rātryāgame pralīyante jagat sarvaṃ carācaram |}%

  \maintext{ahāgame tathaiveha utpadyante carācaram }||\thinspace1:26\thinspace||%
\translation{When [Brahmā's] night falls, the whole moving and unmoving universe dissolves. And when [his] daylight comes, the moving and unmoving [universe] is born. \blankfootnote{1.26 The plural form \textit{pralīyante} in \textit{pāda} a is metri causa for \textit{pralīyate},
  perhaps also influencing \textit{utpadyante} {\rm (}for \textit{utpadyate}{\rm )} in \textit{pāda} d,
  which in turn is used here to avoid an iambic pattern
  {\rm (}- - \shortsyllable\ - \shortsyllable\ - \shortsyllable\ -{\rm )}.
 }}

  \maintext{parārdhaparakalpāni atītāni dvijottama |}%

  \maintext{anāgataṃ tathaivāhur bhṛgurādimaharṣayaḥ }||\thinspace1:27\thinspace||%
\translation{One \textit{para} times \textit{parārdha} [number of, i.e. two hundred quadrillion times a hundred quadrillion] \textit{kalpas} have passed [so far], O great Brahmin. Bhṛgu and the other sages say that the future is the same [time span]. \blankfootnote{1.27 On the definition of the numbers \textit{para} and \textit{parārdha}, see verses 1.32--36.
 Note the peculiar compound \textit{bhṛgu-r-ādi-maharṣayaḥ}, for
  \textit{bhṛgvādimaharṣayaḥ}.
 }}

  \maintext{yathārkagrahatārendu bhramato dṛśyate tv iha |}%

  \maintext{kālacakraṃ bhramitvaiva viśramaṃ na ca vidmahe }||\thinspace1:28\thinspace||%
\translation{Just as the sun, the planets, the stars and the moon are percieved in this world as wandering around, wandering on the wheel of time {\rm (}\textit{kālacakra}{\rm )}, we can never have a rest. \blankfootnote{1.28 \textit{bhramato} in \textit{pāda} b seems to stand for the neuter participle \textit{bhramat}.
  Alternatively, \textit{bhramato} might mean `erroneously' {\rm (}\textit{bhrama-tas}, abl.{\rm )}, but this would
  make the verse difficult to interpret.
 I have corrected \textit{bhramatvaiva} to the standard form \textit{bhramitvaiva}, although the former
  might conceal a final verb {\rm (}\textit{bhramāmaḥ}?{\rm )}.
 }}

  \maintext{kālaḥ sṛjati bhūtāni kālaḥ saṃharate punaḥ |}%

  \maintext{kālasya vaśagāḥ sarve na kālavaśakṛt kvacit }||\thinspace1:29\thinspace||%
\translation{Time creates living beings and time destroys them again. Everything is under the control of time. There is nothing that can bring time under control. }

  \maintext{caturdaśa parārdhāni devarājā dvijottama |}%

  \maintext{kālena samatītāni kālo hi duratikramaḥ }||\thinspace1:30\thinspace||%
\translation{Fourteen \textit{parārdha} [fourteen hundred quadrillion] god kings, O Brahmin, have passed by over time, for time is difficult to overcome. \blankfootnote{1.30 Note that \textit{samatītāni} {\rm (}neuter{\rm )} most probably picks up \textit{devarājāḥ}
  {\rm (}masculine{\rm )} in this verse, or rather \textit{devarājā} stands for
  \textit{devarājānāṃ} and \textit{samatītāni} picks up \textit{°parārdhāni}. It is not clear to me
  what \textit{devarāja} {\rm (}`god king'{\rm )} means exactly {\rm (}Indra?{\rm )}.
 }}

  \maintext{eṣa kālo mahāyogī brahmā viṣṇuḥ paraḥ śivaḥ |}%

  \maintext{anādinidhano dhātā sa mahātmā namaskuru }||\thinspace1:31\thinspace||%
\translation{Time is [manifest] as a great yogin, as Brahmā, Viṣṇu and supreme Śiva, is beginningless and endless, is the creator, the great soul. Pay homage [to Time]. }

  \subchptr{parārdhādi}%

  \trsubchptr{\textit{Parārdha} etc.: numbers}%

  \maintext{vigatarāga uvāca |}%

  \maintext{śrutaṃ vai kālacakraṃ tu mukhapadmaviniḥsṛtam |}%

  \maintext{parārdhaṃ ca paraṃ caiva śrotuṃ vaḥ pratidīpitam }||\thinspace1:32\thinspace||%
\translation{Vigatarāga spoke: I have now heard about the `wheel of time' {\rm (}\textit{kālacakra}{\rm )} from [your] lotus mouth. [I wish] to hear about [the terms] \textit{parārdha} and \textit{para} [mentioned above], as elaborated by you. \blankfootnote{1.32 Although I have corrected the unmetrical \textit{vinisṛtam} in \textit{pāda} b to \textit{viniḥsṛtam},
  originally a metrical \textit{vinisritam} may have been intended.
 The reading of all manuscripts consulted, \textit{vinisṛtam}, 
  may be considered metrical if we interpret it, loosely, as \textit{vinisritam}.
  
 
  \textit{Pāda} d is suspect and my translation tentative.
  \msM's reading in \textit{pāda} d {\rm (}\textit{śrotuṃ naḥ pratidīyatāṃ}{\rm )} might make sense 
  {\rm (}`give it back/repeat it for us to hear'{\rm )}, but it sounds forced,
  as if the scribe tried to come up with a reading that he understood
  better than \textit{śrotuṃ vaḥ pratidīpitam}, the reading of the majority of the witnesses,
  which is in fact not easy to interpret. One would expect a phrase meaning
  `please tell me about these.' Finally, I have deicided to take \textit{vaḥ} as 
  instrumental {\rm (}`by you'{\rm )}.
 }}

  \maintext{anarthayajña uvāca |}%

  \maintext{ekaṃ daśaṃ śataṃ caiva sahasram ayutaṃ tathā |}%

  \maintext{prayutaṃ niyutaṃ koṭim arbudaṃ vṛndam eva ca }||\thinspace1:33\thinspace||%
\translation{Anarthayajña spoke: One, ten, a hundred, a thousand, and ten thousand {\rm (}\textit{ayuta}{\rm )}, a hundred thousand {\rm (}\textit{prayuta}{\rm )}, a million {\rm (}\textit{niyuta}{\rm )}, ten million {\rm (}\textit{koṭi}{\rm )}, a hundred million {\rm (}\textit{arbuda}{\rm )}, and one billion {\rm (}\textit{vṛnda}, 10\raise .5em\hbox{\footnotesize 9\thinspace}{\rm )}, \blankfootnote{1.33 See a similar teaching of numbers in \BRAHMANDAPUR\ 3.2.91ff.
 }}

  \maintext{kharvaṃ caiva nikharvaṃ ca śaṅku padmaṃ tathaiva ca |}%

  \maintext{samudro madhyam antaṃ ca parārdhaṃ ca paraṃ tathā }||\thinspace1:34\thinspace||%
\translation{ten billion {\rm (}\textit{kharva}{\rm )}, a hundred billion {\rm (}\textit{nikharva}{\rm )}, one trillion {\rm (}\textit{śaṅku}, 10\raise .5em\hbox{\footnotesize 12\thinspace}{\rm )}, and ten trillion {\rm (}\textit{padma}{\rm )}, a hundred trillion {\rm (}\textit{samudra}{\rm )}, one quadrillion {\rm (}\textit{madhya}, 10\raise .5em\hbox{\footnotesize 15\thinspace}{\rm )}, ten quadrillion {\rm (}\textit{[an]anta}{\rm )}, a hundred quadrillion {\rm (}\textit{parārdha}{\rm )}, and two hundred quadrillion {\rm (}\textit{para}{\rm )}. \blankfootnote{1.34 Note that \msPaperA\ inserts a line here. See apparatus.
 For \textit{anta} meaning \textit{ananta}, see 1.58cd--59ab. \msM's reading in \textit{pāda} d
  may be a result of an eyeskip to 1.35c.
 }}

  \maintext{sarve daśaguṇā jñeyāḥ parārdhaṃ yāvad eva hi |}%

  \maintext{parārdhadviguṇenaiva parasaṃkhyā vidhīyate }||\thinspace1:35\thinspace||%
\translation{Each should be known as powers of ten up to \textit{parārdha}. The number corresponding to \textit{para} is double that of \textit{parārdha}. }

  \maintext{parāt parataraṃ nāsti iti me niścitā matiḥ |}%

  \maintext{purāṇavedapaṭhitā mayākhyātā dvijottama }||\thinspace1:36\thinspace||%
\translation{There is no higher number than \textit{para}. This is my firm conviction, which is based on my readings of the Purāṇas and the Vedas and [which I have now] taught [to you], O great Brahmin. \blankfootnote{1.36 Note that \Ed\ inserts the line here that \msPaperA\ inserted above. See apparatus.
 }}

  \subchptr{brahmāṇḍam}%

  \trsubchptr{Brahmā's Egg: the Universe}%

  \maintext{vigatarāga uvāca |}%

  \maintext{brahmāṇḍaṃ kati vijñeyaṃ pramāṇaṃ jñāpitaṃ kvacit |}%

  \maintext{kati cāṅguli{-}m{-}ūrdhveṣu sūryas tapati vai mahīm }||\thinspace1:37\thinspace||%
\translation{Vigatarāga spoke: What is the extent of the Brahmāṇḍa [i.e. the universe]? Is it disclosed anywhere? From how many finger's breadths high does the sun heat the earth? \blankfootnote{1.37 The use of the singular next to numerals is one of the hallmarks of the \VSS\ 
  {\rm (}see p.~\pageref{singularwithnumerals}{\rm )}. This means that \textit{pāda} a may well refer to multiple \textit{brahmāṇḍa}s.
  Nevertheless, in the light of \VSS\ 2.2d {\rm (}\textit{pramāṇaṃ tasya vā kati}{\rm )}, I suspect that 
  the first question here could be rendered in slightly more standard Sanskrit as
  \textit{brahmāṇḍasya pramāṇaṃ kati yojanāni vijñeyaṃ}.
  \textit{cāpitaṃ kvacit} in \textit{pāda} b in the witnesses is enigmatic.
  One may conjecture \textit{prāpitaṃ} {\rm (}perhaps: `is it available somewhere?'{\rm )}, 
  or \textit{jñāpitaṃ} {\rm (}`is it disclosed somewhere?'{\rm )}. I have chosen the latter.
  The intended form may have been \textit{jñātaṃ kenacit} {\rm (}`is it known by anyone?'{\rm )},
  to which 1.38 below could be a reply. Of course, \textit{cāpitaṃ} could be analysed as
  \textit{cāpi taṃ} {\rm (}possibly for \textit{cāpi tat}{\rm )}, but that would help little, unless we
  imagine that the question is `and where is it?' {\rm (}\textit{cāpi tat kva}{\rm )}.
 
  
 My emendation of \textit{cāṅguli-mūrdheṣu} to \textit{cāṅguli{-}m{-}ūrdhveṣu} {\rm (}with a hiatus-filler{\rm )} 
  is based on \textit{ūrdhvatas} in 1.61d, which is part of the reply to the question posed in this line.
  In turn, \textit{aṅguli} here triggered an conjecture in 1.61c.
 }}

  \maintext{anarthayajña uvāca |}%

  \maintext{brahmāṇḍānāṃ prasaṃkhyātuṃ mayā śakyaṃ kathaṃ dvija |}%

  \maintext{devās te 'pi na jānanti mānuṣāṇāṃ ca kā kathā }||\thinspace1:38\thinspace||%
\translation{Anarthayajña spoke: How could I enumerate [all the details of] the Brahmāṇḍa, O twice-born? Even the gods do not know, not to mention humans. \blankfootnote{1.38 One would expect \textit{brahmāṇḍāni} in \textit{pāda} a instead of \textit{brahmāṇḍānāṃ},
  but we should probably understand \textit{brahmāṇḍānāṃ viśeṣān prasaṃkhyātuṃ...}
  The structure noun in genitive + verb meaning `telling' occurs also in 4.69a and \verify .
 }}

  \maintext{paryāyeṇa tu vakṣyāmi yathāśakyaṃ dvijottama |}%

  \maintext{brahmaṇā yat purākhyāto mātariśvā yathā tathā }||\thinspace1:39\thinspace||%
\translation{I shall teach [you], as far as I can, in due order and truthfully, that, O great Brahmin, which Mātariśvan was taught by Brahmā in the past. \blankfootnote{1.39 The claim that Brahmā taught Mātariśvan is confirmed in 1.64cd, and
  also, e.g., in \BrahmandaPur\ 3.4.58cd {\rm (}see the apparatus{\rm )}.
 }}

  \maintext{śivāṇḍābhyantareṇaiva sarveṣām iva bhūbhṛtām |}%

  \maintext{daśa nāma diśāṣṭānāṃ brahmāṇḍe kīrtitaṃ śṛṇu }||\thinspace1:40\thinspace||%
\translation{Ten names of all the [cosmic] rulers of each of the eight directions in Brahmā's Egg, [which is] inside Śiva's Egg, are being taught now, listen. \blankfootnote{1.40 My conjecture in \textit{pāda} b {\rm (}\textit{bhūbhṛtām}{\rm )} is based on the fact that the 
  readings transmitted in the MSS seem unintelligible, and, more importantly, that
  these names are said, in the subsequent verses, to belong to \textit{nāyaka}s {\rm (}`chiefs, lords'{\rm )},
  a possible synonym of \textit{bhūbhṛt} {\rm (}`a king'{\rm )}, and also that it is a minute intervention.
 
  In \textit{pāda} c, understand \textit{diśāṣṭānāṃ} as \textit{diśām aṣṭānāṃ} or \textit{digaṣṭakānāṃ}, 
  and note that one of the hallmarks of the language of the \VSS\ is the use
  of the singular in the proximity of numbers, where a plural would be expected {\rm (}\textit{daśa nāma}{\rm )}.
 }}
