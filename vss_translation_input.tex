
  \chptr{ekādaśamo 'dhyāyaḥ}
\addcontentsline{toc}{section}{Chapter 11}
\fancyhead[CO]{{\footnotesize\textit{Translation of chapter 11}}}%

  \trchptr{ Chapter Eleven }%

  \subchptr{caturāśramadharmavidhānaḥ}%

  \trsubchptr{Regulations on the Dharma of the four social disciplines}%

  \maintext{devy uvāca |}%

  \maintext{sarvayajñaḥ paraśreṣṭha asti anyaḥ surottama | }%

  \maintext{alpakleśa{-}m{-}anāyāsa arthaprāyaṃ vineśvara }||\thinspace11:1\thinspace||%
\translation{The Goddess spoke: O Paraśreṣṭha, O Surottama! Is there another [form of] sacrifice that is for all {\rm (}\textit{sarvayajña}{\rm )}, which is free of pain, easy, and which does not require an abundance of materials, O Īśvara? \blankfootnote{11.1 I understand \textit{pāda} c as containing a sandhi bridge thus: \textit{alpakleśa-m-anāyāsa}.
  The sandhi between \textit{pāda}s c and d is irregular, understand °\textit{anāyāsaḥ artha}°, or rather
  °\textit{anāyāso 'rtha}°.
 }}

  \maintext{sarvayajñaphalāvāpti daivataiś cāpi pūjitam |}%

  \maintext{kathayasva suraśreṣṭha mānuṣāṇāṃ hitāya vai }||\thinspace11:2\thinspace||%
\translation{For the benefit of mankind, teach me, O Suraśreṣṭha, how one obtains the fruits of [this] universal sacrifice {\rm (}\textit{sarvayajña}{\rm )}, which is praised even by the gods. \blankfootnote{11.2 \SDHS\ 1.7--11ab express a similar sentiment,
  using the word \textit{āyāsa}, similarly to \VSS\ 11.1c above, but giving a somewhat clearer 
  reason for asking for a new form of devotion, namely that twice-born members of society
  with limited financial resources struggle to perform 
  expensive Vedic rituals {\rm (}\textit{na śakyante yataḥ kartum alpavittair dvijātibhiḥ}{\rm )}:
  
 
  \textit{sanatkumāra uvāca}\thinspace | 
 
  \textit{bhagavan sarvadharmajña śivadharmaparāyaṇaḥ}\thinspace |
 
  \textit{śrotukāmāḥ paraṃ dharmam imaṃ sarve samāgatāḥ}\thinspace || 
 
  \textit{agniṣṭomādayo yajñā bahuvittakriyānvitāḥ}\thinspace |
 
  \textit{nātyantaphalabhūyiṣṭhā bahvāyāsasamanvitāḥ}\thinspace ||
 
  \textit{na śakyante yataḥ kartum alpavittair dvijātibhiḥ}\thinspace |
 
  \textit{sukhopāyam ato brūhi sarvakāmārthasādhakam}\thinspace | 
 
  \textit{hitāya sarvasatvānāṃ śivadharmaṃ sanātanam}\thinspace ||
 
  \textit{nandikeśvara uvāca}\thinspace |
 
  \textit{śrūyatām abhidhāsyāmi sukhopāyamahatphalam}\thinspace |
 
  \textit{paramasarvadharmāṇāṃ śivadharmaṃ śivātmakam}\thinspace ||
 
  \textit{śivena kathitaṃ pūrvaṃ pārvatyāḥ ṣaṇmukhasya ca}\thinspace |
 }}

  \maintext{maheśvara uvāca |}%

  \maintext{na tulyaṃ tava paśyāmi dayā bhūteṣu bhāmini |}%

  \maintext{kim anyat kathayiṣyāmi dayā yatra na vidyate }||\thinspace11:3\thinspace||%
\translation{Maheśvara spoke: I cannot see anything comparable to your compassion towards living beings, O Bhāminī. What else could I teach [you] with respect to which [your] compassion is not evident? \blankfootnote{11.3 I understand \textit{dayā} in \textit{pāda} b as if it were instrumental: \textit{tava dayayā bhūteṣu na tulyaṃ paśyāmi}.
  Alternatively, as suggested by Csaba Dezső, \textit{pāda}s ab could be interpreted as
  two sentences: `I cannot see anything comparable to you. [You have great]
  compassion towards living beings, O Bhāminī.'
 }}

  \maintext{sadāśivamukhāt pūrvaṃ śrutaṃ me varasundari |}%

  \maintext{śṛṇu devi pravakṣyāmi dharmasāram anuttamam }||\thinspace11:4\thinspace||%
\translation{I heard [the following] previously from Sadāśiva's mouth, O Varasundarī. Listen, O Goddess, I shall teach you the ultimate essence of Dharma. \blankfootnote{11.4 Note \textit{me} for \textit{mayā} in \textit{pāda} b, and the evident distinction here between Maheśvara,
  the interlocutor, and Sadāśiva, who, in this context seems to be superior, being the 
  ultimate source here of the following teaching. This might hint at a familiarity with 
  the Tantric sequence of \textit{tattva}s, on which see, e.g., \mycitep{NisvasaGoodall}{45}.
 }}

  \subchptr{gṛhasthaḥ{\rm (}?{\rm )}}%

  \trsubchptr{The householder}%

  \maintext{vinārthena tu yo yajñaḥ sa yajñaḥ sārvakāmikaḥ |}%

  \maintext{akṣayaś cāvyayaś caiva sarvapātakanāśanaḥ }||\thinspace11:5\thinspace||%
\translation{Sacrifice which [is performed] without materials satisfies all desires. It is undecaying and imperishable, and it removes all sins. \blankfootnote{11.5 I put a question mark after the subchapter heading here because in this
  chapter the category of the \textit{gṛhastha} never gets mentioned. This category is simply labelled 
  \textit{āśramaḥ prathamaḥ} in 11.25a. Nevertheless, it is most probably the \textit{gṛhastha} that is 
  implied, and it is mentioned elsewhere {\rm (}see 4.74c, 5.9a, and 15.17a, 
  which reads \textit{āśramāṇāṃ gṛhī śreṣṭho}{\rm )}. 
  The teaching on sacrifice without materials {\rm (}\textit{vinārthena yajñaḥ} or \textit{anarthayajñaḥ}{\rm )},
  which is fundamentally internalised sacrifice, is a central teaching of the \VSS:
  in addition to the present chapter, the expression 
  appears as the main interlocutor's name {\rm (}Anarthayajña{\rm )} in chapters 1--9 and 19--21, 
  and his life is discussed in chapter 22. Thus the name Anarthayajña or the concept of \textit{anarthayajña}
  appears in each major layer of the text. On this see pp.~\pageref{structure} ff,
  and \mycite{KissVolume2021}.
  That \textit{anarthayajña} is basically internalised worship is also hinted at in 10.12cd above in 13.2:
  
 
  \textit{svaśarīre sthito yajñaḥ svaśarīre sthitaṃ tapaḥ}\thinspace | 
 
  \textit{svaśarīre sthitaṃ tīrthaṃ śruto vistarato mayā}\thinspace ||
 }}

  \maintext{bahuvighnakaro hy artho bahvāyāsakaras tathā |}%

  \maintext{brahmahatyā ivendrasya pravibhāgaphalā smṛtā }||\thinspace11:6\thinspace||%
\translation{Material things {\rm (}\textit{artha}{\rm )} present many kinds of obstacle and [their acquisition causes] much trouble, similarly to Indra's murder of the Brahmin [Viśvarūpa], which yielded results [i.e. sins] that were distributed [among trees, lands etc.]. \blankfootnote{11.6 The context of \textit{pāda}s cd is this: Viśvarūpa was a son of Tvaṣṭṛ. 
  Viśvarūpa's heads were struck off by Indra and Indra's sins were 
  distributed among the earth, water, trees, and women. See, e.g., \BHAGP\ 6.9.6:
  
 
  \textit{brahmahatyām añjalinā jagrāha yad apīśvaraḥ}\thinspace | 
 
  \textit{saṃvatsarānte tad aghaṃ bhūtānāṃ sa viśuddhaye}\thinspace | 
 
  \textit{bhūmyambudrumayoṣidbhyaś caturdhā vyabhajad dhariḥ}\thinspace ||
  
 
  `Even though [Indra was] the Lord, he took on himself, with folded hands,
  the sin of killing a Brāhmaṇa. At the end of the year,
  Hari [= Indra] distributed that sin in four parts to the earth, water, trees and women
  for the self-purification of living beings.'
 }}

  \maintext{pañcaśodhyena śodhyeta arthayajño varānane |}%

  \maintext{śodhite tu phalaṃ śuddham aśuddhe niṣphalaṃ bhavet }||\thinspace11:7\thinspace||%
\translation{Material sacrifice can be purified with the five purifications, O Varānanā. When it is purified, the fruits are also pure. If it is not purified, it is fruitless. }

  \maintext{devy uvāca |}%

  \maintext{pañcaśodhye suraśreṣṭha saṃśayo 'tra bhaven mama |}%

  \maintext{kathayasva vibhāgena śrotum icchāmi tattvataḥ }||\thinspace11:8\thinspace||%
\translation{The Goddess spoke: I am not sure about the five purifications, O Sura\-śreṣṭha. Please teach [them to] me one by one, I want to hear [them] as [they] really [are]. }

  \maintext{rudra uvāca |}%

  \maintext{manaḥśuddhis tu prathamaṃ dravyaśuddhir ataḥ param |}%

  \maintext{mantraśuddhis tṛtīyā tu karmaśuddhir ataḥ param |}%

  \maintext{pañcamī sattvaśuddhis tu kratuśuddhiś ca pañcadhā }||\thinspace11:9\thinspace||%
\translation{Rudra spoke: First [there is] the purification of the mind, then [comes] the purification of the substances. The third is the purification of mantras, then the purification of the ritual. The fifth is the purification of Sattva. The purification of the sacrifice is [thus] fivefold. \blankfootnote{11.9 \textit{Pāda} a is unmetrical unless the \mutacumliquida\ is applied for the 
  first syllable of \textit{prathamaṃ}, turning the line into a \textit{na-vipulā}.
 
  
  Sets of five types of purification are a commonplace in Tantric Śaivism, but
  they are usually somewhat different form what we see here. They usually include
  \textit{ātmaśuddhi, sthānaśuddhi, dravyaśuddhi, mantraśuddhi} and \textit{liṅgaśuddhi}. See
  Goodall's article on this in \TAKIII\ s.v. \textit{dravyaśuddhi}.
 }}

  \maintext{manaḥśuddhir nāma aviparītabhāvanayā | }%

  \maintext{dravyaśuddhir nāma ananyāyopārjitadravyena  }||\thinspace11:10\thinspace||%
\translation{The purification of the mind is [achieved] by mentally creating what is not against [the rules]. The purification of the substances is [achieved] by [using] substances that were not obtained by unlawful means. \blankfootnote{11.10 The passage 11.10-11 is in fact prose.
 }}

  \maintext{mantraśuddhir nāma svaravyañjanayuktatayā | }%

  \maintext{kriyāśuddhir nāma yathākramāviparītatayā | }%

  \maintext{sattvaśuddhir nāma rajastama-apradhānatayā  }||\thinspace11:11\thinspace||%
\translation{Purification of the mantras is [achieved] by properly connecting vowels to consonants. Purification of the ritual is [achieved] by not altering the proper sequence [of the elements of ritual]. The purification of Sattva is [achieved] by the non-prevalence of Rajas and Tamas. }

  \maintext{vidhim evaṃ yadā śudhyed yadi yajñaṃ karoti hi |}%

  \maintext{tasya yajñaphalāvāptir janmamṛtyuś ca no bhavet }||\thinspace11:12\thinspace||%
\translation{When he has purified the ritual {\rm (}\textit{vidhi}{\rm )} thus and performs the sacrifice, he will obtain the fruits of the sacrifice, and will not undergo births and deaths [any more]. \blankfootnote{11.12 An alternative to my conjecture in \textit{pāda} a 
  {\rm (}\textit{yadā śudhyed} for \textit{yadā sūyed}, \textit{sūryed}, \textit{pūrya}, and \textit{pūyed}{\rm )}
  has been suggested by Dominic Goodall, namely that one could apply the reading of \msCb\ thus:
  \textit{yadāpūrya} {\rm (}`when having completed'{\rm )}.
 }}

  \maintext{vinārthena tu yo yajñaṃ karoti varasundari |}%

  \maintext{na tasya tatphalāvāptiḥ sarvayajñeṣv aśeṣataḥ }||\thinspace11:13\thinspace||%
\translation{But he who performs sacrifice without materials, O Varasundarī, will not [only] obtain its fruits, [but] of all sacrifices, without exception. \blankfootnote{11.13 I tentatively interpret \textit{sarvayajñeṣu} in \textit{pāda} d as a locative for genitive, and
  in a sense that does not reflect the meaning in which I took \textit{sarvayajñaḥ} in 11.1a above.
  Compare the conclusion of this section, 11.24cd: 
  \textit{āsahasrasya yajñānāṃ phalaṃ prāpnoti nityaśaḥ}.
 }}

  \maintext{yajñavāṭa kurukṣetraṃ sattvāvāsakṛtālayaḥ |}%

  \maintext{pratyāhāra mahāvedi kuśaprastara saṃyamaḥ }||\thinspace11:14\thinspace||%
\translation{The sacrificial ground is [the internal] Kurukṣetra. The abode made is [now:] dwelling in Truth {\rm (}\textit{sattva}{\rm )}. The great altar is the withdrawal of the senses {\rm (}\textit{pratyāhāra}{\rm )}. The seat made of \textit{kuśa} grass is constraint {\rm (}\textit{saṃyama}{\rm )} [in internalised sacrifice]. \blankfootnote{11.14 It would be easy to correct \textit{yajñavāṭa} in \textit{pāda} a to \textit{yajñavāṭaḥ}, and to normalise
  all the similarly positioned stem form nouns in the following verses because there are no
  metrical constrains that would prevent us from doing so,
  but it seems to me that there is a pattern here and that these stem forms
  are being emphasised, highlighted, or being items in a list
  {\rm (}see 11.14c and d, 15a, 16a and b, 17a, 18d, etc.{\rm )}. Nevertheless,
  some of the expression in the upcoming verses should be interpreted as bahuvrīhis
  qualifying the sacrificer or yogin. In fact, we could read \textit{yajñavāṭakurukṣetraḥ} and
  \textit{pratyāhāramahāvediḥ} as bahuvrīhis here.
 
  Kurukṣetra was defined as an internalised pilgrimage place in 10.12, which
  fits well the presently introduced teaching of internalised sacrifice.
  Both are summarised, together with bodily penance, in 13.2 {\rm (}see note to 11.5{\rm )}.
  The term \textit{sattvāvāsa} has elsewhere, but probably not here, a distinctively Buddhist flavour,
  denoting the seven or nine `abodes of beings,' see, e.g.,
  \mycitep{EdgertonHybrid}{vol.~2, s.v. \textit{sattvāvāsa}}, and
  \mycitep{SferraTorellaVolArticle}{1155}.
 Note that if \textit{pāda} c followed the pattern of \textit{pāda} a, namely that `X in Vedic ritual is now Y in this
  internalised sacrifice,' we would need to read \textit{mahāvedi pratyāhāra}, but that would be unmetrical.
 
  \textit{saṃyama} is mentioned only a few times in the \VSS\ {\rm (}e.g., in a similar
  context, in 22.12{\rm )}, and is never explained, in contrast with 
  the \textit{niyama}-rules mentioned in the next verse, which are expounded in detail in 5.1--8.44.
  \textit{saṃyama} may perhaps be used here in the sense in which it appears in the \YS:
  the yogic application, or appearance, of \textit{dhāraṇā}, \textit{dhyāna}, and \textit{samādhi} at
  the same time {\rm (}see \YogaS\ 3.1--4{\rm )}.
 }}

  \maintext{vidhi niyamavistāro dhyānavahniḥ pradīpitaḥ |}%

  \maintext{yogendhanasamijjvālatapodhūmasamākulaḥ }||\thinspace11:15\thinspace||%
\translation{Vedic injunction {\rm (}\textit{vidhi}{\rm )} is the large group of Niyama-rules. [Instead of the Vedic ritual fire, it is now] the fire of meditation {\rm (}\textit{dhyāna}{\rm )} [that] is lighted, which is flaring up by the fuel of the firewood of yoga and is abounding in the smoke of penance. \blankfootnote{11.15 I have chosen the reading in \textit{pāda} b that is the easiest to interpret. Alternatively,
  the intended expression may have been \textit{dhyānena vahniḥ pradīpitaḥ}.
 Instead of taking °\textit{samijjvāla}° as a tatpuruṣa compound in \textit{pāda} c {\rm (}°\textit{samidh-jvāla}°{\rm )}, 
  consider emending it to °\textit{samujjvāla}°, which would stand metri causa for °\textit{samujjvala}°.
 }}

  \maintext{pātranyāsa śivajñānaṃ sthālīpāka śivātmakaḥ |}%

  \maintext{ājyāhutim avicchinnaṃ lambakasruvapātitaḥ }||\thinspace11:16\thinspace||%
\translation{The placing down of the chalice is knowledge of Śiva. [The oblation of] boiled rice is [now the process of] be[com]ing Śiva. The continuous oblation of clarified butter {\rm (}\textit{ājyāhuti}{\rm )} is poured with the ritual ladle {\rm (}\textit{sruva}{\rm )} of the uvula {\rm (}\textit{lambaka}{\rm )}. \blankfootnote{11.16 The interpretation of \textit{pāda} b is tentative.
 Ignoring the problems concerning grammatical gender and case, 
  we may presume that the intended meaning in \textit{pāda}s cd could be expressed thus:
  \textit{ājyāhutir avicchinnā lambikāsruvena pātitā}. I suspect that \textit{lambaka} simply
  stands for \textit{lambikā} {\rm (}`uvula'{\rm )}, which fits the internalised nature of 
  this ritual. See also \textit{ghaṇṭikā} possibly as `uvula' in 10.32d.
 }}

  \maintext{dhāraṇādhvaryuvat kṛtvā prāṇāyāmaś ca ṛtvijaḥ |}%

  \maintext{tarkayuktaḥ savistāraḥ samādhir vayatāpanaḥ }||\thinspace11:17\thinspace||%
\translation{Transforming concentration {\rm (}\textit{dhāraṇā}{\rm )} into an Adhvaryu [priest, the phases of] breath control will be the [other Vedic] priests[, the Hotṛ, the Brahman, and the Udgātṛ]. Samādhi which involves reflection {\rm (}\textit{tarka}{\rm )} and which is extensive is the [Vedic ritual of] burning the oblation {\rm (}\textit{vaya}[\textit{s}]-\textit{tāpana}{\rm )}. \blankfootnote{11.17 Understand \textit{pāda}s a as \textit{dhāraṇām adhvaryuvat kṛtvā} {\rm (}\textit{dhāraṇā} in the MSS being in stem form{\rm )}.
 Note how taking 11.14c and 15b together with the present verse,
  all six auxiliaries of the \textit{ṣaḍaṅgayoga} of \VSS\ chapter 16 
  have now been mentioned in this chapter.
  See 16.18:
  
 
  \textit{pratyāhāras tathā dhyānaṃ prāṇāyāmaś ca dhāraṇā}\thinspace |
 
  \textit{tarkaś caiva samādhiś ca ṣaḍaṅgo yoga ucyate}\thinspace ||
  
 
  My interpretation of \textit{vayatāpana} in \textit{pāda} d as `burning of oblation' 
  {\rm (}\textit{vaya} possibly standing for \textit{vayas} metri causa{\rm )} is tentative.
 }}

  \maintext{brahmavidyāmayo yūpaḥ paśubandho manonmanaḥ |}%

  \maintext{śraddhā patnī viśālākṣi saṃkalpa pada śāśvatam }||\thinspace11:18\thinspace||%
\translation{The sacrificial post is made up of the knowledge about the Brahman. The tying of the sacrificial animal is [the mental state called] Manonmanas. [The householder's] wife is Faith, O Viśālākṣī. [His] ritual intention {\rm (}\textit{saṃkalpa}{\rm )} is [reaching] the eternal abode. \blankfootnote{11.18 The final section of \VSS\ chapter 20, a chapter on the \textit{tattva}s of Sāṃkhya,
  discusses the mental state of \textit{unmanas}:
  
 
  \textit{unmanastvaṃ gate vipra nibodha daśalakṣaṇam}\thinspace | 
 
  \textit{na śabdaṃ śṛṇute śrotraṃ śaṅkhabherīsvanād api}\thinspace || etc.
  
 
  Verse 11.49 below mentions \textit{manonmanas} in a similar context.
 In \textit{pāda} d, understand \textit{saṃkalpaḥ padaṃ śāśvatam} {\rm (}both \textit{saṃkalpa} and \textit{pada} are
  stem form nouns in the verse, the latter metri causa{\rm )}.
 }}

  \maintext{pañcendriyajayotpannaḥ puroḍāśo 'mṛtāśanaḥ |}%

  \maintext{brahmanādo mahāmantraḥ prāyaścittānilo jayaḥ }||\thinspace11:19\thinspace||%
\translation{Rice oblation is the consumption of the nectar of immortality that arises from the victory over the five senses. The great [Vedic] mantra is [now] Brahmā's sound. Expiation is victory over the breath. \blankfootnote{11.19 The term \textit{brahmanāda} in \textit{pāda} c may refer to the same concept 
  as \textit{brahmabilasvara} does in 11.29d. It may be the same as the {\rm (}haṭha{\rm )}yogic
  concept of \textit{mahānāda} {\rm (}`great sound' or `unstruck sound'{\rm )}, on which see
  \mycitep{MallinsonKhecari}{225, nn.~359 and 361}.
  My translation tentatively presupposes that \textit{mantra} in \textit{mahāmatra}
  refers to Vedic mantras, now contrasted with a yogic experience.
  {\rm (}See \textit{mahāmantra} referring to Vedic/Śrauta mantras in \SKANDAP\ 13.132cd: 
  \textit{śrutigītair mahāmantrair mūrtimadbhir upasthitaiḥ}.{\rm )}
 
  Understand \textit{pāda} d as \textit{prāyaścitto 'nilajayaḥ}. It would be possible
  to correct °\textit{cittānilo} to °\textit{citto 'nilo}, but since \textit{'nilajayaḥ} would
  be unmetrical and since stem form nouns abound in this chapter, 
  I believe that \textit{prāyaścittānilo} could be original.
 }}

  \maintext{somapāna parijñānam upākarma caturyamaḥ |}%

  \maintext{itihāsa jalasnānaṃ purāṇakṛta{-}m{-}ambaraḥ }||\thinspace11:20\thinspace||%
\translation{The consumption of Soma is [substituted now with] complete knowledge. The commencement [of the Vedic ritual] is the four Yama-rules. The ritual water-bath is [the study of] the Itihāsa. His garment is made of [his study of] the Purāṇas. \blankfootnote{11.20 \textit{caturyamaḥ} in \textit{pāda} b is baffling. The \VSS\ teaches ten 
  Yama-rules in 3.16--4.89. Dominic Goodall has suggested that \textit{caturyamaḥ} could
  stand for \textit{ca tu yamāḥ} metri causa. Another possibility would be 
  to interpret \textit{catur} as \textit{caturtha} {\rm (}`fourth'{\rm )} and then the phrase
  may refer to the fourth Yama-rule, absence of hostility {\rm (}\textit{ānṛśaṃsya}, 4.31--49{\rm )}.
 Note the stem form \textit{itihāsa} in \textit{pāda} c, and see the notes to verses 6.5 and 8.6
  to clarify what \textit{itihāsa} most probably means in the \VSS\ {\rm (}the \MBh{\rm )}. 
  There is a hiatus-filler {\rm (}\textit{-m-}{\rm )} in \textit{pāda} c in
  °\textit{kṛta-m-ambaraḥ}, which is a metrical solution for °\textit{kṛto 'mbaraḥ}.
 }}

  \maintext{iḍāsuṣumnāsaṃvedye snānam ācamanaṃ sakṛt |}%

  \maintext{saṃtoṣātithim ādṛtya dayābhūtadvijārcitaḥ }||\thinspace11:21\thinspace||%
\translation{Ritual bathing and sipping water once are [to be performed] at the confluence of the Iḍā and the Suṣumnā. Having honoured Contentment as a guest, he salutes the Brahmin that is [now] Compassion. \blankfootnote{11.21 For the teaching on the internalised pilgrimage places Gaṅgā, i.e. Suṣumnā, and
  Yamunā, i.e. Iḍā, and their internalised confluence, Prayāga, see 10.17. Note that
  Iḍā and Suṣumnā are then reinterpreted as Somatīrtha and Sūryatīrtha, respectively,
  in 10.20--21.
 \textit{saṃtoṣa}° is either meant to be compounded with °\textit{atithim} in \textit{pāda} c or 
  is in stem form for \textit{saṃtoṣam atithiṃ}; for the latter possibility cf., e.g., 
  11.17a above. Similarly, °\textit{dvija}° may be in stem form in \textit{pāda} d, for
  °\textit{dvijo 'rcitaḥ}, or simply correct it to the same.
 }}

  \maintext{brahmakūrca guṇātīta havirgandha nirañjanaḥ |}%

  \maintext{brahmasūtraṃ trayas tattvaṃ bodhanā muṇḍitaṃ śiraḥ }||\thinspace11:22\thinspace||%
\translation{The Brahmakūrca [observance] is the [state of mind called] `beyond the Qualities' {\rm (}\textit{guṇātīta}{\rm )}, the scent of the sacrifice is the `spotless' {\rm (}\textit{nirañjana}{\rm )} [state of mind]. [His] sacred thread is the three truths {\rm (}\textit{tattva}{\rm )}. The shaven head [of the \textit{snātaka}] is [now] enlightenment. \blankfootnote{11.22 Note the stem form nouns in \textit{pāda}s ab.
 
  On the \textit{brahmakūrca} observance, see, e.g., \mycitep{KaneHistory}{vol.~4, 146},
  where the references given include \MITAKSARA\ ad \YAJNS\ 3.314: 
  \textit{yadā punaḥ pūrvedyur upoṣyāparedyuḥ samantrakaṃ saṃyujya}
  \textit{samantrakam eva pañcagavyaṃ pīyate tadā brahmakūrca ity ākhyāyate};
  `And when one fasts one day, and on the next day mixes the five products of the cow
  together while reciting mantras, and drinks [the mixture] while reciting mantras again,
  that is called \textit{brahmakūrca}.'
  
  On the \textit{guṇātīta} state of mind, see 9.39--43. See the term \textit{nirañjana} mentioned
  as a quality of the soul {\rm (}\textit{jīva}{\rm )} in 1.11 and 15.4, of the \textit{puruṣa} in 20.3, 
  as a state of mind in 11.48, and as one of ten meditative states in 22.30.
 
  
 It is difficult to know what the three \textit{tattva}s mentioned in \textit{pāda} c are.
  {\rm (}Understand \textit{trayas tattvaṃ} as \textit{tattvatrayaṃ}, \textit{trīṇi tattvāni}, \textit{tritattvāni}, or
  \textit{tritattvaṃ}.{\rm )}
  \VSS\ chapter 4 teaches four \textit{tattva}s as objects of meditation:
  \textit{ātman}, \textit{vidyā}, \textit{bhava}, and \textit{sūkṣma} {\rm (}see, e.g., 4.72{\rm )}. \VSS\ chapter 6
  discusses five \textit{tattva}s: \textit{sūrya}, \textit{soma}, \textit{agni}, \textit{sphaṭika}, and \textit{sūkṣma} 
  {\rm (}see, e.g., 6.7{\rm )}. \VSS\ chapter 20 enumerates the 25 \textit{tattva}s of Sāṃkhya.
  One possibility would be to interpret the set of three \textit{tattva}s as
  the three \textit{padārtha}s of the Śaivasiddhānta, \textit{pati}, \textit{paśu}, and \textit{pāśa};
  see, e.g., \TAKIII, s.v. \textit{patipaśupāśa}.
  Dominic Goodall has tentatively suggested reading here in \VSS\ 11.22c, with \msNa,
  \textit{brahmasūtratrayaṃ tattvaṃ} {\rm (}`the three strands of the sacred thread is truth'{\rm )}.
  The problem is firstly that we have \textit{trayas tattvaṃ} repeated in 11.29c below, 
  and secondly that what we need here is three entities compared to the three strands
  of the sacred thread. What is clear here is that even the investiture of the
  sacred thread {\rm (}\textit{upanayana}{\rm )} is supposed to be internalised in this teaching
  of non-material sacrifice.
 }}

  \maintext{nivṛttyādi caturvedaś catuḥprakaraṇāsanaḥ |}%

  \maintext{dakṣiṇām abhayaṃ bhūte dattvā yajñaṃ yajet sadā }||\thinspace11:23\thinspace||%
\translation{The four Vedas are [now] \textit{nivṛtti} etc. His seat is the four \textit{prakaraṇa}s. He should always perform a[n internalised] sacrifice after donating the priestly fee of providing being[s] with freedom from danger. \blankfootnote{11.23 My assumption is that \textit{pāda} a here hints at those four, later five, categories,
  called \textit{kalā}s, that are well-known from Tantric Śaivism: 
  \textit{nivṛtti}, \textit{pratiṣṭhā}, \textit{vidyā}, \textit{śānti}, and \textit{śāntyatīta}.
  For this, I had to emend the reading found in all witnesses consulted, \textit{nivṛtyā}°.
  I consider \textit{nivṛti} for \textit{nivṛtti} a common and plausible error. 
  As Dominic Goodall has suggested, here the four \textit{kalā}s, 
  originally possibly the four Śaktis of the Lord, may be reinterpreted as yogic states.
  The fact that the \VSS\ is aware of only four \textit{kalā}s here may hint at a relatively
  early date of composition of this section {\rm (}see Introduction pp.~\pageref{dating} ff{\rm )}.
  On the history and interpretation of these \textit{kalā}s,
  see \TAKII\ s.v. \textit{kalā} 6.
 
  \textit{catuḥprakaraṇāsanaḥ} may be taken as \textit{catuḥprakaraṇāṇy āsanam}, or, as I take it in my
  translation, a bahuvrīhi compound qualifying the practitioner. As to
  what the four \textit{prakaraṇa}s {\rm (}`chapters'?{\rm )} refer to here, I am without a clue.
  Perhaps the phrase was meaningful in a context whereof this section was
  taken out. It may stand for yogic \textit{karaṇa}s, postures, which are 
  mentioned, but then not clearly described, in 16.1:
  
 
  \textit{adhunā śrotum icchāmi yogasadbhāvanirṇayam}\thinspace | 
 
  \textit{karaṇaṃ ca yathānyāyaṃ kathayasva sureśvara}\thinspace ||
 }}

  \maintext{vinārthaṃ yajñasamprāptiḥ kathitā te varānane |}%

  \maintext{āsahasrasya yajñānāṃ phalaṃ prāpnoti nityaśaḥ }||\thinspace11:24\thinspace||%
\translation{The attainment of sacrifice without materials has been taught to you, O Varānanā. [The sacrificer] will in any case obtain the fruits of up to a thousand [ordinary Vedic] sacrifices. }

  \maintext{āśramaḥ prathamas tubhyaṃ kathito 'sti varānane |}%

  \maintext{sadāśivena saddharmaṃ daivatair api pūjitam }||\thinspace11:25\thinspace||%
\translation{The first discipline {\rm (}\textit{āśrama}{\rm )} has been taught to you, O Varānanā, through Sadāśiva; [this is] the true Dharma, revered also by the gods. \blankfootnote{11.25 \textit{sadāśivena} in \textit{pāda} c could also be interpreted as the agent of \textit{pūjitam} in \textit{pāda} d
  {\rm (}`it is revered by Sadāśiva'{\rm )}, but Sadāśiva was mentioned as the original
  teacher of this ritual in 11.4 above, which makes it probable that
  he is being referred to in a similar manner here. Cf. also 11.30 below.
 }}

  \subchptr{brahmacārī}%

  \trsubchptr{The chaste one}%

  \maintext{brahmacaryaṃ nibodhedaṃ śṛṇuṣvāvahitā śubhe |}%

  \maintext{dvitīyam āśramaṃ devi sarvapāpavināśanam }||\thinspace11:26\thinspace||%
\translation{[Now] learn about this, about the practice of chastity {\rm (}\textit{brahmacarya}{\rm )}. Listen with attentively, O Śubhā. [It is] the second discipline {\rm (}\textit{āśrama}{\rm )}, O Devī, the destroyer of all sins. \blankfootnote{11.26 \textit{idaṃ} in \textit{nibodhedaṃ} in \textit{pāda} a sounds clumsy with \textit{brahmacaryaṃ} {\rm (}lit. `listen to this
  practice of chastity'{\rm )} but in fact the \MBH\ and the Purāṇas contain countless similar,
  albeit smoother, expressions, e.g., \MBH\ 5.145.15ab 
  {\rm (}\textit{duryodhana nibodhedaṃ kulārthe yad bravīmi te}{\rm )},
  \BRAHMAP\ 133.10ab
  {\rm (}\textit{bharadvāja nibodhedaṃ vākyaṃ mama samāsataḥ}{\rm )}, etc.
 See some remarks on the disciplines, or life-stages {\rm (}\textit{āśrama}{\rm )},
  and especially on their order, in the \VSS\ in \mycite{KissVolume2021}.
 }}

  \maintext{vrataṃ brahmaparaṃ dhyānaṃ sāvitrī prakṛti{-}r{-}layam |}%

  \maintext{brahmasūtrākṣaraṃ sūkṣmaṃ triguṇālaya mekhalam }||\thinspace11:27\thinspace||%
\translation{Religious observance is [now] meditation focussed on the Brahman. The Sāvitrī [hymn] is absorption in Prakṛti. The Brahmanical cord {\rm (}\textit{brahmasūtra}{\rm )} is the subtle syllable. His girdle is now the abode of the three Qualities {\rm (}\textit{guṇa}{\rm )}. \blankfootnote{11.27 One could emend \textit{prakṛtir layam} in \textit{pāda} b to the expected \textit{prakṛtau layaḥ}
  {\rm (}see, e.g., \AGNIP\ 379.1d: \textit{vairāgyāt prakṛtau layam}{\rm )}.
  Nevertheless, I retained the reading of \msCa\msNa\msNc\Ed\ because
  it may have been the way in which the compound \textit{prakṛtilaya} was originally made 
  metrical. In other words, I suspect the \textit{-r-} to be only a link
  between the two elements of this compound. I also retained the neuter ending.
  Compare 16.8d, where the same expression is transmitted in all the witnesses
  consulted so far as \textit{prakṛtālayam}.
 
  
 Note the stem form nouns in \textit{pāda}s cd {\rm (}°\textit{sūtra} and °\textit{ālaya}{\rm )}.
  The `subtle syllable' may be \textit{oṃ} {\rm (}cf. 1.9--10{\rm )}, traditionally analysed as
  made up of three sounds, here corresponding to the three strands of the
  sacred thread. In \textit{pāda} d, \textit{triguṇālaya} might rather
  mean `absorption in the three Qualities' {\rm (}\textit{triguṇeṣu layaḥ}{\rm )}
  although in my translation I translate it as \textit{triguṇa-ālayaḥ}.
 }}

  \maintext{dama daṇḍa dayā pātraṃ bhikṣā saṃsāramocanam |}%

  \maintext{tryāyuṣaṃ dvyakṣarātītaṃ jñānabhasma-alaṅkṛtam }||\thinspace11:28\thinspace||%
\translation{His staff is self-restraint, his bowl compassion. Alms are liberation from transmigration {\rm (}\textit{saṃsāra}{\rm )}. The Tryāyuṣa is the one beyond the two syllables. [The three lines are] prepared with the ashes of knowledge. \blankfootnote{11.28 The Tryāyuṣa is a Vedic mantra, see, e.g., \textit{Ṛgveda-khila} 5.3.6:
  \textit{tryāyuṣam jamadagneḥ kaśyapasya tryāyuṣam\thinspace |}
  \textit{agastyasya tryāyuṣam yad devānām tryāyuṣam tan no astu tryāyuṣam}\thinspace |;
  `The threefold vitality of [the sage] Jamadagni, that of [the sage] Kaśyapa, 
  that which is that of the gods---may it be ours!' {\rm (}translation based on 
  \mycitep{SaivaUtopia}{28}{\rm )}. `In the Vedic domestic ritual codes, 
  this is the mantra to be recited over the razor or over the student who is
  about to be shaven before bathing at the end of his studies' {\rm (}ibid.{\rm )}.
  In \SIVAUP\ 5.20ab, this mantra is prescribed to accompany the application of the 
  three lines on the forehead. Thus here in \VSS\ 11:28cd, \textit{tryāyuṣa} and
  the mention of ashes make it clear that the next element of the ritual
  life of the \textit{brahmacārin} to be internalised is the application 
  of the \textit{tripuṇḍra}. As for the \textit{dvyakṣarātīta}, which should be a mantra,
  it perhaps means a three-syllable mantra, possibly \textit{a-u-m} or \textit{śivāya}.
 }}

  \maintext{snānavrataṃ sadāsatyaṃ śīlaśaucasamanvitam |}%

  \maintext{agnihotra trayas tattvaṃ japa brahmabilasvaraḥ }||\thinspace11:29\thinspace||%
\translation{The bath-vow is life-long truthfulness, accompanied by the purity and moral conduct. The Agnihotra sacrifice is the three \textit{tattva}s. Recitation is the sound at the aperture of Brahmā. \blankfootnote{11.29 On the problem of understanding what the three \textit{tattva}s are in this text, and on the 
  phrase \textit{trayas tattvaṃ}, see notes on verse 11.22 above.
  Perhaps \textit{brahmabilasvara} in \textit{pāda} d refers to the same concept as \textit{brahmanāda} does in 11.19c.
 }}

  \maintext{dvitīya āśramo devi yathāha bhagavān śivaḥ |}%

  \maintext{mamāpi kathitaṃ tubhyaṃ janmamṛtyuvināśanam }||\thinspace11:30\thinspace||%
\translation{The second discipline {\rm (}\textit{āśrama}{\rm )} has [now] been taught also to you as Lord Śiva taught it, O Devī, to me. It is the destruction of birth and death. \blankfootnote{11.30 One may consider correcting \textit{mamā}° to \textit{mayā}° {\rm (}`it has been taught by \textit{me}'{\rm )},
  but \textit{mama}, linked to the first hemistich, may be original, and \textit{api}, then slightly 
  unusually placed in the sense of `too/also' {\rm (}as, e.g., in \RAGHU\ 5.44 and 9.8c{\rm )},
  starting a new clause.
 }}

  \subchptr{vānaprasthaḥ}%

  \trsubchptr{The forest-dweller}%

  \maintext{vānaprasthavidhiṃ vakṣye śṛṇuṣvāyatalocane |}%

  \maintext{yathāśrutaṃ yathātathyam ṛṣidaivatapūjitam }||\thinspace11:31\thinspace||%
\translation{Listen, O Long-eyed goddess, I shall teach you the forest-dweller's way of life, which is revered by the sages and the gods, as I heard it, as it [really] is. }

  \maintext{vairāgyavanam āśritya niyamāśramam āharet |}%

  \maintext{śīlaśailadṛḍhadvāre prākāre vijitendriyaḥ }||\thinspace11:32\thinspace||%
\translation{Having taken to the forest of indifference, he should take residence in the ashram of Niyama-rules, within walls that have the stone-strong gate of moral conduct, with his sense faculties conquered. \blankfootnote{11.32 \textit{āharet} {\rm (}`should take away, get, use'{\rm )} in \textit{pāda} b is suspect; 
  \textit{āvaset} {\rm (}`should settle'{\rm )} or \textit{āśrayet} {\rm (}`should take refuge'{\rm )} 
  would make more sense in this context.
 }}

  \maintext{adhibhūtaḥ smṛto mātā adhyātmaś ca pitā tathā |}%

  \maintext{adhidaivikam ācāryo vyavasāyāś ca bhrātaraḥ }||\thinspace11:33\thinspace||%
\translation{One's mother is the material realm, one's father the Self, one's guru the divine. Resolutions are one's brothers. \blankfootnote{11.33 Note the \mutacumliquida\ applied in \textit{pāda} d: the syllable \textit{brā}
  does not make the previous syllable long.
  
 
  I have accepted Dominic Goodall's suggestion to emend \textit{adhibhautika} in \textit{pāda} c
  to \textit{adhidaivika}. In this way, we arrive at the well-know triad of \textit{adhibhūta},
  \textit{adhyātma}, and \textit{adhidaivika} {\rm (}or more often: \textit{ādhibhautika}, \textit{ādhyātmika}, and
  \textit{ādhidaivika}; see, e.g. \YBH\ ad \YS\ 1.31 and 3.22, and \SK\ 1.1 in most
  commentators' interpretation{\rm )}. \textit{adhibhautika} in \textit{pāda} c may be the result of
  an eyeskip to \textit{pāda} a, and the final \textit{-m} of \textit{adhidaivika} could be 
  interpreted as a hiatus-filler. The triad in question usually qualify
  three types of suffering or bad omen: pertaining to the material world, 
  one's own self or body, and to the world of gods, respectively.
  Here in the \VSS, they seem to refer to 
  realms of knowledge, or as \BhG\ 8.1--4, a possible source for the present verse,
  define them, \textit{adhibhūta} is mundane existence {\rm (}\textit{kṣaro bhāvaḥ}{\rm )},
  \textit{adhyātma} is one's true nature {\rm (}\textit{svabhāvaḥ}{\rm )}, 
  and \textit{adhidaivata} the \textit{puruṣa}.
 }}

  \maintext{śrutiḥ smṛtiḥ smṛtā bhāryā prajñā putraḥ kṣamānujaḥ |}%

  \maintext{maitrī bandhur jaṭā cāpaṃ karuṇā supavitrakam |}%

  \maintext{muditā mauna catvāraḥ sarvakāryam upekṣakā }||\thinspace11:34\thinspace||%
\translation{Śruti and Smṛti are his wives, Wisdom his son, Patience his little brother. Benevolence is his kinsman, his twisted hair [and] his bow. Compassion his sacred thread. Sympathy is the four ways of observing silence. All his religious duties are equanimity. \blankfootnote{11.34 \textit{bhāryā} in \textit{pāda} a is probably meant to be in the dual {\rm (}\textit{bhārye}{\rm )} but
  the use of the singular could be original. Note how notions expressed by
  feminine nouns in \textit{pāda} b are associated with male relatives {\rm (}\textit{prajñā} is a son, 
  \textit{kṣamā} a brother{\rm )}.
 
  
 In \textit{pāda} c, \textit{jaṭā cāpaṃ} is problematic. One would expect here an abstract notion 
  corresponding to a real-life element of the forest-dweller life, as in the above verses.
  Also, a bow is not naturally associated with the life of a forest hermit.
  \textit{jaṭā} and \textit{cāpa} are either still identified with \textit{maitrī} 
  {\rm (}that is how I translate the \textit{pāda}{\rm )} 
  or there is a need to emend, e.g., to \textit{jaṭācāraḥ} {\rm (}`good conduct is his twisted hair'{\rm )}.
  I prefer the former solution because in this way the four Buddhist \textit{brahmavihāra}s,
  \textit{maitrī-karuṇā-muditā-upekṣā}, appear in one uninterrupted sequence. 
  One could even emend to \textit{jaṭā cāyaṃ} or \textit{jaṭā cāpi}. The \textit{brahmavihāra}s may seem to be
  out of context in a Brahmanical text but the source for them may have been
  \YS\ 1.33: 
  \textit{maitrīkaruṇāmuditopekṣāṇāṃ sukhaduḥkhapuṇyāpuṇyaviṣayāṇāṃ bhāvanātaś}
  \textit{cittaprasādanam}. See them mentioned also in verse 4.72 above, and in 11.56 below.
  
  
 Note \textit{mauna} in \textit{pāda} e in stem form, and \textit{upekṣakā} for \textit{upekṣā}, both metri causa.
  For the four \textit{mauns}s, see 4.69.
 }}

  \maintext{yamavalkalasaṃvītas tapaḥkṛṣṇājinādharaḥ |}%

  \maintext{uttarāsaṅgam āsīno yogapaṭṭadṛḍhavrataḥ }||\thinspace11:35\thinspace||%
\translation{He is clothed in the Yama-rules instead of a garment made of bark, and he wears penance instead of the skin of a black antelope. He is seated on the highest level of non-attachment, and a firm observance is his yoga-belt. \blankfootnote{11.35 I think that \msNc's \textit{jinādharaḥ} in \textit{pāda} b may be the original reading, and
  it lengthens the final \textit{a} of \textit{jina}° metri causa, and the remaining sources
  try to restore the standard form of \textit{ajina} and thus 
  ruin the metre. Cf., e.g., \MBH\ 1.123.18:
  
 
  \textit{sa kṛṣṇaṃ maladigdhāṅgaṃ kṛṣṇājinadharaṃ vane}\thinspace |
 
  \textit{naiṣādiṃ śvā samālakṣya bhaṣaṃs tasthau tadantike}\thinspace ||
 
 
  The accusative \textit{uttarāsaṅgam} in \textit{pāda} c is acceptable, but one may
  understand the final \textit{-m} as a hiatus-filler after a locative {\rm (}°\textit{saṅga āsīno}{\rm )},
  or in the middle of a compound {\rm (}°\textit{saṅgāsīno}{\rm )}.
 }}

  \maintext{vedaghoṣeṇa ghoṣeṇa prāṇāyāmo 'gnihāvanam |}%

  \maintext{jitaprāṇa mṛgākūlo dhṛti yajñaḥ kriyā japaḥ }||\thinspace11:36\thinspace||%
\translation{Fire sacrifice accompanied by the sound of murmuring the Vedas is breath-control accompanied by [its] hissing. The herd of deer [in the forest where the forest-dweller normally lives] is [now his] conquered breaths. [Now] sacrifice is resolution, ritual is mantra-recitation. \blankfootnote{11.36 \textit{hāvana} in \textit{pāda} b stands for \textit{havana} metri causa.
 I suspect that °\textit{mṛgākūlo} in \textit{pāda} c stands for an unmetrical \textit{mṛgakulo}.
  Incidentally, even by inverting the order of the two elements in this \textit{pāda}, there would
  remain the metrical error of two \textit{laghu}s: \textit{mṛgakulo jitaprāṇo}.
  Also, note °\textit{prāṇa} and \textit{dhṛti} in \textit{pāda}s cd as nouns in stem form.
 }}

  \maintext{arthasaṃgraha śāstreṣu sakhā damadayādayaḥ |}%

  \maintext{śivayajñaṃ prayuñjīta sādhanāṣṭakapūjanam }||\thinspace11:37\thinspace||%
\translation{His treasures are in the \textit{śāstra}s, his companions are self-control, compassion, etc. He should perform sacrifice to Śiva as worship of the eight [yogic] practices {\rm (}\textit{sādhana}{\rm )}. \blankfootnote{11.37 See the word \textit{saṃgraha} {\rm (}here in stem form{\rm )} used probably 
  in a similar sense in 11.45 below.
 See a reference to eight \textit{sādhana}s in \DHARMP\ 2.1 {\rm (}quoted in the apparatus
  to the present verse in the critical edition{\rm )}. These may or may not point to
  the same set of practices.
 }}

  \maintext{pañcabrahmajalaiḥ pūtaḥ satyatīrthaśivahrade |}%

  \maintext{snānam ācamanaṃ kṛtvā saṃdhyātrayam upāsayet }||\thinspace11:38\thinspace||%
\translation{Purified by the water of the five Brahma[-mantras], bathing and sipping water in the auspicious {\rm (}\textit{śiva}{\rm )} lake at the pilgrimage place of truthfulness, he should honour the three junctures of the day. \blankfootnote{11.38 The reading of the witnesses in \textit{pāda} d, \textit{upāśrayet}, might be acceptable, but
  I consider my emendation, \textit{upāsayet}, better, especially because that is
  the verb used in 11.58d below, in a similar context.
 }}

  \maintext{akṣamālā purāṇārthaṃ japa śāntaṃ divāniśam |}%

  \maintext{jñānasalilasampūrṇa{-}m{-}itihāsakamaṇḍaluḥ }||\thinspace11:39\thinspace||%
\translation{The rosary is [now] the meaning of the Purāṇas. Recitation is [now his] peace of mind day and night. His jar of epics is filled with the water of knowledge. \blankfootnote{11.39 \textit{Pāda} b may allow for various interpretations. The one I have chosen seems to be the
  simplest. It involves a stem form noun, \textit{japa}, and \textit{śāntaṃ}
  in the sense of \textit{śāntiḥ}.
 Understand the middle of \textit{pāda}s cd as containing a hiatus-filler to bridge the vowels
  in a standard °\textit{pūrṇa itihāsa}°.
 }}

  \maintext{pañcakarmakriyotkrānti japa pañcavidhaḥ sukham |}%

  \maintext{sādhanaṃ śivasaṃkalpo yogasiddhiphalapradaḥ }||\thinspace11:40\thinspace||%
\translation{The actions of the five [medical] procedures are yogic suicide. Recitation is the five kinds of pleasure. The \textit{Śivasaṃkalpa} [hymn] is [yogic] practice {\rm (}\textit{sādhana}{\rm )}, which yields fruits of yoga accomplishments. \blankfootnote{11.40 My translation of this verse is tentative. Note that \textit{utkrānti} 
  {\rm (}usually in similar contexts: `yogic suicide'{\rm )} is a \textit{yogāṅga} in chapter 16.
  See also 17.31, which mentions suicide by entering fire.
  I take \textit{japa} tentatively as a stem form noun, and \textit{pañcavidhaḥ} as if
  it read \textit{pañcavidhaṃ}. \BODHISBH\ 1.3.4 teaches five kinds of \textit{sukha}:
  \textit{hetusukhaṃ veditasukhaṃ duḥkhaprātipakṣikaṃ sukhaṃ veditopacchedasukham avyabādhyañ 
  ca pañcamaṃ sukham}. This would not be the first occasion in this
  chapter to see Buddhist categories introduced, see 11.34 above.
 
  
 I think that \Ed's silent correction of °\textit{pradaḥ} to °\textit{pradam},
  making \textit{pāda} d qualifying \textit{sādhanaṃ} in \textit{pāda} c, is
  reasonable, but since this form is not attasted in any of the witnesses
  consulted, I hesitate to follow it. Nevertheless, I understand the
  sentence thus: that which is normally the \textit{śivasaṃkalpa} is now,
  in this internalised version of the forest-dweller's life,
  {\rm (}yogic{\rm )} practice that yields \textit{siddhi}s. I suppose that the reference is
  to \VajasaneyiS\ 34.1--6, usually called \textit{Śivasaṃkalpa}:
  
 
  \textit{yaj jāgrato dūram udaiti daivaṃ} 
 
  \textit{tad u suptasya tathaivaiti}\thinspace |
 
  \textit{dūraṃgamaṃ jyotiṣāṃ jyotir ekaṃ}
 
  \textit{tan me manaḥ śivasaṃkalpam astu}\thinspace || etc.
  
 
  See this hymn referred to in \MANU\ 11.251 in a context of expiation: 
  
 
  \textit{sakṛj japtvāsyavāmīyaṃ śivasaṃkalpam eva ca}\thinspace |
 
  \textit{apahṛtya suvarṇaṃ tu kṣaṇād bhavati nirmalaḥ}\thinspace ||
  
 
  In Olivelle's translation: `A man who has stolen gold, on the other hand, 
  becomes instantly stainless by reciting softly[? rather: once] the Asyavāmīya hymn 
  and the Śivasaṃkalpa formulas.' Other texts that reference the \Sivasamkalpa\
  include \NISVGUHYA\ 2.77, \AGNIP\ 259.74, and \LINPU\ 1.64.76. See more on
  the \Sivasamkalpa\ in \mycite{RgvedaKhila} and \citeyear{Sivasankalpopanisad}.
 }}

  \maintext{saṃtoṣaphalam āhāraḥ kāmakrodhaparājitaḥ |}%

  \maintext{āśāpāśajayābhyāso dhyānayogaratipriyaḥ |}%

  \maintext{atithibhyo 'bhayaṃ dattvā vānaprasthaś cared vratam }||\thinspace11:41\thinspace||%
\translation{His food is the fruit of contentment. He conquers lust and anger. His practice is the victory over the trap of hope. He loves the joy of yoga meditation. The forest-dweller should observe his vow by providing guests with fearlessness. \blankfootnote{11.41 Cf. 11.23 above on giving \textit{abhaya} to guests.
 }}

  \maintext{vānaprastham ayaṃ dharmaṃ gadita yat pūrvam avadhāritaṃ}%

 \nonanustubhindent \maintext{saṃsāroddharaṇam anityaharaṇam ajñānanirmūlanam |}%

  \maintext{prajñāvṛddhikaram amoghakaraṇaṃ kleśārṇavottāraṇaṃ}%

 \nonanustubhindent \maintext{janmavyādhiharam akarmadahanaṃ sevet sa dharmottamam }||\thinspace11:42\thinspace||%
\translation{One should follow the Dharma of the forest-dweller, the supreme Dharma, which has been taught and which, if first understood, will deliver one from transmigration, will remove transient existence, uproot ignorance, increase wisdom, will be fruitful, will deliver one from the flood of affliction, will remove rebirth and disease, and will burn one's bad karma. \blankfootnote{11.42 In some MSS, \textit{pāda} a gives a first impression of being an \textit{anuṣṭubh} line 
  with metrical problems. But, as Dominic Goodall remarked, 
  the variants suggest that it may belong to the upcoming Śārdūlavikrīḍita verse.
  This is all the more so because that verse would otherwise 
  contain only three \textit{pāda}s. My reconstruction of the now \textit{pāda} a 
  is still highly problematic;
  \textit{gadita} is in stem form, and the final syllable of \textit{pūrvam} scans as heavy.
  While these are acceptable in the language of the \VSS\ {\rm (}see pp.~\pageref{language} ff{\rm )},
  some elements remain questionable, namely
  the first syllable of \textit{dharmaṃ} as a short syllable, and the second
  syllable of \textit{avadhāritaṃ} as long. The \textit{pāda} may have gone through some
  heavy corruption, possibly involving an eyeskip to 11.43a.
  It is also unclear if the first half of the \textit{pāda} is to be interpreted as 
  \textit{vānaprastham ayaṃ}, \textit{vānaprastho 'yaṃ} [\textit{sevet}], 
  \textit{vānaprastham idaṃ}, or \textit{vānaprasthamayaṃ}. 
  I translate the first of these options, taking both \textit{ayaṃ} and \textit{dharmaṃ} 
  as neuter nominative.
 Word-final short syllables treated as heavy also 
  appear in \textit{pāda}s bcd: °\textit{haraṇam} {\rm (}twice{\rm )}, °\textit{karam}, and °\textit{haram}.
 }}

  \subchptr{parivrājakaḥ}%

  \trsubchptr{The wandering mendicant}%

  \maintext{parivrājakadharmo 'yaṃ kīrtayiṣyāmi tac chṛṇu |}%

  \maintext{sukhaduḥkhaṃ samaṃ kṛtvā lobhamohavivarjitaḥ }||\thinspace11:43\thinspace||%
\translation{Here follows the wandering religious mendicant's Dharma. Listen, I shall teach it to you. Making joy and pain equal, he gets rid of greed and folly. }

  \maintext{varjayen madhu māṃsāni paradārāṃś ca varjayet |}%

  \maintext{varjayec ciravāsaṃ ca paravāsaṃ ca varjayet }||\thinspace11:44\thinspace||%
\translation{He should avoid honey and meat, as well as others' wives. He should avoid staying [at one place] for long and also staying at others' places. }

  \maintext{varjayet sṛṣṭabhojyāni bhikṣām ekāṃ ca varjayet |}%

  \maintext{varjayet saṃgrahaṃ nityam abhimānaṃ ca varjayet }||\thinspace11:45\thinspace||%
\translation{He should avoid food that has been thrown away and he should avoid getting alms [always] from the same household. He should always refrain from accumulating wealth and from self-conceit. \blankfootnote{11.45 See the term \textit{arthasaṃgraha} in 11.37c, probably in the same meaning as
  \textit{saṃgraha} here in \textit{pāda} c.
 }}

  \maintext{susūkṣmaṃ manasā dhyātvā dṛśau pādaṃ vinikṣipet |}%

  \maintext{na kupyeta anālābhe lābhe vāpi na harṣayet }||\thinspace11:46\thinspace||%
\translation{Meditating on the extremely subtle one, he should cast his eyes on his feet [when begging]. He should not get angry when he does not receive anything, and when he does, he should not rejoice. \blankfootnote{11.46 On meditation on the subtle one {\rm (}\textit{susūkṣma}{\rm )}, see Intro\verify.
 
  \textit{Pāda} b is suspect as it is transmitted in the MSS {\rm (}in most sources it is
  \textit{śucau pādaṃ vinikṣipet}: `he should place his foot in the pure'?{\rm )}. 
  My conjecture {\rm (}\textit{dṛśau}{\rm )} results in something close to 
  the early Buddhist rule given in the Pāli \Patimokkha\ on begging that says
  that the monk should not make eye-contact with the donor.
  See \Patimokkha\ Sekhiyā 7--8 and 28:
  
 
  \textit{okkhittacakkhu antaraghare gamissāmīti sikkhā karaṇīyā}\thinspace |
 
  \textit{okkhittacakkhu antaraghare nisīdissāmīti sikkhā karaṇīyā}\thinspace | [...] 
 
  \textit{pattasaññī piṇḍapātaṃ paṭiggahessāmīti sikkhā karaṇīyā}\thinspace |
  
 
  In Bhikkhu Ñāṇatusita's translation {\rm (}\mycitep{Patimokkha}{294 and 303}{\rm )}:
  ` ``I shall go with the eyes cast down inside an inhabited area,''
  thus the training is to be done.
  ``I shall sit with the eyes cast down inside an inhabited area,''
  thus the training is to be done. [...]
  ``I shall accept alms-food paying attention to the bowl,'' thus the training is to be done.'
  The last of these sentences opens up another possibility for emending the text of the \VSS:
  \textit{pādaṃ} might perhaps be considered as a corruption from \textit{pātraṃ} {\rm (}`on his bowl'{\rm )}.
  I am not aware of similar Dharmaśāstric teachings on avoiding eye-contact. 
  The closest could be \BAUDHDHS\ 1.5.11 on observing silence while begging {\rm (}\textit{vāgyatas tiṣṭhet}{\rm )}.
  Not even \MANU\ 5.50--60, a longer section on begging, prohibits eye-contact.
  If there are indeed no Brahmanical rules on this topic, the verse above in the \VSS\ 
  could be another piece of evidence for Buddhist influence.
 }}

  \maintext{arthatṛṣṇāsv anudvigno roṣe vāpi sudāruṇe |}%

  \maintext{stutinindā samaṃ kṛtvā priyaṃ vāpriyam eva vā }||\thinspace11:47\thinspace||%
\translation{He should not be agitated with regards to thirst for material things, or to violent anger. He should take praise and reproach equal, as well as pleasant and unpleasant things. \blankfootnote{11.47 In \textit{pāda} c, understand \textit{stutinindā} as a dual {\rm (}or singular{\rm )} accusative.
 }}

  \maintext{niyamās tu parīdhānaṃ saṃyamāvṛtamekhalaḥ |}%

  \maintext{nirālambaṃ manaḥ kṛtvā buddhiṃ kṛtvā nirañjanām }||\thinspace11:48\thinspace||%
\translation{His garment is the Niyama-rules, and he is girded by the girdle of constraint {\rm (}\textit{saṃyama}{\rm )}. He should make his mind supportless, his intellect spotless, \blankfootnote{11.48 On \textit{saṃyama}, see notes on 11.14 above.
 }}

  \maintext{ātmānaṃ pṛthivīṃ kṛtvā khaṃ ca kṛtvā manonmanam |}%

  \maintext{tridaṇḍaṃ triguṇaṃ kṛtvā pātraṃ kṛtvākṣaro 'vyayaḥ }||\thinspace11:49\thinspace||%
\translation{the ground his self, the sky the mind-nonmind [state of mind] {\rm (}\textit{manonmana}{\rm )}, the three staffs [of the \textit{parivrājaka}] the three qualities {\rm (}\textit{guṇa}{\rm )}, and the bowl the imperishable syllable. \blankfootnote{11.49 °\textit{kṣaram avyayam} in \textit{pāda} d would be hypermetrical, that is probably 
  why the nominative appears here.
 }}

  \maintext{nyased dharmam adharmaṃ ca īrṣyādveṣaṃ parityajet |}%

  \maintext{nirdvandvo nityasatyastho nirmamo nirahaṃkṛtaḥ }||\thinspace11:50\thinspace||%
\translation{He should throw away Dharma and Adharma, and should give up envy and hatred. He should be indifferent to opposites, always dwell in truthfulness, being unselfish, humble. }

  \maintext{divasasyāṣṭame bhāge bhikṣāṃ saptagṛhaṃ caret |}%

  \maintext{na cāsīta na tiṣṭheta na ca dehīti vā vadet }||\thinspace11:51\thinspace||%
\translation{He should go on his alms round visiting seven houses at the eighth part of the day. He should not sit down, he should not stay, and he should not say `Give me!' \blankfootnote{11.51 According to \Manu\ 6.56, the wandering ascetic should go around begging
  after people have finished their meal.
  \MBH\ Suppl. 1.52.36 {\rm (}= \LAKSMINARS\ 1.238.18ab = \VASISTHADHS\ 11.36ab{\rm )} suggests 
  that the `eighth part of the day' is around sunset:
  \textit{divasasyāṣṭame bhāge mandībhūte divākare}.
 }}

  \maintext{yathālābhena varteta aṣṭau piṇḍān dine dine |}%

  \maintext{vastrabhojanaśayyāsu na prasajyeta vistaram }||\thinspace11:52\thinspace||%
\translation{He should live on what is available, on eight bites a day. He should not stick to items of clothes, food, or a bed, for long. }

  \maintext{nābhinandeta maraṇaṃ nābhinandeta jīvitam |}%

  \maintext{indriyāṇi vaśaṃkṛtvā kāmaṃ hatvā yatavrataḥ }||\thinspace11:53\thinspace||%
\translation{He should nor rejoice in death, he should not rejoice in life. Having conquered his senses, and having killed his desire, firm in his observances, }

  \maintext{atītaṃ ca bhaviṣyaṃ ca na bhikṣuś cintayet sadā |}%

  \maintext{krodhamānamadadarpān parivrāḍ varjayet sadā }||\thinspace11:54\thinspace||%
\translation{the mendicant {\rm (}\textit{bhikṣu}{\rm )} should never think about the past or the future. The wandering mendicant {\rm (}\textit{parivrāj}{\rm )} should always avoid anger, self-conceit, intoxication, and pride. \blankfootnote{11.54 \textit{Pāda} c is a \textit{sa-vipulā}.
 }}

  \maintext{virāgaṃ tu dhanuḥ kṛtvā prāṇāyāmaguṇair yutam |}%

  \maintext{dhāraṇāśaratīkṣṇena mṛgaṃ hatvā manendriyam }||\thinspace11:55\thinspace||%
\translation{Making indifference a bow which is strung with the strings of breath-control, he should kill the beast that is the mind and the sense-faculties with the sharp-pointed arrow of concentration. \blankfootnote{11.55 Understand \textit{pāda} c as \textit{dhāraṇātīkṣṇaśareṇa}.
 }}

  \maintext{maitrīkhaḍgasutīkṣṇena saṃsārāriṃ nikṛntayet |}%

  \maintext{karuṇāvartacakreṇa krodhamattagajaṃ jayet |}%

  \maintext{muditāvarmabaddhāṅgas tūṇaṃ pūrṇam upekṣayā }||\thinspace11:56\thinspace||%
\translation{He should stab the enemy that is transmigration with the extremely sharp knife of friendliness. He should defeat the rutting elephant of anger with the whirling discus of compassion. His body should be clad in the armour of sympathy, his quiver full of equanimity. \blankfootnote{11.56 Understand \textit{pāda} a as \textit{maitrīsutīkṣṇakhaḍgena}, which is even metrical.
 Note the four Buddhist \textit{brahmavihāra}s, \textit{maitrī}, \textit{karuṇā}, \textit{muditā}, and \textit{upekṣā},
  mentioned in this verse. They appear also in verses 4.71 and 11.56 above.
 }}

  \maintext{anakṣaraṃ paraṃ brahma cintayet satataṃ dvija |}%

  \maintext{brahmaṇo hṛdayaṃ viṣṇur viṣṇoś ca hṛdayaṃ śivaḥ |}%

  \maintext{śivasya hṛdayaṃ saṃdhyā tasmāt saṃdhyām upāsayet }||\thinspace11:57\thinspace||%
\translation{He should constantly recall the unutterable syllable which is the supreme Brahman, O Brahmin. Brahmā's heart is Viṣṇu. Viṣṇu's heart is Śiva. Śiva's heart is the junctures of the day. Therefore he should worship the junctures. }

  \maintext{saṃsārārṇavatāraṇaṃ śubhagatiḥ sa brahma saṃdhyākṣaraṃ}%

 \nonanustubhindent \maintext{dhyāyen nityam atandrito hy anupamaṃ vyaktātmavedyaṃ śivam |}%

  \maintext{rūpair varṇaguṇādibhiś ca vihitaṃ durlakṣyalakṣyottamaṃ}%

 \nonanustubhindent \maintext{yatnoddhṛtya samāśrayet suraguruṃ sarvārtihartā haram }||\thinspace11:58\thinspace||%
\translation{[Śiva] is deliverance from the ocean of mundane existence, the path to happiness, the Brahman, the junctures, the [sacred] syllable. One should always, unweariedly, meditate on matchless Śiva, who is to be recognized as the manifest soul. He should take refuge in Hara, who is devoid of form, colour, qualities etc., who is the supreme aim which is difficult to discern, honouring the divine guru with effort, who removes all pain. \blankfootnote{11.58 Note \textit{vihita} in \textit{pāda} c probably in the sense of `devoid of.'
 I take \textit{yatnoddhṛtya} in \textit{pāda} d as \textit{yatnenoddhṛtya}, \textit{yatna} being in stem form, and
  °\textit{hartā} as nominative for accusative.
 }}

\centerline{\maintext{\dbldanda\thinspace iti vṛṣasārasaṃgrahe caturāśramadharmavidhāno nāmādhyāya ekādaśamaḥ\thinspace\dbldanda}}
\translation{Here ends the eleventh chapter in the \textit{Vṛṣasārasaṃgraha} called Regulations concerning the four life-stages.}
