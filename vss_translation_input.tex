
  \chptr{aṣṭamo 'dhyāyaḥ}
\addcontentsline{toc}{section}{Chapter 8}
\fancyhead[CO]{{\footnotesize\textit{Translation of chapter 8}}}%

  \trchptr{ Chapter Eight}%

  \subchptr{niyameṣu svādhyāyaḥ {\rm {\rm (}5{\rm )}}}%

  \trsubchptr{The fifth Niyama-rule: Study}%

  \maintext{pañcasvādhyāyanaṃ kāryam ihāmutra sukhārthinā |}%

  \maintext{śaivaṃ sāṃkhyaṃ purāṇaṃ ca smārtaṃ bhāratasaṃhitām }||\thinspace8:1\thinspace||%
\translation{Five kinds of study are to be pursued by those who wish to be happy in this life and in the other. [One should study] Śaiva [teachings], Sāṃkhya [philosophy], the Purāṇa[s], the Smārta [tradition] and the \textit{Bhāratasaṃhitā} [i.e. the \textit{Mahābhārata}]. \blankfootnote{8.1 The form \textit{svādhyāyana}, for the more standard \textit{svādhyayana},
  does occur in several, typically Buddhist, texts. See, e.g., 
  the \textit{Mahāpratisarā-mahāvidyārājñī} {\rm (}\mycitep{HidasMahapratisara}{153}{\rm )}:
  \textit{mahāyānodgrahaṇa\-likhana\-vācana\-paṭhana\-svādhyāyana\-śravaṇa\-dhāraṇā\-bhi\-yuktānāṃ 
  pari\-pālikeyaṃ mahādhāraṇī}.
 Note the accusative ending of \textit{°saṃhitām}. One may correct it to the nominative, or rather 
  supply an active verb such as \textit{adhīyāt}. I have choosen the latter
  in the translation.
 }}

  \maintext{śaivaṃ tattvaṃ vicinteta śaivapāśupatadvaye |}%

  \maintext{atra vistarataḥ proktaṃ tattvasārasamuccayam }||\thinspace8:2\thinspace||%
\translation{He should reflect on the Śaiva truth in both Śaiva and Pāśupata [teachings]. In those teachings the whole essence of truth is taught extensively. \blankfootnote{8.2 Note that both \textit{śaivaṃ tattvaṃ} in \textit{pāda} a 
  and the reading \textit{śaivapāśupatadvaye} 
  in \textit{pāda} b are weakly attested. In spite of these uncertainties, 
  I think that this form of the current half-verse is the only one that 
  yields the appropriate meaning. Alternatively, read \textit{śaivatattvaṃ} in \textit{pāda} a; compare 8.3a.
 }}

  \maintext{saṃkhyātattvaṃ tu sāṃkhyeṣu boddhavyaṃ tattvacintakaiḥ |}%

  \maintext{pañcatattvavibhāgena kīrtitāni maharṣibhiḥ }||\thinspace8:3\thinspace||%
\translation{Those who reflect on the truth {\rm (}\textit{tattva}{\rm )} can grasp the truth of enumeration [of ontological principles/reality levels] {\rm (}\textit{saṃkhyātattva}{\rm )} from Sāṃkhya [texts]. The great sages taught [those twenty-five] \textit{tattva}s [of Sāṃkhya] as being in groups of five. \blankfootnote{8.3 In \textit{pāda} d, \textit{kīrtitāni} picks up an implied \textit{tattvāni}.
 }}

  \maintext{purāṇeṣu mahīkoṣo vistareṇa prakīrtitaḥ |}%

  \maintext{adhordhvamadhyatiryaṃ ca yatnataḥ sampraveśayet }||\thinspace8:4\thinspace||%
\translation{In the Purāṇas it is the sheath[s]/layers of the world that are described extensively. One can definitely enter [the realms] below, above, in the middle, and horizontally [all around]. \blankfootnote{8.4 Note that \textit{tirya} seems to be an acceptable nominal stem in this text for \textit{tiryañc}. 
  I understand the causative form \textit{sampraveśayet} as non-causative.
  \Ed's silent emendation to \textit{samprabodhayet} is understandable since to `enter' these realms
  the study of the Purāṇas makes little sense, at least when taken literally.
  Kengo Harimoto has suggested emending to \textit{sampradeśayet}.
 }}

  \maintext{smārtaṃ varṇāśramācāraṃ dharmanyāyapravartanam |}%

  \maintext{śiṣṭācāro 'vikalpena grāhyas tatra aśaṅkitaḥ }||\thinspace8:5\thinspace||%
\translation{The Smārta [tradition] deals with the conduct of the social classes {\rm (}\textit{varṇa}{\rm )} and disciplines {\rm (}\textit{āśrama}{\rm )}, and with the procedures of Dharma and lawsuits. Good conduct is to be gathered from it without hesitation, with certainty. \blankfootnote{8.5 Compare \textit{pāda}s ab with 3.15cd:
  \textit{smārto varṇāśramācāro yamaiś ca niyamair yutaḥ}. 
  The term \textit{smārta} seems to be used here in the sense of Dharmaśāstra.
 The \textit{avagraha} in \textit{'vikalpena} is not to be found in the witnesses and has
  therefore been supplied by the editor. The form \textit{aśaṅkitaḥ} is less then perfect here,
  and may have been intended as an ablative {\rm (}\textit{aśaṅkā-taḥ}{\rm )}, as suggested by Judit Törzsök,
  or adverbially {\rm (}\textit{aśaṅkitam}{\rm )}, or even more probably as a loosely added subject {\rm (}for 
  \textit{aśaṅkitena}{\rm )}.
 }}

  \maintext{itihāsam adhīyānaḥ sarvajñaḥ sa naro bhavet |}%

  \maintext{dharmārthakāmamokṣeṣu saṃśayas tena chidyate }||\thinspace8:6\thinspace||%
\translation{A man who studies the legend[s] {\rm (}\textit{itihāsa}{\rm )} will become omniscient. [All his] doubts about religious duty {\rm (}\textit{dharma}{\rm )}, finanicial gain {\rm (}\textit{artha}{\rm )}, carnal desires {\rm (}\textit{kāma}{\rm )} and liberation {\rm (}\textit{mokṣa}{\rm )} will be eliminated. \blankfootnote{8.6 As it is clear from 8.1d, what is primarily meant by \textit{itihāsa} is the \textit{Mahābhārata}.
 }}

  \subchptr{niyameṣv upasthanigrahaḥ {\rm {\rm (}6{\rm )}}}%

  \trsubchptr{The sixth Niyama-rule: Sexual restraint}%

  \maintext{śṛṇuṣvāvahito vipra pañcopasthavinigraham |}%

  \maintext{striyo vā garhitotsargaḥ svayaṃmuktiś ca kīrtyate |}%

  \maintext{svapnopaghātaṃ viprendra divāsvapnaṃ ca pañcamaḥ }||\thinspace8:7\thinspace||%
\translation{Listen with great attention, O Brahmin, to the five [spheres of] sexual restraint. Women, forbidden ejaculation, and masturbation are mentioned [in this context, as well as] offence while sleeping, O Brahmin, and fantasising, as the fifth. }

  \subsubchptr{striyaḥ}%

  \trsubsubchptr{Women}%

  \maintext{agamyā strī divā parve dharmapatny api vā bhavet |}%

  \maintext{viruddhastrīṃ na seveta varṇabhraṣṭādhikāsu ca }||\thinspace8:8\thinspace||%
\translation{A woman is not to be approached sexually in the daytime and on the four nights of the changes of the Moon {\rm (}\textit{parvan}{\rm )}, even if she is one's lawful wife. One should not have sex with a woman who is taboo or with one who has lost her class {\rm (}\textit{varṇa}{\rm )} or is [of a] superior [\textit{varṇa} than oneself]. \blankfootnote{8.8 Understand \textit{parve} as \textit{parvani} {\rm (}thematisation of the stem in \textit{-an}{\rm )}.
 
  Compare \MANU\ 11.175 {\rm (}Olivelle's edition and translation{\rm )}:
  \textit{maithunaṃ tu samāsevya puṃsi yoṣiti vā dvijaḥ\thinspace |
  goyāne 'psu divā caiva savāsāḥ snānam ācaret\thinspace ||}
  {\rm (}`If a twice-born has sexual intercourse with
  a man or a woman in an ox-cart, on water,
  or during the day, he should bathe with his clothes on.'{\rm )};
  and \MANU\ 3.45 {\rm (}Olivelle's edition and translation{\rm )}:
  \textit{ṛtukālābhigāmī syāt svadāranirataḥ sadā\thinspace |
  parvavarjaṃ vrajec caināṃ tadvrato ratikāmyayā\thinspace ||}
  {\rm (}`Finding his gratification always in his wife, he should have sex 
  with her during her season. Devoted solely to her, he may go to 
  her also when he wants sexual pleasure, except on the days of the moon's change.'{\rm )}
 The nominative °\textit{strī} in \textit{pāda} in most witnesses may 
  be the result of an eyeskip to \textit{strī} in \textit{pāda} a. Note how the paper MS
  is the only one transmitting a fully correct form.
 }}

  \subsubchptr{garhitotsargaḥ}%

  \trsubsubchptr{Forbidden ejaculation}%

  \maintext{ajameṣagavādīnāṃ vaḍavāmahiṣīṣu ca |}%

  \maintext{garhitotsargam ity etad yatnena parivarjayet }||\thinspace8:9\thinspace||%
\translation{Intercourse with goats, sheep, cows, mares, buffalo-cows is called forbidden ejaculation, which is to be avoided at all cost. \blankfootnote{8.9 Understand \textit{°ādīnāṃ} in \textit{pāda} a as standing for the locative case.
 Understand \textit{°sargam} as neuter nominative {\rm (}instead of °\textit{sargaḥ}{\rm )} or alternatively
  understand \textit{pāda} c with a hiatus bridge: \textit{garhitotsarga-m-ity etad}.
 }}

  \subsubchptr{svayaṃmuktiḥ}%

  \trsubsubchptr{Masturbation}%

  \maintext{ayonikaṣaṇā vāpi apānakaṣaṇāpi vā |}%

  \maintext{svayaṃmuktir iyaṃ jñeyā tasmāt tāṃ parivarjayet }||\thinspace8:10\thinspace||%
\translation{Rubbing himself against something else than a female sexual organ or rubbing his anus, are called masturbation {\rm (}\textit{svayaṃmukti}{\rm )}, therefore these are to be avoided. \blankfootnote{8.10 The conjecture in \textit{pāda} a involves minimal intervention 
  and makes the sentence much more meaningful than the 
  version transmitted. {\rm (}Consider also \textit{ayonya}°.{\rm )}
  Compare \MANU\ 11.174 {\rm (}Olivelle's edition and translation{\rm )}:
  \textit{amānuṣīṣu puruṣa udakyāyām ayoniṣu\thinspace |
  retaḥ siktvā jale caiva kṛcchraṃ sāṃtapanaṃ caret\thinspace ||}
  'If someone ejaculates his semen in non-human females, in a man, in a
  menstruating woman, in any place other than the vagina, or on water, he should
  perform the Sāntapana penance...'
 The variant \textit{strī} for \textit{tāṃ} in \textit{pāda} d in the \Ed\ may be one example of the numerous
  silent interventions made by Naraharināth in his edition.
 }}

  \subsubchptr{svapnaghātam}%

  \trsubsubchptr{Offence while sleeping}%

  \maintext{svapnaghātaṃ dvijaśreṣṭha aniṣṭaṃ paṇḍitaiḥ sadā |}%

  \maintext{svapne strīṣu ramante ca retaḥ prakṣarate tataḥ }||\thinspace8:11\thinspace||%
\translation{Offence while sleeping, O best of Brahmins, has always been [considered] undesirable by the learned. [If] one enjoys women while dreaming, his semen will issue. }

  \subsubchptr{divāsvapnam}%

  \trsubsubchptr{Daydreaming}%

  \maintext{divāśayaṃ na kartavyaṃ nityaṃ dharmapareṇa tu | }%

  \maintext{svargamārgārgalā hy etāḥ striyo nāma prakīrtitāḥ }||\thinspace8:12\thinspace||%
\translation{Sleeping by day should always be avoided by those who are intent on Dharma. These women are called `the bolts [that block the gate to] the path to heaven.' \blankfootnote{8.12 It is not crystal clear why `sleeping by day' or `daydreaming/fantasising' should count as
  one of the offences against sexual restraint. A line may have dropped out here.
 \textit{Pāda}s cd are clumsy and out of context. They would fit verse 8.8 better.
 }}

  \subchptr{niyameṣu vratapañcakam {\rm {\rm (}7{\rm )}}}%

  \trsubchptr{The seventh Niyama-rule: religious observances}%

  \maintext{mārjārakabakaśvānagomahīvratapañcakam |}%

  \subsubchptr{mārjārakavratam}%

  \trsubsubchptr{The Cat Observance}%

  \maintext{svaviṣṭhamūtraṃ bhūmīṣu chādayed dvijasattama |}%

  \maintext{sūryasomānumodanti mārjāravratikeṣu ca }||\thinspace8:13\thinspace||%
\translation{[Hear about] the five religious observances [called] the cat, the heron, the dog, the cow, and the earth. He buries his own urine and f\ae ces in the ground, O truest Brahmin. He rejoices [seeing] the sun and the moon when performing the cat observance. \blankfootnote{8.13 Note \textit{°viṣṭha°} for \textit{viṣṭhā} metri causa in \textit{pāda} c {\rm (}\textit{ma-vipulā}{\rm )}.
  Alternatively, read \textit{svaviṣṭhāmūtra bhūmīṣu} {\rm (}\textit{pathyā} with
  stem form noun{\rm )}.
 Note the stem form \textit{sūryasoma} for \textit{sūryasomau} {\rm (}\textit{sūryasomāv anu°}{\rm )} in \textit{pāda} e. 
  It is not entirely clear why cats would rejoice seeing the Sun and the Moon.
  Perhaps this remark refers to the fact that cats can be active both
  in the daytime and at night.
 }}

  \subsubchptr{bakavratam}%

  \trsubsubchptr{The Heron Observance}%

  \maintext{bakavac cendriyagrāmaṃ suniyamya tapodhana |}%

  \maintext{sādhayec ca manastuṣṭiṃ mokṣasādhanatatparaḥ }||\thinspace8:14\thinspace||%
\translation{O great ascetic, one should suppress all his senses like a heron, and should cultivate the peace of the mind, focusing on achieving liberation. \blankfootnote{8.14 Cranes are compared to ascetics here probably because of the similarity of
  their posture when relaxing standing on one leg to ascetics performing penance 
  standing on one leg {\rm (}such as the ascetic, and a cat, depicted on the 
  famous relief in Mahabalipuram{\rm )}. More specifically, herons apply `meditation,' so to say,
  when fishing, as Olivelle points out commenting on \MANU\ 7.106a {\rm (}\textit{bakavac cintayed arthān}{\rm )},
  quoting Bhāruci's explanation ad loc.:
  `Just as naturally in order to catch a mass of fish who are safe in 
  their water-fort the `heron' finds an effective means to take them 
  by employing meditation, after dedicating himself to the task, so 
  the king should not be despondent realizing that if one employs
  abundant thought on one's affairs even aims very difficult to achieve are attained.'
  \mycitep{OlivelleManu}{298}.
  {\rm (}\textit{yathā abdurgāśrayaṃ matsyabalaṃ svabhāvatas tadgrahaṇārthaṃ bakaḥ paryupāsanayā
  tadgrahaṇopāyaṃ dhyānayogād āsādayati, evam arthacintābhiyogātiśayena suduṣprāpā 
  apy arthā āsādyanta iti matvā na nirvedaṃ gacchet\thinspace |}{\rm )}
 }}

  \subsubchptr{śvānavratam}%

  \trsubsubchptr{The Dog Observance}%

  \maintext{mūtraviṣṭhe na bhūmīṣu kurute dhunadaṃ sadā |}%

  \maintext{tuṣyate bhagavān śarvaḥ śvānavratacaro yadi }||\thinspace8:15\thinspace||%
\translation{[He does] not [bury] his urine and f\ae ces in the ground, and he barks constantly. Lord Śarva [i.e. Śiva] is satisfied when one practises the dog observance. \blankfootnote{8.15 \textit{dhunadaṃ} {\rm (}`barking'?{\rm )} in \textit{pāda} b may not be the intended form;
  perhaps understand \textit{dhunanaṃ} {\rm (}related to \textit{dhvanana}{\rm )}, 
  or emend to \textit{dhvananaṃ}.
 A possible expanation for Śiva being satisfied with an ascetic practising 
  this observance is that Śiva's Bhairava form often has a dog as his mount. See, e.g.,
  \mycitep{BakkerWorld2014}{232--233} on a 5-6th-century 
  image of Bhairava and a dog carved in rock at Muṇḍeśvarī Hill not far from Vārāṇasī, 
  and Mirnig 2013, 334 ?\verify 
  This observance has ancient roots. Its practitioner, the \textit{kukkuravatika}
  appears in \textit{Majjhimanikāya} 2.1.7, in the \textit{Kukkuravatiyasutta}, 
  alongside with a practitioner of the \textit{govrata} {\rm (}\textit{govatika}{\rm )}, an observance
  that comes up in the next verse in the \VSS:
  \textit{evaṃ me sutaṃ. ekaṃ samayaṃ bhagavā koliyesu viharati haliddavasanaṃ nāma koliyānaṃ nigamo.
  atha kho puṇṇo ca koliyaputto govatiko, acelo ca seniyo kukkuravatiko yena 
  bhagavā tenupasaṅkamiṃsu...}
  See \mycitep{AcharyaBull}{127--128}. Acharya 
  summarises the \textit{Kukkuravatiyasutta} thus:
  `The \textit{Kukkuravatiyasutta} from the \textit{Majjhimanikāya} {\rm (}II.1.7{\rm )} 
  presents a \textit{govatika} together with a \textit{kukkuravatika}. They are observing 
  their vows, and have adopted the behaviour of a bull and a dog respectively. 
  The Buddha tells them that as they are cultivating bullness and dogness, 
  the state of mind of these animals, they will go to hell or become reborn as animal.
  They are alarmed at this and take refuge in the Buddha.'
 }}

  \subsubchptr{govratam}%

  \trsubsubchptr{The Cow Observance}%

  \maintext{mūtravarco na rudhyeta sadā govratiko naraḥ |}%

  \maintext{bhīmas tuṣṭikaraś caiva purāṇeṣu nigadyate }||\thinspace8:16\thinspace||%
\translation{A person practising the Cow Observance should never hold back his urine and f\ae ces. This is a terrifying [observance] that gives satisfaction, [as] stated in the Purāṇas. \blankfootnote{8.16 I prefer reading \textit{bhīma} and \textit{tuṣṭi°} as two separate words, the first
  one either in stem form {\rm (}\msCa\msCb\msNa\msNc\msParis{\rm )} or as \textit{bhīmas} {\rm (}\msCc\msNb\Ed{\rm )}
  or \textit{bhīmaṃ} {\rm (}\eme{\rm )}, to reading these two words as a compound because
  of the following \textit{caiva}.
  I suspect that both \textit{bhīma} and \textit{tuṣṭikara} refer to the \textit{vrata}, rather than its practitioner,
  but I have not emended \textit{bhīmas tuṣṭikaraś} to \textit{bhīmaṃ tuṣṭikaraṃ}
  because \textit{vrata} appears as a masculine noun, e.g., in 8.17d below.
 
  \mycite{AcharyaBull} gives a number of significant clues about the origins 
  of this observance. After exploring its links to Pāśupatas, 
  \mycitep{AcharyaBull}{116--118},
  quotes \textit{Jaiminīyabrāhmaṇa} 2.113, which contains the phrase 
  \textit{yatra yatrainaṃ viṣṭhā vindet tat tad vitiṣṭheta}, in Acharya's translation:
  `Wherever he feels the urge to evacuate f\ae ces, right there he should evacuate.'
  This is an instruction in a Vedic text that is close to what the \VSS\ teaches above.
  Incidentaly, the \textit{Jaiminīyabrāhmaṇa} adds:
  \textit{tena haitenottaravayasy e} [\textit{va}] \textit{yajeta}
  {\rm (}translated in \mycitep{AcharyaBull}{118} as: 
  `One should perform this [sacrifice] in the final years of one's life'{\rm )}.
 }}

  \subsubchptr{mahīvratam}%

  \trsubsubchptr{The Earth Observance}%

  \maintext{kuddālair dārayanto 'pi kīlakoṭiśataiś citaḥ |}%

  \maintext{kṣamate pṛthivī devī evam eva mahīvrataḥ }||\thinspace8:17\thinspace||%
\translation{Splitting [the earth] with spades and laid out on hundreds of pointed wedges: Goddess Earth bears [this] patiently. This is exactly how one can practise the earth vow. \blankfootnote{8.17 While \textit{dārayanto} as an active participle in the masculine nominative is acceptable
  as an irregular form, the precise interpretation of \textit{pāda}s a and b is still problematic,
  therefore my translation of this verse is tentative and the description seems too condensed to be
  intelligible. Kengo Harimoto suggested that \msCc\ and \Ed\ might be transmitting
  the correct reading, and then the reference would be to soil 
  piled up by millions of insects {\rm (}\textit{kīṭakoṭi}°{\rm )}, instead of points of
  wedges {\rm (}\textit{kīlakoṭi}°{\rm )}. Nevertheless, now I think that the reference point could
  be Bhīṣma's dying scene in the \MBH, where the great warrior is lying on a bed of hundreds of arrows: 
  \textit{sa śete śaratalpastho medinīm aspṛśaṃs tadā}: `Then he lay there on his bed of arrows,
  without touching the ground' {\rm (}\MBH\ 6.115.8ab{\rm )}. The word \textit{cita} is used in the same context in 
  \MBH\ 12.47.4ab: \textit{vikīrṇāṃśur ivādityo bhīṣmaḥ śaraśataiś citaḥ}: 
  `Bhīṣma, laid on a hundred arrows, was like the Sun with its scattered rays of light.'
  If this interpretation of \VSS\ 8.17 is correct, the observance described here may require
  one to dig the ground, install wedges, and lie on them, in the manner of fakirs.
  The reference to the Earth in \textit{pāda} c may have been inspired by lines such as \MBH\
  6.115.11cd: \textit{rarāsa pṛthivī caiva bhīṣme śāṃtanave hate}:
  `The Earth cried out when Bhīṣma, the son of Śaṃtanu, was killed.'
 
  In \BHAVP\ 4.121, called `The Description of eighty-five observances' {\rm (}\textit{vratapañcāśītivarṇana}{\rm )},
  we find this on \textit{mahīvrata}:
  \textit{dadyāt triṃśatpalād ūrdhvaṃ mahīṃ kṛtvā tu kāṃcanīm\thinspace |
  kulācalādrisahitāṃ tilavastrasamanvitām\thinspace || 152\thinspace || 
  tiladroṇopari gatāṃ brāhmaṇāya kuṭuṃbine\thinspace | 
  dinaṃ payovratas tiṣṭhed rudraloke mahīyate\thinspace || 153\thinspace || 
  etan mahīvrataṃ proktaṃ saptakalpānuvartakam\thinspace |}.
 
  A tentative translation of this passage would go as follows: `One should donate a golden [model of] Earth
  that weighs more than thirty \textit{pala}s {\rm (}appr. one kilogram{\rm )}, showing the chief mountain-ranges,
  together with [donations of] sesamum seeds and clothes, the sesamum seeds [weighing] more than
  a \textit{droṇa} {\rm (}appr. ten kilograms{\rm )}, to a householder Brāhmin. One should keep the milk-observance 
  [i.e. subsisting on nothing but milk] for one day, and one will have fun in Rudraloka.
  This is called the Earth Observance whose range is seven \ae ons.' {\rm (}I take the values for weights
  from \mycitep{OlivelleManu}{997}.{\rm )} \MATSP\ 101.52 gives similar instructions, 
  as are the descriptions of the \textit{dharāvrata} and the \textit{śubhadvādaśī} observance in 
  \mycitep{KaneHistory}{v. 5, 321 and 429}.
  The \VSS's \textit{mahīvrata} seems different, and more in line with 
  the somewhat transgressive and wild, perhaps Pāśupata-oriented, nature of the
  four preceding observances.
 }}

  \maintext{vratapañcakam ity etad yaś careta jitendriyaḥ |}%

  \maintext{sa cottamam idaṃ lokaṃ prāpnoti na ca saṃśayaḥ }||\thinspace8:18\thinspace||%
\translation{He who practises these five religious observances with his senses subdued will, without doubt, reach this superior world [i.e. heaven?]. \blankfootnote{8.18 Note the neuter \textit{idaṃ} picking up the normally masculine \textit{lokaṃ} in \textit{pāda} c,
  and that the same \textit{idaṃ} would make more sense if the interlocutor were a deity, e.g.,
  Śiva, referring to his abode, and not Anarthayajña, the ascetic. 
  Perhaps emend to \textit{paraṃ}, as suggested by Florinda De Simini.
 }}

  \subchptr{niyameṣv upavāsaḥ {\rm {\rm (}8{\rm )}}}%

  \trsubchptr{The eighth Niyama-rule: Eating restrictions}%

  \maintext{śeṣānnam antarānnaṃ ca naktāyācitam eva ca |}%

  \maintext{upavāsaṃ ca pañcaitat kathayiṣyāmi tac chṛṇu }||\thinspace8:19\thinspace||%
\translation{Eating leftovers, [not] eating in-between [breakfast and dinner], eating [only] at night, eating food obtained without solicitation, and fasting: listen, I shall teach you these five. \blankfootnote{8.19 Note how this category of \textit{niyama}-rules was called \textit{upavāsa} {\rm (}`fasting'{\rm )} 
  in 5.3c above but how in fact \textit{upavāsa} is just the fifth 
  subcategory withing this group of eating restrictions.
 }}

  \subsubchptr{śeṣānnam}%

  \trsubsubchptr{Eating leftovers}%

  \maintext{vaiśvadevātithiśeṣaṃ pitṛśeṣaṃ ca yad bhavet |}%

  \maintext{bhṛtyaputrakalatrebhyaḥ śeṣāśī vighasāśanaḥ }||\thinspace8:20\thinspace||%
\translation{[He who eats] the leftovers belonging to all the gods, to guests, and to the ancestors, he who eats the leftovers {\rm (}\textit{śeṣāśin}{\rm )} of servants, sons and wives, is [called in general] the one who consumes the remains of food {\rm (}\textit{vighasāśana}{\rm )}. \blankfootnote{8.20 \textit{Pāda} a is a \textit{sa-vipulā}.
 }}

  \subsubchptr{antarānnam}%

  \trsubsubchptr{{\rm [}Not{\rm ]} eating in-between breakfast and dinner}%

  \maintext{antarā prātarāśī ca sāyamāśī tathaiva ca |}%

  \maintext{sadopavāsī bhavati yo na bhuṅkte kadācana }||\thinspace8:21\thinspace||%
\translation{if he never eats between breakfast and dinner, he will be regarded as one who is always fasting. \blankfootnote{8.21 My translation here follows the parallel verse in the \MBH\ and 
  is based on the one in \mycite{GanguliMBh}. 
  The syntax of the version here in the \VSS\ is less
  smooth than that in the \MBH, and the \VSS's reading \textit{prāntarāśī} 
  definitely required an emendation.
 }}

  \subsubchptr{naktānnam}%

  \trsubsubchptr{Eating {\rm [}only{\rm ]} at night}%

  \maintext{na divā bhojanaṃ kāryaṃ rātrau naiva ca bhojayet |}%

  \maintext{naktavele ca bhoktavyaṃ naktadharmaṃ samīhatā }||\thinspace8:22\thinspace||%
\translation{One should eat neither in the daytime nor in the evening, and should eat [only] at nighttime {\rm (}\textit{naktavelā}{\rm )} if he wishes to follow the practice of [eating only at] night {\rm (}\textit{naktadharma}{\rm )}. \blankfootnote{8.22 Note \textit{°vele} for \textit{°velāyāṃ} in \textit{pāda} c. On \textit{naktabhojana}, see \SDHS\ 10.
 }}

  \subsubchptr{ayācitānnam}%

  \trsubsubchptr{Eating food obtained without solicitation}%

  \maintext{anārabhya ya āhāraṃ kuryān nityam ayācitam |}%

  \maintext{parair dattaṃ tu yo bhuṅkte tam ayācitam ucyate }||\thinspace8:23\thinspace||%
\translation{He who consumes food only without initiating [the donation], without asking for it, and eats [only] that which has been given by others is called [one who eats] unsolicited [food]. \blankfootnote{8.23 \textit{anārambhasya} {\rm (}`of someone who has not yet started/initiated'{\rm )} in \textit{pāda} a seems suspect, hence
  my conjecture {\rm (}\textit{anārabhya ya}{\rm )} that involves mininal intervention and yields better sense.
  I take \textit{ayācitam} in \textit{pāda} b adverbially.
 Note the accusative with the passive in \textit{pāda} d {\rm (}\textit{tam... ucyate}{\rm )}.
 }}

  \subsubchptr{upavāsaḥ}%

  \trsubsubchptr{Fasting}%

  \maintext{bhakṣyaṃ bhojyaṃ ca lehyaṃ ca coṣyaṃ peyaṃ ca pañcamam |}%

  \maintext{na kāṅkṣen nopayuñjīta upavāsaḥ sa ucyate }||\thinspace8:24\thinspace||%
\translation{Chewable and unchewable food, food to be sipped or sucked or drunk, as the fifth [category]: if one does not long for and does not consume [any of the above], that is called fasting {\rm (}\textit{upavāsa}{\rm )}. \blankfootnote{8.24 For a detailed discussion of the categories \textit{bhakṣya, bhojya, lehya} and \textit{coṣya},
  see \mycitep{KafleNisvasaBook}{245, n. 534}. 
  See also \SDHU\ 8.13:
  \textit{bhakṣyaṃ bhojyaṃ ca peyaṃ ca lehyaṃ coṣyaṃ ca picchilam\thinspace |
  iti bhedāḥ ṣaḍannasya madhurādyāś ca ṣaḍguṇāḥ}\thinspace ||.
 }}

  \subchptr{niyameṣu maunavratam {\rm {\rm (}9{\rm )}}}%

  \trsubchptr{The ninth Niyama-rule: Silence}%

  \maintext{mithyāpiśunapāruṣyatīkṣṇavāg apralāpanam |}%

  \maintext{maunapañcakam ity etad dhārayen niyatavrataḥ }||\thinspace8:25\thinspace||%
\translation{One who is disciplined in religious observances should observe silence [i.e. should avoid] with regards these five: deceitful speech, malignant speech, insult, abusive speech, and babble. \blankfootnote{8.25 \textit{pāruṣya} seems to be the correct reading in \textit{pāda} a, as opposed
  to \msCc's \textit{saṃbhinnā}, because in the following 
  a short section on the category of \textit{pāruṣya} is coming up {\rm (}in 8.28{\rm )}.
  As far as the readings \textit{spṛṣṭavāg} and \textit{pṛṣṭavāg} are concerned, I suppose 
  \textit{pṛṣṭavāg} is not inconceivable {\rm (}as suggested by Judit Törzsök{\rm )}, 
  for in 8.29 it is, in a way, questions that are given as relevant examples. 
  Another possibility, as suggested by Kengo Harimoto, could be \textit{mṛṣāvāg} {\rm (}`lying'{\rm )},
  although this does not fully fit the corresponding examples.
  All in all, I conjectured \textit{tīkṣṇavāg} here, relying on the same verse, 8.29.
  As it will become clear below, \textit{apralāpa} stands for \textit{asatpralāpa}.
 }}

  \subsubchptr{mithyāvacanam}%

  \trsubsubchptr{Deceitful speech}%

  \maintext{asambhūtam adṛṣṭaṃ ca dharmāc cāpi bahiṣkṛtam |}%

  \maintext{anarthāpriyavākyaṃ yat tan mithyāvacanaṃ smṛtam }||\thinspace8:26\thinspace||%
\translation{Fictitious [speech], [speech about] unknown [things], [speech about things] outside the range of Dharma, meaningless and unfriendly speech: these are called deceitful speech. }

  \subsubchptr{piśunaḥ}%

  \trsubsubchptr{Malignancy}%

  \maintext{paraśrīṃ nābhinandanti parasyaiśvaryam eva ca |}%

  \maintext{aniṣṭadarśanākāṅkṣī piśunaḥ samudāhṛtaḥ }||\thinspace8:27\thinspace||%
\translation{One who does not rejoice in others' fortune or in others' power, one who would like to see something disadvantageous [for others] is called somebody utters malignant speech. }

  \subsubchptr{pāruṣyam}%

  \trsubsubchptr{Insult}%

  \maintext{mṛtamātā pitā caiva hāni sthānaṃ kathaṃ bhavet |}%

  \maintext{bhuṅkṣva kāmam amṛṣṭānāṃ pāruṣyaṃ samudāhṛtam }||\thinspace8:28\thinspace||%
\translation{`[May your] mother and father be dead! [May you have] failure {\rm (}\textit{hāni}{\rm )}! Why [do you] even exist? Enjoy the love of unclean women!' [These are] called insult. \blankfootnote{8.28 My translation of \textit{pāda} b, or rather of the whole verse, is tentative.
  I am not at all certain that I understand correctly what these abusive words imply.
 }}

  \subsubchptr{tīkṣṇavāk}%

  \trsubsubchptr{Verbal abuse}%

  \maintext{hṛdi na sphuṭase mūḍha śiro vā na vidāryase |}%

  \maintext{evamādīny anekāni tīkṣṇavādī sa ucyate }||\thinspace8:29\thinspace||%
\translation{`Won't you burst in your heart, stupid? [Why] don't you break your head?' [If one utters] these or similar [curses], he is said to be using verbal abuse. }

  \subsubchptr{asatpralāpaḥ}%

  \trsubsubchptr{Babble}%

  \maintext{dyūtabhojanayuddhaṃ ca madyastrīkatham eva ca |}%

  \maintext{asatpralāpaḥ pañcaitat kīrtitaṃ me dvijottama }||\thinspace8:30\thinspace||%
\translation{Stories about gambling, food, fights, drinking, and women are five [examples of] babble. [Thus] have I taught [reasons for observing silence], O excellent Brahmin. \blankfootnote{8.30 I take \textit{°katham} in \textit{pāda} b as an alternative nominative form of \textit{°kathā} metri causa and as 
  belonging to all the categories here thus: \textit{dyūtakathā, bhojanakathā, yuddhakathā, madyakathā,
  strīkathā}. There are various definitions of \textit{asatpralāpa}, of which the most
  useful for understanding this verse is perhaps
  Siṃhabhūpāla's {\rm (}\textit{Rasārṇavasudhākara} 3.382--383{\rm )}:
  \textit{asambaddhakathālāpo 'satpralāpa itīritiaḥ} {\rm (}`Relating something incoherent is called
  \textit{asatpralāpa}.'{\rm )} This is illustrated with an incoherent and illogical verse from the play
  \textit{Vīrabhadravijṛmbhaṇa}.
 Note the use of the singular next to a number in \textit{pāda} c and
  understand \textit{me} in \textit{pāda} d as \textit{mayā}. The latter usage appears in the epics,
  see \mycitep{OberliesEpicSkt}{102--103 {\rm (}4.1.3{\rm )}}.
 }}

  \maintext{maunam eva sadā kāryaṃ vākyasaubhāgyam icchatā |}%

  \maintext{apāruṣyam asambhinnaṃ vākyaṃ satyam udīrayet }||\thinspace8:31\thinspace||%
\translation{Those who long for eloquent speech should always observe silence. One should speak true words without insult and idle talk. }

  \maintext{yas tu maunasya no kartā dūṣitaḥ sa kulādhamaḥ |}%

  \maintext{janme janme ca durgandho mūkaś caivopajāyate }||\thinspace8:32\thinspace||%
\translation{He who does not observe silence is defiled and is the black sheep of the family. For a number of rebirths, [his mouth] will stink and he will become mute. \blankfootnote{8.32 The form \textit{janme} for \textit{janmani} often occurs in Śaiva tantras as a tipically Aiśa phenomenon.
  See, e.g., \NISVNAYA\ 1.86a {\rm (}\textit{janme janme vimūḍhātmā}, 
  see \mycitep{NisvasaGoodall}{114 and 191}{\rm )}
  and \BRAYA\ 45.8b, 452a, 559a {\rm (}the last one reads \textit{janme janme tu yā jātiṃ}, 
  see \mycitep{KissBraYa}{83 and 128ff}{\rm )}.
  Thematisation of stems in \textit{-an} occurs also in the epics, see 
  \mycitep{OberliesEpicSkt}{88 {\rm (}3.10{\rm )}}.
 }}

  \maintext{tasmān maunavrataṃ sadaiva sudṛḍhaṃ kurvīta yo niścitaṃ}%

 \nonanustubhindent \maintext{vācā tasya alaṅghyatā ca bhavati sarvāṃ sabhāṃ nandati |}%

  \maintext{vaktrāc cotpalagandham asya satataṃ vāyanti gandhotkaṭāḥ}%

 \nonanustubhindent \maintext{śāstrānekasahasraśo giri naraḥ proccāryate nirmalam }||\thinspace8:33\thinspace||%
\translation{Therefore the speech of a person who always observes silence firmly, with resolution, will be impossible to ignore and it will make everybody in the assembly rejoice. The fragrance of lotuses CA [and other kinds of] SATATAM ! rich fragrances will blow from his mouth. Thousands of faultless \textit{śāstra}s will be declared in the words of this person. \blankfootnote{8.33 Note the `muta cum liquida' licence in °\textit{vrataṃ}: the last syllable of \textit{mauna}° counts as light.
 In \textit{pāda} b, understand \textit{nandati} as causative.
 To make sense of \textit{pāda} d, we are forced to take \textit{śāstra} as a stem form noun 
  and \textit{naraḥ} as a {\rm (}regular{\rm )} genitive from \textit{nṛ}. 
  {\rm (}I thank Judit Törzsök for this interpretation.{\rm )}
 }}

  \subchptr{niyameṣu snānam {\rm {\rm (}10{\rm )}}}%

  \trsubchptr{The tenth Niyama-rule: Bathing}%

  \maintext{snānaṃ pañcavidhaṃ caiva pravakṣyāmi yathātatham |}%

  \maintext{āgneyaṃ vāruṇaṃ brāhmyaṃ vāyavyaṃ divyam eva ca }||\thinspace8:34\thinspace||%
\translation{I shall teach you the five kinds of bathing as they really are: fire bath, water bath, Vedic bath, wind bath and divine bath. \blankfootnote{8.34 For a similar set of five types of baths, see, e.g., Parāśarasmṛti 12.9--11:
  \textit{snānāni pañca puṇyāni kīrtitāni manīṣibhiḥ\thinspace |
  āgneyaṃ vāruṇaṃ brāhmaṃ vāyavyaṃ divyam eva ca\thinspace ||
  āgneyaṃ bhasmanā snānam avagāhya tu vāruṇam\thinspace |
  āpo hi ṣṭheti ca brāhmaṃ vāyavyaṃ gorajaḥ smṛtam\thinspace ||
  yat tu sātapavarṣeṇa tat snānaṃ divyam ucyate\thinspace |
  tatra snātvā tu gaṅgāyāṃ snāto bhavati mānavaḥ\thinspace ||}.
  Similar passages are to be found, e.g., at \PADMAP\ 1.47.4ff, 
  Revakhanda 177.6ff, and in a citation attributed to Bhṛgu in Maskari's commentary ad Gautamadharmasūtra 2.14.
 }}

  \subsubchptr{āgneyaṃ snānam}%

  \trsubsubchptr{Fire bath}%

  \maintext{āgneyaṃ bhasmanā snānaṃ toyāc chataguṇaṃ phalam |}%

  \maintext{bhasmapūtaṃ pavitraṃ ca bhasma pāpapraṇāśanam }||\thinspace8:35\thinspace||%
\translation{Fire bath is [performed] with ashes. Its fruits are a hundred times bigger than [those of a] water [bath]. For [anything] cleaned with ashes is pure. Ashes destroy sin. }

  \maintext{tasmād bhasma prayuñjīta dehināṃ tu malāpaham |}%

  \maintext{sarvaśāntikaraṃ bhasma bhasma rakṣakam uttamam }||\thinspace8:36\thinspace||%
\translation{Therefore one should use ash for it purifies humans of their defilement. Ashes yield appeasement for everyone. Ash is the ultimate protector. }

  \maintext{bhasmanā tryāyuṣaṃ kṛtvā brahmacaryavrate sthitam |}%

  \maintext{bhasmanā ṛṣayaḥ sarve pavitrīkṛtam ātmanaḥ }||\thinspace8:37\thinspace||%
\translation{Drawing [the sectarian marks on their foreheads while reciting] the Tryāyuṣa [mantra], observing chastity, all the sages purified themselves with ashes. \blankfootnote{8.37 Note \textit{tryāyuṣa} in the sense of the three \textit{puṇḍra}-lines on the
  forehead and compare with 11.28c. Understand \textit{sthitam} as 
  \textit{sthitaḥ} or rather \textit{sthitāḥ} if we are to connect this line
  to the next {\rm (}8.37cd{\rm )}.
 Understand \textit{pavitrīkṛtam} as \textit{pavitrīkṛtvantaḥ}.
  The reference here may be to a story in which Kaśyapa and 
  other Ṛṣis are burnt to ashes, to be later reanimated by Vīrabhadra, 
  in the Śokara forest. See \PADMAP\ 5.107.1--14ff:
 
  \textit{śucismitovāca~|}
  \textit{kaśyapaṃ jamadagniṃ ca devānāṃ ca purā katham~|}
  \textit{rarakṣa bhasma tad brahman samācakṣva mune mama}~||~1~||
  \textit{dadhīca uvāca~|}
  \textit{kaśyapādiyutā devāḥ pūrvam abhyāgaman girim}~|
  \textit{śokaraṃ nāma vikhyātaṃ girimadhye suśobhanam~}||~2~||
  \textit{nānāvihaṃgasaṃkīrṇaṃ nānāmunigaṇāśrayam}~| 
  \textit{vāsudevāśrayaṃ ramyam apsarogaṇasevitam}~||~3~||
  \textit{vicitravṛkṣasaṃvītaṃ sarvartukusumojjvalam}~|
  \textit{tathāvidhaṃ praviśyaite giriṃ vayam athāpare}~||~4~|| 
  \textit{stuvantaḥ keśavaṃ tatra gatāḥ sma giriśeśvaram}~|
  \textit{dṛṣṭvā tatra mahājvālāṃ praviṣṭāś ca vayaṃ ca tām~}||~5~|| 
  \textit{māmekaṃ tu tiraskṛtya hy adahad devatā munīn~}|
  \textit{māṃ dadāha tataḥ paścād bhasmībhūtā vayaṃ śubhe~}||~6~|| 
  \textit{asmān etādṛśān dṛṣṭvā vīrabhadraḥ pratāpavān~}| 
  \textit{kenāpi kāraṇenāsau gatavān parvataṃ ca tam~}||~7~|| 
  \textit{bhasmoddhūlitasarvāṅgo mastakasthaśivaḥ śuciḥ~}|
  \textit{ekākī niḥspṛhaḥ śānto hāhāśabdam athāśṛṇot~}||~8~|| 
  \textit{atha cintāparaś cāsīn mriyamāṇa śavadhvaniḥ~}| 
  \textit{śavānām iva gandhaś ca dṛśyate tannirīkṣaṇe~}||~9~|| 
  \textit{iti niścitya manasā jagāmāgnim atiprabham~}|
  \textit{sa vahnir vīrabhadraṃ ca dagdhum ārabdhavān atha~}||~10~|| 
  \textit{tṛṇāgnir iva śānto 'bhūd āsādya salilaṃ yathā~}| 
  \textit{tato 'parāṃ mahājvālāṃ vīrabhadras tu dṛṣṭavān~}||~11~||
  \textit{khaṃ gacchantīṃ mahākālo jvālāṃ nipatitām api~}|
  \textit{manasā cintayac cāpi vīrabhadraḥ pratāpavān}~||~12~|| 
  \textit{sarveṣāṃ nāśinī jvālā prāṇināṃ śatakoṭiśaḥ}~| 
  \textit{tat sarvaṃ rakṣaṇārthaṃ hi pipāsuś cāpy ahaṃ tv imām}~||~13~|| 
  \textit{prāśnāmi mahatīṃ jvālāṃ salilaṃ tṛṣito yathā}~| 
  \textit{etasminn antare vīraṃ vāg āha cāśarīriṇī}~||~14~||.
 }}

  \maintext{bhasmanā vibudhā muktā vīrabhadrabhayārditāḥ |}%

  \maintext{bhasmānuśaṃsaṃ dṛṣṭvaiva brahmanānumatiḥ kṛtā }||\thinspace8:38\thinspace||%
\translation{The gods, afflicted by their fear of Vīrabhadra, were set free with the help of ashes. Seeing the glory of ashes, Brahmā consented [to the use of this otherwise impure substance]. \blankfootnote{8.38 The verse may refer to the destruction of Dakṣa's sacrifice, after which the gods were 
  relieved. See old \SKANDAP\ 180.1--4ab {\rm (}in which our \textit{pāda} b is echoed{\rm )}:
  \textit{sanatkumāra uvāca\thinspace |}
  \textit{brahmādyā devatā vyāsa dakṣayajñavadhe purā\thinspace |}
  \textit{śaṅkaraṃ śaraṇaṃ jagmur vīrabhadrabhayārditāḥ}~||~1~||
  \textit{gaṇendreṇābhiyuktās tu bhasmakūṭāni bhejire}~|
  \textit{yadā bhasma praviṣṭās te tejaḥ śāṅkaram uttamam}~||~2~||
  \textit{abhavan te tadā raudrāḥ paśavo dīkṣitā iva}~| 
  \textit{bhasmābhasitagātrāṇāṃ śaṅkaravratacāriṇām}~||~3~||
  \textit{svaṃ yogaṃ pradadau teṣāṃ tadā deva umāpatiḥ~|.}
 }}

  \maintext{caturāśramato 'dhikyaṃ vrataṃ pāśupataṃ kṛtam |}%

  \maintext{tasmāt pāśupataṃ śreṣṭhaṃ bhasmadhāraṇahetutaḥ }||\thinspace8:39\thinspace||%
\translation{[Thus] the Pāśupata observance was created, which ranks above [the system of] the four \textit{āśrama}s. Therefore the Pāśupata [observance] is the best because it involves carrying ashes [on one's body]. \blankfootnote{8.39 One could simply accept the reading of \msCc\ {\rm (}\textit{°hetunā}{\rm )} in \textit{pāda} d, but all other rejected 
  readings hint at an original \textit{hetutaḥ} {\rm (}as remarked by Judit Törzsök{\rm )}.
 }}

  \subsubchptr{vāruṇaṃ snānam}%

  \trsubsubchptr{Water bath}%

  \maintext{vāruṇaṃ salilaṃ snānaṃ kartavyaṃ vividhaṃ naraiḥ |}%

  \maintext{nadītoyataḍāgeṣu prasraveṣu hradeṣu ca }||\thinspace8:40\thinspace||%
\translation{A water bath {\rm (}\textit{vāruṇa}{\rm )} is to be performed with water in different ways by [different] people: in the water of rivers, water tanks, streams and ponds. \blankfootnote{8.40 The reading \textit{vividhaṃ} in \textit{pāda} b seems to be the lectio difficilior as opposed to
  the rejected \textit{vidhivat}.
 }}

  \subsubchptr{brāhmyaṃ snānam}%

  \trsubsubchptr{Vedic bath}%

  \maintext{brahmasnānaṃ ca viprendra āpohiṣṭhaṃ vidur budhāḥ |}%

  \maintext{trisaṃdhyam eva kartavyaṃ brahmasnānaṃ tad ucyate }||\thinspace8:41\thinspace||%
\translation{The wise know the Vedic bath as [the one performed with the Vedic mantra beginning with] \textit{āpo hi ṣṭhā}, O excellent Brahmin. It is to be performed at the three junctures of the day [dawn, noon, and evening]. It is called the Vedic bath. \blankfootnote{8.41 The Ṛgvedic mantra starting with \textit{āpo hi ṣṭhā} {\rm (}\RV\ 10.9.1--3{\rm )} is traditionally associated with 
  \textit{mārjana} {\rm (}`cleaning, wiping'{\rm )}. According to \mycitep{KaneHistory}{v. 4, 120},
  a Brahmin `should bathe thrice in the day, should perform \textit{mārjana} {\rm (}splashing
  or sprinkling water on the head and other limbs by means of \textit{kuśas} 
  dipped in water after repeating sacred mantras{\rm )} with the three verses 
  `apo hi sthā' [sic] {\rm (}Ṛg. X.9.1--3{\rm )} [...]'
  This suggests a method of bathing that is more of a ritual than an actual bath.
 }}

  \subsubchptr{vāyavyaṃ snānam}%

  \trsubsubchptr{Wind bath}%

  \maintext{goṣu saṃcāramārgeṣu yatra godhūlisambhavaḥ |}%

  \maintext{tatra gatvāvasīdeta snānam uktaṃ manīṣibhiḥ }||\thinspace8:42\thinspace||%
\translation{He should go where dust rises among the cows on the paths as they pass by, and he should sit down there. This is [also] called a bath, [namely the \textit{vāyavya} or wind-bath]. \blankfootnote{8.42 This version of bathing seems to be a way of taking a shower 
  in the holy dust raising from under the hooves of cows.
 }}

  \subsubchptr{divyaṃ snānam}%

  \trsubsubchptr{Heavenly bath}%

  \maintext{varṣatoyāmbudhārābhiḥ plāvayitvā svakāṃ tanum |}%

  \maintext{snānaṃ divyaṃ vadaty eva jagadādimaheśvaraḥ }||\thinspace8:43\thinspace||%
\translation{One should immerse one's own body in the water-showers of the rain. The first and foremost Lord {\rm (}\textit{maheśvara}{\rm )} of the universe declares it as the heavenly bath. }

  \maintext{iti niyamavibhāgaḥ pañcabhedena vipra}%

 \nonanustubhindent \maintext{nigadita tava pṛṣṭaḥ sarvalokānukampya |}%

  \maintext{sakalamalapahārī dharmapañcāśad etan}%

 \nonanustubhindent \maintext{na bhavati punajanma kalpakoṭyāyute 'pi }||\thinspace8:44\thinspace||%
\translation{Thus have I taught you the section on the Niyama-rules in divisions of five [sub-categories to each], O Brahmin because you asked me to, to favour the whole world. These fifty Dharmic [teachings] wipe off all the defilement. There will not be rebirth [for one who follows these rules], not even in millions of \ae ons. \blankfootnote{8.44 This verse marks not only the end of a long section on the Niyama rules,
  but also the end of a major part of the text that discusses the ten Yama and ten Niyama rules,
  spanning 3.16--8.44.
 There are two stem form nouns in \textit{pāda} b: I suspect that \Ed\ is right
  assuming that in order to restore the metre, we must have \textit{nigadita}, as opposed to
  \textit{nigaditas}, the reading trasmitted in all the witnesses;
  also understand \textit{sarvalokānukampya} in \textit{pāda} b as \textit{sarvalokān anukampya}.
  Also, understand \textit{nigadita tava pṛṣṭaḥ} as \textit{nigadito mayā tvayā pṛṣṭena}.
 
 Understand \textit{sakalamalapahārī} in \textit{pāda} c as \textit{sakala-mala-apahārī}, which would be unmetrical,
  and compare it with \textit{duritamalapahārī} in 4.89c.
  Take \textit{etan/etad} as either picking up °\textit{pahārī} or rather
  a plural corresponding to °\textit{pañcāśad}. The latter phenomenon, namely the use of
  the singular after numbers, is one of the hallmarks of the text.
 
  By `fifty Dharmas,' the text refers to the ten main Niyama-rules, each
  having five subcategories {\rm (}10 × 5 = 50{\rm )}.
  
 
 The licence of an word-ultimate short syllable treated as long {\rm (}°\textit{janma} in \textit{pāda} d{\rm )} is
  also freqently seen in this text. Note also \textit{puna} for \textit{punar} metri causa.
 }}

\centerline{\maintext{\dbldanda\thinspace iti vṛṣasārasaṃgrahe niyamapraśaṃsā nāmādhyāyo 'ṣṭamaḥ\thinspace\dbldanda}}
\translation{Here ends the eighth chapter in the \textit{Vṛṣasārasaṃgraha} called the Praise of the Niyama-rules.}

  \chptr{saptadaśamo 'dhyāyaḥ}
\addcontentsline{toc}{section}{Chapter 17}
\fancyhead[CO]{{\footnotesize\textit{Translation of chapter 17}}}%

  \trchptr{Chapter Seventeen}%

  \maintext{satyabhāmā svakaṃ bhartrā dattvā nāradasatkṛtam |}%

  \maintext{dānasyāsya prabhāvena akṣayaṃ tridivaṃ gataḥ }||\thinspace17:50\thinspace||%
\translation{Satyabhāmā gave her own wealth {\rm (}\textit{svaka}{\rm )} [equal in weight to the wishing-tree together] with [her] husband [Kṛṣṇa] as a way to honour Nārada. By the force of this donation, he [i.e. Nārada] went to the third heaven. \blankfootnote{17.50 The interpretation of this verse is tentative. It seems to 
  refer to the episode when Kṛṣṇa was given flowers of the heavenly wishing tree by Nārada.
  Kṛṣṇa failed to pass any of them to his favourite wife Satyabhāmā {\rm (}note emendation in 
  \textit{pāda} a: an original -\textit{ā} may have been misread as a visarga{\rm )}. 
  Kṛṣṇa's blunder was remedied by a journey to heaven together, and
  by bringing the wishing tree from the world of gods to Satyabhāmā's garden. 
  Nārada told Satyabhāmā that in order for her to have the tree in 
  all of her births, all she has to do is \textit{tulāpuruṣadāna},
  one of the \textit{mahādāna}s. This involves donating as much gold as the weight of the donor. 
  She gave Nārada as much gold as the weight of her husband, Kṛṣṇa, plus
  the tree. After this, Nārada departed to heaven. 
  See \mycite{PuranicEnc} s.v. `Satyabhāmā', and 
  \PadmaP\ 6.88.15--17:
  \textit{satyabhāmovāca\thinspace | 
  īdṛśaḥ kalpavṛkṣo 'yaṃ patir etādṛśaḥ prabhuḥ\thinspace | 
  bhave bhave kathaṃ prāpyas tad ākhyātu bhavān mama\thinspace || 
  iti pṛṣṭas tadā prāha nārado munisattamaḥ\thinspace | 
  prāpyate satyabhāme 'yaṃ tulāpuruṣadānataḥ\thinspace || 
  satyabhāmā tadā kṛṣṇaṃ kalpavṛkṣasamanvitam\thinspace | 
  nāradāyaiva sā prādāt tolayitvā vidhānataḥ\thinspace | 
  sarvopaskaram ākṛṣya nāradas tridivaṃ yayau\thinspace ||}
 }}
