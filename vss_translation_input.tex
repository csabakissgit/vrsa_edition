
  \chptr{prathamo 'dhyāyaḥ}
\addcontentsline{toc}{section}{Chapter 1}
\fancyhead[CO]{{\footnotesize\textit{Translation of chapter 1}}}%

  \trchptr{Chapter One}%

  \subchptr{stutiḥ}%

  \trsubchptr{Invocation}%

  \maintext{anādimadhyāntam anantapāraṃ}%

 \nonanustubhindent \maintext{susūkṣmam avyaktajagatsusāram |}%

  \maintext{harīndrabrahmādibhir āsamagraṃ}%

 \nonanustubhindent \maintext{praṇamya vakṣye vṛṣasārasaṃgraham }||\thinspace1:1\thinspace||%
\translation{Having bowed to the One who has no beginning, no middle part and no end, whose boundaries are limitless, who is very subtle and who is the unmanifest and fine essence of the world, and also to Indra, Brahmā and the other [gods], I shall recite [the work called] `A Compendium on the Essence of the Bull [of Dharma]'. \blankfootnote{1.1 This verse echoes \VSS\ 20.3:
 
  \textit{nādimadhyaṃ na cāntaṃ ca yan na vedyaṃ surair api}\thinspace | 
 
  \textit{atisūkṣmo hy atisthūlo nirālambo nirañjanaḥ\thinspace ||} 
 
  This could suggest
  that \textit{pāda} c above might be parallel with \textit{na vedyaṃ surair api}. 
  Perhaps understand \textit{asamagraṃ} [\textit{vedyaṃ}] {\rm (}`incompletely [known]{\rm )}.
 
  \textit{Pāda} a is also reminiscent of, among other famous passages, \BHG\ 11.19:
  
 
  \textit{anādimadhyāntam anantavīryam} 
  \textit{anantabāhuṃ śaśisūryanetram}\thinspace | 
 
  \textit{paśyāmi tvāṃ dīptahutāśavaktraṃ} 
  \textit{svatejasā viśvam idaṃ tapantam}\thinspace ||
  
 
  See also \BHG\ 10.20cd:
  
 
  \textit{aham ādiś ca madhyaṃ ca bhūtānām anta eva ca}\thinspace ||
  
 
  A faint reference to the \BHG\ seems proper at the
  beginning of a work that claims to deliver a teaching 
  based on, but also to surpass, the \MBH\ {\rm (}see following verses of the \VSS{\rm )}.
 
  Compare also, e.g., \KURMP\ 1.11.237:
  
 
  \textit{rūpaṃ tavāśeṣakalāvihīnam}
  \textit{agocaraṃ nirmalam ekarūpam}\thinspace |
  
  \textit{anādimadhyāntam anantam ādyaṃ} 
  \textit{namāmi satyaṃ tamasaḥ parastāt}\thinspace ||
  
 
  In general, to say that a god has no beginning and no end in a temporal or spacial sense is natural
  {\rm (}\textit{anādi \dots\ antam}{\rm )}, but to have no `middle part' {\rm (}\textit{madhya}{\rm )} in these senses is slightly less so.
  Thus the rather commonly occuring phrase \textit{anādimadhyāntam} is probably a fixed expression usually 
  referring to a deity that is endless, eternal and immaterial. 
  As to which deity or what form of a deity this stanza refers to, one could argue that 
  it is Śiva, his name missing in \textit{pāda} c, but the phrasing of the verse 
  is vague enough to keep the question somewhat open: the impersonal Brahman 
  might be another option, even more so if we look at verses 1.9--10, whose
  topic is \textit{brahmavidyā}.
 
 
  
 
  In \textit{pāda} b \textit{jagat-susāraṃ} is most probably not 
  to be interpreted as \textit{jagatsu sāraṃ} {\rm (}`the essence in the worlds'{\rm )}.
  Another way to translate \textit{avyaktajagatsusāraṃ} would be: 
  `who is the fine essence of the unmanifest world.'
 
  
 
  Strictly speaking, \textit{pāda} c is unmetrical, but it is better to 
  simply acknowledge here the phenomenon of `muta cum liquida', namely
  that syllables followed by consonant clusters such as 
  \textit{ra, bra, hra, kra, śra, śya, śva, sva, dva} can be treated as short {\rm (}\textit{laghu}{\rm )}.
  {\rm (}See Introduction pp.~\pageref{muta}\thinspace ff.{\rm )}
  Thus \textit{harīndrabrahmā°} can be treated as a regular beginning
  of an \textit{upajāti} {\rm (}\shortsyllable\ - \shortsyllable\ - -{\rm )}, the syllable 
  \textit{bra} not turning the previous syllable long {\rm (}\textit{guru}{\rm )}.
 
  
 
  The reading \textit{āsamagraṃ} in \textit{pāda} c is suspect 
  {\rm (}see a preliminary comment on this above{\rm )},
  although the initial \textit{ā-} might convey some sort of
  completeness, meaning `all round'
  {\rm (}see e.g. \mycitep{KaleHigherGrammar}{226}{\rm )}.
  The fact that we could percieve the ending of \textit{pāda}s a and b 
  {\rm (}\textit{pāraṃ}--\textit{sāram}{\rm )}, as well as \textit{pāda}s c and d, as {\rm (}in the 
  latter case, oddly{\rm )} rhyming pairs {\rm (}\textit{graṃ}-\textit{graham}{\rm )}
  suggests that accepting the reading \textit{āsamagram} could be 
  the right decision {\rm (}as suggested by Alessandro Battistini{\rm )}.
  I translate this verse accordingly. \msM\ gives an exciting,
  albeit unmetrical, alternative {\rm (}\textit{yat samagraṃ}{\rm )}, but
  this seems more of a guess than the correct reading.
  For some time I was considering emending \textit{āsamagraṃ}.
  The most tempting of all the possible options 
  {\rm (}\textit{arcyam/arhyam/arghyam/īḍyam/āḍhyam/āptam agraṃ, āsamastaṃ}{\rm )} 
  seemed to be \textit{āptam agraṃ},
  meaning `appointed/received/respected [by Hari, Indra,
  Brahmā etc.] as the foremost one'. The fact that 
  the \textit{akṣara}s \textit{āsam} and \textit{āptam} look similar in most
  of the scripts used in the witnesses could support this
  conjecture. \textit{āptam} could also
  possibly refer to the text itself, although then the
  syntax becomes slightly confusing: `I shall recite the
  \textit{Vṛṣasārasaṃgraha} that was first received by Hari...' etc.
  Another candidate was \textit{āḍhyam agram}:
  `Having bowed to [Him] who contains/is rich with Hari, Indra, Brahmā
  etc.' I have not emended the text because it is difficult
  to know if there is any need for change and if there is, which reading 
  to chose. There was no consensus when this verse was discussed 
  in our extended Śivadharma reading group.
 
  
 
  Pāda d seems hypermetrical, but it can be interpreted as a \textit{vaṃśastha}
  line, a change from \textit{triṣṭubh} to \textit{jagatī} {\rm (}as suggested by Dominic Goodall{\rm )}.
 }}

  \subchptr{janamejayavaiśampāyanasaṃvādaḥ}%

  \trsubchptr{Dialogue of Janamejaya and Vaiśampāyana}%

  \maintext{śatasāhasrikaṃ granthaṃ sahasrādhyāyam uttamam |}%

  \maintext{parva cāsya śataṃ pūrṇaṃ śrutvā bhāratasaṃhitām }||\thinspace1:2\thinspace||%
\translation{Having listened to the \textit{Bhāratasaṃhitā} [i.e. the \textit{Mahābhārata}], the supreme book of a hundred thousand [verses] and a thousand chapters {\rm (}\textit{adhyāya}{\rm )}, with all its hundred sections {\rm (}\textit{parvan}{\rm )}, \blankfootnote{1.2 The dialogue of Janamejaya and Vaiśampāyana makes up the outermost layer of the \VSS\ 
  {\rm (}see Introduction p.~\pageref{structure}{\rm )}, mostly containing
  general \textit{dharmaśāstric} material.
  
 
  That the \MBH\ should contain a hundred thousand verses is hinted at, e.g., in line 19 of
  the Khoh Charter 2 of Śarvanātha, year 214 {\rm (}Siddham Database IN00088; 
  \textit{uktañ ca mahābhārate śatasāhasryaṃ} [understand °\textit{ryāṃ}] \textit{saṃhitāyāṃ}...{\rm )}.
  The hundred \textit{parvan}s of the \textit{Mahābhārata} are listed in \MBH\ 1.2.33--70.
  Note the use of the singular {\rm (}\textit{parva}{\rm )} in connection with numerals {\rm (}\textit{śataṃ}{\rm )},
  one of the hallmarks of this text {\rm (}see p.~\pageref{singularwithnumerals}{\rm )}.
 }}

  \maintext{atṛptaḥ puna papraccha vaiśampāyanam eva hi |}%

  \maintext{janamejayena yat pūrvaṃ tac chṛṇu tvam atandritam }||\thinspace1:3\thinspace||%
\translation{Janamejaya remained unsatisfied. Listen attentively to what he asked Vaiśampāyana in the past. \blankfootnote{1.3 My emendation from the unmetrical \textit{punaḥ} to the unusual, or rather, Middle Indic
  {\rm (}\mycitep{EdgertonHybrid}{vol. 2, p. 347}{\rm )},
  and Newar {\rm (}\mycitep{JorgensenGrammar}{113}{\rm )}, \textit{puna} is based
  on the assumption that in the original the metre must have overridden 
  morphology, similarily to what may have happened in 8.44d {\rm (}Mālinī metre{\rm )}:
  \textit{na bhavati punajanma kalpakoṭyāyute 'pi}, and in 12.151c {\rm (}Sragdharā metre{\rm )}:
  \textit{garbhāvāsaṃ na ca tvan na ca punamaraṇaṃ kleśam āyāsapūrṇam}.
 
  
 
  For an unsatisfaction or dissatisfaction {\rm (}\textit{atṛpti}{\rm )} with previous 
  teachings in a somewhat similar manner to what
  Janamejaya experiences here, see, e.g., \textit{Niśvāsa} mūla 1.9:
  
 
  \textit{vedāntaṃ viditaṃ deva sāṃkhyaṃ vai pañcaviṃśakam}\thinspace |
 
  \textit{na ca tṛptiṃ gamiṣyāmo hy ṛte śaivād anugrahāt}\thinspace ||
  
 
  
 Vaiśampāyana, a Ṛṣi, disciple of Vyāsa, great-grandson to Arjuna,
  recited the \MBh\ at the snake sacrifice of 
  Janamejaya. This setting is an echo of the starting point of the \MBH, see \MBH\ 1.1.8ff.
  In fact the next few verses in the \VSS\ make it clear that the \VSS\
  picks up where the \MBH\ left off: Janamejaya has heard the whole \MBh\ from
  Vaiśampāyana, but he is eager to hear more, or rather a concised version
  of the Dharmic teachings of the \MBh.
  
 
  It is tempting to emend \textit{pāda} c to contain a stem form proper noun {\rm (}\textit{janamejaya}{\rm )}
  in order to maintain the metre, 
  and note how the manuscripts struggle with this \textit{pāda}. 
  Stem form nouns, \textit{prātipadika}s, abound in the \VSS: see Introduction p.~\pageref{stemform}.
  On the other hand, the contracted/syncopated form \textit{janmejaya} occurs, 
  e.g., in \BHAGP\ 12.06.16 and \BRAHMAVP\ 4.14.41 and 46. {\rm (}It is even
  lexicalised in \Monier{\rm )} 
  The hypermetrical form \textit{janamejayena}, and the construction finite verb + instrumental
  {\rm (}\textit{papraccha... janamejayena}{\rm )}, could be original; compare 1.8 and 4.75 below.
  Alternatively, 1.3cd could be taken as a separate, and elliptical,
  sentence standing for \textit{janamejayena yac chrutaṃ pūrvaṃ tac chṛṇu}.
 }}

  \maintext{janamejaya uvāca |}%

  \maintext{bhagavan sarvadharmajña sarvaśāstraviśārada |}%

  \maintext{asti dharmaṃ paraṃ guhyaṃ saṃsārārṇavatāraṇam }||\thinspace1:4\thinspace||%
\translation{Janamejaya spoke: O venerable sir, O knower of the entire Dharma, O you who are well-versed in all the sciences {\rm (}\textit{śāstra}{\rm )}! There is a supreme and secret Dharma [that brings about] liberation from the ocean of mundane existence {\rm (}\textit{saṃsāra}{\rm )}, \blankfootnote{1.4 Note \textit{dharma} as a neuter noun in \textit{pāda} c and in the next verse.
 }}

  \maintext{dvaipāyanamukhodgīrṇaṃ dharmaṃ vā yad dvijottama |}%

  \maintext{kathayasva hi me tṛptiṃ kuru yatnāt tapodhana }||\thinspace1:5\thinspace||%
\translation{that is, the Dharma that emerged from [Vyāsa] Dvaipāyana's mouth, O best of Brahmins. Teach [it] to me and help me find satisfaction at all cost, O great ascetic! \blankfootnote{1.5 The majority of the MSS consulted include a \textit{vā} in \textit{pāda} b, 
  and although \msCb's reading seems a bit smoother, that manuscript rarely gives superior readings.
  Therefore I have chosen \textit{dharmaṃ vā yad}, in which \textit{vā} functions probably in a weak sense
  {\rm (}`that is'{\rm )}.
  That the secret Dharma Janamejaya is seeking is the one taught by Vyāsa Dvaipāyana, and
  thus no real options are involved here, becomes clear in 1.6cd.
  The reading of \msM\ in \textit{pāda} b {\rm (}\textit{dharmavākyaṃ}{\rm )} is tempting but could be a later correction.
 In general, \msM's readings here are unique but probably secondary: 
  \textit{hi me tṛptiṃ} in \textit{pāda} c seems more attractive than \msM's 
  \textit{prasādena} because it echoes \textit{atṛptaḥ} in 1.3a
 }}

  \maintext{vaiśampāyana uvāca |}%

  \maintext{śṛṇu rājann avahito dharmākhyānam anuttamam |}%

  \maintext{vyāsānugrahasamprāptaṃ guhyadharmaṃ śṛṇotu me }||\thinspace1:6\thinspace||%
\translation{Vaiśampāyana spoke: Listen with great attention, O king, to this unsurpassed narration of Dharma. Hear the secret Dharma that I received through the grace of Vyāsa. }

  \maintext{anarthayajñakartāraṃ tapovrataparāyaṇam |}%

  \maintext{śīlaśaucasamācāraṃ sarvabhūtadayāparam }||\thinspace1:7\thinspace||%
{\blankfootnote{1.7 On Anarthayajña, the interlocutor of VSS 1.9--10.2 and 19.1--21.22, and
  an important figure discussed in 22.3ff, as well as a concept {\rm (}`nonmaterial sacrifice'{\rm )},
  see \mycite{KissVolume2021} and Introduction p.~\pageref{anarthayajna_person}.
 }}

  \maintext{jijñāsanārthaṃ praśnaikaṃ viṣṇunā prabhaviṣṇunā |}%

  \maintext{dvijarūpadharo bhūtvā papraccha vinayānvitaḥ }||\thinspace1:8\thinspace||%
\translation{Viṣṇu, the great Lord, assuming the form of a twice-born [Brahmin], wanted to test [Anarthayajña, the ascetic yogin] who practised nonmaterial sacrifices {\rm (}\textit{anarthayajña}{\rm )}, focused on his austerities and observances, whose conduct was virtuous and pure, and who was intent on compassion towards all living beings; therefore he [Viṣṇu] humbly asked him a question. \blankfootnote{1.8 Note the syntax here involving the agent in the instrumental
  with a finite verb {\rm (}ergative structure{\rm )}: \textit{viṣṇunā... dvijarūpadharo bhūtvā papraccha}.
  Compare 1.3.
 }}

  \subchptr{brahmavidyā}%

  \trsubchptr{Knowledge of Brahman}%

  \maintext{{\rm [}vigatarāga uvāca | {\rm ]}}%

  \maintext{brahmavidyā kathaṃ jñeyā rūpavarṇavivarjitā |}%

  \maintext{svaravyañjananirmuktam akṣaraṃ kimu tat param }||\thinspace1:9\thinspace||%
\translation{[Vigatarāga spoke:] How is the knowledge of the Brahman to be understood if it is devoid of form and colour? Why is that supreme syllable which is devoid of vowels and consonants the supreme one? \blankfootnote{1.9 The translation of this verse, and the reconstruction and interpretation
  of \textit{pāda} d, which is echoed in 1.10d, is slightly tentative.
  I doubt if \textit{kimu} could have the standard {\rm (}Vedic{\rm )} meaning `how much more/less'
  here. Rather \textit{u} is probably just an expletive. In general it seems that
  this verse references the syllable \textit{oṃ}.
 }}

  \maintext{anarthayajña uvāca |}%

  \maintext{anuccāryam asandigdham avicchinnam anākulam |}%

  \maintext{nirmalaṃ sarvagaṃ sūkṣmam akṣaraṃ kim ataḥ param }||\thinspace1:10\thinspace||%
\translation{Anarthayajña replied: That syllable is not to be pronounced, is unquestionable, non-dividable, consistent, spotless, all-pervading and subtle: what could be higher than that? \blankfootnote{1.10 In \textit{pāda} d, I have choosen, somewhat randomly, \textit{kim ataḥ} instead of \textit{kimu tat},
  trying to make sense of 10.9--10.
 }}

  \subchptr{kālapāśaḥ}%

  \trsubchptr{Noose of death and time}%

  \maintext{vigatarāga uvāca |}%

  \maintext{dehī dehe kṣayaṃ yāte bhūjalāgniśivādibhiḥ |}%

  \maintext{yamadūtaiḥ kathaṃ nīto nirālambo nirañjanaḥ }||\thinspace1:11\thinspace||%
\translation{Vigatarāga spoke: When the body disintegrates in the ground, in water, in fire, or [is torn apart] by jackals and other [animals], how is the supportless and spotless soul led [to the netherworld] by Yama's messengers? \blankfootnote{1.11 The word \textit{°śivā°} in \textit{pāda} b is slightly suspect, and could be the result
  of metathesis, from \textit{°viṣā°} {\rm (}`by poison'{\rm )}. Nevertheless, 
  jackals seems appropriate in this context, for they 
  are commonly associated with human corpses, death and the cremation ground
  {\rm (}see e.g. \mycite{Ohnuma2019}{\rm )}. Furthermore, \textit{pāda} b lists phenomena
  that cause the body to disintegrate, and not causes of death; thus the reading \textit{śiva}
  is probably correct.
 }}

  \maintext{kālapāśaiḥ kathaṃ baddho nirdehaś ca kathaṃ vrajet |}%

  \maintext{svargaṃ vā sa kathaṃ yāti nirdeho bahudharmakṛt |}%

  \maintext{etan me saṃśayaṃ brūhi jñātum icchāmi tattvataḥ }||\thinspace1:12\thinspace||%
\translation{How is it bound by the nooses of death [/ time] {\rm (}\textit{kālapāśa}{\rm )}? And if it is bodiless, how can it move? And how does the [soul of a] virtuous [person] {\rm (}\textit{bahudharmakṛt}{\rm )} reach heaven if it has no body? This is my doubt. Teach me. I want to know the truth. \blankfootnote{1.12 The word \textit{kāla} has, as usual, a double meaning here: \textit{kālapāśa}
  is both Yama's noose, and also the limitations and bondage caused by time, 
  as becomes clear at the discussion on the different time units in verses 1.18--30.
 \textit{saṃśaya} seems to be treated as neuter in \textit{pāda} e.
 }}

  \maintext{anarthayajña uvāca |}%

  \maintext{atisaṃśayakaṣṭaṃ te pṛṣṭo 'haṃ dvijasattama |}%

  \maintext{durvijñeyaṃ manuṣyais tu devadānavapannagaiḥ }||\thinspace1:13\thinspace||%
\translation{Anarthayajña spoke: You are asking me about an extremely doubtful and problematic matter, O truest of the twice-born. [This is a matter that] is difficult to understand by humans, and [even] by gods {\rm (}\textit{deva}{\rm )}, demons {\rm (}\textit{dānava}{\rm )} and serpents {\rm (}\textit{pannaga}{\rm )}. \blankfootnote{1.13 Note \textit{te} used for \textit{tvayā} in \textit{pāda} a. Alternatively, taking \textit{te} as genitive, the line
  could be translatied as: `I am being asked about a great 
  problem of yours that originates in doubts\dots'
 }}

  \maintext{karmahetu śarīrasya utpatti nidhanaṃ ca yat |}%

  \maintext{sukṛtaṃ duṣkṛtaṃ caiva pāśadvayam udāhṛtam }||\thinspace1:14\thinspace||%
\translation{The cause of both the birth and death of the body is karma. Good and bad deeds are called the two nooses. \blankfootnote{1.14 The MSS give \textit{karmahetu} in \textit{pāda} a overwhelmingly, which could work as a neuter
  \textit{bahuvrīhi} compound picking up both a stem-form \textit{utpatti} and \textit{nidhanaṃ}. 
  \textit{karmahetuḥ} {\rm (}\msCb{\rm )} is grammatically more correct, picking up the feminine \textit{utpatti},
  but a neuter stem-form \textit{utpatti} is unsurprising in this text.
 }}

  \maintext{tenaiva saha saṃyāti narakaṃ svargam eva vā |}%

  \maintext{sukhaduḥkhaṃ śarīreṇa bhoktavyaṃ karmasambhavam }||\thinspace1:15\thinspace||%
\translation{[The soul] goes to hell or heaven [bound and led] by the same [nooses of Yama's messengers, or the karmas]. Happiness and suffering, both arising from karma, are to be experienced by the body. }

  \maintext{hetunānena viprendra dehaḥ sambhavate nṛṇām |}%

  \maintext{yaṃ kālapāśam ity āhuḥ śṛṇu vakṣyāmi suvrata }||\thinspace1:16\thinspace||%
\translation{It is for this reason, O great Brahmin, that the human body is born. Now learn about that which they call the noose of time {\rm (}\textit{kālapāśa}{\rm )}, I shall teach you, O you of great observances. }

  \maintext{na tvayā viditaṃ kiñcij jijñāsyasi kathaṃ dvija |}%

  \maintext{kālapāśaṃ ca viprendra sakalaṃ vettum arhasi }||\thinspace1:17\thinspace||%
\translation{[If] you do not know anything, how could you start your investigation, O twice-born? O great Brahmin, you should know the noose of time {\rm (}\textit{kālapāśa}{\rm )} in its entirety. \blankfootnote{1.17 The variant \textit{jijñāsyasi} seems to be the lectio difficilior as opposed to
  \textit{vijñāsyasi}, but the latter could also work fine here.
 Note how \msM\ {\rm (}agreeing with two paper MSS, \msPaperA\ and \msPaperC, as well as \Ed{\rm )} 
  gives a reading {\rm (}\textit{vaktum arhasi}{\rm )} that is clearly out
  of context. This confirms that while \msM\ comes up with interesting readings, 
  they are mostly to be ignored.
 }}

  \maintext{kalākalitakālaṃ ca kālatattvakalāṃ śṛṇu |}%

  \maintext{truṭidvayaṃ nimeṣas tu nimeṣadviguṇā kalā }||\thinspace1:18\thinspace||%
\translation{Learn about time {\rm (}\textit{kāla}{\rm )} which is divided into digits {\rm (}\textit{kalā}{\rm )}, [i.e. about] the division[s] {\rm (}\textit{kalā}{\rm )} of the entity [called] time {\rm (}\textit{kālatattva}{\rm )}. Two atomic units of time {\rm (}\textit{truṭi}{\rm )} are one twinkling {\rm (}\textit{nimeṣa}{\rm )}. One digit {\rm (}\textit{kalā}, cca. 1.6 second{\rm )} is twice a twinkling. \blankfootnote{1.18 1.18d and 1.19a are problematic in the light of 1.19b, which 
  redefines \textit{kalā} in harmony with the traditional
  interpretation, see e.g. \Arthasastra\ 2.20.33: \textit{trimśatkāṣṭhāḥ kalāḥ}.
  \nocite{Arthasastra1969}
  On divisions of time, see also, e.g., \MANU\ 1.64ff.
  I have calculated 1.6 second for one \textit{kalā} backwards, starting from one day {\rm (}see 1.20ab{\rm )}.
 }}

  \maintext{kalādviguṇitā kāṣṭhā kāṣṭhā vai triṃśatiḥ kalā |}%

  \maintext{triṃśatkalā muhūrtaś ca mānuṣena dvijottama }||\thinspace1:19\thinspace||%
\translation{Two digits {\rm (}\textit{kalā}{\rm )} form one bit {\rm (}\textit{kāṣṭhā}, 3.2 seconds{\rm )}. Thirty bits {\rm (}\textit{kāṣṭhā}{\rm )} make one digit {\rm (}\textit{kalā}?, 1.6 minutes{\rm )}. Thirty digits {\rm (}\textit{kalā}{\rm )} make up one section {\rm (}\textit{muhūrta}, 48 minutes{\rm )} in human terms, O great Brahmin. \blankfootnote{1.19 Underestand \textit{mānuṣena} as \textit{mānuṣasaṃkhyayā} {\rm (}1.21d{\rm )}.
 }}

  \maintext{muhūrtatriṃśakenaiva ahorātraṃ vidur budhāḥ |}%

  \maintext{ahorātraṃ punas triṃśan māsam āhur manīṣiṇaḥ }||\thinspace1:20\thinspace||%
\translation{Thirty sections {\rm (}\textit{muhūrta}{\rm )} are known to the wise as one night and day [i.e. a full day]. Thirty days and nights are taught by the wise to be one month. }

  \maintext{samā dvādaśa māsāś ca kālatattvavido janāḥ |}%

  \maintext{śataṃ varṣasahasrāṇi trīṇi mānuṣasaṃkhyayā |}%

  \maintext{ṣaṣṭiṃ caiva sahasrāṇi kālaḥ kaliyugaḥ smṛtaḥ }||\thinspace1:21\thinspace||%
\translation{One year is twelve months [according to] people who know the entity of time. The time span of three hundred and sixty thousand years by human counting is said to be the Kali age {\rm (}\textit{kali\-yuga}{\rm )}. \blankfootnote{1.21 Note how a verb {\rm (}e.g. \textit{iti vadanti, iti prāhur}{\rm )} is missing in \textit{pāda}s ab.
 }}

  \maintext{dviguṇaḥ kalisaṃkhyāto dvāparo yuga saṃjñitaḥ |}%

  \maintext{tretā tu triguṇā jñeyā catuḥ kṛtayugaḥ smṛtaḥ }||\thinspace1:22\thinspace||%
\translation{The Dvāpara age is known to be twice as long as the Kali age. The Tretā age is thrice [as long], the Kṛta age four [times as long as the Kali age]. \blankfootnote{1.22 Note the stem form noun \textit{yuga} in \textit{pāda} b metri causa, or rather the compound \textit{dvāparo-yuga-saṃjñitaḥ}
  {\rm (}the end of \textit{dvāparo} lengthened to avoid the metrical fault of two \textit{laghu}s{\rm )},
  and also \msM's unique but confused readings.
 }}

  \maintext{eṣā caturyugāsaṃkhyā kṛtvā vai hy ekasaptatiḥ |}%

  \maintext{manvantarasya caikasya jñānam uktaṃ samāsataḥ }||\thinspace1:23\thinspace||%
\translation{This is the figure related to the four ages {\rm (}\textit{yuga}{\rm )}. Multiplying it by seventy-one, the knowledge about one time-span of a Manu {\rm (}\textit{manvantara}{\rm )} has been briefly taught. \blankfootnote{1.23 Note the lengthened vowel in °\textit{yugā} {\rm (}metri causa{\rm )}.
 
  The `figure' mentioned in this verse is the sum of the duration of the four \textit{yuga}s, 
  which makes up one \textit{mahāyuga}:
  Kaliyuga = 360,000 years,
  Dvāparayuga = 720,000 years,
  Tretāyuga = 1,080,000 years,
  Dvāparayuga = 1,440,000 years; altogether 3,600,000 years. 71 \textit{mahāyuga}s make up
  a \textit{manvantara} {\rm (}= 255,600,000 years; cf. Manu 1.79{\rm )}. 
  One \textit{kalpa} is 14 \textit{manvantara}s {\rm (}= 3,578,400,000 years{\rm )}. 
  Ten thousand \textit{kalpa}s are one day of Brahmā, and his night is of the same length, which
  would make one full day of Brahmā 71,568,000,000,000 human years. See next verses and,
  e.g., \mycite{GonzalezCosmic}.
 See \VSS\ 21.34ff on \textit{kalpa} etc.
 }}

  \maintext{kalpo manvantarāṇāṃ tu caturdaśa tu saṃkhyayā |}%

  \maintext{daśa kalpasahasrāṇi brahmāhaḥ parikalpitam |}%

  \maintext{rātrir etāvatī proktā munibhis tattvadarśibhiḥ }||\thinspace1:24\thinspace||%
\translation{One \ae on {\rm (}\textit{kalpa}{\rm )} is fourteen \textit{manvantara}s in total. Brahmā's day {\rm (}\textit{brahmāhar}{\rm )} is made up of ten thousand \ae ons {\rm (}\textit{kalpa}{\rm )}. [Brahmā's] night is of the same duration according to the wise who know the truth. \blankfootnote{1.24 The accepted reading \textit{kalpo} in \textit{pāda} a is probably not original.
 \msM\ has a separator sign {\rm (}|o|{\rm )} at the end of \textit{pāda} b, as if a section ended here.
 }}

  \maintext{rātryāgame pralīyante jagat sarvaṃ carācaram |}%

  \maintext{ahāgame tathaiveha utpadyante carācaram }||\thinspace1:25\thinspace||%
\translation{When [Brahmā's] night falls, the whole moving and unmoving universe dissolves. And when [his] daylight arrives, similarly, the moving and unmoving [universe] is born here. \blankfootnote{1.25 The plural form \textit{pralīyante} in \textit{pāda} a is metri causa for \textit{pralīyate},
  perhaps also influencing \textit{utpadyante} {\rm (}for \textit{utpadyate}{\rm )} in \textit{pāda} d,
  which in turn is used here to avoid an iambic pattern
  {\rm (}- - \shortsyllable\ - \shortsyllable\ - \shortsyllable\ -{\rm )}.
  Note a general lack of a sense of grammatical number {\rm (}see p.~\pageref{number}{\rm )}.
 }}

  \maintext{parārdhaparakalpāni atītāni dvijottama |}%

  \maintext{anāgataṃ tathaivāhur bhṛgurādimaharṣayaḥ }||\thinspace1:26\thinspace||%
\translation{One \textit{para} times \textit{parārdha} [number of, i.e. two hundred quadrillion times a hundred quadrillion] \ae ons {\rm (}\textit{kalpa}{\rm )} have passed [thus far], O great Brahmin. Bhṛgu and the other sages say that the future is the same [time span]. \blankfootnote{1.26 On the definition of the numbers \textit{para} and \textit{parārdha}, see verses 1.31--35.
 Note the peculiar compound \textit{bhṛgu-r-ādi-maharṣayaḥ}, for
  \textit{bhṛgvādimaharṣayaḥ}.
 }}

  \maintext{yathārkagrahatārendu bhramato dṛśyate tv iha |}%

  \maintext{kālacakraṃ bhramitvaiva viśramaṃ na ca vidmahe }||\thinspace1:27\thinspace||%
\translation{Just as the sun, the planets, the stars and the moon are perceived in this world as circling around, we, wandering around riding the wheel of time {\rm (}\textit{kālacakra}{\rm )}, can never have a rest. \blankfootnote{1.27 \textit{bhramato} in \textit{pāda} b seems to stand for the neuter participle \textit{bhramat}.
  Alternatively, \textit{bhramato} might mean `erroneously' {\rm (}\textit{bhrama-tas}, abl.{\rm )}, but this would
  make the verse difficult to interpret.
 I have corrected \textit{bhramatvaiva} to the standard form \textit{bhramitvaiva}, although the former
  might conceal a finate verb {\rm (}\textit{bhramāmaḥ}?{\rm )}.
 }}

  \maintext{kālaḥ sṛjati bhūtāni kālaḥ saṃharate punaḥ |}%

  \maintext{kālasya vaśagāḥ sarve na kālavaśakṛt kvacit }||\thinspace1:28\thinspace||%
\translation{Time creates living beings and time destroys them again. Everything is under the control of time. There is nothing that can bring time under control. }

  \maintext{caturdaśa parārdhāni devarājā dvijottama |}%

  \maintext{kālena samatītāni kālo hi duratikramaḥ }||\thinspace1:29\thinspace||%
\translation{Fourteen \textit{parārdha} [fourteen hundred quadrillion] god kings, O Brahmin, have passed with time, for time is difficult to overcome. \blankfootnote{1.29 Note that \textit{samatītāni} {\rm (}neuter{\rm )} most probably picks up \textit{devarājāḥ}
  {\rm (}masculine{\rm )} in this verse, or rather \textit{devarājā} stands for
  \textit{devarājānāṃ} and \textit{samatītāni} picks up \textit{°parā\-rdhāni}. It is not clear to me
  what \textit{devarāja} {\rm (}`god king'{\rm )} means exactly {\rm (}Indra?{\rm )}.
 }}

  \maintext{eṣa kālo mahāyogī brahmā viṣṇuḥ paraḥ śivaḥ |}%

  \maintext{anādinidhano dhātā sa mahātmā namaskuru }||\thinspace1:30\thinspace||%
\translation{Time is [manifest] as a great yogin, as Brahmā, Viṣṇu and supreme Śiva, is beginningless and endless, is the Creator and the great soul. Pay homage [to Time]. }

  \subchptr{parārdhādi}%

  \trsubchptr{\textit{Parārdha} etc.: numbers}%

  \maintext{vigatarāga uvāca |}%

  \maintext{śrutaṃ vai kālacakraṃ tu mukhapadmaviniḥsṛtam |}%

  \maintext{parārdhaṃ ca paraṃ caiva śrotuṃ vaḥ pratidīpitam }||\thinspace1:31\thinspace||%
\translation{Vigatarāga spoke: I have now heard about the `wheel of time' {\rm (}\textit{kālacakra}{\rm )} from [your] lotus mouth. [I wish] to hear about [the terms] \textit{parārdha} and \textit{para} [mentioned above], as elaborated by you. \blankfootnote{1.31 I have corrected the unmetrical \textit{vinisṛtam} in \textit{pāda} b to \textit{viniḥsṛtam}.
  The reading of all manuscripts consulted, \textit{vinisṛtam}, 
  may be considered metrical if we interpret it, loosely, as \textit{vinisritam}.
  Read \textit{tvanmukhapadma}° {\rm (}`your lotus mouth'{\rm )} over the \textit{pāda}-boundary?
  See, e.g., \SIVP\ 2.3.27.6ab:
  \textit{taj jñātvā nikhilaṃ devi śrutvā tvanmukhapaṃkajāt}.
  
 
  \textit{Pāda} d is suspect and my translation tentative.
  \msM's reading in \textit{pāda} d {\rm (}\textit{śrotuṃ naḥ pratidīyatāṃ}{\rm )} might make sense 
  {\rm (}`give it back/repeat it for us to hear'{\rm )}, but it sounds forced,
  as if the scribe tried to come up with a reading that he understood
  better than \textit{śrotuṃ vaḥ pratidīpitam}, the reading of the majority of the witnesses,
  which is in fact not easy to interpret. One would expect a phrase meaning
  `please tell me about these.' Finally, I have deicided to take \textit{vaḥ} as 
  instrumental {\rm (}`by you'{\rm )}. Still, a verb is missing.
 }}

  \maintext{anarthayajña uvāca |}%

  \maintext{ekaṃ daśaṃ śataṃ caiva sahasram ayutaṃ tathā |}%

  \maintext{prayutaṃ niyutaṃ koṭim arbudaṃ vṛndam eva ca }||\thinspace1:32\thinspace||%
\translation{Anarthayajña spoke: One, ten, a hundred, a thousand, ten thousand {\rm (}\textit{ayuta}{\rm )}, a hundred thousand {\rm (}\textit{prayuta}{\rm )}, a million {\rm (}\textit{niyuta}{\rm )}, ten million {\rm (}\textit{koṭi}{\rm )}, a hundred million {\rm (}\textit{arbuda}{\rm )}, one billion {\rm (}\textit{vṛnda}, 10$^{\englishfont\tiny 9\thinspace}${\rm )}, \blankfootnote{1.32 See a similar teaching of numbers in \BRAHMANDAPUR\ 3.2.91ff.
 }}

  \maintext{kharvaṃ caiva nikharvaṃ ca śaṅku padmaṃ tathaiva ca |}%

  \maintext{samudro madhyam antaṃ ca parārdhaṃ ca paraṃ tathā }||\thinspace1:33\thinspace||%
\translation{ten billion {\rm (}\textit{kharva}{\rm )}, a hundred billion {\rm (}\textit{nikharva}{\rm )}, one trillion {\rm (}\textit{śaṅku}, 10$^{\englishfont\tiny 12\thinspace}${\rm )}, ten trillion {\rm (}\textit{padma}{\rm )}, a hundred trillion {\rm (}\textit{samudra}{\rm )}, one quadrillion {\rm (}\textit{madhya}, 10$^{\englishfont\tiny 15\thinspace}${\rm )}, ten quadrillion {\rm (}\textit{[an]anta}{\rm )}, a hundred quadrillion {\rm (}\textit{parārdha}{\rm )}, and two hundred quadrillion {\rm (}\textit{para}{\rm )}. \blankfootnote{1.33 Note that \msPaperA\ inserts a line here. See apparatus.
 For \textit{anta} meaning \textit{ananta}, see 1.57. \msM's reading in \textit{pāda} d
  may be a result of an eyeskip to 1.34c.
 }}

  \maintext{sarve daśaguṇā jñeyāḥ parārdhaṃ yāvad eva hi |}%

  \maintext{parārdhadviguṇenaiva parasaṃkhyā vidhīyate }||\thinspace1:34\thinspace||%
\translation{Each should be known as powers of ten up to \textit{parārdha}. The number corresponding to \textit{para} is double that of \textit{parārdha}. }

  \maintext{parāt parataraṃ nāsti iti me niścitā matiḥ |}%

  \maintext{purāṇavedapaṭhitā mayākhyātā dvijottama }||\thinspace1:35\thinspace||%
\translation{There is no higher number than \textit{para}. This is my firm conviction, which is based on my readings of the Purāṇas and the Vedas and [which I have now] taught [to you], O great Brahmin. \blankfootnote{1.35 Note that \Ed\ inserts the line here that \msPaperA\ inserted above. See apparatus.
 }}

  \subchptr{brahmāṇḍam}%

  \trsubchptr{Brahmā's Egg: the Universe}%

  \maintext{vigatarāga uvāca |}%

  \maintext{brahmāṇḍaṃ kati vijñeyaṃ pramāṇaṃ jñāpitaṃ kvacit |}%

  \maintext{kati cāṅguli{-}m{-}ūrdhveṣu sūryas tapati vai mahīm }||\thinspace1:36\thinspace||%
\translation{Vigatarāga spoke: What is the extent of the Brahmāṇḍa [i.e. the universe]? Is it disclosed anywhere? From how many finger's breadths high does the sun heat the earth? \blankfootnote{1.36 The use of the singular next to numerals is one of the hallmarks of the \VSS\ 
  {\rm (}see p.~\pageref{singularwithnumerals}{\rm )}. This means that \textit{pāda} a may well refer to multiple \textit{brahmāṇḍa}s.
  Nevertheless, in the light of \VSS\ 2.2d {\rm (}\textit{pramāṇaṃ tasya vā kati}{\rm )}, I suspect that 
  the first question here could be rendered in slightly more standard Sanskrit as
  \textit{brahmāṇḍasya pramāṇaṃ kati yojanāni vijñeyaṃ}.
  \textit{cāpitaṃ kvacit} in \textit{pāda} b in the witnesses is enigmatic.
  One may conjecture \textit{prāpitaṃ} {\rm (}perhaps: `is it available somewhere?'{\rm )}, 
  The intended form may have been \textit{jñātaṃ kenacit} {\rm (}`is it known by anyone?'{\rm )},
  or \textit{jñāpitaṃ} {\rm (}`is it disclosed somewhere?'{\rm )}. I have chosen the latter.
  to which 1.37 below could be a reply. Of course, \textit{cāpitaṃ} could be analysed as
  \textit{cāpi taṃ} {\rm (}possibly for \textit{cāpi tat}{\rm )}, but that would help little, unless we
  imagine that the question is `and where is it?' {\rm (}\textit{cāpi tat kva}{\rm )}.
 
  
 My emendation of \textit{cāṅguli-mūrdheṣu} to \textit{cāṅguli{-}m{-}ūrdhveṣu} {\rm (}with a hiatus-filler{\rm )} 
  is based on \textit{ūrdhvatas} in 1.60d, which is part of the reply to the question posed in this line.
  In turn, \textit{aṅguli} here triggered an conjecture in 1.60c.
 }}

  \maintext{anarthayajña uvāca |}%

  \maintext{brahmāṇḍānāṃ prasaṃkhyātuṃ mayā śakyaṃ kathaṃ dvija |}%

  \maintext{devās te 'pi na jānanti mānuṣāṇāṃ ca kā kathā }||\thinspace1:37\thinspace||%
\translation{Anarthayajña spoke: How could I enumerate [all the details of] the Brahmāṇḍa, O twice-born? Even the gods do not know, not to mention humans. \blankfootnote{1.37 One would expect \textit{brahmāṇḍāni} in \textit{pāda} a instead of \textit{brahmāṇḍānāṃ},
  but we should probably understand \textit{brahmāṇḍānāṃ viśeṣān prasaṃkhyātuṃ...}
  The structure noun in genitive + verb meaning `telling' occurs also in 4.69a and \verify .
 }}

  \maintext{paryāyeṇa tu vakṣyāmi yathāśakyaṃ dvijottama |}%

  \maintext{brahmaṇā yat purākhyāto mātariśvā yathā tathā }||\thinspace1:38\thinspace||%
\translation{I shall teach [you], as far as I can, in due order and truthfully, that, O great Brahmin, which Mātariśvan was taught by Brahmā in the past. \blankfootnote{1.38 The claim that Brahmā taught Mātariśvan is confirmed in 1.62cd, and
  also, e.g., in \BrahmandaPur\ 3.4.58cd {\rm (}see the apparatus{\rm )}.
 }}

  \maintext{śivāṇḍābhyantareṇaiva sarveṣām iva bhūbhṛtām |}%

  \maintext{daśa nāma diśāṣṭānāṃ brahmāṇḍe kīrtitaṃ śṛṇu }||\thinspace1:39\thinspace||%
\translation{Ten names of all the [cosmic] rulers of each of the eight directions in Brahmā's Egg, [which is] inside Śiva's Egg, are being taught now, listen. \blankfootnote{1.39 My conjecture in \textit{pāda} b {\rm (}\textit{bhūbhṛtām}{\rm )} is based on the fact that the 
  readings transmitted in the MSS seem unintelligible, and, more importantly, that
  these names are said, in the subsequent verses, to belong to \textit{nāyaka}s {\rm (}`chiefs, lords'{\rm )},
  a possible synonym of \textit{bhūbhṛt} {\rm (}`a king'{\rm )}, and also that it is a minute intervention.
 
  In \textit{pāda} c, understand \textit{diśāṣṭānāṃ} as \textit{diśām aṣṭānāṃ} or \textit{digaṣṭakānāṃ}, 
  and note that one of the hallmarks of the language of the \VSS\ is the use
  of the singular in the proximity of numbers, where a plural would be expected {\rm (}\textit{daśa nāma}{\rm )}.
 }}

  \subchptr{bhūbhṛtāṃ nāmāni}%

  \trsubchptr{Names of the cosmic rulers}%

  \subsubchptr{pūrvataḥ}%

  \trsubsubchptr{East}%

  \maintext{sahāsahaḥ sahaḥ sahyo visahaḥ saṃhato 'sabhā |}%

  \maintext{prasaho 'prasahaḥ sānuḥ pūrvato daśa nāyakāḥ }||\thinspace1:40\thinspace||%
\translation{[1] Sahā, [2] Asaha, [3] Saha, [4] Sahya, [5] Visaha, [6] Saṃhata, [7] Asabhā, [8] Prasaha, [9] Aprasaha, [10] Sānu: [these are] the ten Leaders in the East. \blankfootnote{1.40 Note that many of the names here and in the following verses are,
  in the absence of any parallel passage, rather insecure.
  In order to avoid the repetition of the name Saha, I take the first name here
  as feminine; Asabhā seems also to be a feminine ruler's name. Later on there
  seem to come more feminine names {\rm (}Tejā, Yamunā, Naganā, etc.{\rm )}, therefore it 
  may be correct to interpret some of the names as names of queens.
  What is clear here is that the list evokes the name Sahasrākṣa,
  one of the appellations of Indra, the guardian of the eastern direction.
 }}

  \subsubchptr{āgneye}%

  \trsubsubchptr{South-East}%

  \maintext{prabhāso bhāsano bhānuḥ pradyoto dyutimo dyutiḥ |}%

  \maintext{dīptatejāś ca tejāś ca tejā tejavaho daśa |}%

  \maintext{āgneye tv etad ākhyātaṃ yāmye śṛṇv atha bho dvija }||\thinspace1:41\thinspace||%
\translation{[1] Prabhāsa, [2] Bhāsana, [3] Bhānu, [4] Pradyota, [5] Dyutima, [6] Dyuti, [7] Dīptatejas, [8] Tejas, [9] Tejā, [10] Tejavaho: [these are] the ten [rulers] in the direction of Agni [SE]. Now listen to [the names for] Yama's region, O twice-born. \blankfootnote{1.41 Here, in the region of Agni, the names evidently evoke the image of flames.
 }}

  \subsubchptr{yāmye}%

  \trsubsubchptr{South}%

  \maintext{yamo 'tha yamunā yāmaḥ saṃyamo yamuno 'yamaḥ |}%

  \maintext{saṃyano yamanoyāno yaniyugmā yanoyanaḥ }||\thinspace1:42\thinspace||%
\translation{[1] Yama, [2] Yamunā, [3] Yāma, [4] Saṃyama, [5] Yamuna, [6] Ayama, [7] Saṃyana, [8] Yamanoyāna, [9] Yaniyugmā, [10] Yanoyana. \blankfootnote{1.42 I have chosen the variant \textit{saṃyano} in \textit{pāda} c only to avoid the repetition of
  the name \textit{saṃyama}, and the variant \textit{yanoyanaḥ} in \textit{pāda} d because I suspect that
  most of the names here should begin with \textit{ya}, except for \textit{ayamaḥ} in 
  in \textit{pāda} b, which is a guess to avoide the repetition of \textit{yamaḥ}.
  All the name forms in this verse are to be taken as tentative. The only 
  guiding light is the presence of \textit{ya}, reinforcing a connection with Yama.
 }}

  \subsubchptr{nairṛte}%

  \trsubsubchptr{South-West}%

  \maintext{nagajo naganā nando nagaro naga nandanaḥ |}%

  \maintext{nagarbho gahano guhyo gūḍhajo daśa tatparaḥ }||\thinspace1:43\thinspace||%
\translation{[1] Nagaja, [2] Naganā, [3] Nanda, [4] Nagara, [5] Naga, [6] Nandana, [7] Nagarbha, [8] Gahana, [9] Guhyo, [10] Gūḍhaja: [these are] the ten associated with [the South-West]. \blankfootnote{1.43 \textit{naga} in \textit{pāda} b is a stem form noun metri causa
 \textit{tatparaḥ} in \textit{pāda} d might be another example of a singular form next to a number {\rm (}see 1.39c above{\rm )}.
  Note that the reconstruction of these names are tentative. What is clear here is that the
  initials should be \textit{na} and \textit{ga}, probably suggesting a connection with \textit{nirṛti}, \textit{naraka}s, and \textit{nāga}s.
 }}

  \subsubchptr{vāruṇe}%

  \trsubsubchptr{West}%

  \maintext{vāruṇena pravakṣyāmi śṛṇu vipra nibodha me |}%

  \maintext{babhraḥ setur bhavodbhadraḥ prabhavodbhavabhājanaḥ |}%

  \maintext{bharaṇo bhuvano bhartā daśaite varuṇālayāḥ }||\thinspace1:44\thinspace||%
\translation{I shall teach you [the names] in Varuṇa's region [in the west]. Listen, O Brahmin, learn from me. [1] Babhra, [2] Setu, [3] Bhava, [4] Udbhadra, [5] Prabhava, [6] Udbhava, [7] Bhājana, [8] Bharaṇa, [9] Bhuvana, and [10] Bhartṛ: these ten dwell in Varuṇa's region [in the west]. \blankfootnote{1.44 Varuṇa upholds {\rm (}\textit{bibharti/bharati}{\rm )} the sky and the earth. This could be the reason why 
  these names include \textit{bharaṇa} and \textit{bhartṛ}.
 }}

  \subsubchptr{vāyavye}%

  \trsubsubchptr{North-West}%

  \maintext{nṛgarbho 'suragarbhaś ca devagarbho mahīdharaḥ |}%

  \maintext{vṛṣabho vṛṣagarbhaś ca vṛṣāṅko vṛṣabhadhvajaḥ }||\thinspace1:45\thinspace||%
\translation{[1] Nṛgarbha, [2] Asuragarbha, [3] Devagarbha, [4] Mahīdhara, [5] Vṛṣabha, [6] Vṛṣagarbha, [7] Vṛṣāṅka, [8] Vṛṣabhadhvaja, \blankfootnote{1.45 The connection between \textit{vṛṣa} and the north-west or Vāyu is not evident to me. \verify
  In a tantric context, a western position is more standard for \textit{vṛṣa}, see e.g.
  \mycitep{Pancavaranastava}{40}.
 }}

  \maintext{jñātavyaś ca tathā samyag vṛṣajo vṛṣanandanaḥ |}%

  \maintext{nāyakā daśa vāyavye kīrtitā ye mayā dvija }||\thinspace1:46\thinspace||%
\translation{[9] Vṛṣaja, and [10] Vṛṣanandana: these are to be known properly as the ten leaders in Vāyu's region [in the north-west], as I taught them, O twice-born. \blankfootnote{1.46 Note how \msM\ deviates here again in a significant way.
 }}

  \subsubchptr{uttare}%

  \trsubsubchptr{North}%

  \maintext{sulabhaḥ sumanaḥ saumyaḥ suprajaḥ sutanuḥ śivaḥ |}%

  \maintext{sataḥ satya layaḥ śambhur daśa nāyakam uttare }||\thinspace1:47\thinspace||%
\translation{[1] Sulabha, [2] Sumana, [3] Saumya, [4] Supraja, [5] Sutanu, [6] Śiva, [7] Sata, [8] Satya, [9] Laya, [10] Śambhu: [these are] the ten leaders in the north. \blankfootnote{1.47 I prefer the form \textit{sumanaḥ} to the more standard \textit{sumanāḥ} {\rm (}\msNc{\rm )} in \textit{pāda} a
  because it suits the slightly irregluar language of the \VSS\ {\rm (}see pp. \verify{\rm )},
  and because the solitary reading of \msNc\ may well only be an attempt to
  standardise. It is also not inconceivable that \textit{sumanaḥ} stands compounded 
  with \textit{saumyaḥ}.
 Note how \textit{daśa nāyakam} {\rm (}neuter singular for masculine plural{\rm )}
  could again be an example for the use of the singular 
  next to a number in \textit{pāda d}. It seems that here it is the northern region
  that is associated with Śiva, rather than the north-east, the \textit{īśāna} direction, 
  which is here occupied by Brahmā: see next verse. 
  In a tantric context, Brahmā is sometimes associated with the north-east, see, e.g.,
  \mycitep{Pancavaranastava}{39}.
  I have left \textit{satya} in stem form.
 }}

  \subsubchptr{īśāne}%

  \trsubsubchptr{North-East}%

  \maintext{indu bindu bhuvo vajra varado vara varṣaṇaḥ |}%

  \maintext{ilano valino brahmā daśeśāneṣu nāyakāḥ }||\thinspace1:48\thinspace||%
\translation{[1] Indu, [2] Bindu, [3] Bhuva, [4] Vajra, [5] Varada, [6] Vara, [7] Varṣaṇa, [8] Ilana, [9] Valina, [10] Brahmā: [these are] the ten rulers in the Īśāna direction [i.e. in the north-east]. \blankfootnote{1.48 I consider \textit{indu, bindu} and \textit{vajra} stem form nouns.
 The north-east seems to be occupied by Brahmā, and by rulers whose names should
  somehow evoke Brahmā's name.
 }}

  \subsubchptr{madhyame}%

  \trsubsubchptr{Center}%

  \maintext{aparo vimalo moho nirmalo mana mohanaḥ |}%

  \maintext{akṣayaś cāvyayo viṣṇur varado madhyame daśa }||\thinspace1:49\thinspace||%
\translation{[1] Apara, [2] Vimala, [3] Moha, [4] Nirmala, [5] Mana, [6] Mohana, [7] Akṣaya, [8] Avyaya, [9] Viṣṇu, [10] Varada: [these are] the ten [leaders] in the centre. \blankfootnote{1.49 Note that the last three lists above have been associated 
  with Śiva, Brahmā and Viṣṇu, respectively, and here, in a layer
  of the text that can be labelled Vaiṣṇava {\rm (}see pp. \verify{\rm )}, it is Viṣṇu that
  seems to occupy a central position. \textit{mana mohanaḥ} {\rm (}or \textit{nirmalonmana}{\rm )} in \textit{pāda} b may
  sound like one single name, but we are forced to separate these two words
  {\rm (}\textit{mana} being in stem form metri causa{\rm )} to arrive at a list of ten names.
 }}

  \subsubchptr{parivārāḥ}%

  \trsubsubchptr{Subordinates}%

  \maintext{sarveṣāṃ daśa{-}m{-}īśānāṃ parivāraśataṃ śatam |}%

  \maintext{śatānāṃ pṛthag ekaikaṃ sahasraiḥ parivāritam }||\thinspace1:50\thinspace||%
\translation{Each of the ten rulers has a retinue of a hundred subordinates. Each one of [these] hundred is surrounded by a thousand subordinates. \blankfootnote{1.50 I take \textit{daśa-m-īśānāṃ} as a split compound {\rm (}\textit{daśeśānāṃ}{\rm )}.
  It is conceivable that each of the above ninety rulers has ten subordinates, 
  therefore each group of ten rulers has a hundred subordinates altogether,
  but the original idea may have been that each one of the above ninety 
  rulers has a hundred subordinates. Alternatively, this verse may only refer to 
  the central group of ten rulers mentioned in 1.49, and each one of them has
  a hundred subordinates.
 }}

  \maintext{sahasreṣu ca ekaikam ayutaiḥ parivāritam |}%

  \maintext{ayutaṃ prayutair vṛndaiḥ prayutaṃ niyutair vṛtam }||\thinspace1:51\thinspace||%
\translation{Each one of the thousand is surrounded by ten thousand [subordinates], the ten thousand is surrounded by a multitude of a hundred thousand, the hundred thousand by a million, \blankfootnote{1.51 We are forced to follow \Ed's reading in \textit{pāda} c in order to make sense of this passage.
  My correction in \textit{pāda} d is motivated by the same. Note that \textit{vṛnda} is not a number in this line. 
  Elsewhere in this chapter \textit{vṛnda} is the word that signifies `a billion.'
 }}

  \maintext{ekaikasya parīvāro niyutaḥ pṛthag eva ca |}%

  \maintext{koṭibhir daśakoṭyena ekaikaḥ parivāritaḥ }||\thinspace1:52\thinspace||%
\translation{[that is] each one has a retinue of a million {\rm (}\textit{niyuta}{\rm )} [subordinates]. [Then those] are surrounded by ten million {\rm (}\textit{koṭi}{\rm )} [subordinates], [they in turn] by a hundred million {\rm (}\textit{daśakoṭi}{\rm )}. \blankfootnote{1.52 It seems that \textit{pāda}s ab repeat what has been stated in 1.51cd.
 \textit{°koṭyena} stands for \textit{°koṭyā} {\rm (}thematisation{\rm )}.
  Note how the scribe of \msM\ gets confused at 1.52c due to an eye-skip and 
  fully regains control only at 1.54b.
 }}

  \maintext{daśakoṭiṣu ekaikaṃ vṛndavṛndabhṛtair vṛtam |}%

  \maintext{vṛndavargeṣu ekaikaṃ kharvabhiḥ parivāritam }||\thinspace1:53\thinspace||%
\translation{Each one of the hundred million is surrounded by a billion {\rm (}\textit{vṛnda}{\rm )} subordinates {\rm (}\textit{bhṛta}{\rm )}. Each one in these groups of a billion {\rm (}\textit{vṛnda}{\rm )} is surrounded by ten billion {\rm (}\textit{kharva}{\rm )} [subordinates]. }

  \maintext{kharvavargeṣu ekaikaṃ daśakharvagaṇair vṛtam |}%

  \maintext{daśakharveṣu ekaikaṃ śaṅkubhiḥ parivāritam }||\thinspace1:54\thinspace||%
\translation{Each in these groups of ten billion {\rm (}\textit{kharva}{\rm )} is surrounded by a hundred billion {\rm (}\textit{daśakharva}{\rm )}. Each of those hundred billion is surrounded by a trillion {\rm (}\textit{śaṅku}{\rm )} [deities]. }

  \maintext{śaṅkubhiḥ pṛthag ekaikaṃ padmena parivāritam |}%

  \maintext{padmavargeṣu ekaikaṃ samudraiḥ parivāritam }||\thinspace1:55\thinspace||%
\translation{Each of those one trillion is surrounded by ten trillion {\rm (}\textit{padma}{\rm )}. Each of those ten trillion is surrounded by a hundred trillion {\rm (}\textit{samudra}{\rm )}. \blankfootnote{1.55 Note that in \textit{pāda} a \textit{śaṅkubhiḥ} stands for \textit{śaṅkūṣu} {\rm (}instrumental for locative{\rm )}.
 }}

  \maintext{samudreṣu tathaikaikaṃ madhyasaṃkhyais tu tair vṛtam |}%

  \maintext{madhyasaṃkhyeṣu ekaikam anantaiḥ parivāritam }||\thinspace1:56\thinspace||%
\translation{And each of those hundred trillion is surrounded by those whose number is one quadrillion {\rm (}\textit{madhya}{\rm )}. Each of those quadrillion is surrounded by ten quadrillion {\rm (}\textit{ananta}{\rm )}. }

  \maintext{ananteṣu ca ekaikaṃ parārdhaparivāritam |}%

  \maintext{parārdheṣu ca ekaikaṃ pareṇa parivāritam |}%

  \maintext{eṣa vai kathito vipra śakyaṃ sāṃkhyam udīritam }||\thinspace1:57\thinspace||%
\translation{Each of those ten quadrillion is surrounded by a hundred quadrillion {\rm (}\textit{parārdha}{\rm )}. Each of those hundred quadrillion is surrounded by two hundred quadrillion {\rm (}\textit{para}{\rm )}. This is how it is taught, O Brahmin. The enumeration [of the rulers of the Brahmāṇḍa] has been taught as much as it is possible. }

  \subchptr{pramāṇam}%

  \trsubchptr{Measurements}%

  \maintext{pramāṇaṃ śṛṇu me vipra saṃkṣepād bruvato mama |}%

  \maintext{candrodaye pūrṇamāsyāṃ vapur aṇḍasya tādṛśam }||\thinspace1:58\thinspace||%
\translation{Listen to me and learn about the measurements [of the universe, or Brahmā's Egg], O Brahmin, I shall teach [you] in a concise manner. The body of the Egg is like that of [the moon] at moonrise on the day of the full moon. }

  \maintext{koṭikoṭisahasraṃ tu yojanānāṃ samantataḥ |}%

  \maintext{aṇḍānāṃ ca parīmāṇaṃ brahmaṇā parikīrtitam }||\thinspace1:59\thinspace||%
\translation{The whole circumference of the Egg has been declared by Brahmā to be ten million {\rm (}\textit{koṭi}{\rm )} times a thousand times ten million \textit{yojana}s. \blankfootnote{1.59 Note on plural \verify 
 }}

  \maintext{saptakoṭisahasrāṇi saptakoṭiśatāni ca |}%

  \maintext{viṃśakoṭiṣv aṅgulīṣu ūrdhvatas tapate raviḥ }||\thinspace1:60\thinspace||%
\translation{The Sun shines from the height of seven thousand seven hundred and twenty \textit{koṭi} finger's breadth. \blankfootnote{1.60 This verse is the reply to the question in 1.36cd, which contains the word \textit{aṅguli}:
  this hints at the possibility that the unintelligible \textit{gulmeṣu} transmitted in most of the
  witnesses might be corrupted from \textit{aṅguīṣu}; hence my conjecture, resulting
  in a \textit{ra-vipulā}.
 }}

  \maintext{pramāṇaṃ nāma saṃkhyā ca kīrtitāni samāsataḥ |}%

  \maintext{brahmāṇḍaṃ cāprameyāṇāṃ lakṣaṇaṃ parikīrtitam }||\thinspace1:61\thinspace||%
\translation{The numbers pertaining to the measurements have been taught in brief. The characteristics of the unmeasurable Brahmāṇḍa[s] have been taught. \blankfootnote{1.61 Note the mixture of different grammatical genders and numbers in this verse. 
  Understand \textit{pramāṇeṣu saṃkhyāḥ kīrtitāḥ samāsataḥ} and 
  \textit{brahmāṇḍānām aprameyānāṃ}...
 }}

  \subchptr{purāṇam}%

  \trsubchptr{Redactors of the Purāṇa{\rm [}s{\rm ]}}%

  \maintext{purāṇāśīsahasrāṇi śatāni dvijasattama |}%

  \maintext{brahmaṇā kathitaṃ pūrṇaṃ mātariśvā yathātatham }||\thinspace1:62\thinspace||%
\translation{O truest of the twice-born, the Purāṇa[s of] 8,000,000 [verses] were taught by [1] Brahmā to [2] Mātariśvan [= Vāyu] in their entirety, in their true form. \blankfootnote{1.62 \textit{Pāda} a should probably be analysed and interpreted as 
  \textit{purāṇam brahmaṇā kathitam} {\rm (}\textit{purāṇānām aśītisahasrāṇi śatāni ślokāni
  brahmaṇā kathitāni}{\rm )}.
  Alternatively, \textit{pāda} a may have originally read \textit{purāṇāni sahasrāṇi},
  and then the initial number of verses transmitted by Brahmā is
  a hundred thousand. That the number refers to the number of \textit{śloka}s
  transmitted is confirmed in 1.65d: \textit{viṃśatślokasahasrikam}.
 
  On the idea that initially there was only one Purāṇa, see, e.g.,
  \mycitep{RocherPuranas1986}{41ff}.
 
  
 
  In \textit{pāda} d, either understand \textit{mātariśvā} {\rm (}nom.{\rm )} as \textit{mātariśvānaṃ} {\rm (}acc.{\rm )} or emend
  \textit{kathitaṃ} to \textit{kathitaḥ} in the sense `Mātariśvan was taught,' echoing 1.38cd:
  \textit{brahmaṇā yat purākhyāto mātariśvā yathā tathā}.
 
  
 
  Compare this list to a list of twenty-eight \textit{vedavyāsa}s, from
  Brahmā to Vyāsa Dvaipāyana, in \VISNUP\ 3.3.10--19,
  taught by Parāśara, the twenty-sixth \textit{vyāsa}
  of this list and our text {\rm (}in the numbering that I add here I follow
  the translation in \mycitep{Visnupurana_tr}{178--179}{\rm )}:
  
 
  \textit{vedavyāsā vyatītā ye aṣṭāviṃśati sattama~|
  caturdhā yaiḥ kṛto vedo dvāpareṣu punaḥ punaḥ~||
  dvāpare prathame vyastāḥ svayaṃ vedāḥ }[1]\textit{ svayaṃbhuvā~|
  dvitīye dvāpare caiva vedavyāsaḥ }[2]\textit{ prajāpati~||
  tṛtīye }[3]\textit{ cośanā vyāsaś caturthe ca }[4]\textit{ bṛhaspatiḥ~| 
  }[5]\textit{ savitā pañcame vyāsaḥ }[6]\textit{ mṛtyuḥ ṣaṣṭhe smṛtaḥ prabhuḥ~|| 
  saptame ca }[7]\textit{ tathaivendro }[8]\textit{ vasiṣṭhaś cāṣṭame smṛtaḥ~| 
  }[9]\textit{ sārasvataś ca navame }[10]\textit{ tridhāmā daśame smṛtaḥ~|| 
  ekādaśe tu }[11]\textit{ trivṛṣā }[12]\textit{ bhāradvājas tataḥ param~| 
  trayodaśe }[13]\textit{ cāntarikṣo }[14]\textit{ varṇī cāpi caturdaśe~|| 
  }[15]\textit{ trayyāruṇaḥ pañcadaśe ṣoḍaśe tu }[16]\textit{ dhanaṃjayaḥ~| 
  }[17]\textit{ kratuṃjayaḥ saptadaśe }[18]\textit{ ṛṇajyo 'ṣṭādaśe smṛtaḥ~|| 
  tato vyāso }[19]\textit{ bharadvājo bharadvājāt tu }[20]\textit{ gautamaḥ~| 
  gautamād uttamo vyāso }[21]\textit{ haryātmā yo 'bhidhīyate~|| 
  atha haryātmano }[22]\textit{ venaḥ smṛto vājaśravās tu yaḥ~| 
  somaḥ śuṣmāyaṇas tasmāt }[23]\textit{ tṛṇabindur iti smṛtaḥ~|| 
  }[24]\textit{ ṛkṣo 'bhūd bhārgavas tasmād vālmīkir yo 'bhidhīyate~| 
  tasmād asmatpitā }[25]\textit{ śaktir vyāsas tasmād }[26]\textit{ ahaṃ mune~|| 
  }[27]\textit{ jātukarṇo 'bhavan mattaḥ kṛṣṇadvaipāyanas }[28]\textit{ tataḥ~| 
  aṣṭaviṃśatir ity ete vedavyāsāḥ purātanāḥ~||. }
 
  
 
  Another relevant passage is \BRAHMANDAPUR\ 3.4.58cd--67 {\rm (}\similar\ \VAYUP\ 2.41.58--67{\rm )}.
  Note how Tṛṇabindu is, perhaps by mistake, different from Somaśuṣma/Śuṣmāyaṇa here,
  but, more importantly, note Amitabuddhi of \VSS\ 1.75b appearing at the end of this list:
  
 
  [1] \textit{brahmā dadau śāstram idaṃ purāṇaṃ }[2]\textit{ mātariśvane~|| 
  tasmāc }[3]\textit{ cośanasā prāptaṃ tasmāc cāpi }[4]\textit{ bṛhaspatiḥ~| 
  bṛhaspatis tu provāca }[5]\textit{ savitre tadanantaram~|| 
  savitā }[6]\textit{ mṛtyave prāha mṛtyuś }[7]\textit{ cendrāya vai punaḥ~| 
  indraś cāpi }[8]\textit{ vasiṣṭāya so 'pi }[9]\textit{ sārasvatāya ca~|| 
  sārasvatas }[10]\textit{ tridhāmne 'tha tridhāmā ca }[11]\textit{ śaradvate~| 
  śaradvāṃs tu }[12]\textit{ triviṣṭāya so }[13]\textit{ 'ntarikṣāya dattavān~|| 
  }[14]\textit{ carṣiṇe cāntarikṣo vai so 'pi }[15]\textit{ trayyāruṇāya ca~| 
  trayyāruṇād }[16]\textit{ dhanañjayaḥ sa vai prādāt }[17]\textit{ kṛtañjaye~|| 
  kṛtañjayāt }[18]\textit{ tṛṇañjayo }[19]\textit{ bharadvājāya so 'py atha~| 
  }[20]\textit{ gautamāya bharadvājaḥ so 'pi }[21]\textit{ niryyantare punaḥ~|| 
  niryyantaras tu provāca tathā }[22]\textit{ vājaśravāya vai~| 
  sa dadau }[23]\textit{ somaśuṣmāya sa cādāt }[24]\textit{ tṛṇabindave~|| 
  tṛṇabindus tu }[25]\textit{ dakṣāya dakṣaḥ provāca }[26]\textit{ śaktaye~| 
  śakteḥ }[27]\textit{ parāśaraś cāpi garbhasthaḥ śrutavān idam~|| 
  parāśarāj }[28]\textit{ jātukarṇyas tasmād }[29]\textit{ dvaipāyanaḥ prabhuḥ~| 
  dvaipāyanāt punaś cāpi }[30]\textit{ mayā prāptaṃ dvijottama~|| 
  mayā caitat punaḥ proktaṃ }[31]\textit{ putrāyāmitabuddhaye~| 
  ity eva vākyaṃ brahmādiguruṇāṃ samudāhṛtam~||.} 
 
  
 
  The list of \textit{vedavyāsa}s in \LINPU\ 1.7.15--18 includes these twenty-five names:
  Kratu, Satya, Bhārgava, Aṅgiras, Savitṛ,
  Mṛtyu, Śatakratu, Vasiṣṭha, Sārasvata, Tridhāman,
  Trivṛta, Śatatejas, Tarakṣu, Āruṇi, Kṛtaṃjaya,
  Ṛtaṃjayo, Bharadvāja, Gautama, Vācaśravas, Tṛṇabindu,
  Rūkṣa, Śakti, Jātūkarṇya, Kṛṣṇa Dvaipāyana.
 }}

  \maintext{vāyunā pāda saṃkṣipya prāptaṃ cośanasaṃ purā |}%

  \maintext{tenāpi pāda saṃkṣipya prāptavāṃś ca bṛhaspatiḥ }||\thinspace1:63\thinspace||%
\translation{Vāyu abridged the verses and then gave [the Purāṇas] to [3] Uśanas. He [Uśanas] also abridged the verses, and [4] Bṛhaspati received them. \blankfootnote{1.63 Note the stem form noun \textit{pāda} twice in this verse and 
  the slightly odd grammatical structure in \textit{pāda} b, {\rm (}\textit{purāṇaṃ}{\rm )} \textit{prāptam uśanasam} {\rm (}`the Purāṇa reached Uśanas'{\rm )},
  as opposed to the solution in \textit{pāda} d with \textit{prāptavān}.
 }}

  \maintext{bṛhaspatis tu provāca sūryaṃ triṃśatsahasrikam |}%

  \maintext{pañcaviṃśatsahasrāṇi mṛtyuṃ prāha divākaraḥ }||\thinspace1:64\thinspace||%
\translation{Bṛhaspati taught 30,000 [verses] to [5] Sūrya [the Sun]. Divākara [= the Sun] taught 25,000 [verses] to [6] Mṛtyu [Death]. \blankfootnote{1.64 \textit{Pāda}s ab are a ma-\textit{vipulā}, or simply a \textit{pathyā} if \textit{pra} in \textit{provāca}
  does not turn the previous syllable long {\rm (}muta cum liquida{\rm )}.
 }}

  \maintext{ekaviṃśatsahasrāṇi mṛtyunendrāya kīrtitam | }%

  \maintext{indreṇāha vasiṣṭhāya viṃśatślokasahasrikam }||\thinspace1:65\thinspace||%
\translation{Mṛtyu taught 21,000 [verses] to [7] Indra. Indra taught 20,000 verses to [8] Vasiṣṭha. }

  \maintext{aṣṭādaśasahasrāṇi tena sārasvatāya tu |}%

  \maintext{sārasvatas tridhāmāya sahasradaśa sapta ca }||\thinspace1:66\thinspace||%
\translation{And he[, Vasiṣṭha taught] 18,000 [verses] to [9] Sārasvata. Sārasvata [taught] 17,000 [verses] to [10] Tridhāma[n]. }

  \maintext{ṣoḍaśānāṃ sahasrāṇi bharadvājāya vai tataḥ |}%

  \maintext{daśa pañcasahasrāṇi trivṛṣāya abhāṣata }||\thinspace1:67\thinspace||%
\translation{[He taught] 16,000 verses to [11] Bharadvāja. [Bharadvāja] taught 15,000 verses to [12] Trivṛṣa. }

  \maintext{caturdaśasahasrāṇi antarīkṣāya vai tataḥ |}%

  \maintext{trayyāruṇiṃ sahasrāṇi trayodaśa abhāṣata }||\thinspace1:68\thinspace||%
\translation{[Trivṛṣa] then [taught] 14,000 verses to [13] Antarīkṣa. [Antarīkṣa] taught 13,000 [verses] to [14] Trayyāruṇi. }

  \maintext{trayyāruṇis tu viprendro dhanaṃjayam abhāṣata |}%

  \maintext{dvādaśāni sahasrāṇi saṃkṣipya punar abravīt }||\thinspace1:69\thinspace||%
\translation{Trayyāruṇi, the great Brahmin, having abridged them again, taught 12,000 [verses] to [15] Dhanaṃjaya. }

  \maintext{kṛtaṃjayāya samprāpto dhanaṃjayamahāmuniḥ |}%

  \maintext{kṛtaṃjayād dvijaśreṣṭha ṛṇaṃjayamahātmane }||\thinspace1:70\thinspace||%
\translation{Dhanaṃjaya, the great sage, handed [them] over to [16] Kṛtaṃjaya. [That recension was transmitted] from Kṛtaṃjaya, O best of the twice-born, to [17] noble Ṛṇaṃjaya. \blankfootnote{1.70 Note the odd structure in \textit{pāda}s ab: \textit{dhanaṃjayaḥ kṛtaṃjayāya samprāptaḥ}, 
  for a more standard \textit{dhanaṃjayena {\rm (}\textit{purāṇam}{\rm )} samprāpitaṃ kṛtaṃjayam} 
  {\rm (}`the Purāṇa was transmitted to Kṛtaṃjaya'{\rm )}.
 }}

  \maintext{ṛṇañjayāt punaḥ prāpto gautamāya maharṣiṇe |}%

  \maintext{gautamāc ca bharadvājas tasmād dharyadvatāya tu }||\thinspace1:71\thinspace||%
\translation{Then from Ṛṇaṃjaya it was given to [18] Gautama, the great sage, from Gautama to [19] Bharadvāja, from him to [20] Haryātman. \blankfootnote{1.71 The structure of \textit{pāda}s ab is as odd as that of 1.70ab. What was
  intended is probably \textit{ṛṇañjayena prāpitaṃ gautamāya}.
 The name Haryadvata in \textit{pāda} d seem to be a variant on the attested
  Haryatvata and Haryātman {\rm (}the latter is in the list of \textit{vedavyāsa}s in 
  \VISNUP\ 3.3.16--17, see note to 1.62 above{\rm )}.
 }}

  \maintext{rājaśravās tataḥ prāptaḥ somaśuṣmāya vai tataḥ |}%

  \maintext{somaśuṣmāt tataḥ prāptas tṛṇabindus tu bho dvija }||\thinspace1:72\thinspace||%
\translation{Then [21] Rājaśravas received it, then [22] Somaśuṣma. Then from Somaśuṣma [23] Tṛṇabindu received it, O twice-born. \blankfootnote{1.72 The syntax is again slightly odd here. The intention may have been
  \textit{prāpitaṃ rājaśravasā somaśuṣmāya... tatas tṛṇabindunā prāptam}.
 }}

  \maintext{tṛṇabindus tu vṛkṣāya vṛkṣaḥ śaktim abhāṣata |}%

  \maintext{śaktiḥ parāśaraṃ prāha jatukarṇāya vai tataḥ }||\thinspace1:73\thinspace||%
\translation{Tṛṇabindu taught it to [24] Vṛkṣa, Vṛkṣa to [25] Śakti [the father of Parāśara]. Śakti taught it to [26] Parāśara, then [Parāśara] to [27] Jatukarṇa. \blankfootnote{1.73 In other list of \textit{vedavyāsa}s, Tṛṇabindu hands the Purāṇas down to 
  Ṛkṣa, Rūkṣa or Dakṣa {\rm (}see note to 1.62 above{\rm )}. \textit{vṛkṣa} in \textit{pāda} a
  is probably a corrupted form.
 The name Jatukarṇa may be a corrupted form of Jātū- or Jātukarṇa.
 }}

  \maintext{dvaipāyanaṃ tu provāca jatukarṇo maharṣiṇam |}%

  \maintext{romaharṣāya samprāpto dvaipāyanamahāmuniḥ }||\thinspace1:74\thinspace||%
\translation{Jatukarṇa taught it to [28] [Vyāsa] Dvaipāyana, the great sage. Dvaipāyana, the great sage, gave it to [29] Romaharṣa. \blankfootnote{1.74 \textit{Pāda}s ab are a \textit{pathyā} if \textit{pra} in \textit{provāca}
  does not turn the previous syllable long {\rm (}muta cum liquida{\rm )}.
 The syntax of \textit{pāda}s cd echoes that of 1.70ab above.
 }}

  \maintext{romaharṣeṇa provāca putrāyāmitabuddhaye |}%

  \maintext{daśa dve ca sahasrāṇi purāṇaṃ samprakāśitam |}%

  \maintext{mānuṣāṇāṃ hitārthāya kiṃ bhūyaḥ śrotum icchasi }||\thinspace1:75\thinspace||%
\translation{Romaharṣa taught the Purāṇa[s] of 12,000 [verses], now fully revealed, to his son, [30] Amitabuddhi, for the benefit of humankind. What else do you wish to know? \blankfootnote{1.75 Romaharṣa is usually considered to be the same person as Sūta, disciple of Vyāsa Dvaipāyana.
  
  
 
  In \BrahmandaPur\ 3.4.67ab {\rm (}\textit{mayā caitat punaḥ proktaṃ putrāyāmitabuddhaye}, see note to 
  1.62 above{\rm )} Amitabuddhi is clearly the name {\rm (}or epithet{\rm )} of Romaharṣa's son. 
  This suggests that the reading \textit{romaharṣāya} in some of the MSS
  in \textit{pāda} a is a mistake for \textit{romaharṣaś ca}, or similar. MS \msM\ is either transmitting an
  otherwise syntactically problematic reading {\rm (}\textit{romaharṣeṇa}{\rm )} that is more 
  original than that in most other witnesses, or \msM's scribe is trying to correct the text.
  Supposing the former, in this case I accepted \msM's reading.
 
 
  Manuscripts \msCc\ and \msM\ place the \textit{iti} of the colophon at the end of the last \textit{śloka}, before
  the \textit{daṇḍa}s, thus: \textit{icchasīti}\thinspace ||O|| {\rm (}\msCc{\rm )} and \textit{icchasi iti}\thinspace ||o|| {\rm (}\msM{\rm )}.
  Note also that \msM\ gives the number of \textit{śloka}s in this chapter, 77, which is almost exactly
  the number of verses this critical edition has produced. The scribe of \msM\ struggled 
  with eyeskips in this chapter, therefore it seems unlikely that he himself
  counted the number of verses he had copied and arrived at this very figure.
  Rather, he copied the number from his exemplar.
 }}

\centerline{\maintext{\dbldanda\thinspace iti vṛṣasārasaṃgrahe brahmāṇḍasaṃkhyā nāmādhyāyaḥ prathamaḥ\thinspace\dbldanda}}
\translation{Here ends the first chapter in the \textit{Vṛṣasārasaṃgraha} called the Description of the Brahmāṇḍa[s].}
