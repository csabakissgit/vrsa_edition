
  \chptr{caturtho 'dhyāyaḥ}
\addcontentsline{toc}{section}{Chapter 4}
\fancyhead[CO]{{\footnotesize\textit{Translation of chapter 4}}}%

  \trchptr{ Chapter Four }%

  \subchptr{yameṣu satyam {\rm {\rm (}2{\rm )}}}%

  \trsubchptr{Second Yama-rule: truthfulness}%

  \maintext{anarthayajña uvāca |}%

  \maintext{sadbhāvaḥ satyam ity āhur dṛṣṭapratyayam eva vā |}%

  \maintext{yathābhūtārthakathanaṃ tat satyakathanaṃ smṛtam }||\thinspace4:1\thinspace||%
\translation{Anarthayajña spoke: The state of being real {\rm (}\textit{sad-bhāva}{\rm )} is called truth {\rm (}\textit{sat-ya}{\rm )}. Alternatively, it is also a certainty {\rm (}\textit{pratyaya}{\rm )} that originates in perception {\rm (}\textit{dṛṣṭa}{\rm )}. Relating things in a way that corresponds to reality is called `speaking the truth.' \blankfootnote{4.1 Compare \SDHS\ 11.105:
  
 
  \textit{svānubhūtaṃ svadṛṣṭaṃ ca yaḥ pṛṣṭārthaṃ na gūhati}\thinspace |
 
  \textit{yathābhūtārthakathanam ity etat satyalakṣaṇam}\thinspace ||
  
 
  Translation in \mycitep{SaivaUtopia}{p.~124}:
  `If one does not conceal a matter one is asked about, whether
  it was experienced by oneself or witnessed with one's own eyes,
  but gives an account of things as they happened, this is the definition of `truth.' '
  This verse makes it tempting to emend \textit{satyakathanaṃ} to \textit{satyalakṣaṇaṃ} in \VSS\ 4.1d, but 
  I rather take the \VSS\ verse to introduce two views on truth: one philosophical,
  and one ordinary that relates to the moral aspect of truthfulness.
  Also consider the commentator's remark on the same verse in the \SDHS\ 
  {\rm (}11.105; \mycitep{SaivaUtopia}{p.~124 n.~181 and p.~143}{\rm )}:
  \textit{yathābhūtārthakathane prāṇivadhaprāptāv asatyasya sādhutvāt para pīḍāvinirmuktam eva satyam ity āha}.
  Translation ibid.: `\dots\ he states that [speech is] truth
  only as long as it is devoid of harm of others, for untruth is good when giving an
  account of something as it really happened will result in the slaughter of a living
  creature.'
 }}

  \maintext{ākrośatāḍanādīni yaḥ saheta suduḥsaham |}%

  \maintext{kṣamate yo jitātmā tu sa ca satyam udāhṛtam }||\thinspace4:2\thinspace||%
\translation{He who endures severe abuse and beating etc. and resists [giving\linebreak away secrets], his self being conquered, is said to be [an example of] truth[ful\-ness]. \blankfootnote{4.2 \textit{suduḥsaham} {\rm (}singular{\rm )} in \textit{pāda} b picks up °\textit{ādīni} {\rm (}plural{\rm )} in \textit{pāda} a.
 The \textit{-m} in \textit{satyam} may be a sandhi-bridge and the phrase may refer to a
  masculine subject {\rm (}`a truthful person'{\rm )} thus: \textit{sa ca satya-m-udāhṛtaḥ}.
  Compare with \SDHS\ 11.82 {\rm (}see apparatus{\rm )}, which is a definition of
  forbearance {\rm (}\textit{kṣānti}{\rm )}.
 }}

  \maintext{vadhārtham udyataḥ śastraṃ yadi pṛccheta karhicit |}%

  \maintext{na tatra satyaṃ vaktavyam anṛtaṃ satyam ucyate }||\thinspace4:3\thinspace||%
\translation{If one is being interrogated at any time with a sword lifted to strike him down, in this case the truth is not to be spoken, and a lie can be called truth. \blankfootnote{4.3 Understand \textit{udyataḥ} {\rm (}nom.{\rm )} in an active sense {\rm (}`holding/lifting'{\rm )}.
 }}

  \maintext{vadhārhaḥ puruṣaḥ kaścid vrajet pathi bhayāturaḥ |}%

  \maintext{pṛcchato 'pi na vaktavyaṃ satyaṃ tad vāpi ucyate }||\thinspace4:4\thinspace||%
\translation{A person who is walking on the road and is afraid of being killed should not reply to [people who are potentially dangerous] even if they ask him. This is also called truth[fulness]. \blankfootnote{4.4 `being killed' is not the most obvious translation for
  \textit{vadhārhaḥ} in \textit{pāda} a, but the context suggests that it is not
  a person who `deserves death' that may have been intended.
 }}

  \maintext{na narmayuktam anṛtaṃ hinasti}%

 \nonanustubhindent \maintext{na strīṣu rājan na vivāhakāle |}%

  \maintext{prāṇātyaye sarvadhanāpahāre}%

 \nonanustubhindent \maintext{pañcānṛtaṃ satyam udāharanti }||\thinspace4:5\thinspace||%
\translation{A lie does not hurt when it is connected with joking, with women, O king, at the time of marriage, at the departure from life and when one's entire wealth is about to be taken away. They call these five kinds of lies truths. \blankfootnote{4.5 This \textit{upajāti} verse appears in countless sources, beginning with 
  the \MBH\ {\rm (}see the apparatus{\rm )}. All versions 
  contain a vocative addressing a king, which is out of context in the \VSS, the addressee being Vigatarāga,
  i.e. Viṣṇu disguised as a Brahmin. The redactors did not notice or did not care about this
  small inconsistency. Note the metrical licence that allows the last syllable
  of °\textit{yuktam} to count as long {\rm (}see p.~\pageref{short2long}{\rm )}.
  The reading with \textit{anṛtaṃ}, as opposed to \textit{vacanaṃ}, in \textit{pāda} a, can be found 
  in the apparatus of the \MBH\ critical edition.
 }}

  \maintext{devamānuṣatiryeṣu satyaṃ dharmaḥ paro yataḥ |}%

  \maintext{satyaṃ śreṣṭhaṃ variṣṭhaṃ ca satyaṃ dharmaḥ sanātanaḥ }||\thinspace4:6\thinspace||%
\translation{Since truth is the supreme Dharma in [the world of] gods, humans and animals, truth is the best, the most preferable. Truth is the eternal Dharma. }

  \maintext{satyaṃ sāgaram avyaktaṃ satyam akṣayabhogadam |}%

  \maintext{satyaṃ potaḥ paratrārthaṃ satyaṃ panthāna vistaram }||\thinspace4:7\thinspace||%
\translation{Truth is an unmanifest ocean. Truth yields imperishable pleasures. Truth is a ship bound for the other world. Truth is the wide path. \blankfootnote{4.7 \textit{Pāda} d is slightly problematic because it is difficult to ascertain if some of the
  MSS actually read \textit{panthāna} or \textit{pasthāna} {\rm (}or \textit{yasthāna}{\rm )}. I suspect that \textit{panthāna} 
  is a stem form noun formed {\rm (}metri causa{\rm )} to stand for an irregular nominative of \textit{pathin}.
 }}

  \maintext{satyam iṣṭagatiḥ proktaṃ satyaṃ yajñam anuttamam |}%

  \maintext{satyaṃ tīrthaṃ paraṃ tīrthaṃ satyaṃ dānam anantakam }||\thinspace4:8\thinspace||%
\translation{Truth is said to be the desired path. Truth is the supreme sacrifice. Truth is a pilgrimage place, a supreme pilgrimage place. Truth is endless donation. }

  \maintext{satyaṃ śīlaṃ tapo jñānaṃ satyaṃ śaucaṃ damaḥ śamaḥ |}%

  \maintext{satyaṃ sopānam ūrdhvasya satyaṃ kīrtir yaśaḥ sukham }||\thinspace4:9\thinspace||%
\translation{Truth is virtue, austerity, knowledge. Truth is purity, self-control, and\linebreak tranquillity. Truth is the ladder [that leads] upwards. Truth is fame and glory and happiness. \blankfootnote{4.9 Considering a similar line in the \VARP\ {\rm (}193.36cd, see the apparatus{\rm )}, one 
  wonders if the slightly odd \textit{ūrdhvasya} in \textit{pāda} c is not a corrupt form of 
  \textit{svargasya} somehow.
 }}

  \maintext{aśvamedhasahasraṃ ca satyaṃ ca tulayā dhṛtam |}%

  \maintext{aśvamedhasahasrād dhi satyam eva viśiṣyate }||\thinspace4:10\thinspace||%
\translation{[When] a thousand Aśvamedha sacrifices and truth are measured on a pair of scales, truth indeed surpasses a thousand Aśvamedha sacrifices. }

  \maintext{satyena tapate sūryaḥ satyena pṛthivī sthitā |}%

  \maintext{satyena vāyavo vānti satye toyaṃ ca śītalam }||\thinspace4:11\thinspace||%
\translation{The Sun shines because of truth. The Earth stays in place by truth. The winds blow because of truth. Water has a cooling effect through truth. \blankfootnote{4.11 In general, see sections similar to \VSS\ 4.11--17 on \textit{satya} in \MBH\ 12.192.63--72,
  \RevKhS\ 91.68--70, \VDH\ 55.1ff, \VDHU\ 3.265.1ff, etc.
 Here in \VSS\ 4.11d, and several times below, \textit{satye} is probably 
  to be taken as standing for \textit{satyena}.
 }}

  \maintext{tiṣṭhanti sāgarāḥ satye samayena priyavrataḥ |}%

  \maintext{satye tiṣṭhati govindo balibandhanakāraṇāt }||\thinspace4:12\thinspace||%
\translation{The oceans exist by the truthful encounter with Priyavrata. Govinda abides in truth because He [as Vāmana] stopped [Mahā]Bali [in spite of the fact that this was achieved by a trick]. \blankfootnote{4.12 \textit{Pāda} b, \textit{samayena priyavrataḥ}, probably stand for \textit{samayena priyavratasya} although
  it is unclear to me what exactly \textit{samaya} refers to here.
  
 
 
  For the story of Priyavrata, Manu's son, in which he wanted to turn nights into days by 
  circling aroung Mount Meru in a chariot, and by this produced the seven oceans,
  see, e.g., \BHAGP\ 5.1.30--31: 
  \textit{yāvad avabhāsayati suragirim anuparikrāman bhagavān ādityo
  vasudhātalam ardhenaiva pratapaty ardhenāvacchādayati, tadā hi [priyavrataḥ]
  bhagavadupāsanopacitātipuruṣaprabhāvas tad anabhinandan samajavena
  rathena jyotirmayena rajanīm api dinaṃ kariṣyāmīti saptakṛtvas 
  taraṇim anuparyakrāmad dvitīya iva pataṅgaḥ\thinspace |
  ye vā u ha tadrathacaraṇanemikṛtaparikhātās te sapta sindhava āsan
  yata eva kṛtāḥ sapta bhuvo dvīpāḥ\thinspace |}.
  
 
 
  
  \textit{Pāda}s cd: for a somewhat similar reference to the story of Mahābali, see, e.g., \VAMP\ 65.66:
  
 
  \textit{evaṃ purā cakradhareṇa viṣṇunā} 
 
  \textit{baddho balir vāmanarūpadhāriṇā}\thinspace |
 
  \textit{śakrapriyārthaṃ surakāryasiddhaye} 
 
  \textit{hitāya viprarṣabhagodvijānām}\thinspace || 
 }}

  \maintext{agnir dahati satyena satyena śaśinaś caraḥ |}%

  \maintext{satyena vindhyās tiṣṭhanti vardhamāno na vardhate }||\thinspace4:13\thinspace||%
\translation{Fire burns according to truth. The Moon's course is [governed] by truth. It is because of truth that the Vindhya mountain stands in place and that although it was growing, it is not growing [anymore]. \blankfootnote{4.13 \textit{Pāda} a might as well be a reference to a story mentioned in \MANU\ 8.116:
  
 
  \textit{vatsasya hy abhiśastasya purā bhrātrā yavīyasā}\thinspace |
 
  \textit{nāgnir dadāha romāpi satyena jagataḥ spaśaḥ}\thinspace ||
  
 
  Olivelle's translation {\rm (}\citeyear{OlivelleManu}, 311{\rm )}: 
  `Long ago when Vatsa was accused by his younger brother, 
  Fire, the world's spy, did not burn a single hair of his because he told the truth.'
  Olivelle's note on this verse {\rm (}ibid. 311{\rm )} reads:
  `Vatsa was accused by his brother of being the son of a Śūdra woman and thus not 
  a pure Brahmin. Vatsa went through fire to prove his pedigree. See \textit{Pañcaviṃśa Brāhmaṇa}
  14.6.6.'
 
  Since \textit{śaśi} {\rm (}instead of \textit{śaśin}{\rm )} is a possible stem in this text, 
  \textit{śaśir ācaraḥ} {\rm (}\msNa\msNb\msNc{\rm )} in \textit{pāda} b could be acceptable here,
  perhaps standing metri causa for the compound \textit{śaśicaraḥ}.
  Nevertheless, I have chosen to conjecture \textit{śaśinaś caraḥ}, now preferring it
  to my previous conjecture, \textit{śaśinā caraḥ}.
  Other possibilities, suggested by Judit Törzsök and other colleagues, include \textit{śaśibhāskaraḥ}, 
  \textit{śaśigocaraḥ}, \textit{śiśiro 'caraḥ}, and \textit{śiśirāmbhasaḥ}. Similar passages quoted in the apparatus
  suggest that the Moon vaxes, or shines, by truth {\rm (}\textit{satyena vardhate}/\textit{rājate}{\rm )}.
  Compare also a passage in the \MBH\ {\rm (}quoted in the apparatus{\rm )} that 
  compares Hariścandra, renowned for his truthfulness, to the Moon, 
  using the verb \textit{carati}. These passages seem to support a reading close to
  my conjecture.
 
 
 
  While it is not clear if \textit{pāda}s ab refer to specific legends or not,
  \textit{pāda}s cd hint at the story of Agastya and the Vindhya mountain {\rm (}as pointed out to me
  by Judit Törzsök{\rm )}:
  Vindhya became jealous of the Sun's revolving around Mount Meru, and when the Sun 
  refused him the same favour, he decided to grow higher and obstruct the Sun's movement.
  As a solution to this situation, Agastya asked Vidhya to bend down to make 
  it easier for him to reach the south and to remain thus until he retured. 
  Vindhya agreed to do what Agastya asked him but Agastya never returned. 
  See \MBH\ 3.102.1--14 {\rm (}see the word \textit{samaya} in verse 13 in this passage, and compare it to \VSS\ 4.12b{\rm )}:
  
 
  \textit{yudhiṣṭhira uvāca}\thinspace |
 
  \textit{kimarthaṃ sahasā vindhyaḥ pravṛddhaḥ krodhamūrchitaḥ}\thinspace |
 
  \textit{etad icchāmy ahaṃ śrotuṃ vistareṇa mahāmune}\thinspace ||
 
  \textit{lomaśa uvāca}\thinspace |
 
  \textit{adrirājaṃ mahāśailaṃ meruṃ kanakaparvatam}\thinspace |
 
  \textit{udayāstamaye bhānuḥ pradakṣiṇam avartata}\thinspace ||
 
  \textit{taṃ tu dṛṣṭvā tathā vindhyaḥ śailaḥ sūryam athābravīt}\thinspace |
 
  \textit{yathā hi merur bhavatā nityaśaḥ parigamyate}\thinspace |
 
  \textit{pradakṣiṇaṃ ca kriyate mām evaṃ kuru bhāskara}\thinspace ||
 
  \textit{evam uktas tataḥ sūryaḥ śailendraṃ pratyabhāṣata}\thinspace |
 
  \textit{nāham ātmecchayā śaila karomy enaṃ pradakṣiṇam}\thinspace |
 
  \textit{eṣa mārgaḥ pradiṣṭo me yenedaṃ nirmitaṃ jagat}\thinspace ||
 
  \textit{evam uktas tataḥ krodhāt pravṛddhaḥ sahasācalaḥ}\thinspace |
 
  \textit{sūryācandramasor mārgaṃ roddhum icchan paraṃtapa}\thinspace || 5\thinspace ||
 
  \textit{tato devāḥ sahitāḥ sarva eva; sendrāḥ samāgamya mahādrirājam}\thinspace |
 
  \textit{nivārayām āsur upāyatas taṃ; na ca sma teṣāṃ vacanaṃ cakāra}\thinspace ||
 
  \textit{athābhijagmur munim āśramasthaṃ; tapasvinaṃ dharmabhṛtāṃ variṣṭham}\thinspace |
 
  \textit{agastyam atyadbhutavīryadīptaṃ; taṃ cārtham ūcuḥ sahitāḥ surās te}\thinspace ||
 
  \textit{devā ūcuḥ}\thinspace |
 
  \textit{sūryācandramasor mārgaṃ nakṣatrāṇāṃ gatiṃ tathā}\thinspace |
 
  \textit{śailarājo vṛṇoty eṣa vindhyaḥ krodhavaśānugaḥ}\thinspace ||
 
  \textit{taṃ nivārayituṃ śakto nānyaḥ kaś cid dvijottama}\thinspace |
 
  \textit{ṛte tvāṃ hi mahābhāga tasmād enaṃ nivāraya}\thinspace ||
 
  \textit{lomaśa uvāca}\thinspace |
 
  \textit{tac chrutvā vacanaṃ vipraḥ surāṇāṃ śailam abhyagāt}\thinspace |
 
  \textit{so 'bhigamyābravīd vindhyaṃ sadāraḥ samupasthitaḥ}\thinspace || 10\thinspace ||
 
  \textit{mārgam icchāmy ahaṃ dattaṃ bhavatā parvatottama}\thinspace |
 
  \textit{dakṣiṇām abhigantāsmi diśaṃ kāryeṇa kena cit}\thinspace ||
 
  \textit{yāvadāgamanaṃ mahyaṃ tāvat tvaṃ pratipālaya}\thinspace |
 
  \textit{nivṛtte mayi śailendra tato vardhasva kāmataḥ}\thinspace ||
 
  \textit{evaṃ sa samayaṃ kṛtvā vindhyenāmitrakarśana}\thinspace |
 
  \textit{adyāpi dakṣiṇād deśād vāruṇir na nivartate}\thinspace ||
 
  \textit{etat te sarvam ākhyātaṃ yathā vindhyo na vardhate}\thinspace |
 
  \textit{agastyasya prabhāvena yan māṃ tvaṃ paripṛcchasi}\thinspace || 14\thinspace ||
 }}

  \maintext{lokālokaḥ sthitaḥ satye meruḥ satye pratiṣṭhitaḥ |}%

  \maintext{vedās tiṣṭhanti satyeṣu dharmaḥ satye pratiṣṭhati }||\thinspace4:14\thinspace||%
\translation{The [mythical] Lokāloka mountains are located in truth. Mount Meru stands by truth. The Vedas abide in truth. Dharma is rooted in truth. }

  \maintext{satyaṃ gauḥ kṣarate kṣīraṃ satyaṃ kṣīre ghṛtaṃ sthitam |}%

  \maintext{satye jīvaḥ sthito dehe satyaṃ jīvaḥ sanātanaḥ }||\thinspace4:15\thinspace||%
\translation{The milk a cow yields is truth. Ghee in milk is present as truth. The soul dwells in the body by truth. The eternal soul is truth. \blankfootnote{4.15 \textit{satye} {\rm (}for \textit{satyena}?{\rm )} in \textit{pāda} c may also stand for \textit{satyaṃ}: `The soul dwells in the body as truth.'
 }}

  \maintext{satyam ekena samprāpto dharmasādhananiścayaḥ |}%

  \maintext{rāmarāghavavīryeṇa satyam ekaṃ surakṣitam }||\thinspace4:16\thinspace||%
\translation{If truth is obtained by somebody {\rm (}\textit{ekena}{\rm )}, he/she will be one for whom Dharma is surely accomplished. By the heroism of Rāma Rāghava, the only truth was well-guarded. \blankfootnote{4.16 Or: `If truth alone {\rm (}\textit{ekena}{\rm )} is obtained, Dharma is surely accomplished.'
 }}

  \maintext{evaṃ satyavidhānasya kīrtitaṃ tava suvrata |}%

  \maintext{sarvalokahitārthāya kim anyac chrotum icchasi }||\thinspace4:17\thinspace||%
\translation{Thus have [I] taught the rules of truth to you, O virtuous one, to favour the whole world. What else do you wish to hear? }

  \subchptr{yameṣv asteyam {\rm {\rm (}3{\rm )}}}%

  \trsubchptr{Third Yama-rule: refraining from stealing}%

  \maintext{vigatarāga uvāca |}%

  \maintext{na hi tṛptiṃ vijānāmi śrutvā dharmaṃ tavāpy aham |}%

  \maintext{upariṣṭād ato bhūyaḥ kathayasva tapodhana }||\thinspace4:18\thinspace||%
\translation{Vigatarāga spoke: I can't have enough of learning about [this teaching of] your[s on] Dharma. Teach me further than this, O great ascetic. \blankfootnote{4.18 It is not inconceivable that \textit{tava} is meant to carry the sense of the ablative
  {\rm (}`I can't have enough of learning about Dharma from you'{\rm )}.
 }}

  \maintext{anarthayajña uvāca |}%

  \maintext{steyaṃ śṛṇv atha viprendra pañcadhā parikīrtitam |}%

  \maintext{adattādānam ādau tu utkocaṃ ca tataḥ param |}%

  \maintext{prasthavyājas tulāvyājaḥ prasahyasteya pañcamam }||\thinspace4:19\thinspace||%
\translation{Anarthayajña spoke: Now listen to [my teaching about] stealing, O great Brahmin, which is taught to be of five kinds. Firstly, [listen to] theft, then bribery, cheating with weights, cheating with scales, and the fifth kind, robbery. \blankfootnote{4.19 `Theft' {\rm (}\textit{adattādāna}{\rm )}: literally `taking what has not been given.'
 Note the stem form °\textit{steya} in \textit{pāda} f.
 }}

  \maintext{dhṛṣṭaduṣṭaprabhāvena paradravyāpakarṣaṇam |}%

  \maintext{vāryamāṇo 'pi durbuddhir adattādānam ucyate }||\thinspace4:20\thinspace||%
\translation{When somebody's wealth is taken away by an impudent and wicked person, it is called theft, even if that fool is prevented [from committing the\linebreak crime]. \blankfootnote{4.20 My impression is that \textit{prabhāva} in \textit{pāda} a stands for \textit{bhāva}, \textit{duṣṭabhāva} {\rm (}`vicious'{\rm )}
  being a common expression.
 The implications of \textit{vāryamāṇo} in \textit{pāda} c are unclear to me, therefore
  my translation is tentative. One could consider emending to \textit{vāryamāṇāpi},
  possibly suggesting that `it is a wicked thought {\rm (}\textit{durbuddhi}{\rm )} even if suppressed {\rm (}\textit{vāryamāṇa}{\rm )}.'
 }}

  \maintext{utkocaṃ śṛṇu viprendra dharmasaṃkarakārakam |}%

  \maintext{mūlyaṃ kāryavināśārtham utkocaḥ parigṛhyate |}%

  \maintext{tena cāsau vijānīyād dravyalobhabalāt kṛtam }||\thinspace4:21\thinspace||%
\translation{O great Brahmin, listen to bribery, which causes confusion in Dharma. A sum of money taken in order to dismiss a lawsuit is a bribe. Therefore this [also] should be considered as such [i.e.\ as stealing because] it is committed out of greed for material goods. \blankfootnote{4.21 Note that \textit{mūlyaṃ} in \textit{pāda} c is a conjecture for \textit{mūla}. It is partly based on 
  a relevant passage in the \Mitaksara\ {\rm (}ad \YajnS\ 2.176cd{\rm )}:
  \textit{paṇyasya krītadravyasya yan mūlyaṃ dattam, bhṛtir vetanaṃ kṛtakarmaṇe dattam}\dots\ 
  \textit{utkocena kāryapratibandhanirāsārtham adhikṛtebhyo dattam}\dots\ 
 Note \textit{asau} in \textit{pāda} e as an accusative form {\rm (}for \textit{amum} or \textit{adaḥ}{\rm )}. It is not unlikely that 
  \textit{tena} is a corruption from \textit{stena}, and the \textit{pāda} may have originally read 
  \textit{stenaṃ taṃ ca vijānīyād} {\rm (}`he should be known as a thief'{\rm )}, or similar {\rm (}cf. 4.22c below{\rm )}. 
  \msM\ {\rm (}f. 7r{\rm )} reads \textit{tena steya vijānīyād} here.
 }}

  \maintext{prasthavyāja-upāyena kuṭumbaṃ trātum icchati |}%

  \maintext{taṃ ca stenaṃ vijānīyāt paradravyāpahārakam }||\thinspace4:22\thinspace||%
\translation{[Even if] somebody wants to protect a family by the method of cheating with weights, that person should be considered a thief, because he takes away other people's wealth. }

  \maintext{tulāvyāja-upāyena parasvārthaṃ hared yadi |}%

  \maintext{cauralakṣaṇakāś cānye kūṭakāpaṭikā narāḥ }||\thinspace4:23\thinspace||%
\translation{If somebody takes away somebody else's belongings by the method of cheating with scales, that person is another kind of a deceitful swindler {\rm (}\textit{kūṭa-kāpaṭika}{\rm )} having the characteristics of thieves. \blankfootnote{4.23 I take \textit{anye} in \textit{pāda} c rather liberally, and as connected to
  \textit{pāda}s ab, because I suspect that this verse introduces one single
  category, albeit using clumsy syntax.
 }}

  \maintext{durbalārjavabāleṣu cchadmanā vā balena vā |}%

  \maintext{apahṛtya dhanaṃ mūḍhaḥ sa cauraś cora ucyate }||\thinspace4:24\thinspace||%
\translation{If someone, by deceit or by force, snatches away the wealth of weak and honest people and simpletons, that morally corrupt usurper is [simply] a thief. \blankfootnote{4.24 It is possible that \textit{pāda} d read differently originally, 
  e.g., \textit{sa coraś cora ucyate}, meaning `that thief is [rightly] called a thief'.
 }}

  \maintext{nāsti steyasamaṃ pāpaṃ nāsty adharmaś ca tatsamaḥ |}%

  \maintext{nāsti stenasamākīrtir nāsti stenasamo 'nayaḥ }||\thinspace4:25\thinspace||%
\translation{There is no sin equal to stealing. There is no crime {\rm (}\textit{adharma}{\rm )} equal to it. There is no ill-fame comparable to that of being a thief. There is no bad-conduct comparable to being a thief. }

  \maintext{nāsti steyasamāvidyā nāsti stenasamaḥ khalaḥ |}%

  \maintext{nāsti stenasama ajño nāsti stenasamo 'lasaḥ }||\thinspace4:26\thinspace||%
\translation{There is no greater ignorance than stealing. There are no bigger rouges than thieves. There is nobody as ignorant as a thief. There is no lazy person that is comparable to a thief. \blankfootnote{4.26 Note the peculiar sandhi in \textit{pāda} c {\rm (}°\textit{sama ajño}{\rm )}, which still leaves the \textit{pāda} 
  a \textit{sa-vipulā}.
 }}

  \maintext{nāsti stenasamo dveṣyo nāsti stenasamo 'priyaḥ |}%

  \maintext{nāsti steyasamaṃ duḥkhaṃ nāsti steyasamo 'yaśaḥ }||\thinspace4:27\thinspace||%
\translation{There is nobody as detestable as a thief. There is nobody disliked as much as a thief. There is no greater suffering than stealing. There is no greater disgrace than theft. \blankfootnote{4.27 Note how \textit{stena} and \textit{steya} are used interchangeably {\rm (}or chaotically{\rm )}
  in the above passages in the MSS to denote both `thief' and 'theft/stealing'.
  The scribe of \msNc\ ends up writing \textit{stenya} in 4.27e.
 }}

  \maintext{pracchanno hriyate 'rtham anyapuruṣaḥ pratyakṣam anyo haret}%

 \nonanustubhindent \maintext{nikṣepād dhanahāriṇo 'nya{-}m{-}adhamo vyājena cānyo haret |}%

  \maintext{anye lekhyavikalpanāhṛtadhanā {\rm †}anyo hṛtād vai hṛtā{\rm †}}%

 \nonanustubhindent \maintext{anyaḥ krītadhano 'paro dhayahṛta ete jaghanyāḥ smṛtāḥ }||\thinspace4:28\thinspace||%
\translation{Some [thieves] take away [other people's] wealth in disguise, some in broad daylight. Other wicked people take money from deposits, and some people steal through fraud. Some gather wealth by forging documents, others steal from stolen money[?]. Some people's wealth is from purchased [children?] {\rm (}\textit{krīta}{\rm )}. Others take away others' inheritance[?]. These are considered the vilest. \blankfootnote{4.28 Metre \textit{śārdūlavikrīḍita}. It appears that \textit{hriyate} in \textit{pāda} a is to be taken as an active verb {\rm (}\textit{harate}{\rm )}.
  Note also how \msCb\ and \msNc\ read the same here against the other witnesses.
 Take °\textit{hāriṇo} in \textit{pāda} b as singular and \textit{m} in \textit{'nya-m-adhamo} as a sandhi-bridge.
  Alternatively, read as plural: °\textit{hāriṇo 'nya adhamo}\dots\ 
 The second half of \textit{pāda} c is difficult to reconstruct.
 The translation of \textit{pāda} d is mostly guesswork. Tentatively, I take \textit{krīta} as \textit{krītaka} {\rm (}`a purchased son', see
  \MANU\ 9.174{\rm )}. \textit{dhayahṛta} makes little sense to me. Florinda De Simini suggested that
  \textit{dhaya} might stand for \textit{daya}, which in turn may stand for \textit{dāya} {\rm (}`inheritance'{\rm )} metri causa.
  Lacking any better solution, I supplied these in my translation, marked with question marks.
  Note also the metrical licence that the last syllable of \textit{dhayahṛta} counts as long.
 }}

  \maintext{stenatulya na mūḍham asti puruṣo dharmārthahīno 'dhamaḥ}%

 \nonanustubhindent \maintext{yāvaj jīvati śaṅkayā narapateḥ saṃtrasyamāno raṭan |}%

  \maintext{prāptaḥśāsana tīvrasahyaviṣamaṃ prāpnoti karmeritaḥ}%

 \nonanustubhindent \maintext{kālena mriyate sa yāti nirayam ākrandamāno bhṛśam }||\thinspace4:29\thinspace||%
\translation{There is no bigger idiot than a thief, who is a wicked person without Dharma and financial gain {\rm (}\textit{artha}{\rm )}. As long as he lives, he trembles in fear of the king, wailing. Having received his punishment, he gets into severe and [in]tolerable difficulties, propelled by [his] karma. When his time comes, he dies and goes to hell, weeping vehemently. \blankfootnote{4.29 For some time I was wondering if one should accept \Ed's reading \textit{stenastulya na mūḍham asti} 
  as a metri causa version of \textit{stenatulyo na mūḍho 'sti}; see a similar case of a nominative ending
  inside of compound in \textit{pāda} c below. One major concern remained:
  the accepted reading would be of an edition that rarely emerges as 
  the sole transmitter of the best reading. Another possible solution could be 
  to emend to \textit{stenaṃtulya}\dots, meaning `there is no bigger foolishness than theft',
  but then the second part of \textit{pāda} a is difficult to connect. In the end,
  I decided to go for the most widely attested reading {\rm (}\textit{stenatulya}{\rm )},
  which is unmetrical.
 
  
  
 Understand \textit{prāptaḥśāsana tīvrasahyaviṣamaṃ} in \textit{pāda} c as \textit{prāptaśāsanas tīvram asahyaṃ ca viṣamaṃ prāpnoti}.
  Alternatively, understand \textit{tīvrasahya}° as \textit{duḥsahya}°.
  The actual reading of \msCa, \textit{prāptaś}, lost in the process of normalization and standing
  in contrast with that of all other MSS that read \textit{prāptaḥ}, may suggest
  a doubling of the \textit{ś} of \textit{śāsana} metri causa.
  More likely is that a licence of having a nominative ending inside of a compound
  is applied here, as may have been the case above in \textit{pāda} a.
 }}

  \maintext{nītvā durgatikoṭikalpa nirayāt tiryatvam āyānti te}%

 \nonanustubhindent \maintext{tiryatve ca tathaivam ekaśatikaṃ prabhramya varṣārbudam |}%

  \maintext{mānuṣyaṃ tad avāpnuvanti vipule dāridryarogākulaṃ}%

 \nonanustubhindent \maintext{tasmād durgatihetu karma sakalaṃ tyaktvā śivaṃ cāśrayet }||\thinspace4:30\thinspace||%
\translation{Having spent ten million \ae ons of suffering, they emerge from hell to the state of animal existence. Similarly, they roam about in animal existence for a hundred and one times ten million years. Then they reach the status of human existence on earth which is full of poverty and disease. Then abandoning all one's karmas, the causes of suffering, one seeks refuge in Śiva. \blankfootnote{4.30 Note the stem form °\textit{kalpa} for °\textit{kalpaṃ} metri causa in \textit{pāda} a.
 In \textit{pāda} c, \textit{tathaivam}, or \textit{tathaikam}, and \textit{ekaśatikaṃ} are suspect.
 I understand \textit{vipule} as \textit{vipulāyāṃ}, \textit{vipulā} appearing in \Amara\ 2.1.7 as a synonym of
  \textit{dhātrī}, `earth.' It is difficult to interpret it otherwise.
  This is still problematic because both human and
  animal existence takes place on earth, thus, if \textit{tiryatva} {\rm (}i.e. \textit{tiryaktva}{\rm )} 
  indeed means `animal existence,' there is no contrast between \textit{pāda}s b and c as
  regards location. As for \textit{tiryaktva}, see, e.g., \MANU\ 12.40:
  
 
  \textit{devatvaṃ sāttvikā yānti manuṣyatvaṃ ca rājasāḥ}\thinspace |
  \textit{tiryaktvaṃ tāmasā nityam ity eṣā trividhā gatiḥ}\thinspace ||
  
 
  It is not unlikely that the original form of \textit{dāridryarogākulam} was \textit{dāridryarogākule},
  picking up \textit{vipule}.
 Note the switch from plural to singular in \textit{pāda} d {\rm (}\textit{āśrayet}{\rm )}.
 }}

  \subchptr{yameṣv ānṛśaṃsyam {\rm {\rm (}4{\rm )}}}%

  \trsubchptr{Fourth Yama-rule: absence of hostility}%

  \maintext{aṣṭamūrtiśivadveṣṭā pitur mātuś ca yo dviṣet |}%

  \maintext{gavāṃ vā atither dveṣṭā nṛśaṃsāḥ pañca eva te }||\thinspace4:31\thinspace||%
\translation{The one who is hostile towards the eight-formed Śiva, he who hurts his mother or father, he who is hostile towards cows or guests: these are the five types of hostile people. \blankfootnote{4.31 Note \textit{pitur} and \textit{mātur} used as accusative forms in \textit{pāda} b, or rather,
  understand \textit{pitur mātuś ca yo dveṣṭā}, i.e. \textit{dviṣet} is
  metri causa for \textit{dveṣṭā}.
 }}

  \maintext{aṣṭamūrtiḥ śivaḥ sākṣāt pañcavyomasamanvitaḥ |}%

  \maintext{sūryaḥ somaś ca dīkṣaś ca dūṣakaḥ sa nṛśaṃsakaḥ }||\thinspace4:32\thinspace||%
\translation{Śiva, when manifest {\rm (}\textit{sākṣāt}{\rm )}, has eight form, possessing the five elements {\rm (}\textit{vyoman}{\rm )}, and the Sun, the Moon, and the sacrificer. [He who] disgraces [any of these] is a hostile person. \blankfootnote{4.32 Törzsök has suggested emending \textit{sa nṛśaṃsakaḥ} in \textit{pāda} d to \textit{tannṛṃśakaḥ}. I don't think that it is
  inevitably necessary. I think that \textit{pāda}s a-c form a list that is meant to be in the genitive, understanding
  \dots\ \textit{ity eteṣāṃ dūṣakaḥ sa nṛśaṃsakaḥ} or similar. This is clumsy but in a way that is
  more than possible within the style of this text.
 
  I have not been able find any other attestation of \textit{vyoman} meaning the five elements. Perhaps it is meant
  to mean \textit{vyomādi} {\rm (}`the atmosphere/sky and the other four elements'{\rm )}. 
  
  For Śiva of eight forms, see, e.g., \textit{Śakuntalā} 1.1:
  
 
  [1] \textit{yā sṛṣṭiḥ sraṣṭur ādyā vahati} [2] \textit{vidhihutaṃ yā havir} [3] \textit{yā ca hotrī}
 
  [4, 5] \textit{ye dve kālaṃ vidhattaḥ} [6] \textit{śruti-viṣaya-guṇā yā sthitā vyāpya viśvam}\thinspace |
 
  [7] \textit{yām āhuḥ sarva-bīja-prakṛtir iti yayā prāṇinaḥ prāṇavantaḥ} [8]
 
  \textit{pratyakṣābhiḥ prapannas tanubhir avatu vas tābhir aṣṭābhir īśaḥ}\thinspace ||
  
 
  Here the eight \textit{mūrti}s, or rather, \textit{tanu}s, are: 
  [1] \textit{jala}, [2] \textit{agni}, [3] \textit{hotrī} {\rm (}`the form that sacrifices'{\rm )}, [4 + 5] \textit{sūrya} + \textit{candra},
  [6] \textit{ākāśa}, [7] \textit{bhūmi}, [8] \textit{vāyu}.
 
  For a similar interpretation of \textit{aṣṭamūrti}, see, e.g., 
  \Isanasiva\ 2.29.34 {\rm (}\textit{mantrapāda}; note \textit{yajamāna} for our \textit{dīkṣa}{\rm )}:
  
 
  \textit{kṣmā-vahni-yajamānārka-jala-vāyv-indu-puṣkaraiḥ}\thinspace |
 
  \textit{aṣṭābhir mūrtibhiḥ śambhor dvitīyāvaraṇaṃ smṛtam}\thinspace ||
  
 
  {\rm (}For \textit{puṣkara} as `sky, atmosphere', see, e.g., \Amara\ 1.2.167:
  \textit{dyodivau dve striyām abhraṃ vyoma puṣkaram ambaram}.{\rm )}
 
  A closely related Aṣṭamūrti-hymn appears in \Nisvmukh\ 1.30--41 {\rm (}I owe thanks to Nirajan Kafle
  for drawing my attention to this{\rm )}; see \mycitep{KafleNisvasaBook}{62, 63, 116, 119}. 
  Kafle notes that this hymn is closely parallel to some passages in the \textit{Prayogamañjarī} {\rm (}1.19--26{\rm )},
  the \textit{Tantrasamuccaya} {\rm (}1.16--23{\rm )}, and the \textit{Īśānaśivagurudevapaddhati} {\rm (}\textit{kriyāpāda} 26.56--63{\rm )}. 
  See also \TAKI\ s.v. \textit{aṣṭamūrti}.
 }}

  \maintext{pitākāśasamo jñeyo janmotpattikaraḥ pitā |}%

  \maintext{pitṛdaivata{\rm †}m ādiś cam ānṛśaṃsa tamanvitaḥ{\rm †} }||\thinspace4:33\thinspace||%
\translation{The father is to be considered similar to the [element] sky, he is the cause of one's birth. One should not be hostile to a father, god\dots[?]. \blankfootnote{4.33 It is difficult to restore \textit{pāda}s cd, although the general meaning of this line is
  predictable. Some questions remain. Is \textit{āditya} a good reading or is \textit{mātṛ} hidden in
  \textit{daivata-mādiśca}? Is \textit{ānṛśaṃsa} right or was it \textit{nṛśaṃsa} that was meant by the author of this line?
  Does \textit{tamanvitaḥ} {\rm (}or \textit{tamānvitaḥ}{\rm )} has anything to do with \textit{tamas} {\rm (}`darkness'{\rm )}?
 }}

  \maintext{pṛthvyā gurutarī mātā ko na vandeta mātaram |}%

  \maintext{yajñadānatapovedās tena sarvaṃ kṛtaṃ bhavet }||\thinspace4:34\thinspace||%
\translation{The mother is more venerable than the earth. Who would not praise a mother? By that [praise], sacrifices, donations, austerities and [the study of] the Vedas, all will be completed. }

  \maintext{gāvaḥ pavitraṃ maṅgalyaṃ devatānāṃ ca devatāḥ |}%

  \maintext{sarvadevamayā gāvas tasmād eva na hiṃsayet }||\thinspace4:35\thinspace||%
\translation{Cows are an auspicious blessing, they are the gods of the gods. Cows contain in themselves all the gods. That is exactly why one should not hurt them. }

  \maintext{jātamātrasya lokasya gāvas trātā na saṃśayaḥ |}%

  \maintext{ghṛtaṃ kṣīraṃ dadhi mūtraṃ śakṛtkarṣaṇam eva ca }||\thinspace4:36\thinspace||%
\translation{Cows are the protectors of the world as if the world were their new-born [calf], there is no doubt about it. Collecting [the five products of the cow, the \textit{pañcagavya},] ghee, milk, curd, and [the cow's] urine and dung [is auspicious]. \blankfootnote{4.36 Note the number confusion in the phrase \textit{gāvas trātā}, for \textit{gāvas trātāras}. Alternatively,
  this line might try to echo \Harivamsa\ 45.30ab: 
  \textit{trātavyāḥ prathamaṃ gāvas trātās trāyanti tā dvijān} 
  {\rm (}`First the cows should be protected. When protected, they protect the Brahmins'{\rm )}.
 \textit{Pāda} c is a \textit{sa-viplulā}. 
  The use of \textit{karsaṇa} in \textit{pāda} d, most probably in the sense of `collecting,' is slightly odd.
 }}

  \maintext{pañcāmṛtaṃ pañcapavitrapūtaṃ}%

 \nonanustubhindent \maintext{ye pañcagavyaṃ puruṣāḥ pibanti |}%

  \maintext{te vājimedhasya phalaṃ labhanti}%

 \nonanustubhindent \maintext{tad akṣayaṃ svargam avāpnuvanti }||\thinspace4:37\thinspace||%
\translation{People who drink the five products of the cow, the five nectars, purified by the five Pavitras, will obtain the fruits of a horse sacrifice, and then reach the undecaying heavens. \blankfootnote{4.37 The five \textit{pavitra}s can be the five \textit{brahmamantras}, see, e.g., \TAKIII\ s.v. \textit{pavitra} 1.
 }}

  \maintext{gobhir na tulyaṃ dhanam asti kiṃcid}%

 \nonanustubhindent \maintext{duhyanti vāhyanti bahiś caranti |}%

  \maintext{tṛṇāni bhuktvā amṛtaṃ sravanti}%

 \nonanustubhindent \maintext{vipreṣu dattāḥ kulam uddharanti }||\thinspace4:38\thinspace||%
\translation{There is no wealth comparable to a cow. They yield milk, they carry things, they roam under the sky. Feeding on grass, they issue nectar. When given to Brahmins, they deliver the family [from \textit{saṃsāra} or the suffering experienced in hell]. \blankfootnote{4.38 Note that \textit{duhyanti} and \textit{vāhyanti} are supposed to be understood as passive,
  as in the similar verse in \SDHU\ 12.92 {\rm (}see apparatus{\rm )}.
 }}

  \maintext{gavāhnikaṃ yaś ca karoti nityaṃ}%

 \nonanustubhindent \maintext{śuśrūṣaṇaṃ yaḥ kurute gavāṃ tu |}%

  \maintext{aśeṣayajñatapadānapuṇyaṃ}%

 \nonanustubhindent \maintext{labhaty asau tām anṛśaṃsakartā }||\thinspace4:39\thinspace||%
\translation{He who never fails to serve the cow daily [e.g. with a handful of grass], he who tends to the cows' service, he who is kind to her [i.e. to the cow], will obtain the merits of all sacrifices, austerities and donation. \blankfootnote{4.39 Strictly speaking, \textit{pāda} c is unmetrical. The second syllable of \textit{tapa} counts as
  long {\rm (}see Intro \verify{\rm )}.
 Although the accusative with °\textit{kartā} in \textit{pāda} d is still not optimal, my 
  emendation of \textit{tam} to \textit{tām} at least restores the metre and improves 
  upon the meaning of the sentece. Alternatively, as suggested by Törzsök,
  \textit{taṃ} could be understood as \textit{tad}, picking up \textit{puṇyaṃ} in \textit{pāda} c,
  but in this way any reference to cows here is only implied.
 }}

  \maintext{atithiṃ yo 'nugaccheta atithiṃ yo 'numanyate |}%

  \maintext{atithiṃ yo 'nupūjyeta atithiṃ yaḥ praśaṃsate }||\thinspace4:40\thinspace||%
\translation{One who looks after a guest, one who respects a guest, one who worships a guest, one who praises a guest, \blankfootnote{4.40 Note the peculiar active verb forms \textit{anugaccheta} and \textit{anupūjyeta}.
  On this formation, see a remark about \Nisvmul\ 2.8 in \mycitep{NisvasaGoodall}{247}:
  `We have assumed that \textit{pūjyeta} is intended to mean \textit{pūjayet} and is
  perhaps a contraction of \textit{pūjayeta}.'
 }}

  \maintext{atithiṃ yo na pīḍyeta atithiṃ yo na duṣyati |}%

  \maintext{atithipriyakartā yaḥ atitheḥ paricārakaḥ |}%

  \maintext{atitheḥ kṛtasaṃtoṣas tasya puṇyam anantakam }||\thinspace4:41\thinspace||%
\translation{one who does not harm a guest, one who does not commit a fault towards a guest, one who keeps the guest happy, one who attends to the needs of a guest, one who makes a guest satisfied: his merits are endless. \blankfootnote{4.41 On the form \textit{pīḍyeta}, see previous note.
 }}

  \maintext{āsanenārghapātreṇa pādaśaucajalena ca |}%

  \maintext{annavastrapradānair vā sarvaṃ vāpi nivedayet }||\thinspace4:42\thinspace||%
\translation{He should offer [the guest] a seat, a vessel with water-offering, and water for washing his feet, or gifts of food and clothes, or all [of these]. \blankfootnote{4.42 My conjecture in \textit{pāda} a {\rm (}°\textit{pātreṇa} for °\textit{pādyena}{\rm )} was inspired by the fact that 
  \textit{pāda} b seems to awkwardly repeat what °\textit{pādyena} in \textit{pāda} a signifies.
  Other possibilities could include taking into account bathing {\rm (}\textit{snāna}{\rm )} or 
  an unguent {\rm (}\textit{abhyaṅga}{\rm )}.
 }}

  \maintext{putradārātmanā vāpi yo 'tithim anupūjayet |}%

  \maintext{śraddhayā cāvikalpena aklībamānasena ca }||\thinspace4:43\thinspace||%
\translation{He who worships the guest by [offering him] his own son, wife or himself with willingness, without hesitation, and with a brave heart, \blankfootnote{4.43 For the requirement that one could part with his wife or son, or his own life,
  for the benefit of someone else, see \VSS\ 2.38 and the narrative in \VSS\ chapter 12
  which tells about a Brahmin giving away his own wife to a guest;
  these influenced my decision to emend °\textit{ātmano} to °\textit{ātmanā} in \textit{pāda} a.
  Note that in fact \VSS\ 4.44cd below echoes verse 37cd in the above mentioned chapter 12,
  which reads: \textit{dvijarūpadharo dharmaḥ svayam eva ihāgataḥ}.
 }}

  \maintext{na pṛcched gotracaraṇaṃ svādhyāyaṃ deśajanmanī |}%

  \maintext{cintayen manasā bhaktyā dharmaḥ svayam ihāgataḥ }||\thinspace4:44\thinspace||%
\translation{and does not ask [the guests about their] lineage, Vedic affiliation {\rm (}\textit{caraṇa}{\rm )}, studies, country or birth, and imagines mentally, with devotion, that it is Dharma himself who has arrived, }

  \maintext{aśvamedhasahasrāṇi rājasūyaśatāni ca |}%

  \maintext{puṇḍarīkasahasraṃ ca sarvatīrthatapaḥphalam }||\thinspace4:45\thinspace||%
\translation{[will obtain all the fruits of] thousands of Aśvamedha sacrifices and hundreds of Rājasūya sacrifices, a thousand Puṇḍarīka sacrifices and the fruit of [visiting] all the pilgrimage places and [performing] all the austerities; }

  \maintext{atithir yasya tuṣyeta nṛśaṃsamatam utsṛjet |}%

  \maintext{sa tasya sakalaṃ puṇyaṃ prāpnuyān nātra saṃśayaḥ }||\thinspace4:46\thinspace||%
\translation{he whose guest is satisfied [and] he who can abandon the sentiment of cruelty, will obtain all the merits of the above, there is no doubt about it. \blankfootnote{4.46 The demonstrative pronoun \textit{tasya} in \textit{pāda} c may refer to the guest:
  `he will obtain all his [i.e. the guest's] merits,' hinting at some sort of karmic exchange.
  Nevertheless, I think rather that \textit{tasya} points to the merits one can obtain by the rituals listed 
  in the previous verse. This is suggested by passages such as the following:
  
 
  \MBH\ Supp. 13.14.379 ff.:
  \textit{ahany ahani yo dadyāt kapilāṃ dvādaśīḥ samāḥi\thinspace |
  māsi māsi ca satreṇa yo yajeta sadā naraḥ\thinspace || 
  gavāṃ śatasahasraṃ ca yo dadyāj jyeṣṭhapuṣkare\thinspace | 
  na taddharmaphalaṃ tulyam atithir yasya tuṣyati\thinspace ||}.
  
 
  \BRAHMAVP\ 3.44--46:
  \textit{atithiḥ pūjito yena pūjitāḥ sarvadevatāḥ\thinspace |
  atithir yasya saṃtuṣṭas tasya tuṣṭo hariḥ svayam\thinspace ||
  snānena sarvatīrtheṣu sarvadānena yat phalam\thinspace | 
  sarvavratopavāsena sarvayajñeṣu dīkṣayā\thinspace || 
  sarvais tapobhir vividhair nityair naimittikādibhiḥ\thinspace | 
  tad evātithisevāyāḥ kalāṃ nārhanti ṣoḍaśīm\thinspace ||}.
 }}

  \maintext{{\rm †}na gatim atithijñasya{\rm †} gatim āpnoti karhicit |}%

  \maintext{tasmād atithim āyāntam abhigacchet kṛtāñjaliḥ }||\thinspace4:47\thinspace||%
\translation{\dots\ will ever reach the path. Therefore one should go up to the arriving guest with respectfully joined palms. \blankfootnote{4.47 Something has gone wrong with \textit{pāda}s ab and I am unable to reconstruct the
  meaning. The line may have begun with something like \textit{nāgatātithyavajña}°
  {\rm (}`he who despises a guest that has arrived will not\dots'{\rm )}.
 }}

  \maintext{saktuprasthena caikena yajña āsīn mahādbhutaḥ |}%

  \maintext{atithiprāptadānena svaśarīraṃ divaṃ gatam }||\thinspace4:48\thinspace||%
\translation{By one \textit{prastha} [a small unit of weight] of coarsely ground grains given to a guest, an extremely wonderful sacrifice was performed [so to say], and his body [i.e. the protagonist in his mortal form] reached heaven. \blankfootnote{4.48 This verse is a reference to the story related by a mongoose in \MBH\ 14.92--93: 
  A Brahmin who practises the vow of gleaning {\rm (}\textit{uñcha}{\rm )} and his family
  receive a guest. They feed the guest with the last morsels of the little food
  they have. In the end, the guest reveals that he is in fact Dharma {\rm (}14.93.80cd{\rm )} and as 
  a reward the family departs to heaven. The noble act of the poor Brahmin and his family
  is depicted as yielding greater rewards than Yudhiṣṭhira's grandiose horse-sacrifice. 
  {\rm (}See an analysis of this story in \mycite{TakahashiUnca}.{\rm )}
 
  
 We would be forced to accept the reading of \Ed\ in \textit{pāda} d {\rm (}\textit{saśarīro}{\rm )} 
  if the expression were in the masculine {\rm (}\textit{divaṃ gataḥ}{\rm )}. This would make sense
  and it would also echo expressions occuring, e.g., in the \MBH:
  3.164.33cd: \textit{paśya puṇyakṛtāṃ lokān saśarīro divaṃ vraja};
  14.5.10cd: \textit{saṃjīvya kālam iṣṭaṃ ca saśarīro divaṃ gataḥ}.
  It is tempting to emend accordingly, but instead I have retained 
  \textit{svaśarīraṃ divaṃ gatam}, and I interpret it in a general way.
 }}

  \maintext{nakulena purādhītaṃ vistareṇa dvijottama |}%

  \maintext{viditaṃ ca tvayā pūrvaṃ prasthavārttā ca kīrtitā }||\thinspace4:49\thinspace||%
\translation{The mongoose related [this story in the \textit{Mahābhārata}] in the past in detail, O great Brahmin, and you known it already. The story of the \textit{prastha} is well-known. }

  \subchptr{yameṣu damaḥ {\rm {\rm (}5{\rm )}}}%

  \trsubchptr{Fifth Yama-rule: self-restraint}%

  \maintext{dama eva manuṣyāṇāṃ dharmasārasamuccayaḥ |}%

  \maintext{damo dharmo damaḥ svargo damaḥ kīrtir damaḥ sukham }||\thinspace4:50\thinspace||%
\translation{Self-restraint is in itself the collected essence of Dharma for humans. Self-restraint is Dharma, self-restraint is heaven, self-restraint is fame, self-restraint is happiness. }

  \maintext{damo yajño damas tīrthaṃ damaḥ puṇyaṃ damas tapaḥ |}%

  \maintext{damahīna{-}m{-}adharmaś ca damaḥ kāmakulapradaḥ }||\thinspace4:51\thinspace||%
\translation{Self-restraint is sacrifice, self-restraint is a pilgrimage-place, self-restraint is merit, self-restraint is religious austerity. If one has no self-restraint, one is a sinner {\rm (}\textit{adharma}{\rm )}, [while] self-restraint yields a multitude of desired objects. \blankfootnote{4.51 I suspect that the final \textit{m} in \textit{dhamahīnam} in \textit{pāda} c is a hiatus-filler: \textit{dhamahīna-m-adharmaś ca}.
  \textit{kāmakulapradaḥ} in \textit{pāda} d is suspect, and my translation is 
  unsatisfactory. This compound could be interpreted as `fulfilling desires and giving a family' or 
  it may have originally read \textit{sarvakāmapradaḥ} {\rm (}`fulfilling all desires'{\rm )} or
  \textit{kulakāmapradaḥ} {\rm (}`fulfilling the desires of the family'{\rm )}.
  \SDHS\ 4.28b reads \textit{sarvakāmasukhapradam}, which opens up further possibilities.
 }}

  \maintext{nirdamaḥ kari mīnaś ca pataṅgabhramaramṛgāḥ |}%

  \maintext{tvag jihvā ca tathā ghrāṇā cakṣuḥ śravaṇam indriyāḥ }||\thinspace4:52\thinspace||%
\translation{The elephant, the fish, the moth, the bee and the deer are without self-restraint. The senses are the skin, the tongue, the nose, the eye and the ear. \blankfootnote{4.52 Note \textit{kari} for \textit{karī} metri causa, and the end of \textit{pāda} b {\rm (}°\textit{mṛgāḥ}{\rm )}, which 
  should be treated metrically as if it read °\textit{mrigāḥ}.
 }}

  \maintext{durjayendriyam ekaikaṃ sarve prāṇaharāḥ smṛtāḥ |}%

  \maintext{damaṃ yo jayate 'samyag nirdamo nidhanaṃ vrajet }||\thinspace4:53\thinspace||%
\translation{Each of these sense faculties are hard to conquer and all are known to be fatal [if unconquered]. If one masters self-restraint in a less than proper way, one remains unrestrained and will die . \blankfootnote{4.53 The only way to make sense of \textit{pāda}s cd is to supply and \textit{avagraha} before
  \textit{samyag}. Otherwise some text may have dropped out here.
 }}

  \maintext{mṛge śrotravaśān mṛtyuḥ pataṅgāś cakṣuṣor mṛtāḥ |}%

  \maintext{ghrāṇayā bhramaro naṣṭo naṣṭo mīnaś ca jihvayā }||\thinspace4:54\thinspace||%
\translation{In the case of the deer, death comes about because of hearing [when, e.g., hunters use buck grunts]. Moths die because of their eyes [as they are attracted to the light of a lamp]. Bees perish because of their smelling [as they are attracted to smells], fish because of their tongues [when fishermen feed them]. \blankfootnote{4.54 My comments in square brackets in the translation are tentative.
 }}

  \maintext{sparśena ca karī naṣṭo bandhanāvāsaduḥsahaḥ |}%

  \maintext{kiṃ punaḥ pañcabhuktānāṃ mṛtyus tebhyaḥ kim adbhutam }||\thinspace4:55\thinspace||%
\translation{The elephant perishes because of touch, not tolerating to be kept in fetters. How much more true it is for those who enjoy all five [senses]! Why should death come as a surprise for them? \blankfootnote{4.55 \textit{Mātaṅgalīlā} 11.1 may shed some light on elephants dying in captivity:
  \textit{vānyas tatra sukhoṣitā vidhivaśād grāmāvatīrṇā gajā baddhās tīkṣṇakaṭūgravāgbhir atiśugbhīmohabandhādibhiḥ\thinspace |
  udvignāś ca manaḥśarīrajanitair duḥkhair atīvākṣamāḥ prāṇān dhārayituṃ ciraṃ naravaśaṃ prāptāḥ svayūthād atha}\thinspace ||.
  In Edgerton's translation \nocite{EdgertonElephant}{\rm (}1931, 92{\rm )}: 
  `Forest elephants who dwell there happily and by
  the power of fate have been brought to town in bonds, afflicted by harsh, bitter, cruel words,
  by excessive grief, fear, bewilderment, bondage, etc., and by sufferings of mind and body,
  are quite unable for long to sustain life, when from their own herds they have come into
  the control of men.'
 }}

  \maintext{purūravo 'tilobhena atikāmena daṇḍakaḥ |}%

  \maintext{sāgarāś cātidarpeṇa atimānena rāvaṇaḥ }||\thinspace4:56\thinspace||%
\translation{Purūravas [perished] by excessive greed, Daṇḍaka by excessive desire, Sagara's sons by excessive pride, Rāvaṇa by excessive haughtiness, \blankfootnote{4.56 We may treat \textit{purūravo} in \textit{pāda} a as a stem form noun or thematised stem, or imagine that the
  original reading was \textit{purūravā}° with double sandhi:
  \textit{purūravās ati}° $\rightarrow$\ \textit{purūravā ati}° $\rightarrow$\ \textit{purūravāti}°.
 
  \textit{Pāda} a may refer to the following passage in the \MBH\ {\rm (}1.70.16--18, 20ab{\rm )}:
  \textit{purūravās tato vidvān ilāyāṃ samapadyata\thinspace |
  sā vai tasyābhavan mātā pitā ceti hi naḥ śrutam\thinspace ||
  trayodaśa samudrasya dvīpān aśnan purūravāḥ\thinspace |
  amānuṣair vṛtaḥ sattvair mānuṣaḥ san mahāyaśāḥ\thinspace ||
  vipraiḥ sa vigrahaṃ cakre vīryonmattaḥ purūravāḥ\thinspace |
  jahāra ca sa viprāṇāṃ ratnāny utkrośatām api\thinspace ||
  [\dots] 
  tato maharṣibhiḥ kruddhaiḥ śaptaḥ sadyo vyanaśyata\thinspace |}.
  
 
  ``The wise Purūravas was born to Ilā. We heard that Ilā 
  was both his mother and his father. 
  The great Purūravas ruled over thirteen islands of the ocean
  and, though human, he was always surrounded by superhuman beings.
  Intoxicated with his power, Purūravas quarrelled with some Brahmins 
  and robbed them of their wealth even though they were protesting. [...]
  Therefore, cursed by the great Ṛṣis, he perished.''
  See also \BUDDHACARITA\ 11.15 {\rm (}Aiḍa = Purūravas{\rm )}:
 
  \textit{ aiḍaś ca rājā tridivaṃ vigāhya 
  nītvāpi devīṃ vaśam urvaśīṃ tām\thinspace |
  lobhād ṛṣibhyaḥ kanakaṃ jihīrṣur 
  jagāma nāśaṃ viṣayeṣv atṛptaḥ\thinspace ||}.
  
 
 
  For Daṇḍa{\rm (}ka{\rm )}'s story, see \RAMAYANA\ 7.71.31 ff.:
  Daṇḍa meets Arajā, a beautiful girl, in a forest and rapes her. As a consequence, her father, Śukra/Bhārgava,
  destroyes Daṇḍa's kingdom, which thus becomes the desolate Daṇḍaka-forest.
 
  
 
  For two versions of the destruction of
  Sagara's sons, who were chasing the sacrificial horse of their father's Aśvamedha sacrifice,
  and by doing so disturbed Kapila's meditation, and who in turn burnt them to ashes,
  see \MBH\ 3.105.9 ff. and \BRAHMANDAPUR\ 2.52--53.
 
  
 
  As for Rāvaṇa's haughtiness,
  especially the fact that he chose to be invincible by all creatures except humans,
  and its consequences,
  one should recall the story of the \RAMAYANA\ and Rāvaṇa's destruction brought about by Rāma therein.
 }}

  \maintext{atikrodhena saudāsa atipānena yādavāḥ |}%

  \maintext{atitṛṣṇāc ca māndhātā nahuṣo dvijavajñayā }||\thinspace4:57\thinspace||%
\translation{Saudāsa by excessive anger, the Yādavas by excessive drinking, Māndhātṛ by excessive desire, Nahuṣa by contempt for Brahmins, \blankfootnote{4.57 Saudāsa, also known as Kalmāṣapāda, hit Śakti, Vasiṣṭha's son, with a whip because
  the latter did not give way to him, and as a consequence Śakti cursed Saudāsa:
  Saudāsa had to roam the world as a Rākṣasa for twelve years. 
  See \MBH\ 1.166.1 ff.
 
  
 
  As for the end of the Yādavas, see the short \textit{Mausalaparvan} of the \MBH\ {\rm (}canto 16{\rm )}:
  cursed by the sages Viśvāmitra, Kaṇva and Nārada, and seeing menacing omens,
  the Yādavas take to drinking in Prabhāsa and destroy each other.
 
 Most probably, \textit{atitṛṣṇā} in the MSS stand for \textit{atitṛṣṇāt} {\rm (}intending \textit{atitṛṣṇayā}{\rm )}.
  The form \textit{māndhāto} in \msCb\ stands for \textit{māndhātā} {\rm (}nominative of \textit{māndhātṛ}{\rm )}.
  I have corrected it in spite of the fact that the authors' knowledge about his story may
  come from \DIVYAV\ 17, where it sometimes appears to be an a-stem noun {\rm (}\textit{māndhāta}{\rm )}.
  \textit{dvijavajñayā} in \textit{pāda} d stands for \textit{dvijāvajñayā} metri causa.
 
  
 
  Māndhātṛ was born from his father's body who, being excessively thirsty once,
  had drank some decoction prepared for ritual purposes and as a result become pregnant with him.
  Nevertheless, \BUDDHACARITA\ 11.13 suggests that Māndhātṛ himself was still unsatisfied
  with wordly objects even after he had obtained half of Indra's throne:
  \textit{devena vṛṣṭe 'pi hiraṇyavarṣe 
  dvīpān samagrāṃś caturo 'pi jitvā\thinspace | 
  śakrasya cārdhāsanam apy avāpya 
  māndhātur āsīd viṣayeṣv atṛptiḥ\thinspace ||}. 
  In fact, as Monika Zin points out {\rm (}\mycitep{ZinMandhatar}{149}{\rm )},
  Māndhātṛ/Māndhāta's rise and fall is a very popular theme
  in the `Narrative Art of the Amaravati School': 
  `Statistics show that in the Amaravati School the most frequently represented narrative is
  the story of King Māndhātar, which appears 47 times.'
  
  
 
  Nahuṣa was elevated to the position of Indra for a period of time and he also wanted
  to take Śacī, Indra's wife. Indra instructed Śacī to tell Nahuṣa to 
  harness some Ṛsis to a vehicle and use this vehicle to take Śacī. 
  Agastya, one of the Ṛṣis, was insulted even further by Nahuṣa, therefore
  he cursed Nahuṣa, who then fell from the vehicle. See \MBH\ 12.329.35 ff. and
  a verse in the \BUDDHACARITA\ {\rm (}11.14{\rm )} that follows the one about Māndhātṛ:
  
 
  \textit{bhuktvāpi rājyaṃ divi devatānāṃ 
  śatakratau vṛtrabhayāt pranaṣṭe\thinspace |
  darpān maharṣīn api vāhayitvā 
  kāmeṣv atṛpto nahuṣaḥ papāta\thinspace ||}.
 }}

  \maintext{atidānād balir naṣṭa atiśauryeṇa arjunaḥ |}%

  \maintext{atidyūtān nalo rājā nṛgo goharaṇena tu }||\thinspace4:58\thinspace||%
\translation{[Mahā]bali perished by excessive donations, Arjuna by excessive heroism, King Nala by excessive gambling, Nṛga by taking a cow. \blankfootnote{4.58 \textit{Pāda} a is most probably a reference to Mahābali's promises made to Vāmana that caused his own fall. 
  The ultimate cause of Arjuna' death while the Pāṇḍavas were on the way to the underworld 
  was summarised by Yudhiṣṭhira thus {\rm (}\MBH\ 17.2.21ab{\rm )}:
  \textit{ekāhnā nirdaheyaṃ vai śatrūn ity arjuno 'bravīt\thinspace |
  na ca tat kṛtavān eṣa śūramānī tato 'patat}\thinspace ||.
  {\rm (}`Arjuna claimed that he could destroy the enemy in one single day. He failed to do so.
  He was a boaster, that is why he fell.'{\rm )}
 
  
  
  King Nala was an expert in the game of dice but once he lost his kingdom to Puṣkara.
  See, e.g., \MBH\ 3.56.1 ff. 
 
  
 
  As for Nṛga, see \MBH\ 14.93.74: 
  \textit{gopradānasahasrāṇi dvijebhyo 'dān nṛgo nṛpaḥ\thinspace |
  ekāṃ dattvā sa pārakyāṃ narakaṃ samavāptavān\thinspace ||.}
  {\rm (}``King Nṛga had made gifts of thousands of cows for the twice-born.
  By giving away one single cow that belonged to someone else, 
  he fell into hell.''{\rm )} 
 }}

  \maintext{damena hīnaḥ puruṣo dvijendra}%

 \nonanustubhindent \maintext{svargaṃ ca mokṣaṃ ca sukhaṃ ca nāsti |}%

  \maintext{vijñānadharmakulakīrtināśa}%

 \nonanustubhindent \maintext{bhavanti vipra damayā vihīnāḥ }||\thinspace4:59\thinspace||%
\translation{[For] a person who is without self-restraint, O great Brahmin, there is no heaven, liberation or happiness. O Brahmin, people without self-restraint are the destruction of knowledge, Dharma, family and fame. \blankfootnote{4.59 \textit{Pāda} b: \textit{svarga} and \textit{mokṣa} are usually masculine in standard Sanskrit.
 The majority of the witnesses suggest that \textit{pāda} c ends in a stem form noun {\rm (}°\textit{nāśa}{\rm )},
  although a singular masculine nominative {\rm (}as in \Ed{\rm )} may work.
  This \textit{pāda} is unmetrical, or rather it applies the licence of a word-final
  short syllable being counted as potentially long {\rm (}°\textit{dharMA}°{\rm )}. 
 Note how \textit{viprā} in \textit{pāda} d is probably an attempt in some MSS to restore the metre.
  This \textit{pāda} is also unmetrical, or rather the licence of a word-final
  short syllable being counted as potentially long is again applied {\rm (}\textit{viPRA}{\rm )}.
 }}

  \subchptr{yameṣu ghṛṇā {\rm {\rm (}6{\rm )}}}%

  \trsubchptr{Sixth Yama-rule: taboos}%

  \maintext{nirghṛṇo na paratrāsti nirghṛṇo na ihāsti vai |}%

  \maintext{nirghṛṇe na ca dharmo 'sti nirghṛṇe na tapo 'sti vai }||\thinspace4:60\thinspace||%
\translation{A person without taboos does not exists either in this or the other world. In a person without taboos there is no Dharma or religious austerity. \blankfootnote{4.60 The implications of \textit{pāda}s ab are not crystal clear to me. Perhaps:
  such a person has no right for existence in society and has no place in heaven.
 }}

  \maintext{parastrīṣu parārtheṣu parajīvāpakarṣaṇe |}%

  \maintext{paranindāparānneṣu ghṛṇāṃ pañcasu kārayet }||\thinspace4:61\thinspace||%
\translation{These five should be treated as taboo: women who are not depending on oneself, others' wealth, taking away others' lives, hurting others and [consuming] others' food. }

  \maintext{parastrī śṛṇu viprendra ghṛṇīkāryā sadā budhaiḥ |}%

  \maintext{rājñī viprī parivrājā svayoniparayoniṣu }||\thinspace4:62\thinspace||%
\translation{Listen, O great Brahmin, the wise should always treat women who are not dependent on oneself as taboo, [be she] a queen, a Brahmin's wife, a wandering religious mendicant, a relative or of another caste. \blankfootnote{4.62 The translation of \textit{parayoni} in \textit{pāda} d is tentative.
 }}

  \maintext{parārthe śṛṇu bhūyo 'nya anyāyārtha{-}m{-}upārjanam |}%

  \maintext{āḍhaprasthatulāvyājaiḥ parārthaṃ yo 'pakarṣati }||\thinspace4:63\thinspace||%
\translation{Listen further to something else, with regards to others' wealth. [It may include] gaining wealth through unlawful means, when somebody takes away other people's wealth by cheating with weights of one \textit{āḍha[ka]} or a \textit{prastha} and with scales. \blankfootnote{4.63 Although \textit{'nya} in \textit{pāda} a could be interpreted several ways {\rm (}e.g. \textit{anye} for \textit{anyasmin}, 
  or taken to be the first element of a compound: \textit{anya-anyāyārtha-}{\rm )},
  I think that \textit{bhūyo 'nyat} is a fixed expression meaning `something/anything more.' 
  See, e.g., \BHG\ 7.2cd:
  \textit{yaj jñātvā neha bhūyo 'nyaj jñātavyam avaśiṣyate}.
 }}

  \maintext{jīvāpakarṣaṇe vipra ghṛṇīkurvīta paṇḍitaḥ |}%

  \maintext{vanajāvanajā jīvā vilagāś caraṇācarāḥ }||\thinspace4:64\thinspace||%
\translation{O Brahmin, the wise should regard the taking away [of others'] lives as taboo. Wild and domesticated animals, serpents, [in general,] plants and animals [are examples of life forms not to destroy]. \blankfootnote{4.64 In \textit{pāda} d, I take \textit{caraṇācarāḥ} as standing for \textit{carācarāḥ} {\rm (}\textit{cara-acarāḥ}{\rm )} metri causa.
  Alternatively, one may understand it as \textit{caraṇacarāḥ} {\rm (}metri causa{\rm )}, 
  meaning `those who move on their feet,' perhaps as opposed to snakes {\rm (}\textit{bilaga} or \textit{bilaṃga}{\rm )}.
  Neither solution is fully satisfactory. Note that this \textit{pāda} also involves a small correction.
 }}

  \maintext{paranindā ca kā vipra śṛṇu vakṣye samāsataḥ |}%

  \maintext{devānāṃ brāhmaṇānāṃ ca gurumātātithidviṣaḥ }||\thinspace4:65\thinspace||%
\translation{And what is the hurting of others? Listen, O Brahmin, I'll tell you briefly. He who is hostile to the gods, Brahmins, gurus, mothers and guests [hurts others]. \blankfootnote{4.65 Note \textit{mātā} as a stem form in \textit{pāda} d.
 }}

  \maintext{parānneṣu ghṛṇā kāryā abhojyeṣu ca bhojanam |}%

  \maintext{sūtake mṛtake śauṇḍe varṇabhraṣṭakule naṭe }||\thinspace4:66\thinspace||%
\translation{As regards other people's food, eating together with people whose food is not to be accepted {\rm (}\textit{abhojyeṣu}{\rm )} is taboo, [e.g.] after birth or death [in a family], in case of vendors of alcohol, or a family having lost their caste, and in the case of a [member of the] Naṭa [caste of dancers]. \blankfootnote{4.66 One should probably understand \textit{śauṇḍe} in \textit{pāda} c as \textit{śauṇḍike}, `a distiller,' or, alternatively,
  it may be corrupted from \textit{ṣaṇḍhe}, `a eunuch'; see both in \VasDh\ 14.1--3:
  \textit{athāto bhojyābhojyaṃ ca varṇayiṣyāmaḥ\thinspace |
  cikitsaka-mṛgayu-puṃścalī-ḍaṇḍika-stenābhiśastar-ṣaṇḍha-patitānām annam abhojyam\thinspace |
  kadarya-dīkṣita-baddhātura-somavikrayi-takṣa-rajaka-śauṇḍika-sūcaka-vārdhuṣika-carmāvakṛntānām\thinspace ||} etc.
  
 
  Translated in \mycitep{OlivelleDharmasutras}{285} as:
  `Next we will describe food that is fit and food that is
  unfit to be eaten [\dots] The following are unfit
  to be eaten: food given by a physician, a hunter, a harlot, a law
  enforcement agent, a thief, a heinous sinner [...] a
  eunuch, or an outcaste; as also that given by a miser, a man
  consecrated for a sacrifice, a prisoner, a sick person, a man who
  sells Soma, a carpenter, a washerman, a liquor dealer, a spy, an
  usurer, a leather worker\dots'
  
 
  In support of reading \textit{ṣaṇḍhe}, one might consult \MANU\ 3.239:
  
 
  \textit{cāṇḍālaś ca varāhaś ca kukkuṭaḥ śvā tathaiva ca\thinspace |
  rajasvalā ca ṣaṇḍhaś ca nekṣerann aśnato dvijān\thinspace ||.}
  Translated in \mycitep{OlivelleDharmasutras}{120} as:
  `A Cāṇḍāla, a pig, a cock, a dog, a menstruating woman, or a eunuch must not
  look at the Brahmins while they are eating.'
 }}

  \maintext{ete pañcaghṛṇāsu saktapuruṣāḥ svargārthamokṣārthino}%

 \nonanustubhindent \maintext{loke 'nindanam āpnuvanti satataṃ kīrtir yaśo'laṃkṛtāḥ |}%

  \maintext{prajñābodhaśrutiṃ smṛtiṃ ca labhate mānaṃ ca nityaṃ labhed}%

 \nonanustubhindent \maintext{dākṣiṇyaṃ sabhavet sa āyuṣa paraṃ prāpnoti niḥsaṃśayaḥ }||\thinspace4:67\thinspace||%
\translation{Those people who stick to the five kinds of taboo [and thus] seek heaven, wealth and liberation, will reach eternal faultlessness in this world, embellished with fame and glory. [A person like that] will obtain wisdom, intelligence, [knowledge of] the Śruti and Smṛti traditions, and honour forever. Kindness will arise and he will obtain an extra long life, no doubt. \blankfootnote{4.67 Understand \textit{kīrtir-yaśo}° as \textit{kīrtiyaśo}° {\rm (}'r' being an intrusive consonant here metri causa{\rm )}, 
  as in 5.20 below. Alternatively, as suggested by Francesco Sferra, emend to \textit{kīrtiṃ yaśo'laṃkṛtām}.
  My emendation of °\textit{kṛtam} to °\textit{kṛtāḥ} is influenced be 5.20b.
 In \textit{pāda} c, note the muta cum liquida licence that allows °\textit{bodhaśrutiṃ}°
  to scan as - \shortsyllable\ \shortsyllable\ - , the consonant cluster 
  \textit{śr} not turning the previous syllable long.
 \textit{Pāda} d has several problems. I take \textit{sabhavet} as standing for \textit{sambhavet} metri causa,
  and I had to emend \textit{samāyuṣa} to \textit{sa āyuṣa} to make sense of it.
  Understand \textit{āyuṣa} as \textit{āyuḥ} {\rm (}metri causa{\rm )}, otherwise emend to \textit{sa mānuṣya}.
  Also consider correcting \textit{niḥsaṃśayaḥ} to \textit{niḥsaṃśayam}.
 }}

  \subchptr{yameṣu pañcavidho dhanyaḥ {\rm {\rm (}7{\rm )}}}%

  \trsubchptr{Seventh Yama-rule: five methods of virtue}%

  \maintext{caturmaunaṃ catuḥśatruś caturāyatanaṃ tathā |}%

  \maintext{caturdhyānaṃ catuṣpādaṃ pañcadhanyavidhocyate }||\thinspace4:68\thinspace||%
\translation{The four cases of observing silence, [victory over] the four enemies, the four sanctuariess, the four meditations, and the four legged [Dharma] are called the five ways of being virtuous. \blankfootnote{4.68 Understand \textit{pāda} d as \textit{pañcavidho dhanya ucyate}.
 }}

  \maintext{caturmaunasya vakṣyāmi śṛṇuṣvāvahito bhava |}%

  \maintext{pāruṣyapiśunāmithyāsambhinnāni ca varjayet }||\thinspace4:69\thinspace||%
\translation{I shall tell you about the four cases of observing silence. Listen, be attentive. One should avoid violent and slanderous [words], lies, and idle [talk]. \blankfootnote{4.69 Note the genitive with a verb meaning `to tell' in \textit{pāda} a, similarly to 1.37a and \verify.
 Similar teachings on \textit{mauna} in \DHARMP\ 1.31cd--32ab and \DIVYAV\ 186.21 are quoted in the apparatus.
 }}

  \maintext{kāmaḥ krodhaś ca lobhaś ca mohaś caiva caturvidhaḥ | }%

  \maintext{catuḥśatrur nihantavyaḥ so 'rihā vītakalmaṣaḥ }||\thinspace4:70\thinspace||%
\translation{The fourfold enemy [made up of] desire, anger, greed and delusion is to be destroyed. He who destroys [these] enemies will become sinless. \blankfootnote{4.70 Possible direct sources for the idea that \textit{kāma} is an enemy to be defeated or avoided include
  \BUDDHACARITA\ 11.17:
  
 
  \textit{cīrāmbarā mūlaphalāmbubhakṣā 
  jaṭā vahanto 'pi bhujaṃgadīrghāḥ\thinspace | 
  yair nānyakāryā munayo 'pi bhagnāḥ 
  kaḥ kāmasaṃjñān mṛgayeta śatrūn\thinspace ||};
  
 
  see also \BHG\ 3.43:
  
 
  \textit{evaṃ buddheḥ paraṃ buddhvā saṃstabhyātmānam ātmanā\thinspace |
  jahi śatruṃ mahābāho kāmarūpaṃ durāsadam\thinspace ||}.
  As for \textit{arihā} in \textit{pāda} d, the notion that a saint is a `destroyer of the enemies' 
  [that are evil states of mind] {\rm (}\textit{arihanta/arahanta}{\rm )}
  in Jainism, but less so in Buddhism, is discussed in \mycitep{GombrichWhat2013}{57--58}.
 }}

  \maintext{caturāyatanaṃ vipra kathayiṣyāmi tac chṛṇu |}%

  \maintext{karuṇā muditopekṣā maitrī cāyatanaṃ smṛtam }||\thinspace4:71\thinspace||%
\translation{I shall teach you the four sanctuaries. Listen, O Brahmin. Compassion, sympathy in joy, indifference, and benevolence are the four sanctuaries. \blankfootnote{4.71 This verse teaches the four Buddhist \textit{brahmavihāra}s under the label
  \textit{caturāyatana}. Therfore the word \textit{āyatana} seems to be a synonym of \textit{vihāra} here,
  and its use a simple method of appropriating it, turning the list into a Brahmanical one.
 }}

  \maintext{caturdhyānādhunā vakṣye saṃsārārṇavatāraṇam |}%

  \maintext{ātmavidyābhavaḥ sūkṣmaṃ dhyānam uktaṃ caturvidham }||\thinspace4:72\thinspace||%
\translation{I shall now teach you the four meditations, which will liberate you from transmigration. Meditation is taught to be fourfold: of the Self, \textit{vidyā}, \textit{bhava} [= Śiva] and the subtle one {\rm (}\textit{sūkṣma}{\rm )}. \blankfootnote{4.72 Note the stem form \textit{dhyāna} in °\textit{dhyānādhunā} {\rm (}for °\textit{dhyānam adhunā}{\rm )} in \textit{pāda} a.
 For contrast, but also for similarities, see the \textit{dhyānayajña} section in \VSS\ 6.7ff, in which
  five types of related meditations are taught. See analysis on pp. Intro \verify.
 }}

  \maintext{ātmatattvaḥ smṛto dharmo vidyā pañcasu pañcadhā |}%

  \maintext{ṣaṭtriṃśākṣaram ityāhuḥ sūkṣmatattvam alakṣaṇam }||\thinspace4:73\thinspace||%
\translation{The \textit{tattva} of the Self is Dharma. \textit{Vidyā} is in the five in a fivefold way[??]. They call the thirty-sixth the imperishable one, [and] the subtle \textit{tattva} has no attributes. \blankfootnote{4.73 This verse is difficult to interpret. \textit{Pāda}s a to d should define \textit{ātman}, \textit{vidyā}, \textit{bhava}, and \textit{sūkṣma},
  objects of meditation, respectively. In \textit{pāda} a, \textit{dharmo} is suspect: it may be the result of
  an eyeskip to \textit{pāda} a of the next verse. \textit{Pāda} b might refer to \textit{tattva}s in an ontological
  system of 25, 26 or 36 \textit{tattva}s.
 If \textit{pāda} c is in fact a reference to a 36-\textit{tattva} philosophical system,
  it is in striking contrast with the 25-\textit{tattva} system described in \VSS\ chapter 20.
  I take \textit{ṣaṭtriṃśa} as being in stem form.
 }}

  \maintext{catuṣpādaḥ smṛto dharmaś caturāśramam āśritaḥ |}%

  \maintext{gṛhastho brahmacārī ca vānaprastho 'tha bhaikṣukaḥ }||\thinspace4:74\thinspace||%
\translation{The four-legged one is said to be Dharma [as] it rests on the four \textit{āśrama}s, [those of] the householder, the chaste one, the forest-dweller and the mendicant. }

  \maintext{dhanyās te yair idaṃ vetti nikhilena dvijottama |}%

  \maintext{pāvanaṃ sarvapāpānāṃ puṇyānāṃ ca pravardhanam }||\thinspace4:75\thinspace||%
\translation{Virtuous are those who know these thoroughly, O great Brahmin. [They will experience] the purification of all sins and the growth of merits. \blankfootnote{4.75 Note the plural instrumental {\rm (}\textit{yair}{\rm )} with a singular active verb {\rm (}\textit{vetti}; anacoluthic structure CHECK{\rm )}.
 }}

  \maintext{āyuḥ kīrtir yaśaḥ saukhyaṃ dhanyād eva pravardhate |}%

  \maintext{śāntiḥ puṣṭiḥ smṛtir medhā jāyate dhanyamānave }||\thinspace4:76\thinspace||%
\translation{One's life-span, fame and glory and happiness grow only through virtue {\rm (}\textit{dhanya}{\rm )}. In a virtuous person piece, prosperity, tradition {\rm (}\textit{smṛti}{\rm )} and intelligence will arise. \blankfootnote{4.76 Emending °\textit{mānavaḥ} to °\textit{mānave} might err by overcorrection, and °\textit{mānavaḥ} may have originally
  been felt like a genitive {\rm (}`for a person\dots'{\rm )}.
 }}

  \subchptr{yameṣv apramādaḥ {\rm {\rm (}8{\rm )}}}%

  \trsubchptr{Eighth Yama-rule: lack of negligence}%

  \maintext{pramādasthāna pañcaiva kīrtayiṣyāmi tac chṛṇu |}%

  \maintext{brahmahatyā surāpānaṃ steyo gurvaṅganāgamam |}%

  \maintext{mahāpātakam ity āhus tatsaṃyogī ca pañcamaḥ }||\thinspace4:77\thinspace||%
\translation{There are five areas of negligence. I shall teach them to you, listen. Murdering a Brahmin, drinking alcohol, stealing, having sex with the guru's wife: they call these grievous sins. The fifth is when one is connected with them [i.e. with these sins or with people involved in these sinful acts]. \blankfootnote{4.77 Note the stem form noun in \textit{pāda} a {\rm (}°\textit{sthāna}{\rm )} metri causa, and also 
  that this stem form noun may function as a singular noun
  next to a number {\rm (}\textit{pañca}{\rm )}, a frequently seen phenomenon in this text.
 See the apparatus to the Sanskrit text for very similar verses in the \MBH, \MANU\ and 
  the \YAJNS, and note how \textit{pāda} f slightly deviates from \MANU\ 11.55, which is translated in
  \mycitep{OlivelleManu}{217--218} as: 
  `Killing a Brahmin, drinking liquor, stealing, and having sex with an elder's 
  wife---they call these ``grievous sins causing loss of caste''; 
  and so is establishing any links with such individuals.'
 }}

  \maintext{anṛtaṃ ca samutkarṣe rājagāmī ca paiśunaḥ |}%

  \maintext{guroś cālīkanirbandhaḥ samāni brahmahatyayā }||\thinspace4:78\thinspace||%
\translation{A lie concerning one's superiority, a slander that reaches the king's ear, and false accusations against an elder are equal to killing a Brahmin. \blankfootnote{4.78 This verse being a quotation of \MANU\ 11.56, my translation 
  is based on \mycitep{OlivelleManu}{218}.
 }}

  \maintext{brahmojjhaṃ vedanindā ca kūṭasākṣī suhṛdvadhaḥ |}%

  \maintext{garhitānādyayor jagdhiḥ surāpānasamāni ṣaṭ }||\thinspace4:79\thinspace||%
\translation{Abandoning the Vedas, reviling the Vedas, being a false witness, murdering a friend, eating unfit or forbidden food are six [deeds that are] equal to drinking alcohol. \blankfootnote{4.79 This verse continues quoting \MANU. \textit{Pāda} a in the witnesses may actually be no more than the result of 
  misreading of the syllable \textit{jjha} in \MANU\ 11.57. Note the variant \textit{brahmojjhaṃ vedanindā ca}
  in both the `Northern' and `Southern' transmissions in Olivelle's critical edition 
  of \MANU\ {\rm (}\mycitep{OlivelleManu}{847}{\rm )}.
 }}

  \maintext{retotsekaḥ svayonyāsu kumārīṣv antyajāsu ca |}%

  \maintext{sakhyuḥ putrasya ca strīṣu gurutalpasamaḥ smṛtaḥ }||\thinspace4:80\thinspace||%
\translation{Sexual intercourse with a female relative, with an unmarried girl, with women of the lowest castes, with the wife of a friend or of one's own son are said to be equal to violating the guru's bed. \blankfootnote{4.80 The text, and my emendation in \textit{pāda} c, still follow \MANU\ {\rm (}11.59{\rm )}.
 }}

  \maintext{nikṣepasyāpaharaṇaṃ narāśvarajatasya ca |}%

  \maintext{bhūmivajramaṇīnāṃ ca rukmasteyasamaḥ smṛtaḥ }||\thinspace4:81\thinspace||%
\translation{Stealing deposits, people, horses, silver, land, diamonds, or gems are said to be equal to stealing gold. \blankfootnote{4.81 This is \MANU\ 11.58. I have emended \textit{rugma}° to \textit{rukma}° in \textit{pāda} d, although
  \textit{rugma}° is attested in a great number of Southern MSS and one Śāradā MS in \mycitep{OlivelleManu}{847}.
 }}

  \maintext{catvāra ete sambhūya yat pāpaṃ kurute naraḥ |}%

  \maintext{mahāpātakapañcaitat tena sarvaṃ prakāśitam |}%

  \maintext{pañcapramādam etāni varjanīyaṃ dvijottama }||\thinspace4:82\thinspace||%
\translation{If a man is associated with [any of these] four [i.e. \textit{brahmahatyā, surāpāna, stena, gurvaṅganāgama}], he commits sin. By this all the five grievous sins have been explained. These five kinds of negligence are to be avoided, O great Brahmin. \blankfootnote{4.82 Perhaps understand \textit{pāda} c as \textit{etan mahāpātakapañcakaṃ}.
 Note the confusion of number and gender: understand \textit{pañca pramādāḥ etā varjanīyāḥ}
  or \textit{pañca prāmādāny etāni varjanīyāni}.
 }}

  \subchptr{yameṣu mādhuryam {\rm {\rm (}9{\rm )}}}%

  \trsubchptr{Ninth Yama-rule: charm}%

  \maintext{kāyavāṅmanamādhuryaś cakṣur buddhiś ca pañcamaḥ |}%

  \maintext{saumyadṛṣṭipradānaṃ ca krūrabuddhiṃ ca varjayet }||\thinspace4:83\thinspace||%
\translation{[Charm has five types:] bodily, verbal and mental charm, [charm of] the eyes and [of one's] thoughts as fifth. Giving [others] a friendly glance [is commendable] and one should avoid cruel thoughts. \blankfootnote{4.83 My emendation from °\textit{manasā dhūryaś} to °\textit{mana-mādhuryaś} is based on the fact that following the list
  of \textit{yama}s in 3.16cd--17ab, we need some reference to \textit{mādhurya} here and that it is easy to see how this
  corruption came about: °\textit{mano-mādhurya}° would be unmetrical, hence the form °\textit{mana-mādhurya};
  °\textit{mana-mā}° is easily corrupted to °\textit{manasā}° {\rm (}not to mention the fact 
  that \textit{manasā} comes up in the next verse{\rm )}. 
  In addition, we need five items in this line because of \textit{pañcamaḥ}.
  As always, I correct \textit{mādhūrya} to \textit{mādhurya}, although it seems that 
  the former is acceptable in this text. 
  I did not correct \textit{mādhuryaś} to \textit{mādhuryaṃ} because of the corresponding
  \textit{pañcamaḥ}.
 }}

  \maintext{prasannamanasā dhyāyet priyavākyam udīrayet |}%

  \maintext{yathāśaktipradānaṃ ca svāśramābhyāgato guruḥ }||\thinspace4:84\thinspace||%
\translation{One should meditate with a tranquil mind and should speak [to other people using] gentle words. [When] respectable people arrive at one's own hermitage, [one should] present them with as many gifts as one can, \blankfootnote{4.84 \textit{Pāda}s cd of the previous verse, and \textit{pāda}s ab of the present one cover
  four categories of the above: \textit{cakṣurmādhurya}, \textit{buddhimādhurya}, \textit{dṛṣṭimādhurya} and \textit{vāgmādhurya}.
  This suggests that what follows is on \textit{kāyamādhurya}.
 Emending \textit{pāda} d to \textit{svāśramābhyāgate gurau} would make the line smoother, as
  suggested by Kengo Harimoto.
 }}

  \maintext{indhanodakadānaṃ ca jātavedam athāpi vā |}%

  \maintext{sulabhāni na dattāni indhanāgnyudakāni ca |}%

  \maintext{kṣute jīveti vā noktaṃ tasya kiṃ parataḥ phalam }||\thinspace4:85\thinspace||%
\translation{with gifts of fire-wood, water and fire. [If] fire-wood, fire and water are easily available [but] are not given [as gift] or [if the phrase] `Live [for a hundred years]!' is not uttered when [somebody] sneezes, what reward could there be for such a person in the afterlife? \blankfootnote{4.85 Understand \textit{jātavedam} in \textit{pāda} b as \textit{jātavedasam} or \textit{jātavedāḥ},
  or rather as belonging to the compound °\textit{dānaṃ}: \textit{jātavedodānaṃ}.
 For \textit{pāda} e, see an Āryāgīti verse in the \MAHASUBHS\ {\rm {\rm (}2558{\rm )}}: 
  \textit{amṛtāyatām iti vadet pīte bhukte kṣute ca śataṃ jīva\thinspace |
  choṭikayā saha jṛmbhāsamaye syātāṃ cirāyurānandau\thinspace ||}
  {\rm (}`When eating or drinking, one should say: ``May it turn into nectar!''; 
  and after sneezing: ``Live for a hundred years!''
  By snapping the thumb and forefinger when yawning, there will be long life and happiness.'{\rm )}
 }}

  \subchptr{yameṣv ārjavam {\rm {\rm (}10{\rm )}}}%

  \trsubchptr{Tenth Yama-rule: sincerity}%

  \maintext{pañcārjavāḥ praśaṃsanti munayas tattvadarśinaḥ |}%

  \maintext{karmavṛttyābhivṛddhiṃ ca pāratoṣikam eva ca |}%

  \maintext{strīdhanotkocavittaṃ ca ārjavo nābhinandati }||\thinspace4:86\thinspace||%
\translation{The sages who see the truth praise five types of sincerity. A sincere person does not rejoice in prosperity arising from the operation of karma or by a reward, in riches from women, from property, and bribery. \blankfootnote{4.86 °\textit{ārjavāḥ} should be in the accusative, therefore it is to be taken as feminine {\rm (}rather than neuter{\rm )} or as
  an irregular form for °\textit{ārjavāni}. I have emended \textit{pāratoṣikam} to \textit{pāritoṣikam}.
 My translation of the categories listed here is tentative, the only guiding light being
  that, if the first line is right, there should be five of them. In addition, I have tried to
  find categories that seem to be, more or less, in conflict with `sincerity' or `straightness.'
 }}

  \maintext{ārjavo na vṛthā yajña ārjavo na vṛthā tapaḥ |}%

  \maintext{ārjavo na vṛthā dānam ārjavo na vṛthāgnayaḥ }||\thinspace4:87\thinspace||%
\translation{If one is not sincere, sacrifice is in vain. If one is not sincere, austerity is in vain. If one is not sincere, donation is in vain. If one is not sincere, [sacrificial] fires are in vain. \blankfootnote{4.87 I thank Nirajan Kafle for helping me interpret this verse.
 }}

  \maintext{ārjavasyendriyagrāmaḥ suprasanno 'pi tiṣṭhati |}%

  \maintext{ārjavasya sadā devāḥ kāye tasya caranti te }||\thinspace4:88\thinspace||%
\translation{The sense faculties of a sincere person are firm even when he is delighted. The gods are always present in the body of a sincere person. }

  \maintext{iti yamapravibhāgaḥ kīrtito 'yaṃ dvijendra}%

 \nonanustubhindent \maintext{iha parata sukhārthaṃ kārayet taṃ manuṣyaḥ |}%

  \maintext{duritamalapahārī śaṅkarasyājñayāste}%

 \nonanustubhindent \maintext{bhavati pṛthivibhartā hy ekachatrapravartā }||\thinspace4:89\thinspace||%
\translation{Thus has been taught this section on the \textit{yama}-rules, O great Brahmin. Humans should follow them to reach happiness here and in the other world. One will stand removing one's filth of sins, and shall by Śaṅkara's command become a ruler of the world [that he subjugates] under one royal umbrella. \blankfootnote{4.89 In \textit{pāda} a °\textit{pra}° does not make the previous syllable long: this is the phenomenon of
  `muta cum liquida,' one of the hallmarks of the \VSS, 
  that is, syllables such as \textit{tra, pra, bra, dra} do not necessarily make the 
  previous syllable long.
 In \textit{pāda} b, \textit{parata} most probably stands for \textit{paratra} or \textit{parataḥ} metri causa. 
  We may correct it to \textit{paratra}, presupposing the presence of the licence `muta cum liquida.'
 °\textit{malapahārī} in the MSS stands either for °\textit{malāpahārī} or °\textit{malaprahārī} metri causa. 
  I could have chosen to emend it to °\textit{malaprahārī} {\rm (}again applying the licence `muta cum liquida'{\rm )},
  but I decided not to because \textit{apahārin}, \textit{apahāra}, \textit{apahāraka} are used in the text very frequently. 
  See also 8.44c, which contains a very similar expression: \textit{sakalamalapahāre dharmapañcāśad etat}.
 }}

\centerline{\maintext{\dbldanda\thinspace iti vṛṣasārasaṃgrahe yamavibhāgo nāmādhyāyaś{ }caturthaḥ\thinspace\dbldanda}}
\translation{Here ends the fourth chapter in the \textit{Vṛṣasārasaṃgraha} called the Section on the Yama-rules.}
