
  \chptr{tṛtīyo 'dhyāyaḥ}
\addcontentsline{toc}{section}{Chapter 3}
\fancyhead[CO]{{\footnotesize\textit{Translation of chapter 3}}}%

  \trchptr{  Chapter Three }%

  \subchptr{dharmapravacanam}%

  \trsubchptr{Exposition of Dharma}%

  \maintext{vigatarāga uvāca |}%

  \maintext{kimarthaṃ dharmam ity āhuḥ katimūrtiś ca kīrtyate |}%

  \maintext{katipādavṛṣo jñeyo gatis tasya kati smṛtāḥ }||\thinspace3:1\thinspace||%
\translation{Vigatarāga spoke: Why do they call it Dharma? And how many embodiments {\rm (}\textit{mūrti}{\rm )} is he known to have? He is known as a bull: how many legs does it have? How many are his paths? \blankfootnote{3.1 For the correct interpretation of \textit{pāda} a, namely to decide whether these questions
  focus on the bull of Dharma or Dharma itself/himself, see 
  the end of the previous chapter, where \textit{dharma} was mentioned {\rm (}2.40b{\rm )},
  and to which the present verse is a reaction; see also
  \MBH\ 12.110.10--11:
  
 
  \textit{prabhāvārthāya bhūtānāṃ dharmapravacanaṃ kṛtam\thinspace |
  yat syād ahiṃsāsaṃyuktaṃ sa dharma iti niścayaḥ\thinspace ||
  dhāraṇād dharma ity āhur dharmeṇa vidhṛtāḥ prajāḥ\thinspace |
  yat syād dhāraṇasaṃyuktaṃ sa dharma iti niścayaḥ\thinspace ||}
  
 
  Note the similarities of \MBH\ this passage with this chapter: the phrase \textit{dharma ity āhur},
  the fact that the present chapter from verse 18 on is actually a chapter on \textit{ahiṃsā},
  and that the etimological explanation involves the word [\textit{ā}]\textit{dhāraṇa} in
  both cases. These lead me to think that in \textit{pāda}s ab of this verse in the \VSS,
  it is Dharma that is the focus of the inquiry and not the bull.
 
  
 Understand \textit{pāda} d as \textit{gatayas tasya kati smṛtāḥ}. I have accepted
  \textit{smṛtāḥ} because this plural signals that \textit{gatis} is meant to be plural,
  similarly to what happens in 3.6cd {\rm (}\textit{tasya patnī... mahābhāgāḥ}{\rm )}.
  The use of the singular in a context of numbers and quantities is one of 
  the hallmarks of the language of the \VSS, see p. \verify.
 
  On Dharma as a bull, see Introduction, pp. \verify.
 }}

  \maintext{kautūhalaṃ mamotpannaṃ saṃśayaṃ chindhi tattvataḥ |}%

  \maintext{kasya putro muniśreṣṭha prajās tasya kati smṛtāḥ }||\thinspace3:2\thinspace||%
\translation{I have become curious [about these questions]. Put an end to my doubts for good. Whose son is [Dharma], O best of sages? How many children does he have? }

  \maintext{anarthayajña uvāca |}%

  \maintext{dhṛtir ity eṣa dhātur vai paryāyaḥ parikīrtitaḥ |}%

  \maintext{ādhāraṇān mahattvāc ca dharma ity abhidhīyate  }||\thinspace3:3\thinspace||%
\translation{Anarthayajña spoke: Well, \textit{dhṛti} {\rm (}`firmness'{\rm )} is [of the same] verbal root [as \textit{dharma}], and is said to be [its] synonym. It is called \textit{dharma} because it supports {\rm (}\textit{āDHĀRaṇa}{\rm )} and because it is great {\rm (}\textit{MAhattva}{\rm )}. \blankfootnote{3.3 For similar Purāṇic passages on the etimology of \textit{dharma}, see the apparatus to
  this verse.
 
  The insertion in my translation `[of the same]' solves the problem of a noun {\rm (}\textit{dhṛti}{\rm )} seemingly
  being considered a verbal root {\rm (}\textit{dhātu}{\rm )} here. I owe thanks to Judit Törzsök for this interpretation.
  For similar passages with nominal stems appearently being treated as \textit{dhātu}s, see e.g. 
  \VAYUP\ 3.17cd:
  \textit{bhāvya ity eṣa dhātur vai bhāvye kāle vibhāvyate};
  \VAYUP\ 3.19cd {\rm (}= \BRAHMANDAPUR\ 1.38.21ab{\rm )}:
  \textit{nātha ity eṣa dhātur vai dhātujñaiḥ pālane smṛtaḥ};
  \LINPU\ 2.9.19:
  \textit{bhaja ity eṣa dhātur vai sevāyāṃ parikīrtitaḥ}.
 }}

  \maintext{śrutismṛtidvayor mūrtiś catuṣpādavṛṣaḥ sthitaḥ |}%

  \maintext{caturāśrama yo dharmaḥ kīrtitāni manīṣibhiḥ }||\thinspace3:4\thinspace||%
\translation{The four-legged Bull is the embodiment of both Śruti and Smṛti. It is Dharma, as made up of the four \textit{āśrama}s. \blankfootnote{3.4 A similar image of the legs of the Bull of Dharma being the four {\rm (}and not three, at least according to
  \mycitep{OlivelleAsrama}{55} and
  \mycitep{GanguliMBh}{Śāntiparvan CCLXX}{\rm )} 
  \textit{āśrama}s is hinted at \MBH\ 12.262.19--21: 
  
 
  \textit{dharmam ekaṃ catuṣpādam āśritās te nararṣabhāḥ\thinspace |
  taṃ santo vidhivat prāpya gacchanti paramāṃ gatim\thinspace ||
  gṛhebhya eva niṣkramya vanam anye samāśritāḥ\thinspace |
  gṛham evābhisaṃśritya tato 'nye brahmacāriṇaḥ\thinspace ||
  dharmam etaṃ catuṣpādam āśramaṃ brāhmaṇā viduḥ\thinspace |
  ānantyaṃ brahmaṇaḥ sthānaṃ brāhmaṇā nāma niścayaḥ\thinspace ||}.
  
 
  On the more frequently quoted interpretation of the four legs, see 
  \mycitep{OlivelleAsrama}{235}, a translation of \MANU\ 1.81--82:
  `Dharma and truth possess all four feet and are whole during the Kṛta yuga, 
  and people did not obtain anything unrighteously {\rm (}\textit{adharmeṇa}{\rm )}. 
  By obtaining, however, \textit{dharma} has lost one foot during each of the other \textit{yuga}s 
  and righteousness {\rm (}\textit{dharma}{\rm )} likewise has diminished by one quarter due to theft, 
  falsehood, and deceit. {\rm (}MDh 1.81--82{\rm )}.'
  
 
  Understand \textit{pāda}s c and d as \textit{catvāri āśramāṇi kīrtitāni dharmo manīṣibhiḥ} or
  \textit{yo dharmaḥ kīrtitaś caturāśramāṇi manīṣibhiḥ} or 
  \textit{yo dharmaś caturāśramaḥ kīrtito manīṣibhiḥ}. Judit Törzsök suggested
  that \textit{caturāśrama} and \textit{dharmaḥ} may be interpreted as a compound here.
 }}

  \maintext{gatiś ca pañca vijñeyāḥ śṛṇu dharmasya bho dvija |}%

  \maintext{devamānuṣatiryaṃ ca narakasthāvarādayaḥ }||\thinspace3:5\thinspace||%
\translation{And the paths of Dharma are five. Listen, O Brahmin: [existence as] gods, men, animals, [existence in] hell and [as] immovable things [such as plants and rocks] etc. \blankfootnote{3.5 Note the use of the singular next to numbers in \textit{pāda} a, as in 3.1d, and that
  \textit{vijñeyāḥ} is an emendation from \textit{vijñeyaḥ} following the logic of 3.1d.
 \textit{tirya} seems to be an acceptable nominal stem in this text for \textit{tiryañc}. See,
  e.g., 4.6a: \textit{devamānuṣatiryeṣu}. \textit{°ādayaḥ} in \textit{pāda} d seems superfluous.
 }}

  \maintext{brahmaṇo hṛdayaṃ bhittvā jāto dharmaḥ sanātanaḥ |}%

  \maintext{tasya patnī mahābhāgā trayodaśa sumadhyamāḥ }||\thinspace3:6\thinspace||%
\translation{Eternal Dharma was born after splitting Brahmā's heart. He has beautiful wives, thirteen in number, with nice waists. \blankfootnote{3.6 Note the use of the singular in \textit{pāda}s cd. I have left \textit{sumadhyamāḥ} as the
  manuscripts transmit it: it signals the presence of the plural. And consider 
  correcting \textit{mahābhāgā} to \textit{mahābhāgās}. In sum, understand
  \textit{tasya patnyo mahābhāgās trayodaśa sumadhyamāḥ}.
 }}

  \maintext{dakṣakanyā viśālākṣī śraddhādyāḥ sumanoharāḥ |}%

  \maintext{tasya putrāś ca pautrāś ca anekāś ca babhūva ha |}%

  \maintext{eṣa dharmanisargo 'yaṃ kiṃ bhūyaḥ śrotum icchasi }||\thinspace3:7\thinspace||%
\translation{They are Dakṣa's daughters, [called] Śraddhā and so on. They have huge eyes and they are beautiful. Numerous sons and grandsons were born to him. This is the nature of Dharma. What more do you wish to hear? \blankfootnote{3.7 \textit{śraddhāḍhyāḥ} in \textit{pāda} b is an attractive lectio difficilior {\rm (}`they were rich in faith/devotion'{\rm )}, but I have finally 
  decided to accept the easier and better-attested \textit{śraddhādyā}[\textit{ḥ}].
  Again, I have chosen/applied the plural forms \textit{°ādyāḥ} and \textit{sumanoharāḥ} in \textit{pāda} b to hint at the fact
  that the presence of the plural is to be preferred here; thus only \textit{viśālākṣī} is 
  problematic. As \textit{patnī} in the previous verse, it should be treated as a plural.
  Note the use of the singular for the plural also in \textit{pāda}s cd, especially \textit{babhūva ha} for \textit{babhūvuḥ}
  {\rm (}\textit{babhūva ha} perhaps being a phonetic and metrically `adjusted' equivalent, so to say, of \textit{babhūvuḥ}{\rm )}.
 }}

  \maintext{vigatarāga uvāca |}%

  \maintext{dharmapatnī viśeṣeṇa putras tābhyaḥ pṛthak pṛthak |}%

  \maintext{śrotum icchāmi tattvena kathayasva tapodhana }||\thinspace3:8\thinspace||%
\translation{Vigatarāga spoke: I would like to hear about Dharma's wives truly and about each one of the sons born to them. Teach me, O great ascetic. \blankfootnote{3.8 I have emended \textit{tebhyaḥ} to the correct feminine form \textit{tābhyaḥ}
  because I suspect that it is only the result of some early confusion
  brought about by \textit{putras}, although \textit{tebhyaḥ} might be original.
  Note again the use of the singular {\rm (}nominative{\rm )} for the plural {\rm (}accusative{\rm )} in \textit{pāda}s ab.
  Alternatively, emend \textit{dharmapatnī} to \textit{dharmapatnīr} {\rm (}plural accusative{\rm )} and 
  \textit{putras} to \textit{putrān} to make them work with \textit{śrotum icchāmi}.
 }}

  \maintext{anarthayajña uvāca |}%

  \maintext{śraddhā lakṣmīr dhṛtis tuṣṭiḥ puṣṭir medhā kriyā lajjā |}%

  \maintext{buddhiḥ śāntir vapuḥ kīrtiḥ siddhiḥ prasūtisambhavāḥ }||\thinspace3:9\thinspace||%
\translation{Anarthayajña spoke: [Dharma's wives are] [1] Śraddhā {\rm (}`Faith'{\rm )}, [2] Lakṣmī {\rm (}`Prosperity'{\rm )}, [3] Dhṛti {\rm (}`Resolution'{\rm )}, [4] Tuṣṭi {\rm (}`Satisfaction'{\rm )}, [5] Puṣṭi {\rm (}`Growth'{\rm )}, [6] Medhā {\rm (}`Wisdom'{\rm )}, [7] Kriyā {\rm (}`Labour'{\rm )}, [8] Lajjā {\rm (}`Modesty'{\rm )}, [9] Buddhi {\rm (}`Intelligence'{\rm )}, [10] Śānti {\rm (}`Tranquillity'{\rm )}, [11] Vapus {\rm (}`Beauty'{\rm )}, [12] Kīrti {\rm (}`Fame'{\rm )}, [13] Siddhi {\rm (}`Success'{\rm )}, [all] born to Prasūti [Dakṣa's wife]. \blankfootnote{3.9 Note how \textit{lajjā} in \textit{pāda} b makes the line unumetrical.
 
  For Dharma's thirteen wives and their sons, see, e.g., \LINPU\ 1.5.34--37 {\rm (}note the 
  similarity between the first line and \VSS\ 3.6cd--7ab above{\rm )}:
  
 
  \textit{dharmasya patnyaḥ śraddhādyāḥ kīrtitā vai trayodaśa\thinspace |
  tāsu dharmaprajāṃ vakṣye yathākramam anuttamam\thinspace ||
  kāmo darpo 'tha niyamaḥ saṃtoṣo lobha eva ca\thinspace |
  śrutas tu daṇḍaḥ samayo bodhaś caiva mahādyutiḥ\thinspace ||
  apramādaś ca vinayo vyavasāyo dvijottamāḥ\thinspace |
  kṣemaṃ sukhaṃ yaśaś caiva dharmaputrāś ca tāsu vai\thinspace || 
  dharmasya vai kriyāyāṃ tu daṇḍaḥ samaya eva ca\thinspace |
  apramādas tathā bodho buddher dharmasya tau sutau\thinspace ||}.
  
 
 
  \textit{prasūtisambhavāḥ} in \textit{pāda} d is a rather bold conjecture that can be supported by two facts:
  firstly, the readings of the manuscripts are difficult to make sense of and thus are
  probably corrupt; secondly, a corruption from the name Prasūti,
  traditionally the name of Dakṣa's wife, to \textit{ābhūti}
  is relatively easily to explain, \textit{sū} and \textit{bhū} being close enough in some scripts 
  {\rm (}e.g. in \msCa{\rm )} to cause confusion. Another option would be to accept 
  Ābhūti as the name of Dakṣa's wife.
  
 
  For Prasūti being Dakṣa's wife in other sources,
  see, e.g., \LINPU\ 1.5.20--21 {\rm (}but also note the presence of the name Sambhūti{\rm )}:
  \textit{prasūtiḥ suṣuve dakṣāc caturviṃśatikanyakāḥ\thinspace |
  śraddhāṃ lakṣmīṃ dhṛtiṃ puṣṭiṃ tuṣṭiṃ medhāṃ kriyāṃ tathā\thinspace ||
  buddhi lajjāṃ vapuḥ śāntiṃ siddhiṃ kīrtiṃ mahātapāḥ\thinspace |
  khyātiṃ śāntiś ca saṃbhūtiṃ smṛtiṃ prītiṃ kṣamāṃ tathā\thinspace ||}.
 }}

  \maintext{śraddhā kāmaḥ suto jāto darpo lakṣmīsutaḥ smṛtaḥ |}%

  \maintext{dhṛtyās tu niyamaḥ putraḥ saṃtoṣas tuṣṭijaḥ smṛtaḥ }||\thinspace3:10\thinspace||%
\translation{Śraddhā's son is Kāma {\rm (}`Desire'{\rm )}. Darpa {\rm (}`Pride'{\rm )} is said to be Lakṣmī's son. Dhṛti's son is Niyama {\rm (}`Rule'{\rm )}. Saṃtoṣa {\rm (}`Satisfaction'{\rm )} is Tuṣṭi's son. \blankfootnote{3.10 Understand \textit{śraddhā} as a stem form noun for \textit{śraddhāyāḥ} {\rm (}gen./abl., cf. 3.11a{\rm )}.
  Alternatively, take \textit{śraddhā} and \textit{suto} as elements of a split compound, and understand
  \textit{śraddhāsuto jātaḥ kāmaḥ}.
 }}

  \maintext{puṣṭyā lābhaḥ suto jāto medhāputraḥ śrutas tathā |}%

  \maintext{kriyāyās tv abhavat putro daṇḍaḥ samaya eva ca }||\thinspace3:11\thinspace||%
\translation{To Puṣṭi was born a son [called] Lābha {\rm (}`Profit'{\rm )}. Medhā's son is Śruta {\rm (}`Sacred Knowledge'{\rm )}. Kriyā's sons are Daṇḍa {\rm (}`Punishment'{\rm )} and Samaya {\rm (}`Law'{\rm )}. \blankfootnote{3.11 I have emended \textit{abhayaḥ} to \textit{abhavat} in \textit{pāda} c, following the relevant line in the \KURMP\ cited above
  {\rm (}\textit{kriyāyāś cābhavat putro daṇḍaḥ samaya eva ca}{\rm )} and also \LINPU\ 1.5.37 quoted in the 
  apparatus to this verse, allotting only two sons to Kriyā. Thus I don't think
  that Kriyā is supposed to have a son called Abhaya {\rm (}`Freedom from danger'; \BHAGP\ 4.1.50ab 
  claims that Dayā had a son called Abhaya:
  \textit{śraddhāsūta śubhaṃ maitrī prasādam abhayaṃ dayā}{\rm )}.
  Nevertheless, in a number of sources Kriyā actually has three sons, 
  see, e.g., \VISNUP\ 1.7.26ab,
  where they are named as Daṇḍa, Naya and Vinaya:
  \textit{medhā śrutaṃ kriyā daṇḍaṃ nayaṃ vinayam eva ca}. 
  Perhaps read \textit{kriyāyās tu nayaḥ putro} in \textit{pāda} c? Compare \VAYUP\ 1.10.34cd
  {\rm (}\textit{kriyāyās tu nayaḥ prokto daṇḍaḥ samaya eva ca}{\rm )} 
  with \BRAHMANDAPUR\ 1.9.60ab {\rm (}\textit{kriyāyās tanayau proktau damaś ca śama eva ca}{\rm )}.
 }}

  \maintext{lajjāyā vinayaḥ putro buddhyā bodhaḥ sutaḥ smṛtaḥ |}%

  \maintext{lajjāyāḥ sudhiyaḥ putra apramādaś ca tāv ubhau }||\thinspace3:12\thinspace||%
\translation{Lajjā's son is Vinaya {\rm (}`Discipline'{\rm )}, Buddhi's son is Bodha {\rm (}`Intelligence'{\rm )}. Lajjā has two [more] sons: Sudhiya[/Sudhī] {\rm (}`Wise'{\rm )} and Apramāda {\rm (}`Cautiousness'{\rm )}. \blankfootnote{3.12 In a very similar passages in \KURMP\ 1.8.20 ff., Apramāda is Buddhi's son and 
  Lajjā has only one son, Vinaya. In the above verse {\rm (}\VSS\ 3.12{\rm )}, \textit{sudhiyaḥ} {\rm (}for \textit{sudhīḥ}{\rm )} may only be 
  qualifying \textit{apramāda}, thus Lajjā may have two sons: Vinaya and the wise Apramāda.
  Alternatively, \textit{pāda}s cd might be a extra line inserted accidentally.
 }}

  \maintext{kṣemaḥ śāntisuto vindyād vyavasāyo vapoḥ sutaḥ |}%

  \maintext{yaśaḥ kīrtisuto jñeyaḥ sukhaṃ siddher vyajāyata |}%

  \maintext{svāyambhuve 'ntare tv āsan kīrtitā dharmasūnavaḥ }||\thinspace3:13\thinspace||%
\translation{Kṣema {\rm (}`Peace'{\rm )} is to be known as Śānti's son, Vyavasāya {\rm (}`Resolution'{\rm )} is Vapus' son. Yaśas {\rm (}`Fame'{\rm )} is Kīrti's son, Sukha {\rm (}`Joy'{\rm )} was born to Siddhi. [This is how] the sons of Dharma in the [\textit{manvantara}] era of Svāyambhuva [Manu] were known. \blankfootnote{3.13 Note that \textit{sukhaṃ} in \textit{pāda} d is probably meant to be masculine {\rm (}\textit{sukhaḥ}{\rm )}, but e.g. in the 
  \KURMP\ passage quoted above it is also neuter. For the emendation in \textit{pāda} e, 
  see \MATSP\ 9.2cd: 
  \textit{yāmā nāma purā devā āsan svāyambhuvāntare},
  and \BHAGP\ 6.4.1: 
  \textit{devāsuranṛṇāṃ sargo nāgānāṃ mṛgapakṣiṇām\thinspace |
  sāmāsikas tvayā prokto yas tu svāyambhuve 'ntare\thinspace ||}.
 }}

  \maintext{vigatarāga uvāca |}%

  \maintext{mūrtidvayaṃ kathaṃ dharmaṃ kathayasva tapodhana |}%

  \maintext{kautūhalam atīvaṃ me kartaya jñānasaṃśayam }||\thinspace3:14\thinspace||%
\translation{Vigatarāga spoke: How come Dharma has two embodiments? Tell me, O great ascetic. I am extremely intrigued. Cut my doubts concerning [this] knowledge. \blankfootnote{3.14 Note \textit{dharma} as a neuter noun and the form \textit{atīvaṃ} for \textit{atīva} metri causa. 
  My emen\-dation from \textit{kīrtaya} {\rm (}`declare'{\rm )} to \textit{kartaya} {\rm (}`cut'{\rm )} was influenced by the combination
  of \textit{chindhi} and \textit{saṃśaya}, often with \textit{kautūhala}, elsewhere in the \VSS:
  3.2ab: \textit{kautūhalaṃ mamotpannaṃ saṃśayaṃ chindhi tattvataḥ}; 
  10.10cd: \textit{kautūhalaṃ mahaj jātaṃ chindhi saṃśayakārakam};
  15.2ab: \textit{etat kautūhalaṃ chindhi saṃśayaṃ parameśvara}. 
  The reading \textit{kīrtaya} may have been the result of the influence of \textit{kīrtitā} in 3.13b above 
  {\rm (}De Simini's observation{\rm )}.
 }}

  \maintext{anarthayajña uvāca |}%

  \maintext{śrutismṛtidvayor mūrtir dharmasya parikīrtitā |}%

  \maintext{dārāgnihotrasambandham ijyā śrautasya lakṣaṇam |}%

  \maintext{smārto varṇāśramācāro yamaiś ca niyamair yutaḥ }||\thinspace3:15\thinspace||%
\translation{Anarthayajña spoke: Dharma's embodiment is said to consist of Śruti and Smṛti. The characteristics of the Śrauta [tradition] are an association with a wife [i.e.\ marriage] and with the fire ritual, and sacrifice. The Smārta [tradition] [focuses on] the conduct {\rm (}\textit{ācāra}{\rm )} of the classes {\rm (}\textit{varṇa}{\rm )} and life-stages {\rm (}\textit{āśrama}{\rm )} which is connected to rules and regulations {\rm (}\textit{yama-niyama}{\rm )}. \blankfootnote{3.15 The reading \textit{°dvayī} in \msNc\ in \textit{pāda} a is attractive, but as Judit 
  Törzsök has pointed out to me, it is more likely that
  the slightly less convincing but widespread variant \textit{°dvayor} is original.
 
  As for Dharma being based on \textit{śruti} and \textit{smṛti}, see, e.g., \MANU\ 2.10:
  \textit{śrutis tu vedo vijñeyo dharmaśāstraṃ tu vai smṛtiḥ\thinspace |
  te sarvārtheṣv amīmāṃsye tābhyāṃ dharmo hi nirbabhau\thinspace ||}.
  In Olivelle's translation {\rm (}\mycitep{OlivelleManu}{94}{\rm )}:
  `\thinspace ``Scripture'' should be recognized as ``Veda,'' and ``tradition''
  as ``Law Treatise.'' These two should never be called into question in any matter,
  for it is from them that the Law shines forth.'
 
  
  There may be a hiatus-filler in \textit{pāda}s cd: \textit{°sambandha-m-ijyā} for \textit{°sambandha ijyā}.
 
  To state that the Smārta tradition is connected to \textit{yama}s and \textit{niyama}s and the \textit{āśrama}s and
  then to discuss these at length {\rm (}principally in chapters 3--8 and 11{\rm )} can be seen 
  as a clear self-identification with the Smārta tradition.
 }}

  \subchptr{yamaniyamabhedaḥ}%

  \trsubchptr{Yama and Niyama rules}%

  \maintext{yamaś ca niyamaś caiva dvayor bhedam ataḥ śṛṇu |}%

  \maintext{ahiṃsā satyam asteyam ānṛśaṃsyaṃ damo ghṛṇā |}%

  \maintext{dhanyāpramādo mādhuryam ārjavaṃ ca yamā daśa }||\thinspace3:16\thinspace||%
\translation{Now hear the classification of both the \textit{yama} and \textit{niyama} rules. Non-violence, truthfulness, not stealing, absence of hostility, self-restraint, taboos, virtue, carefulness, charm, honesty: these are the ten \textit{yama}s. \blankfootnote{3.16 \textit{Pāda} a should be understood as \textit{yamaniyamayoś caiva}, but the author of this line
  may have tried to avoid the metrical fault of having two short syllables 
  in second and third position.
 Note that this is the beginning of a long section in our text
  that describes the \textit{yama-niyama} rules, reaching up to the end of chapter eight. 
  The title given in the colophon of the next chapter, chapter four, namely \textit{yamavibhāga},
  would fit this locus better than the beginning of that chapter, which 
  commences with a discussion on the second of the \textit{yama}s, \textit{satya}.
 Note how all witnesses read \textit{mādhūrya} in \textit{pāda} e instead of \textit{mādhurya}. The former may have been
  acceptable originally in this text. \textit{Pāda} e is a \textit{ma-vipulā}.
 }}

  \maintext{ekaikasya punaḥ pañcabhedam āhur manīṣiṇaḥ |}%

  \maintext{ahiṃsādi pravakṣyāmi śṛṇuṣvāvahito dvija }||\thinspace3:17\thinspace||%
\translation{The wise say that there are five subclasses to each. I shall teach you about non-violence and the other [\textit{yama}-rules]. Listen carefully, O twice-born. \blankfootnote{3.17 In \textit{pāda} a, \textit{pañca} and \textit{bheda} may be typeset as two separate words since
  the use of the singular after numbers is one of the hallmarks of the text {\rm (}see \verify{\rm )}.
 }}

  \subchptr{yameṣv ahiṃsā {\rm {\rm (}1{\rm )}}}%

  \trsubchptr{First Yama-rule: non-violence}%

  \subsubchptr{pañcavidhā hiṃsā}%

  \trsubsubchptr{Five types of violence}%

  \maintext{trāsanaṃ tāḍanaṃ bandho māraṇaṃ vṛttināśanam |}%

  \maintext{hiṃsāṃ pañcavidhām āhur munayas tattvadarśinaḥ }||\thinspace3:18\thinspace||%
\translation{Frightening and beating [other people], tying [someone] up, killing and the destruction of [other people's] livelihood: violence is said by the wise who see the truth to be of [these] five types. }

  \maintext{kāṣṭhaloṣṭakaśādyais tu tāḍayantīha nirdayāḥ |}%

  \maintext{tatprahāravibhinnāṅgo mṛtavadhyam avāpnuyāt }||\thinspace3:19\thinspace||%
\translation{Cruel people beat [other people] with sticks, clods of earth [understand: they stone them], with whips and other [objects] in the everyday world. Their bodies broken by the same blows, they receive the capital punishment. \blankfootnote{3.19 Note the use of the singular {\rm (}°\textit{āṅgo}... \textit{avāpnuyāt}{\rm )} in \textit{pāda}s cd referring back to the agents of the previous sentence.
  Most probably, °\textit{vadhyam} is to be understand as °\textit{vadham} and the form 
  \textit{vadhyam} serves only to avoid two \textit{laghu} syllables in \textit{pāda} d.
 }}

  \maintext{baddhvā pādau bhujoraś ca śirorukkaṇṭhapāśitāḥ |}%

  \maintext{anāhatā mriyanty evaṃ vadho bandhanajaḥ smṛtaḥ }||\thinspace3:20\thinspace||%
\translation{[Others,] tie up [people] at their feet and their arms and chests. [These,] hung by their hair and neck, die in this way without being wounded. This is the capital punishment for tying up [other people]. \blankfootnote{3.20 Understand \textit{bhujoraś ca} in \textit{pāda} a as \textit{bhuje, urasi ca}, in this case with an instance of double sandhi,
  and in stem form: \textit{bhuje urasi ca} $\rightarrow$\ \textit{bhuja urasi ca} 
  $\rightarrow$\ \textit{bhujorasi ca} $\rightarrow$\ \textit{bhujoraś ca}.
  Alternatively, understand it as a compound {\rm (}\textit{bhujorasi}{\rm )}. 
  In \textit{pāda} b, my emendation is only one of the possible interpretations. We might accept
  \textit{śiroru}° as consisting of \textit{śira} + \textit{ūru} {\rm (}`head and thigh'{\rm )}, or emend it 
  to \textit{śiroraḥ}° for \textit{śira} + \textit{uraḥ} {\rm (}`head and chest'{\rm )}. Also note my conjecture
  in \textit{pāda} d, without which this \textit{pāda} is difficult to interpret.
 }}

  \maintext{śatrucaurabhayair ghoraiḥ siṃhavyāghragajoragaiḥ |}%

  \maintext{trāsanād vadham āpnoti anyair vāpi suduḥsahaiḥ }||\thinspace3:21\thinspace||%
\translation{He who frightens [other people] with the terrible danger of enemies and thieves, with lions, tigers, elephants or snakes, or by other horrors, will be executed. }

  \maintext{yasya yasya hared vittaṃ tasya tasya vadhaḥ smṛtaḥ |}%

  \maintext{vṛttijīvābhibhūtānāṃ taddvārā nihataḥ smṛtaḥ }||\thinspace3:22\thinspace||%
\translation{He who robs somebody's money is to be punished by the same person. He is [to be] struck down by those whose livelihood got damaged by him. \blankfootnote{3.22 Understand \textit{vadhaḥ} in \textit{pāda} b as \textit{vadhyaḥ} metri causa.
 My translation of the second line of this verse reflects a conjecture {\rm (}\textit{taddvārā}{\rm )}
  understood as connected to both \textit{pāda} c and \textit{nihataḥ} in \textit{pāda} d.
 }}

  \maintext{viṣavahniśaraśastrair māyāyogabalena vā |}%

  \maintext{hiṃsakāny āhu viprendra munayas tattvadarśinaḥ }||\thinspace3:23\thinspace||%
\translation{[Those who kill other people] with poison, fire, arrows, swords, or by the force of magic or yoga, are called murderers by the sages who see the truth, O great Brahmin. \blankfootnote{3.23 \textit{Pāda} a is a \textit{sa-vipulā} with two \textit{laghu}s.
  Note how elliptical this verse is and that \textit{hiṃsakāni} is neuter although it refers to 
  people, perhaps implying \textit{bhūtāni}. Alternatively, take \textit{y} in \textit{hiṃsakāny} as a 
  rather unusual sandhi-bridge {\rm (}\textit{hiṃsakān-y-āhu}{\rm )}, or simply delete this \textit{y}. 
  Note also that \textit{āhu} stands for \textit{āhur} metri causa.
 }}

  \subsubchptr{ahiṃsāpraśaṃsā}%

  \trsubsubchptr{Praise of non-violence}%

  \maintext{ahiṃsā paramaṃ dharmaṃ yas tyajet sa durātmavān |}%

  \maintext{kleśāyāsavinirmuktaṃ sarvadharmaphalapradam }||\thinspace3:24\thinspace||%
\translation{Non-violence is the highest Dharma. He who abandons it is a wicked person. It is free of pain and trouble, it yields the fruits of all [other] Dharmic teachings [in itself]. \blankfootnote{3.24 Note \textit{dharma} as a neuter noun in \textit{pāda} a and that \textit{°vinirmuktaṃ} and
  \textit{°pradam} are neuter accordingly.
 }}

  \maintext{nātaḥ parataro mūrkho nātaḥ parataraṃ tamaḥ |}%

  \maintext{nātaḥ parataraṃ duḥkhaṃ nātaḥ parataro 'yaśaḥ }||\thinspace3:25\thinspace||%
\translation{There isn't a bigger fool than he [who abandons it]. There is no bigger mental darkness [than the abandonment of non-violence]. There is no greater suffering or greater infamy. \blankfootnote{3.25 Note that \textit{parataro} is masculine in \textit{pāda} d, picking up a neuter \textit{'yaśaḥ}.
  This phenomenon is probably the result of \textit{'yaśaḥ} resembling a masculine noun ending in \textit{-aḥ}
  and also of the metrical problem with a grammatically correct \textit{nātaḥ parataram ayaśaḥ}.
 }}

  \maintext{nātaḥ parataraṃ pāpaṃ nātaḥ parataraṃ viṣam |}%

  \maintext{nātaḥ paratarāvidyā nātaḥ paraṃ tapodhana }||\thinspace3:26\thinspace||%
\translation{There is no greater sin or a more effective poison. There is no greater ignorance, there is nothing worse, O great ascetic. \blankfootnote{3.26 \textit{Pāda} d {\rm (}\textit{nātaḥ paraṃ tapodhana}{\rm )} is slightly suspect. 
  The vocative \textit{tapodhana} usually refers to Anarthayajña in these
  passages, and not to Vigatarāga, as here. The text may have read \textit{nātaḥ paratamo 'dhanaḥ} 
  {\rm (}`There is no bigger loss of wealth'{\rm )} or possibly something starting with
  \textit{nātaḥ paraṃ tapo ...} {\rm (}`There is no greater\dots\ of austerity'{\rm )}.
 }}

  \maintext{yo hinasti na bhūtāni udbhijjādi caturvidham |}%

  \maintext{sa bhavet puruṣaḥ śreṣṭhaḥ sarvabhūtadayānvitaḥ }||\thinspace3:27\thinspace||%
\translation{He who does not harm the four types of living beings beginning with plants is the best person, having compassion for all creatures. }

  \maintext{sarvabhūtadayāṃ nityaṃ yaḥ karoti sa paṇḍitaḥ |}%

  \maintext{sa yajvā sa tapasvī ca sa dātā sa dṛḍhavrataḥ }||\thinspace3:28\thinspace||%
\translation{He who always has compassion for all creatures is the [true] Pandit. He is the [true] sacrificer, the [true] ascetic, he is the donor, the one with a firm vow. }

  \maintext{ahiṃsā paramaṃ tīrtham ahiṃsā paramaṃ tapaḥ |}%

  \maintext{ahiṃsā paramaṃ dānam ahiṃsā paramaṃ sukham }||\thinspace3:29\thinspace||%
\translation{Non-violence is the supreme pilgrimage place. Non-violence is the highest austerity. Non-violence is the highest donation. Non-violence is the highest joy. }

  \maintext{ahiṃsā paramo yajñaḥ ahiṃsā paramaṃ vratam |}%

  \maintext{ahiṃsā paramaṃ jñānam ahiṃsā paramā kriyā }||\thinspace3:30\thinspace||%
\translation{Non-violence is the supreme sacrifice. Non-violence is the supreme religious observance. Non-violence is supreme knowledge. Non-violence is the supreme ritual. }

  \maintext{ahiṃsā paramaṃ śaucam ahiṃsā paramo damaḥ |}%

  \maintext{ahiṃsā paramo lābhaḥ ahiṃsā paramaṃ yaśaḥ }||\thinspace3:31\thinspace||%
\translation{Non-violence is the highest purity. Non-violence is the highest self-restraint. Non-violence is the highest profit. Non-violence is the greatest fame. }

  \maintext{ahiṃsā paramo dharmaḥ ahiṃsā paramā gatiḥ |}%

  \maintext{ahiṃsā paramaṃ brahma ahiṃsā paramaḥ śivaḥ }||\thinspace3:32\thinspace||%
\translation{Non-violence is the supreme Dharma. Non-violence is the supreme path. Non-violence is the supreme Brahman. Non-violence is supreme Śiva. }

  \subsubchptr{māṃsāhāraḥ}%

  \trsubsubchptr{On meet-consumption}%

  \maintext{māṃsāśanān nivarteta manasāpi na kāṅkṣayet |}%

  \maintext{sa mahat phalam āpnoti yas tu māṃsaṃ vivarjayet }||\thinspace3:33\thinspace||%
\translation{One should refrain from meat-consumption. One should not even desire it mentally. He who abandons meat will receive a great reward. }

  \maintext{svamāṃsaṃ paramāṃsena yo vardhayitum icchati |}%

  \maintext{anabhyarcya pitṝn devān na tato 'nyo 'sti pāpakṛt }||\thinspace3:34\thinspace||%
\translation{He who wishes to nourish his own flesh with the flesh of other [beings], outside of worshipping the ancestors and the gods, is the biggest sinner of all. \blankfootnote{3.34 See \UUMS\ chapter two for a similar section on meat-consumption.
 }}

  \maintext{madhuparke ca yajñe ca pitṛdaivatakarmaṇi |}%

  \maintext{atraiva paśavo hiṃsyā nānyatra manur abravīt }||\thinspace3:35\thinspace||%
\translation{During the \textit{madhuparka} offering and during a sacrifice, during rituals for the ancestors and the gods: only in these cases are animals to be slaughtered and not in any other case. [This is what] Manu taught. \blankfootnote{3.35 This verse is a variant of \MANU\ 5.41.
 }}

  \maintext{krītvā svayaṃ vāpy utpādya paropahṛtam eva vā |}%

  \maintext{devān pitṝṃś cārcayitvā khādan māṃsaṃ na doṣabhāk }||\thinspace3:36\thinspace||%
\translation{Should he buy it or procure it himself or should it be offered by others, if he eats meat, he will not sin if he first worships the gods and the ancestors. }

  \maintext{vedayajñatapastīrthadānaśīlakriyāvrataiḥ |}%

  \maintext{māṃsāhāranivṛttānāṃ ṣoḍaśāṃśaṃ na pūryate }||\thinspace3:37\thinspace||%
\translation{[People who perform] Vedic sacrifices and austerities, and [visit] sacred places, donate, [those who are of] good conduct, [perform] rituals and [keep] religious vows, [but eat meat] will not [be able to] enjoy even a tiny portion of [such rewards that] [those] people [receive] who have given up meat. \blankfootnote{3.37 As for \textit{pāda} d, see a similarly phrased comparison in \MANU\ 2.86:
  
 
  \textit{ye pākayajñās catvāro vidhiyajñasamanvitāḥ\thinspace |
  sarve te japayajñasya kalāṃ nārhanti ṣoḍaśīm\thinspace ||}.
 }}

  \maintext{mṛgāḥ parṇatṛṇāhārād ajameṣagavādibhiḥ |}%

  \maintext{sukhino balavantaś ca vicaranti mahītale }||\thinspace3:38\thinspace||%
\translation{Deer and goats, sheep, cows and other [animals] wander in the world happily and in great strength [just] from eating leaves and grass. }

  \maintext{vānarāḥ phala{-}m{-}āhārā rākṣasā rudhirapriyāḥ |}%

  \maintext{nihatā rākṣasāḥ sarve vānaraiḥ phalabhojibhiḥ }||\thinspace3:39\thinspace||%
\translation{Monkeys eat fruits, Rākṣasas prefer blood. The fruit-eating monkeys defeated all the Rākṣasas. \blankfootnote{3.39 Understand \textit{phalam āhārā} as \textit{phalāhārā} {\rm (}\textit{-m-} is a sandhi-bridge{\rm )}.
 This verse clearly refers to the story of the \textit{Rāmāyaṇa}.
 }}

  \maintext{tasmān māṃsaṃ na hīheta balakāmena bho dvija |}%

  \maintext{balena ca guṇākarṣāt parato bhayabhīruṇā }||\thinspace3:40\thinspace||%
\translation{Therefore one should not crave meat in the hope of gaining strength, O Brahmin, in order to be able to draw a bow with force, or out of fear of the danger coming from the enemy. \blankfootnote{3.40 \textit{guṇākāśāt} in pāda c is difficult to interpret and 
  \textit{guṇākarṣāt} is a conjecture by Judit Törzsök which fits the context well,
  although the polysemy of \textit{guṇa} may allow for other solutions.
  
 
  Verses 3.40--42 may be echoing \BRAHMANDAPUR\ 216.64--66:
  
 
  \textit{ māṃsān miṣṭataraṃ nāsti bhakṣyabhojyādikeṣu ca\thinspace |
  tasmān māṃsaṃ na bhuñjīta nāsti miṣṭaiḥ sukhodayaḥ\thinspace || 
  gosahasraṃ tu yo dadyād yas tu māṃsaṃ na bhakṣayet\thinspace |
  samāv etau purā prāha brahmā vedavidāṃ varaḥ\thinspace ||
  sarvatīrtheṣu yat puṇyaṃ sarvayajñeṣu yat phalam\thinspace |
  amāṃsabhakṣaṇe viprās tac ca tac ca ca tatsamam\thinspace ||}.
 }}

  \maintext{ahiṃsakasamo nāsti dānayajñasamīhayā |}%

  \maintext{iha loke yaśaḥ kīrtiḥ paratra ca parā gatiḥ }||\thinspace3:41\thinspace||%
\translation{By wishing to make donations and perform sacrifices no one will become comparable to someone who refrains from violence. [He will have] fame and glory in this world and the supreme path in the other. \blankfootnote{3.41 \textit{Pāda}s ab are reminescent of \SDHS\ 11.92: 
 
  \textit{ahiṃsaikā paro dharmaḥ śaktānāṃ parikīrtitam\thinspace |
  aśaktānām ayaṃ dharmo dānayajñādipūrvakaḥ\thinspace ||}. 
  On this verse see also \mycitep{SaivaUtopia2021}{15--16}.
 
  Note the variant \textit{°dharma°} in both \msCc\ and \Ed\ in \textit{pāda} b.
 }}

  \maintext{trailokyaṃ maṇiratnapūrṇam akhilaṃ dattvottame brāhmaṇe}%

 \nonanustubhindent \maintext{koṭīyajñasahasrapadmam ayutaṃ dattvā mahīṃ dakṣiṇām |}%

  \maintext{tīrthānāṃ ca sahasrakoṭiniyutaṃ snātvā sakṛn mānava}%

 \nonanustubhindent \maintext{etatpuṇyaphalam ahiṃsakajanaḥ prāpnoti niḥsaṃśayaḥ }||\thinspace3:42\thinspace||%
\translation{A person who refrains from violence will gain, no doubt about it, the [same] meritorious rewards that others would get by donating the three worlds filled with jewels and gems in their entirety to an excellent Brahmin, by [performing] a thousand [times] ten trillion {\rm (}\textit{padma}{\rm )} [times] ten thousand {\rm (}\textit{ayuta}{\rm )} \textit{koṭīyajña} sacrifices, by donating the earth [to a priest] as sacrificial fee, and by bathing [at] a thousand times ten million times a million {\rm (}\textit{niyuta}{\rm )} sacred places at once. \blankfootnote{3.42 Metre: \textit{śārdūlavikrīḍita}. 
  On \textit{padma} meaning `ten trillion', and on other words for numbers, see 1.31--35. 
  \textit{koṭīyajña} in \textit{pāda} d may refer to a special kind of sacrifice, 
  mostly known as \textit{koṭihoma} in the Purāṇas and in inscriptions 
  {\rm (}see, e.g., \mycitep{Fleming2010}{and 2013}\nocite{Fleming2013}{\rm )}.
  It involves a hundred fire-pits 
  and a hundred times one thousand Brahmins {\rm (}hence the name `the ten-million sacrifice'{\rm )}.
  See, e.g., \BHAVP\ \textit{uttaraparvan} 4.142.54--58:
  \textit{śatānano daśamukho dvimukhaikamukhas tathā\thinspace |
  caturvidho mahārāja koṭihomo vidhīyate\thinspace || 
  kāryasya gurutāṃ jñātvā naiva kuryād aparvaṇi\thinspace |
  yathā saṃkṣepataḥ kāryaḥ koṭihomas tathā śṛṇu\thinspace ||
  kṛtvā kuṇḍaśataṃ divyaṃ yathoktaṃ hastasaṃmitam\thinspace |
  ekaikasmiṃs tataḥ kuṇḍe śataṃ viprān niyojayet\thinspace ||
  sadyaḥ pakṣe tu viprāṇāṃ sahasraṃ parikīrtitam\thinspace |
  ekasthānapraṇīte 'gnau sarvataḥ paribhāvite\thinspace || 
  homaṃ kuryur dvijāḥ sarve kuṇḍe kuṇḍe yathoditam\thinspace |
  yathā kuṇḍabahutve 'pi rājasūye mahākratau\thinspace ||.}
  
 
 
  Note that the second syllable of \textit{phalam} in \textit{pāda} d is treated as long: this
  happens often at word-boundaries in this text; and 
  note how \msNc\ aims to restore the metre by inserting \textit{tv} after its \textit{phalaṃ}.
 }}

\centerline{\maintext{\dbldanda\thinspace iti vṛṣasārasaṃgrahe ahiṃsāpraśaṃsā nāmādhyāyas{ }tṛtīyaḥ\thinspace\dbldanda}}
\translation{Here ends the third chapter in the \textit{Vṛṣasārasaṃgraha} called the Praise of Non-violence.}
