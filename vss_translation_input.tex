
  \chptr{prathamo 'dhyāyaḥ}
\addcontentsline{toc}{subsection}{Chapter 1}
\fancyhead[CO]{{\footnotesize\textit{Translation of chapter 1}}}%

  \trchptr{Chapter One}%

  \subchptr{stutiḥ}%

  \trsubchptr{Invocation}%

  \maintext{anādimadhyāntam anantapāraṃ}%

 \nonanustubhindent \maintext{susūkṣmam avyaktajagatsusāram |}%

  \maintext{harīndrabrahmādibhir āsamagraṃ}%

 \nonanustubhindent \maintext{praṇamya vakṣye vṛṣasārasaṃgraham }||\thinspace1:1\thinspace||%
\translation{Having bowed to the One who has no beginning, no middle part and no end, whose boundaries are limitless, who is very subtle and who is the unmanifest and fine essence of the world, to the One who is wholly complete with Hari, Indra, Brahmā and the other [gods], I shall recite [the work called] `A Compendium on the Essence of the Bull [of Dharma]'. \blankfootnote{1.1 \textit{Pāda} a is reminiscent of, among other famous passages, \BHG\ 11.19:
  %
 \textit{anādimadhyāntam anantavīryam  
 anantabāhuṃ śaśisūryanetram\thinspace |  
 paśyāmi tvāṃ dīptahutāśavaktraṃ  
 svatejasā viśvam idaṃ tapantam\thinspace ||}.
  %
 See also \BHG\ 10.20cd:
  %
 \textit{aham ādiś ca madhyaṃ ca bhūtānām anta eva ca}\thinspace ||.
  %
 
 A faint reference to the \BHG\ seems proper at the
 beginning of a work that claims to deliver a teaching 
 based on, but also to surpass, the \MBH\ {\rm (}see following verses of the \VSS{\rm )}.
 Compare also, e.g., \KURMP\ 1.11.237:
  %
  \textit{rūpaṃ tavāśeṣakalāvihīnam 
  agocaraṃ nirmalam ekarūpam\thinspace |  
  anādimadhyāntam anantam ādyaṃ  
  namāmi satyaṃ tamasaḥ parastāt}\thinspace ||.
  %
 To say that a god has no beginning and no end in a temporal or spacial sense is natural
 {\rm (}\textit{anādi°...°antam}{\rm )}, but to have no `middle part' {\rm (}\textit{°madhya°}{\rm )} in these senses is slightly less so.
 Thus the rather commonly occuring phrase \textit{anādimadhyāntam} is probably a fixed expression usually 
 referring to a deity that is endless, eternal and immaterial. 
 As to which deity or what form of a deity this stanza refers to, 
 it may be Śiva, his name missing in pāda c, but the phrasing of the verse 
 is vague enough to keep the question somewhat open: the impersonal Brahman 
 might be another option, even more so if we look at verses 1.9--10, whose
 topic is \textit{brahmavidyā}.
 
 
  %
 In \textit{pāda} b \textit{jagat-susāraṃ} is most probably not 
 to be interpreted as \textit{jagatsu sāraṃ} {\rm (}`the essence in the worlds'{\rm )}.
 Another way to translate \textit{avyaktajagatsusāraṃ} would be: 
 `who is the fine essence of the unmanifest world.'
 
  %
 Strictly speaking, \textit{pāda} c is unmetrical, but it is better to 
 simply acknowledge here the phenomenon of `muta cum liquida', namely
 that syllables followed by consonant clusters such as 
 \textit{ra, bra, hra, kra, śra, śya, śva, sva, dva} can be treated as short {\rm (}\textit{laghu}{\rm )}.
 {\rm (}See Introduction \CHECK{\rm )}
 Thus \textit{harīndrabrahmā°} can be treated as a regular beginning
 of an \textit{upajāti} {\rm (}\shortsyllable\ - \shortsyllable\ - -{\rm )}, the syllable 
 \textit{bra} not turning the previous syllable long {\rm (}\textit{guru}{\rm )}.
 
  %
 The reading \textit{āsamagraṃ} in \textit{pāda} c is suspect,
 although the initial \textit{ā-} might convey some sort of
 completeness, meaning `all round'
 {\rm (}see e.g. \mycitep{KaleHigherGrammar}{226}{\rm )}.
 The fact that we could percieve
 the ending of \textit{pāda}s a and b {\rm (}\textit{pāraṃ}--\textit{sāram}{\rm )}, 
 as well as \textit{pāda}s c and d, as {\rm (}in the latter case, oddly{\rm )} rhyming pairs {\rm (}\textit{graṃ}-\textit{graham}{\rm )}
 suggests that accepting the reading
 \textit{āsamagram} could be the right decision
 {\rm (}as suggested by Alessandro Battistini{\rm )}.
 I translate this verse accordingly. \msM\ gives an exciting,
 albeit unmetrical, alternative {\rm (}\textit{yat samagraṃ}{\rm )}, but
 this seems more like a guess to me than the correct reading.
 For some time I was considering emending \textit{āsamagraṃ}.
 The most tempting of all the possible options 
 {\rm (}\textit{arcyam/arhyam/arghyam/īḍyam/āḍhyam agraṃ, āsamastaṃ}{\rm )} 
 seemed to be \textit{āptam agraṃ},
 meaning `appointed/received/respected [by Hari, Indra,
 Brahmā etc.] as the foremost one'. The fact that 
 the \textit{akṣara}s \textit{āsam} and \textit{āptam} look similar in most
 of the scripts used in our manuscripts could support this
 conjecture. \textit{āptam} could also
 possibly refer to the text itself, although then the
 syntax becomes slightly confusing: `I shall recite the
 \textit{Vṛṣasārasaṃgraha} that was first received by Hari...' etc.
 Another candidate was \textit{āḍhyam agram}:
 `Having bowed to [Him] who contains/is rich with Hari, Indra, Brahmā
 etc.' I have not emended the text because it is difficult
 to know if there is any need for change and if there is, which reading 
 to chose. There was no consensus when this verse was discussed 
 in our extended Śivadharma reading group.
 
  %
 Pāda d seems hypermetrical, but it can be interpreted as a \textit{vaṃśastha}
 line, a change from \textit{triṣṭubh} to \textit{jagatī} {\rm (}as suggested by Dominic Goodall{\rm )}.
 }}

  \subchptr{janamejayavaiśampāyanasaṃvādaḥ}%

  \trsubchptr{The dialogue of Janamejaya and Vaiśampāyana}%

  \maintext{śatasāhasrikaṃ granthaṃ sahasrādhyāyam uttamam |}%

  \maintext{parva cāsya śataṃ pūrṇaṃ śrutvā bhāratasaṃhitām }||\thinspace1:2\thinspace||%
\translation{Having listened to the \textit{Bhāratasaṃhitā} [i.e. the \textit{Mahābhārata}], the supreme book of a hundred thousand [verses] and a thousand chapters {\rm (}\textit{adhyāya}{\rm )}, with all its hundred sections {\rm (}\textit{parvan}{\rm )}, \blankfootnote{1.2 The dialogue of Janamejaya and Vaiśampāyana makes up the outermost layer of the \VSS\ 
  {\rm (}except for the introductory stanzas 1.1--3; see Introduction \CHECK{\rm )}, mostly containing
  general \textit{dharmaśāstric} material.
   %
  That the \MBH\ should contain a hundred thousand verses is hinted at e.g. in line 19 of
  the Khoh Charter 2 of Śarvanātha, year 214 {\rm (}Siddham IN00088: 
  \textit{uktañ ca mahābhārate śatasāhasryaṃ} {\rm (}understand °\textit{ryāṃ}{\rm )} \textit{saṃhitāyāṃ}... 
  The hundred \textit{parvan}s of the \textit{Mahābhārata} are listed in \MBH\ 1.2.33--70.
 }}

  \maintext{atṛptaḥ puna papraccha vaiśampāyanam eva hi |}%

  \maintext{janamejaya yat pūrvaṃ tac chṛṇu tvam atandritaḥ }||\thinspace1:3\thinspace||%
\translation{Janamejaya remained unsatisfied. Listen unweariedly to what he asked Vaiśampāyana in the past. \blankfootnote{1.3 My emendation from the unmetrical \textit{punaḥ} to the unusual, or rather, Middle Indic 
  {\rm (}\mycitep{EdgertonHybrid}{vol. 2, p. 347}{\rm )}, \textit{puna} is based
  on the assumption that in the original the metre must have overridden 
  morphology, similarily to what may have happened in 8.44d {\rm (}Mālinī metre{\rm )}:
  \textit{na bhavati punajanma kalpakoṭyāyute 'pi}, and in 12.151c {\rm (}Sragdharā metre{\rm )}:
  \textit{garbhāvāsaṃ na ca tvan na ca punamaraṇaṃ kleśam āyāsapūrṇam}.
 
   %
  For an unsatisfaction or dissatisfaction {\rm (}\textit{atṛpti}{\rm )} with previous 
  teachings in a somewhat similar manner to what
  Janamejaya experiences here, see e.g. \textit{Niśvāsa} mūla 1.9:
   %
  \textit{vedāntaṃ viditaṃ deva sāṃkhyaṃ vai pañcaviṃśakam\thinspace |
  na ca tṛptiṃ gamiṣyāmo hy ṛte śaivād anugrahāt\thinspace ||};
   %
  and the \SDhS\... \CHECK.
 Vaiśampāyana, a Ṛṣi, disciple of Vyāsa, great-grandson to Arjuna,
  recited the Mahābhārata at the snake sacrifice of 
  Janamejaya. This setting is an echo of the starting point of the Mahābhārata, see \MBH\ 1.1.8ff.
  In fact the next few verses in the \VSS\ make it clear that the \VSS\
  picks up where the Mahābhārata left off: Janamejaya has heard the whole Mahābhārata from
  Vaiśampāyana, but he is eager to hear more.
   %
  Note how we are forced to emend \textit{pāda} c to contain a stem form proper noun {\rm (}\textit{janamejaya}{\rm )}
  to maintain the metre, and note how the manuscripts struggle with this \textit{pāda}.
  Stem form nouns, \textit{prātipadika}s, abound in the \VSS: see Introduction p. \CHECK. 
 }}

  \maintext{janamejaya uvāca |}%

  \maintext{bhagavan sarvadharmajña sarvaśāstraviśārada |}%

  \maintext{asti dharmaṃ paraṃ guhyaṃ saṃsārārṇavatāraṇam }||\thinspace1:4\thinspace||%
\translation{Janamejaya spoke: O venerable sir, O knower of the entire Dharma, O you who are well-versed in all the sciences {\rm (}\textit{śāstra}{\rm )}! There is a supreme and secret Dharma [that causes] liberation from the ocean of mundane existence {\rm (}\textit{saṃsāra}{\rm )}. \blankfootnote{1.4 Note \textit{dharma} as a neuter noun in \textit{pāda} c and in the next verse.
 }}

  \maintext{dvaipāyanamukhodgīrṇaṃ dharmaṃ vā yad dvijottama |}%

  \maintext{kathayasva hi me tṛptiṃ kuru yatnāt tapodhana }||\thinspace1:5\thinspace||%
\translation{Teach me the Dharma that emerged from [Vyāsa] Dvaipāyana's mouth, O best of Brahmins. Help me find satisfaction at all cost, O great ascetic! \blankfootnote{1.5 The majority of the MSS consulted include a \textit{vā} in \textit{pāda} b, 
  and although \msCb's reading seems a bit smoother, that manuscript rarely gives superior readings.
  Therefore I have chosen \textit{dharmaṃ vā yad}, in which \textit{vā} functions probably in a weak sense.
  That the secret Dharma Janamejaya is seeking is the one taught by Vyāsa Dvaipāyana, and
  thus no real options are involved here, becomes clear in 1.6cd.
  The reading of \msM\ in \textit{pāda} b {\rm (}\textit{dharmavākyaṃ}{\rm )} is tempting but could be a later correction.
 In general, \msM's readings here are unique but probably secondary: \textit{hi me tṛptiṃ} in \textit{pāda} c seems more
  attractive than \msM's \textit{prasādena} because it echoes \textit{atṛptaḥ} in 1.3a
 }}

  \maintext{vaiśampāyana uvāca |}%

  \maintext{śṛṇu rājann avahito dharmākhyānam anuttamam |}%

  \maintext{vyāsānugrahasamprāptaṃ guhyadharmaṃ śṛṇotu me }||\thinspace1:6\thinspace||%
\translation{Vaiśampāyana spoke: Listen with great attention, O king, to this unsurpassed narration of Dharma. Hear the secret Dharma that I received by Vyāsa's favour. }

  \maintext{anarthayajñakartāraṃ tapovrataparāyaṇam |}%

  \maintext{śīlaśaucasamācāraṃ sarvabhūtadayāparam }||\thinspace1:7\thinspace||%
{\blankfootnote{1.7 Note the odd syntax here: \textit{viṣṇunā... dvijarūpadharo bhūtvā papraccha}.
  The agent of the active verb is in the instrumental case {\rm (}anacoluthic structure{\rm )}.
   %
  On Anarthayajña, the interlocutor of VSS 1.9--10.2 and 19.1--21.22, and
  an important figure discussed in 22.3ff, as well as a concept {\rm (}`nonmaterial sacrifice'{\rm )},
  see \mycite{KissVolume2021} and Introduction \CHECK.
 }}

  \maintext{jijñāsanārthaṃ praśnaikaṃ viṣṇunā prabhaviṣṇunā |}%

  \maintext{dvijarūpadharo bhūtvā papraccha vinayānvitaḥ }||\thinspace1:8\thinspace||%
\translation{Viṣṇu, the great Lord, assuming the form of a twice-born [Brahmin], wanted to test [Anarthayajña, the ascetic yogin] who performed nonmaterial sacrifices {\rm (}\textit{anarthayajña}{\rm )}, focused on his austerities and observances, whose conduct was virtuous and pure, and who was intent on compassion towards all living beings; therefore he [Viṣṇu] humbly asked him a question. }

  \subchptr{brahmavidyā}%

  \trsubchptr{The knowledge of Brahman}%

  \maintext{{\rm [}vigatarāga uvāca |{\rm ]}}%

  \maintext{brahmavidyā kathaṃ jñeyā rūpavarṇavivarjitā |}%

  \maintext{svaravyañjananirmuktam akṣaraṃ kimu tatparam }||\thinspace1:9\thinspace||%
\translation{[Vigatarāga spoke:] How is the knowledge of the Brahman to be understood if it is devoid of form and colour? The syllable that is devoid of vowels and consonants: is there anything higher than that? \blankfootnote{1.9 The translation of this verse, and the reconstruction and interpretation
  of \textit{pāda} d, which is echoed in 1.10d, is slightly tentative.
  I doubt if \textit{kimu} could have the standard {\rm (}Vedic{\rm )} meaning `how much more/less'
  here. Rather \textit{u} is probably just an expletive. In general it seems that
  this verse references the syllable \textit{oṃ}.
 }}

  \maintext{anarthayajña uvāca |}%

  \maintext{anuccāryam asandigdham avicchinnam anākulam |}%

  \maintext{nirmalaṃ sarvagaṃ sūkṣmam akṣaraṃ kimu tatparam }||\thinspace1:10\thinspace||%
\translation{Anarthayajña replied: That syllable is not to be pronounced, is unquestionable, non-dividable, consistent, spotless, all-pervading and subtle: what could be higher than that? }

  \subchptr{kālapāśaḥ}%

  \trsubchptr{The noose of death and time}%

  \maintext{vigatarāga uvāca |}%

  \maintext{dehī dehe kṣayaṃ yāte bhūjalāgniśivādibhiḥ |}%

  \maintext{yamadūtaiḥ kathaṃ nīto nirālambo nirañjanaḥ }||\thinspace1:11\thinspace||%
\translation{Vigatarāga spoke: When the body disintegrates in the ground, in water, in fire or [is torn apart] by jackals and other [animals], how is the supportless and spotless soul led [to the netherworld] by Yama's messengers? \blankfootnote{1.11 The word \textit{°śivā°} in \textit{pāda} b is slightly suspect, and could be the result
  of metathesis, from \textit{°viṣā°} {\rm (}`by poison'{\rm )}. Nevertheless, 
  jackals seems appropriate in this context, for they 
  are commonly associated with human corpses, death and the cremation ground
  {\rm (}see e.g. \mycite{Ohnuma2019}{\rm )}. Furthermore, \textit{pāda} b lists phenomena
  that cause the body to disintegrate, and not causes of death; thus the reading \textit{śiva}
  is probably correct.
 }}

  \maintext{kālapāśaiḥ kathaṃ baddho nirdehaś ca kathaṃ vrajet |}%

  \maintext{svargaṃ vā sa kathaṃ yāti nirdeho bahudharmakṛt |}%

  \maintext{etan me saṃśayaṃ brūhi jñātum icchāmi tattvataḥ }||\thinspace1:12\thinspace||%
\translation{How is it bound by the nooses of death/time? And if it is bodiless, how can it move? And how does the [soul of a] virtuous [person] {\rm (}\textit{bahudharmakṛt}{\rm )} reach heaven if it has no body? This is my doubt. Teach me. I want to know the truth. \blankfootnote{1.12 The word \textit{kāla} has, as usual, a double meaning here: \textit{kālapāśa}
  is both Yama's noose, and also the limitations and bondage caused by time, 
  as becomes clear at the discussion on the different time units in verses 1.18--31.
 }}

  \maintext{anarthayajña uvāca |}%

  \maintext{atisaṃśayakaṣṭaṃ te pṛṣṭo 'haṃ dvijasattama |}%

  \maintext{durvijñeyaṃ manuṣyais tu devadānavapannagaiḥ }||\thinspace1:13\thinspace||%
\translation{Anarthayajña spoke: You are asking me about an extremely doubtful and problematic matter, O truest of the twice-born. [This is something that] is difficult to understand by humans, and [even] by gods {\rm (}\textit{deva}{\rm )}, demons {\rm (}\textit{dānava}{\rm )} and serpents {\rm (}\textit{pannaga}{\rm )}. \blankfootnote{1.13 Note \textit{te} used for \textit{tvayā} in \textit{pāda} a. Alternatively, taking \textit{te} as genitive, the line
  could be translatied as: `I am being asked about a great 
  problem of yours that originates in doubts\dots'
 }}

  \maintext{karmahetuḥ śarīrasya utpattir nidhanaṃ ca yat |}%

  \maintext{sukṛtaṃ duṣkṛtaṃ caiva pāśadvayam udāhṛtam }||\thinspace1:14\thinspace||%
\translation{The cause of both the birth and death of the body is karma. Good and bad deeds are called the two nooses. \blankfootnote{1.14 The MSS give \textit{karmahetu} in \textit{pāda} a overwhelmingly, which could work as a neuter
  \textit{bahuvrīhi} compound picking up both \textit{utpattir} and \textit{nidhanaṃ} but \textit{karmahetuḥ} is
  grammatically more correct, picking up the feminine \textit{utpatti}.
  I suspect that there may have been a confusion,
  scribes taking \textit{karmahetuśarīrasya} as one single compound; but this would make
  it difficult to interpret the verse.
 }}

  \maintext{tenaiva saha saṃyāti narakaṃ svargam eva vā |}%

  \maintext{sukhaduḥkhaṃ śarīreṇa bhoktavyaṃ karmasambhavam }||\thinspace1:15\thinspace||%
\translation{[The soul] goes to hell or heaven accordingly. Happiness and suffering, both arising from karma, are to be experienced by the body. }

  \maintext{hetunānena viprendra dehaḥ sambhavate nṛṇām |}%

  \maintext{yaṃ kālapāśam ity āhuḥ śṛṇu vakṣyāmi suvrata }||\thinspace1:16\thinspace||%
\translation{It is for this reason, O great Brahmin, that the human body is born. Now learn about that which they call the noose of time, I shall teach you, O you of great observances. }

  \maintext{na tvayā viditaṃ kiñcij jijñāsyasi kathaṃ dvija |}%

  \maintext{kālapāśaṃ ca viprendra sakalaṃ vettum arhasi }||\thinspace1:17\thinspace||%
\translation{[If] you don't know anything, how could you start your investigation, O twice-born? O great Brahmin, you should know the noose of time in its entirety. \blankfootnote{1.17 The variant \textit{jijñāsyasi} seems to be the lectio difficilior as opposed to
  \textit{vijñāsyasi}, but the latter could also work fine here.
 Note how \msM\ {\rm (}agreeing with \Ed{\rm )} gives a reading {\rm (}\textit{vaktum arhasi}{\rm )} that is clearly out
  of context. This confirms that while \msM\ comes up with interesting readings, 
  they are mostly to be ignored.
 }}

  \maintext{kalākalitakālaṃ ca kālatattvakalāṃ śṛṇu |}%

  \maintext{truṭidvayaṃ nimeṣas tu nimeṣadviguṇā kalā }||\thinspace1:18\thinspace||%
\translation{Learn about time {\rm (}\textit{kāla}{\rm )} which is divided into digits {\rm (}\textit{kalā}{\rm )}, [i.e. about] the division[s] {\rm (}\textit{kalā}{\rm )} of the entity [called] time {\rm (}\textit{kālatattva}{\rm )}. Two atomic units of time {\rm (}\textit{truṭi}{\rm )} is one twinkling {\rm (}\textit{nimeṣa}{\rm )}. One digit {\rm (}\textit{kalā}, cca. 1.6 second{\rm )} is twice a twinkling. \blankfootnote{1.18 1.18d and 1.19a are problematic in the light of 1.19b, which 
  redefines \textit{kalā} in harmony with the traditional
  interpretation, see e.g. \textit{Arthaśāstra} 2.20.33: \textit{trimśatkāṣṭhāḥ kalāḥ}.
  \nocite{Arthasastra1969}
  On divisions of time, see also, e.g., \MANU\ 1.64ff.
  I have calculated 1.6 second for one \textit{kalā} backwards, starting from one day {\rm (}see 1.20ab{\rm )}.
 }}

  \maintext{kalādviguṇitā kāṣṭhā kāṣṭhā vai triṃśatiḥ kalā |}%

  \maintext{triṃśatkalā muhūrtaś ca mānuṣena dvijottama }||\thinspace1:19\thinspace||%
\translation{Two digits {\rm (}\textit{kalā}{\rm )} form one bit {\rm (}\textit{kāṣṭhā}, 3.2 seconds{\rm )}. Thirty bits {\rm (}\textit{kāṣṭhā}{\rm )} is one digit {\rm (}\textit{kalā}?, 1.6 minutes{\rm )}. Thirty digits {\rm (}\textit{kalā}{\rm )} make up one section {\rm (}\textit{muhūrta}, 48 minutes{\rm )} in human terms, O great Brahmin. }

  \maintext{muhūrtatriṃśakenaiva ahorātraṃ vidur budhāḥ |}%

  \maintext{ahorātraṃ punas triṃśan māsam āhur manīṣiṇaḥ }||\thinspace1:20\thinspace||%
\translation{Thirty sections {\rm (}\textit{muhūrta}{\rm )} are known to the wise as night and day [i.e. a full day]. Thirty days and nights are taught by the wise to be one month. }

  \maintext{samā dvādaśa māsāś ca kālatattvavido janāḥ |}%

  \maintext{śataṃ varṣasahasrāṇi trīṇi mānuṣasaṃkhyayā }||\thinspace1:21\thinspace||%
\translation{One year is twelve months [according to] people who know the entity of time. The time span of three hundred \blankfootnote{1.21 Note how a verb {\rm (}e.g. \textit{iti vadanti, iti prāhur}{\rm )} is missing in the first half-verse.
 }}

  \maintext{ṣaṣṭiṃ caiva sahasrāṇi kālaḥ kaliyugaḥ smṛtaḥ |}%

  \maintext{dviguṇaḥ kalisaṃkhyāto dvāparo yuga saṃjñitaḥ }||\thinspace1:22\thinspace||%
\translation{and sixty thousand years by human terms is said to be the Kali age {\rm (}\textit{yuga}{\rm )}. The Dvāpara age is known to be twice as long as the Kali age. \blankfootnote{1.22 Note the stem form noun \textit{yuga} metri causa, and also \msM's unique but confused readings.
 }}

  \maintext{tretā tu triguṇā jñeyā catuḥ kṛtayugaḥ smṛtaḥ |}%

  \maintext{eṣā caturyugā saṃkhyā kṛtvā vai hy ekasaptatiḥ }||\thinspace1:23\thinspace||%
\translation{The Tretā age is thrice [as long], the Kṛta age four [times as long as the Kali age]. This is the figure related to the four ages {\rm (}\textit{yuga}{\rm )}. Taking it seventy-one [times], \blankfootnote{1.23 The `figure' mentioned in this verse is the sum of the duration of the four \textit{yuga}s, 
  which makes up one \textit{mahāyuga}:
  Kaliyuga = 360,000 years,
  Dvāparayuga = 720,000 years,
  Tretāyuga = 1,080,000 years,
  Dvāparayuga = 1,440,000 years; altogether 3,600,000 years. 72 \textit{mahāyuga}s make up
  a \textit{manvantara} {\rm (}= 259,200,000 years{\rm )}. One \textit{kalpa} is 14 \textit{manvantara}s {\rm (}= 3,628,800,000 years{\rm )}. 
  Ten thousand \textit{kalpa}s are one day of Brahmā, and his night is of the same length, which
  makes one full day of Brahmā 72,576,000,000,000 years. See next verses.
 }}

  \maintext{manvantarasya caikasya jñānam uktaṃ samāsataḥ |}%

  \maintext{kalpo manvantarāṇāṃ tu caturdaśa tu saṃkhyayā }||\thinspace1:24\thinspace||%
\translation{the knowledge about one time-span of a Manu {\rm (}\textit{manvantara}{\rm )} has been taught briefly. One aeon {\rm (}\textit{kalpa}{\rm )} is fourteen \textit{manvantara}s in total. \blankfootnote{1.24 See 21.34ff.
 }}

  \maintext{daśa kalpasahasrāṇi brahmāhaḥ parikalpitam |}%

  \maintext{rātrir etāvatī proktā munibhis tattvadarśibhiḥ }||\thinspace1:25\thinspace||%
\translation{Brahmā's day {\rm (}\textit{brahmāhar}{\rm )} is made up of ten thousand Kalpas. [Brahmā's] night is of the same [duration] according to the wise who know the truth. \blankfootnote{1.25 \msM\ has a separator sign {\rm (}|o|{\rm )} at the end of \textit{pāda} b, as if a section ended here.
 }}

  \maintext{rātryāgame pralīyante jagat sarvaṃ carācaram |}%

  \maintext{ahāgame tathaiveha utpadyante carācaram }||\thinspace1:26\thinspace||%
\translation{When [Brahmā's] night falls, the whole moving and unmoving universe dissolves. And when [his] daylight comes, the moving and unmoving [universe] is born. \blankfootnote{1.26 The plural form \textit{pralīyante} in \textit{pāda} a is metri causa for \textit{pralīyate},
  perhaps also influencing \textit{utpadyante} {\rm (}for \textit{utpadyate}{\rm )} in \textit{pāda} d,
  which in turn is used here to avoid an iambic pattern
  {\rm (}- - \shortsyllable\ - \shortsyllable\ - \shortsyllable\ -{\rm )}.
 }}

  \maintext{parārdhaparakalpāni atītāni dvijottama |}%

  \maintext{anāgataṃ tathaivāhur bhṛgurādimaharṣayaḥ }||\thinspace1:27\thinspace||%
\translation{One \textit{para} times \textit{parārdha} [number of, i.e. two hundred quadrillion times a hundred quadrillion] \textit{kalpas} have passed [so far], O great Brahmin. Bhṛgu and the other sages say that the future is the same [time span]. \blankfootnote{1.27 On the definition of the numbers \textit{para} and \textit{parārdha}, see verses 1.32--36.
 Note the peculiar compound \textit{bhṛgu-r-ādi-maharṣayaḥ}.
 }}

  \maintext{yathārkagrahatārendu bhramato dṛśyate tv iha |}%

  \maintext{kālacakraṃ bhramatvaiva viśramaṃ na ca vidmahe }||\thinspace1:28\thinspace||%
\translation{Just as the sun, the planets, the stars and the moon are percieved in this world as wandering around, the wheel of time {\rm (}\textit{kālacakra}{\rm )} keeps spinning and we never experience its halting. \blankfootnote{1.28 \textit{bhramato} {\rm (}gen.{\rm )} in \textit{pāda} b seems to stand for the neuter participle \textit{bhramat}.
  Alternatively, \textit{bhramato} might mean `erroneously' {\rm (}\textit{brama-tas}, abl.{\rm )}, but this
  makes the verse difficult to interpret.
 }}

  \maintext{kālaḥ sṛjati bhūtāni kālaḥ saṃharate punaḥ |}%

  \maintext{kālasya vaśagāḥ sarve na kālavaśakṛt kvacit }||\thinspace1:29\thinspace||%
\translation{Time creates living beings and time destroys them again. Everything is under the control of time. There is nothing that can bring time under control. }

  \maintext{caturdaśaparārdhāni devarājā dvijottama |}%

  \maintext{kālena samatītāni kālo hi duratikramaḥ }||\thinspace1:30\thinspace||%
\translation{Fourteen \textit{parārdha} [fourteen hundred quadrillion] god kings, O Brahmin, have passed by over time, for time is difficult to overcome. \blankfootnote{1.30 Note that \textit{samatītāni} {\rm (}neuter{\rm )} most probably picks up \textit{devarājāḥ}
  {\rm (}masculine{\rm )} in this verse, or rather \textit{devarājā} stands for
  \textit{devarājānāṃ} and \textit{samatītāni} picks up \textit{°parārdhāni}. It is not clear to me
  what \textit{devarāja} {\rm (}`god king'{\rm )} means exactly {\rm (}Indra?{\rm )}.
 }}

  \maintext{eṣa kālo mahāyogī brahmā viṣṇuḥ paraḥ śivaḥ |}%

  \maintext{anādinidhano dhātā sa mahātmā namaskuru }||\thinspace1:31\thinspace||%
\translation{Time is [manifest] as a great yogin, as Brahmā, Viṣṇu and supreme Śiva, is beginningless and endless, is the creator, the great soul. Pay homage [to Time]. }

  \subchptr{parārdhādi}%

  \trsubchptr{The \textit{parārdha} etc.: numbers}%

  \maintext{vigatarāga uvāca |}%

  \maintext{śrutaṃ vai kālacakraṃ tu mukhapadmaviniḥsṛtam |}%

  \maintext{parārdhaṃ ca paraṃ caiva śrotuṃ vaḥ pratidīpitam }||\thinspace1:32\thinspace||%
\translation{Vigatarāga spoke: I have just heard [the term] `wheel of time' {\rm (}\textit{kālacakra}{\rm )} uttered from [your] lotus mouth, as well as \textit{parārdha} and \textit{para}. You have made these things appear as exciting, as things to hear. \blankfootnote{1.32 The reading of all manuscripts consulted, \textit{vinisṛtam}, 
  may be considered metrical if we interpret it, loosely, as \textit{vinisritam}.
   %
  \textit{Pāda} d is suspect and my translation is tentative.
  \msM's reading in \textit{pāda} d {\rm (}\textit{srotuṃ naḥ pratidīyatāṃ}{\rm )} might make sense 
  {\rm (}`give it back/repeat it for us again'{\rm )}, but it sounds forced,
  as if the scribe tried to come up with a reading that he understood
  better than \textit{srotuṃ vaḥ pratidīpitam}, which is in fact not easy to interpret.
  One would expect a phrase meaning `please tell me about these.'
 }}

  \maintext{anarthayajña uvāca |}%

  \maintext{ekaṃ daśaṃ śataṃ caiva sahasram ayutaṃ tathā |}%

  \maintext{prayutaṃ niyutaṃ koṭim arbudaṃ vṛndam eva ca }||\thinspace1:33\thinspace||%
\translation{Anarthayajña spoke: One, ten, a hundred, a thousand, and ten thousand {\rm (}\textit{ayuta}{\rm )}, a hundred thousand {\rm (}\textit{prayuta}{\rm )}, a million {\rm (}\textit{niyuta}{\rm )}, ten million {\rm (}\textit{koṭi}{\rm )}, a hundred million {\rm (}\textit{arbuda}{\rm )}, and one billion {\rm (}\textit{vṛnda}, 10\raise .5em\hbox{\footnotesize 9\thinspace}{\rm )}, \blankfootnote{1.33 See a similar teaching of numbers in \BRAHMANDAPUR\ 3.2.91ff.
 }}

  \maintext{kharvaṃ caiva nikharvaṃ ca śaṅkuḥ padmaṃ tathaiva ca |}%

  \maintext{samudro madhyam antaṃ ca parārdhaṃ ca paraṃ tathā }||\thinspace1:34\thinspace||%
\translation{ten billion {\rm (}\textit{kharva}{\rm )}, a hundred billion {\rm (}\textit{nikharva}{\rm )}, one trillion {\rm (}\textit{śaṅku}, 10\raise .5em\hbox{\footnotesize 12\thinspace}{\rm )}, and ten trillion {\rm (}\textit{padma}{\rm )}, a hundred trillion {\rm (}\textit{samudra}{\rm )}, one quadrillion {\rm (}\textit{madhya}, 10\raise .5em\hbox{\footnotesize 15\thinspace}{\rm )}, ten quadrillion {\rm (}\textit{[an]anta}{\rm )}, a hundred quadrillion {\rm (}\textit{parārdha}{\rm )}, and two hundred quadrillion {\rm (}\textit{para}{\rm )}. \blankfootnote{1.34 For \textit{anta} meaning \textit{ananta}, see 1.58cd--59ab. \msM's reading in \textit{pāda} d
  may be a result of an eyeskip to 1.35c.
 }}

  \maintext{sarve daśaguṇā jñeyāḥ parārdhaṃ yāvad eva hi |}%

  \maintext{parārdhadviguṇenaiva parasaṃkhyā vidhīyate }||\thinspace1:35\thinspace||%
\translation{Each should be known as powers of ten up to \textit{parārdha}. The number corresponding to \textit{para} is double that of \textit{parārdha}. }

  \maintext{parāt parataraṃ nāsti iti me niścitā matiḥ |}%

  \maintext{purāṇavedapaṭhitā mayākhyātā dvijottama }||\thinspace1:36\thinspace||%
\translation{There is no higher number than \textit{para}. This is my firm conviction, which is based on my readings of the Purāṇas and the Vedas and [which I have now] taught [to you], O great Brahmin. \blankfootnote{1.36 Note that \Ed, after omitting three lines, inserts this: \textit{vṛndañ caiva mahāvṛnda dviparānantam eva ca}.
 }}

  \subchptr{brahmāṇḍam}%

  \trsubchptr{Brahmā's Egg}%

  \maintext{vigatarāga uvāca |}%

  \maintext{brahmāṇḍaṃ kati vijñeyaṃ pramāṇaṃ prāpitaṃ kvacit |}%

  \maintext{kati cāṅguli{-}m{-}ūrdhveṣu sūryas tapati vai mahīm }||\thinspace1:37\thinspace||%
\translation{Vigatarāga spoke: How many eggs of Brahmā are there? And are its measurements available anywhere? From how many finger's breadths high does the sun heat the earth? \blankfootnote{1.37 The use of the singular next to numerals is one of the hallmarks of the \VSS\ 
  {\rm (}see p. \verify{\rm )}. As an introduction to this phenomenon, \textit{pāda} a has
  \textit{brahmāṇḍaṃ} in the singular where we would expect a plural form.
  The word \textit{prāpitaṃ} is a conjecture for \textit{cāpitaṃ}, which I find unintelligible. 
  Another possibility could be \textit{jñāpitaṃ}.
 My emendation of \textit{cāṅguli-mūrdheṣu} to \textit{cāṅguli{-}m{-}ūrdhveṣu} {\rm (}with a hiatus filler{\rm )} 
  is based on \textit{ūrdhvatas} in 1.61d, which is part of the reply to the question posed in this line.
  In turn, \textit{aṅguli} here triggered an conjecture in 1.61c.
 }}

  \maintext{anarthayajña uvāca |}%

  \maintext{brahmāṇḍānāṃ prasaṃkhyātuṃ mayā śakyaṃ kathaṃ dvija |}%

  \maintext{devās te 'pi na jānanti mānuṣāṇāṃ ca kā kathā }||\thinspace1:38\thinspace||%
\translation{Anarthayajña spoke: How could I enumerate [all] the eggs of Brahmā, O twice-born? Even the gods don't know [all the details], not to mention humans. \blankfootnote{1.38 One would expect \textit{brahmāṇḍāni} in \textit{pāda} a instead of \textit{brahmāṇḍānāṃ},
  but we should probably understand \textit{brahmāṇḍānāṃ viśeṣān prasaṃkhyātuṃ...}
  The structure noun in genitive + verb meaning 'telling' occurs also in 4.69a and \verify .
 }}

  \maintext{paryāyeṇa tu vakṣyāmi yathāśakyaṃ dvijottama |}%

  \maintext{brahmaṇā yat purākhyāto mātariśvā yathā tathā }||\thinspace1:39\thinspace||%
\translation{I shall teach [you], as far as I can, in due order and truthfully, that, O great Brahmin, which Mātariśvan was taught by Brahmā in the past. \blankfootnote{1.39 The claim that Brahmā taught Mātariśvan is confirmed in 1.64cd, and
  also, e.g., in \BrahmandaPur\ 3.4.58cd {\rm (}see the apparatus{\rm )}.
 }}

  \maintext{śivāṇḍābhyantareṇaiva sarveṣām iva bhūbhṛtām |}%

  \maintext{daśa nāma diśāṣṭānāṃ brahmāṇḍe kīrtitaṃ śṛṇu }||\thinspace1:40\thinspace||%
\translation{Ten names of all the [cosmic] rulers of each of the eight directions in Brahmā's Egg, [which is] inside Śiva's Egg, are being taught now, listen. \blankfootnote{1.40 My conjecture in \textit{pāda} b {\rm (}\textit{bhūbhṛtām}{\rm )} is based on the fact that the 
  readings transmitted in the MSS seem unintelligible and, more importantly, that
  these names are said to belong to \textit{nāyaka}s in the subsequent verses,
  a possible synonym of \textit{bhūbhṛt} {\rm (}`a king'{\rm )}, and also that it is a minute intervention.
  In \textit{pāda} c, understand \textit{diśāṣṭānāṃ} as \textit{diśām aṣṭānāṃ} or \textit{digaṣṭakānāṃ}, 
  and note that one of the hallmarks of the language of the \VSS\ is the use
  of the singular in the proximity of numbers, where a plural would be expected {\rm (}\textit{daśa nāma}{\rm )}.
 }}

  \subchptr{bhūbhṛtāṃ nāmāni}%

  \trsubchptr{The names of the cosmic rulers}%

  \subsubchptr{pūrvataḥ}%

  \trsubsubchptr{East}%

  \maintext{sahāsahaḥ sahaḥ sahyo visahaḥ saṃhato 'sabhā |}%

  \maintext{prasaho 'prasahaḥ sānuḥ pūrvato daśa nāyakāḥ }||\thinspace1:41\thinspace||%
\translation{[1] Sahā, [2] Asaha, [3] Saha, [4] Sahya, [5] Visaha, [6] Saṃhata, [7] Asabhā, [8] Prasaha, [9] Aprasaha, [10] Sānu: [these are] the ten Leaders in the East. \blankfootnote{1.41 Note that many of the names here and in the following verses are,
  in the absence of any parallel passage, rather insecure.
  In order to avoid the repetition of the name Saha, I take the first name here
  as feminine; Asabhā seems also to be a feminine ruler's name. Later on there
  seem to come more feminine names {\rm (}Tejā, Yamunā, Naganā, etc.{\rm )}, therefore it 
  may be correct to interpret some of the names as names of queens.
  What is clear here is that the list evokes the name Sahasrākṣa,
  one of the appellations of Indra, the guardian of the eastern direction.
 }}

  \subsubchptr{āgneye}%

  \trsubsubchptr{South-East}%

  \maintext{prabhāso bhāsano bhānuḥ pradyoto dyutimo dyutiḥ |}%

  \maintext{dīptatejāś ca tejāś ca tejā tejavaho daśa |}%

  \maintext{āgneye tv etad ākhyātaṃ yāmye śṛṇv atha bho dvija }||\thinspace1:42\thinspace||%
\translation{[1] Prabhāsa, [2] Bhāsana, [3] Bhānu, [4] Pradyota, [5] Dyutima, [6] Dyuti, [7] Dīptatejas, [8] Tejas, [9] Tejā, [10] Tejavaho: [these are] the ten [rulers] in the direction of Agni [SE]. Now listen to [the names for] the Yama's region, O twice-born. \blankfootnote{1.42 Here, in the region of Agni, the names evidently evoke the image of flames.
 }}

  \subsubchptr{yāmye}%

  \trsubsubchptr{South}%

  \maintext{yamo 'tha yamunā yāmaḥ saṃyamo yamuno 'yamaḥ |}%

  \maintext{saṃyano yamanoyāno yaniyugmā yanoyanaḥ }||\thinspace1:43\thinspace||%
\translation{[1] Yama, [2] Yamunā, [3] Yāma, [4] Saṃyama, [5] Yamuna, [6] Ayama, [7] Saṃyana, [8] Yamanoyāna, [9] Yaniyugmā, [10] Yanoyana. \blankfootnote{1.43 I have choosen the variant \textit{saṃyano} in \textit{pāda} c only to avoid the repetition of
  the name \textit{saṃyama}, and the variant \textit{yanoyanaḥ} in \textit{pāda} d because I suspect that
  most of the names here should begin with \textit{ya}. All the name forms
  in this verse are to be taken as tentative. The only 
  guiding light is the presence of \textit{ya}, reinforcing a connection with Yama.
 }}

  \subsubchptr{nairṛte}%

  \trsubsubchptr{South-West}%

  \maintext{nagajo naganā nando nagaro naga nandanaḥ |}%

  \maintext{nagarbho gahano guhyo gūḍhajo daśa tatparaḥ }||\thinspace1:44\thinspace||%
\translation{[1] Nagaja, [2] Naganā, [3] Nanda, [4] Nagara, [5] Naga, [6] Nandana, [7] Nagarbha, [8] Gahana, [9] Guhyo, [10] Gūḍhaja: [these are] the ten associated with [the South-West]. \blankfootnote{1.44 \textit{naga} in \textit{pāda} b is a stem form noun metri causa
 \textit{tatparaḥ} in \textit{pāda} d might be another example of a singular form next to a number {\rm (}see 1.40c above{\rm )}.
  Note that the reconstruction of these names are tentative. What is clear here is that the
  initials should be \textit{na} and \textit{ga}, probably suggesting a connection with \textit{nirṛti}, \textit{naraka}s and \textit{nāga}s.
 }}

  \subsubchptr{vāruṇe}%

  \trsubsubchptr{West}%

  \maintext{vāruṇena pravakṣyāmi śṛṇu vipra nibodha me |}%

  \maintext{babhraḥ setur bhavodbhadraḥ prabhavodbhavabhājanaḥ |}%

  \maintext{bharaṇo bhuvano bhartā daśaite varuṇālayāḥ }||\thinspace1:45\thinspace||%
\translation{I shall teach you the [names] in Varuṇa's region [in the west]. Listen, O Brahmin, learn from me. [1] Babhra, [2] Setu, [3] Bhava, [4] Udbhadra, [5] Prabhava, [6] Udbhava, [7] Bhājana, [8] Bharaṇa, [9] Bhuvana, and [10] Bhartṛ: these ten dwell in Varuṇa's region [in the west]. \blankfootnote{1.45 Varuṇa upholds the sky and the earth. This could be the reason why 
  these names include \textit{bharaṇa} and \textit{bhartṛ}.
 }}

  \subsubchptr{vāyavye}%

  \trsubsubchptr{North-West}%

  \maintext{nṛgarbho 'suragarbhaś ca devagarbho mahīdharaḥ |}%

  \maintext{vṛṣabho vṛṣagarbhaś ca vṛṣāṅko vṛṣabhadhvajaḥ }||\thinspace1:46\thinspace||%
\translation{[1] Nṛgarbha, [2] Asuragarbha, [3] Devagarbha, [4] Mahīdhara, [5] Vṛṣabha, [6] Vṛṣagarbha, [7] Vṛṣāṅka, [8] Vṛṣabhadhvaja, \blankfootnote{1.46 The connection between \textit{vṛṣa} and the north-west or Vāyu is not evident to me. \verify
  In a tantric context, a western position is more standard for \textit{vṛṣa}, see e.g.
  \mycitep{Pancavaranastava}{40}.
 }}

  \maintext{jñātavyaś ca tathā samyag vṛṣajo vṛṣanandanaḥ |}%

  \maintext{nāyakā daśa vāyavye kīrtitā ye mayā dvija }||\thinspace1:47\thinspace||%
\translation{and [9] Vṛṣaja and [10] Vṛṣanandana: these are to be known properly as the ten leaders in Vāyu's region [in the north-west], as I taught them, O twice-born. \blankfootnote{1.47 Note how \msM\ deviates here again in a significant way.
 }}

  \subsubchptr{uttare}%

  \trsubsubchptr{North}%

  \maintext{sulabhaḥ sumanaḥ saumyaḥ suprajaḥ sutanuḥ śivaḥ |}%

  \maintext{sataḥ satya layaḥ śambhur daśa nāyakam uttare }||\thinspace1:48\thinspace||%
\translation{[1] Sulabha, [2] Sumana, [3] Saumya, [4] Supraja, [5] Sutanu, [6] Śiva, [7] Sata, [8] Satya, [9] Laya, [10] Śambhu: [these are] the ten leaders in the north. \blankfootnote{1.48 I prefer the form \textit{sumanaḥ} to the more standard \textit{sumanāḥ} {\rm (}\msNc{\rm )} in \textit{pāda} a, 
  because it suits the slightly irregluar language of the \VSS\ {\rm (}see pp. \verify{\rm )},
  and because the solitary reading of \msNc\ may well only be an attempt to
  standardise. It is also not inconceivable that \textit{sumanaḥ} stands compounded 
  with \textit{saumyaḥ}.
 Note how \textit{daśa nāyakam} could again be an example for the use of the singular 
  next to a number in \textit{pāda d}. It seems that here the northern region
  is associated with Śiva, rather than the north-east, the \textit{īśāna} direction, which
  is occupied by Brahmā: see next verse. 
  In a tantric context, Brahmā is sometimes associated with the north-east, see, e.g.,
  \mycitep{Pancavaranastava}{39}.\verify
  I have left \textit{satya} in stem form.
 }}

  \subsubchptr{īśāne}%

  \trsubsubchptr{North-East}%

  \maintext{indu bindu bhuvo vajra varado vara varṣaṇaḥ |}%

  \maintext{ilano valino brahmā daśeśāneṣu nāyakāḥ }||\thinspace1:49\thinspace||%
\translation{[1] Indu, [2] Bindu, [3] Bhuva, [4] Vajra, [5] Varada, [6] Vara, [7] Varṣaṇa, [8] Ilana, [9] Valina, [10] Brahmā: [these are] the ten rulers in the Īśāna direction [i.e. in the north-east]. \blankfootnote{1.49 I consider \textit{indu, bindu} and \textit{vajra} stem form nouns.
 The north-east seems to be occupied by Brahmā, and by rulers whose names should
  somehow evoke Brahmā's name.
 }}

  \subsubchptr{madhyame}%

  \trsubsubchptr{Center}%

  \maintext{aparo vimalo moho nirmalo mana mohanaḥ |}%

  \maintext{akṣayaś cāvyayo viṣṇur varado madhyame daśa }||\thinspace1:50\thinspace||%
\translation{[1] Apara, [2] Vimala, [3] Moha, [4] Nirmala, [5] Mana, [6] Mohana, [7] Akṣaya, [8] Avyaya, [9] Viṣṇu, [10] Varada: [these are] the ten [leaders] in the centre. \blankfootnote{1.50 Note that the last three lists above have been associated 
  with Śiva, Brahmā and Viṣṇu, respectively, and here, in a layer
  of the text that can be labelled Vaiṣṇava {\rm (}see pp. \verify{\rm )}, it is Viṣṇu that
  seems to occupy a central position. \textit{mana mohanaḥ} in \textit{pāda} b may
  sound like one single name, but we are forced to separate these two words
  {\rm (}\textit{mana} being in stem form metri causa{\rm )} to arrive at a list of ten names.
 }}

  \subsubchptr{parivārāḥ}%

  \trsubsubchptr{Subordinates}%

  \maintext{sarveṣāṃ daśa{-}m{-}īśānāṃ parivāraśataṃ śatam |}%

  \maintext{śatānāṃ pṛthag ekaikaṃ sahasraiḥ parivāritam }||\thinspace1:51\thinspace||%
\translation{Each of the ten rulers has a retinue of a hundred subordinates. Each one of [these] hundred is surrounded by a thousand subordinates. \blankfootnote{1.51 I take \textit{daśa-m-īśānāṃ} as a split compound {\rm (}\textit{daśeśānāṃ}{\rm )}.
  It is conceivable that each of the above ninety rulers has ten subordinates, 
  therefore each group of ten rulers has a hundred subordinates altogether,
  but the original idea may have been that each one of the above ninety 
  rulers has a hundred subordinates. Alternatively, this verse may only refer to 
  the central group of ten rulers mentioned in 1.50, and each one of them has
  a hundred subordinates.
 }}

  \maintext{sahasreṣu ca ekaikam ayutaiḥ parivāritam |}%

  \maintext{ayutaṃ prayutair vṛndaiḥ prayutaṃ niyutair vṛtam }||\thinspace1:52\thinspace||%
\translation{Each one of the thousand is surrounded by ten thousand [subordinates], the ten thousand is surrounded by a multitude of a hundred thousand, the hundred thousand by a million, \blankfootnote{1.52 We are forced to follow \Ed's reading in \textit{pāda} c in order to make sense of this passage.
  My correction in \textit{pāda} d is motivated by the same. Note that \textit{vṛnda} is not a number in this line. 
  Elsewhere in this chapter \textit{vṛnda} is the word that signifies `a billion.'
 }}

  \maintext{ekaikasya parīvāro niyutaḥ pṛthag eva ca |}%

  \maintext{koṭibhir daśakoṭyena ekaikaḥ parivāritaḥ }||\thinspace1:53\thinspace||%
\translation{[that is] each one has a retinue of a million {\rm (}\textit{niyuta}{\rm )} [subordinates]. [Then those] are surrounded by ten million {\rm (}\textit{koṭi}{\rm )} [subordinates], [they in turn] by a hundred million {\rm (}\textit{daśakoṭi}{\rm )}. \blankfootnote{1.53 It seems that \textit{pāda}s ab repeat what has been stated in 1.52cd.
 \textit{°koṭyena} stands for \textit{°koṭyā} {\rm (}thematisation{\rm )}.
  Note how the scribe of \msM\ gets confused at 1.53c due to an eye-skip and 
  fully regains control only at 1.55b.
 }}

  \maintext{daśakoṭiṣu ekaikaṃ vṛndavṛndabhṛtair vṛtam |}%

  \maintext{vṛndavargeṣu ekaikaṃ kharvabhiḥ parivāritam }||\thinspace1:54\thinspace||%
\translation{Each one of the hundred million is surrounded by a billion {\rm (}\textit{vṛnda}{\rm )} subordinates {\rm (}\textit{bhṛta}{\rm )}. Each one in these groupsof a billion {\rm (}\textit{vṛnda}{\rm )} is surrounded by ten billion {\rm (}\textit{kharva}{\rm )} [subordinates]. }

  \maintext{kharvavargeṣu ekaikaṃ daśakharvagaṇair vṛtam |}%

  \maintext{daśakharveṣu ekaikaṃ śaṅkubhiḥ parivāritam }||\thinspace1:55\thinspace||%
\translation{Each in these gourps of ten billion {\rm (}\textit{kharva}{\rm )} is surrounded by a hundred billion {\rm (}\textit{daśakharva}{\rm )}. Each of those hundred billion {\rm (}\textit{daśakharva}{\rm )} is surrounded by a trillion {\rm (}\textit{śaṅku}{\rm )} [deities]. }

  \maintext{śaṅkubhiḥ pṛthag ekaikaṃ padmena parivāritam |}%

  \maintext{padmavargeṣu ekaikaṃ samudraiḥ parivāritam }||\thinspace1:56\thinspace||%
\translation{Each of those one trillion {\rm (}\textit{śaṅku}{\rm )} is surrounded by ten trillion {\rm (}\textit{padma}{\rm )}. Each of those ten trillion {\rm (}\textit{padma}{\rm )} is surrounded by a hundred trillion {\rm (}\textit{samudra}{\rm )}. \blankfootnote{1.56 Note that \textit{śaṅkubhiḥ} stands for \textit{śaṅkūṣu} {\rm (}instrumental for locative{\rm )}.
 }}

  \maintext{samudreṣu tathaikaikaṃ madhyasaṃkhyais tu tair vṛtam |}%

  \maintext{madhyasaṃkhyeṣu ekaikam anantaiḥ parivāritam }||\thinspace1:57\thinspace||%
\translation{And each of those hundred trillion {\rm (}\textit{samudra}{\rm )} is surrounded by those whose number is one quadrillion {\rm (}\textit{madhya}{\rm )}. Each of those quadrillion {\rm (}\textit{madhya}{\rm )} is surrounded by ten quadrillion {\rm (}\textit{ananta}{\rm )}. }

  \maintext{ananteṣu ca ekaikaṃ parārdhaparivāritam |}%

  \maintext{parārdheṣu ca ekaikaṃ pareṇa parivāritam |}%

  \maintext{eṣa vai kathito vipra śakyaṃ sāṃkhyam udīritam }||\thinspace1:58\thinspace||%
\translation{Each of those ten quadrillion {\rm (}\textit{ananta}{\rm )} is surrounded by a hundred quadrillion {\rm (}\textit{parārdha}{\rm )}. Each of those hundred quadrillion {\rm (}\textit{parārdha}{\rm )} is surrounded by two hundred quadrillion {\rm (}\textit{para}{\rm )}. This is how it is taught, O Brahmin. The enumeration [of the rulers of the Brahmāṇḍa] has been taught as much as it is possible. }

  \subchptr{pramāṇam}%

  \trsubchptr{Measurements}%

  \maintext{pramāṇaṃ śṛṇu me vipra saṃkṣepād bruvato mama |}%

  \maintext{candrodaye pūrṇamāsyāṃ vapur aṇḍasya tādṛśam }||\thinspace1:59\thinspace||%
\translation{Listen to me and learn about the measurements [of the universe], O Brahmin, I shall teach [you] in a concise manner. The body of the Egg is like that of [the moon] at moonrise on the day of the full moon. }

  \maintext{koṭikoṭisahasraṃ tu yojanānāṃ samantataḥ |}%

  \maintext{aṇḍānāṃ ca parīmāṇaṃ brahmaṇā parikīrtitam }||\thinspace1:60\thinspace||%
\translation{The whole circumference of the Eggs has been declared by Brahmā to be ten million {\rm (}\textit{koṭi}{\rm )} times a thousand times ten million \textit{yojana}s. }

  \maintext{saptakoṭisahasrāṇi saptakoṭiśatāni ca |}%

  \maintext{viṃśakoṭiṣv aṅgulīṣu ūrdhvatas tapate raviḥ }||\thinspace1:61\thinspace||%
\translation{The Sun shines from the height of seven thousand seven hundred and twenty \textit{koṭi} finger's breath. \blankfootnote{1.61 This verse is the reply to the question in 1.37cd, which contains the word \textit{aṅguli}:
  this hints at the possibility that the unintelligible \textit{gulmeṣu} transmitted in most of the
  witnesses might be corrupted from \textit{aṅguīṣu}; hence my conjecture, resulting
  in a \textit{ra-vipulā}.
 }}

  \maintext{pramāṇaṃ nāma saṃkhyā ca kīrtitāni samāsataḥ |}%

  \maintext{brahmāṇḍaṃ cāprameyāṇāṃ lakṣaṇaṃ parikīrtitam }||\thinspace1:62\thinspace||%
\translation{The numbers pertaining to the measurements have been taught in brief. The characteristics of the unmeasurable Brahmāṇḍa[s] have been taught. \blankfootnote{1.62 Note the mixture of different grammatical genders and numbers in this verse. 
  Understand \textit{pramāṇeṣu saṃkhyāḥ kīrtitāḥ samāsataḥ} and 
  \textit{brahmāṇḍānām aprameyānāṃ}...
 }}

  \subchptr{vyāsāḥ}%

  \trsubchptr{The redactors {\rm (}of the Purāṇas{\rm )}}%

  \maintext{purāṇāśīsahasrāṇi śatāni dvijasattama |}%

  \maintext{brahmaṇā kathitaṃ pūrṇaṃ mātariśvā yathātatham }||\thinspace1:63\thinspace||%
\translation{O truest of the twice-born, the Purāṇa[s of] 8,000,000 [verses] were taught by [1] Brahmā to [2] Mātariśvan [= Vāyu] in their entirety, in their true form. \blankfootnote{1.63 \textit{Pāda} a should probably be analysed and interpreted as 
  \textit{purāṇam {\rm (}purāṇānām aśītisahasrāṇi śatāni ślokāni{\rm )} brahmaṇā kathitam}.
  Alternatively, \textit{pāda} a may have originally read \textit{purāṇāni sahasrāṇi},
  and then the initial number of verses transmitted by Brahmā is
  a hundred thousand. That the number refers to the number of \textit{śloka}s
  transmitted is confirmed in 1.66d: \textit{viṃśatślokasahasrikam}.
 
  On the idea that initially there was only one Purāṇa, see, e.g.,
  \mycitep{RocherPuranas1986}{41ff}.
 
   %
  In \textit{pāda} d, either understand \textit{mātariśvā} {\rm (}nom.{\rm )} as \textit{mātariśvānaṃ} {\rm (}acc.{\rm )} or emend
  \textit{kathitaṃ} to \textit{kathitaḥ} in the sense `Mātariśvan was taught,' echoing 1.39cd:
  \textit{brahmaṇā yat purākhyāto mātariśvā yathā tathā}.
 
   %
  Compare this list to a list of twenty-eight \textit{vedavyāsa}s, from
  Brahmā to Vyāsa Dvaipāyana, in \VISNUP\ 3.3.10--19,
  taught by Parāśara, the twenty-sixth \textit{vyāsa}
  of this list and our text {\rm (}in the numbering that I add here I follow
  the translation in \mycitep{Visnupurana_tr}{178--179}{\rm )}:
   %
  \textit{vedavyāsā vyatītā ye aṣṭāviṃśati sattama~|
  caturdhā yaiḥ kṛto vedo dvāpareṣu punaḥ punaḥ~||
  dvāpare prathame vyastāḥ svayaṃ vedāḥ }[1]\textit{ svayaṃbhuvā~| 
  dvitīye dvāpare caiva vedavyāsaḥ }[2]\textit{ prajāpati~|| 
  tṛtīye }[3]\textit{ cośanā vyāsaś caturthe ca }[4]\textit{ bṛhaspatiḥ~|  
  }[5]\textit{ savitā pañcame vyāsaḥ }[6]\textit{ mṛtyuḥ ṣaṣṭhe smṛtaḥ prabhuḥ~||  
  saptame ca }[7]\textit{ tathaivendro }[8]\textit{ vasiṣṭhaś cāṣṭame smṛtaḥ~|  
  }[9]\textit{ sārasvataś ca navame }[10]\textit{ tridhāmā daśame smṛtaḥ~||  
  ekādaśe tu }[11]\textit{ trivṛṣā }[12]\textit{ bhāradvājas tataḥ param~|  
  trayodaśe }[13]\textit{ cāntarikṣo }[14]\textit{ varṇī cāpi caturdaśe~||  
  }[15]\textit{ trayyāruṇaḥ pañcadaśe ṣoḍaśe tu }[16]\textit{ dhanaṃjayaḥ~|  
  }[17]\textit{ kratuṃjayaḥ saptadaśe }[18]\textit{ ṛṇajyo 'ṣṭādaśe smṛtaḥ~||  
  tato vyāso }[19]\textit{ bharadvājo bharadvājāt tu }[20]\textit{ gautamaḥ~|  
  gautamād uttamo vyāso }[21]\textit{ haryātmā yo 'bhidhīyate~||  
  atha haryātmano }[22]\textit{ venaḥ smṛto vājaśravās tu yaḥ~|  
  somaḥ śuṣmāyaṇas tasmāt }[23]\textit{ tṛṇabindur iti smṛtaḥ~||  
  }[24]\textit{ ṛkṣo 'bhūd bhārgavas tasmād vālmīkir yo 'bhidhīyate~|  
  tasmād asmatpitā }[25]\textit{ śaktir vyāsas tasmād }[26]\textit{ ahaṃ mune~||  
  }[27]\textit{ jātukarṇo 'bhavan mattaḥ kṛṣṇadvaipāyanas }[28]\textit{ tataḥ~|  
  aṣṭaviṃśatir ity ete vedavyāsāḥ purātanāḥ~||.  }
 
   %
  Another relevant passage is \BRAHMANDAPUR\ 3.4.58cd--67 {\rm (}\similar\ \VAYUP\ 2.41.58--67{\rm )}.
  Note how Tṛṇabindu is, perhaps by mistake, different from Somaśuṣma/Śuṣmāyaṇa here,
  but, more importantly, note Amitabuddhi of \VSS\ 1.76 appear at the end of this list:
   %
  [1] \textit{brahmā dadau śāstram idaṃ purāṇaṃ }[2]\textit{ mātariśvane~||  
  tasmāc }[3]\textit{ cośanasā prāptaṃ tasmāc cāpi }[4]\textit{ bṛhaspatiḥ~|  
  bṛhaspatis tu provāca }[5]\textit{ savitre tadanantaram~||  
  savitā }[6]\textit{ mṛtyave prāha mṛtyuś }[7]\textit{ cendrāya vai punaḥ~|  
  indraś cāpi }[8]\textit{ vasiṣṭāya so 'pi }[9]\textit{ sārasvatāya ca~||  
  sārasvatas }[10]\textit{ tridhāmne 'tha tridhāmā ca }[11]\textit{ śaradvate~|  
  śaradvāṃs tu }[12]\textit{ triviṣṭāya so }[13]\textit{ 'ntarikṣāya dattavān~||  
  }[14]\textit{ carṣiṇe cāntarikṣo vai so 'pi }[15]\textit{ trayyāruṇāya ca~|  
  trayyāruṇād }[16]\textit{ dhanañjayaḥ sa vai prādāt }[17]\textit{ kṛtañjaye~||  
  kṛtañjayāt }[18]\textit{ tṛṇañjayo }[19]\textit{ bharadvājāya so 'py atha~|  
  }[20]\textit{ gautamāya bharadvājaḥ so 'pi }[21]\textit{ niryyantare punaḥ~||  
  niryyantaras tu provāca tathā }[22]\textit{ vājaśravāya vai~|  
  sa dadau }[23]\textit{ somaśuṣmāya sa cādāt }[24]\textit{ tṛṇabindave~||  
  tṛṇabindus tu }[25]\textit{ dakṣāya dakṣaḥ provāca }[26]\textit{ śaktaye~|  
  śakteḥ }[27]\textit{ parāśaraś cāpi garbhasthaḥ śrutavān idam~||  
  parāśarāj }[28]\textit{ jātukarṇyas tasmād }[29]\textit{ dvaipāyanaḥ prabhuḥ~|  
  dvaipāyanāt punaś cāpi }[30]\textit{ mayā prāptaṃ dvijottama~||  
  mayā caitat punaḥ proktaṃ }[31]\textit{ putrāyāmitabuddhaye~|  
  ity eva vākyaṃ brahmādiguruṇāṃ samudāhṛtam~||.}  
 
   %
  The list of \textit{vedavyāsa}s in \LINPU\ 1.7.15--18 includes these twenty-five names:
  Kratu, Satya, Bhārgava, Aṅgiras, Savitṛ,
  Mṛtyu, Śatakratu, Vasiṣṭha, Sārasvata, Tridhāman,
  Trivṛta, Śatatejas, Tarakṣu, Āruṇi, Kṛtaṃjaya,
  Ṛtaṃjayo, Bharadvāja, Gautama, Vācaśravas, Tṛṇabindu,
  Rūkṣa, Śakti, Jātūkarṇya, Kṛṣṇa Dvaipāyano.
 }}

  \maintext{vāyunā pāda saṃkṣipya prāptaṃ cośanasaṃ purā |}%

  \maintext{tenāpi pāda saṃkṣipya prāptavāṃś ca bṛhaspatiḥ }||\thinspace1:64\thinspace||%
\translation{Vāyu abridged the verses and then gave [the Purāṇas] to [3] Uśanas. He [Uśanas] also abridged the verses, and [4] Bṛhaspati received them. \blankfootnote{1.64 Note the stem form noun \textit{pāda} twice in this verse and 
  the slightly odd grammatical structure {\rm (}\textit{purāṇaṃ}{\rm )} \textit{prāptam uśanasam} {\rm (}`the Purāṇa reached Uśanas'{\rm )},
  as opposed to the solution in \textit{pāda} d {\rm (}\textit{prāptavān}{\rm )}.
 }}

  \maintext{bṛhaspatis tu provāca sūryaṃ triṃśatsahasrikam |}%

  \maintext{pañcaviṃśatsahasrāṇi mṛtyuṃ prāha divākaraḥ }||\thinspace1:65\thinspace||%
\translation{Bṛhaspati taught 30,000 [verses] to [5] Sūrya [the Sun]. Divākara [= the Sun] taught 25,000 [verses] to [6] Mṛtyu [Death]. }

  \maintext{ekaviṃśatsahasrāṇi mṛtyunendrāya kīrtitam | }%

  \maintext{indreṇāha vasiṣṭhāya viṃśatślokasahasrikam }||\thinspace1:66\thinspace||%
\translation{Mṛtyu taught 21,000 [verses] to [7] Indra. Indra taught 20,000 verses to [8] Vasiṣṭha. }

  \maintext{aṣṭādaśasahasrāṇi tena sārasvatāya tu |}%

  \maintext{sārasvatas tridhāmāya sahasradaśa sapta ca }||\thinspace1:67\thinspace||%
\translation{And he[, Vasiṣṭha taught] 18,000 [verses] to [9] Sārasvata. Sārasvata [taught] 17,000 [verses] to [10] Tridhāma[n]. }

  \maintext{ṣoḍaśānāṃ sahasrāṇi bharadvājāya vai tataḥ |}%

  \maintext{daśa pañcasahasrāṇi trivṛṣāya abhāṣata }||\thinspace1:68\thinspace||%
\translation{[He taught] 16,000 verses to [11] Bharadvāja. [Bharadvāja] taught 15,000 verses to [12] Trivṛṣa. }

  \maintext{caturdaśasahasrāṇi antarīkṣāya vai tataḥ |}%

  \maintext{trayyāruṇiṃ sahasrāṇi trayodaśa abhāṣata }||\thinspace1:69\thinspace||%
\translation{[Trivṛṣa] then [taught] 14,000 verses to [13] Antarīkṣa. [Antarīkṣa] taught 13,000 [verses] to [14] Trayyāruṇi. }

  \maintext{trayyāruṇis tu viprendro dhanaṃjayam abhāṣata |}%

  \maintext{dvādaśāni sahasrāṇi saṃkṣipya punar abravīt }||\thinspace1:70\thinspace||%
\translation{Trayyāruṇi, the great Brahmin, having abridged them again, taught 12,000 [verses] to [15] Dhanaṃjaya. }

  \maintext{kṛtaṃjayāya samprāpto dhanaṃjayamahāmuniḥ |}%

  \maintext{kṛtaṃjayād dvijaśreṣṭha ṛṇaṃjayamahātmane }||\thinspace1:71\thinspace||%
\translation{Dhanaṃjaya, the great sage, handed [them] over to [16] Kṛtaṃjaya. [That recension was transmitted] from Kṛtaṃjaya, O best of the twice-born, to [17] noble Ṛṇaṃjaya. \blankfootnote{1.71 Note the odd structure in \textit{pāda}s ab: \textit{dhanaṃjayaḥ kṛtaṃjayāya samprāptaḥ}, 
  for a more standard \textit{dhanaṃjayena {\rm (}\textit{purāṇam}{\rm )} samprāpitaṃ kṛtaṃjayam} 
  {\rm (}`the Purāṇa was transmitted to Kṛtaṃjaya'{\rm )}.
 }}

  \maintext{ṛṇañjayāt punaḥ prāpto gautamāya maharṣiṇe |}%

  \maintext{gautamāc ca bharadvājas tasmād dharyātmanāya tu }||\thinspace1:72\thinspace||%
\translation{Then from Ṛṇaṃjaya it was given to [18] Gautama, the great sage, from Gautama to [19] Bharadvāja, from him to [20] Haryātman. \blankfootnote{1.72 The structure of \textit{pāda}s ab is as odd as that of 1.71ab. What was
  intended is probably \textit{ṛṇañjayena prāpitaṃ gautamāya}.
 My emendation in \textit{pāda} d of \textit{haryadvatāya} to \textit{haryātmanāya} {\rm (}for a standard \textit{haryātmane}{\rm )} 
  is based on the list of \textit{vedavyāsa}s in \VISNUP\ 3.3.16--17 {\rm (}see note to 1.63 above{\rm )}.
 }}

  \maintext{rājaśravās tataḥ prāptaḥ somaśuṣmāya vai tataḥ |}%

  \maintext{somaśuṣmāt tataḥ prāptas tṛṇabindus tu bho dvija }||\thinspace1:73\thinspace||%
\translation{Then [21] Rājaśravas received it, then [22] Somaśuṣma. Then from Somaśuṣma [23] Tṛṇabindu received it, O twice-born. \blankfootnote{1.73 The syntax is again slightly odd here. The indention may have been
  \textit{prāpitaṃ rājaśavasā somaśuṣmāya... tatas tṛṇabindunā prāptam}.
 }}

  \maintext{tṛṇabindus tu vṛkṣāya vṛkṣaḥ śaktim abhāṣata |}%

  \maintext{śaktiḥ parāśaraṃ prāha jatukarṇāya vai tataḥ }||\thinspace1:74\thinspace||%
\translation{Tṛṇabindu taught it to [24] Vṛkṣa, Vṛkṣa to [25] Śakti [the father of Parāśara]. Śakti taught it to [26] Parāśara, then [Parāśara] to [27] Jatukarṇa. \blankfootnote{1.74 In other list of \textit{vedavyāsa}s, Tṛṇabindu hands the Purāṇas down to 
  Ṛkṣa, Rūkṣa or Dakṣa {\rm (}see note to 1.63 above{\rm )}. \textit{vṛkṣa} in \textit{pāda} a
  is probably a corrupted form.
 The name Jatukarṇa may be a corrupted form of Jātū- or Jātukarṇa.
 }}

  \maintext{dvaipāyanaṃ tu provāca jatukarṇo maharṣiṇam |}%

  \maintext{romaharṣāya samprāpto dvaipāyanamahāmuniḥ }||\thinspace1:75\thinspace||%
\translation{Jatukarṇa taught it to [28] [Vyāsa] Dvaipāyana, the great sage. Dvaipāyana, the great sage, gave it to [29] Romaharṣa. \blankfootnote{1.75 The syntax of \textit{pāda}s cd echoes that of 1.73ab above.
 }}

  \maintext{romaharṣeṇa provāca putrāyāmitabuddhaye |}%

  \maintext{daśa dve ca sahasrāṇi purāṇaṃ samprakāśitam |}%

  \maintext{mānuṣāṇāṃ hitārthāya kiṃ bhūyaḥ śrotum icchasi }||\thinspace1:76\thinspace||%
\translation{Romaharṣa taught the Purāṇa[s] of 12,000 [verses], now fully revealed, to his son, [30] Amitabuddhi, for the benefit of humankind. What else do you wish to know? \blankfootnote{1.76 Romaharṣa is usually considered to be the same person as Sūta, disciple of Vyāsa Dvaipāyana.
  
   %
  In \BrahmandaPur\ 3.4.67ab {\rm (}\textit{mayā caitat punaḥ proktaṃ putrāyāmitabuddhaye}, see note to 
  1.63 above{\rm )} Amitabuddhi is clearly the name {\rm (}or epithet{\rm )} of Romaharṣa's son. This suggests that the form \textit{romaharṣāya}
  in \textit{pāda} a is a mistake for \textit{romaharṣaś ca}, or similar. MS \msM\ is either transmitting an
  otherwise syntactically problematic reading {\rm (}\textit{romaharṣeṇa}{\rm )} that is more 
  original than that of most other witnesses or \msM's scribe is trying to correct the text.
  Supposing the former, in this case I accepted \msM's reading.
  %
 
  Manuscripts \msCc\ and \msM\ place the \textit{iti} of the colophon at the end of the last \textit{śloka}, before
  the \textit{daṇḍa}s, thus: \textit{icchasīti}\thinspace ||O|| {\rm (}\msCc{\rm )} and \textit{icchasi iti}\thinspace ||o|| {\rm (}\msM{\rm )}.
  Note also that \msM\ gives the number of \textit{śloka}s in this chapter, 77, which is almost exactly
  the number of verses this critical edition has produced. The scribe of \msM\ struggled 
  with eyeskips in this chapter, therefore it seems unlikely that he himself
  counted the number of verses he had copied and arrived at this very figure.
  Rather, he copied the number from his exemplar.
 }}
\center{\maintext{\dbldanda\thinspace iti vṛṣasārasaṃgrahe brahmāṇḍasaṃkhyā nāmādhyāyaḥ prathamaḥ\thinspace\dbldanda}}
\translation{Here ends the first chapter in the \textit{Vṛṣasārasaṃgraha} called the Description of the Brahmāṇḍa[s].}

  \chptr{dvitīyo 'dhyāyaḥ}
\addcontentsline{toc}{subsection}{Chapter 2}
\fancyhead[CO]{{\footnotesize\textit{Translation of chapter 2}}}%

  \trchptr{  Chapter Two }%

  \maintext{vigatarāga uvāca |}%

  \maintext{śrutaṃ mayā janāgreṇa brahmāṇḍasya tu nirṇayam |}%

  \maintext{pramāṇaṃ varṇarūpaṃ ca saṃkhyā tasya samāsataḥ }||\thinspace2:1\thinspace||%
\translation{Vigatarāga spoke: I have heard the description of the Brahmāṇḍa from [you,] the best of men, its extent, colour, form and the numbers associated with it, in a concise manner. \blankfootnote{2.1 It is unlikely that \textit{janāgreṇa} picks up \textit{mayā} {\rm (}`by me, the best of men'{\rm )}, instead,
  I supposed that this instrumental stands for the ablative or should be 
  understood as `through the best of man.'
 }}

  \maintext{śivāṇḍeti tvayā prokto brahmāṇḍālayakīrtitaḥ |}%

  \maintext{kīdṛśaṃ lakṣaṇaṃ jñeyaṃ pramāṇaṃ tasya vā kati }||\thinspace2:2\thinspace||%
\translation{You mentioned a Śivāṇḍa as taught to be the receptacle of the Brahmāṇḍa. What are its characteristics and how much is its extent? \blankfootnote{2.2 The location where Śivāṇḍa was mentioned is verse 1.40ab above.
 }}

  \maintext{kasya vā layanaṃ jñeyaṃ pramāṇaṃ vātra vāsinaḥ |}%

  \maintext{kā vā tatra prajā jñeyā ko vā tatra prajāpatiḥ }||\thinspace2:3\thinspace||%
\translation{Whose dwelling place is it? And [what] is the scale of the one[s] who dwell there? What kind of people live there? And who is the ruler {\rm (}\textit{prajāpati}{\rm )} there? \blankfootnote{2.3 \textit{vā layanaṃ} in \textit{pāda} a may stand for \textit{vā-ālayanaṃ}, in the sense of \textit{vā-ālayaṃ}.
  The questions in this verse are most probably answered in verses 2.26--33, and if my
  interpretation is correct there, \textit{pramāṇaṃ vātra vāsinaḥ} {\rm (}understand \textit{vāsināṃ}{\rm )} 
  and \textit{pāda} c should refer to the number of inhabitants in the five regions of Īśāna, Tatpuruṣa etc.,
  who are refered to here in \textit{pāda}s a and possibly d.
 }}

  \subchptr{śivāṇḍasaṃkhyā}%

  \trsubchptr{Summary of the Śivāṇḍa}%

  \maintext{anarthayajña uvāca |}%

  \maintext{śivāṇḍalakṣaṇaṃ vipra na tvaṃ praṣṭum ihārhasi |}%

  \maintext{daivatair api kā śaktir jñātuṃ draṣṭuṃ ca tattvataḥ }||\thinspace2:4\thinspace||%
\translation{Anarthayajña spoke: Please don't ask me about the characteristics of the Śivāṇḍa, O Brahmin. How could even the gods have the power to really know and see [the Śivāṇḍa]? }

  \maintext{agamyagamanaṃ guhyaṃ guhyād api samuddhṛtam |}%

  \maintext{na prabhur netaras tatra na daṇḍyo na ca daṇḍakaḥ }||\thinspace2:5\thinspace||%
\translation{The path leading to it is not to be trodden, it is more secret than any secret, and it is lofty. There is no master or servant there, nobody to be punished and no punisher. \blankfootnote{2.5 My emendation to \textit{samuddhṛtam} in \textit{pāda} b is not fully satisfactory, but the 
  readings transmitted in the witnesses are problematic. \msM, a MS not collated for
  this chapter, gives a confusing reading: \textit{sa\uncl{murdhni}dam}. I doubt
  if \Ed's \textit{samṛddhidam} {\rm (}`yielding success'{\rm )} is the correct reading.
  Perhaps \textit{samudāhṛtam} {\rm (}`declared, talked about as'{\rm )} was meant.
  It is not inconceivable that \msCc's {\rm (}and \msM's{\rm )} \textit{agamyagahanaṃ} 
  {\rm (}`it is inaccessible because of its depth'{\rm )} is original and
  it is to be contrasted with \textit{samuddhṛtam} {\rm (}`lofty'{\rm )}. One also wonders if
  \textit{guhād} could be the right reading, and in what sense, in \textit{pāda} b.
 }}

  \maintext{na satyo nānṛtas tatra suśīlo no duḥśīlavān |}%

  \maintext{nānṛjur na ca dambhitvaṃ na tṛṣṇā na ca īrṣyatā }||\thinspace2:6\thinspace||%
\translation{There are no truthful or untruthful people there, no moral or immoral people, no crooked people, no hypocrisy, no thirst or envy. \blankfootnote{2.6 Strictly speaking \textit{duḥśīlavān} in \textit{pāda} b is unmetrical; understand or pronounce \textit{duśīlavān}.
 \textit{īrṣyatā} {\rm (}for \textit{īrṣyā}, see 2.7a{\rm )} is a form rarely attested.
 }}

  \maintext{na krodho na ca lobho 'sti na māno 'sti na sūyakaḥ |}%

  \maintext{īrṣyā dveṣo na tatrāsti na śaṭho na ca matsaraḥ }||\thinspace2:7\thinspace||%
\translation{There is no anger or desire there, no arrogance or discontent {\rm (}\textit{[a]sūyaka}{\rm )}, no envy or hatred, no cheaters and no jealousy. \blankfootnote{2.7 \textit{na sūyakaḥ} in \textit{pāda} b stands for \textit{na asūyaka} metri causa.
 }}

  \maintext{na vyādhir na jarā tatra na śoko 'sti na viklavaḥ |}%

  \maintext{nādhamaḥ puruṣas tatra nottamo na ca madhyamaḥ }||\thinspace2:8\thinspace||%
\translation{There is no disease, no aging, no grief and no agitation there, there are no inferior or superior people and there is nobody in-between. }

  \maintext{notkṛṣṭo mānavas tasmin striyaś caiva śivālaye |}%

  \maintext{na nindā na praśaṃsāsti matsarī piśuno na ca }||\thinspace2:9\thinspace||%
\translation{There are no privileged men or women there in Śiva's abode, no reproach or praise, no selfish or treacherous people. }

  \maintext{garvadarpaṃ na tatrāsti krūramāyādikaṃ tathā |}%

  \maintext{yācamāno na tatrāsti dātā caiva na vidyate }||\thinspace2:10\thinspace||%
\translation{There is no pride or arrogance there, no cruelty or trickery and so on. There are no beggars and no donors there. }

  \maintext{anarthī vraja tatrasthaḥ kalpavṛkṣasamāśritaḥ |}%

  \maintext{na karma nāpriyas tatra na kaliḥ kalaho na ca }||\thinspace2:11\thinspace||%
\translation{Go without material desires {\rm (}\textit{anarthin}{\rm )}, being there you'll be resting under a wishing tree. There is no karma there and no enemy. No Kali age is there and there is no fighting. \blankfootnote{2.11 Note the term \textit{anartī} in \textit{pāda} a: it might have something to do 
  with non-material sacrifice {\rm (}\textit{anarthayajña}{\rm )}, the topic of chapter 11.
  \textit{vraja} in \textit{pāda} a is suspect.
 }}

  \maintext{dvāparo na ca na tretā kṛtaṃ cāpi na vidyate |}%

  \maintext{manvantaraṃ na tatrāsti kalpaś caiva na vidyate }||\thinspace2:12\thinspace||%
\translation{There is no Dvāpara age or Tretā or Kṛta. There are no \textit{manvantara}s there and no \textit{kalpa}s. \blankfootnote{2.12 On \textit{manvantara}s and \textit{kalpa}s, see 1.23--24 above.
 }}

  \maintext{āhūtasamplavaṃ nāsti brahmarātridinaṃ tathā |}%

  \maintext{na janmamaraṇaṃ tatra āpadaṃ nāpnuyāt kvacit }||\thinspace2:13\thinspace||%
\translation{No universal floods of destruction come, and there are no days and nights of Brahmā. There is no birth and death there and one never encounters catastrophes. \blankfootnote{2.13 \textit{āhūtasamplava} for the more widely attested form \textit{ābhūtasamplava} occurs, e.g.,
  in some MSS transmitting \SDHS\ 10.77 and 81 {\rm (}see \mycite{SDhS10_ed}{\rm )}.
 }}

  \maintext{na cāśāpāśabaddho 'sti rāgamohaṃ na vidyate |}%

  \maintext{na devā nāsurās tatra na yakṣoragarākṣasāḥ }||\thinspace2:14\thinspace||%
\translation{Nobody is tied to the noose of hope and there is no passion or delusion. There are no gods and demons there and no Yakṣas, Serpents and Rākṣasas. }

  \maintext{na bhūtā na piśācāś ca gandharvā ṛṣayas tathā |}%

  \maintext{tārāgrahaṃ na tatrāsti nāgakiṃnaragāruḍam }||\thinspace2:15\thinspace||%
\translation{There are no Ghosts nor Piśācas, no Gandharvas and no Ṛṣis. There are no planets there, no Nāgas, Kiṃnaras or Garuḍa-like creatures. }

  \maintext{na japo nāhnikas tatra nāgnihotrī na yajñakṛt |}%

  \maintext{na vrataṃ na tapaś caiva na tiryannarakaṃ tathā }||\thinspace2:16\thinspace||%
\translation{There are no recitations or daily rituals there, nobody performs the Agnihotra and there are no sacrificers. There are no religious observances and no austerities and no `animal hell'. \blankfootnote{2.16 The phrase of \textit{tiryaṅnaraka} appears in \MBH\ 3.181.18ab: 
  \textit{aśubhaiḥ karmabhiḥ pāpās tiryaṅnarakagāminaḥ}. Here \mycite{GanguliMBh} translates \textit{tiryaṅ} 
  separately as `in a crooked way,' but I suspect that in the \VSS\ \textit{tiryannaraka} has
  more to do with \textit{tiraggati}, being reduced to animal existence, being reborn as an animal or entering a 
  hell in animal form. 
  Cf. \MBH\ 13.134.057 {\rm (}\verify{\rm )}:
  \textit{nṛṣu janma labhante ye karmaṇā madhyamāḥ smṛtāḥ\thinspace |
  tiryaṅnarakagantāro hy adhamās te narādhamāḥ\thinspace ||},
  and \textit{Umāmaheśvarasaṃvāda} 6.1: 
  \textit{avamanyanti ye viprān sarvaloke namaskṛtān\thinspace |
  narakaṃ yānti te sarve tiryagyoniṃ vrajanti ca~||.}
  I suspect that \textit{nātirya°} in the witnesses is only a scribal mistake for \textit{na tirya°}.
 }}

  \maintext{tasyeśānasya devasya aiśvaryaguṇavistaram |}%

  \maintext{api varṣaśatenāpi śakyaṃ vaktuṃ na kenacit }||\thinspace2:17\thinspace||%
\translation{Nobody would be able to tell the extent of the qualities of the god Īśāna's powers, not even in a hundred years. \blankfootnote{2.17 My translation of \textit{aiśvaryaguṇa}° is tentative. It could be taken as a \textit{dvandva} compound
  {\rm (}e.g. `supremacy and qualities'{\rm )}. The expression \textit{sarva}° or \textit{aṣṭaiśvaryaguṇopeta}
  occurs frequently, e.g. in \SIVP\ 7.2.8.28ab and \SKANDAP\ 55.30cd, and \SDHU\ 2.6, 79, 125, 127,
  with \textit{aiśvarya} most probably refering to the eight \textit{siddhi}s \textit{aṇiman, laghiman} etc.
  De Simini {\rm (}2016a, 386{\rm )},\nocite{DeSiminiGods2016} e.g., translates \textit{sarvaiśvaryaguṇopetaḥ} in \SDHU\ 2.127 as
  `endowed with all the qualities of lordship.'
 }}

  \maintext{harecchāprabhavāḥ sarve paryāyeṇa bravīmi te |}%

  \maintext{devamānuṣavarjyāni vṛkṣagulmalatādayaḥ }||\thinspace2:18\thinspace||%
\translation{All are born by Hara's wish. I shall teach [them to] you one by one, excluding gods and people, starting with the trees, the bushes and creepers. \blankfootnote{2.18 Note the gender confusion in this verse, and the way I take \textit{pāda} a as a separate 
  statement to aviod a further confusion of case.
 }}

  \maintext{parārdhadviguṇotsedho vistāraś ca tathāvidhaḥ |}%

  \maintext{anekākārapuṣpāṇi phalāni ca manoharam }||\thinspace2:19\thinspace||%
\translation{The height [of the Śivāṇḍa] is two \textit{parārdha}s, and [its] width is the same. There are lovely flowers of different forms [there] and also lovely fruits. \blankfootnote{2.19 I understand \textit{pāda} a as \textit{parārdhadviguṇa utsedho}, i.e. as an example of double \textit{sandhi}.
  On the other hand, °\textit{sedho} is only my conjecture, and may refer to something else than the Śivāṇḍa.
  Note the number confusion in \textit{pāda} d, and also that two \textit{parārdha}s is one \textit{para}, 
  the highest possible number according to verses 1.35--36 above. The number may refer to
  any unit of length, but 2.23 below suggests that it is \textit{yojana}s.
 }}

  \maintext{anye kāñcanavṛkṣāṇi maṇivṛkṣāṇy athāpare |}%

  \maintext{pravālamaṇiṣaṇḍāś ca padmarāgaruhāṇi ca }||\thinspace2:20\thinspace||%
\translation{There are also golden trees and also gem trees, coral gem thickets and ruby plants. \blankfootnote{2.20 Note that both \textit{anye} and \textit{apare} here pick up neuter nouns {\rm (}gender confusion{\rm )}.
 }}

  \maintext{svādumūlaphalāḥ skandhalatāviṭapapādapāḥ |}%

  \maintext{kāmarūpāś ca te sarve kāmadāḥ kāmabhāṣiṇaḥ }||\thinspace2:21\thinspace||%
\translation{There are tasty roots and fruits and trees with creepers on their twigs. All are shape-shifters and they fulfill man's desires and they whisper seductively. \blankfootnote{2.21 My conjectures in \textit{pāda}s ab result in a compoud spanning the c\ae sura, which
  may have been the reason why the line got corrupted.
 }}

  \maintext{tatra vipra prajāḥ sarve anantaguṇasāgarāḥ |}%

  \maintext{tulyarūpabalāḥ sarve sūryāyutasamaprabhāḥ }||\thinspace2:22\thinspace||%
\translation{There [in the Śivāṇḍa], O Brahmin, all the subjects are the oceans of endless virtues. They are all equally beautiful and strong, and they shine like millions of suns. }

  \maintext{parārdhadvayavistāraṃ parārdhadvayam āyatam |}%

  \maintext{parārdhadvayavikṣepaṃ yojanānāṃ dvijottama }||\thinspace2:23\thinspace||%
\translation{[The Śivāṇḍa] is two \textit{parārdha} long and two \textit{parārdha} wide, and two \textit{parārdha}s is its [vertical] extension, [measured] in \textit{yojana}s, O great Brahmin. }

  \maintext{aiśvaryatvaṃ na saṃkhyāsti balaśaktiś ca bho dvija |}%

  \maintext{adhordhvo na ca saṃkhyāsti na tiryañ caiti kaścana }||\thinspace2:24\thinspace||%
\translation{[Īśāna's] powers cannot be expressed by numbers, neither can [His] powerfulness, O twice-born. [In fact, the distances in the Śivāṇḍa] downwards and upwards cannot be expressed by numbers. Nobody can travere it horizontally. \blankfootnote{2.24 \textit{Pāda}s ab are an echo of 2.17b.
 \textit{kaścana} in \textit{pāda} d forces us to accept the readin in \msNapcorr\msNc\ {\rm (}\textit{caiti}{\rm )},
  as opposed to \textit{ceti} in the remaining witnesses.
 }}

  \maintext{śivāṇḍasya ca vistāram āyāmaṃ ca na vedmy aham |}%

  \maintext{bhogam akṣaya tatraiva janmamṛtyur na vidyate }||\thinspace2:25\thinspace||%
\translation{[In reality,] I do not know the length and width of the Śivāṇḍa. Enjoyment is undecaying there, and there is no birth or death there. \blankfootnote{2.25 \textit{Pāda} c is transmitted in an unmetrical form and with a gender problem in the witnesses
  {\rm (}\textit{bhogam akṣayas}{\rm )}, hence my emendation using a stem form noun,
  a phenomenon frequently seen in this text. But note that \textit{bhoga} is normally masculine;
  there might be a hiatus-filler in-between: \textit{bhoga-m-akṣaya}.
 }}

  \maintext{śivāṇḍamadhyam āśritya gokṣīrasadṛśaprabhāḥ |}%

  \maintext{parārdhaparakoṭīnām īśānānāṃ smṛtālayaḥ }||\thinspace2:26\thinspace||%
\translation{In the centre of the Śivāṇḍa, [creatures] shine like cow's milk. [It is] said to be the region {\rm (}\textit{ālaya}{\rm )} of the one and a half \textit{para} crore Īśānas. \blankfootnote{2.26 Note the stem form \textit{smṛta} in \textit{pāda} d {\rm (}cf. 2.29d{\rm )}. I understand \textit{īśānānāṃ} as \textit{aiśānānāṃ}.
 
  Īśāna is traditionally the upward-looking face of Śiva, his region is positioned in the centre here.
  Note that the somewhat cryptic third \textit{pāda}s here and in the coming verses
  may or may not refer to the number of people living in the given region. 
  They may tell us about the extent of the given region, although the numbers are much
  higher than what one would expect after verse 2.23.
 }}

  \maintext{bālasūryaprabhāḥ sarve jñeyās tatpuruṣālaye |}%

  \maintext{parārdhaparakoṭīnāṃ pūrvasyāṃ diśam āśritāḥ }||\thinspace2:27\thinspace||%
\translation{They are all like the rising sun in the region of Tatpuruṣa. They are one and a half \textit{para} crore [in number], living in the east. \blankfootnote{2.27 The genitive of \textit{parārdhaparakoṭīnāṃ} is baffling here and in the coming verses,
  but I suspect that again the expression gives the number of subjects living in the given region.
  \textit{pūrvasyāṃ} is meant to mean \textit{pūrvāṃ} {\rm (}cf. \textit{dakṣiṇāṃ, paścimāṃ,} and \textit{uttarāṃ} in the next verses{\rm )};
  note how \msNb\ tries to save the construction by reading \textit{diśi}.
 
  This verse conforms to the traditional view that Śiva's Tatpuruṣa-face
  is looking to the east.
 }}

  \maintext{bhinnāñjanaprabhāḥ sarve dakṣiṇāṃ diśam āśritāḥ |}%

  \maintext{parārdhaparakoṭīnām aghorālayam āśritāḥ }||\thinspace2:28\thinspace||%
\translation{Everybody is like collyrium in the southern direction, in the region of Aghora, one and a half \textit{para} crore [in number]. \blankfootnote{2.28 Note the Aiśa form \verify\ REF
  \textit{diśiṃ} in \msCb, and that Aghora is indeed usually south-facing.
 }}

  \maintext{kundenduhimaśailābhāḥ paścimāṃ diśam āśritāḥ |}%

  \maintext{parārdhaparakoṭīnāṃ sadya{-}m{-}iṣṭālayaḥ smṛtaḥ }||\thinspace2:29\thinspace||%
\translation{In the western direction, they are like jasmine, the moon, like snowy rocks. Sadyojāta's lovely region is [home] to one and a half \textit{para} crore [people]. \blankfootnote{2.29 Note the Aiśa form \textit{diśiṃ} in \msNc\ in \textit{pāda} b.
  In \textit{pāda} d, we may presuppose the presence of a \textit{sandhi}-bridge: \textit{sadya-m-iṣṭālayaḥ}.
  Sadyojāta is traditionally associated with the western direction.
 }}

  \maintext{kuṅkumodakasaṃkāśā uttarāṃ diśam āśritāḥ |}%

  \maintext{parārdhaparakotīnāṃ vāmadevālayaḥ smṛtaḥ }||\thinspace2:30\thinspace||%
\translation{In the northern direction, they are like saffron in water. Vāmadeva's region is [home] to one and a half \textit{para} crore [people]. \blankfootnote{2.30 Note the Aiśa form \textit{diśiṃ} in \msCa\ in \textit{pāda} b.
 Vāmadeva is traditionally associated with the western direction.
 }}

  \maintext{īśānasya kalāḥ pañca vaktrasyāpi catuṣ kalāḥ |}%

  \maintext{aghorasya kalā aṣṭau vāmadevās trayodaśa }||\thinspace2:31\thinspace||%
\translation{Īśāna has five parts {\rm (}\textit{kalā}{\rm )}, [his Tatpuruṣa] face has four. Aghora has eight, and there are thirteen Vāmadeva[-\textit{kalā}]s. \blankfootnote{2.31 Note how \textit{vaktrasya} should refer to Śiva's Tatpuruṣa-face, 
  given that the text lists Śiva's five faces: Īśāna, Tatpuruṣa, Aghora, Vāmadeva, Sadyojāta.
 }}

  \maintext{sadyaś cāṣṭau kalā jñeyāḥ saṃsārārṇavatārakāḥ |}%

  \maintext{aṣṭatriṃśat kalā hy etāḥ kīrtitā dvijasattama }||\thinspace2:32\thinspace||%
\translation{Sadyojāta has eight parts. These parts, altogether thirty-eight, which liberate us from the ocean of existence, have been taught, O truest Brahmin. \blankfootnote{2.32 Note \textit{sadyaś} in \textit{pāda} a for \textit{sadyasaś} or \textit{sadyojātasya}.
 }}

  \maintext{saṃkhyā varṇā diśaś caiva ekaikasya pṛthak pṛthak |}%

  \maintext{pūrvoktena vidhānena bodhavyās tattvacintakaiḥ }||\thinspace2:33\thinspace||%
\translation{Those who explore the truth should know the numbers, the colours and directions associated with each one [of Śiva's faces] in the way taught above. }

  \maintext{śivāṇḍagamanākṛṣṭyā śivayogaṃ sadābhyaset |}%

  \maintext{śivayogaṃ vinā vipra tatra gantuṃ na śakyate }||\thinspace2:34\thinspace||%
\translation{If one has the intention to go to the Śivāṇḍa, one should practise Śiva-yoga regularly. Without Śiva-yoga, O Brahmin, it is impossible to go there. \blankfootnote{2.34 \textit{ākṛṣṭyā} in \textit{pāda} a might be corrupt.
 }}

  \maintext{aśvamedhādiyajñānāṃ koṭyāyutaśatāni ca |}%

  \maintext{kṛcchrāditapa sarvāṇi kṛtvā kalpaśatāni ca |}%

  \maintext{tatra gantuṃ na śakyeta devair api tapodhana }||\thinspace2:35\thinspace||%
\translation{[Even] by [performing] millions of sacrifices such as the Aśvamedha, or by performing all the difficult austerities for a hundred \textit{kalpa}s, it is impossible to get there even for the gods, O great ascetic. \blankfootnote{2.35 Understand \textit{kṛcchrāditapa sarvāṇi} as \textit{kṛcchrāditapāṃsi sarvāṇi}. It can be 
  considered an instance of the use of a stem form noun.
 }}

  \maintext{gaṅgādisarvatīrtheṣu snātvā taptvā ca vai punaḥ |}%

  \maintext{tatra gantuṃ na śakyeta ṛṣibhir vā mahātmabhiḥ }||\thinspace2:36\thinspace||%
\translation{By [merely] bathing and performing austerities at all the sacred places such as the Gaṅgā, even the honorable Ṛṣis will not be able to get there. }

  \maintext{saptadvīpasamudrāṇi ratnapūrṇāni bho dvija |}%

  \maintext{dattvā vā vedaviduṣe śraddhābhaktisamanvitaḥ |}%

  \maintext{tatra gantuṃ na śakyeta vinā dhyānena niścayaḥ }||\thinspace2:37\thinspace||%
\translation{Or [even] by donating the oceans of the seven islands with all their gems to a Veda expert, O Brahmin, with faith and devotion, one will not be able to go there without meditation. [This is a] certainty. }

  \maintext{svadehān māṃsam uddhṛtya dattvārthibhyaś ca niścayāt |}%

  \maintext{svadāraputrasarvasvaṃ śiro 'rthibhyaś ca yo dadet |}%

  \maintext{na tatra gantuṃ śakyeta anyair vāpi suduṣkaraiḥ }||\thinspace2:38\thinspace||%
\translation{He who destroys his own body and gives it without hesitation to those who are in need of it, or he who gives away his wife, his son and his possessions or his own head to those in need, or he who [performs] other difficult deeds, will not be able to go there [by merely doing these]. }

  \maintext{yajñatīrthatapodānavedādhyayanapāragaḥ |}%

  \maintext{brahmāṇḍāntasya bhogāṃs tu bhuṅkte kālavaśānugaḥ }||\thinspace2:39\thinspace||%
\translation{He who has completed the sacrifices, the pilgrimages, the austerities, the donations, the study of the Vedas, will experience those enjoyments that the Brahmāṇḍa offers, still being subject to time and death. }

  \maintext{kālena samapreṣyeṇa dharmo yāti parikṣayam |}%

  \maintext{alātacakravat sarvaṃ kālo yāti paribhraman |}%

  \maintext{traikālyakalanāt kālas tena kālaḥ prakīrtitaḥ }||\thinspace2:40\thinspace||%
\translation{Dharma decays tossed forward by time. Time flies moving everything round and round like a circle of burning coal. Time is called \textit{kāla} because of the waves {\rm (}\textit{kalana}{\rm )} of the three divisions of time [past, present, future]. \blankfootnote{2.40 Notice the muta cum liquida licence in \textit{pāda} a: \textit{samapre}° renders as short-short-long.
  I take \textit{samapreṣyena} as if it read \textit{sampreṣito}, picking up \textit{dharmo}; otherwise
  it is difficult to make sense of it.
 As Kenji Takahashi pointed out to me, \mycite{Fitzgerald_Alatacakra2012} is
  a good starting point to understand the implication of \textit{alātacakra}, 
  `a single, rapidly twirled torch creat[ing] the illusion of an apparently real, continuous circle'
  {\rm (}ibid., p. 777{\rm )}. The function of \textit{sarvaṃ} in \textit{pāda} a becomes clear only if
  we understand \textit{paribhraman} in a causative sense {\rm (}for \textit{paribhramayan}{\rm )}.
 One cannot help noticing that this verse would be in a more fitting context after verse 1.31,
  at the end of a section on \textit{kāla}. On the other hand, it leads us to the next topic, Dharma,
  smoothly.
 }}
\center{\maintext{\dbldanda\thinspace iti vṛṣasārasaṃgrahe śivāṇḍasaṃkhyā nāmādhyāyo dvitīyaḥ\thinspace\dbldanda}}
\translation{Here ends the second chapter in the \textit{Vṛṣasārasaṃgraha} called the Description of the Śivāṇḍa.}

  \chptr{tṛtīyo 'dhyāyaḥ}
\addcontentsline{toc}{subsection}{Chapter 3}
\fancyhead[CO]{{\footnotesize\textit{Translation of chapter 3}}}%

  \trchptr{  Chapter Three }%

  \subchptr{dharmapravacanam}%

  \trsubchptr{An Exposition of Dharma}%

  \maintext{vigatarāga uvāca |}%

  \maintext{kimarthaṃ dharmam ity āhuḥ katimūrtiś ca kīrtyate |}%

  \maintext{katipādavṛṣo jñeyo gatis tasya kati smṛtāḥ }||\thinspace3:1\thinspace||%
\translation{Vigatarāga spoke: Why do they call it Dharma? And how many embodiments {\rm (}\textit{mūrti}{\rm )} is he known to have? He is known as a bull: how many legs does it have? How many are his paths? \blankfootnote{3.1 For the correct interpretation of \textit{pāda} a, namely to decide whether these questions
  focus on the bull of Dharma or Dharma itself/himself, see 
  the end of the previous chapter, where \textit{dharma} was mentioned {\rm (}2.40b{\rm )},
  and to which the present verse is a reaction; see also
  \MBH\ 12.110.10--11:
   %
  \textit{prabhāvārthāya bhūtānāṃ dharmapravacanaṃ kṛtam\thinspace | 
  yat syād ahiṃsāsaṃyuktaṃ sa dharma iti niścayaḥ\thinspace || 
  dhāraṇād dharma ity āhur dharmeṇa vidhṛtāḥ prajāḥ\thinspace | 
  yat syād dhāraṇasaṃyuktaṃ sa dharma iti niścayaḥ\thinspace ||}
   %
  Note the similarities of \MBH\ this passage with this chapter: the phrase \textit{dharma ity āhur},
  the fact that the present chapter from verse 18 on is actually a chapter on \textit{ahiṃsā},
  and that the etimological explanation involves the word [\textit{ā}]\textit{dhāraṇa} in
  both cases. These lead me to think that in \textit{pāda}s ab of this verse in the \VSS,
  it is Dharma that is the focus of the inquiry and not the bull.
 
  
 Understand \textit{pāda} d as \textit{gatayas tasya kati smṛtāḥ}. I have accepted
  \textit{smṛtāḥ} because this plural signals that \textit{gatis} is meant to be plural,
  similarly to what happens in 3.6cd {\rm (}\textit{tasya patnī... mahābhāgāḥ}{\rm )}.
  The use of the singular in a context of numbers and quantities is one of 
  the hallmarks of the language of the \VSS, see p. \verify.
 
  On Dharma as a bull, see Introduction, pp. \verify.
 }}

  \maintext{kautūhalaṃ mamotpannaṃ saṃśayaṃ chindhi tattvataḥ |}%

  \maintext{kasya putro muniśreṣṭha prajās tasya kati smṛtāḥ }||\thinspace3:2\thinspace||%
\translation{I have become curious [about these questions]. Put an end to my doubts for good. Whose son is [Dharma], O best of sages? How many children does he have? }

  \maintext{anarthayajña uvāca |}%

  \maintext{dhṛtir ity eṣa dhātur vai paryāyaḥ parikīrtitaḥ |}%

  \maintext{ādhāraṇān mahattvāc ca dharma ity abhidhīyate  }||\thinspace3:3\thinspace||%
\translation{Anarthayajña spoke: Well, \textit{dhṛti} {\rm (}`firmness'{\rm )} is [of the same] verbal root [as \textit{dharma}], and is said to be [its] synonym. It is called \textit{dharma} because it supports {\rm (}\textit{āDHĀRaṇa}{\rm )} and because it is great {\rm (}\textit{MAhattva}{\rm )}. \blankfootnote{3.3 For similar Purāṇic passages on the etimology of \textit{dharma}, see the apparatus to
  this verse.
 
  The insertion in my translation '[of the same]' solves the problem of a noun {\rm (}\textit{dhṛti}{\rm )} seemingly
  being considered a verbal root {\rm (}\textit{dhātu}{\rm )} here. I owe thanks to Judit Törzsök for this interpretation.
  For similar passages with nominal stems appearently being treated as \textit{dhātu}s, see e.g. 
  \VAYUP\ 3.17cd:
  \textit{bhāvya ity eṣa dhātur vai bhāvye kāle vibhāvyate};
  \VAYUP\ 3.19cd {\rm (}= \BRAHMANDAPUR\ 1.38.21ab{\rm )}:
  \textit{nātha ity eṣa dhātur vai dhātujñaiḥ pālane smṛtaḥ};
  \LINPU\ 2.9.19:
  \textit{bhaja ity eṣa dhātur vai sevāyāṃ parikīrtitaḥ}.
 }}

  \maintext{śrutismṛtidvayor mūrtiś catuṣpādavṛṣaḥ sthitaḥ |}%

  \maintext{caturāśrama yo dharmaḥ kīrtitāni manīṣibhiḥ }||\thinspace3:4\thinspace||%
\translation{The four-legged Bull is the embodiment of both Śruti and Smṛti. It is Dharma, as made up of the four \textit{āśrama}s. \blankfootnote{3.4 A similar image of the legs of the Bull of Dharma being the four {\rm (}and not three, at least according to
  \mycitep{OlivelleAsrama}{55} and
  \mycitep{GanguliMBh}{Śāntiparvan CCLXX}{\rm )} 
  \textit{āśrama}s is hinted at \MBH\ 12.262.19--21: 
   %
  \textit{dharmam ekaṃ catuṣpādam āśritās te nararṣabhāḥ\thinspace |
   taṃ santo vidhivat prāpya gacchanti paramāṃ gatim\thinspace ||
   gṛhebhya eva niṣkramya vanam anye samāśritāḥ\thinspace |
   gṛham evābhisaṃśritya tato 'nye brahmacāriṇaḥ\thinspace ||
   dharmam etaṃ catuṣpādam āśramaṃ brāhmaṇā viduḥ\thinspace |
   ānantyaṃ brahmaṇaḥ sthānaṃ brāhmaṇā nāma niścayaḥ\thinspace ||}.
   %
  On the more frequently quoted interpretation of the four legs, see 
  \mycitep{OlivelleAsrama}{235}, a translation of \MANU\ 1.81--82:
  `Dharma and truth possess all four feet and are whole during the Kṛta yuga, 
  and people did not obtain anything unrighteously {\rm (}\textit{adharmeṇa}{\rm )}. 
  By obtaining, however, \textit{dharma} has lost one foot during each of the other \textit{yuga}s 
  and righteousness {\rm (}\textit{dharma}{\rm )} likewise has diminished by one quarter due to theft, 
  falsehood, and deceit. {\rm (}MDh 1.81--82{\rm )}.'
   %
  Understand \textit{pāda}s c and d as \textit{catvāri āśramāṇi kīrtitāni dharmo manīṣibhiḥ} or
  \textit{yo dharmaḥ kīrtitaś caturāśramāṇi manīṣibhiḥ} or 
  \textit{yo dharmaś caturāśramaḥ kīrtito manīṣibhiḥ}. Judit Törzsök suggested
  that \textit{caturāśrama} and \textit{dharmaḥ} may be interpreted as a compound here.
 }}

  \maintext{gatiś ca pañca vijñeyāḥ śṛṇu dharmasya bho dvija |}%

  \maintext{devamānuṣatiryaṃ ca narakasthāvarādayaḥ }||\thinspace3:5\thinspace||%
\translation{And the paths of Dharma are five. Listen, O Brahmin: [existence as] gods, men, animals, [existence in] hell and [as] immovable things [such as plants and rocks] etc. \blankfootnote{3.5 Note the use of the singular next to numbers in \textit{pāda} a, as in 3.1d, and that
  \textit{vijñeyāḥ} is an emendation from \textit{vijñeyaḥ} following the logic of 3.1d.
 \textit{tirya} seems to be an acceptable nominal stem in this text for \textit{tiryañc}. See,
  e.g., 4.6a: \textit{devamānuṣatiryeṣu}. \textit{°ādayaḥ} in \textit{pāda} d seems superfluous.
 }}

  \maintext{brahmaṇo hṛdayaṃ bhittvā jāto dharmaḥ sanātanaḥ |}%

  \maintext{tasya patnī mahābhāgā trayodaśa sumadhyamāḥ }||\thinspace3:6\thinspace||%
\translation{Eternal Dharma was born after splitting Brahmā's heart. He has beautiful wives, thirteen in number, with nice waists. \blankfootnote{3.6 Note the use of the singular in \textit{pāda}s cd. I have left \textit{sumadhyamāḥ} as the
  manuscripts transmit it: it signals the presence of the plural. And consider 
  correcting \textit{mahābhāgā} to \textit{mahābhāgās}. In sum, understand
  \textit{tasya patnyo mahābhāgās trayodaśa sumadhyamāḥ}.
 }}

  \maintext{dakṣakanyā viśālākṣī śraddhādyāḥ sumanoharāḥ |}%

  \maintext{tasya putrāś ca pautrāś ca anekāś ca babhūva ha |}%

  \maintext{eṣa dharmanisargo 'yaṃ kiṃ bhūyaḥ śrotum icchasi }||\thinspace3:7\thinspace||%
\translation{They are Dakṣa's daughters, [called] Śraddhā and so on. They have huge eyes and they are beautiful. Numerous sons and grandsons were born to him. This is the emergence of Dharma. What more do you wish to hear? \blankfootnote{3.7 \textit{śraddhāḍhyāḥ} in \textit{pāda} b is an attractive lectio difficilior {\rm (}`they were rich in faith/devotion'{\rm )}, but I have finally 
  decided to accept the easier and better-attested \textit{śraddhādyā}[\textit{ḥ}].
  Again, I have chosen/applied the plural forms \textit{°ādyāḥ} and \textit{sumanoharāḥ} in \textit{pāda} b to hint at the fact
  that the presence of the plural is to be preferred here; thus only \textit{viśālākṣī} is 
  problematic. As \textit{patnī} in the previous verse, it should be treated as a plural.
  Note the use of the singular for the plural also in \textit{pāda}s cd, especially \textit{babhūva ha} for \textit{babhūvuḥ}
  {\rm (}\textit{babhūva ha} perhaps being a phonetic and metrically `adjusted' equivalent, so to say, of \textit{babhūvuḥ}{\rm )}.
 }}

  \maintext{vigatarāga uvāca |}%

  \maintext{dharmapatnī viśeṣeṇa putras tābhyaḥ pṛthak pṛthak |}%

  \maintext{śrotum icchāmi tattvena kathayasva tapodhana }||\thinspace3:8\thinspace||%
\translation{Vigatarāga spoke: I would like to hear about Dharma's wives truly and about each one of the sons born to them. Teach me, O great ascetic. \blankfootnote{3.8 I have emended \textit{tebhyaḥ} to the correct feminine form \textit{tābhyaḥ}
  because I suspect that it is only the result of some early confusion
  brought about by \textit{putras}, although \textit{tebhyaḥ} might be original.
  Note again the use of the singular {\rm (}nominative{\rm )} for the plural {\rm (}accusative{\rm )} in \textit{pāda}s ab.
  Alternatively, emend \textit{dharmapatnī} to \textit{dharmapatnīr} {\rm (}plural accusative{\rm )} and 
  \textit{putras} to \textit{putrān} to make them work with \textit{śrotum icchāmi}.
 }}

  \maintext{anarthayajña uvāca |}%

  \maintext{śraddhā lakṣmīr dhṛtis tuṣṭiḥ puṣṭir medhā kriyā lajjā |}%

  \maintext{buddhiḥ śāntir vapuḥ kīrtiḥ siddhiḥ prasūtisambhavāḥ }||\thinspace3:9\thinspace||%
\translation{Anarthayajña spoke: [Dharma's wives are] [1] Śraddhā {\rm (}`Faith'{\rm )}, [2] Lakṣmī {\rm (}`Prosperity'{\rm )}, [3] Dhṛti {\rm (}`Resolution'{\rm )}, [4] Tuṣṭi {\rm (}`Satisfaction'{\rm )}, [5] Puṣṭi {\rm (}`Growth'{\rm )}, [6] Medhā {\rm (}`Wisdom'{\rm )}, [7] Kriyā {\rm (}`Labour'{\rm )}, [8] Lajjā {\rm (}`Modesty'{\rm )}, [9] Buddhi {\rm (}`Intelligence'{\rm )}, [10] Śānti {\rm (}`Tranquillity'{\rm )}, [11] Vapus {\rm (}`Beauty'{\rm )}, [12] Kīrti {\rm (}`Fame'{\rm )}, [13] Siddhi {\rm (}`Success'{\rm )}, [all] born to Prasūti [Dakṣa's wife]. \blankfootnote{3.9 Note how \textit{lajjā} in \textit{pāda} b makes the line unumetrical.
 
  For Dharma's thirteen wives and their sons, see, e.g., \LINPU\ 1.5.34--37 {\rm (}note the 
  similarity between the first line and \VSS\ 3.6cd--7ab above{\rm )}:
   %
  \textit{dharmasya patnyaḥ śraddhādyāḥ kīrtitā vai trayodaśa\thinspace |
   tāsu dharmaprajāṃ vakṣye yathākramam anuttamam\thinspace ||
   kāmo darpo 'tha niyamaḥ saṃtoṣo lobha eva ca\thinspace |
   śrutas tu daṇḍaḥ samayo bodhaś caiva mahādyutiḥ\thinspace ||
   apramādaś ca vinayo vyavasāyo dvijottamāḥ\thinspace |
   kṣemaṃ sukhaṃ yaśaś caiva dharmaputrāś ca tāsu vai\thinspace || 
   dharmasya vai kriyāyāṃ tu daṇḍaḥ samaya eva ca\thinspace |
   apramādas tathā bodho buddher dharmasya tau sutau\thinspace ||}.
   %
 
  \textit{prasūtisambhavāḥ} in \textit{pāda} d is a rather bold conjecture that can be supported by two facts:
  firstly, the readings of the manuscripts are difficult to make sense of and thus are
  probably corrupt; secondly, a corruption from the name Prasūti,
  traditionally the name of Dakṣa's wife, to \textit{ābhūti}
  is relatively easily to explain, \textit{sū} and \textit{bhū} being close enough in some scripts 
  {\rm (}e.g. in \msCa{\rm )} to cause confusion. Another option would be to accept 
  Ābhūti as the name of Dakṣa's wife.
   %
  For Prasūti being Dakṣa's wife in other sources,
  see, e.g., \LINPU\ 1.5.20--21 {\rm (}but also note the presence of the name Sambhūti{\rm )}:
  \textit{prasūtiḥ suṣuve dakṣāc caturviṃśatikanyakāḥ\thinspace |
  śraddhāṃ lakṣmīṃ dhṛtiṃ puṣṭiṃ tuṣṭiṃ medhāṃ kriyāṃ tathā\thinspace ||
  buddhi lajjāṃ vapuḥ śāntiṃ siddhiṃ kīrtiṃ mahātapāḥ\thinspace |
  khyātiṃ śāntiś ca saṃbhūtiṃ smṛtiṃ prītiṃ kṣamāṃ tathā\thinspace ||}.
 }}

  \maintext{śraddhā kāmaḥ suto jāto darpo lakṣmīsutaḥ smṛtaḥ |}%

  \maintext{dhṛtyās tu niyamaḥ putraḥ saṃtoṣas tuṣṭijaḥ smṛtaḥ }||\thinspace3:10\thinspace||%
\translation{Śraddhā's son is Kāma {\rm (}`Desire'{\rm )}. Darpa {\rm (}`Pride'{\rm )} is said to be Lakṣmī's son. Dhṛti's son is Niyama {\rm (}`Rule'{\rm )}. Saṃtoṣa {\rm (}`Satisfaction'{\rm )} is Tuṣṭi's son. \blankfootnote{3.10 Understand \textit{śraddhā} as a stem form noun for \textit{śraddhāyāḥ} {\rm (}gen./abl., cf. 3.11a{\rm )}.
  Alternatively, take \textit{śraddhā} and \textit{suto} as elements of a split compound, and understand
  \textit{śraddhāsuto jātaḥ kāmaḥ}.
 }}

  \maintext{puṣṭyā lābhaḥ suto jāto medhāputraḥ śrutas tathā |}%

  \maintext{kriyāyās tv abhavat putro daṇḍaḥ samaya eva ca }||\thinspace3:11\thinspace||%
\translation{To Puṣṭi was born a son [called] Lābha {\rm (}`Profit'{\rm )}. Medhā's son is Śruta {\rm (}`Sacred Knowledge'{\rm )}. Kriyā's sons are Daṇḍa {\rm (}`Punishment'{\rm )} and Samaya {\rm (}`Law'{\rm )}. \blankfootnote{3.11 I have emended \textit{abhayaḥ} to \textit{abhavat} in \textit{pāda} c, following the relevant line in the \KURMP\ cited above
  {\rm (}\textit{kriyāyāś cābhavat putro daṇḍaḥ samaya eva ca}{\rm )} and also \LINPU\ 1.5.37 quoted in the 
  apparatus to this verse, allotting only two sons to Kriyā. Thus I don't think
  that Kriyā is supposed to have a son called Abhaya {\rm (}`Freedom from danger'; \BHAGP\ 4.1.50ab 
  claims that Dayā had a son called Abhaya:
  \textit{śraddhāsūta śubhaṃ maitrī prasādam abhayaṃ dayā}{\rm )}.
  Nevertheless, in a number of sources Kriyā actually has three sons, 
  see, e.g., \VISNUP\ 1.7.26ab,
  where they are named as Daṇḍa, Naya and Vinaya:
  \textit{medhā śrutaṃ kriyā daṇḍaṃ nayaṃ vinayam eva ca}. 
  Perhaps read \textit{kriyāyās tu nayaḥ putro} in \textit{pāda} c? Compare \VAYUP\ 1.10.34cd
  {\rm (}\textit{kriyāyās tu nayaḥ prokto daṇḍaḥ samaya eva ca}{\rm )} 
  with \BRAHMANDAPUR\ 1.9.60ab {\rm (}\textit{kriyāyās tanayau proktau damaś ca śama eva ca}{\rm )}.
 }}

  \maintext{lajjāyā vinayaḥ putro buddhyā bodhaḥ sutaḥ smṛtaḥ |}%

  \maintext{lajjāyāḥ sudhiyaḥ putra apramādaś ca tāv ubhau }||\thinspace3:12\thinspace||%
\translation{Lajjā's son is Vinaya {\rm (}`Discipline'{\rm )}, Buddhi's son is Bodha {\rm (}`Intelligence'{\rm )}. Lajjā has two [more] sons: Sudhiya[/Sudhī] {\rm (}`Wise'{\rm )} and Apramāda {\rm (}`Cautiousness'{\rm )}. \blankfootnote{3.12 In a very similar passages in \KURMP\ 1.8.20 ff., Apramāda is Buddhi's son and 
  Lajjā has only one son, Vinaya. In the above verse {\rm (}\VSS\ 3.12{\rm )}, \textit{sudhiyaḥ} {\rm (}for \textit{sudhīḥ}{\rm )} may only be 
  qualifying \textit{apramāda}, thus Lajjā may have two sons: Vinaya and the wise Apramāda.
  Alternatively, \textit{pāda}s cd might be a extra line inserted accidentally.
 }}

  \maintext{kṣemaḥ śāntisuto vindyād vyavasāyo vapoḥ sutaḥ |}%

  \maintext{yaśaḥ kīrtisuto jñeyaḥ sukhaṃ siddher vyajāyata |}%

  \maintext{svāyambhuve 'ntare tv āsan kīrtitā dharmasūnavaḥ }||\thinspace3:13\thinspace||%
\translation{Kṣema {\rm (}`Peace'{\rm )} is to be known as Śānti's son, Vyavasāya {\rm (}`Resolution'{\rm )} is Vapus' son. Yaśas {\rm (}`Fame'{\rm )} is Kīrti's son, Sukha {\rm (}`Joy'{\rm )} was born to Siddhi. [This is how] the sons of Dharma in the [\textit{manvantara}] era of Svāyambhuva [Manu] were known. \blankfootnote{3.13 Note that \textit{sukhaṃ} in \textit{pāda} d is probably meant to be masculine {\rm (}\textit{sukhaḥ}{\rm )}, but e.g. in the 
  \KURMP\ passage quoted above it is also neuter. For the emendation in \textit{pāda} e, 
  see \MATSP\ 9.2cd: 
  \textit{yāmā nāma purā devā āsan svāyambhuvāntare},
  and \BHAGP\ 6.4.1: 
  \textit{devāsuranṛṇāṃ sargo nāgānāṃ mṛgapakṣiṇām\thinspace |
  sāmāsikas tvayā prokto yas tu svāyambhuve 'ntare\thinspace ||}.
 }}

  \maintext{vigatarāga uvāca |}%

  \maintext{mūrtidvayaṃ kathaṃ dharmaṃ kathayasva tapodhana |}%

  \maintext{kautūhalam atīvaṃ me kartaya jñānasaṃśayam }||\thinspace3:14\thinspace||%
\translation{Vigatarāga spoke: How come Dharma has two embodiments? Tell me, O great ascetic. I am extremely intrigued. Cut my doubts concerning [this] knowledge. \blankfootnote{3.14 Note \textit{dharma} as a neuter noun and the form \textit{atīvaṃ} for \textit{atīva} metri causa. 
  My emen\-dation from \textit{kīrtaya} {\rm (}`declare'{\rm )} to \textit{kartaya} {\rm (}`cut'{\rm )} was influenced by the combination
  of \textit{chindhi} and \textit{saṃśaya}, often with \textit{kautūhala}, elsewhere in the \VSS:
  3.2ab: \textit{kautūhalaṃ mamotpannaṃ saṃśayaṃ chindhi tattvataḥ}; 
  10.10cd: \textit{kautūhalaṃ mahaj jātaṃ chindhi saṃśayakārakam};
  15.2ab: \textit{etat kautūhalaṃ chindhi saṃśayaṃ parameśvara}. 
  The reading \textit{kīrtaya} may have been the result of the influence of \textit{kīrtitā} in 3.13b above 
  {\rm (}De Simini's observation{\rm )}.
 }}

  \maintext{anarthayajña uvāca |}%

  \maintext{śrutismṛtidvayor mūrtir dharmasya parikīrtitā |}%

  \maintext{dārāgnihotrasambandham ijyā śrautasya lakṣaṇam |}%

  \maintext{smārto varṇāśramācāro yamaiś ca niyamair yutaḥ }||\thinspace3:15\thinspace||%
\translation{Anarthayajña spoke: Dharma's embodiment is said to consist of Śruti and Smṛti. The characteristics of the Śrauta [tradition] are an association with a wife [i.e.\ marriage] and with the fire ritual, and sacrifice. The Smārta [tradition] [focuses on] the conduct {\rm (}\textit{ācāra}{\rm )} of the classes {\rm (}\textit{varṇa}{\rm )} and life-stages {\rm (}\textit{āśrama}{\rm )} which is connected to rules and regulations {\rm (}\textit{yama-niyama}{\rm )}. \blankfootnote{3.15 The reading \textit{°dvayī} in \msNc\ in \textit{pāda} a is attractive, but as Judit 
  Törzsök has pointed out to me, it is more likely that
  the slightly less convincing but widespread variant \textit{°dvayor} is original.
 
  As for Dharma being based on \textit{śruti} and \textit{smṛti}, see, e.g., \MANU\ 2.10:
  \textit{śrutis tu vedo vijñeyo dharmaśāstraṃ tu vai smṛtiḥ\thinspace |
  te sarvārtheṣv amīmāṃsye tābhyāṃ dharmo hi nirbabhau\thinspace ||}.
  In Olivelle's translation {\rm (}\mycitep{OlivelleManu}{94}{\rm )}:
  `\thinspace ``Scripture'' should be recognized as ``Veda,'' and ``tradition''
  as ``Law Treatise.'' These two should never be called into question in any matter,
  for it is from them that the Law shines forth.'
 
  
  There may be a hiatus filler in \textit{pāda}s cd: \textit{°sambandha-m-ijyā} for \textit{°sambandha ijyā}.
 
  To state that the Smārta tradition is connected to \textit{yama}s and \textit{niyama}s and the \textit{āśrama}s and
  then to discuss these at length {\rm (}principally in chapters 3--8 and 11{\rm )} can be seen 
  as a clear self-identification with the Smārta tradition.
 }}

  \subchptr{yamaniyamabhedaḥ}%

  \trsubchptr{Yama and Niyama rules}%

  \maintext{yamaś ca niyamaś caiva dvayor bhedam ataḥ śṛṇu |}%

  \maintext{ahiṃsā satyam asteyam ānṛśaṃsyaṃ damo ghṛṇā |}%

  \maintext{dhanyāpramādo mādhuryam ārjavaṃ ca yamā daśa }||\thinspace3:16\thinspace||%
\translation{Now hear the classification of both the \textit{yama} and \textit{niyama} rules. Non-violence, truthfulness, not stealing, absence of hostility, self-restraint, taboos, virtue, carefulness, charm, honesty: these are the ten \textit{yama}s. \blankfootnote{3.16 \textit{Pāda} a should be understood as \textit{yamaniyamayoś caiva}, but the author of this line
  may have tried to avoid the metrical fault of having two short syllables 
  in second and third position.
 Note that this is the beginning of a long section in our text
  that describes the \textit{yama-niyama} rules, reaching up to the end of chapter eight. 
  The title given in the colophon of the next chapter, chapter four, namely \textit{yamavibhāga},
  would fit this locus better than the beginning of that chapter, which 
  commences with a discussion on the second of the \textit{yama}s, \textit{satya}.
 Note how all witnesses read \textit{mādhūrya} in \textit{pāda} e instead of \textit{mādhurya}. The former may have been
  acceptable originally in this text. \textit{Pāda} e is a \textit{ma-vipulā}.
 }}

  \maintext{ekaikasya punaḥ pañcabhedam āhur manīṣiṇaḥ |}%

  \maintext{ahiṃsādi pravakṣyāmi śṛṇuṣvāvahito dvija }||\thinspace3:17\thinspace||%
\translation{The wise say that there are five subclasses to each. I shall teach you about non-violence and the other [\textit{yama}-rules]. Listen carefully, O twice-born. \blankfootnote{3.17 In \textit{pāda} a, \textit{pañca} and \textit{bheda} may be typeset as two separate words since
  the use of the singular after numbers is one of the hallmarks of the text {\rm (}see \verify{\rm )}.
 }}

  \subchptr{yameṣv ahiṃsā {\rm {\rm (}1{\rm )}}}%

  \trsubchptr{The first Yama-rule: Non-violence}%

  \subsubchptr{pañcavidhā hiṃsā}%

  \trsubsubchptr{Five types of violence}%

  \maintext{trāsanaṃ tāḍanaṃ bandho māraṇaṃ vṛttināśanam |}%

  \maintext{hiṃsāṃ pañcavidhām āhur munayas tattvadarśinaḥ }||\thinspace3:18\thinspace||%
\translation{Frightening and beating [other people], tying [someone] up, killing and the destruction of [other people's] livelihood: violence is said by the wise who see the truth to be of [these] five types. }

  \maintext{kāṣṭhaloṣṭakaśādyais tu tāḍayantīha nirdayāḥ |}%

  \maintext{tatprahāravibhinnāṅgo mṛtavadhyam avāpnuyāt }||\thinspace3:19\thinspace||%
\translation{Cruel people beat [other people] with sticks, clods of earth [understand: they stone them], with whips and other [objects] in the everyday world. Their bodies broken by the same blows, they receive the capital punishment. \blankfootnote{3.19 Note the use of the singular in \textit{pāda}s cd referring back to the agents of the previous sentence.
  Most probably, °\textit{vadhyam} is to be understand as °\textit{vadham} and the form 
  \textit{vadhyam} serves only to avoid two \textit{laghu} syllables in \textit{pāda} d.
 }}

  \maintext{baddhvā pādau bhujoraś ca śirorukkaṇṭhapāśitāḥ |}%

  \maintext{anāhatā mriyanty evaṃ vadho bandhanajaḥ smṛtaḥ }||\thinspace3:20\thinspace||%
\translation{[Others,] tie up [people] at their feet and their arms and chests. [These,] hung by their hair and neck, die in this way without being wounded. This is the capital punishment for tying up [other people]. \blankfootnote{3.20 Understand \textit{bhujoraś ca} in \textit{pāda} a as \textit{bhuje, urasi ca}, in this case with an instance of double sandhi,
  and in stem form: \textit{bhuje urasi ca} $\rightarrow$\ \textit{bhuja urasi ca} 
  $\rightarrow$\ \textit{bhujorasi ca} $\rightarrow$\ \textit{bhujoraś ca}.
  Alternatively, understand it as a compound {\rm (}\textit{bhujorasi}{\rm )}. 
  In \textit{pāda} b, my emendation is only one of the possible interpretations. We might accept
  \textit{śiroru}° as consisting of \textit{śira} + \textit{ūru} {\rm (}`head and thigh'{\rm )}, or emend it 
  to \textit{śiroraḥ}° for \textit{śira} + \textit{uraḥ} {\rm (}`head and chest'{\rm )}. Also note my conjecture
  in \textit{pāda} d, without which this \textit{pāda} is difficult to interpret.
 }}

  \maintext{śatrucaurabhayair ghoraiḥ siṃhavyāghragajoragaiḥ |}%

  \maintext{trāsanād vadham āpnoti anyair vāpi suduḥsahaiḥ }||\thinspace3:21\thinspace||%
\translation{He who frightens [other people] with the terrible danger of enemies and thieves, with lions, tigers, elephants or snakes, or by other horrors, will be executed. }

  \maintext{yasya yasya hared vittaṃ tasya tasya vadhaḥ smṛtaḥ |}%

  \maintext{vṛttijīvābhibhūtānāṃ taddvārā nihataḥ smṛtaḥ }||\thinspace3:22\thinspace||%
\translation{He who robs somebody's money is to be punished by the same person. He is [to be] struck down by those whose livelihood got damaged by him. \blankfootnote{3.22 Understand \textit{vadhaḥ} in \textit{pāda} b as \textit{vadhyaḥ} metri causa.
 My translation of the second line of this verse reflects a conjecture {\rm (}\textit{taddvārā}{\rm )}
  understood as connected to both \textit{pāda} c and \textit{nihataḥ} in \textit{pāda} d.
 }}

  \maintext{viṣavahniśaraśastrair māyāyogabalena vā |}%

  \maintext{hiṃsakāny āhu viprendra munayas tattvadarśinaḥ }||\thinspace3:23\thinspace||%
\translation{[Those who kill other people] with poison, fire, arrows, swords, or by the force of magic or yoga are called murderers by the sages who see the truth, O great Brahmin. \blankfootnote{3.23 \textit{Pāda} a is a \textit{sa-vipulā} with two \textit{laghu}s.
  Note how elliptical this verse is and that \textit{hiṃsakāni} is neuter although it refers to 
  people, perhaps implying \textit{bhūtāni}. Alternatively, take \textit{y} in \textit{hiṃsakāny} as a 
  rather unusual sandhi-bridge {\rm (}\textit{hiṃsakān-y-āhu}{\rm )}, or simply delete this \textit{y}. 
  Note also that \textit{āhu} stands for \textit{āhur} metri causa.
 }}

  \subsubchptr{ahiṃsāpraśaṃsā}%

  \trsubsubchptr{Praise of non-violence}%

  \maintext{ahiṃsā paramaṃ dharmaṃ yas tyajet sa durātmavān |}%

  \maintext{kleśāyāsavinirmuktaṃ sarvadharmaphalapradam }||\thinspace3:24\thinspace||%
\translation{Non-violence is the highest Dharma. He who abandons it is a wicked person. It is free of pain and trouble, it yields the fruits of all [other] Dharmic teachings [in itself]. \blankfootnote{3.24 Note \textit{dharma} as a neuter noun in \textit{pāda} a and that \textit{°vinirmuktaṃ} and
  \textit{°pradam} are neuter accordingly.
 }}

  \maintext{nātaḥ parataro mūrkho nātaḥ parataraṃ tamaḥ |}%

  \maintext{nātaḥ parataraṃ duḥkhaṃ nātaḥ parataro 'yaśaḥ }||\thinspace3:25\thinspace||%
\translation{There isn't a bigger fool than he [who abandons it]. There is no bigger mental darkness [than the abandonment of non-violence]. There is no greater suffering or greater infamy. \blankfootnote{3.25 Note that \textit{parataro} is masculine in \textit{pāda} d, picking up a neuter \textit{'yaśaḥ}.
  This phenomenon is probably the result of \textit{'yaśaḥ} resembling a masculine noun ending in \textit{-aḥ}
  and also of the metrical problem with a grammatically correct \textit{nātaḥ parataram ayaśaḥ}.
 }}

  \maintext{nātaḥ parataraṃ pāpaṃ nātaḥ parataraṃ viṣam |}%

  \maintext{nātaḥ paratarāvidyā nātaḥ paraṃ tapodhana }||\thinspace3:26\thinspace||%
\translation{There is no greater sin or a more effective poison. There is no greater ignorance, there is nothing worse, O great ascetic. \blankfootnote{3.26 \textit{Pāda} d {\rm (}\textit{nātaḥ paraṃ tapodhana}{\rm )} is slightly suspect. 
  The vocative \textit{tapodhana} usually refers to Anarthayajña in these
  passages, and not to Vigatarāga, as here. The text may have read \textit{nātaḥ paratamo 'dhanaḥ} 
  {\rm (}`There is no bigger loss of wealth'{\rm )} or possibly something starting with
  \textit{nātaḥ paraṃ tapo ...} {\rm (}`There is no greater\dots\ of austerity'{\rm )}.
 }}

  \maintext{yo hinasti na bhūtāni udbhijjādi caturvidham |}%

  \maintext{sa bhavet puruṣaḥ śreṣṭhaḥ sarvabhūtadayānvitaḥ }||\thinspace3:27\thinspace||%
\translation{He who does not harm the four types of living beings beginning with plants is the best person, having compassion for all creatures. }

  \maintext{sarvabhūtadayāṃ nityaṃ yaḥ karoti sa paṇḍitaḥ |}%

  \maintext{sa yajvā sa tapasvī ca sa dātā sa dṛḍhavrataḥ }||\thinspace3:28\thinspace||%
\translation{He who always has compassion for all creatures is the [true] Pandit. He is the [true] sacrificer, the [true] ascetic, he is the donor, the one with a firm vow. }

  \maintext{ahiṃsā paramaṃ tīrtham ahiṃsā paramaṃ tapaḥ |}%

  \maintext{ahiṃsā paramaṃ dānam ahiṃsā paramaṃ sukham }||\thinspace3:29\thinspace||%
\translation{Non-violence is the supreme pilgrimage place. Non-violence is the highest austerity. Non-violence is the highest donation. Non-violence is the highest joy. }

  \maintext{ahiṃsā paramo yajñaḥ ahiṃsā paramaṃ vratam |}%

  \maintext{ahiṃsā paramaṃ jñānam ahiṃsā paramā kriyā }||\thinspace3:30\thinspace||%
\translation{Non-violence is the supreme sacrifice. Non-violence is the supreme religious observance. Non-violence is supreme knowledge. Non-violence is the supreme ritual. }

  \maintext{ahiṃsā paramaṃ śaucam ahiṃsā paramo damaḥ |}%

  \maintext{ahiṃsā paramo lābhaḥ ahiṃsā paramaṃ yaśaḥ }||\thinspace3:31\thinspace||%
\translation{Non-violence is the highest purity. Non-violence is the highest self-restraint. Non-violence is the highest profit. Non-violence is the greatest fame. }

  \maintext{ahiṃsā paramo dharmaḥ ahiṃsā paramā gatiḥ |}%

  \maintext{ahiṃsā paramaṃ brahma ahiṃsā paramaḥ śivaḥ }||\thinspace3:32\thinspace||%
\translation{Non-violence is the supreme Dharma. Non-violence is the supreme path. Non-violence is the supreme Brahman. Non-violence is supreme Śiva. }

  \subsubchptr{māṃsāhāraḥ}%

  \trsubsubchptr{On meet-consumption}%

  \maintext{māṃsāśanān nivarteta manasāpi na kāṅkṣayet |}%

  \maintext{sa mahat phalam āpnoti yas tu māṃsaṃ vivarjayet }||\thinspace3:33\thinspace||%
\translation{One should refrain from meat-consumption. One should not even desire it mentally. He who abandons meat will receive a great reward. }

  \maintext{svamāṃsaṃ paramāṃsena yo vardhayitum icchati |}%

  \maintext{anabhyarcya pitṝn devān na tato 'nyo 'sti pāpakṛt }||\thinspace3:34\thinspace||%
\translation{He who wishes to nourish his own flesh with the flesh of other [beings], outside of worshipping the ancestors and the gods, is the biggest sinner of all. \blankfootnote{3.34 See \UUMS\ chapter two for a similar section on meat-consumption.
 }}

  \maintext{madhuparke ca yajñe ca pitṛdaivatakarmaṇi |}%

  \maintext{atraiva paśavo hiṃsyā nānyatra manur abravīt }||\thinspace3:35\thinspace||%
\translation{During the \textit{madhuparka} offering and during a sacrifice, during rituals for the ancestors and the gods: only in these cases are animals to be slaughtered and not in any other case. [This is what] Manu taught. \blankfootnote{3.35 This verse is a variant of \MANU\ 5.41.
 }}

  \maintext{krītvā svayaṃ vāpy utpādya paropahṛtam eva vā |}%

  \maintext{devān pitṝṃś cārcayitvā khādan māṃsaṃ na doṣabhāk }||\thinspace3:36\thinspace||%
\translation{Should he buy it or procure it himself or should it be offered by others, if he eats meat, he will not sin if he first worships the gods and the ancestors. }

  \maintext{vedayajñatapastīrthadānaśīlakriyāvrataiḥ |}%

  \maintext{māṃsāhāranivṛttānāṃ ṣoḍaśāṃśaṃ na pūryate }||\thinspace3:37\thinspace||%
\translation{[People who perform] Vedic sacrifices and austerities, and [visit] sacred places, donate, [those who are of] good conduct, [perform] rituals and [keep] religious vows, [but eat meat] will not [be able to] enjoy even a tiny portion of [such rewards that] [those] people [receive] who have given up meat. \blankfootnote{3.37 As for \textit{pāda} d, see a similarly phrased comparison in \MANU\ 2.86:
   %
  \textit{ye pākayajñās catvāro vidhiyajñasamanvitāḥ\thinspace | 
  sarve te japayajñasya kalāṃ nārhanti ṣoḍaśīm\thinspace ||}.
 }}

  \maintext{mṛgāḥ parṇatṛṇāhārād ajameṣagavādibhiḥ |}%

  \maintext{sukhino balavantaś ca vicaranti mahītale }||\thinspace3:38\thinspace||%
\translation{Deer and goats, sheep, cows and other [animals] wander in the world happily and in great strength [just] from eating leaves and grass. }

  \maintext{vānarāḥ phala{-}m{-}āhārā rākṣasā rudhirapriyāḥ |}%

  \maintext{nihatā rākṣasāḥ sarve vānaraiḥ phalabhojibhiḥ }||\thinspace3:39\thinspace||%
\translation{Monkeys eat fruits, Rākṣasas prefer blood. The fruit-eating monkeys defeated all the Rākṣasas. \blankfootnote{3.39 Understand \textit{phalam āhārā} as \textit{phalāhārā} {\rm (}\textit{-m-} is a sandhi-bridge{\rm )}.
 This verse clearly refers to the story of the \textit{Rāmāyaṇa}.
 }}

  \maintext{tasmān māṃsaṃ na hīheta balakāmena bho dvija |}%

  \maintext{balena ca guṇākarṣāt parato bhayabhīruṇā }||\thinspace3:40\thinspace||%
\translation{Therefore one should not crave meat in the hope of gaining strength, O Brahmin, in order to be able to draw a bow with force, or out of fear of the danger coming from the enemy. \blankfootnote{3.40 \textit{guṇākāśāt} in pāda c is difficult to interpret and 
  \textit{guṇākarṣāt} is a conjecture by Judit Törzsök which fits the context well,
  although the polysemy of \textit{guṇa} may allow for other solutions.
   %
  Verses 3.40--42 may be echoing \BRAHMANDAPUR\ 216.64--66:
   %
  \textit{ māṃsān miṣṭataraṃ nāsti bhakṣyabhojyādikeṣu ca\thinspace | 
  tasmān māṃsaṃ na bhuñjīta nāsti miṣṭaiḥ sukhodayaḥ\thinspace ||  
  gosahasraṃ tu yo dadyād yas tu māṃsaṃ na bhakṣayet\thinspace | 
  samāv etau purā prāha brahmā vedavidāṃ varaḥ\thinspace || 
  sarvatīrtheṣu yat puṇyaṃ sarvayajñeṣu yat phalam\thinspace | 
  amāṃsabhakṣaṇe viprās tac ca tac ca ca tatsamam\thinspace ||}.
 }}

  \maintext{ahiṃsakasamo nāsti dānayajñasamīhayā |}%

  \maintext{iha loke yaśaḥ kīrtiḥ paratra ca parā gatiḥ }||\thinspace3:41\thinspace||%
\translation{By wishing to make donations and perform sacrifices no one will become comparable to someone who refrains from violence. [He will have] fame and glory in this world and the supreme path in the other. \blankfootnote{3.41 \textit{Pāda}s ab are reminescent of \SDHS\ 11.92:  %
  \textit{ahiṃsaikā paro dharmaḥ śaktānāṃ parikīrtitam\thinspace | 
  aśaktānām ayaṃ dharmo dānayajñādipūrvakaḥ\thinspace ||}. 
  On this verse see also \mycitep{SaivaUtopia2021}{15--16}.
 
  Note the variant \textit{°dharma°} in both \msCc\ and \Ed\ in \textit{pāda} b.
 }}

  \maintext{trailokyaṃ maṇiratnapūrṇam akhilaṃ dattvottame brāhmaṇe}%

 \nonanustubhindent \maintext{koṭīyajñasahasrapadmam ayutaṃ dattvā mahīṃ dakṣiṇām |}%

  \maintext{tīrthānāṃ ca sahasrakoṭiniyutaṃ snātvā sakṛn mānava}%

 \nonanustubhindent \maintext{etatpuṇyaphalam ahiṃsakajanaḥ prāpnoti niḥsaṃśayaḥ }||\thinspace3:42\thinspace||%
\translation{A person who refrains from violence will gain, no doubt about it, the [same] meritorious rewards that others would get by donating the three worlds filled with jewels and gems in their entirety to an excellent Brahmin, by [performing] a thousand [times] ten trillion {\rm (}\textit{padma}{\rm )} [times] ten thousand {\rm (}\textit{ayuta}{\rm )} \textit{koṭīyajña} sacrifices, by donating the earth [to a priest] as sacrificial fee, and by bathing [at] a thousand times ten million times a million {\rm (}\textit{niyuta}{\rm )} sacred places at once. \blankfootnote{3.42 Metre: \textit{śārdūlavikrīḍita}. 
  On \textit{padma} meaning `ten trillion', and on other words for numbers, see 1.32--35. 
  \textit{koṭīyajña} in \textit{pāda} d may refer to a special kind of sacrifice, 
  mostly known as \textit{koṭihoma} in the Purāṇas and in inscriptions 
  {\rm (}see, e.g., \mycitep{Fleming2010}{and 2013}\nocite{Fleming2013}{\rm )}.
  It involves a hundred fire-pits 
  and a hundred times one thousand Brahmins {\rm (}hence the name `the ten-million sacrifice'{\rm )}.
  See, e.g., \BHAVP\ \textit{uttaraparvan} 4.142.54--58:
  \textit{śatānano daśamukho dvimukhaikamukhas tathā\thinspace |
  caturvidho mahārāja koṭihomo vidhīyate\thinspace || 
  kāryasya gurutāṃ jñātvā naiva kuryād aparvaṇi\thinspace |
  yathā saṃkṣepataḥ kāryaḥ koṭihomas tathā śṛṇu\thinspace ||
  kṛtvā kuṇḍaśataṃ divyaṃ yathoktaṃ hastasaṃmitam\thinspace |
  ekaikasmiṃs tataḥ kuṇḍe śataṃ viprān niyojayet\thinspace ||
  sadyaḥ pakṣe tu viprāṇāṃ sahasraṃ parikīrtitam\thinspace |
  ekasthānapraṇīte 'gnau sarvataḥ paribhāvite\thinspace || 
  homaṃ kuryur dvijāḥ sarve kuṇḍe kuṇḍe yathoditam\thinspace |
  yathā kuṇḍabahutve 'pi rājasūye mahākratau\thinspace ||.}
   %
 
  Note that the second syllable of \textit{phalam} in \textit{pāda} d is treated as long: this
  happens often at word-boundaries in this text; and 
  note how \msNc\ aims to restore the metre by inserting \textit{tv} after its \textit{phalaṃ}.
 }}
\center{\maintext{\dbldanda\thinspace iti vṛṣasārasaṃgrahe ahiṃsāpraśaṃsā nāmādhyāyas{ }tṛtīyaḥ\thinspace\dbldanda}}
\translation{Here ends the third chapter in the \textit{Vṛṣasārasaṃgraha} called the Praise of Non-violence.}

  \chptr{caturtho 'dhyāyaḥ}
\addcontentsline{toc}{subsection}{Chapter 4}
\fancyhead[CO]{{\footnotesize\textit{Translation of chapter 4}}}%

  \trchptr{ Chapter Four }%

  \subchptr{yameṣu satyam {\rm {\rm (}2{\rm )}}}%

  \trsubchptr{The second Yama-rule: Truthfulness}%

  \maintext{anarthayajña uvāca |}%

  \maintext{sadbhāvaḥ satyam ity āhur dṛṣṭapratyayam eva vā |}%

  \maintext{yathābhūtārthakathanaṃ tat satyakathanaṃ smṛtam }||\thinspace4:1\thinspace||%
\translation{Anarthayajña spoke: The state of being real {\rm (}\textit{sad-bhāva}{\rm )} is called truth {\rm (}\textit{sat-ya}{\rm )}. Alternatively, it is also a certainty {\rm (}\textit{pratyaya}{\rm )} that originates in perception {\rm (}\textit{dṛṣṭa}{\rm )}. Relating things in a way that corresponds to reality is called `speaking the truth.' \blankfootnote{4.1 Although the rather similar line in the \SDHS\ 
  {\rm (}11.105cd: \textit{yathābhūtārthakathanam ity etat satyalakṣaṇam}{\rm )} 
  makes it tempting to emend \textit{satyakathanaṃ} to \textit{satyalakṣaṇaṃ} in \textit{pāda} d, 
  I rather take this verse to introduce two views on truth: one philosophical, and one ordinary that
  relates to the moral aspect of truthfulness.
 }}

  \maintext{ākrośatāḍanādīni yaḥ saheta suduḥsaham |}%

  \maintext{kṣamate yo jitātmā tu sa ca satyam udāhṛtam }||\thinspace4:2\thinspace||%
\translation{He who endures severe abuse and beating etc. but keeps quiet, his self being conquered, is said to be [an example of] truth[fulness]. \blankfootnote{4.2 \textit{suduḥsaham} {\rm (}singular{\rm )} in \textit{pāda} b picks up \textit{°ādīni} {\rm (}plural{\rm )} in \textit{pāda} a.
  The \textit{-m} in \textit{satyam} may be a sandhi-bridge and the phrase may refer to a
  masculine subject thus: \textit{sa ca satya-m-udāhṛtaḥ}.
 }}

  \maintext{vadhārtham udyataḥ śastraṃ yadi pṛccheta karhicit |}%

  \maintext{na tatra satyaṃ vaktavyam anṛtaṃ satyam ucyate }||\thinspace4:3\thinspace||%
\translation{If one is being interrogated at any time with a sword lifted to strike him down, in this case the truth is not to be spoken, and a lie is can be called truth. \blankfootnote{4.3 Understand \textit{udyataḥ} {\rm (}nom.{\rm )} in an active sense {\rm (}`holding/lifting'{\rm )}.
 }}

  \maintext{vadhārhaḥ puruṣaḥ kaścid vrajet pathi bhayāturaḥ |}%

  \maintext{pṛcchato 'pi na vaktavyaṃ satyaṃ tad vāpi ucyate }||\thinspace4:4\thinspace||%
\translation{A person who is walking on the road and is afraid of being killed \verify should not reply [to people who are potentially dangerous] even if they ask him. This is also called truth[fulness]. }

  \maintext{na narmayuktam anṛtaṃ hinasti}%

 \nonanustubhindent \maintext{na strīṣu rājan na vivāhakāle |}%

  \maintext{prāṇātyaye sarvadhanāpahāre}%

 \nonanustubhindent \maintext{pañcānṛtaṃ satyam udāharanti }||\thinspace4:5\thinspace||%
\translation{A lie does not hurt when it is connected with joking, with women, O king, at the time of marriage, at the departure from life and when one's entire wealth is about to be taken away. They call these five kinds of lies truths. \blankfootnote{4.5 This \textit{upajāti} verse appears in countless sources, beginning with 
  the \MBH\ {\rm (}see the apparatus{\rm )}. All versions 
  contain a vocative addressing a king, which is out of context in the \VSS, the addressee being Vigatarāga,
  i.e. Viṣṇu diguised as a Brahmin. The redactors did not notice or did not care about this
  small inconsistency. Note the metrical licence that allows the last syllable
  of °\textit{yuktam} to count as long.
  The reading with \textit{anṛtaṃ}, as opposed to \textit{vacanaṃ}, in \textit{pāda} a, can be found 
  in the apparatus in the \MBH\ critical edition.
 }}

  \maintext{devamānuṣatiryeṣu satyaṃ dharmaḥ paro yataḥ |}%

  \maintext{satyaṃ śreṣṭhaṃ variṣṭhaṃ ca satyaṃ dharmaḥ sanātanaḥ }||\thinspace4:6\thinspace||%
\translation{Since truth is the supreme Dharma in [the world of] gods, humans and animals, truth is the best, the most preferable. Truth is the eternal Dharma. }

  \maintext{satyaṃ sāgaram avyaktaṃ satyam akṣayabhogadam |}%

  \maintext{satyaṃ potaḥ paratrārthaṃ satyaṃ panthāna vistaram }||\thinspace4:7\thinspace||%
\translation{Truth is an unmanifest ocean. Truth yields imperishable pleasures. Truth is the ship that carries you to the other world. Truth is the wide path. \blankfootnote{4.7 \textit{Pāda} d is slightly problematic because it is difficult to ascertain if some of the
  MSS actually read \textit{panthāna} or \textit{pasthāna} {\rm (}or \textit{yasthāna}{\rm )}. I suspect that \textit{panthāna} 
  is a stem form noun formed {\rm (}metri causa{\rm )} to stand for an irregular nominative of \textit{pathin}.
 }}

  \maintext{satyam iṣṭagatiḥ proktaṃ satyaṃ yajñam anuttamam |}%

  \maintext{satyaṃ tīrthaṃ paraṃ tīrthaṃ satyaṃ dānam anantakam }||\thinspace4:8\thinspace||%
\translation{Truth is said to be the desired path. Truth is the supreme sacrifice. Truth is a pilgrimage place, a supreme pilgrimage place. Truth is an endless donation. \blankfootnote{4.8 The repetition of \textit{tīrthaṃ} in \textit{pāda} c is sightly suspect. Cf., e.g., \MATSP\ 22.79ab:
  \textit{satyaṃ tīrthaṃ dayā tīrthaṃ tīrtham indriyanigrahaḥ}.
 }}

  \maintext{satyaṃ śīlaṃ tapo jñānaṃ satyaṃ śaucaṃ damaḥ śamaḥ |}%

  \maintext{satyaṃ sopānam ūrdhvasya satyaṃ kīrtir yaśaḥ sukham }||\thinspace4:9\thinspace||%
\translation{Truth is morality, austerity, knowledge. Truth is purity, self-control and tranquillity. Truth is the ladder upwards. Truth is fame and glory and happiness. \blankfootnote{4.9 Considering a similar line in the \VARP\ {\rm (}193.36cd, see the apparatus{\rm )}, one 
  wonders if the slightly odd \textit{ūrdhvasya} in \textit{pāda} c is not a corrupt form of 
  \textit{svargasya}.
 }}

  \maintext{aśvamedhasahasraṃ ca satyaṃ ca tulayā dhṛtam |}%

  \maintext{aśvamedhasahasrād dhi satyam eva viśiṣyate }||\thinspace4:10\thinspace||%
\translation{[When] a thousand Aśvamedha sacrifices and truth are measured on a pair of scales, truth indeed surpasses a thousand Aśvamedha sacrifices. }

  \maintext{satyena tapate sūryaḥ satyena pṛthivī sthitā |}%

  \maintext{satyena vāyavo vānti satye toyaṃ ca śītalam }||\thinspace4:11\thinspace||%
\translation{The Sun shines because of truth. The Earth stays in place by truth. The winds blow because of truth. Water is cooling through truth. \blankfootnote{4.11 Here and several times below, \textit{satye} is probably to be taken as standing for \textit{satyena}.
 }}

  \maintext{tiṣṭhanti sāgarāḥ satye samayena priyavrataḥ |}%

  \maintext{satye tiṣṭhati govindo balibandhanakāraṇāt }||\thinspace4:12\thinspace||%
\translation{The oceans exist by the truthful encounter with Priyavrata. Govinda abides in truth because He [as Vāmana] stopped [Mahā]Bali [in spite of the fact that this was achieved by a trick]. \blankfootnote{4.12 \textit{Pāda} b, \textit{samayena priyavrataḥ}, probably stand for \textit{samayena priyavratasya} although
  it is unclear to me what exactly \textit{samaya} refers to here.
   %
 
  For the story of Priyavrata, Manu's son, in which he wanted to turn nights into days by 
  circling aroung Mount Meru in a chariot, and by this produced the seven oceans,
  see, e.g., \BHAGP\ 5.1.30--31:  
  \textit{yāvad avabhāsayati suragirim anuparikrāman bhagavān ādityo
  vasudhātalam ardhenaiva pratapaty ardhenāvacchādayati, tadā hi [priyavrataḥ]
  bhagavadupāsanopacitātipuruṣaprabhāvas tad anabhinandan samajavena
  rathena jyotirmayena rajanīm api dinaṃ kariṣyāmīti saptakṛtvas 
  taraṇim anuparyakrāmad dvitīya iva pataṅgaḥ\thinspace |
  ye vā u ha tadrathacaraṇanemikṛtaparikhātās te sapta sindhava āsan
  yata eva kṛtāḥ sapta bhuvo dvīpāḥ\thinspace |}.
  
  %
  
  \textit{Pāda}s cd: for a somewhat similar reference to the story of Mahābali, see, e.g., \VAMP\ 65.66:
  \textit{evaṃ purā cakradhareṇa viṣṇunā baddho balir vāmanarūpadhāriṇā\thinspace |
  śakrapriyārthaṃ surakāryasiddhaye hitāya viprarṣabhagodvijānām\thinspace ||}. 
 }}

  \maintext{agnir dahati satyena satyena śaśinā caraḥ |}%

  \maintext{satyena vindhyās tiṣṭhanti vardhamāno na vardhate }||\thinspace4:13\thinspace||%
\translation{Fire burns with truth. The Moon rises by truth. It is because of truth that the Vindhya mountain stands in place and that although is was growing it is not growing [anymore]. \blankfootnote{4.13 Since \textit{śaśi} {\rm (}instead of \textit{śaśin}{\rm )} is a possible stem in this text, 
  \textit{śaśir ācaraḥ} could be acceptable here in \textit{pāda} b {\rm (}see \msNa\msNb\msNc{\rm )}, perhaps standing for 
  \textit{śaśinaś caraṇam} or \textit{śaśiś carati}. My conjecture {\rm (}\textit{śaśinā caraḥ}{\rm )} 
  could stand for \textit{śaśinā/śaśinaś cāraḥ} metri causa. Other possibilities, suggested by
  colleagues, include \textit{śaśibhāskaraḥ}, \textit{śaśigocaraḥ} and \textit{śiśirāmbhasaḥ}.
  %
 
  \textit{Pāda}s cd refer to the story of Agastya and the Vindhya mountain:
  Vindhya became jealous of the Sun's revolving around Mount Meru and when the Sun 
  refused him the same favour, he decided to grow higher and obstruct the Sun's movement.
  As a solution to this situation, Agastya asked Vidhya to bend down to make 
  it easier for him to reach the south and to remain thus until he retured. 
  Vindhya agreed to do what Agastya asked him but Agastya never returned. 
  See \MBH\ 3.102.1--14 {\rm (}see the word \textit{samaya} in verse 13 in this passage, and compare it to \VSS\ 4.12b{\rm )}:
   %
  \textit{ yudhiṣṭhira uvāca\thinspace | 
  kimarthaṃ sahasā vindhyaḥ pravṛddhaḥ krodhamūrchitaḥ\thinspace | 
  etad icchāmy ahaṃ śrotuṃ vistareṇa mahāmune\thinspace || 
  lomaśa uvāca\thinspace | 
  adrirājaṃ mahāśailaṃ meruṃ kanakaparvatam\thinspace | 
  udayāstamaye bhānuḥ pradakṣiṇam avartata\thinspace || 
  taṃ tu dṛṣṭvā tathā vindhyaḥ śailaḥ sūryam athābravīt\thinspace | 
  yathā hi merur bhavatā nityaśaḥ parigamyate\thinspace || 
  pradakṣiṇaṃ ca kriyate mām evaṃ kuru bhāskara\thinspace | 
  evam uktas tataḥ sūryaḥ śailendraṃ pratyabhāṣata\thinspace || 
  nāham ātmecchayā śaila karomy enaṃ pradakṣiṇam\thinspace | 
  eṣa mārgaḥ pradiṣṭo me yenedaṃ nirmitaṃ jagat\thinspace || 
  evam uktas tataḥ krodhāt pravṛddhaḥ sahasācalaḥ\thinspace | 
  sūryācandramasor mārgaṃ roddhum icchan paraṃtapa\thinspace || 
  tato devāḥ sahitāḥ sarva eva; sendrāḥ samāgamya mahādrirājam\thinspace | 
  nivārayām āsur upāyatas taṃ; na ca sma teṣāṃ vacanaṃ cakāra\thinspace || 
  athābhijagmur munim āśramasthaṃ; tapasvinaṃ dharmabhṛtāṃ variṣṭham\thinspace | 
  agastyam atyadbhutavīryadīptaṃ; taṃ cārtham ūcuḥ sahitāḥ surās te\thinspace || 
  devā ūcuḥ\thinspace | 
  sūryācandramasor mārgaṃ nakṣatrāṇāṃ gatiṃ tathā\thinspace | 
  śailarājo vṛṇoty eṣa vindhyaḥ krodhavaśānugaḥ\thinspace || 
  taṃ nivārayituṃ śakto nānyaḥ kaś cid dvijottama\thinspace | 
  ṛte tvāṃ hi mahābhāga tasmād enaṃ nivāraya\thinspace || 
  lomaśa uvāca\thinspace | 
  tac chrutvā vacanaṃ vipraḥ surāṇāṃ śailam abhyagāt\thinspace | 
  so 'bhigamyābravīd vindhyaṃ sadāraḥ samupasthitaḥ\thinspace || 
  mārgam icchāmy ahaṃ dattaṃ bhavatā parvatottama\thinspace | 
  dakṣiṇām abhigantāsmi diśaṃ kāryeṇa kena cit\thinspace || 
  yāvadāgamanaṃ mahyaṃ tāvat tvaṃ pratipālaya\thinspace | 
  nivṛtte mayi śailendra tato vardhasva kāmataḥ\thinspace || 
  evaṃ sa samayaṃ kṛtvā vindhyenāmitrakarśana\thinspace | 
  adyāpi dakṣiṇād deśād vāruṇir na nivartate\thinspace || 
  etat te sarvam ākhyātaṃ yathā vindhyo na vardhate\thinspace | 
  agastyasya prabhāvena yan māṃ tvaṃ paripṛcchasi\thinspace ||}.
 }}

  \maintext{lokālokaḥ sthitaḥ satye meruḥ satye pratiṣṭhitaḥ |}%

  \maintext{vedās tiṣṭhanti satyeṣu dharmaḥ satye pratiṣṭhati }||\thinspace4:14\thinspace||%
\translation{The [mythical] Lokāloka mountains are located in truth. Mount Meru stands by truth. The Vedas abide in truth. Dharma is rooted in truth. }

  \maintext{satyaṃ gauḥ kṣarate kṣīraṃ satyaṃ kṣīre ghṛtaṃ sthitam |}%

  \maintext{satye jīvaḥ sthito dehe satyaṃ jīvaḥ sanātanaḥ }||\thinspace4:15\thinspace||%
\translation{The milk a cow yields is truth. Ghee in milk is present as truth. The soul dwells in the body in truth. The eternal soul is truth. \blankfootnote{4.15 \textit{satye} in \textit{pāda} c may stand for \textit{satyaṃ}: `The soul dwells in the body as truth.'
 }}

  \maintext{satyam ekena samprāpto dharmasādhananiścayaḥ |}%

  \maintext{rāmarāghavavīryeṇa satyam ekaṃ surakṣitam }||\thinspace4:16\thinspace||%
\translation{If truth is obtained by somebody {\rm (}\textit{ekena}{\rm )}, he/she will be one for whom Dharma is surely accomplished. By the heroism of Rāma Rāghava, the only truth was well-guarded. \blankfootnote{4.16 Or: `If truth alone {\rm (}\textit{ekena}{\rm )} is obtained, Dharma is surely accomplished.'
 }}

  \maintext{evaṃ satyavidhānasya kīrtitaṃ tava suvrata |}%

  \maintext{sarvalokahitārthāya kim anyac chrotum icchasi }||\thinspace4:17\thinspace||%
\translation{Thus have [I] taught the rules of truth to you, O virtuous one, to favour the whole world. What else do you wish to hear? }

  \subchptr{yameṣv asteyam {\rm {\rm (}3{\rm )}}}%

  \trsubchptr{The third Yama-rule: Refraining from stealing}%

  \maintext{vigatarāga uvāca |}%

  \maintext{na hi tṛptiṃ vijānāmi śrutvā dharmaṃ tavāpy aham |}%

  \maintext{upariṣṭād ato bhūyaḥ kathayasva tapodhana }||\thinspace4:18\thinspace||%
\translation{Vigatarāga spoke: I can't have enough of learning about [this teaching of] your[s on] Dharma. Teach me further than this, O great ascetic. \blankfootnote{4.18 It is not inconceivable that \textit{tava} is meant to carry the sense of the ablative,
  as Kenji Takahashi has suggested to me:
  `I can't have enough of learning about Dharma from you.' 
 }}

  \maintext{anarthayajña uvāca |}%

  \maintext{steyaṃ śṛṇv atha viprendra pañcadhā parikīrtitam |}%

  \maintext{adattādānam ādau tu utkocaṃ ca tataḥ param |}%

  \maintext{prasthavyājas tulāvyājaḥ prasahyasteya pañcamam }||\thinspace4:19\thinspace||%
\translation{Anarthayajña spoke: Now listen to [my teaching about] stealing, O great Brahmin, which is taught to be of five kinds. Firstly, [listen to] theft, then bribery, cheating with weights, cheating with scales, and the fifth kind, robbery. \blankfootnote{4.19 `Theft' {\rm (}\textit{adattādāna}{\rm )}: literally `taking what has not been given.'
 }}

  \maintext{dhṛṣṭaduṣṭaprabhāvena paradravyāpakarṣaṇam |}%

  \maintext{vāryamāṇāpi durbuddhir adattādānam ucyate }||\thinspace4:20\thinspace||%
\translation{When somebody's wealth is taken away by an impudent and wicked person is called theft. It is a foolish thought even if suppressed. \blankfootnote{4.20 My impression is that \textit{prabhāva} in \textit{pāda} a stands for \textit{bhāva}, \textit{duṣṭabhāva} {\rm (}`vicious'{\rm )}
  being a common expression.
 The implications of \textit{vāryamāṇo} in \textit{pāda} c are unclear to me, hence my emendation to \textit{vāryamāṇā}.
  My translation is thus tentative and still not satisfactory.
 }}

  \maintext{utkocaṃ śṛṇu viprendra dharmasaṃkarakārakam |}%

  \maintext{mūlyaṃ kāryavināśārtham utkocaḥ parigṛhyate |}%

  \maintext{tena cāsau vijānīyād dravyalobhabalāt kṛtam }||\thinspace4:21\thinspace||%
\translation{O great Brahmin, listen to bribery, which defiles Dharma. A sum of money taken in order to exempt somebody from a duty is a bribe. Therefore this [also] should be considered as such [i.e.\ as stealing because] it is committed out of greed for material goods. \blankfootnote{4.21 Note that \textit{mūlyaṃ} in \textit{pāda} c is a conjecture for \textit{mūla}. It is partly based on 
  a relevant passage in the \Mitaksara\ {\rm (}ad \YajnS\ 2.176cd{\rm )}:
  \textit{paṇyasya krītadravyasya yan mūlyaṃ dattam, bhṛtir vetanaṃ kṛtakarmaṇe dattam\dots\ 
  utkocena kāryapratibandhanirāsārtham adhikṛtebhyo dattam}\dots\ 
 Note \textit{asau} in \textit{pāda} e as an accusative form {\rm (}for \textit{amum} or \textit{adaḥ}{\rm )}. It is not unlikely that 
  \textit{tena} is a corruption from \textit{stena}, and the \textit{pāda} may have originally read 
  \textit{stenaṃ taṃ ca vijānīyād} {\rm (}`he should be known as a thief'{\rm )}, or similar {\rm (}cf. 4.22c below{\rm )}. 
  \msM\ {\rm (}f. 7r{\rm )} reads \textit{tena steya vijānīyād} here.
 }}

  \maintext{prasthavyāja-upāyena kuṭumbaṃ trātum icchati |}%

  \maintext{taṃ ca stenaṃ vijānīyāt paradravyāpahārakam }||\thinspace4:22\thinspace||%
\translation{[Even if] somebody wants to protect a family by the method of cheating with weights, that person should be considered a thief, because he takes away other people's wealth. }

  \maintext{tulāvyāja-upāyena parasvārthaṃ hared yadi |}%

  \maintext{cauralakṣaṇakāś cānye kūṭakāpaṭikā narāḥ }||\thinspace4:23\thinspace||%
\translation{[The case is similar] if somebody takes away somebody else's belongings by the method of cheating with scales. Other people, deceitful swindlers {\rm (}\textit{kūṭa-kāpaṭika}{\rm )} share the characteristics of thieves. \blankfootnote{4.23 A line may have dropped out after \textit{pāda} b, perhaps because a line 
  similar to 4.22cd caused an eyeskip. Alternatively, this line may simply be
  elliptical.
 }}

  \maintext{durbalārjavabāleṣu cchadmanā vā balena vā |}%

  \maintext{apahṛtya dhanaṃ mūḍhaḥ sa cauraś cora ucyate }||\thinspace4:24\thinspace||%
\translation{If someone, by deceit or by force, snatches away the wealth of weak and honest people and simpletons, that morally corrupt usurper is [simply] a thief. \blankfootnote{4.24 It is possible that \textit{pāda} d read differently, e.g. \textit{sa coraś cora ucyate}, meaning `that thief is [rightly] called a thief'.
 }}

  \maintext{nāsti steyasamaṃ pāpaṃ nāsty adharmaś ca tatsamaḥ |}%

  \maintext{nāsti stenasamākīrtir nāsti stenasamo 'nayaḥ }||\thinspace4:25\thinspace||%
\translation{There is no sin equal to stealing. There is no crime {\rm (}\textit{adharma}{\rm )} equal to it. There is no ill-fame comparable to that of being a thief. There is no bad-conduct comparable to being a thief. }

  \maintext{nāsti steyasamāvidyā nāsti stenasamaḥ khalaḥ |}%

  \maintext{nāsti stenasama ajño nāsti stenasamo 'lasaḥ }||\thinspace4:26\thinspace||%
\translation{There is no greater ignorance than stealing. There are no bigger rouges than thieves. There is nobody as ignorant as a thief. There is not a lazy person who is comparable to a thief. \blankfootnote{4.26 Note the peculiar sandhi in \textit{pāda} c {\rm (}\textit{°sama ajño}{\rm )}, which still leaves the \textit{pāda} 
  a \textit{sa-vipulā}.
 }}

  \maintext{nāsti stenasamo dveṣyo nāsti stenasamo 'priyaḥ |}%

  \maintext{nāsti steyasamaṃ duḥkhaṃ nāsti steyasamo 'yaśaḥ }||\thinspace4:27\thinspace||%
\translation{There is nobody as detestable as a thief. There is nobody disliked as much as a thief. There is no greater suffering than stealing. There is no greater disgrace than theft. \blankfootnote{4.27 Note how \textit{stena} and \textit{steya} are used interchangeably {\rm (}or chaotically{\rm )}
  in the above passages in the MSS to denote both `thief' and 'theft/stealing'.
  The scribe of \msNc\ ends up writing \textit{stenya} in 4.27e.
 }}

  \maintext{pracchanno hriyate 'rtham anyapuruṣaḥ pratyakṣam anyo haret}%

 \nonanustubhindent \maintext{nikṣepād dhanahāriṇo 'nya{-}m{-}adhamo vyājena cānyo haret |}%

  \maintext{anye lekhyavikalpanāhṛtadhanā {\rm †}anyo hṛtād vai hṛtā{\rm †}}%

 \nonanustubhindent \maintext{anyaḥ krītadhano 'paro dhayahṛta ete jaghanyāḥ smṛtāḥ }||\thinspace4:28\thinspace||%
\translation{Some [thieves] take away [other people's] wealth in disguise, some in broad daylight. Other wicked people take money from deposits, and some people steal through fraud. Some gather wealth by forging documents, others steal from stolen money[?]. Some people's wealth is from purchased [children?] {\rm (}\textit{krīta}{\rm )}. Others take away others' inheritance[?]. These are considered the vilest. \blankfootnote{4.28 Metre \textit{śārdūlavikrīḍita}. It appears that \textit{hriyate} in \textit{pāda} a is to be taken as an active verb {\rm (}\textit{harate}{\rm )}.
  Note also how \msCb\ and \msNc\ read the same here against the other witnesses.
 Take \textit{°hariṇo} in \textit{pāda} b as singular and \textit{m} in \textit{'nya-m-adhamo} as a sandhi-bridge.
  Alternatively, read as plural: °\textit{hariṇo 'nya adhamo}\dots\ 
 The second half of \textit{pāda} c is difficult to reconstruct.
 The translation of \textit{pāda} d is mostly guesswork. Tentatively, I take \textit{krīta} as \textit{krītaka} {\rm (}`a purchased son', see
  \MANU\ 9.174{\rm )}. \textit{dhayahṛta} makes little sense to me. Florinda De Simini suggested that
  \textit{dhaya} might stand for \textit{daya}, which in turn may stand for \textit{dāya} {\rm (}`inheritance'{\rm )} metri causa.
  Lacking any better solution, I supplied these in my translation, marked with question marks.
  Note also the metrical licence that the last syllable of \textit{dhayahṛta} counts as long.
 }}

  \maintext{stenatulya na mūḍham asti puruṣo dharmārthahīno 'dhamaḥ}%

 \nonanustubhindent \maintext{yāvaj jīvati śaṅkayā narapateḥ saṃtrasyamāno raṭan |}%

  \maintext{prāptaḥśāsana tīvrasahyaviṣamaṃ prāpnoti karmeritaḥ}%

 \nonanustubhindent \maintext{kālena mriyate sa yāti nirayam ākrandamāno bhṛśam }||\thinspace4:29\thinspace||%
\translation{There isn't a bigger idiot than a thief, who is a wicked person without Dharma and Artha. As long as he lives, he trembles in fear of the king, wailing. Having received his punishment, he gets into severe and [in]tolerable difficulties, propelled by [his] karma. When his time comes, he dies and goes to hell, weeping vehemently. \blankfootnote{4.29 For some time I was wondering if one should accept \Ed's reading \textit{stenastulya na mūḍham asti} 
  as a metri causa version of \textit{stenatulyo na mūḍho 'sti}; see a similar case of a nominative ending
  inside of compound in \textit{pāda} c below. One major concern remained:
  the accepted reading would be of an edition that rarely emerges as 
  the sole transmitter of the best reading. Another possible solution could be 
  to emend to \textit{stenaṃtulya}\dots, meaning `There is no bigger foolishness than theft',
  but then the second part of \textit{pāda} a is difficult to connect. In the end,
  I decided to to go for the most widely attested reading {\rm (}\textit{stenatulya}{\rm )},
  which is unmetrical.
 
   % 
 Understand \textit{prāptaḥśāsana tīvrasahyaviṣamaṃ} in \textit{pāda} c as \textit{prāptaśāsanas tīvram asahyaṃ ca viṣamaṃ prāpnoti}.
  Alternatively, understand \textit{tīvrasahya°} as \textit{duḥsahya°} {\rm (}suggested by Törzsök{\rm )}.
 
   %
  The actual reading of \msCa, \textit{prāptaś} {\rm (}lost in the process of normalization and standing
  in contrast with that of all other MSS that read \textit{prāptaḥ}{\rm )} may suggest
  a doubling of the \textit{ś} of \textit{śāsana} metri causa {\rm (}suggestion by Törzsök{\rm )}.
  More likely is that a licence of having a nominative ending inside of a compound
  is applied here, as may have been the case above in \textit{pāda} a {\rm (}also remarked by Törzsök{\rm )}.
 }}

  \maintext{nītvā durgatikoṭikalpa nirayāt tiryatvam āyānti te}%

 \nonanustubhindent \maintext{tiryatve ca tathaivam ekaśatikaṃ prabhramya varṣārbudam |}%

  \maintext{mānuṣyaṃ tad avāpnuvanti vipule dāridryarogākulaṃ}%

 \nonanustubhindent \maintext{tasmād durgatihetu karma sakalaṃ tyaktvā śivaṃ cāśrayet }||\thinspace4:30\thinspace||%
\translation{Having spent ten million \ae ons of suffering, they emerge from hell to the state of animal existence. Thus, they roam about in animal existence for a hundred and one times ten million years. Then they reach the status of human existence on earth which is full of poverty and disease. Then abandoning all one's karmas, the causes of suffering, one seeks refuge in Śiva. \blankfootnote{4.30 Note the stem form °\textit{kalpa} for °\textit{kalpaṃ} metri causa.
 In \textit{pāda} c, \textit{tathaivam}, or \textit{tathaikam}, and \textit{ekaśatikaṃ} are suspect.
 I understand \textit{vipule} as \textit{vipulāyāṃ}, \textit{vipulā} appearing in \Amara\ 2.1.7 as a synonym of
  \textit{dhātrī}, `earth.' It is difficult to interpret it otherwise.
  This is still problematic because both human and
  animal existence takes place on earth, thus, if \textit{tiryatva} {\rm (}i.e. \textit{tiryaktva}{\rm )} 
  indeed means `animal existence,' there is no contrast between \textit{pāda}s b and c as
  regards location. As for \textit{tiryaktva}, see, e.g., \MANU\ 12.40:
  \textit{devatvaṃ sāttvikā yānti manuṣyatvaṃ ca rājasāḥ\thinspace |
  tiryaktvaṃ tāmasā nityam ity eṣā trividhā gatiḥ\thinspace ||}.
  It is not unlikely that the original form of \textit{dāridryarogākulam} was \textit{dāridryarogākule},
  picking up \textit{vipule}.
 Note the switch from plural to singular in \textit{pāda} d.
 }}

  \subchptr{yameṣv ānṛśaṃsyam {\rm {\rm (}4{\rm )}}}%

  \trsubchptr{The fourth Yama-rule: Absence of hostility}%

  \maintext{aṣṭamūrtiśivadveṣṭā pitur mātuś ca yo dviṣet |}%

  \maintext{gavāṃ vā atither dveṣṭā nṛśaṃsāḥ pañca eva te }||\thinspace4:31\thinspace||%
\translation{The one who is hostile towards the eight-formed Śiva, he who hurts his mother or father, he who is hostile towards cows or guests: these are the five types of cruel people. \blankfootnote{4.31 Note \textit{pitur} and \textit{mātur} used as accusative forms in \textit{pāda} b, or alternatively
  understand: `who are hateful towards their fathers and mothers'.
 }}

  \maintext{aṣṭamūrtiḥ śivaḥ sākṣāt pañcavyomasamanvitaḥ |}%

  \maintext{sūryaḥ somaś ca dīkṣaś ca dūṣakaḥ sa nṛśaṃsakaḥ }||\thinspace4:32\thinspace||%
\translation{Śiva in his manifest form {\rm (}\textit{sākṣāt}{\rm )} is of eight forms, with the five elements {\rm (}\textit{vyoman}{\rm )}, the Sun, the Moon, and the sacrificer. [He who] disgraces [any of these] is a hostile person. \blankfootnote{4.32 Törzsök has suggested emending \textit{sa nṛśaṃsakaḥ} in \textit{pāda} d to \textit{tannṛṃśakaḥ}. I don't think that it is
  inevitably necessary. I think that \textit{pāda}s a-c form a list that is meant to be in the genitive, understanding
  \dots\ \textit{ity eteṣāṃ dūṣakaḥ sa nṛśaṃsakaḥ} or similar. This is clumsy but in a way that is
  more than possible within the style of this text.
 
  I have not been able find any other attestation of \textit{vyoman} meaning the five elements. Perhaps it is meant
  to mean \textit{vyomādi} {\rm (}`the atmosphere/sky and the other four elements'{\rm )}. 
  
  For Śiva of eight forms, see, e.g., \textit{Śakuntalā} 1.1:
   %
  \textit{yā sṛṣṭiḥ sraṣṭur ādyā [1] vahati vidhihutaṃ yā havir [2] yā ca hotrī [3] 
  ye dve kālaṃ vidhattaḥ [4, 5] śruti-viṣaya-guṇā yā [6] sthitā vyāpya viśvam\thinspace | 
  yām āhuḥ sarva-bīja-prakṛtir [7] iti yayā prāṇinaḥ prāṇavantaḥ [8] 
  pratyakṣābhiḥ prapannas tanubhir avatu vas tābhir aṣṭābhir īśaḥ\thinspace ||}. 
  The eight \textit{mūrti}s, or rather, \textit{tanu}s, here are: 
  [1] \textit{jala} [2] \textit{agni} [3] \textit{yajamāna} [4,5] \textit{sūrya} + \textit{candra} [6] \textit{ākāśa} [7] \textit{bhūmi} [8] \textit{vāyu}.
 
  For a similar interpretation of \textit{aṣṭamūrti}, see, e.g., \textit{Īśānaśivagurudevapaddhati} 2.29.34 {\rm (}\textit{mantrapāda};
  note \textit{yajamāna} for our \textit{dīkṣa}{\rm )}:
  \textit{kṣmā-vahni-yajamānārka-jala-vāyv-indu-puṣkaraiḥ}\thinspace |
  \textit{aṣṭābhir mūrtibhiḥ śambhor dvitīyāvaraṇaṃ smṛtam}\thinspace ||.
  {\rm (}For \textit{puṣkara} as `sky, atmosphere', see, e.g., \Amara\ 1.2.167:
  \textit{dyodivau dve striyām abhraṃ vyoma puṣkaram ambaram}.{\rm )}
 
  A closely related Aṣṭamūrti-hymn appears in \Nisvmukh\ 1.30--41 {\rm (}I owe thanks to Nirajan Kafle
  for drawing my attention to this{\rm )}; see \mycitep{KafleNisvasaBook}{62, 63, 116, 119}. 
  Kafle notes that this hymn is closely parallel to some passages in the \textit{Prayogamañjarī} {\rm (}1.19--26{\rm )},
  the \textit{Tantrasamuccaya} {\rm (}1.16--23{\rm )}, and the \textit{Īśānaśivagurudevapaddhati} {\rm (}\textit{kriyāpāda} 26.56--63{\rm )}. 
  See also \TAKI\ s.v. \textit{aṣṭamūrti}.
 }}

  \maintext{pitākāśasamo jñeyo janmotpattikaraḥ pitā |}%

  \maintext{pitṛdaivata{\rm †}m ādiś cam ānṛśaṃsa tamanvitaḥ{\rm †} }||\thinspace4:33\thinspace||%
\translation{The father is to be considered similar to the [element] sky, he is the cause of one's birth. One should not be hostile to the forefathers, gods\dots[?]. \blankfootnote{4.33 It is difficult to restore \textit{pāda}s cd, although the general meaning of this line is
  predictable. Some questions remain. Is \textit{āditya} a good reading or is \textit{mātṛ} hidden in
  \textit{daivata-mādiśca}? Is \textit{ānṛśaṃsa} right or was it \textit{nṛśaṃsa} that was meant by the author of this line?
  Does \textit{tamanvitaḥ} {\rm (}or \textit{tamānvitaḥ}{\rm )} has anything to do with \textit{tamas} {\rm (}`darkness'{\rm )}?
 }}

  \maintext{pṛthvyā gurutarī mātā ko na vandeta mātaram |}%

  \maintext{yajñadānatapovedās tena sarvaṃ kṛtaṃ bhavet }||\thinspace4:34\thinspace||%
\translation{The mother is more venerable than the earth. Who would not praise a mother? By that [praise], sacrifices, donations, austerities and [the study of] the Vedas, all will be completed. }

  \maintext{gāvaḥ pavitraṃ maṅgalyaṃ devatānāṃ ca devatāḥ |}%

  \maintext{sarvadevamayā gāvas tasmād eva na hiṃsayet }||\thinspace4:35\thinspace||%
\translation{Cows are an auspicious blessing, they are the gods of the gods. Cows contain in themselves all the gods. That is exactly why one should not hurt them. }

  \maintext{jātamātrasya lokasya gāvas trātā na saṃśayaḥ |}%

  \maintext{ghṛtaṃ kṣīraṃ dadhi mūtraṃ śakṛtkarṣaṇam eva ca }||\thinspace4:36\thinspace||%
\translation{Cows are the protectors of the world as if the world were their new-born [calf], there is no doubt about it. Collecting [the five products of the cow, the \textit{pañcagavya},] ghee, milk, curd, and [the cow's] urine and dung [is auspicious]. \blankfootnote{4.36 \textit{Pāda} c is a \textit{sa-viplulā}. 
  The use of \textit{karsaṇa} in \textit{pāda} d, most probably in the sense of `collecting,' is slightly odd.
 }}

  \maintext{pañcāmṛtaṃ pañcapavitrapūtaṃ}%

 \nonanustubhindent \maintext{ye pañcagavyaṃ puruṣāḥ pibanti |}%

  \maintext{te vājimedhasya phalaṃ labhanti}%

 \nonanustubhindent \maintext{tad akṣayaṃ svargam avāpnuvanti }||\thinspace4:37\thinspace||%
\translation{People who drink the five products of the cow, the five nectars, purified by the five Pavitras, will obtain the fruits of a horse sacrifice, and then reach the undecaying heavens. \blankfootnote{4.37 The five Pavitras are most probably the five \textit{brahmamantras}, see, e.g., \TAKIII\ s.v. \textit{pavitra} 1.
 }}

  \maintext{gobhir na tulyaṃ dhanam asti kiṃcid}%

 \nonanustubhindent \maintext{duhyanti vāhyanti bahiś caranti |}%

  \maintext{tṛṇāni bhuktvā amṛtaṃ sravanti}%

 \nonanustubhindent \maintext{vipreṣu dattāḥ kulam uddharanti }||\thinspace4:38\thinspace||%
\translation{There is no wealth comparable to [having] a cow. They yield milk, they draw [a plough etc.], they roam under the sky. Feeding on grass, they issue nectar. When given to Brahmins, they deliver the family [from \textit{saṃsāra} or the suffering experienced in hell]. \blankfootnote{4.38 Note that \textit{duhyanti} and \textit{vāhyanti} are supposed to be understood as passive,
  as in the similar verse in \SDHU\ 12.92 {\rm (}see apparatus{\rm )}.
 }}

  \maintext{gavāhnikaṃ yaś ca karoti nityaṃ}%

 \nonanustubhindent \maintext{śuśrūṣaṇaṃ yaḥ kurute gavāṃ tu |}%

  \maintext{aśeṣayajñatapadānapuṇyaṃ}%

 \nonanustubhindent \maintext{labhaty asau tām anṛśaṃsakartā }||\thinspace4:39\thinspace||%
\translation{He who never fails to serve the cow daily [e.g. with a handful of grass], he who tends to the cows' service, he who is kind to her [i.e. to the cow], will obtain the merits of all sacrifices, austerities and donation. \blankfootnote{4.39 Strictly speaking, \textit{pāda} c is unmetrical. The second syllable of \textit{tapa} counts as
  long {\rm (}see Intro \verify{\rm )}.
 Although the accusative with °\textit{kartā} in \textit{pāda} d is still not optimal, my 
  emendation of \textit{tam} to \textit{tām} at least restores the metre and improves 
  upon the meaning of the sentece. Alternatively, as suggested by Törzsök,
  \textit{taṃ} could be understood as \textit{tad}, picking up \textit{puṇyaṃ} in \textit{pāda} c,
  but in this way any reference to cows here is only implied.
 }}

  \maintext{atithiṃ yo 'nugaccheta atithiṃ yo 'numanyate |}%

  \maintext{atithiṃ yo 'nupūjyeta atithiṃ yaḥ praśaṃsate }||\thinspace4:40\thinspace||%
\translation{One who looks after a guest, one who respects a guest, one who worships a guest, one who praises a guest, \blankfootnote{4.40 Note the peculiar active verb forms \textit{anugaccheta} and \textit{anupūjyeta}.
  On this formation, see a remark about \Nisvmul\ 2.8 in \mycitep{NisvasaGoodall}{247}:
  `We have assumed that \textit{pūjyeta} is intended to mean \textit{pūjayet} and is
  perhaps a contraction of \textit{pūjayeta}.'
 }}

  \maintext{atithiṃ yo na pīḍyeta atithiṃ yo na duṣyati |}%

  \maintext{atithipriyakartā yaḥ atitheḥ paricārakaḥ |}%

  \maintext{atitheḥ kṛtasaṃtoṣas tasya puṇyam anantakam }||\thinspace4:41\thinspace||%
\translation{one who does not harm a guest, one who does not commit a fault towards a guest, one who keeps the guest happy, one who attends to the needs of a guest, one who makes a guest satisfied: his merits are endless. \blankfootnote{4.41 On the form \textit{pīḍyeta}, see previous note.
 }}

  \maintext{āsanenārghapātreṇa pādaśaucajalena ca |}%

  \maintext{annavastrapradānair vā sarvaṃ vāpi nivedayet }||\thinspace4:42\thinspace||%
\translation{He should offer [the guest] a seat, a vessel with water-offering, and water for washing his feet, or gifts of food and clothes, or all [of these]. \blankfootnote{4.42 My conjecture in \textit{pāda} a {\rm (}°\textit{pātreṇa} for °\textit{pādyena}{\rm )} was inspired by the fact that 
  \textit{pāda} b seems to awkwardly repeat what °\textit{pādyena} in \textit{pāda} a signifies.
  Other possibilities could include taking into account bathing {\rm (}\textit{snāna}{\rm )} or 
  an unguent {\rm (}\textit{abhyaṅga}{\rm )}.
 }}

  \maintext{putradārātmanā vāpi yo 'tithim anupūjayet |}%

  \maintext{śraddhayā cāvikalpena aklībamānasena ca }||\thinspace4:43\thinspace||%
\translation{He who worships the guest by [offering him] his own son, wife or himself with willingness, without hesitation, and with a brave heart, \blankfootnote{4.43 For the requirement that one could part with his wife or son, or his own life,
  for the benefit of someone else, see \VSS\ 2.38 and the narrative in \VSS\ chapter 12
  which tells about a Brahmin giving away his own wife to a guest;
  these influenced my decision to emend \textit{°ātmano} to \textit{°ātmanā} in \textit{pāda} a.
  Note that in fact \VSS\ 4.44cd below echoes verse 37cd in the above mentioned chapter 12,
  which reads: \textit{dvijarūpadharo dharmaḥ svayam eva ihāgataḥ}.
 }}

  \maintext{na pṛcched gotracaraṇaṃ svādhyāyaṃ deśajanmanī |}%

  \maintext{cintayen manasā bhaktyā dharmaḥ svayam ihāgataḥ }||\thinspace4:44\thinspace||%
\translation{and does not ask [the guests about their] lineage, Vedic affiliation {\rm (}\textit{caraṇa}{\rm )}, studies, country or birth, and imagines mentally, with devotion, that it is Dharma himself who has arrived, }

  \maintext{aśvamedhasahasrāṇi rājasūyaśatāni ca |}%

  \maintext{puṇḍarīkasahasraṃ ca sarvatīrthatapaḥphalam }||\thinspace4:45\thinspace||%
\translation{[will obtain all the fruits of] thousands of Aśvamedha sacrifices and hundreds of Rājasūya sacrifices, a thousand Puṇḍarīka sacrifices and the fruit of [visiting] all the pilgrimage places and [performing] all the austerities; }

  \maintext{atithir yasya tuṣyeta nṛśaṃsamatam utsṛjet |}%

  \maintext{sa tasya sakalaṃ puṇyaṃ prāpnuyān nātra saṃśayaḥ }||\thinspace4:46\thinspace||%
\translation{he whose guest is satisfied [and] he who can abandon the sentiment of cruelty, will obtain all the merits of the above, there is no doubt about it. \blankfootnote{4.46 The demonstrative pronoun \textit{tasya} in \textit{pāda} c may refer to the guest:
  `he will obtain all his [i.e. the guest's] merits,' hinting at some sort of karmic exchange.
  Nevertheless, I think rather that \textit{tasya} points to the merits one can obtain by the rituals listed 
  in the previous verse. This is suggested by passages such as the following:
   %
  \MBH\ Supp. 13.14.379 ff.: 
  \textit{ahany ahani yo dadyāt kapilāṃ dvādaśīḥ samāḥi\thinspace | 
  māsi māsi ca satreṇa yo yajeta sadā naraḥ\thinspace ||  
  gavāṃ śatasahasraṃ ca yo dadyāj jyeṣṭhapuṣkare\thinspace |  
  na taddharmaphalaṃ tulyam atithir yasya tuṣyati\thinspace ||}.
   %
  \BRAHMAVP\ 3.44--46: 
  \textit{atithiḥ pūjito yena pūjitāḥ sarvadevatāḥ\thinspace | 
  atithir yasya saṃtuṣṭas tasya tuṣṭo hariḥ svayam\thinspace || 
  snānena sarvatīrtheṣu sarvadānena yat phalam\thinspace |  
  sarvavratopavāsena sarvayajñeṣu dīkṣayā\thinspace ||  
  sarvais tapobhir vividhair nityair naimittikādibhiḥ\thinspace |  
  tad evātithisevāyāḥ kalāṃ nārhanti ṣoḍaśīm\thinspace ||}.
 }}

  \maintext{{\rm †}na gatim atithijñasya{\rm †} gatim āpnoti karhicit |}%

  \maintext{tasmād atithim āyāntam abhigacchet kṛtāñjaliḥ }||\thinspace4:47\thinspace||%
\translation{\dots\ will ever reach the path. Therefore one should go up to the arriving guest with respectfully joined palms. \blankfootnote{4.47 Something has gone wrong with \textit{pāda}s ab and I am unable to reconstruct the
  meaning. The line may have begun with something like \textit{nāgatātithyavajña}°
  {\rm (}`he who despise a guest that has arrived will not\dots'{\rm )}.
 }}

  \maintext{saktuprasthena caikena yajña āsīn mahādbhutaḥ |}%

  \maintext{atithiprāptadānena svaśarīraṃ divaṃ gatam }||\thinspace4:48\thinspace||%
\translation{By one \textit{prastha} [a small unit of weight] of coarsely ground grains given to a guest, an extremely wonderful sacrifice was performed [so to say], and his body [i.e. the protagonist in his mortal form] reached heaven. \blankfootnote{4.48 This verse is a reference to the story related by a mongoose in \MBH\ 14.92--93: 
  A Brahmin who practises the vow of gleaning {\rm (}\textit{uñcha}{\rm )} and his family
  receive a guest. They feed the guest with the last morsels of the little food
  they have. In the end, the guest reveals that he is in fact Dharma {\rm (}14.93.80cd{\rm )} and as 
  a reward the family departs to heaven. The noble act of the poor Brahmin and his family
  is depicted as yielding greater rewards than Yudhiṣṭhira's grandiose horse-sacrifice. 
  {\rm (}See an analysis of this story in \mycite{TakahashiUnca}.{\rm )}
 
  
 We would be forced to accept the reading of \Ed\ in \textit{pāda} d {\rm (}\textit{saśarīro}{\rm )} 
  if the expression were in the masculine {\rm (}\textit{divaṃ gataḥ}{\rm )}. This would make sense
  and it would also echo expressions occuring, e.g., in the \MBH:
  3.164.33cd: \textit{paśya puṇyakṛtāṃ lokān saśarīro divaṃ vraja};
  14.5.10cd: \textit{saṃjīvya kālam iṣṭaṃ ca saśarīro divaṃ gataḥ}.
  It is tempting to emend accordingly, but instead I have retained 
  \textit{svaśarīraṃ divaṃ gatam}, and I interpret it in a general way.
 }}

  \maintext{nakulena purādhītaṃ vistareṇa dvijottama |}%

  \maintext{viditaṃ ca tvayā pūrvaṃ prasthavārttā ca kīrtitā }||\thinspace4:49\thinspace||%
\translation{The mongoose related [this story in the \textit{Mahābhārata}] in the past in detail, O great Brahmin, and you known it already. The story of the \textit{prastha} is well-known. }

  \subchptr{yameṣu damaḥ {\rm {\rm (}5{\rm )}}}%

  \trsubchptr{The fifth Yama-rule: Self-restraint}%

  \maintext{dama eva manuṣyāṇāṃ dharmasārasamuccayaḥ |}%

  \maintext{damo dharmo damaḥ svargo damaḥ kīrtir damaḥ sukham }||\thinspace4:50\thinspace||%
\translation{Self-restraint is in itself the collected essence of Dharma for humans. Self-restraint is Dharma, self-restraint is heaven, self-restraint is fame, self-restraint is happiness. }

  \maintext{damo yajño damas tīrthaṃ damaḥ puṇyaṃ damas tapaḥ |}%

  \maintext{damahīna{-}m{-}adharmaś ca damaḥ kāmakulapradaḥ }||\thinspace4:51\thinspace||%
\translation{Self-restraint is sacrifice, self-restraint is a pilgrimage-place, self-restraint is merit, self-restraint is religious austerity. If one has no self-restraint, one is a sinner {\rm (}\textit{adharma}{\rm )}, [while] self-restraint yields a multitude of desired objects. \blankfootnote{4.51 I suspect that the final \textit{m} in \textit{dhamahīnam} in \textit{pāda} c is a hiatus filler: \textit{dhamahīna-m-adharmaś ca}.
  \textit{kāmakulapradaḥ} in \textit{pāda} d is suspect, and my translation is 
  unsatisfactory. This compound could be interpreted as `fullfilling desires and giving a family' or 
  it may have originally read \textit{sarvakāmapradaḥ} {\rm (}`fullfilling all desires'{\rm )} or
  \textit{kulakāmapradaḥ} {\rm (}`fullfilling the desires of the family'{\rm )}.
  \SDHS\ 4.28b reads \textit{sarvakāmasukhapradam}, which opens up further possibilities.
 }}

  \maintext{nirdamaḥ kari mīnaś ca pataṅgabhramaramṛgāḥ |}%

  \maintext{tvag jihvā ca tathā ghrāṇā cakṣuḥ śravaṇam indriyāḥ }||\thinspace4:52\thinspace||%
\translation{The elephant, the fish, the moth, the bee and the deer are without self-restraint. The senses are the skin, the tongue, the nose, the eye and the ear. \blankfootnote{4.52 Note \textit{kari} for \textit{karī} metri causa, and the end of \textit{pāda} b {\rm (}\textit{°mṛgāḥ}{\rm )}, which 
  should be treated metrically as if it read \textit{°mrigāḥ}.
 }}

  \maintext{durjayendriyam ekaikaṃ sarve prāṇaharāḥ smṛtāḥ |}%

  \maintext{damaṃ yo jayate 'samyag nirdamo nidhanaṃ vrajet }||\thinspace4:53\thinspace||%
\translation{Each of these sense faculties are hard to conquer and all are known to be fatal [if unconquered]. If one masters self-restraint in a less than proper way, one remains unrestrained and will die . \blankfootnote{4.53 The only way to make sense of \textit{pāda}s cd is to supply and \textit{avagraha} before
  \textit{samyag}. Otherwise some text may have dropped out here.
 }}

  \maintext{mṛge śrotravaśān mṛtyuḥ pataṅgāś cakṣuṣor mṛtāḥ |}%

  \maintext{ghrāṇayā bhramaro naṣṭo naṣṭo mīnaś ca jihvayā }||\thinspace4:54\thinspace||%
\translation{In the case of the deer, death comes about because of hearing [when, e.g., hunters use buck grunts]. Moths die because of their eyes [as they are attracted to the light of a lamp]. Bees perish because of their smelling [as they are attracted to smells], fish because of their tongues [when fishermen feed them]. \blankfootnote{4.54 My comments in square brackets in the translation are tentative.
 }}

  \maintext{sparśena ca karī naṣṭo bandhanāvāsaduḥsahaḥ |}%

  \maintext{kiṃ punaḥ pañcabhuktānāṃ mṛtyus tebhyaḥ kim adbhutam }||\thinspace4:55\thinspace||%
\translation{The elephant perishes because of touch, not tolerating to be kept in fetters. How much more true it is for those who enjoy all five [senses]! Why should death come as a surprise for them? \blankfootnote{4.55 \textit{Mātaṅgalīlā} 11.1 may shed some light on elephants dying in captivity:
  \textit{vānyas tatra sukhoṣitā vidhivaśād grāmāvatīrṇā gajā baddhās tīkṣṇakaṭūgravāgbhir atiśugbhīmohabandhādibhiḥ\thinspace |
  udvignāś ca manaḥśarīrajanitair duḥkhair atīvākṣamāḥ prāṇān dhārayituṃ ciraṃ naravaśaṃ prāptāḥ svayūthād atha}\thinspace ||.
  In Edgerton's translation \nocite{EdgertonElephant}{\rm (}1931, 92{\rm )}: 
  `Forest elephants who dwell there happily and by
  the power of fate have been brought to town in bonds, afflicted by harsh, bitter, cruel words,
  by excessive grief, fear, bewilderment, bondage, etc., and by sufferings of mind and body,
  are quite unable for long to sustain life, when from their own herds they have come into
  the control of men.'
 }}

  \maintext{purūravo 'tilobhena atikāmena daṇḍakaḥ |}%

  \maintext{sāgarāś cātidarpeṇa atimānena rāvaṇaḥ }||\thinspace4:56\thinspace||%
\translation{Purūravas [perished] by excessive greed, Daṇḍaka by excessive desire, Sagara's sons by excessive pride, Rāvaṇa by excessive haughtiness, \blankfootnote{4.56 We may treat \textit{purūravo} in \textit{pāda} a as a stem form noun or thematised stem, or imagine that the
  original reading was \textit{purūravā}° with double sandhi:
  \textit{purūravās ati}° $\rightarrow$\ \textit{purūravā ati}° $\rightarrow$\ \textit{purūravāti}°.
 
  \textit{Pāda} a may refer to the following passage in the \MBH\ {\rm (}1.70.16--18, 20ab{\rm )}:
  \textit{purūravās tato vidvān ilāyāṃ samapadyata\thinspace | 
  sā vai tasyābhavan mātā pitā ceti hi naḥ śrutam\thinspace || 
  trayodaśa samudrasya dvīpān aśnan purūravāḥ\thinspace | 
  amānuṣair vṛtaḥ sattvair mānuṣaḥ san mahāyaśāḥ\thinspace || 
  vipraiḥ sa vigrahaṃ cakre vīryonmattaḥ purūravāḥ\thinspace | 
  jahāra ca sa viprāṇāṃ ratnāny utkrośatām api\thinspace || 
  [\dots] 
  tato maharṣibhiḥ kruddhaiḥ śaptaḥ sadyo vyanaśyata\thinspace |}.
   %
  ``The wise Purūravas was born to Ilā. We heard that Ilā 
  was both his mother and his father. 
  The great Purūravas ruled over thirteen islands of the ocean
  and, though human, he was always surrounded by superhuman beings.
  Intoxicated with his power, Purūravas quarrelled with some Brahmins 
  and robbed them of their wealth even though they were protesting. [...]
  Therefore, cursed by the great Ṛṣis, he perished.''
  See also \BUDDHACARITA\ 11.15 {\rm (}Aiḍa = Purūravas{\rm )}: %
  \textit{ aiḍaś ca rājā tridivaṃ vigāhya  
  nītvāpi devīṃ vaśam urvaśīṃ tām\thinspace | 
  lobhād ṛṣibhyaḥ kanakaṃ jihīrṣur  
  jagāma nāśaṃ viṣayeṣv atṛptaḥ\thinspace ||}.
   %
 
  For Daṇḍa{\rm (}ka{\rm )}'s story, see \RAMAYANA\ 7.71.31 ff.:
  Daṇḍa meets Arajā, a beautiful girl, in a forest and rapes her. As a consequence, her father, Śukra/Bhārgava,
  destroyes Daṇḍa's kingdom, which thus becomes the desolate Daṇḍaka-forest.
 
   %
  For two versions of the destruction of
  Sagara's sons, who were chasing the sacrificial horse of their father's Aśvamedha sacrifice,
  and by doing so disturbed Kapila's meditation, and who in turn burnt them to ashes,
  see \MBH\ 3.105.9 ff. and \BRAHMANDAPUR\ 2.52--53.
 
   %
  As for Rāvaṇa's haughtiness,
  especially the fact that he chose to be invincible by all creatures except humans,
  and its consequences,
  one should recall the story of the \RAMAYANA\ and Rāvaṇa's destruction brought about by Rāma therein.
 }}

  \maintext{atikrodhena saudāsa atipānena yādavāḥ |}%

  \maintext{atitṛṣṇāc ca māndhātā nahuṣo dvijavajñayā }||\thinspace4:57\thinspace||%
\translation{Saudāsa by excessive anger, the Yādavas by excessive drinking, Māndhātṛ by excessive desire, Nahuṣa by contempt for Brahmins, \blankfootnote{4.57 Saudāsa, also known as Kalmāṣapāda, hit Śakti, Vasiṣṭha's son, with a whip because
  the latter did not give way to him, and as a consequence Śakti cursed Saudāsa:
  Saudāsa had to roam the world as a Rākṣasa for twelve years. 
  See \MBH\ 1.166.1 ff.
 
   %
  As for the end of the Yādavas, see the short \textit{Mausalaparvan} of the \MBH\ {\rm (}canto 16{\rm )}:
  cursed by the sages Viśvāmitra, Kaṇva and Nārada, and seeing menacing omens,
  the Yādavas take to drinking in Prabhāsa and destroy each other.
  %Most probably, \textit{atitṛṣṇā} in the MSS stand for \textit{atitṛṣṇāt} {\rm (}intending \textit{atitṛṣṇayā}{\rm )}.
  The form \textit{māndhāto} in \msCb\ stands for \textit{māndhātā} {\rm (}nominative of \textit{māndhātṛ}{\rm )}.
  I have corrected it in spite of the fact that the authors' knowledge about his story may
  come from \DIVYAV\ 17, where it sometimes appears to be an a-stem noun {\rm (}\textit{māndāta}{\rm )}.
  \textit{dvijavajñayā} in \textit{pāda} d stands for \textit{dvijāvajñayā} metri causa.
 
   %
  Māndhātṛ was born from his father's body who, being excessively thirsty once,
  had drank some decoction prepared for ritual purposes and as a result become pregnant with him.
  Nevertheless, \BUDDHACARITA\ 11.13 suggests that Māndhātṛ himself was still unsatisfied
  with wordly objects even after he had obtained half of Indra's throne:
  \textit{devena vṛṣṭe 'pi hiraṇyavarṣe  
  dvīpān samagrāṃś caturo 'pi jitvā\thinspace |  
  śakrasya cārdhāsanam apy avāpya  
  māndhātur āsīd viṣayeṣv atṛptiḥ\thinspace ||}. 
  In fact, as Monika Zin points out {\rm (}\mycitep{ZinMandhatar}{149}{\rm )},
  Māndhātṛ/Māndhāta's rise and fall is a very popular theme
  in the `Narrative Art of the Amaravati School': 
  `Statistics show that in the Amaravati School the most frequently represented narrative is
  the story of King Māndhātar, which appears 47 times.'
  
   %
  Nahuṣa was elevated to the position of Indra for a period of time and he also wanted
  to take Śacī, Indra's wife. Indra instructed Śacī to tell Nahuṣa to 
  harness some Ṛsis to a vehicle and use this vehicle to take Śacī. 
  Agastya, one of the Ṛṣis, was insulted even further by Nahuṣa, therefore
  he cursed Nahuṣa, who then fell from the vehicle. See \MBH\ 12.329.35 ff. and
  a verse in the \BUDDHACARITA\ {\rm (}11.14{\rm )} that follows the one about Māndhātṛ:
   %
  \textit{bhuktvāpi rājyaṃ divi devatānāṃ  
  śatakratau vṛtrabhayāt pranaṣṭe\thinspace | 
  darpān maharṣīn api vāhayitvā  
  kāmeṣv atṛpto nahuṣaḥ papāta\thinspace ||}.
 }}

  \maintext{atidānād balir naṣṭa atiśauryeṇa arjunaḥ |}%

  \maintext{atidyūtān nalo rājā nṛgo goharaṇena tu }||\thinspace4:58\thinspace||%
\translation{[Mahā]bali perished by excessive donations, Arjuna by excessive heroism, King Nala by excessive gambling, Nṛga by taking a cow. \blankfootnote{4.58 \textit{Pāda} a is most probably a reference to Mahābali's promises made to Vāmana that caused his own fall. 
  The ultimate cause of Arjuna' death while the Pāṇḍavas were on the way to the underworld 
  was summarised by Yudhiṣṭhira thus {\rm (}\MBH\ 17.2.21ab{\rm )}:
  \textit{ekāhnā nirdaheyaṃ vai śatrūn ity arjuno 'bravīt\thinspace |
  na ca tat kṛtavān eṣa śūramānī tato 'patat}\thinspace ||.
  {\rm (}`Arjuna claimed that he could destroy the enemy in one single day. He failed to do so.
  He was a boaster, that is why he fell.'{\rm )}
 
   % 
  King Nala was an expert in the game of dice but once he lost his kingdom to Puṣkara.
  See, e.g., \MBH\ 3.56.1 ff. 
 
   %
  As for Nṛga, see \MBH\ 14.93.74:  
  \textit{gopradānasahasrāṇi dvijebhyo 'dān nṛgo nṛpaḥ\thinspace | 
  ekāṃ dattvā sa pārakyāṃ narakaṃ samavāptavān\thinspace ||.}
  {\rm (}``King Nṛga had made gifts of thousands of cows for the twice-born.
  By giving away one single cow that belonged to someone else, 
  he fell into hell.''{\rm )} 
 }}

  \maintext{damena hīnaḥ puruṣo dvijendra}%

 \nonanustubhindent \maintext{svargaṃ ca mokṣaṃ ca sukhaṃ ca nāsti |}%

  \maintext{vijñānadharmakulakīrtināśa}%

 \nonanustubhindent \maintext{bhavanti vipra damayā vihīnāḥ }||\thinspace4:59\thinspace||%
\translation{[For] a person who is without self-restraint, O great Brahmin, there is no heaven, liberation or happiness. O Brahmin, people without self-restraint are the destruction of knowledge, Dharma, family and fame. \blankfootnote{4.59 Note how flexible the gender of most nouns is in \textit{pāda} b: 
  \textit{svarga}, \textit{mokṣa} and \textit{dama} are usually masculine in standard Sanskrit.
 The majority of the witnesses suggest that \textit{pāda} c ends in a stem form noun {\rm (}\textit{°nāśa}{\rm )},
  although a singular masculine nominative {\rm (}as in \Ed{\rm )} may work.
  This \textit{pāda} is unmetrical, or rather it applies the licence of a word-final
  short syllable being counted as potentially long {\rm (}\textit{°dharMA°}{\rm )}. 
 Note how \textit{viprā} in \textit{pāda} d is probably an attempt in some MSS to restore the metre.
  This \textit{pāda} is also unmetrical, or rather the licence of a word-final
  short syllable being counted as potentially long is again applied {\rm (}\textit{viPRA}{\rm )}.
 }}

  \subchptr{yameṣu ghṛṇā {\rm {\rm (}6{\rm )}}}%

  \trsubchptr{The sixth Yama-rule: Taboos}%

  \maintext{nirghṛṇo na paratrāsti nirghṛṇo na ihāsti vai |}%

  \maintext{nirghṛṇe na ca dharmo 'sti nirghṛṇe na tapo 'sti vai }||\thinspace4:60\thinspace||%
\translation{A person without taboos does not exists either in this or the other world. In a person without taboos there is no Dharma or religious austerity. \blankfootnote{4.60 The implications of \textit{pāda}s ab are not crystal clear to me. Perhaps:
  such a person has no right for existence in society and has no place in heaven.
 }}

  \maintext{parastrīṣu parārtheṣu parajīvāpakarṣaṇe |}%

  \maintext{paranindāparānneṣu ghṛṇāṃ pañcasu kārayet }||\thinspace4:61\thinspace||%
\translation{These five should be treated as taboo: women who are not depending on oneself, others' wealth, taking away others' lives, hurting others and [consuming] others' food. }

  \maintext{parastrī śṛṇu viprendra ghṛṇīkāryā sadā budhaiḥ |}%

  \maintext{rājñī viprī parivrājā svayoniparayoniṣu }||\thinspace4:62\thinspace||%
\translation{Listen, O great Brahmin, the wise should always treat women who are not dependent on oneself as taboo, [be she] a queen, a Brahmin's wife, a wandering religious mendicant, a relative or of another caste. \blankfootnote{4.62 The translation of \textit{parayoni} in \textit{pāda} d is tentative.
 }}

  \maintext{parārthe śṛṇu bhūyo 'nya anyāyārtha{-}m{-}upārjanam |}%

  \maintext{āḍhaprasthatulāvyājaiḥ parārthaṃ yo 'pakarṣati }||\thinspace4:63\thinspace||%
\translation{Listen further to something else, with regards to others' wealth. [It may include] gaining wealth through unlawful means, when somebody takes away other people's wealth by cheating with weights of one \textit{āḍha[ka]} or a \textit{prastha} and with scales. \blankfootnote{4.63 Although \textit{'nya} in \textit{pāda} a could be interpreted several ways {\rm (}e.g. \textit{anye} for \textit{anyasmin}, 
  or taken to be the first element of a compound: \textit{anya-anyāyārtha-}{\rm )},
  I think that \textit{bhūyo 'nyat} is a fixed expression meaning `something/anything more.' 
  See, e.g., \BHG\ 7.2cd:
  \textit{yaj jñātvā neha bhūyo 'nyaj jñātavyam avaśiṣyate}.
 }}

  \maintext{jīvāpakarṣaṇe vipra ghṛṇīkurvīta paṇḍitaḥ |}%

  \maintext{vanajāvanajā jīvā vilagāś caraṇācarāḥ }||\thinspace4:64\thinspace||%
\translation{O Brahmin, the wise should regard the taking away [of others'] lives as taboo. Wild and domesticated animals, serpents, [in general,] plants and animals [are examples of life forms not to destroy]. \blankfootnote{4.64 In \textit{pāda} d, I take \textit{caraṇācarāḥ} as standing for \textit{carācarāḥ} {\rm (}\textit{cara-acarāḥ}{\rm )} metri causa.
  Alternatively, one may understand it as \textit{caraṇacarāḥ} {\rm (}metri causa{\rm )}, 
  meaning `those who move on their feet,' perhaps as opposed to snakes {\rm (}\textit{bilaga} or \textit{bilaṃga}{\rm )}.
  Neither solution is fully satisfactory. Note that this \textit{pāda} also involves a small correction.
 }}

  \maintext{paranindā ca kā vipra śṛṇu vakṣye samāsataḥ |}%

  \maintext{devānāṃ brāhmaṇānāṃ ca gurumātātithidviṣaḥ }||\thinspace4:65\thinspace||%
\translation{And what is the hurting of others? Listen, O Brahmin, I'll tell you briefly. He who is hostile to the gods, Brahmins, gurus, mothers and guests [hurts others]. \blankfootnote{4.65 Note \textit{mātā} as a stem form in \textit{pāda} d.
 }}

  \maintext{parānneṣu ghṛṇā kāryā abhojyeṣu ca bhojanam |}%

  \maintext{sūtake mṛtake śauṇḍe varṇabhraṣṭakule naṭe }||\thinspace4:66\thinspace||%
\translation{As regards other people's food, eating together with people whose food is not to be accepted {\rm (}\textit{abhojyeṣu}{\rm )} is taboo, [e.g.] after birth or death [in a family], in case of vendors of alcohol, or a family having lost their caste, and in the case of a [member of the] Naṭa [caste of dancers]. \blankfootnote{4.66 One should probably understand \textit{śauṇḍe} in \textit{pāda} c as \textit{śauṇḍike}, `a distiller,' or, alternatively,
  it may be corrupted from \textit{ṣaṇḍhe}, `a eunuch'; see both in \VasDh\ 14.1--3:
  \textit{athāto bhojyābhojyaṃ ca varṇayiṣyāmaḥ\thinspace |
  cikitsaka-mṛgayu-puṃścalī-ḍaṇḍika-stenābhiśastar-ṣaṇḍha-patitānām annam abhojyam\thinspace |
  kadarya-dīkṣita-baddhātura-somavikrayi-takṣa-rajaka-śauṇḍika-sūcaka-vārdhuṣika-carmāvakṛntānām\thinspace ||} etc.
   %
  Translated in \mycitep{OlivelleDharmasutras}{285} as:
  `Next we will describe food that is fit and food that is
  unfit to be eaten [\dots] The following are unfit
  to be eaten: food given by a physician, a hunter, a harlot, a law
  enforcement agent, a thief, a heinous sinner [...] a
  eunuch, or an outcaste; as also that given by a miser, a man
  consecrated for a sacrifice, a prisoner, a sick person, a man who
  sells Soma, a carpenter, a washerman, a liquor dealer, a spy, an
  usurer, a leather worker\dots'
   %
  In support of reading \textit{ṣaṇḍhe}, one might consult \MANU\ 3.239:
   %
  \textit{cāṇḍālaś ca varāhaś ca kukkuṭaḥ śvā tathaiva ca\thinspace | 
  rajasvalā ca ṣaṇḍhaś ca nekṣerann aśnato dvijān\thinspace ||.}
  Translated in \mycitep{OlivelleDharmasutras}{120} as:
  `A Cāṇḍāla, a pig, a cock, a dog, a menstruating woman, or a eunuch must not
  look at the Brahmins while they are eating.'
 }}

  \maintext{ete pañcaghṛṇāsu saktapuruṣāḥ svargārthamokṣārthino}%

 \nonanustubhindent \maintext{loke 'nindanam āpnuvanti satataṃ kīrtir yaśo'laṃkṛtāḥ |}%

  \maintext{prajñābodhaśrutiṃ smṛtiṃ ca labhate mānaṃ ca nityaṃ labhed}%

 \nonanustubhindent \maintext{dākṣiṇyaṃ sabhavet sa āyuṣa paraṃ prāpnoti niḥsaṃśayaḥ }||\thinspace4:67\thinspace||%
\translation{Those people who stick to the five kinds of taboo [and thus] seek heaven, wealth and liberation, will reach eternal faultlessness in this world, embellished with fame and glory. [A person like that] will obtain wisdom, intelligence, [knowledge of] the Śruti and Smṛti traditions, and honour forever. Kindness will arise and he will obtain an extra long life, no doubt. \blankfootnote{4.67 Understand \textit{kīrtir-yaśo°} as \textit{kīrtiyaśo°} {\rm (}'r' being an intrusive consonant here metri causa{\rm )}, 
  as in 5.20 below. Alternatively, as suggested by Francesco Sferra, emend to \textit{kīrtiṃ yaśo'laṃkṛtām}.
  My emendation of °\textit{kṛtam} to °\textit{kṛtāḥ} is influenced be 5.20b.
 In \textit{pāda} c, note the muta cum liquida licence that allows °\textit{bodhaśrutiṃ}°
  to scan as - \shortsyllable\ \shortsyllable\ - , the consonant cluster 
  \textit{śr} not turning the previous syllable long.
 \textit{Pāda} d has several problems. I take \textit{sabhavet} as standing for \textit{sambhavet} metri causa,
  and I had to emend \textit{samāyuṣa} to \textit{sa āyuṣa} to make sense of it.
  Understand \textit{āyuṣa} as \textit{āyuḥ} {\rm (}metri causa{\rm )}, otherwise emend to \textit{sa mānuṣya}.
  Also consider correcting \textit{niḥsaṃśayaḥ} to \textit{niḥsaṃśayam}.
 }}

  \subchptr{yameṣu pañcavidho dhanyaḥ {\rm {\rm (}7{\rm )}}}%

  \trsubchptr{The seventh Yama-rule: The five methods of virtue?}%

  \maintext{caturmaunaṃ catuḥśatruś caturāyatanaṃ tathā |}%

  \maintext{caturdhyānaṃ catuṣpādaṃ pañcadhanyavidhocyate }||\thinspace4:68\thinspace||%
\translation{The four cases of observing silence, [victory over] the four enemies, the four sanctuariess, the four meditations, and the four legged [Dharma] are called the five ways of being virtuous. \blankfootnote{4.68 Understand \textit{pāda} d as \textit{pañcavidho dhanya ucyate}.
 }}

  \maintext{caturmaunasya vakṣyāmi śṛṇuṣvāvahito bhava |}%

  \maintext{pāruṣyapiśunāmithyāsambhinnāni ca varjayet }||\thinspace4:69\thinspace||%
\translation{I shall tell you about the four cases of observing silence. Listen, be attentive. One should avoid violent and slanderous [words], lies, and idle [talk]. \blankfootnote{4.69 Note the genitive with a verb meaning `to tell' in \textit{pāda} a, similarly to 1.38a and \verify.
 Similar teachings on \textit{mauna} in \DHARMP\ 1.31cd--32ab and \DIVYAV\ 186.21 are quoted in the apparatus.
 }}

  \maintext{kāmaḥ krodhaś ca lobhaś ca mohaś caiva caturvidhaḥ | }%

  \maintext{catuḥśatrur nihantavyaḥ so 'rihā vītakalmaṣaḥ }||\thinspace4:70\thinspace||%
\translation{The fourfold enemy [made up of] desire, anger, greed and delusion is to be destroyed. He who destroys [these] enemies will become sinless. \blankfootnote{4.70 Possible direct sources for the idea that \textit{kāma} is an enemy to be defeated or avoided include
  \BUDDHACARITA\ 11.17:
   %
  \textit{cīrāmbarā mūlaphalāmbubhakṣā  
  jaṭā vahanto 'pi bhujaṃgadīrghāḥ\thinspace |  
  yair nānyakāryā munayo 'pi bhagnāḥ  
  kaḥ kāmasaṃjñān mṛgayeta śatrūn\thinspace ||};
   %
  see also \BHG\ 3.43:
   %
  \textit{evaṃ buddheḥ paraṃ buddhvā saṃstabhyātmānam ātmanā\thinspace | 
  jahi śatruṃ mahābāho kāmarūpaṃ durāsadam\thinspace ||}.
  As for \textit{arihā} in \textit{pāda} d, the notion that a saint is a `destroyer of the enemies' 
  [that are evil states of mind] {\rm (}\textit{arihanta/arahanta}{\rm )}
  in Jainism, but less so in Buddhism, is discussed in \mycitep{GombrichWhat2013}{57--58}.
 }}

  \maintext{caturāyatanaṃ vipra kathayiṣyāmi tac chṛṇu |}%

  \maintext{karuṇā muditopekṣā maitrī cāyatanaṃ smṛtam }||\thinspace4:71\thinspace||%
\translation{I shall teach you the four sanctuaries. Listen, O Brahmin. Compassion, sympathy in joy, indifference, and benevolence are the four sanctuaries. \blankfootnote{4.71 This verse teaches the four Buddhist \textit{brahmavihāra}s under the label
  \textit{caturāyatana}. Therfore the word \textit{āyatana} seems to be a synonym of \textit{vihāra} here,
  and its use a simple method of appropriating it, turning the list into a Brahmanical one.
 }}

  \maintext{caturdhyānādhunā vakṣye saṃsārārṇavatāraṇam |}%

  \maintext{ātmavidyābhavaḥ sūkṣmaṃ dhyānam uktaṃ caturvidham }||\thinspace4:72\thinspace||%
\translation{I shall now teach you the four meditations, which will liberate you from transmigration. Meditation is taught to be fourfold: of the Self, \textit{vidyā}, \textit{bhava} [= Śiva] and the subtle one {\rm (}\textit{sūkṣma}{\rm )}. \blankfootnote{4.72 Note the stem form \textit{dhyāna} in \textit{°dhyānādhunā} {\rm (}for \textit{°dhyānam adhunā}{\rm )} in \textit{pāda} a.
 For contrast, but also for similarities, see the \textit{dhyānayajña} section in \VSS\ 6.7ff, in which
  five types of related meditations are taught. See analysis on pp. Intro \verify.
 }}

  \maintext{ātmatattvaḥ smṛto dharmo vidyā pañcasu pañcadhā |}%

  \maintext{ṣaṭtriṃśākṣaram ityāhuḥ sūkṣmatattvam alakṣaṇam }||\thinspace4:73\thinspace||%
\translation{The \textit{tattva} of the Self is Dharma. \textit{Vidyā} is in the five in a fivefold way[??]. They call the thirty-sixth the imperishable one, [and] the subtle \textit{tattva} has no attributes. \blankfootnote{4.73 This verse is difficult to interpret. \textit{Pāda}s a to d should define \textit{ātman}, \textit{vidyā}, \textit{bhava}, and \textit{sūkṣma},
  objects of meditation, respectively. In \textit{pāda} a, \textit{dharmo} is suspect: it may be the result of
  an eye-skip to \textit{pāda} a of the next verse. \textit{Pāda} b might refer to \textit{tattva}s in an ontological
  system of 25, 26 or 36 \textit{tattva}s.
 If \textit{pāda} c is in fact a reference to a 36-\textit{tattva} philosophical system,
  it is in striking contrast with the 25-\textit{tattva} system described in \VSS\ chapter 20.
  I take \textit{ṣaṭtriṃśa} as being in stem form.
 }}

  \maintext{catuṣpādaḥ smṛto dharmaś caturāśramam āśritaḥ |}%

  \maintext{gṛhastho brahmacārī ca vānaprastho 'tha bhaikṣukaḥ }||\thinspace4:74\thinspace||%
\translation{The four-legged one is said to be Dharma [as] it rests on the four \textit{āśrama}s, [those of] the householder, the chaste one, the forest-dweller and the mendicant. }

  \maintext{dhanyās te yair idaṃ vetti nikhilena dvijottama |}%

  \maintext{pāvanaṃ sarvapāpānāṃ puṇyānāṃ ca pravardhanam }||\thinspace4:75\thinspace||%
\translation{Virtuous are those who know these thoroughly, O great Brahmin. [They will experience] the purification of all sins and the growth of merits. \blankfootnote{4.75 Note the plural instrumental {\rm (}\textit{yair}{\rm )} with a singular active verb {\rm (}\textit{vetti}; anacoluthic structure{\rm )}.
 }}

  \maintext{āyuḥ kīrtir yaśaḥ saukhyaṃ dhanyād eva pravardhate |}%

  \maintext{śāntiḥ puṣṭiḥ smṛtir medhā jāyate dhanyamānave }||\thinspace4:76\thinspace||%
\translation{One's life-span, fame and glory and happiness grow only through virtue {\rm (}\textit{dhanya}{\rm )}. In a virtuous person piece, prosperity, tradition {\rm (}\textit{smṛti}{\rm )} and intelligence will arise. \blankfootnote{4.76 Emending °\textit{mānavaḥ} to °\textit{mānave} might err by overcorrection, and °\textit{mānavaḥ} may have originally
  been felt like a genitive {\rm (}`for a person\dots'{\rm )}.
 }}

  \subchptr{yameṣv apramādaḥ {\rm {\rm (}8{\rm )}}}%

  \trsubchptr{The eighth Yama-rule: Lack of negligence}%

  \maintext{pramādasthāna pañcaiva kīrtayiṣyāmi tac chṛṇu |}%

  \maintext{brahmahatyā surāpānaṃ steyo gurvaṅganāgamam |}%

  \maintext{mahāpātakam ity āhus tatsaṃyogī ca pañcamaḥ }||\thinspace4:77\thinspace||%
\translation{There are five areas of negligence. I shall teach them to you, listen. Murdering a Brahmin, drinking alcohol, stealing, having sex with the guru's wife: they call these grievous sins. The fifth is when one is connected with them [i.e. with these sins or with people involved in these sinful acts]. \blankfootnote{4.77 Note the stem form noun in \textit{pāda} a {\rm (}°\textit{sthāna}{\rm )} metri causa, and also 
  that this stem form noun may function as a singular noun
  next to a number {\rm (}\textit{pañca}{\rm )}, a frequently seen phenomenon in this text.
 See the apparatus to the Sanskrit text for very similar verses in the \MBH, \MANU\ and 
  the \YAJNS, and note how \textit{pāda} f slightly deviates from \MANU\ 11.55, which is translated in
  \mycitep{OlivelleManu}{217--218} as: 
  `Killing a Brahmin, drinking liquor, stealing, and having sex with an elder's 
  wife---they call these ``grievous sins causing loss of caste''; 
  and so is establishing any links with such individuals.'
 }}

  \maintext{anṛtaṃ ca samutkarṣe rājagāmī ca paiśunaḥ |}%

  \maintext{guroś cālīkanirbandhaḥ samāni brahmahatyayā }||\thinspace4:78\thinspace||%
\translation{A lie concerning one's superiority, a slander that reaches the king's ear, and false accusations against an elder are equal to killing a Brahmin. \blankfootnote{4.78 This verse being a quotation of \MANU\ 11.56, my translation 
  is based on \mycitep{OlivelleManu}{218}.
 }}

  \maintext{brahmojjhaṃ vedanindā ca kūṭasākṣī suhṛdvadhaḥ |}%

  \maintext{garhitānādyayor jagdhiḥ surāpānasamāni ṣaṭ }||\thinspace4:79\thinspace||%
\translation{Abandoning the Vedas, reviling the Vedas, being a false witness, murdering a friend, eating unfit or forbidden food are six [deeds that are] equal to drinking alcohol. \blankfootnote{4.79 This verse continues quoting \MANU. \textit{Pāda} a in the witnesses may actually be no more than the result of 
  misreading of the syllable \textit{jjha} in \MANU\ 11.57. Note the variant \textit{brahmojjhaṃ vedanindā ca}
  in both the `Northern' and `Southern' transmissions in Olivelle's critical edition 
  of \MANU\ {\rm (}\mycitep{OlivelleManu}{847}{\rm )}.
 }}

  \maintext{retotsekaḥ svayonyāsu kumārīṣv antyajāsu ca |}%

  \maintext{sakhyuḥ putrasya ca strīṣu gurutalpasamaḥ smṛtaḥ }||\thinspace4:80\thinspace||%
\translation{Sexual intercourse with a female relative, with an unmarried girl, with women of the lowest castes, with the wife of a friend or of one's own son are said to be equal to violating the guru's bed. \blankfootnote{4.80 The text, and my emendation in \textit{pāda} c, still follow \MANU\ {\rm (}11.59{\rm )}.
 }}

  \maintext{nikṣepasyāpaharaṇaṃ narāśvarajatasya ca |}%

  \maintext{bhūmivajramaṇīnāṃ ca rukmasteyasamaḥ smṛtaḥ }||\thinspace4:81\thinspace||%
\translation{Stealing deposits, people, horses, silver, land, diamonds, or gems are said to be equal to stealing gold. \blankfootnote{4.81 This is \MANU\ 11.58. I have emended \textit{rugma}° to \textit{rukma}° in \textit{pāda} d, although
  \textit{rugma}° is attested in a great number of Southern MSS and one Śāradā MS in \mycitep{OlivelleManu}{847}.
 }}

  \maintext{catvāra ete sambhūya yat pāpaṃ kurute naraḥ |}%

  \maintext{mahāpātakapañcaitat tena sarvaṃ prakāśitam |}%

  \maintext{pañcapramādam etāni varjanīyaṃ dvijottama }||\thinspace4:82\thinspace||%
\translation{If a man is associated with [any of these] four [i.e. \textit{brahmahatyā, surāpāna, stena, gurvaṅganāgama}], he commits sin. By this all the five grievous sins have been explained. These five kinds of negligence are to be avoided, O great Brahmin. \blankfootnote{4.82 Perhaps understand \textit{pāda} c as \textit{etan mahāpātakapañcakaṃ}.
 Note the confusion of number and gender: understand \textit{pañca pramādāḥ etā varjanīyāḥ}
  or \textit{pañca prāmādāny etāni varjanīyāni}.
 }}

  \subchptr{yameṣu mādhuryam {\rm {\rm (}9{\rm )}}}%

  \trsubchptr{The ninth Yama-rule: Charm}%

  \maintext{kāyavāṅmanamādhuryaś cakṣur buddhiś ca pañcamaḥ |}%

  \maintext{saumyadṛṣṭipradānaṃ ca krūrabuddhiṃ ca varjayet }||\thinspace4:83\thinspace||%
\translation{[Charm has five types:] bodily, verbal and mental charm, [charm of] the eyes and [of one's] thoughts as fifth. Giving [others] a friendly glance [is commendable] and one should avoid cruel thoughts. \blankfootnote{4.83 My emendation from \textit{°manasā dhūryaś} to \textit{°mana-mādhuryaś} is based on the fact that following the list
  of \textit{yama}s in 3.16cd--17ab, we need some reference to \textit{mādhurya} here and that it is easy to see how this
  corruption came about: \textit{°mano-mādhurya°} would be unmetrical, hence the form \textit{°mana-mādhurya};
  \textit{°mana-mā°} is easily corrupted to \textit{°manasā°} {\rm (}not to mention the fact 
  that \textit{manasā} comes up in the next verse{\rm )}. 
  In addition, we need five items in this line because of \textit{pañcamaḥ}.
  As always, I correct \textit{mādhūrya} to \textit{mādhurya}, although it seems that 
  the former is acceptable in this text. 
  I did not correct \textit{mādhuryaś} to \textit{mādhuryaṃ} because of the corresponding
  \textit{pañcamaḥ}.
 }}

  \maintext{prasannamanasā dhyāyet priyavākyam udīrayet |}%

  \maintext{yathāśaktipradānaṃ ca svāśramābhyāgato guruḥ }||\thinspace4:84\thinspace||%
\translation{One should meditate with a tranquil mind and should speak [to other people using] gentle words. [When] respectable people arrive at one's own hermitage, [one should] present them with as many gifts as one can, \blankfootnote{4.84 \textit{Pāda}s cd of the previous verse, and \textit{pāda}s ab of the present one cover
  four categories of the above: \textit{cakṣurmādhurya}, \textit{buddhimādhurya}, \textit{dṛṣṭimādhurya} and \textit{vāgmādhurya}.
  This suggests that what follows is on \textit{kāyamādhurya}.
 Emending \textit{pāda} d to \textit{svāśramābhyāgate gurau} would make the line smoother, as
  suggested by Kengo Harimoto.
 }}

  \maintext{indhanodakadānaṃ ca jātavedam athāpi vā |}%

  \maintext{sulabhāni na dattāni indhanāgnyudakāni ca |}%

  \maintext{kṣute jīveti vā noktaṃ tasya kiṃ parataḥ phalam }||\thinspace4:85\thinspace||%
\translation{with gifts of fire-wood, water and fire. [If] fire-wood, fire and water are easily available [but] are not given [as gift] or [if the phrase] `Live [for a hundred years]!' is not uttered when [somebody] sneezes, what reward could there be for such a person in the afterlife? \blankfootnote{4.85 Understand \textit{jātavedam} in \textit{pāda} b as \textit{jātavedasam} or \textit{jātavedāḥ},
  or rather as belonging to the compound °\textit{dānaṃ}: \textit{jātavedodānaṃ}.
 For \textit{pāda} e, see an Āryāgīti verse in the \MAHASUBHS\ {\rm {\rm (}2558{\rm )}}: 
  \textit{amṛtāyatām iti vadet pīte bhukte kṣute ca śataṃ jīva\thinspace |
  choṭikayā saha jṛmbhāsamaye syātāṃ cirāyurānandau\thinspace ||}
  {\rm (}`When eating or drinking, one should say: ``May it turn into nectar!''; 
  and after sneezing: ``Live for a hundred years!''
  By snapping the thumb and forefinger when yawning, there will be long life and happiness.'{\rm )}
 }}

  \subchptr{yameṣv ārjavam {\rm {\rm (}10{\rm )}}}%

  \trsubchptr{The tenth Yama-rule: Sincerity}%

  \maintext{pañcārjavāḥ praśaṃsanti munayas tattvadarśinaḥ |}%

  \maintext{karmavṛttyābhivṛddhiṃ ca pāratoṣikam eva ca |}%

  \maintext{strīdhanotkocavittaṃ ca ārjavo nābhinandati }||\thinspace4:86\thinspace||%
\translation{The sages who see the truth praise five types of sincerity. A sincere person does not rejoice in prosperity arising from the operation of karma or by a reward, in riches from women, from property, and bribery. \blankfootnote{4.86 °\textit{ārjavāḥ} should be in the accusative, therefore it is to be taken as feminine {\rm (}rather than neuter{\rm )} or as
  an irregular form for °\textit{ārjavāni}. I have emended \textit{pāratoṣikam} to \textit{pāritoṣikam}.
 My translation of the categories listed here is tentative, the only guiding light being
  that, if the first line is right, there should be five of them. In addition, I have tried to
  find categories that seem to be, more or less, in conflict with `sincerity' or `straightness.'
 }}

  \maintext{ārjavo na vṛthā yajña ārjavo na vṛthā tapaḥ |}%

  \maintext{ārjavo na vṛthā dānam ārjavo na vṛthāgnayaḥ }||\thinspace4:87\thinspace||%
\translation{If one is not sincere, sacrifice is in vain. If one is not sincere, austerity is in vain. If one is not sincere, donation is in vain. If one is not sincere, [sacrificial] fires are in vain. \blankfootnote{4.87 I thank Nirajan Kafle for helping me interpret this verse.
 }}

  \maintext{ārjavasyendriyagrāmaḥ suprasanno 'pi tiṣṭhati |}%

  \maintext{ārjavasya sadā devāḥ kāye tasya caranti te }||\thinspace4:88\thinspace||%
\translation{The sense faculties of a sincere person are firm even when he is delighted. The gods are always present in the body of a sincere person. }

  \maintext{iti yamapravibhāgaḥ kīrtito 'yaṃ dvijendra}%

 \nonanustubhindent \maintext{iha parata sukhārthaṃ kārayet taṃ manuṣyaḥ |}%

  \maintext{duritamalapahārī śaṅkarasyājñayāste}%

 \nonanustubhindent \maintext{bhavati pṛthivibhartā hy ekachatrapravartā }||\thinspace4:89\thinspace||%
\translation{Thus has been taught this section on the \textit{yama}-rules, O great Brahmin. Humans should follow them to reach happiness here and in the other world. One will stand removing one's filth of sins, and shall by Śaṅkara's command become a ruler of the world [that he subjugates] under one royal umbrella. \blankfootnote{4.89 In \textit{pāda} a °\textit{pra}° does not make the previous syllable long: this is the phenomenon of
  `muta cum liquida,' one of the hallmarks of the \VSS, 
  that is, syllables such as \textit{tra, pra, bra, dra} do not necessarily make the 
  previous syllable long.
 In \textit{pāda} b, \textit{parata} most probably stands for \textit{paratra} or \textit{parataḥ} metri causa. 
  We may correct it to \textit{paratra}, presupposing the presence of the licence `muta cum liquida.'
 \textit{°malapahārī} in the MSS stands either for \textit{°malāpahārī} or \textit{°malaprahārī} metri causa. 
  I could have choosen to emend it to \textit{°malaprahārī} {\rm (}again applying the licence `muta cum liquida'{\rm )},
  but I decided not to because \textit{apahārin}, \textit{apahāra}, \textit{apahāraka} are used in the text very frequently. 
  See also 8.44c, which contains a very similar expression: \textit{sakalamalapahāre dharmapañcāśad etat}.
 }}
\center{\maintext{\dbldanda\thinspace iti vṛṣasārasaṃgrahe yamavibhāgo nāmādhyāyaś{ }caturthaḥ\thinspace\dbldanda}}
\translation{Here ends the fourth chapter in the \textit{Vṛṣasārasaṃgraha} called the Section on the Yama-rules.}

  \chptr{pañcamo 'dhyāyaḥ}
\addcontentsline{toc}{subsection}{Chapter 5}
\fancyhead[CO]{{\footnotesize\textit{Translation of chapter 5}}}%

  \trchptr{ Chapter Five }%

  \subchptr{niyamāḥ}%

  \trsubchptr{The Niyama-rules}%

  \maintext{vigatarāga uvāca |}%

  \maintext{kathaya niyamatattvaṃ sāmprataṃ tvaṃ viśeṣād}%

 \nonanustubhindent \maintext{amṛtavacanatulyaṃ śrotukāmo gato 'smi |}%

  \maintext{prakṛtidahanadagdhaṃ jñānatoyair niṣiktam}%

 \nonanustubhindent \maintext{apara vada{-}m{-}atajjñaṃ nāsti dharmeṣu tṛptiḥ }||\thinspace5:1\thinspace||%
\translation{Vigatarāga spoke: Now teach me the true nature of the Niyama-rules in detail. I have become desirious to hear [your] teaching that is comparable to ambrosia. Tell me more {\rm (}\textit{apara vada}{\rm )}, [to the one who had been] burnt by the fire of materiality {\rm (}\textit{prakṛti}{\rm )}, [but is now] sprinkled with the water of knowledge, and is ignorant of [the topic]. One can't have enough of the [teaching on] Dharmas {\rm (}\textit{nāsti dharmeṣu tṛptiḥ}{\rm )}. \blankfootnote{5.1 Most witnesses read °\textit{vadana}° in \textit{pāda} b. This is slightly odd in the sense of `speech,' the meaning
  required here, therefore I follow \msM\ here. One wonders if it is not \textit{amṛtasvādana} or
  °\textit{svadana} {\rm (}`tasting nectar'{\rm )} what was meant originally. I translate the phrase in question as if it read
  \textit{amṛtatulyavacanaṃ}.
 The first half of \textit{pāda} d is difficult to interpret safely. \textit{apara vada} {\rm (}`tell me more'{\rm )} might be original,
  with \textit{apara} in stem form. The phrase \textit{matajñā} is now emended to \textit{-m-atajjñaṃ},
  containing a hiatus breaker but making the line metrical.
  Otherwise it could be emended to \textit{matajña} {\rm (}with the last syllable taken as long{\rm )} and translated as a vocative {\rm (}`O knower of [my] thoughts{\rm )}.
  Note \msM's reading for the end of the line {\rm (}\textit{me dharmatṛptiḥ}{\rm )}.
 }}

  \maintext{anarthayajña uvāca |}%

  \maintext{śravaṇasukham ato 'nyat kīrtayiṣye dvijendra}%

 \nonanustubhindent \maintext{niyamakalaviśeṣaḥ pañca pañca prakāraḥ |}%

  \maintext{hariharamunibhīṣṭaṃ dharmasāraṃ dvijendra}%

 \nonanustubhindent \maintext{kalikaluṣavināśaṃ prāyamokṣaprasiddham }||\thinspace5:2\thinspace||%
\translation{Anarthayajña spoke: I shall teach you something more that is nice to hear, O best of the twice-born. The specific sections of the Niyamas are of five types [each]. It is the essence of Dharma, dear to Hari, Hara and the sages, O great Brahmin, the destruction of the impurity of the Kali age, generally known as liberation. \blankfootnote{5.2 My suspicion is that °\textit{kala}° in \textit{pāda} b stands for \textit{kalā} metri causa. 
 Similarly, °\textit{munibhīṣṭaṃ} is metri causa, for °\textit{munyabhīṣṭaṃ} {\rm (}`dear the the sages'{\rm )}.
 In \textit{pāda} d, \textit{prāya}° is suspect. Compare with 6.1c: \textit{dharmamokṣaprasiddhyarthaṃ}.
 }}

  \maintext{śaucam ijyā tapo dānaṃ svādhyāyopasthanigrahaḥ |}%

  \maintext{vratopavāsamaunaṃ ca snānaṃ ca niyamā daśa }||\thinspace5:3\thinspace||%
\translation{Purification, sacrifice, penance, donation, Vedic study and the restraint of sexual desire, religious observances, fasting, observing silence, and bathing: these are the ten Niyamas. }

  \subchptr{niyameṣu śaucam {\rm {\rm (}1{\rm )}}}%

  \trsubchptr{The first Niyama-rule: Purity}%

  \maintext{tatra śaucādinirdeśaṃ vakṣyāmīha dvijottama |}%

  \maintext{śārīraśaucam āhāro mātrā bhāvaś ca pañcamaḥ }||\thinspace5:4\thinspace||%
\translation{From among these, now I shall tell you the particulars of purification [first], and [then] the others. [1] Bodily purity, [2] [purity of] food, [3] [purity of] property[?] {\rm (}\textit{mātrā}{\rm )}, [4] [purity of] character[?] {\rm (}\textit{bhāva}{\rm )}, and the fifth, [5]...? \blankfootnote{5.4 The chapter deals with \textit{śārīraśauca} {\rm (}5.5--9{\rm )} and \textit{āhāraśauca} {\rm (}5.10--16{\rm )}, therefore \textit{pāda} c is probably correct, 
  and \msM's reading {\rm (}\textit{śārīrasrotam āhāra}{\rm )} is wrong. Even if we could interpret \textit{pāda} d with any certainty, there
  is one missing element of this list of allegedly five items. Something must have dropped out here.
  Oddly enought, the chapter stops after teaching the second type of purity, \textit{āhāraśauca}, so we are left without a clue.
  \MBH\ Indices 14.4.3229--3230 is not very helpful: \textit{manaḥśaucaṃ karmaśaucaṃ kulaśaucaṃ ca bhārata\thinspace |
  śarīraśaucaṃ vākśaucaṃ śaucaṃ pañcavidhaṃ smṛtam}\thinspace ||.
 }}

  \subsubchptr{śarīraśaucam}%

  \trsubsubchptr{Purity of the Body}%

  \maintext{tāḍayen na ca bandheta na ca prāṇair viyojayet |}%

  \maintext{parastrīparadravyeṣu śaucaṃ kāyikam ucyate }||\thinspace5:5\thinspace||%
\translation{He should not beat, tie or kill [any living being]. [This and] purity concerning others' wives and property is called bodily purity. \blankfootnote{5.5 Note the application of the licence muta cum liquida in \textit{pāda} c: the first syllable
  of \textit{dravyeṣu} does not make the previous syllable heavy.
 }}

  \maintext{śrotraśaucaṃ dvijaśreṣṭha gudopasthamukhādayaḥ |}%

  \maintext{mukhasyācamanaṃ śaucam āhāravacaneṣu ca }||\thinspace5:6\thinspace||%
\translation{The cleanliness of the ears, O great Brahmin, and of the anus, the loins, the mouth etc. [also contributes to bodily purity]. The purity of the mouth [comes from] sipping water before eating, speaking. }

  \maintext{mūtraviṣṭāsamutsarge devatārādhaneṣu ca |}%

  \maintext{mṛttoyais tu gudopasthaṃ śaucayīta vicakṣaṇaḥ }||\thinspace5:7\thinspace||%
\translation{After the emission of urine and f\ae ces, and before the worship of gods, the wise one should clean his anus and his loins with clay and water. \blankfootnote{5.7 Note the peculiar verb form \textit{śaucayīta} {\rm (}for a more standard \textit{śocayeta}{\rm )}. \msM's \textit{śaucaye}[\textit{c}] \textit{ca}
  may be close to an original reading.
 }}

  \maintext{ekopasthe gude pañca tathaikatra kare daśa |}%

  \maintext{ubhayoḥ sapta dātavyā mṛdaḥ śuddhiṃ samīhatā }||\thinspace5:8\thinspace||%
\translation{One [portion of clay] for the loins, five for the anus, ten for one hand, [then] seven [portions] of clay are to be applied for both [hands] by him who wishes cleanliness. \blankfootnote{5.8 In essence, this verse is \MANU\ 5.136. Olivelle's notes on this verse read:
  `\textit{on one hand:} within the context the meaning is clear: ``one hand'' refers to the left
  hand, with which the person applied the earth and water to the penis and anus. All purifications
  below the navel are carried out using the left hand. Variant reading: ``on the left hand.''\thinspace '
  {\rm (}\mycitep{OlivelleManu}{287}.{\rm )}
 }}

  \maintext{etac chaucaṃ gṛhasthānāṃ dviguṇaṃ brahmacāriṇām |}%

  \maintext{vānaprasthasya triguṇaṃ yatīnāṃ tu caturguṇam }||\thinspace5:9\thinspace||%
\translation{This is the purification for the householder {\rm (}\textit{gṛhastha}{\rm )}. It is twice as much for the chaste one {\rm (}\textit{brahmacārin}{\rm )}, three times as much for the forest-dweller {\rm (}\textit{vānaprastha}{\rm )}, four times as much for the ascetic {\rm (}\textit{yati}{\rm )}. \blankfootnote{5.9 This verse corresponds to \MANU\ 5.137.
 Note the muta cum liquida licence in \textit{pāda} c: \textit{tr} does not turn the previous syllable heavy and 
  the \textit{pāda} becomes a \textit{na-vipulā}.
 }}

  \subsubchptr{āhāraśaucam}%

  \trsubsubchptr{Purity of the food}%

  \maintext{āhāraśaucaṃ vakṣyāmi śṛṇuṣvāvahito bhava |}%

  \maintext{bhāgadvayaṃ tu bhuñjīta bhāgam ekaṃ jalaṃ pibet |}%

  \maintext{vāyusaṃcāradānārthaṃ caturtham avaśeṣayet }||\thinspace5:10\thinspace||%
\translation{I shall teach you the rules of purity concerning food. Listen, pay great attention. One should eat [as much] food [that fills] two quarters [of the stomach] and drink water [that fills] one quarter. In order to give passage to the air, one should save the remaining quarter. \blankfootnote{5.10 Śaṅkara quotes a similar verse in his commentary ad \BHG\ 6.16 {\rm (}see apparatus{\rm )}.
  It translates as:
  `Half is for saucy food, the third part for water, but in order to be able to move the air,
  one should leave the fourth part [empty].' This verse and one in the \SANNYASUP\ {\rm (}see apparatus{\rm )} have
  \textit{saṃcaraṇārthaṃ tu} and \textit{saṃcaraṇārthāya}, respectively, where our verse in the \VSS\ has \textit{saṃcāradānārthaṃ}.
  It would be tempting to emend but the \VSS\ version more or less works fine, therefore 
  there is no need to alter the text.
 }}

  \maintext{snigdhasvādurasaiḥ ṣaḍbhir āhāraṣaḍrasair budhaḥ |}%

  \maintext{dhātuvaiṣamyanāśo 'sti na ca rogāḥ sudāruṇāḥ }||\thinspace5:11\thinspace||%
\translation{[By] the wise one['s applying] the six soft and sweet juices, [which are] the six flavours in food, the disturbances of the \textit{dhātu}s will disappear and the terrible illnesses will not arise. \blankfootnote{5.11 The readings may suggest that \textit{pāda} b contains \textit{sadrava} or maybe \textit{sudrava}, but it is difficult to make
  sense of the sentence. We are lacking a verb; \textit{āhāra} might be wrong for \textit{āharet} {\rm (}see \msM{\rm )}.
  The Āyurvedic implications of this clumsy verse are obscue to me. What is clear is that traditionally there are
  six basic flavours or `juices' in food. See, e.g. \BHELAS\ 1.28.1:
  \textit{yad bhakṣayati bhuṅkte vā vidhivac cāpi mānavaḥ\thinspace |
  anyac ca kiñcit pibati tat sarvaṃ ṣaḍrasānvitam\thinspace ||.}
  {\rm (}`All that a human eats or enjoys according to the rules, and furthermore all 
  that he or she drinks, is endowed with the six flavours.'{\rm )}
  To repair \textit{pāda}s ab, one should perhaps imagine that the intended meaning was that the
  six flavours/juices should be present in a harmonious proportion in a wise man's food. Cf. \BHELAS\ 3.1.1:
  \textit{śarīraṃ dhārayantīha ṣaḍrasāḥ samam āhṛtāḥ\thinspace |
  ato 'nyathā vikārāṃs tu janayanti śarīriṇām\thinspace ||.}
  {\rm (}`The six flavours will support the body in this world when brought to a balanced state.
  Otherwise they will produce defects to people.'{\rm )}
 On \textit{dhātuvaiṣamya}, see, e.g., \CARAKA\ 1.9.4:
  \textit{vikāro dhātuvaiṣamyaṃ sāmyaṃ prakṛtir ucyate\thinspace |
  sukhasaṃjñakam ārogyaṃ vikāro duḥkham eva ca\thinspace ||}
  {\rm (}`The imbalance of the \textit{dhātu}s means defects. Balance is said to be natural.
  Health is happiness, defects are suffering.'{\rm )}
 }}

  \maintext{abhakṣyaṃ ca na bhakṣeta apeyaṃ na ca pāyayet |}%

  \maintext{agamyaṃ na ca gamyeta avācyaṃ na ca bhāṣayet }||\thinspace5:12\thinspace||%
\translation{He should not eat what is forbidden and he should not drink what is forbidden. He should not go where he is not allowed to and he should not say what is improper. \blankfootnote{5.12 Understand the causative \textit{pāyayet} as simplex.
 }}

  \maintext{laśunaṃ ca palāṇḍuṃ ca gṛñjanaṃ kavakāni ca |}%

  \maintext{gauraṃ ca sūkaraṃ māṃsaṃ varjayec ca vidhānataḥ }||\thinspace5:13\thinspace||%
\translation{He should avoid garlic, onion, \textit{gṛñjana} onion, mushrooms, buffalo meat and pork, following the rules. }

  \maintext{chattrākaṃ viḍvarāhaṃ ca gomāṃsaṃ ca na bhakṣayet |}%

  \maintext{caṭakaṃ ca kapotaṃ ca jālapādāṃś ca varjayet }||\thinspace5:14\thinspace||%
\translation{He should not eat \textit{chattrāka} mushrooms, village hog, and cow flesh. He should also avoid sparrows, pigeons, and water-birds. }

  \maintext{haṃsasārasacakrāhvakukkuṭān śukaśyenakān |}%

  \maintext{kākolūkaṃ balākaṃ ca matsyādīṃś cāpi varjayet }||\thinspace5:15\thinspace||%
\translation{He should also avoid [eating] geese, cranes, \textit{cakravāka} birds, cocks, parrots and hawks, crows, owls, \textit{balāka} cranes, fish etc. \blankfootnote{5.15 Note that in \textit{pāda} b the first syllable of \textit{śyenakān} does not turn the previous syllable, \textit{śu},
  heavy. This is an extension of the muta cum liquida licence.
 }}

  \maintext{amedhyāṃś cāpavitrāṃś ca sarvān eva vivarjayet |}%

  \maintext{śākamūlaphalānāṃ ca abhakṣyaṃ parivarjayet }||\thinspace5:16\thinspace||%
\translation{He should avoid everything that is ritually impure or polluted. He should also completely avoid those vegetables, roots and fruits, that are prohibited. }

  \maintext{mānaveṣu purāṇeṣu śaivabhāratasaṃhite |}%

  \maintext{kīrtitāni viśeṣeṇa śaucācāram aśeṣataḥ |}%

  \maintext{tvayā jijñāsito 'smy adya saṃkṣiptaḥ kathito mayā }||\thinspace5:17\thinspace||%
\translation{In the books of Manu, in the Purāṇas, in Śaiva texts, and in the \textit{Bhāratasaṃhitā} {\rm (}i.e. the \textit{Mahābhārata}{\rm )}, the practice of purity is definitely expanded in great detail. Now you have asked me [about it], and I taught it [to you] in a condensed form. \blankfootnote{5.17 In \textit{pāda} b, since °\textit{saṃhite} is not a correct locative of °\textit{saṃhitā}, 
  instead of emending to \textit{śaive bhāratasaṃhite}, we may take the compound as a \textit{samāhāra\-dvandva\-samāsa} in the neuter locative.
 Note the gender and number confusion between \textit{kīrtitāni} and °\textit{ācāram} in \textit{pāda}s cd.
  This and the next verse sound as if the author had been aware of the fact that he 
  left the remaining three categories of purity {\rm (}see 5.4{\rm )} unexplained.
 }}

  \maintext{satyavādī śucir nityaṃ dhyānayogarataḥ śuciḥ |}%

  \maintext{ahiṃsakaḥ śucir dānto dayābhūtakṣamā śuciḥ }||\thinspace5:18\thinspace||%
\translation{He who speaks the truth is pure. He who engages in yogic meditation is pure. He who avoids violence and is restrained is pure. Compassion towards living beings and patience is purity. \blankfootnote{5.18 My impression is that \textit{dayābhūtakṣamā} in \textit{pāda} d may stand for \textit{bhūtadayā kṣamā} {\rm (}\textit{bhūtadayā} occurring in
  1.7 and 3.27--28{\rm )}, and I translate accordingly.
 }}

  \maintext{sarveṣām eva śaucānām arthaśaucaṃ paraṃ smṛtam |}%

  \maintext{yo 'rthe hi śuciḥ sa śucir na mṛdvāriśuciḥ śuciḥ |}%

  \maintext{kāyavāṅmanasāṃ śaucaṃ sa śuciḥ sarvavastuṣu }||\thinspace5:19\thinspace||%
\translation{Of all the [ways of] purification, material purification is taught to be the highest. For he who is pure with regards to material things is truly pure, and not the one who [only] uses clay and water [i.e. the one who performs only ordinary baths]. When purification pertains to the body, to speech and to the mind, he is pure in all respects. \blankfootnote{5.19 \textit{Pāda}s a-d are quoting \MANU\ 5.106 {\rm (}in most witnesses, unmetrically{\rm )}; it is translated in
  \mycitep{OlivelleManu}{144} as:
  `Purifying oneself with respect to wealth, tradition tells us, is the highest of all
  purifications; for the truly pure man is the one who is pure with respect to wealth, not
  the one who becomes pure by using earth and water.'
 }}

  \maintext{śaucāśaucavidhijña mānava yadi kālakṣaye niścayaḥ}%

 \nonanustubhindent \maintext{saubhāgyatvam avāpnuvanti satataṃ kīrtir yaśo'laṅkṛtāḥ |}%

  \maintext{prāptaṃ tena ihaiva puṇyasakalaṃ saddharmaśāstreritaṃ}%

 \nonanustubhindent \maintext{jīvānte ca paratra{-}m{-}īhitagatiṃ prāpnoti niḥsaṃśayam }||\thinspace5:20\thinspace||%
\translation{If a person knows the rules of purity and impurity, he will surely gain happiness at the end of time, eternally embellished with glory and fame. He has reached here in this world all the merits that the books on true Dharma teach, and at the end of his life he will undoubtedly reach the desired path in the other world. \blankfootnote{5.20 Note the stem form adjective \textit{°jña} and noun \textit{°mānava} metri causal and
  the second syllable of \textit{yadi} as a long syllable at the c\ae sura in \textit{pāda} a 
  {\rm (}see \msM's reading{\rm )}, the plural \textit{āpnuvanti} where one would expect a verb in the singular and
  \textit{kīrtir} metri causa for a compounded stem form {\rm (}\textit{kīrti°}{\rm )} in \textit{pāda} b,
  and the sandhi-bridge \textit{-m-} in \textit{paratra-m-īhita°} in \textit{pāda} d. Compare with 4.67b above.
 }}
\center{\maintext{\dbldanda\thinspace iti vṛṣasārasaṃgrahe śaucācāravidhir{ }nāmādhyāyaḥ pañcamaḥ\thinspace\dbldanda}}
\translation{Here ends the fifth chapter in the \textit{Vṛṣasārasaṃgraha} called the Method of Purification.}

  \chptr{ṣaṣṭho 'dhyāyaḥ}
\addcontentsline{toc}{subsection}{Chapter 6}
\fancyhead[CO]{{\footnotesize\textit{Translation of chapter 6}}}%

  \trchptr{ Chapter Six }%

  \subchptr{niyameṣv ijyā {\rm {\rm (}2{\rm )}}}%

  \trsubchptr{The second Niyama-rule: Sacrifice}%

  \maintext{atha pañcavidhām ijyāṃ pravakṣyāmi dvijottama |}%

  \maintext{dharmamokṣaprasiddhyarthaṃ śṛṇuṣvāvahito dvija }||\thinspace6:1\thinspace||%
\translation{[Anarthayajña continued:] Now I shall teach you the five types of sacrifice {\rm (}\textit{ijyā}{\rm )}, O excellent Brahmin, for success in Dharma and liberation. Listen carefully, O Brahmin. }

  \maintext{arthayajñaḥ kriyāyajño japayajñas tathaiva ca |}%

  \maintext{jñānaṃ dhyānaṃ ca pañcaitat pravakṣyāmi pṛthak pṛthak }||\thinspace6:2\thinspace||%
\translation{Material sacrifice, sacrifice through work, sacrifice through recitation, knowledge and meditation: I shall teach you these five one by one. \blankfootnote{6.2 Note the singular \textit{etat} after a number {\rm (}see Intro \verify{\rm )}.
 
  Compare this list of five to the somewhat similar \BHG\ 4.28:
  \textit{dravyayajñās tapoyajñā yogayajñās tathāpare\thinspace |
  svādhyāyajñānayajñāś ca yatayaḥ saṃśita-vratāḥ\thinspace ||}.
  \SDHU\ chapter 3 can be also relevant since it uses the terms
  \textit{japayajña}, \textit{jñānayajña}, and \textit{dhyānayajña}. See also \SDHU\ 1.10 {\rm (}\msCa\ f.\thinspace 42v l4{\rm )}:
  \textit{karmayajñas tapoyajñaḥ svādhyāyo dhyānam eva ca\thinspace | 
  jñānayajñaś ca pañcaite mahāyajñāḥ prakīrtitāḥ\thinspace ||}.
  Note how this definition of the five \textit{mahāyajña}s in the \SDHU\ 
  is different from the one, e.g., in \MANU\ 3.69--71
  {\rm (}\textit{brahma}°, \textit{pitṛ}°, \textit{daiva}°, \textit{bhauta}°, and \textit{nṛyajña}{\rm )}.
 }}

  \subsubchptr{arthayajñaḥ}%

  \trsubsubchptr{Material sacrifice}%

  \maintext{agnyupāsanakarmādi agnihotrakratukriyā |}%

  \maintext{aṣṭakā pārvaṇī śrāddhaṃ dravyayajñaḥ sa ucyate }||\thinspace6:3\thinspace||%
\translation{Material sacrifice includes the following: the domestic ritual fire worship etc., the public performance of the ritual of Agnihotra, [and the so-called \textit{pākayajña}s such as] the Aṣṭakā oblation, the Pārvaṇī oblation, and the ancestral ritual {\rm (}\textit{śrāddha}{\rm )}. \blankfootnote{6.3 By somewhat overtranslating the items in this list, I want to emphasise that
  the text introduces three categories of sacrifical rituals well-known from
  the time of the Gṛhyasūtras and Śrautasūtras: those of the domestic or \textit{aupāsana} fire {\rm (}\textit{gṛhyakarman}{\rm )},
  the Śrauta rituals such as the Agnihotra, and the Smārta \textit{pākayajña}s, such as the \textit{aṣṭakā}, 
  the \textit{pārvaṇī} and the \textit{śrāddha}. For a mention of the \textit{pākayajña}s in a manner similar to 
  our \textit{pāda}s cd here, see, e.g. the \Diksottara\ quoted in \mycitep{NisvasaGoodall}{275}:
  \textit{aṣṭakāḥ pārvaṇī śrāddhaṃ śrāvaṇy āgrāyaṇī tathā\thinspace |
  caitrī cāśvayujī caiva pākayajñāḥ prakīrtitāḥ\thinspace ||}.
  For an earlier list of \textit{pākayajña}s, see \GAUTDHS\ 1.8.19: 
  \textit{aṣṭakā pārvaṇaḥ śrāddham śrāvaṇy\-āgrahāyaṇī\-caitry\-āśvayujīti sapta pākayajñasamsthāḥ}.
 }}

  \subsubchptr{kriyāyajñaḥ}%

  \trsubsubchptr{Sacrifice through work}%

  \maintext{ārāmodyānavāpīṣu devatāyataneṣu ca |}%

  \maintext{svahastakṛtasaṃskāraḥ kriyāyajña sa ucyate }||\thinspace6:4\thinspace||%
\translation{Sacrifice through work is taking care of/ cleaning/ embellishing {\rm (}\textit{saṃskāra}{\rm )} a grove, a park, a pond or a temple with one's own hands. }

  \subsubchptr{japayajñaḥ}%

  \trsubsubchptr{Sacrifice through recitation}%

  \maintext{japayajñaṃ tato vakṣye svargamokṣaphalapradam |}%

  \maintext{vedādhyayana kartavyaṃ śivasaṃhitam eva ca |}%

  \maintext{itihāsapurāṇaṃ ca japayajñaḥ sa ucyate }||\thinspace6:5\thinspace||%
\translation{Next I shall teach you the sacrifice through recitation, the bestower of the fruits of heaven and liberation. One should recite the Vedas, Śaiva texts or the \textit{Mahābhārata}, the epics and the Purāṇas: this is called sacrifice with recitation. \blankfootnote{6.5 Note the stem form \textit{vedādhyayana} in \textit{pāda} c metri causa. As for the interpretation of
  \textit{śivasaṃhitam} in \textit{pāda} d, see 5.17b above: \textit{śaivabhāratasaṃhite}. 
  The proximity of these two phrases, and the fact that both give instructions
  on using texts, suggest that we should interpret them similarly. 
  It is then a \textit{samāhāra\-dvandva\-samāsa} again, in the neuter.
  Both \textit{śivasaṃhitam} and \textit{itihāsapurāṇaṃ} should be interpreted as
  being part of the compound in \textit{pāda} c: \textit{śiva\-saṃhitādhyayanaṃ} and 
  \textit{itihāsapurāṇādhyayanaṃ}. 
 
  See \textit{japayajña} mentioned, e.g., in \BHG\ 10.25c {\rm (}\textit{yajñānāṃ japayajño 'smi}{\rm )} 
  and \MANU\ 2.86 {\rm (}\textit{vidhiyajñāj japayajño viśiṣṭo daśabhir guṇaiḥ}{\rm )}.
 }}

  \subsubchptr{jñānayajñaḥ}%

  \trsubsubchptr{Sacrifice through knowledge}%

  \maintext{idaṃ karma akarmedam ūhāpohaviśāradaḥ |}%

  \maintext{śāstracakṣuḥ samālokya jñānayajñaḥ sa ucyate }||\thinspace6:6\thinspace||%
\translation{[He who can decide if] `this is [proper] action; the other is improper action' because he is knowledgeable about reasoning pro and contra, and investigates with his eyes on the Śāstras, is called [a person performing] sacrifice through knowledge. \blankfootnote{6.6 For the expression \textit{śāstracakṣuḥ}, see, e.g., \BRAHMAP\ 24.21:
  \textit{tena yajñān yathāproktān mānavāḥ śāstracakṣuṣaḥ\thinspace |
  kurvate 'harahaś caiva devān āpyāyayanti te\thinspace ||}.
  In G. P. Bhatt's translation {\rm (}\mycitep{BrahmapuranaTr1}{126}{\rm )}:
  `Day by day men with the sacred scriptures as their guides
  perform sacrifices in the manner they have been laid down and thereby nourish the gods.'
 }}

  \subsubchptr{dhyānayajñaḥ}%

  \trsubsubchptr{Sacrifice through meditation}%

  \maintext{dhyānayajñaṃ samāsena kathayiṣyāmi te śṛṇu |}%

  \maintext{dhyānaṃ pañcavidhaṃ caiva kīrtitaṃ hariṇā purā |}%

  \maintext{sūryaḥ somo 'gni sphaṭikaḥ sūkṣmaṃ tattvaṃ ca pañcamam }||\thinspace6:7\thinspace||%
\translation{I shall teach you concisely about sacrifice through meditation. Listen to me. Meditation was taught by Hari in the past as of five kinds. [Meditation on] the Sun, the Moon, Fire, Crystal and the subtle \textit{tattva} as fifth. \blankfootnote{6.7 For an analysis of this fivefold method of meditation, and this ancient-looking
  \textit{tattva}-system, see Intro \verify, and for different
  versions of the same teaching of meditation, see \VSS\ 22.19--28 and \DHARMP\ 4.5--14.
 }}

  \maintext{sūryamaṇḍalam ādau tu tattvaṃ prakṛtir ucyate |}%

  \maintext{tasya madhye śaśiṃ dhyāyet tattvaṃ puruṣa ucyate }||\thinspace6:8\thinspace||%
\translation{First it is the Sun [that should be meditated upon], which is said to be \textit{prakṛti-tattva}. He should visualize the Moon in its centre: that \textit{tattva} is said to be \textit{puruṣa}. \blankfootnote{6.8 Note the form \textit{śaśiṃ} for \textit{śaśinaṃ}.
 }}

  \maintext{candramaṇḍalamadhye tu jvālām agniṃ vicintayet |}%

  \maintext{prabhutattvaḥ sa vijñeyo janmamṛtyuvināśanaḥ }||\thinspace6:9\thinspace||%
\translation{In the centre of the Moon's disk, he should visualise a flame, a fire. That is said to be \textit{prabhu}-\textit{tattva}, the destroyer of [the circle of] birth and death. }

  \maintext{agnimaṇḍalamadhye tu dhyāyet sphaṭika nirmalam |}%

  \maintext{vidyātattvaḥ sa vijñeyaḥ kāraṇam ajam avyayam }||\thinspace6:10\thinspace||%
\translation{In the centre of the ring of Fire, he should visualize a spottless crystal. That is said to be \textit{vidyā}-\textit{tattva}, the never-born, imperishable cause. \blankfootnote{6.10 Note the stem form \textit{sphaṭika} in \textit{pāda} b metri causa.
 }}

  \maintext{vidyāmaṇḍalamadhye tu dhyāyet tattvam anuttamam |}%

  \maintext{akīrtitam anaupamyaṃ śivam akṣayam avyayam |}%

  \maintext{pañcamaṃ dhyānayajñasya tattvam uktaṃ samāsataḥ }||\thinspace6:11\thinspace||%
\translation{In the centre of the disk of \textit{vidyā}, he should visualize the highest \textit{tattva}, never-heard, unparalleled, undecaying and imperishable Śiva. The fifth \textit{tattva} of the sacrifice through meditation has been taught in short. }

  \maintext{vigatarāga uvāca |}%

  \maintext{ekaikasya tu tattvasya phalaṃ kīrtaya kīdṛśam |}%

  \maintext{kāni lokāḥ prapadyante kālaṃ vāsya tapodhana }||\thinspace6:12\thinspace||%
\translation{Vigatarāga spoke: Teach me, what are the fruits of [reaching] each \textit{tattva}? Which worlds can be attained and how much time [can one spend there], O great ascetic? \blankfootnote{6.12 The reading \textit{tritattvasya} in \textit{pāda} a in the MSS is a problem 
  because we have just finished a section mentioning five \textit{tattva}s. 
  {\rm (}This was probably noticed by \Ed, hence printing \textit{hi} for \textit{tri}°.{\rm )}
  My conjecture {\rm (}\textit{tu}{\rm )} is based on the assumption that \textit{tri} is ofter written as \textit{tṛ} 
  in Nepalese MSS {\rm (}e.g. in \msM\ at this point{\rm )} and that \textit{tṛ} may then easily get corrupted to \textit{tu}.
 }}

  \maintext{anarthayajña uvāca |}%

  \maintext{brahmalokaṃ tu prathamaṃ tattvaprakṛticintayā |}%

  \maintext{kalpakoṭisahasrāṇi śivavan modate sukhī }||\thinspace6:13\thinspace||%
\translation{Anarthayajña spoke: Through meditation on the first \textit{tattva}, \textit{prakṛti}, [one reaches] Brahmaloka. He will rejoice [there] happily like Śiva for millions of \ae ons. \blankfootnote{6.13 Understand \textit{pāda}s ab as \textit{brahmalokaṃ prathamatattvacintayā prakṛtitattvacintayā}. 
  One might take \textit{prathamaṃ} adverbially {\rm (}`firstly': \textit{prathamaṃ brahmalokaṃ prakṛtitattvacintayā}{\rm )},
  but in the next verses, the ordinal numbers {\rm (}\textit{dvitīyaṃ, tṛtīyaṃ, pañcamaṃ}{\rm )}
  always refer to the \textit{tattva}s.
 }}

  \maintext{dvitīyaṃ tattva puruṣaṃ dhyāyamāno mṛto yadi |}%

  \maintext{viṣṇulokam ito yāti kalpakoṭyayutaṃ sukhī }||\thinspace6:14\thinspace||%
\translation{If one dies while meditating on the second \textit{tattva}, \textit{puruṣa}, one goes to Viṣṇuloka from this world, [and will live there] happily for billions of \ae ons. \blankfootnote{6.14 Note the stem form \textit{tattva} in \textit{pāda} a metri causa.
 }}

  \maintext{prabhutattvaṃ tṛtīyaṃ tu dhyāyamāno mariṣyati |}%

  \maintext{śivaloke vasen nityaṃ kalpakoṭyayutaṃ śatam }||\thinspace6:15\thinspace||%
\translation{Should one die while meditating on the third, the \textit{prabhu-tattva}, one can live in Śivaloka continuously for a hundred billion \ae ons. \blankfootnote{6.15 \Ed\ changes \textit{śivaloka} to \textit{rudraloka}, probably for more contrast with
  \textit{sadāśiva} in 6.16 and \textit{śivatattva} in 6.17. \verify
 }}

  \maintext{vidyātattvāmṛtaṃ dhyāyet sadāśivam anāmayam |}%

  \maintext{akṣayaṃ lokam āpnoti kalpānāntaparaṃ tathā  }||\thinspace6:16\thinspace||%
\translation{If he visualizes the nectar of \textit{vidyā-tattva}, [i.e.] Sadāśiva, he can reach [His] diseaseless, imperishable world [and can live there] well beyond endless \ae ons. \blankfootnote{6.16 In \textit{pāda} a, \textit{amṛta} is suspect. It may refer to the world of Sadāśiva and 
  then \textit{vidyātattva} is in stem form. Alternatively, since this verse is the only one in
  this list of worlds {\rm (}6.13--17{\rm )} without an ordinal number, \textit{amṛtaṃ} may mean `four' or possibly `fourth,'
  as suggested by Monier-Williams and Apte in their dictionaries. This meaning would fit in nicely.
  In addition, dying has been mentioned above, thus \textit{amṛtaṃ} might be a corrupted form of 
  a participle from the verbal root \textit{mṛ} {\rm (}\textit{mṛyan} or \textit{maran}?{\rm )}: e.g., 
  \textit{vidyātattvaṃ mṛyan dhyāyet...} {\rm (}`should he meditation upon Vidyātattva while dying...'{\rm )}.
 }}

  \maintext{pañcamaṃ śivatattvaṃ tu sūkṣmaṃ cātmani saṃsthitam |}%

  \maintext{na kālasaṃkhyā tatrāsti śivena saha modate }||\thinspace6:17\thinspace||%
\translation{The fifth one, the subtle \textit{śiva-tattva} dwells in the Self. There is no counting of time there and he will be rejoicing [there] together with Śiva. }

  \maintext{pañcadhyānābhiyukto bhavati ca na punarjanmasaṃskārabandhaḥ}%

 \nonanustubhindent \maintext{jijñāsyantāṃ dvijendra bhavadahanakaraḥ prārthanākalpavṛkṣaḥ |}%

  \maintext{janmenaikena muktir bhavati kimu na vā mānavāḥ sādhayantu}%

 \nonanustubhindent \maintext{pratyakṣān nānumānaṃ sakalamalaharaṃ svātmasaṃvedanīyam }||\thinspace6:18\thinspace||%
\translation{[If] he practises the five meditations, there is no rebirth and no more fetters of transmigration. O excellent Brahmin, [the Lord] should be seeked, a wishing tree of desires, [as] he burns away existence. Liberation comes within one single birth! People, why should you not strive [for it]! [It is known] as the destroyer of all impurity. [It's ascertainable] by direct perception. It is not inference. It is to be experienced by one's own Self. \blankfootnote{6.18 Note how a plural passive imperative form {\rm (}\textit{jijñāsyantāṃ}{\rm )} stands for the singular
  {\rm (}\textit{jijñāsyatāṃ}{\rm )} metri causa. Note also that the last syllable of
  \textit{dvijendra} {\rm (}at the c\ae sura{\rm )} counts here as long: this phenomenon of a word-ending
  syllable becoming long by position is common in the \VSS.
 The non-standard \textit{janmena} in \textit{pāda} d seems superior to \textit{janmanā} for it
  preserves the metre.
 }}

  \subchptr{niyameṣu tapaḥ {\rm {\rm (}3{\rm )}}}%

  \trsubchptr{The third Niyama-rule: Penance}%

  \maintext{mānasaṃ tapa ādau tu dvitīyaṃ vācikaṃ tapaḥ |}%

  \maintext{kāyikaṃ ca tṛtīyaṃ tu manovākkarma tatparam |}%

  \maintext{kāyikaṃ vācikaṃ caiva tapo miśraka pañcamam }||\thinspace6:19\thinspace||%
\translation{The first type of penance is mental penance, the second is verbal penance, the third is the bodily one, the next one is the one which is [characterised by] both mental and verbal action. The fifth type of penance is a mixture of the bodily and the verbal ones. \blankfootnote{6.19 Note the stem form \textit{miśraka} in \textit{pāda} f metri causa.
 }}

  \maintext{manaḥsaumyaṃ prasādaś ca ātmanigraham eva ca |}%

  \maintext{maunaṃ bhāvaviśuddhiś ca pañcaitat tapa mānasam }||\thinspace6:20\thinspace||%
\translation{Gentleness of the mind, calmness, self-control, observing silence, and the purification of one's state of mind: mental penance comprises these five. \blankfootnote{6.20 Again, we can see the use of the singular {\rm (}\textit{etat}{\rm )} next to numbers; note also 
  the stem form \textit{tapa} in \textit{pāda} d metri causa. This verse is a paraphrase of \MBH\ 3.39.16 {\rm (}\BHG\ 17.16; see text in the
  apparatus{\rm )}.
 }}

  \maintext{anudvegakarā vāṇī priyaṃ satyaṃ hitaṃ ca yat |}%

  \maintext{svādhyāyābhyasanaṃ caiva vācikaṃ tapa ucyate }||\thinspace6:21\thinspace||%
\translation{Verbal penance is taught as speech that causes no anxiety, which is kind, true and useful, and it includes also the practice of recitation. \blankfootnote{6.21 This verse is a version of \MBH\ 6.39.15 {\rm (}\BHG\ 17.15; see it in the apparatus{\rm )}.
 }}

  \maintext{ārjavaṃ ca ahiṃsā ca brahmacaryaṃ surārcanam |}%

  \maintext{śaucaṃ pañcamam ity etat kāyikaṃ tapa ucyate }||\thinspace6:22\thinspace||%
\translation{Bodily penance is taught as the following: honesty, harmlessness, chastity, the worship of gods, and purity as the fifth. \blankfootnote{6.22 This verse seems to be a paraphrase of \MBH\ 6.39.14 {\rm (}\BHG\ 17.14; see it in the apparatus{\rm )}.
 }}

  \maintext{iṣṭaṃ kalyāṇabhāvaṃ ca dhanyaṃ pathyaṃ hitaṃ vadet |}%

  \maintext{manomiśraka pañcaitat tapa uktaṃ maharṣibhiḥ }||\thinspace6:23\thinspace||%
\translation{[Penance] which is a mixture of the mental [and the verbal] is taught by the great sages to be these five: he should speak [about things that are] agreeable, of a virtuous character, auspicious, salutary and useful. \blankfootnote{6.23 Note the use of the singular {\rm (}\textit{etat}{\rm )} next to a number and the stem form noun in \textit{pāda} c.
 }}

  \maintext{svasti maṅgalam āśīrbhir atithigurupūjanam |}%

  \maintext{kāyamiśraka pañcaitat tapa uktaṃ mahātmabhiḥ }||\thinspace6:24\thinspace||%
\translation{[Penance] in which bodily [and verbal actions] mix is taught by the great-souled ones to be these five: the worship of the guest and the guru, benediction, greetings, and blessings. \blankfootnote{6.24 See \SDHS\ 11.73--79 {\rm (}and \mycitep{SaivaUtopia2021}{91--93 and 120--121}{\rm )} 
  for a somewhat similar discussion on `kind speach.'
 }}

  \maintext{maṇḍūkayogī hemante grīṣme pañcatapās tathā |}%

  \maintext{abhrāvakāśo varṣāsu tapaḥ sādhanam ucyate }||\thinspace6:25\thinspace||%
\translation{[Being] a [so-called] frog-yogin in the winter, or one with the five fires in the summer, or one who has the clouds [i.e. the open sky] for shelter in the rainy season: these kinds of penance is called \textit{sādhana}. \blankfootnote{6.25 \Manu\ 6.23 mentions three kins of penance that corresponds to three seasons:
  \textit{grīṣme pañcatapās tu syād varṣāsv abhrāvakāśikaḥ\thinspace |
  ārdravāsās tu hemante kramaśo vardhayaṃs tapaḥ\thinspace ||}. 
  Translated in \mycitep{OlivelleManu}{149} as:
  `[He should] surround himself with the five fires in the summer; live in the open air during the rainy season;
  and wear wet clothes in the winter---gradually intensifying his ascetic toil.'
  This and \SDHSANGR\ 9.32ab {\rm (}quoted in the apparatus{\rm )} may suggest that being 
  a `frog-yogin' could be the same as wearing wet clothes or standing in water for a long time.
  A footnote to verse \MBH\ 12.309.9 in the Kumbakonam edition of the \MBH\ {\rm (}\mycite{MBhKumbakonaEd}{\rm )} suggests otherwise:
  \textit{maṇḍūkavat pāṇipādaṃ saṅkocya nyubjaḥ śete iti maṇḍūkaśāyī}. {\rm (}`The word `frog-sleeper' means
  somebody who sleeps like a frog, with his hands and feet withdrawn and with his back humped.'{\rm )} 
 }}

  \maintext{svamāṃsoddhṛtya dānaṃ ca hastapādaśiras tathā |}%

  \maintext{puṣpam utpādya dānaṃ ca sarve te tapasādhanāḥ }||\thinspace6:26\thinspace||%
\translation{Carving out his own flesh as a donation, or [offering his own] hand, feet and head, or drawing [his own] blood {\rm (}\textit{puṣpa}{\rm )} as a donation: all these are \textit{sādhana}-penances, \blankfootnote{6.26 Note the stem form \textit{svamāṃsa} in \textit{pāda} a for the accusative.
 The translation of \textit{pāda} c is tentative, but taking \textit{puṣpa} as `blood' is not only
  normal e.g. in tantric texts {\rm (}see e.g. \verify{\rm )}, but \VSS\ 17.38--39 suggest the same
  in a similar context:
  \textit{devī uvāca\thinspace |
  svamāṃsarudhiraṃ dānaṃ dānaṃ putrakalatrayoḥ\thinspace |
  kiṃ praśasyaṃ mahādeva tattvaṃ vaktum ihārhasi\thinspace ||
  maheśvara uvāca\thinspace |
  svamāṃsarudhiraṃ dānaṃ praśaṃsanti manīṣiṇaḥ\thinspace |
  śrūyatāṃ pūrvavṛttāni saṃkṣipya kathayāmy aham\thinspace ||}.
  {\rm (}`Devī spoke: Why are one's own flesh and blood and one's son and wife praised as donation, O Mahādeva?
  Tell me the truth please. Maheśvara spoke: The wise praise one's own flesh and blood as donation.
  Let's hear the old legends, I shall tell you briefly.'{\rm )}
 }}

  \maintext{kṛcchrātikṛcchraṃ naktaṃ ca taptakṛcchram ayācitam |}%

  \maintext{cāndrāyaṇaṃ parākaṃ ca tapaḥ sāṃtapanādayaḥ }||\thinspace6:27\thinspace||%
\translation{[as also] the `painful penance' and the `extremely paniful one', [eating only] at night, the `hot and painful' and [the one in which only food obtained] without solicitation [can be eaten], the \textit{cāndrāyaṇa} and \textit{parāka} penances, the `sāṃtapana,' etc. \blankfootnote{6.27 For short descriptions and the loci classici of these penances, see, e.g.,
  \mycitep{KaneHistory}{v. 4, 130--152}.
  For \textit{nakta}/\textit{naktānna} see \VSS\ 8.22 below and, e.g., \SDHS\ chapter 10, and for \textit{ayācita}, \VSS\ 8.23 below.
 }}

  \maintext{yenedaṃ tapa tapyate sumanasā saṃsāraduḥkhacchidam}%

 \nonanustubhindent \maintext{āśāpāśa vimucya nirmalamatis tyaktvā jaghanyaṃ phalam |}%

  \maintext{svargākāṅkṣyanṛpatvabhogaviṣayaṃ sarvāntikaṃ tatphalaṃ}%

 \nonanustubhindent \maintext{jantuḥ śāśvatajanmamṛtyubhavane tanniṣṭhasādhyaṃ vahet }||\thinspace6:28\thinspace||%
\translation{He who performs with a well-disposed mind this penance that puts an end to the suffering caused by transmigration {\rm (}\textit{saṃsāra}{\rm )}, abandoning the trap of hope, with a spotless mind, giving up the lowest rewards [such as] wishing for heaven, being a king and having enjoyments for the senses, will have an ultimate {\rm (}\textit{sarvāntika}{\rm )} reward. In this home of eternal births and deaths, man can bring about an accomplishment that puts an end to them. \blankfootnote{6.28 Note my emendation in \textit{pāda} a {\rm (}\textit{sumanasā} from \textit{sumanasaḥ}{\rm )} and that
  in order to restore the metre, I accepted \Ed's stem form \textit{tapa}.
 Note the stem form \textit{°pāśa} in \textit{pāda} b metri causa.
 }}
\center{\maintext{\dbldanda\thinspace iti vṛṣasārasaṃgrahe ṣaṣṭho 'dhyāyaḥ\thinspace\dbldanda}}
\translation{Here ends the sixth chapter in the \textit{Vṛṣasārasaṃgraha}.}

  \chptr{saptamo 'dhyāyaḥ}
\addcontentsline{toc}{subsection}{Chapter 7}
\fancyhead[CO]{{\footnotesize\textit{Translation of chapter 7}}}%

  \trchptr{ Chapter Seven }%

  \subchptr{niyameṣu dānam {\rm {\rm (}4{\rm )}}}%

  \trsubchptr{The fourth Niyama-rule: Donation}%

  \maintext{dānāni ca tathety āhuḥ pañcadhā munibhiḥ purā |}%

  \maintext{annaṃ vastraṃ hiraṇyaṃ ca bhūmi godāna pañcamam }||\thinspace7:1\thinspace||%
\translation{In the past the wise declared that, again, there were five kinds of donation. Donation of food, clothes, gold, land and the fifth, donation of cows. \blankfootnote{7.1 \textit{tathety} in \textit{pāda} a is suspicious and my translation of it {\rm (}`again'{\rm )} is tentative and
  is supposed to refer back to the fact that all \textit{yama}s so far have been 
  devided into five types.
  Note how \textit{annaṃ}, \textit{vastraṃ}, \textit{hiraṇyaṃ} and 
  \textit{bhūmi} {\rm (}the latter treated as neuter, or given in
  stem form{\rm )} are all meant to go with °\textit{dāna} {\rm (}again, in stem form, metri causa{\rm )}.
 }}

  \subsubchptr{annadānam}%

  \trsubsubchptr{Donation of food}%

  \maintext{annāt tejaḥ smṛtiḥ prāṇaḥ annāt puṣṭir vapuḥ sukham |}%

  \maintext{annāc chrīḥ kānti vīryaṃ ca annāt sattvaṃ ca jāyate }||\thinspace7:2\thinspace||%
\translation{From food [comes] energy, memory, the vital breath, growth, body, happiness. From food arise grace and beauty, heroism, strength. \blankfootnote{7.2 Note the stem form noun \textit{kānti} metri causa in \textit{pāda} c.
 }}

  \maintext{annāj jīvanti bhūtāni annaṃ tuṣṭikaraṃ sadā |}%

  \maintext{ānnāt kāmo mado darpaḥ annāc chauryaṃ ca jāyate }||\thinspace7:3\thinspace||%
\translation{Living beings live on food. Food always satisfies. From food arise desire, rapture, pride and valour. }

  \maintext{annaṃ kṣudhātṛṣāvyādhīn sadya eva vināśayet |}%

  \maintext{annadānāc ca saubhāgyaṃ khyātiḥ kīrtiś ca jāyate }||\thinspace7:4\thinspace||%
\translation{Food drives away hunger and thirst and disease instantly. From donations of food arise happiness, fame and glory. }

  \maintext{annadaḥ prāṇadaś caiva prāṇadaś cāpi sarvadaḥ |}%

  \maintext{tasmād annasamaṃ dānaṃ na bhūtaṃ na bhaviṣyati }||\thinspace7:5\thinspace||%
\translation{He who donates food donates life. He who donates life donates everything. Therefore nothing is equal to the donation of food, nothing was, nothing will be. }

  \subsubchptr{vastradānam}%

  \trsubsubchptr{Donation of clothes}%

  \maintext{vastrābhāvān manuṣyasya śriyād api parityajet |}%

  \maintext{vastrahīno na pūjyeta bhāryāputrasakhādibhiḥ }||\thinspace7:6\thinspace||%
\translation{In the absence of [proper] clothes, a man will also lose his fortunes. A person without clothes may not be respected by his wife, son, friends etc. \blankfootnote{7.6 \textit{Pāda} b is difficult to interpret securely. I translate it as if reading
  \textit{śrīs tam api parityajet}. Consider also \BRAHMAP\ 220.139:
  \textit{vastrābhāve kriyā nāsti yajñā vedās tapāṃsi ca\thinspace |
  tasmād vāsāṃsi deyāni śrāddhakāle viśeṣataḥ\thinspace ||}.
 }}

  \maintext{vidyāvān sukulīno 'pi jñānavān guṇavān api |}%

  \maintext{vastrahīnaḥ parādhīnaḥ paribhūtaḥ pade pade }||\thinspace7:7\thinspace||%
\translation{Be it a learned person from a good family or an intelligent and virtuous person, anybody without clothes is subdued and humiliated on every occasion }

  \maintext{apamānam avajñāṃ ca vastrahīno hy avāpnuyāt |}%

  \maintext{jugupsati mahātmāpi sabhāstrījanasaṃsadi }||\thinspace7:8\thinspace||%
\translation{because a man without clothes receives contempt and disrespect. Even a great soul will despise [him] at the court, among women, in an assembly. \blankfootnote{7.8 The intention originally may have been this: ``Even if he is a great soul, he will be avoided...''
 }}

  \maintext{tasmād vastrapradānāni praśaṃsanti manīṣiṇaḥ |}%

  \maintext{na jīrṇaṃ sphuṭitaṃ dadyād vastraṃ kutsitam eva vā }||\thinspace7:9\thinspace||%
\translation{Therefore the wise praise donations of clothes. One should not give away old, torn or dirty clothes. }

  \maintext{navaṃ purāṇarahitaṃ mṛdu sūkṣmaṃ suśobhanam |}%

  \maintext{susaṃskṛtya pradātavyaṃ śraddhābhaktisamanvitam }||\thinspace7:10\thinspace||%
\translation{[Clothes] should be donated [only if they are] new, not worn, soft, delicate and beautiful, ornamented, and accompanied by willingness and devotion. }

  \maintext{śraddhāsattvaviśeṣeṇa deśakālavidhena ca |}%

  \maintext{pātradravyaviśeṣeṇa phalam āhuḥ pṛthak pṛthak }||\thinspace7:11\thinspace||%
\translation{They say that the reward [of donation/generosity] is in every case dependent on the particular [donor's] willingness and character, the choice of place and time, and on the particular recipient and material. \blankfootnote{7.11 It seems that \textit{vidhena ca} stands for \textit{vidhinā ca} or rather \textit{vidhānena} metri causa in \textit{pāda} b.
  CHECK also ŚDhU, and Florinda's article, etc.
 }}

  \maintext{yādṛśaṃ dīyate vastraṃ tādṛśaṃ prāpyate phalam |}%

  \maintext{jīrṇavastrapradānena jīrṇavastram avāpnuyāt |}%

  \maintext{śobhanaṃ dīyate vastraṃ śobhanaṃ vastram āpnuyāt }||\thinspace7:12\thinspace||%
\translation{The reward received will similar to the clothes donated. By donating old clothes, one would receive old clothes [as a reward]. By donating beautiful clothes, one would receive beautiful clothes [as a reward]. }

  \maintext{dadyād vastra suśobhanaṃ dvijavare kāle śubhe sādaram}%

 \nonanustubhindent \maintext{saubhāgyam atulaṃ labheta sa naro rūpaṃ tathā śobhanam |}%

  \maintext{tasmin yāti suvastrakoṭi śataśaḥ prāpnoti niḥsaṃśayam}%

 \nonanustubhindent \maintext{tasmāt tvaṃ kuru vastradānam asakṛt pāratrikotkarṣaṇam }||\thinspace7:13\thinspace||%
\translation{Should one bestow very beautiful clothes on a Brahmin at an auspicious time, respectfully, he [i.e. the donor] will receive unequalled happiness and a beautiful appearance. When he departs, he will be given hundreds of millions of items of nice clothes, no doubt about that. Therefore do donate clothes often. It is the way up to the other world. \blankfootnote{7.13 Note the stem form \textit{vastra} in \textit{pāda} a metri causa.
 `on a Brahmin' {\rm (}in \textit{pāda} a{\rm )}: literally, `on a person who is first among the twice-born'
  {\rm (}\textit{dvijavare}{\rm )}.
 The final syllable of \textit{saubhāgyam} in \textit{pāda} b counts as long by licence; see, e.g., 5.20 and 6.18b.
  This time the c\ae sura is not involved.
 In \textit{pāda} c, °\textit{koṭi} is treated as neuter or as a stem form {\rm (}metri causa{\rm )}.
 }}

  \subsubchptr{suvarṇadānam}%

  \trsubsubchptr{Donation of gold}%

  \maintext{suvarṇadānaṃ viprendra saṃkṣipya kathayāmy aham |}%

  \maintext{pavitraṃ maṅgalaṃ puṇyaṃ sarvapātakanāśanam }||\thinspace7:14\thinspace||%
\translation{O great Brahmin, now I shall teach you about the donation of gold in a concise manner. It is a pure, auspicious and meritorious [act] and it washes off all sins. }

  \maintext{dhārayet satataṃ vipra suvarṇakaṭakāṅgulim |}%

  \maintext{mucyate sarvapāpebhyo rāhuṇā candramā yathā }||\thinspace7:15\thinspace||%
\translation{Should one hand over [to someone] a golden bracelet or ring, O Brahmin, he will be freed of all sins, just as the moon is freed from [the demon] Rāhu [after an eclipse]. \blankfootnote{7.15 I suspect that \textit{aṅguli} is used in \textit{pāda} b in the sense of \textit{aṅgulīya} {\rm (}`finger-ring'{\rm )}.
 }}

  \maintext{dattvā suvarṇaṃ viprebhyo devebhyaś ca dvijarṣabha |}%

  \maintext{tuṭimātre 'pi yo dadyāt sarvapāpaiḥ pramucyate }||\thinspace7:16\thinspace||%
\translation{If a person donates gold to Brahmins or gods, O excellent Brahmin, even if it is only in a minute quantity, he will be freed of all sins. \blankfootnote{7.16 The form \textit{tuṭi} as a widespread variant of \textit{truṭi}, see e.g. \verify.
 }}

  \maintext{raktimāṣakakarṣaṃ vā palārdhaṃ palam eva vā |}%

  \maintext{evam eva phalaṃvṛddhir jñeyā dānaviśeṣataḥ }||\thinspace7:17\thinspace||%
\translation{[The amount can be just] one \textit{rakti}, a \textit{māṣaka}, a \textit{karṣa}, half a \textit{pala} or a \textit{pala}: this is exactly how the increase in the [size of the corresponding] reward will be, in proportion to the properties [i.e.\ amount] of the donation. \blankfootnote{7.17 I suspect that \textit{phalaṃ vṛddhir}, or \textit{phalaṃvṛddhir}, stands for 
  \textit{phalavṛddhir} {\rm (}\textit{phalasya vṛddhiḥ}{\rm )} metri causa, meaning `the increase of the reward.'
  \textit{rakti}, \textit{māṣaka}, \textit{karṣa}, and \textit{pala} are units of weight.
 }}

  \subsubchptr{bhūmidānam}%

  \trsubsubchptr{Donation of land}%

  \maintext{sarvādhāraṃ mahīdānaṃ praśaṃsanti manīṣiṇaḥ |}%

  \maintext{annavastrahiraṇyādi sarvaṃ vai bhūmisambhavam }||\thinspace7:18\thinspace||%
\translation{The wise praise the donation of land as the basis of everything [else]. Food, clothes, gold etc., all these originate in the land. }

  \maintext{bhūmidānena viprendra sarvadānaphalaṃ labhet |}%

  \maintext{bhūmidānasamaṃ vipra yady asti vada tattvataḥ }||\thinspace7:19\thinspace||%
\translation{O Brahmin, one can obtain all the rewards of donation by donating land. If there is anything that equals the donation of land, O Brahmin, you should definitely tell me. }

  \maintext{mātṛkukṣivimuktas tu dharaṇīśaraṇo bhavet |}%

  \maintext{carācarāṇāṃ sarveṣāṃ bhūmiḥ sādhāraṇā smṛtā }||\thinspace7:20\thinspace||%
\translation{[Humans] have the earth as their abode as soon as they get out of their mother's womb. Land is said to be common to all that are mobile and immobile. \blankfootnote{7.20 I take \textit{sādhāraṇā} as one word, but it is possible that the intention of the author
  was \textit{sā dhāraṇā} in two words, in fact meaning \textit{sādhāraḥ} {\rm (}\textit{sā ādhāraḥ}, `it is the basis'{\rm )}.
 }}

  \maintext{ekahastaṃ dvihastaṃ vā pañcāśac chatam eva vā |}%

  \maintext{sahasrāyutalakṣaṃ vā bhūmidānaṃ praśasyate }||\thinspace7:21\thinspace||%
\translation{Be it [only a land of] one forearm, two forearms, fifty or a hundred, a thousand, ten thousand, a hundred thousand, donations of land are held in great esteem. }

  \maintext{ekahastāṃ ca yo bhūmiṃ dadyād dvijavarāya tu |}%

  \maintext{varṣakoṭiśataṃ divyaṃ svargaloke mahīyate }||\thinspace7:22\thinspace||%
\translation{Should he donate a piece of land of [only] one forearm to a Brahmin, he will enjoy a billion divine years in heaven. }

  \maintext{evaṃ bahuṣu hasteṣu guṇāguṇi phalaṃ smṛtam |}%

  \maintext{śraddhādhikaṃ phalaṃ dānaṃ kathitaṃ te dvijottama }||\thinspace7:23\thinspace||%
\translation{Thus in case of [donating] many forearms [of land], the reward is said to be proportional to the properties [of the land]. O Brahmin, I have taught you about the rewards of donation that is made willingly. \blankfootnote{7.23 I think that \textit{guṇāguṇi}, or perhaps \textit{guṇaguṇi} {\rm (}which would be unmetrical, containing
  two \textit{laghu}s in both the second and third syllables of the \textit{pāda}{\rm )}, should refer to the idea
  that, e.g., the donation of a piece of land of 2 × 2 \textit{hasta}s would result in 
  2 or 4 × \textit{koṭiśata} years in heaven, \textit{guṇa} generally meaning `times.' 
  I take \textit{guṇā}° as referring to the size of the land donated, and °\textit{guṇi}[\textit{n}] as 
  `amounting to that many times,' but this is only a guess, 
  and it would need to be supported by some similar passage, other than 7.17 above.
 
 
  I suspect that \textit{pāda} c is an awkward attempt at saying \textit{śraddhādhikadāna{\rm (}sya{\rm )} phalaṃ}.
 }}

  \maintext{jāmadagnyena rāmeṇa bhūmiṃ dattvā dvijāya vai |}%

  \maintext{āyur akṣayam āptaṃ tu ihaiva ca dvijottama }||\thinspace7:24\thinspace||%
\translation{[Paraśu]rāma, the son of Jamadagni, having donated land to the Brahmin [Kaśyapa], obtained eternal life in this very world, O excellent Brahmin. \blankfootnote{7.24 See a summary of the corresponding episode \verify\ in the \MBH\ in 
  \mycitep{PuranicEnc}{570--571}, s.v. Paraśurāma:
  `To atone for the sin of slaughtering even
  innocent Kṣatriyas, Paraśurāma gave away all his
  riches as gifts to brahmins. He invited all the brahmins
  to Samantapañcaka and conducted a great Yāga there.
  The chief Ṛtvik {\rm (}officiating priest{\rm )} of the Yāga was
  the sage Kaśyapa and Paraśurāma gave all the lands
  he conquered till that time to Kaśyapa. Then a platform 
  of gold ten yards long and nine yards wide was
  made and Kaśyapa was installed there and worshipped.
  After the worship was over according to the instructions
  from Kaśyapa the gold platform was cut into pieces
  and the gold pieces were offered to brahmins.
 
  When Kaśyapa got all the lands from Paraśurāma he
  said thus:---``Oh Rāma, you have given me all your
  land and it is not now proper for you to live in my
  soil. You can go to the south and live somewhere on
  the shores of the ocean there.'' Paraśurāma walked
  south and requested the ocean to give him some land to
  live.'
 Note that without applying the muta cum liquida licence {\rm (}\textit{ca dvi}°{\rm )}, \textit{pāda} d would be iambic and thus
  metrically problematic.
 }}

  \subsubchptr{godānam}%

  \trsubsubchptr{Donation of cows}%

  \maintext{hemaśṛṅgāṃ raupyakhurāṃ cailaghaṇṭāṃ dvijottama |}%

  \maintext{viprāya vedaviduṣe dattvānantaphalaṃ smṛtam }||\thinspace7:25\thinspace||%
\translation{[A cow] with golden horns, silver hooves, garment and bell, O Brahmin, when given to a Veda-knowing Brahmin, [produces] rewards that are said to be endless. }

  \subsubchptr{dānapraśaṃsā}%

  \trsubsubchptr{Praise of donation}%

  \maintext{dānābhyāsarataḥ pravartanabhavāṃ śakyānurūpaṃ sadā}%

 \nonanustubhindent \maintext{annaṃ vastrahiraṇyaraupyam udakaṃ gāvas tilān medinīm |}%

  \maintext{dadyāt pādukachattrapīṭhakalaśaṃ pātrādyam anyac ca vā}%

 \nonanustubhindent \maintext{śraddhādānam abhinnarāgavadanaṃ kṛtvā mano nirmalam }||\thinspace7:26\thinspace||%
\translation{Always rejoicing in the practice of giving, \dots, as far as one's capacities go, one should give food, clothes, gold and silver, water, cows, sesamum seeds, land, sandals, parasols, seats, jars, cups or anything else. Making the [deed of] giving willingly {\rm (}\textit{śraddhādāna}{\rm )} something done with an unconditioned affection {\rm (}\textit{rāga}{\rm )} and reverence {\rm (}\textit{vadana}{\rm )}, one's mind [becomes] spotless. \blankfootnote{7.26 I am unable to interpret \textit{pravartanabhavāṃ} in \textit{pāda} a and
  I suspect that \textit{śakyānurūpaṃ} in the same \textit{pāda} stands for \textit{śaktyanurūpaṃ} metri causa.
 }}

  \maintext{dānād eva yaśaḥ śriyaḥ sukhakarāḥ khyātim atulyāṃ labhet}%

 \nonanustubhindent \maintext{dānād eva nigarhaṇaṃ ripugaṇe ānandadaṃ saukhyadam |}%

  \maintext{dānād ūrjayatā prasādam atulaṃ saubhāgya dānāl labhet}%

 \nonanustubhindent \maintext{dānād eva anantabhoga niyataṃ svargaṃ ca tasmād bhavet }||\thinspace7:27\thinspace||%
\translation{Glory and fortune that makes us happy come about only by donations, and one can gain unequalled fame. Only from donations will reproach [exercised by] the enemy [turn into] pleasure and happiness. Vigour and unequalled graciousness come from donation. One can reach happiness thought donations. Endless enjoyments surely come only from donations, and heaven is [reached] also because of it. \blankfootnote{7.27 I suspect that \textit{khyātiś ca tulyaṃ} in the MSS stands for \textit{khyātim atulyāṃ} {\rm (}`and unequalled fame'{\rm )} and
  that it is not a clumsy attempt to restore the metre, but rather a later correction gone wrong.
  I have emended the phrase believing that the second {\rm (}last{\rm )} syllable of \textit{khyātim} may be treated as \textit{guru}.
  See the same licence applied in non-\textit{anuṣṭubh} verses above,
  e.g., in 5.20a, 6.18b, 7.13b {\rm (}just before \textit{atula}{\rm )}.
 I doubt if \Ed's reading in \textit{pāda} c, \textit{durjayatā} {\rm (}`invincibility'{\rm )} were better than \textit{ūrjayatā} transmitted in
  all the MSS consulted. While \textit{ūrjayatā} is still problematic, it is not inconceivable that it
  stands for \textit{ūrjatā} meaning most probably `being powerful, strength, vigour.' Also, note here
  the stem form noun \textit{saubhāgya} metri causa.
 Note \textit{svargaṃ} as a neuter noun, and the stem form °\textit{bhoga} metri causa in \textit{pāda} d. 
  The lack of sandhi between \textit{eva} and \textit{ananta}° helps restore the metre.
 }}

  \maintext{dānād eva ca śakralokasakalaṃ dānāj janānandanam}%

 \nonanustubhindent \maintext{dānād eva mahīṃ samasta bubhuje samrāḍ mahīmaṇḍale |}%

  \maintext{dānād eva surūpayonisubhagaś candrānano vīkṣyate}%

 \nonanustubhindent \maintext{dānād eva anekasambhavasukhaṃ prāpnoti niḥsaṃśayam }||\thinspace7:28\thinspace||%
\translation{The whole world of Śakra [i.e. Indra can be taken as one's possession] by donations only. Donations make people happy. Supreme ruler[s] enjoyed all the land in the world only because of donations. Skanda {\rm (}\textit{candrānana}{\rm )} appears as handsome and fortunate, with a [good] family[? \verify] only because of donations. One can reach happiness that lasts countless births only through donations, there is no doubt about that. \blankfootnote{7.28 °\textit{lokasakalaṃ} in \textit{pāda} a is suspect and \Ed's silent emendation {\rm (}°\textit{lokam atulaṃ}{\rm )} is
  not without reason.
 I translate \textit{pāda} b as a general statement although \textit{samrāṭ} may 
  refer to a specific figure and story in mythology. The perfect form \textit{bubhuje}, and 
  the next \textit{pāda}, at least point to this direction.
 }}
\center{\maintext{\dbldanda\thinspace iti vṛṣasārasaṃgrahe dānapraśaṃsādhyāyaḥ saptamaḥ\thinspace\dbldanda}}
\translation{Here ends the seventh chapter in the \textit{Vṛṣasārasaṃgraha} called Praise of Donations.}

  \chptr{aṣṭamo 'dhyāyaḥ}
\addcontentsline{toc}{subsection}{Chapter 8}
\fancyhead[CO]{{\footnotesize\textit{Translation of chapter 8}}}%

  \trchptr{ Chapter Eight}%

  \subchptr{niyameṣu svādhyāyaḥ {\rm {\rm (}5{\rm )}}}%

  \trsubchptr{The fifth Niyama-rule: Study}%

  \maintext{pañcasvādhyāyanaṃ kāryam ihāmutra sukhārthinā |}%

  \maintext{śaivaṃ sāṃkhyaṃ purāṇaṃ ca smārtaṃ bhāratasaṃhitām }||\thinspace8:1\thinspace||%
\translation{Five kinds of study are to be pursued by those who wish to be happy in this life and in the other: [one has to study the] Śaiva [teachings], Sāṃkhya [philosophy], the Purāṇa[s], the Smārta [tradition] and the \textit{Bhāratasaṃhitā} [i.e. the \textit{Mahābhārata}]. \blankfootnote{8.1 Note the accusative ending of \textit{°saṃhitām} after a list consisting of words probably in the
  nominative. One may correct it to \textit{°saṃhitā} or rather 
  supply an active verb such as \textit{adhigacchet} {\rm (}`he should study'{\rm )}.
 }}

  \maintext{śaivatattvaṃ vicinteta śaivapāśupatadvaye |}%

  \maintext{atra vistarataḥ proktaṃ tattvasārasamuccayam }||\thinspace8:2\thinspace||%
\translation{He should reflect on the Śaiva truth in both Śaiva and Pāśupata [teachings]. In those teachings the whole essence of truth is taught extensively. \blankfootnote{8.2 Note that \textit{śaivatattvaṃ} in \textit{pāda} a is the result of a conjecture and that the reading \textit{śaivapāśupatadvaye} 
  in \textit{pāda} b is based on one single manuscript {\rm (}\msP{\rm )}. In spite of these uncertainties, 
  I think that this form of the current half-verse is the only one that yields the appropriate meaning.
 }}

  \maintext{saṃkhyātattvaṃ tu sāṃkhyeṣu boddhavyaṃ tattvacintakaiḥ |}%

  \maintext{pañcatattvavibhāgena kīrtitāni maharṣibhiḥ }||\thinspace8:3\thinspace||%
\translation{Those who reflect on the truth {\rm (}\textit{tattva}{\rm )} can grasp the truth of enumeration [of ontological principles/reality levels] {\rm (}\textit{saṃkhyātattva}{\rm )} from Sāṃkhya [texts]. The great sages taught [those twenty-five] \textit{tattva}s [of Sāṃkhya] as being in groups of five. \blankfootnote{8.3 In \textit{pāda} d, \textit{kīrtitāni} picks up an implied \textit{tattvāni}.
 }}

  \maintext{purāṇeṣu mahīkoṣo vistareṇa prakīrtitaḥ |}%

  \maintext{adhordhvamadhyatiryaṃ ca yatnataḥ sampraveśayet }||\thinspace8:4\thinspace||%
\translation{In the Purāṇas it is the sheath[s] of the world that are described extensively. One can definitely enter [the realm] of the lower [world, i.e. hell], the upper [world, i.e. heaven], and middle [world, i.e. the human world], and the horizontal [world, i.e. of animals, by studying the Purāṇas]. \blankfootnote{8.4 Note that \textit{tirya} seems to be an acceptable nominal stem in this text for \textit{tiryañc}. 
  I understand the causative form \textit{sampraveśayet} as non-causative, and 
  I interpret °\textit{madhya}° as the `human world' tentatively. \Ed's silent emendation
  to \textit{samprabodhayet} is understandable since to `enter' these worlds 
  {\rm (}especially the hells and the human world{\rm )} through the study of the Purāṇas makes little sense,
  at least when taken literally.
 }}

  \maintext{smārtaṃ varṇāśramācāraṃ dharmanyāyapravartanam |}%

  \maintext{śiṣṭācāro 'vikalpena grāhyas tatra aśaṅkitaḥ }||\thinspace8:5\thinspace||%
\translation{The Smārta [tradition] deals with the conduct of the social classes {\rm (}\textit{varṇa}{\rm )} and disciplines {\rm (}\textit{āśrama}{\rm )}, and with the procedures of Dharma and lawsuits. Good conduct is to be gathered from that [source] without hesitation, with certainty. \blankfootnote{8.5 Compare \textit{pāda} a with 3.15c.
 }}

  \maintext{itihāsam adhīyānaḥ sarvajñaḥ sa naro bhavet |}%

  \maintext{dharmārthakāmamokṣeṣu saṃśayas tena chidyate }||\thinspace8:6\thinspace||%
\translation{A man who studies the epics {\rm (}\textit{itihāsa}{\rm )} will become omniscient. [All his] doubts about Dharma, Artha, Kāma and Mokṣa will be eliminated. }

  \subchptr{niyameṣv upasthanigrahaḥ {\rm {\rm (}6{\rm )}}}%

  \trsubchptr{The sixth Niyama-rule: Sexual restraint}%

  \maintext{śṛṇuṣvāvahito vipra pañcopasthavinigraham |}%

  \maintext{striyo vā garhitotsargaḥ svayaṃmuktiś ca kīrtyate |}%

  \maintext{svapnopaghātaṃ viprendra divāsvapnaṃ ca pañcamaḥ }||\thinspace8:7\thinspace||%
\translation{Listen with great attention, O Brahmin, to the five [spheres of] sexual restraint. Women, forbidden ejaculation, and masturbation are mentioned [in this context, as well as] offence while sleeping, O Brahmin, and sleeping by day as the fifth. }

  \subsubchptr{striyaḥ}%

  \trsubsubchptr{Women}%

  \maintext{agamyā strī divā parve dharmapatny api vā bhavet |}%

  \maintext{viruddhastrīṃ na seveta varṇabhraṣṭādhikāsu ca }||\thinspace8:8\thinspace||%
\translation{A woman is not to be approached sexually in daytime and on the four days of the changes of the Moon {\rm (}\textit{parvan}{\rm )}, even if she is one's lawful wife. One should not have sex with a woman who is taboo or with one of those who have lost their class {\rm (}\textit{varṇa}{\rm )} or are [of a] superior [\textit{varṇa} than oneself]. \blankfootnote{8.8 Understand \textit{parve} as \textit{parvani} {\rm (}thematisation of the stem in \textit{-an}{\rm )}.
 The nominative °\textit{strī} in \textit{pāda} c, now corrected to the accusative, may 
  be the result of an eyeskip to \textit{strī} in \textit{pāda} a.
 }}

  \subsubchptr{garhitotsargaḥ}%

  \trsubsubchptr{Forbidden ejaculation}%

  \maintext{ajameṣagavādīnāṃ vaḍavāmahiṣīṣu ca |}%

  \maintext{garhitotsargam ity etad yatnena parivarjayet }||\thinspace8:9\thinspace||%
\translation{Intercourse with goats, sheep, cows, mares, buffalo-cows is called forbidden ejaculation, which is to be avoided at all cost. \blankfootnote{8.9 Understand \textit{°ādīnāṃ} in \textit{pāda} a as standing for the locative case.
 Understand \textit{°sargam} as neuter nominative {\rm (}instead of \textit{°sargaḥ}{\rm )} or alternatively
  understand \textit{pāda} c with a hiatus bridge: \textit{garhitotsarga-m-ity etad}.
 }}

  \subsubchptr{svayaṃmuktiḥ}%

  \trsubsubchptr{Masturbation}%

  \maintext{ayonyakaṣaṇā vāpi apānakaṣaṇāpi vā |}%

  \maintext{svayaṃmuktir iyaṃ jñeyā tasmāt tāṃ parivarjayet }||\thinspace8:10\thinspace||%
\translation{Rubbing himself against something else than a female sexual organ or rubbing his anus, are called masturbation, therefore these are to be avoided. \blankfootnote{8.10 The conjecture that changes \textit{anyonya°} to \textit{ayonya°} in \textit{pāda} a involves 
  minimal intervention and makes the sentence much more meaningful than the 
  version transmitted. Also consider \textit{ayoni°}.
 The variant \textit{strī} for \textit{tāṃ} in \textit{pāda} d in the \Ed\ may be one example of the numerous
  silent intervention made by Naraharināth in his edition.
 }}

  \subsubchptr{svapnaghātam}%

  \trsubsubchptr{Offence while sleeping}%

  \maintext{svapnaghātaṃ dvijaśreṣṭha aniṣṭaṃ paṇḍitaiḥ sadā |}%

  \maintext{svapne strīṣu ramante ca retaḥ prakṣarate tataḥ }||\thinspace8:11\thinspace||%
\translation{Offence while sleeping, O best of Brahmins, has always been [considered] undesirable by the learned. [If] one enjoys women while sleeping, his semen will issue. }

  \subsubchptr{divāsvapnam}%

  \trsubsubchptr{Sleeping by day}%

  \maintext{divāśayaṃ na kartavyaṃ nityaṃ dharmapareṇa tu | }%

  \maintext{svargamārgārgalā hy etāḥ striyo nāma prakīrtitāḥ }||\thinspace8:12\thinspace||%
\translation{Sleeping by day should always be avoided by those who are intent on Dharma. These women are called `the bolts [that block the gate to] the path to heaven.' \blankfootnote{8.12 It is not crystal clear why `sleeping by day' should count as
  one of the offences against sexual restraint. Even if we translated \textit{divāsvapna} and
  \textit{divāśaya} as `daydreaming,' this category would stil seem out of context.
 \textit{Pāda}s cd are clumsy and out of context. They would fit verse 8.8 better.
 }}

  \subchptr{niyameṣu vratapañcakam {\rm {\rm (}7{\rm )}}}%

  \trsubchptr{The seventh Niyama-rule: religious observances}%

  \maintext{mārjārakabakaśvānagomahīvratapañcakam |}%

  \subsubchptr{mārjārakavratam}%

  \trsubsubchptr{The Cat Vow}%

  \maintext{svaviṣṭhamūtraṃ bhūmīṣu chādayed dvijasattama |}%

  \maintext{sūryasomānumodanti mārjāravratikeṣu ca }||\thinspace8:13\thinspace||%
\translation{[Hear about] the five religious observances [called] the cat, the crane, the dog, the cow, and the earth. He buries his own urine and f\ae ces in the ground, O truest Brahmin. He rejoices [seeing] the sun and the moon when performing the cat observance. \blankfootnote{8.13 Note \textit{°viṣṭha°} for \textit{viṣṭhā} metri causa in \textit{pāda} c {\rm (}\textit{ma-vipulā}{\rm )}.
  Alternatively, read \textit{svaviṣṭhāmūtra bhūmīṣu} {\rm (}\textit{pathyā}{\rm )}.
 Note the stem form \textit{sūryasoma} for \textit{sūryasomau} in \textit{pāda} e. 
  It is not entirely clear why cats would rejoice seeing the Sun and the Moon.
  Perhaps this remark refers to the fact that cats can be active both
  in the daytime and at night.
 }}

  \subsubchptr{bakavratam}%

  \trsubsubchptr{The Crane Vow}%

  \maintext{bakavac cendriyagrāmaṃ suniyamya tapodhana |}%

  \maintext{sādhayec ca manastuṣṭiṃ mokṣasādhanatatparaḥ }||\thinspace8:14\thinspace||%
\translation{O great ascetic, one should suppress all his senses like a crane, and should cultivate the peace of the mind, focusing on achieving liberation. \blankfootnote{8.14 Cranes are compared to ascetics here probably because of the similarity of
  their posture when relaxing standing on one leg to ascetics performing penance 
  standing on one leg {\rm (}such as the ascetic, and a cat, depicted on the famous relief in Mahabalipuram{\rm )}. 
 }}

  \subsubchptr{śvānavratam}%

  \trsubsubchptr{The Dog Vow}%

  \maintext{mūtraviṣṭhe na bhūmīṣu kurute śvānadaḥ sadā |}%

  \maintext{tuṣyate bhagavān śarvaḥ śvānavratacaro yadi }||\thinspace8:15\thinspace||%
\translation{He does not bury his urine and f\ae ces in the ground, and he barks constantly. Lord Śarva [i.e. Śiva] is satisfied when one practises the dog observance. \blankfootnote{8.15 A possible expanation for Śiva being satisfied with an ascetic practising 
  this observance is that Śiva's Bhairava form often has a dog as his mount. See, e.g.,
  \mycitep{BakkerWorld2014}{232--233} on a 5-6th-century image of Bhairava and a dog carved in
  rock at Muṇḍeśvarı̄ Hill not far from Vārāṇasī, and Mirnig 2013, 334 ?\verify 
  This observance has ancient roots. Its practitioner, the \textit{kukkuravatika}
  appears in \textit{Majjhimanikāya} 2.1.7, in the \textit{Kukkuravatiyasutta}, 
  alongside with a practitioner of the \textit{govrata} {\rm (}\textit{govatika}{\rm )}, an observance
  that comes up in the next verse in the \VSS:
  \textit{evaṃ me sutaṃ. ekaṃ samayaṃ bhagavā koliyesu viharati haliddavasanaṃ nāma koliyānaṃ nigamo.
  atha kho puṇṇo ca koliyaputto govatiko, acelo ca seniyo kukkuravatiko yena bhagavā tenupasaṅkamiṃsu...}
  See \mycitep{AcharyaBull}{127--128}. Acharya 
  summarises the \textit{Kukkuravatiyasutta} thus:
  `The \textit{Kukkuravatiyasutta} from the \textit{Majjhimanikāya} {\rm (}II.1.7{\rm )} 
  presents a \textit{govatika} together with a \textit{kukkuravatika}. They are observing 
  their vows, and have adopted the behaviour of a bull and a dog respectively. 
  The Buddha tells them that as they are cultivating bullness and dogness, 
  the state of mind of these animals, they will go to hell or become reborn as animal.
  They are alarmed at this and take refuge in the Buddha.'
 }}

  \subsubchptr{govratam}%

  \trsubsubchptr{The Cow Vow}%

  \maintext{mūtravarco na rudhyeta sadā govratiko naraḥ |}%

  \maintext{bhīmas tuṣṭikaraś caiva purāṇeṣu nigadyate }||\thinspace8:16\thinspace||%
\translation{A person practising the Cow Vow should never hold back his urine and f\ae ces. This is a terrifying [observance] that gives satisfaction, [as] stated in the Purāṇas. \blankfootnote{8.16 I prefer reading \textit{bhīma} and \textit{tuṣṭi°} as two separate words, the first
  one either in stem form {\rm (}\msCa\msCb\msNa\msNc\msP{\rm )} or as \textit{bhīmas} {\rm (}\msCc\msNb\Ed{\rm )}
  or \textit{bhīmaṃ} {\rm (}\eme{\rm )},
  to reading these two words as a compound because
  of the following \textit{caiva}.
  I suspect that both \textit{bhīma} and \textit{tuṣṭikara} refer to the \textit{vrata}, rather than its practitioner,
  but I have not emended \textit{bhīmas tuṣṭikaraś} to \textit{bhīmaṃ tuṣṭikaraṃ}
  because \textit{vrata} appears as a masculine noun, e.g., in 8.17d below.
 
  \mycite{AcharyaBull} gives a number of significant clues about the origins 
  of this observance. After exploring its links to Pāśupatas, \mycitep{AcharyaBull}{116--118},
  quotes \textit{Jaiminīyabrāhmaṇa} 2.113, which contains the phrase 
  \textit{yatra yatrainaṃ viṣṭhā vindet tat tad vitiṣṭheta}, in Acharya's translation:
  `Wherever he feels the urge to evacuate f\ae ces, right there he should evacuate.'
  This is an instruction in a Vedic text that is close to what the \VSS\ teaches above.
  Incidentaly, the \textit{Jaiminīyabrāhmaṇa} adds:
  \textit{tena haitenottaravayasy e} [\textit{va}] \textit{yajeta}
  {\rm (}translated in \mycitep{AcharyaBull}{118} as: 
  `One should perform this [sacrifice] in the final years of one's life'{\rm )}.
 }}

  \subsubchptr{mahīvratam}%

  \trsubsubchptr{The Earth Vow}%

  \maintext{kuddālair dārayanto 'pi kīlakoṭiśataiś citaḥ |}%

  \maintext{kṣamate pṛthivī devī evam eva mahīvrataḥ }||\thinspace8:17\thinspace||%
\translation{Splitting [the earth] with spades and laid on hundreds of pointed wedges: Goddess Earth bears [this] patiently. This is exactly how one can practise the earth vow. \blankfootnote{8.17 While \textit{dārayanto} as an active participle in the masculine nominative is acceptable
  as an irregular form, the precise interpretation of \textit{pāda}s a and b is still problematic therefore
  my translation of this verse is tentative and the description seems too condensed to be
  intelligible. Kengo Harimoto suggested that \msCc\ and \Ed\ might be transmitting
  the correct reading, and then the reference would be to soil 
  piled up by millions of insects {\rm (}\textit{kīṭakoṭi}°{\rm )}, instead of points of
  wedges {\rm (}\textit{kīlakoṭi}°{\rm )}. Nevertheless, now I think that the reference point could
  be Bhīṣma's dying scene in the \MBH, where the great warrior is lying on a bed of hundreds of arrows: 
  \textit{sa śete śaratalpastho medinīm aspṛśaṃs tadā}: `Then he lay there on his bed of arrows,
  without touching the ground' {\rm (}\MBH\ 6.115.8ab{\rm )}. The word \textit{cita} is used in the same context in 
  \MBH\ 12.47.4ab: \textit{vikīrṇāṃśur ivādityo bhīṣmaḥ śaraśataiś citaḥ}: 
  `Bhīṣma, laid on a hundred arrows, was like the Sun with its scattered rays of light.'
  If this interpretation of \VSS\ 8.17 is correct, the observance described here may require
  one to dig the ground, install wedges, and lie on them, in the manner of fakirs.
  The reference to the Earth in \textit{pāda} c may have been inspired by lines such as \MBH\
  6.115.11cd: \textit{rarāsa pṛthivī caiva bhīṣme śāṃtanave hate}:
  `The Earth cried out when Bhīṣma, the son of Śaṃtanu, was killed.'
 
  In \BHAVP\ 4.121, called `The Description of eighty-five observances' {\rm (}\textit{vratapañcāśītivarṇana}{\rm )},
  we find this on \textit{mahīvrata}:
  \textit{dadyāt triṃśatpalād ūrdhvaṃ mahīṃ kṛtvā tu kāṃcanīm\thinspace |
  kulācalādrisahitāṃ tilavastrasamanvitām\thinspace || 152\thinspace || 
  tiladroṇopari gatāṃ brāhmaṇāya kuṭuṃbine\thinspace | 
  dinaṃ payovratas tiṣṭhed rudraloke mahīyate\thinspace || 153\thinspace || 
  etan mahīvrataṃ proktaṃ saptakalpānuvartakam\thinspace |}.
 
  A tentative translation of this passage would go as follows: `One should donate a golden [model of] Earth
  that weighs more than thirty \textit{pala}s {\rm (}appr. one kilogram{\rm )}, showing the chief mountain-ranges,
  together with [donations of] sesamum seeds and clothes, the sesamum seeds [weighing] more than
  a \textit{droṇa} {\rm (}appr. ten kilograms{\rm )}, to a householder Brāhmin. One should keep the milk-observance 
  [i.e. subsisting on nothing but milk] for one day, and one will have fun in Rudraloka.
  This is called the Earth Observance whose range is seven \ae ons.' {\rm (}I take the values for weights
  from \mycitep{OlivelleManu}{997}.{\rm )} 
  The descriptions of the \textit{dharāvrata} and the \textit{śubhadvādaśī} observance in 
  \mycitep{KaneHistory}{v. 5, 321 and 429} are similar.
  The \VSS's \textit{mahīvrata} seems different, and more in line with 
  the somewhat transgressive and wild, perhaps Pāśupata-oriented, nature of the
  four preceding observances.
 }}

  \maintext{vratapañcakam ity etad yaś careta jitendriyaḥ |}%

  \maintext{sa cottamam idaṃ lokaṃ prāpnoti na ca saṃśayaḥ }||\thinspace8:18\thinspace||%
\translation{He who practises these five religious observances with his senses subdued will, without doubt, reach this superior world [i.e. heaven?]. \blankfootnote{8.18 Note the neuter \textit{idaṃ} picking up the normally masculine \textit{lokaṃ} in \textit{pāda} c,
  and that the same \textit{idaṃ} would make more sense if the interlocutor were a deity, e.g.,
  Śiva, referring to his abode, and not Anarthayajña, the ascetic.
 }}

  \subchptr{niyameṣv upavāsaḥ {\rm {\rm (}8{\rm )}}}%

  \trsubchptr{The eighth Niyama-rule: Eating restrictions}%

  \maintext{śeṣānnam antarānnaṃ ca naktāyācitam eva ca |}%

  \maintext{upavāsaṃ ca pañcaitat kathayiṣyāmi tac chṛṇu }||\thinspace8:19\thinspace||%
\translation{Eating leftovers, [not] eating in-between [breakfast and dinner], eating [only] at night, eating food obtained without solicitation, and fasting: listen, I shall teach you these five. \blankfootnote{8.19 Note how this category of \textit{niyama}-rules was called \textit{upavāsa} {\rm (}`fasting'{\rm )} in 5.3c above but how in fact
  \textit{upavāsa} is just the fifth subcategory withing this group of eating restrictions.
 }}

  \subsubchptr{śeṣānnam}%

  \trsubsubchptr{Eating leftovers}%

  \maintext{vaiśvadevātithiśeṣaṃ pitṛśeṣaṃ ca yad bhavet |}%

  \maintext{bhṛtyaputrakalatrebhyaḥ śeṣāśī vighasāśanaḥ }||\thinspace8:20\thinspace||%
\translation{[He who eats] the leftovers belonging to all the gods, to guests, and to the ancestors, he who eats the leftovers {\rm (}śeṣāśin{\rm )} of servants, sons and wives, is [called in general] the one who consumes the remains of food {\rm (}\textit{vighasāśana}{\rm )}. \blankfootnote{8.20 \textit{Pāda} a is a \textit{sa-vipulā}.
 }}

  \subsubchptr{antarānnam}%

  \trsubsubchptr{{\rm [}Not{\rm ]} eating in-between breakfast and dinner}%

  \maintext{antarā prātarāśī ca sāyamāśī tathaiva ca |}%

  \maintext{sadopavāsī bhavati yo na bhuṅkte kadācana }||\thinspace8:21\thinspace||%
\translation{He will be regarded as one that is always fasting if he never eats between breakfast and dinner. \blankfootnote{8.21 My translation here follows the parallel verse in the \MBH\ and 
  is based on that of Kisari Mohan Ganguli {\rm (}\mycite{GanguliMBh}{\rm )}. 
  The syntax of the version here in the \VSS\ is less
  smooth than that in the \MBH, and the \VSS's reading \textit{prāntarāśī} 
  definitely required an emendation.
 }}

  \subsubchptr{naktānnam}%

  \trsubsubchptr{Eating {\rm [}only{\rm ]} at night}%

  \maintext{na divā bhojanaṃ kāryaṃ rātrau naiva ca bhojayet |}%

  \maintext{naktavele ca bhoktavyaṃ naktadharmaṃ samīhatā }||\thinspace8:22\thinspace||%
\translation{One should eat neither in the daytime nor in the evening, and should eat [only] at midnight if he wishes to follow the practice of [eating only at] night {\rm (}\textit{naktadharma}{\rm )}. \blankfootnote{8.22 Note \textit{°vele} for \textit{°velāyāṃ} in \textit{pāda} c.
 }}

  \subsubchptr{ayācitānnam}%

  \trsubsubchptr{Eating food obtained without solicitation}%

  \maintext{anārambhya ya āhāraṃ kuryān nityam ayācitam |}%

  \maintext{parair dattaṃ tu yo bhuṅkte tam ayācitam ucyate }||\thinspace8:23\thinspace||%
\translation{He who consumes food only without initiating [the donation], without asking for it, and eats [only] that which has been given by others is called [one who eats] unsolicited [food]. \blankfootnote{8.23 \textit{anārambhasya} {\rm (}`of someone who has not yet started/initiated'{\rm )} in \textit{pāda} a seems suspect, hence
  my conjecture {\rm (}\textit{anārambhya ya}{\rm )} that involves mininal intervention and yields better sense.
  I take \textit{ayācitam} in \textit{pāda} b adverbially.
 }}

  \subsubchptr{upavāsaḥ}%

  \trsubsubchptr{Fasting}%

  \maintext{bhakṣyaṃ bhojyaṃ ca lehyaṃ ca coṣyaṃ peyaṃ ca pañcamam |}%

  \maintext{na kāṅkṣen nopayuñjīta upavāsaḥ sa ucyate }||\thinspace8:24\thinspace||%
\translation{Chewable and unchewable food, food to be sipped or sucked or drunk, as the fifth [category]: if one does not long for and does not consume [any of the above], that is called fasting {\rm (}\textit{upavāsa}{\rm )}. \blankfootnote{8.24 For a detailed discussion of the categories \textit{bhakṣya, bhojya, lehya} and \textit{coṣya},
  see \mycitep{KafleNisvasaBook}{245, n. 534}. 
  See also \SDHU\ 8.13: %
  \textit{bhakṣyaṃ bhojyaṃ ca peyaṃ ca lehyaṃ coṣyaṃ ca picchilam}\thinspace |
  \textit{iti bhedāḥ ṣaḍannasya madhurādyāś ca ṣaḍguṇāḥ}\thinspace ||
 }}

  \subchptr{niyameṣu maunavratam {\rm {\rm (}9{\rm )}}}%

  \trsubchptr{The ninth Niyama-rule: Silence}%

  \maintext{mithyāpiśunapāruṣyatīkṣṇavāg apralāpanam |}%

  \maintext{maunapañcakam ity etad dhārayen niyatavrataḥ }||\thinspace8:25\thinspace||%
\translation{One who is disciplined in religious observances should observe silence in [i.e. should avoid] these five: deceitful speech, envious speech, insult, harsh speech and bragging. \blankfootnote{8.25 \textit{pāruṣya} seems to be the good reading in \textit{pāda} a, as opposed
  to \msCc's \textit{saṃbhinnā}, because in the following 
  a short section on the category of \textit{pāruṣya} is coming up {\rm (}in 8.28{\rm )}.
  As far as the readings \textit{spṛṣṭavāg} and \textit{pṛṣṭavāg} are concerned, I suppose 
  \textit{pṛṣṭavāg} is not inconceivable {\rm (}as suggested by Judit Törzsök{\rm )}, 
  for in 8.29 it is, in a way, questions that are given as relevant examples. 
  Nevertheless I conjectured \textit{tīkṣṇavāg} here, relying on the same verse, 8.29.
 }}

  \subsubchptr{mithyāvacanam}%

  \trsubsubchptr{Deceitful speech}%

  \maintext{asambhūtam adṛṣṭaṃ ca dharmāc cāpi bahiṣkṛtam |}%

  \maintext{anarthāpriyavākyaṃ yat tan mithyāvacanaṃ smṛtam }||\thinspace8:26\thinspace||%
\translation{Fictitious [speech], [speech about] unknown [things], [speech about things] outside the range of Dharma, meaningless and unfriendly speech: these are called deceitful speech. }

  \subsubchptr{piśunaḥ}%

  \trsubsubchptr{Envy}%

  \maintext{paraśrīṃ nābhinandanti parasyaiśvaryam eva ca |}%

  \maintext{aniṣṭadarśanākāṅkṣī piśunaḥ samudāhṛtaḥ }||\thinspace8:27\thinspace||%
\translation{One who does not rejoice in others' fortune or in others' power, one who would like to see something disadvantageous [for others] is called envious. }

  \subsubchptr{pāruṣyam}%

  \trsubsubchptr{Insult}%

  \maintext{mṛtā mātā pitā caiva hānisthānaṃ kathaṃ bhavet |}%

  \maintext{bhuṅkṣva kāmam amṛṣṭānāṃ pāruṣyaṃ samudāhṛtam }||\thinspace8:28\thinspace||%
\translation{`[Your] mother and father are dead. How can this be a condition for deficit? Enjoy the love of unclean women!' [These are] called insult. \blankfootnote{8.28 My translation of \textit{pāda} b, or rather of the whole verse, is tentative, and to make
  sense of \textit{pāda} a, I have chosen a reading {\rm (}\textit{mṛtā}{\rm )} that is not well attested.
  I am not at all certain that I understand what these abusive words imply.
 }}

  \subsubchptr{tīkṣṇavāk}%

  \trsubsubchptr{Verbal abuse}%

  \maintext{hṛdi na sphuṭase mūḍha śiro vā na vidāryase |}%

  \maintext{evamādīny anekāni tīkṣṇavādī sa ucyate }||\thinspace8:29\thinspace||%
\translation{`Won't you burst in your heart, stupid? [Why] don't you break your head?' [If one utters] these or similar [curses], he is said to be using verbal abuse. }

  \subsubchptr{asatpralāpaḥ}%

  \trsubsubchptr{Bragging}%

  \maintext{dyūtabhojanayuddhaṃ ca madyastrīkatham eva ca |}%

  \maintext{asatpralāpaḥ pañcaitat kīrtitaṃ me dvijottama }||\thinspace8:30\thinspace||%
\translation{Relating fancy stories about gambling, enjoyments, fights, drinking and women are the five types of bragging. [Thus] have I taught [reasons for observing silence], O excellent Brahmin. \blankfootnote{8.30 I take \textit{°katham} in \textit{pāda} b as an alternative nominative form of \textit{°kathā} metri causa and as 
  belonging to all the categories here thus: \textit{dyūtakathā, bhojanakathā, yuddhakathā, madyakathā,
  strīkathā}.
 Note the use of the singular next to a number in \textit{pāda} c and understand \textit{me} in \textit{pāda} d as \textit{mayā}. 
  The latter usage appears in the epics, see \mycitep{OberliesEpicSkt}{102--103 {\rm (}4.1.3{\rm )}}.
 }}

  \maintext{maunam eva sadā kāryaṃ vākyasaubhāgyam icchatā |}%

  \maintext{apāruṣyam asambhinnaṃ vākyaṃ satyam udīrayet }||\thinspace8:31\thinspace||%
\translation{Those who long for speech eloquent speech should always observe silence. One should speak true words without insult and idle talk. }

  \maintext{yas tu maunasya no kartā dūṣitaḥ sa kulādhamaḥ |}%

  \maintext{janme janme ca durgandho mūkaś caivopajāyate }||\thinspace8:32\thinspace||%
\translation{He who does not observe silence is defiled and he is the black sheep of the family. For a number of rebirths, [his mouth] will stink and he will become mute. \blankfootnote{8.32 The form \textit{janme} for \textit{janmani} often occurs in Śaiva tantras as a tipically Aiśa phenomenon.
  See, e.g., \NISVNAYA\ 1.86a {\rm (}\textit{janme janme vimūḍhātmā}, see \mycitep{NisvasaGoodall}{114 and 191}{\rm )}
  and \BRAYA\ 45.8b, 452a, 559a {\rm (}the last reads \textit{janme janme tu yā jātiṃ}, 
  see \mycitep{KissBraYa}{83 and 128ff}{\rm )}.
  Thematisation of stems in \textit{-an} occurs in the epics, see 
  \mycitep{OberliesEpicSkt}{88 {\rm (}3.10{\rm )}}.
 }}

  \maintext{tasmān maunavrataṃ sadaiva sudṛḍhaṃ kurvīta yo niścitaṃ}%

 \nonanustubhindent \maintext{vācā tasya alaṅghyatā ca bhavati sarvāṃ sabhāṃ nandati |}%

  \maintext{vaktrāc cotpalagandham asya satataṃ vāyanti gandhotkaṭāḥ}%

 \nonanustubhindent \maintext{śāstrānekasahasraśo giri naraḥ proccāryate nirmalam }||\thinspace8:33\thinspace||%
\translation{Therefore the speech of a person who always observes silence firmly, with resolution, will be impossible to ignore and it will make the community rejoice. The fragrance of lotuses and [other kinds of] rich fragrances will blow from his mouth. Thousands of faultless \textit{śāstra}s will be declared in the words of this person. \blankfootnote{8.33 To make sense of \textit{pāda} d, we are forced to take \textit{śāstra} as a stem form noun and 
  \textit{naraḥ} as a {\rm (}regular{\rm )} genitive from \textit{nṛ}. {\rm (}I thank Judit Törzsök for this interpretation.{\rm )}
  Another way of understanding the beginning of this sentence would be to separate \textit{śāstrāneka°} as
  \textit{śāstrān eka°}, treating the word \textit{śāstra} as masculine.
 }}

  \subchptr{niyameṣu snānam {\rm {\rm (}10{\rm )}}}%

  \trsubchptr{The tenth Niyama-rule: Bathing}%

  \maintext{snānaṃ pañcavidhaṃ caiva pravakṣyāmi yathātatham |}%

  \maintext{āgneyaṃ vāruṇaṃ brāhmyaṃ vāyavyaṃ divyam eva ca }||\thinspace8:34\thinspace||%
\translation{I shall teach you the five kinds of bathing as they really are: fire bath, water bath, Vedic bath, wind bath and divine bath. }

  \subsubchptr{āgneyaṃ snānam}%

  \trsubsubchptr{Fire bath}%

  \maintext{āgneyaṃ bhasmanā snānaṃ toyāc chataguṇaṃ phalam |}%

  \maintext{bhasmapūtaṃ pavitraṃ ca bhasma pāpapraṇāśanam }||\thinspace8:35\thinspace||%
\translation{Fire bath is [performed] with ashes. Its fruits are a hundred times bigger than [those of] a water [bath]. [Things] purified with ashes are holy. Ashes destroy sin. }

  \maintext{tasmād bhasma prayuñjīta dehināṃ tu malāpaham |}%

  \maintext{sarvaśāntikaraṃ bhasma bhasma rakṣakam uttamam }||\thinspace8:36\thinspace||%
\translation{Therefore one should use ash for it purifies humans of their defilement. Ashes yield appeasement for everyone. Ash is the ultimate protector. }

  \maintext{bhasmanā tryāyuṣaṃ kṛtvā brahmacaryavrate sthitam |}%

  \maintext{bhasmanā ṛṣayaḥ sarve pavitrīkṛtam ātmanaḥ }||\thinspace8:37\thinspace||%
\translation{Drawing [the sectarian marks on their foreheads while reciting] the Tryāyuṣa [mantra], observing chastity, all the sages purified themselves with ashes. \blankfootnote{8.37 Note \textit{tryāyuṣa} in the sense of the three \textit{puṇḍra}-lines on the
  forehead and compare with 11.28c. Understand \textit{sthitam} as 
  \textit{sthitaḥ} or rather \textit{sthitāḥ} if we are to connect this line
  to the next {\rm (}8.37cd{\rm )}.
 Understand \textit{pavitrīkṛtam} as \textit{pavitrīkṛtvantaḥ}.
 
  The reference here may be a story in which Kaśyapa and 
  other Ṛṣis are burnt to ashes, to be later reanimated by Vīrabhadra, 
  in the Śokara forest. See \PADMAP\ 5.107.1--14ff: %
  \textit{śucismitovāca~|}
  \textit{kaśyapaṃ jamadagniṃ ca devānāṃ ca purā katham~|}
  \textit{rarakṣa bhasma tad brahman samācakṣva mune mama}~||~1~||
  \textit{dadhīca uvāca~|}
  \textit{kaśyapādiyutā devāḥ pūrvam abhyāgaman girim}~|
  \textit{śokaraṃ nāma vikhyātaṃ girimadhye suśobhanam~}||~2~||
  \textit{nānāvihaṃgasaṃkīrṇaṃ nānāmunigaṇāśrayam}~| 
  \textit{vāsudevāśrayaṃ ramyam apsarogaṇasevitam}~||~3~||
  \textit{vicitravṛkṣasaṃvītaṃ sarvartukusumojjvalam}~|
  \textit{tathāvidhaṃ praviśyaite giriṃ vayam athāpare}~||~4~|| 
  \textit{stuvantaḥ keśavaṃ tatra gatāḥ sma giriśeśvaram}~|
  \textit{dṛṣṭvā tatra mahājvālāṃ praviṣṭāś ca vayaṃ ca tām~}||~5~|| 
  \textit{māmekaṃ tu tiraskṛtya hy adahad devatā munīn~}|
  \textit{māṃ dadāha tataḥ paścād bhasmībhūtā vayaṃ śubhe~}||~6~|| 
  \textit{asmān etādṛśān dṛṣṭvā vīrabhadraḥ pratāpavān~}| 
  \textit{kenāpi kāraṇenāsau gatavān parvataṃ ca tam~}||~7~|| 
  \textit{bhasmoddhūlitasarvāṅgo mastakasthaśivaḥ śuciḥ~}|
  \textit{ekākī niḥspṛhaḥ śānto hāhāśabdam athāśṛṇot~}||~8~|| 
  \textit{atha cintāparaś cāsīn mriyamāṇa śavadhvaniḥ~}| 
  \textit{śavānām iva gandhaś ca dṛśyate tannirīkṣaṇe~}||~9~|| 
  \textit{iti niścitya manasā jagāmāgnim atiprabham~}|
  \textit{sa vahnir vīrabhadraṃ ca dagdhum ārabdhavān atha~}||~10~|| 
  \textit{tṛṇāgnir iva śānto 'bhūd āsādya salilaṃ yathā~}| 
  \textit{tato 'parāṃ mahājvālāṃ vīrabhadras tu dṛṣṭavān~}||~11~||
  \textit{khaṃ gacchantīṃ mahākālo jvālāṃ nipatitām api~}|
  \textit{manasā cintayac cāpi vīrabhadraḥ pratāpavān}~||~12~|| 
  \textit{sarveṣāṃ nāśinī jvālā prāṇināṃ śatakoṭiśaḥ}~| 
  \textit{tat sarvaṃ rakṣaṇārthaṃ hi pipāsuś cāpy ahaṃ tv imām}~||~13~|| 
  \textit{prāśnāmi mahatīṃ jvālāṃ salilaṃ tṛṣito yathā}~| 
  \textit{etasminn antare vīraṃ vāg āha cāśarīriṇī}~||~14~||.
 }}

  \maintext{bhasmanā vibudhā muktā vīrabhadrabhayārditāḥ |}%

  \maintext{bhasmānuśaṃsaṃ dṛṣṭvaiva brahmaṇānumatiḥ kṛtā }||\thinspace8:38\thinspace||%
\translation{The gods, afflicted by their fear of Vīrabhadra, were set free with the help of ashes. Seeing the glory of ashes, Brahmā consented [to the use of this otherwise impure substance]. \blankfootnote{8.38 The verse may refer to the destruction of Dakṣa's sacrifice, after which the gods were 
  relieved. See old \SKANDAP\ 180.1--4ab {\rm (}in which our \textit{pāda} b is echoed{\rm )}:
  \textit{sanatkumāra uvāca\thinspace |}
  \textit{brahmādyā devatā vyāsa dakṣayajñavadhe purā\thinspace |}
  \textit{śaṅkaraṃ śaraṇaṃ jagmur vīrabhadrabhayārditāḥ}~||~1~||
  \textit{gaṇendreṇābhiyuktās tu bhasmakūṭāni bhejire}~|
  \textit{yadā bhasma praviṣṭās te tejaḥ śāṅkaram uttamam}~||~2~||
  \textit{abhavan te tadā raudrāḥ paśavo dīkṣitā iva}~| 
  \textit{bhasmābhasitagātrāṇāṃ śaṅkaravratacāriṇām}~||~3~||
  \textit{svaṃ yogaṃ pradadau teṣāṃ tadā deva umāpatiḥ~|.}
 }}

  \maintext{caturāśramato 'dhikyaṃ vrataṃ pāśupataṃ kṛtam |}%

  \maintext{tasmāt pāśupataṃ śreṣṭhaṃ bhasmadhāraṇahetutaḥ }||\thinspace8:39\thinspace||%
\translation{[Thus] the Pāśupata observance was created, which is above [the system of] the four \textit{āśrama}s. Therefore the Pāśupata [observance] is the best because it involves carrying ashes [on one's body]. \blankfootnote{8.39 One could simply accept the reading of \msCc\ {\rm (}\textit{°hetunā}{\rm )} in \textit{pāda} d, but all other rejected 
  readings hint at an original \textit{hetutaḥ} {\rm (}as remarked by Judit Törzsök{\rm )}.
 }}

  \subsubchptr{vāruṇaṃ snānam}%

  \trsubsubchptr{Water bath}%

  \maintext{vāruṇaṃ salilaṃ snānaṃ kartavyaṃ vividhaṃ naraiḥ |}%

  \maintext{nadītoyataḍāgeṣu prasraveṣu hradeṣu ca }||\thinspace8:40\thinspace||%
\translation{A water bath {\rm (}\textit{vāruṇa}{\rm )} is to be performed with water in different ways by [different] people: in the water of rivers, water tanks, streams and ponds. \blankfootnote{8.40 The reading \textit{vividhaṃ} in \textit{pāda} b seems to be the lectio difficilior as opposed to
  the rejected \textit{vidhivat}.
 }}

  \subsubchptr{brāhmyaṃ snānam}%

  \trsubsubchptr{Vedic bath}%

  \maintext{brahmasnānaṃ ca viprendra āpohiṣṭhaṃ vidur budhāḥ |}%

  \maintext{trisaṃdhyam eva kartavyaṃ brahmasnānaṃ tad ucyate }||\thinspace8:41\thinspace||%
\translation{The wise know the Vedic bath as [the one performed with the Vedic mantra beginning] \textit{āpo hi ṣṭhā}, O excellent Brahmin. It is to be performed at the three junctures of the day [dawn, noon, evening]. It is called the Vedic bath. \blankfootnote{8.41 The Ṛgvedic mantra starting with \textit{āpo hi ṣṭhā} {\rm (}ṚV 10.9.1--3{\rm )} is traditionally associated with 
  \textit{mārjana} {\rm (}`cleaning, wiping'{\rm )}. According to \mycitep{KaneHistory}{v. 4, 120},
  a Brahmin ``should bathe thrice in the day, should perform \textit{mārjana} {\rm (}splashing
  or sprinkling water on the head and other limbs by means of \textit{kuśas} 
  dipped in water after repeating sacred mantras{\rm )} with the three verses `apo hi sthā' [sic] {\rm (}Ṛg. X.9.1--3{\rm )} [...]''
  This suggests a method of bathing that is more of a ritual than an actual bath.
 }}

  \subsubchptr{vāyavyaṃ snānam}%

  \trsubsubchptr{Wind bath}%

  \maintext{goṣu saṃcāramārgeṣu yatra godhūlisambhavaḥ |}%

  \maintext{tatra gatvāvasīdeta snānam uktaṃ manīṣibhiḥ }||\thinspace8:42\thinspace||%
\translation{He should go where, on the paths where cows roam, dust is rising, and he should sit down there. This is called [a kind of] bath, [namely the \textit{vāyavya} or wind-bath]. \blankfootnote{8.42 Understand \textit{goṣu} in \textit{pāda} a as \textit{gavāṃ} {\rm (}genitive{\rm )}.
 This version of bathing seems to be a way of taking a shower 
  in the holy dust raising from under the hooves of cows.
 }}

  \subsubchptr{divyaṃ snānam}%

  \trsubsubchptr{Heavenly bath}%

  \maintext{varṣatoyāmbudhārābhiḥ plāvayitvā svakāṃ tanum |}%

  \maintext{snānaṃ divyaṃ vadaty eva jagadādimaheśvaraḥ }||\thinspace8:43\thinspace||%
\translation{One should immerse one's own body in the water-showers of rain water. The one and only great Lord {\rm (}\textit{maheśvara}{\rm )} of the universe calls it heavenly bath. }

  \maintext{iti niyamavibhāgaḥ pañcabhedena vipra}%

 \nonanustubhindent \maintext{nigadita tava pṛṣṭaḥ sarvalokānukampya |}%

  \maintext{sakalamalapahārī dharmapañcāśad etan}%

 \nonanustubhindent \maintext{na bhavati punajanma kalpakoṭyāyute 'pi }||\thinspace8:44\thinspace||%
\translation{Thus have I taught you the section on the Niyama-rules in divisions of five [sub-categories to each] because you asked me to, favouring the whole world. These fifty Dharmic [teachings], wipe off all the defilement. There will not be rebirth [for one who keeps these rules], not even in millions of \ae ons. \blankfootnote{8.44 This verse marks not only the end of a long section on the Niyama rules,
  but also the end of a major part of the text that discusses the ten Yama and ten Niyama rules,
  spanning 3.16--8.44.
 
  
 There are two stem form nouns in \textit{pāda} b: I suspect that \Ed\ is right
  assuming that in order to restore the metre, we must have \textit{nigadita} and not
  \textit{nigaditas}, which is trasmitted in all the witnesses;
  also understand \textit{sarvalokānukampya} in \textit{pāda} b as \textit{sarvalokān anukampya}.
 Understand \textit{sakalamalapahārī} in \textit{pāda} c as \textit{sakala-mala-apahārī}, which would be unmetrical,
  and compare it with \textit{duritamalapahārī} in 4.89c.
  Take \textit{etan/etad} as either picking up °\textit{pahārī} or rather
  a plural corresponding to °\textit{pañcāśad}. The latter phenomenon, namely the use of
  the singular after numbers, is one of the hallmarks of the text.
 
  By `fifty Dharmas,' the text refers to the ten main Niyama-rules × five subcategories.
 
  
 The licence of an word-ultimate short syllable treated as long {\rm (}°\textit{janma} in \textit{pāda} d{\rm )} is
  also freqently seen here. Note also \textit{puna} for \textit{punar} metri causa.
 }}
\center{\maintext{\dbldanda\thinspace iti vṛṣasārasaṃgrahe niyamapraśaṃsā nāmādhyāyo 'ṣṭamaḥ\thinspace\dbldanda}}
\translation{Here ends the eighth chapter in the \textit{Vṛṣasārasaṃgraha} called the Praise of the Niyama-rules}

  \chptr{navamo 'dhyāyaḥ}
\addcontentsline{toc}{subsection}{Chapter 9}
\fancyhead[CO]{{\footnotesize\textit{Translation of chapter 9}}}%

  \trchptr{ Chapter Nine}%

  \subchptr{traiguṇyam}%

  \trsubchptr{The system of three qualities}%

  \maintext{trikālaguṇabhedena bhinnaṃ sarvacarācaram |}%

  \maintext{tasmāt triguṇabandhena veṣṭitaṃ nikhilaṃ jagat }||\thinspace9:1\thinspace||%
\translation{The whole universe with its moving and unmoving elements is divided by the three subdivisions {\rm (}\textit{guṇa}{\rm )} of time. Therefore the whole world is bound by the fetters of three qualities {\rm (}\textit{guṇa}{\rm )}. \blankfootnote{9.1 It is only \msM, a MS not collated for this chapter, that inserts, post correctionem,
  \textit{anarthayajña uvāca} at the beginning of this chapter. It is not really needed:
  Anarthayajña's teaching continues without interruption here. 
  Another possibility is that this verse was originally the continuation
  of the end of chapter two {\rm (}2:40ef: \textit{traikālyakalanāt kālas tena kālaḥ prakīrtitaḥ}{\rm )}.
  At least it seems to directly connect there topic-wise.
  My translation of \textit{guṇa} in \textit{pāda} a is tentative.
 }}

  \maintext{vigatarāga uvāca |}%

  \maintext{traikālyam iti kiṃ jñeyaṃ traidhātukaśarīriṇaḥ |}%

  \maintext{kiṃcid vistaram eveha kathayasva tapodhana }||\thinspace9:2\thinspace||%
\translation{Vigatarāga spoke: What does the term `the three times' mean for an embodied creature that is made up of three constituents {\rm (}\textit{dhātuka}{\rm )}? Teach me about this in a somewhat more extended manner, O great ascetic. \blankfootnote{9.2 I have included the element \textit{trai°} in the lemma in \textit{pāda}s ab only because \msCc\ 
  has a slightly unusual ligature there {\rm (}\textit{mtrai}{\rm )}. 
  
  As for the interpretation of
  \textit{traidhātuka} in \textit{pāda} b, an intelligent guess would be a reference to the three so-called
  `humours' of the body, namely \textit{pitta, vāyu/anila/vāta}, and \textit{śleṣman}.
  These are discussed later in, e.g., \VSS\ 23:31--32ab, in the context of types of sleep:
  \textit{śleṣmapittānilasthāne trīṇi pakṣāṇi vāsinaḥ\thinspace |
  tamaḥ śleṣmāśrayā nidrā rajonidrā tu vātikā\thinspace ||
  pittāśrayāṃ smṛtāṃ nidrāṃ sāttvikāṃ viddhi bhūpate\thinspace |}.
 
  \MBH\ 12.330.21--22ab clearly states that the three \textit{dhātu}s, \textit{pitta, śleṣma} and \textit{vāyu},
  keep the body alive:
  \textit{trayo hi dhātavaḥ khyātāḥ karmajā iti ca smṛtāḥ\thinspace |
  pittaṃ śleṣmā ca vāyuś ca eṣa saṃghāta ucyate\thinspace ||
  etaiś ca dhāryate jantur etaiḥ kṣīṇaiś ca kṣīyate\thinspace |}.
 
  The present verse in the \VSS\ contains the only occurence 
  of the term \textit{traidhātuka} in the text.
  In 5.11cd, \textit{dhātu} is probably used in the same Ayurvedic sense 
  that I am proposing here {\rm (}\textit{dhātuvaiṣamyanāśo 'sti na ca rogāḥ sudāruṇāḥ}{\rm )}.
  Elsewhere \textit{dhātu} means `verbal root' {\rm (}3.3{\rm )}, `metal' {\rm (}16.6: 
  \textit{yathā vai sarvadhātūnāṃ doṣā dahyanti dhāmyatām\thinspace | 
  tathā pāpāḥ pradahyante dhruvaṃ prāṇasya nigrahāt}\thinspace ||{\rm )},
  and `gross element' {\rm (}for Sāṃkhya-style \textit{mahābhūta}s in chapter 20{\rm )}.
  To slightly complicate things, chapter thirteen claims that the human body is made up
  of two \textit{dhātu}s, \textit{somadhātu} and \textit{agnidhātu}. Semen contains \textit{somadhātu},
  menstrual blood \textit{agnidhātu}, and the new-born baby is thus made up of both. See e.g. 13.20cd--13.21:
  \textit{śukraśoṇitasaṃyogād garbhotpattis tataḥ smṛtaḥ\thinspace ||}
  \textit{agnisomātmakaṃ devi śarīradvayadhātutaḥ\thinspace |}
  \textit{somadhātu smṛtaṃ śukram agnidhātu rajaḥ smṛtam\thinspace |}
  \textit{agnisomāśrayaṃ devi śarīram iti saṃjñitam\thinspace ||}.
 }}

  \maintext{anarthayajña uvāca |}%

  \maintext{traikālyaṃ triguṇaṃ jñeyaṃ vyāpī prakṛtisambhavaḥ |}%

  \maintext{anyonyam upajīvanti anyonyam anuvartinaḥ }||\thinspace9:3\thinspace||%
\translation{Anarthayajña spoke: The three times are the three qualities {\rm (}\textit{guṇa}{\rm )}. They are [all-]pervading and are born from Prakṛti. They support each other, they follow each other. \blankfootnote{9.3 Understand \textit{pāda} b as referring to the neuter \textit{traikālyaṃ} or rather 
  \textit{triguṇaṃ} {\rm (}gender confusion{\rm )}.
 }}

  \maintext{sattvaṃ rajas tamaś caiva rajaḥ sattvaṃ tamas tathā |}%

  \maintext{tamaḥ sattvaṃ rajaś caiva anyonyamithunāḥ smṛtāḥ }||\thinspace9:4\thinspace||%
\translation{Sattva, Rajas and Tamas; Rajas, Sattva and Tamas; Tamas, Sattva and Rajas; they are mutually each other's pairs. }

  \maintext{sāttviko bhagavān viṣṇū rājasaḥ kamalodbhavaḥ |}%

  \maintext{tāmaso bhagavān īśaḥ sakalaṃvikaleśvaraḥ }||\thinspace9:5\thinspace||%
\translation{Lord Viṣṇu is Sattvic. [Brahmā], the one who was born on a lotus, is Rājasa. Lord Īśa is Tāmasa, [both in his] complete {\rm (}\textit{sakala}{\rm )} [form] and [as] formless {\rm (}\textit{vikala}{\rm )} Īśvara. \blankfootnote{9.5 My altering the reading \textit{viṣṇu} to \textit{viṣṇū} in \textit{pāda} a against all witnesses may
  be regarded as an overcorrection and the stem form could be original.
 My translation of \textit{pāda}s cd is tentative. I suspect that \textit{pāda} d is one single compound,
  the \textit{anusvāra} is only inserted to avoid the metric fault of two \textit{laghu} syllables 
  at the second and third position.
  I understand \textit{vikala} as a synonym of \textit{niṣkala}. For the tantric connotations of the pair
  \textit{sakala-niṣkala} see, e.g., \TAKIII\ s.v. \textit{niṣkala}.
 }}

  \maintext{sattvaṃ kundenduvarṇābhaṃ padmarāganibhaṃ rajaḥ |}%

  \maintext{tamaś cāñjanaśailābhaṃ kīrtitāni manīṣibhiḥ }||\thinspace9:6\thinspace||%
\translation{Sattva is of the colour of jasmine and the moon. Rajas is of the colour of ruby. Tamas is of the colour of lamp-black and colliryum. [This is how the colours of the qualities] are taught by the wise. }

  \maintext{sattvaṃ jalaṃ rajo 'ṅgāraṃ tamo dhūmasamākulam |}%

  \maintext{etadguṇamayair baddhāḥ pacyante sarvadehinaḥ }||\thinspace9:7\thinspace||%
\translation{Sattva is water, Rajas is charcoal, Tamas is filled with smoke. All living creature are being burnt away by [the fire] of these qualities {\rm (}\textit{guṇa}{\rm )}. }

  \maintext{vigatarāga uvāca |}%

  \maintext{kena kena prakāreṇa guṇapāśena badhyate |}%

  \maintext{cihnam eṣāṃ pṛthaktvena kathayasva tapodhana }||\thinspace9:8\thinspace||%
\translation{Vigatarāga spoke: By what sort of nooses of the qualities {\rm (}\textit{guṇa}{\rm )} is [a person] bound? Teach me the signs connected to them one by one, O great ascetic. }

  \maintext{anarthayajña uvāca |}%

  \maintext{anekākārabhāvena badhyante guṇabandhanaiḥ |}%

  \maintext{mohitā nābhijānanti jānanti śivayoginaḥ }||\thinspace9:9\thinspace||%
\translation{Anarthayajña spoke: [Creatures] are bound in many ways and by many conditions by the fetters of the qualities {\rm (}\textit{guṇa}{\rm )}. Those who are deluded do not know. The Śivayogins do know. \vfill\pagebreak }

  \maintext{ūrdhvaṃgo nityasattvastho madhyago rajasāvṛtaḥ |}%

  \maintext{adhogatis tamo'vasthā bhavanti puruṣādhamāḥ }||\thinspace9:10\thinspace||%
\translation{He who is always established in Sattva goes upwards. He who is covered with Rajas goes in the middle. Those lowest of men in the state of Tamas go downward. \blankfootnote{9.10 Understand \textit{adhogatis} in \textit{pāda} c as a \textit{bahuvrīhi} in the plural {\rm (}\textit{adhogatayas}{\rm )}.
 }}

  \maintext{svarge 'pi hi trayo vaite bhāvanīyās tapodhana |}%

  \maintext{mānuṣeṣu ca tiryeṣu guṇabhedās trayas trayaḥ }||\thinspace9:11\thinspace||%
\translation{These three kinds of [\textit{guṇa}s] are to be acknowledged even in heaven, O great ascetic, and among humans and also among animals. }

  \subsubchptr{sāttvikottamāḥ}%

  \trsubsubchptr{Superior Sattva-type}%

  \maintext{brahmā viṣṇuś ca rudraś ca dharma indraḥ prajāpatiḥ |}%

  \maintext{somo 'gnir varuṇaḥ sūryo daśa sattvottamāḥ smṛtāḥ }||\thinspace9:12\thinspace||%
\translation{The ten superior Sattva [beings] are: Brahmā, Viṣṇu, Rudra, Dharma, Indra, Prajāpati, Soma, Agni, Varuṇa and Sūrya. \blankfootnote{9.12 Note that Brahmā was labelled as Rajas-type in 9.5b above.
 }}

  \subsubchptr{sāttvikamadhyamāḥ}%

  \trsubsubchptr{Middle Sattva-type}%

  \maintext{rudrādityā vasusādhyā viśveśamaruto dhruvaḥ |}%

  \maintext{ṛṣayaḥ pitaraś caiva daśaite sattvamadhyamāḥ }||\thinspace9:13\thinspace||%
\translation{The ten middle-ranking Sattva [beings] are: Rudra[s], Ādityas, Vasus, Sādhyas, Viśveśa [or the Viśvedevas and Īśa?], the Maruts, Dhruva, the sages, and the ancestors. \blankfootnote{9.13 \textit{Pāda} a is a \textit{sa-vipulā}. Note that there seems to be only nine names/categories listed here unless
  we try to interpret \textit{viśveśa} as \textit{viśvedevāḥ} and \textit{īśaḥ}.
 }}

  \subsubchptr{sāttvikādhamāḥ}%

  \trsubsubchptr{Low Sattva-type}%

  \maintext{tārā grahāḥ surā yakṣā gandharvāḥ kiṃnaroragāḥ |}%

  \maintext{rakṣobhūtapiśācāś ca daśaite sāttvikādhamāḥ }||\thinspace9:14\thinspace||%
\translation{The ten low-ranking Sattva [beings] are the stars, the planets, the Suras, the Yakṣas, the Gandharvas, the Kiṃnaras, the Serpents, the Rakṣases, the Ghosts, and the Piśācas. }

  \subsubchptr{rājasottamāḥ}%

  \trsubsubchptr{Superior Rajas-type}%

  \maintext{ṛtvik purohitācāryayajvāno 'tithi vijñanī |}%

  \maintext{rājā mantrī vratī vedī daśaite rājasottamāḥ }||\thinspace9:15\thinspace||%
\translation{The ten superior Rājasa [categories] are Ṛtvij priests, domestic Purohita chaplains, teachers, sacrificers, guests, the wise, kings, ministers, people engaged in religious observances, and learned [Brahmins]. \blankfootnote{9.15 I take \textit{'tithi} as a stem form noun and \textit{vijñanī} as \textit{vijñānī}, both metri causa.
 \textit{rājamantrī} as `minister' makes sense, but by emendading \textit{rāja}° to \textit{rājā} 
  in \textit{pāda} c I aim to arrive at a list of ten categories instead of nine.
 }}

  \subsubchptr{rājasamadhyamāḥ}%

  \trsubsubchptr{Middle Rajas-type}%

  \maintext{sūto 'mbaṣṭhavaṇiś cograḥ śilpikārukamāgadhāḥ |}%

  \maintext{veṇavaidehakāmātyā daśaite rajamadhyamāḥ }||\thinspace9:16\thinspace||%
\translation{The ten middle-ranking Rājasa [categories] are [the following castes and professions]: Sūta [coachman/bard], Ambaṣṭha [doctor], Vaṇij [merchant caste], Ugra [combatant?], Śilpin and Kāruka [both artisans], Māgadha [bard] Veṇa [musician], Vaidehaka [guard], and Āmātya [counsellor]. \blankfootnote{9.16 Since all the wittnesses consulted treat \textit{vaṇi} as an acceptable stem in \textit{pāda} a,
  I have refrained from correcting it to \textit{vaṇij/vaṇik}. The English equivalents that
  I give in square brackets are in some cases not more than traditionally accepted guesses.
 }}

  \subsubchptr{rājasādhamāḥ}%

  \trsubsubchptr{Low Rajas-type}%

  \maintext{carmakṛt kumbhakṛt kolī lohakṛttrapunīlikāḥ |}%

  \maintext{naṭamuṣṭikacaṇḍālā daśaite rajasādhamāḥ }||\thinspace9:17\thinspace||%
\translation{The low-ranking Rājasa [professions] are: leathersmith, potter, Kolī, blacksmith, tinsmith, dyer. dancer, goldsmith, Caṇḍāla. \blankfootnote{9.17 Problems with this verse include the following. There are only nine 
  professions/castes listed here instead of the expected ten.
  \textit{kolī} is difficult to interpret; later texts of the Jātiviveka
  genre such as Gopinātha's \textit{Jātiviveka} 
  {\rm (}see \mycite{OHanlonHidasKiss}{\rm )} mention \textit{kolī} as a regional 
  name for the caste Niṣāda {\rm (}sometimes: a falconer{\rm )}. I take \textit{trapu} tentatively as \textit{trapukṛt}
  although I can't see any attestation of that form. And taking \textit{nīlikā} as a {\rm (}female{\rm )} dyer
  is again tentative.
 }}

  \subsubchptr{tāmasottamāḥ}%

  \trsubsubchptr{Superior Tamas-type}%

  \maintext{gogajagavayā aśvamṛgacāmarakiṃnarāḥ |}%

  \maintext{siṃhavyāghravarāhāś ca daśaite tāmasottamāḥ }||\thinspace9:18\thinspace||%
\translation{These are the ten superior Tāmasa [creatures]: cows, elephants, Gayal oxen, horses, deer, Yaks, Kiṃnaras, lions, tigers, wild boar. }

  \subsubchptr{tāmasamadhyamāḥ}%

  \trsubsubchptr{Middle Tamas-type}%

  \maintext{ajameṣamahiṣyāś ca mūṣikānakulādayaḥ |}%

  \maintext{uṣṭraraṅkuśaśagaṇḍā daśaite tamamadhyamāḥ }||\thinspace9:19\thinspace||%
\translation{The ten middle-ranking Tāmasa [animals] are: goats, sheep, buffaloes, mice, mongooses etc., camels, Raṅku deer, hares, rhinoceroses. \blankfootnote{9.19 \textit{°mahiṣyāś} seems to be an equivalent of \textit{°mahiṣāś} metri causa. 
  Again, we expect ten items in this list but we find only nine.
 \textit{Pāda} c is a \textit{sa-vipulā}.
 }}

  \subsubchptr{tāmasādhamāḥ}%

  \trsubsubchptr{Low Tamas-type}%

  \maintext{ṛkṣagodhāmṛgaśṛṅgibakavānaragardabhāḥ |}%

  \maintext{sūkaraśvānagomāyur daśaite tāmasādhamāḥ }||\thinspace9:20\thinspace||%
\translation{The ten low-ranking Tāmasa [beings] are: bears, alligators, deer, sheat-fish, cranes, apes, donkeys, boar, dogs and frogs. \blankfootnote{9.20 \textit{Pāda} a is a \textit{sa-vipulā}. Translating \textit{śṛṅgi}, \textit{śṛṅgin}, or perhaps \textit{śṛṅgī} as `sheat-fish' is not
  much more than a guess. Other possibilities such as `elephants' or `bulls'
  are less likely because we have had them above in other categories, 
  although repetitions do occur across, and sometimes within, these lists:
  see, e.g. \textit{mṛga} mentioned both in 9.18 and 20, \textit{śyena} in both 9.21 and 22, 
  and \textit{śuka} repeated in 9.21.
 }}

  \subsubchptr{tamasāttvikāḥ}%

  \trsubsubchptr{The Tamas-Sattva category}%

  \maintext{krauñcahaṃsaśukaśyenabhāsabāruṇḍasārasāḥ |}%

  \maintext{cakrāhvaśukamāyūrā daśaite tamasāttvikāḥ }||\thinspace9:21\thinspace||%
\translation{The ten Tāmasa-Sāttvika [beings] are: curlews, geese, parrots, falcons, vultures, B[h]āruṇḍa birds, cranes, Cakra[vāka] birds, parrots, and peacocks. \blankfootnote{9.21 Although all the manuscripts consulted read \textit{kroñca°} in \textit{pāda} a, I have decided
  to accept \Ed's standard spelling in this case. In \textit{pāda} b, I left \textit{°bāruṇḍa°}
  thus, although what is really meant is probably \textit{bhāraṇḍa}, \textit{bhāruṇḍa} or \textit{bhuruṇḍa}.
 Note the repetition of \textit{śuka} in this stanza.
 }}

  \subsubchptr{tamarājasāḥ}%

  \trsubsubchptr{The Tamas-Rajas category}%

  \maintext{balākāḥ kukkuṭāḥ kākāś cillalāvakatittirāḥ |}%

  \maintext{gṛdhrakaṅkabakaśyena daśaite tamarājasāḥ }||\thinspace9:22\thinspace||%
\translation{The ten Tāmasa-Rājasa [beings] are: Balāka-cranes, cocks, crows, Bengal kites, painted quails, partridges, vultures, herons, Bakas and hawks. \blankfootnote{9.22 It would be easy to correct the stem form °\textit{śyena} in \textit{pāda} c to \textit{śyeno} but I suspect
  that the form could be original, possibly because it was confused with an instrumental.
 }}

  \subsubchptr{tāmasādhamādi}%

  \trsubsubchptr{Low Tamas-type etc.}%

  \maintext{kokilolūkakañjalyakapotāḥ pañca eva ca |}%

  \maintext{śārikāś ca kuliṅgāś ca daśaite tamasādhamāḥ }||\thinspace9:23\thinspace||%
\translation{The ten lowest Tāmasa [beings also include]: cuckoos, owls, Kañjala-birds, doves, and the five[?], Śārika birds and sparrows. \blankfootnote{9.23 My impression is that the reading °\textit{kiñjalka}° {\rm (}usually: `the filament of a lotus'{\rm )} in \textit{pāda} a
  is either a mistake for, or rather an altered form metri causa, maybe a regional form, 
  of \textit{kañjala} {\rm (}a kind of bird{\rm )}. \msCa\msCc\msNa\ {\rm (}\textit{kiñjalya}{\rm )} may be slightly closer 
  to the required form {\rm (}\textit{kañjalaka}/\textit{kañjalka}?{\rm )}. My emendation is a compromise.
  Note that there are only six items in this list and that \textit{pāda} b is 
  difficult to make sense of in this context. Something must have gone wrong here.
 }}

  \maintext{makaragohanakrāś ca ṛkṣāś ca tamasāttvikāḥ |}%

  \maintext{kacchapaśiśukumbhīramaṇḍūkās tamarājasāḥ |}%

  \maintext{śaṅkhaśuktikaśambūkāḥ kavayyas tamatāmasāḥ }||\thinspace9:24\thinspace||%
\translation{Makara crocodiles, cow-killing alligators and bears are of Tamas-Sattva. Tortoises, porpoises, crocodiles of the Ganges and frogs are of Tamas-Rajas. Conch-shells, pearl-oysters, shells, and Kavayī fish are Tamas-Tāmasa. \blankfootnote{9.24 Note the two \textit{laghu}s in \textit{pāda} a. 
  The reading that yields `and bears' {\rm (}\textit{ṛkṣāś ca}{\rm )} is my conjecture
  for a problematic \textit{ṛṣā ca}. It is far from satisfactory since bears have already appeared in 
  verse 9.20 above.
 My emendation of the word \textit{śuśu} to \textit{śisu} {\rm (}`porpoise,' for \textit{śiśuka} or \textit{śiśumāra}, lit. 
  `child-killer'{\rm )} in \textit{pāda} c is based on the fact that, most probably,
  we need an aquatic animal here, rather than a hare {\rm (}\textit{śaśa}{\rm )}.
 The readings \textit{kabandhyās} and \textit{kabanas} in \textit{pāda} f make no sense. I conjecture \textit{kavayyas} {\rm (}the plural of
  \textit{kavayī}{\rm )}, which is a type of fish. See them mentioned in \MAHASUBHS\ 388:
  \textit{ajājījambāle rajasi maricānāṃ ca luṭhitāḥ
  kaṭutvād uṣṇatvāj janitarasanauṣṭhavyatikarāḥ\thinspace |
  anirvāṇotthena prabalataratailāktatanavo 
  mayā sadyo bhṛṣṭāḥ katipayakavayyaḥ kavalitāḥ\thinspace ||}.
  See a translation of this verse in the \MAHASUBHS\ {\rm (}ed. Sternbach, vol. 1, p. 67{\rm )}:
  `I rolled them in a cumin swamp / and in a heap of pepper dust / till they were spiced and hot enough /
  to twist your tongue and mouth. / When they were basted well with oil, / I didn't wait to wash or sit ; /
  I gobbled that mess of \textit{koji} fish / as soon as they were fried. {\rm (}D.H.H. Ingalls's translation{\rm )}.'
 }}

  \maintext{candanāgarupadmaṃ ca plakṣodumbarapippalāḥ |}%

  \maintext{vaṭadāruśamībilvā daśaite tamasāttvikāḥ }||\thinspace9:25\thinspace||%
\translation{Sandal tree, Aloe wood, lotus, waved-leaf fig-tree, Ficus Glomerata, holy fig-tree, Banyan, Devadāru tree, Śamī tree, wood-apple tree: these ten are Tamas-Sattva. \blankfootnote{9.25 In \textit{pāda} d, \textit{tamas}° or \textit{tamaḥ}° are unmetrical and might be the result
  of scribal correction. The original may have been the metrical \textit{tama}°, here
  transmitted only in \Ed. Cf. 9.27d.
 }}

  \maintext{jāmbīralakucāmrātadāḍimākolavetasāḥ |}%

  \maintext{nimbanīpo {\rm †}dhravāvaś ca{\rm †} daśaite tamarājasāḥ }||\thinspace9:26\thinspace||%
\translation{The ten Tamas-Rajas [trees] are: Citron trees, bread-fruit trees, hog-plum trees, pomegranate trees, jujube trees, rattan trees, Neemb trees, Kadamba trees and ... \blankfootnote{9.26 There seems to be only nine items here instead of the expected ten. I have not been able
  to interpret the last one, \textit{dhravāvaś}.
 }}

  \maintext{vṛkṣavallīlatāveṇutvaksāratṛṇabhūruhāḥ |}%

  \maintext{mīrajāś ca śilāśasyā daśaite tamasāttvikāḥ }||\thinspace9:27\thinspace||%
\translation{Trees, creepers, winding plants, cane, bamboo, grass, plants, seaweed, rocks, grains are the ten Tamas-Sattva ones. }

  \maintext{bhramarādipataṅgāś ca krimikīṭajalaukasaḥ |}%

  \maintext{yūkoddaṃśamaśānāṃ ca viṣṭhājās tamasāttvikāḥ }||\thinspace9:28\thinspace||%
\translation{Bees, butterflies etc., worms, insects, aquatic animals, lice, bugs, mosquitoes, creatures in f\ae ces are Tamas-Sattva ones. \blankfootnote{9.28 \textit{ādi} in \textit{pāda} a is misplaced, in order to avoid the metrical fault of 
  two \textit{laghu} syllables in the second and third syllables; understand \textit{bhramarapataṅgādayaś ca}.
 }}

  \maintext{dayā satyaṃ damaḥ śaucaṃ jñānaṃ maunaṃ tapaḥ kṣamā |}%

  \maintext{śīlaṃ ca nābhimānaṃ ca sāttvikāś cottamā janāḥ }||\thinspace9:29\thinspace||%
\translation{[These ten words describe] the people who are the best among the Sāttvika [type]: compassion, truthfulness, self-control, purity, knowledge, observing silence, penance, patience, integrity, lack of self-conceit. }

  \maintext{kāmatṛṣṇāratidyūtamāno yuddhaṃ madaḥ spṛhā |}%

  \maintext{nirghṛṇāḥ kalikartāro rājaseṣūttamā janāḥ }||\thinspace9:30\thinspace||%
\translation{[These ten words describe] the people who are the best among the Rājasa [type]: desire, thirst, pleasure, gambling, arrogance, fight, intoxication, delight, cruel, quarrelling. }

  \maintext{hiṃsāsūyāghṛṇāmūḍhanidrātandrībhayālasāḥ |}%

  \maintext{krodho matsaramāyī ca tāmaseṣūttamā janāḥ }||\thinspace9:31\thinspace||%
\translation{[These words describe] people who are the best among the Tāmasa [type]: violence, envy, incompassionate, stupid, sleepy, lazy, cowardly, idle, anger, greedy, cheating. }

  \maintext{laghuprītiprakāśī ca dhyānayoge sadotsukaḥ |}%

  \maintext{prajñābuddhivirāgī ca sāttvikaṃ guṇalakṣaṇam }||\thinspace9:32\thinspace||%
\translation{The Sāttvika can be characterised as follows: light, joyful, bright, always eager for yoga meditation, wise, intelligent and dispassionate. }

  \maintext{bālako nipuṇo rāgī māno darpaś ca lobhakaḥ |}%

  \maintext{spṛhā īrṣā pralāpī ca rājasaṃ guṇalakṣaṇam }||\thinspace9:33\thinspace||%
\translation{The Rājasa can be characterised as follows: childish, skilful, passionate, proud, arrogant, greedy, desirous, jealous, and chattering. }

  \maintext{udvega ālaso mohaḥ krūras taskaranirdayaḥ |}%

  \maintext{krodhaḥ piśuna nidrā ca tāmasaṃ guṇalakṣaṇam }||\thinspace9:34\thinspace||%
\translation{The Tāmasa can be characterised as follows: anxious, lazy, delusion, cruel, a thief, pitiless, angry, wicked and sleepy. \blankfootnote{9.34 In \textit{pāda} a, \textit{piśuno} might be the right choice: the \textit{pāda} is a \textit{ra-vipulā} 
  if \textit{dr} in \textit{nidrā} does not make the previous syllable long, a licence
  often occuring in this text {\rm (}`muta cum liquida'{\rm )}.
 }}

  \subsubchptr{āhāras traiguṇye}%

  \trsubsubchptr{Food and the three qualities}%

  \maintext{vigatarāga uvāca |}%

  \maintext{kena cihnena vijñeya āhāraḥ sarvadehinām |}%

  \maintext{traiguṇyasya pṛthaktvena kathayasva tapodhana }||\thinspace9:35\thinspace||%
\translation{Vigatarāga spoke: By what signs can the food of each [category of] humans be characterised? Teach me with regards to the three qualities {\rm (}\textit{guṇa}{\rm )}, O great ascetic. }

  \maintext{anarthayajña uvāca |}%

  \maintext{āyuḥ kīrtiḥ sukhaṃ prītir balārogyavivardhanam |}%

  \maintext{hṛdyasvādurasaṃ snigdha āhāraḥ sāttvikapriyaḥ }||\thinspace9:36\thinspace||%
\translation{Anarthayajña spoke: The Sāttvikas prefer food that yields [long] life, fame, happiness, joy, which increases strength and health, which is savoury and which tastes nice, and which is soft. }

  \maintext{atyuṣṇam āmlalavaṇaṃ rūkṣaṃ tīkṣṇaṃ vidāhi ca |}%

  \maintext{rājasaśreṣṭha-āhāro duḥkhaśokāmayapradaḥ }||\thinspace9:37\thinspace||%
\translation{The best food for the Rājasas is rather warm, acidic, salty, hard, hot and pungent. It gives you pain, a burning sensation and indigestion. \blankfootnote{9.37 Note the lack of sandhi within what was meant to be a compund in \textit{pāda} c {\rm (}understand
  \textit{rājaśreṣṭhāhāro}{\rm )}, and the total lack of gender agreement between the adjectives in \textit{pāda}s ab, and
  \textit{āhāro} and \textit{pradaḥ}.
 }}

  \maintext{abhakṣyāmedhyapūtī ca pūti paryuṣitaṃ ca yat |}%

  \maintext{āmayārasavisvāda āhāras tāmasapriyaḥ }||\thinspace9:38\thinspace||%
\translation{Tāmasas prefer food that is prohibited, impure and foul-smelling, stinky and stale. It causes indigestion, is sapless and tasteless. \blankfootnote{9.38 Understand \textit{°pūtī} \textit{in} pāda a as standing for \textit{°pūti} metri causa {\rm (}which is oddly repeated in 
  \textit{pāda} b{\rm )}, and note that °āmedhya° in the same \textit{pāda} is an emendation {\rm (}correcting \msNc's reading{\rm )}.
 I have conjectured \textit{āmayārasa}° for \textit{āyāmarasa}° in \textit{pāda} c because the transmitted readings
  make little sense and because \textit{āmaya} appeared in 9.37d above.
 }}

  \subsubchptr{guṇātītam}%

  \trsubsubchptr{Beyond the qualities}%

  \maintext{vigatarāga uvāca |}%

  \maintext{guṇātītaṃ kathaṃ jñeyaṃ saṃsāraparapāragam |}%

  \maintext{guṇapāśanibaddhānāṃ mokṣaṃ kathaya tattvataḥ }||\thinspace9:39\thinspace||%
\translation{Vigatarāga spoke: How can one recognize [the state of getting] beyond the \textit{guṇa}s, which leads one to the other shore of [the ocean] of mundane existence? Tell me truly about the liberation of those who are bound by the noose of the \textit{guṇa}s. }

  \maintext{anarthayajña uvāca |}%

  \maintext{ātmavat sarvabhūtāni samyak paśyeta bho dvija |}%

  \maintext{guṇātītaḥ sa vijñeyaḥ saṃsāraparapāragaḥ }||\thinspace9:40\thinspace||%
\translation{Anarthayajña spoke: Well, he who looks at all living beings in the correct way, as his own Self, O Brahmin, is to be known as one beyond the qualities {\rm (}\textit{guṇa}{\rm )}, as one who has departed to the other shore of [the ocean of] mundane existence. \blankfootnote{9.40 Note passages in the \BHG\ {\rm (}6.32, 12.13, 14.24--25{\rm )}
  quoted in the apparatus to the critical edition, of which \VSS\ 9.40--42 seem
  to be echoes of.
 }}

  \maintext{īrṣādveṣasamo yas tu sukhaduḥkhasamāś ca ye |}%

  \maintext{stutinindāsamā ye ca guṇātītaḥ sa ucyate }||\thinspace9:41\thinspace||%
\translation{He who is indifferent to envy and hate, treats happiness and sorrow as equal, treats praise and reproach as equal, is called `one who is beyond the qualities {\rm (}\textit{guṇa}{\rm )}'. }

  \maintext{tulyapriyāpriyo yaś ca arimitrasamas tathā |}%

  \maintext{mānāpamānayos tulyo guṇātītaḥ sa ucyate  }||\thinspace9:42\thinspace||%
\translation{He who treats pleasant and unpleasant things, enemy and friend, respect and contempt equally, is called `one who is beyond the qualities {\rm (}\textit{guṇa}{\rm )}'. }

  \maintext{eṣa te kathito vipra guṇasadbhāvanirṇayaḥ |}%

  \maintext{guṇayuktas tu saṃsārī guṇātītaḥ parāṅgatiḥ }||\thinspace9:43\thinspace||%
\translation{O Brahmin, thus has the exposition of the essence of the qualities {\rm (}\textit{guṇa}{\rm )} been taught to you. Those who are connected with the qualities {\rm (}\textit{guṇa}{\rm )} are mundane {\rm (}\textit{saṃsārin}{\rm )}, those beyond the qualities {\rm (}\textit{guṇa}{\rm )} are on the supreme path. }
\center{\maintext{\dbldanda\thinspace iti vṛṣasārasaṃgrahe traiguṇyaviśeṣaṇīyo nāmādhyāyo navamaḥ\thinspace\dbldanda}}
\translation{Here ends the ninth chapter in the \textit{Vṛṣasārasaṃgraha} called the Particulars of the Three Guṇas}

  \chptr{daśamo 'dhyāyaḥ}
\addcontentsline{toc}{subsection}{Chapter 10}
\fancyhead[CO]{{\footnotesize\textit{Translation of chapter 10}}}%

  \trchptr{ Chapter Seven }%

  \subchptr{kāyatīrthopavarṇanam}%

  \trsubchptr{The description of the pilgrimage places in the body}%

  \maintext{vigatarāga uvāca |}%

  \maintext{katamaṃ sarvatīrthānāṃ śreṣṭham āhur manīṣinaḥ |}%

  \maintext{kathayasva muniśreṣṭha yady asti bhuvi kāmadam }||\thinspace10:1\thinspace||%
\translation{Vigatarāga spoke: Which pilgrimage place {\rm (}\textit{tīrtha}{\rm )} do the wise consider the best of all? Tell me, O best of sages, if there is one in the world that fulfills [all] desires. }

  \maintext{anarthayajña uvāca |}%

  \maintext{atiguhyam idaṃ praśnaṃ pṛṣṭaḥ snehād dvijottama |}%

  \maintext{bravīmi vaḥ purāvṛttaṃ nandinā kathito 'smy aham }||\thinspace10:2\thinspace||%
\translation{Anarthayajña spoke: This question [that I have been] asked is an extremely deep secret. Out of fondness, O excellent Brahmin, I'll teach you an ancient legend that Nandi told me. }

  \maintext{nandikeśvara uvāca |}%

  \maintext{kailāsaśikhare ramye siddhacāraṇasevite |}%

  \maintext{tatrāsīnaṃ śivaṃ sākṣād devī vacanam abravīt }||\thinspace10:3\thinspace||%
\translation{Nandikeśvara spoke: On the beautiful peak of Mount Kailāsa, which is frequented by Siddhas and celestial singers {\rm (}\textit{cāraṇa}{\rm )}, Devī asked Śiva, who was sitting there in his manifest form. \blankfootnote{10.3 Note the change of speaker here: Nandikeśvara is also the main
  interlocutor of the \SDHS\ and the \SDHSANGR.
  This verse marks the beginning of the layer that can be labelled Śaiva.
  On Nandi/Nandin/Nandikeśvara not being Śiva's bull, see
  \mycite{bhattacharya_nandin_1977} and p. \pageref{nandi_not_bull} above.
 }}

  \maintext{devy uvāca |}%

  \maintext{bhagavan devadeveśa sarvabhūtajagatpate |}%

  \maintext{praṣṭum icchāmy ahaṃ tv ekaṃ dharmaguhyaṃ sanātanam }||\thinspace10:4\thinspace||%
\translation{Devī spoke: O Lord, Lord of the chiefs of the gods, O ruler of all beings and of all the world, I would like to ask you about an eternal secret concerning Dharma, \blankfootnote{10.4 It is not unlikely that in \textit{pāda} d, \textit{sanātanam} was intended to refer to
  \textit{dharma}° {\rm (}`eternal Dharma'{\rm )}, or that \textit{dharmaguhya} should be corrected
  to \textit{dharmaṃ guhyaṃ} {\rm (}`...ask you about a secret and eternal Dharma'{\rm )}.
 }}

  \maintext{atitīrthaṃ paraṃ guhyaṃ saṃsārād yena mucyate |}%

  \maintext{manuṣyāṇāṃ hitārthāya brūhi tattvaṃ maheśvara }||\thinspace10:5\thinspace||%
\translation{about the transcendental and highly secret pilgrimage place at which one can be liberated from mundane existence {\rm (}\textit{saṃsāra}{\rm )}. O Maheśvara, teach me the truth for the benefit of mankind. }

  \maintext{maheśvara uvāca |}%

  \maintext{ko māṃ pṛcchati taṃ praśnaṃ muktvā tvām eva sundari |}%

  \maintext{śṛṇu vakṣyāmi taṃ praśnaṃ devair api sudurlabham }||\thinspace10:6\thinspace||%
\translation{Maheśvara spoke: Who else could ask me that question except for you, O Sundarī? Listen, I shall expound that question which is difficult to grasp even for the gods. \blankfootnote{10.6 Although most witnesses consulted treat \textit{praśna} as neuter, and it can 
  be original, it could as well be just a minor error transmitted widely.
  This is why I have chosen \msNc's reading {\rm (}\textit{taṃ praśnaṃ}{\rm )}.
 }}

  \maintext{kurukṣetraṃ prayāgaṃ ca vārāṇasīm ataḥ param |}%

  \maintext{gaṅgāgniṃ somatīrthaṃ ca sūryapuṣkaramānasam }||\thinspace10:7\thinspace||%
\translation{If one gets to know Kurukṣetra, Prayāga, Vārāṇasī, Gaṅgā, Agni[tīrtha], Somatīrtha, Sūrya[tīrtha], Puṣkara, Mānasa, }

  \maintext{naimiṣaṃ bindusāraṃ ca setubandhaṃ suradraham |}%

  \maintext{ghaṇṭikeśvaravāgīśaṃ jñātvā niścayapāpahā }||\thinspace10:8\thinspace||%
\translation{Naimiṣa, Bindusaras, Setubandha, Suradraha, Ghaṇṭikeśvara, and Vāgīśa, one will certainly be able to destroy one's sins. \blankfootnote{10.8 Note \textit{bindusāraṃ} for \textit{bindusaras/°saraṃ/°sarasaṃ} metri causa.
 Although some of these toponyms are difficult to identify and some may refer to
  southern locations {\rm (}e.g. Setubandha{\rm )}, in general they suggest a North Indian focus.
  See details on pp. CHECK
 }}

  \maintext{umovāca |}%

  \maintext{evamādi mahādeva pūrvavat kathitāsmy aham |}%

  \maintext{svargabhogapradaṃ tīrtham eteṣāṃ suranāyaka }||\thinspace10:9\thinspace||%
\translation{Umā spoke: I have been taught this previously, O Mahādeva. [Which is] the pilgrimage place that yields all kinds of enjoyment, O Suranāyaka? \blankfootnote{10.9 I take \textit{pūrvavat} in \textit{pāda} b as if used in the sense of \textit{pūrvaṃ} {\rm (}`previously'{\rm )},
  and \textit{eteṣāṃ} in \textit{pāda} d as \textit{eteṣu}. It would also be possible to take \textit{eteṣāṃ}
  in 10.9d and \textit{jñānamātreṇa} in 10.10b as connected {\rm (}`by the 
  mere knowledge of them'{\rm )} but the former solution seems to work better 
  with 10.14, where again a genitive {\rm (}\textit{teṣāṃ}{\rm )} may stand for a locative {\rm (}\textit{teṣu}{\rm )}.
 }}

  \maintext{kathaṃ mucyeta saṃsārāj jñānamātreṇa īśvara |}%

  \maintext{kautūhalaṃ mahaj jātaṃ chindhi saṃśayakārakam }||\thinspace10:10\thinspace||%
\translation{[And] how is one liberated from mundane existence by merely knowing [the pilgrimage places], O Īśvara? Cut [this] great curiosity arising [in me] that causes doubt. \blankfootnote{10.10 We are forced to agree with \Ed's printing °\textit{kārakam} in \textit{pāda} d because
  all the other readings seem out of context, whether they refer to Śiva as a vocative or 
  a nominative.
 }}

  \maintext{rudra uvāca |}%

  \maintext{kiṃ na jānāmi tat tīrthaṃ sulabhaṃ durlabhaṃ ca yat |}%

  \maintext{sulabhaṃ gurusevīnāṃ durlabhaṃ tad vivarjayet }||\thinspace10:11\thinspace||%
\translation{Rudra spoke: How could I not know [the difference between] that pilgrimage place that is easy to reach and that which is difficult to reach? It is easy to reach for those who serve their guru. One can abandon the one that is difficult to reach. \blankfootnote{10.11 My translation here is slightly tentative and is fashioned to make sense
  in light of what is coming: the praise of internalised pilgrimage places,
  contrasting them with real, external pilgrimage places.
 }}

  \subsubchptr{kurukṣetram}%

  \trsubsubchptr{Kurukṣetra}%

  \maintext{kuruḥ puruṣa vijñeyaḥ śarīraṃ kṣetra ucyate |}%

  \maintext{śarīrasthaṃ kurukṣetraṃ sarvatīrthaphalapradam }||\thinspace10:12\thinspace||%
\translation{\textit{kuru} [in Kurukṣetra] is to be known as the soul {\rm (}\textit{puruṣa}{\rm )}, \textit{kṣetra} as the body. Kurukṣetra is in the body, and it yields the fruits of [visiting] all pilgrimage places. \blankfootnote{10.12 In \textit{pāda} b, one could apply \msNa's reading that has the standard neuter nominative form \textit{kṣetram}
  as opposed to the form transmitted in all other witnesses {\rm (}\textit{kṣetra}{\rm )} but
  the latter might be original, influenced by the stem form \textit{puruṣa} in \textit{pāda} a.
 }}

  \maintext{sarvayajñaphalāvāptiḥ sarvadānaphalāni ca |}%

  \maintext{sarvavratatapaś cīrṇaṃ tatphalaṃ sakalaṃ bhavet }||\thinspace10:13\thinspace||%
\translation{[And there will be] the obtaining of the fruits of all sacrifices, the fruits of all [possible] donations, and all the fruits of all religious observances and penance observed. }

  \maintext{evam eva phalaṃ teṣāṃ tīrthapañcadaśeṣu ca |}%

  \maintext{anaghānaṃ mahāpuṇyaṃ mahātīrthaṃ mahāsukham }||\thinspace10:14\thinspace||%
\translation{This is how the fruits [are said to be also] in the case of those fifteen pilgrimage places [from Kurukṣetra to Vāgīśa]. [Kurukṣetra,] the great and faultless pilgrimage place is extremely auspicious and pleasant. \blankfootnote{10.14 \textit{anaghānaṃ} in \textit{pāda} c is problematic. It may simply stand for \textit{anaghaṃ} {\rm (}`faultless'{\rm )}.
  That is how I translate it. Originally it may have involved a stem form adjective:
  \textit{anaghaitan} {\rm (}\textit{anagha + etad}{\rm )}.
 }}

  \maintext{devy uvāca |}%

  \maintext{atīva romaharṣo me jāto 'sti tridaśeśvara |}%

  \maintext{sulabhaṃ sukaraṃ sūkṣmaṃ śrutvā tuṣṭiś ca me gatā }||\thinspace10:15\thinspace||%
\translation{Devī spoke: I am extremely thrilled, O Tridaśeśvara. Hearing about that which is easy to obtain, easy to perform and is subtle, my contentment has left me [that is, I want to hear more]. }

  \maintext{caturdaśa paro bhūyaḥ kathayasva manoharam |}%

  \maintext{prayāgādi pṛthaktvena tattvatas tu sureśvara }||\thinspace10:16\thinspace||%
\translation{Teach me further about the remaining fourteen pleasant [pilgrimage places], Prayāga and the others, one by one, as they really are, O Sureśvara. \blankfootnote{10.16 Note again the use of the singular next to numbers {\rm (}\textit{caturdaśa... manoharam prayāgādi}{\rm )},
  a frequent phenomenon in this text.
 }}

  \subsubchptr{prayāgo vārāṇasī ca}%

  \trsubsubchptr{Prayāga and Vārāṇasī}%

  \maintext{rudra uvāca |}%

  \maintext{suṣumnā bhagavatī gaṅgā iḍā ca yamunā nadī |}%

  \maintext{etāḥ srotovahā nadyaḥ prayāgaḥ sa vidhīyate }||\thinspace10:17\thinspace||%
\translation{The Suṣumnā[-tube] is the Honourable Gaṅgā, Iḍā[-tube] is the river Yamunā. [At the confluence of] these rivers is [the pilgrimage place] called Prayāga. \blankfootnote{10.17 There seems to be only two yogic tubes mentioned here {\rm (}and in 10.20--21{\rm )}: 
  Suṣumnā and Iḍā, instead of the more usual triad of Iḍā, Piṅgalā, and Suṣumnā. This is strikingly similar to
  what we see in the archaic yoga of the \NISVNAYA, see \mycitep{NisvasaGoodall}{33--34}.
  According to \mycitep{BaroisDhP}{23 and 46} the case is similar in the \DHARMP.
  This is slightly doubtful because a third tube, called Turyā, is mentioned immediately after
  Iḍā and Suṣumnā in \DHARMP\ 4.57: 
  \textit{iḍā vāmā suṣumnā ca dve nāḍī nāsikāśrite\thinspace |
  bhruvor madhye parā nāḍī tajjñais turyeti kīrttitā\thinspace ||}.
  It is also possible that the third tube is there, as Prayāga, in our obscure \VSS\ 10.17cd,
  which may want to say that at the confluence of the Gaṅgā/Suṣumnā and the Yamunā/Iḍā,
  there is the internalised pilgrimage place, or tube, called Prayāga.
  Compare \MBH\ Indices 6.3A.41--44:
  \textit{iḍā bhagavatī gaṅgā piṅgalā yamunā nadī\thinspace |
  tayor madhye tṛtīyā tu tat prayāgam anusmaret\thinspace ||
  iḍā vai vaiṣṇavī nāḍī brahmanāḍī tu piṅgalā\thinspace |
  suṣumṇā caiśvarī nāḍī tridhā prāṇavahā smṛtā\thinspace ||}.
  Note that Yamunā has not been mentioned as a \textit{tīrtha} in the \VSS.
  See also \HYP\ 3.110:
  \textit{iḍā bhagavatī gaṅgā piṅgalā yamunā nadī\thinspace |
  iḍāpiṅgalayor madhye bālaraṇḍā ca kuṇḍalī\thinspace ||}.
  
  Note also \Ed's attempt to make \textit{pāda} a metrical.
 }}

  \maintext{dakṣiṇā vāruṇī nāsā vāmanāsā asi smṛtā |}%

  \maintext{vāruṇā-asimadhyena tena vārāṇasī smṛtā }||\thinspace10:18\thinspace||%
\translation{The right nostril is [the river] Vāruṇī, the left nostril is known as [the river] Asi. Because [it is] at the confluence of Vāruṇā and Asi, [the city/internalised pilgrimage place there] is known as Vārāṇasī. \blankfootnote{10.18 This verse most probably describes the spot between the eyebrows as an
  internalised pilgrimage place.
 }}

  \subsubchptr{gaṅgā}%

  \trsubsubchptr{The Gaṅgā}%

  \maintext{ākāśagaṅgā vikhyātā tasyāḥ sravati cāmṛtam |}%

  \maintext{ahorātram avicchinnaṃ gaṅgā sā tena ucyate }||\thinspace10:19\thinspace||%
\translation{She is called the ethereal Gaṅgā, and the nectar of immortality issues from her day and night uninterruptedly. That's why [this internalised pilgrimage place] is called Gaṅgā. \blankfootnote{10.19 This verse may describe a bodily location such as the soft palate as an
  internalised pilgrimage place.
 The word \textit{gaṅga} is interpreted here as an intensive form from the root\verbalroot{\textit{gam}},
  related to the better-attested intensive stems \textit{jaṅgam} and \textit{ganīgam} {\rm (}see the latter two, e.g., in 
  \mycitep{WhitneyGrammar}{§1003}{\rm )}.
 }}

  \subsubchptr{somatīrtham}%

  \trsubsubchptr{Somatīrtha}%

  \maintext{somatīrtham iḍā nāḍī kiṅkiṇīravacihnitā |}%

  \maintext{taṃ tu śrutvā na saṃdehaḥ sarvapāpakṣayo bhavet }||\thinspace10:20\thinspace||%
\translation{Somatīrtha is the tube Iḍā. It is characterised by the ringing of small bells. Upon hearing that [ringing], all of one's sins will be destroyed. \blankfootnote{10.20 Note that Iḍā has already been identified as the Yamunā in 10.17b.
 }}

  \subsubchptr{sūryatīrtham}%

  \trsubsubchptr{Sūryatīrtha}%

  \maintext{sūryatīrthaṃ suṣumnā ca nīravāravasaṃyutā |}%

  \maintext{śrutimātrād vimucyeta pāparāśir mahān api }||\thinspace10:21\thinspace||%
\translation{Sūryatīrtha is the [tube] Suṣumnā, the one that has a soundless thundering. By merely hearing about it one is liberated, even if one has a huge heap of sins. \blankfootnote{10.21 Suṣumnā has already been identified as the the Gaṅgā in 10.17a.
  \textit{nīravā-rava} in \textit{pāda} b probably stands for \textit{nīrava-rava} metri causa.
 }}

  \subsubchptr{agnitīrtham}%

  \trsubsubchptr{Agnitīrtha}%

  \maintext{agnitīrthārjunā nāḍī brahmaghoṣamanoramā |}%

  \maintext{tat tad akṣaram ākarṇya amṛtatvāya kalpate }||\thinspace10:22\thinspace||%
\translation{Agnitīrtha is the Arjuna tube. It is charming because of the hum of Veda recitation. Upon hearing this or that syllable, one will become immortal. \blankfootnote{10.22 CHECK Arjunā nāḍī
 }}

  \subsubchptr{puṣkaram}%

  \trsubsubchptr{Puṣkara}%

  \maintext{puṣkaraṃ hṛdi madhyastham aṣṭapattraṃ sakarṇikam |}%

  \maintext{cintayet sūkṣma tanmadhye janmamṛtyuvināśanam }||\thinspace10:23\thinspace||%
\translation{Puṣkara is a lotus with eight petals and a pericarp in the centre of the heart. One should visualize the Subtle One in its centre [and] it'll destroy birth and death. \blankfootnote{10.23 \textit{hṛdi} might be meant to be a nominative, as in 12.17, here potentially compounded with \textit{madhyastham}.
 On \textit{sūkṣma} {\rm (}here in stem form metri causa{\rm )}, see CHECK.
 }}

  \subsubchptr{mānasam}%

  \trsubsubchptr{Mānasa}%

  \maintext{mānasasaramadhyasthaṃ sa haṃsaḥ kamalopari |}%

  \maintext{salīlo līlayācārī parataḥ parapāragaḥ }||\thinspace10:24\thinspace||%
\translation{That goose on a lotus in the middle of the Mānasa lake is playful, acting gracefully, rising far beyond the other shore. \blankfootnote{10.24 Understand \textit{mānasasara}° in \textit{pāda} a as \textit{mānasasaro}° {\rm (}metri causa{\rm )}.
  To make sense of this verse, especially the masculine nominatives in
  \textit{pāda}s cd, I have conjectured \textit{sa haṃsaḥ} for what seems to 
  a compound: \textit{sahaṃsakamalopari}. I suspect \textit{pāda} a to qualify, clumsily,
  \textit{kamala} in \textit{pāda} b. Other possibilities include \textit{sahaṃsa}° meaning
  `with the syllables HAṂ and SA on it.' 
 
  The association of Lake Mānasa on Mount Kailāsa with lotuses, and especialy with geese or swans, is
  well-known. See, e.g., \MBH\ 6.114.90ff: Gaṅgā sends the great sages, who inhabit
  Lake Mānasa in the form of geese, to visit the dying Bhīṣma.
  Although the interpretation of this verse, which obviously refers to an internalised
  form of this pilgrimage place, is still problematic, the goose/swan
  most probably signifies to the soul.
 }}

  \subsubchptr{naimiṣam}%

  \trsubsubchptr{Naimiṣa}%

  \maintext{naimiṣaṃ śṛṇu deveśi nimiṣā pratyayo bhavet |}%

  \maintext{samyag chāyāṃ nirīkṣeta ātmāno vā parasya vā }||\thinspace10:25\thinspace||%
\translation{Listen to Naimiṣa, O Deveśī. It yields assurance in a moment. One can observe the shadow of one's own and others' soul properly. \blankfootnote{10.25 This obscure verse {\rm (}coupled with the next one{\rm )} might have something to do with a type of
  meditation, \textit{chāyādhyāna}, mentioned in \NISVUTTARA\ 5.6:
  \textit{tattvadhyānaṃ prathamakaṃ chāyādhyānaṃ dvitīyakam\thinspace |
  ghoṣadhyānan tṛtīyan tu lakṣadhyānañ caturthakam\thinspace ||}.
  Later on in the same text {\rm (}5.12 and 16{\rm )}, this meditation on `the shadow of the 
  soul/\textit{puruṣa}' is mentioned again.
  \NISVUTTARA\ 5.16 states that 
  `Focussing on[?] one's awareness on [one's] "shadow" {\rm (}\textit{chāyācittam}{\rm )},
  one will see the soul {\rm (}\textit{pumān} = \textit{pumāṃsam}?{\rm )} in the sky {\rm (}\textit{viyatstham}{\rm )}.
  Practising in this way, one attains success and becomes Śiva.' 
  {\rm (}tr. in \mycitep{NisvasaGoodall}{391};
  \textit{chāyācittaṃ samālambya viyatsthaṃ paśyate pumān\thinspace |
  evam abhyasyamānas tu siddhyate ca śivo bhavet\thinspace ||}.
  But as the editors of the \NISV\ put it with reference to the
  four elements of meditation given there:
  `Very little of this is clear and almost nothing is certain'
  {\rm (}\mycitep{NisvasaGoodall}{389}{\rm )}.
 }}

  \maintext{āyatam aṅgulīmātraṃ nimiṣākṣiḥ sa paśyati |}%

  \maintext{dṛṣṭvā pratyayam evaṃ hi naimiṣajñaḥ sa ucyate }||\thinspace10:26\thinspace||%
\translation{He will see [the soul's] length with his eyes shut as one finger-breath. When one has seen the proof thus, one is called the knower of Naimiṣa. \blankfootnote{10.26 \textit{Pāda}s ab involve an emendation and a conjecture, without which it is
  difficult to understand this line.
 }}

  \subsubchptr{bindusaraḥ}%

  \trsubsubchptr{Bindusaras}%

  \maintext{tīrthaṃ bindusaraṃ nāma śṛṇu vakṣyāmi sundari |}%

  \maintext{dehamadhye hṛdi jñeyaṃ hṛdimadhye tu paṅkajam }||\thinspace10:27\thinspace||%
\translation{Listen O Sundarī, I shall teach you the pilgrimage place called Bindusaras. The heart is to be known to be located in the centre of the body. In the centre of the heart, there is a lotus. \blankfootnote{10.27 Understand °\textit{saraṃ} in \textit{pāda} a as °\textit{saro} {\rm (}thematisation{\rm )}.
 Take \textit{hṛdi} as a nominative in \textit{pāda} c and possibly also in \textit{pāda} d {\rm (}and see 10.23a{\rm )}.
 }}

  \maintext{karṇikā padmamadhye tu binduḥ karṇikamadhyataḥ |}%

  \maintext{bindumadhye sthito nādaḥ sa nādaḥ kena bhidyate }||\thinspace10:28\thinspace||%
\translation{There is a pericarp in the centre of the lotus, and the subtle sonic matter {\rm (}\textit{bindu}{\rm )} in the centre of the pericarp. In the centre of the subtle sonic matter {\rm (}\textit{bindu}{\rm )}, there is the subtle sound {\rm (}\textit{nāda}{\rm )}. How is that subtle sound {\rm (}\textit{nāda}{\rm )} divided? \blankfootnote{10.28 For a general discussion on \textit{nāda} and \textit{bindu}, see, e.g., \TAKIII\ s.v. \textit{nāda}.
  Our text considers the internalised manifestation of the pilgrimage place Bindusaras
  to be \textit{bindu}, or subtle sonic matter.
 }}

  \maintext{ukāraṃ ca makāraṃ ca bhittvā nādo vinirgataḥ |}%

  \maintext{taṃ viditvā viśālākṣi so 'mṛtatvaṃ labheta ca }||\thinspace10:29\thinspace||%
\translation{The subtle sound {\rm (}\textit{nāda}{\rm )} departs divided by the sounds U and M. Realizing that [subtle sound], O Viśālākṣi, one can obtain immortality. \blankfootnote{10.29 \VSS\ 10.27--29ab seem to paraphrase \NISVK\ 5.55--57ab.
 }}

  \subsubchptr{setubandham}%

  \trsubsubchptr{Setubandha}%

  \maintext{vakṣye te setubandhaṃ duritamalaharaṃ nādatoyapravāhaṃ}%

 \nonanustubhindent \maintext{jihvākaṇṭhorakūlā svaragaṇapulināvartaghoṣā taraṅgā |}%

  \maintext{kumbhīrāghoṣamīnā daśagaṇamakarā bhīmanakrā visargā}%

 \nonanustubhindent \maintext{sānusvāre gabhīre madasukharasanaṃ setubandhaṃ vrajasva }||\thinspace10:30\thinspace||%
\translation{I shall teach you Setubandha, which sports a current whose water of subtle sound {\rm (}\textit{nāda}{\rm )} cleanses you of the dirt of your sins. [It is a river whose] banks are the tongue, the throat, and the chest, and its sandbanks are the group of vowels {\rm (}\textit{svara}{\rm )}. It is wavy because its whirlpools are the voiced consonants {\rm (}\textit{ghoṣa}{\rm )}. Voiceless consonants {\rm (}\textit{aghoṣa}{\rm )} are its crocodiles and fish, the ten verbal classes {\rm (}\textit{gaṇa}{\rm )} are its sea-monsters, \textit{visarga}s are its terrifying alligators. It is in the deep-sounding \textit{anusvāra} {\rm (}\textit{sā-anusvāre}{\rm )}. Go to Setubandha, have a taste of the pleasure of intoxication. \blankfootnote{10.30 Note that °\textit{kaṇṭhora}° is a conjecture based on the context: this line
  talks about sounds and the production of sounds. For this 
  \textit{uraḥ}/\textit{ura} {\rm (}`chest'{\rm )} seems better that \textit{ūru} {\rm (}`thigh'{\rm )}.
  It is not evident at first sight why \textit{pāda}s b and c stick to feminine endings. I take this
  as qualifying an implied \textit{nadī}, partly because the similarly structured 10.33 below
  explicitely mentions \textit{nadī}. Some of the compounds here are inverted or split:
  understand \textit{āvartaghoṣā taraṅgā} as \textit{ghoṣāvartataraṅgā}, 
  \textit{kumbhīrāghoṣamīnā} as \textit{aghoṣakumbhīramīnā}, and 
  \textit{bhīmanakrā visargā} as \textit{visargabhīmanakrā}.
 Nevertheless, the general idea seems to be clear: the internalised
  version of the pilgrimage place Setubandha, externally usually understood as 
  Rameśvara in the South, is now the sounds of recitation.
 }}

  \subsubchptr{suradrahaḥ}%

  \trsubsubchptr{Suradraha}%

  \maintext{saptadvīpāntamadhye śṛṇu śaśivadane sarvaduḥkhāntalābham}%

 \nonanustubhindent \maintext{īśānenābhijuṣṭaṃ hṛdi hrada vimalaṃ nādaśītāmbupūrṇam |}%

  \maintext{tatraikaṃ jātapadmaṃ prakṛtidalayutaṃ keśaraṃ śaktibhinnaṃ}%

 \nonanustubhindent \maintext{pañcavyomapraśastaṃ gatiparamapadaṃ prāptukāmena sevyam }||\thinspace10:31\thinspace||%
\translation{O Moon-faced goddess, listen to [the description of Suradraha], the way to the cessation of all sorrow, in the centre of the seven islands. It is frequented by Īśāna, a spotless lake in the heart full of the cool water of sound {\rm (}\textit{nāda}{\rm )}. There is a lotus arising there whose petals are Prakṛti and whose filaments are split between Śaktis, praised as the five gross elements {\rm (}\textit{vyoman}{\rm )}. It is to be honoured if one wishes to obtain the path to the supreme abode. \blankfootnote{10.31 The first syllable of \textit{hrada} in \textit{pāda} b does not make the previous syllable long {\rm (}`muta cum liquida'
  licence{\rm )}, otherwise the line would be unmetrical. Understand the same \textit{hrada} as a stem form metri causa
  standing for the accusative.
 \textit{keśaraṃ śaktibhinnaṃ} in \textit{pāda} c should probably be understood as a bahuvrīhi compound
  thus: \textit{śaktibhinnakeśaraṃ}.
 For \textit{vyoman} as `gross element,' see notes to \VSS\ 4.32 above, but note
  that the expression `fifty voids' {\rm (}\textit{pañcāśadvyoman}{\rm )} also comes up in 
  \VSS\ 20.7 and also in 10.33 below. It is not clear why this internalised pilgrimage place,
  or the filaments of the lotus mentioned, would be praised as the five elements.
 }}

  \subsubchptr{ghaṇṭikeśvaram}%

  \trsubsubchptr{Ghaṇṭikeśvara}%

  \maintext{{\rm †}nāḍyaikāsaṅgatāni{\rm †} nipatitam amṛtaṃ ghaṇṭikāpārakeṇa}%

 \nonanustubhindent \maintext{tṛpyante tena nityaṃ hṛdi kamalapuṭaṃ sthāṇubhūtāntarātmā |}%

  \maintext{yaṃ paśyantīśabhaktāḥ kalikaluṣaharaṃ vyāpinaṃ niṣprapañcaṃ}%

 \nonanustubhindent \maintext{deveśaṃ ghaṇṭikeśāmarabhavam abhavaṃ tīrtham ākāśabindum }||\thinspace10:32\thinspace||%
\translation{The tubes join[?]. The nectar of immortality {\rm (}\textit{amṛta}{\rm )} has descended by the Saviour Ghaṇṭikā. Those whose inner selves have become Sthāṇu [i.e. Śiva] are continuously delighted in Him, as he is embraced by the lotus in their hearts. [He is the one] whom Īśa's devotees can behold, who drives off the impurity of the Kali age, who is all-pervading {\rm (}\textit{vyāpin}{\rm )} and non-manifest {\rm (}\textit{niṣprapañca}{\rm )}, the lord of gods, Ghaṇṭikeśa of undying existence. The \ae rial \textit{bindu} is a non-mundane {\rm (}\textit{abhava}{\rm )} pilgrimage place. \blankfootnote{10.32 The interpretation of this verse is not without problems. The cruxed expression
  in \textit{pāda} a is difficult to repair; it may involve \textit{nāḍī} or \textit{nāḍyā}, \textit{ekā}, and 
  \textit{saṃgata}. These suggest that it may hint at a point of confluence where the bodily tubes {\rm (}\textit{nāḍī}{\rm )}
  join. {\rm (}Possibly understand \textit{nāḍya ekasaṃgatāḥ}.{\rm )}
 In \textit{pāda} b, \textit{sthāṇu} is my conjecture for \textit{sthānu}, and 
  I understand °\textit{ātmā} as standing for the plural nominative.
 I take \textit{ghaṇṭikeśa} in \textit{pāda} d as a stem form noun in sandhi with \textit{amara}°,
  notwithstanding the reading \textit{ghaṇṭikeśamara}° in \msCa\msCb\msNb\msNc.
 
  The external pilgrimage place related to Ghaṇṭikeśvara the \VSS\ has in mind here may or may not be 
  `Virajā, modern Jajpur in the Cuttack District of Orissa' presided over by
  Ghaṇṭīśa, Mahāghaṇṭeśvara or Mahāghaṇṭa Bhairava {\rm (}\mycitep{SandersonSaivaAge}{113, n. 241}{\rm )}.
  See Introduction \verify.
 
  As for the yogic interpretation of this verse, it seems plausible that \textit{ghaṇṭikā} is
  taken here as the uvula, from which \textit{amṛta} is said to be dripping down the throat.
  See \TAKII\ s.v. \textit{ghaṇṭikā} and \mycite{MallinsonKhecari}.
 }}

  \subsubchptr{vāgīśvaratīrtham}%

  \trsubsubchptr{Vāgīśvaratīrtha}%

  \maintext{mīmāṃsāratnakūlā kramapadapulinā śaivaśāstrārthatoyā}%

 \nonanustubhindent \maintext{mīnaughā pañcarātraṃ śrutikuṭilagatiḥ smārtavegā taraṅgā |}%

  \maintext{yogāvartātiśobhā upaniṣadivahā bhāratāvartaphenā}%

 \nonanustubhindent \maintext{pañcāśadvyomarūpī rasabhavananadī tīrtha vāgīśvarīyam }||\thinspace10:33\thinspace||%
\translation{The banks [of Vāgīśvaratīrtha] are the gems of Mīmāṃsā, its sandbanks the [Vedic] \textit{kramapada}s, its water the meaning of the Śaiva manuals. Its flock of fish is the Pañcarātra [tradition], its winding path is the Śruti [tradition], its rapid waves the Smārta [tradition]. It is beautiful with its whirlpools of yoga. Its currents are the Upaniṣads. The foam made by its whirlpools is the Mahābhārata. This river, whose form is the fifty voids {\rm (}\textit{vyoman}{\rm )}, is the abode of the elixir. [This is the description of] the pilgrimage place Vāgīśvara. \blankfootnote{10.33 \textit{kramapada} {\rm (}better known as \textit{padakrama}{\rm )} is a particular method
  of reciting Vedic texts. \verify\ REF
 Note the split compounds in \textit{pāda} b. Understand \textit{mīnaughā pañcarātraṃ} as
  \textit{pañcarātramīnaughā}, and \textit{smārtavegā taraṅgā} as \textit{smārtavegataraṅgā}.
 Note the form \textit{upaniṣadi} for a stem form of \textit{upaniṣad} in \textit{upaniṣadi-vahā} in
  \textit{pāda} c. This phenomenon is similar to what we see in 10:23 and 27 above with \textit{hṛdi}.
  The lack of sandhi between °\textit{śobhā} and \textit{upaniṣadi}° is also notable.
 \textit{tīrtha} in \textit{pāda} d is a stem form noun metri causa. The exact meaning of
  \textit{pañcāśadvyoma}° is not clear to me. Could it be the fifty sounds of Sanskrit?
  All in all, Vāgīśvaratīrtha here represents the religious traditions and scriptures.
 }}

  \maintext{yas taṃ vetti sa vetti vedanikhilaṃ saṃsāraduḥkhacchidaṃ}%

 \nonanustubhindent \maintext{janmavyādhiviyogatāpamaraṇaṃ kleśārṇavaṃ duḥsaham |}%

  \maintext{garbhāvāsam atīva sahyaviṣayaṃ dustīryaduḥkhālayaṃ}%

 \nonanustubhindent \maintext{prāptaṃ tena na saṃśayaḥ śivapadaṃ duṣprāpya devair api }||\thinspace10:34\thinspace||%
\translation{One will know all the Vedas by knowing Him who puts an end to the suffering of transmigration, to birth, disease, separation, suffering, death, the floods of unbearable pain, to dwelling in the womb, to extremely insufferable sensations, and to places of suffering that are difficult to escape from. Such a person will, without doubt, reach Śiva's world that is difficult to enter even for the gods. \blankfootnote{10.34 I take \textit{pāda}s b and c as if °\textit{chidaṃ} in \textit{pāda} a were implied for
  each element there,
 and \textit{atīva sahya}° as standing for \textit{atīvāsahya}° metri causa.
 Understand \textit{duṣprāpya} as a stem form adjective {\rm (}for \textit{duṣprāpyaṃ}{\rm )} metri causa.
 }}
\center{\maintext{\dbldanda\thinspace iti vṛṣasārasaṃgrahe kāyatīrthopavarṇano nāmādhyāyo daśamaḥ\thinspace\dbldanda}}
\translation{Here ends the tenth chapter in the \textit{Vṛṣasārasaṃgraha} called the Description of the bodily pilgrimage places}

  \chptr{ekādaśamo 'dhyāyaḥ}
\addcontentsline{toc}{subsection}{Chapter 11}
\fancyhead[CO]{{\footnotesize\textit{Translation of chapter 11}}}%

  \trchptr{ Chapter Eleven }%

  \subchptr{caturāśramadharmavidhānaḥ}%

  \trsubchptr{The regulations on the Dharma of the four āśramas}%

  \maintext{devy uvāca |}%

  \maintext{sarvayajñaḥ paraśreṣṭha asti anyaḥ surottama | }%

  \maintext{alpakleśa{-}m{-}anāyāsa arthaprāyaṃ vineśvara }||\thinspace11:1\thinspace||%
\translation{The Goddess spoke: O Paraśreṣṭha, O Surottama! Is there another [form of] sacrifice that is for all {\rm (}\textit{sarvayajña}{\rm )}, which is free of pain, is easy, and which does not require an abundance of materials, O Īśvara? \blankfootnote{11.1 I understand \textit{pāda} c as containing a sandhi bridge thus: \textit{alpakleśa-m-anāyāsa}.
  The sandhi between \textit{pāda}s c and d is irregular, understand °\textit{anāyāsaḥ artha}°, or rather
  °\textit{anāyāso 'rtha}°.
 }}

  \maintext{sarvayajñaphalāvāpti daivataiś cāpi pūjitam |}%

  \maintext{kathayasva suraśreṣṭha mānuṣāṇāṃ hitāya vai }||\thinspace11:2\thinspace||%
\translation{For the benefit of mankind, teach me, O Suraśreṣṭha, how one obtains the fruits of [this] universal sacrifice, [a sacrifice] praised even by the gods. \blankfootnote{11.2 The beginning of the \SDHS\ {\rm (}\SDHS\ 1.7--11{\rm )} expresses a similar sentiment,
  using the word \textit{āyāsa}, similarly to \VSS\ 11.1c above, but giving a somewhat clearer 
  reason for asking for a new form of devotion, namely that twice-born members of society
  with limited financial resources struggle to perform 
  expensive Vedic rituals {\rm (}\textit{na śakyante yataḥ kartim alpavittair dvijātibhiḥ}{\rm )}:
  \textit{sanatkumāra uvāca\thinspace | 
  bhagavan sarvadharmajña śivadharmaparāyaṇaḥ\thinspace |
  śrotukāmāḥ paraṃ dharmam imaṃ sarve samāgatāḥ\thinspace || 
  agniṣṭomādayo yajñā bahuvittakriyānvitāḥ\thinspace |
  nātyantaphalabhūyiṣṭhā bahvāyāsasamanvitāḥ\thinspace ||
  na śakyante yataḥ kartum alpavittair dvijātibhiḥ\thinspace |
  sukhopāyam ato brūhi sarvakāmārthasādhakam\thinspace || 
  hitāya sarvasatvānāṃ śivadharmaṃ sanātanam\thinspace |
  nandikeśvara uvāca\thinspace |
  śrūyatām abhidhāsyāmi sukhopāyamahatphalam\thinspace ||
  paramasarvadharmāṇāṃ śivadharmaṃ śivātmakam\thinspace |
  śivena kathitaṃ pūrvaṃ pārvatyāḥ ṣaṇmukhasya ca\thinspace ||.}
 }}

  \maintext{maheśvara uvāca |}%

  \maintext{na tulyaṃ tava paśyāmi dayā bhūteṣu bhāmini |}%

  \maintext{kim anyat kathayiṣyāmi dayā yatra na vidyate }||\thinspace11:3\thinspace||%
\translation{Maheśvara spoke: I cannot see anything comparable to your compassion towards living beings, O Bhāminī. What else could I teach [you] with respect to which [your] compassion is not evident? \blankfootnote{11.3 I understand \textit{dayā} in \textit{pāda} b as instrumental: \textit{tava dayayā bhūteṣu na tulyaṃ paśyāmi}.
  Alternatively, as suggested by Csaba Dezső, \textit{pāda}s ab could be interpreted as
  two sentences: `I cannot see anything comparable to you. [You have great]
  compassion towards living beings, O Bhāminī.'
 }}

  \maintext{sadāśivamukhāt pūrvaṃ śrutaṃ me varasundari |}%

  \maintext{śṛṇu devi pravakṣyāmi dharmasāram anuttamam }||\thinspace11:4\thinspace||%
\translation{I heard [the following] previously from Sadāśiva's mouth, O Varasundarī. Listen, O Goddess, I shall teach you the ultimate essence of Dharma. \blankfootnote{11.4 Note \textit{me} for \textit{mayā} in \textit{pāda} b, and the evident distinction here between Maheśvara,
  the interlocutor, and Sadāśiva, who, in this context seems to be superior, being the 
  ultimate source here of the following teaching. This might hint at a familiarity with 
  the Tantric sequence of \textit{tattva}s, on which see, e.g., \mycitep{NisvasaGoodall}{45}.
 }}

  \subchptr{gṛhasthaḥ{\rm (}?{\rm )}}%

  \trsubchptr{The householder{\rm (}?{\rm )}}%

  \maintext{vinārthena tu yo yajñaḥ sa yajñaḥ sārvakāmikaḥ |}%

  \maintext{akṣayaś cāvyayaś caiva sarvapātakanāśanaḥ }||\thinspace11:5\thinspace||%
\translation{Sacrifice which [is performed] without materials satisfies all desires. It is undecaying and imperishable, and it removes all sins. \blankfootnote{11.5 I put a question mark after the subchapter heading here because in this
  chapter the category of the \textit{gṛhastha} never gets mentioned. It is simply labelled 
  \textit{āśramaḥ prathamaḥ} in 11.25a. Nevertheless, the category \textit{gṛhastha} is most probably
  implied and elsewhere mentioned {\rm (}see 4.74c, 5.9a, and 15.17a, which reads \textit{āśramāṇāṃ gṛhī śreṣṭho}{\rm )}. 
  The teaching on sacrifice without materials {\rm (}\textit{vinārthena yajñaḥ} or \textit{anarthayajñaḥ}{\rm )},
  which is fundamentally internalised sacrifice, is a central teaching of the \VSS:
  in addition to the present chapter, the expression 
  appears as the main interlocutor's name {\rm (}Anarthayajña{\rm )} in chapters 1--9 and 19--21, 
  and his life is discussed in chapter 22. Thus the name Anarthayajña or the concept of \textit{anarthayajña}
  appears in each major layer of the text. On this see p. \verify, and \mycite{KissVolume2021}.
 }}

  \maintext{bahuvighnakaro hy artho bahvāyāsakaras tathā |}%

  \maintext{brahmahatyā ivendrasya pravibhāgaphalā smṛtā }||\thinspace11:6\thinspace||%
\translation{Material things {\rm (}\textit{artha}{\rm )} present many kinds of obstacle and [their acquisition causes] great fatigue, similarly to Indra's murder of the Brahmin [Viśvarūpa], which yielded results [i.e. sins] that were distributed [among trees, lands etc.]. \blankfootnote{11.6 The context of \textit{pāda}s cd is this: Viśvarūpa was a son of Tvaṣṭṛ. 
  Viśvarūpa's heads were struck off by Indra and Indra's sin were distributed among the ground, 
  water, trees and women. See e.g.\ \BHAGP\ 6.9.6:
  \textit{brahmahatyām añjalinā jagrāha yad apīśvaraḥ\thinspace |
  saṃvatsarānte tad aghaṃ bhūtānāṃ sa viśuddhaye\thinspace |
  bhūmyambudrumayoṣidbhyaś caturdhā vyabhajad dhariḥ\thinspace ||.}
  {\rm (}`Even though [Indra was] the Lord, he took on himself, with folded hands,
  the sin of killing a Brāhmaṇa. At the end of the year,
  [he,] Hari distributed that sin in four parts to the earth, water, trees and women
  for the self-purification of living beings.'{\rm )}
 }}

  \maintext{pañcaśodhyena śodhyeta arthayajño varānane |}%

  \maintext{śodhite tu phalaṃ śuddham aśuddhe niṣphalaṃ bhavet }||\thinspace11:7\thinspace||%
\translation{Material sacrifice can be purified with the five purifications, O Varānanā. When it is purified, the fruits are also pure. If it is not purified, there is no fruit. }

  \maintext{devy uvāca |}%

  \maintext{pañcaśodhye suraśreṣṭha saṃśayo 'tra bhaven mama |}%

  \maintext{kathayasva vibhāgena śrotum icchāmi tattvataḥ }||\thinspace11:8\thinspace||%
\translation{The Goddess spoke: I am not sure about the five purifications, O Sura\-śreṣṭha. Please teach [them to] me one by one, I want to hear [them] as [they] really [are]. }

  \maintext{rudra uvāca |}%

  \maintext{manaḥśuddhis tu prathamaṃ dravyaśuddhir ataḥ param |}%

  \maintext{mantraśuddhis tṛtīyā tu karmaśuddhir ataḥ param |}%

  \maintext{pañcamī sattvaśuddhis tu kratuśuddhiś ca pañcadhā }||\thinspace11:9\thinspace||%
\translation{Rudra spoke: First [there is] the purification of the mind, then [comes] the purification of the substances. The third is the purification of mantras, then the purification of the ritual. The fifth is the purification of Sattva. The purification of the sacrifice is [thus] fivefold. \blankfootnote{11.9 \textit{Pāda} a is unmetrical unless the so-called muta cum liquida licence is applied for the 
  first syllable of \textit{prathamaṃ}, turning the line into a \textit{na-vipulā}.
 
  
  Sets of five types of purification are a commonplace in Tantric Śaivism, but
  they are usually somewhat different form what we see here. They usually include
  \textit{ātmaśuddhi, sthānaśuddhi, dravyaśuddhi, mantraśuddhi} and \textit{liṅgaśuddhi}. See
  Goodall's article on this in \TAKIII\ s.v. \textit{dravyaśuddhi}.
 }}

  \maintext{manaḥśuddhir nāma aviparītabhāvanayā | }%

  \maintext{dravyaśuddhir nāma ananyāyopārjitadravyena  }||\thinspace11:10\thinspace||%
\translation{The purification of the mind is [achived] by mentally creating what is not against [the rules]. The purification of the substances is [achieved] by [using] substances that were not obtained by unlawful means. \blankfootnote{11.10 The passage 11.10-11 is in fact prose.
 }}

  \maintext{mantraśuddhir nāma svaravyañjanayuktatayā | }%

  \maintext{kriyāśuddhir nāma yathākramāviparītatayā | }%

  \maintext{sattvaśuddhir nāma rajastama-apradhānatayā  }||\thinspace11:11\thinspace||%
\translation{Purification of the mantras is [achived] by properly applying {\rm (}\textit{yuktatayā}{\rm )} vowels to consonants. Purification of the ritual is [achived] by not altering the proper sequence [of the elements of ritual]. The purification of Sattva is [achived] by the non-prevalence of Rajas and Tamas. }

  \maintext{vidhim evaṃ yadā śudhyed yadi yajñaṃ karoti hi |}%

  \maintext{tasya yajñaphalāvāptir janmamṛtyuś ca no bhavet }||\thinspace11:12\thinspace||%
\translation{When he has purified the ritual {\rm (}\textit{vidhi}{\rm )} thus and performs the sacrifice, he will obtain the fruits of the sacrifice, and will not undergo births and deaths [any more]. \blankfootnote{11.12 An alternative to my conjecture in \textit{pāda} a {\rm (}\textit{yadā śudhyed} for \textit{yadā sūyed}, \textit{sūryed}, \textit{pūrya}, and \textit{pūyed}{\rm )}
  has been suggested by Dominic Goodall. One could apply the reading of \msCb\ thus:
  \textit{yadāpūrya} {\rm (}`when having completed'{\rm )}.
 }}

  \maintext{vinārthena tu yo yajñaṃ karoti varasundari |}%

  \maintext{na tasya tatphalāvāptiḥ sarvayajñeṣv aśeṣataḥ }||\thinspace11:13\thinspace||%
\translation{But he who performs sacrifice without materials, O Varasundarī, will not [only] obtain its fruits, [but] of all sacrifices, without exception. \blankfootnote{11.13 I tentatively interpret \textit{sarvayajñeṣu} in \textit{pāda} d as a locative for genitive, and
  in a sense that does not reflect the meaning in which I took \textit{sarvayajñaḥ} in 11.1a above.
  Compare the conclusion of this section, 11.24cd: 
  \textit{āsahasrasya yajñānāṃ phalaṃ prāpnoti nityaśaḥ}.
 }}

  \maintext{yajñavāṭa kurukṣetraṃ sattvāvāsakṛtālayaḥ |}%

  \maintext{pratyāhāra mahāvedi kuśaprastara saṃyamaḥ }||\thinspace11:14\thinspace||%
\translation{The sacrificial ground is [the internal] Kurukṣetra, he has made his abode in the house of Truth {\rm (}\textit{sattva}{\rm )}. The great altar is the withdrawal of the senses {\rm (}\textit{pratyāhāra}{\rm )}. The seat made of \textit{kuśa} grass is constraint {\rm (}\textit{saṃyama}{\rm )}. \blankfootnote{11.14 It would be easy to correct \textit{yajñavāṭa} in \textit{pāda} a to \textit{yajñavāṭaḥ}, and to normalise
  all the similarly positioned stem form nouns in the following verses because there are no
  metrical constrains that would prevent us from doing it,
  but it seems to me that there is a pattern here and these stem forms
  give the impression of being emphasised, highlighted, or being items in a list
  {\rm (}see 11.14c and d, 15a, 16a and b, 17a, 18d, etc.{\rm )}. Nevertheless,
  some of the expression in the upcoming verses should be interpreted as bahuvrīhis
  qualifying the sacrificer/yogin. In fact, we could read \textit{yajñavāṭakurukṣetraḥ} and
  \textit{pratyāhāramahāvediḥ} as bahuvrīhis here.
 
  Kurukṣetra was defined as an internalised pilgrimage place in 10.12, which
  fits well the presently introduced teaching of internalised sacrifice.
  Both are summarised, together with bodily penance, in 13.2 as:
  \textit{svaśarīrasthito yajñaḥ svaśarīre sthitaṃ tapaḥ\thinspace |
  svaśarīre sthitaṃ tīrthaṃ śruto vistarato mayā\thinspace ||}.
  The term \textit{sattvāvāsa} has elsewhere, but probably not here, a distinctively Buddhist flavour,
  denoting the seven or nine `abodes of beings,' see, e.g. 
  \mycitep{EdgertonHybrid}{vol.~2, s.v. \textit{sattvāvāsa}}, and
  \mycitep{SferraTorellaVolArticle}{1155}.
 Note that if \textit{pāda} c followed the pattern of \textit{pāda} a, namely that `X in Vedic ritual is now Y in this
  internalised sacrifice,' we would need to read \textit{mahāvedi pratyāhāra}, but that would be unmetrical.
 
  \textit{saṃyama} is mentioned only a few times in the \VSS\ {\rm (}e.g., in a similar
  context, in 22.12{\rm )}, and is never explained, in contrast with 
  the \textit{niyama}-rules mentioned in the next verse, which are expounded in detail in 5.1--8.44.
  \textit{saṃyama} may perhaps be used here in the sense in which it appears in the \YS:
  the yogic application, or appearance, of \textit{dhāraṇā}, \textit{dhyāna}, and \textit{samādhi} at
  the same time {\rm (}see \YS\ 3.ff{\rm )}.
 }}

  \maintext{vidhi niyamavistāro dhyānavahniḥ pradīpitaḥ |}%

  \maintext{yogendhanasamijjvālatapodhūmasamākulaḥ }||\thinspace11:15\thinspace||%
\translation{Vedic injunction {\rm (}\textit{vidhi}{\rm )} is the enumeration of Niyama-rules. [For the Vedic ritual fire it is now] the fire of meditation {\rm (}\textit{dhyāna}{\rm )} [that] is lighted. which is flaring up by the fuel of the firewood of yoga and is abounding in the smoke of penance. \blankfootnote{11.15 I have chosen the reading of \textit{pāda} b that is the easiest to interpret. Alternatively,
  the intended expression may have been \textit{dhyānena vahniḥ pradīpitaḥ}.
 Instead of taking °\textit{samijjvāla}° as a tatpuruṣa compound in \textit{pāda} c {\rm (}°\textit{samidh-jvāla}°{\rm )}, 
  consider emending it to °\textit{samujjvāla}°, which would stand metri causa for °\textit{samujjvala}°.
 }}

  \maintext{pātranyāsa śivajñānaṃ sthālīpāka śivātmakaḥ |}%

  \maintext{ājyāhutim avicchinnaṃ lambakasruvapātitaḥ }||\thinspace11:16\thinspace||%
\translation{The placing down of the chalice is knowledge of Śiva. [The oblation of] boiled rice is [now the process of] be[com]ing Śiva. The continuous oblation of clarified butter {\rm (}\textit{ājyāhuti}{\rm )} is poured with the ritual ladle {\rm (}\textit{sruva}{\rm )} of the uvula {\rm (}\textit{lambaka}{\rm )}. \blankfootnote{11.16 The interpretation of \textit{pāda} b is tentative.
 Ignoring the problems concerning grammatical gender and case, 
  we may presume that the intended meaning in \textit{pāda}s cd could be expressed thus:
  \textit{ājyāhutir avicchinnā lambikāsruvena pātitā}. I suspect that \textit{lambaka} simply
  stands for \textit{lambikā} {\rm (}`uvula'{\rm )}, which fits the internalised nature of 
  this ritual. See also \textit{ghaṇṭikā} possibly as `uvula' in 10.32d.
 }}

  \maintext{dhāraṇādhvaryuvat kṛtvā prāṇāyāmaś ca ṛtvijaḥ |}%

  \maintext{tarkayuktaḥ savistāraḥ samādhir vayatāpanaḥ }||\thinspace11:17\thinspace||%
\translation{Transforming concentration into an Adhvaryu [priest, the phases of] breath control will be the [other Vedic] priests[, the Hotṛ, the Brahman, and the Udgātṛ]. Samādhi which involves reflection {\rm (}\textit{tarka}{\rm )} and which is extensive is the [Vedic ritual of] burning the oblation {\rm (}\textit{vayas-tāpana}?{\rm )}. \blankfootnote{11.17 Understand \textit{pāda}s a as \textit{dhāraṇām adhvaryuvat kṛtvā} {\rm (}\textit{dhāraṇā} in the MSS being in stem form{\rm )}.
 Note how taking 11.14c and 15b together with the present verse,
  all six auxiliaries of the \textit{ṣaḍaṅgayoga} of \VSS\ chapter 16 have now been mentioned in this chapter.
  See 16.18:
  \textit{pratyāhāras tathā dhyānaṃ prāṇāyāmaś ca dhāraṇā\thinspace |
  tarkaś caiva samādhiś ca ṣaḍaṅgo yoga ucyate\thinspace ||}
  My interpretation of \textit{vayatāpana} in \textit{pāda} d as `burning of oblation' {\rm (}\textit{vaya} possibly standing for \textit{vayas}
  metri causa{\rm )} is tentative.
 }}

  \maintext{brahmavidyāmayo yūpaḥ paśubandho manonmanaḥ |}%

  \maintext{śraddhā patnī viśālākṣi saṃkalpa pada śāśvatam }||\thinspace11:18\thinspace||%
\translation{The sacrificial post is made up of the knowledge about the Brahman. The tying of the sacrificial animal is [the mental state called] Manonmanas. [The householder's] wife is Faith, O Viśālākṣī. [His] ritual intention {\rm (}\textit{saṃkalpa}{\rm )} is [reaching] the eternal abode. \blankfootnote{11.18 The final section of \VSS\ chapter 20, a chapter on the \textit{tattva}s of Sāṃkhya,
  discusses the mental state of \textit{unmanas}:
  \textit{unmanastvaṃ gate vipra nibodha daśalakṣaṇam\thinspace |
  na śabdaṃ śṛṇute śrotraṃ śaṅkhabherīsvanād api\thinspace ||}, etc. 
  Verse 11.50 below mentions \textit{manonmanas} in a similar context.
 In \textit{pāda} d, understand \textit{saṃkalpaḥ padaṃ śāśvatam} {\rm (}both \textit{saṃkalpa} and \textit{pada} are
  stem form nouns in the verse, the latter metri causa{\rm )}.
 }}

  \maintext{pañcendriyajayotpannaḥ puroḍāśo 'mṛtāśanaḥ |}%

  \maintext{brahmanādo mahāmantraḥ prāyaścittānilo jayaḥ }||\thinspace11:19\thinspace||%
\translation{Rice oblation is the consumption of the nectar of immortality that arises from the victory over the five senses. The great [Vedic] mantra is [now] Brahmā's sound. Expiation is victory over the breath. \blankfootnote{11.19 The term \textit{brahmanāda} in \textit{pāda} c may refer to the same concept 
  as \textit{brahmabilasvara} does in 11.29d. It may be the same as the {\rm (}haṭha{\rm )}yogic
  concept of \textit{mahānāda} {\rm (}`great sound' or `unstruck sound'{\rm )}, on which see
  \mycitep{MallinsonKhecari}{225, nn.~359 and 361}.
  My translation tentatively presupposes that \textit{mantra} in \textit{mahāmatra}
  refers to Vedic mantras, now contrasted with a yogic experience.
  {\rm (}See \textit{mahāmantra} referring to Vedic/Śrauta mantras in \SKANDAP\ 13.132cd: 
  \textit{śrutigītair mahāmantrair mūrtimadbhir upasthitaiḥ}.{\rm )}
 
  Understand \textit{pāda} d as \textit{prāyaścitto 'nilajayaḥ}. It would be possible
  to correct °\textit{cittānilo} to °\textit{citto 'nilo}, but since \textit{'nilajayaḥ} would
  be unmetrical and since stem form nouns abound in this chapter, 
  I believe that \textit{prāyaścittānilo} could be original.
 }}

  \maintext{somapāna parijñānam upākarma caturyamaḥ |}%

  \maintext{itihāsa jalasnānaṃ purāṇakṛta{-}m{-}ambaraḥ }||\thinspace11:20\thinspace||%
\translation{The consumption of Soma is [substituted now with] complete knowledge. The commencement [of the Vedic ritual] is the four Yama-rules. The ritual water-bath is [the study of] the epics. His garment is made of [his study of] the Purāṇas. \blankfootnote{11.20 \textit{caturyamaḥ} in \textit{pāda} b is baffling. The \VSS\ teaches ten 
  Yama-rules in 3.16--4.89. Dominic Goodall has suggested that \textit{caturyamaḥ} could
  stand for \textit{ca tu yamāḥ} metri causa. Another possibility would be 
  to interpret \textit{catur} as \textit{caturtha} {\rm (}`fourth'{\rm )} and then the phrase
  may refer to the fourth Yama-rule, absence of hostility {\rm (}\textit{ānṛśaṃsya}, 4.31--49{\rm )}.
 Note the stem form \textit{itihāsa} in \textit{pāda} c, and the hiatus-filler \textit{-m-} in \textit{pāda} c in
  °\textit{kṛta-m-ambaraḥ} which is a metrical solution for °\textit{kṛto 'mbaraḥ}.
 }}

  \maintext{iḍāsuṣumnāsaṃvedye snānam ācamanaṃ sakṛt |}%

  \maintext{saṃtoṣātithim ādṛtya dayābhūtadvijārcitaḥ }||\thinspace11:21\thinspace||%
\translation{Ritual bathing and sipping water once are [to be performed] at the confluence of the Iḍā and the Suṣumnā. Having honoured Contentment as a guest, he salutes the Brahmin that is [now] Compassion. \blankfootnote{11.21 For the teaching on the internalised pilgrimage places Gaṅgā, i.e. Suṣumnā, and
  Yamunā, i.e. Iḍā, and their internalised confluence, Prayāga, see 10.17. Note that
  Iḍā and Suṣumnā are then reinterpreted as Somatīrtha and Sūryatīrtha, respectively,
  in 10.20--21.
 \textit{saṃtoṣa}° is either meant to be compounded with °\textit{atithim} in \textit{pāda} c or 
  is in stem form for \textit{saṃtoṣam atithiṃ}; for the latter possibility cf. e.g. 
  11.17a above. Similarly, °\textit{dvija}° may be in stem form in \textit{pāda} d, for
  °\textit{dvijo 'rcitaḥ}, or simply correct it to °\textit{dvijo 'rcitaḥ}.
 }}

  \maintext{brahmakūrca guṇātīta havirgandha nirañjanaḥ |}%

  \maintext{brahmasūtraṃ trayas tattvaṃ bodhanā muṇḍitaṃ śiraḥ }||\thinspace11:22\thinspace||%
\translation{The Brahmakūrca [observance] is the [state of mind called] `beyond the Qualities' {\rm (}\textit{guṇātīta}{\rm )}, the scent of the sacrifice is the `spotless' {\rm (}\textit{nirañjana}{\rm )} [state of mind]. [His] sacred thread is the three truths {\rm (}\textit{tattva}{\rm )}. The shaven head [of the \textit{snātaka}] is [now] enlightenment. \blankfootnote{11.22 Note the stem form nouns in \textit{pāda}s ab.
 
  On the \textit{brahmakūrca} observance, see, e.g., \mycitep{KaneHistory}{vol.~4, 146},
  where the references given include \MITAKSARA\ ad \YAJNS\ 3.314: 
  \textit{yadā punaḥ pūrvedyur upoṣyāparedyuḥ samantrakaṃ saṃyujya 
  samantrakam eva pañcagavyaṃ pīyate tadā brahmakūrca ity ākhyāyate};
  `And when one fasts one day, and on the next day mixes the five products of the cow
  together while reciting mantras, and drinks [the mixture] while reciting mantras again,
  that is called \textit{brahmakūrca}.'
  
  On the \textit{guṇātīta} state of mind, see 9.39--43. See the term \textit{nirañjana} mentioned
  as a quality of the soul {\rm (}\textit{jīva}{\rm )} in 1.11 and 15.4, of the \textit{puruṣa} in 20.3, 
  as a state of mind in 11.49, and as one of ten meditative states in 22.30.
 
  
 It is difficult to know what are the three \textit{tattva}s mentioned in \textit{pāda} c.
  {\rm (}Understand \textit{trayas tattvaṃ} as \textit{tattvatrayaṃ}, \textit{trīṇi tattvāni}, \textit{tritattvāni}, or
  \textit{tritattvaṃ}.{\rm )}
  \VSS\ chapter 4 teaches four \textit{tattva}s as objects of meditation:
  \textit{ātman}, \textit{vidyā}, \textit{bhava}, and \textit{sūkṣma} {\rm (}see, e.g., 4.73{\rm )}. \VSS\ chapter 6
  discusses five \textit{tattva}s: \textit{sūrya}, \textit{soma}, \textit{agni}, \textit{sphaṭika}, and \textit{sūkṣma} 
  {\rm (}see, e.g., 6.7{\rm )}. \VSS\ chapter 20 enumerates the 25 \textit{tattva}s of Sāṃkhya.
  One possibility would be to interpret the set of three \textit{tattva}s as
  the three \textit{padārtha}s of the Śaivasiddhānta, \textit{pati}, \textit{paśu}, and \textit{pāśa};
  see, e.g., \TAKIII, s.v. \textit{patipaśupāśa}.
  Dominic Goodall has tentatively suggested reading here in 11.22c, with \msNa,
  \textit{brahmasūtratrayaṃ tattvaṃ} {\rm (}`the three strands of the sacred thread is truth'{\rm )}.
  The problem is firstly that we have \textit{trayas tattvaṃ} repeated in 11.29c below, 
  and secondly that what we need here is three entities compared to the three strands
  of the sacred thread. What is clear here is that even the investiture of the
  sacred thread {\rm (}\textit{upanayana}{\rm )} is supposed to be internalised in this teaching
  of non-material sacrifice.
 }}

  \maintext{nivṛttyādi caturvedaś catuḥprakaraṇāsanaḥ |}%

  \maintext{dakṣiṇām abhayaṃ bhūte dattvā yajñaṃ yajet sadā }||\thinspace11:23\thinspace||%
\translation{The four Vedas are [now] \textit{nivṛtti} etc. His seat is the four \textit{prakaraṇa}s. He should always perform a[n internalised] sacrifice after donating the priestly fee of providing being[s] with freedom from danger. \blankfootnote{11.23 My assumption is that \textit{pāda} a here hints at those four, later five, categories,
  called \textit{kalā}s, that are well-known from Tantric Śaivism: 
  \textit{nivṛtti}, \textit{pratiṣṭhā}, \textit{vidyā}, \textit{śānti}, and \textit{śāntyatīta}.
  For this, I had to emend the reading found in all witnesses consulted, \textit{nivṛtyā}°.
  I consider \textit{nivṛti} for \textit{nivṛtti} a common and plausible error. 
  As Dominic Goodall has suggested, here the four \textit{kalā}s, 
  originally possibly the four Śaktis of the Lord, may be reinterpreted as yogic states.
  The fact that the \VSS\ is aware of only four \textit{kalā}s here may hint at a relatively
  early date of composition of this section {\rm (}see Introduction pp. \verify{\rm )}.
  On the history and interpretation of these \textit{kalā}s,
  see \TAKII\ s.v. \textit{kalā} 6.
 
  \textit{catuḥprakaraṇāsanaḥ} may be taken as \textit{catuḥprakaraṇāṇy āsanam}, or, as I take it in my
  translation, a bahuvrīhi compound qualifying the practitioner. As to
  what the four \textit{prakaraṇa}s {\rm (}`chapters'?{\rm )} refer to here, I am without a clue.
  Perhaps the phrase was meaningful in a context where this section was
  taken out. {\rm (}The \textit{Mokṣopāya}, a text of Kashmiri origin from the tenth century and
  made up of \textit{prakaraṇa}s would be an interesting candidate for being a point of reference.
  Unfortunately from this point of view, there are six \textit{prakaraṇa}s in the 
  \textit{Mokṣopāya}, and not four.
  See, e.g., \mycite{SlajeMoksopaya1996}\index{Moksopaya@\textsl{Mo\-kṣo\-pā\-ya}}.
 }}

  \maintext{vinārthaṃ yajñasamprāptiḥ kathitā te varānane |}%

  \maintext{āsahasrasya yajñānāṃ phalaṃ prāpnoti nityaśaḥ }||\thinspace11:24\thinspace||%
\translation{The attainment of sacrifice without materials has been taught to you, O Varānanā. [The sacrificer] will in any case obtain the fruits of up to a thousand [ordinary Vedic] sacrifices. }

  \maintext{āśramaḥ prathamas tubhyaṃ kathito 'sti varānane |}%

  \maintext{sadāśivena saddharmaṃ daivatair api pūjitam }||\thinspace11:25\thinspace||%
\translation{The first life-stage [life option] has been taught to you, O Varānanā, through Sadāśiva; [this is] the true Dharma, revered also by the gods. \blankfootnote{11.25 \textit{sadāśivena} in \textit{pāda} c could also be interpreted as the agent of \textit{pūjitam} in \textit{pāda} d
  {\rm (}`it is revered by Sadāśiva'{\rm )}, but Sadāśiva was mentioned as the original
  teacher of this ritual in 11.4 above, which makes it probable that
  he is being referred to in a similar manner here. Cf. also 11.30 below.
 }}

  \subchptr{brahmacārī}%

  \trsubchptr{The chaste one}%

  \maintext{brahmacaryaṃ nibodhedaṃ śṛṇuṣvāvahitā śubhe |}%

  \maintext{dvitīyam āśramaṃ devi sarvapāpavināśanam }||\thinspace11:26\thinspace||%
\translation{[Now] learn about this, about the practice of chastity {\rm (}\textit{brahmacarya}{\rm )}. Listen with attentively, O Śubhā. [It is] the second life-stage {\rm (}\textit{āśrama}{\rm )}, O Devī, the destroyer of all sins. \blankfootnote{11.26 \textit{idaṃ} in \textit{nibodhedaṃ} in \textit{pāda} a sounds clumsy with \textit{brahmacaryaṃ} {\rm (}lit. `listen to this
  practice of chastity'{\rm )} but in fact the \MBH\ and the Purāṇas contain countless similar,
  albeit smoother, expressions, e.g., \MBH\ 5.145.15ab 
  {\rm (}\textit{duryodhana nibodhedaṃ kulārthe yad bravīmi te}{\rm )},
  \BRAHMAP\ 133.10ab
  {\rm (}\textit{bharadvāja nibodhedaṃ vākyaṃ mama samāsataḥ}{\rm )}, etc.
 See some remarks on the life-stages, or social order of disciplines {\rm (}\textit{āśrama}{\rm )},
  and especially on their order, in the \VSS\ in \mycite{KissVolume2021} and
  above on p.\verify 
 }}

  \maintext{vrataṃ brahmaparaṃ dhyānaṃ sāvitrī prakṛti{-}r{-}layam |}%

  \maintext{brahmasūtrākṣaraṃ sūkṣmaṃ triguṇālaya mekhalam }||\thinspace11:27\thinspace||%
\translation{Religious observances are [now] meditation focussed on the Brahman. The Sāvitrī [hymn] is absorption in Prakṛti. The Brahmanical cord {\rm (}\textit{brahmasūtra}{\rm )} is the subtle syllable. His girdle is now the abode of the three Qualities {\rm (}\textit{guṇa}{\rm )}. \blankfootnote{11.27 One could emend \textit{prakṛtir layam} in \textit{pāda} b to the expected \textit{prakṛtau layaḥ}
  {\rm (}see, e.g., \AGNIP\ 379.1d: \textit{vairāgyāt prakṛtau layam}{\rm )}.
  Nevertheless, I retained the reading of \msCa\msNa\msNc\Ed\ because
  it may have been the way in which the compound \textit{prakṛtilaya} was originally made 
  metrical. In other words, I suspect the \textit{-r-} to be only a link
  between the two elements of this compound. I also retained the neuter ending.
  Compare 16.8d, where the same expression is transmitted in all the witnesses
  so far consulted as \textit{prakṛtālayam}.
 
  
 Note the stem form nouns in \textit{pāda}s cd {\rm (}°\textit{sūtra} and °\textit{ālaya}{\rm )}.
  The `subtle syllable' may be \textit{oṃ} {\rm (}cf. 1.9--10{\rm )}, traditionally analysed as
  made up of three sounds, here corresponding to the three strands of the
  sacred thread. In \textit{pāda} d, \textit{triguṇālaya} might rather
  mean `absorption in the three Qualities' {\rm (}\textit{triguṇeṣu layaḥ}{\rm )}
  although in my translation I translate it as \textit{triguṇa-ālayaḥ}.
 }}

  \maintext{dama daṇḍa dayā pātraṃ bhikṣā saṃsāramocanam |}%

  \maintext{tryāyuṣaṃ dvyakṣarātītaṃ jñānabhasma-alaṅkṛtam }||\thinspace11:28\thinspace||%
\translation{His staff is self-restraint, his bowl compassion. Alms are liberation from transmigration {\rm (}\textit{saṃsāra}{\rm )}. The Tryāyuṣa is the one beyond the two syllables. [The three lines are] prepared with the ashes of knowledge. \blankfootnote{11.28 The Tryāyuṣa is a Vedic mantra, see, e.g., \textit{Ṛgveda-khila} 5.3.6:
  \textit{tryāyuṣam jamadagneḥ kaśyapasya tryāyuṣam\thinspace |
  agastyasya tryāyuṣam yad devānām tryāyuṣam tan no astu tryāyuṣam}\thinspace |;
  `The threefold vitality of [the sage] Jamadagni, that of [the sage] Kaśyapa, 
  that which is that of the gods---may it be ours!' {\rm (}translation based on 
  \mycitep{SaivaUtopia2021}{28}{\rm )}. `In the Vedic domestic ritual codes, 
  this is the mantra to be recited over the razor or over the student who is
  about to be shaven before bathing at the end of his studies' {\rm (}ibid.{\rm )}.
  In \SIVAUP\ 5.20ab, this mantra is prescribed to accompany the application of the 
  three lines on the forehead. Thus here in \VSS\ 11:28cd, \textit{tryāyuṣa} and
  the mention of ashes make it clear that the next element of the ritual
  life of the \textit{brahmacārin} to be internalised is the application 
  of the \textit{tripuṇḍra}. As for the \textit{dvyakṣarātīta}, which should be a mantra,
  it perhaps means a three-syllable mantra, possibly \textit{a-u-m} or \textit{śivāya}.
 }}

  \maintext{snānavrataṃ sadāsatyaṃ śīlaśaucasamanvitam |}%

  \maintext{agnihotra trayas tattvaṃ japa brahmabilasvaraḥ }||\thinspace11:29\thinspace||%
\translation{The bath-vow is life-long truthfulness, accompanied by the purity of moral conduct. The Agnihotra sacrifice is the three \textit{tattva}s. Recitation is the sound at the aperture of Brahmā. \blankfootnote{11.29 On the problem of understanding what the three \textit{tattva}s are in this text, and on the 
  phrase \textit{trayas tattvaṃ}, see notes on verse 11.22 above.
  Perhaps \textit{brahmabilasvara} in \textit{pāda} d refers to the same concept as \textit{brahmanāda} does in 11.19c.
 }}

  \maintext{dvitīya āśramo devi yathāha bhagavān śivaḥ |}%

  \maintext{mamāpi kathitaṃ tubhyaṃ janmamṛtyuvināśanam }||\thinspace11:30\thinspace||%
\translation{The second life-stage has [now] been taught also to you as Lord Śiva taught it, O Devī, to me. It is the destruction of birth and death. \blankfootnote{11.30 One may consider correcting \textit{mamā}° to \textit{mayā}° {\rm (}`it has been taught by \textit{me}'{\rm )},
  but \textit{mama}, linked to the first hemistich, may be original, and \textit{api}, then slightly 
  unusually placed in the sense of `too/also' {\rm (}as, e.g., in \RAGHU\ 5.44 and 9.8c{\rm )},
  starting a new clause.
 }}

  \subchptr{vānaprasthaḥ}%

  \trsubchptr{The forest-dweller}%

  \maintext{vānaprasthavidhiṃ vakṣye śṛṇuṣvāyatalocane |}%

  \maintext{yathāśrutaṃ yathātathyam ṛṣidaivatapūjitam }||\thinspace11:31\thinspace||%
\translation{Listen, O Long-eyed goddess, I shall teach you the forest-dweller's way of life, which is revered by the sages and the gods, as I heard it, as it [really] is. }

  \maintext{vairāgyavanam āśritya niyamāśramam āharet |}%

  \maintext{śīlaśailadṛḍhadvāre prākāre vijitendriyaḥ }||\thinspace11:32\thinspace||%
\translation{Having taken to the forest of indifference, he should take residence in the ashram of Niyama-rules, within walls that have the stone-strong gate of moral conduct, with his sense faculties conquered. \blankfootnote{11.32 \textit{āharet} {\rm (}`should take away, get, use'{\rm )} in \textit{pāda} b is suspect; 
  \textit{āvaset} {\rm (}`should settle'{\rm )} or \textit{āśrayet} {\rm (}`should take refuge'{\rm )} 
  would make more sense in this context.
 }}

  \maintext{adhibhūtaḥ smṛto mātā adhyātmaś ca pitā tathā |}%

  \maintext{adhidaivikam ācāryo vyavasāyāś ca bhrātaraḥ }||\thinspace11:33\thinspace||%
\translation{One's mother is the material realm, one's father the Self, one's guru the divine. Resolutions are one's brothers. \blankfootnote{11.33 I have accepted Dominic Goodall's suggestion to emend \textit{adhibhautika} in \textit{pāda} c
  to \textit{adhidaivika}. In this way, we arrive at the well-know triad of \textit{adhibhūta},
  \textit{adhyātma}, and \textit{adhidaivika} {\rm (}or more often: \textit{ādhibhautika}, \textit{ādhyātmika}, and
  \textit{ādhidaivika}; see, e.g. \YBH\ ad \YS\ 1.31 and 3.22, and \SK\ 1.1 in most
  commentators' interpretation{\rm )}. \textit{adhibhautika} in \textit{pāda} c may be the result of
  an eye-skip to \textit{pāda} a, and the final \textit{-m} of \textit{adhidaivika} could be 
  interpreted as a hiatus-filler. The triad in question usually qualify
  three types of suffering or bad omen: pertaining to the material world, 
  one's own self or body, and to the world of gods, respectively. Here in the \VSS, they seem to refer to 
  realms of knowledge, or as \BhG\ 8.1--4, a possible source for the present verse,
  define them, \textit{adhibhūta} is mundane existence, \textit{adhyātma} is the eternal Brahman that is
  one's true nature, and \textit{adhidaivata} is the \textit{puruṣa}.
 }}

  \maintext{śrutiḥ smṛtiḥ smṛtā bhāryā prajñā putraḥ kṣamānujaḥ |}%

  \maintext{maitrī bandhur jaṭā cāpaṃ karuṇā supavitrakam |}%

  \maintext{muditā mauna catvāraḥ sarvakāryam upekṣakā }||\thinspace11:34\thinspace||%
\translation{Śruti and Smṛti are his wives, Wisdom his son, Patience his little brother. Benevolence is his kinsman, his twisted hair [and] his bow. Compassion his sacred thread. Sympathy is the four ways of observing silence. All his religious duties are equanimity. \blankfootnote{11.34 \textit{bhāryā} in \textit{pāda} a is probably meant to be in the dual {\rm (}\textit{bhārye}{\rm )} but
  the use of the singular could be original. Note how notions expressed by
  feminine nouns are associated with male relatives {\rm (}\textit{prajñā} is a son, 
  \textit{kṣamā} a brother{\rm )}.
 
  
 In \textit{pāda} c, \textit{jaṭā cāpaṃ} is problematic. One would expect here an abstract notion 
  corresponding to a real-life element of the forest-dweller life, as in the above verses.
  \textit{jaṭā} and \textit{cāpa} are either still identified with \textit{maitrī} {\rm (}that is how I translate the \textit{pāda}{\rm )} 
  or there is a need to emend, e.g. to \textit{jaṭācāraḥ} {\rm (}`good conduct is his twisted hair'{\rm )}. 
  I prefer the former solution because in this way the four Buddhist \textit{brahmavihāra}s,
  \textit{maitrī-karuṇā-muditā-upekṣā}, appear in one uninterrupted sequence. These may seem 
  as being out of context in a Brahmanical text but the source for them may have been
  \YS\ 1.33: \textit{maitrīkaruṇāmuditopekṣāṇāṃ sukhaduḥkhapuṇyāpuṇyaviṣayāṇāṃ bhāvanātaś 
  cittaprasādanam}. See them mentioned also in verse 4.72 above, and in 11.56 below.
  
  
 Note \textit{mauna} in \textit{pāda} e in stem form, and \textit{upekṣakā} for \textit{upekṣā}, both metri causa.
 }}

  \maintext{yamavalkalasaṃvītas tapaḥkṛṣṇājinādharaḥ |}%

  \maintext{uttarāsaṅgam āsīno yogapaṭṭadṛḍhavrataḥ }||\thinspace11:35\thinspace||%
\translation{He is clothed in the Yama-rules instead of a garment made of bark, and he wears penance instead of the skin of a black antelope. He is seated on the highest level of non-attachment, and a firm observance is his yoga-belt. \blankfootnote{11.35 I think that \msNc's \textit{jinādharaḥ} in \textit{pāda} b may be the original reading:
  it lengthens the final \textit{a} of \textit{jina}° metri causa, and that the remaining sources
  try to restore the standard form of \textit{ajina} and thus ruin the metre. Cf., e.g., \MBH\ 1.123.18:
  \textit{sa kṛṣṇaṃ maladigdhāṅgaṃ kṛṣṇājinadharaṃ vane\thinspace |
  naiṣādiṃ śvā samālakṣya bhaṣaṃs tasthau tadantike\thinspace ||}.
 The accusative \textit{uttarāsaṅgam} in \textit{pāda} c is acceptable, but one may
  understand the final \textit{-m} as a hiatus filler after a locative {\rm (}°\textit{saṅga āsīno}{\rm )},
  or in the middle of a compound {\rm (}°\textit{saṅgāsīno}{\rm )}.
 }}

  \maintext{vedaghoṣeṇa ghoṣeṇa prāṇāyāmo 'gnihāvanam |}%

  \maintext{jitaprāṇa mṛgākūlo dhṛti yajñaḥ kriyā japaḥ }||\thinspace11:36\thinspace||%
\translation{Fire sacrifice accompanied by the sound of murmuring the Vedas is breath-control accompanied by [its] hissing. The herd of deer [in the forest where the forest-dweller normally lives] is [now his] conquered breaths. [Now] sacrifice is resolution, ritual is mantra-recitation. \blankfootnote{11.36 \textit{hāvana} in \textit{pāda} b stands fot \textit{havana} metri causa.
 I suspect that \textit{°mṛgākūlo} in \textit{pāda} c stands for an unmetrical \textit{mṛgakulo}.
  Incidentally, even by inverting the order of the two elements in this \textit{pāda}, there would
  remain the metrical error of two \textit{laghu}s: \textit{mṛgakulo jitaprāṇo}.
  Also, note °\textit{prāṇa} and \textit{dhṛti} in \textit{pāda}s cd as nouns in stem form.
 }}

  \maintext{arthasaṃgraha śāstreṣu sakhā damadayādayaḥ |}%

  \maintext{śivayajñaṃ prayuñjīta sādhanāṣṭakapūjanam }||\thinspace11:37\thinspace||%
\translation{His treasures are in the \textit{śāstra}s, his companions are self-control, compassion, etc. He should perform sacrifice to Śiva [as] the worship of the eight [yogic] practices {\rm (}\textit{sādhana}{\rm )}. \blankfootnote{11.37 See the word \textit{saṃgraha} {\rm (}here in stem form{\rm )} used probably in a similar sense in 11.46 below.
 See a reference to eight \textit{sādhana}s in \DHARMP\ 2.1 {\rm (}quoted in the apparatus
  to the present verse in the critical edition{\rm )}. These may or may not reference
  the same set of practices.
 }}

  \maintext{pañcabrahmajalaiḥ pūtaḥ satyatīrthaśivahrade |}%

  \maintext{snānam ācamanaṃ kṛtvā saṃdhyātrayam upāsayet }||\thinspace11:38\thinspace||%
\translation{Purified by the water of the five Brahma[-mantras], bathing and sipping water in the auspicious {\rm (}\textit{śiva}{\rm )} lake at the pilgrimage place of truthfulness, he should honour the three junctures of the day. \blankfootnote{11.38 The reading of the witnesses in \textit{pāda} d, \textit{upāśrayet}, might be acceptable, but
  I consider my emendation, \textit{upāsayet}, better, especially because that is
  the verb used in 11.59d below, in a similar context.
 }}

  \maintext{akṣamālā purāṇārthaṃ japa śāntaṃ divāniśam |}%

  \maintext{jñānasalilasampūrṇa{-}m{-}itihāsakamaṇḍaluḥ }||\thinspace11:39\thinspace||%
\translation{The rosary is [now] the meaning of the Purāṇas. Recitation is [now his] peace of mind day and night. His jar of epics is filled with the water of knowledge. \blankfootnote{11.39 \textit{Pāda} b may allow for various interpretations. The one I have choosen seems to be the
  simplest. It involves a stem form noun, \textit{japa}, and \textit{śāntaṃ} in the sense of \textit{śāntiḥ}.
 Understand the middle of \textit{pāda}s cd as containing a hiatus filler to bridge the vowels
  in a standard °\textit{pūrṇa itihāsa}°.
 }}

  \maintext{pañcakarmakriyotkrānti japa pañcavidhaḥ sukham |}%

  \maintext{sādhanaṃ śivasaṃkalpo yogasiddhiphalapradaḥ }||\thinspace11:40\thinspace||%
\translation{The actions of the five [medical] procedures are yogic suicide. Recitation is the five kinds of pleasure. The \textit{Śivasaṃkalpa} [hymn] is [yogic] practice {\rm (}\textit{sādhana}{\rm )}, which yields fruits of yoga accomplishments. \blankfootnote{11.40 My translation of this verse is tentative. Note that \textit{utkrānti} 
  {\rm (}usually in similar contexts: `yogic suicide'{\rm )} is a \textit{yogāṅga} in chapter 16.
  I take \textit{japa} tentatively as a stem form noun, and \textit{pañcavidhaḥ} as if
  it read \textit{pañcavidhaṃ}. The \BODHISBH\ 1.3.4 teaches five kinds of \textit{sukha}:
  \textit{hetusukhaṃ veditasukhaṃ duḥkhaprātipakṣikaṃ sukhaṃ veditopacchedasukham avyabādhyañ 
  ca pañcamaṃ sukham}. This would not be the first occasion in this
  chapter to see Buddhist categories introduced, see 11.34 above.
 
  
 I think that \Ed's silent correction of °\textit{pradaḥ} to °\textit{pradam},
  making \textit{pāda} d qualifying \textit{sādhanaṃ} in \textit{pāda} c, is
  reasonable, but since this form is not attasted in any of the witnesses
  consulted, I hesitate to follow it. Nevertheless, I understand the
  sentence thus: that which is normally the \textit{śivasaṃkalpa} is now,
  in this internalised version of the forest-dweller's life,
  {\rm (}yogic{\rm )} practice that yields \textit{siddhi}s. I suppose that the reference is
  to \VajasaneyiS\ 34.1--6, usually called \textit{Śivasaṃkalpa}:
  \textit{yaj jāgrato dūram udaiti daivaṃ tad u suptasya tathaivaiti\thinspace |
  dūraṃgamaṃ jyotiṣāṃ jyotir ekaṃ tan me manaḥ śivasaṃkalpam astu\thinspace ||}, etc.
  See this hymn referred to in \MANU\ 11.251 in a context of expiation: 
  \textit{sakṛj japtvāsyavāmīyaṃ śivasaṃkalpam eva ca\thinspace |
  apahṛtya suvarṇaṃ tu kṣaṇād bhavati nirmalaḥ\thinspace ||}.
  In Olivelle's translation: `A man who has stolen gold, on the other hand, 
  becomes instantly stainless by reciting softly[? rather: once] the Asyavāmīya hymn 
  and the Śivasaṃkalpa formulas.' Other texts that reference the \Sivasamkalpa\
  include \NISVGUHYA\ 2.77, \AGNIP\ 259.74, and \LINPU\ 1.64.76. See more on
  the \Sivasamkalpa\ in \mycite{RgvedaKhila} and \citeyear{Sivasankalpopanisad}.
 }}

  \maintext{saṃtoṣaphalam āhāraḥ kāmakrodhaparājitaḥ |}%

  \maintext{āśāpāśajayābhyāso dhyānayogaratipriyaḥ |}%

  \maintext{atithibhyo 'bhayaṃ dattvā vānaprasthaś cared vratam }||\thinspace11:41\thinspace||%
\translation{His food is the fruit of contentment. He conquers lust and anger. His practice is the victory over the trap of hope. He prefers the joy of yoga meditation. The forest-dweller should observe his vow by providing guests with fearlessness. \blankfootnote{11.41 Cf. 11.23 above on giving \textit{abhaya} to guests.
 }}

  \maintext{vānaprastham ayaṃ dharmaṃ gadita yat pūrvam avadhāritaṃ}%

 \nonanustubhindent \maintext{saṃsāroddharaṇam anityaharaṇam ajñānanirmūlanam |}%

  \maintext{prajñāvṛddhikaram amoghakaraṇaṃ kleśārṇavottāraṇaṃ}%

 \nonanustubhindent \maintext{janmavyādhiharam akarmadahanaṃ sevet sa dharmottamam }||\thinspace11:42\thinspace||%
\translation{One should follow the Dharma of the forest-dweller, the supreme Dharma, which has been taught and which, if first understood, will deliver one from transmigration, will remove transient existence, uproot ignorance, increase wisdom, will be fruitful, will deliver one from the flood of affliction, will remove rebirth and disease, and will burn one's bad karma. \blankfootnote{11.42 In some MSS, \textit{pāda} a gives a first impression of being an \textit{anuṣṭubh} line with metrical problems.
  But, as Dominic Goodall remarked, the variants suggest that it may belong to the
  upcoming Śārdūlavikrīḍita verse. This is all the more so because that verse would otherwise 
  contain only three \textit{pāda}s. My reconstruction of the now \textit{pāda} a is still highly problematic;
  \textit{gadita} is in stem form, and the final syllable of \textit{pūrvam} scans as heavy.
  While these are acceptable in the language of the \VSS, some elements remain questionable, namely
  the first syllable of \textit{dharmaṃ} as a short syllable, and the second
  syllable of \textit{avadhāritaṃ} as long. The \textit{pāda} may have gone through some
  heavy corruption. It is also unclear if the first half of the \textit{pāda} is to be interpreted as
  \textit{vānaprastham ayaṃ}, \textit{vānaprastho 'yaṃ} [\textit{sevet}], \textit{vānaprastham idaṃ}, or \textit{vānaprasthamayaṃ}. I translate the
  first of these options, taking both \textit{ayaṃ} and \textit{dharmaṃ} as neuter nominative.
 Word final syllables treated as heavy also appear in \textit{pāda}s bcd: °\textit{haraṇam} {\rm (}twice{\rm )},
  °\textit{karam}, and °\textit{haram}.
 }}

  \subchptr{parivrājakaḥ}%

  \trsubchptr{The wandering mendicant}%

  \maintext{parivrājakadharmo 'yaṃ kīrtayiṣyāmi tac chṛṇu |}%

  \maintext{sukhaduḥkhaṃ samaṃ kṛtvā lobhamohavivarjitaḥ }||\thinspace11:43\thinspace||%
\translation{Here follows the wandering religious mendicant's Dharma. Listen, I shall teach it to you. Making joy and pain equal, he gets rid of greed and folly. }

  \maintext{varjayen madhu māṃsāni paradārāṃś ca varjayet |}%

  \maintext{varjayec ciravāsaṃ ca paravāsaṃ ca varjayet }||\thinspace11:44\thinspace||%
\translation{He should avoid honey and meat, as well as others' wives. He should avoid staying [at one place] for long and also staying at others' places. }

  \maintext{varjayet sṛṣṭabhojyāni bhikṣām ekāṃ ca varjayet |}%

  \maintext{varjayet saṃgrahaṃ nityam abhimānaṃ ca varjayet }||\thinspace11:45\thinspace||%
\translation{He should avoid food that has been thrown away and he should avoid getting alms [always] from the same household. He should always refrain from accumulating wealth and from self-conceit. \blankfootnote{11.45 See the term \textit{arthasaṃgraha} in 11.37c, probably in the same meaning as
  \textit{saṃgraha} here in \textit{pāda} c.
 }}

  \maintext{susūkṣmaṃ manasā dhyātvā dṛśau pādaṃ vinikṣipet |}%

  \maintext{na kupyeta anālābhe lābhe vāpi na harṣayet }||\thinspace11:46\thinspace||%
\translation{Meditating on the subtle one, he should cast his eyes on his feet [when begging]. He should not get angry when he does not receive anything, and when he does, he should not rejoice. \blankfootnote{11.46 On meditation on the subtle one {\rm (}\textit{susūkṣma}{\rm )}, see Intro\verify.
 
  \textit{Pāda} b is suspect as it is transmitted in the MSS {\rm (}in most sources:
  \textit{śucau pādaṃ vinikṣipet}: `he should place his foot in the pure'?{\rm )}. 
  My conjecture {\rm (}\textit{dṛśau}{\rm )} results in something close to 
  the early Buddhist rule given in the Pāli \Patimokkha\ on begging that says
  that the monk should not make eye-contact with the donor. See \Patimokkha\ Sekhiyā 7--8 and 28:
  \textit{okkhittacakkhu antaraghare gamissāmīti sikkhā karaṇīyā.
  okkhittacakkhu antaraghare nisīdissāmīti sikkhā karaṇīyā. [...]
  pattasaññī piṇḍapātaṃ paṭiggahessāmīti sikkhā karaṇīyā.}
  In Bhikkhu Ñāṇatusita's translation {\rm (}\mycitep{Patimokkha}{294 and 303}{\rm )}:
  ` ``I shall go with the eyes cast down inside an inhabited area,'' thus the training is to be done.
  ``I shall sit with the eyes cast down inside an inhabited area,'' thus the training is to be done. [...]
  ``I shall accept alms-food paying attention to the bowl,'' thus the training is to be done.'
  The last of these sentences opens up another possibility for emending the text of the \VSS:
  \textit{pādaṃ} might perhaps be considered as a corruption from \textit{pātraṃ} {\rm (}`on his bowl'{\rm )}.
  I am not aware of similar Dharmaśāstric teachings on avoiding eye-contact. 
  The closest could be \BAUDHDHS\ 1.5.11 on observing silence while begging {\rm (}\textit{vāgyatas tiṣṭhet}{\rm )}.
  Not even \MANU\ 5.50--60, a longer section on begging, prohibits eye-contact.
  If there are indeed no Brahmanical rules on this topic, the verse above in the \VSS\ 
  could be another piece of evidence for Buddhist influence.
 }}

  \maintext{arthatṛṣṇāsv anudvigno roṣe vāpi sudāruṇe |}%

  \maintext{stutinindā samaṃ kṛtvā priyaṃ vāpriyam eva vā }||\thinspace11:47\thinspace||%
\translation{He should not be agitated with regards to thirst for material things, or to violent anger. He should take praise and reproach equal, as well as pleasant and unpleasant things. \blankfootnote{11.47 In \textit{pāda} c, understand \textit{stutinindā} as a dual {\rm (}or singular{\rm )} accusative.
 }}

  \maintext{niyamās tu parīdhānaṃ saṃyamāvṛtamekhalaḥ |}%

  \maintext{nirālambaṃ manaḥ kṛtvā buddhiṃ kṛtvā nirañjanām }||\thinspace11:48\thinspace||%
\translation{His garment is the Niyama-rules, and he is girded by the girdle of constraint {\rm (}\textit{saṃyama}{\rm )}. He makes his mind supportless, his intellect spotless, \blankfootnote{11.48 On \textit{saṃyama}, see notes on 11.14 above.
 }}

  \maintext{ātmānaṃ pṛthivīṃ kṛtvā khaṃ ca kṛtvā manonmanam |}%

  \maintext{tridaṇḍaṃ triguṇaṃ kṛtvā pātraṃ kṛtvākṣaro 'vyayaḥ }||\thinspace11:49\thinspace||%
\translation{the ground his self, the sky the \textit{manonmana} [state of mind], the three staffs [of the \textit{parivrājaka}] the three qualities {\rm (}\textit{guṇa}{\rm )}, the bowl the imperishable syllable. \blankfootnote{11.49 \textit{°kṣaram avyayam} in \textit{pāda} d would be hypermetrical, that is probably why the nominative appears here.
 }}

  \maintext{nyased dharmam adharmaṃ ca īrṣyādveṣaṃ parityajet |}%

  \maintext{nirdvandvo nityasatyastho nirmamo nirahaṃkṛtaḥ }||\thinspace11:50\thinspace||%
\translation{He should throw away Dharma and Adharma, and should give up envy and hatred. He should be indifferent to opposites, always dwell in truthfulness, being unselfish, humble. }

  \maintext{divasasyāṣṭame bhāge bhikṣāṃ saptagṛhaṃ caret |}%

  \maintext{na cāsīta na tiṣṭheta na ca dehīti vā vadet }||\thinspace11:51\thinspace||%
\translation{He should go on his alms round visiting seven houses at the eighth part of the day. He should not sit down, he should not stay, and he should not say `Give me!' }

  \maintext{yathālābhena varteta aṣṭau piṇḍān dine dine |}%

  \maintext{vastrabhojanaśayyāsu na prasajyeta vistaram }||\thinspace11:52\thinspace||%
\translation{He should live on what is available, on eight bites a day. He should not stick to items of clothes, food, or a bed, for long. }

  \maintext{nābhinandeta maraṇaṃ nābhinandeta jīvitam |}%

  \maintext{indriyāṇi vaśaṃkṛtvā kāmaṃ hatvā yatavrataḥ }||\thinspace11:53\thinspace||%
\translation{He should nor rejoice in death, he should not rejoice in life. Having conquered his senses, and having killed his desire, firm in his observances, }

  \maintext{atītaṃ ca bhaviṣyaṃ ca na bhikṣuś cintayet sadā |}%

  \maintext{krodhamānamadadarpān parivrāḍ varjayet sadā }||\thinspace11:54\thinspace||%
\translation{the mendicant {\rm (}\textit{bhikṣu}{\rm )} should never think about the past or the future. The wandering mendicant should always avoid anger, self-conceit, intoxication, and pride. \blankfootnote{11.54 \textit{Pāda} c is a \textit{sa-vipulā}, which is rare and is usually treated as 
  unmetrical.
 }}

  \maintext{virāgaṃ tu dhanuḥ kṛtvā prāṇāyāmaguṇair yutam |}%

  \maintext{dhāraṇāśaratīkṣṇena mṛgaṃ hatvā manendriyam }||\thinspace11:55\thinspace||%
\translation{Making indifference a bow which is strung with the strings of breath-control, he should kill the beast that is the mind and the sense-faculties, with the sharp-pointed arrow of concentration. \blankfootnote{11.55 Understand \textit{pāda} c as \textit{dhāraṇātīkṣṇaśareṇa}.
 }}

  \maintext{maitrīkhaḍgasutīkṣṇena saṃsārāriṃ nikṛntayet |}%

  \maintext{karuṇāvartacakreṇa krodhamattagajaṃ jayet |}%

  \maintext{muditāvarmabaddhāṅgas tūṇaṃ pūrṇam upekṣayā }||\thinspace11:56\thinspace||%
\translation{He should stab the enemy that is transmigration with the extremely sharp knife of friendliness. He should defeat the rutting elephant of anger with the whirling discus of compassion. His body is clad in the armour of sympathy, his quiver is full of equanimity. \blankfootnote{11.56 Understand \textit{pāda} a as \textit{maitrīsutīkṣṇakhaḍgena}, which is even metrical.
 Note the four Buddhist \textit{brahmavihāra}s, \textit{maitrī}, \textit{karuṇā}, \textit{muditā}, and \textit{upekṣā},
  mentioned in this verse. They appear also in verses 4.72 and 11.56 above.
 }}

  \maintext{anakṣaraṃ paraṃ brahma cintayet satataṃ dvija |}%

  \maintext{brahmaṇo hṛdayaṃ viṣṇur viṣṇoś ca hṛdayaṃ śivaḥ |}%

  \maintext{śivasya hṛdayaṃ saṃdhyā tasmāt saṃdhyām upāsayet }||\thinspace11:57\thinspace||%
\translation{He should constantly recall the unutterable syllable which is the supreme Brahman, O Brahmin. Brahmā's heart is Viṣṇu. Viṣṇu's heart is Śiva. Śiva's heart is the junctures of the day. Therefore he should worship the junctures. }

  \maintext{saṃsārārṇavatāraṇaṃ śubhagatiḥ sa brahma saṃdhyākṣaraṃ}%

 \nonanustubhindent \maintext{dhyāyen nityam atandrito hy anupamaṃ vyaktātmavedyaṃ śivam |}%

  \maintext{rūpair varṇaguṇādibhiś ca vihitaṃ durlakṣyalakṣyottamaṃ}%

 \nonanustubhindent \maintext{yatnoddhṛtya samāśrayet suraguruṃ sarvārtihartā haram }||\thinspace11:58\thinspace||%
\translation{[Śiva] is deliverance from the ocean of mundane existence, the path to happiness, the Brahman, the junctures, the [sacred] syllable. One should always, unweariedly, meditate on matchless Śiva, who is to be recognized as the manifest soul. He should take refuge in Hara, who is devoid of form, colour, qualities etc., who is the supreme aim which is difficult to discern, honouring the divine guru with effort, who removes all pain. \blankfootnote{11.58 Note \textit{vihita} in \textit{pāda} c probably in the sense of `devoid of.'
 I take \textit{yatnoddhṛtya} in \textit{pāda} d as \textit{yatnenoddhṛtya}, \textit{yatna} being in stem form, and
  °\textit{hartā} as nominative for accusative.
 }}
\center{\maintext{\dbldanda\thinspace iti vṛṣasārasaṃgrahe caturāśramadharmavidhāno nāmādhyāya ekādaśamaḥ\thinspace\dbldanda}}
\translation{Here ends the eleventh chapter in the \textit{Vṛṣasārasaṃgraha} called the Regulations concerning the four life-stages.}

  \chptr{dvādaśamo 'dhyāyaḥ}
\addcontentsline{toc}{subsection}{Chapter 12}
\fancyhead[CO]{{\footnotesize\textit{Translation of chapter 12}}}%

  \trchptr{ Chapter Twelve }%

  \subchptr{ātithyadharmaḥ}%

  \trsubchptr{The rules of hospitality}%

  \maintext{devy uvāca |}%

  \maintext{ahiṃsā paramo dharmaḥ satataṃ parikīrtyate |}%

  \maintext{ātithyakānāṃ dharmaṃ ca kathayasva yad uttamam }||\thinspace12:1\thinspace||%
\translation{The Goddess spoke: Non-violence is always praised as the highest Dharma. Also, teach me the ultimate Dharma of the hospitable ones. }

  \maintext{maheśvara uvāca |}%

  \maintext{ahiṃsātithyakānāṃ ca śṛṇu dharmaṃ yad uttamam |}%

  \maintext{trailokyam akhilaṃ devi ratnapūrṇaṃ sulocane }||\thinspace12:2\thinspace||%
\translation{Maheśvara spoke: Hear the ultimate Dharma of non-violence and that of hospitality. O beautiful-eyed goddess, [if] all the three worlds, full of wealth, \blankfootnote{12.2 Understand \textit{ahiṃsātithyakāmāṃ} as \textit{ahiṃsakānām ātithyakānāṃ ca} or
  \textit{ahiṃsāyā ātithyakāṇāṃ ca}.
 }}

  \maintext{caturvedavide dānaṃ na tattulyam ahiṃsakaḥ |}%

  \maintext{śṛṇu dharmam atithyānāṃ kīrtayiṣyāmi sundari }||\thinspace12:3\thinspace||%
\translation{[were handed over as] a gift to [a Brahmin who] knows the four Vedas, [that gift] cannot be compared to somebody who avoids doing harm. Hear the Dharma of the hospitable ones. I shall teach it [to you], O beautiful one. \blankfootnote{12.3 Note that this verse seems to be all that Maheśvara teaches in this chapter on 
  \textit{ahiṃsā}, and that \textit{tattulyam ahiṃsakaḥ} may either contain a sandhi bridge
  {\rm (}\textit{tattulya-m-ahiṃsakaḥ}{\rm )} or be interpreted as \textit{dānaṃ na tat tulyam ahiṃsakena} 
  {\rm (}`that gift is not comparable to a non-violent person'{\rm )}.
 \textit{atithyānāṃ} in \textit{pāda} c stands for \textit{ātithyānāṃ} or \textit{ātithyakānāṃ} metri causa.
 }}

  \subchptr{vipulopākhyānam}%

  \trsubchptr{The Story of Vipula}%

  \maintext{āsīd vṛttaṃ purākhyānaṃ nagare kusumāhvaye |}%

  \maintext{kapilasya suto vidvān vipulo nāma viśrutaḥ }||\thinspace12:4\thinspace||%
\translation{This is an old story of what happened once in a city called Kusuma. There was a famous and wise man called Vipula, Kapila's son. \blankfootnote{12.4 Kusumapura is Pāṭaliputra, or modern Patna. This is confirmed in verse 12.12, where 
  the confluence of the Gaṇḍakī and the Gaṅgā is mentioned as a local spot.
  The \textit{dramatis person\ae} in the following story are the following:
  Vipula---a merchant, Kapila's son;
  Vipula's wife;
  a Brahmin guest {\rm (}Dharma in diguise?{\rm )};
  a monkey;
  Bhīmabala---a traveller;
  Puṇḍaka---the foremand of the guild;
  King Siṃhajaṭa;
  Queen Kekayī;
  Caṇḍa and Vicaṇḍa---two envoys of the king;
  Citraratha---the king of the Gandharvas;
  Sūrya, Soma, Indra, Viṣṇu, Brahmā---gods.
 }}

  \maintext{dharmanityo jitakrodhaḥ satyavādī jitendriyaḥ |}%

  \maintext{brahmaṇyaś ca kṛtajñaś ca madbhaktaḥ kṛtaniścayaḥ }||\thinspace12:5\thinspace||%
\translation{He always followed Dharma, he conquered anger, he spoke only the truth, and he conquered his senses. He was pious and grateful, and he was my determined devotee. \blankfootnote{12.5 \textit{Pāda} d implies that Vipula is a Śaiva devotee, but there is little indication 
  in this story of Vipula's affiliation, except for 11.44, where Maheśvara is mentioned.
  The story as we have it here ends with a praise of Brahmā.
 }}

  \maintext{dhanāḍhyo 'tithipūjyaś ca dātā dānto dayālukaḥ |}%

  \maintext{nyāyārjitadhano nityam anyāyaparivarjitaḥ }||\thinspace12:6\thinspace||%
\translation{He was rich and he worshipped his guests. He was generous, restrained, and kind. His wealth always came through just means. He always stayed away from illegal transactions. \blankfootnote{12.6 While one would normally translate \textit{atithipūjya} {\rm (}in \textit{pāda} a{\rm )} as `to be worshipped by guests,'
  in the light of the story I suspect that the intended meaning is that he worshipped his guests.
 }}

  \maintext{bhāryā ca rūpiṇī tasya candrabimbaśubhānanā |}%

  \maintext{pīnottuṅgastanī kāntā sakalānandakāriṇī |}%

  \maintext{pativratā patiratā patiśuśrūṣaṇe ratā }||\thinspace12:7\thinspace||%
\translation{He had a pretty wife whose face was as beautiful as the disk of the moon. Her breasts were round and elevated, she was lovely, a source of all pleasures. She was faithful, devoted to her husband and his needs. }

  \maintext{atha kenāpi kālena sūryarāga{-}m{-}abhūt tataḥ |}%

  \maintext{grastabhāgatrayas tv āsīt kṛṣṇamādhavamāsike }||\thinspace12:8\thinspace||%
\translation{Now, once there was an eclipse of the sun. Three quarters [of the sun] were eclipsed, and it was in the dark half of the month of Mādhava [April-May]. \blankfootnote{12.8 In \textit{pāda} b, understand \textit{sūryarāgam} as \textit{sūryoparāgaḥ} {\rm (}`eclipse of the sun'{\rm )}. 
  I take °\textit{rāga-m-abhūt} an example of irregular sandhi for °\textit{rāgo 'bhūt}.
 }}

  \maintext{snātukāmāvatīryante sarve pauranṛpādayaḥ |}%

  \maintext{devāś ca pitaraś caiva tarpyante vidhivat tathā }||\thinspace12:9\thinspace||%
\translation{Eager to take a ritual bath, the king and all the citizens the went down [to the riverbank]. Then they worshipped the gods and the deceased ancestors according to the rules. \blankfootnote{12.9 Understand \textit{pāda} a as \textit{snātukāmā avatīryante}. It is an instance of double sandhi or
  of a stem form noun in sandhi with the following verb.
 }}

  \maintext{kecij juhvati tatrāgniṃ kecid viprāṃś ca tarpayet |}%

  \maintext{kecid dānopatiṣṭhanti kecit stuvanti devatām }||\thinspace12:10\thinspace||%
\translation{Some sacrificed in the fire, some fed the Brahmins, some were of service with donations, others praised the deity. \blankfootnote{12.10 Understand \textit{agniṃ} in \textit{pāda} a as locative, and \textit{tarpayet} in \textit{pāda} b as plural.
 Note \textit{dāna} in \textit{pāda} c in stem form {\rm (}for the instrumental{\rm )}.
 }}

  \maintext{dhyānayogaratāḥ kecit kecit pañcatape ratāḥ |}%

  \maintext{evaṃ pravartamāneṣu rājanādiṣu sarvaśaḥ }||\thinspace12:11\thinspace||%
\translation{Some people practised yoga meditation, others were engrossed in five-fire penance. While the ritual waving of lamps etc. were being performed all around the place, \blankfootnote{12.11 \textit{rājanādiṣu} in \textit{pāda} d is suspect. The intended meaning may be
  `the royals and other people,' but I prefer now the option to
  take it as a shortened form of \textit{nīrājanādiṣu}, and that is how
  I translate it. Cf., e.g., \SIVP\ 7.30.81cd: 
  \textit{nīrājanādikaṃ kṛtvā pūjāśeṣaṃ samāpayet}.
 }}

  \maintext{vipulo 'pi hi tatraiva gaṅgāgaṇḍakisaṃgame |}%

  \maintext{bhāryayā saha tatraiva snātvā kṣomavibhūṣaṇaḥ }||\thinspace12:12\thinspace||%
\translation{Vipula also, there at the confluence of the Gaṅgā and the Gaṇḍakī, attired in linen clothes, performing a bath, together with his wife, \blankfootnote{12.12 Note \textit{gaṇḍaki} metri causa for \textit{gaṇḍakī} in \textit{pāda} b.
 }}

  \maintext{devatāguruviprāṇām anyeṣāṃ tarpaṇe rataḥ |}%

  \maintext{tatrāvasarasamprāpto brāhmaṇo 'tithir āgataḥ }||\thinspace12:13\thinspace||%
\translation{was engrossed in satiating the deities, the gurus, the Brahmins and others. Then, jumping on the possibility, a Brahmin came up [to them] as a guest. }

  \maintext{bhāryā tasyātirūpeṇa mohitā brahmaṇas tadā |}%

  \maintext{brāhmaṇo 'pi tathaiveha rūpeṇāpratimo bhavet }||\thinspace12:14\thinspace||%
\translation{The wife got infatuated with that Brahmin's extreme beauty. The Brahmin [felt] the same. His beauty was unparalleled. \blankfootnote{12.14 \textit{Pāda} d is suspect and the translation of \textit{pāda}s cd is
  tentative. The expression \textit{rūpeṇāpratimo/°pratimā bhuvi} {\rm (}`his/her beauty is unparalleled in the world'{\rm )} is 
  common in the \MBH\ and in the Purāṇas. Is that what was meant here?
  May a dual have been intended? An alternative reading, albeit requiring substantial
  emendations, could be: \textit{brāhmaṇo 'pi tathaivāha rūpeṇāpratimā bhuvi}; 
  `The Brahmin [felt the same] and said [to himself,] her figure is unparalleled in the world.'
  Nevertheless, I retained the reading found in the MSS, and I interpret \textit{pāda} d
  as an indication that this Brahmin was extraordinary, in fact a manifestation of
  Dharma.
 }}

  \maintext{anyonyadṛṣṭisaṃsaktau jātau tau tu parasparam |}%

  \maintext{vipulenāñjaliṃ kṛtvā brāhmaṇa saṃśitavrata }||\thinspace12:15\thinspace||%
\translation{Their gaze got fixed on each other mutually. Vipula joined his hands [and said:] `O virtuous Brahmin, }

  \maintext{ājñāpaya dvijaśreṣṭha adya me 'nugrahaṃ kuru |}%

  \maintext{bhāryābhṛtyapaśugrāma ratnāni vividhāni ca }||\thinspace12:16\thinspace||%
\translation{I am at your service, be gracious to me now, O great Brahmin. [My] wife, servants, cattle, village and all kinds of jewels [are all at your service].' \blankfootnote{12.16 °\textit{grāma} in \textit{pāda} c is in stem form, although it would be unproblematic
  to correct it to the neuter singular {\rm (}to form a \textit{samāhāra\-samāsa}{\rm )}.
 }}

  \maintext{vipulenaivam uktas tu gṛhīto brāhmaṇo 'bravīt |}%

  \maintext{yadi satyaṃ pradātāsi suprasannaṃ manas tava }||\thinspace12:17\thinspace||%
\translation{Having been addressed and greeted thus, in a hospitable way, by Vipula, the Brahmin spoke: `If you really mean to give, your heart is very generous.' \blankfootnote{12.17 Note that \msCc's omission of \textit{pāda}s cd here could be due to an eye-skip from \textit{suprasannaṃ} in
  12.17d to \textit{suprasannaṃ} in 12.18a, although this would have also led to an omission of
  the next \textit{vipula uvāca}.
 }}

  \maintext{vipula uvāca |}%

  \maintext{suprasannaṃ mano me 'dya suprasannaṃ tapaḥphalam |}%

  \maintext{śīghram ājñāpaya vipra yac cābhilaṣitaṃ tava |}%

  \maintext{adeyaṃ nāsti viprasya svaśiraḥprabhṛti dvija }||\thinspace12:18\thinspace||%
\translation{Vipula spoke: `My heart is generous today, generosity is the fruit of austerity. Just command me quickly, O Brahmin. What is your desire? There is nothing that should not be donated to a Brahmin, including one's own head, O Brahmin.' \blankfootnote{12.18 \textit{Pāda} c is either a \textit{sa-vipulā} or by applying the muta cum liquida metrical licence,
  by which °\textit{pra} does not make \textit{vi}° a long syllable, a \textit{na-vipulā}.
 }}

  \maintext{brāhmaṇa uvāca |}%

  \maintext{yady evaṃ vadase bhadra bhāryāṃ me dehi rūpiṇīm |}%

  \maintext{svasti bhavatu bhadraṃ vaḥ kalyāṇaṃ bhava śāśvatam }||\thinspace12:19\thinspace||%
\translation{The Brahmin spoke: `If you talk like this, dear Sir, give me your beautiful wife. May there be happiness, may you be fortunate, and may you prosper eternally!' \blankfootnote{12.19 \textit{Pāda} c has the metrical fault of two \textit{laghu}s in the second and third position.
 
  In \textit{pāda} d, \textit{bhava} is less than satisfactory. One would normally expect 
  \textit{bhavate/bhavatāṃ/bhavatu} in this context. Alternatively, it is possible that
  \textit{kalyāṇo bhava} {\rm (}`be happy'{\rm )} was meant, or \Ed's reading {\rm (}\textit{tava}{\rm )}
  could be accepted as a conjecture.
 }}

  \maintext{vipula uvāca |}%

  \maintext{pratīccha bhāryāṃ suśroṇīṃ rūpayauvanaśālinīm |}%

  \maintext{akutsitāṃ viśālākṣīṃ pūrṇacandranibhānanām }||\thinspace12:20\thinspace||%
\translation{Vipula spoke: `Accept my nice-buttocked, young and beautiful wife, who is blameless, large-eyed and whose face resembles the full-moon.' }

  \maintext{bhāryovāca |}%

  \maintext{parityājyā kathaṃ nātha apāpāṃ tyajase katham |}%

  \maintext{atīva hi priyāṃ bhāryāṃ nirdoṣāṃ ca kathaṃ tyajeḥ }||\thinspace12:21\thinspace||%
\translation{The wife spoke: `How can you abandon me, my lord? How can you dismiss a woman who is sinless? How can you abandon a wife who is extremely kind and faultless? \blankfootnote{12.21 All witnesses consulted read \textit{sa} instead of my conjectured \textit{ca} in \textit{pāda} d.
  \textit{sa} might work if we read \textit{tyajet} {\rm (}\msCb\msCc{\rm )} instead of \textit{tyajeḥ} {\rm (}\msCa\msNa\msNc{\rm )},
  but even this version sounds a bit out of context {\rm (}`how can he abandon...'{\rm )}.
 }}

  \maintext{sakhā bhāryā manuṣyāṇām iha loke paratra ca |}%

  \maintext{dānaṃ vā sumahad dattvā yajño vā subahuḥ kṛtaḥ }||\thinspace12:22\thinspace||%
\translation{A wife is a man's friend in this world and in the other world. [Even if] a man gives enormous donations or performs numerous sacrifices, }

  \maintext{aputro nāpnuyāt svargaṃ tapobhir vā suduṣkaraiḥ |}%

  \maintext{śruto me pitṛbhiḥ prokto brāhmaṇaiś ca mamāntike }||\thinspace12:23\thinspace||%
\translation{or performs hard penance, he cannot reach heaven without having a son. I have heard this as taught by my father and my uncles, and by Brahmins in my presence. \blankfootnote{12.23 Note \textit{me} as instrumental in \textit{pāda} c. I translate \textit{pitṛbhiḥ} in the same \textit{pāda} as
  `father and uncles,' and not as `ancestors' because the former fits the context better.
 }}

  \maintext{aputro nāpnuyāt svargaṃ śrutaṃ me bahuśaḥ purā |}%

  \maintext{mandapālo dvijaśreṣṭho gataḥ svargaṃ tapobalāt }||\thinspace12:24\thinspace||%
\translation{A sonless man cannot reach heaven. I have heard this so many times! Mandapāla, the great Brahmin, went to heaven as a reward of his austerities, \blankfootnote{12.24 See details of Mandapāla's story, here summarised, in \MBH\ 1.220.5ff.
 }}

  \maintext{dānāni ca bahūn dattvā yajñāṃś ca vividhāṃs tathā |}%

  \maintext{vedāṃś ca japayajñāṃś ca kṛtvā sa dvijasattamaḥ }||\thinspace12:25\thinspace||%
\translation{having made numerous donations, having performed various sacrifices, Vedic sacrifices and sacrifices of recitation, that great Brahmin. \blankfootnote{12.25 On \textit{japayajña}, see \VSS\ 6.1--2 and 5 above, as well as, e.g.,
  \BHG\ 10.25c {\rm (}\textit{yajñānāṃ japayajño 'smi}{\rm )} and \MANU\ 2.86 
  {\rm (}\textit{vidhiyajñāj japayajño viśiṣṭo daśabhir guṇaiḥ}{\rm )}.
  Understand \textit{pāda} c as \textit{vedayajñāñ japayajñāṃś ca kṛtvā}.
  {\rm (}See \textit{vedayajña} mentioned in \VSS\ 3.37a above.{\rm )}
 }}

  \maintext{prāptadvāro 'pi yasyāpi devadūtair nivāritaḥ |}%

  \maintext{aputro nāpnuyāt svargaṃ yadi yajñaśatair api }||\thinspace12:26\thinspace||%
\translation{But even he, even when he reached the gate [of heaven], was stopped by the celestial messengers [saying:] ``The sonless cannot enter heaven, not even by hundreds of sacrifices.'' \blankfootnote{12.26 \textit{Pāda}s ab are not perfectly smooth syntactically, \textit{yasyāpi} is difficult to fit in.
  Perhaps understand \textit{prāptadvāre 'pi yasmin sa devatūtair nivāritaḥ}. Alternatively, 
  \textit{yasya} might reference \textit{svargaḥ}.
 }}

  \maintext{ity uktas tu cyutaḥ svargān mandapālo mahān ṛṣiḥ |}%

  \maintext{putrān utpādayām āsa śāraṅgāṃś caturo dvijaḥ }||\thinspace12:27\thinspace||%
\translation{Mandapāla, the great sage, having been thus informed fell from heaven. The Brahmin begot four sons with a Śāraṅga-bird. }

  \maintext{tena puṇyaprabhāveṇa svargaṃ prāpto hy avāritaḥ |}%

  \maintext{kulatrāṇāt kalatrāsmi bharaṇād bhārya eva ca }||\thinspace12:28\thinspace||%
\translation{By the virtue of this, he reached heaven unobstructed. I am a wife {\rm (}\textit{kalatra}{\rm )} [because] I protect the family {\rm (}\textit{kulatrāṇa}{\rm )}, and I am a wife to be supported {\rm (}\textit{bhārya}{\rm )} because I bear [sons] {\rm (}\textit{bharaṇa}{\rm )}. \blankfootnote{12.28 Note that \textit{pāda} c is the result of emendations 
  {\rm (}the majority of the MSS read \textit{kalatrāṇāṃ kalatrāsmi}{\rm )},
  and that \textit{bhārya} in \textit{pāda} d is to be understood as \textit{bhāryā} 
  metri causa. I added `to be supported' in the translation to convey 
  the general meaning of the word \textit{bhārya}, 
  which seemed to fit the context well.
 }}

  \maintext{dārasaṃgraha putrārthe kriyate śāstradarśanāt |}%

  \maintext{yāni santi gṛhe dravyaṃ grāmaghoṣagṛhāṇi ca }||\thinspace12:29\thinspace||%
\translation{Taking a wife is for the sake of having sons according to the Śāstras. Please give that Brahmin all the wealth at home, the village, the stations of herdsmen, and the houses, \blankfootnote{12.29 Note the stem form °\textit{saṃgraha} metri causa in \textit{pāda} a.
 Note the number discrepancy between \textit{yāni santi} and \textit{dravyaṃ} in \textit{pāda} c,
  which is repeated in 12:42a.
 }}

  \maintext{dātum arhasi viprāya na māṃ dātum ihārhasi |}%

  \maintext{bhāryāyā vacanaṃ śrutvā vipulaḥ punar abravīt }||\thinspace12:30\thinspace||%
\translation{but please don't give me away this time!' Having heard his wife's speech, Vipula spoke again. \blankfootnote{12.30 I have not included \msCcpcorr's \textit{vipula uvāca} {\rm (}echoed in \Ed{\rm )}
  because after \textit{punar abravīt} is seems secondary and unnecessary.
  Note that the correction in \msCc\ is in a second hand.
 }}

  \maintext{sādhu bhāmini jānāmi sādhu sādhu pativrate |}%

  \maintext{jito 'smy anena vākyena anenāsmi hi toṣitaḥ }||\thinspace12:31\thinspace||%
\translation{`Alright, my beautiful wife, I know! Good, good, my faithful wife! I am beaten by this speach and I am satisfied with it. }

  \maintext{adya grahaṇakāle ca dvija āgatya yācate |}%

  \maintext{dadāmīti pratijñāya adattvā narakaṃ vraje }||\thinspace12:32\thinspace||%
\translation{Today the Brahmin came up to me at the time of eclipse, and he asked me. I promised him that I would give [you away]. If I didn't give [you to him], I would go to hell. }

  \maintext{narakaṃ yadi gacchāmi kulena saha sundari |}%

  \maintext{kalpakoṭisahasre 'pi narakastho yaśasvini |}%

  \maintext{muktim eva na paśyāmi janmakoṭiśatair api }||\thinspace12:33\thinspace||%
\translation{If I go to hell along with my family, I will be in hell, O brilliant woman, for millions of \ae ons, and will not see release for millions of births. \blankfootnote{12.33 The reading \textit{narakastho} in \textit{pāda} b {\rm (}\msNc\Ed{\rm )} might not be the original one but it is 
  definitely the simplest solution. \textit{narakasthād} may be original, possibly meaning \textit{narakasthānād}.
 }}

  \maintext{adānāc cāśubhaṃ devi paśyāmi varavarṇini |}%

  \maintext{dānena tu śubhaṃ paśye svargaloke yad akṣayam }||\thinspace12:34\thinspace||%
\translation{I can see something bad, my Princess, from not giving, O woman with a nice compexion, but from giving I can see something good in heaven that is eternal. }

  \maintext{noktaṃ mayānṛtaṃ pūrvaṃ nityaṃ satyavrate sthitaḥ |}%

  \maintext{satyadharmam atikramya nānyadharmaṃ samācare }||\thinspace12:35\thinspace||%
\translation{I have never ever lied, I always observe the vow of truthfulness. If I transgressed the Dharma of truthfullness, [by this] I would stop following all other Dharmas [too]. }

  \maintext{bhāryā dharmasakhety evaṃ tvayā pūrvam udāhṛtam |}%

  \maintext{yadi dharmasakhāyāsi so 'dya kāla ihāgataḥ }||\thinspace12:36\thinspace||%
\translation{You mentioned earlier that the wife is one's Dharmic friend. If you are indeed Dharma's friend, it was actually the perfect time for him to come up to us today. \blankfootnote{12.36 I have emended \textit{tvayi} in \textit{pāda} d to \textit{tvayā} because it 
  seems an early random scribal mistake, rather than some 
  linguistic pecularity.
 Note the form \textit{sakhāyā} for a feminine \textit{sakhī} or \textit{sahāyā}. I sense a touch
  of humour or sarcasm in Vipula's spin on his wife's claim in 12.22a that
  `a wife is a man's friend': now he suggests that his wife, his `Dharmic friend,'
  is actually friends with Dharma.
 }}

  \maintext{dvijarūpadharo dharmaḥ svayam eva ihāgataḥ |}%

  \maintext{jijñāsārtham ahaṃ bhadre na vighnaṃ kartum arhasi }||\thinspace12:37\thinspace||%
\translation{[For] Dharma himself visited us, disguised as a Brahmin. I am being tested. My dear, please don't cause me trouble. }

  \maintext{mātāvyaktaḥ pitā brahmā buddhir bhāryā damaḥ sakhā |}%

  \maintext{putro dharmaḥ kriyācārya ity ete mama bāndhavāḥ }||\thinspace12:38\thinspace||%
\translation{The Unmanifest {\rm (}\textit{avyakta}{\rm )} is my mother, Brahmā is my father, intelligence is my wife, self-control is my friend. Dharma is my son, ritual is my teacher. These are my relatives. }

  \maintext{kālaśreṣṭho grahaḥ sūryo gaṅgā śreṣṭhā nadīṣu ca |}%

  \maintext{candrakṣaye dinaṃ śreṣṭhaṃ naraśreṣṭho dvijottamaḥ }||\thinspace12:39\thinspace||%
\translation{The best time is the time of the eclipse of the Sun. The best one among the rivers is the Gaṅgā. The best day is at new moon, the best man is the Brahmin. \blankfootnote{12.39 I understand \textit{grahaḥ sūryo} in \textit{pāda} a as \textit{sūryagrahaḥ} {\rm (}or \textit{sūryagrahaṇam}{\rm )}:
  the eclipse of the Sun, which appears to be an auspicious day.
  See, e.g., \AgamaKL\ 3.128:
  \textit{sūryagrahaṇakālasya samānā nāsti bhūtale\thinspace |
  atra yad yat kṛtaṃ karma anantaphaladaṃ bhavet\thinspace ||}
 This short list of `best of' items anticipates \VSS\ 15.16--29, a longer
  list of what is best in every possible category, not entirely differently from 
  the manner of \BHG\ 10.21--38.
 }}

  \maintext{śuśrūṣaṇārthaṃ viprasya mayā dattāsi sundari |}%

  \maintext{sarvasvaṃ brāhmaṇe dattvā vanam evāśrayāmy aham }||\thinspace12:40\thinspace||%
\translation{I have given you to the Brahmin to serve him, O beautiful woman. After I have given all my riches to the Brahmin, I shall resort to the forest.' \blankfootnote{12.40 \textit{Pāda} d may give a hint at the connection between this chapter and the
  end of the previous one: this story is partly a propagation 
  of the life of the \textit{vānaprastha}.
 }}

  \maintext{śaṅkara uvāca |}%

  \maintext{tūṣṇīmbhūtā tato bhāryā aśrupūrṇākulekṣaṇā |}%

  \maintext{kare gṛhya viśālākṣī brāhmaṇāya niveditā }||\thinspace12:41\thinspace||%
\translation{Śaṅkara spoke: The wife remained silent, her bewildered eyes filled with tears. [Vipula] took her by the hand and the long-eyed woman was presented to the Brahmin. }

  \maintext{yāni santi gṛhe dravyaṃ hiraṇyaṃ paśavas tathā |}%

  \maintext{dadāmi te dvijaśreṣṭha grāmaghoṣagṛhādikam }||\thinspace12:42\thinspace||%
\translation{`I am ready to give you all the wealth I have at home, all the gold and the cattle, O great Brahmin, the village, the stations of herdsmen, and the houses, and everything else, }

  \maintext{muktāvaiḍūryavāsāṃsi divyāṇy ābharaṇāni ca |}%

  \maintext{sarvān gṛhāṇa viprendra śraddhayā dattasatkṛtān }||\thinspace12:43\thinspace||%
\translation{pearls, gems, clothes, and exquisite ornaments. Accept all these, O best of Brahmins. It's given in good faith and with respect. }

  \maintext{prīyatāṃ bhagavān dharmaḥ prīyatāṃ ca maheśvaraḥ |}%

  \maintext{prīyantāṃ pitaraḥ sarve yady asti sukṛtaṃ phalam }||\thinspace12:44\thinspace||%
\translation{May Lord Dharma be pleased and may Maheśvara be pleased. May all the ancestors rejoice if there is reward for meritorious acts.' \blankfootnote{12.44 Note Śivadharmaśāstra 10.11cd, in a similar context of donations:
  \textit{bhojayitvā tato brūyāt prīyatāṃ bhagavān śivaḥ}
 Understand \textit{sukṛtaṃ phalam} as \textit{sukṛtaphalam} {\rm (}metri causa{\rm )}.
 }}

  \maintext{rudra uvāca |}%

  \maintext{vipulasya vacaḥ śrutvā brāhmaṇena tapasvinā |}%

  \maintext{āśīḥ suvipulaṃ dattvā vipulāya mahātmane }||\thinspace12:45\thinspace||%
\translation{Rudra spoke: Having heard Vipula's speech, the ascetic Brahmin blessed the good-souled Vipula a good number of times, \blankfootnote{12.45 There are several ways to explain the form \textit{āśīḥ} in \textit{pāda} c.
  The easiest is to treat it as a singular accusative neuter.
  Alternatively, it could be a plural accusative feminine from \textit{āśī} and
  then \textit{suvipulaṃ} is either to be understood adverbially or as \textit{suvipulā}[\textit{ḥ}].
  Another way to treat \textit{āśīḥ} would be to take it as a nominative standing
  for the accusative.
 }}

  \maintext{vaset tatra gṛhe ramye bhāryām ādāya tasya ca |}%

  \maintext{vipulas tu namaskṛtvā kṛtvā cāpi pradakṣiṇam }||\thinspace12:46\thinspace||%
\translation{and then went off to live in a nice house, taking Vipula's wife with him. As for Vipula, he said good-bye and circulambulated him. }

  \maintext{brāhmaṇam abhivādyaivaṃ gataḥ śīghraṃ vanāntaram |}%

  \maintext{vane mūlaphalāhāro vicareta mahītale }||\thinspace12:47\thinspace||%
\translation{Thus saluting the Brahmin, he departed quickly into the forest. In the forest, he lived off roots and fruits, and roamed the world. \blankfootnote{12.47 Note the metrical problem in \textit{pāda} a {\rm (}two \textit{laghu}s{\rm )}.
 }}

  \maintext{ekākī vijane śūnye cintayā ca pariplutaḥ |}%

  \maintext{kva gacchāmi kva bhokṣyāmi kutra vā kiṃ karomy aham }||\thinspace12:48\thinspace||%
\translation{But being alone in an abandoned and deserted place, he got overwhelmed with worry. `Where should I go? Where could I find food? From whom? What shall I do? }

  \maintext{na pathaṃ viṣayaṃ vedmi grāmaṃ vā nagarāṇi vā |}%

  \maintext{kheṭakharvaṭadeśaṃ vā jānāmīha na kaṃcana }||\thinspace12:49\thinspace||%
\translation{I don't know these roads, this country, these villages and these cities, towns, mountain settlements. I don't know anybody here. \blankfootnote{12.49 In \textit{pāda} c, I accepted \Ed's reading {\rm (}°\textit{kharvaṭa}°, `a mountain village'{\rm )}
  against all witnesses consulted. The MSS transmit a reading that is 
  difficult to make sense of {\rm (}°\textit{kharpaṭa}, `ragged garment'{\rm )}.
  In \textit{pāda} d, the reading of all the witnesses, \textit{kaścana}, seems to be
  an early scribal mistake for \textit{kañcana}. But note that the same happens at 12.55d.
 }}

  \maintext{amuṃ suśailaṃ paśyāmi vipulodarakandaram |}%

  \maintext{tam āruhya nirīkṣyāmi grāmaṃ nagarapattanam }||\thinspace12:50\thinspace||%
\translation{I can see a nice mountain yonder with large cavities and caves. I'll climb it and try to figure out if there is a village, town or city [nearby].' }

  \maintext{evam uktvā tu vipulaḥ śanaiḥ parvatam āruhat |}%

  \maintext{vṛkṣacchāyāṃ samālokya niṣasāda śramānvitaḥ }||\thinspace12:51\thinspace||%
\translation{Having said this, Vipula climbed the mountain slowly. He caught sight of the shades of a tree, and, being exhausted, sat down [there]. \blankfootnote{12.51 I have accepted the reading {\rm (}emendation?{\rm )} of \Ed\ in \textit{pāda} d {\rm (}\textit{āruhat}{\rm )}
  because I think that \textit{āruhet} is an early scribal mistake that
  is easy to make, and because \textit{°āruhat} comes up again in 12.53d.
 }}

  \maintext{etasminn eva kāle tu vṛkṣaśākhāvatārya ca |}%

  \maintext{apūrvaṃ ca surūpaṃ ca sugandhatvaṃ ca śobhanam }||\thinspace12:52\thinspace||%
\translation{In the same moment, descending from among the branches of the tree, [a monkey appeared and] carrying an extraordinary, beautiful, fragrant, exquisite, \blankfootnote{12.52 Note the stem form noun °\textit{śākhā} in \textit{pāda} b. Understand °\textit{śākhād avatārya}.
  From this point on, the story might be interpreted as a dream. See especially 12.149ab:
  \textit{svapnabhūtam ivāścāryaṃ paśyāmi...}.
 }}

  \maintext{phalaṃ gṛhya vicitraṃ ca hṛdayānandanaṃ śubham |}%

  \maintext{vipulasyāgrataḥ kṛtvā punar vṛkṣaṃ samāruhat }||\thinspace12:53\thinspace||%
\translation{lovely, delightful and pleasant-looking fruit, it put it in front of Vipula, and then climbed back onto the tree. \blankfootnote{12.53 Note how the agent of this sentence is omitted here. That it was a monkey
  that gave Vipula the fruit becomes clear in 12.94 below.
 }}

  \maintext{vipulaś citravad dṛṣṭvā vismayaṃ paramaṃ gataḥ |}%

  \maintext{aho vā svapnabhūto 'smi aho vā tapasaḥ phalam }||\thinspace12:54\thinspace||%
\translation{Vipula, looking [at it] as if seeing a miracle, was perplexed. Wow, am I sleeping? Or is this the fruit of my penance? \blankfootnote{12.54 See notes on 12.52 above on how most of the story could be interpreted as a 
  dream.
 }}

  \maintext{na paśyāmi na jighrāmi na ca svādaṃ ca vedmy aham |}%

  \maintext{vārttāpi na ca me śrotā pratijānāmi kaṃcana }||\thinspace12:55\thinspace||%
\translation{I have never seen, smelt, tasted anything like this. I have not even heard of anything like this. I shall let somebody know about it. \blankfootnote{12.55 Note the use of the {\rm (}non-historical{\rm )} present tense
  in \textit{pāda}s ab clearly pointing to past events.
 I suspect that \textit{śrotā} in \textit{pāda} c is meant to be feminine participle \textit{śrutā}, but
  the metre required the first vowel to be lengthened; understand \textit{me} as \textit{mayā}.
  In \textit{pāda} d, the reading of all the witnesses, \textit{kaścana}, seems to be
  an early scribal mistake for \textit{kañcana}. Note that the same happens at 12.49d.
 }}

  \maintext{evam uktvā hy anekāni phalaṃ gṛhya manoramam |}%

  \maintext{sunirīkṣya punar jighran punar jighran nirīkṣya ca }||\thinspace12:56\thinspace||%
\translation{Having repeated this several times, taking that nice fruit, he kept observing it smelling it again and again. \blankfootnote{12.56 Since one of the main points, and a source of conflict, in the story is 
  that there was only one single fruit, we have to interpret \textit{anekāni} in 
  \textit{pāda} a as a shortened form of \textit{anekavāram} {\rm (}`repeatedly'{\rm )}.
 Most sources consulted read \textit{jighra} or \textit{jighraṃ} in both \textit{pāda} c and d, i.e. most of them do not
  suggest the participle \textit{jighran}, which seems to be the correct reading. I have altered this
  part of the text silently.
 }}

  \maintext{phalaṃ cātra nirūpyanto deśaṃ vāpy avalokayan |}%

  \maintext{pātheyarahitaś cāsmi devadattaṃ phalaṃ mama }||\thinspace12:57\thinspace||%
\translation{`While gazing at this fruit, and observing the countryside, I have run out of provisions. This fruit is godsent. \blankfootnote{12.57 Understand \textit{nirūpyanto} in \textit{pāda} a as a thematised present participle 
  in the nominative {\rm (}\textit{nirūpayan}{\rm )}. This is also suggested by the standard \textit{avalokayan}
  in \textit{pāda} b.
 }}

  \maintext{tat phalaṃ pratigṛhyaiva nagaraṃ praviśāmy aham |}%

  \maintext{prārthayitvā tu yat kiṃcij jīvanārthaṃ carāmy aham }||\thinspace12:58\thinspace||%
\translation{Therefore I shall take this fruit and enter that city, and I shall go and seek something to live on.' }

  \maintext{tataḥ śailam atikramya nagaraṃ praviveśa ha |}%

  \maintext{pathi kaścij janaḥ pṛṣṭhaḥ kiṃnāma nagaraṃ tv idam }||\thinspace12:59\thinspace||%
\translation{Then crossing that mountain, he entered the city. He asked a man on the road: `What is the name of this city?'. }

  \maintext{sa hovāca pathīkena kim apūrvam ihāgataḥ |}%

  \maintext{dakṣiṇāpathadeśo 'yaṃ naravīrapuraṃ tv adaḥ }||\thinspace12:60\thinspace||%
\translation{The traveller replied: `Have you never been here before? This is the Deccan region, and this is the city of Naravīra. \blankfootnote{12.60 I understand \textit{pathīkena} as standing for \textit{pathikena} metri causa {\rm (}see 12.64b{\rm )} and not
  as two words, \textit{pathī kena}. This means that we are forced to accept an instrumental as the agent 
  of the finite verb \textit{uvāca}. I suspect that \msNc's reading {\rm (}\textit{pathīko na}{\rm )} 
  is an attempt to correct the syntax, but in this way \textit{na... apūrvam} becomes 
  problematic.
 \textit{ayam} as the end of this verse may have been the original reading and
  \msCb\ may have corrected it to \textit{adaḥ}. Another possibility is that
  an original \textit{adaḥ} is preserved in \msCb, and it got corrupted to
  \textit{ayaḥ} {\rm (}\msCa{\rm )}, and then to \textit{ayaṃ} {\rm (}\msCc\msNa{\rm )}. 
  In any case, in this case I have chosen the not-so-well attested reading \textit{adaḥ} 
  simply because it works better. Another possibility would be to echo 12.59d and
  correct to \textit{idam}.
 
  Since I am not aware of any attestation of Naravīrapura as a city,
  I suspect that this name is either a mistake for or a pun on
  Karavīrapura, possibly modern Kolhapur in Maharashtra. See more in Intro\verify.
 }}

  \maintext{rājā siṃhajaṭo nāma rājñī tasya ca kekayī |}%

  \maintext{ativṛddho jarāgrastaḥ kekayī ca tathaiva ca }||\thinspace12:61\thinspace||%
\translation{The king is called Siṃhajaṭa, his queen is Kekayī. The king is very old, afflicted by old age, the queen likewise. }

  \maintext{dātā sarvakalājñaś ca yuddhe vīryabalānvitaḥ |}%

  \maintext{brahmaṇyo vatsalo loke sarvaśāstraviśāradaḥ }||\thinspace12:62\thinspace||%
\translation{He is generous, an expert in all the arts, and he possesses the virtue of heroism in battle. He is pious and devoted to his subjects, and he is well-versed in the Śāstras.' \blankfootnote{12.62 Oddly, I had to accept \Ed's reading in \textit{pāda} a {\rm (}°\textit{kalā} as opposed to °\textit{kala}{\rm )} because it is the
  only one that makes sense. A faint possibility would be correcting the
  text to \textit{sarvakālajñaś} {\rm (}`knowing all the times'{\rm )}, but that sounds out of context,
  being usually the epithet of gods and Buddhas.
 }}

  \maintext{vipula uvāca |}%

  \maintext{atra śreṣṭhim upāsyāmi nāma vā tasya kiṃ vada |}%

  \maintext{katamo deśa tadvāsaḥ kathayasva na saṃśayaḥ }||\thinspace12:63\thinspace||%
\translation{Vipula spoke: `As a matter of fact, I am seeking audience with the foreman of the guild {\rm (}\textit{śreṣṭhi}{\rm )}. What is his name? Tell me. In which district is his dwelling? Tell me without any hesitation.' \blankfootnote{12.63 Note the thematised stem \textit{śreṣṭhi} from \textit{śreṣṭhin} in \textit{pāda} a.
 I have chosen a variant containing a stem form in \textit{pāda} c {\rm (}\textit{deśa}{\rm )} for metrical
  reasons. One may even read \textit{katamoddeśa} in a similar sense, or as containing
  \textit{uddeśa} {\rm (}for \textit{uddiśa}{\rm )} as an imperative: `Where is his house, give me directions!'
 }}

  \maintext{vipulenaivam uktas tu pathikovāca taṃ punaḥ |}%

  \maintext{mama bhīmabalo nāma śreṣṭhikasya gṛhāgataḥ }||\thinspace12:64\thinspace||%
\translation{Having been addressed by Vipula thus, the traveller replied: `My name is Bhīmabala and I have come to visit the house of the foreman of the guild. \blankfootnote{12.64 Note the stem form \textit{pathika} in \textit{pathikovāca} in \textit{pāda} b. Alternatively,
  it is an instance of double sandhi {\rm (}\textit{pathika uvāca} $\rightarrow$\ \textit{pathikovāca}{\rm )}.
 }}

  \maintext{śreṣṭhikaḥ puṇḍako nāma khyātaḥ śreṣṭhika ucyate |}%

  \maintext{kautukaṃ tava yady asti tad āgaccha mayā saha }||\thinspace12:65\thinspace||%
\translation{The foreman of the guild is called Puṇḍaka and he is said to be a famous foreman. If you are eager [to see him], come with me.' }

  \maintext{evam astv iti tenokto vipulena mahātmanā |}%

  \maintext{tenaiva saha niryātaḥ śreṣṭhikasya gṛhaṃ prati }||\thinspace12:66\thinspace||%
\translation{`Alright, let it be,' replied to him great-souled Vipula, and they set off to visit the foreman's house together. }

  \maintext{śreṣṭhikaḥ svagṛhāsīno dṛṣṭaḥ sa vipulena tu |}%

  \maintext{tasyāntikam upāgamya tat phalaṃ sa niveditaḥ }||\thinspace12:67\thinspace||%
\translation{When Vipula saw the foreman sitting in his house, he went up to him and offered him that fruit. \blankfootnote{12.67 Understand the construction in \textit{pāda} d as \textit{tasmai tena tat phalaṃ niveditam},
  or read {\rm (}partly with \msNa\msNc{\rm )} \textit{tat phalaṃ saṃniveditam}.
 }}

  \maintext{aho phalam idaṃ śreṣṭham aho phalam ihānitam |}%

  \maintext{aho rūpam aho gandha{-}m{-}aho phalaṃ suśobhanam }||\thinspace12:68\thinspace||%
\translation{`Wow, what an excellent fruit! Hey, what a fruit we have here! Wow, what a form, what a smell, wow what a splendid fruit! \blankfootnote{12.68 Note \textit{ihānitam} for \textit{ihānītam} in \textit{pāda} b for metrical reasons.
 I consider the \textit{-m-} between \textit{gandha} and \textit{aho} in \textit{pāda}s cd a hiatus filler.
 }}

  \maintext{tat phalaṃ na mahījātaṃ na merau na ca mandare |}%

  \maintext{devalokika suvyaktaṃ na martya{-}m{-}upajāyate }||\thinspace12:69\thinspace||%
\translation{This fruit did not grow on earth, not even on Mount Meru or Mount Mandara. It is clearly from the world of gods, it does not grow in the world of humans. \blankfootnote{12.69 \textit{kandare} {\rm (}`in a cave'{\rm )} in \textit{pāda} b must be an early
  mistake in the MSS for \textit{mandare} {\rm (}`on Mount Mandara'{\rm )}, a location that
  appears frequently in the epics and the Purāṇas next
  to Mount Meru, see, e.g., \MBH\ 3.187.10:
  \textit{catuḥsamudraparyantāṃ merumandarabhūṣaṇām\thinspace |
  śeṣo bhūtvāham evaitāṃ dhārayāmi vasuṃdharām\thinspace ||}.
  This is why I conjecture \textit{mandare} here.
 Understand \textit{devalokika} in \textit{pāda} c as being in stem form {\rm (}metri causa{\rm )} 
  for a more standard \textit{devalaukikaṃ}. Understand \textit{martya-m-upajāyate} in \textit{pāda} d 
  as \textit{martya upajāyate} {\rm (}i.e. \textit{martye...}{\rm )} with \textit{-m-} as a sandhi bridge.
 }}

  \maintext{aho 'smi sa phalaṃ bhoktā rājārhaṃ ca na saṃśayaḥ |}%

  \maintext{ḍhaukayitvā phalaṃ divyaṃ rājānaṃ toṣayāmy aham }||\thinspace12:70\thinspace||%
\translation{Alas! Is it me who will enjoy this fruit? No doubt, [only] a king is worthy of it. Offering this divine fruit to the king, I shall please him.' \blankfootnote{12.70 \textit{Pāda} a is slightly suspect. It is possible that originally it contained a 
  negation: \textit{aho 'smi na phalaṃ bhoktā} {\rm (}`Ah! I will not eat this fruit'{\rm )}.
  I have chosen to translate this \textit{pāda} as a question, interpreting \textit{sa} as
  giving emphasis to the grammatical subject. Nevertheless, the slightly 
  odd reoccurrence of the phrase \textit{sa phalaṃ} in 12.71 and 72 might suggest 
  that sometimes we could interpret it, somewhat surprisingly, as \textit{tat phalaṃ}.
 }}

  \maintext{tatas tvarita gatvaiva phalaṃ gṛhya manoharam |}%

  \maintext{ādareṇopasṛtyaiva rājānaṃ sa phalaṃ dadau }||\thinspace12:71\thinspace||%
\translation{Then grabbing that pleasant fruit, he left hastily [together with Bhīmabala]. He approached the king respectfully, and gave him the fruit. \blankfootnote{12.71 In \textit{pāda} a, \textit{tvarita}, for the adverb \textit{tvaritaṃ}, is in stem form metri causa.
 }}

  \maintext{rājā ca sa phalaṃ dṛṣṭvā vismayaṃ paramaṃ gataḥ |}%

  \maintext{kutaḥ śreṣṭhi tvayā nītaṃ phalaṃ pūrvaṃ manoharam }||\thinspace12:72\thinspace||%
\translation{And seeing the fruit, the king was highly amazed. `O foreman, from where have you brought this charming fruit previously? \blankfootnote{12.72 On the possibility that \textit{saphala} is a form in this text simply signifying \textit{phala},
  see notes on 12.70 and 72.
 \textit{pūrva}[\textit{ṃ}] in \textit{pāda} d is suspect and \Ed\ is probably trying to silently emend it.
  One possibility is that the \textit{pāda} originally contained a stem form noun:
  \textit{phalāpūrvaṃ manoharam} {\rm (}`an unparalleled and charming fruit'{\rm )}.
  Alternatively, \textit{pūrva} is an eye-skip to 12.73b.
 }}

  \maintext{svādumūlaṃ phalaṃ kandaṃ dṛṣṭaṃ pūrvaṃ na tādṛśam |}%

  \maintext{rūpagandhaguṇopetaṃ hṛdayānandakārakam }||\thinspace12:73\thinspace||%
\translation{I have never seen such a palatable root or fruit or bulb, one with such beauty, fragrance, and qualities, one that gladdens the heart. }

  \maintext{sadya evopayuñjāmi tvayā dattam idaṃ phalam |}%

  \maintext{kīdṛśaṃ svāda vijñānam icchāmi kuru māciram }||\thinspace12:74\thinspace||%
\translation{I shall eat this fruit that you have given me instantly. What does it taste like? I want to know. Give it to me quickly.' \blankfootnote{12.74 I take \textit{svāda} ain \textit{pāda} c as a stem form noun that stands for the accusative metri causa.
 }}

  \maintext{tataḥ sa bhakṣayām āsa phalaṃ cāmṛtasaṃnibham |}%

  \maintext{amṛtopamasusvādaṃ sarvaṃ ca bubhuje nṛpaḥ }||\thinspace12:75\thinspace||%
\translation{Then he ate the fruit that looked like the ambrosia. The king devoured all of it, and its taste was like that of nectar. }

  \maintext{sadyaḥ ṣoḍaśavarṣasya yauvanaṃ samapadyata |}%

  \maintext{na valīpalitaṃ sadyo na jarā na ca durbalaḥ }||\thinspace12:76\thinspace||%
\translation{In an instant, he obtained the youthfulness of a sixteen-year-old boy. In a moment, there were no wrinkles or grey hair, no illness, no weakness. \blankfootnote{12.76 I have corrected \textit{sadya} in \textit{pāda} a to \textit{sadyaḥ} because there is no
  metrical reason to retain this thematised stem form here {\rm (}cf. \textit{sadyo} in \textit{pāda} c{\rm )}.
 }}

  \maintext{keśadantanakhasnigdho dṛḍhadanto dṛḍhendriyaḥ |}%

  \maintext{tejaścakṣurbalaprāṇān sadyaḥ sarvān avāptavān }||\thinspace12:77\thinspace||%
\translation{His hair, teeth, and nails, all became smooth and shiny, his teeth and senses strong, he regained his vital powers, his vision, strength, and his life energies in a moment. \blankfootnote{12.77 I have corrected \textit{sadya} to \textit{sadyaḥ} in \textit{pāda} d, similarly to what I did in 12.76a.
 }}

  \maintext{mantrī purohito 'mātyaḥ sarve bhṛtyajanās tathā |}%

  \maintext{paurastrī bālavṛddhāś ca sarve te vismayaṃ gatāḥ }||\thinspace12:78\thinspace||%
\translation{The minister, the domestic chaplain, the counsellor, all the servants, the townswomen, and all the children, and all the elderly people, everybody was amazed. \blankfootnote{12.78 Note the singular \textit{paurastrī} in \textit{pāda} c clearly for a plural.
 }}

  \maintext{rājā siṃhajaṭo nāma tuṣṭim eva parāṃ gataḥ |}%

  \maintext{praharṣam atulaṃ caiva prāptavān sa nareśvaraḥ }||\thinspace12:79\thinspace||%
\translation{The sovereign, namely king Siṃhajaṭa, became extremely satisfied and very happy. }

  \maintext{uvāca rājā taṃ śreṣṭhiṃ svārthatatparanirdayaḥ |}%

  \maintext{kuru bhīmabalas tv evaṃ phalam ānaya adya vai }||\thinspace12:80\thinspace||%
\translation{The king, who was selfish and cruel, spoke to that foreman of the guild: `Tell Bhīmabala to bring another fruit today. \blankfootnote{12.80 Note the thematised \textit{śreṣṭhiṃ} in \textit{pāda} a {\rm (}for \textit{śreṣṭhinaṃ}{\rm )}.
 The syntax of \textit{pāda} c is confusing. I translate it as if it carried 
  a causative meaning {\rm (}e.g. \textit{kāraya bhīmabalaṃ tv evaṃ}: `make Bhīmabala act like this'{\rm )}.
  On the other hand, an instrumental {\rm (}\textit{bhīmabalena}{\rm )} would be better {\rm (}`act like this, together with
  Bhīmabala'{\rm )}.
 }}

  \maintext{punar me yauvanaprāptis tvatprasādān narottama |}%

  \maintext{kekayīṃ durbalāṃ vṛddhāṃ punaḥ prāpaya yauvanam }||\thinspace12:81\thinspace||%
\translation{I have regained my youthfulness by your kindness, O excellent man. Help Kekayī, who is weak and old, regain her youthfulness.' }

  \maintext{sa rājñā evam uktas tu śreṣṭhī bhīmabalas tathā |}%

  \maintext{pratyuvāca ha rājānaṃ prāñjaliḥ praṇataḥ sthitaḥ }||\thinspace12:82\thinspace||%
\translation{This is how the king addressed the foreman. Bhīmabala replied to the king, joining his hands reverentially, and remaining standing with his head bowed down. \blankfootnote{12.82 I accepted the reading \textit{śreṣṭhī} {\rm (}\msCc{\rm )} in \textit{pāda} b although it may be a 
  correction of \textit{śreṣṭhi} {\rm (}\msCa\msCb\msNa\msNc{\rm )}, 
  an original \textit{prātipadika} of the thematised form of \textit{śreṣṭhin} {\rm (}see 1.63a{\rm )}.
  All in all, the latter reading is more likely to be the result of
  a bit of confusion about the two nominatives \textit{śreṣṭhī} and \textit{bhīmabalas},
  referring to two different persons. That it is Bhīmabala that replies to
  the king, and not Puṇḍaka the foreman, becomes clear in 12.85a {\rm (}\textit{śrutvā bhīmabalavākyaṃ}{\rm )}.
 }}

  \maintext{na vanena vane rājan na vāṇijyakṛṣeṇa vā |}%

  \maintext{kenāpi kulaputreṇa tava darśanakāṃkṣayā }||\thinspace12:83\thinspace||%
\translation{`Your majesty, one cannot obtain [such a fruit by wondering] from forest to forest. It cannot be obtained through merchants or by cultivating the land. Some noble man, seeking your audience, \blankfootnote{12.83 \textit{Pāda} a could be construed as \textit{na vane na vane rājan}
  {\rm (}`Your majesty, there is no [such fruit] in any
  forest'{\rm )}, but a similar expression, \textit{vanena
  vanaṃ}, occurs, e.g., in \MBH\ 1.144.1 meaning `from forest to forest' 
  {\rm (}\textit{te vanena vanaṃ vīrā ghnanto mṛgagaṇān bahūn\thinspace | 
  apakramya yayū rājaṃs tvaramāṇā mahārathāḥ\thinspace ||}{\rm )}, 
  and this made me choose another option {\rm (}\textit{na vanena vane rājan}{\rm )}. 
  \Ed's variant {\rm (}\textit{na phaledaṃ vane rājan}{\rm )} is likely an attempt to `correct' the text.
 }}

  \maintext{datto 'smi tena rājendra mayā datto 'si bhūpate |}%

  \maintext{na te śaknomy ahaṃ rājan vaktuṃ vaideśinaṃ naram }||\thinspace12:84\thinspace||%
\translation{gave it to me, and, O supreme king, I gave it to you, your majesty. Your majesty, I cannot tell you who this foreigner is.' \blankfootnote{12.84 Note the form \textit{vaideśin} for the better-attested \textit{videśin} or \textit{vaideśika} in \textit{pāda} d.
 }}

  \maintext{śrutvā bhīmabalavākyaṃ pratyuvāca tataḥ punaḥ |}%

  \maintext{amātyakulaputras tvaṃ brūhi madvacanaṃ punaḥ }||\thinspace12:85\thinspace||%
\translation{Having heard Bhīmabala's reply, [the king] said: `You are the son of a noble family of ministers. Repeat my words [to Vipula]: \blankfootnote{12.85 \textit{Pāda} a, as transmitted in \msCa\msCb, is a rare \textit{sa-vipulā}. 
  Some MSS {\rm (}\msCc\msNa\msNb\msNc{\rm )} read °\textit{balaṃ} to avoid this.
 }}

  \maintext{yadi nāsti kiṃ me dattaṃ mayā vā mārgito bhavān |}%

  \maintext{yatra hy eko bahavo 'tra jāyante nātra saṃśayaḥ }||\thinspace12:86\thinspace||%
\translation{If there are no more [fruits], why did you give me one? This is what I request from you, sir. Where there is one, there will be many, that is for sure. \blankfootnote{12.86 \textit{Pāda} c is a rare \textit{sa-vipulā} {\rm (}cf. 12.85a above{\rm )}, as transmitted in \msCa\msNa\msNb\msNc.
  It seems that \msCb\ and \msCc\ try to `correct' it in different ways.
 }}

  \maintext{āgamopāyamārgaṃ ca tenaiva sa tu gamyatām |}%

  \maintext{avaśyaṃ tena gantavyaṃ tena mārgeṇa mārgaya }||\thinspace12:87\thinspace||%
\translation{[There is a] path by which it arrived. He [Vipula] should go [back] by the same route. By all means, that's the way to go. Track it down by that route. }

  \maintext{adattvā phalam anyac ca śiraś chedyāmi durmate |}%

  \maintext{chedyaś caṇḍavicaṇḍābhyāṃ rakṣa bhīmabalādhamaḥ }||\thinspace12:88\thinspace||%
\translation{If you are unable to provide another [fruit], I'll have your head cut off, you fool. [Vipula] will be slain by Caṇḍa and Vicaṇḍa. Beware, Bhīmabala, he is a vile person!' \blankfootnote{12.88 Understand \textit{chedyāmi} in \textit{pāda} b as \textit{chedayāmi}. It is difficult to see how
  the readings \textit{chedye} and \textit{chede} appeared in \msCa\msNb\ and \msCb\msNc, respectively.
  The only MS transmitting \textit{chedyaś} is \msNa, but I suppose that this phrase should
  refer to Vipula being potentially slain by Caṇḍa and Vicaṇḍa, could be the two royal envoys mentioned 
  in verse 12.126 {\rm (}\textit{rājadūtadvayam}{\rm )}, sent along with Bhīmabala to make sure he obeys the king's command.
  Compare with \SDHU\ 7.101, where Yama's attendants are called Caṇḍa and Mahācaṇḍa.
 }}

  \maintext{tato bhīmabalaḥ kruddhaḥ khaḍgaṃ gṛhya śaśiprabham |}%

  \maintext{alaṅghya vacanaṃ rājñaḥ kulaputra vraja tvaram }||\thinspace12:89\thinspace||%
\translation{Then Bhīmabala got angry and drew his sword that looked like the [crescent] moon. [He spoke to Vipula:] `Obeying the king's orders, O son of a noble family, go hastily! \blankfootnote{12.89 The reconstruction of \textit{pāda} d is tentative.
 }}

  \maintext{mā ruṣa kulaputra tvaṃ mayā vadhyo bhaviṣyasi |}%

  \maintext{sadyo 'sti phalam anyad vā dehi rājānam adya vai }||\thinspace12:90\thinspace||%
\translation{O son of a noble family, don't take it as an offence, but I have a licence to kill you, unless you have more of this fruit. Give another one to the king before the end of the day! }

  \maintext{yatra prāptaṃ phalaṃ divyaṃ tatra vādeśaya tvaram |}%

  \maintext{tatphalena vinā bhadra durlabhaṃ tava jīvitam }||\thinspace12:91\thinspace||%
\translation{Reveal to me quickly where you found that exquisite fruit. Without that fruit, my friend, your life is in danger.' \blankfootnote{12.91 I have conjectured \textit{tvaram} for \textit{tava} in \textit{pāda} b because \textit{tava} is both
  unmetrical and meaningless in this context. \textit{tava} might have
  been the result of an eyeskip to \textit{pāda} d, or rather to \textit{pāda} b of 12.92.
 }}

  \maintext{vipula uvāca |}%

  \maintext{jīvitāśām ahaṃ prāpto vaideśī bhavanaṃ tava |}%

  \maintext{kṛtakartā kathaṃ vadhyaḥ prāpnuyām aham adya vai }||\thinspace12:92\thinspace||%
\translation{Vipula spoke: 'As a foreigner, when I reached your house, I also regained my hope of life. How could one who does his duty be slain? I would fetch [another fruit] right now, \blankfootnote{12.92 I emended \textit{vaideśibhavanaṃ} in \textit{pāda} b to \textit{vaideśī bhavanaṃ} to arrive at a much smoother
  interpretation. 
 }}

  \maintext{phalaṃ vā na punas tv anyad dātuṃ śakyaṃ na kenacit |}%

  \maintext{sahyaparvataśailāgre āsīnaḥ śrāntamānasaḥ }||\thinspace12:93\thinspace||%
\translation{but there is no other fruit. Nobody can provide any. Up on the rocky peak of Mount Sahya, I sat down, disheartened. }

  \maintext{vānaras tat phalaṃ gṛhya mama dattvā punar gataḥ |}%

  \maintext{mayā dattam idaṃ tubhyaṃ tvayāpi ca narādhipe }||\thinspace12:94\thinspace||%
\translation{It was a monkey that took that fruit, gave it to me, and then disappeared. I gave it to you, you gave it to the king. }

  \maintext{tatra gacchāva bho śreṣṭhi dṛśyate yadi vānaraḥ |}%

  \maintext{tvayā mayā ca gatvaiva yācāvaḥ plavagādhipam }||\thinspace12:95\thinspace||%
\translation{Hey, let's go to that place, O foreman, to see if the monkey is still there. When we get there together, we can ask the monkey king [for more fruit].' \blankfootnote{12.95 I have accepted \msCb's reading in \textit{pāda} d against all other witnesses.
  The dual seems to nicely follow \textit{gacchāva} in \textit{pāda} a, and the verb
  \verbalroot{\textit{yāc}} also appears in 12.105d {\rm (}\textit{yācasva}{\rm )}. 
  Nevertheless, \msCb\ may only be trying to
  correct the problematic reading found in all the other witnesses.
  \textit{yo vāsaḥ plavagādhipaḥ} could be just an awkward way of saying
  \textit{yatra plavagādhipasya vāsaḥ} or \textit{yatra vasati plavagādhipaḥ}.
 }}

  \maintext{śreṣṭhinā ca tathety āha gacchāmaḥ sahitā vayam |}%

  \maintext{yatra prāptaṃ phalaṃ tubhyaṃ mokṣayāmo na saṃśayaḥ }||\thinspace12:96\thinspace||%
\translation{The foreman said: `Alright, let us all go together to the place where you found that fruit. We shall be saved, no doubt.' \blankfootnote{12.96 Bhīmabala switches to the plural in his reply, possibly referring to
  Vipula, Puṇḍaka, and himself, and also perhaps to the two envoys of the
  king, Caṇḍa and Vicaṇḍa {\rm (}see 12.126cd{\rm )}. Note also \textit{tubhyaṃ} in \textit{pāda} c as instrumental.
 }}

  \maintext{rudra uvāca |}%

  \maintext{tam āruhya giriṃ sahyaṃ mārgamāṇaḥ samantataḥ |}%

  \maintext{vipulena tato dṛṣṭo vānaraḥ plavagādhipaḥ }||\thinspace12:97\thinspace||%
\translation{Rudra spoke: Climbing that mountain, Mount Sahya, searching the place all over, Vipula then caught glimpse of that monkey, the monkey king. }

  \maintext{ayaṃ sa vānaraśreṣṭho vṛkṣacchāyāṃ samāśritaḥ |}%

  \maintext{mama puṇyabalenaiva dṛśyate 'dyāpi vānaraḥ }||\thinspace12:98\thinspace||%
\translation{`It's that extraordinary monkey there lurking in the shade of that tree. This monkey has showed up today again merely by the force of my meritious acts. }

  \maintext{vānara kuru mitrārthaṃ sadyo mṛtyur bhaven mama |}%

  \maintext{pūrvadattaṃ phalam anyad dehi vānara jīvaya }||\thinspace12:99\thinspace||%
\translation{Hey, monkey, do me a friendly favour or I will perish very quickly. Give me another one of that fruit that you gave me, O monkey, and keep me alive.' \blankfootnote{12.99 Note the two \textit{laghu} syllables in \textit{pāda} a in second and third position.
 }}

  \maintext{vānara uvāca |}%

  \maintext{gandharveṇa tu me dattaṃ phalaṃ dattaṃ tu te mayā |}%

  \maintext{punar anyat kathaṃ dāsye tatra gaccha yadīcchasi }||\thinspace12:100\thinspace||%
\translation{The monkey spoke: `It was a Gandharva that had given me the fruit that I gave you. How could I give you another one? Go there [where Gandharvas live] if you wish. }

  \maintext{vipula uvāca |}%

  \maintext{adattvā tat phalaṃ tubhyaṃ jīvituṃ saṃśayo bhavet |}%

  \maintext{athavā tatra gacchāmo yatra citrarathaḥ svayam }||\thinspace12:101\thinspace||%
\translation{Vipula spoke: `If you cannot give me another fruit, [my] staying alive is doubtful. Another alternative is that we go where Citraratha himself dwells.' \blankfootnote{12.101 Note \textit{tubhyaṃ} in \textit{pāda} a again in the sense of \textit{tvayā}.
 Citraratha is the king of the Gandharvas.
 }}

  \maintext{vānaraḥ punar evāha evaṃ kurvāmahe vayam |}%

  \maintext{tataś citrarathāvāsam upagamyedam abravīt }||\thinspace12:102\thinspace||%
\translation{The monkey replied: `Let's do it.' Then, upon reaching Citraratha's dwelling place, and having gone up to him, he said this: }

  \maintext{gandharvarāja kāryārthī tvām ahaṃ punar āgataḥ |}%

  \maintext{pūrvadattaphalaṃ tv anyad dehi māṃ yadi śakyate  }||\thinspace12:103\thinspace||%
\translation{`O king of the Gandharvas, I have come back to you with a request. Give me another of that fruit that you gave me, if you can.' \blankfootnote{12.103 Variants for \textit{pāda} b are problematic. I conjectured \textit{tvām ahaṃ} because \textit{ahaṃ} 
  {\rm (}in \msCb\msNb{\rm )} seems to work better with \textit{punar} than \textit{ayaṃ} {\rm (}after all,
  it is the monkey who returns to the Citraratha, and not Vipula{\rm )}, and because 
  it is difficult to accept the ablative \textit{tvat} as meaning `to you.' The original 
  may have read the enclitic form \textit{tvā}. Considering \textit{tvatsakāśaṃ} in 12.107b opens up 
  other possibilities, such as conjecturing \textit{tvadvāsaṃ}.
 }}

  \maintext{gandharvarāja uvāca |}%

  \maintext{sūryalokagataś cāsmi tena dattaṃ phalottamam |}%

  \maintext{mayā dattaṃ phalaṃ tubhyam atyantasuhṛdo 'si me }||\thinspace12:104\thinspace||%
\translation{The king of the Gandharvas spoke: `I went to the world of Sūrya, and it was him who gave me that extraordinary fruit. I gave that fruit to you [because] you are my very best friend. \blankfootnote{12.104 Understand \textit{suhṛdo} in \textit{pāda} d as a singular nominative of the rare \textit{suhṛda}.
 }}

  \maintext{kuto 'nyat phalam ādāsye mama nāsti plavaṅgama |}%

  \maintext{sūryalokaṃ gamiṣyāmas tatra yācasva bhāskaram }||\thinspace12:105\thinspace||%
\translation{Where could I find another fruit, I don't have any, O monkey. Let us go to the world of Sūrya, and ask the Sun there.' }

  \maintext{gandharvenaivam uktas tu tathety āha plavaṅgamaḥ |}%

  \maintext{sūryalokaṃ tataḥ prāptā gandharvādaya sarvaśaḥ }||\thinspace12:106\thinspace||%
\translation{Having been addressed thus by the Gandharva, the monkey consented. They reached the world of Sūrya all together, the Gandharva and the others. \blankfootnote{12.106 I have emended the correct but unmetrical °\textit{ādayaḥ} in \textit{pāda} d to a stem form 
  to restore the metre.
 }}

  \maintext{gandharva uvāca |}%

  \maintext{kāryārthena punaḥ prāptas tvatsakāśaṃ khageśvara |}%

  \maintext{pūrvadattaphalaṃ tv anyad dehi jīvam anāśaya }||\thinspace12:107\thinspace||%
\translation{The Gandharva spoke: `I have come back to you with a request, O Sky-goer lord. Give me another of that fruit you gave me, and spare a life.' }

  \maintext{sūrya uvāca |}%

  \maintext{somalokagataś cāsmi tena dattaṃ phalottamam |}%

  \maintext{sa phalaṃ dattam evāsi suhṛdatvān mayā tava }||\thinspace12:108\thinspace||%
\translation{Sūrya spoke: `I went to Soma's world, and it was he who gave me the magical fruit. That is how you were given that fruit, by me, out of my friendship to you. \blankfootnote{12.108 Note the odd syntax of \textit{pāda}s cd. \textit{sa phalaṃ} may have been influenced 
  by 12.71d and 72a. Here in 12.108 \textit{tat phalaṃ} would work
  better but see \textit{sa phalaṃ} in a similarly odd position in 
  12.113d. I translate \textit{sa} again as standing for emphasis 
  {\rm (}`it was like that that you...'; cf. 12.70a{\rm )}.
  \textit{dattam evāsi} is also problematic although similar
  structures do appear in this text, e.g., in 12.113c. The original may have read
  \textit{tat phalam datta evāsi}; or take \textit{dattam evāsi} as \textit{datta-m-evāsi},
  with a hiatus breaker \textit{-m-}.
 }}

  \maintext{anyad dātuṃ na śaknomi gaccha somapurādya vai |}%

  \maintext{taṃ prārthayāvikalpena atriputraṃ graheśvaram }||\thinspace12:109\thinspace||%
\translation{I cannot give you another one. Go now to Soma's city. Ask him, [the Moon], the son of Atri, the lord of planets, without hesitation. \blankfootnote{12.109 Understand \textit{purādya} as \textit{puram adya} {\rm (}stem form metri causa{\rm )}.
 }}

  \maintext{rudra uvāca |}%

  \maintext{gatāḥ sūryāgrataḥ kṛtvā somalokaṃ tathaiva hi |}%

  \maintext{uvāca sūryaḥ somāya kāraṇāpekṣayā śaśim }||\thinspace12:110\thinspace||%
\translation{Rudra spoke: Led by Sūrya, they went to the world of Soma. Sūrya spoke to Soma, hoping for action on the Moon's part. \blankfootnote{12.110 Understand \textit{sūryāgrataḥ} in \textit{pāda} a as \textit{sūryam agrataḥ} {\rm (}stem form noun{\rm )}.
 Note the thematised form \textit{śaśim} for \textit{śaśinam} in \textit{pāda} d. 
  \textit{somāya... śaśiṃ} could be just a clumsy way of saying \textit{somaṃ... śaśinaṃ} or
  \textit{somāya... śaśine}. My interpretation of \textit{pāda} d is tentatitve,
  and it is not inconceivable that \msCb\ is right reading
  \textit{karuṇāpekṣayā} {\rm (}`hoping for compassion'{\rm )}.
 }}

  \maintext{soma uvāca |}%

  \maintext{kimartham āgato bhūyaḥ kartavyaṃ tatra bhāskara |}%

  \maintext{phalaṃ dātuṃ punas tv anyan muktvā tv anyat karomy aham }||\thinspace12:111\thinspace||%
\translation{Soma spoke: For what purpose have you returned? O Sun, there will be a solution for that. Except for giving another fruit, I shall do anything. }

  \maintext{sūrya uvāca |}%

  \maintext{yadi śakyaṃ phalaṃ dehi anyan na prārthayāmy aham |}%

  \maintext{na dattāsi phalam anyan mayā vadhyo bhaviṣyasi }||\thinspace12:112\thinspace||%
\translation{Sūrya spoke: `If you can, give me a fruit, I am not asking for anything else. If you do not give me another fruit, I'll kill you.' \blankfootnote{12.112 Understand \textit{pāda} c either as \textit{na dattaṃ tvayā phalam anyat} or
  \textit{na dātāsi phalam anyat}. This \textit{pāda} is a \textit{sa-vipulā}, or if we
  apply a licence mostly seen in the non-\textit{anuṣṭhubh} verses in this
  text, namely that a word-final syllable can count as \textit{guru},
  it is a standard \textit{anuṣṭubh} {\rm (}\textit{pathyā}{\rm )}.
 }}

  \maintext{soma uvāca |}%

  \maintext{āgamaṃ tasya vakṣyāmi śṛṇuṣvāvahito bhava |}%

  \maintext{indreṇāsmi phalaṃ dattaṃ sa phalaṃ datta me bhavān }||\thinspace12:113\thinspace||%
\translation{Soma spoke: `I shall tell you the way by which it arrived. Listen carefully. It was Indra who gave me the fruit and I gave that fruit to you. \blankfootnote{12.113 Note \textit{sa phalaṃ}, potentially for \textit{tat phalaṃ}, or for emphasis, again, as in 12.108c. 
  The syntax of \textit{pāda}s cd is rather confused and \textit{datta} in \textit{pāda} d
  is a stem form participle metri causa. Note also \textit{me} for \textit{mayā}.
 }}

  \maintext{gatvaivendrasadas tv anyat prārthayāmaḥ sahaiva tu |}%

  \maintext{evaṃ kurma iti prāha gatvendrasadanaṃ prati }||\thinspace12:114\thinspace||%
\translation{If we go to Indra's palace, we can ask for another one together. Let us do it!' he said and left for Indra's residence. }

  \maintext{soma indram uvācedaṃ phalakāmā ihāgatāḥ |}%

  \maintext{pūrvadattaphalam anyad dehi śakra mamādya vai }||\thinspace12:115\thinspace||%
\translation{Soma said this to Indra: `We have come here seeking a fruit.' Give me now another of the fruit that you gave me before, O Śakra. \blankfootnote{12.115 \textit{soma indram} in pāda a in \msNc\ may be a correction of
  the reading in all the other sources. On the other hand,
  it can be original, and the hiatus may have confused an early scribe.
 \textit{Pāda} c is either a \textit{sa-vipulā} or a \textit{pathyā} if the final syllable
  of °\textit{phalam} counts as \textit{guru}. Cf. 12.112 above.
 }}

  \maintext{indra uvāca |}%

  \maintext{yadartham iha samprāptaḥ sa ca nāsti niśākara |}%

  \maintext{viṣṇuhastān mayā prāptam ekam eva phalaṃ śubham }||\thinspace12:116\thinspace||%
\translation{Indra spoke: `The reason for which you came here does not exist, O Night-maker!. I received only a single one of that nice fruit, out of Viṣṇu's hands. }

  \maintext{sarva eva hi gacchāmo viṣṇulokaṃ graheśvara |}%

  \maintext{sarva evopajagmus te phalārthaṃ madhusūdanam }||\thinspace12:117\thinspace||%
\translation{Let us all go to Viṣṇu's world, O lord of the planets.' They all went to Madhusūdana for the fruit. }

  \maintext{evam uktvā gatāḥ sarve devarājapuraskṛtāḥ |}%

  \maintext{muhūrtenaiva samprāptā viṣṇulokaṃ yaśasvini }||\thinspace12:118\thinspace||%
\translation{After he spoke thus, they all left, led by the king of the gods. They reached the world of Viṣṇu in a moment, O Yaśasvinī. \blankfootnote{12.118 Note how there is a minor confusion here with the order of events. 
  12.117 informs us that Indra spoke and then they all left. Then 12.118ab
  restates that after Indra spoke they left.
 }}

  \maintext{upasṛtya tata indraḥ praṇipatya janārdanam |}%

  \maintext{sarveṣām uparodhena prārthayāmi yaśodhara }||\thinspace12:119\thinspace||%
\translation{Indra then approached Janārdana, bowing down respectfully. `I have a request, O Yaśodhara, for something that troubles everybody [here]'. \blankfootnote{12.119 Note that \textit{pāda} a is unmetrical or rather a \textit{sa-vipulā}.
 }}

  \maintext{viṣṇur uvāca |}%

  \maintext{pūrvadattaphalasyārthe tac ca sarva{-}m{-}ihāgatāḥ |}%

  \maintext{na śaknomi phalaṃ dātuṃ kiṃ vā tv anyat karomy aham }||\thinspace12:120\thinspace||%
\translation{Viṣṇu spoke: `You all have come here for the fruit that I donated previously. I cannot give you the fruit. Otherwise, what else can I do for you?' \blankfootnote{12.120 The function of \textit{tac ca} in \textit{pāda} b is unclear. Perhaps understand \textit{atra} {\rm (}`here'{\rm )}
  or, less likely, \textit{tvaṃ ca} {\rm (}`you and [everybody else]'{\rm )}.
  Understand \textit{sarvam ihāgatāḥ} as \textit{sarva-m-ihāgatāḥ}, with a hiatus filler \textit{-m-}
  for \textit{sarva} {\rm (}i.e. \textit{sarve}{\rm )} \textit{ihāgatāḥ}.
 The non-standard form \textit{anyaṃ} transmitted in most witnesses consulted
  might be original but I have not found any clear occurences of
  it in this text elsewhere. That is why I have chosen \msNc's reading,
  the standard \textit{anyat}.
 }}

  \maintext{indra uvāca |}%

  \maintext{brahmāṇḍam api bhettuṃ tvaṃ śaknoṣi garuḍadhvaja |}%

  \maintext{aśakyaṃ tava nāstīti jānāmi puruṣottama }||\thinspace12:121\thinspace||%
\translation{Indra spoke: `You are even capable of splitting Brahmā's Egg, O you of the banner with Garuḍa on it. I know that there is nothing that you cannot do, O Puruṣottama.' }

  \maintext{evam uktaḥ punar viṣṇuḥ pratyuvāca purandaram |}%

  \maintext{phalam ekaṃ parityajya sarvaṃ śaknomi kauśika }||\thinspace12:122\thinspace||%
\translation{Having been addressed thus, Viṣṇu replied to Purandara [Indra]: `O Kauśika, I can do everything with the only exception of the fruit. }

  \maintext{upāyo 'tra pravakṣyāmi āgamaṃ śṛṇu gopate |}%

  \maintext{brahmaṇā ca mama dattaṃ tat phalaikaṃ purandara }||\thinspace12:123\thinspace||%
\translation{I shall tell you the means of obtaining it. Listen to where it came from, O Gopati. It was Brahmā who gave me that one single piece of fruit, O Purandara. \blankfootnote{12.123 Note that \textit{pāda} c is a \textit{sa-vipulā}, and that \textit{phala} is in stem form in
  \textit{pāda} d {\rm (}understand \textit{phalam ekaṃ}; see 12.124a{\rm )}.
 }}

  \maintext{mayā dattaṃ phalaṃ tv ekaṃ kim anyad dātum icchasi |}%

  \maintext{prārthayāmo 'tra gatvaikaṃ parameṣṭhiprajāpatim  }||\thinspace12:124\thinspace||%
\translation{I have given you that single piece of fruit, why do you want me to give you another one? Let us now go to the highest creator Prajāpati [Brahmā], and ask him for one. \blankfootnote{12.124 In \textit{pāda} b, by \textit{dātum icchasi}, Viṣṇu probably means to say
  \textit{prāptum icchasi}.
 For the expression \textit{parameṣṭhiprajāpati} see \MBH\ 6.15.35ab:
  \textit{sarvalokeśvarasyeva parameṣṭhiprajāpateḥ}
 }}

  \maintext{tavoparodhād devendra prārthayāmi pitāmaham |}%

  \maintext{evam uktvā gatāḥ sarve puraskṛtya janārdanam }||\thinspace12:125\thinspace||%
\translation{I shall ask Grandfather Brahmā, O king of the gods, to solve your problem.' After he said this, they all left together, led by Janārdana: }

  \maintext{indraḥ sūryaḥ śaśī caiva gandharvo vānaras tathā |}%

  \maintext{vipulaḥ śreṣṭhikaś caiva rājadūtadvayaṃ tathā }||\thinspace12:126\thinspace||%
\translation{Indra, Sūrya, the Moon, the Gandharva, the monkey, Vipula, the foreman, and the two envoys of the king. \blankfootnote{12.126 Reading this list of characters, the careful reader may ask the question: what happened
  to Bhīmabala?
 }}

  \maintext{brahmalokaṃ muhūrtena prāptavān surasundari |}%

  \maintext{dṛṣṭvā brahmasado ramyaṃ sarvakāmaparicchadam }||\thinspace12:127\thinspace||%
\translation{They reached Brahmā's world in a moment, O Surasundarī. Seeing Brahmā's beautiful palace filled with all desirable things, }

  \maintext{anekāni vicitrāṇi ratnāni vividhāni ca |}%

  \maintext{mandāratala śobhāni vaiḍūryamaṇikuṭṭimān }||\thinspace12:128\thinspace||%
\translation{the innumerable wonders and different kinds of gems, the beautiful coral-tree roofs, the floors inlaid with cat's-eye gems, \blankfootnote{12.128 I take \textit{mandāratala} as a stem form compound {\rm (}for \textit{mandāratalāni}{\rm )}.
  Note that all witnesses read °\textit{kuṭṭimāṃ} or °\textit{kuṭṭimām} for the 
  masculine plural accusative.
 }}

  \maintext{pravālamaṇistambhāni vajrakāñcanavedikām |}%

  \maintext{pravālasphāṭiko jāla indranīlagavākṣakaḥ }||\thinspace12:129\thinspace||%
\translation{the coral-gem pillars, and the diamond and golden altar, the coral-gem and crystalline lattice-window[s] and sapphire window[s], \blankfootnote{12.129 \textit{Pāda} a is unmetrical.
 Understand the nominatives in \textit{pāda}s cd as {\rm (}plural{\rm )} accusatives.
 }}

  \maintext{paśyate vipulas tatra nānāvṛkṣa manoramāḥ |}%

  \maintext{puṣpānāmitavṛkṣāgrāḥ phalānāmitakā bhavet }||\thinspace12:130\thinspace||%
\translation{Vipula [also] saw [that there were] various beautiful trees there, with their tops bent down with [the burden of] the blossom and the fruits. \blankfootnote{12.130 Note \textit{°vṛkṣa} in \textit{pāda} b as a stem form noun for \textit{°vṛkṣā} or \textit{°vṛkṣān}
  {\rm (}\textit{manoramāḥ/-ān}{\rm )}. One could simply correct the \textit{pāda} to
  \textit{nānāvṛkṣān manoramān}, but then the next line should also
  be altered.
 \textit{bhavet} in \textit{pāda} d is out of context.
 }}

  \maintext{sarvaratnamayā vṛkṣāḥ sarvaratnamayaṃ jalam |}%

  \maintext{vṛkṣagulmalatāvallī kandamūlaphalāni ca }||\thinspace12:131\thinspace||%
\translation{The trees and the water seemed to be made of all kinds of gems. The trees, bushes, creepers, winding plants, and bulbous roots, and fruits, }

  \maintext{sarve ratnamayā dṛṣṭā vipulo vipulekṣaṇaḥ |}%

  \maintext{anekabhaumaṃ prāsādaṃ muktādāmavibhūṣitam }||\thinspace12:132\thinspace||%
\translation{Vipula, with his eyes open wide, saw all these consisting of gems. [There was] a multi-storeyed palace decorated with garlands of pearls, \blankfootnote{12.132 Note the odd syntax of \textit{pāda}s ab. \textit{Pāda} b should be understood as a 
  phrase in the instrumental case. \msCb\ tries to correct the syntax by
  reading \textit{dṛṣṭvā}.
 }}

  \maintext{apsarogaṇakoṭībhiḥ sarvābharaṇabhūṣitam |}%

  \maintext{vimānakoṭikoṭīnāṃ sarvakāmasamanvitam }||\thinspace12:133\thinspace||%
\translation{embellished with millions of groups of Apsarases wearing all kinds of ornaments, and millions and millions of floating \ae rial vehicles, and possessing everything wished for. \blankfootnote{12.133 I understand \textit{pāda}s ab as if it read \textit{apsarogaṇakoṭībhiḥ sarvābharaṇabhūṣitair bhūṣitam}.
 Perhaps understand \textit{vimānakoṭikoṭīnāṃ} as \textit{vimānakoṭīnāṃ koṭibhiḥ} and
  \textit{°samanvitam} as \textit{°samanvitānām}. This is what, e.g., \SDHS\ 10.41 suggests
  {\rm (}see the apparatus{\rm )}.
 }}

  \maintext{brahmalokasabhā ramyā sūryakoṭisamaprabhā |}%

  \maintext{tatra brahmā sukhāsīno nānāratnopaśobhite }||\thinspace12:134\thinspace||%
\translation{The assembly hall in Brahmā's world was charming and it shone like millions of suns. Brahmā was sitting there comfortably, [on a throne] decorated with various jewels, \blankfootnote{12.134 \textit{Pāda}s c may have indended to read \textit{tatra brahmā sukhāsane}, or at least
  \textit{āsane} is implied to go with \textit{pāda} d.
 }}

  \maintext{caturmūrtiś caturvaktraś caturbāhuś caturbhujaḥ |}%

  \maintext{caturvedadharo devaś caturāśramanāyakaḥ }||\thinspace12:135\thinspace||%
\translation{with his four embodiments, four heads, four arms, and four hands. The god who is the governor of the four social disciplines {\rm (}\textit{āśrama}{\rm )} was holding the four Vedas. }

  \maintext{caturvedāvṛtas tatra mūrtimanta{-}m{-}upāsate |}%

  \maintext{gāyatrī vedamātā ca sāvitrī ca surūpiṇī }||\thinspace12:136\thinspace||%
\translation{He was at the same time surronded by the four Vedas: they were worshipping Him in their embodied forms. Gāyatrī, the mother of the Vedas, and beautiful Sāvitrī, \blankfootnote{12.136 The context dictates that \textit{pāda} b is to be understood in the plural 
  {\rm (}\textit{mūrtimanta upāsate}{\rm )}, with a hiatus filler \textit{-m-} {\rm (}cf. \DEVIP\ 12.12.53cd:
  \textit{saptakoṭimahāmantrā mūrtimanta upāsate}{\rm )}.
 For Gāyatrī being `the mother of the Vedas,' see, e.g. \MBH\ Appendices 14.4.494: 
  \textit{yo japet pāvanīṃ devīṃ gāyatrīṃ vedamātaram}.
 }}

  \maintext{vyāhṛtiḥ praṇavaś caiva mūrtimān samupāsate |}%

  \maintext{vauṣaṭkāro vaṣaṭkāro namaskāraḥ sa mūrtimān }||\thinspace12:137\thinspace||%
\translation{as well as the Vyāhṛti[s] [Bhur, Bhuvaḥ, Svar], and Praṇava [Oṃ], were serving [him] in their embodied forms, as well as [the mantras] Vauṣaṭ, Vaṣaṭ and Namaḥ in their embodied forms, \blankfootnote{12.137 Note the singular \textit{mūrtimān} in \textit{pāda} b governing each singular subject 
  in 12.136cd and 137a.
 }}

  \maintext{śrutiḥ smṛtiś ca nītiś ca dharmaśāstraṃ samūrtimat |}%

  \maintext{itihāsaḥ purāṇaṃ ca sāṃkhyayogaḥ patañjalam }||\thinspace12:138\thinspace||%
\translation{and Śruti and Smṛti and Nīti and Dharmaśāstra in their embodied forms, as well as the Epics, the Purāṇas, and Pātañjala Sāṃkhyayoga, \blankfootnote{12.138 Understand \textit{samūrtimat} simply as \textit{mūrtimat}.
 It is difficult to say if \textit{sāṃkhya-yoga} in \textit{pāda} d signifies one or two
  things. I could have chosen to separate them, interpreting \textit{sāṃkhya} as a stem form
  noun, because in other parts of the text, \textit{sāṃkhya} and \textit{yoga} are usually treated as
  two different traditions. See 8.1--3, 16.36--37 {\rm (}here clearly separate{\rm )}, and 23.5c
  {\rm (}again, clearly separate{\rm )}. In any case, one should probably understand \textit{patañjalam} as \textit{pātañjalaḥ} 
  metri causa, with gender confusion. Another, less likely, possibility is that \textit{sāṃkhyayoga} and
  \textit{pātañjalayoga} are somehow contrasted here.
 }}

  \maintext{āyurvedo dhanurvedo vedo gāndharva{-}m{-}eva ca |}%

  \maintext{arthavedo 'nyavedāś ca mūrtimān samupāsate }||\thinspace12:139\thinspace||%
\translation{Āyurveda, Dhanurveda, and Gāndharvaveda, Arthaveda, and other Vedas, in their embodied forms. \blankfootnote{12.139 Note \msCb\ and \msCc's attempt to include the Atharvaveda in this list. I find it more likely that by
  \textit{arthaveda} Kauṭilya's Arthaśāstra is being referred to here.
 }}

  \maintext{tato brahmā samutthāya abhigamya janārdanam |}%

  \maintext{gāṃ ca arghaṃ ca dattvaivam āsyatām iti cābravīt }||\thinspace12:140\thinspace||%
\translation{Then Brahmā rose and went up to Janārdana [Viṣṇu]. Gifting him a cow and guest-water, he said `Please take a seat. }

  \maintext{maṇiratnamaye divye āsane garuḍadhvajaḥ |}%

  \maintext{devarājo raviḥ somo gandharvaḥ plavageśvaraḥ }||\thinspace12:141\thinspace||%
\translation{The one of the banner with Garuḍa on it [should please sit] on [this] divine throne made of gems and jewels. The king of the gods [Indra], the Sun, the Moon, the Gandharva, the monkey king, }

  \maintext{vipulaś ca mahāsattva āsyatāṃ ratna-āsane |}%

  \maintext{sādhu bho vipula śreṣṭha sādhu bho vipulaṃ tapaḥ }||\thinspace12:142\thinspace||%
\translation{and Vipula the great man should sit on [these] gem-throne[s]. Well done, excellent Vipula! Congratulations for your enormous {\rm (}\textit{vipula}{\rm )} austerity! \blankfootnote{12.142 Note how Bhimabala and Puṇḍaka are not mentioned here. They have either not made
  it to Brahmā's palace, or are kept standing.
 Note Brahmā's puns on Vipula's name in \textit{pāda} d and in the next verse.
 }}

  \maintext{sādhu bho vipulaprājña sādhu bho vipulaśriya |}%

  \maintext{toṣitāḥ sma vayaṃ sarve brahmaviṣṇumaheśvarāḥ }||\thinspace12:143\thinspace||%
\translation{Well done, you of enourmous wisdom {\rm (}\textit{vipulaprajña}{\rm )}! Well done, you of enormous fortune! We, Brahmā, Viṣṇu, and Maheśvara, are all pleased, \blankfootnote{12.143 Understand \textit{°śriya} as the singular masculine vocative of °\textit{śrī}.
 }}

  \maintext{ādityā vasavo rudrāḥ sādhyāśvinau marut tathā |}%

  \maintext{bhuṅkṣva bhogān yathotsāhaṃ mama loke yathāsukham }||\thinspace12:144\thinspace||%
\translation{[as well as] the Ādityas, the Vasus, the Rudras, the Sādhyas, the Aśvins, and the Marut[s]. Dive into the enjoyments in my world as much as you want, as you please. \blankfootnote{12.144 \textit{Pāda} b is iambic.
 MSS \msCa\msCb\msNa\msNc\ read \textit{bhogāṃ} for the plural accusative \textit{bhogān}
  {\rm (}silently corrected{\rm )}.
 }}

  \maintext{iyaṃ vimānakoṭīnāṃ tavārthāyopakalpitā |}%

  \maintext{sahasrāṇāṃ sahasrāṇi apsarā kāmarūpiṇī }||\thinspace12:145\thinspace||%
\translation{This one amongst the millions of \ae rial vehices has been built for you. There are thousands and thousands of sexy Apsarases, \blankfootnote{12.145 \textit{iyaṃ} {\rm (}f.{\rm )} in \textit{pāda} a stands for either \textit{ayaṃ} {\rm (}m.{\rm )} or \textit{idaṃ} {\rm (}n.{\rm )}, agreeing
  with the gender of \textit{vimāna}. Alternatively, the sentence 
  wants, rather clumsily, to convey the meaning `all these millions of \ae rial vehicles...'
 Note that here, as often in this text, nouns and adjectives stand in the singular
  after numbers such as a thousand.
 }}

  \maintext{tavārthīyopasarpanti sarvālaṃkārabhūṣitāḥ |}%

  \maintext{yāvat kalpasahasrāṇi parārdhāni tapodhana |}%

  \maintext{yatra yatra prayāsitvaṃ tatra tatropabhujyatām }||\thinspace12:146\thinspace||%
\translation{adorned with all kinds of ornaments, making advances to you. [This state of affairs will go on] for a thousand hundred quadrillion \ae ons, O great ascetic. Where there is effort, there one can enjoy [the results].' \blankfootnote{12.146 Understand \textit{tavārthīyopasarpanti} as \textit{tavārthīyā upasarpanti} {\rm (}double sandhi{\rm )}.
  \textit{tavārthāyo°} may work as well {\rm (}\msCb\ and \msNa{\rm )} but I consider 
  \textit{tavārtīyo°} the lectio difficilior, thus potentially the original reading.
 }}

  \maintext{maheśvara uvāca |}%

  \maintext{iti śrutvā vacas tasya vipulo vipulekṣaṇaḥ |}%

  \maintext{vepamāno bhayatrasta aśrupūrṇākulekṣaṇaḥ }||\thinspace12:147\thinspace||%
\translation{Maheśvara spoke: Listening to His speech, Vipula, with his eyes wild open, shaking, trembling with fear, his bewildered eyes filled with tears, \blankfootnote{12.147 We are forced to accept \Ed's reading of \textit{bhayatrasta} in \textit{pāda} c because it
  is far superior to the readings of all other witnesses. 
  The rejected reading {\rm (}\textit{bhayas tatra}{\rm )} may have been the result of
  a simple metathesis {\rm (}\textit{tra-sta} to \textit{sta-tra}{\rm )}.
 }}

  \maintext{praṇamya śirasā bhūmau praṇipatya punaḥ punaḥ |}%

  \maintext{uvāca madhuraṃ vākyaṃ brahmalokapitāmaham }||\thinspace12:148\thinspace||%
\translation{bowing down his head, prostrating himself on the ground again and again, delivered a sweet speech to [Brahmā,] the Grandfather of Brahmaloka: \blankfootnote{12.148 The compound \textit{brahmalokapitāmahaḥ} may sound 
  tautological as an epithet of Brahmā but it does occur 
  in the \MBH\ {\rm (}12.336.30b{\rm )} and in other texts {\rm (}\PADMAS\ 3.193d, \JRY\ 3.14.198b{\rm )}.
  Otherwise, the word \textit{brahma} may stand for the accusative here {\rm (}\textit{brahmānaṃ}{\rm )},
  or may be corrupted from \textit{sarva}° {\rm (}see next verse{\rm )}.
 }}

  \maintext{vipula uvāca |}%

  \maintext{bhagavan sarvalokeśa sarvalokapitāmaha |}%

  \maintext{svapnabhūtam ivāścaryaṃ paśyāmi tridaśeśvara |}%

  \maintext{smṛtibhraṃśaś ca me jāto buddhir jātāndhacetanā }||\thinspace12:149\thinspace||%
\translation{Vipula spoke: `Venerable Sir, Lord of all the worlds, Grandfather of all people, I can see a dream-like wonder, O Lord of the thirty[-three] gods. My memory abandons me, my mind's intelligence is darkened. \blankfootnote{12.149 Note that \Ed\ adds a line here {\rm (}see the apparatus; translation:
  `I am a fool, how could I praise you? You are beyond knowledge, beyond the ultimate'{\rm )}.
  I have not been able to locate this line in any of the available sources,
  not even in paper manuscripts.
 }}

  \maintext{tubhyaṃ trailokyabandho bhava mama śaraṇaṃ trāhi saṃsāraghorād}%

 \nonanustubhindent \maintext{bhīto 'haṃ garbhavāsāj jaramaraṇabhayāt trāhi māṃ mohabandhāt |}%

  \maintext{nityaṃ rogādhivāsam aniyatavapuṣaṃ trāhi māṃ kālapāśāt}%

 \nonanustubhindent \maintext{tiryaṃ cānyonyabhakṣaṃ bahuyugaśataśas trāhi mohāndhakārāt }||\thinspace12:150\thinspace||%
\translation{You keep the three worlds under control. Be my refuge. Protect [me] from terrible transmigration. I am afraid of being in a womb, and of the terror of old age and death. Protect me from the fetter of illusions. Dwelling in illness is eternal. Protect me, whose body is not controlled, from the noose of time. Animals existence means eating each other for many hundreds of \textit{yuga}s. Protect [me] from the darkness of illusions.' \blankfootnote{12.150 We have to understand \textit{tubhyaṃ}, as often in this text, as an instrumental.
 Note that in \textit{pāda} c, the final syllable of \textit{rogādhivāsam} scans as long. This
  is a phenomenon seen many times in this text.
 }}

  \maintext{śrutvaivovāca brahmā vipulamati punar mānayitvā yathāvad}%

 \nonanustubhindent \maintext{āhūtasamplavānte bhaviṣyasi tava me janmalobho na bhūyaḥ |}%

  \maintext{garbhāvāsaṃ na ca tvan na ca punamaraṇaṃ kleśam āyāsapūrṇaṃ}%

 \nonanustubhindent \maintext{chittvā mohāndhaśatruṃ vrajasi ca paramaṃ brahmabhūyatvam eṣi }||\thinspace12:151\thinspace||%
\translation{Hearing [this] Brahmā spoke to [Vipula] of huge intellect, honouring [him] duly. `You will live until the universal floods of destruction. You will not have any longing for being reborn any more. There will be no dwelling in a womb for you, no rebirth, no anguish full of weariness. Killing the enemy that is the darkness of illusions, you will reach the ultimate, the absorption into the Brahman.' \blankfootnote{12.151 The stem form noun °\textit{mati} of the bahuvrīhi compound 
  in \textit{pāda} a may stand for \textit{matiḥ} {\rm (}see the unmetrical reading in \msCa\msCb\msNa{\rm )}, and
  then it should refer to Brahmā himself {\rm (}`Brahmā, the one with a huge intellect...'{\rm )}.
  I have choosen to take \textit{mati} as a stem form noun standing for the accusative,
  referring to Vipula. This works better because \textit{mānayitvā} {\rm (}and \textit{śrutvā}{\rm )} requires an object.
 Note \textit{āhūtasamplava} instead of the more common \textit{ābhūtasamplava} {\rm (}both unmetrical here; see also 2.13{\rm )}.
  \textit{me} in \textit{pāda} b is difficult to interpret {\rm (}perhaps `you will live with me'?{\rm )}.
 I take \textit{tvan na} in \textit{pāda} c as the ablative \textit{tvad} used as a genitive, plus \textit{na}.
 }}

  \maintext{maheśvara uvāca |}%

  \maintext{brahmaṇā evam uktas tu viṣṇunā prabhaviṣṇunā |}%

  \maintext{evaṃ bhavatu bhadraṃ vo yathovāca pitāmahaḥ }||\thinspace12:152\thinspace||%
\translation{Maheśvara spoke: When [Vipula] was addressed thus by Brahmā, Lord Viṣṇu [said:] `Let it be like that, bless your soul, just as the Grandfather said.' }

  \maintext{indreṇa raviṇā caiva somena ca punaḥ punaḥ |}%

  \maintext{sādhyādityair marudrudrair viśvebhir vasavais tathā }||\thinspace12:153\thinspace||%
\translation{[Then] Indra, Ravi and Soma, the Sādhyas, the Ādityas, the Maruts, the Rudras, the Viśve[śas] and the Vasus [spoke:] \blankfootnote{12.153 Note that I had to accept \Ed's reading in \textit{pāda} d, and note
  \textit{vasavais} probably for \textit{vasubhiḥ}.
 }}

  \maintext{aho tapaḥphalaṃ divyaṃ vipulasya mahātmanaḥ |}%

  \maintext{svaśarīro divaṃ prāptaḥ śraddhayātithipūjayā }||\thinspace12:154\thinspace||%
\translation{`Wow, what a divine reward for great-souled Vipula's penance! He has reached heaven in his own [mortal] body by virtue of his worshipping a guest in good faith.' }

  \maintext{evam ādīny anekāni vipule parikīrtitam |}%

  \maintext{brahmāṇaṃ punar evāha viṣṇur viśvajagatprabhuḥ }||\thinspace12:155\thinspace||%
\translation{This and many other things are related in the Vipula section. Viṣṇu, the lord of the whole universe, turned back to Brahmā. \blankfootnote{12.155 The reference here to a `Vipula section' is probably to \MBH\ 13.39.1ff, although this story
  is not to be found there. \verify MORE
 The story ends abruptly here in the \VSS. The next chapter starts with a short
  summary by Devī of the previous chapters: 
  \textit{devy uvāca}\thinspace | 
  \textit{ahiṃsātithyakānāṃ ca śruto dharmaḥ suvistaraḥ}\thinspace |
  \textit{kiṃ na kurvanti manujāḥ sukhopāyaṃ mahat phalam}\thinspace ||13.1||
  \textit{svaśarīrasthito yajñaḥ svaśarīre sthitaṃ tapaḥ}\thinspace |
  \textit{svaśarīre sthitaṃ tīrthaṃ śruto vistarato mayā}\thinspace ||13.2||.
 }}
\center{\maintext{\dbldanda\thinspace iti vṛṣasārasaṃgrahe vipulopākhyāno nāmādhyāyo dvādaśamaḥ\thinspace\dbldanda}}
\translation{Here ends the twelfth chapter in the \textit{Vṛṣasārasaṃgraha} called The Story of Vipula.}
