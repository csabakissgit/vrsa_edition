
  \chptr{dvādaśamo 'dhyāyaḥ}
\addcontentsline{toc}{section}{Chapter 12}
\fancyhead[CO]{{\footnotesize\textit{Translation of chapter 12}}}%

  \trchptr{ Chapter Twelve }%

  \subchptr{ātithyadharmaḥ}%

  \trsubchptr{Rules of hospitality}%

  \maintext{devy uvāca |}%

  \maintext{ahiṃsā paramo dharmaḥ satataṃ parikīrtyate |}%

  \maintext{ātithyakānāṃ dharmaṃ ca kathayasva yad uttamam }||\thinspace12:1\thinspace||%
\translation{The Goddess spoke: Non-violence is always praised as the highest Dharma. Teach me also the ultimate Dharma of hospitality. \blankfootnote{12.1 One could read \textit{ahiṃsāparamo dharmaḥ} in \textit{pāda} a. This would
  translate as `A Dharma beyond non-violence is always being praised.'
  It is not crystal clear why \textit{ahiṃsā} is mentioned at all 
  at the beginning of this chapter. I suspect that by \textit{ātithyakānāṃ dharmaṃ},
  one should simply understand \textit{ātithyadharmaṃ}.
 }}

  \maintext{maheśvara uvāca |}%

  \maintext{ahiṃsātithyakānāṃ ca śṛṇu dharmaṃ yad uttamam |}%

  \maintext{trailokyam akhilaṃ devi ratnapūrṇaṃ sulocane }||\thinspace12:2\thinspace||%
\translation{Maheśvara spoke: Hear the ultimate Dharma of non-violence and that of hospitality. O beautiful-eyed goddess, [if] all the three worlds, full of wealth, \blankfootnote{12.2 Understand \textit{ahiṃsātithyakāmāṃ} as \textit{ahiṃsakānām ātithyakānāṃ ca} or
  \textit{ahiṃsāyā ātithyakāṇāṃ ca}.
 }}

  \maintext{caturvedavide dānaṃ na tattulyam ahiṃsakaḥ |}%

  \maintext{śṛṇu dharmam atithyānāṃ kīrtayiṣyāmi sundari }||\thinspace12:3\thinspace||%
\translation{[were handed over as] a gift to [a Brahmin who] knows the four Vedas, [that gift] cannot be compared to somebody who avoids causing harm. Hear the Dharma of the hospitable ones. I shall teach it [to you], O beautiful one. \blankfootnote{12.3 Note that this verse seems to be all that Maheśvara teaches in this chapter on 
  \textit{ahiṃsā}, and that \textit{tattulyam ahiṃsakaḥ} may either contain a sandhi bridge
  {\rm (}\textit{tattulya-m-ahiṃsakaḥ}{\rm )} or be interpreted as \textit{dānaṃ na tat tulyam ahiṃsakena} 
  {\rm (}`that gift is not comparable to a non-violent person'{\rm )}.
 \textit{atithyānāṃ} in \textit{pāda} c stands for \textit{ātithyānāṃ}, \textit{ātithyasya},
  or \textit{ātithyakānāṃ} metri causa.
 }}

  \subchptr{vipulopākhyānam}%

  \trsubchptr{Story of Vipula}%

  \maintext{āsīd vṛttaṃ purākhyānaṃ nagare kusumāhvaye |}%

  \maintext{kapilasya suto vidvān vipulo nāma viśrutaḥ }||\thinspace12:4\thinspace||%
\translation{This is an old story of what happened once in a city called Kusuma. There was a famous and wise man called Vipula, Kapila's son. \blankfootnote{12.4 Kusumapura is Pāṭaliputra, or modern Patna in Bihar.
  This is confirmed in verse 12.12, where 
  the confluence of the Gaṇḍakī and the Gaṅgā is mentioned as a local spot.
  The \textit{dramatis person\ae} in the following story are the following:
  Vipula---a merchant, Kapila's son;
  Vipula's wife;
  a Brahmin guest {\rm (}Dharma in diguise?{\rm )};
  a monkey;
  Bhīmabala---a traveller;
  Puṇḍaka---the foreman of the guild;
  King Siṃhajaṭa;
  Queen Kekayī;
  Caṇḍa and Vicaṇḍa---two envoys of the king;
  Citraratha---the king of the Gandharvas;
  Sūrya, Soma, Indra, Viṣṇu, and Brahmā.
 }}

  \maintext{dharmanityo jitakrodhaḥ satyavādī jitendriyaḥ |}%

  \maintext{brahmaṇyaś ca kṛtajñaś ca madbhaktaḥ kṛtaniścayaḥ }||\thinspace12:5\thinspace||%
\translation{He always followed Dharma, he conquered anger, he spoke only the truth, and he conquered his senses. He was pious and knowledgeable, and he was my determined devotee. \blankfootnote{12.5 \textit{Pāda} d implies that Vipula is a Śaiva devotee, but there is little indication 
  in this story of Vipula's affiliation, except for 12.44, where Maheśvara is mentioned.
  The story as we have it here ends with a praise of Brahmā.
 }}

  \maintext{dhanāḍhyo 'tithipūjyaś ca dātā dānto dayālukaḥ |}%

  \maintext{nyāyārjitadhano nityam anyāyaparivarjitaḥ }||\thinspace12:6\thinspace||%
\translation{He was rich and he worshipped his guests. He was generous, restrained, and kind. His wealth always came through just means. He always stayed away from illegal transactions. \blankfootnote{12.6 While one would normally translate \textit{atithipūjya} {\rm (}in \textit{pāda} a{\rm )} 
  as `to be worshipped by guests,' in the light of the story I
  suspect that the intended meaning is that he worshipped his guests.
 }}

  \maintext{bhāryā ca rūpiṇī tasya candrabimbaśubhānanā |}%

  \maintext{pīnottuṅgastanī kāntā sakalānandakāriṇī |}%

  \maintext{pativratā patiratā patiśuśrūṣaṇe ratā }||\thinspace12:7\thinspace||%
\translation{He had a pretty wife whose face was as beautiful as the disk of the moon. Her breasts were round and elevated, she was lovely, a source of all pleasures. She was faithful, devoted to her husband and his needs. }

  \maintext{atha kenāpi kālena sūryarāga{-}m{-}abhūt tataḥ |}%

  \maintext{grastabhāgatrayas tv āsīt kṛṣṇamādhavamāsike }||\thinspace12:8\thinspace||%
\translation{Now, once there was an eclipse of the sun. Three quarters [of the sun] were eclipsed, and it was in the dark half of the month of Mādhava [April-May]. \blankfootnote{12.8 In \textit{pāda} b, understand \textit{sūryarāgam} as \textit{sūryoparāgaḥ} {\rm (}`eclipse of the sun'{\rm )}. 
  I take °\textit{rāga-m-abhūt} an example of irregular sandhi for °\textit{rāgo 'bhūt}.
 }}

  \maintext{snātukāmāvatīryante sarve pauranṛpādayaḥ |}%

  \maintext{devāś ca pitaraś caiva tarpyante vidhivat tathā }||\thinspace12:9\thinspace||%
\translation{Eager to take a ritual bath, the king and all the citizens went down [to the riverbank]. Then they worshipped the gods and the deceased ancestors according to the rules. \blankfootnote{12.9 Understand \textit{pāda} a as \textit{snātukāmā avatīryante}. It is an instance of double sandhi or
  of a stem form noun in sandhi with the following verb.
 }}

  \maintext{kecij juhvati tatrāgniṃ kecid viprāṃś ca tarpayet |}%

  \maintext{kecid dānopatiṣṭhanti kecit stuvanti devatām }||\thinspace12:10\thinspace||%
\translation{Some sacrificed in the fire, some fed the Brahmins, some were of service with donations, others praised the deity. \blankfootnote{12.10 Understand \textit{agniṃ} in \textit{pāda} a as locative, and \textit{tarpayet} in \textit{pāda} b as plural.
 Note \textit{dāna} in \textit{pāda} c in stem form {\rm (}for the instrumental{\rm )}.
 }}

  \maintext{dhyānayogaratāḥ kecit kecit pañcatape ratāḥ |}%

  \maintext{evaṃ pravartamāneṣu rājanādiṣu sarvaśaḥ }||\thinspace12:11\thinspace||%
\translation{Some people practised yoga meditation, others were engrossed in five-fire penance. While the ritual waving of lamps etc. were being performed all around the place, \blankfootnote{12.11 \textit{rājanādiṣu} in \textit{pāda} d is suspect. The intended meaning may be
  `the royals and other people,' but I prefer now the option to
  take it as a shortened form of \textit{nīrājanādiṣu}, and that is how
  I translate it. Cf., e.g., \SIVP\ 7.30.81cd: 
  \textit{nīrājanādikaṃ kṛtvā pūjāśeṣaṃ samāpayet}.
 }}

  \maintext{vipulo 'pi hi tatraiva gaṅgāgaṇḍakisaṃgame |}%

  \maintext{bhāryayā saha tatraiva snātvā kṣomavibhūṣaṇaḥ }||\thinspace12:12\thinspace||%
\translation{Vipula also, performing a bath there at the confluence of the Gaṅgā and the Gaṇḍakī, attired in linen clothes, together with his wife, \blankfootnote{12.12 Note \textit{gaṇḍaki} metri causa for \textit{gaṇḍakī} in \textit{pāda} b.
 }}

  \maintext{devatāguruviprāṇām anyeṣāṃ tarpaṇe rataḥ |}%

  \maintext{tatrāvasarasamprāpto brāhmaṇo 'tithir āgataḥ }||\thinspace12:13\thinspace||%
\translation{was engrossed in satiating the deities, the gurus, the Brahmins and others. Then, jumping on the possibility, a Brahmin came up [to them] as a guest. }

  \maintext{bhāryā tasyātirūpeṇa mohitā brahmaṇas tadā |}%

  \maintext{brāhmaṇo 'pi tathaiveha rūpeṇāpratimo bhavet }||\thinspace12:14\thinspace||%
\translation{The wife got infatuated with that Brahmin's extreme beauty. The Brahmin [felt] the same. His beauty was unparalleled in the world. \blankfootnote{12.14 \textit{Pāda} d is suspect and the translation of \textit{pāda}s cd is
  tentative. The expression \textit{rūpeṇāpratimo/°pratimā bhuvi} 
  {\rm (}`his/her beauty is unparalleled in the world'{\rm )} is 
  common in the \MBH\ and in the Purāṇas. Is that what was meant here?
  May a dual have been intended? An alternative reading, albeit requiring substantial
  emendations, could be: \textit{brāhmaṇo 'pi tathaivāha rūpeṇāpratimā bhuvi}; 
  `The Brahmin [felt the same] and said [to himself,] 
  her figure is unparalleled in the world.'
  Nevertheless, I retained the reading found in the MSS, and I interpret \textit{pāda} d
  as an indication that this Brahmin was extraordinary, in fact a manifestation of
  Dharma.
 }}

  \maintext{anyonyadṛṣṭisaṃsaktau jātau tau tu parasparam |}%

  \maintext{vipulenāñjaliṃ kṛtvā brāhmaṇa saṃśitavrata }||\thinspace12:15\thinspace||%
\translation{Their gaze got fixed on each other mutually. Vipula joined his hands [and said:] `O virtuous Brahmin, \blankfootnote{12.15 While the apparatus here appears to indicate that in \textit{pāda} a I am following \Ed, in fact
  the majority of the remaining witnesses suggest the same reading.
 }}

  \maintext{ājñāpaya dvijaśreṣṭha adya me 'nugrahaṃ kuru |}%

  \maintext{bhāryābhṛtyapaśugrāma ratnāni vividhāni ca }||\thinspace12:16\thinspace||%
\translation{I am at your service, be gracious to me now, O great Brahmin. [My] wife, servants, cattle, village, and all kinds of jewels [are all at your service].' \blankfootnote{12.16 °\textit{grāma} in \textit{pāda} c is in stem form, although it would be unproblematic
  to correct it to the neuter singular {\rm (}to form a \textit{samāhāra\-samāsa}{\rm )}.
 }}

  \maintext{vipulenaivam uktas tu gṛhīto brāhmaṇo 'bravīt |}%

  \maintext{yadi satyaṃ pradātāsi suprasannaṃ manas tava }||\thinspace12:17\thinspace||%
\translation{Having been addressed and greeted hospitably by Vipula, the Brahmin spoke: `If you really mean to give, your heart is very generous.' \blankfootnote{12.17 Note that \msCc's omission of \textit{pāda}s cd here could be due
  to an eyeskip from \textit{suprasannaṃ} in 12.17d to \textit{suprasannaṃ} in 12.18a, 
  although this would have also led to an omission of the next \textit{vipula uvāca}.
 }}

  \maintext{vipula uvāca |}%

  \maintext{suprasannaṃ mano me 'dya suprasannaṃ tapaḥphalam |}%

  \maintext{śīghram ājñāpaya vipra yac cābhilaṣitaṃ tava |}%

  \maintext{adeyaṃ nāsti viprasya svaśiraḥprabhṛti dvija }||\thinspace12:18\thinspace||%
\translation{Vipula spoke: `My heart is generous today, generosity is the fruit of austerity. Just command me quickly, O Brahmin. What is your desire? There is nothing that should not be donated to a Brahmin, including one's own head, O Brahmin.' \blankfootnote{12.18 \textit{Pāda} c is either a \textit{sa-vipulā} or by applying the \mutacumliquida,
  by which °\textit{pra} does not make \textit{vi}° a long syllable, a \textit{na-vipulā}.
 }}

  \maintext{brāhmaṇa uvāca |}%

  \maintext{yady evaṃ vadase bhadra bhāryāṃ me dehi rūpiṇīm |}%

  \maintext{svasti bhavatu bhadraṃ vaḥ kalyāṇaṃ bhava śāśvatam }||\thinspace12:19\thinspace||%
\translation{The Brahmin spoke: `If you talk like this, dear Sir, give me your beautiful wife. May there be happiness, may you be fortunate, and may you prosper eternally!' \blankfootnote{12.19 \textit{Pāda} c has the metrical fault of two \textit{laghu}s in the second and third position.
 
  In \textit{pāda} d, \textit{bhava} is less than satisfactory. One would normally expect 
  \textit{bhavate/bhavatāṃ/bhavatu} in this context. Alternatively, it is possible that
  \textit{kalyāṇo bhava} {\rm (}`be happy'{\rm )} was meant, or \Ed's reading {\rm (}\textit{tava}{\rm )}
  could be accepted as a conjecture.
 }}

  \maintext{vipula uvāca |}%

  \maintext{pratīccha bhāryāṃ suśroṇīṃ rūpayauvanaśālinīm |}%

  \maintext{akutsitāṃ viśālākṣīṃ pūrṇacandranibhānanām }||\thinspace12:20\thinspace||%
\translation{Vipula spoke: `Accept my nice-buttocked, young and beautiful wife, who is blameless, large-eyed, and whose face resembles the full-moon.' }

  \maintext{bhāryovāca |}%

  \maintext{parityājyā kathaṃ nātha apāpāṃ tyajase katham |}%

  \maintext{atīva hi priyāṃ bhāryāṃ nirdoṣāṃ ca kathaṃ tyajeḥ }||\thinspace12:21\thinspace||%
\translation{The wife spoke: `How can you abandon me, my lord? How can you dismiss a woman who is sinless? How can you abandon a wife who is extremely kind and faultless? \blankfootnote{12.21 All witnesses consulted read \textit{sa} instead of my conjectured \textit{ca} in \textit{pāda} d.
  \textit{sa} might work if we read \textit{tyajet} {\rm (}\msCb\msCc{\rm )} instead of \textit{tyajeḥ} {\rm (}\msCa\msNa\msNc{\rm )},
  but even this version sounds a bit out of context {\rm (}`how can he abandon...'{\rm )}.
 }}

  \maintext{sakhā bhāryā manuṣyāṇām iha loke paratra ca |}%

  \maintext{dānaṃ vā sumahad dattvā yajño vā subahuḥ kṛtaḥ }||\thinspace12:22\thinspace||%
\translation{A wife is a man's friend in this world and in the other world. [Even if] a man gives enormous donations or performs numerous sacrifices, }

  \maintext{aputro nāpnuyāt svargaṃ tapobhir vā suduṣkaraiḥ |}%

  \maintext{śruto me pitṛbhiḥ prokto brāhmaṇaiś ca mamāntike }||\thinspace12:23\thinspace||%
\translation{or performs hard penance, he cannot reach heaven without having a son. I have heard this as taught by my father and my uncles, and by Brahmins in my presence. \blankfootnote{12.23 Note \textit{me} as instrumental in \textit{pāda} c
  {\rm (}\mycitep{OberliesEpicSkt}{102--103 [4.1.3]}{\rm )}.
  I translate \textit{pitṛbhiḥ} in the same \textit{pāda} as
  `father and uncles,' and not as `ancestors' because the former 
  fits the context better.
 }}

  \maintext{aputro nāpnuyāt svargaṃ śrutaṃ me bahuśaḥ purā |}%

  \maintext{mandapālo dvijaśreṣṭho gataḥ svargaṃ tapobalāt }||\thinspace12:24\thinspace||%
\translation{A sonless man cannot reach heaven. I have heard this so many times! Mandapāla, the great Brahmin, went to heaven as a reward of his austerities, \blankfootnote{12.24 Note \textit{me} as instrumental again in \textit{pāda} b.
 See details of Mandapāla's story, here summarised, in \MBH\ 1.220.5ff.
 }}

  \maintext{dānāni ca bahūn dattvā yajñāṃś ca vividhāṃs tathā |}%

  \maintext{vedāṃś ca japayajñāṃś ca kṛtvā sa dvijasattamaḥ }||\thinspace12:25\thinspace||%
\translation{having made numerous donations, having performed various sacrifices, Vedic sacrifices and sacrifices of recitation, that great Brahmin. \blankfootnote{12.25 Note \textit{dānānī bahūn} for \textit{dānāni bahūni} in \textit{pāda} a.
  Understand \textit{pāda} c as \textit{vedayajñāñ japayajñāṃś ca kṛtvā}.
  {\rm (}See \textit{vedayajña} mentioned in \VSS\ 3.37a above.{\rm )}
  On \textit{japayajña}, see \VSS\ 6.1--2 and 5 above, as well as, e.g.,
  \BHG\ 10.25c {\rm (}\textit{yajñānāṃ japayajño 'smi}{\rm )} and \MANU\ 2.86 
  {\rm (}\textit{vidhiyajñāj japayajño viśiṣṭo daśabhir guṇaiḥ}{\rm )}.
 }}

  \maintext{prāptadvāro 'pi yasyāpi devadūtair nivāritaḥ |}%

  \maintext{aputro nāpnuyāt svargaṃ yadi yajñaśatair api }||\thinspace12:26\thinspace||%
\translation{But even he, even when he reached the gate [of heaven], was stopped by the celestial messengers. [They said:] ``The sonless cannot enter heaven, not even by hundreds of sacrifices.'' \blankfootnote{12.26 \textit{Pāda}s ab are not perfectly smooth syntactically, \textit{yasyāpi} is difficult to fit in.
  Perhaps understand \textit{prāptadvāre 'pi yasmin sa devatūtair nivāritaḥ}. Alternatively, 
  \textit{yasya} might reference \textit{svargaḥ}.
 }}

  \maintext{ity uktas tu cyutaḥ svargān mandapālo mahān ṛṣiḥ |}%

  \maintext{putrān utpādayām āsa śāraṅgāṃś caturo dvijaḥ }||\thinspace12:27\thinspace||%
\translation{Mandapāla, the great sage, having been thus informed fell from heaven. The Brahmin begot four sons with a Śāraṅga-bird. }

  \maintext{tena puṇyaprabhāveṇa svargaṃ prāpto hy avāritaḥ |}%

  \maintext{kulatrāṇāt kalatrāsmi bharaṇād bhārya eva ca }||\thinspace12:28\thinspace||%
\translation{By the virtue of this, he reached heaven unobstructed. I am a wife {\rm (}\textit{kalatra}{\rm )} because I protect the family {\rm (}\textit{kulatrāṇāt}{\rm )}, and I am a wife to be supported {\rm (}\textit{bhāryā}{\rm )} because I bear [sons] {\rm (}\textit{bharaṇa}{\rm )}. \blankfootnote{12.28 Note that \textit{pāda} c is the result of emendations 
  {\rm (}the majority of the MSS read \textit{kalatrāṇāṃ kalatrāsmi}{\rm )},
  and that \textit{bhārya} in \textit{pāda} d is to be understood as \textit{bhāryā} 
  metri causa. I added `to be supported' in the translation to convey 
  the general meaning of the word \textit{bhārya}, 
  which seemed to fit the context well.
 }}

  \maintext{dārasaṃgraha putrārthe kriyate śāstradarśanāt |}%

  \maintext{yāni santi gṛhe dravyaṃ grāmaghoṣagṛhāṇi ca }||\thinspace12:29\thinspace||%
\translation{Taking a wife is for the sake of having sons according to the Śāstras. Please give that Brahmin all the wealth you find at home, the village, the stations of herdsmen, and the houses, \blankfootnote{12.29 Note the stem form °\textit{saṃgraha} metri causa in \textit{pāda} a.
 Note the number discrepancy between \textit{yāni santi} and \textit{dravyaṃ} in \textit{pāda} c,
  which is repeated in 12:42a.
 }}

  \maintext{dātum arhasi viprāya na māṃ dātum ihārhasi |}%

  \maintext{bhāryāyā vacanaṃ śrutvā vipulaḥ punar abravīt }||\thinspace12:30\thinspace||%
\translation{but please don't give me away this time!' Having heard his wife's speech, Vipula spoke again. \blankfootnote{12.30 I have not included \msCcpcorr's \textit{vipula uvāca} {\rm (}echoed in \Ed{\rm )}
  because after \textit{punar abravīt} is seems secondary and unnecessary.
  Note that the correction in \msCc\ is in a second hand and it is also 
  to be found in \msPaperA\ and \msPaperC\ {\rm (}see p.~\pageref{vipulauvaca}{\rm )}.
 }}

  \maintext{sādhu bhāmini jānāmi sādhu sādhu pativrate |}%

  \maintext{jito 'smy anena vākyena anenāsmi hi toṣitaḥ }||\thinspace12:31\thinspace||%
\translation{`Alright, my beautiful wife, I know! Good, good, my faithful wife! I am beaten by this speach and I am satisfied with it. }

  \maintext{adya grahaṇakāle ca dvija āgatya yācate |}%

  \maintext{dadāmīti pratijñāya adattvā narakaṃ vraje }||\thinspace12:32\thinspace||%
\translation{Today the Brahmin came up to me at the time of eclipse, and he asked me. I promised him that I would give [you away]. If I don't give [you to him], I will go to hell. }

  \maintext{narakaṃ yadi gacchāmi kulena saha sundari |}%

  \maintext{kalpakoṭisahasre 'pi narakastho yaśasvini |}%

  \maintext{muktim eva na paśyāmi janmakoṭiśatair api }||\thinspace12:33\thinspace||%
\translation{If I go to hell along with my family, I will be in hell, O brilliant woman, for millions of \ae ons, and will not see release for millions of births. \blankfootnote{12.33 The reading \textit{narakastho} in \textit{pāda} b {\rm (}\msNc\Ed{\rm )} might not be the original one but it is 
  definitely the simplest solution. \textit{narakasthād} may be original,
  possibly meaning \textit{narakasthānād}.
 }}

  \maintext{adānāc cāśubhaṃ devi paśyāmi varavarṇini |}%

  \maintext{dānena tu śubhaṃ paśye svargaloke yad akṣayam }||\thinspace12:34\thinspace||%
\translation{I can see something bad [coming], my Princess, from not giving, O woman with a nice complexion, but from giving I can see something good in heaven that is eternal. }

  \maintext{noktaṃ mayānṛtaṃ pūrvaṃ nityaṃ satyavrate sthitaḥ |}%

  \maintext{satyadharmam atikramya nānyadharmaṃ samācare }||\thinspace12:35\thinspace||%
\translation{I have never ever lied, I always observe the vow of truthfulness. If I transgressed the Dharma of truthfullness, [by this] I would stop following all other Dharmas [too]. }

  \maintext{bhāryā dharmasakhety evaṃ tvayā pūrvam udāhṛtam |}%

  \maintext{yadi dharmasakhāyāsi so 'dya kāla ihāgataḥ }||\thinspace12:36\thinspace||%
\translation{You mentioned earlier that the wife is one's Dharmic friend. If you are indeed Dharma's friend, it was actually the perfect time for him to come up to us today. \blankfootnote{12.36 I have emended \textit{tvayi} in \textit{pāda} d to \textit{tvayā} because it 
  seems an early random scribal mistake, rather than some 
  linguistic pecularity.
 Note the form \textit{sakhāyā} for a feminine \textit{sakhī} or \textit{sahāyā}. I sense a touch
  of humour or sarcasm in Vipula's spin on his wife's claim in 12.22a that
  `a wife is a man's friend': now he suggests that his wife, his `Dharmic friend,'
  is actually friends with Dharma.
 }}

  \maintext{dvijarūpadharo dharmaḥ svayam eva ihāgataḥ |}%

  \maintext{jijñāsārtham ahaṃ bhadre na vighnaṃ kartum arhasi }||\thinspace12:37\thinspace||%
\translation{[For] Dharma himself visited us, disguised as a Brahmin. I am being tested. My dear, please don't cause me trouble. \blankfootnote{12.37 \textit{jijñāsārtham ahaṃ} in \textit{pāda} c is slightly clumsy. Understand
  \textit{maj-jijñāsārthaṃ} {\rm (}`in order to test me'{\rm )}.
 }}

  \maintext{mātāvyaktaḥ pitā brahmā buddhir bhāryā damaḥ sakhā |}%

  \maintext{putro dharmaḥ kriyācārya ity ete mama bāndhavāḥ }||\thinspace12:38\thinspace||%
\translation{The unmanifest {\rm (}\textit{avyakta}{\rm )} is my mother, Brahmā is my father, intelligence my wife, self-control my friend. Dharma is my son, ritual is my teacher. These are my relatives. }

  \maintext{kālaśreṣṭho grahaḥ sūryo gaṅgā śreṣṭhā nadīṣu ca |}%

  \maintext{candrakṣaye dinaṃ śreṣṭhaṃ naraśreṣṭho dvijottamaḥ }||\thinspace12:39\thinspace||%
\translation{The best time is the time of the eclipse of the Sun. The best one among the rivers is the Gaṅgā. The best day is at new moon, the best man is the Brahmin. \blankfootnote{12.39 I understand \textit{grahaḥ sūryo} in \textit{pāda} a as \textit{sūryagrahaḥ} {\rm (}or \textit{sūryagrahaṇam}{\rm )}:
  the eclipse of the Sun, which appears to be an auspicious day.
  See, e.g., \AgamaKL\ 3.128:
  
 
  \textit{sūryagrahaṇakālasya samānā nāsti bhūtale}\thinspace | 
 
  \textit{atra yad yat kṛtaṃ karma anantaphaladaṃ bhavet}\thinspace ||
 
 
  This short list of `best of' items anticipates \VSS\ 15.16--29, a longer
  list of what is best in every possible category, not entirely differently in
  manner from \BHG\ 10.21--38.
 }}

  \maintext{śuśrūṣaṇārthaṃ viprasya mayā dattāsi sundari |}%

  \maintext{sarvasvaṃ brāhmaṇe dattvā vanam evāśrayāmy aham }||\thinspace12:40\thinspace||%
\translation{I have given you to the Brahmin to serve him, O beautiful woman. After I have given all my riches to the Brahmin, I shall resort to the forest.' \blankfootnote{12.40 \textit{Pāda} d may give a hint at the connection between this chapter and the
  end of the previous one: this story is partly a propagation 
  of the life of the \textit{vānaprastha}.
 }}

  \maintext{śaṅkara uvāca |}%

  \maintext{tūṣṇīmbhūtā tato bhāryā aśrupūrṇākulekṣaṇā |}%

  \maintext{kare gṛhya viśālākṣī brāhmaṇāya niveditā }||\thinspace12:41\thinspace||%
\translation{Śaṅkara spoke: The wife remained silent, her bewildered eyes filled with tears. [Vipula] took her by the hand and the long-eyed woman was presented to the Brahmin. \blankfootnote{12.41 Note that the variant \textit{maheśvara uvāca} in \Ed\ may seem as an odd
  alteration by Naraharinath, but in fact \msPaperA\ and \msPaperC\
  {\rm (}neither collated for this chapter{\rm )} also read the same. 
  See pp.~\pageref{msPaperAdesc} ff.
 }}

  \maintext{yāni santi gṛhe dravyaṃ hiraṇyaṃ paśavas tathā |}%

  \maintext{dadāmi te dvijaśreṣṭha grāmaghoṣagṛhādikam }||\thinspace12:42\thinspace||%
\translation{`I am ready to give you all the wealth I have at home, all the gold and cattle, O great Brahmin, the village, the stations of herdsmen, and the houses, and everything else, }

  \maintext{muktāvaiḍūryavāsāṃsi divyāṇy ābharaṇāni ca |}%

  \maintext{sarvān gṛhāṇa viprendra śraddhayā dattasatkṛtān }||\thinspace12:43\thinspace||%
\translation{pearls, gems, clothes, and exquisite jewellery. Accept all these, O best of Brahmins. It's given in good faith and with respect. }

  \maintext{prīyatāṃ bhagavān dharmaḥ prīyatāṃ ca maheśvaraḥ |}%

  \maintext{prīyantāṃ pitaraḥ sarve yady asti sukṛtaṃ phalam }||\thinspace12:44\thinspace||%
\translation{May Lord Dharma be pleased and may Maheśvara be pleased. May all the ancestors rejoice if there is reward for meritorious acts.' \blankfootnote{12.44 Note \SDHS\ 10.11cd, in a similar context of donations:
  \textit{bhojayitvā tato brūyāt prīyatāṃ bhagavān śivaḥ}.
 Understand \textit{sukṛtaṃ phalam} as \textit{sukṛtaphalam} {\rm (}metri causa{\rm )}.
 }}

  \maintext{rudra uvāca |}%

  \maintext{vipulasya vacaḥ śrutvā brāhmaṇena tapasvinā |}%

  \maintext{āśīḥ suvipulaṃ dattvā vipulāya mahātmane }||\thinspace12:45\thinspace||%
\translation{Rudra spoke: Having heard Vipula's speech, the ascetic Brahmin blessed the good-souled Vipula a good number of times, \blankfootnote{12.45 Note that the variant \textit{maheśvara uvāca} in \Ed\ again is to be found in 
  \msPaperA, but this time not in \msPaperC\ {\rm (}compare note to 12.41{\rm )}.
 One may wonder why the Brahmin is labelled as ascetic {\rm (}\textit{tapasvin}{\rm )} 
  in \textit{pāda} b. 
  
  
  There are several ways to explain the form \textit{āśīḥ} in \textit{pāda} c.
  The easiest is to treat it as a singular accusative neuter.
  Alternatively, it could be a plural accusative feminine of \textit{āśī} and
  then \textit{suvipulaṃ} is either to be understood adverbially or as \textit{suvipulā}[\textit{ḥ}].
  Another way to treat \textit{āśīḥ} would be to take it as a nominative standing
  for the accusative.
 }}

  \maintext{vaset tatra gṛhe ramye bhāryām ādāya tasya ca |}%

  \maintext{vipulas tu namaskṛtvā kṛtvā cāpi pradakṣiṇam }||\thinspace12:46\thinspace||%
\translation{and then went off to live in a nice house, taking Vipula's wife with him. As for Vipula, he saluted and circulambulated him. }

  \maintext{brāhmaṇam abhivādyaivaṃ gataḥ śīghraṃ vanāntaram |}%

  \maintext{vane mūlaphalāhāro vicareta mahītale }||\thinspace12:47\thinspace||%
\translation{Thus saying good-bye to the Brahmin, he departed quickly into the forest. In the forest, he lived off roots and fruits, and roamed the world. \blankfootnote{12.47 Note the metrical problem in \textit{pāda} a {\rm (}two \textit{laghu}s{\rm )}.
 }}

  \maintext{ekākī vijane śūnye cintayā ca pariplutaḥ |}%

  \maintext{kva gacchāmi kva bhokṣyāmi kutra vā kiṃ karomy aham }||\thinspace12:48\thinspace||%
\translation{But being alone in an abandoned and deserted place, he got overwhelmed with worry. `Where should I go? Where could I find food? From whom? What shall I do? }

  \maintext{na pathaṃ viṣayaṃ vedmi grāmaṃ vā nagarāṇi vā |}%

  \maintext{kheṭakharvaṭadeśaṃ vā jānāmīha na kaṃcana }||\thinspace12:49\thinspace||%
\translation{I don't know these roads, this country, these villages, and these cities, towns, and mountain settlements. I don't know anybody here. \blankfootnote{12.49 In \textit{pāda} c, I accepted \Ed's reading {\rm (}°\textit{kharvaṭa}°, `a mountain village'{\rm )}
  against all witnesses consulted. The MSS transmit a reading that is 
  difficult to make sense of {\rm (}°\textit{kharpaṭa}, `ragged garment'{\rm )}.
  In \textit{pāda} d, the reading of all the witnesses, \textit{kaścana}, seems to be
  an early scribal mistake for \textit{kañcana}. But note that the same happens in 12.55d.
 }}

  \maintext{amuṃ suśailaṃ paśyāmi vipulodarakandaram |}%

  \maintext{tam āruhya nirīkṣyāmi grāmaṃ nagarapattanam }||\thinspace12:50\thinspace||%
\translation{I can see a nice mountain yonder with large cavities and caves. I'll climb it and try to figure out if there is a village, town, or city [nearby].' \blankfootnote{12.50 \textit{Pāda} a is a \textit{ma-vipulā}.
 }}

  \maintext{evam uktvā tu vipulaḥ śanaiḥ parvatam āruhat |}%

  \maintext{vṛkṣacchāyāṃ samālokya niṣasāda śramānvitaḥ }||\thinspace12:51\thinspace||%
\translation{Having said this, Vipula climbed the mountain slowly. He caught sight of the shades of a tree, and, being exhausted, sat down [there]. \blankfootnote{12.51 I have accepted the reading of \Ed\ in \textit{pāda} d {\rm (}\textit{āruhat}{\rm )}
  because I think that \textit{āruhet} is an early scribal mistake that
  is easy to make, and because °\textit{āruhat} comes up again in 12.53d.
  Additionally, \msPaperA\ {\rm (}not collated here{\rm )}
  seems to read \textit{āruhat} too {\rm (}f.~220r{\rm )}.
 }}

  \maintext{etasminn eva kāle tu vṛkṣaśākhāvatārya ca |}%

  \maintext{apūrvaṃ ca surūpaṃ ca sugandhatvaṃ ca śobhanam }||\thinspace12:52\thinspace||%
\translation{In the same moment, descending from among the branches of the tree, [a monkey appeared and] carrying an extraordinary, beautiful, fragrant, exquisite, \blankfootnote{12.52 Note the stem form noun °\textit{śākhā} in \textit{pāda} b. Understand °\textit{śākhāyā avatārya}
  or \textit{śākhayāvatārya}. Understand \textit{sugandhatvaṃ} in \textit{pāda} d as
  \textit{sugandhi}. 
  
 
  From this point on, the story might be interpreted as a dream. See especially 12.149ab:
  \textit{svapnabhūtam ivāścāryaṃ paśyāmi...}.
 }}

  \maintext{phalaṃ gṛhya vicitraṃ ca hṛdayānandanaṃ śubham |}%

  \maintext{vipulasyāgrataḥ kṛtvā punar vṛkṣaṃ samāruhat }||\thinspace12:53\thinspace||%
\translation{lovely, delightful and pleasant-looking fruit, it put it in front of Vipula, and then climbed back onto the tree. \blankfootnote{12.53 Note how the agent of this sentence is omitted here. That it was a monkey
  that gave Vipula the fruit becomes clear in 12.94 below.
 }}

  \maintext{vipulaś citravad dṛṣṭvā vismayaṃ paramaṃ gataḥ |}%

  \maintext{aho vā svapnabhūto 'smi aho vā tapasaḥ phalam }||\thinspace12:54\thinspace||%
\translation{Vipula, looking [at it] as if seeing a miracle, was perplexed. Wow, am I sleeping? Or is this the fruit of my penance? \blankfootnote{12.54 See notes on 12.52 above on how most of the story could be interpreted as a 
  dream.
 }}

  \maintext{na paśyāmi na jighrāmi na ca svādaṃ ca vedmy aham |}%

  \maintext{vārttāpi na ca me śrotā pratijānāmi kaṃcana }||\thinspace12:55\thinspace||%
\translation{I have never seen, smelt, or tasted anything like this. I have not even heard of anything like this. I shall let somebody know about it. \blankfootnote{12.55 Note the use of the {\rm (}non-historical{\rm )} present tense
  in \textit{pāda}s ab clearly pointing to past events.
 I suspect that \textit{śrotā} in \textit{pāda} c is meant to be feminine participle \textit{śrutā}, but
  the metre required the first vowel to be lengthened.
  Understand \textit{me} as \textit{mayā} {\rm (}\mycitep{OberliesEpicSkt}{102--103 [4.1.3]}{\rm )}.
  In \textit{pāda} d, the reading of all the witnesses, \textit{kaścana}, seems to be
  an early scribal mistake for \textit{kañcana}. Note that the same happens in 12.49d.
 }}

  \maintext{evam uktvā hy anekāni phalaṃ gṛhya manoramam |}%

  \maintext{sunirīkṣya punar jighran punar jighran nirīkṣya ca }||\thinspace12:56\thinspace||%
\translation{Having repeated this several times, taking that nice fruit, he kept observing it smelling it again and again. \blankfootnote{12.56 Since one of the main points, and a source of conflict, in the story is 
  that there was only one single fruit, we have to interpret \textit{anekāni} in 
  \textit{pāda} a as a shortened form of \textit{anekavāram} {\rm (}`repeatedly'{\rm )}.
 Most sources consulted read \textit{jighra} or \textit{jighraṃ} in both 
  \textit{pāda} c and d, i.e. most of them do not suggest the participle \textit{jighran}, 
  which seems to be the correct reading. I have altered this
  part of the text silently.
 }}

  \maintext{phalaṃ cātra nirūpyanto deśaṃ vāpy avalokayan |}%

  \maintext{pātheyarahitaś cāsmi devadattaṃ phalaṃ mama }||\thinspace12:57\thinspace||%
\translation{`While gazing at this fruit, and observing the countryside, I have run out of provisions. This fruit is godsent. \blankfootnote{12.57 Understand \textit{nirūpyanto} in \textit{pāda} a as a thematised present participle 
  in the nominative {\rm (}\textit{nirūpayan}{\rm )}. This is also suggested by the standard \textit{avalokayan}
  in \textit{pāda} b.
 }}

  \maintext{tat phalaṃ pratigṛhyaiva nagaraṃ praviśāmy aham |}%

  \maintext{prārthayitvā tu yat kiṃcij jīvanārthaṃ carāmy aham }||\thinspace12:58\thinspace||%
\translation{Therefore I shall take this fruit and enter that city, and I shall go and seek something to live on.' }

  \maintext{tataḥ śailam atikramya nagaraṃ praviveśa ha |}%

  \maintext{pathi kaścij janaḥ pṛṣṭhaḥ kiṃnāma nagaraṃ tv idam }||\thinspace12:59\thinspace||%
\translation{Then crossing that mountain, he entered the city. He asked a man on the road: `What is the name of this city?' }

  \maintext{sa hovāca pathīkena kim apūrvam ihāgataḥ |}%

  \maintext{dakṣiṇāpathadeśo 'yaṃ naravīrapuraṃ tv adaḥ }||\thinspace12:60\thinspace||%
\translation{The traveller replied: `Have you never been here before? This is the Deccan region, and this is the city of Naravīra. \blankfootnote{12.60 I understand \textit{pathīkena} as standing for \textit{pathikena} metri causa {\rm (}see 12.64b{\rm )} and not
  as two words, \textit{pathī kena}. This means that we are forced to accept 
  an instrumental as the agent of the finite verb \textit{uvāca} 
  {\rm (}ergative structure, see p.~\pageref{ergative}{\rm )}.
  I suspect that \msNc's reading {\rm (}\textit{pathīko na}{\rm )} 
  is an attempt to correct the syntax, but in this way \textit{na \dots\ apūrvam} becomes 
  problematic.
 
 
  \textit{ayam} as the end of this verse may have been the original reading and
  \msCb\ may have corrected it to \textit{adaḥ}. Another possibility is that
  an original \textit{adaḥ} is preserved in \msCb, and it got corrupted to
  \textit{ayaḥ} {\rm (}\msCa{\rm )}, and then to \textit{ayaṃ} {\rm (}\msCc\msNa{\rm )}. 
  In any case, in this case I have chosen the not-so-well attested reading \textit{adaḥ} 
  simply because it works better. Another possibility would be to echo 12.59d and
  correct to \textit{idam}.
 
  Since I am not aware of any attestation of Naravīrapura as a city,
  I suspect that this name is either a mistake for or a pun on
  Karavīrapura, possibly modern Kolhapur in Maharashtra. 
  See p.~\pageref{naravirapura}.
 }}
