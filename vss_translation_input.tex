
  \chptr{daśamo 'dhyāyaḥ}
\addcontentsline{toc}{section}{Chapter 10}
\fancyhead[CO]{{\footnotesize\textit{Translation of chapter 10}}}%

  \trchptr{ Chapter Seven }%

  \subchptr{kāyatīrthopavarṇanam}%

  \trsubchptr{Description of the pilgrimage places in the body}%

  \maintext{vigatarāga uvāca |}%

  \maintext{katamaṃ sarvatīrthānāṃ śreṣṭham āhur manīṣinaḥ |}%

  \maintext{kathayasva muniśreṣṭha yady asti bhuvi kāmadam }||\thinspace10:1\thinspace||%
\translation{Vigatarāga spoke: Which pilgrimage place {\rm (}\textit{tīrtha}{\rm )} do the wise consider the best of all? Tell me, O best of sages, if there is one in the world that fulfills [all] desires. }

  \maintext{anarthayajña uvāca |}%

  \maintext{atiguhyam idaṃ praśnaṃ pṛṣṭaḥ snehād dvijottama |}%

  \maintext{bravīmi vaḥ purāvṛttaṃ nandinā kathito 'smy aham }||\thinspace10:2\thinspace||%
\translation{Anarthayajña spoke: This question is an extremely deep secret. [Now that you] ask [me], O excellent Brahmin, I shall teach you, out of fondness, an ancient legend that Nandi told me. \blankfootnote{10.2 On the syntax of \textit{pāda} d, see pp.~\pageref{kathita} ff.
 }}

  \maintext{nandikeśvara uvāca |}%

  \maintext{kailāsaśikhare ramye siddhacāraṇasevite |}%

  \maintext{tatrāsīnaṃ śivaṃ sākṣād devī vacanam abravīt }||\thinspace10:3\thinspace||%
\translation{Nandikeśvara spoke: On the beautiful peak of Mount Kailāsa, which is frequented by Siddhas and celestial singers {\rm (}\textit{cāraṇa}{\rm )}, Devī asked Śiva, who was sitting there in his manifest form {\rm (}\textit{sākṣāt}{\rm )}. \blankfootnote{10.3 Note the change of speaker here: Nandikeśvara is also the main
  interlocutor of the \SDHS\ and the \SDHSAMGR.
  This verse marks the beginning of the layer that can be labelled Śaiva
  {\rm (}see pp.~\pageref{structure} ff{\rm )}.
  On Nandi/Nandin/Nandikeśvara not being Śiva's bull, see
  \mycite{bhattacharya_nandin_1977} and pp.~\pageref{bull} ff. above.
 }}

  \maintext{devy uvāca |}%

  \maintext{bhagavan devadeveśa sarvabhūtajagatpate |}%

  \maintext{praṣṭum icchāmy ahaṃ tv ekaṃ dharmaguhyaṃ sanātanam }||\thinspace10:4\thinspace||%
\translation{Devī spoke: O Lord, Lord of the chiefs of the gods, O ruler of all beings and of the whole world, I would like to ask you about an eternal secret concerning Dharma, \blankfootnote{10.4 It is not unlikely that in \textit{pāda} d, \textit{sanātanam} was intended to refer to
  \textit{dharma}° {\rm (}`eternal Dharma'{\rm )}, or that \textit{dharmaguhya} should be corrected
  to \textit{dharmaṃ guhyaṃ} {\rm (}`\dots~ask you about the secret and eternal Dharma'{\rm )}.
 }}

  \maintext{atitīrthaṃ paraṃ guhyaṃ saṃsārād yena mucyate |}%

  \maintext{manuṣyāṇāṃ hitārthāya brūhi tattvaṃ maheśvara }||\thinspace10:5\thinspace||%
\translation{about the transcendental and highly secret pilgrimage place at which one can be liberated from mundane existence {\rm (}\textit{saṃsāra}{\rm )}. O Maheśvara, teach me the truth for the benefit of mankind. }

  \maintext{maheśvara uvāca |}%

  \maintext{ko māṃ pṛcchati taṃ praśnaṃ muktvā tvām eva sundari |}%

  \maintext{śṛṇu vakṣyāmi taṃ praśnaṃ devair api sudurlabham }||\thinspace10:6\thinspace||%
\translation{Maheśvara spoke: Who else could ask me that question except for you, O Sundarī? Listen, I shall expound that question, which is difficult to grasp even for the gods. }

  \maintext{kurukṣetraṃ prayāgaṃ ca vārāṇasīm ataḥ param |}%

  \maintext{gaṅgāgniṃ somatīrthaṃ ca sūryapuṣkaramānasam }||\thinspace10:7\thinspace||%
\translation{If one gets to know Kurukṣetra, Prayāga, Vārāṇasī, Gaṅgā, Agni[tīrtha], Somatīrtha, Sūrya[tīrtha], Puṣkara, Mānasa, }

  \maintext{naimiṣaṃ bindusāraṃ ca setubandhaṃ suradraham |}%

  \maintext{ghaṇṭikeśvaravāgīśaṃ jñātvā niścayapāpahā }||\thinspace10:8\thinspace||%
\translation{Naimiṣa, Bindusaras, Setubandha, Suradraha, Ghaṇṭikeśvara, and Vāgīśa, one will certainly be able to destroy one's sins. \blankfootnote{10.8 Note \textit{bindusāraṃ} for \textit{bindusaras/°saraṃ/°sarasaṃ} metri causa.
 Although some of these toponyms are difficult to identify and some may refer to
  southern locations {\rm (}e.g. Setubandha{\rm )}, in general they suggest a North Indian focus.
  See details on the pilgrimage places in this chapter on pp.~\pageref{provenance} ff.
 }}

  \maintext{umovāca |}%

  \maintext{evamādi mahādeva pūrvavat kathitāsmy aham |}%

  \maintext{svargabhogapradaṃ tīrtham eteṣāṃ suranāyaka }||\thinspace10:9\thinspace||%
\translation{Umā spoke: I have been taught this previously, O Mahādeva. [Which is] the pilgrimage place that yields all kinds of enjoyment, O Suranāyaka? \blankfootnote{10.9 I take \textit{pūrvavat} in \textit{pāda} b as if used in the sense of \textit{pūrvaṃ} {\rm (}`previously'{\rm )},
  and \textit{eteṣāṃ} in \textit{pāda} d as \textit{eteṣu}. It would also be possible to take \textit{eteṣāṃ}
  in 10.9d and \textit{jñānamātreṇa} in 10.10b as connected {\rm (}`by the 
  mere knowledge of them'; actually, one should understand 
  \textit{svargabhogapradānāṃ tīrthānām eteṣāṃ}{\rm )},
  but the former solution, namely taking \textit{eteṣāṃ} as \textit{eteṣu},
  seems to work also in 10.14, where again a genitive {\rm (}\textit{teṣāṃ}{\rm )} 
  may stand for a locative {\rm (}\textit{teṣu}{\rm )}. On the syntax of \textit{pāda} b,
  see p.~\pageref{kathita}.
 }}

  \maintext{kathaṃ mucyeta saṃsārāj jñānamātreṇa īśvara |}%

  \maintext{kautūhalaṃ mahaj jātaṃ chindhi saṃśayakārakam }||\thinspace10:10\thinspace||%
\translation{[And] how is one liberated from mundane existence by merely knowing [the pilgrimage places], O Īśvara? Cut [this] great curiosity arising [in me] that causes doubt. \blankfootnote{10.10 We are forced to agree with \Ed's printing °\textit{kārakam} in \textit{pāda} d because
  all the other readings seem out of context, whether they refer to Śiva
  in the vocative or nominative.
 }}

  \maintext{rudra uvāca |}%

  \maintext{kiṃ na jānāmi tat tīrthaṃ sulabhaṃ durlabhaṃ ca yat |}%

  \maintext{sulabhaṃ gurusevīnāṃ durlabhaṃ tad vivarjayet }||\thinspace10:11\thinspace||%
\translation{Rudra spoke: How could I not know [the difference between] that pilgrimage place that is easy to reach and that which is difficult to reach? It is easy to reach for those who serve their guru. One can abandon the one which is difficult to reach. \blankfootnote{10.11 Note \textit{sevīnāṃ} for \textit{sevināṃ} in \textit{pāda} c metri causa.
  
 My translation here is slightly tentative and is fashioned to make sense
  in light of what is coming: the praise of internalised pilgrimage places,
  contrasting them with real, external pilgrimage places.
 }}

  \subsubchptr{kurukṣetram}%

  \trsubsubchptr{Kurukṣetra}%

  \maintext{kuruḥ puruṣa vijñeyaḥ śarīraṃ kṣetra ucyate |}%

  \maintext{śarīrasthaṃ kurukṣetraṃ sarvatīrthaphalapradam }||\thinspace10:12\thinspace||%
\translation{\textit{Kuru}- [in Kurukṣetra] is to be known as the soul {\rm (}\textit{puruṣa}{\rm )}, -\textit{kṣetra} as the body. Kurukṣetra that is in the body yields the fruits of [visiting] all pilgrimage places. \blankfootnote{10.12 In \textit{pāda} b, one could apply \msNa's reading that has the standard neuter nominative form \textit{kṣetram}
  as opposed to the form transmitted in all other witnesses {\rm (}\textit{kṣetra}{\rm )} but
  the latter might be original, influenced by the stem form \textit{puruṣa} in \textit{pāda} a.
 }}

  \maintext{sarvayajñaphalāvāptiḥ sarvadānaphalāni ca |}%

  \maintext{sarvavratatapaś cīrṇaṃ tatphalaṃ sakalaṃ bhavet }||\thinspace10:13\thinspace||%
\translation{[And there will be] the obtaining of the fruits of all sacrifices, the fruits of all [possible] donations, and all the fruits of all religious observances and penance performed. }

  \maintext{evam eva phalaṃ teṣāṃ tīrthapañcadaśeṣu ca |}%

  \maintext{anaghānaṃ mahāpuṇyaṃ mahātīrthaṃ mahāsukham }||\thinspace10:14\thinspace||%
\translation{This is how the fruits [are said to be also] in the case of those fifteen pilgrimage places [from Kurukṣetra to Vāgīśa]. [Kurukṣetra,] the great and faultless pilgrimage place is extremely auspicious and pleasant. \blankfootnote{10.14 \textit{anaghānaṃ} in \textit{pāda} c is problematic. It may simply stand for \textit{anaghaṃ} {\rm (}`faultless'{\rm )}.
  That is how I translate it. Originally it may have involved a stem form adjective:
  \textit{anaghaitan} {\rm (}\textit{anagha + etad}{\rm )}.
 }}

  \maintext{devy uvāca |}%

  \maintext{atīva romaharṣo me jāto 'sti tridaśeśvara |}%

  \maintext{sulabhaṃ sukaraṃ sūkṣmaṃ śrutvā tuṣṭiś ca me gatā }||\thinspace10:15\thinspace||%
\translation{Devī spoke: I am extremely thrilled, O Tridaśeśvara. Hearing about that which is easy to obtain, easy to perform, and is subtle, my contentment has left me [that is, I want to hear more]. \blankfootnote{10.15 We could read \textit{śrutvātuṣṭiś} {\rm (}i.e. \textit{śrutvā-atuṣṭiś}{\rm )} in \textit{pāda} d
  {\rm (}`hearing this, my discontent is gone'{\rm )}, but interlocutors in this
  text usually want to hear more when they are still unsatisfied, and
  hungry for more teaching. See, e.g., \mycite{KissVolume2021}.
  {\rm (}Or shall we read \textit{śrutvātuṣṭiś ca me 'gatā}, `hearing this my
  discontent has not yet disappeared'?{\rm )}
 }}

  \maintext{caturdaśa paro bhūyaḥ kathayasva manoharam |}%

  \maintext{prayāgādi pṛthaktvena tattvatas tu sureśvara }||\thinspace10:16\thinspace||%
\translation{Teach me further about the remaining fourteen pleasant [pilgrimage places], Prayāga and the others, one by one, as they really are, O Sureśvara. \blankfootnote{10.16 Note again the use of the singular next to numbers {\rm (}\textit{caturdaśa \dots\ manoharam prayāgādi}{\rm )},
  a frequent phenomenon in this text.
 }}

  \subsubchptr{prayāgo vārāṇasī ca}%

  \trsubsubchptr{Prayāga and Vārāṇasī}%

  \maintext{rudra uvāca |}%

  \maintext{suṣumnā bhagavatī gaṅgā iḍā ca yamunā nadī |}%

  \maintext{etāḥ srotovahā nadyaḥ prayāgaḥ sa vidhīyate }||\thinspace10:17\thinspace||%
\translation{The Suṣumnā[-tube] is the Honourable Gaṅgā, Iḍā[-tube] is the river Yamunā. [At the confluence of] these surging rivers is [the pilgrimage place] called Prayāga. \blankfootnote{10.17 There seems to be only two yogic tubes mentioned 
  here {\rm (}and in 10.20--21{\rm )}: Suṣumnā and Iḍā, instead 
  of the more usual triad of Iḍā, Piṅgalā, 
  and Suṣumnā. This is strikingly similar to
  what we see in the archaic yoga of the \NISVNAYA, 
  see 
  \mycitep{NisvasaGoodall}{33--34}.
  According to 
  \mycitep{BaroisDhP}{23 and 46} 
  the case is similar in the \DHARMP.
  This is slightly doubtful because a third tube, 
  called Turyā, is mentioned immediately after
  Iḍā and Suṣumnā in \DHARMP\ 4.57: 
  
 
  \textit{iḍā vāmā suṣumnā ca dve nāḍī nāsikāśrite}\thinspace | 
 
  \textit{bhruvor madhye parā nāḍī tajjñais turyeti kīrttitā}\thinspace ||
  
 
  It is also possible that the third tube is there, 
  as Prayāga, in our obscure \VSS\ 10.17cd,
  which may want to say that at the confluence of 
  the Gaṅgā/Suṣumnā and the Yamunā/Iḍā,
  there is the internalised pilgrimage place, 
  or tube, called Prayāga.
  Compare \MBH\ Suppl. 6.3A.41--44:
  
 
  \textit{iḍā bhagavatī gaṅgā piṅgalā yamunā nadī}\thinspace |
 
  \textit{tayor madhye tṛtīyā tu tat prayāgam anusmaret}\thinspace ||
 
  \textit{iḍā vai vaiṣṇavī nāḍī brahmanāḍī tu piṅgalā}\thinspace |
 
  \textit{suṣumṇā caiśvarī nāḍī tridhā prāṇavahā smṛtā}\thinspace ||
  
 
  Note that Yamunā has not been mentioned as 
  a \textit{tīrtha} in \VSS\ 10.7--8 above.
  See also \HYP\ 3.110:
  
 
  \textit{iḍā bhagavatī gaṅgā piṅgalā yamunā nadī}\thinspace | 
 
  \textit{iḍāpiṅgalayor madhye bālaraṇḍā ca kuṇḍalī}\thinspace ||
  
  Note also \Ed's attempt to make \textit{pāda} a metrical.
 }}

  \maintext{dakṣiṇā vāruṇī nāsā vāmanāsā asi smṛtā |}%

  \maintext{vāruṇā-asimadhyena tena vārāṇasī smṛtā }||\thinspace10:18\thinspace||%
\translation{The right nostril is [the river] Vāruṇī, the left nostril is known as [the river] Asi. Because [it is] at the confluence of Vāruṇā and Asi, [the city/internalised pilgrimage place there] is known as Vārāṇasī. \blankfootnote{10.18 This verse most probably describes the spot between the eyebrows as an
  internalised pilgrimage place.
 }}

  \subsubchptr{gaṅgā}%

  \trsubsubchptr{Gaṅgā}%

  \maintext{ākāśagaṅgā vikhyātā tasyāḥ sravati cāmṛtam |}%

  \maintext{ahorātram avicchinnaṃ gaṅgā sā tena ucyate }||\thinspace10:19\thinspace||%
\translation{[There is] the famous ethereal Gaṅgā. The nectar of immortality issues from her day and night uninterruptedly. That is why [this internalised pilgrimage place] is called Gaṅgā. \blankfootnote{10.19 This verse may describe a bodily location such as the soft palate as an
  internalised pilgrimage place.
 The word \textit{gaṅga} is interpreted here as an intensive form from the root\verbalroot{\textit{gam}},
  related to the better-attested intensive stems \textit{jaṅgam} and \textit{ganīgam} {\rm (}see the latter two, e.g., in 
  \mycitep{WhitneyGrammar}{§1003}{\rm )}.
 }}

  \subsubchptr{somatīrtham}%

  \trsubsubchptr{Somatīrtha}%

  \maintext{somatīrtham iḍā nāḍī kiṅkiṇīravacihnitā |}%

  \maintext{taṃ tu śrutvā na saṃdehaḥ sarvapāpakṣayo bhavet }||\thinspace10:20\thinspace||%
\translation{Somatīrtha is the tube Iḍā. It is characterised by the ringing of small bells. Upon hearing that [ringing], all of one's sins will be destroyed. \blankfootnote{10.20 Note that Iḍā has already been identified as the Yamunā in 10.17b.
 }}

  \subsubchptr{sūryatīrtham}%

  \trsubsubchptr{Sūryatīrtha}%

  \maintext{sūryatīrthaṃ suṣumnā ca nīravāravasaṃyutā |}%

  \maintext{śrutimātrād vimucyeta pāparāśir mahān api }||\thinspace10:21\thinspace||%
\translation{Sūryatīrtha is the [tube] Suṣumnā, the one that emits a soundless thunder. One is liberated by merely hearing it, even if one has mountains of sin. \blankfootnote{10.21 Suṣumnā has already been identified as the Gaṅgā in 10.17a.
 }}

  \subsubchptr{agnitīrtham}%

  \trsubsubchptr{Agnitīrtha}%

  \maintext{agnitīrthārjunā nāḍī brahmaghoṣamanoramā |}%

  \maintext{tat tad akṣaram ākarṇya amṛtatvāya kalpate }||\thinspace10:22\thinspace||%
\translation{Agnitīrtha is the Arjuna tube. It is charming because of the hum of Veda recitation. Upon hearing this or that syllable, one's share will be immortality. \blankfootnote{10.22 \textit{agnitīrtha} is most probably in stem form in \textit{pāda} a.
  
  
  I am not aware of any yogic teachings that involve a \textit{nāḍī} called \textit{arjunā}.
  Maybe \textit{aruṇā} or \textit{varuṇā} was meant? A \textit{vāruṇī nāḍī} does occur in 
  some texts, such as the \YogasikhaU\ {\rm (}5.26, \mycitep{YogaUpanisads}{444}{\rm )}, 
  the \Hatharatnavali\ {\rm (}4.34--35, \mycitep{RootsOfYoga}{5.1.10}{\rm )}, 
  and the \Sivasamhita\ {\rm (}2.15, ibid. 5.2.4{\rm )}. On the other hand, 
  `red' {\rm (}\textit{aruṇa}{\rm )} would be an appropriate label for Agnitīrtha.
 }}

  \subsubchptr{puṣkaram}%

  \trsubsubchptr{Puṣkara}%

  \maintext{puṣkaraṃ hṛdi madhyastham aṣṭapattraṃ sakarṇikam |}%

  \maintext{cintayet sūkṣma tanmadhye janmamṛtyuvināśanam }||\thinspace10:23\thinspace||%
\translation{Puṣkara is a lotus with eight petals and a pericarp in the centre of the heart. One should visualize the Subtle One in its centre. It will destroy birth and death. \blankfootnote{10.23 \textit{hṛdi} was probably meant to be nominative, as in 10.27, 
  here potentially compounded with \textit{madhyastham}.
 On \textit{sūkṣma} {\rm (}here in stem form metri causa{\rm )}, see \CHECK.
 }}

  \subsubchptr{mānasam}%

  \trsubsubchptr{Mānasa}%

  \maintext{mānasasaramadhyasthaṃ sa haṃsaḥ kamalopari |}%

  \maintext{salīlo līlayācārī parataḥ parapāragaḥ }||\thinspace10:24\thinspace||%
\translation{That goose on a lotus in the middle of the Mānasa lake is playful, acting gracefully, rising far beyond the other shore. \blankfootnote{10.24 Understand \textit{mānasasara}° in \textit{pāda} a as \textit{mānasasaro}° {\rm (}metri causa{\rm )}.
  To make sense of this verse, especially the masculine nominatives in
  \textit{pāda}s cd, I have conjectured \textit{sa haṃsaḥ} for what seems to 
  a compound: \textit{sahaṃsakamalopari}. I suspect \textit{pāda} a to qualify, clumsily,
  \textit{kamala} in \textit{pāda} b. Other possibilities include \textit{sahaṃsa}° meaning
  `with the syllables HAṂ and SA on it.' 
 
  The association of Lake Mānasa on Mount Kailāsa with lotuses, and especialy with geese or swans, is
  well-known. See, e.g., \MBH\ 6.114.90ff: Gaṅgā sends the great sages, who inhabit
  Lake Mānasa in the form of geese, to visit the dying Bhīṣma.
  Although the interpretation of this verse, which obviously refers to an internalised
  form of this pilgrimage place, is still problematic, the goose/swan
  most probably signifies to the soul.
 }}

  \subsubchptr{naimiṣam}%

  \trsubsubchptr{Naimiṣa}%

  \maintext{naimiṣaṃ śṛṇu deveśi nimiṣā pratyayo bhavet |}%

  \maintext{samyag chāyāṃ nirīkṣeta ātmāno vā parasya vā }||\thinspace10:25\thinspace||%
\translation{Listen to Naimiṣa, O Deveśī. It yields assurance in a moment. One can observe the shadow of one's own and others' soul properly. \blankfootnote{10.25 This obscure verse {\rm (}coupled with the next one{\rm )} might have something to do with a type of
  meditation, \textit{chāyādhyāna}, mentioned in \NISVUTTARA\ 5.6:
  
 
  \textit{tattvadhyānaṃ prathamakaṃ chāyādhyānaṃ dvitīyakam}\thinspace |
 
  \textit{ghoṣadhyānan tṛtīyan tu lakṣadhyānañ caturthakam}\thinspace ||
  
 
  Later on in the same text {\rm (}5.12 and 16{\rm )}, this meditation on `the shadow of the 
  soul/\textit{puruṣa}' is mentioned again.
  \NISVUTTARA\ 5.16 states that 
  `[f]ocussing on[?] one's awareness on [one's] ``shadow'' {\rm (}\textit{chāyācittam}{\rm )},
  one will see the soul {\rm (}\textit{pumān} = \textit{pumāṃsam}?{\rm )} in the sky {\rm (}\textit{viyatstham}{\rm )}.
  Practising in this way, one attains success and becomes Śiva.' 
  {\rm (}Translation from \mycitep{NisvasaGoodall}{391}.{\rm )}
  The Sanskrit reads:
  
 
  \textit{chāyācittaṃ samālambya viyatsthaṃ paśyate pumān}\thinspace |
 
  \textit{evam abhyasyamānas tu siddhyate ca śivo bhavet}\thinspace ||
  
 
  But as the editors of the \NISV\ put it with reference to the
  four elements of meditation given there:
  `[v]ery little of this is clear and almost nothing is certain'
  {\rm (}\mycitep{NisvasaGoodall}{389}{\rm )}.
 }}

  \maintext{āyatam aṅgulīmātraṃ nimiṣākṣiḥ sa paśyati |}%

  \maintext{dṛṣṭvā pratyayam evaṃ hi naimiṣajñaḥ sa ucyate }||\thinspace10:26\thinspace||%
\translation{He will see [the soul's] length with his eyes shut as one finger-breadth. When one has seen the proof thus, one is called the knower of Naimiṣa. \blankfootnote{10.26 \textit{Pāda}s ab involve an emendation and a conjecture, without which it is
  difficult to understand this line.
 }}

  \subsubchptr{bindusaraḥ}%

  \trsubsubchptr{Bindusaras}%

  \maintext{tīrthaṃ bindusaraṃ nāma śṛṇu vakṣyāmi sundari |}%

  \maintext{dehamadhye hṛdi jñeyaṃ hṛdimadhye tu paṅkajam }||\thinspace10:27\thinspace||%
\translation{Listen, O Sundarī, I shall teach you the pilgrimage place called Bindusaras. The heart is to be known to be located in the centre of the body. In the centre of the heart, there is a lotus. \blankfootnote{10.27 Understand °\textit{saraṃ} in \textit{pāda} a as °\textit{saro} {\rm (}thematisation{\rm )}.
 Take \textit{hṛdi} as a nominative in \textit{pāda} c and possibly also in \textit{pāda} d {\rm (}and see 10.23a{\rm )}.
 }}

  \maintext{karṇikā padmamadhye tu binduḥ karṇikamadhyataḥ |}%

  \maintext{bindumadhye sthito nādaḥ sa nādaḥ kena bhidyate }||\thinspace10:28\thinspace||%
\translation{There is a pericarp in the centre of the lotus, and the subtle sonic matter {\rm (}\textit{bindu}{\rm )} in the centre of the pericarp. In the centre of the subtle sonic matter {\rm (}\textit{bindu}{\rm )}, there is the subtle sound {\rm (}\textit{nāda}{\rm )}. How is that subtle sound {\rm (}\textit{nāda}{\rm )} divided? \blankfootnote{10.28 For a general discussion on \textit{nāda} and \textit{bindu}, see, e.g., \TAKIII\ s.v. \textit{nāda}.
  Our text considers the internalised manifestation of the pilgrimage place Bindusaras
  to be \textit{bindu}, or subtle sonic matter.
 }}

  \maintext{ukāraṃ ca makāraṃ ca bhittvā nādo vinirgataḥ |}%

  \maintext{taṃ viditvā viśālākṣi so 'mṛtatvaṃ labheta ca }||\thinspace10:29\thinspace||%
\translation{The subtle sound {\rm (}\textit{nāda}{\rm )} departs divided by the sounds U and M. Realizing that [subtle sound], O Viśālākṣi, one can obtain immortality. \blankfootnote{10.29 \VSS\ 10.27--29ab seem to paraphrase \NISVK\ 5.55--57ab.
 }}

  \subsubchptr{setubandham}%

  \trsubsubchptr{Setubandha}%

  \maintext{vakṣye te setubandhaṃ duritamalaharaṃ nādatoyapravāhaṃ}%

 \nonanustubhindent \maintext{jihvākaṇṭhorakūlā svaragaṇapulināvartaghoṣā taraṅgā |}%

  \maintext{kumbhīrāghoṣamīnā daśagaṇamakarā bhīmanakrā visargā}%

 \nonanustubhindent \maintext{sānusvāre gabhīre madasukharasanaṃ setubandhaṃ vrajasva }||\thinspace10:30\thinspace||%
\translation{I shall teach you Setubandha, which sports a current whose water of subtle sound {\rm (}\textit{nāda}{\rm )} cleanses you of the dirt of your sins. [It is a river whose] banks are the tongue, the throat, and the chest, and its sandbanks are the group of vowels {\rm (}\textit{svara}{\rm )}. It is wavy with its whirlpools of voiced consonants {\rm (}\textit{ghoṣa}{\rm )}. Voiceless consonants {\rm (}\textit{aghoṣa}{\rm )} are its crocodiles and fish, the ten verbal classes {\rm (}\textit{gaṇa}{\rm )} are its sea-monsters, \textit{visarga}s are its terrifying alligators. It is in the deep-sounding \textit{anusvāra} {\rm (}\textit{sā-anusvāre}{\rm )}. Go to Setubandha, have a taste of the pleasure of intoxication. \blankfootnote{10.30 Note that °\textit{kaṇṭhora}° is a conjecture based on the context: this line
  speaks about sounds and the production of sounds. For this,
  \textit{uraḥ}/\textit{ura} {\rm (}`chest'{\rm )} seems better that \textit{ūru} {\rm (}`thigh'{\rm )}.
  It is not evident at first sight why \textit{pāda}s b and c stick to feminine endings. I take this
  as qualifying an implied \textit{nadī}, partly because the similarly structured 10.33 below
  explicitly mentions \textit{nadī}. Some of the compounds here are inverted or split:
  understand \textit{āvartaghoṣā taraṅgā} as \textit{ghoṣāvartataraṅgā}, 
  \textit{kumbhīrāghoṣamīnā} as \textit{aghoṣakumbhīramīnā}, and 
  \textit{bhīmanakrā visargā} as \textit{visargabhīmanakrā}.
 Nevertheless, the general idea seems to be clear: the internalised
  version of the pilgrimage place Setubandha, externally usually understood as 
  Rameśvara in the South, is now the sounds of recitation.
 }}

  \subsubchptr{suradrahaḥ}%

  \trsubsubchptr{Suradraha}%

  \maintext{saptadvīpāntamadhye śṛṇu śaśivadane sarvaduḥkhāntalābham}%

 \nonanustubhindent \maintext{īśānenābhijuṣṭaṃ hṛdi hrada vimalaṃ nādaśītāmbupūrṇam |}%

  \maintext{tatraikaṃ jātapadmaṃ prakṛtidalayutaṃ keśaraṃ śaktibhinnaṃ}%

 \nonanustubhindent \maintext{pañcavyomapraśastaṃ gatiparamapadaṃ prāptukāmena sevyam }||\thinspace10:31\thinspace||%
\translation{O Moon-faced goddess, listen to [the description of Suradraha], the way to the cessation of all sorrow, in the centre of the seven islands. It is frequented by Īśāna, a spotless lake in the heart full of the cool water of sound {\rm (}\textit{nāda}{\rm )}. There is a lotus arising there whose petals are Prakṛti and whose filaments are split between Śaktis, praised as the five gross elements {\rm (}\textit{vyoman}{\rm )}. It is to be honoured if one wishes to obtain the path to the supreme abode. \blankfootnote{10.31 The first syllable of \textit{hrada} in \textit{pāda} b does not make the previous syllable long {\rm (}`muta cum liquida'
  licence{\rm )}, otherwise the line would be unmetrical. Understand the same \textit{hrada} as a stem form metri causa
  standing for the accusative.
 \textit{keśaraṃ śaktibhinnaṃ} in \textit{pāda} c should probably be understood as a bahuvrīhi compound
  thus: \textit{śaktibhinnakeśaraṃ}.
 For \textit{vyoman} as `gross element,' see notes to \VSS\ 4.32 above, but note
  that the expression `fifty voids' {\rm (}\textit{pañcāśadvyoman}{\rm )} also comes up in 
  \VSS\ 20.7 and also in 10.33 below. It is not clear why this internalised pilgrimage place,
  or the filaments of the lotus mentioned, would be praised as the five elements.
 }}

  \subsubchptr{ghaṇṭikeśvaram}%

  \trsubsubchptr{Ghaṇṭikeśvara}%

  \maintext{{\rm †}nāḍyaikāsaṅgatāni{\rm †} nipatitam amṛtaṃ ghaṇṭikāpārakeṇa}%

 \nonanustubhindent \maintext{tṛpyante tena nityaṃ hṛdi kamalapuṭaṃ sthāṇubhūtāntarātmā |}%

  \maintext{yaṃ paśyantīśabhaktāḥ kalikaluṣaharaṃ vyāpinaṃ niṣprapañcaṃ}%

 \nonanustubhindent \maintext{deveśaṃ ghaṇṭikeśāmarabhavam abhavaṃ tīrtham ākāśabindum }||\thinspace10:32\thinspace||%
\translation{The tubes join[?]. The nectar of immortality {\rm (}\textit{amṛta}{\rm )} has descended by the Saviour Ghaṇṭikā. Those whose inner selves have become Sthāṇu [i.e. Śiva] are continuously delighted in Him, as he is embraced by the lotus in their hearts. [He is the one] whom Īśa's devotees can behold, who drives off the impurity of the Kali age, who is all-pervading {\rm (}\textit{vyāpin}{\rm )} and non-manifest {\rm (}\textit{niṣprapañca}{\rm )}, the lord of gods, Ghaṇṭikeśa of undying existence. The \ae rial \textit{bindu} is a non-mundane {\rm (}\textit{abhava}{\rm )} pilgrimage place. \blankfootnote{10.32 The interpretation of this verse is not without problems. The cruxed expression
  in \textit{pāda} a is difficult to repair; it may involve \textit{nāḍī} or \textit{nāḍyā}, \textit{ekā}, and 
  \textit{saṃgata}. These suggest that it may hint at a point of confluence where the bodily tubes {\rm (}\textit{nāḍī}{\rm )}
  join. Possibly understand \textit{nāḍya ekasaṃgatāḥ}.
 In \textit{pāda} b, \textit{sthāṇu} is my conjecture for \textit{sthānu}, and 
  I understand °\textit{ātmā} as standing for the plural nominative.
 I take \textit{ghaṇṭikeśa} in \textit{pāda} d as a stem form noun in sandhi with \textit{amara},
  notwithstanding the {\rm (}unmetrical{\rm )} reading \textit{ghaṇṭikeśamara}° in \msCa\msCb\msNb\msNc.
 
  The external pilgrimage place related to Ghaṇṭikeśvara the redactors of the 
  \VSS\ may have had in mind here may or may not be 
  `Virajā, modern Jajpur in the Cuttack District of Orissa' presided over by
  Ghaṇṭīśa, Mahāghaṇṭeśvara or 
  Mahāghaṇṭa Bhairava {\rm (}\mycitep{SandersonSaivaAge}{113, n. 241}{\rm )}.
  See Introduction pp.~\pageref{provenance} ff.
 
  As for the yogic interpretation of this verse, it seems plausible that \textit{ghaṇṭikā} is
  taken here as the uvula, from which \textit{amṛta} is said to be dripping down the throat.
  See \TAKII\ s.v. \textit{ghaṇṭikā} and \mycite{MallinsonKhecari}.
 }}

  \subsubchptr{vāgīśvaratīrtham}%

  \trsubsubchptr{Vāgīśvaratīrtha}%

  \maintext{mīmāṃsāratnakūlā kramapadapulinā śaivaśāstrārthatoyā}%

 \nonanustubhindent \maintext{mīnaughā pañcarātraṃ śrutikuṭilagatiḥ smārtavegā taraṅgā |}%

  \maintext{yogāvartātiśobhā upaniṣadivahā bhāratāvartaphenā}%

 \nonanustubhindent \maintext{pañcāśadvyomarūpī rasabhavananadī tīrtha vāgīśvarīyam }||\thinspace10:33\thinspace||%
\translation{The banks [of Vāgīśvaratīrtha] are the gems of Mīmāṃsā, its sandbanks the [Vedic] \textit{kramapada}s, its water the meaning of the Śaiva manuals. Its flock of fish is the Pañcarātra [tradition], its winding path is the Śruti [tradition], its rapid waves the Smārta [tradition]. It is beautiful with its whirlpools of yoga. Its currents are the Upaniṣads. The foam made by its whirlpools is the \textit{Mahābhārata}. This river, whose form is the fifty voids {\rm (}\textit{vyoman}{\rm )}, is the abode of the elixir. [This is the description of] the pilgrimage place Vāgīśvara. \blankfootnote{10.33 By \textit{kramapada}, most probably a particular method of reciting Vedic texts
  {\rm (}better known as \textit{padakrama}{\rm )} is meant.
 Note the split compounds in \textit{pāda} b. Understand \textit{mīnaughā pañcarātraṃ} as
  \textit{pañcarātramīnaughā}, and \textit{smārtavegā taraṅgā} as \textit{smārtavegataraṅgā}.
 Note the form \textit{upaniṣadi} for a stem form of \textit{upaniṣad} in \textit{upaniṣadi-vahā} in
  \textit{pāda} c. This phenomenon is similar to what we see in 10.23 and 27 above with \textit{hṛdi}.
  The lack of sandhi between °\textit{śobhā} and \textit{upaniṣadi}° is also notable.
 \textit{tīrtha} in \textit{pāda} d is a stem form noun metri causa. The exact meaning of
  \textit{pañcāśadvyoma}° is not clear to me. Could it be the fifty sounds of Sanskrit?
  All in all, Vāgīśvaratīrtha here represents the religious traditions and scriptures.
 }}

  \maintext{yas taṃ vetti sa vetti vedanikhilaṃ saṃsāraduḥkhacchidaṃ}%

 \nonanustubhindent \maintext{janmavyādhiviyogatāpamaraṇaṃ kleśārṇavaṃ duḥsaham |}%

  \maintext{garbhāvāsam atīva sahyaviṣayaṃ dustīryaduḥkhālayaṃ}%

 \nonanustubhindent \maintext{prāptaṃ tena na saṃśayaḥ śivapadaṃ duṣprāpya devair api }||\thinspace10:34\thinspace||%
\translation{One will know all the Vedas by knowing Him who puts an end to the suffering of transmigration, to birth, disease, separation, suffering, death, the floods of unbearable pain, to dwelling in the womb, to extremely insufferable sensations, and to places of suffering that are difficult to escape from. Such a person will, without doubt, reach Śiva's world that is difficult to enter even for the gods. \blankfootnote{10.34 I take \textit{pāda}s b and c as if °\textit{chidaṃ} in \textit{pāda} a were implied for
  each element there,
 and \textit{atīva sahya}° as standing for \textit{atīvāsahya}° metri causa.
 Understand \textit{duṣprāpya} as a stem form adjective {\rm (}for \textit{duṣprāpyaṃ}{\rm )} metri causa.
 }}

\centerline{\maintext{\dbldanda\thinspace iti vṛṣasārasaṃgrahe kāyatīrthopavarṇano nāmādhyāyo daśamaḥ\thinspace\dbldanda}}
\translation{Here ends the tenth chapter in the \textit{Vṛṣasārasaṃgraha} called the Description of the bodily pilgrimage places.}
