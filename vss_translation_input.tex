
  \chptr{aṣṭādaśamo 'dhyāyaḥ}
\addcontentsline{toc}{section}{Chapter 18}
\fancyhead[CO]{{\footnotesize\textit{Translation of chapter 18}}}%

  \trchptr{Chapter Eighteen}%

  \subchptr{svargopāgatānāṃ cihnāni}%

  \trsubchptr{Marks of those who return from heaven}%

  \maintext{devy uvāca |}%

  \maintext{bhuktvā tu bhogān suciraṃ yatheṣṭaṃ}%

 \nonanustubhindent \maintext{puṇyakṣayān martyam upāgatānām |}%

  \maintext{cihnāni teṣāṃ kathayasva me 'dya}%

 \nonanustubhindent \maintext{yathākramaṃ karmaphalaṃ viśeṣāt }||\thinspace18:1\thinspace||%
\translation{Devī spoke: Please tell me now about the characteristic marks of those who, after having experienced enjoyable things as they please for a long time, their merits thus having worn away, return to the mortal world, and especially about the fruits of their deeds, one by one. }

  \subchptr{dānāṣṭakam}%

  \trsubchptr{Eight kinds of donation}%

  \maintext{maheśvara uvāca |}%

  \maintext{sadānnadātā kṛpaṇārtidīnāṃ}%

 \nonanustubhindent \maintext{sa varṣakoṭyāyutam īśaloke |}%

  \maintext{bhuktvā ca bhogān samam apsarobhiḥ}%

 \nonanustubhindent \maintext{prakṣīṇapuṇyaḥ punar eti martyam }||\thinspace18:2\thinspace||%
\translation{Maheśvara spoke: He who regularly gives to the poor and to the ones afflicted by pain will experience enjoyments in Īśaloka together with Apsarases for millions of years, before he returns to the world of mortals, his merits having worn away. }

  \maintext{jāyanti divyeṣu kuleṣu puṃsaḥ}%

 \nonanustubhindent \maintext{sastrīsamṛddhe bahubhṛtyapūrṇe |}%

  \maintext{gauraśvaratnādidhanākuleṣu}%

 \nonanustubhindent \maintext{rūpojjvalaḥ kāntisamāyutaś ca  }||\thinspace18:3\thinspace||%
\translation{[These] men will be [re-]born in divine families, [later] having a wife and wealth and many servants, into families that are stuffed with wealth that consists of cows, horses, jewels etc., he himself possessing shining beauty and loveliness. \blankfootnote{18.3 Note the change from plural {\rm (}\textit{kuleṣu}{\rm )} to singular {\rm (}°\textit{samṛddhe}, °\textit{pūrṇe}{\rm )}
  in \textit{pāda}s a and b, the slightly irregular plural nominative \textit{puṃsaḥ} in \textit{pāda} a,
  and that \textit{sastrī} might have been meant as a separate word, in the sense of
  \textit{sastrīkaḥ} {\rm (}`married'{\rm )}.
 I take \textit{gaur aśva}° in \textit{pāda} c as if it were part of the compound:
  \textit{go'śva}°. See \NARADAP\ 1.71.77ab for a compound similar to the one here:
  \textit{gajāśvaratharatnaiś ca grāmakṣetradhanādibhiḥ}
 }}

  \maintext{vastraṃ susatkṛtya dvijasya dānāt}%

 \nonanustubhindent \maintext{svargeṣu modanti sa varṣakoṭyaḥ |}%

  \maintext{punaś ca te martyam upāgatāś ca}%

 \nonanustubhindent \maintext{cihnaṃ mahacchrīpadam āpnuvanti }||\thinspace18:4\thinspace||%
\translation{[If one] donates clothes to a Brahmin with utmost respect, he will have fun in the heavens for millions of years. When they return to the world of mortals, their characteristics mark is that they rise to an extremely glorious rank. \blankfootnote{18.4 Note that \textit{pāda} a can be considered metrical only if 
  \textit{dvi} in \textit{dvijasya} does not make the previous syllable heavy 
  {\rm (}muta cum liquida{\rm )}.
 Note the plural nominative °\textit{koṭyaḥ} in \textit{pāda} b for
  a more standard accusative °\textit{koṭīḥ} {\rm (}from \textit{koṭi} or \textit{koṭī}{\rm )}.
 }}

  \maintext{kūpaprapāpuṣkariṇīpradātā}%

 \nonanustubhindent \maintext{sa lokam āpnoti jaleśvarasya |}%

  \maintext{tataḥ sa tasmāc cyutim āpya lokāt}%

 \nonanustubhindent \maintext{sukhī sutṛpteṣu kuleṣu jāyet }||\thinspace18:5\thinspace||%
\translation{He who donates wells, fountains, or lotus-ponds will reach the world of Jaleśvara [i.e.\ Varuṇa]. Then falling from that world, he will be [re-]born into a very comfortable [well-to-do?] family, and will be happy. \blankfootnote{18.5 The phrase \textit{sutṛpteṣu kuleṣu} {\rm (}lit.\ `into satified families'{\rm )} is 
  slightly odd.
 }}

  \maintext{ratnipramāṇād api hemadānāt}%

 \nonanustubhindent \maintext{surendralokaṃ samavāpnuvanti |}%

  \maintext{tasmāc cyuto martyam upāgatānāṃ}%

 \nonanustubhindent \maintext{cihnaṃ samṛddhir dhanadhānyalakṣmyāḥ }||\thinspace18:6\thinspace||%
\translation{By donating as little gold as a cubit, people can reach the world of Surendra [i.e.\ Indra]. The characteristic mark of those who fall from there to the world of mortals is prosperity, riches, wealth, and good fortune. \blankfootnote{18.6 I have chosen \textit{lakṣmyāḥ} in \textit{pāda} d against \textit{lakṣyāḥ} as
  a lectio difficilior. It is supposed to stand for a 
  plural nominative.
 }}

  \maintext{adūṣya bhūmīvaravipradānāt}%

 \nonanustubhindent \maintext{sa lokam āpnoti sureśvarasya |}%

  \maintext{bhuktvā tu bhogān cyuta martyaloke}%

 \nonanustubhindent \maintext{cihnaṃ labhed vai viṣayādhipatvam }||\thinspace18:7\thinspace||%
\translation{By donating an excellent piece of land to a Brahmin without corruption[?], he will reach the world of Sureśvara [Śiva/Brahmā?]. After experiencing enjoyments, he falls back to the world of mortals, And will possess the characteristic mark of sovereignty. }

  \maintext{dvijasya satkṛtya tilapradātā}%

 \nonanustubhindent \maintext{sa lokam āpnoti ca keśavasya |}%

  \maintext{bhraṣṭas tato martyam upāgatas tu}%

 \nonanustubhindent \maintext{cihnaṃ labhed akṣayam arthalābham }||\thinspace18:8\thinspace||%
\translation{He who donates sesame seeds to a Brahmin respectfully will reach the world of Keśava [i.e.\ Viṣṇu]. Then, having fallen and returned to the world of mortals, he will possess the characteristic mark of undiminishing acquisition of wealth. }

  \maintext{gavāṃ surūpāṃ vidhivad dvijānāṃ}%

 \nonanustubhindent \maintext{dattvā ca golokam avāpnuvanti |}%

  \maintext{kalpāvasāne samupetya martye}%

 \nonanustubhindent \maintext{cihnaṃ gavāḍhyaṃ śatagoyutaṃ ca }||\thinspace18:9\thinspace||%
\translation{By donating beautiful cows to Brahmins, people reach Goloka. At the end of the \ae on, they return to the world of mortals. Their characteristic mark would be an abundance of cows with a hundred cows. \blankfootnote{18.9 It seems that \textit{gavāṃ} is meant to be a singular accusative 
  of \textit{go}. Or is it a different animal? See 18.9d.
 }}

  \maintext{svargaṃ gatānāṃ puruṣasya cihnaṃ}%

 \nonanustubhindent \maintext{dhanāḍhyatā śrī sukhabhogalābham |}%

  \maintext{āyuryaśorūpakalatraputraṃ}%

 \nonanustubhindent \maintext{sampad vibhūtikulakīrtim artham }||\thinspace18:10\thinspace||%
\translation{The characteristic marks of those who have been in heaven are: an abundance of wealth, grace, the attainment of happiness and enjoyment, [a long] life, glory, beauty, family, sons, success, power, a glorious family, and riches. \blankfootnote{18.10 Note the discrepancy in grammatical number in \textit{pāda} a.
 Note the seemingly accusative forms °\textit{lābham} and °\textit{kīrtim} {\rm (}for \textit{lābhaḥ}
  and \textit{kīrtir}{\rm )}. The last syllable of \textit{vibhūti} is treated as long.
 }}

  \subchptr{nirayāgatānāṃ cihnāni}%

  \trsubchptr{Marks of those who return from hell}%

  \maintext{dānāṣṭakaṃ cottama kīrtitaṃ te}%

 \nonanustubhindent \maintext{cihnaṃ ca lokaṃ ca samāsato me |}%

  \maintext{śṛṇotu devī nirayāgatānāṃ}%

 \nonanustubhindent \maintext{cihnaṃ ca karmaṃ ca vipākatāṃ ca }||\thinspace18:11\thinspace||%
\translation{I have taught you the eight supreme kinds of donation, the characteristic marks, and the [corresponding] worlds in brief. Listen, O Goddess, to the characteristic marks of those who have returned from hell, and to their actions and the fruition thereof. \blankfootnote{18.11 Note the stem form adjective \textit{uttama}, probably metri causa, in \textit{pāda} a.
 The slightly odd phrase \textit{śṛṇotu devī}, instead of a vocative with \textit{śṛṇu},
  is metri causa.
 Note the accusative form \textit{karmaṃ}, probably metri causa, in \textit{pāda} d.
 }}

  \maintext{hatvā ca vipraṃ manasā ca vācā}%

 \nonanustubhindent \maintext{sa yāti pāraṃ nirayasya ghoram |}%

  \maintext{aśītikalpaṃ niraye krameṇa}%

 \nonanustubhindent \maintext{bhuktvā punas tirya śatāyutānām }||\thinspace18:12\thinspace||%
\translation{If one hurts a Brahmin mentally of verbally, one goes to the boundaries of terrible hell. Gradually experiencing [his karmas] for eighty years in hell, he will live as an animal for millions [of years/lives]. \blankfootnote{18.12 Note the stem form \textit{tirya} in \textit{pāda} d {\rm (}metri causa{\rm )},
  and that the phrase \textit{śatāyutānām} is ambiguous. Perhaps
  \textit{śatāyutābdam} {\rm (}for \textit{śatāyutāny abdāni}{\rm )} or \textit{śatāyutāni janmāni}
  was meant.
 }}

  \maintext{jāyanti te mānuṣa hīnavidyāḥ}%

 \nonanustubhindent \maintext{pratyantavāsāḥ kulavittahīnāḥ |}%

  \maintext{nityaṃ ca tasyākṣayarogapīḍā}%

 \nonanustubhindent \maintext{idaṃ tu cihnaṃ dvijajīvahartuḥ }||\thinspace18:13\thinspace||%
\translation{Those men will be [re-]born without any knowledge, will live on the fringes of town, and will lack family and wealth. They will always be tormented by incurable diseases. These are the characteristic marks of one who endangers the life of a Brahmin. \blankfootnote{18.13 In \textit{pāda} a, I take \textit{mānuṣa} as a stem form noun {\rm (}metri causa{\rm )}.
 }}

  \maintext{pītvā ca madyaṃ dvija kāmato vā}%

 \nonanustubhindent \maintext{āghrāti gandhaṃ svamanīṣikeṇa |}%

  \maintext{sa yāti ghoraṃ narakam asahyaṃ}%

 \nonanustubhindent \maintext{yāvac ca kalpaṃ daśa atra bhuktvā }||\thinspace18:14\thinspace||%
\translation{If a Brahmin drinks alcohol definitely intentionally, smells [its] odour on his own accord, he will go to the terrible and unbearable hell for ten \ae ons to experience [his karmas] there. \blankfootnote{18.14 I take \textit{dvija} in \textit{pāda} a as a stem form noun {\rm (}for \textit{dvijaḥ}{\rm )}.
  If standard sandhi is expected between \textit{pāda}s a and be, then \textit{vā} in \textit{pāda} a
  is to be understood to stand for \textit{vai} {\rm (}`definitely'{\rm )}.
 Strictly speaking, \textit{pāda} c is unmetrical, the last
  syllable of \textit{narakam} ending in a light syllable.
  Word-ending syllables are often treated as heavy in this text.
 In \textit{pāda} d \textit{atra} probably stands for \textit{tatra {\rm (}narake{\rm )}}. 
  It is not clear why \textit{atra} seemed better to the redactors.
 }}

  \maintext{tiryaṃ ca sarvam anubhūya duḥkhaṃ}%

 \nonanustubhindent \maintext{sa kaṣṭakaṣṭena manuṣyajanma |}%

  \maintext{caṇḍālaśaunaśvapacatvam eti}%

 \nonanustubhindent \maintext{śyāmaṃ ca tālu bhavatīha cihnam }||\thinspace18:15\thinspace||%
\translation{Experiencing all the pain of animal existence, he will, with great difficulty, [reach] a human birth. He will go through [states of being] a Caṇḍāla, a butcher, and a dog-cooker. In this case, the characteristic mark is that his palate becomes black. \blankfootnote{18.15 The syntax of \textit{pāda} a is obscure. Either understand \textit{tiryaṃ} as \textit{tiryaś} {\rm (}`being an animal,'
  `in an animal form'{\rm )} or \textit{tiryaṃ} as qualifying \textit{duḥkhaṃ} {\rm (}`the pain of animal existence'{\rm )}.
  The last syllable of \textit{sarvam} in \textit{pāda} a is treated as long.
 The two syllables of \textit{tālu} scan as long-long.
 }}

  \maintext{nindanti ye veda {\rm †}sambhūya{\rm †} jihvā}%

 \nonanustubhindent \maintext{yaḥ kūṭasākṣī sa ca khalv alāndhau |}%

  \maintext{suhṛdvadhā mṛtyuśataṃ hi garbhe}%

 \nonanustubhindent \maintext{garhāśanocchiṣṭabhujo bhavanti }||\thinspace18:16\thinspace||%
\translation{Those who despise the Vedas will [be reborn] with their tongues ... He who gives false testimony will [be reborn] blind[?]. [In case of] the murder of a friend, [one will experience] a hundred deaths in the womb. Those who eat forbidden food will eat [only] leftovers [in their next lives]. \blankfootnote{18.16 I take \textit{veda} as a stem form noun in \textit{pāda} a. 
  I suspect that \textit{pāda} a may have contained a reference to \textit{upajihvā}, a disease of the tongue.
 Understans \textit{garhāśanocchiṣṭabhujo} in \textit{pāda} d as \textit{garhitāśanā 
  ucchiṣṭabhujo} with double sandhi.
 }}

  \maintext{stainyaṃ tu yaḥ kurvati pāpasattvaṃ}%

 \nonanustubhindent \maintext{te pāpadoṣān narakaṃ vrajanti |}%

  \maintext{manvantarādīny anubhūya duḥkhaṃ}%

 \nonanustubhindent \maintext{punaś ca tiryaṃ śataśo 'nubhūyāt }||\thinspace18:17\thinspace||%
\translation{Those wicked people who steal will, because of this sinful crime, go to hell. Suffering pain for at least[?] a Manu-era, one will again and again, for a hundred times, experience animal existence. \blankfootnote{18.17 Note the discrepancy between \textit{yaḥ kurvati} and \textit{te vrajanti} in \textit{pāda}s a and b, and 
  the corresponding attempt in \msCa\ to correct \textit{yaḥ} to \textit{ye}. One could also emend °\textit{sattvaṃ} to °\textit{sattvaḥ}.
 }}

  \maintext{mānuṣyajanmeṣu ca duḥkhabhāgī}%

 \nonanustubhindent \maintext{stenatvam āyāti punaś ca mūḍhaḥ |}%

  \maintext{suvarṇacorī kunakhatva cihnam}%

 \nonanustubhindent \maintext{viśīrṇagātro rajatāpahārī }||\thinspace18:18\thinspace||%
\translation{When born as a human, he will suffer. The fool will become a thief again. If one steals gold, the characteristic mark will be that one will have ugly nails. One who steals silver will have broken limbs. }

  \maintext{tāmrāpahārī sphuṭitāgrapāṇir}%

 \nonanustubhindent \maintext{lohāpahārī bhujacheda cihnam |}%

  \maintext{kāṃsāpahārī karabhagna cihnaṃ}%

 \nonanustubhindent \maintext{hṛtvā ca rīti-trapu-sīsakānām }||\thinspace18:19\thinspace||%
\translation{If one steals copper, the fore part of one's hand will be split. If one steals steel, the characteristic mark will be a broken arm. If one steals brass, the characteristic mark will be a broken hand. Stealing bell-metal, tin or lead \blankfootnote{18.19 Note the stem forms °\textit{cheda} and °\textit{bhagna} in \textit{pāda}s b and c.
 Note \textit{kāṃsa} as an alternative form of \textit{kāṃsya}, and ˚\textit{bhagna} as a stem form in \textit{pāda} c.
 }}

  \maintext{nāsoṣṭhakarṇaśravaṇasya chedaś}%

 \nonanustubhindent \maintext{cihnaṃ nṛṇāṃ vastraharaḥ kucailaḥ |}%

  \maintext{dhānyāpahārī bhavaty aṅgahīno}%

 \nonanustubhindent \maintext{dīpāpahārī bhavaty andha cihnam }||\thinspace18:20\thinspace||%
\translation{will cause clefts in the nose, lips, ears, and problems with hearing[?]. The characteristic mark of one who stole people's clothes is being badly-dressed. Those who steal grain will lack some of their limbs. If one steals lamps, the characteristic mark is that he will become blind. \blankfootnote{18.20 Note stem form metri causa in \textit{pāda} d {\rm (}\textit{andha}{\rm )}.
 }}

  \maintext{nirvāpahā kāṇa bhaveta cihnaṃ}%

 \nonanustubhindent \maintext{yaḥ strīṃ haret so 'pi jitaḥ striyā syāt |}%

  \maintext{sasyāpahārī bhavate 'nnahīno}%

 \nonanustubhindent \maintext{hṛtvāyudham astrahatatva cihnam }||\thinspace18:21\thinspace||%
\translation{The sing of one who takes away sacrifical offerings is becoming one-eyed. He who abducts women will himself be overcome by women. Somebody who steals corn will lack food. The characteristic mark of being a thief of weapons is death by a missile. }

  \maintext{annāpahārī paradattabhoktā}%

 \nonanustubhindent \maintext{hṛtvā tu gāvaḥ sa bhaved daridraḥ |}%

  \maintext{hariṃ haret tad dhariṇā dahanti}%

 \nonanustubhindent \maintext{hṛtvā tu meṣān ajagardabhaṃ vā }||\thinspace18:22\thinspace||%
\translation{One who steals food will live on [food] given by others. One who steals cows will become poor. One who steals horses will be destroyed by a horse. One who steals sheep, goats, donkeys, }

  \maintext{sa bhārabhṛjjīvya{-}m{-}udāharanti}%

 \nonanustubhindent \maintext{ratnāpahārī anapatyatā ca |}%

  \maintext{chatrāpahārī apavitratā ca}%

 \nonanustubhindent \maintext{hṛtvā ca bījaṃ sa bhaved abījaḥ }||\thinspace18:23\thinspace||%
\translation{will lead[?] a burdened life, they say[?]. One who steals jewels: [the sign is] childlessness. One who steals parasols: [the sign is] impurity. Stealing seeds, one becomes seedless. }

  \maintext{godhūmaśāliyavamudgamāṣān}%

 \nonanustubhindent \maintext{hṛtvā masūraṃ vilayaṃ vrajanti |}%

  \maintext{kāmāturo mātara mātṛputrīṃ}%

 \nonanustubhindent \maintext{mātṛsvasāṃ gacchati mātulānīm }||\thinspace18:24\thinspace||%
\translation{If one steals wheat, rice, barley, mungo beans, wild beans, or lentils, one will die. If somebody, being sick with desire, sexually approaches his mother, his mother's daughter, his mother's sister, or the wife of a maternal uncle, }

  \maintext{rājāṅganāṃ putrasutāṃ snuṣāṃ ca}%

 \nonanustubhindent \maintext{pravrājinīṃ brāhmaṇīm antyajāṃ ca | }%

  \maintext{ajāśvameṣaṃ surabhīsutāṃ ca}%

 \nonanustubhindent \maintext{yat kāmayet teṣu vimūḍhacetāḥ }||\thinspace18:25\thinspace||%
\translation{or if somebody desires a royal concubine, his son's daughter, a female religious mendicant, a Brahmin's wife, or the wife of a low-born, a goat, horse, sheep, or a cow, with a foolish mind, }

  \maintext{sa yāti kṛcchraṃ narakaṃ sughoraṃ}%

 \nonanustubhindent \maintext{sa varṣakoṭīśataśo bhramitvā |}%

  \maintext{tīryañ ca bhūyaḥ śataśo vyatītya}%

 \nonanustubhindent \maintext{kaṣṭena vai jāyati mānuṣatvam }||\thinspace18:26\thinspace||%
\translation{he will go to the painful and extremely terrible hell. Wandering [through transmigration] a million times, dying as an animal again and again a hundred times, he will, with great difficulty, be born as a human. }

  \maintext{hīnāṅgatā dīnaśarīratāś ca}%

 \nonanustubhindent \maintext{yo mātṛgāmī sa bhaved aliṅgaḥ |}%

  \maintext{mātṛsvasātalpagavānaliṅgā}%

 \nonanustubhindent \maintext{liṅgāparodhaḥ sutaputrikāmaḥ }||\thinspace18:27\thinspace||%
\translation{He who had sex with his mother will lack some limbs and will have a miserable body, and will have no penis. ...  }

  \maintext{snuṣāṃ ca yaḥ sevati raktam ehī}%

 \nonanustubhindent \maintext{dauḥ carmatāś ca dvijasundarīṣu |}%

  \maintext{rājāṅganāyāsu ca liṅgacchedaḥ}%

 \nonanustubhindent \maintext{pravrājinī kāmukamūtrakṛcchram }||\thinspace18:28\thinspace||%


  \maintext{savyādhiliṅga labhatenty ajāsu}%

 \nonanustubhindent \maintext{vilīnaliṅgaḥ paśuyonigāmī |}%

  \maintext{jāyanti te mūṣika dhānyacaurī}%

 \nonanustubhindent \maintext{kṣīraṃ hared vāyasatāṃ prayāti }||\thinspace18:29\thinspace||%


  \maintext{haṃsāpahārī sa bhaven nihaṃsaḥ}%

 \nonanustubhindent \maintext{śvānatvam āyāti rasāpahārī |}%

  \maintext{hṛtvā ca sūcīn tu bhavet sa daṃśaḥ}%

 \nonanustubhindent \maintext{hṛtvā tu sarpir vṛṣatāṃ prayāti }||\thinspace18:30\thinspace||%


  \maintext{māṃsaṃ tu hṛtvā sa bhaveta gṛdhras}%

 \nonanustubhindent \maintext{tailāpahārī khagatāṃ prayāti |}%

  \maintext{guḍaṃ ca hṛtvā guḍikā bhavanti}%

 \nonanustubhindent \maintext{śākāpahārī sa bhaven mayūram }||\thinspace18:31\thinspace||%


  \maintext{hṛtvā paśuṃ paṅgurajāyatehaḥ}%

 \nonanustubhindent \maintext{citratvam āyāti suvastrahārī |}%

  \maintext{hṛtvā dukūlaṃ sa ca sārasattvaṃ}%

 \nonanustubhindent \maintext{kṣaumaṃ ca hṛtvā sa ca durbalatvam }||\thinspace18:32\thinspace||%


  \maintext{ūrnāni vastrāṇy apahṛtya meṣaḥ}%

 \nonanustubhindent \maintext{chuchundarī jāyati gandhahārī |}%

  \maintext{brahmasvam alpam apahṛtya bhoktā}%

 \nonanustubhindent \maintext{sa gṛdhra ucchiṣṭabhujo bhavanti }||\thinspace18:33\thinspace||%


  \maintext{pādena yaḥ sparśayate dvijāṅghriṃ}%

 \nonanustubhindent \maintext{tacchītaraktaṃ caraṇau bhaveta |}%

  \maintext{pādena yaḥ sparśayate ca gāvaḥ}%

 \nonanustubhindent \maintext{sa pādarogān vividhāṃl labheta }||\thinspace18:34\thinspace||%


  \maintext{yo mātaraḥ tāḍayate pādena}%

 \nonanustubhindent \maintext{pāde tadīye kṛmayaḥ patanti |}%

  \maintext{pādāt pṛśed yaḥ pitaraṃ durātmā}%

 \nonanustubhindent \maintext{sūnonnapādaḥ sa bhavet paratra }||\thinspace18:35\thinspace||%


  \maintext{padāt pṛśet toyam anādareṇa}%

 \nonanustubhindent \maintext{saślīpadīpādayuge bhaveta |}%

  \maintext{pādena ya sparśayate hutāśaṃ}%

 \nonanustubhindent \maintext{sa cāgnipādaḥ satataṃ bhaveta }||\thinspace18:36\thinspace||%


  \maintext{pādena yaś cāryam upaspṛśeta}%

 \nonanustubhindent \maintext{sa pādacchedaṃ bahuśo labheta |}%

  \maintext{granthāpahārī sa bhaveta mūkaḥ}%

 \nonanustubhindent \maintext{durgandhavaktraḥ parichidravādī }||\thinspace18:37\thinspace||%


  \maintext{paiśunyavādī sa ca pūtināsām}%

 \nonanustubhindent \maintext{anamravaktras tv anṛtāpavādī |}%

  \maintext{pāruṣyavaktā mukhapākarāgī}%

 \nonanustubhindent \maintext{asat pralāpī sa ca dantarogaḥ }||\thinspace18:38\thinspace||%


  \maintext{stīkṣṇapradāyī sa ca vakranāsa}%

 \nonanustubhindent \maintext{sambhinnavaktā sa ca kaṇṭharogī |}%

  \maintext{kruddhekṣaṇaḥ paśyati yas tu vipraṃ}%

 \nonanustubhindent \maintext{tīvrākṣirogī sa tu jāyate hi }||\thinspace18:39\thinspace||%


  \maintext{pradveṣayālokayate 'tithīn ya}%

 \nonanustubhindent \maintext{utpāditākṣis sa bhavet paratra |}%

  \maintext{vairūpya cakṣus tv atisūkṣmacakṣuḥ}%

 \nonanustubhindent \maintext{sa jāyate kekarapiṅgayakṣuḥ }||\thinspace18:40\thinspace||%


  \maintext{gartākṣikādīni vipāṇḍurāṇi}%

 \nonanustubhindent \maintext{netrāmayāny eva ca pāpadoṣāt |}%

  \maintext{śṛṇvanti ye pāpakathāṃ praśastāṃ}%

 \nonanustubhindent \maintext{tāṃ karṇasarpiḥ paripīḍiyeta }||\thinspace18:41\thinspace||%


  \maintext{śṛṇvanti nindāṃ hariśarvayor yaḥ}%

 \nonanustubhindent \maintext{sa karṇaśūlena tu jīvatī vā |}%

  \maintext{mātāpitṝṇāṃ śṛṇute 'pavādaḥ}%

 \nonanustubhindent \maintext{sa karṇasāphena vināśam eti }||\thinspace18:42\thinspace||%


  \maintext{śṛṇoti nindāṃ guruviprajā yaḥ}%

 \nonanustubhindent \maintext{sa karṇapūyaṃ sravate saraktam |}%

  \maintext{virūpyadāridhrakulādhameṣu}%

 \nonanustubhindent \maintext{aniṣṭakarmabhṛtijīvanāś ca }||\thinspace18:43\thinspace||%


  \maintext{akīrtanaṃ darśanavarjanaṃ ca}%

 \nonanustubhindent \maintext{śvāpākato śvādiṣu jāyate saḥ |}%

  \maintext{etāni cihnaṃ nirayāgatānāṃ}%

 \nonanustubhindent \maintext{mānuṣyaloke kukṛtasya dṛṣṭam |}%

  \maintext{samāsataḥ kīrtita eva devi}%

 \nonanustubhindent \maintext{yathaiva muktis tv iha karmabhaṅgaḥ }||\thinspace18:44\thinspace||%


  \maintext{mātāpitroghato yāsutaduhitṛvahā bhrātṛgambhīravegā}%

 \nonanustubhindent \maintext{bhāryāvartā vivartā kuṭilagativadhur bāndhavormītaraṅgā |}%

  \maintext{kāmakrodhobhakūlā karimakarajhaṣā grāhakāmā bhayante}%

 \nonanustubhindent \maintext{mṛtyor ākhyārṇave 'smin na śaraṇavivaśākāladṛṣṭo prayāti }||\thinspace18:45\thinspace||%


  \maintext{nityaṃ yena vinā na yāti divasaṃ pañcatvam āpadyate}%

 \nonanustubhindent \maintext{tyaktvā deha vanāntareṣu viṣame śvānaśrigālākule |}%

  \maintext{bandhuḥ sarvanivartate gatadayā dharmaika tatra sthitaḥ}%

 \nonanustubhindent \maintext{tasmād dharmaparo na cānyaḥ suhṛdaḥ sevet paratrārthinaḥ }||\thinspace18:46\thinspace||%

\centerline{\maintext{\dbldanda\thinspace iti vṛṣasārasaṃgrahe pūrvakarmavipākacihnāṣṭādaśo 'dhyāyaḥ\thinspace\dbldanda}}
