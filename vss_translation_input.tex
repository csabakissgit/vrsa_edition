
  \chptr{ekonaviṃśatimo 'dhyāyaḥ}
\addcontentsline{toc}{section}{Chapter 19}
\fancyhead[CO]{{\footnotesize\textit{Translation of chapter 19}}}%

  \trchptr{Chapter Nineteen}%

  \subchptr{gāvaḥ}%

  \trsubchptr{Cows}%

  \maintext{vigatarāga uvāca |}%

  \maintext{kriyāsūkṣmo mahādharmaḥ karmaṇā kena prāpyate |}%

  \maintext{alpopāyaṃ narārthāya pṛcchāmi kathayasva me }||\thinspace19:1\thinspace||%
\translation{Vigatarāga spoke: By what action can the great Dharma, whose rituals are subtle[?], be reached? I ask for an easy method for mankind, tell me about it. }

  \maintext{anarthayajña uvāca |}%

  \maintext{alpopāyaṃ mahādharmaṃ kathayāmi dvijottama |}%

  \maintext{sukhena labhate svargaṃ karmaṇā yena tac chṛṇu }||\thinspace19:2\thinspace||%
\translation{Anarthayajña spoke: I shall teach you the great Dharma that is the easy method, O Brahmin. Listen to that action by which heaven can be reached easily. }

  \maintext{lokānāṃ mātaro gāvo gobhiḥ sarvaṃ jagad dhṛtam |}%

  \maintext{gomayam amṛtaṃ sarvaṃ jātaṃ sarvaṃ śivecchayā }||\thinspace19:3\thinspace||%
\translation{Cows are the mothers of the worlds. Cows hold all the world. All cow-dung is nectar, all are produced by Śiva's will. }

  \maintext{sarvadevamayā gāvaḥ sarvadevamayo dvijaḥ |}%

  \maintext{sarvadevamayī bhūmiḥ sarvadevamayaḥ śivaḥ }||\thinspace19:4\thinspace||%
\translation{Cows contain all the gods. The Brahmin contains all the gods. Earth contains all the gods. Śiva contains all the gods. }

  \maintext{tasmād gāvaḥ sadā sevyā dharmamokṣārthasiddhidāḥ |}%

  \maintext{paricaryā yathāśaktyā grāsavāsajalādibhiḥ }||\thinspace19:5\thinspace||%
\translation{Therefore cows are always to be served because they give religious duties, liberation, financial gain and success. They should be provided with food, shelter, water, etc., with all one's effort. }

  \maintext{tāḍayen nātivegena vācayen mṛdunācaret |}%

  \maintext{pālayeta ghanāḍhyeṣu bhagnodvigneṣu yatnataḥ }||\thinspace19:6\thinspace||%
\translation{One should not beat them too hard...? Protect them in thick [darkness or when in multitude], in case something is broken or they are startled. }

  \maintext{vyādhivraṇaparikleśa oṣadhopakramaṃ caret |}%

  \maintext{kaṇḍūyanaṃ ca kartavyaṃ yathāsaukhyaṃ bhaved gavām }||\thinspace19:7\thinspace||%
\translation{In case of pain from disease or wound, one should apply remedy using medicine. Rubbing should be done as much as it is pleasurable for cows. }

  \maintext{gavāṃ pradakṣiṇaṃ kṛtvā śraddhābhaktisamanvitaḥ |}%

  \maintext{sāgarāntā mahī sarvā pradakṣiṇīkṛtā bhavet }||\thinspace19:8\thinspace||%
\translation{By circumabulating cows with faith and devotion, the whole Earth up to the oceans gets circulambulated. }

  \maintext{spṛṣṭasaṃsparśanādye ca śraddhayā yadi mānavaḥ |}%

  \maintext{ahorātrakṛtaṃ pāpaṃ naśyate nātra saṃśayaḥ }||\thinspace19:9\thinspace||%
\translation{If a man touches a cow with faith ...... his sins, be them committed at daylight or at night, will disappeat, no doubt. }

  \maintext{lāṅgūlenoddhṛtaṃ toyaṃ mūrdhnā gṛhṇāti yo naraḥ |}%

  \maintext{yāvaj jīvakṛtaṃ pāpaṃ naśyate nātra saṃśayaḥ }||\thinspace19:10\thinspace||%
\translation{He who applies the water that has been dispersed by a [cow's] tail onto his head, will have his sins accumulated throughout his life destroyed, no doubt. }

  \maintext{vidhivat snāpayed gāṃś ca mantrayuktena vāriṇā |}%

  \maintext{tenāmbhasā svayaṃ snātvā sarvapāpakṣayo bhavet }||\thinspace19:11\thinspace||%
\translation{One should bathe the cows as prescribed, using water onto which mantras have been recited. If he himself bathes in the same water, he will have all his sins destroyed. }

  \maintext{vyādhir vighnam alakṣmītvaṃ naśyate sadya eva ca |}%

  \maintext{mṛtāpatyānapatyāś ca snānam eva praśasyate }||\thinspace19:12\thinspace||%
\translation{Diseases, obstructing forces, and bad luck will disappear instantly. Those with dead offspring or without offspring praise this very bath. \blankfootnote{19.12 Understand \textit{praśasyate} in \textit{pāda} d as active and plural {\rm (}\textit{prasaṃsanti}{\rm )}.
 }}

  \maintext{gavāṃ śṛṅgodakaṃ gṛhya mūrdhni yo dhārayen naraḥ |}%

  \maintext{sa sarvatīrthasnānasya phalaṃ prāpnoti mānavaḥ }||\thinspace19:13\thinspace||%
\translation{If a man collects the `horn-water' of cows and applies it on his head, he will receive the fruits of bathing at all the sacred pilgrimage places. \blankfootnote{19.13 Applying `horn water' means sprinkling with water filled into a cow's horn, 
  while reciting the Gāyatrī matra a hundred times. See note to \SDHS\ 10.24
  in \mycite{SDhS10_ed}.
 }}

  \maintext{grāsamuṣṭipradānena goṣu bhaktisamanvitaḥ |}%

  \maintext{agnihotraṃ hutaṃ tena sarvadevāḥ sutarpitāḥ }||\thinspace19:14\thinspace||%
\translation{If somebody gives a handful of food to cows with devotion, by this an Agnihotra is being performed and all the gods become satisfied. }

  \maintext{catvāraḥ stanadhārās tu yas tu mūrdhnā pratīcchati |}%

  \maintext{sa catuḥsāgaraṃ gatvā snānapuṇyaphalaṃ labhet }||\thinspace19:15\thinspace||%
\translation{He who collects on his head the four streams [of milk] from the teats will receive the meritous fruits of visiting the four oceans. }

  \maintext{gavārthaṃ yas tyajet prāṇān gograheṣu dvijottama |}%

  \maintext{kalpakoṭiśataṃ divyaṃ śivaloke mahīyate }||\thinspace19:16\thinspace||%
\translation{He who gives his life for cows during an attempt at stealing them, O greatest of Brahmins will prosper in Śivaloka for millions of years. }

  \maintext{cyutabhagnādisaṃskāraṃ sarvaṃ yaḥ kurute naraḥ |}%

  \maintext{bhāryākoṭiśataṃ dānaṃ yat phalaṃ parikīrtitam  }||\thinspace19:17\thinspace||%
\translation{If a man rears all [the cows] that have missing or broken [limbs] CHECK, will get all the fruits that are said to be produced by donating millions of wives, }

  \maintext{tatphalaṃ labhate martyaḥ śivalokaṃ ca gacchati |}%

  \maintext{śivalokaparibhraṣṭaḥ pṛthivyām ekarāḍ bhavet }||\thinspace19:18\thinspace||%
\translation{and will go to Śivaloka. When descended from Śivaloka, he will become a universal monarch on Earth. }

  \maintext{samāsataḥ samākhyātaṃ yathātattvaṃ dvijottama |}%

  \maintext{na śakyaṃ vistarād vaktuṃ gomahābhāgyam uttamam }||\thinspace19:19\thinspace||%
\translation{I have taught [about cows] truly, in brief, O supreme Brahmin. It is impossible to talk about the excellence of cows in more detail. }

  \subchptr{cāturvarṇyam}%

  \maintext{vigatarāga uvāca |}%

  \maintext{devā aṣṭavidhāḥ proktās tiryak pañcavidhaḥ smṛtaḥ |}%

  \maintext{mānuṣam ekam evāhuś cāturvarṇaḥ kathaṃ bhavet }||\thinspace19:20\thinspace||%
\translation{Vigatarāga spoke: The gods are of eight kinds, animals are of five kinds. Mankind is said to be only one single [kind]. How come that there is the system of four social classes {\rm (}\textit{varṇa}{\rm )}? \blankfootnote{19.20 cāturvarṇ[y]aṃ
 }}

  \maintext{anarthayajña uvāca |}%

  \maintext{pūrvakalpasṛjas tv eṣa viṣṇunā prabhaviṣṇunā |}%

  \maintext{ekavarṇo dvijaś cāsīt sarvakalpāgram agrataḥ }||\thinspace19:21\thinspace||%
\translation{Anarthayajña spoke: It [i.e. the system of four social classes] was created by Lord Viṣṇu in the previous \ae on[s].  Before the very beginning of all \ae ons, there was a single class {\rm (}\textit{varṇa}{\rm )} of Brahmins {\rm (}\textit{dvija}{\rm )}. \blankfootnote{19.21 See above \textit{sṛja} for \textit{sṛṣṭa} in XXX.
 }}

  \maintext{sarvavedavido viprāḥ sarvayajñavidas tathā |}%

  \maintext{teṣāṃ viprasahasrāṇāṃ yajñotsāhamano bhavet }||\thinspace19:22\thinspace||%
\translation{The Brahmins {\rm (}\textit{vipra}{\rm )} got to know all the Vedas and all the sacrifices. These thousands of Brahmins {\rm (}\textit{vipra}{\rm )} developed an inclination to make a resolution to perform sacrifices. }

  \maintext{vṛddhaviprasahasrāṇāṃ matam ājñāya brāhmaṇaiḥ |}%

  \maintext{kartuṃ karma samārabdhaṃ karma cāpi vibhajyate }||\thinspace19:23\thinspace||%
\translation{Having understood the intention of the thousands of senior Brahmins {\rm (}\textit{vipra}{\rm )}, the Brahmins {\rm (}\textit{brāhmaṇa}{\rm )} commenced performing rituals {\rm (}\textit{karman}{\rm )} and the tasks {\rm (}\textit{karman}{\rm )} were distributed. }

  \maintext{ṛtvijatve sthitāḥ kecit kecit saṃrakṣaṇe sthitāḥ |}%

  \maintext{arthopārjanayuktānye anye śilpe niyojitāḥ }||\thinspace19:24\thinspace||%
\translation{Some took on the function of being priests {\rm (}\textit{ṛtvij}{\rm )}, some took on the task of protection. Some got engaged in the acquisition of materials and others were appointed to do manual crafts. \blankfootnote{19.24 Note the form \textit{ṛtvijatva}.
 Note the double sandhi in °\textit{yuktānye} {\rm (}\textit{yuktā anye}{\rm )}.
 }}

  \maintext{evaṃ yajñavidhānena kartum ārebhire purā |}%

  \maintext{yathoddiṣṭena karmeṇa yajñotsāha{-}m{-}avartata }||\thinspace19:25\thinspace||%
\translation{This is how they started performing sacrifices in the beginning. With the tasks {\rm (}\textit{karman}{\rm )} thus distributed, the will to perform sacrifices increased. \blankfootnote{19.25 Perhaps understand pāda a as \textit{evaṃvidhānena yajñaṃ kartum}.
 }}

  \maintext{āgatā ṛṣayaḥ sarve devatāḥ pitaras tathā |}%

  \maintext{anyonyam abruvan tatra devarṣipitṛdevatāḥ }||\thinspace19:26\thinspace||%
\translation{Then came all the Ṛṣis and all the gods and the Ancestors. They discussed it among themselves, the divine Ṛṣis, the Ancestors and the gods. }

  \maintext{yajñārtham asṛjad varṇaṃ vidhinā kratuhetavaḥ |}%

  \maintext{evam eva pravartantu bhavadbhir dvijasattamāḥ }||\thinspace19:27\thinspace||%
\translation{Brahmā/Viṣṇu {\rm (}\textit{vidhi}{\rm )} created class for the sake of sacrifice. [The classes are] for the purpose of rituals {\rm (}\textit{kratu}{\rm )}. Proceed in this very manner, Sirs, O excellent twice-born! \blankfootnote{19.27 Note the confused syntax both in pādas ab and cd.
 }}

  \maintext{ijyādhyayanasampannā brāhmaṇā ye 'tra kalpitāḥ |}%

  \maintext{suviprā vipratāṃ yāntu ṣaṭkarmaniratāḥ sadā }||\thinspace19:28\thinspace||%
\translation{Those Brahmins {\rm (}\textit{brāhmaṇa}{\rm )} who are now engaged in sacrifice and recitation, those good Brahmins {\rm (}\textit{suvipra}{\rm )} shall become Brahmins {\rm (}\textit{vipratāṃ yāntu}{\rm )}, always engaged in the six duties [of Brahmins] {\rm (}\textit{ṣaṭkarman}{\rm )}. }

  \maintext{rakṣaṇārthaṃ tu ye viprāḥ kalpitāḥ śastrapāṇayaḥ |}%

  \maintext{kṣatatrāṇāya viprāṇāṃ nityakṣatravratodbhavāḥ }||\thinspace19:29\thinspace||%
\translation{As for those Brahmins {\rm (}\textit{vipra}{\rm )} that have been appointed to protect [the sacrifice] with weapons in their hands, to protect the Brāhmins {\rm (}\textit{vipra}{\rm )} from injury, they shall eternally follow[?] the vow of Kṣatras. }

  \maintext{arthopārjanam uddiśya kalpitā ye dvijātayaḥ |}%

  \maintext{te tu vaiśyatvam āyāntu vārttopāyaratodbhavāḥ }||\thinspace19:30\thinspace||%
\translation{As for those twice-born who have been appointed for the acquisition of materials, they shall become Vaiśyas, involved in the means of trade. }

  \maintext{vadhabandhanakarmasu śilpasthānavidheṣu ca |}%

  \maintext{kalpitā ye dvijātīnāṃ sarve śūdrā bhavantu te }||\thinspace19:31\thinspace||%
\translation{Those of the twice-born who have been appointed to the tasks of slaughering and tying [animals] and of manual skills, they all shall become Śūdras. }

  \maintext{prājāpatyaṃ brāhmaṇānām ijyādhyayanatatparāt |}%

  \maintext{sthānam aindraṃ kṣatriyāṇāṃ prajāpālanatatparāt }||\thinspace19:32\thinspace||%
\translation{The [world] of Prajāpati belongs to the Brahmins {\rm (}\textit{brāhmaṇa}{\rm )} [after death] because they are devoted to the sacrifice and to recitation. The [world] of Indra belongs to the Kṣatriyas because they are devoted to the protection of the people. }

  \maintext{vaiśyānāṃ vāsavasthānaṃ vāṇijyakṛṣijīvinām |}%

  \maintext{śūdrāṇāṃ marutaḥ sthānaṃ śuśrūṣāniratātmanām }||\thinspace19:33\thinspace||%
\translation{The [world] of Vāsus belongs to the Vaiśyas who earn their living by trade and agriculture. The [world] of Marut belongs to the Śūdras who devote themselves to sevice. }

  \maintext{maharṣipitṛdevānāṃ matam ājñāya niścitaḥ |}%

  \maintext{eṣa saṃkalpito brahmā padmayoniḥ pitāmahaḥ }||\thinspace19:34\thinspace||%
\translation{Understanding the intention of the great Ṛṣis, the Ancestors and the gods, lotus-born Brahmā, the Grandfather, it [i.e. the system of \textit{varṇa}s] was established firmly. }

  \maintext{saṃkalpaprabhavāḥ sarve devadānavamānavāḥ |}%

  \maintext{paśupakṣimṛgā mukhyā yāvanti jagasambhavāḥ }||\thinspace19:35\thinspace||%
\translation{All the main domestic animals, birds and wild animals that are born in the world, }

  \maintext{bhūtasaṃkalpakaṃ nāma kalpam āsīd dvijottama |}%

  \maintext{kīrtitāni samāsena kim anyac chrotum icchasi }||\thinspace19:36\thinspace||%
\translation{CHECK .... [The social classes] have been taught briefly. What else do you wish to hear? }

  \maintext{vigatarāga uvāca |}%

  \maintext{kiṃ tapaḥ sarvavarṇānāṃ vṛttir vāpi tapodhana |}%

  \maintext{yajñāś caiva pṛthaktvena śrotum icchāmi tattvataḥ }||\thinspace19:37\thinspace||%


  \maintext{anarthayajña uvāca |}%

  \maintext{brāhmaṇasya tapo yajñāḥ - tapaḥ kṣātrasya rakṣaṇam |}%

  \maintext{vaiśyaś ca tapa vāṇijya tapaḥ śūdrasya sevanam }||\thinspace19:38\thinspace||%


  \maintext{pratigrahadhano vipraḥ kṣatriyasya dhanur dhanam |}%

  \maintext{kṛṣir dhanaṃ tathā vaiśyaḥ śūdraḥ śuśrūṣaṇaṃ dhanam }||\thinspace19:39\thinspace||%


  \maintext{ārambhayajñaḥ kṣatrasya havir yajño viśas tathā |}%

  \maintext{śūdraḥ paricaro yajño japayajño dvijātayaḥ }||\thinspace19:40\thinspace||%


  \maintext{satya tīrtha dvijātīnāṃ raṇa tīrthaṃ tu kṣatriyāḥ |}%

  \maintext{āryā tīrthaṃ tu vaiśānāṃ ! śūdratīrthaṃ tu vai dvijāḥ }||\thinspace19:41\thinspace||%


  \maintext{nāsti vidyāsamo mitro nāsti dānasamaḥ sakhā |}%

  \maintext{nāsti jñānasamo bandur nāsti yajño japaḥ samaḥ }||\thinspace19:42\thinspace||%


  \maintext{dharmahīno mṛtas tulyo devatulyo jitendriyaḥ |}%

  \maintext{yajñatulyo 'bhayaṃ dātā śivatulyo manonmanaḥ }||\thinspace19:43\thinspace||%


  \maintext{vigatarāga uvāca |}%

  \maintext{dāna yajñas tapas tīrthaṃ saṃnyāsaṃ yoga eva ca |}%

  \maintext{eteṣu katamaḥ śreṣṭhaḥ śrotum icchāmi kīrtaya }||\thinspace19:44\thinspace||%


  \maintext{anarthayajña uvāca |}%

  \maintext{dānadharmasahasrebhyaḥ yajñayājī viśiṣyate |}%

  \maintext{yajñayājīsahasrebhyas tīrthayātrī viśiṣyate }||\thinspace19:45\thinspace||%


  \maintext{tīrthayātrisahasrebhyas tapaniṣṭo viśiṣyate |}%

  \maintext{tapaniṣṭhasahasrebhyaḥ śreṣṭhaḥ saṃnyāsikaḥ smṛtaḥ }||\thinspace19:46\thinspace||%


  \maintext{saṃnyāsīnāṃ sahasrebhyaḥ śreṣṭho yac ya jitendriyaḥ |}%

  \maintext{jitendriyasahasrebhyaḥ yogayukto viśiṣyate }||\thinspace19:47\thinspace||%


  \maintext{yogayuktasahasrebhyaḥ śreṣṭho līnamanaḥ smṛtaḥ |}%

  \maintext{tasmāt sarvaprayatnena ādau mana viśodhayet }||\thinspace19:48\thinspace||%


  \maintext{nigṛhītendriyagrāmaḥ svargamokṣau tu sādhanam |}%

  \maintext{viśiṣṭhe tv indriyagrāme tiryannarakasādhanam }||\thinspace19:49\thinspace||%


  \maintext{vigatarāga uvāca |}%

  \maintext{carācarāṇāṃ bhūtānāṃ katamaḥ śreṣṭha ucyate |}%

  \maintext{kathayasva mamādya tvaṃ chettum arhasi saṃśayam }||\thinspace19:50\thinspace||%


  \maintext{anarthayajña uvāca |}%

  \maintext{carācarāṇāṃ bhūtānāṃ tatra śreṣṭho - carāḥ smṛtāḥ |}%

  \maintext{carāṇāṃ caiva sarveṣāṃ buddhimān śreṣṭha ucyate }||\thinspace19:51\thinspace||%


  \maintext{buddhimānṣu ! ca sarveṣu tataḥ śreṣṭha narāḥ smṛtāḥ |}%

  \maintext{narāṇāṃ caiva sarveṣāṃ brāhmaṇaḥ śreṣṭha ucyate }||\thinspace19:52\thinspace||%


  \maintext{vidvarsv api ca sarveṣu kṛtabuddhir viśiṣyate |}%

  \maintext{kṛtabuddhiṣu sarveṣu śreṣṭhaḥ kartā sa ucyate }||\thinspace19:53\thinspace||%


  \maintext{kartṛṣv api ca sarveṣu brahmavedī viśiṣyate |}%

  \maintext{brahmavedi paraṃ ! vipraḥ nānyaṃ vedmi paraṃ tapaḥ |}%

  \maintext{sa vipraḥ sa tapasvī ca sa yogī sa śivaḥ smṛtaḥ }||\thinspace19:54\thinspace||%


\centerline{\maintext{\dbldanda\thinspace iti vṛṣasārasaṃgrahe dānayajñaviśeṣo nāma unaviṃśatitamo 'dhyāyaḥ\thinspace\dbldanda}}
