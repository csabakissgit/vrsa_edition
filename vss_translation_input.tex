
  \chptr{prathamo 'dhyāyaḥ}
\addcontentsline{toc}{section}{Chapter 1}
\fancyhead[CO]{{\footnotesize\textit{Translation of chapter 1}}}%

  \trchptr{Chapter One}%

  \subchptr{stutiḥ}%

  \trsubchptr{Invocation}%

  \maintext{anādimadhyāntam anantapāraṃ}%

 \nonanustubhindent \maintext{susūkṣmam avyaktajagatsusāram |}%

  \maintext{harīndrabrahmādibhir āsamagraṃ}%

 \nonanustubhindent \maintext{praṇamya vakṣye vṛṣasārasaṃgraham }||\thinspace1:1\thinspace||%
\translation{Having bowed to the One who has no beginning, no middle part and no end, whose boundaries are limitless, who is very subtle and who is the unmanifest and fine essence of the world, and also to Indra, Brahmā and the other [gods], I shall recite [the work called] `A Compendium on the Essence of the Bull [of Dharma]'. \blankfootnote{1.1 This verse echoes \VSS\ 20.3:
 
  \textit{nādimadhyaṃ na cāntaṃ ca yan na vedyaṃ surair api}\thinspace | 
 
  \textit{atisūkṣmo hy atisthūlo nirālambo nirañjanaḥ\thinspace ||} 
 
  This could suggest
  that \textit{pāda} c above might be parallel with \textit{na vedyaṃ surair api}. 
  Perhaps understand \textit{asamagraṃ} [\textit{vedyaṃ}] {\rm (}`incompletely [known]{\rm )}.
 
  \textit{Pāda} a is also reminiscent of, among other famous passages, \BHG\ 11.19:
  
 
  \textit{anādimadhyāntam anantavīryam} 
  \textit{anantabāhuṃ śaśisūryanetram}\thinspace | 
 
  \textit{paśyāmi tvāṃ dīptahutāśavaktraṃ} 
  \textit{svatejasā viśvam idaṃ tapantam}\thinspace ||
  
 
  See also \BHG\ 10.20cd:
  
 
  \textit{aham ādiś ca madhyaṃ ca bhūtānām anta eva ca}\thinspace ||
  
 
  A faint reference to the \BHG\ seems proper at the
  beginning of a work that claims to deliver a teaching 
  based on, but also to surpass, the \MBH\ {\rm (}see following verses of the \VSS{\rm )}.
 
  Compare also, e.g., \KURMP\ 1.11.237:
  
 
  \textit{rūpaṃ tavāśeṣakalāvihīnam}
  \textit{agocaraṃ nirmalam ekarūpam}\thinspace |
  
  \textit{anādimadhyāntam anantam ādyaṃ} 
  \textit{namāmi satyaṃ tamasaḥ parastāt}\thinspace ||
  
 
  In general, to say that a god has no beginning and no end in a temporal or spacial sense is natural
  {\rm (}\textit{anādi \dots\ antam}{\rm )}, but to have no `middle part' {\rm (}\textit{madhya}{\rm )} in these senses is slightly less so.
  Thus the rather commonly occurring phrase \textit{anādimadhyāntam} is probably a fixed expression usually 
  referring to a deity that is endless, eternal and immaterial. 
  As to which deity or what form of a deity this stanza refers to, one could argue that 
  it is Śiva, his name missing in \textit{pāda} c, but the phrasing of the verse 
  is vague enough to keep the question somewhat open: the impersonal Brahman 
  might be another option, even more so if we look at verses 1.9--10, whose
  topic is \textit{brahmavidyā}.
 
 
  
 
  In \textit{pāda} b \textit{jagat-susāraṃ} is most probably not 
  to be interpreted as \textit{jagatsu sāraṃ} {\rm (}`the essence in the worlds'{\rm )}.
  Another way to translate \textit{avyaktajagatsusāraṃ} would be: 
  `who is the fine essence of the unmanifest world.'
 
  
 
  Strictly speaking, \textit{pāda} c is unmetrical, but it is better to 
  simply acknowledge here the phenomenon of `muta cum liquida', namely
  that syllables followed by consonant clusters such as 
  \textit{ra, bra, hra, kra, śra, śya, śva, sva, dva} can be treated as short {\rm (}\textit{laghu}{\rm )}.
  {\rm (}See Introduction pp.~\pageref{muta}\thinspace ff.{\rm )}
  Thus \textit{harīndrabrahmā}° can be treated as a regular beginning
  of an \textit{upajāti} {\rm (}\shortsyllable\ - \shortsyllable\ - -{\rm )}, the syllable 
  \textit{bra} not turning the previous syllable long {\rm (}\textit{guru}{\rm )}.
 
  
 
  The reading \textit{āsamagraṃ} in \textit{pāda} c is suspect 
  {\rm (}see a preliminary comment on this above{\rm )},
  although the initial \textit{ā-} might convey some sort of
  completeness, meaning `all round'
  {\rm (}see e.g. \mycitep{KaleHigherGrammar}{226}{\rm )}.
  The fact that we could perceive the ending of \textit{pāda}s a and b 
  {\rm (}\textit{pāraṃ}--\textit{sāram}{\rm )}, as well as \textit{pāda}s c and d, as {\rm (}in the 
  latter case, oddly{\rm )} rhyming pairs {\rm (}\textit{graṃ}-\textit{graham}{\rm )}
  suggests that accepting the reading \textit{āsamagram} could be 
  the right decision {\rm (}as suggested by Alessandro Battistini{\rm )}.
  I translate this verse accordingly. \msM\ gives an exciting,
  albeit unmetrical, alternative {\rm (}\textit{yat samagraṃ}{\rm )}, but
  this seems more of a guess than the correct reading.
  For some time I was considering emending \textit{āsamagraṃ}.
  The most tempting of all the possible options 
  {\rm (}\textit{arcyam/arhyam/arghyam/īḍyam/āḍhyam/āptam agraṃ, āsamastaṃ}{\rm )} 
  seemed to be \textit{āptam agraṃ},
  meaning `appointed/received/respected [by Hari, Indra,
  Brahmā etc.] as the foremost one'. The fact that 
  the \textit{akṣara}s \textit{āsam} and \textit{āptam} look similar in most
  of the scripts used in the witnesses could support this
  conjecture. \textit{āptam} could also
  possibly refer to the text itself, although then the
  syntax becomes slightly confusing: `I shall recite the
  \textit{Vṛṣasārasaṃgraha} that was first received by Hari...' etc.
  Another candidate was \textit{āḍhyam agram}:
  `Having bowed to [Him] who contains/is rich with Hari, Indra, Brahmā
  etc.' I have not emended the text because it is difficult
  to know if there is any need for change and if there is, which reading 
  to chose. There was no consensus when this verse was discussed 
  in our extended Śivadharma reading group.
 
  
 
  \textit{Pāda} d seems hypermetrical, but it can be interpreted as a \textit{vaṃśastha}
  line, a change from \textit{triṣṭubh} to \textit{jagatī} {\rm (}as suggested by Dominic Goodall{\rm )}.
 }}

  \subchptr{janamejayavaiśampāyanasaṃvādaḥ}%

  \trsubchptr{Dialogue of Janamejaya and Vaiśampāyana}%

  \maintext{śatasāhasrikaṃ granthaṃ sahasrādhyāyam uttamam |}%

  \maintext{parva cāsya śataṃ pūrṇaṃ śrutvā bhāratasaṃhitām }||\thinspace1:2\thinspace||%
\translation{Having listened to the \textit{Bhāratasaṃhitā} [i.e. the \textit{Mahābhārata}], the supreme book of a hundred thousand [verses] and a thousand chapters {\rm (}\textit{adhyāya}{\rm )}, with all its hundred sections {\rm (}\textit{parvan}{\rm )}, \blankfootnote{1.2 The dialogue of Janamejaya and Vaiśampāyana makes up the outermost layer of the \VSS\ 
  {\rm (}see Introduction p.~\pageref{structure}{\rm )}, mostly containing
  general \textit{dharmaśāstric} material.
  
 
  That the \MBH\ should contain a hundred thousand verses is hinted at, e.g., in line 19 of
  the Khoh Charter 2 of Śarvanātha, year 214 {\rm (}Siddham Database IN00088; 
  \textit{uktañ ca mahābhārate śatasāhasryaṃ} [understand °\textit{ryāṃ}] \textit{saṃhitāyāṃ}...{\rm )}.
  The hundred \textit{parvan}s of the \textit{Mahābhārata} are listed in \MBH\ 1.2.33--70.
  Note the use of the singular {\rm (}\textit{parva}{\rm )} in connection with numerals {\rm (}\textit{śataṃ}{\rm )},
  one of the hallmarks of this text {\rm (}see p.~\pageref{singularwithnumerals}{\rm )}.
 }}

  \maintext{atṛptaḥ puna papraccha vaiśampāyanam eva hi |}%

  \maintext{janamejayena yat pūrvaṃ tac chṛṇu tvam atandritam }||\thinspace1:3\thinspace||%
\translation{Janamejaya remained unsatisfied. Listen attentively to what he asked Vaiśampāyana in the past. \blankfootnote{1.3 My emendation from the unmetrical \textit{punaḥ} to the unusual, or rather, Middle Indic
  {\rm (}\mycitep{EdgertonHybrid}{vol. 2, p. 347}{\rm )},
  and Newar {\rm (}\mycitep{JorgensenGrammar}{113}{\rm )}, \textit{puna} is based
  on the assumption that in the original the metre must have overridden 
  morphology, similarly to what may have happened in 8.44d {\rm (}Mālinī metre{\rm )}:
  \textit{na bhavati punajanma kalpakoṭyāyute 'pi}, and in 12.151c {\rm (}Sragdharā metre{\rm )}:
  \textit{garbhāvāsaṃ na ca tvan na ca punamaraṇaṃ kleśam āyāsapūrṇam}.
 
  
 
  For an unsatisfaction or dissatisfaction {\rm (}\textit{atṛpti}{\rm )} with previous 
  teachings in a somewhat similar manner to what
  Janamejaya experiences here, see, e.g., \textit{Niśvāsa} mūla 1.9:
  
 
  \textit{vedāntaṃ viditaṃ deva sāṃkhyaṃ vai pañcaviṃśakam}\thinspace |
 
  \textit{na ca tṛptiṃ gamiṣyāmo hy ṛte śaivād anugrahāt}\thinspace ||
  
 
  
 Vaiśampāyana, a Ṛṣi, disciple of Vyāsa, great-grandson to Arjuna,
  recited the \MBh\ at the snake sacrifice of 
  Janamejaya. This setting is an echo of the starting point of the \MBH, see \MBH\ 1.1.8ff.
  In fact the next few verses in the \VSS\ make it clear that the \VSS\
  picks up where the \MBH\ left off: Janamejaya has heard the whole \MBh\ from
  Vaiśampāyana, but he is eager to hear more, or rather a concise version
  of the Dharmic teachings of the \MBh.
  
 
  It is tempting to emend \textit{pāda} c to contain a stem form proper noun {\rm (}\textit{janamejaya}{\rm )}
  in order to maintain the metre, 
  and note how the manuscripts struggle with this \textit{pāda}. 
  Stem form nouns, \textit{prātipadika}s, abound in the \VSS: see Introduction p.~\pageref{stemform}.
  On the other hand, the contracted/syncopated form \textit{janmejaya} occurs, 
  e.g., in \BHAGP\ 12.06.16 and \BRAHMAVP\ 4.14.41 and 46. {\rm (}It is even
  lexicalised in \Monier{\rm )} 
  The hypermetrical form \textit{janamejayena}, and the construction finite verb + instrumental
  {\rm (}\textit{papraccha... janamejayena}{\rm )}, could be original; compare 1.8 and 4.75 below.
  Alternatively, 1.3cd could be taken as a separate, and elliptical,
  sentence standing for \textit{janamejayena yac chrutaṃ pūrvaṃ tac chṛṇu}.
 }}

  \maintext{janamejaya uvāca |}%

  \maintext{bhagavan sarvadharmajña sarvaśāstraviśārada |}%

  \maintext{asti dharmaṃ paraṃ guhyaṃ saṃsārārṇavatāraṇam }||\thinspace1:4\thinspace||%
\translation{Janamejaya spoke: O venerable sir, O knower of the entire Dharma, O you who are well-versed in all the sciences {\rm (}\textit{śāstra}{\rm )}! There is a supreme and secret Dharma [that brings about] liberation from the ocean of mundane existence {\rm (}\textit{saṃsāra}{\rm )}, \blankfootnote{1.4 Note \textit{dharma} as a neuter noun in \textit{pāda} c and in the next verse.
 }}

  \maintext{dvaipāyanamukhodgīrṇaṃ dharmaṃ vā yad dvijottama |}%

  \maintext{kathayasva hi me tṛptiṃ kuru yatnāt tapodhana }||\thinspace1:5\thinspace||%
\translation{that is, the Dharma that emerged from [Vyāsa] Dvaipāyana's mouth, O best of Brahmins. Teach [it] to me and help me find satisfaction at all cost, O great ascetic! \vfill\pagebreak \blankfootnote{1.5 The majority of the MSS consulted include a \textit{vā} in \textit{pāda} b, 
  and although \msCb's reading seems a bit smoother, that manuscript rarely gives superior readings.
  Therefore I have chosen \textit{dharmaṃ vā yad}, in which \textit{vā} functions probably in a weak sense
  {\rm (}`that is'{\rm )}.
  That the secret Dharma Janamejaya is seeking is the one taught by Vyāsa Dvaipāyana, and
  thus no real options are involved here, becomes clear in 1.6cd.
  The reading of \msM\ in \textit{pāda} b {\rm (}\textit{dharmavākyaṃ}{\rm )} is tempting but could be a later correction.
 In general, \msM's readings here are unique but probably secondary: 
  \textit{hi me tṛptiṃ} in \textit{pāda} c seems more attractive than \msM's 
  \textit{prasādena} because it echoes \textit{atṛptaḥ} in 1.3a
 }}

  \maintext{vaiśampāyana uvāca |}%

  \maintext{śṛṇu rājann avahito dharmākhyānam anuttamam |}%

  \maintext{vyāsānugrahasamprāptaṃ guhyadharmaṃ śṛṇotu me }||\thinspace1:6\thinspace||%
\translation{Vaiśampāyana spoke: Listen with great attention, O king, to this unsurpassed narration of Dharma. Hear the secret Dharma that I received through the grace of Vyāsa. }

  \maintext{anarthayajñakartāraṃ tapovrataparāyaṇam |}%

  \maintext{śīlaśaucasamācāraṃ sarvabhūtadayāparam }||\thinspace1:7\thinspace||%
{\blankfootnote{1.7 On Anarthayajña, the interlocutor of \VSS\ 1.9--10.2 and 19.1--21.22, and
  an important figure discussed in 22.3ff, as well as a concept {\rm (}`nonmaterial sacrifice'{\rm )},
  see \mycite{KissVolume2021} and Introduction p.~\pageref{anarthayajna_person}.
 }}

  \maintext{jijñāsanārthaṃ praśnaikaṃ viṣṇunā prabhaviṣṇunā |}%

  \maintext{dvijarūpadharo bhūtvā papraccha vinayānvitaḥ }||\thinspace1:8\thinspace||%
\translation{Viṣṇu, the great Lord, assuming the form of a twice-born [Brahmin], wanted to test [Anarthayajña, the ascetic yogin] who practised nonmaterial sacrifices {\rm (}\textit{anarthayajña}{\rm )}, focused on his austerities and observances, whose conduct was virtuous and pure, and who was intent on compassion towards all living beings; therefore he [Viṣṇu] humbly asked him a question. \blankfootnote{1.8 Note the syntax here involving the agent in the instrumental
  with a finite verb {\rm (}ergative structure{\rm )}: \textit{viṣṇunā... dvijarūpadharo bhūtvā papraccha}.
  Compare 1.3.
 }}

  \subchptr{brahmavidyā}%

  \trsubchptr{Knowledge of Brahman}%

  \maintext{{\rm [}vigatarāga uvāca | {\rm ]}}%

  \maintext{brahmavidyā kathaṃ jñeyā rūpavarṇavivarjitā |}%

  \maintext{svaravyañjananirmuktam akṣaraṃ kimu tat param }||\thinspace1:9\thinspace||%
\translation{[Vigatarāga spoke:] How is the knowledge of the Brahman to be understood if it is devoid of form and colour? Why is that supreme syllable which is devoid of vowels and consonants the supreme one? \blankfootnote{1.9 The translation of this verse, and the reconstruction and interpretation
  of \textit{pāda} d, which is echoed in 1.10d, is slightly tentative.
  I doubt if \textit{kimu} could have the standard {\rm (}Vedic{\rm )} meaning `how much more/less'
  here. Rather \textit{u} is probably just an expletive. In general it seems that
  this verse references the syllable \textit{oṃ}.
 }}

  \maintext{anarthayajña uvāca |}%

  \maintext{anuccāryam asandigdham avicchinnam anākulam |}%

  \maintext{nirmalaṃ sarvagaṃ sūkṣmam akṣaraṃ kim ataḥ param }||\thinspace1:10\thinspace||%
\translation{Anarthayajña replied: That syllable is not to be pronounced, is unquestionable, non-dividable, consistent, spotless, all-pervading and subtle: what could be higher than that? \blankfootnote{1.10 In \textit{pāda} d, I have chosen, somewhat randomly, \textit{kim ataḥ} instead of \textit{kimu tat},
  trying to make sense of 10.9--10.
 }}

  \subchptr{kālapāśaḥ}%

  \trsubchptr{Noose of death and time}%

  \maintext{vigatarāga uvāca |}%

  \maintext{dehī dehe kṣayaṃ yāte bhūjalāgniśivādibhiḥ |}%

  \maintext{yamadūtaiḥ kathaṃ nīto nirālambo nirañjanaḥ }||\thinspace1:11\thinspace||%
\translation{Vigatarāga spoke: When the body disintegrates in the ground, in water, in fire, or [is torn apart] by jackals and other [animals], how is the supportless and spotless soul led [to the netherworld] by Yama's messengers? \blankfootnote{1.11 The word °\textit{śivā}° in \textit{pāda} b is slightly suspect, and could be the result
  of metathesis, from °\textit{viṣā}° {\rm (}`by poison'{\rm )}. Nevertheless, 
  jackals seems appropriate in this context, for they 
  are commonly associated with human corpses, death and the cremation ground
  {\rm (}see e.g. \mycite{Ohnuma2019}{\rm )}. Furthermore, \textit{pāda} b lists phenomena
  that cause the body to disintegrate, and not causes of death; thus the reading \textit{śiva}
  is probably correct.
 }}

  \maintext{kālapāśaiḥ kathaṃ baddho nirdehaś ca kathaṃ vrajet |}%

  \maintext{svargaṃ vā sa kathaṃ yāti nirdeho bahudharmakṛt |}%

  \maintext{etan me saṃśayaṃ brūhi jñātum icchāmi tattvataḥ }||\thinspace1:12\thinspace||%
\translation{How is it bound by the nooses of death [/ time] {\rm (}\textit{kālapāśa}{\rm )}? And if it is bodiless, how can it move? And how does the [soul of a] virtuous [person] {\rm (}\textit{bahudharmakṛt}{\rm )} reach heaven if it has no body? This is my doubt. Teach me. I want to know the truth. \blankfootnote{1.12 The word \textit{kāla} has, as usual, a double meaning here: \textit{kālapāśa}
  is both Yama's noose, and also the limitations and bondage caused by time, 
  as becomes clear at the discussion on the different time units in verses 1.18--30.
 \textit{saṃśaya} seems to be treated as neuter in \textit{pāda} e.
 }}

  \maintext{anarthayajña uvāca |}%

  \maintext{atisaṃśayakaṣṭaṃ te pṛṣṭo 'haṃ dvijasattama |}%

  \maintext{durvijñeyaṃ manuṣyais tu devadānavapannagaiḥ }||\thinspace1:13\thinspace||%
\translation{Anarthayajña spoke: You are asking me about an extremely doubtful and problematic matter, O truest of the twice-born. [This is a matter that] is difficult to understand by humans, and [even] by gods {\rm (}\textit{deva}{\rm )}, demons {\rm (}\textit{dānava}{\rm )} and serpents {\rm (}\textit{pannaga}{\rm )}. \blankfootnote{1.13 Note \textit{te} used for \textit{tvayā} in \textit{pāda} a. Alternatively, taking \textit{te} as genitive, the line
  could be translated as: `I am being asked about a great 
  problem of yours that originates in doubts\dots'
 }}

  \maintext{karmahetu śarīrasya utpatti nidhanaṃ ca yat |}%

  \maintext{sukṛtaṃ duṣkṛtaṃ caiva pāśadvayam udāhṛtam }||\thinspace1:14\thinspace||%
\translation{The cause of both the birth and death of the body is karma. Good and bad deeds are called the two nooses. \blankfootnote{1.14 The MSS give \textit{karmahetu} in \textit{pāda} a overwhelmingly, which could work as a neuter
  \textit{bahuvrīhi} compound picking up both a stem-form \textit{utpatti} and \textit{nidhanaṃ}. 
  \textit{karmahetuḥ} {\rm (}\msCb{\rm )} is grammatically more correct, picking up the feminine \textit{utpatti},
  but a neuter stem-form \textit{utpatti} is unsurprising in this text.
 }}

  \maintext{tenaiva saha saṃyāti narakaṃ svargam eva vā |}%

  \maintext{sukhaduḥkhaṃ śarīreṇa bhoktavyaṃ karmasambhavam }||\thinspace1:15\thinspace||%
\translation{[The soul] goes to hell or heaven [bound and led] by the same [nooses of Yama's messengers, or the karmas]. Happiness and suffering, both arising from karma, are to be experienced by the body. }

  \maintext{hetunānena viprendra dehaḥ sambhavate nṛṇām |}%

  \maintext{yaṃ kālapāśam ity āhuḥ śṛṇu vakṣyāmi suvrata }||\thinspace1:16\thinspace||%
\translation{It is for this reason, O great Brahmin, that the human body is born. Now learn about that which they call the noose of time {\rm (}\textit{kālapāśa}{\rm )}, I shall teach you, O you of great observances. }

  \maintext{na tvayā viditaṃ kiñcij jijñāsyasi kathaṃ dvija |}%

  \maintext{kālapāśaṃ ca viprendra sakalaṃ vettum arhasi }||\thinspace1:17\thinspace||%
\translation{[If] you do not know anything, how could you start your investigation, O twice-born? O great Brahmin, you should know the noose of time {\rm (}\textit{kālapāśa}{\rm )} in its entirety. \blankfootnote{1.17 The variant \textit{jijñāsyasi} seems to be the lectio difficilior as opposed to
  \textit{vijñāsyasi}, but the latter could also work fine here.
 Note how \msM\ {\rm (}agreeing with two paper MSS, \msPaperA\ and \msPaperC, as well as \Ed{\rm )} 
  gives a reading {\rm (}\textit{vaktum arhasi}{\rm )} that is clearly out
  of context. This confirms that while \msM\ comes up with interesting readings, 
  they are mostly to be ignored.
 }}

  \maintext{kalākalitakālaṃ ca kālatattvakalāṃ śṛṇu |}%

  \maintext{truṭidvayaṃ nimeṣas tu nimeṣadviguṇā kalā }||\thinspace1:18\thinspace||%
\translation{Learn about time {\rm (}\textit{kāla}{\rm )} which is divided into digits {\rm (}\textit{kalā}{\rm )}, [i.e. about] the division[s] {\rm (}\textit{kalā}{\rm )} of the entity [called] time {\rm (}\textit{kālatattva}{\rm )}. Two atomic units of time {\rm (}\textit{truṭi}{\rm )} are one twinkling {\rm (}\textit{nimeṣa}{\rm )}. One digit {\rm (}\textit{kalā}, cca. 1.6 second{\rm )} is twice a twinkling. \blankfootnote{1.18 1.18d and 1.19a are problematic in the light of 1.19b, which 
  redefines \textit{kalā} in harmony with the traditional
  interpretation, see e.g. \Arthasastra\ 2.20.33: \textit{trimśatkāṣṭhāḥ kalāḥ}.
  \nocite{Arthasastra1969}
  On divisions of time, see also, e.g., \MANU\ 1.64ff.
  I have calculated 1.6 second for one \textit{kalā} backwards, starting from one day {\rm (}see 1.20ab{\rm )}.
 }}

  \maintext{kalādviguṇitā kāṣṭhā kāṣṭhā vai triṃśatiḥ kalā |}%

  \maintext{triṃśatkalā muhūrtaś ca mānuṣena dvijottama }||\thinspace1:19\thinspace||%
\translation{Two digits {\rm (}\textit{kalā}{\rm )} form one bit {\rm (}\textit{kāṣṭhā}, 3.2 seconds{\rm )}. Thirty bits {\rm (}\textit{kāṣṭhā}{\rm )} make one digit {\rm (}\textit{kalā}?, 1.6 minutes{\rm )}. Thirty digits {\rm (}\textit{kalā}{\rm )} make up one section {\rm (}\textit{muhūrta}, 48 minutes{\rm )} in human terms, O great Brahmin. \blankfootnote{1.19 Understand \textit{mānuṣena} as \textit{mānuṣasaṃkhyayā} {\rm (}1.21d{\rm )}.
 }}

  \maintext{muhūrtatriṃśakenaiva ahorātraṃ vidur budhāḥ |}%

  \maintext{ahorātraṃ punas triṃśan māsam āhur manīṣiṇaḥ }||\thinspace1:20\thinspace||%
\translation{Thirty sections {\rm (}\textit{muhūrta}{\rm )} are known to the wise as one night and day [i.e. a full day]. Thirty days and nights are taught by the wise to be one month. }

  \maintext{samā dvādaśa māsāś ca kālatattvavido janāḥ |}%

  \maintext{śataṃ varṣasahasrāṇi trīṇi mānuṣasaṃkhyayā |}%

  \maintext{ṣaṣṭiṃ caiva sahasrāṇi kālaḥ kaliyugaḥ smṛtaḥ }||\thinspace1:21\thinspace||%
\translation{One year is twelve months [according to] people who know the entity of time. The time span of three hundred and sixty thousand years by human counting is said to be the Kali age {\rm (}\textit{kali\-yuga}{\rm )}. \blankfootnote{1.21 Note how a verb {\rm (}e.g. \textit{iti vadanti, iti prāhur}{\rm )} is missing in \textit{pāda}s ab.
 }}

  \maintext{dviguṇaḥ kalisaṃkhyāto dvāparo yuga saṃjñitaḥ |}%

  \maintext{tretā tu triguṇā jñeyā catuḥ kṛtayugaḥ smṛtaḥ }||\thinspace1:22\thinspace||%
\translation{The Dvāpara age is known to be twice as long as the Kali age. The Tretā age is thrice [as long], the Kṛta age four [times as long as the Kali age]. \blankfootnote{1.22 Note the stem form noun \textit{yuga} in \textit{pāda} b metri causa, or rather the compound \textit{dvāparo-yuga-saṃjñitaḥ}
  {\rm (}the end of \textit{dvāparo} lengthened to avoid the metrical fault of two \textit{laghu}s{\rm )},
  and also \msM's unique but confused readings.
 }}

  \maintext{eṣā caturyugāsaṃkhyā kṛtvā vai hy ekasaptatiḥ |}%

  \maintext{manvantarasya caikasya jñānam uktaṃ samāsataḥ }||\thinspace1:23\thinspace||%
\translation{This is the figure related to the four ages {\rm (}\textit{yuga}{\rm )}. Multiplying it by seventy-one, the knowledge about one time-span of a Manu {\rm (}\textit{manvantara}{\rm )} has been briefly taught. \blankfootnote{1.23 Note the lengthened vowel in °\textit{yugā} {\rm (}metri causa{\rm )}.
 
  The `figure' mentioned in this verse is the sum of the duration of the four \textit{yuga}s, 
  which makes up one \textit{mahāyuga}:
  Kaliyuga = 360,000 years,
  Dvāparayuga = 720,000 years,
  Tretāyuga = 1,080,000 years,
  Kṛtayuga = 1,440,000 years; altogether 3,600,000 years. 71 \textit{mahāyuga}s make up
  a \textit{manvantara} {\rm (}= 255,600,000 years; cf. \Manu\ 1.79{\rm )}. 
  One \textit{kalpa} is 14 \textit{manvantara}s {\rm (}= 3,578,400,000 years{\rm )}. 
  Ten thousand \textit{kalpa}s are one day of Brahmā, and his night is of the same length, which
  would make one full day of Brahmā 71,568,000,000,000 human years. See next verses and,
  e.g., \mycite{GonzalezCosmic}.
 See \VSS\ 21.34ff on \textit{kalpa} etc.
 }}

  \maintext{kalpo manvantarāṇāṃ tu caturdaśa tu saṃkhyayā |}%

  \maintext{daśa kalpasahasrāṇi brahmāhaḥ parikalpitam |}%

  \maintext{rātrir etāvatī proktā munibhis tattvadarśibhiḥ }||\thinspace1:24\thinspace||%
\translation{One \ae on {\rm (}\textit{kalpa}{\rm )} is fourteen \textit{manvantara}s in total. Brahmā's day {\rm (}\textit{brahmāhar}{\rm )} is made up of ten thousand \ae ons {\rm (}\textit{kalpa}{\rm )}. [Brahmā's] night is of the same duration according to the wise who know the truth. \blankfootnote{1.24 The accepted reading \textit{kalpo} in \textit{pāda} a is probably not original.
 \msM\ has a separator sign {\rm (}|o|{\rm )} at the end of \textit{pāda} b, as if a section ended here.
 }}

  \maintext{rātryāgame pralīyante jagat sarvaṃ carācaram |}%

  \maintext{ahāgame tathaiveha utpadyante carācaram }||\thinspace1:25\thinspace||%
\translation{When [Brahmā's] night falls, the whole moving and unmoving universe dissolves. And when [his] daylight arrives, similarly, the moving and unmoving [universe] is born here. \blankfootnote{1.25 The plural form \textit{pralīyante} in \textit{pāda} a is metri causa for \textit{pralīyate},
  perhaps also influencing \textit{utpadyante} {\rm (}for \textit{utpadyate}{\rm )} in \textit{pāda} d,
  which in turn is used here to avoid an iambic pattern
  {\rm (}- - \shortsyllable\ - \shortsyllable\ - \shortsyllable\ -{\rm )}.
  Note a general lack of a sense of grammatical number {\rm (}see p.~\pageref{number}{\rm )}.
 }}

  \maintext{parārdhaparakalpāni atītāni dvijottama |}%

  \maintext{anāgataṃ tathaivāhur bhṛgurādimaharṣayaḥ }||\thinspace1:26\thinspace||%
\translation{One \textit{para} times \textit{parārdha} [number of, i.e. two hundred quadrillion times a hundred quadrillion] \ae ons {\rm (}\textit{kalpa}{\rm )} have passed [thus far], O great Brahmin. Bhṛgu and the other sages say that the future is the same [time span]. \blankfootnote{1.26 On the definition of the numbers \textit{para} and \textit{parārdha}, see verses 1.31--35.
 Note the peculiar compound \textit{bhṛgu-r-ādi-maharṣayaḥ}, for
  \textit{bhṛgvādimaharṣayaḥ}.
 }}

  \maintext{yathārkagrahatārendu bhramato dṛśyate tv iha |}%

  \maintext{kālacakraṃ bhramitvaiva viśramaṃ na ca vidmahe }||\thinspace1:27\thinspace||%
\translation{Just as the sun, the planets, the stars and the moon are perceived in this world as circling around, we, wandering around riding the wheel of time {\rm (}\textit{kālacakra}{\rm )}, can never have a rest. \blankfootnote{1.27 \textit{bhramato} in \textit{pāda} b seems to stand for the neuter participle \textit{bhramat}.
  Alternatively, \textit{bhramato} might mean `erroneously' {\rm (}\textit{bhrama-tas}, abl.{\rm )}, but this would
  make the verse difficult to interpret.
 I have corrected \textit{bhramatvaiva} to the standard form \textit{bhramitvaiva}, although the former
  might conceal a finite verb {\rm (}\textit{bhramāmaḥ}?{\rm )}.
 }}

  \maintext{kālaḥ sṛjati bhūtāni kālaḥ saṃharate punaḥ |}%

  \maintext{kālasya vaśagāḥ sarve na kālavaśakṛt kvacit }||\thinspace1:28\thinspace||%
\translation{Time creates living beings and time destroys them again. Everything is under the control of time. There is nothing that can bring time under control. }

  \maintext{caturdaśa parārdhāni devarājā dvijottama |}%

  \maintext{kālena samatītāni kālo hi duratikramaḥ }||\thinspace1:29\thinspace||%
\translation{Fourteen \textit{parārdha} [fourteen hundred quadrillion] god kings, O Brahmin, have passed with time, for time is difficult to overcome. \blankfootnote{1.29 Note that \textit{samatītāni} {\rm (}neuter{\rm )} most probably picks up \textit{devarājāḥ}
  {\rm (}masculine{\rm )} in this verse, or rather \textit{devarājā} stands for
  \textit{devarājānāṃ} and \textit{samatītāni} picks up °\textit{parā\-rdhāni}. It is not clear to me
  what \textit{devarāja} {\rm (}`god king'{\rm )} means exactly {\rm (}Indra?{\rm )}.
 }}

  \maintext{eṣa kālo mahāyogī brahmā viṣṇuḥ paraḥ śivaḥ |}%

  \maintext{anādinidhano dhātā sa mahātmā namaskuru }||\thinspace1:30\thinspace||%
\translation{Time is [manifest] as a great yogin, as Brahmā, Viṣṇu and supreme Śiva, is beginningless and endless, is the Creator and the great soul. Pay homage [to Time]. }

  \subchptr{parārdhādi}%

  \trsubchptr{\textit{Parārdha} etc.: numbers}%

  \maintext{vigatarāga uvāca |}%

  \maintext{śrutaṃ vai kālacakraṃ tu mukhapadmaviniḥsṛtam |}%

  \maintext{parārdhaṃ ca paraṃ caiva śrotuṃ vaḥ pratidīpitam }||\thinspace1:31\thinspace||%
\translation{Vigatarāga spoke: I have now heard about the `wheel of time' {\rm (}\textit{kālacakra}{\rm )} from [your] lotus mouth. [I wish] to hear about [the terms] \textit{parārdha} and \textit{para} [mentioned above], as elaborated by you. \blankfootnote{1.31 I have corrected the unmetrical \textit{vinisṛtam} in \textit{pāda} b to \textit{viniḥsṛtam}.
  The reading of all manuscripts consulted, \textit{vinisṛtam}, 
  may be considered metrical if we interpret it, loosely, as \textit{vinisritam}.
  Read \textit{tvanmukhapadma}° {\rm (}`your lotus mouth'{\rm )} over the \textit{pāda}-boundary?
  See, e.g., \SIVP\ 2.3.27.6ab:
  \textit{taj jñātvā nikhilaṃ devi śrutvā tvanmukhapaṃkajāt}.
  
 
  \textit{Pāda} d is suspect and my translation tentative.
  \msM's reading in \textit{pāda} d {\rm (}\textit{śrotuṃ naḥ pratidīyatāṃ}{\rm )} might make sense 
  {\rm (}`give it back/repeat it for us to hear'{\rm )}, but it sounds forced,
  as if the scribe tried to come up with a reading that he understood
  better than \textit{śrotuṃ vaḥ pratidīpitam}, the reading of the majority of the witnesses,
  which is in fact not easy to interpret. One would expect a phrase meaning
  `please tell me about these.' Finally, I have decided to take \textit{vaḥ} as 
  instrumental {\rm (}`by you'{\rm )}. Still, a verb is missing.
 }}

  \maintext{anarthayajña uvāca |}%

  \maintext{ekaṃ daśaṃ śataṃ caiva sahasram ayutaṃ tathā |}%

  \maintext{prayutaṃ niyutaṃ koṭim arbudaṃ vṛndam eva ca }||\thinspace1:32\thinspace||%
\translation{Anarthayajña spoke: One, ten, a hundred, a thousand, ten thousand {\rm (}\textit{ayuta}{\rm )}, a hundred thousand {\rm (}\textit{prayuta}{\rm )}, a million {\rm (}\textit{niyuta}{\rm )}, ten million {\rm (}\textit{koṭi}{\rm )}, a hundred million {\rm (}\textit{arbuda}{\rm )}, one billion {\rm (}\textit{vṛnda}, 10$^{\englishfont\tiny 9\thinspace}${\rm )}, \blankfootnote{1.32 See a similar teaching of numbers in \BRAHMANDAPUR\ 3.2.91ff.
 }}

  \maintext{kharvaṃ caiva nikharvaṃ ca śaṅku padmaṃ tathaiva ca |}%

  \maintext{samudro madhyam antaṃ ca parārdhaṃ ca paraṃ tathā }||\thinspace1:33\thinspace||%
\translation{ten billion {\rm (}\textit{kharva}{\rm )}, a hundred billion {\rm (}\textit{nikharva}{\rm )}, one trillion {\rm (}\textit{śaṅku}, 10$^{\englishfont\tiny 12\thinspace}${\rm )}, ten trillion {\rm (}\textit{padma}{\rm )}, a hundred trillion {\rm (}\textit{samudra}{\rm )}, one quadrillion {\rm (}\textit{madhya}, 10$^{\englishfont\tiny 15\thinspace}${\rm )}, ten quadrillion {\rm (}\textit{[an]anta}{\rm )}, a hundred quadrillion {\rm (}\textit{parārdha}{\rm )}, and two hundred quadrillion {\rm (}\textit{para}{\rm )}. \blankfootnote{1.33 Note that \msPaperA\ inserts a line here. See apparatus.
 For \textit{anta} meaning \textit{ananta}, see 1.57. \msM's reading in \textit{pāda} d
  may be a result of an eyeskip to 1.34c.
 }}

  \maintext{sarve daśaguṇā jñeyāḥ parārdhaṃ yāvad eva hi |}%

  \maintext{parārdhadviguṇenaiva parasaṃkhyā vidhīyate }||\thinspace1:34\thinspace||%
\translation{Each should be known as powers of ten up to \textit{parārdha}. The number corresponding to \textit{para} is double that of \textit{parārdha}. }

  \maintext{parāt parataraṃ nāsti iti me niścitā matiḥ |}%

  \maintext{purāṇavedapaṭhitā mayākhyātā dvijottama }||\thinspace1:35\thinspace||%
\translation{There is no higher number than \textit{para}. This is my firm conviction, which is based on my readings of the Purāṇas and the Vedas and [which I have now] taught [to you], O great Brahmin. \blankfootnote{1.35 Note that \Ed\ inserts the line here that \msPaperA\ inserted above. See apparatus.
 }}

  \subchptr{brahmāṇḍam}%

  \trsubchptr{Brahmā's Egg: the Universe}%

  \maintext{vigatarāga uvāca |}%

  \maintext{brahmāṇḍaṃ kati vijñeyaṃ pramāṇaṃ jñāpitaṃ kvacit |}%

  \maintext{kati cāṅguli{-}m{-}ūrdhveṣu sūryas tapati vai mahīm }||\thinspace1:36\thinspace||%
\translation{Vigatarāga spoke: What is the extent of Brahmā's Egg {\rm (}\textit{brahmāṇḍa}{\rm )} [i.e. the universe]? Is it disclosed anywhere? From how many finger's breadths high does the sun heat the earth? \blankfootnote{1.36 The use of the singular next to numerals is one of the hallmarks of the \VSS\ 
  {\rm (}see p.~\pageref{singularwithnumerals}{\rm )}. This means that \textit{pāda} a may well refer to multiple \textit{brahmāṇḍa}s.
  Nevertheless, in the light of \VSS\ 2.2d {\rm (}\textit{pramāṇaṃ tasya vā kati}{\rm )}, I suspect that 
  the first question here could be rendered in slightly more standard Sanskrit as
  \textit{brahmāṇḍasya pramāṇaṃ kati yojanāni vijñeyaṃ}.
  \textit{cāpitaṃ kvacit} in \textit{pāda} b in the witnesses is enigmatic.
  One may conjecture \textit{prāpitaṃ} {\rm (}perhaps: `is it available somewhere?'{\rm )}, 
  The intended form may have been \textit{jñātaṃ kenacit} {\rm (}`is it known by anyone?'{\rm )},
  or \textit{jñāpitaṃ} {\rm (}`is it disclosed somewhere?'{\rm )}. I have chosen the latter,
  to which 1.37 below could be a reply. Of course, \textit{cāpitaṃ} could be analysed as
  \textit{cāpi taṃ} {\rm (}possibly for \textit{cāpi tat}{\rm )}, but that would help little, unless we
  imagine that the question is `and where is it?' {\rm (}\textit{cāpi tat kva}{\rm )}.
 
  
 My emendation of \textit{cāṅguli-mūrdheṣu} to \textit{cāṅguli{-}m{-}ūrdhveṣu} {\rm (}with a hiatus-filler{\rm )} 
  is based on \textit{ūrdhvatas} in 1.60d, which is part of the reply to the question posed in this line.
  In turn, \textit{aṅguli} here triggered a conjecture in 1.60c.
 }}

  \maintext{anarthayajña uvāca |}%

  \maintext{brahmāṇḍānāṃ prasaṃkhyātuṃ mayā śakyaṃ kathaṃ dvija |}%

  \maintext{devās te 'pi na jānanti mānuṣāṇāṃ ca kā kathā }||\thinspace1:37\thinspace||%
\translation{Anarthayajña spoke: How could I enumerate [all the details of] Brahmā's Egg, O twice-born? Even the gods do not know, not to mention humans. \blankfootnote{1.37 One would expect \textit{brahmāṇḍāni} in \textit{pāda} a instead of \textit{brahmāṇḍānāṃ},
  but we should probably understand \textit{brahmāṇḍānāṃ viśeṣān prasaṃkhyātuṃ...}, or rather,
  \textit{brahmāṇḍasya viśeṣān prasaṃkhyātuṃ}.
  The structure noun in genitive + verb meaning `to tell' occurs also, e.g., in 4.69a.
 }}

  \maintext{paryāyeṇa tu vakṣyāmi yathāśakyaṃ dvijottama |}%

  \maintext{brahmaṇā yat purākhyāto mātariśvā yathā tathā }||\thinspace1:38\thinspace||%
\translation{I shall teach [you], as far as I can, in due order and truthfully, that, O great Brahmin, which Mātariśvan was taught by Brahmā in the past. \blankfootnote{1.38 The claim that Brahmā taught Mātariśvan is confirmed in 1.62cd, and
  also, e.g., in \BrahmandaPur\ 3.4.58cd {\rm (}see the apparatus{\rm )}.
 }}

  \maintext{śivāṇḍābhyantareṇaiva sarveṣām iva bhūbhṛtām |}%

  \maintext{daśa nāma diśāṣṭānāṃ brahmāṇḍe kīrtitaṃ śṛṇu }||\thinspace1:39\thinspace||%
\translation{The ten names of all the [cosmic] rulers in each of the eight directions in Brahmā's Egg, [which is] inside Śiva's Egg {\rm (}\textit{śivāṇḍa}{\rm )}, are being taught now, listen. \blankfootnote{1.39 My conjecture in \textit{pāda} b {\rm (}\textit{bhūbhṛtām}{\rm )} is based on the fact that the 
  readings transmitted in the MSS seem unintelligible, and, more importantly, that
  these names are said, in the subsequent verses, to belong to \textit{nāyaka}s {\rm (}`chiefs, lords'{\rm )},
  a possible synonym of \textit{bhūbhṛt} {\rm (}`a king'{\rm )}. Also, it is a minute intervention.
 
  In \textit{pāda} c, understand \textit{diśāṣṭānāṃ} as \textit{diśām aṣṭānāṃ} or \textit{digaṣṭakānāṃ}:
  again, the use of the singular in the proximity of numbers is normal in the \VSS\ {\rm (}\textit{daśa nāma}{\rm )}.
 }}

  \subchptr{bhūbhṛtāṃ nāmāni}%

  \trsubchptr{Names of the cosmic rulers}%

  \subsubchptr{pūrvataḥ}%

  \trsubsubchptr{East}%

  \maintext{sahāsahaḥ sahaḥ sahyo visahaḥ saṃhato 'sabhā |}%

  \maintext{prasaho 'prasahaḥ sānuḥ pūrvato daśa nāyakāḥ }||\thinspace1:40\thinspace||%
\translation{[1] Sahā, [2] Asaha, [3] Saha, [4] Sahya, [5] Visaha, [6] Saṃhata, [7] Asabhā, [8] Prasaha, [9] Aprasaha, [10] Sānu: [these are] the ten Leaders in the East. \blankfootnote{1.40 Note that many of the names here and in the following verses are,
  in the absence of any close parallel passage, rather insecure.
  In order to avoid the repetition of the name Saha, I take the first name here
  as feminine; Asabhā seems also to be a feminine ruler's name. Later on there
  seem to come more feminine names {\rm (}Tejā, Yamunā, Naganā, etc.{\rm )}, therefore it 
  might be correct to interpret some of the names as names of queens.
  What is clear here is that the list evokes the name Sahasrākṣa,
  one of the appellations of Indra, the guardian of the eastern direction.
 }}

  \subsubchptr{āgneye}%

  \trsubsubchptr{South-East}%

  \maintext{prabhāso bhāsano bhānuḥ pradyoto dyutimo dyutiḥ |}%

  \maintext{dīptatejāś ca tejāś ca tejā tejavaho daśa |}%

  \maintext{āgneye tv etad ākhyātaṃ yāmye śṛṇv atha bho dvija }||\thinspace1:41\thinspace||%
\translation{[1] Prabhāsa, [2] Bhāsana, [3] Bhānu, [4] Pradyota, [5] Dyutima, [6] Dyuti, [7] Dīptatejas, [8] Tejas, [9] Tejā, [10] Tejavaha: [these are] the ten [rulers] in the direction of Agni [SE]. Now listen to [the names for] Yama's region, O twice-born. \blankfootnote{1.41 Here, in the region of Agni, the names evidently evoke the image of flames.
 }}

  \subsubchptr{yāmye}%

  \trsubsubchptr{South}%

  \maintext{yamo 'tha yamunā yāmaḥ saṃyamo yamuno 'yamaḥ |}%

  \maintext{saṃyano yamanoyāno yaniyugmā yanoyanaḥ }||\thinspace1:42\thinspace||%
\translation{[1] Yama, [2] Yamunā, [3] Yāma, [4] Saṃyama, [5] Yamuna, [6] Ayama, [7] Saṃyana, [8] Yamanoyāna, [9] Yaniyugmā, [10] Yanoyana. \blankfootnote{1.42 I have chosen the variant \textit{saṃyano} in \textit{pāda} c only to avoid the repetition of
  the name \textit{saṃyama}, and the variant \textit{yanoyanaḥ} in \textit{pāda} d because I suspect that
  most of the names here should begin with \textit{ya}, except for \textit{ayamaḥ}
  in \textit{pāda} b, which is little more than a guess in order to avoid the repetition of \textit{yamaḥ}.
  All the name forms in this verse are to be taken as tentative. The only 
  guiding light is the presence of \textit{ya}, reinforcing a connection with Yama.
 }}

  \subsubchptr{nairṛte}%

  \trsubsubchptr{South-West}%

  \maintext{nagajo naganā nando nagaro naga nandanaḥ |}%

  \maintext{nagarbho gahano guhyo gūḍhajo daśa tatparaḥ }||\thinspace1:43\thinspace||%
\translation{[1] Nagaja, [2] Naganā, [3] Nanda, [4] Nagara, [5] Naga, [6] Nandana, [7] Nagarbha, [8] Gahana, [9] Guhya, [10] Gūḍhaja: [these are] the ten associated with [the South-West]. \blankfootnote{1.43 \textit{naga} in \textit{pāda} b is a stem form noun metri causa.
 \textit{tatparaḥ} in \textit{pāda} d is be another example of a singular form next to a number {\rm (}see 1.39c above{\rm )}.
  Note that the reconstruction of these names is tentative. What is clear here is that the
  initials should be \textit{na} and \textit{ga}, probably suggesting a connection with \textit{nirṛti}, \textit{naraka}s, and \textit{nāga}s.
 }}

  \subsubchptr{vāruṇe}%

  \trsubsubchptr{West}%

  \maintext{vāruṇena pravakṣyāmi śṛṇu vipra nibodha me |}%

  \maintext{babhraḥ setur bhavodbhadraḥ prabhavodbhavabhājanaḥ |}%

  \maintext{bharaṇo bhuvano bhartā daśaite varuṇālayāḥ }||\thinspace1:44\thinspace||%
\translation{I shall teach you [the names] in Varuṇa's region [in the west]. Listen, O Brahmin, learn from me. [1] Babhra, [2] Setu, [3] Bhava, [4] Udbhadra, [5] Prabhava, [6] Udbhava, [7] Bhājana, [8] Bharaṇa, [9] Bhuvana, and [10] Bhartṛ: these ten dwell in Varuṇa's region [in the west]. \blankfootnote{1.44 Varuṇa upholds {\rm (}\textit{bibharti/bharati}{\rm )} the sky and the earth. This could be the reason why 
  these names include \textit{bharaṇa} and \textit{bhartṛ}.
 }}

  \subsubchptr{vāyavye}%

  \trsubsubchptr{North-West}%

  \maintext{nṛgarbho 'suragarbhaś ca devagarbho mahīdharaḥ |}%

  \maintext{vṛṣabho vṛṣagarbhaś ca vṛṣāṅko vṛṣabhadhvajaḥ }||\thinspace1:45\thinspace||%
\translation{[1] Nṛgarbha, [2] Asuragarbha, [3] Devagarbha, [4] Mahīdhara, [5] Vṛṣabha, [6] Vṛṣagarbha, [7] Vṛṣāṅka, [8] Vṛṣabhadhvaja, \blankfootnote{1.45 The connection between \textit{vṛṣa} and the north-west or Vāyu is not evident to me. 
  In a tantric context, a western position is more standard for \textit{vṛṣa}, see e.g.
  \mycitep{Pancavaranastava}{40}.
 }}

  \maintext{jñātavyaś ca tathā samyag vṛṣajo vṛṣanandanaḥ |}%

  \maintext{nāyakā daśa vāyavye kīrtitā ye mayā dvija }||\thinspace1:46\thinspace||%
\translation{[9] Vṛṣaja, and [10] Vṛṣanandana: these are to be known properly as the ten leaders in Vāyu's region [in the north-west], as I taught them, O twice-born. \blankfootnote{1.46 Note how \msM\ deviates here again in a significant way.
 }}

  \subsubchptr{uttare}%

  \trsubsubchptr{North}%

  \maintext{sulabhaḥ sumanaḥ saumyaḥ suprajaḥ sutanuḥ śivaḥ |}%

  \maintext{sataḥ satya layaḥ śambhur daśa nāyakam uttare }||\thinspace1:47\thinspace||%
\translation{[1] Sulabha, [2] Sumana, [3] Saumya, [4] Supraja, [5] Sutanu, [6] Śiva, [7] Sata, [8] Satya, [9] Laya, [10] Śambhu: [these are] the ten leaders in the north. \blankfootnote{1.47 I prefer the form \textit{sumanaḥ} to the more standard \textit{sumanāḥ} {\rm (}\msNc{\rm )} in \textit{pāda} a
  because it suits the slightly irregular language of the \VSS\ {\rm (}see pp.~\pageref{language}{\rm )}
  and because the solitary reading of \msNc\ may well only be an attempt to
  standardise. It is also not inconceivable that \textit{sumanaḥ} stands compounded 
  with \textit{saumyaḥ}.
 Note how \textit{daśa nāyakam} {\rm (}neuter singular for masculine plural{\rm )}
  could again be an example for the use of the singular 
  next to a number in \textit{pāda d}. It seems that here it is the northern region
  that is associated with Śiva, rather than the north-east, the \textit{īśāna} direction, 
  which is occupied by Brahmā in the next verse. 
  {\rm (}In a tantric context, Brahmā is sometimes associated with the north-east, see, e.g.,
  \mycitep{Pancavaranastava}{39}.{\rm )}
  I have left \textit{satya} in stem form.
 }}

  \subsubchptr{īśāne}%

  \trsubsubchptr{North-East}%

  \maintext{indu bindu bhuvo vajra varado vara varṣaṇaḥ |}%

  \maintext{ilano valino brahmā daśeśāneṣu nāyakāḥ }||\thinspace1:48\thinspace||%
\translation{[1] Indu, [2] Bindu, [3] Bhuva, [4] Vajra, [5] Varada, [6] Vara, [7] Varṣaṇa, [8] Ilana, [9] Valina, [10] Brahmā: [these are] the ten rulers in the Īśāna direction [i.e. in the north-east]. \blankfootnote{1.48 I consider \textit{indu, bindu} and \textit{vajra} stem form nouns.
 The north-east seems to be occupied by Brahmā, and by rulers whose names should
  somehow evoke Brahmā's name.
 }}

  \subsubchptr{madhyame}%

  \trsubsubchptr{Center}%

  \maintext{aparo vimalo moho nirmalo mana mohanaḥ |}%

  \maintext{akṣayaś cāvyayo viṣṇur varado madhyame daśa }||\thinspace1:49\thinspace||%
\translation{[1] Apara, [2] Vimala, [3] Moha, [4] Nirmala, [5] Mana, [6] Mohana, [7] Akṣaya, [8] Avyaya, [9] Viṣṇu, [10] Varada: [these are] the ten [leaders] in the centre. \blankfootnote{1.49 Note that the last three lists above have been associated 
  with Śiva, Brahmā and Viṣṇu, respectively, and here, in a layer
  of the text that can be labelled Vaiṣṇava {\rm (}see pp.~\pageref{structure}{\rm )}, it is Viṣṇu that
  seems to occupy a central position. \textit{mana mohanaḥ} {\rm (}or \textit{nirmalonmana}{\rm )} in \textit{pāda} b may
  sound like one single name, but we are forced to separate these two words
  {\rm (}\textit{mana} being in stem form metri causa{\rm )} to arrive at a list of ten names.
 }}

  \subsubchptr{parivārāḥ}%

  \trsubsubchptr{Subordinates}%

  \maintext{sarveṣāṃ daśa{-}m{-}īśānāṃ parivāraśataṃ śatam |}%

  \maintext{śatānāṃ pṛthag ekaikaṃ sahasraiḥ parivāritam }||\thinspace1:50\thinspace||%
\translation{Each of the ten rulers has a retinue of a hundred subordinates. Each one of [these] hundred is surrounded by a thousand subordinates. \blankfootnote{1.50 I take \textit{daśa-m-īśānāṃ} as a split compound {\rm (}\textit{daśeśānāṃ}{\rm )}.
  It is conceivable that each of the above ninety rulers has ten subordinates, 
  therefore each group of ten rulers has a hundred subordinates altogether,
  but the original idea may have been that each one of the above ninety 
  rulers has a hundred subordinates. Alternatively, this verse may only refer to 
  the central group of ten rulers mentioned in 1.49, and each one of them has
  a hundred subordinates.
 }}

  \maintext{sahasreṣu ca ekaikam ayutaiḥ parivāritam |}%

  \maintext{ayutaṃ prayutair vṛndaiḥ prayutaṃ niyutair vṛtam }||\thinspace1:51\thinspace||%
\translation{Each one of the thousand is surrounded by ten thousand [subordinates], the ten thousand is surrounded by a multitude of a hundred thousand, the hundred thousand by a million, \blankfootnote{1.51 We are forced to follow \Ed's reading in \textit{pāda} c in order to make sense of this passage.
  My correction in \textit{pāda} d is motivated by the same. Note that \textit{vṛnda} is not a number in this line. 
  Elsewhere in this chapter \textit{vṛnda} is the word that signifies `a billion.'
 }}

  \maintext{ekaikasya parīvāro niyutaḥ pṛthag eva ca |}%

  \maintext{koṭibhir daśakoṭyena ekaikaḥ parivāritaḥ }||\thinspace1:52\thinspace||%
\translation{[that is] each one has a retinue of a million {\rm (}\textit{niyuta}{\rm )} [subordinates]. [Then those] are surrounded by ten million {\rm (}\textit{koṭi}{\rm )} [subordinates], [they in turn] by a hundred million {\rm (}\textit{daśakoṭi}{\rm )}. \blankfootnote{1.52 It seems that \textit{pāda}s ab repeat what has been stated in 1.51cd.
 °\textit{koṭyena} stands for °\textit{koṭyā} {\rm (}thematisation{\rm )}.
  Note how the scribe of \msM\ gets confused at 1.52c due to an eyeskip and 
  fully regains control only at 1.54b.
 }}

  \maintext{daśakoṭiṣu ekaikaṃ vṛndavṛndabhṛtair vṛtam |}%

  \maintext{vṛndavargeṣu ekaikaṃ kharvabhiḥ parivāritam }||\thinspace1:53\thinspace||%
\translation{Each one of the hundred million is surrounded by a billion {\rm (}\textit{vṛnda}{\rm )} subordinates {\rm (}\textit{bhṛta}{\rm )}. Each one in these groups of a billion {\rm (}\textit{vṛnda}{\rm )} is surrounded by ten billion {\rm (}\textit{kharva}{\rm )} [subordinates]. }

  \maintext{kharvavargeṣu ekaikaṃ daśakharvagaṇair vṛtam |}%

  \maintext{daśakharveṣu ekaikaṃ śaṅkubhiḥ parivāritam }||\thinspace1:54\thinspace||%
\translation{Each in these groups of ten billion {\rm (}\textit{kharva}{\rm )} is surrounded by a hundred billion {\rm (}\textit{daśakharva}{\rm )}. Each of those hundred billion is surrounded by a trillion {\rm (}\textit{śaṅku}{\rm )} [deities]. }

  \maintext{śaṅkubhiḥ pṛthag ekaikaṃ padmena parivāritam |}%

  \maintext{padmavargeṣu ekaikaṃ samudraiḥ parivāritam }||\thinspace1:55\thinspace||%
\translation{Each of those one trillion is surrounded by ten trillion {\rm (}\textit{padma}{\rm )}. Each of those ten trillion is surrounded by a hundred trillion {\rm (}\textit{samudra}{\rm )}. \blankfootnote{1.55 Note that in \textit{pāda} a \textit{śaṅkubhiḥ} stands for \textit{śaṅkūṣu} {\rm (}instrumental for locative{\rm )}.
 }}

  \maintext{samudreṣu tathaikaikaṃ madhyasaṃkhyais tu tair vṛtam |}%

  \maintext{madhyasaṃkhyeṣu ekaikam anantaiḥ parivāritam }||\thinspace1:56\thinspace||%
\translation{And each of those hundred trillion is surrounded by those whose number is one quadrillion {\rm (}\textit{madhya}{\rm )}. Each of those quadrillion is surrounded by ten quadrillion {\rm (}\textit{ananta}{\rm )}. }

  \maintext{ananteṣu ca ekaikaṃ parārdhaparivāritam |}%

  \maintext{parārdheṣu ca ekaikaṃ pareṇa parivāritam |}%

  \maintext{eṣa vai kathito vipra śakyaṃ sāṃkhyam udīritam }||\thinspace1:57\thinspace||%
\translation{Each of those ten quadrillion is surrounded by a hundred quadrillion {\rm (}\textit{parā\-rdha}{\rm )}. Each of those hundred quadrillion is surrounded by two hundred quadrillion {\rm (}\textit{para}{\rm )}. This is how it is taught, O Brahmin. The enumeration [of the rulers of the Brahmāṇḍa] has been taught as much as it is possible. }

  \subchptr{pramāṇam}%

  \trsubchptr{Measurements}%

  \maintext{pramāṇaṃ śṛṇu me vipra saṃkṣepād bruvato mama |}%

  \maintext{candrodaye pūrṇamāsyāṃ vapur aṇḍasya tādṛśam }||\thinspace1:58\thinspace||%
\translation{Listen to me and learn about the measurements [of Brahmā's Egg], O Brahmin, I shall teach [you] in a concise manner. The body of the Egg is like that of [the moon] at moonrise on the day of the full moon. }

  \maintext{koṭikoṭisahasraṃ tu yojanānāṃ samantataḥ |}%

  \maintext{aṇḍānāṃ ca parīmāṇaṃ brahmaṇā parikīrtitam }||\thinspace1:59\thinspace||%
\translation{The whole circumference of the Egg has been declared by Brahmā to be ten million {\rm (}\textit{koṭi}{\rm )} times a thousand times ten million \textit{yojana}s. \blankfootnote{1.59 I suspect that the plural form \textit{aṇḍānāṃ} is accidental and what is meant
  is a singular.
 }}

  \maintext{saptakoṭisahasrāṇi saptakoṭiśatāni ca |}%

  \maintext{viṃśakoṭiṣv aṅgulīṣu ūrdhvatas tapate raviḥ }||\thinspace1:60\thinspace||%
\translation{The Sun shines from the height of seven thousand seven hundred and twenty \textit{koṭi} finger's breadth. \blankfootnote{1.60 This verse is the reply to the question in 1.36cd, which contains the word \textit{aṅguli}:
  this hints at the possibility that the unintelligible \textit{gulmeṣu} transmitted in most of the
  witnesses might be corrupted from \textit{aṅgulīṣu}; hence my conjecture, resulting
  in a \textit{ra-vipulā}.
 }}

  \maintext{pramāṇaṃ nāma saṃkhyā ca kīrtitāni samāsataḥ |}%

  \maintext{brahmāṇḍaṃ cāprameyāṇāṃ lakṣaṇaṃ parikīrtitam }||\thinspace1:61\thinspace||%
\translation{The numbers pertaining to the measurements have been taught in brief. The characteristics of the unmeasurable Brahmāṇḍa[s] have been taught. \blankfootnote{1.61 Note the mixture of different grammatical genders and numbers in this verse. 
  Understand \textit{pramāṇeṣu saṃkhyāḥ kīrtitāḥ samāsataḥ} and 
  \textit{brahmāṇḍānām aprameyānāṃ}, or \textit{brahmāṇḍasyāprameyasya}, which is even metrical.
 }}

  \subchptr{purāṇam}%

  \trsubchptr{Redactors of the Purāṇa{\rm [}s{\rm ]}}%

  \maintext{purāṇāśīsahasrāṇi śatāni dvijasattama |}%

  \maintext{brahmaṇā kathitaṃ pūrṇaṃ mātariśvā yathātatham }||\thinspace1:62\thinspace||%
\translation{O truest of the twice-born, the Purāṇa[s of] 8,000,000 [verses] were taught by [1] Brahmā to [2] Mātariśvan [= Vāyu] in their entirety, in their true form. \blankfootnote{1.62 \textit{Pāda} a should probably be analysed and interpreted as 
  \textit{purāṇam brahmaṇā kathitam}, or rather, \textit{purāṇānām aśītisahasrāṇi śatāni ślokāni
  brahmaṇā kathitāni}.
  Alternatively, \textit{pāda} a may have originally read \textit{purāṇāni sahasrāṇi},
  and then the initial number of verses transmitted by Brahmā is
  a hundred thousand. That the number refers to the number of \textit{śloka}s
  transmitted is confirmed in 1.65d: \textit{viṃśatślokasahasrikam}.
 
  
 
  In \textit{pāda} d, either understand \textit{mātariśvā} {\rm (}nom.{\rm )} as \textit{mātariśvānaṃ} {\rm (}acc.{\rm )} or emend
  \textit{kathitaṃ} to \textit{kathitaḥ} in the sense `Mātariśvan was taught,' echoing 1.38cd:
  \textit{brahmaṇā yat purākhyāto mātariśvā yathā tathā}.
  
 
  On the idea that initially there was only one Purāṇa, see, e.g.,
  \mycitep{RocherPuranas1986}{41ff}.
  Compare the list in the \VSS\ to a list of twenty-eight \textit{vedavyāsa}s, from
  Brahmā to Vyāsa Dvaipāyana, in \VISNUP\ 3.3.10--19,
  taught by Parāśara, the twenty-sixth \textit{vyāsa}
  of this list and our text {\rm (}in the numbering that I add here I follow
  the translation in \mycitep{Visnupurana_tr}{178--179}{\rm )}:
  
 
  \textit{vedavyāsā vyatītā ye aṣṭāviṃśati sattama~|} 
 
  \textit{caturdhā yaiḥ kṛto vedo dvāpareṣu punaḥ punaḥ~||} 
 
  \textit{dvāpare prathame vyastāḥ svayaṃ vedāḥ }[1]\textit{ svayaṃbhuvā~|} 
 
  \textit{dvitīye dvāpare caiva vedavyāsaḥ }[2]\textit{ prajāpati~||} 
 
  \textit{tṛtīye }[3]\textit{ cośanā vyāsaś caturthe ca }[4]\textit{ bṛhaspatiḥ~| } 
 
  [5]\textit{ savitā pañcame vyāsaḥ }[6]\textit{ mṛtyuḥ ṣaṣṭhe smṛtaḥ prabhuḥ~|| } 
 
  \textit{saptame ca }[7]\textit{ tathaivendro }[8]\textit{ vasiṣṭhaś cāṣṭame smṛtaḥ~| } 
 
  [9]\textit{ sārasvataś ca navame }[10]\textit{ tridhāmā daśame smṛtaḥ~|| } 
 
  \textit{ekādaśe tu }[11]\textit{ trivṛṣā }[12]\textit{ bhāradvājas tataḥ param~| } 
 
  \textit{trayodaśe }[13]\textit{ cāntarikṣo }[14]\textit{ varṇī cāpi caturdaśe~|| } 
 
  [15]\textit{ trayyāruṇaḥ pañcadaśe ṣoḍaśe tu }[16]\textit{ dhanaṃjayaḥ~| } 
 
  [17]\textit{ kratuṃjayaḥ saptadaśe }[18]\textit{ ṛṇajyo 'ṣṭādaśe smṛtaḥ~|| } 
 
  \textit{tato vyāso }[19]\textit{ bharadvājo bharadvājāt tu }[20]\textit{ gautamaḥ~| } 
 
  \textit{gautamād uttamo vyāso }[21]\textit{ haryātmā yo 'bhidhīyate~|| } 
 
  \textit{atha haryātmano }[22]\textit{ venaḥ smṛto vājaśravās tu yaḥ~| } 
 
  \textit{somaḥ śuṣmāyaṇas tasmāt }[23]\textit{ tṛṇabindur iti smṛtaḥ~|| } 
 
  [24]\textit{ ṛkṣo 'bhūd bhārgavas tasmād vālmīkir yo 'bhidhīyate~| } 
 
  \textit{tasmād asmatpitā }[25]\textit{ śaktir vyāsas tasmād }[26]\textit{ ahaṃ mune~|| } 
 
  [27]\textit{ jātukarṇo 'bhavan mattaḥ kṛṣṇadvaipāyanas }[28]\textit{ tataḥ~| } 
 
  \textit{aṣṭaviṃśatir ity ete vedavyāsāḥ purātanāḥ~|| }
 
  
 
  Another relevant passage is \BRAHMANDAPUR\ 3.4.58cd--67 {\rm (}\similar\ \VAYUP\ 2.41.58--67{\rm )}.
  Note how Tṛṇabindu is, perhaps by mistake, different from Somaśuṣma/Śuṣmāyaṇa here,
  but, more importantly, note Amitabuddhi of \VSS\ 1.75b appearing at the end of this list:
  
 
  [1] \textit{brahmā dadau śāstram idaṃ purāṇaṃ }[2]\textit{ mātariśvane~|| } 
 
  \textit{tasmāc }[3]\textit{ cośanasā prāptaṃ tasmāc cāpi }[4]\textit{ bṛhaspatiḥ~| } 
 
  \textit{bṛhaspatis tu provāca }[5]\textit{ savitre tadanantaram~|| } 
 
  \textit{savitā }[6]\textit{ mṛtyave prāha mṛtyuś }[7]\textit{ cendrāya vai punaḥ~| } 
 
  \textit{indraś cāpi }[8]\textit{ vasiṣṭāya so 'pi }[9]\textit{ sārasvatāya ca~|| } 
 
  \textit{sārasvatas }[10]\textit{ tridhāmne 'tha tridhāmā ca }[11]\textit{ śaradvate~| } 
 
  \textit{śaradvāṃs tu }[12]\textit{ triviṣṭāya so }[13]\textit{ 'ntarikṣāya dattavān~|| } 
 
  [14]\textit{ carṣiṇe cāntarikṣo vai so 'pi }[15]\textit{ trayyāruṇāya ca~| } 
 
  \textit{trayyāruṇād }[16]\textit{ dhanañjayaḥ sa vai prādāt }[17]\textit{ kṛtañjaye~|| } 
 
  \textit{kṛtañjayāt }[18]\textit{ tṛṇañjayo }[19]\textit{ bharadvājāya so 'py atha~| } 
 
  [20]\textit{ gautamāya bharadvājaḥ so 'pi }[21]\textit{ niryyantare punaḥ~|| } 
 
  \textit{niryyantaras tu provāca tathā }[22]\textit{ vājaśravāya vai~| } 
 
  \textit{sa dadau }[23]\textit{ somaśuṣmāya sa cādāt }[24]\textit{ tṛṇabindave~|| } 
 
  \textit{tṛṇabindus tu }[25]\textit{ dakṣāya dakṣaḥ provāca }[26]\textit{ śaktaye~| } 
 
  \textit{śakteḥ }[27]\textit{ parāśaraś cāpi garbhasthaḥ śrutavān idam~|| } 
 
  \textit{parāśarāj }[28]\textit{ jātukarṇyas tasmād }[29]\textit{ dvaipāyanaḥ prabhuḥ~| } 
 
  \textit{dvaipāyanāt punaś cāpi }[30]\textit{ mayā prāptaṃ dvijottama~|| } 
 
  \textit{mayā caitat punaḥ proktaṃ }[31]\textit{ putrāyāmitabuddhaye~| } 
 
  \textit{ity eva vākyaṃ brahmādiguruṇāṃ samudāhṛtam~||} 
 
  
 
  The list of \textit{vedavyāsa}s in \LINPU\ 1.7.15--18 includes these twenty-five names:
  Kratu, Satya, Bhārgava, Aṅgiras, Savitṛ,
  Mṛtyu, Śatakratu, Vasiṣṭha, Sārasvata, Tridhāman,
  Trivṛta, Śatatejas, Tarakṣu, Āruṇi, Kṛtaṃjaya,
  Ṛtaṃjayo, Bharadvāja, Gautama, Vācaśravas, Tṛṇabindu,
  Rūkṣa, Śakti, Jātūkarṇya, Kṛṣṇa Dvaipāyana.
 }}

  \maintext{vāyunā pāda saṃkṣipya prāptaṃ cośanasaṃ purā |}%

  \maintext{tenāpi pāda saṃkṣipya prāptavāṃś ca bṛhaspatiḥ }||\thinspace1:63\thinspace||%
\translation{Vāyu abridged the verses and then gave [the Purāṇas] to [3] Uśanas. He [Uśanas] also abridged the verses, and [4] Bṛhaspati received them. \blankfootnote{1.63 Note the stem form noun \textit{pāda} twice in this verse and 
  the slightly odd grammatical structure in \textit{pāda} b, {\rm (}\textit{purāṇaṃ}{\rm )} \textit{prāptam uśanasam} {\rm (}`the Purāṇa reached Uśanas'{\rm )},
  as opposed to the solution in \textit{pāda} d with \textit{prāptavān}.
 }}

  \maintext{bṛhaspatis tu provāca sūryaṃ triṃśatsahasrikam |}%

  \maintext{pañcaviṃśatsahasrāṇi mṛtyuṃ prāha divākaraḥ }||\thinspace1:64\thinspace||%
\translation{Bṛhaspati taught 30,000 [verses] to [5] Sūrya [the Sun]. Divākara [= the Sun] taught 25,000 [verses] to [6] Mṛtyu [Death]. \blankfootnote{1.64 \textit{Pāda} a is a ma-\textit{vipulā}, or simply a \textit{pathyā} if \textit{pra} in \textit{provāca}
  does not turn the previous syllable long {\rm (}muta cum liquida{\rm )}.
 }}

  \maintext{ekaviṃśatsahasrāṇi mṛtyunendrāya kīrtitam | }%

  \maintext{indreṇāha vasiṣṭhāya viṃśatślokasahasrikam }||\thinspace1:65\thinspace||%
\translation{Mṛtyu taught 21,000 [verses] to [7] Indra. Indra taught 20,000 verses to [8] Vasiṣṭha. }

  \maintext{aṣṭādaśasahasrāṇi tena sārasvatāya tu |}%

  \maintext{sārasvatas tridhāmāya sahasradaśa sapta ca }||\thinspace1:66\thinspace||%
\translation{And he[, Vasiṣṭha, taught] 18,000 [verses] to [9] Sārasvata. Sārasvata [taught] 17,000 [verses] to [10] Tridhāma[n]. }

  \maintext{ṣoḍaśānāṃ sahasrāṇi bharadvājāya vai tataḥ |}%

  \maintext{daśa pañcasahasrāṇi trivṛṣāya abhāṣata }||\thinspace1:67\thinspace||%
\translation{[He taught] 16,000 verses to [11] Bharadvāja. [Bharadvāja] taught 15,000 verses to [12] Trivṛṣa. }

  \maintext{caturdaśasahasrāṇi antarīkṣāya vai tataḥ |}%

  \maintext{trayyāruṇiṃ sahasrāṇi trayodaśa abhāṣata }||\thinspace1:68\thinspace||%
\translation{[Trivṛṣa] then [taught] 14,000 verses to [13] Antarīkṣa. [Antarīkṣa] taught 13,000 [verses] to [14] Trayyāruṇi. }

  \maintext{trayyāruṇis tu viprendro dhanaṃjayam abhāṣata |}%

  \maintext{dvādaśāni sahasrāṇi saṃkṣipya punar abravīt }||\thinspace1:69\thinspace||%
\translation{Trayyāruṇi, the great Brahmin, having abridged them again, taught 12,000 [verses] to [15] Dhanaṃjaya. }

  \maintext{kṛtaṃjayāya samprāpto dhanaṃjayamahāmuniḥ |}%

  \maintext{kṛtaṃjayād dvijaśreṣṭha ṛṇaṃjayamahātmane }||\thinspace1:70\thinspace||%
\translation{Dhanaṃjaya, the great sage, handed [them] over to [16] Kṛtaṃjaya. [That recension was transmitted] from Kṛtaṃjaya, O best of the twice-born, to [17] noble Ṛṇaṃjaya. \blankfootnote{1.70 Note the odd structure in \textit{pāda}s ab: \textit{dhanaṃjayaḥ kṛtaṃjayāya samprāptaḥ}, 
  for a more standard \textit{dhanaṃjayena {\rm (}\textit{purāṇam}{\rm )} samprāpitaṃ kṛtaṃjayam} 
  {\rm (}`the Purāṇa was transmitted to Kṛtaṃjaya'{\rm )}.
 }}

  \maintext{ṛṇañjayāt punaḥ prāpto gautamāya maharṣiṇe |}%

  \maintext{gautamāc ca bharadvājas tasmād dharyadvatāya tu }||\thinspace1:71\thinspace||%
\translation{Then from Ṛṇaṃjaya it was given to [18] Gautama, the great sage, from Gautama to [19] Bharadvāja, from him to [20] Haryadvata. \blankfootnote{1.71 The structure of \textit{pāda}s ab is as odd as that of 1.70ab. What was
  intended is probably \textit{ṛṇañjayena prāpitaṃ gautamāya}. Many of the
  syntactic oddities in this and other chapters might betray an influence
  of classical Newar. See pp.~\pageref{newar}.
 The name Haryadvata in \textit{pāda} d seem to be a variant on the attested forms
  Haryatvata and Haryātman {\rm (}the latter is in the list of \textit{vedavyāsa}s in 
  \VISNUP\ 3.3.16--17, see note to 1.62 above{\rm )}.
 }}

  \maintext{rājaśravās tataḥ prāptaḥ somaśuṣmāya vai tataḥ |}%

  \maintext{somaśuṣmāt tataḥ prāptas tṛṇabindus tu bho dvija }||\thinspace1:72\thinspace||%
\translation{Then [21] Rājaśravas received it, then [22] Somaśuṣma. Then from Somaśuṣma [23] Tṛṇabindu received it, O twice-born. \blankfootnote{1.72 The syntax is again slightly odd here. The intention may have been
  \textit{prāpitaṃ rājaśravasā somaśuṣmāya... tatas tṛṇabindunā prāptam}.
 }}

  \maintext{tṛṇabindus tu vṛkṣāya vṛkṣaḥ śaktim abhāṣata |}%

  \maintext{śaktiḥ parāśaraṃ prāha jatukarṇāya vai tataḥ }||\thinspace1:73\thinspace||%
\translation{Tṛṇabindu taught it to [24] Vṛkṣa, Vṛkṣa to [25] Śakti [the father of Parāśara]. Śakti taught it to [26] Parāśara, then [Parāśara] to [27] Jatukarṇa. \blankfootnote{1.73 In other list of \textit{vedavyāsa}s, Tṛṇabindu hands the Purāṇas down to 
  Ṛkṣa, Rūkṣa or Dakṣa {\rm (}see note to 1.62 above{\rm )}. \textit{vṛkṣa} in \textit{pāda} a
  is probably a corrupted form.
 The name Jatukarṇa may be a corrupted form of Jātū- or Jātukarṇa.
 }}

  \maintext{dvaipāyanaṃ tu provāca jatukarṇo maharṣiṇam |}%

  \maintext{romaharṣāya samprāpto dvaipāyanamahāmuniḥ }||\thinspace1:74\thinspace||%
\translation{Jatukarṇa taught it to [28] [Vyāsa] Dvaipāyana, the great sage. Dvaipāyana, the great sage, gave it to [29] Romaharṣa. \blankfootnote{1.74 \textit{Pāda}s ab are a \textit{pathyā} if \textit{pra} in \textit{provāca}
  does not turn the previous syllable long {\rm (}muta cum liquida{\rm )}.
 The syntax of \textit{pāda}s cd echoes that of 1.70ab above.
 }}

  \maintext{romaharṣeṇa provāca putrāyāmitabuddhaye |}%

  \maintext{daśa dve ca sahasrāṇi purāṇaṃ samprakāśitam |}%

  \maintext{mānuṣāṇāṃ hitārthāya kiṃ bhūyaḥ śrotum icchasi }||\thinspace1:75\thinspace||%
\translation{Romaharṣa taught the Purāṇa[s] of 12,000 [verses], now fully revealed, to his son, [30] Amitabuddhi, for the benefit of humankind. What else do you wish to know? \blankfootnote{1.75 Romaharṣa is usually considered to be the same person as Sūta, disciple of Vyāsa Dvaipāyana.
  
  
 
  In \BrahmandaPur\ 3.4.67ab {\rm (}\textit{mayā caitat punaḥ proktaṃ putrāyāmitabuddhaye}, see note to 
  1.62 above{\rm )} Amitabuddhi is clearly the name {\rm (}or epithet{\rm )} of Romaharṣa's son. 
  This suggests that the reading \textit{romaharṣāya} in some of the MSS
  in \textit{pāda} a is a mistake for \textit{romaharṣaś ca}, or similar. MS \msM\ is either transmitting an
  otherwise syntactically problematic reading {\rm (}\textit{romaharṣeṇa}{\rm )} that is more 
  original than that in most other witnesses, or \msM's scribe is trying to correct the text.
  Supposing the former, in this case I accepted \msM's reading.
 Note that the extent of the transmitted text {\rm (}12,000 \textit{śloka}s{\rm )} has not changed since Trayyāruṇi
  {\rm (}1.69{\rm )}.
 
 
  Manuscripts \msCc\ and \msM\ place the \textit{iti} of the colophon at the end of the last \textit{śloka}, before
  the \textit{daṇḍa}s, thus: \textit{icchasīti}\thinspace ||O|| {\rm (}\msCc{\rm )} and \textit{icchasi iti}\thinspace ||o|| {\rm (}\msM{\rm )}.
  Note also that \msM\ gives the number of \textit{śloka}s in this chapter, 77, which is close to
  the number of verses this critical edition has produced. The scribe of \msM\ struggled 
  with eyeskips in this chapter, therefore it seems unlikely that he himself
  counted the number of verses he had copied and arrived at this very figure.
  Rather, he copied the number from his exemplar.
 }}

\centerline{\maintext{\dbldanda\thinspace iti vṛṣasārasaṃgrahe brahmāṇḍasaṃkhyā nāmādhyāyaḥ prathamaḥ\thinspace\dbldanda}}
\translation{Here ends the first chapter in the \textit{Vṛṣasārasaṃgraha} called the Description of Śiva's Egg.}

  \chptr{dvitīyo 'dhyāyaḥ}
\addcontentsline{toc}{section}{Chapter 2}
\fancyhead[CO]{{\footnotesize\textit{Translation of chapter 2}}}%

  \trchptr{Chapter Two}%

  \maintext{vigatarāga uvāca |}%

  \maintext{śrutaṃ mayā janāgreṇa brahmāṇḍasya tu nirṇayam |}%

  \maintext{pramāṇaṃ varṇarūpaṃ ca saṃkhyā tasya samāsataḥ }||\thinspace2:1\thinspace||%
\translation{Vigatarāga spoke: I have heard the description of Brahmā's Egg {\rm (}\textit{brahmāṇḍa}{\rm )} from [you,] the best of men, its extent, colour, form, and the numbers associated with it, in a concise manner. \blankfootnote{2.1 It is unlikely that \textit{janāgreṇa} picks up \textit{mayā} {\rm (}`by me, the best of men'{\rm )}.
  Instead, I suppose that this instrumental could be understood as `through the best of man,' 
  or rather, simply taken as an ablative {\rm (}`from the best of men'{\rm )}.
 }}

  \maintext{śivāṇḍeti tvayā prokto brahmāṇḍālayakīrtitaḥ |}%

  \maintext{kīdṛśaṃ lakṣaṇaṃ jñeyaṃ pramāṇaṃ tasya vā kati }||\thinspace2:2\thinspace||%
\translation{You mentioned Śiva's Egg {\rm (}\textit{śivāṇḍa}{\rm )} as taught to be the receptacle of\linebreak Brahmā's Egg {\rm (}\textit{brahmāṇḍa}{\rm )}. What are its characteristics and how much is its extent? \blankfootnote{2.2 The location where the Śivāṇḍa was mentioned is verse 1.39a above.
 }}

  \maintext{kasya vā layanaṃ jñeyaṃ pramāṇaṃ vātra vāsinaḥ |}%

  \maintext{kā vā tatra prajā jñeyā ko vā tatra prajāpatiḥ }||\thinspace2:3\thinspace||%
\translation{And whose dwelling place is it? And [what] is the extent of the inhabitants thereof? What kind of subjects live there? And who is the ruler {\rm (}\textit{prajāpati}{\rm )} there? \blankfootnote{2.3 \textit{vā layanaṃ} in \textit{pāda} a may stand for \textit{vā-ālayanaṃ}, in the sense of \textit{vā-ālayaṃ}.
  The questions in this verse are most probably answered in verses 2.26--33, and if my
  interpretation is correct there, \textit{pramāṇaṃ vātra vāsinaḥ} {\rm (}understand \textit{vāsināṃ}{\rm )} 
  and \textit{pāda}~c should refer to the number of inhabitants in the five regions of Īśāna, Tatpuruṣa, etc.,
  deities who are referred to here in \textit{pāda}s a and possibly d.
 }}

  \subchptr{śivāṇḍasaṃkhyā}%

  \trsubchptr{Summary of the Śivāṇḍa}%

  \maintext{anarthayajña uvāca |}%

  \maintext{śivāṇḍalakṣaṇaṃ vipra na tvaṃ praṣṭum ihārhasi |}%

  \maintext{daivatair api kā śaktir jñātuṃ draṣṭuṃ ca tattvataḥ }||\thinspace2:4\thinspace||%
\translation{Anarthayajña spoke: Please don't ask me about the characteristics of Śiva's Egg {\rm (}\textit{śivāṇḍa}{\rm )}, O Brahmin. How could even the gods have the power to really know and see Śiva's Egg? }

  \maintext{agamyagamanaṃ guhyaṃ guhyād api samuddhitam |}%

  \maintext{na prabhur netaras tatra na daṇḍyo na ca daṇḍakaḥ }||\thinspace2:5\thinspace||%
\translation{The path leading to it is not to be trodden, it is more secret than any secret, and it is lofty. There is no master or servant there, nobody to be punished and no punisher. \blankfootnote{2.5 \textit{samuddhitam} in \textit{pāda} b is suspect. Emending it to \textit{samuddhṛtam} would not be 
  fully satisfactory, and the readings transmitted in the witnesses are problematic. 
  \msM, a MS not collated for this chapter, gives a confusing reading: \textit{sa\uncl{murdhni}dam}. 
  I doubt if \Ed's \textit{samṛddhidam} {\rm (}`yielding success'{\rm )} is the correct reading.
  Perhaps \textit{samudāhṛtam} {\rm (}`declared, talked about as'{\rm )}, or \textit{samāvṛtam} {\rm (}`guarded'{\rm )} was meant.
  It is not inconceivable that \textit{agamyagahanaṃ} in \msCc\ {\rm (}and \msM\msPaperA;
  `it is inaccessible because of its depth'{\rm )} is original and
  is to be contrasted with \textit{samuddhṛtam} {\rm (}`lofty'{\rm )}. One also wonders if
  \textit{guhād} could be the right reading, and in what sense, in \textit{pāda} b.
 }}

  \maintext{na satyo nānṛtas tatra suśīlo no duḥśīlavān |}%

  \maintext{nānṛjur na ca dambhitvaṃ na tṛṣṇā na ca īrṣyatā }||\thinspace2:6\thinspace||%
\translation{There are no truthful or untruthful people there, no moral or immoral people, no crooked people, no hypocrisy, no thirst or envy. \blankfootnote{2.6 Strictly speaking \textit{duḥśīlavān} in \textit{pāda} b is unmetrical; understand or pronounce \textit{duśīlavān}.
 \textit{īrṣyatā} {\rm (}for \textit{īrṣyā}, see 2.7a{\rm )} is a form rarely attested.
 }}

  \maintext{na krodho na ca lobho 'sti na māno 'sti na sūyakaḥ |}%

  \maintext{īrṣyā dveṣo na tatrāsti na śaṭho na ca matsaraḥ }||\thinspace2:7\thinspace||%
\translation{There is no anger or greed there, no arrogance or discontent {\rm (}\textit{[a]sūyaka}{\rm )}, no envy or hatred, no cheaters and no jealousy. \blankfootnote{2.7 \textit{na sūyakaḥ} in \textit{pāda} b stands for \textit{na asūyaka} metri causa.
 }}

  \maintext{na vyādhir na jarā tatra na śoko 'sti na viklavaḥ |}%

  \maintext{nādhamaḥ puruṣas tatra nottamo na ca madhyamaḥ }||\thinspace2:8\thinspace||%
\translation{There is no disease, no aging, no grief and no agitation there, there are no inferior or superior people and there is nobody in-between. }

  \maintext{notkṛṣṭo mānavas tasmin striyaś caiva śivālaye |}%

  \maintext{na nindā na praśaṃsāsti matsarī piśuno na ca }||\thinspace2:9\thinspace||%
\translation{There are no privileged men or women there in Śiva's abode, no reproach or praise, no selfish or treacherous people. }

  \maintext{garvadarpaṃ na tatrāsti krūramāyādikaṃ tathā |}%

  \maintext{yācamāno na tatrāsti dātā caiva na vidyate }||\thinspace2:10\thinspace||%
\translation{There is no pride or arrogance there, no cruelty or trickery and so on. There are no beggars and no donors there. }

  \maintext{anarthī vraja tatrasthaḥ kalpavṛkṣasamāśritaḥ |}%

  \maintext{na karma nāpriyas tatra na kaliḥ kalaho na ca }||\thinspace2:11\thinspace||%
\translation{Go without material desires {\rm (}\textit{anarthin}{\rm )}. Being there you'll be resting under a wishing tree. There is no karma there and no enemy. No Kali age is there and there is no fighting. \blankfootnote{2.11 Note the term \textit{anartī} in \textit{pāda} a: it might have something to do 
  with non-material sacrifice {\rm (}\textit{anarthayajña}{\rm )}, the topic of chapter 11,
  and with our interlocutor Anarthayajña.
 My emendation in \textit{pāda} c from \textit{na priyas} {\rm (}`no lover/husband'{\rm )} to \textit{nāpriyas} {\rm (}`no enemy'{\rm )} 
  might not be necessary but it seems more meaningful than the transmitted readings.
 }}

  \maintext{dvāparo na ca na tretā kṛtaṃ cāpi na vidyate |}%

  \maintext{manvantaraṃ na tatrāsti kalpaś caiva na vidyate }||\thinspace2:12\thinspace||%
\translation{There is no Dvāpara age or Tretā or Kṛta. There are no Manu-eras {\rm (}\textit{manvantara}{\rm )} there and no \ae ons {\rm (}\textit{kalpa}{\rm )}. \blankfootnote{2.12 On \textit{manvantara}s and \textit{kalpa}s, see 1.22--23 above.
 }}

  \maintext{āhūtasamplavaṃ nāsti brahmarātridinaṃ tathā |}%

  \maintext{na janmamaraṇaṃ tatra āpadaṃ nāpnuyāt kvacit }||\thinspace2:13\thinspace||%
\translation{No universal floods of destruction come, and there are no days and nights of Brahmā. There is no birth and death there and one never encounters catastrophes. \blankfootnote{2.13 \textit{āhūtasamplava} for the more widely attested form \textit{ābhūtasamplava} occurs, e.g.,
  in some MSS transmitting \SDHS\ 10.77 and 81 {\rm (}see \mycite{SDhS10_ed}{\rm )}.
 }}

  \maintext{na cāśāpāśabaddho 'sti rāgamohaṃ na vidyate |}%

  \maintext{na devā nāsurās tatra na yakṣoragarākṣasāḥ }||\thinspace2:14\thinspace||%
\translation{Nobody is tied to the noose of hope and there is no passion or delusion. There are no gods and demons there and no Yakṣas, Serpents and Rākṣasas. }

  \maintext{na bhūtā na piśācāś ca gandharvā ṛṣayas tathā |}%

  \maintext{tārāgrahaṃ na tatrāsti nāgakiṃnaragāruḍam }||\thinspace2:15\thinspace||%
\translation{There are neither Ghosts nor Piśācas, no Gandharvas and no Ṛṣis. There are no planets there, no Nāgas, Kiṃnaras or Garuḍa-like creatures. }

  \maintext{na japo nāhnikas tatra nāgnihotrī na yajñakṛt |}%

  \maintext{na vrataṃ na tapaś caiva na tiryaṅnarakaṃ tathā }||\thinspace2:16\thinspace||%
\translation{There are no recitations or daily rituals there, nobody performs the Agnihotra and there are no sacrificers. There are no religious observances and no austerities and no `animal hell'. \blankfootnote{2.16 The phrase of \textit{tiryaṅnaraka} appears in \MBH\ 3.181.18ab: 
  \textit{aśubhaiḥ karmabhiḥ pāpās tiryaṅnarakagāminaḥ}. Here \mycite{GanguliMBh} translates \textit{tiryaṅ} 
  separately as `in a crooked way,' but I suspect that in the \VSS\ \textit{tiryaṅnaraka} has
  more to do with \textit{tiraggati}, being reduced to animal existence, being reborn as an animal or entering a 
  hell in animal form. 
  Cf. \MBH\ Suppl. 13.15.2615--16:
 
  \textit{nṛṣu janma labhante ye karmaṇā madhyamāḥ smṛtāḥ}\thinspace |
 
  \textit{tiryaṅnarakagantāro hy adhamās te narādhamāḥ}\thinspace || 
 
  and \UMS\ 6.1:
 
  \textit{avamanyanti ye viprān sarvaloke namaskṛtān}\thinspace |
 
  \textit{narakaṃ yānti te sarve tiryagyoniṃ vrajanti ca~||}. 
  
 I suspect that \textit{nātirya}° in the witnesses is only a scribal mistake for \textit{na tirya}°.
 }}

  \maintext{tasyeśānasya devasya aiśvaryaguṇavistaram |}%

  \maintext{api varṣaśatenāpi śakyaṃ vaktuṃ na kenacit }||\thinspace2:17\thinspace||%
\translation{Nobody would be able to tell the extent of the qualities of the god Īśāna's powers, not even in a hundred years. \blankfootnote{2.17 My translation of \textit{aiśvaryaguṇa}° is tentative. It could be taken as a \textit{dvandva} compound
  {\rm (}e.g. `supremacy and qualities'{\rm )}. The expression \textit{sarva}° or \textit{aṣṭaiśvaryaguṇopeta}
  occurs frequently, e.g. in \SIVP\ 7.2.8.28ab and \SKANDAP\ 55.30cd, and \SDHU\ 2.6, 79, 125, 127,
  with \textit{aiśvarya} most probably referring to the eight \textit{siddhi}s \textit{aṇiman, laghiman} etc.
  De Simini {\rm (}2016a, 386{\rm )},\nocite{DeSiminiGods2016} e.g., translates \textit{sarvaiśvaryaguṇopetaḥ} in \SDHU\ 2.127 as
  `endowed with all the qualities of lordship.'
 }}

  \maintext{harecchāprabhavāḥ sarve paryāyeṇa bravīmi te |}%

  \maintext{devamānuṣavarjyāni vṛkṣagulmalatādayaḥ }||\thinspace2:18\thinspace||%
\translation{All are born by Hara's wish. I shall teach [them] to you one by one, gods and people, and trees, bushes, creepers, etc. \blankfootnote{2.18 Treat \textit{pāda} a as if the object of \textit{bravīmi}.
 Note the gender confusion in this verse. In \textit{pāda} c, °\textit{varjyāni} is suspect. 
  I take it as if it stood for \textit{vargāḥ/vargāṇi}, and not in the sense of `excluding,'
  because gods and people are in fact, albeit vaguely, mentioned below.
 }}

  \maintext{parārdhadviguṇotsedho vistāraś ca tathāvidhaḥ |}%

  \maintext{anekākārapuṣpāṇi phalāni ca manoharam }||\thinspace2:19\thinspace||%
\translation{The height [of the Śivāṇḍa] is two \textit{parārdha}s, and [its] width is the same. There are lovely flowers of different forms [there] and also lovely fruits. \blankfootnote{2.19 I understand \textit{pāda} a as \textit{parārdhadviguṇa utsedho}, i.e. as an example of double \textit{sandhi}.
  On the other hand, °\textit{sedho} is only my conjecture, and may refer to something else than the Śivāṇḍa.
  Note the number confusion in \textit{pāda} d, and also that two \textit{parārdha}s is one \textit{para}, 
  the highest possible number according to verses 1.34--35 above. The number may refer to
  any unit of length, but 2.23 below suggests that it is \textit{yojana}s.
 }}

  \maintext{anye kāñcanavṛkṣāṇi maṇivṛkṣāṇy athāpare |}%

  \maintext{pravālamaṇiṣaṇḍāś ca padmarāgaruhāṇi ca }||\thinspace2:20\thinspace||%
\translation{There are also golden trees and also gem trees, coral gem thickets and ruby plants. \blankfootnote{2.20 Note that both \textit{anye} and \textit{apare} here pick up neuter nouns {\rm (}gender confusion{\rm )}.
 }}

  \maintext{svādumūlaphalāḥ skandhalatāviṭapapādapāḥ |}%

  \maintext{kāmarūpāś ca te sarve kāmadāḥ kāmabhāṣiṇaḥ }||\thinspace2:21\thinspace||%
\translation{There are tasty roots and fruits and trees with creepers on their twigs. All are shape-shifters and they fulfill desires and they whisper seductively. \blankfootnote{2.21 My conjectures in \textit{pāda}s ab result in a compound spanning the c\ae sura, which
  may have been the reason why the line got corrupted.
 }}

  \maintext{tatra vipra prajāḥ sarve anantaguṇasāgarāḥ |}%

  \maintext{tulyarūpabalāḥ sarve sūryāyutasamaprabhāḥ }||\thinspace2:22\thinspace||%
\translation{There [in the Śivāṇḍa], O Brahmin, all the subjects are oceans of endless virtues. They are all equally beautiful and strong, and they shine like millions of suns. }

  \maintext{parārdhadvayavistāraṃ parārdhadvayam āyatam |}%

  \maintext{parārdhadvayavikṣepaṃ yojanānāṃ dvijottama }||\thinspace2:23\thinspace||%
\translation{[Śiva's Egg] is two \textit{parārdha}s long and two \textit{parā\-rdha}s wide, and two \textit{parā\-rdha}s is its [vertical] extension, [measured] in \textit{yojana}s, O great Brahmin. }

  \maintext{aiśvaryatvaṃ na saṃkhyāsti balaśaktiś ca bho dvija |}%

  \maintext{adhordhvo na ca saṃkhyāsti na tiryañ caiti kaścana }||\thinspace2:24\thinspace||%
\translation{[Īśāna's] powers cannot be expressed by numbers, neither can [His] powerfulness, O twice-born. [In fact, the distances in Śiva's Egg] downwards and upwards cannot be expressed by numbers. Nobody can traverse through it. \blankfootnote{2.24 \textit{Pāda}s ab are an echo of 2.17b.
 \textit{kaścana} in \textit{pāda} d forces us to accept the reading in \msNapcorr\msNc\ {\rm (}\textit{caiti}{\rm )},
  as opposed to \textit{ceti} in the remaining witnesses. Alternatively translate as
  '[The distances in Śiva's Egg] downwards and 
  upwards and horizontally cannot be expressed by numbers, some people say.'
 }}

  \maintext{śivāṇḍasya ca vistāram āyāmaṃ ca na vedmy aham |}%

  \maintext{bhogam akṣaya tatraiva janmamṛtyur na vidyate }||\thinspace2:25\thinspace||%
\translation{[In reality,] I do not know the length and width of Śiva's Egg. Enjoyment is undecaying there, and there is no birth or death there. \blankfootnote{2.25 \textit{Pāda} c is transmitted in an unmetrical form and with a gender problem in the witnesses
  {\rm (}\textit{bhogam akṣayas}, including \msPaperA, not collated here{\rm )}, hence my emendation using a stem form noun,
  a phenomenon frequently seen in this text. But note that \textit{bhoga} is normally masculine;
  there might be a hiatus-filler in-between: \textit{bhoga-m-akṣaya}.
 }}

  \maintext{śivāṇḍamadhyam āśritya gokṣīrasadṛśaprabhāḥ |}%

  \maintext{parārdhaparakoṭīnām īśānānāṃ smṛtālayaḥ }||\thinspace2:26\thinspace||%
\translation{In the centre of Śiva's Egg, [creatures] shine like cow's milk. [It is] said to be the region {\rm (}\textit{ālaya}{\rm )} of those belonging to Iśāna, one and a half \textit{para} crore in number. \blankfootnote{2.26 Note the stem form \textit{smṛta} in \textit{pāda} d {\rm (}cf. 2.29d{\rm )}. I understand \textit{īśānānāṃ} as \textit{aiśānānāṃ}.
  
 
  Īśāna is traditionally the upward-looking face of Śiva, his region is positioned in the centre here.
  Note that the somewhat cryptic third \textit{pāda}s here and in the coming verses
  may or may not refer to the number of creatures living in the given region. 
  They may tell us about the extent of the given region, although the numbers are much
  higher than what one would expect after verse 2.23.
 }}

  \maintext{bālasūryaprabhāḥ sarve jñeyās tatpuruṣālaye |}%

  \maintext{parārdhaparakoṭīnāṃ pūrvasyāṃ diśam āśritāḥ }||\thinspace2:27\thinspace||%
\translation{They are all like the rising sun in the region of Tatpuruṣa. They are one and a half \textit{para} crore in number, living in the east. \blankfootnote{2.27 The genitive of \textit{parārdhaparakoṭīnāṃ} is baffling here and in the coming verses,
  but I suspect that again the expression gives the number of subjects living in the given region.
  \textit{pūrvasyāṃ} is meant to mean \textit{pūrvāṃ} {\rm (}cf. \textit{dakṣiṇāṃ, paścimāṃ}, and \textit{uttarāṃ} in the next verses{\rm )};
  note how \msNb\ tries to save the construction by reading \textit{diśi-m-}.
 
  This verse conforms to the traditional view that Śiva's Tatpuruṣa-face
  is looking towards the eastern direction.
 }}

  \maintext{bhinnāñjanaprabhāḥ sarve dakṣiṇāṃ diśam āśritāḥ |}%

  \maintext{parārdhaparakoṭīnām aghorālayam āśritāḥ }||\thinspace2:28\thinspace||%
\translation{Everybody is like collyrium in the southern direction, in the region of Aghora, one and a half \textit{para} crore in number. \blankfootnote{2.28 Note the Aiśa form \textit{diśiṃ} in \msCb\ {\rm (}see, e.g., \mycitep{KissBraYa}{83, \S 26}{\rm )},
  and that Aghora is indeed usually south-facing.
 }}

  \maintext{kundenduhimaśailābhāḥ paścimāṃ diśam āśritāḥ |}%

  \maintext{parārdhaparakoṭīnāṃ sadya{-}m{-}iṣṭālayaḥ smṛtaḥ }||\thinspace2:29\thinspace||%
\translation{In the western direction, they are like jasmine, the moon, like snowy rocks. Sadyojāta's lovely region is [home] to one and a half \textit{para} crore [people]. \blankfootnote{2.29 Note the Aiśa form \textit{diśiṃ} in \msNc\ in \textit{pāda} b.
  In \textit{pāda} d, we may presuppose the presence of a \textit{sandhi}-bridge: \textit{sadya-m-iṣṭālayaḥ}.
  Sadyojāta is traditionally associated with the western direction.
 }}

  \maintext{kuṅkumodakasaṃkāśā uttarāṃ diśam āśritāḥ |}%

  \maintext{parārdhaparakotīnāṃ vāmadevālayaḥ smṛtaḥ }||\thinspace2:30\thinspace||%
\translation{In the northern direction, they are like saffron in water. Vāmadeva's region is [home] to one and a half \textit{para} crore [people]. \blankfootnote{2.30 Note the Aiśa form \textit{diśiṃ} in \msCa\ in \textit{pāda} b.
 Vāmadeva is traditionally associated with the western direction.
 }}

  \maintext{īśānasya kalāḥ pañca vaktrasyāpi catuṣkalāḥ |}%

  \maintext{aghorasya kalā aṣṭau vāmadevās trayodaśa }||\thinspace2:31\thinspace||%
\translation{Īśāna has five parts {\rm (}\textit{kalā}{\rm )}, [his Tatpuruṣa] face has four. Aghora has eight, and there are thirteen Vāmadeva[-\textit{kalā}]s. \blankfootnote{2.31 Note how \textit{vaktrasya} should refer to Śiva's Tatpuruṣa-face, 
  given that the text lists Śiva's five faces: Īśāna, Tatpuruṣa, Aghora, Vāmadeva, Sadyojāta.
 }}

  \maintext{sadyaś cāṣṭau kalā jñeyāḥ saṃsārārṇavatārakāḥ |}%

  \maintext{aṣṭatriṃśat kalā hy etāḥ kīrtitā dvijasattama }||\thinspace2:32\thinspace||%
\translation{Sadyojāta has eight parts. These parts, altogether thirty-eight, which\linebreak liberate us from the ocean of existence, have been taught, O truest Brahmin. \blankfootnote{2.32 Note \textit{sadyaś} in \textit{pāda} a for \textit{sadyasaś} or \textit{sadyojātasya}.
 }}

  \maintext{saṃkhyā varṇā diśaś caiva ekaikasya pṛthak pṛthak |}%

  \maintext{pūrvoktena vidhānena bodhavyās tattvacintakaiḥ }||\thinspace2:33\thinspace||%
\translation{Those who explore the truth should know the numbers, the colours, and directions associated with each one [of Śiva's faces] in the way taught above. }

  \maintext{śivāṇḍagamanākṛṣṭyā śivayogaṃ sadābhyaset |}%

  \maintext{śivayogaṃ vinā vipra tatra gantuṃ na śakyate }||\thinspace2:34\thinspace||%
\translation{If one has the intention to go to the Śiva's Egg, one should practise\linebreak Śiva-yoga regularly. Without Śiva-yoga, O Brahmin, it is impossible to go there. \blankfootnote{2.34 °\textit{ākṛṣṭyā} {\rm (}`because of being drawn to' or `with the intention of'{\rm )} in \textit{pāda} a might be corrupt.
  Perhaps understand °\textit{ākṛṣṭaḥ} {\rm (}`he who is attracted to'{\rm )}.
 }}

  \maintext{aśvamedhādiyajñānāṃ koṭyāyutaśatāni ca |}%

  \maintext{kṛcchrāditapa sarvāṇi kṛtvā kalpaśatāni ca |}%

  \maintext{tatra gantuṃ na śakyeta devair api tapodhana }||\thinspace2:35\thinspace||%
\translation{[Even] by [performing] millions of sacrifices such as the Aśvamedha, or by performing all the difficult austerities such as the \textit{kṛcchra} for a hundred \textit{kalpa}s, it is impossible to get there even for the gods, O great ascetic. \blankfootnote{2.35 Understand \textit{kṛcchrāditapa sarvāṇi} as \textit{kṛcchrāditapāṃsi sarvāṇi}. It can be 
  considered an instance of the use of a stem form noun.
  On the specific penance called \textit{kṛcchra}, which involves having to sleep
  in a sitting position, see, e.g., \mycitep{KaneHistory}{120}.
 }}

  \maintext{gaṅgādisarvatīrtheṣu snātvā taptvā ca vai punaḥ |}%

  \maintext{tatra gantuṃ na śakyeta ṛṣibhir vā mahātmabhiḥ }||\thinspace2:36\thinspace||%
\translation{By [merely] bathing and performing austerities at all the sacred places such as the Gaṅgā, even the honorable Ṛṣis will not be able to get there. }

  \maintext{saptadvīpasamudrāṇi ratnapūrṇāni bho dvija |}%

  \maintext{dattvā vā vedaviduṣe śraddhābhaktisamanvitaḥ |}%

  \maintext{tatra gantuṃ na śakyeta vinā dhyānena niścayaḥ }||\thinspace2:37\thinspace||%
\translation{Or [even] by donating the oceans of the seven islands with all their gems to a Veda expert, O Brahmin, with faith and devotion, one will not be able to go there without meditation. [This is a] certainty. }

  \maintext{svadehān māṃsam uddhṛtya dattvārthibhyaś ca niścayāt |}%

  \maintext{svadāraputrasarvasvaṃ śiro 'rthibhyaś ca yo dadet |}%

  \maintext{na tatra gantuṃ śakyeta anyair vāpi suduṣkaraiḥ }||\thinspace2:38\thinspace||%
\translation{He who carves out flesh from his own body and gives it without hesitation to those who are in need of it, or he who gives away his wife, his son and his possessions or his own head to those in need, or he who [performs] other difficult deeds, will not be able to go there [by merely doing these]. \blankfootnote{2.38 For examples of legends that involve donating one's own flesh, see \VSS\ 17.37--40 {\rm (}Uśīnara, Alarka{\rm )}.
  See also 6.26.
 Examples of people donating family members include \VSS\ chapter 12 {\rm (}Vipula giving away his wife{\rm )},
  and 17.41 {\rm (}Sudāsa's story{\rm )}.
 }}

  \maintext{yajñatīrthatapodānavedādhyayanapāragaḥ |}%

  \maintext{brahmāṇḍāntasya bhogāṃs tu bhuṅkte kālavaśānugaḥ }||\thinspace2:39\thinspace||%
\translation{He who has completed the sacrifices, the pilgrimages, the austerities, the donations, the study of the Vedas, will experience [only] those enjoyments that Brahmā's Egg offers, still being subject to time and death. }

  \maintext{kālena samapreṣyeṇa dharmo yāti parikṣayam |}%

  \maintext{alātacakravat sarvaṃ kālo yāti paribhraman |}%

  \maintext{traikālyakalanāt kālas tena kālaḥ prakīrtitaḥ }||\thinspace2:40\thinspace||%
\translation{Dharma decays tossed forward by time. Time flies moving everything round and round like a circle of burning coal. Time is called \textit{kāla} because of the waves {\rm (}\textit{kalana}{\rm )} of the three divisions of time [past, present, future]. \blankfootnote{2.40 Notice the muta cum liquida licence in \textit{pāda} a: \textit{samapre}° renders as short-short-long.
  I take \textit{samapreṣyena} as if it read \textit{sampreṣito}, picking up \textit{dharmo}; otherwise
  it is difficult to make sense of it.
 As Kenji Takahashi pointed out to me, \mycite{Fitzgerald_Alatacakra2012} is
  a good starting point to understand the implication of \textit{alātacakra}, 
  `a single, rapidly twirled torch creat[ing] the illusion of an apparently real, continuous circle'
  {\rm (}ibid., p. 777{\rm )}. The function of \textit{sarvaṃ} in \textit{pāda} a becomes clear only if
  we understand \textit{paribhraman} in a causative sense {\rm (}for \textit{paribhramayan}{\rm )}.
 One cannot help noticing that this verse would be in a more fitting context after verse 1.30,
  at the end of a section on \textit{kāla}. On the other hand, it leads us to the next topic, Dharma,
  smoothly.
 }}

\centerline{\maintext{\dbldanda\thinspace iti vṛṣasārasaṃgrahe śivāṇḍasaṃkhyā nāmādhyāyo dvitīyaḥ\thinspace\dbldanda}}
\translation{Here ends the second chapter in the \textit{Vṛṣasārasaṃgraha} called the Description of the Śiva's Egg.}

  \chptr{tṛtīyo 'dhyāyaḥ}
\addcontentsline{toc}{section}{Chapter 3}
\fancyhead[CO]{{\footnotesize\textit{Translation of chapter 3}}}%

  \trchptr{  Chapter Three }%

  \subchptr{dharmapravacanam}%

  \trsubchptr{Exposition of Dharma}%

  \maintext{vigatarāga uvāca |}%

  \maintext{kimarthaṃ dharmam ity āhuḥ katimūrtiś ca kīrtyate |}%

  \maintext{katipādavṛṣo jñeyo gatis tasya kati smṛtāḥ }||\thinspace3:1\thinspace||%
\translation{Vigatarāga spoke: Why do they call it Dharma? And how many embodiments {\rm (}\textit{mūrti}{\rm )} is it known to have? It is known as a bull: how many legs does it have? How many are its paths? \blankfootnote{3.1 For the correct interpretation of \textit{pāda} a, namely to decide whether these questions
  focus on the bull of Dharma {\rm (}`Why do they call the bull Dharma?'{\rm )} 
  or Dharma itself/himself {\rm (}`Why is Dharma called Dharma?'{\rm )}, see 
  the end of the previous chapter, where \textit{dharma} was mentioned {\rm (}2.40b{\rm )},
  and to which the present verse is a reaction, i.e. the focus is not so much the bull
  but Dharma. Compare also \MBH\ 12.110.10--11:
  
 
  \textit{prabhāvārthāya bhūtānāṃ dharmapravacanaṃ kṛtam}\thinspace |
 
  \textit{yat syād ahiṃsāsaṃyuktaṃ sa dharma iti niścayaḥ}\thinspace ||
 
  \textit{dhāraṇād dharma ity āhur dharmeṇa vidhṛtāḥ prajāḥ}\thinspace |
 
  \textit{yat syād dhāraṇasaṃyuktaṃ sa dharma iti niścayaḥ}\thinspace ||
 
  
 
  Note the similarities of the above passage from the \MBH\ with this present \VSS\ chapter:
  the phrase \textit{dharma ity āhur},
  the fact that the present chapter from verse 18 on is actually a chapter on \textit{ahiṃsā},
  and that the etymological explanation involves the word [\textit{ā}]\textit{dhāraṇa} in
  both cases. These have led me to think that in \textit{pāda}s ab of the verse in the \VSS\
  it is Dharma that is the focus of the inquiry, as in the \MBH, and not the bull.
 
  
 Understand \textit{pāda} d as \textit{gatayas tasya kati smṛtāḥ}. I have accepted
  \textit{smṛtāḥ} because this plural at the end of the phrase
  signals that \textit{gatis} is meant to be plural,
  similarly to what happens in 3.6cd {\rm (}\textit{tasya patnī... mahābhāgāḥ}{\rm )}.
  On this, see p.~\pageref{number} in the Introduction.
  On Dharma as a bull, see Introduction, pp.~\pageref{bull}.
 }}

  \maintext{kautūhalaṃ mamotpannaṃ saṃśayaṃ chindhi tattvataḥ |}%

  \maintext{kasya putro muniśreṣṭha prajās tasya kati smṛtāḥ }||\thinspace3:2\thinspace||%
\translation{I have become curious [about these questions]. Put an end to my doubts for good. Whose son is [Dharma], O best of sages? How many children does he have? }

  \maintext{anarthayajña uvāca |}%

  \maintext{dhṛtir ity eṣa dhātur vai paryāyaḥ parikīrtitaḥ |}%

  \maintext{ādhāraṇān mahattvāc ca dharma ity abhidhīyate  }||\thinspace3:3\thinspace||%
\translation{Anarthayajña spoke: Well, \textit{dhṛti} {\rm (}`firmness'{\rm )}, [of] the [same] verbal root [as \textit{dharma}], is said to be [its] synonym. It is called \textit{dharma} because it supports {\rm (}\textit{āDHĀRaṇa}{\rm )} and because it is great {\rm (}\textit{MAhattva}{\rm )}. \blankfootnote{3.3 For similar Purāṇic passages on the etymology of \textit{dharma}, see the apparatus to
  this verse.
 
  The insertion `[of] the [same]' in my translation solves the problem of a noun {\rm (}\textit{dhṛti}{\rm )} seemingly
  being considered a verbal root {\rm (}\textit{dhātu}{\rm )} here.
  For similar passages with nominal stems apparently being treated as \textit{dhātu}s, see, e.g., 
  \VAYUP\ 3.17cd:
  \textit{bhāvya ity eṣa dhātur vai bhāvye kāle vibhāvyate};
  \VAYUP\ 3.19cd {\rm (}= \BRAHMANDAPUR\ 1.38.21ab{\rm )}:
  \textit{nātha ity eṣa dhātur vai dhātujñaiḥ pālane smṛtaḥ};
  \LINPU\ 2.9.19:
  \textit{bhaja ity eṣa dhātur vai sevāyāṃ parikīrtitaḥ}.
 }}

  \maintext{śrutismṛtidvayor mūrtiś catuṣpādavṛṣaḥ sthitaḥ |}%

  \maintext{caturāśrama yo dharmaḥ kīrtitāni manīṣibhiḥ }||\thinspace3:4\thinspace||%
\translation{The four-legged Bull is the embodiment of both Śruti and Smṛti. It is Dharma as made up of the four \textit{āśrama}s. \blankfootnote{3.4 A similar image of the legs of the Bull of Dharma being the four \textit{āśrama}s 
  {\rm (}and not three, as it may seem, at least according to
  \mycitep{OlivelleAsrama}{55} and
  \mycitep{GanguliMBh}{Śāntiparvan CCLXX}{\rm )} 
  is hinted at \MBH\ 12.262.19--21: 
  
 
  \textit{dharmam ekaṃ catuṣpādam āśritās te nararṣabhāḥ}\thinspace |
 
  \textit{taṃ santo vidhivat prāpya gacchanti paramāṃ gatim}\thinspace ||
 
  \textit{gṛhebhya eva niṣkramya vanam anye samāśritāḥ}\thinspace |
 
  \textit{gṛham evābhisaṃśritya tato 'nye brahmacāriṇaḥ}\thinspace ||
 
  \textit{dharmam etaṃ catuṣpādam āśramaṃ brāhmaṇā viduḥ}\thinspace |
 
  \textit{ānantyaṃ brahmaṇaḥ sthānaṃ brāhmaṇā nāma niścayaḥ}\thinspace ||
  
 
  On the more frequently quoted interpretation of the four legs, see 
  \mycitep{OlivelleAsrama}{235}, a translation of \MANU\ 1.81--82:
  `Dharma and truth possess all four feet and are whole during the Kṛta yuga, 
  and people did not obtain anything unrighteously {\rm (}\textit{adharmeṇa}{\rm )}. 
  By obtaining, however, \textit{dharma} has lost one foot during each of the other \textit{yuga}s 
  and righteousness {\rm (}\textit{dharma}{\rm )} likewise has diminished by one quarter due to theft, 
  falsehood, and deceit.'
  
 
  Understand \textit{pāda}s c and d as \textit{catvāri āśramāṇi kīrtitāni dharmo manīṣibhiḥ} or
  \textit{yo dharmaḥ kīrtitaś caturāśramāṇi manīṣibhiḥ} or 
  \textit{yo dharmaś caturāśramaḥ kīrtito manīṣibhiḥ}. Judit Törzsök suggested
  that \textit{caturāśrama} and \textit{dharmaḥ} may be interpreted as a split compound here.
 }}

  \maintext{gatiś ca pañca vijñeyāḥ śṛṇu dharmasya bho dvija |}%

  \maintext{devamānuṣatiryaṃ ca narakasthāvarādayaḥ }||\thinspace3:5\thinspace||%
\translation{And the paths of Dharma are five. Listen, O Brahmin: [existence as] gods, men, animals, [existence in] hell and [as] vegetables, etc. \blankfootnote{3.5 Note the use of the singular next to a number in \textit{pāda} a, as in 3.1d, and that
  \textit{vijñeyāḥ} is an emendation from \textit{vijñeyaḥ} following the logic of 3.1d.
 \textit{tirya} seems to be an acceptable nominal stem in this text for \textit{tiryañc}. See,
  e.g., 4.6a: \textit{devamānuṣatiryeṣu}. °\textit{ādayaḥ} in \textit{pāda} d seems superfluous,
  the verse having already listed five items.
 }}

  \maintext{brahmaṇo hṛdayaṃ bhittvā jāto dharmaḥ sanātanaḥ |}%

  \maintext{tasya patnī mahābhāgā trayodaśa sumadhyamāḥ }||\thinspace3:6\thinspace||%
\translation{Eternal Dharma was born after splitting Brahmā's heart. He has beautiful wives, thirteen in number, with nice waists. \blankfootnote{3.6 Note the use of the singular in \textit{pāda}s cd. I have left \textit{sumadhyamāḥ} as the
  manuscripts transmit it: it signals the presence of the plural. One might consider 
  correcting \textit{mahābhāgā} to \textit{mahābhāgās}, but cf. p.~\pageref{number} on grammatical number.
  In sum, understand \textit{tasya patnyo mahābhāgās trayodaśa sumadhyamāḥ}.
 }}

  \maintext{dakṣakanyā viśālākṣī śraddhādyā sumanoharāḥ |}%

  \maintext{tasya putrāś ca pautrāś ca anekāś ca babhūva ha |}%

  \maintext{eṣa dharmanisargo 'yaṃ kiṃ bhūyaḥ śrotum icchasi }||\thinspace3:7\thinspace||%
\translation{They are Dakṣa's daughters, [called] Śraddhā and so on. They have huge eyes and they are beautiful. Numerous sons and grandsons were born to him. This is the nature of Dharma. What more do you wish to hear? \blankfootnote{3.7 \textit{śraddhāḍhyāḥ} in \textit{pāda} b is an attractive lectio difficilior {\rm (}`they were rich in faith/devotion'{\rm )}, but I have finally 
  decided to accept the easier and better-attested \textit{śraddhādyā}[\textit{ḥ}].
  {\rm (}Note that in fact the wives' names start with Śraddhā in 3.9.{\rm )}
  Again, the plural forms °\textit{ādyāḥ} could have been applied. I have chosen
  \textit{sumanoharāḥ} in \textit{pāda} b because the pattern singular-singular-{\rm (}singular{\rm )}-plural,
  i.e. having the required plural ending only at the end of the noun phrase,
  seems to be natural in the language of the \VSS. 
  Note the use of a singular verb instead of the required the plural in \textit{pāda}s cd,
  \textit{babhūva ha} perhaps being a phonetic and metrically `adjusted' equivalent, so to say, of \textit{babhūvuḥ}.
 }}

  \maintext{vigatarāga uvāca |}%

  \maintext{dharmapatnī viśeṣeṇa putras tebhyaḥ pṛthak pṛthak |}%

  \maintext{śrotum icchāmi tattvena kathayasva tapodhana }||\thinspace3:8\thinspace||%
\translation{Vigatarāga spoke: I would like to hear about Dharma's wives truly and about each one of the sons born to them. Teach me, O great ascetic. \blankfootnote{3.8 I could have emended \textit{tebhyaḥ} to the correct feminine form \textit{tābhyaḥ},
  suspecting that it is only the result of some early confusion
  brought about by \textit{putras}, but \textit{tebhyaḥ} might be original, and
  it even might mean '[hear] about them.'
  Note again the use of the singular {\rm (}nominative{\rm )} for the plural {\rm (}accusative{\rm )} in \textit{pāda}s ab.
  Alternatively, emend \textit{dharmapatnī} to \textit{dharmapatnīr} {\rm (}plural accusative{\rm )} and 
  \textit{putras} to \textit{putrān} to make them work with \textit{śrotum icchāmi}.
 }}

  \maintext{anarthayajña uvāca |}%

  \maintext{śraddhā lakṣmīr dhṛtis tuṣṭiḥ puṣṭir medhā kriyā lajjā |}%

  \maintext{buddhiḥ śāntir vapuḥ kīrtiḥ siddhiḥ prasūtisambhavāḥ }||\thinspace3:9\thinspace||%
\translation{Anarthayajña spoke: [Dharma's wives are] [1] Śraddhā {\rm (}`Faith'{\rm )}, [2] Lakṣmī {\rm (}`Prosperity'{\rm )}, [3] Dhṛti {\rm (}`Resolution'{\rm )}, [4] Tuṣṭi {\rm (}`Satisfaction'{\rm )}, [5] Puṣṭi\linebreak {\rm (}`Growth'{\rm )}, [6] Medhā {\rm (}`Wisdom'{\rm )}, [7] Kriyā {\rm (}`Labour'{\rm )}, [8] Lajjā {\rm (}`Modesty'{\rm )}, [9] Buddhi {\rm (}`Intelligence'{\rm )}, [10] Śānti {\rm (}`Tranquillity'{\rm )}, [11] Vapus {\rm (}`Beauty'{\rm )}, [12] Kīrti {\rm (}`Fame'{\rm )}, [13] Siddhi {\rm (}`Success'{\rm )}, [all] born to Prasūti[, Dakṣa's wife]. \blankfootnote{3.9 Note how \textit{lajjā} in \textit{pāda} b makes the line unmetrical.
 
  For Dharma's thirteen wives and their sons, see, e.g., \LINPU\ 1.5.34--37 {\rm (}note the 
  similarity between the first line and \VSS\ 3.6cd--7ab above{\rm )}:
  
 
  \textit{dharmasya patnyaḥ śraddhādyāḥ kīrtitā vai trayodaśa}\thinspace |
 
  \textit{tāsu dharmaprajāṃ vakṣye yathākramam anuttamam}\thinspace ||
 
  \textit{kāmo darpo 'tha niyamaḥ saṃtoṣo lobha eva ca}\thinspace |
 
  \textit{śrutas tu daṇḍaḥ samayo bodhaś caiva mahādyutiḥ}\thinspace ||
 
  \textit{apramādaś ca vinayo vyavasāyo dvijottamāḥ}\thinspace |
 
  \textit{kṣemaṃ sukhaṃ yaśaś caiva dharmaputrāś ca tāsu vai}\thinspace || 
 
  \textit{dharmasya vai kriyāyāṃ tu daṇḍaḥ samaya eva ca}\thinspace |
 
  \textit{apramādas tathā bodho buddher dharmasya tau sutau}\thinspace ||
  
 
 
  \textit{prasūtisambhavāḥ} in \textit{pāda} d is a rather bold conjecture that can be supported by two facts:
  firstly, the readings of the manuscripts are difficult to make sense of and thus are
  probably corrupt; secondly, a corruption from the name Prasūti,
  traditionally the name of Dakṣa's wife, to \textit{ābhūti}
  is relatively easily to explain, \textit{sū} and \textit{bhū} being close enough in some scripts 
  {\rm (}e.g. in \msCa{\rm )} to cause confusion. Another option would be to accept 
  Ābhūti as the name of Dakṣa's wife.
  
 
  For Prasūti being Dakṣa's wife in other sources,
  see, e.g., \LINPU\ 1.5.20--21 {\rm (}but also note the presence of the name Sambhūti{\rm )}:
  
 
  \textit{prasūtiḥ suṣuve dakṣāc caturviṃśatikanyakāḥ}\thinspace |
 
  \textit{śraddhāṃ lakṣmīṃ dhṛtiṃ puṣṭiṃ tuṣṭiṃ medhāṃ kriyāṃ tathā}\thinspace ||
 
  \textit{buddhi lajjāṃ vapuḥ śāntiṃ siddhiṃ kīrtiṃ mahātapāḥ}\thinspace |
 
  \textit{khyātiṃ śāntiś ca saṃbhūtiṃ smṛtiṃ prītiṃ kṣamāṃ tathā}\thinspace ||
 }}

  \maintext{śraddhā kāmaḥ suto jāto darpo lakṣmīsutaḥ smṛtaḥ |}%

  \maintext{dhṛtyās tu niyamaḥ putraḥ saṃtoṣas tuṣṭijaḥ smṛtaḥ }||\thinspace3:10\thinspace||%
\translation{Śraddhā's son is Kāma {\rm (}`Desire'{\rm )}. Darpa {\rm (}`Pride'{\rm )} is said to be Lakṣmī's son. Dhṛti's son is Niyama {\rm (}`Rule'{\rm )}. Saṃtoṣa {\rm (}`Satisfaction'{\rm )} is Tuṣṭi's son. \blankfootnote{3.10 Understand \textit{śraddhā} as a stem form noun for \textit{śraddhāyāḥ} {\rm (}gen./abl., cf. 3.11a{\rm )}.
  Alternatively, take \textit{śraddhā} and \textit{suto} as elements of a split compound, and understand
  \textit{śraddhāsuto jātaḥ kāmaḥ}.
 }}

  \maintext{puṣṭyā lābhaḥ suto jāto medhāputraḥ śrutas tathā |}%

  \maintext{kriyāyās tv abhavat putro daṇḍaḥ samaya eva ca }||\thinspace3:11\thinspace||%
\translation{To Puṣṭi was born a son [called] Lābha {\rm (}`Profit'{\rm )}. Medhā's son is Śruta {\rm (}`Sacred Knowledge'{\rm )}. Kriyā's sons are Daṇḍa {\rm (}`Punishment'{\rm )} and Samaya {\rm (}`Law'{\rm )}. \blankfootnote{3.11 I have emended \textit{abhayaḥ} to \textit{abhavat} in \textit{pāda} c, following the relevant line in the \KURMP\ cited 
  in the apparatus to this verse
  {\rm (}\textit{kriyāyāś cābhavat putro daṇḍaḥ samaya eva ca}{\rm )} and also \LINPU\ 1.5.37 quoted also 
  in the apparatus, allotting only two sons to Kriyā. Thus I don't think
  that Kriyā is supposed to have a son called Abhaya {\rm (}`Freedom from danger'; \BHAGP\ 4.1.50ab 
  claims that Dayā had a son called Abhaya:
  \textit{śraddhāsūta śubhaṃ maitrī prasādam abhayaṃ dayā}{\rm )}.
  Nevertheless, in a number of sources Kriyā actually has three sons, 
  see, e.g., \VISNUP\ 1.7.26ab,
  where they are named as Daṇḍa, Naya and Vinaya:
  \textit{medhā śrutaṃ kriyā daṇḍaṃ nayaṃ vinayam eva ca}. 
  Perhaps read \textit{kriyāyās tu nayaḥ putro} in \textit{pāda} c? Compare \VAYUP\ 1.10.34cd
  {\rm (}\textit{kriyāyās tu nayaḥ prokto daṇḍaḥ samaya eva ca}{\rm )} 
  with \BRAHMANDAPUR\ 1.9.60ab {\rm (}\textit{kriyāyās tanayau proktau damaś ca śama eva ca}{\rm )}.
 }}

  \maintext{lajjāyā vinayaḥ putro buddhyā bodhaḥ sutaḥ smṛtaḥ |}%

  \maintext{lajjāyāḥ sudhiyaḥ putra apramādaś ca tāv ubhau }||\thinspace3:12\thinspace||%
\translation{Lajjā's son is Vinaya {\rm (}`Discipline'{\rm )}, Buddhi's son is Bodha {\rm (}`Intelligence'{\rm )}. Lajjā has two [more] sons: Sudhiya[/Sudhī] {\rm (}`Wise'{\rm )} and Apramāda {\rm (}`Cautiousness'{\rm )}. \blankfootnote{3.12 In a very similar passages in \KURMP\ 1.8.20 ff., Apramāda is Buddhi's son and 
  Lajjā has only one son, Vinaya. In the above verse {\rm (}\VSS\ 3.12{\rm )}, \textit{sudhiyaḥ} {\rm (}for \textit{sudhīḥ}{\rm )} may only be 
  qualifying \textit{apramāda}, thus Lajjā may have two sons: Vinaya and the wise Apramāda.
  Alternatively, \textit{pāda}s cd might be a extra line inserted accidentally.
 }}

  \maintext{kṣemaḥ śāntisuto vindyād vyavasāyo vapoḥ sutaḥ |}%

  \maintext{yaśaḥ kīrtisuto jñeyaḥ sukhaṃ siddher vyajāyata |}%

  \maintext{svāyambhuve 'ntare tv āsan kīrtitā dharmasūnavaḥ }||\thinspace3:13\thinspace||%
\translation{Kṣema {\rm (}`Peace'{\rm )} is to be known as Śānti's son, Vyavasāya {\rm (}`Resolution'{\rm )} is Vapus' son. Yaśas {\rm (}`Fame'{\rm )} is Kīrti's son, Sukha {\rm (}`Joy'{\rm )} was born to Siddhi. [This is how] the sons of Dharma in the [\textit{manvantara}] era of Svāyambhuva [Manu] were known. \blankfootnote{3.13 Note that \textit{sukhaṃ} in \textit{pāda} d is probably meant to be masculine {\rm (}\textit{sukhaḥ}{\rm )}, but e.g. in the 
  \KURMP\ passage quoted above it is also neuter. For the emendation in \textit{pāda} e, 
  see \MATSP\ 9.2cd: 
  
 
  \textit{yāmā nāma purā devā āsan svāyambhuvāntare},
  
 
  and \BHAGP\ 6.4.1: 
  
 
  \textit{devāsuranṛṇāṃ sargo nāgānāṃ mṛgapakṣiṇām}\thinspace |
  \textit{sāmāsikas tvayā prokto yas tu svāyambhuve 'ntare}\thinspace ||
 }}

  \maintext{vigatarāga uvāca |}%

  \maintext{mūrtidvayaṃ kathaṃ dharmaṃ kathayasva tapodhana |}%

  \maintext{kautūhalam atīvaṃ me kartaya jñānasaṃśayam }||\thinspace3:14\thinspace||%
\translation{Vigatarāga spoke: How come Dharma has two embodiments? Tell me, O great ascetic. I am extremely intrigued. Cut my doubts concerning [this] knowledge. \blankfootnote{3.14 Note \textit{dharma} as a neuter noun and the form \textit{atīvaṃ} for \textit{atīva} metri causa. 
  My emen\-dation from \textit{kīrtaya} {\rm (}`declare'{\rm )} to \textit{kartaya} {\rm (}`cut'{\rm )} was influenced by the combination
  of \textit{chindhi} and \textit{saṃśaya}, often with \textit{kautūhala}, elsewhere in the \VSS:
  3.2ab: \textit{kautūhalaṃ mamotpannaṃ saṃśayaṃ chindhi tattvataḥ}; 
  10.10cd: \textit{kautūhalaṃ mahaj jātaṃ chindhi saṃśayakārakam};
  15.2ab: \textit{etat kautūhalaṃ chindhi saṃśayaṃ parameśvara}. 
  The reading \textit{kīrtaya} may have been the result of the influence of \textit{kīrtitā} in 3.13f above.
 }}

  \maintext{anarthayajña uvāca |}%

  \maintext{śrutismṛtidvayor mūrtir dharmasya parikīrtitā |}%

  \maintext{dārāgnihotrasambandha ijyā śrautasya lakṣaṇam |}%

  \maintext{smārto varṇāśramācāro yamaiś ca niyamair yutaḥ }||\thinspace3:15\thinspace||%
\translation{Anarthayajña spoke: Dharma's embodiment is said to consist of Scripture {\rm (}\textit{śruti}{\rm )} and Tradition {\rm (}\textit{smṛti}{\rm )}. The characteristics of the Śrauta [tradition] are an association with a wife [i.e.\ marriage] and with the fire ritual, and sacrifice. The Smārta [tradition focuses on] the conduct {\rm (}\textit{ācāra}{\rm )} of the social classes {\rm (}\textit{varṇa}{\rm )} and disciplines {\rm (}\textit{āśrama}{\rm )} which is connected to rules and regulations {\rm (}\textit{yama-niyama}{\rm )}. \blankfootnote{3.15 The reading °\textit{dvayī} in \msNc\ in \textit{pāda} a is attractive, but
  it could well be only an attempt to improve upon the text.
  The emendation in \textit{pāda} c is based on parallel passages in \MANU\ and the \MATSP\
  {\rm (}see the apparatus{\rm )}.
 
  As for Dharma being based on \textit{śruti} and \textit{smṛti}, see, e.g., \MANU\ 2.10:
  
 
  \textit{śrutis tu vedo vijñeyo dharmaśāstraṃ tu vai smṛtiḥ}\thinspace |
 
  \textit{te sarvārtheṣv amīmāṃsye tābhyāṃ dharmo hi nirbabhau}\thinspace ||
  
 
  In Olivelle's translation {\rm (}\citeyear{OlivelleManu}, 94{\rm )}:
  `\thinspace ``Scripture'' should be recognized as ``Veda,'' and ``tradition''
  as ``Law Treatise.'' These two should never be called into question in any matter,
  for it is from them that the Law shines forth.'
 
  
  To state that the Smārta tradition is connected to \textit{yama}s and \textit{niyama}s and the \textit{āśrama}s and
  then to discuss these at length {\rm (}principally in chapters 3--8 and 11{\rm )} can be seen 
  as a clear self-identification with the Smārta tradition.
 }}

  \subchptr{yamaniyamabhedaḥ}%

  \trsubchptr{Yama and Niyama rules}%

  \maintext{yamaś ca niyamaś caiva dvayor bhedam ataḥ śṛṇu |}%

  \maintext{ahiṃsā satyam asteyam ānṛśaṃsyaṃ damo ghṛṇā |}%

  \maintext{dhanyāpramādo mādhuryam ārjavaṃ ca yamā daśa }||\thinspace3:16\thinspace||%
\translation{Now hear the classification of both the \textit{yama} and \textit{niyama} rules. Non-vio\-lence, truthfulness, refraining from stealing, absence of hostility, self-re\-straint, taboos, virtue, avoiding mistakes, charm, sincerity: these are the ten \textit{yama}s. \blankfootnote{3.16 \textit{Pāda} a should be understood as \textit{yamaniyamayoś}, but the author of this line
  may have tried to avoid the metrical fault of having two short syllables 
  in second and third position. Note how all witnesses read \textit{mādhūrya} 
  in \textit{pāda} e instead of \textit{mādhurya}. The former may have been
  acceptable originally in this text. \textit{Pāda} e is a \textit{ma-vipulā}.
 
 
  As noted above, this is the beginning of a long section in our text
  that describes the \textit{yama-niyama} rules, reaching up to the end of chapter eight. 
  The title given in the colophon of the next chapter, chapter four, namely \textit{yamavibhāga},
  would fit this locus better than the beginning of that chapter, which 
  commences with a discussion on the second of the \textit{yama}s, \textit{satya}.
 }}

  \maintext{ekaikasya punaḥ pañcabhedam āhur manīṣiṇaḥ |}%

  \maintext{ahiṃsādi pravakṣyāmi śṛṇuṣvāvahito dvija }||\thinspace3:17\thinspace||%
\translation{The wise say that there are five subclasses to each. I shall teach you about non-violence and the other [\textit{yama}-rules]. Listen carefully, O twice-born. \blankfootnote{3.17 In \textit{pāda} a, \textit{pañca} and \textit{bheda} may be typeset as two separate words since
  the use of the singular after numbers is one of the hallmarks of the text 
  {\rm (}see p.~\pageref{number}{\rm )}.
 }}

  \subchptr{yameṣv ahiṃsā {\rm {\rm (}1{\rm )}}}%

  \trsubchptr{First Yama-rule: non-violence}%

  \subsubchptr{pañcavidhā hiṃsā}%

  \trsubsubchptr{Five types of violence}%

  \maintext{trāsanaṃ tāḍanaṃ bandho māraṇaṃ vṛttināśanam |}%

  \maintext{hiṃsāṃ pañcavidhām āhur munayas tattvadarśinaḥ }||\thinspace3:18\thinspace||%
\translation{Frightening and beating [other people], tying [someone] up, killing, and the destruction of [other people's] livelihood: violence is said by the wise who see the truth to be of [these] five types. }

  \maintext{kāṣṭhaloṣṭakaśādyais tu tāḍayantīha nirdayāḥ |}%

  \maintext{tatprahāravibhinnāṅgo mṛtavadhyam avāpnuyāt }||\thinspace3:19\thinspace||%
\translation{Cruel people beat [other people] with sticks, clods of earth [i.e. they stone them], with whips and other [objects] in the everyday world. Their bodies broken by the same blows, they receive the capital punishment. \blankfootnote{3.19 Note the use of the singular {\rm (}°\textit{āṅgo}... \textit{avāpnuyāt}{\rm )} in \textit{pāda}s cd 
  referring back to the plural agents of the previous sentence.
  Most probably, °\textit{vadhyam} is to be understood as °\textit{vadham} and the form 
  \textit{vadhyam} serves only to avoid two \textit{laghu} syllables in \textit{pāda} d.
  {\rm (}See the word \textit{vadha} in the next three verses.{\rm )}
 }}

  \maintext{baddhvā pādau bhujoraś ca śirorukkaṇṭhapāśitāḥ |}%

  \maintext{anāhatā mriyanty evaṃ vadho bandhanajaḥ smṛtaḥ }||\thinspace3:20\thinspace||%
\translation{[Others] tie up [people] at their feet, arms and chest. [These,] hung by their hair and neck, die in this way without being wounded. This is the capital punishment for tying up [other people]. \blankfootnote{3.20 Understand \textit{bhujoraś ca} in \textit{pāda} a as \textit{bhuje, urasi ca}, in this case with an instance of double sandhi,
  and in stem form: \textit{bhuje urasi ca} $\rightarrow$\ \textit{bhuja urasi ca} 
  $\rightarrow$\ \textit{bhujorasi ca} $\rightarrow$\ \textit{bhujoraś ca}.
  Alternatively, understand it as a compound {\rm (}\textit{bhujorasi}{\rm )}. 
  In \textit{pāda} b, my emendation is only one of the possible interpretations. We might accept
  \textit{śiroru}° as consisting of \textit{śira} + \textit{ūru} {\rm (}`head and thigh'{\rm )}, or emend it 
  to \textit{śiroraḥ}° for \textit{śira} + \textit{uraḥ} {\rm (}`head and chest'{\rm )}. Also note my conjecture
  in \textit{pāda} d, without which this \textit{pāda} is difficult to interpret.
 }}

  \maintext{śatrucaurabhayair ghoraiḥ siṃhavyāghragajoragaiḥ |}%

  \maintext{trāsanād vadham āpnoti anyair vāpi suduḥsahaiḥ }||\thinspace3:21\thinspace||%
\translation{He who frightens [other people] with the terrible danger of enemies and thieves, with lions, tigers, elephants or snakes, or by other horrors, will be executed. }

  \maintext{yasya yasya hared vittaṃ tasya tasya vadhaḥ smṛtaḥ |}%

  \maintext{vṛttijīvābhibhūtānāṃ taddvārā nihataḥ smṛtaḥ }||\thinspace3:22\thinspace||%
\translation{He who robs somebody's money is to be punished by the same person. He is [to be] struck down by those whose livelihood got damaged by him. \blankfootnote{3.22 Perhaps understand \textit{vadhaḥ} in \textit{pāda} b as \textit{vadhyaḥ} metri causa.
 My translation of the second line of this verse reflects a conjecture {\rm (}\textit{taddvārā}{\rm )}
  understood as connected to both \textit{pāda} c and \textit{nihataḥ} in \textit{pāda} d.
  The plural genitive in \textit{pāda} c and the instrumental \textit{taddvārā} are perhaps to be taken as 
  plural instrumentals: °\textit{bhibhūtais tair}.
 }}

  \maintext{viṣavahniśaraśastrair māyāyogabalena vā |}%

  \maintext{hiṃsakāny āhu viprendra munayas tattvadarśinaḥ }||\thinspace3:23\thinspace||%
\translation{[Those who kill other people] with poison, fire, arrows, swords, or by the force of magic or yoga, are called murderers by the sages who see the truth, O great Brahmin. \blankfootnote{3.23 \textit{Pāda} a is a \textit{sa-vipulā}.
  Note how elliptical this verse is and that \textit{hiṃsakāni} is neuter although it refers to 
  people, perhaps implying \textit{bhūtāni}. Alternatively, take \textit{y} in \textit{hiṃsakāny} as a 
  rather unusual sandhi-bridge {\rm (}\textit{hiṃsakān-y-āhu}{\rm )}, or simply delete this \textit{y}. 
  Note also that \textit{āhu} stands for \textit{āhur} metri causa.
 }}

  \subsubchptr{ahiṃsāpraśaṃsā}%

  \trsubsubchptr{Praise of non-violence}%

  \maintext{ahiṃsā paramaṃ dharmaṃ yas tyajet sa durātmavān |}%

  \maintext{kleśāyāsavinirmuktaṃ sarvadharmaphalapradam }||\thinspace3:24\thinspace||%
\translation{Non-violence is the highest Dharma. He who abandons it is a wicked person. It is free of pain and trouble, it yields the fruits of all [other] Dharmic teachings [in itself]. \blankfootnote{3.24 Note \textit{dharma} as a neuter noun in \textit{pāda} a and that °\textit{vinirmuktaṃ} and
  °\textit{pradam} are neuter accordingly.
 }}

  \maintext{nātaḥ parataro mūrkho nātaḥ parataraṃ tamaḥ |}%

  \maintext{nātaḥ parataraṃ duḥkhaṃ nātaḥ parataro 'yaśaḥ }||\thinspace3:25\thinspace||%
\translation{There is no bigger fool than one [that abandons it]. There is no bigger mental darkness [than the abandonment of non-violence]. There is no greater suffering or greater infamy. \blankfootnote{3.25 Note that \textit{parataro} is masculine in \textit{pāda} d, picking up a neuter \textit{'yaśaḥ}.
  This phenomenon is probably the result of \textit{'yaśaḥ} resembling a masculine noun ending in \textit{-aḥ}
  and also of the metrical problem with a grammatically correct \textit{nātaḥ parataram ayaśaḥ}.
 }}

  \maintext{nātaḥ parataraṃ pāpaṃ nātaḥ parataraṃ viṣam |}%

  \maintext{nātaḥ paratarāvidyā nātaḥ paraṃ tapodhana }||\thinspace3:26\thinspace||%
\translation{There is no greater sin or a more effective poison. There is no greater ignorance, there is nothing worse, O great ascetic. \blankfootnote{3.26 \textit{Pāda} d is slightly suspect. 
  The vocative \textit{tapodhana} usually refers to Anarthayajña in these
  passages, and not to Vigatarāga, as here. The text may have read \textit{nātaḥ paratamo 'dhanaḥ} 
  {\rm (}`There is no bigger loss of wealth'{\rm )} or possibly something starting with
  \textit{nātaḥ paraṃ tapo ...} {\rm (}`There is no greater\dots\ of austerity'{\rm )}.
  Perhaps \textit{nātaḥ paraṃ tapo'ntakam} {\rm (}`There is no greater destroyer of penance'{\rm )}?
 }}

  \maintext{yo hinasti na bhūtāni udbhijjādi caturvidham |}%

  \maintext{sa bhavet puruṣaḥ śreṣṭhaḥ sarvabhūtadayānvitaḥ }||\thinspace3:27\thinspace||%
\translation{He who does not harm [any of] the four types of living beings, beginning with plants, is the best person, because he has compassion for all creatures. }

  \maintext{sarvabhūtadayāṃ nityaṃ yaḥ karoti sa paṇḍitaḥ |}%

  \maintext{sa yajvā sa tapasvī ca sa dātā sa dṛḍhavrataḥ }||\thinspace3:28\thinspace||%
\translation{He who always has compassion for all creatures is the [true] Paṇḍit. He is the [true] sacrificer, the [true] ascetic, he is a [real] donor, one with a firm vow. }

  \maintext{ahiṃsā paramaṃ tīrtham ahiṃsā paramaṃ tapaḥ |}%

  \maintext{ahiṃsā paramaṃ dānam ahiṃsā paramaṃ sukham }||\thinspace3:29\thinspace||%
\translation{Non-violence is the supreme pilgrimage place. Non-violence is the highest austerity. Non-violence is the highest donation. Non-violence is the highest joy. }

  \maintext{ahiṃsā paramo yajñaḥ ahiṃsā paramaṃ vratam |}%

  \maintext{ahiṃsā paramaṃ jñānam ahiṃsā paramā kriyā }||\thinspace3:30\thinspace||%
\translation{Non-violence is the supreme sacrifice. Non-violence is the supreme religious observance. Non-violence is supreme knowledge. Non-violence is the supreme ritual. }

  \maintext{ahiṃsā paramaṃ śaucam ahiṃsā paramo damaḥ |}%

  \maintext{ahiṃsā paramo lābhaḥ ahiṃsā paramaṃ yaśaḥ }||\thinspace3:31\thinspace||%
\translation{Non-violence is the highest purity. Non-violence is the highest self-restraint. Non-violence is the highest profit. Non-violence is the greatest fame. }

  \maintext{ahiṃsā paramo dharmaḥ ahiṃsā paramā gatiḥ |}%

  \maintext{ahiṃsā paramaṃ brahma ahiṃsā paramaḥ śivaḥ }||\thinspace3:32\thinspace||%
\translation{Non-violence is the supreme Dharma. Non-violence is the supreme path. Non-violence is the supreme Brahman. Non-violence is supreme Śiva. \blankfootnote{3.32 \textit{śiva} in \textit{pāda} d may or may not refer to the deity Śiva. The last sentence may simply
  mean: `Non-violence is the supreme good.'
 }}

  \subsubchptr{māṃsāhāraḥ}%

  \trsubsubchptr{Meat-consumption}%

  \maintext{māṃsāśanān nivarteta manasāpi na kāṅkṣayet |}%

  \maintext{sa mahat phalam āpnoti yas tu māṃsaṃ vivarjayet }||\thinspace3:33\thinspace||%
\translation{One should refrain from meat-consumption. One should not even desire it mentally. He who abandons meat will receive a great reward. }

  \maintext{svamāṃsaṃ paramāṃsena yo vardhayitum icchati |}%

  \maintext{anabhyarcya pitṝn devān na tato 'nyo 'sti pāpakṛt }||\thinspace3:34\thinspace||%
\translation{He who wishes to nourish his own flesh with the flesh of other [beings], outside of worshipping the ancestors and the gods, is the biggest sinner of all. \blankfootnote{3.34 See \UUMS\ chapter two for a similar section on meat-consumption. 
  The present verse is a variant on \MANU\ 5.52 {\rm (}see apparatus{\rm )}.
 }}

  \maintext{madhuparke ca yajñe ca pitṛdaivatakarmaṇi |}%

  \maintext{atraiva paśavo hiṃsyā nānyatra manur abravīt }||\thinspace3:35\thinspace||%
\translation{During the honey-mixture offering {\rm (}\textit{madhuparka}{\rm )} and during a sacrifice, during rituals for the ancestors and the gods: only in these cases are animals to be slaughtered and not in any other case. [This is what] Manu taught. \blankfootnote{3.35 This verse is a variant of \MANU\ 5.41.
 }}

  \maintext{krītvā svayaṃ vāpy utpādya paropahṛtam eva vā |}%

  \maintext{devān pitṝṃś cārcayitvā khādan māṃsaṃ na doṣabhāk }||\thinspace3:36\thinspace||%
\translation{Should he buy it or procure it himself or should it be offered by others, if he eats meat, he will not sin if he first worships the gods and the ancestors. \blankfootnote{3.36 This verse is \MANU\ 5.32.
 }}

  \maintext{vedayajñatapastīrthadānaśīlakriyāvrataiḥ |}%

  \maintext{māṃsāhāranivṛttānāṃ ṣoḍaśāṃśaṃ na pūryate }||\thinspace3:37\thinspace||%
\translation{[People who perform] Vedic sacrifices and austerities, and [visit] sacred places, donate, [those who are of] good conduct, [perform] rituals and [keep] religious vows, [but eat meat] will not [be able to] enjoy even the sixteenth part of [such rewards that those] people [receive] who have given up meat. \blankfootnote{3.37 As for \textit{pāda} d, see a similarly phrased comparison in \MANU\ 2.86:
  
 
  \textit{ye pākayajñās catvāro vidhiyajñasamanvitāḥ}\thinspace |
 
  \textit{sarve te japayajñasya kalāṃ nārhanti ṣoḍaśīm}\thinspace ||
  
 
  In Olivelle's translation {\rm (}\citeyear{OlivelleManu}, 99{\rm )}:
  `The four types of cooked oblations along with the sacrifices 
  consisting of prescribed rites---all these are not worth a sixteenth part 
  of the sacrifice consisting of soft recitation.'
 }}

  \maintext{mṛgāḥ parṇatṛṇāhārād ajameṣagavādibhiḥ |}%

  \maintext{sukhino balavantaś ca vicaranti mahītale }||\thinspace3:38\thinspace||%
\translation{Deer and goats, sheep, cows and other [animals] wander in the world happily and in great strength [just] from eating leaves and grass. }

  \maintext{vānarāḥ phala{-}m{-}āhārā rākṣasā rudhirapriyāḥ |}%

  \maintext{nihatā rākṣasāḥ sarve vānaraiḥ phalabhojibhiḥ }||\thinspace3:39\thinspace||%
\translation{Monkeys eat fruits, Rākṣasas prefer blood. The fruit-eating monkeys defeated all the Rākṣasas. \blankfootnote{3.39 Understand \textit{phalam āhārā} as \textit{phalāhārā} {\rm (}\textit{-m-} is a sandhi-bridge{\rm )}.
 This verse clearly refers to the story of the \textit{Rāmāyaṇa}.
 }}

  \maintext{tasmān māṃsaṃ na hīheta balakāmena bho dvija |}%

  \maintext{balena ca guṇākarṣāt parato bhayabhīruṇā }||\thinspace3:40\thinspace||%
\translation{Therefore one should not crave meat in the hope of gaining strength, O Brahmin, in order to be able to draw a bow with force, or out of fear of the danger coming from the enemy. \blankfootnote{3.40 \textit{guṇākāśāt} in \textit{pāda} c is difficult to interpret and 
  \textit{guṇākarṣāt} is a conjecture by Judit Törzsök which fits the context well,
  although the polysemy of \textit{guṇa} may allow for other solutions.
 }}

  \maintext{ahiṃsakasamo nāsti dānayajñasamīhayā |}%

  \maintext{iha loke yaśaḥ kīrtiḥ paratra ca parā gatiḥ }||\thinspace3:41\thinspace||%
\translation{By wishing to make donations and perform sacrifices no one will become comparable to someone who refrains from violence. [Such a person will have] fame and glory in this world and the supreme path in the other. \blankfootnote{3.41 Note the variant °\textit{dharma}° in both \msCc\ and \Ed\ in \textit{pāda} b.
  \textit{Pāda}s ab are reminiscent of \SDHS\ 11.92: 
 
  \textit{ahiṃsaikā paro dharmaḥ śaktānāṃ parikīrtitam}\thinspace |
 
  \textit{aśaktānām ayaṃ dharmo dānayajñādipūrvakaḥ}\thinspace ||
  
 
  On the above verse see also \mycitep{SaivaUtopia}{15--16}.
 }}

  \maintext{trailokyaṃ maṇiratnapūrṇam akhilaṃ dattvottame brāhmaṇe}%

 \nonanustubhindent \maintext{koṭīyajñasahasrapadmam ayutaṃ dattvā mahīṃ dakṣiṇām |}%

  \maintext{tīrthānāṃ ca sahasrakoṭiniyutaṃ snātvā sakṛn mānava}%

 \nonanustubhindent \maintext{etatpuṇyaphalam ahiṃsakajanaḥ prāpnoti niḥsaṃśayaḥ }||\thinspace3:42\thinspace||%
\translation{A person who refrains from violence will gain, no doubt about it, the [same] meritorious rewards that others would get by donating the three worlds filled with jewels and gems in their entirety to an excellent Brahmin, by [performing] a thousand times ten trillion {\rm (}\textit{padma}{\rm )} times ten thousand {\rm (}\textit{ayuta}{\rm )} \textit{koṭīyajña} sacrifices, by donating the earth [to a priest] as sacrificial fee, and by bathing [at] a thousand times ten million times a million {\rm (}\textit{niyuta}{\rm )} sacred places at once. \blankfootnote{3.42 Metre: \textit{śārdūlavikrīḍita}. 
  Note that the second syllable of \textit{phalam} in \textit{pāda} d is treated as long: this
  happens often at word-boundaries in this text {\rm (}see p.~\pageref{muta}{\rm )}; and 
  note how \msNc\ aims to restore the metre by inserting \textit{tv} after its \textit{phalaṃ}.
  On \textit{padma} meaning `ten trillion', and on other words for numbers, see 1.31--35. 
  
 
  \textit{koṭīyajña} in \textit{pāda} d may refer to a special kind of sacrifice, 
  mostly known as \textit{koṭihoma} in the Purāṇas and in inscriptions 
  {\rm (}see, e.g., \mycitep{Fleming2010}{and 2013}\nocite{Fleming2013}{\rm )}.
  It involves a hundred fire-pits 
  and a hundred times one thousand Brahmins {\rm (}hence the name `the ten-million sacrifice'{\rm )}.
  See, e.g., \BHAVP\ \textit{uttaraparvan} 4.142.54--58:
  
 
  \textit{śatānano daśamukho dvimukhaikamukhas tathā}\thinspace |
 
  \textit{caturvidho mahārāja koṭihomo vidhīyate}\thinspace || 
 
  \textit{kāryasya gurutāṃ jñātvā naiva kuryād aparvaṇi}\thinspace |
 
  \textit{yathā saṃkṣepataḥ kāryaḥ koṭihomas tathā śṛṇu}\thinspace ||
 
  \textit{kṛtvā kuṇḍaśataṃ divyaṃ yathoktaṃ hastasaṃmitam}\thinspace |
 
  \textit{ekaikasmiṃs tataḥ kuṇḍe śataṃ viprān niyojayet}\thinspace ||
 
  \textit{sadyaḥ pakṣe tu viprāṇāṃ sahasraṃ parikīrtitam}\thinspace |
 
  \textit{ekasthānapraṇīte 'gnau sarvataḥ paribhāvite}\thinspace || 
 
  \textit{homaṃ kuryur dvijāḥ sarve kuṇḍe kuṇḍe yathoditam}\thinspace |
 
  \textit{yathā kuṇḍabahutve 'pi rājasūye mahākratau}\thinspace ||
  
 
  Note \SDHS\ 10.91 {\rm (}see apparatus{\rm )}, a statement on \textit{ahiṃsā} which is
  similar to the present verse.
 }}

\centerline{\maintext{\dbldanda\thinspace iti vṛṣasārasaṃgrahe ahiṃsāpraśaṃsā nāmādhyāyas{ }tṛtīyaḥ\thinspace\dbldanda}}
\translation{Here ends the third chapter in the \textit{Vṛṣasārasaṃgraha} called the Praise of Non-violence.}

  \chptr{caturtho 'dhyāyaḥ}
\addcontentsline{toc}{section}{Chapter 4}
\fancyhead[CO]{{\footnotesize\textit{Translation of chapter 4}}}%

  \trchptr{ Chapter Four }%

  \subchptr{yameṣu satyam {\rm {\rm (}2{\rm )}}}%

  \trsubchptr{Second Yama-rule: truthfulness}%

  \maintext{anarthayajña uvāca |}%

  \maintext{sadbhāvaḥ satyam ity āhur dṛṣṭapratyayam eva vā |}%

  \maintext{yathābhūtārthakathanaṃ tat satyakathanaṃ smṛtam }||\thinspace4:1\thinspace||%
\translation{Anarthayajña spoke: The state of being real {\rm (}\textit{sad-bhāva}{\rm )} is called truth {\rm (}\textit{sat-ya}{\rm )}. Alternatively, it is also a certainty {\rm (}\textit{pratyaya}{\rm )} that originates in perception {\rm (}\textit{dṛṣṭa}{\rm )}. Relating things in a way that corresponds to reality is called `speaking the truth.' \blankfootnote{4.1 Compare \SDHS\ 11.105:
  
 
  \textit{svānubhūtaṃ svadṛṣṭaṃ ca yaḥ pṛṣṭārthaṃ na gūhati}\thinspace |
 
  \textit{yathābhūtārthakathanam ity etat satyalakṣaṇam}\thinspace ||
  
 
  Translation in \mycitep{SaivaUtopia}{p.~124}:
  `If one does not conceal a matter one is asked about, whether
  it was experienced by oneself or witnessed with one's own eyes,
  but gives an account of things as they happened, this is the definition of `truth.' '
  This verse makes it tempting to emend \textit{satyakathanaṃ} to \textit{satyalakṣaṇaṃ} in \VSS\ 4.1d, but 
  I rather take the \VSS\ verse to introduce two views on truth: one philosophical,
  and one ordinary that relates to the moral aspect of truthfulness.
  Also consider the commentator's remark on the same verse in the \SDHS\ 
  {\rm (}11.105; \mycitep{SaivaUtopia}{p.~124 n.~181 and p.~143}{\rm )}:
  \textit{yathābhūtārthakathane prāṇivadhaprāptāv asatyasya sādhutvāt para pīḍāvinirmuktam eva satyam ity āha}.
  Translation ibid.: `\dots\ he states that [speech is] truth
  only as long as it is devoid of harm of others, for untruth is good when giving an
  account of something as it really happened will result in the slaughter of a living
  creature.'
 }}

  \maintext{ākrośatāḍanādīni yaḥ saheta suduḥsaham |}%

  \maintext{kṣamate yo jitātmā tu sa ca satyam udāhṛtam }||\thinspace4:2\thinspace||%
\translation{He who endures severe abuse and beating etc. and resists [giving\linebreak away secrets], his self being conquered, is said to be [an example of] truth[ful\-ness]. \blankfootnote{4.2 \textit{suduḥsaham} {\rm (}singular{\rm )} in \textit{pāda} b picks up °\textit{ādīni} {\rm (}plural{\rm )} in \textit{pāda} a.
 The \textit{-m} in \textit{satyam} may be a sandhi-bridge and the phrase may refer to a
  masculine subject {\rm (}`a truthful person'{\rm )} thus: \textit{sa ca satya-m-udāhṛtaḥ}.
  Compare with \SDHS\ 11.82 {\rm (}see apparatus{\rm )}, which is a definition of
  forbearance {\rm (}\textit{kṣānti}{\rm )}.
 }}

  \maintext{vadhārtham udyataḥ śastraṃ yadi pṛccheta karhicit |}%

  \maintext{na tatra satyaṃ vaktavyam anṛtaṃ satyam ucyate }||\thinspace4:3\thinspace||%
\translation{If one is being interrogated at any time with a sword lifted to strike him down, in this case the truth is not to be spoken, and a lie can be called truth. \blankfootnote{4.3 Understand \textit{udyataḥ} {\rm (}nom.{\rm )} in an active sense {\rm (}`holding/lifting'{\rm )}.
 }}

  \maintext{vadhārhaḥ puruṣaḥ kaścid vrajet pathi bhayāturaḥ |}%

  \maintext{pṛcchato 'pi na vaktavyaṃ satyaṃ tad vāpi ucyate }||\thinspace4:4\thinspace||%
\translation{A person who is walking on the road and is afraid of being killed should not reply to [people who are potentially dangerous] even if they ask him. This is also called truth[fulness]. \blankfootnote{4.4 `being killed' is not the most obvious translation for
  \textit{vadhārhaḥ} in \textit{pāda} a, but the context suggests that it is not
  a person who `deserves death' that may have been intended.
 }}

  \maintext{na narmayuktam anṛtaṃ hinasti}%

 \nonanustubhindent \maintext{na strīṣu rājan na vivāhakāle |}%

  \maintext{prāṇātyaye sarvadhanāpahāre}%

 \nonanustubhindent \maintext{pañcānṛtaṃ satyam udāharanti }||\thinspace4:5\thinspace||%
\translation{A lie does not hurt when it is connected with joking, with women, O king, at the time of marriage, at the departure from life and when one's entire wealth is about to be taken away. They call these five kinds of lies truths. \blankfootnote{4.5 This \textit{upajāti} verse appears in countless sources, beginning with 
  the \MBH\ {\rm (}see the apparatus{\rm )}. All versions 
  contain a vocative addressing a king, which is out of context in the \VSS, the addressee being Vigatarāga,
  i.e. Viṣṇu disguised as a Brahmin. The redactors did not notice or did not care about this
  small inconsistency. Note the metrical licence that allows the last syllable
  of °\textit{yuktam} to count as long {\rm (}see p.~\pageref{short2long}{\rm )}.
  The reading with \textit{anṛtaṃ}, as opposed to \textit{vacanaṃ}, in \textit{pāda} a, can be found 
  in the apparatus of the \MBH\ critical edition.
 }}

  \maintext{devamānuṣatiryeṣu satyaṃ dharmaḥ paro yataḥ |}%

  \maintext{satyaṃ śreṣṭhaṃ variṣṭhaṃ ca satyaṃ dharmaḥ sanātanaḥ }||\thinspace4:6\thinspace||%
\translation{Since truth is the supreme Dharma in [the world of] gods, humans and animals, truth is the best, the most preferable. Truth is the eternal Dharma. }

  \maintext{satyaṃ sāgaram avyaktaṃ satyam akṣayabhogadam |}%

  \maintext{satyaṃ potaḥ paratrārthaṃ satyaṃ panthāna vistaram }||\thinspace4:7\thinspace||%
\translation{Truth is an unmanifest ocean. Truth yields imperishable pleasures. Truth is a ship bound for the other world. Truth is the wide path. \blankfootnote{4.7 \textit{Pāda} d is slightly problematic because it is difficult to ascertain if some of the
  MSS actually read \textit{panthāna} or \textit{pasthāna} {\rm (}or \textit{yasthāna}{\rm )}. I suspect that \textit{panthāna} 
  is a stem form noun formed {\rm (}metri causa{\rm )} to stand for an irregular nominative of \textit{pathin}.
 }}

  \maintext{satyam iṣṭagatiḥ proktaṃ satyaṃ yajñam anuttamam |}%

  \maintext{satyaṃ tīrthaṃ paraṃ tīrthaṃ satyaṃ dānam anantakam }||\thinspace4:8\thinspace||%
\translation{Truth is said to be the desired path. Truth is the supreme sacrifice. Truth is a pilgrimage place, a supreme pilgrimage place. Truth is endless donation. }

  \maintext{satyaṃ śīlaṃ tapo jñānaṃ satyaṃ śaucaṃ damaḥ śamaḥ |}%

  \maintext{satyaṃ sopānam ūrdhvasya satyaṃ kīrtir yaśaḥ sukham }||\thinspace4:9\thinspace||%
\translation{Truth is virtue, austerity, knowledge. Truth is purity, self-control, and\linebreak tranquillity. Truth is the ladder [that leads] upwards. Truth is fame and glory and happiness. \blankfootnote{4.9 Considering a similar line in the \VARP\ {\rm (}193.36cd, see the apparatus{\rm )}, one 
  wonders if the slightly odd \textit{ūrdhvasya} in \textit{pāda} c is not a corrupt form of 
  \textit{svargasya} somehow.
 }}

  \maintext{aśvamedhasahasraṃ ca satyaṃ ca tulayā dhṛtam |}%

  \maintext{aśvamedhasahasrād dhi satyam eva viśiṣyate }||\thinspace4:10\thinspace||%
\translation{[When] a thousand Aśvamedha sacrifices and truth are measured on a pair of scales, truth indeed surpasses a thousand Aśvamedha sacrifices. }

  \maintext{satyena tapate sūryaḥ satyena pṛthivī sthitā |}%

  \maintext{satyena vāyavo vānti satye toyaṃ ca śītalam }||\thinspace4:11\thinspace||%
\translation{The Sun shines because of truth. The Earth stays in place by truth. The winds blow because of truth. Water has a cooling effect through truth. \blankfootnote{4.11 In general, see sections similar to \VSS\ 4.11--17 on \textit{satya} in \MBH\ 12.192.63--72,
  \RevKhS\ 91.68--70, \VDH\ 55.1ff, \VDHU\ 3.265.1ff, etc.
 Here in \VSS\ 4.11d, and several times below, \textit{satye} is probably 
  to be taken as standing for \textit{satyena}.
 }}

  \maintext{tiṣṭhanti sāgarāḥ satye samayena priyavrataḥ |}%

  \maintext{satye tiṣṭhati govindo balibandhanakāraṇāt }||\thinspace4:12\thinspace||%
\translation{The oceans exist by the truthful encounter with Priyavrata. Govinda abides in truth because He [as Vāmana] stopped [Mahā]Bali [in spite of the fact that this was achieved by a trick]. \blankfootnote{4.12 \textit{Pāda} b, \textit{samayena priyavrataḥ}, probably stand for \textit{samayena priyavratasya} although
  it is unclear to me what exactly \textit{samaya} refers to here.
  
 
 
  For the story of Priyavrata, Manu's son, in which he wanted to turn nights into days by 
  circling aroung Mount Meru in a chariot, and by this produced the seven oceans,
  see, e.g., \BHAGP\ 5.1.30--31: 
  \textit{yāvad avabhāsayati suragirim anuparikrāman bhagavān ādityo
  vasudhātalam ardhenaiva pratapaty ardhenāvacchādayati, tadā hi [priyavrataḥ]
  bhagavadupāsanopacitātipuruṣaprabhāvas tad anabhinandan samajavena
  rathena jyotirmayena rajanīm api dinaṃ kariṣyāmīti saptakṛtvas 
  taraṇim anuparyakrāmad dvitīya iva pataṅgaḥ\thinspace |
  ye vā u ha tadrathacaraṇanemikṛtaparikhātās te sapta sindhava āsan
  yata eva kṛtāḥ sapta bhuvo dvīpāḥ\thinspace |}.
  
 
 
  
  \textit{Pāda}s cd: for a somewhat similar reference to the story of Mahābali, see, e.g., \VAMP\ 65.66:
  
 
  \textit{evaṃ purā cakradhareṇa viṣṇunā} 
  \textit{baddho balir vāmanarūpadhāriṇā}\thinspace |
 
  \textit{śakrapriyārthaṃ surakāryasiddhaye} 
  \textit{hitāya viprarṣabhagodvijānām}\thinspace || 
 }}

  \maintext{agnir dahati satyena satyena śaśinaś caraḥ |}%

  \maintext{satyena vindhyās tiṣṭhanti vardhamāno na vardhate }||\thinspace4:13\thinspace||%
\translation{Fire burns according to truth. The Moon's course is [governed] by truth. It is because of truth that the Vindhya mountain stands in place and that although it was growing, it is not growing [anymore]. \blankfootnote{4.13 \textit{Pāda} a might as well be a reference to a story mentioned in \MANU\ 8.116:
  
 
  \textit{vatsasya hy abhiśastasya purā bhrātrā yavīyasā}\thinspace |
 
  \textit{nāgnir dadāha romāpi satyena jagataḥ spaśaḥ}\thinspace ||
  
 
  Olivelle's translation {\rm (}\citeyear{OlivelleManu}, 311{\rm )}: 
  `Long ago when Vatsa was accused by his younger brother, 
  Fire, the world's spy, did not burn a single hair of his because he told the truth.'
  Olivelle's note on this verse {\rm (}ibid. 311{\rm )} reads:
  `Vatsa was accused by his brother of being the son of a Śūdra woman and thus not 
  a pure Brahmin. Vatsa went through fire to prove his pedigree. See \textit{Pañcaviṃśa Brāhmaṇa}
  14.6.6.'
 
  Since \textit{śaśi} {\rm (}instead of \textit{śaśin}{\rm )} is a possible stem in this text, 
  \textit{śaśir ācaraḥ} {\rm (}\msNa\msNb\msNc{\rm )} in \textit{pāda} b could be acceptable here,
  perhaps standing metri causa for the compound \textit{śaśicaraḥ}.
  Nevertheless, I have chosen to conjecture \textit{śaśinaś caraḥ}, now preferring it
  to my previous conjecture, \textit{śaśinā caraḥ}.
  Other possibilities, suggested by Judit Törzsök and other colleagues, include \textit{śaśibhāskaraḥ}, 
  \textit{śaśigocaraḥ}, \textit{śiśiro 'caraḥ}, and \textit{śiśirāmbhasaḥ}. Similar passages quoted in the apparatus
  suggest that the Moon vaxes, or shines, by truth {\rm (}\textit{satyena vardhate}/\textit{rājate}{\rm )}.
  Compare also a passage in the \MBH\ {\rm (}quoted in the apparatus{\rm )} that 
  compares Hariścandra, renowned for his truthfulness, to the Moon, 
  using the verb \textit{carati}. These passages seem to support a reading close to
  my conjecture.
 
 
 
  While it is not clear if \textit{pāda}s ab refer to specific legends or not,
  \textit{pāda}s cd hint at the story of Agastya and the Vindhya mountain {\rm (}as pointed out to me
  by Judit Törzsök{\rm )}:
  Vindhya became jealous of the Sun's revolving around Mount Meru, and when the Sun 
  refused him the same favour, he decided to grow higher and obstruct the Sun's movement.
  As a solution to this situation, Agastya asked Vidhya to bend down to make 
  it easier for him to reach the south and to remain thus until he retured. 
  Vindhya agreed to do what Agastya asked him but Agastya never returned. 
  See \MBH\ 3.102.1--14 {\rm (}see the word \textit{samaya} in verse 13 in this passage, and compare it to \VSS\ 4.12b{\rm )}:
  
 
  \textit{yudhiṣṭhira uvāca}\thinspace |
 
  \textit{kimarthaṃ sahasā vindhyaḥ pravṛddhaḥ krodhamūrchitaḥ}\thinspace |
 
  \textit{etad icchāmy ahaṃ śrotuṃ vistareṇa mahāmune}\thinspace ||
 
  \textit{lomaśa uvāca}\thinspace |
 
  \textit{adrirājaṃ mahāśailaṃ meruṃ kanakaparvatam}\thinspace |
 
  \textit{udayāstamaye bhānuḥ pradakṣiṇam avartata}\thinspace ||
 
  \textit{taṃ tu dṛṣṭvā tathā vindhyaḥ śailaḥ sūryam athābravīt}\thinspace |
 
  \textit{yathā hi merur bhavatā nityaśaḥ parigamyate}\thinspace |
 
  \textit{pradakṣiṇaṃ ca kriyate mām evaṃ kuru bhāskara}\thinspace ||
 
  \textit{evam uktas tataḥ sūryaḥ śailendraṃ pratyabhāṣata}\thinspace |
 
  \textit{nāham ātmecchayā śaila karomy enaṃ pradakṣiṇam}\thinspace |
 
  \textit{eṣa mārgaḥ pradiṣṭo me yenedaṃ nirmitaṃ jagat}\thinspace ||
 
  \textit{evam uktas tataḥ krodhāt pravṛddhaḥ sahasācalaḥ}\thinspace |
 
  \textit{sūryācandramasor mārgaṃ roddhum icchan paraṃtapa}\thinspace || 5\thinspace ||
 
  \textit{tato devāḥ sahitāḥ sarva eva; sendrāḥ samāgamya mahādrirājam}\thinspace |
 
  \textit{nivārayām āsur upāyatas taṃ; na ca sma teṣāṃ vacanaṃ cakāra}\thinspace ||
 
  \textit{athābhijagmur munim āśramasthaṃ; tapasvinaṃ dharmabhṛtāṃ variṣṭham}\thinspace |
 
  \textit{agastyam atyadbhutavīryadīptaṃ; taṃ cārtham ūcuḥ sahitāḥ surās te}\thinspace ||
 
  \textit{devā ūcuḥ}\thinspace |
 
  \textit{sūryācandramasor mārgaṃ nakṣatrāṇāṃ gatiṃ tathā}\thinspace |
 
  \textit{śailarājo vṛṇoty eṣa vindhyaḥ krodhavaśānugaḥ}\thinspace ||
 
  \textit{taṃ nivārayituṃ śakto nānyaḥ kaś cid dvijottama}\thinspace |
 
  \textit{ṛte tvāṃ hi mahābhāga tasmād enaṃ nivāraya}\thinspace ||
 
  \textit{lomaśa uvāca}\thinspace |
 
  \textit{tac chrutvā vacanaṃ vipraḥ surāṇāṃ śailam abhyagāt}\thinspace |
 
  \textit{so 'bhigamyābravīd vindhyaṃ sadāraḥ samupasthitaḥ}\thinspace || 10\thinspace ||
 
  \textit{mārgam icchāmy ahaṃ dattaṃ bhavatā parvatottama}\thinspace |
 
  \textit{dakṣiṇām abhigantāsmi diśaṃ kāryeṇa kena cit}\thinspace ||
 
  \textit{yāvadāgamanaṃ mahyaṃ tāvat tvaṃ pratipālaya}\thinspace |
 
  \textit{nivṛtte mayi śailendra tato vardhasva kāmataḥ}\thinspace ||
 
  \textit{evaṃ sa samayaṃ kṛtvā vindhyenāmitrakarśana}\thinspace |
 
  \textit{adyāpi dakṣiṇād deśād vāruṇir na nivartate}\thinspace ||
 
  \textit{etat te sarvam ākhyātaṃ yathā vindhyo na vardhate}\thinspace |
 
  \textit{agastyasya prabhāvena yan māṃ tvaṃ paripṛcchasi}\thinspace || 14\thinspace ||
 }}

  \maintext{lokālokaḥ sthitaḥ satye meruḥ satye pratiṣṭhitaḥ |}%

  \maintext{vedās tiṣṭhanti satyeṣu dharmaḥ satye pratiṣṭhati }||\thinspace4:14\thinspace||%
\translation{The [mythical] Lokāloka mountains are located in truth. Mount Meru stands by truth. The Vedas abide in truth. Dharma is rooted in truth. }

  \maintext{satyaṃ gauḥ kṣarate kṣīraṃ satyaṃ kṣīre ghṛtaṃ sthitam |}%

  \maintext{satye jīvaḥ sthito dehe satyaṃ jīvaḥ sanātanaḥ }||\thinspace4:15\thinspace||%
\translation{The milk a cow yields is truth. Ghee in milk is present as truth. The soul dwells in the body by truth. The eternal soul is truth. \blankfootnote{4.15 \textit{satye} {\rm (}for \textit{satyena}?{\rm )} in \textit{pāda} c may also stand for \textit{satyaṃ}: `The soul dwells in the body as truth.'
 }}

  \maintext{satyam ekena samprāpto dharmasādhananiścayaḥ |}%

  \maintext{rāmarāghavavīryeṇa satyam ekaṃ surakṣitam }||\thinspace4:16\thinspace||%
\translation{If truth is obtained by somebody {\rm (}\textit{ekena}{\rm )}, he/she will be one for whom Dharma is surely accomplished. By the heroism of Rāma Rāghava, the only truth was well-guarded. \blankfootnote{4.16 Or: `If truth alone {\rm (}\textit{ekena}{\rm )} is obtained, Dharma is surely accomplished.'
 }}

  \maintext{evaṃ satyavidhānasya kīrtitaṃ tava suvrata |}%

  \maintext{sarvalokahitārthāya kim anyac chrotum icchasi }||\thinspace4:17\thinspace||%
\translation{Thus have [I] taught the rules of truth to you, O virtuous one, to favour the whole world. What else do you wish to hear? }

  \subchptr{yameṣv asteyam {\rm {\rm (}3{\rm )}}}%

  \trsubchptr{Third Yama-rule: refraining from stealing}%

  \maintext{vigatarāga uvāca |}%

  \maintext{na hi tṛptiṃ vijānāmi śrutvā dharmaṃ tavāpy aham |}%

  \maintext{upariṣṭād ato bhūyaḥ kathayasva tapodhana }||\thinspace4:18\thinspace||%
\translation{Vigatarāga spoke: I can't have enough of learning about [this teaching of] your[s on] Dharma. Teach me further than this, O great ascetic. \blankfootnote{4.18 It is not inconceivable that \textit{tava} is meant to carry the sense of the ablative
  {\rm (}`I can't have enough of learning about Dharma from you'{\rm )}.
 }}

  \maintext{anarthayajña uvāca |}%

  \maintext{steyaṃ śṛṇv atha viprendra pañcadhā parikīrtitam |}%

  \maintext{adattādānam ādau tu utkocaṃ ca tataḥ param |}%

  \maintext{prasthavyājas tulāvyājaḥ prasahyasteya pañcamam }||\thinspace4:19\thinspace||%
\translation{Anarthayajña spoke: Now listen to [my teaching about] stealing, O great Brahmin, which is taught to be of five kinds. Firstly, [listen to] theft, then bribery, cheating with weights, cheating with scales, and the fifth kind, robbery. \blankfootnote{4.19 `Theft' {\rm (}\textit{adattādāna}{\rm )}: literally `taking what has not been given.'
 Note the stem form °\textit{steya} in \textit{pāda} f.
 }}

  \maintext{dhṛṣṭaduṣṭaprabhāvena paradravyāpakarṣaṇam |}%

  \maintext{vāryamāṇo 'pi durbuddhir adattādānam ucyate }||\thinspace4:20\thinspace||%
\translation{When somebody's wealth is taken away by an impudent and wicked person, it is called theft, even if that fool is prevented [from committing the\linebreak crime]. \blankfootnote{4.20 My impression is that \textit{prabhāva} in \textit{pāda} a stands for \textit{bhāva}, \textit{duṣṭabhāva} {\rm (}`vicious'{\rm )}
  being a common expression.
 The implications of \textit{vāryamāṇo} in \textit{pāda} c are unclear to me, therefore
  my translation is tentative. One could consider emending to \textit{vāryamāṇāpi},
  possibly suggesting that `it is a wicked thought {\rm (}\textit{durbuddhi}{\rm )} even if suppressed {\rm (}\textit{vāryamāṇa}{\rm )}.'
 }}

  \maintext{utkocaṃ śṛṇu viprendra dharmasaṃkarakārakam |}%

  \maintext{mūlyaṃ kāryavināśārtham utkocaḥ parigṛhyate |}%

  \maintext{tena cāsau vijānīyād dravyalobhabalāt kṛtam }||\thinspace4:21\thinspace||%
\translation{O great Brahmin, listen to bribery, which causes confusion in Dharma. A sum of money taken in order to dismiss a lawsuit is a bribe. Therefore this [also] should be considered as such [i.e.\ as stealing because] it is committed out of greed for material goods. \blankfootnote{4.21 Note that \textit{mūlyaṃ} in \textit{pāda} c is a conjecture for \textit{mūla}. It is partly based on 
  a relevant passage in the \Mitaksara\ {\rm (}ad \YajnS\ 2.176cd{\rm )}:
  \textit{paṇyasya krītadravyasya yan mūlyaṃ dattam, bhṛtir vetanaṃ kṛtakarmaṇe dattam}\dots\ 
  \textit{utkocena kāryapratibandhanirāsārtham adhikṛtebhyo dattam}\dots\ 
 Note \textit{asau} in \textit{pāda} e as an accusative form {\rm (}for \textit{amum} or \textit{adaḥ}{\rm )}. It is not unlikely that 
  \textit{tena} is a corruption from \textit{stena}, and the \textit{pāda} may have originally read 
  \textit{stenaṃ taṃ ca vijānīyād} {\rm (}`he should be known as a thief'{\rm )}, or similar {\rm (}cf. 4.22c below{\rm )}. 
  \msM\ {\rm (}f. 7r{\rm )} reads \textit{tena steya vijānīyād} here.
 }}

  \maintext{prasthavyāja-upāyena kuṭumbaṃ trātum icchati |}%

  \maintext{taṃ ca stenaṃ vijānīyāt paradravyāpahārakam }||\thinspace4:22\thinspace||%
\translation{[Even if] somebody wants to protect a family by the method of cheating with weights, that person should be considered a thief, because he takes away other people's wealth. }

  \maintext{tulāvyāja-upāyena parasvārthaṃ hared yadi |}%

  \maintext{cauralakṣaṇakāś cānye kūṭakāpaṭikā narāḥ }||\thinspace4:23\thinspace||%
\translation{If somebody takes away somebody else's belongings by the method of cheating with scales, that person is another kind of a deceitful swindler {\rm (}\textit{kūṭa-kāpaṭika}{\rm )} having the characteristics of thieves. \blankfootnote{4.23 I take \textit{anye} in \textit{pāda} c rather liberally, and as connected to
  \textit{pāda}s ab, because I suspect that this verse introduces one single
  category, albeit using clumsy syntax.
 }}

  \maintext{durbalārjavabāleṣu cchadmanā vā balena vā |}%

  \maintext{apahṛtya dhanaṃ mūḍhaḥ sa cauraś cora ucyate }||\thinspace4:24\thinspace||%
\translation{If someone, by deceit or by force, snatches away the wealth of weak and honest people and simpletons, that morally corrupt usurper is [simply] a thief. \blankfootnote{4.24 It is possible that \textit{pāda} d read differently originally, 
  e.g., \textit{sa coraś cora ucyate}, meaning `that thief is [rightly] called a thief'.
 }}

  \maintext{nāsti steyasamaṃ pāpaṃ nāsty adharmaś ca tatsamaḥ |}%

  \maintext{nāsti stenasamākīrtir nāsti stenasamo 'nayaḥ }||\thinspace4:25\thinspace||%
\translation{There is no sin equal to stealing. There is no crime {\rm (}\textit{adharma}{\rm )} equal to it. There is no ill-fame comparable to that of being a thief. There is no bad-conduct comparable to being a thief. }

  \maintext{nāsti steyasamāvidyā nāsti stenasamaḥ khalaḥ |}%

  \maintext{nāsti stenasama ajño nāsti stenasamo 'lasaḥ }||\thinspace4:26\thinspace||%
\translation{There is no greater ignorance than stealing. There are no bigger rouges than thieves. There is nobody as ignorant as a thief. There is no lazy person that is comparable to a thief. \blankfootnote{4.26 Note the peculiar sandhi in \textit{pāda} c {\rm (}°\textit{sama ajño}{\rm )}, which still leaves the \textit{pāda} 
  a \textit{sa-vipulā}.
 }}

  \maintext{nāsti stenasamo dveṣyo nāsti stenasamo 'priyaḥ |}%

  \maintext{nāsti steyasamaṃ duḥkhaṃ nāsti steyasamo 'yaśaḥ }||\thinspace4:27\thinspace||%
\translation{There is nobody as detestable as a thief. There is nobody disliked as much as a thief. There is no greater suffering than stealing. There is no greater disgrace than theft. \blankfootnote{4.27 Note how \textit{stena} and \textit{steya} are used interchangeably {\rm (}or chaotically{\rm )}
  in the above passages in the MSS to denote both `thief' and 'theft/stealing'.
  The scribe of \msNc\ ends up writing \textit{stenya} in 4.27e.
 }}

  \maintext{pracchanno hriyate 'rtham anyapuruṣaḥ pratyakṣam anyo haret}%

 \nonanustubhindent \maintext{nikṣepād dhanahāriṇo 'nya{-}m{-}adhamo vyājena cānyo haret |}%

  \maintext{anye lekhyavikalpanāhṛtadhanā {\rm †}anyo hṛtād vai hṛtā{\rm †}}%

 \nonanustubhindent \maintext{anyaḥ krītadhano 'paro dhayahṛta ete jaghanyāḥ smṛtāḥ }||\thinspace4:28\thinspace||%
\translation{Some [thieves] take away [other people's] wealth in disguise, some in broad daylight. Other wicked people take money from deposits, and some people steal through fraud. Some gather wealth by forging documents, others steal from stolen money[?]. Some people's wealth is from purchased [children?] {\rm (}\textit{krīta}{\rm )}. Others take away others' inheritance[?]. These are considered the vilest. \blankfootnote{4.28 Metre \textit{śārdūlavikrīḍita}. It appears that \textit{hriyate} in \textit{pāda} a is to be taken as an active verb {\rm (}\textit{harate}{\rm )}.
  Note also how \msCb\ and \msNc\ read the same here against the other witnesses.
 Take °\textit{hāriṇo} in \textit{pāda} b as singular and \textit{m} in \textit{'nya-m-adhamo} as a sandhi-bridge.
  Alternatively, read as plural: °\textit{hāriṇo 'nya adhamo}\dots\ 
 The second half of \textit{pāda} c is difficult to reconstruct.
 The translation of \textit{pāda} d is mostly guesswork. Tentatively, I take \textit{krīta} as \textit{krītaka} {\rm (}`a purchased son', see
  \MANU\ 9.174{\rm )}. \textit{dhayahṛta} makes little sense to me. Florinda De Simini suggested that
  \textit{dhaya} might stand for \textit{daya}, which in turn may stand for \textit{dāya} {\rm (}`inheritance'{\rm )} metri causa.
  Lacking any better solution, I supplied these in my translation, marked with question marks.
  Note also the metrical licence that the last syllable of \textit{dhayahṛta} counts as long.
 }}

  \maintext{stenatulya na mūḍham asti puruṣo dharmārthahīno 'dhamaḥ}%

 \nonanustubhindent \maintext{yāvaj jīvati śaṅkayā narapateḥ saṃtrasyamāno raṭan |}%

  \maintext{prāptaḥśāsana tīvrasahyaviṣamaṃ prāpnoti karmeritaḥ}%

 \nonanustubhindent \maintext{kālena mriyate sa yāti nirayam ākrandamāno bhṛśam }||\thinspace4:29\thinspace||%
\translation{There is no bigger idiot than a thief, who is a wicked person without Dharma and financial gain {\rm (}\textit{artha}{\rm )}. As long as he lives, he trembles in fear of the king, wailing. Having received his punishment, he gets into severe and [in]tolerable difficulties, propelled by [his] karma. When his time comes, he dies and goes to hell, weeping vehemently. \blankfootnote{4.29 For some time I was wondering if one should accept \Ed's reading \textit{stenastulya na mūḍham asti} 
  as a metri causa version of \textit{stenatulyo na mūḍho 'sti}; see a similar case of a nominative ending
  inside of compound in \textit{pāda} c below. One major concern remained:
  the accepted reading would be of an edition that rarely emerges as 
  the sole transmitter of the best reading. Another possible solution could be 
  to emend to \textit{stenaṃtulya}\dots, meaning `there is no bigger foolishness than theft',
  but then the second part of \textit{pāda} a is difficult to connect. In the end,
  I decided to go for the most widely attested reading {\rm (}\textit{stenatulya}{\rm )},
  which is unmetrical.
 
  
  
 Understand \textit{prāptaḥśāsana tīvrasahyaviṣamaṃ} in \textit{pāda} c as \textit{prāptaśāsanas tīvram asahyaṃ ca viṣamaṃ prāpnoti}.
  Alternatively, understand \textit{tīvrasahya}° as \textit{duḥsahya}°.
  The actual reading of \msCa, \textit{prāptaś}, lost in the process of normalization and standing
  in contrast with that of all other MSS that read \textit{prāptaḥ}, may suggest
  a doubling of the \textit{ś} of \textit{śāsana} metri causa.
  More likely is that a licence of having a nominative ending inside of a compound
  is applied here, as may have been the case above in \textit{pāda} a.
 }}

  \maintext{nītvā durgatikoṭikalpa nirayāt tiryatvam āyānti te}%

 \nonanustubhindent \maintext{tiryatve ca tathaivam ekaśatikaṃ prabhramya varṣārbudam |}%

  \maintext{mānuṣyaṃ tad avāpnuvanti vipule dāridryarogākulaṃ}%

 \nonanustubhindent \maintext{tasmād durgatihetu karma sakalaṃ tyaktvā śivaṃ cāśrayet }||\thinspace4:30\thinspace||%
\translation{Having spent ten million \ae ons of suffering, they emerge from hell to the state of animal existence. Similarly, they roam about in animal existence for a hundred and one times ten million years. Then they reach the status of human existence on earth which is full of poverty and disease. Then abandoning all one's karmas, the causes of suffering, one seeks refuge in Śiva. \blankfootnote{4.30 Note the stem form °\textit{kalpa} for °\textit{kalpaṃ} metri causa in \textit{pāda} a.
 In \textit{pāda} c, \textit{tathaivam}, or \textit{tathaikam}, and \textit{ekaśatikaṃ} are suspect.
 I understand \textit{vipule} as \textit{vipulāyāṃ}, \textit{vipulā} appearing in \Amara\ 2.1.7 as a synonym of
  \textit{dhātrī}, `earth.' It is difficult to interpret it otherwise.
  This is still problematic because both human and
  animal existence takes place on earth, thus, if \textit{tiryatva} {\rm (}i.e. \textit{tiryaktva}{\rm )} 
  indeed means `animal existence,' there is no contrast between \textit{pāda}s b and c as
  regards location. As for \textit{tiryaktva}, see, e.g., \MANU\ 12.40:
  
 
  \textit{devatvaṃ sāttvikā yānti manuṣyatvaṃ ca rājasāḥ}\thinspace |
  \textit{tiryaktvaṃ tāmasā nityam ity eṣā trividhā gatiḥ}\thinspace ||
  
 
  It is not unlikely that the original form of \textit{dāridryarogākulam} was \textit{dāridryarogākule},
  picking up \textit{vipule}.
 Note the switch from plural to singular in \textit{pāda} d {\rm (}\textit{āśrayet}{\rm )}.
 }}

  \subchptr{yameṣv ānṛśaṃsyam {\rm {\rm (}4{\rm )}}}%

  \trsubchptr{Fourth Yama-rule: absence of hostility}%

  \maintext{aṣṭamūrtiśivadveṣṭā pitur mātuś ca yo dviṣet |}%

  \maintext{gavāṃ vā atither dveṣṭā nṛśaṃsāḥ pañca eva te }||\thinspace4:31\thinspace||%
\translation{The one who is hostile towards the eight-formed Śiva, he who hurts his mother or father, he who is hostile towards cows or guests: these are the five types of hostile people. \blankfootnote{4.31 Note \textit{pitur} and \textit{mātur} used as accusative forms in \textit{pāda} b, or rather,
  understand \textit{pitur mātuś ca yo dveṣṭā}, i.e. \textit{dviṣet} is
  metri causa for \textit{dveṣṭā}.
 }}

  \maintext{aṣṭamūrtiḥ śivaḥ sākṣāt pañcavyomasamanvitaḥ |}%

  \maintext{sūryaḥ somaś ca dīkṣaś ca dūṣakaḥ sa nṛśaṃsakaḥ }||\thinspace4:32\thinspace||%
\translation{Śiva, when manifest {\rm (}\textit{sākṣāt}{\rm )}, has eight form, possessing the five elements {\rm (}\textit{vyoman}{\rm )}, and the Sun, the Moon, and the sacrificer. [He who] disgraces [any of these] is a hostile person. \blankfootnote{4.32 Törzsök has suggested emending \textit{sa nṛśaṃsakaḥ} in \textit{pāda} d to \textit{tannṛṃśakaḥ}. I don't think that it is
  inevitably necessary. I think that \textit{pāda}s a-c form a list that is meant to be in the genitive, understanding
  \dots\ \textit{ity eteṣāṃ dūṣakaḥ sa nṛśaṃsakaḥ} or similar. This is clumsy but in a way that is
  more than possible within the style of this text.
 
  I have not been able find any other attestation of \textit{vyoman} meaning the five elements. Perhaps it is meant
  to mean \textit{vyomādi} {\rm (}`the atmosphere/sky and the other four elements'{\rm )}. 
  
  For Śiva of eight forms, see, e.g., \textit{Śakuntalā} 1.1:
  
 
  [1] \textit{yā sṛṣṭiḥ sraṣṭur ādyā vahati} [2] \textit{vidhihutaṃ yā havir} [3] \textit{yā ca hotrī}
 
  [4, 5] \textit{ye dve kālaṃ vidhattaḥ} [6] \textit{śruti-viṣaya-guṇā yā sthitā vyāpya viśvam}\thinspace |
 
  [7] \textit{yām āhuḥ sarva-bīja-prakṛtir iti yayā prāṇinaḥ prāṇavantaḥ} [8]
 
  \textit{pratyakṣābhiḥ prapannas tanubhir avatu vas tābhir aṣṭābhir īśaḥ}\thinspace ||
  
 
  Here the eight \textit{mūrti}s, or rather, \textit{tanu}s, are: 
  [1] \textit{jala}, [2] \textit{agni}, [3] \textit{hotrī} {\rm (}`the form that sacrifices'{\rm )}, [4 + 5] \textit{sūrya} + \textit{candra},
  [6] \textit{ākāśa}, [7] \textit{bhūmi}, [8] \textit{vāyu}.
 
  For a similar interpretation of \textit{aṣṭamūrti}, see, e.g., 
  \Isanasiva\ 2.29.34 {\rm (}\textit{mantrapāda}; note \textit{yajamāna} for our \textit{dīkṣa}{\rm )}:
  
 
  \textit{kṣmā-vahni-yajamānārka-jala-vāyv-indu-puṣkaraiḥ}\thinspace |
 
  \textit{aṣṭābhir mūrtibhiḥ śambhor dvitīyāvaraṇaṃ smṛtam}\thinspace ||
  
 
  {\rm (}For \textit{puṣkara} as `sky, atmosphere', see, e.g., \Amara\ 1.2.167:
  \textit{dyodivau dve striyām abhraṃ vyoma puṣkaram ambaram}.{\rm )}
 
  A closely related Aṣṭamūrti-hymn appears in \Nisvmukh\ 1.30--41 {\rm (}I owe thanks to Nirajan Kafle
  for drawing my attention to this{\rm )}; see \mycitep{KafleNisvasaBook}{62, 63, 116, 119}. 
  Kafle notes that this hymn is closely parallel to some passages in the \textit{Prayogamañjarī} {\rm (}1.19--26{\rm )},
  the \textit{Tantrasamuccaya} {\rm (}1.16--23{\rm )}, and the \textit{Īśānaśivagurudevapaddhati} {\rm (}\textit{kriyāpāda} 26.56--63{\rm )}. 
  See also \TAKI\ s.v. \textit{aṣṭamūrti}.
 }}

  \maintext{pitākāśasamo jñeyo janmotpattikaraḥ pitā |}%

  \maintext{pitṛdaivata{\rm †}m ādiś cam ānṛśaṃsa tamanvitaḥ{\rm †} }||\thinspace4:33\thinspace||%
\translation{The father is to be considered similar to the [element] sky, he is the cause of one's birth. One should not be hostile to a father, god\dots[?]. \blankfootnote{4.33 It is difficult to restore \textit{pāda}s cd, although the general meaning of this line is
  predictable. Some questions remain. Is \textit{āditya} a good reading or is \textit{mātṛ} hidden in
  \textit{daivata-mādiśca}? Is \textit{ānṛśaṃsa} right or was it \textit{nṛśaṃsa} that was meant by the author of this line?
  Does \textit{tamanvitaḥ} {\rm (}or \textit{tamānvitaḥ}{\rm )} has anything to do with \textit{tamas} {\rm (}`darkness'{\rm )}?
 }}

  \maintext{pṛthvyā gurutarī mātā ko na vandeta mātaram |}%

  \maintext{yajñadānatapovedās tena sarvaṃ kṛtaṃ bhavet }||\thinspace4:34\thinspace||%
\translation{The mother is more venerable than the earth. Who would not praise a mother? By that [praise], sacrifices, donations, austerities and [the study of] the Vedas, all will be completed. }

  \maintext{gāvaḥ pavitraṃ maṅgalyaṃ devatānāṃ ca devatāḥ |}%

  \maintext{sarvadevamayā gāvas tasmād eva na hiṃsayet }||\thinspace4:35\thinspace||%
\translation{Cows are an auspicious blessing, they are the gods of the gods. Cows contain in themselves all the gods. That is exactly why one should not hurt them. }

  \maintext{jātamātrasya lokasya gāvas trātā na saṃśayaḥ |}%

  \maintext{ghṛtaṃ kṣīraṃ dadhi mūtraṃ śakṛtkarṣaṇam eva ca }||\thinspace4:36\thinspace||%
\translation{Cows are the protectors of the world as if the world were their new-born [calf], there is no doubt about it. Collecting [the five products of the cow, the \textit{pañcagavya},] ghee, milk, curd, and [the cow's] urine and dung [is auspicious]. \blankfootnote{4.36 Note the number confusion in the phrase \textit{gāvas trātā}, for \textit{gāvas trātāras}. Alternatively,
  this line might try to echo \Harivamsa\ 45.30ab: 
  \textit{trātavyāḥ prathamaṃ gāvas trātās trāyanti tā dvijān} 
  {\rm (}`First the cows should be protected. When protected, they protect the Brahmins'{\rm )}.
 \textit{Pāda} c is a \textit{sa-viplulā}. 
  The use of \textit{karsaṇa} in \textit{pāda} d, most probably in the sense of `collecting,' is slightly odd.
 }}

  \maintext{pañcāmṛtaṃ pañcapavitrapūtaṃ}%

 \nonanustubhindent \maintext{ye pañcagavyaṃ puruṣāḥ pibanti |}%

  \maintext{te vājimedhasya phalaṃ labhanti}%

 \nonanustubhindent \maintext{tad akṣayaṃ svargam avāpnuvanti }||\thinspace4:37\thinspace||%
\translation{People who drink the five products of the cow, the five nectars, purified by the five Pavitras, will obtain the fruits of a horse sacrifice, and then reach the undecaying heavens. \blankfootnote{4.37 The five \textit{pavitra}s can be the five \textit{brahmamantras}, see, e.g., \TAKIII\ s.v. \textit{pavitra} 1.
 }}

  \maintext{gobhir na tulyaṃ dhanam asti kiṃcid}%

 \nonanustubhindent \maintext{duhyanti vāhyanti bahiś caranti |}%

  \maintext{tṛṇāni bhuktvā amṛtaṃ sravanti}%

 \nonanustubhindent \maintext{vipreṣu dattāḥ kulam uddharanti }||\thinspace4:38\thinspace||%
\translation{There is no wealth comparable to a cow. They yield milk, they carry things, they roam under the sky. Feeding on grass, they issue nectar. When given to Brahmins, they deliver the family [from \textit{saṃsāra} or the suffering experienced in hell]. \blankfootnote{4.38 Note that \textit{duhyanti} and \textit{vāhyanti} are supposed to be understood as passive,
  as in the similar verse in \SDHU\ 12.92 {\rm (}see apparatus{\rm )}.
 }}

  \maintext{gavāhnikaṃ yaś ca karoti nityaṃ}%

 \nonanustubhindent \maintext{śuśrūṣaṇaṃ yaḥ kurute gavāṃ tu |}%

  \maintext{aśeṣayajñatapadānapuṇyaṃ}%

 \nonanustubhindent \maintext{labhaty asau tām anṛśaṃsakartā }||\thinspace4:39\thinspace||%
\translation{He who feeds the cows daily, he who serves the cows, he who is kind to her [i.e. to the cow], will obtain the merits of all sacrifices, austerities and donation. \blankfootnote{4.39 Strictly speaking, \textit{pāda} c is unmetrical. The second syllable of \textit{yajña} counts as
  long {\rm (}see Introduction p.~\pageref{short2long}{\rm )}.
 Although the accusative with °\textit{kartā} in \textit{pāda} d is still not optimal, my 
  emendation of \textit{tam} to \textit{tām} at least restores the metre and improves 
  upon the meaning of the sentence. Alternatively, as suggested by Törzsök,
  \textit{taṃ} could be understood as \textit{tad}, picking up \textit{puṇyaṃ} in \textit{pāda} c,
  but in this way any reference to cows here is only implied.
 }}

  \maintext{atithiṃ yo 'nugaccheta atithiṃ yo 'numanyate |}%

  \maintext{atithiṃ yo 'nupūjyeta atithiṃ yaḥ praśaṃsate }||\thinspace4:40\thinspace||%
\translation{He who looks after a guest, he who respects a guest, he who worships a guest, he who praises a guest, \blankfootnote{4.40 Note the peculiar active verb forms \textit{anugaccheta} and \textit{anupūjyeta}.
  On this formation, see a remark about \Nisvmul\ 2.8 in \mycitep{NisvasaGoodall}{247}:
  `We have assumed that \textit{pūjyeta} is intended to mean \textit{pūjayet} and is
  perhaps a contraction of \textit{pūjayeta}.'
 }}

  \maintext{atithiṃ yo na pīḍyeta atithiṃ yo na duṣyati |}%

  \maintext{atithipriyakartā yaḥ atitheḥ paricārakaḥ |}%

  \maintext{atitheḥ kṛtasaṃtoṣas tasya puṇyam anantakam }||\thinspace4:41\thinspace||%
\translation{he who does not harm a guest, he who does not commit a fault towards a guest, he who keeps the guest happy, he who attends to the needs of a guest, he who makes a guest satisfied: his merits are endless. \blankfootnote{4.41 On the form \textit{pīḍyeta}, see previous note.
 }}

  \maintext{āsanenārghapātreṇa pādaśaucajalena ca |}%

  \maintext{annavastrapradānair vā sarvaṃ vāpi nivedayet }||\thinspace4:42\thinspace||%
\translation{He should offer [the guest] a seat, a vessel with water-offering, and water for washing his feet, or gifts of food and clothes, or all [of these]. \blankfootnote{4.42 My conjecture in \textit{pāda} a {\rm (}°\textit{pātreṇa} for °\textit{pādyena}{\rm )} is inspired by the fact that 
  in the MSS \textit{pāda} b seems to awkwardly repeat what °\textit{pādyena} in \textit{pāda} a signifies.
 }}

  \maintext{putradārātmano vāpi yo 'tithim anupūjayet |}%

  \maintext{śraddhayā cāvikalpena aklībamānasena ca }||\thinspace4:43\thinspace||%
\translation{He who worships the guest by [offering him] his own son or wife with willingness, without hesitation, and with a brave heart, \blankfootnote{4.43 I analyse \textit{pāda} a as if it read \textit{putradārair ātmano} {\rm (}\textit{putradāraiḥ} being a
  common expression{\rm )}. Another solution would be to emend to °\textit{ātmanā},
  and thus to include the possibility of sacrificing one's own life for the guest.
  
 
  For the requirement that one should in certain circumstances 
  part with his wife or son, or his own life,
  for the benefit of someone else, see \VSS\ 2.38, and the narrative in \VSS\ chapter 12
  which tells about a Brahmin giving away his own wife to a guest.
  Note that in fact \VSS\ 4.44cd below echoes verse 37cd in the above mentioned chapter 12
  {\rm (}see the apparatus{\rm )}.
 }}

  \maintext{na pṛcched gotracaraṇaṃ svādhyāyaṃ deśajanmanī |}%

  \maintext{cintayen manasā bhaktyā dharmaḥ svayam ihāgataḥ }||\thinspace4:44\thinspace||%
\translation{and does not ask [the guest about his] lineage, Vedic affiliation {\rm (}\textit{caraṇa}{\rm )}, studies, country or birth, and imagines mentally, with devotion, that it is Dharma himself who has come to visit, }

  \maintext{aśvamedhasahasrāṇi rājasūyaśatāni ca |}%

  \maintext{puṇḍarīkasahasraṃ ca sarvatīrthatapaḥphalam }||\thinspace4:45\thinspace||%
\translation{[will obtain all the fruits of] thousands of Aśvamedha sacrifices and hundreds of Rājasūya sacrifices, a thousand Puṇḍarīka sacrifices and the fruit of [visiting] all the pilgrimage places and [performing] all the austerities; }

  \maintext{atithir yasya tuṣyeta nṛśaṃsamatam utsṛjet |}%

  \maintext{sa tasya sakalaṃ puṇyaṃ prāpnuyān nātra saṃśayaḥ }||\thinspace4:46\thinspace||%
\translation{he whose guest is satisfied [and] he who can abandon the sentiment of cruelty, will obtain all the merits of the above, there is no doubt about it. \blankfootnote{4.46 The demonstrative pronoun \textit{tasya} in \textit{pāda} c may refer to the guest:
  `he will obtain all his [i.e. the guest's] merits,' hinting at some sort of karmic exchange.
  Nevertheless, I think rather that \textit{tasya} points to the merits one can obtain by the rituals listed 
  in the previous verse. This is suggested by passages such as the following:
  
 
  \MBH\ Suppl. 13.14.379--380:
  
 
  \textit{ahany ahani yo dadyāt kapilāṃ dvādaśīḥ samāḥi}\thinspace | 
 
  \textit{māsi māsi ca satreṇa yo yajeta sadā naraḥ}\thinspace || 
 
  \textit{gavāṃ śatasahasraṃ ca yo dadyāj jyeṣṭhapuṣkare}\thinspace | 
 
  \textit{na taddharmaphalaṃ tulyam atithir yasya tuṣyati}\thinspace || 
 
  
 
  \BRAHMAVP\ 3.44--46:
  
 
  \textit{atithiḥ pūjito yena pūjitāḥ sarvadevatāḥ}\thinspace | 
 
  \textit{atithir yasya saṃtuṣṭas tasya tuṣṭo hariḥ svayam}\thinspace || 
 
  \textit{snānena sarvatīrtheṣu sarvadānena yat phalam}\thinspace | 
 
  \textit{sarvavratopavāsena sarvayajñeṣu dīkṣayā}\thinspace || 
 
  \textit{sarvais tapobhir vividhair nityair naimittikādibhiḥ}\thinspace | 
 
  \textit{tad evātithisevāyāḥ kalāṃ nārhanti ṣoḍaśīm}\thinspace ||
 }}

  \maintext{{\rm †}na gatim atithijñasya{\rm †} gatim āpnoti karhacit |}%

  \maintext{tasmād atithim āyāntam abhigacchet kṛtāñjaliḥ }||\thinspace4:47\thinspace||%
\translation{One will never reach a path that is the path of one who knows his guest.[?] Therefore one should go up to the arriving guest with respectfully joined palms. \blankfootnote{4.47 Something has gone wrong with \textit{pāda}s ab and I am unable to reconstruct the
  meaning. The translation tries to reflect what is actually transmitted.
  The line may have begun with something like \textit{nāgatātithyavajña}°
  {\rm (}`he who despises a guest that has arrived will not\dots'{\rm )}. I have accepted
  \textit{karhacit} for standard \textit{karhicit} in \textit{pāda} b because it is attested in Buddhist texts,
  see \mycitep{EdgertonHybrid}{s.v. \textit{karhacid}}, and 
  because the readings support it overwhelmingly, unlike in 4.3b above.
 }}

  \maintext{saktuprasthena caikena yajña āsīn mahādbhutaḥ |}%

  \maintext{atithiprāptadānena svaśarīraṃ divaṃ gatam }||\thinspace4:48\thinspace||%
\translation{By one \textit{prastha}[, a small unit of weight] of coarsely ground grains given to a guest, an extremely wonderful sacrifice was performed, and his body [i.e. the protagonist in his mortal form] reached heaven. \blankfootnote{4.48 This verse is a reference to the story related by a mongoose in \MBH\ 14.92--93: 
  A Brahmin who practises the vow of gleaning {\rm (}\textit{uñcha}{\rm )} and his family
  receive a guest. They feed the guest with the last morsels of the little food
  they have. In the end, the guest reveals that he is in fact Dharma {\rm (}14.93.80cd{\rm )} and as 
  a reward the family departs to heaven. The noble act of the poor Brahmin and his family
  is depicted as yielding greater rewards than Yudhiṣṭhira's grandiose horse-sacrifice. 
  {\rm (}See an analysis of this story in \mycite{TakahashiUnca}.{\rm )}
 
  
 We would be forced to accept the reading of \Ed\ in \textit{pāda} d {\rm (}\textit{saśarīro}{\rm )} 
  if the expression were in the masculine {\rm (}\textit{divaṃ gataḥ}{\rm )}. This would make sense
  and it would also echo expressions occuring, e.g., in the \MBH:
  3.164.33cd: \textit{paśya puṇyakṛtāṃ lokān saśarīro divaṃ vraja};
  14.5.10cd: \textit{saṃjīvya kālam iṣṭaṃ ca saśarīro divaṃ gataḥ}.
  It is tempting to emend accordingly, but instead I have retained 
  \textit{svaśarīraṃ divaṃ gatam}, and I interpret it in a general way.
 }}

  \maintext{nakulena purādhītaṃ vistareṇa dvijottama |}%

  \maintext{viditaṃ ca tvayā pūrvaṃ prasthavārttā ca kīrtitā }||\thinspace4:49\thinspace||%
\translation{The mongoose related [this story in the \textit{Mahābhārata}] in the past in detail, O great Brahmin, and you must know it already. The story of the \textit{prastha} is well-known. \vfill\pagebreak }

  \subchptr{yameṣu damaḥ {\rm {\rm (}5{\rm )}}}%

  \trsubchptr{Fifth Yama-rule: self-restraint}%

  \maintext{dama eva manuṣyāṇāṃ dharmasārasamuccayaḥ |}%

  \maintext{damo dharmo damaḥ svargo damaḥ kīrtir damaḥ sukham }||\thinspace4:50\thinspace||%
\translation{Self-restraint is in itself the distilled essence of Dharma for man.\linebreak Self-re\-straint is Dharma, self-restraint is heaven, self-restraint is fame, self-re\-straint is happiness. }

  \maintext{damo yajño damas tīrthaṃ damaḥ puṇyaṃ damas tapaḥ |}%

  \maintext{damahīna{-}m{-}adharmaś ca damaḥ kāmakulapradaḥ }||\thinspace4:51\thinspace||%
\translation{Self-restraint is sacrifice, self-restraint is a pilgrimage-place, self-restraint is merit, self-restraint is religious austerity. If one has no self-restraint, one is a sinner {\rm (}\textit{adharma}{\rm )}, [while] self-restraint yields a multitude of desired objects. \blankfootnote{4.51 I suspect that the final \textit{m} in \textit{dhamahīnam} in \textit{pāda} c is a hiatus-filler.
  Understand \textit{dhamahīno 'dharmaś ca}.
  \textit{kāmakulapradaḥ} in \textit{pāda} d is slightly suspect.
  It may have originally read \textit{sarvakāmapradaḥ} {\rm (}`fulfilling all desires'{\rm )} or
  \textit{kulakāmapradaḥ} {\rm (}`fulfilling the desires of the family'{\rm )}.
  \SDHS\ 4.28b reads \textit{sarvakāmasukhapradam}, which opens up further possibilities.
 }}

  \maintext{nirdamaḥ kari mīnaś ca pataṅgabhramaramṛgāḥ |}%

  \maintext{tvag jihvā ca tathā ghrāṇā cakṣuḥ śravaṇam indriyāḥ }||\thinspace4:52\thinspace||%
\translation{The elephant, the fish, the moth, the bee, and the deer are without self-restraint. The senses are the skin, the tongue, the nose, the eye, and\linebreak the ear. \blankfootnote{4.52 Note \textit{kari} for \textit{karī} metri causa, and the end of \textit{pāda} b, °\textit{mṛgāḥ}, which 
  should be treated metrically as if it read °\textit{mrigāḥ}.
 }}

  \maintext{durjayendriyam ekaikaṃ sarve prāṇaharāḥ smṛtāḥ |}%

  \maintext{damaṃ yo jayate 'samyag nirdamo nidhanaṃ vrajet }||\thinspace4:53\thinspace||%
\translation{Each of these sense faculties are hard to conquer and all are known to be fatal [if unconquered]. If one masters self-restraint in a less than proper way, one remains unrestrained and will die. \blankfootnote{4.53 The only way to make sense of \textit{pāda}s cd is to supply and \textit{avagraha} before
  \textit{samyag}. Otherwise some text may have dropped out here.
 }}

  \maintext{mṛge śrotravaśān mṛtyuḥ pataṅgāś cakṣuṣor mṛtāḥ |}%

  \maintext{ghrāṇayā bhramaro naṣṭo naṣṭo mīnaś ca jihvayā }||\thinspace4:54\thinspace||%
\translation{In the case of the deer, death comes about because of hearing [when, e.g., hunters use buck grunts]. Moths die because of their eyes [as they are attracted to the light of a lamp]. Bees perish because of their smelling [as they are attracted to smells], fish because of their tongues [when attracted by the bait]. \blankfootnote{4.54 My comments in square brackets in the translation are tentative. See
  a verse from the \Buddhacarita\ {\rm (}11.35{\rm )} in the apparatus that may have been
  the inspiration for this verse in the \VSS. In Johnston's translation
  {\rm (}\citeyear{Buddhacarita}, II. 157{\rm )}:
  `For deer are lured to their destruction by songs, moths fly into the fire
  for its brightness, the fish greedy for the bait swallows the hook;
  therefore the objects of sense breed calamity.'
 }}

  \maintext{sparśena ca karī naṣṭo bandhanāvāsaduḥsahaḥ |}%

  \maintext{kiṃ punaḥ pañcabhuktānāṃ mṛtyus tebhyaḥ kim adbhutam }||\thinspace4:55\thinspace||%
\translation{The elephant perishes because of touch, not tolerating to be kept in fetters. How much more true it is for those who enjoy all five [senses]! Why should death come as a surprise for them? \blankfootnote{4.55 \textit{Mātaṅgalīlā} 11.1 may shed some light on elephants dying in captivity:
  
 
  \textit{vānyas tatra sukhoṣitā vidhivaśād grāmāvatīrṇā gajā baddhās tīkṣṇakaṭūgravāgbhir}
  \textit{atiśugbhīmohabandhādibhiḥ\thinspace | udvignāś ca manaḥśarīrajanitair duḥkhair atīvākṣamāḥ}
  \textit{prāṇān dhārayituṃ ciraṃ naravaśaṃ prāptāḥ svayūthād atha}\thinspace ||
  
 
  In Edgerton's translation {\rm (}\citeyear{EdgertonElephant}, 92{\rm )}: 
  
 
  `Forest elephants who dwell there happily and by
  the power of fate have been brought to town in bonds, afflicted by harsh, bitter, cruel words,
  by excessive grief, fear, bewilderment, bondage, etc., and by sufferings of mind and body,
  are quite unable for long to sustain life, when from their own herds they have come into
  the control of men.'
 }}

  \maintext{purūravo 'tilobhena atikāmena daṇḍakaḥ |}%

  \maintext{sāgarāś cātidarpeṇa atimānena rāvaṇaḥ }||\thinspace4:56\thinspace||%
\translation{Purūravas [perished] by excessive greed, Daṇḍaka by excessive desire, Sagara's sons by excessive pride, Rāvaṇa by excessive haughtiness, \blankfootnote{4.56 We may treat \textit{purūravo} in \textit{pāda} a as a stem form noun or thematised stem, or imagine that the
  original reading was \textit{purūravā}° with double sandhi:
  \textit{purūravās ati}° $\rightarrow$\ \textit{purūravā ati}° $\rightarrow$\ \textit{purūravāti}°.
 
  \textit{Pāda} a may refer to the following passage in the \MBH\ {\rm (}1.70.16--18, 20ab{\rm )}:
  
 
  \textit{purūravās tato vidvān ilāyāṃ samapadyata}\thinspace | 
 
  \textit{sā vai tasyābhavan mātā pitā ceti hi naḥ śrutam}\thinspace || 
 
  \textit{trayodaśa samudrasya dvīpān aśnan purūravā}\thinspace | 
 
  \textit{amānuṣair vṛtaḥ sattvair mānuṣaḥ san mahāyaśā}\thinspace || 
 
  \textit{vipraiḥ sa vigrahaṃ cakre vīryonmattaḥ purūravā}\thinspace | 
 
  \textit{jahāra ca sa viprāṇāṃ ratnāny utkrośatām ap}\thinspace || 
 
  [\dots] 
 
  \textit{tato maharṣibhiḥ kruddhaiḥ śaptaḥ sadyo vyanaśyata}\thinspace |
  
 
  `The wise Purūravas was born to Ilā. We heard that Ilā 
  was both his mother and his father. 
  The great Purūravas ruled over thirteen islands of the ocean
  and, though human, he was always surrounded by superhuman beings.
  Intoxicated with his power, Purūravas quarrelled with some Brahmins 
  and robbed them of their wealth even though they were protesting. [...]
  Therefore, cursed by the great Ṛṣis, he perished.'
  
 
  See also \BUDDHACARITA\ 11.15 {\rm (}Aiḍa = Purūravas{\rm )}:
  
 
  \textit{aiḍaś ca rājā tridivaṃ vigāhya} 
  \textit{nītvāpi devīṃ vaśam urvaśīṃ tām}\thinspace | 
 
  \textit{lobhād ṛṣibhyaḥ kanakaṃ jihīrṣur} 
  \textit{jagāma nāśaṃ viṣayeṣv atṛptaḥ}\thinspace ||
  
 
  In Johnston's translation {\rm (}\citeyear{Buddhacarita}, II. 152{\rm )}:
  
 
  `Although the royal son of Iḍā penetrated the triple heaven and brought
  the goddess Urvaśī into his power, he was still unsatisfied with the 
  objects of sense and came to destruction in his greedy desire to seize
  gold from the ṛṣis.'
  
 
  For Daṇḍa{\rm (}ka{\rm )}'s story, see \RAMAYANA\ 7.71.31 ff.:
  Daṇḍa meets Arajā, a beautiful girl, in a forest and rapes her. As a consequence, 
  her father, Śukra/Bhārgava, destroys Daṇḍa's kingdom, which thus
  becomes the desolate Daṇḍaka-forest.
 
  
 
  For two versions of the destruction of
  Sagara's sons {\rm (}note emendation in \textit{pāda} c{\rm )}, 
  who were chasing the sacrificial horse of their father's Aśvamedha sacrifice,
  and by doing so disturbed Kapila's meditation, and who in turn burnt them to ashes,
  see \MBH\ 3.105.9 ff. and \BRAHMANDAPUR\ 2.52--53.
 
  
 
  As for Rāvaṇa's haughtiness,
  especially the fact that he chose to be invincible by all creatures except humans,
  and its consequences,
  one should recall the story of the \RAMAYANA\ and Rāvaṇa's destruction brought about by Rāma therein.
 }}

  \maintext{atikrodhena saudāsa atipānena yādavāḥ |}%

  \maintext{atitṛṣṇāc ca māndhātā nahuṣo dvijavajñayā }||\thinspace4:57\thinspace||%
\translation{Saudāsa by excessive anger, the Yādavas by excessive drinking, Māndhātṛ by excessive desire, Nahuṣa by contempt for Brahmins, \blankfootnote{4.57 Saudāsa {\rm (}note the sandhi between the two \textit{pāda}s{\rm )},
  also known as Kalmāṣapāda, hit Śakti, Vasiṣṭha's son, with a whip because
  the latter did not give way to him, and as a consequence Śakti cursed Saudāsa:
  Saudāsa had to roam the world as a Rākṣasa for twelve years. 
  See \MBH\ 1.166.1ff.
 
  
 
  As for the end of the Yādavas, see the short \textit{Mausalaparvan} of the \MBH\ {\rm (}canto 16{\rm )}:
  cursed by the sages Viśvāmitra, Kaṇva and Nārada, and seeing menacing omens,
  the Yādavas take to drinking in Prabhāsa and destroy each other.
 
 Most probably, \textit{atitṛṣṇā} in the MSS stands for \textit{atitṛṣṇāt} {\rm (}intending \textit{atitṛṣṇayā}{\rm )},
  and the forms \textit{māndhāto}/\textit{mandhāto} in \msCb\ stand for \textit{māndhātā} {\rm (}nominative of \textit{māndhātṛ}{\rm )}.
  I have corrected these in spite of the fact that the authors' knowledge about Māndhātṛ's story may
  come from \DIVYAV\ 17, where it sometimes appears to be an a-stem noun {\rm (}\textit{māndhāta}{\rm )}.
  \textit{dvijavajñayā} in \textit{pāda} d stands for \textit{dvijāvajñayā} metri causa.
 
  
 
  Māndhātṛ was born from his father's body who, being excessively thirsty once,
  had drank some decoction prepared for ritual purposes and as a result become pregnant with him.
  Nevertheless, \BUDDHACARITA\ 11.13 suggests that Māndhātṛ himself was still unsatisfied
  with wordly objects even after he had obtained half of Indra's throne:
  
 
  \textit{devena vṛṣṭe 'pi hiraṇyavarṣe} 
  \textit{dvīpān samagrāṃś caturo 'pi jitvā}\thinspace | 
  
  \textit{śakrasya cārdhāsanam apy avāpya}
  \textit{māndhātur āsīd viṣayeṣv atṛptiḥ}\thinspace ||
  
 
  In Johnston's translation {\rm (}\citeyear{Buddhacarita}, II. 151{\rm )}:
  
 
  `Though the heavens rained gold for him and though he conquered the whole
  of the four continents and won half the seat of Śakra, yet Māndhātṛ's longing
  for the objects of sense remained unappeased.'
  
 
  In fact, as Monika Zin points out {\rm (}\mycitep{ZinMandhatar}{149}{\rm )},
  Māndhātṛ/Māndhāta's rise and fall is a very popular theme
  in the `Narrative Art of the Amaravati School': 
  `Statistics show that in the Amaravati School the most frequently represented narrative is
  the story of King Māndhātar, which appears 47 times.'
  
  
 
  Nahuṣa was elevated to the position of Indra for a period of time and he also wanted
  to take Śacī, Indra's wife. Indra instructed Śacī to tell Nahuṣa to 
  harness some Ṛsis to a vehicle and use this vehicle to take Śacī. 
  Agastya, one of the Ṛṣis, was insulted even further by Nahuṣa, therefore
  he cursed Nahuṣa, who then fell from the vehicle. See \MBH\ 12.329.35ff and
  a verse in the \BUDDHACARITA\ {\rm (}11.14{\rm )} that follows the one about Māndhātṛ:
  
 
  \textit{bhuktvāpi rājyaṃ divi devatānāṃ} 
  \textit{śatakratau vṛtrabhayāt pranaṣṭe}\thinspace | 
 
  \textit{darpān maharṣīn api vāhayitvā} 
  \textit{kāmeṣv atṛpto nahuṣaḥ papāta}\thinspace ||
  
 
  In Johnston's translation {\rm (}\citeyear{Buddhacarita}, II. 151{\rm )}:
  
 
  `Although he enjoyed sovereignty over the gods in heaven, when Śatakratu hid himself
  for fear of Vṛtra, and though out of wanton pride he made the great ṛṣis carry him,
  yet Nahuṣa fell, being still unsatisfied with the passions.'
 }}

  \maintext{atidānād balir naṣṭa atiśauryeṇa arjunaḥ |}%

  \maintext{atidyūtān nalo rājā nṛgo goharaṇena tu }||\thinspace4:58\thinspace||%
\translation{[Mahā]bali perished by excessive donations, Arjuna by excessive heroism, King Nala by excessive gambling, Nṛga by taking a cow. \blankfootnote{4.58 \textit{Pāda} a is most probably a reference to Mahābali's promises made to Vāmana that caused his own fall. 
  The ultimate cause of Arjuna' death while the Pāṇḍavas were on the way to the underworld 
  was summarised by Yudhiṣṭhira thus {\rm (}\MBH\ 17.2.21ab{\rm )}:
  
 
  \textit{ekāhnā nirdaheyaṃ vai śatrūn ity arjuno 'bravīt}\thinspace | 
 
  \textit{na ca tat kṛtavān eṣa śūramānī tato 'patat}\thinspace ||
  
 
  `Arjuna claimed that he could destroy the enemy in one single day. He failed to do so.
  He was a boaster, that is why he fell.'
 
  
  
  King Nala was an expert in the game of dice but once he lost his kingdom to Puṣkara.
  See, e.g., \MBH\ 3.56.1ff. 
 
  
 
  As for Nṛga, see \MBH\ 14.93.74: 
  
 
  \textit{gopradānasahasrāṇi dvijebhyo 'dān nṛgo nṛpaḥ}\thinspace | 
 
  \textit{ekāṃ dattvā sa pārakyāṃ narakaṃ samavāptavān}\thinspace ||
  
 
  `King Nṛga had donated thousands of cows to the twice-born.
  By giving away one single cow that belonged to someone else, 
  he fell into hell.'
 }}

  \maintext{damena hīnaḥ puruṣo dvijendra}%

 \nonanustubhindent \maintext{svargaṃ ca mokṣaṃ ca sukhaṃ ca nāsti |}%

  \maintext{vijñānadharmakulakīrtināśa}%

 \nonanustubhindent \maintext{bhavanti vipra damayā vihīnāḥ }||\thinspace4:59\thinspace||%
\translation{[For] a person who is without self-restraint, O great Brahmin, there is no heaven, liberation or happiness. O Brahmin, people without self-restraint are the destruction of knowledge, Dharma, family and fame. \blankfootnote{4.59 \textit{Pāda} b: \textit{svarga} and \textit{mokṣa} are usually masculine in standard Sanskrit.
 The majority of the witnesses suggest that \textit{pāda} c ends in a stem form noun {\rm (}°\textit{nāśa}{\rm )},
  although a singular masculine nominative {\rm (}as in \Ed{\rm )} may work.
  This \textit{pāda} is unmetrical, or rather it applies the licence of a word-final
  short syllable being counted as potentially long {\rm (}°\textit{dharMA}°; see p.~\pageref{short2long}{\rm )}. 
 Note how \textit{viprā} in \textit{pāda} d is probably an attempt in some MSS to restore the metre.
  This \textit{pāda} is also unmetrical, or rather the licence of a word-final
  short syllable being counted as potentially long is applied again {\rm (}\textit{viPRA}{\rm )}.
 }}

  \subchptr{yameṣu ghṛṇā {\rm {\rm (}6{\rm )}}}%

  \trsubchptr{Sixth Yama-rule: taboos}%

  \maintext{nirghṛṇo na paratrāsti nirghṛṇo na ihāsti vai |}%

  \maintext{nirghṛṇe na ca dharmo 'sti nirghṛṇe na tapo 'sti vai }||\thinspace4:60\thinspace||%
\translation{A person without taboos does not exists either in this or the other world. If one has no taboos, one cannot have Dharma or religious austerity. \blankfootnote{4.60 The implications of \textit{pāda}s ab are not crystal clear to me. Perhaps:
  such a person has no right for existence in society and has no place in heaven.
 }}

  \maintext{parastrīṣu parārtheṣu parajīvāpakarṣaṇe |}%

  \maintext{paranindāparānneṣu ghṛṇāṃ pañcasu kārayet }||\thinspace4:61\thinspace||%
\translation{These five should be treated as taboo: women who are not depending on oneself, others' wealth, taking away others' lives, hurting others and [consuming] others' food. }

  \maintext{parastrī śṛṇu viprendra ghṛṇīkāryā sadā budhaiḥ |}%

  \maintext{rājñī viprī parivrājā svayoniparayoniṣu }||\thinspace4:62\thinspace||%
\translation{Listen, O great Brahmin. The wise should always treat women who are not dependent on oneself as taboo, [be she] a queen, a Brahmin's wife, a wandering religious mendicant, a relative or of another caste. \blankfootnote{4.62 The translation of \textit{parayoni} in \textit{pāda} d is tentative.
 }}

  \maintext{parārthe śṛṇu bhūyo 'nya anyāyārtha{-}m{-}upārjanam |}%

  \maintext{āḍhaprasthatulāvyājaiḥ parārthaṃ yo 'pakarṣati }||\thinspace4:63\thinspace||%
\translation{Listen further to something else, with regards to others' wealth. [It may include] gaining wealth through unlawful means, when somebody takes away other people's wealth by cheating with weights of one \textit{āḍha[ka]} or a \textit{prastha} and with scales. \blankfootnote{4.63 Although \textit{'nya} in \textit{pāda} a could be interpreted several ways {\rm (}e.g. \textit{anye} for \textit{anyasmin}, 
  or taken to be the first element of a compound: \textit{anya-anyāyārtha-}{\rm )},
  I think that \textit{bhūyo 'nyat} is a fixed expression meaning `something/anything more.' 
  See, e.g., \BHG\ 7.2cd:
  \textit{yaj jñātvā neha bhūyo 'nyaj jñātavyam avaśiṣyate}.
  Understand \textit{pāda} b as a compound {\rm (}\textit{anyāya-artha-upārjanam}{\rm )}. 
  See cheating with scales mentioned in 4.23.
 }}

  \maintext{jīvāpakarṣaṇe vipra ghṛṇīkurvīta paṇḍitaḥ |}%

  \maintext{vanajāvanajā jīvā vilagāś caraṇācarāḥ }||\thinspace4:64\thinspace||%
\translation{O Brahmin, the wise should regard taking away lives as taboo, [be they] wild or domesticated living beings, serpents, plants and animals. \blankfootnote{4.64 In \textit{pāda} d, I take \textit{caraṇācarāḥ} as standing for \textit{carācarāḥ} {\rm (}\textit{cara-acarāḥ}{\rm )} metri causa.
  Alternatively, one may understand it as \textit{caraṇacarāḥ} {\rm (}metri causa{\rm )}, 
  meaning `those who move on their feet,' perhaps as opposed to snakes {\rm (}\textit{bilaga} or \textit{bilaṃga}{\rm )}.
  Neither solution is fully satisfactory. Note that this \textit{pāda} also involves a small correction.
 }}

  \maintext{paranindā ca kā vipra śṛṇu vakṣye samāsataḥ |}%

  \maintext{devānāṃ brāhmaṇānāṃ ca gurumātātithidviṣaḥ }||\thinspace4:65\thinspace||%
\translation{And what is the hurting of others? Listen, O Brahmin, I shall tell you briefly. He who is hostile to the gods, Brahmins, the guru, a mother, and guests [hurts others]. \blankfootnote{4.65 Note \textit{mātā} as a stem form in \textit{pāda} d.
 }}

  \maintext{parānneṣu ghṛṇā kāryā abhojyeṣu ca bhojanam |}%

  \maintext{sūtake mṛtake śauṇḍe varṇabhraṣṭakule naṭe }||\thinspace4:66\thinspace||%
\translation{As regards other people's food, eating together with people whose food is not to be accepted {\rm (}\textit{abhojyeṣu}{\rm )} is taboo, [e.g.] after birth or death [in a family], in case of vendors of alcohol, or a family having lost their caste, and in the case of a [member of the] Naṭa [caste of dancers]. \blankfootnote{4.66 One should probably understand \textit{śauṇḍe} in \textit{pāda} c as \textit{śauṇḍike}, `a distiller,' or, alternatively,
  it may be corrupted from \textit{ṣaṇḍhe}, `a eunuch'; see both in \VasDh\ 14.1--3:
  
 
  \textit{athāto bhojyābhojyaṃ ca varṇayiṣyāmaḥ}\thinspace |
  \textit{cikitsaka-mṛgayu-puṃścalī-ḍaṇḍika-stenābhiśastar-ṣaṇḍha-patitānām annam abhojyam}\thinspace |
  \textit{kadarya-dīkṣita-baddhātura-soma-vikrayi-takṣa-rajaka-śauṇḍika-sūcaka-vārdhuṣika-carmāvakṛntānām}\thinspace || etc.
  
 
  It is translated by Olivelle {\rm (}\citeyear{OlivelleDharmasutras}, 285{\rm )} as:
  `Next we will describe food that is fit and food that is
  unfit to be eaten [\dots] The following are unfit
  to be eaten: food given by a physician, a hunter, a harlot, a law
  enforcement agent, a thief, a heinous sinner [...] a
  eunuch, or an outcaste; as also that given by a miser, a man
  consecrated for a sacrifice, a prisoner, a sick person, a man who
  sells Soma, a carpenter, a washerman, a liquor dealer, a spy, an
  usurer, a leather worker\dots'
  
 
  In support of reading \textit{ṣaṇḍhe}, one might consult \MANU\ 3.239:
  
 
  \textit{cāṇḍālaś ca varāhaś ca kukkuṭaḥ śvā tathaiva ca}\thinspace |
 
  \textit{rajasvalā ca ṣaṇḍhaś ca nekṣerann aśnato dvijān}\thinspace ||
  
 
  Translated by Olivelle {\rm (}\citeyear{OlivelleDharmasutras}, 120{\rm )} as:
  
 
  `A Cāṇḍāla, a pig, a cock, a dog, a menstruating woman, or a eunuch must not
  look at the Brahmins while they are eating.'
 }}

  \maintext{ete pañcaghṛṇāsu saktapuruṣāḥ svargārthamokṣārthino}%

 \nonanustubhindent \maintext{loke 'nindanam āpnuvanti satataṃ kīrtir yaśo'laṃkṛtam |}%

  \maintext{prajñābodhaśrutiṃ smṛtiṃ ca labhate mānaṃ ca nityaṃ labhed}%

 \nonanustubhindent \maintext{dākṣiṇyaṃ sabhavet sa āyuṣa paraṃ prāpnoti niḥsaṃśayaḥ }||\thinspace4:67\thinspace||%
\translation{Those people who stick to the five kinds of taboo [and thus] seek heaven, wealth and liberation, will reach eternal faultlessness in this world, embellished with fame and glory. [A person like that] will obtain wisdom, intelligence, [knowledge of] the Śruti and Smṛti traditions, and honour forever. Kindness will arise and he will obtain an extra long life, no doubt. \blankfootnote{4.67 Understand \textit{kīrtir-yaśo}° as \textit{kīrtiyaśo}° {\rm (}'r' being an intrusive consonant here metri causa{\rm )}, 
  as in 5.20b below. Alternatively, emend to \textit{kīrtiṃ yaśo'laṃkṛtām}.
 In \textit{pāda} c, note the muta cum liquida licence that allows °\textit{bodhaśrutiṃ}°
  to scan as - \shortsyllable\ \shortsyllable\ - , the consonant cluster 
  \textit{śr} not turning the previous syllable long.
 \textit{Pāda} d has several problems. I take \textit{sabhavet} as standing for \textit{sambhavet} metri causa,
  and I had to emend \textit{samāyuṣa} to \textit{sa āyuṣa} to make sense of it.
  Understand \textit{āyuṣa} as \textit{āyuḥ} {\rm (}metri causa{\rm )}, otherwise accept \Ed's \textit{sa mānuṣa}.
  Also consider correcting \textit{niḥsaṃśayaḥ} to \textit{niḥsaṃśayam}.
 }}

  \subchptr{yameṣu pañcavidho dhanyaḥ {\rm {\rm (}7{\rm )}}}%

  \trsubchptr{Seventh Yama-rule: five kinds of virtue}%

  \maintext{caturmaunaṃ catuḥśatruś caturāyatanaṃ tathā |}%

  \maintext{caturdhyānaṃ catuṣpādaṃ pañcadhanyavidhocyate }||\thinspace4:68\thinspace||%
\translation{The four cases of observing silence, [victory over] the four enemies, the four sanctuaries, the four meditations, and the four-legged [Dharma] are called the five ways of being virtuous. \blankfootnote{4.68 Understand \textit{pāda} d as \textit{pañcavidho dhanya ucyate}.
 }}

  \maintext{caturmaunasya vakṣyāmi śṛṇuṣvāvahito bhava |}%

  \maintext{pāruṣyapiśunāmithyā sambhinnāni ca varjayet }||\thinspace4:69\thinspace||%
\translation{I shall tell you about the four cases of observing silence. Listen, be attentive. One should avoid violent and slanderous [words], lies, and idle [talk]. \blankfootnote{4.69 Note the genitive with a verb meaning `to tell' in \textit{pāda} a, similarly to 1.37a and \verify\
  {\rm (}See p.~\pageref{tellplusgen}{\rm )}.
 Compare the four types of \textit{mauna} taught here with the five types of
  \textit{maunavrata}, as the ninth Niyama-rule, in \VSS\ 8.25--33 below.
  Similar lists on \textit{mauna} are often found in Buddhist texts:
  see references, e.g., in \mycite{EdgertonHybrid} s.v. 
  \textit{paiśunika} and \textit{saṃbhinnapralāpa}.
  See also the relevant \DIVYAV\ 186.21, as well as \DHARMP\ 1.31cd--32ab 
  quoted in the apparatus.
 }}

  \maintext{kāmaḥ krodhaś ca lobhaś ca mohaś caiva caturvidhaḥ | }%

  \maintext{catuḥśatrur nihantavyaḥ so 'rihā vītakalmaṣaḥ }||\thinspace4:70\thinspace||%
\translation{The fourfold enemy [made up of] desire, anger, greed and delusion is to be destroyed. He who destroys [these] enemies will become sinless. \blankfootnote{4.70 Possible direct sources for the idea that \textit{kāma} is an enemy to be defeated or avoided include
  \BUDDHACARITA\ 11.17:
  
 
  \textit{cīrāmbarā mūlaphalāmbubhakṣā} 
  \textit{jaṭā vahanto 'pi bhujaṃgadīrghāḥ}\thinspace |
  
  \textit{yair nānyakāryā munayo 'pi bhagnāḥ} 
  \textit{kaḥ kāmasaṃjñān mṛgayeta śatrūn}\thinspace ||
  
 
  In Johnston's translation {\rm (}\citeyear{Buddhacarita}, II. 152{\rm )}:
  
 
  `Who would seek after the enemies known as the passions, by whom
  even sages were undone, despite their bark-dresses, their diet
  of roots and water, their coils of hair long as snakes, and their
  lack of worldly interests.'
  
 
  See also \BHG\ 3.37--43 on \textit{kāma} as an enemy.
  As for \textit{arihā} in \textit{pāda} d, the notion that a saint is a `destroyer of the enemies' 
  [that are evil states of mind] {\rm (}\textit{arihanta/arahanta}{\rm )}
  in Jainism, but less so in Buddhism, is discussed in 
  \mycitep{GombrichWhat2013}{57--58}.
 }}

  \maintext{caturāyatanaṃ vipra kathayiṣyāmi tac chṛṇu |}%

  \maintext{karuṇā muditopekṣā maitrī cāyatanaṃ smṛtam }||\thinspace4:71\thinspace||%
\translation{I shall teach you the four sanctuaries. Listen, O Brahmin. Compassion, sympathy in joy, indifference, and benevolence are the four sanctuaries. \blankfootnote{4.71 This verse teaches the four Buddhist \textit{brahmavihāra}s under the label
  \textit{catur\-āyatana}. Therefore the word \textit{āyatana} seems to be a synonym of \textit{vihāra} here,
  and its use a way of appropriating it, turning the list into a Brahmanical one,
  unless the two terms are simply mixed up.
 }}

  \maintext{caturdhyānādhunā vakṣye saṃsārārṇavatāraṇam |}%

  \maintext{ātmavidyābhavaḥ sūkṣmaṃ dhyānam uktaṃ caturvidham }||\thinspace4:72\thinspace||%
\translation{I shall now teach you the four meditations, which will liberate you from transmigration. Meditation is taught to be fourfold: of the Self, \textit{vidyā}, \textit{bhava} [= Śiva] and the subtle one {\rm (}\textit{sūkṣma}{\rm )}. \blankfootnote{4.72 Note the stem form \textit{dhyāna} in °\textit{dhyānādhunā} {\rm (}for °\textit{dhyānam adhunā}{\rm )} in \textit{pāda}~a.
 }}

  \maintext{ātmatattvaḥ smṛto dharmo vidyā pañcasu pañcadhā |}%

  \maintext{ṣaṭtriṃśākṣaram ity āhuḥ sūkṣmatattvam alakṣaṇam }||\thinspace4:73\thinspace||%
\translation{The \textit{tattva} of the Self is Dharma. \textit{Vidyā} is in the five in a fivefold way. They call the thirty-sixth the imperishable one [Śiva]. The subtle \textit{tattva} has no attributes. \blankfootnote{4.73 This verse is difficult to interpret. \textit{Pāda}s a to d should define \textit{ātman}, \textit{vidyā}, 
  \textit{bhava} {\rm (}i.e. Śiva{\rm )}, and \textit{sūkṣma}, objects of meditation, respectively.
  In \textit{pāda} a, \textit{dharmo} is suspect: it may be the result of
  an eyeskip to \textit{pāda} a of the next verse. \textit{Pāda} b might refer to \textit{tattva}s in an ontological
  system of 25 \textit{tattva}s.
 \textit{Pāda} c seems a reference to a tantric 36-\textit{tattva} ontological system,
  in striking contrast with the 25-\textit{tattva} system described in \VSS\ chapter 20.
  Compare the rather similar \textit{dhyānayajña} section in \VSS\ 6.7ff, in which
  five types of meditations are taught. See analysis on pp. Intro \verify.
 }}

  \maintext{catuṣpādaḥ smṛto dharmaś caturāśramam āśritaḥ |}%

  \maintext{gṛhastho brahmacārī ca vānaprastho 'tha bhaikṣukaḥ }||\thinspace4:74\thinspace||%
\translation{The four-legged [bull] is said to be Dharma [as] it rests on the four \textit{āśrama}s, [those of] the householder, the chaste one, the forest-dweller and the mendicant. }

  \maintext{dhanyās te yair idaṃ vetti nikhilena dvijottama |}%

  \maintext{pāvanaṃ sarvapāpānāṃ puṇyānāṃ ca pravardhanam }||\thinspace4:75\thinspace||%
\translation{Virtuous are those who know these thoroughly, O great Brahmin. [They will experience] the purification of all sins and the growth of merits. \blankfootnote{4.75 Note the ergative syntax with the plural instrumental {\rm (}\textit{yair}{\rm )} and a singular active verb.
 }}

  \maintext{āyuḥ kīrtir yaśaḥ saukhyaṃ dhanyād eva pravardhate |}%

  \maintext{śāntiḥ puṣṭiḥ smṛtir medhā jāyate dhanyamānave }||\thinspace4:76\thinspace||%
\translation{One's life-span, fame and glory, and happiness grow only through virtue {\rm (}\textit{dhanya}{\rm )}. In a virtuous person piece, prosperity, tradition {\rm (}\textit{smṛti}{\rm )} and intelligence will arise. \blankfootnote{4.76 Emending °\textit{mānavaḥ} to °\textit{mānave} might err by overcorrection, and °\textit{mānavaḥ} may have originally
  been felt like a genitive {\rm (}`for a person\dots'{\rm )}.
 }}

  \subchptr{yameṣv apramādaḥ {\rm {\rm (}8{\rm )}}}%

  \trsubchptr{Eighth Yama-rule: avoiding mistakes}%

  \maintext{pramādasthāna pañcaiva kīrtayiṣyāmi tac chṛṇu |}%

  \maintext{brahmahatyā surāpānaṃ steyo gurvaṅganāgamam |}%

  \maintext{mahāpātakam ity āhus tatsaṃyogī ca pañcamaḥ }||\thinspace4:77\thinspace||%
\translation{There are five areas of making serious mistakes. I shall teach them to you, listen. Murdering a Brahmin, drinking alcohol, stealing, having sex with the guru's wife: they call these grievous sins. The fifth is when one is connected with them [i.e. with these sins or with people involved in these sinful acts]. \blankfootnote{4.77 Note the stem form noun in \textit{pāda} a {\rm (}°\textit{sthāna}{\rm )} metri causa, and also 
  that this stem form noun may function as a singular noun
  next to a number {\rm (}\textit{pañca}{\rm )}, a frequently seen phenomenon in this text.
 
 
  See the apparatus to the Sanskrit text for very similar verses in the \MBH, \MANU\ and 
  the \YAJNS, and note how \textit{pāda} f slightly deviates from \MANU\ 11.55, which is translated in
  \mycitep{OlivelleManu}{217--218} as: 
  `Killing a Brahmin, drinking liquor, stealing, and having sex with an elder's 
  wife---they call these ``grievous sins causing loss of caste''; 
  and so is establishing any links with such individuals.'
 }}

  \maintext{anṛtaṃ ca samutkarṣe rājagāmī ca paiśunaḥ |}%

  \maintext{guroś cālīkanirbandhaḥ samāni brahmahatyayā }||\thinspace4:78\thinspace||%
\translation{A lie concerning one's superiority, a slander that reaches the king's ear, and false accusations against an elder are equal to killing a Brahmin. \blankfootnote{4.78 This verse being a quotation of \MANU\ 11.56, my translation 
  is based on \mycitep{OlivelleManu}{218}.
  On lies and slander {\rm (}or `malignant speech,' \textit{piśuna}{\rm )}, see also \VSS\ 4.69 and 8.25--28.
 }}

  \maintext{brahmojjhaṃ vedanindā ca kūṭasākṣī suhṛdvadhaḥ |}%

  \maintext{garhitānādyayor jagdhiḥ surāpānasamāni ṣaṭ }||\thinspace4:79\thinspace||%
\translation{Abandoning the Vedas, reviling the Vedas, being a false witness, murdering a friend, eating unfit or forbidden food are six [deeds that are] equal to drinking alcohol. \blankfootnote{4.79 This verse continues quoting \MANU. \textit{Pāda} a in the witnesses may actually be no more than the result of 
  misreading of the syllable \textit{jjha} in \MANU\ 11.57. Note the variant \textit{brahmojjhaṃ vedanindā ca}
  in both the `Northern' and `Southern' transmissions in Olivelle's critical edition 
  of \MANU\ {\rm (}\mycitep{OlivelleManu}{847}{\rm )}.
 }}

  \maintext{retotsekaḥ svayonyāsu kumārīṣv antyajāsu ca |}%

  \maintext{sakhyuḥ putrasya ca strīṣu gurutalpasamaḥ smṛtaḥ }||\thinspace4:80\thinspace||%
\translation{Sexual intercourse with a female relative, with an unmarried girl, with women of the lowest castes, with the wife of a friend or of one's own son are said to be equal to violating the guru's bed. \blankfootnote{4.80 The text, and my emendation in \textit{pāda} c, still follow \MANU\ {\rm (}11.59{\rm )}.
 }}

  \maintext{nikṣepasyāpaharaṇaṃ narāśvarajatasya ca |}%

  \maintext{bhūmivajramaṇīnāṃ ca rukmasteyasamaḥ smṛtaḥ }||\thinspace4:81\thinspace||%
\translation{Stealing deposits, people, horses, silver, land, diamonds, or gems are said to be equal to stealing gold. \blankfootnote{4.81 This is \MANU\ 11.58. I have emended \textit{rugma}° to \textit{rukma}° in \textit{pāda} d, although
  \textit{rugma}° is attested in a great number of Southern MSS and one Śāradā MS in
  \mycitep{OlivelleManu}{847}.
 }}

  \maintext{catvāra ete sambhūya yat pāpaṃ kurute naraḥ |}%

  \maintext{mahāpātaka pañcaitat tena sarvaṃ prakāśitam |}%

  \maintext{pañcapramādam etāni varjanīyaṃ dvijottama }||\thinspace4:82\thinspace||%
\translation{Since a man commits sin if [any of these] four [i.e. \textit{brahmahatyā, surāpāna, stena, gurvaṅganāgama}], occurs, therefore all the five grievous sins have been explained. These five kinds of mistakes are to be avoided, O great Brahmin. \blankfootnote{4.82 Perhaps understand \textit{pāda} c as \textit{etan mahāpātakapañcakaṃ}.
 Note the confusion of number and gender: understand \textit{pañca pramādāḥ etā varjanīyāḥ}.
 }}

  \subchptr{yameṣu mādhuryam {\rm {\rm (}9{\rm )}}}%

  \trsubchptr{Ninth Yama-rule: charm}%

  \maintext{kāyavāṅmanamādhuryaś cakṣur buddhiś ca pañcamaḥ |}%

  \maintext{saumyadṛṣṭipradānaṃ ca krūrabuddhiṃ ca varjayet }||\thinspace4:83\thinspace||%
\translation{[Charm has five types:] bodily, verbal and mental charm, [charm of] the eyes and [of one's] thoughts as fifth. Giving [others] a friendly glance [is commendable] and one should avoid cruel thoughts. \blankfootnote{4.83 My emendation from °\textit{manasā dhūryaś} to °\textit{mana-mādhuryaś} is based on the fact that following the list
  of \textit{yama}s in 3.16, we need some reference to \textit{mādhurya} here and that it is easy to see how this
  corruption came about: °\textit{mano-mādhurya}° would be unmetrical, hence the form °\textit{mana-mādhurya};
  °\textit{mana-mā}° is easily corrupted to °\textit{manasā}° {\rm (}not to mention the fact 
  that \textit{manasā} comes up in the next verse{\rm )}. 
  In addition, we need five items in this line because of \textit{pañcamaḥ}.
  As always, I correct \textit{mādhūrya} to \textit{mādhurya}, although it seems that 
  the former is acceptable in this text. 
  I did not correct \textit{mādhuryaś} to \textit{mādhuryaṃ} because of the corresponding
  \textit{pañcamaḥ}.
 }}

  \maintext{prasannamanasā dhyāyet priyavākyam udīrayet |}%

  \maintext{yathāśaktipradānaṃ ca svāśramābhyāgato guruḥ }||\thinspace4:84\thinspace||%
\translation{One should meditate with a tranquil mind and should speak [to other people using] gentle words. [When] respectable people arrive at one's own hermitage, [one should] present them with as many gifts as one can, \blankfootnote{4.84 \textit{Pāda}s cd of the previous verse, and \textit{pāda}s ab of the present one cover
  four categories of the above: \textit{cakṣurmādhurya}, \textit{buddhimādhurya}, \textit{dṛṣṭimādhurya} and \textit{vāgmādhurya}.
  This suggests that what follows is on \textit{kāyamādhurya}.
 Emending \textit{pāda} d to \textit{svāśramābhyāgate gurau} would make the line smoother.
 }}

  \maintext{indhanodakadānaṃ ca jātavedam athāpi vā |}%

  \maintext{sulabhāni na dattāni indhanāgnyudakāni ca |}%

  \maintext{kṣute jīveti vā noktaṃ tasya kiṃ parataḥ phalam }||\thinspace4:85\thinspace||%
\translation{with gifts of fire-wood, water and fire. [If] fire-wood, fire and water are easily available [but] are not given [as gift] or [if the phrase] `Live [for a hundred years]!' is not uttered when [somebody] sneezes, what reward could there be for such a person in the afterlife? \blankfootnote{4.85 Understand \textit{jātavedam} in \textit{pāda} b as \textit{jātavedasam} or \textit{jātavedāḥ},
  or rather as belonging to the compound °\textit{dānaṃ}: \textit{jātavedodānaṃ}.
 For \textit{pāda} e, see an Āryāgīti verse in the \MAHASUBHS\ {\rm {\rm (}2558{\rm )}}: 
  
 
  \textit{amṛtāyatām iti vadet pīte bhukte kṣute ca śataṃ jīva}\thinspace |
 
  \textit{choṭikayā saha jṛmbhāsamaye syātāṃ cirāyurānandau}\thinspace ||
  
 
  `When eating or drinking, one should say: ``May it turn into nectar!''; 
  and after sneezing: ``Live for a hundred years!''
  By snapping the thumb and forefinger when yawning, there will be long life and happiness.'
 }}

  \subchptr{yameṣv ārjavam {\rm {\rm (}10{\rm )}}}%

  \trsubchptr{Tenth Yama-rule: sincerity}%

  \maintext{pañcārjavāḥ praśaṃsanti munayas tattvadarśinaḥ |}%

  \maintext{karmavṛttyābhivṛddhiṃ ca pāritoṣikam eva ca |}%

  \maintext{strīdhanotkocavittaṃ ca ārjavo nābhinandati }||\thinspace4:86\thinspace||%
\translation{The sages who see the truth praise five types of sincerity. A sincere person does not rejoice in prosperity arising from the operation of karma or by a reward, in riches from women, from property, and bribery. \blankfootnote{4.86 °\textit{ārjavāḥ} should be in the accusative, therefore it is to be taken as feminine {\rm (}rather than neuter{\rm )} or as
  an irregular form for °\textit{ārjavāni}. I have emended \textit{pāratoṣikam} to \textit{pāritoṣikam}.
 My translation of the categories listed here is tentative, the only guiding light being
  that, if the first line is right, there should be five of them. In addition, I have tried to
  find categories that seem to be, more or less, in conflict with `sincerity' or `straightness.'
 }}

  \maintext{ārjavo na vṛthā yajña ārjavo na vṛthā tapaḥ |}%

  \maintext{ārjavo na vṛthā dānam ārjavo na vṛthāgnayaḥ }||\thinspace4:87\thinspace||%
\translation{If one is not sincere, sacrifice is in vain. If one is not sincere, austerity is in vain. If one is not sincere, donation is in vain. If one is not sincere, [sacrificial] fires are in vain. }

  \maintext{ārjavasyendriyagrāmaḥ suprasanno 'pi tiṣṭhati |}%

  \maintext{ārjavasya sadā devāḥ kāye tasya caranti te }||\thinspace4:88\thinspace||%
\translation{The sense faculties of a sincere person are firm even when he is delighted. The gods are always present in the body of a sincere person. }

  \maintext{iti yamapravibhāgaḥ kīrtito 'yaṃ dvijendra}%

 \nonanustubhindent \maintext{iha parata sukhārthaṃ kārayet taṃ manuṣyaḥ |}%

  \maintext{duritamalapahārī śaṅkarasyājñayāste}%

 \nonanustubhindent \maintext{bhavati pṛthivibhartā hy ekachatrapravartā }||\thinspace4:89\thinspace||%
\translation{Thus has been taught this section on the Yama-rules, O great Brahmin. Humans should follow them to reach happiness here and in the other world. One will remove the filth of sins, and shall by Śaṅkara's command become a ruler of the world [that he subjugates] under one royal umbrella. \blankfootnote{4.89 In \textit{pāda} a °\textit{pra}° does not make the previous syllable long: this is the phenomenon of
  `muta cum liquida,' one of the hallmarks of the \VSS, 
  that is, syllables such as \textit{tra, pra, bra, dra} do not necessarily make the 
  previous syllable long.
 In \textit{pāda} b, \textit{parata} most probably stands for \textit{paratra} or \textit{parataḥ} metri causa. 
  We may correct it to \textit{paratra}, presupposing the presence of the licence `muta cum liquida.'
 °\textit{malapahārī} in the MSS stands either for °\textit{malāpahārī} or °\textit{malaprahārī} metri causa. 
  I could have chosen to emend it to °\textit{malaprahārī} again applying the licence of muta cum liquida,
  but I decided not to because \textit{apahārin}, \textit{apahāra}, \textit{apahāraka} are used in the text very frequently. 
  See also 8.44c, which contains a very similar expression: \textit{sakalamalapahāre dharmapañcāśad etat}.
 }}

\centerline{\maintext{\dbldanda\thinspace iti vṛṣasārasaṃgrahe yamavibhāgo nāmādhyāyaś{ }caturthaḥ\thinspace\dbldanda}}
\translation{Here ends the fourth chapter in the \textit{Vṛṣasārasaṃgraha} called the Section on the Yama-rules.}

  \chptr{pañcamo 'dhyāyaḥ}
\addcontentsline{toc}{section}{Chapter 5}
\fancyhead[CO]{{\footnotesize\textit{Translation of chapter 5}}}%

  \trchptr{ Chapter Five }%

  \subchptr{niyamāḥ}%

  \trsubchptr{The Niyama-rules}%

  \maintext{vigatarāga uvāca |}%

  \maintext{kathaya niyamatattvaṃ sāmprataṃ tvaṃ viśeṣād}%

 \nonanustubhindent \maintext{amṛtavacanatulyaṃ śrotukāmo gato 'smi |}%

  \maintext{prakṛtidahanadagdhaṃ jñānatoyair niṣiktam}%

 \nonanustubhindent \maintext{apara vada{-}m{-}atajjñaṃ nāsti dharmeṣu tṛptiḥ }||\thinspace5:1\thinspace||%
\translation{Vigatarāga spoke: Now teach me the true nature of the Niyama-rules in detail. I have become desirous to hear [your] teaching that is comparable to ambrosia. Tell [me] more {\rm (}\textit{apara vada}{\rm )}, [to me who had been] burnt by the fire of materiality {\rm (}\textit{prakṛti}{\rm )}, [but is now] sprinkled with the water of knowledge, and is ignorant of [the topic]. One can't have enough of the [teaching on] Dharmas {\rm (}\textit{nāsti dharmeṣu tṛptiḥ}{\rm )}. \blankfootnote{5.1 Most witnesses read \textit{amṛtavadana}° in \textit{pāda} b. This is slightly odd in the sense of `speech,' the meaning
  required here, therefore I follow \msM\ here. One wonders if it is not \textit{amṛtasvādana} or
  °\textit{svadana} {\rm (}`tasting nectar'{\rm )} what was meant originally. I translate the phrase in question as if it read
  \textit{amṛtatulyavacanaṃ}.
 The first half of \textit{pāda} d is difficult to interpret safely. \textit{apara vada} {\rm (}`tell me more'{\rm )} might be original,
  with \textit{apara} in stem form. The phrase \textit{matajñā} is now emended to \textit{-m-atajjñaṃ},
  containing a hiatus break and making the line metrical.
  Otherwise it could be emended to \textit{matajña} {\rm (}with the last syllable taken as long{\rm )} 
  and translated as a vocative {\rm (}`O knower of the doctrine'{\rm )}.
  Note \msM's reading for the end of the line {\rm (}\textit{me dharmatṛptiḥ}{\rm )}.
 }}

  \maintext{anarthayajña uvāca |}%

  \maintext{śravaṇasukham ato 'nyat kīrtayiṣye dvijendra}%

 \nonanustubhindent \maintext{niyamakalaviśeṣaḥ pañca pañca prakāraḥ |}%

  \maintext{hariharamunibhīṣṭaṃ dharmasāraṃ dvijendra}%

 \nonanustubhindent \maintext{kalikaluṣavināśaṃ prāyamokṣaprasiddham }||\thinspace5:2\thinspace||%
\translation{Anarthayajña spoke: I shall teach you something else that is nice to hear, O best of the twice-born. The [ten] individual Niyamas are fivefold [each]. It is the essence of Dharma, dear to Hari, Hara and the sages, O great Brahmin, the destruction of the impurity of the Kali age, known as almost liberation. \blankfootnote{5.2 My suspicion is that °\textit{kala}° in \textit{pāda} b stands for \textit{kalā} metri causa. 
 Similarly, °\textit{munibhīṣṭaṃ} is metri causa, for °\textit{munyabhīṣṭaṃ} {\rm (}`dear to the sages'{\rm )}.
 In \textit{pāda} d, \textit{prāya}° is suspect. Compare with 6.1c: \textit{dharmamokṣaprasiddhyarthaṃ}.
 }}

  \maintext{śaucam ijyā tapo dānaṃ svādhyāyopasthanigrahaḥ |}%

  \maintext{vratopavāsamaunaṃ ca snānaṃ ca niyamā daśa }||\thinspace5:3\thinspace||%
\translation{Purification, sacrifice, penance, donation, Vedic study, the restraint of sexual desire, religious observances, fasting, observing silence, and bathing: these are the ten Niyamas. \blankfootnote{5.3 See this verse in \LinPu\ 1.8.29cd--30ab and \VDhU\ 3.233.202.
 }}

  \subchptr{niyameṣu śaucam {\rm {\rm (}1{\rm )}}}%

  \trsubchptr{First Niyama-rule: purity}%

  \maintext{tatra śaucādinirdeśaṃ vakṣyāmīha dvijottama |}%

  \maintext{śārīraśaucam āhāro mātrā bhāvaś ca pañcamaḥ }||\thinspace5:4\thinspace||%
\translation{From among these, now I shall tell you the particulars of purification [first], and [then] the others. [1] Bodily purity, [2] [purity of] food, [3] [purity of] the household[?] {\rm (}\textit{mātrā}{\rm )}, [4] [purity of] character[?] {\rm (}\textit{bhāva}{\rm )}, and the fifth, [5]...? \blankfootnote{5.4 The following passages deal with \textit{śārīraśauca} {\rm (}5.5--9{\rm )} and \textit{āhāraśauca} {\rm (}5.10--16{\rm )}, 
  therefore \textit{pāda} c is probably correct, 
  and \msM's reading {\rm (}\textit{śārīrasrotam āhāra}{\rm )} seems wrong. 
  Even if we could interpret \textit{pāda} d with any certainty, there
  is one element missing in this list of allegedly five items. 
  Something must have dropped out here.
  Oddly enough, the chapter stops after teaching the second type of purity,
  \textit{āhāraśauca}, so we are left without a clue.
  \MBH\ Suppl. 14.4.3229--3230 is not very helpful either: 
  
 
  \textit{manaḥśaucaṃ karmaśaucaṃ kulaśaucaṃ ca bhārata}\thinspace | 
 
  \textit{śarīraśaucaṃ vākśaucaṃ śaucaṃ pañcavidhaṃ smṛtam}\thinspace ||
 }}

  \subsubchptr{śarīraśaucam}%

  \trsubsubchptr{Purity of the Body}%

  \maintext{tāḍayen na ca bandheta na ca prāṇair viyojayet |}%

  \maintext{parastrīparadravyeṣu śaucaṃ kāyikam ucyate }||\thinspace5:5\thinspace||%
\translation{He should not beat, tie or kill [any living being]. [This and] purity concerning others' wives and property is called bodily purity. \blankfootnote{5.5 Note the application of the licence muta cum liquida in \textit{pāda} c: the first syllable
  of \textit{dravyeṣu} does not make the previous syllable heavy.
 }}

  \maintext{śrotraśaucaṃ dvijaśreṣṭha gudopasthamukhādayaḥ |}%

  \maintext{mukhasyācamanaṃ śaucam āhāravacaneṣu ca }||\thinspace5:6\thinspace||%
\translation{The cleanliness of the ears, O great Brahmin, and of the anus, the loins, the mouth etc. [also contributes to bodily purity]. The purity of the mouth [comes from] sipping water before eating and speaking. }

  \maintext{mūtraviṣṭāsamutsarge devatārādhaneṣu ca |}%

  \maintext{mṛttoyais tu gudopasthaṃ śaucayīta vicakṣaṇaḥ }||\thinspace5:7\thinspace||%
\translation{After the emission of urine and f\ae ces, and before the worship of gods, the wise one should clean his anus and his loins with clay and water. \blankfootnote{5.7 Note the peculiar verb form \textit{śaucayīta} {\rm (}for a more standard \textit{śocayeta}{\rm )}. \msM's \textit{śaucaye}[\textit{c}] \textit{ca}
  may be close to an original reading.
 }}

  \maintext{ekopasthe gude pañca tathaikatra kare daśa |}%

  \maintext{ubhayoḥ sapta dātavyā mṛdaḥ śuddhiṃ samīhatā }||\thinspace5:8\thinspace||%
\translation{One [portion of clay] for the loins, five for the anus, ten for one hand, [then] seven [portions] of clay are to be applied for both [hands] by him who wishes cleanliness. \blankfootnote{5.8 In essence, this verse is \MANU\ 5.136. Olivelle's notes on this verse read:
  `\textit{on one hand:} within the context the meaning is clear: ``one hand'' refers to the left
  hand, with which the person applied the earth and water to the penis and anus. All purifications
  below the navel are carried out using the left hand. Variant reading: ``on the left hand.''\thinspace '
  {\rm (}\mycitep{OlivelleManu}{287}.{\rm )}
 }}

  \maintext{etac chaucaṃ gṛhasthānāṃ dviguṇaṃ brahmacāriṇām |}%

  \maintext{vānaprasthasya triguṇaṃ yatīnāṃ tu caturguṇam }||\thinspace5:9\thinspace||%
\translation{This is the purification for the householder {\rm (}\textit{gṛhastha}{\rm )}. It is twice as much for the chaste one {\rm (}\textit{brahmacārin}{\rm )}, three times as much for the forest-dweller {\rm (}\textit{vānaprastha}{\rm )}, and four times as much for the ascetic {\rm (}\textit{yati}{\rm )}. \blankfootnote{5.9 This verse corresponds to \MANU\ 5.137.
 Note the muta cum liquida licence in \textit{pāda} c: \textit{tr} does not turn the previous syllable heavy and 
  the \textit{pāda} becomes a \textit{na-vipulā}.
 }}

  \subsubchptr{āhāraśaucam}%

  \trsubsubchptr{Purity of the food}%

  \maintext{āhāraśaucaṃ vakṣyāmi śṛṇuṣvāvahito bhava |}%

  \maintext{bhāgadvayaṃ tu bhuñjīta bhāgam ekaṃ jalaṃ pibet |}%

  \maintext{vāyusaṃcāradānārthaṃ caturtham avaśeṣayet }||\thinspace5:10\thinspace||%
\translation{I shall teach you the rules of purity concerning food. Listen, pay great attention. One should eat [as much] food [that fills] two quarters [of the stomach] and drink water [that fills] one quarter. In order to give passage to the air, one should save the remaining quarter. \blankfootnote{5.10 Śaṅkara quotes a similar verse in his commentary ad \BHG\ 6.16 {\rm (}see apparatus{\rm )}.
  It translates as:
  `Half is for saucy food, the third part for water, but in order to be able to move the air,
  one should leave the fourth part [empty].' This verse and one in the \SANNYASUP\ {\rm (}see apparatus{\rm )} have
  \textit{saṃcaraṇārthaṃ tu} and \textit{saṃcaraṇārthāya}, respectively, where our verse in the \VSS\ has \textit{saṃcāradānārthaṃ}.
  It would be tempting to emend but the \VSS\ version more or less works fine, therefore 
  there is no need to alter the text.
 }}

  \maintext{snigdhasvādurasaiḥ ṣaḍbhir āhāraṣaḍrasair budhaḥ |}%

  \maintext{dhātuvaiṣamyanāśo 'sti na ca rogāḥ sudāruṇāḥ }||\thinspace5:11\thinspace||%
\translation{[By] the wise one['s applying] the six soft and sweet juices, [which are] the six flavours in food, the disturbances of the constituents {\rm (}\textit{dhātu}{\rm )} will disappear and the terrible illnesses will not arise. \blankfootnote{5.11 The readings may suggest that \textit{pāda} b contains \textit{sadrava} or maybe \textit{sudrava}, but it is difficult to make
  sense of the sentence. We are lacking a verb; \textit{āhāra} might be wrong for \textit{āharet} {\rm (}see \msM{\rm )}.
  The Āyurvedic implications of this clumsy verse are not crystal clear to me.
  What is clear is that traditionally there are
  six basic flavours or `juices' in food. See, e.g. \BHELAS\ 1.28.1:
  
 
  \textit{yad bhakṣayati bhuṅkte vā vidhivac cāpi mānavaḥ}\thinspace |
 
  \textit{anyac ca kiñcit pibati tat sarvaṃ ṣaḍrasānvitam}\thinspace ||
  
 
  `All that a human eats or enjoys according to the rules, and furthermore all 
  that he or she drinks, is endowed with the six flavours.'
  
 
  To repair \textit{pāda}s ab, one should perhaps imagine that the intended meaning was that the
  six flavours/juices should be present in a harmonious proportion in a wise man's food. 
  Cf. \BHELAS\ 3.1.1:
  
 
  \textit{śarīraṃ dhārayantīha ṣaḍrasāḥ samam āhṛtāḥ}\thinspace |
 
  \textit{ato 'nyathā vikārāṃs tu janayanti śarīriṇām}\thinspace ||
  
 
  `The six flavours will support the body in this world when brought to a balanced state.
  Otherwise they will produce defects to people.'
 On \textit{dhātuvaiṣamya}, the balanced state of the bodily constituents \textit{pitta}, 
  \textit{kapha} and \textit{vāyu}, see, e.g., \CARAKA\ 1.9.4:
  
 
  \textit{vikāro dhātuvaiṣamyaṃ sāmyaṃ prakṛtir ucyate}\thinspace | 
 
  \textit{sukhasaṃjñakam ārogyaṃ vikāro duḥkham eva ca}\thinspace ||
  
 
  `The imbalance of the \textit{dhātu}s means defects. Balance is said to be natural.
  Health is happiness, defects are suffering.' See also \VSS\ 9.2 below.
 }}

  \maintext{abhakṣyaṃ ca na bhakṣeta apeyaṃ na ca pāyayet |}%

  \maintext{agamyaṃ na ca gamyeta avācyaṃ na ca bhāṣayet }||\thinspace5:12\thinspace||%
\translation{He should not eat what is forbidden and he should not drink what is forbidden. He should not go where he is not allowed to and he should not say what is improper. \blankfootnote{5.12 Understand the causative \textit{pāyayet} as simplex.
 }}

  \maintext{laśunaṃ ca palāṇḍuṃ ca gṛñjanaṃ kavakāni ca |}%

  \maintext{gauraṃ ca sūkaraṃ māṃsaṃ varjayec ca vidhānataḥ }||\thinspace5:13\thinspace||%
\translation{He should avoid garlic, onion, \textit{gṛñjana} onion, mushrooms, buffalo meat, and pork, following the rules. }

  \maintext{chattrākaṃ viḍvarāhaṃ ca gomāṃsaṃ ca na bhakṣayet |}%

  \maintext{caṭakaṃ ca kapotaṃ ca jālapādāṃś ca varjayet }||\thinspace5:14\thinspace||%
\translation{He should not eat \textit{chattrāka} mushrooms, village hog, and cow flesh. He should also avoid sparrows, pigeons, and water-birds. }

  \maintext{haṃsasārasacakrāhvakukkuṭān śukaśyenakān |}%

  \maintext{kākolūkaṃ balākaṃ ca matsyādīṃś cāpi varjayet }||\thinspace5:15\thinspace||%
\translation{He should also avoid [eating] geese, cranes, \textit{cakravāka} birds, cocks, parrots, and hawks, crows, owls, herons, fish etc. \blankfootnote{5.15 Note that in \textit{pāda} b the first syllable of \textit{śyenakān} does not turn the previous syllable, \textit{śu},
  heavy. This is an extension of the muta cum liquida licence.
 }}

  \maintext{amedhyāṃś cāpavitrāṃś ca sarvān eva vivarjayet |}%

  \maintext{śākamūlaphalānāṃ ca abhakṣyaṃ parivarjayet }||\thinspace5:16\thinspace||%
\translation{He should avoid everything that is ritually impure or polluted. He should also completely avoid those vegetables, roots and fruits, that are prohibited. }

  \maintext{mānaveṣu purāṇeṣu śaivabhāratasaṃhite |}%

  \maintext{kīrtitāni viśeṣeṇa śaucācāram aśeṣataḥ |}%

  \maintext{tvayā jijñāsito 'smy adya saṃkṣiptaḥ kathito mayā }||\thinspace5:17\thinspace||%
\translation{In the books of Manu, in the Purāṇas, in Śaiva texts, and in the \textit{Bhāra\-ta\-saṃ\-hitā} {\rm (}i.e. the \textit{Mahābhārata}{\rm )}, the practice of purity is definitely expounded in great detail. Now that you have asked me [about it], I taught it [to you] in a condensed form. \blankfootnote{5.17 In \textit{pāda} b, since °\textit{saṃhite} is not a correct locative of °\textit{saṃhitā}, 
  instead of emending to \textit{śaive bhāratasaṃhite}, we may take the compound 
  as a \textit{samāhāra\-dvandva\-samāsa} in the neuter locative.
 Note the gender and number confusion between \textit{kīrtitāni} and °\textit{ācāram} in \textit{pāda}s cd.
  This and the next verse sound as if the author had been aware of the fact that he 
  left the remaining three categories of purity {\rm (}see 5.4{\rm )} unexplained.
 }}

  \maintext{satyavādī śucir nityaṃ dhyānayogarataḥ śuciḥ |}%

  \maintext{ahiṃsakaḥ śucir dānto dayābhūtakṣamā śuciḥ }||\thinspace5:18\thinspace||%
\translation{He who speaks the truth is pure. He who engages in yogic meditation is pure. He who avoids violence and is restrained is pure. Compassion towards living beings and patience is purity. \blankfootnote{5.18 My impression is that \textit{dayābhūtakṣamā} in \textit{pāda} d may stand for \textit{bhūtadayā kṣamā} {\rm (}\textit{bhūtadayā} occurring in
  1.7 and 3.27--28{\rm )}, and I translate accordingly.
 }}

  \maintext{sarveṣām eva śaucānām arthaśaucaṃ paraṃ smṛtam |}%

  \maintext{yo 'rthe hi śuciḥ sa śucir na mṛdvāriśuciḥ śuciḥ |}%

  \maintext{kāyavāṅmanasāṃ śaucaṃ sa śuciḥ sarvavastuṣu }||\thinspace5:19\thinspace||%
\translation{Of all the [ways of] purification, material purification is taught to be the highest. For he who is pure with regards to material things is truly pure, and not the one who [only] uses clay and water [i.e. the one who performs only ordinary baths]. When purification pertains to the body, to speech and to the mind, he is pure in all respects. \blankfootnote{5.19 \textit{Pāda}s a-d are quoting \MANU\ 5.106 {\rm (}in most witnesses, unmetrically{\rm )}; it is translated by
  Olivelle {\rm (}\citeyear{OlivelleManu}, 144{\rm )} as:
  `Purifying oneself with respect to wealth, tradition tells us, is the highest of all
  purifications; for the truly pure man is the one who is pure with respect to wealth, not
  the one who becomes pure by using earth and water.'
 }}

  \maintext{śaucāśaucavidhijñamānava yadi kālakṣaye niścayaḥ}%

 \nonanustubhindent \maintext{saubhāgyatvam avāpnuvanti satataṃ kīrtir yaśo'laṅkṛtam |}%

  \maintext{prāptaṃ tena ihaiva puṇyasakalaṃ saddharmaśāstreritaṃ}%

 \nonanustubhindent \maintext{jīvānte ca paratra{-}m{-}īhitagatiṃ prāpnoti niḥsaṃśayam }||\thinspace5:20\thinspace||%
\translation{If a person who knows the rules of purity and impurity is determined to destroy aging, he will surely gain attractiveness, eternally embellished with glory and fame. He has obtained here in this world all the merits that the books on true Dharma teach, and at the end of his life he will undoubtedly reach the desired path in the other world. \blankfootnote{5.20 Note the stem form noun °\textit{mānava} metri causa and
  the second syllable of \textit{yadi} as a long syllable at the c\ae sura in \textit{pāda} a 
  {\rm (}see \msM's reading{\rm )}. In place of the plural \textit{āpnuvanti} one 
  would expect a verb in the singular, and \textit{kīrtir} is metri causa 
  for a compounded stem form {\rm (}\textit{kīrti}°{\rm )} in \textit{pāda} b.
  Note also the sandhi-bridge \textit{-m-} in \textit{paratra-m-īhita}° in \textit{pāda} d. 
  Compare with 4.67b above.
 }}

\centerline{\maintext{\dbldanda\thinspace iti vṛṣasārasaṃgrahe śaucācāravidhir{ }nāmādhyāyaḥ pañcamaḥ\thinspace\dbldanda}}
\translation{Here ends the fifth chapter in the \textit{Vṛṣasārasaṃgraha} called the Method of Purification.}

  \chptr{ṣaṣṭho 'dhyāyaḥ}
\addcontentsline{toc}{section}{Chapter 6}
\fancyhead[CO]{{\footnotesize\textit{Translation of chapter 6}}}%

  \trchptr{ Chapter Six }%

  \subchptr{niyameṣv ijyā {\rm {\rm (}2{\rm )}}}%

  \trsubchptr{Second Niyama-rule: sacrifice}%

  \maintext{atha pañcavidhām ijyāṃ pravakṣyāmi dvijottama |}%

  \maintext{dharmamokṣaprasiddhyarthaṃ śṛṇuṣvāvahito dvija }||\thinspace6:1\thinspace||%
\translation{[Anarthayajña continued:] Now I shall teach you the five types of sacrifice {\rm (}\textit{ijyā}{\rm )}, O best of the twice-born, for success in Dharma and liberation. Listen carefully, O Brahmin. }

  \maintext{arthayajñaḥ kriyāyajño japayajñas tathaiva ca |}%

  \maintext{jñānaṃ dhyānaṃ ca pañcaitat pravakṣyāmi pṛthak pṛthak }||\thinspace6:2\thinspace||%
\translation{Material sacrifice, sacrifice through work, sacrifice through recitation, knowledge and meditation: I shall teach you these five one by one. \blankfootnote{6.2 Note the singular \textit{etat} after a number {\rm (}see Introduction p.~\pageref{number}{\rm )}.
 
  Compare this list of five to the somewhat similar \BHG\ 4.28:
  
 
  \textit{dravyayajñās tapoyajñā yogayajñās tathāpare}\thinspace | 
 
  \textit{svādhyāyajñānayajñāś ca yatayaḥ saṃśitavratāḥ}\thinspace ||
  
 
  \SDHU\ chapter 3 can be also relevant since it uses the terms
  \textit{japayajña}, \textit{jñānayajña}, and \textit{dhyānayajña}. 
  See also \SDHU\ 1.10 {\rm (}\msCa\ f.\thinspace 42v l.~4{\rm )}:
  
 
  \textit{karmayajñas tapoyajñaḥ svādhyāyo dhyānam eva ca}\thinspace | 
 
  \textit{jñānayajñaś ca pañcaite mahāyajñāḥ prakīrtitāḥ}\thinspace ||
  
 
  Note how this definition of the five \textit{mahāyajña}s in the \SDHU\ 
  is different from the one, e.g., in \MANU\ 3.69--71
  {\rm (}\textit{brahma}°, \textit{pitṛ}°, \textit{daiva}°, \textit{bhauta}°, and \textit{nṛyajña}{\rm )}.
 }}

  \subsubchptr{arthayajñaḥ}%

  \trsubsubchptr{Material sacrifice}%

  \maintext{agnyupāsanakarmādi agnihotrakratukriyā |}%

  \maintext{aṣṭakā pārvaṇī śrāddhaṃ dravyayajñaḥ sa ucyate }||\thinspace6:3\thinspace||%
\translation{Material sacrifice includes the following: the domestic ritual fire worship etc., the public performance of the ritual of Agnihotra, [and the so-called \textit{pākayajña}s such as] the Aṣṭakā oblation, the Pārvaṇī oblation, and the ancestral ritual {\rm (}\textit{śrāddha}{\rm )}. \blankfootnote{6.3 By somewhat overtranslating the items in this list, I want to emphasise that
  the text introduces three categories of sacrifical rituals well-known from
  the time of the Gṛhyasūtras and Śrautasūtras: those of the domestic or \textit{aupāsana} fire {\rm (}\textit{gṛhyakarman}{\rm )},
  the Śrauta rituals such as the Agnihotra, and the Smārta \textit{pākayajña}s, such as the \textit{aṣṭakā}, 
  the \textit{pārvaṇī} and the \textit{śrāddha}. For a mention of the \textit{pākayajña}s in a manner similar to 
  our \textit{pāda}s cd here, see, e.g., a verse in the \Diksottara\ quoted 
  in \mycitep{NisvasaGoodall}{275}:
  
 
  \textit{aṣṭakāḥ pārvaṇī śrāddhaṃ śrāvaṇy āgrāyaṇī tathā}\thinspace | 
 
  \textit{caitrī cāśvayujī caiva pākayajñāḥ prakīrtitāḥ}\thinspace ||~178~||
  
 
  For an earlier list of \textit{pākayajña}s, see \GAUTDHS\ 1.8.19: 
  \textit{aṣṭakā pārvaṇaḥ śrāddham śrāvaṇy\-āgrahāyaṇī\-caitry\-āśvayujīti sapta pākayajñasamsthāḥ}.
 }}

  \subsubchptr{kriyāyajñaḥ}%

  \trsubsubchptr{Sacrifice through work}%

  \maintext{ārāmodyānavāpīṣu devatāyataneṣu ca |}%

  \maintext{svahastakṛtasaṃskāraḥ kriyāyajña sa ucyate }||\thinspace6:4\thinspace||%
\translation{Sacrifice through work means constructing {\rm (}\textit{saṃskāra}{\rm )} a grove, a park, a pond, or a temple with one's own hands. }

  \subsubchptr{japayajñaḥ}%

  \trsubsubchptr{Sacrifice through recitation}%

  \maintext{japayajñaṃ tato vakṣye svargamokṣaphalapradam |}%

  \maintext{vedādhyayana kartavyaṃ śivasaṃhitam eva ca |}%

  \maintext{itihāsapurāṇaṃ ca japayajñaḥ sa ucyate }||\thinspace6:5\thinspace||%
\translation{Next I shall teach you the sacrifice through recitation, the bestower of the fruits of heaven and liberation. One should recite the Vedas, Śaiva collections, Itihāsas and Purāṇas: this is called sacrifice through recitation. \blankfootnote{6.5 Note the stem form \textit{vedādhyayana} in \textit{pāda} c metri causa. 
  There are several possible interpretations for \textit{pāda}s d and e.
  \textit{śivasaṃhitam} could mean `Śaiva texts and the [Bhārata]saṃhitā,'
  i.e. the \MBh;
  see 5.17b above: \textit{śaivabhāratasaṃhite}. Alternatively, it
  may mean `the collection of Śaiva teachings.'
  As for \textit{itihāsapurāṇaṃ}, it is most probably a dvandva compound, possibly
  denoting primarily the \MBh\ {\rm (}but perhaps not the 
  \Ramayana, contrary to claims such as that, e.g., in 
  \mycitep{SocialLiteraryHistoryDharma}{34, n.~6}{\rm )},
  and the Purāṇas. In case \textit{saṃhitam} in \textit{pāda} d means the \MBh,
  \textit{itihāsapurāṇaṃ} could in general mean `histories and legends.'
  In my translation, I have left these terms untranslated.
  For the debate on what \textit{itihāsa} is, see, e.g., \mycite{ReadingFifthVeda} and
  \mycite{BaileyPuranas}. 
  
  Both \textit{śivasaṃhitam} and \textit{itihāsapurāṇaṃ} should be interpreted as
  being part of the compound in \textit{pāda} c: \textit{śiva\-saṃhitādhyayanaṃ} and 
  \textit{itihāsapurāṇādhyayanaṃ}. 
 
  See \textit{japayajña} mentioned, e.g., in \BHG\ 10.25c {\rm (}\textit{yajñānāṃ japayajño 'smi}{\rm )} 
  and \MANU\ 2.86 {\rm (}\textit{vidhiyajñāj japayajño viśiṣṭo daśabhir guṇaiḥ}{\rm )}.
 }}

  \subsubchptr{jñānayajñaḥ}%

  \trsubsubchptr{Sacrifice through knowledge}%

  \maintext{idaṃ karma akarmedam ūhāpohaviśāradaḥ |}%

  \maintext{śāstracakṣuḥ samālokya jñānayajñaḥ sa ucyate }||\thinspace6:6\thinspace||%
\translation{[He who can decide if] `this is [proper] action; the other is improper action' because he is knowledgeable about reasoning pro and contra, and conducts investigations with his eyes on the Śāstras, is called [a person performing] sacrifice through knowledge. \blankfootnote{6.6 For the expression \textit{śāstracakṣuḥ}, see, e.g., \BRAHMAP\ 24.21:
  
 
  \textit{tena yajñān yathāproktān mānavāḥ śāstracakṣuṣaḥ}\thinspace | 
 
  \textit{kurvate 'harahaś caiva devān āpyāyayanti te}\thinspace ||
  
 
  In G. P. Bhatt's translation {\rm (}\citeyear{BrahmapuranaTr1}, 126{\rm )}:
  `Day by day men with the sacred scriptures as their guides
  perform sacrifices in the manner they have been laid down and thereby nourish the gods.'
 }}

  \subsubchptr{dhyānayajñaḥ}%

  \trsubsubchptr{Sacrifice through meditation}%

  \maintext{dhyānayajñaṃ samāsena kathayiṣyāmi te śṛṇu |}%

  \maintext{dhyānaṃ pañcavidhaṃ caiva kīrtitaṃ hariṇā purā |}%

  \maintext{sūryaḥ somo 'gni sphaṭikaḥ sūkṣmaṃ tattvaṃ ca pañcamam }||\thinspace6:7\thinspace||%
\translation{I shall teach you concisely about sacrifice through meditation. Listen to me. Meditation was taught by Hari in the past as of five kinds. [Meditation on] the Sun, the Moon, Fire, Crystal and the subtle \textit{tattva} as fifth. \blankfootnote{6.7 For an analysis of this fivefold method of meditation, and this ancient-looking
  \textit{tattva}-system, see Intro \verify, and compare with 
  \VSS\ 4.72--73, and the similar teaching in \VSS\ 22.19--28
  and \DHARMP\ 4.5--14. \textit{Pāda} e is unmetrical, or possibly an
  exceptional expansion of the muta cum liquida licence, the
  syllable \textit{spha}° not turning the previous syllable long,
  and thus making the \textit{pāda} a \textit{na-vipulā}.
 }}

  \maintext{sūryamaṇḍalam ādau tu tattvaṃ prakṛtir ucyate |}%

  \maintext{tasya madhye śaśiṃ dhyāyet tattvaṃ puruṣa ucyate }||\thinspace6:8\thinspace||%
\translation{First it is the Sun [that should be meditated upon], which is said to be \textit{prakṛti-tattva}. He should visualize the Moon in its centre: that \textit{tattva} is said to be \textit{puruṣa}. \blankfootnote{6.8 Note the thematised form \textit{śaśiṃ} for \textit{śaśinaṃ}.
 }}

  \maintext{candramaṇḍalamadhye tu jvālām agniṃ vicintayet |}%

  \maintext{prabhutattvaḥ sa vijñeyo janmamṛtyuvināśanaḥ }||\thinspace6:9\thinspace||%
\translation{In the centre of the Moon's disk, he should visualise a flame, a fire. That is said to be \textit{prabhu}-\textit{tattva}, the destroyer of [the circle of] birth and death. }

  \maintext{agnimaṇḍalamadhye tu dhyāyet sphaṭika nirmalam |}%

  \maintext{vidyātattvaḥ sa vijñeyaḥ kāraṇam ajam avyayam }||\thinspace6:10\thinspace||%
\translation{In the centre of the ring of Fire, he should visualize a spotless crystal. That is said to be \textit{vidyā}-\textit{tattva}, the never-born, imperishable cause. \blankfootnote{6.10 Note the stem form \textit{sphaṭika} in \textit{pāda} b metri causa.
 }}

  \maintext{vidyāmaṇḍalamadhye tu dhyāyet tattvam anuttamam |}%

  \maintext{akīrtitam anaupamyaṃ śivam akṣayam avyayam |}%

  \maintext{pañcamaṃ dhyānayajñasya tattvam uktaṃ samāsataḥ }||\thinspace6:11\thinspace||%
\translation{In the centre of the disk of \textit{vidyā}, he should visualize the highest \textit{tattva}, never-heard, unparalleled, undecaying and imperishable Śiva. The fifth \textit{tattva} of the sacrifice through meditation has been taught in short. }

  \maintext{vigatarāga uvāca |}%

  \maintext{ekaikasya tu tattvasya phalaṃ kīrtaya kīdṛśam |}%

  \maintext{kāni lokāḥ prapadyante kālaṃ vāsya tapodhana }||\thinspace6:12\thinspace||%
\translation{Vigatarāga spoke: Teach me, what are the fruits of [reaching] each \textit{tattva}? Which worlds can be attained and how much time [can one spend there], O great ascetic? \blankfootnote{6.12 The reading \textit{tritattvasya} in \textit{pāda} a in the MSS is a problem 
  because we have just finished a section mentioning five \textit{tattva}s. 
  {\rm (}This was probably noticed by \Ed, hence printing \textit{hi} for \textit{tri}°.{\rm )}
  My conjecture {\rm (}\textit{tu}{\rm )} is based on the assumption that \textit{tri} is ofter written as \textit{tṛ} 
  in Nepalese MSS {\rm (}e.g. in \msM\ at this point{\rm )} and that \textit{tṛ} may then easily get corrupted to \textit{tu}.
 }}

  \maintext{anarthayajña uvāca |}%

  \maintext{brahmalokaṃ tu prathamaṃ tattvaprakṛticintayā |}%

  \maintext{kalpakoṭisahasrāṇi śivavan modate sukhī }||\thinspace6:13\thinspace||%
\translation{Anarthayajña spoke: Through meditation on the first \textit{tattva}, \textit{prakṛti}, [one can reach] Brahmaloka. He will rejoice [there] happily like Śiva for millions of \ae ons. \blankfootnote{6.13 Understand \textit{pāda}s ab as \textit{brahmalokaṃ prathamatattvacintayā prakṛtitattvacintayā}. 
  One might take \textit{prathamaṃ} adverbially {\rm (}`firstly': \textit{prathamaṃ brahmalokaṃ prakṛtitattvacintayā}{\rm )},
  but in the next verses, the ordinal numbers {\rm (}\textit{dvitīyaṃ, tṛtīyaṃ, pañcamaṃ}{\rm )}
  always refer to the \textit{tattva}s. \textit{Pāda} a is a \textit{na-vipulā} if the 
  muta cum liquid licence is applied and the syllable \textit{pra}° does not
  turn the previous syllable long.
 }}

  \maintext{dvitīyaṃ tattva puruṣaṃ dhyāyamāno mṛto yadi |}%

  \maintext{viṣṇulokam ito yāti kalpakoṭyayutaṃ sukhī }||\thinspace6:14\thinspace||%
\translation{If one dies while meditating on the second \textit{tattva}, \textit{puruṣa}, one will depart from this world and go to Viṣṇuloka, [and will dwell there] happily for billions of \ae ons. \blankfootnote{6.14 Note the stem form \textit{tattva} in \textit{pāda} a metri causa {\rm (}\textit{na-vipulā}{\rm )}.
 }}

  \maintext{prabhutattvaṃ tṛtīyaṃ tu dhyāyamāno mariṣyati |}%

  \maintext{śivaloke vasen nityaṃ kalpakoṭyayutaṃ śatam }||\thinspace6:15\thinspace||%
\translation{Should one die while meditating on the third, the \textit{prabhu-tattva}, one can live in Śivaloka continuously for a hundred billion \ae ons. \blankfootnote{6.15 \Ed\ changes \textit{śivaloka} to \textit{rudraloka}, probably for more contrast with
  \textit{sadāśiva} in 6.16 and \textit{śivatattva} in 6.17, but this is not
  Naraharinath's intervention since \msPaperA, a MS close to his sources,
  also reads \textit{rudraloka} {\rm (}on \msPaperA, see p.~\pageref{msPaperAdesc}{\rm )}.
 }}

  \maintext{vidyātattvāmṛtaṃ dhyāyet sadāśivam anāmayam |}%

  \maintext{akṣayaṃ lokam āpnoti kalpānāntaparaṃ tathā  }||\thinspace6:16\thinspace||%
\translation{If one visualizes the nectar of \textit{vidyā-tattva}, [i.e.] Sadāśiva, one can reach [His] diseaseless, imperishable world [and can live there] well beyond endless \ae ons. \blankfootnote{6.16 In \textit{pāda} a, \textit{amṛta} is suspect. It may qualify the world of Sadāśiva {\rm (}`immortal'{\rm )} and 
  then \textit{vidyātattva} is in stem form. Alternatively, since this verse is the only one in
  this list of worlds {\rm (}6.13--17{\rm )} without an ordinal number, \textit{amṛtaṃ} may mean `four' or possibly `fourth,'
  as suggested by Monier-Williams and Apte in their dictionaries. This meaning would fit in nicely.
  In addition, dying has been mentioned above, thus \textit{amṛtaṃ} might be a corrupted form of 
  a participle from the verbal root \textit{mṛ} {\rm (}\textit{mṛyan} or \textit{maran}?{\rm )}: e.g., 
  \textit{vidyātattvaṃ mṛyan dhyāyet}\dots\ {\rm (}`should he meditation upon Vidyātattva while dying...'{\rm )}.
 }}

  \maintext{pañcamaṃ śivatattvaṃ tu sūkṣmaṃ cātmani saṃsthitam |}%

  \maintext{na kālasaṃkhyā tatrāsti śivena saha modate }||\thinspace6:17\thinspace||%
\translation{The fifth one, the subtle \textit{śiva-tattva} dwells in the Self. There is no counting of time there and he will be rejoicing [there] together with Śiva. \blankfootnote{6.17 \textit{Pāda} c is a \textit{ma-vipulā}.
 }}

  \maintext{pañcadhyānābhiyukto bhavati ca na punarjanmasaṃskārabandhaḥ}%

 \nonanustubhindent \maintext{jijñāsyantāṃ dvijendra bhavadahanakaraḥ prārthanākalpavṛkṣaḥ |}%

  \maintext{janmenaikena muktir bhavati kimu na vā mānavāḥ sādhayantu}%

 \nonanustubhindent \maintext{pratyakṣān nānumānaṃ sakalamalaharaṃ svātmasaṃvedanīyam }||\thinspace6:18\thinspace||%
\translation{[If] he practises the five meditations, there will be no rebirth and no more fetters of transmigration. O excellent Brahmin, [the five meditation] should be learnt. [They] burn away existence, a wishing tree of desires. Liberation will come within one single birth. Why should people not master [these meditations that] destroy all impurities perceptibly, not only by inference, [since they] are to be experienced by one's own Self. \blankfootnote{6.18 Note how a plural passive imperative form {\rm (}\textit{jijñāsyantāṃ}{\rm )} stands for the singular
  {\rm (}\textit{jijñāsyatāṃ}{\rm )} metri causa, or rather, since probably the five types of
  meditation are meant, the singulars in \textit{pāda} b are somewhat
  out of context. Note also that the last syllable of
  \textit{dvijendra} {\rm (}at the c\ae sura{\rm )} counts here as long: this phenomenon of a word-ending
  syllable becoming long by position is common in the \VSS\ {\rm (}see p.~\pageref{short2long} ff{\rm )}.
 The non-standard \textit{janmena} in \textit{pāda} d seems superior to \textit{janmanā} for it
  preserves the metre.
 }}

  \subchptr{niyameṣu tapaḥ {\rm {\rm (}3{\rm )}}}%

  \trsubchptr{Third Niyama-rule: penance}%

  \maintext{mānasaṃ tapa ādau tu dvitīyaṃ vācikaṃ tapaḥ |}%

  \maintext{kāyikaṃ ca tṛtīyaṃ tu manovākkarma tatparam |}%

  \maintext{kāyikaṃ vācikaṃ caiva tapo miśraka pañcamam }||\thinspace6:19\thinspace||%
\translation{The first type of penance is mental penance, the second is verbal penance, the third is the bodily one, the next one is the one which is [characterised by] both mental and verbal action. The fifth type of penance is a mixture of the bodily and the verbal ones. \blankfootnote{6.19 The reading \textit{manovākkāya}° {\rm (}\msNa\msNb{\rm )} in \textit{pāda} d is probably secondary, influenced by
  such common expressions as, e.g., \textit{manovākkāyakarmabhiḥ} in \YAJNS\ 1.27d.
 Note the stem form \textit{miśraka} in \textit{pāda} f metri causa.
 }}

  \maintext{manaḥsaumyaṃ prasādaś ca ātmanigraham eva ca |}%

  \maintext{maunaṃ bhāvaviśuddhiś ca pañcaitat tapa mānasam }||\thinspace6:20\thinspace||%
\translation{Gentleness of the mind, calmness, self-control, observing silence, and the purification of one's state of mind: mental penance comprises these five. \blankfootnote{6.20 Again, we can see the use of the singular {\rm (}\textit{etat}{\rm )} next to numbers; note also 
  the stem form \textit{tapa} in \textit{pāda} d metri causa.
  This verse is a paraphrase of \MBH\ 3.39.16 {\rm (}\BHG\ 17.16; see text in the
  apparatus{\rm )}.
 }}

  \maintext{anudvegakarā vāṇī priyaṃ satyaṃ hitaṃ ca yat |}%

  \maintext{svādhyāyābhyasanaṃ caiva vācikaṃ tapa ucyate }||\thinspace6:21\thinspace||%
\translation{Verbal penance is taught as speech that causes no anxiety, which is kind, true and useful, and it includes also the practice of recitation. \blankfootnote{6.21 This verse is a variant of \MBH\ 6.39.15 {\rm (}\BHG\ 17.15; see it in the apparatus{\rm )}.
 }}

  \maintext{ārjavaṃ ca ahiṃsā ca brahmacaryaṃ surārcanam |}%

  \maintext{śaucaṃ pañcamam ity etat kāyikaṃ tapa ucyate }||\thinspace6:22\thinspace||%
\translation{Bodily penance is taught as follows: honesty, harmlessness, chastity, the worship of gods, and purity as the fifth. \blankfootnote{6.22 This verse seems to be a paraphrase of \MBH\ 6.39.14 {\rm (}\BHG\ 17.14; see it in the apparatus{\rm )}.
 }}

  \maintext{iṣṭaṃ kalyāṇabhāvaṃ ca dhanyaṃ pathyaṃ hitaṃ vadet |}%

  \maintext{manomiśraka pañcaitat tapa uktaṃ maharṣibhiḥ }||\thinspace6:23\thinspace||%
\translation{[Penance] which is a mixture of the mental [and the verbal] is taught by the great sages to be these five: he should speak [about things that are] agreeable, of a noble character, virtuous, salutary, and useful. \blankfootnote{6.23 Note the use of the singular {\rm (}\textit{etat}{\rm )} next to a number, and the stem form noun in \textit{pāda} c.
 }}

  \maintext{svasti maṅgalam āśīrbhir atithigurupūjanam |}%

  \maintext{kāyamiśraka pañcaitat tapa uktaṃ mahātmabhiḥ }||\thinspace6:24\thinspace||%
\translation{[Penance] in which bodily [and verbal actions] mix is taught by the great-souled ones to be these five: benediction, greetings, blessings, and the worship of the guest and the guru. \blankfootnote{6.24 See \SDHS\ 11.73--79 {\rm (}and \mycitep{SaivaUtopia}{91--93 and 120--121}{\rm )} 
  for a somewhat similar discussion on `kind speach.'
 }}

  \maintext{maṇḍūkayogī hemante grīṣme pañcatapās tathā |}%

  \maintext{abhrāvakāśo varṣāsu tapaḥsādhanam ucyate }||\thinspace6:25\thinspace||%
\translation{[Being] a [so-called] frog-yogin in the winter, or one with the five fires in the summer, or having the clouds [i.e. the open sky] for shelter in the rainy season: these are called accomplishments of penance. \blankfootnote{6.25 \textit{Pāda}s a and c are \textit{ma-vipulā}s.
 \Manu\ 6.23 mentions three kinds of penance that correspond to three seasons:
  
 
  \textit{grīṣme pañcatapās tu syād varṣāsv abhrāvakāśikaḥ}\thinspace | 
 
  \textit{ārdravāsās tu hemante kramaśo vardhayaṃs tapaḥ}\thinspace ||
  
 
  Translated in \mycitep{OlivelleManu}{149} as:
  `[He should] surround himself with the five fires in the summer; live in the open air during the rainy season;
  and wear wet clothes in the winter---gradually intensifying his ascetic toil.'
  This and \SDHSAMGR\ 9.32ab {\rm (}quoted in the apparatus{\rm )} may suggest that being 
  a `frog-yogin' could be the same as wearing wet clothes or standing in water for a long time.
  A footnote to \MBH\ 12.309.9 in the Kumbakonam edition of the 
  \MBH\ {\rm (}\mycite{MBhKumbakonaEd}{\rm )} suggests otherwise:
  \textit{maṇḍūkavat pāṇipādaṃ saṅkocya nyubjaḥ śete iti maṇḍūkaśāyī}. {\rm (}`The word `frog-sleeper' means
  somebody who sleeps like a frog, with his hands and feet withdrawn and with his back humped.'{\rm )} 
 }}

  \maintext{svamāṃsoddhṛtya dānaṃ ca hastapādaśiras tathā |}%

  \maintext{puṣpam utpādya dānaṃ ca sarve te tapasādhanāḥ }||\thinspace6:26\thinspace||%
\translation{Carving out his own flesh as a donation, or [offering his own] hand, feet and head, or drawing [his own] blood {\rm (}\textit{puṣpa}{\rm )} as a donation: all these are accomplishments of penance, \blankfootnote{6.26 Note the stem form \textit{svamāṃsa} in \textit{pāda} a for the accusative.
 The translation of \textit{pāda} c is tentative, but \textit{puṣpa} as `blood' does 
  occur in tantric texts {\rm (}see, e.g., \SYM\ 16.49{\rm )}. \VSS\ 17.37--38 teaches
  blood donation:
  
 
  \textit{devī uvāca}\thinspace |
 
  \textit{svamāṃsarudhiraṃ dānaṃ dānaṃ putrakalatrayoḥ}\thinspace |
 
  \textit{kiṃ praśasyaṃ mahādeva tattvaṃ vaktum ihārhasi}\thinspace ||
 
  \textit{maheśvara uvāca}\thinspace |
 
  \textit{svamāṃsarudhiraṃ dānaṃ praśaṃsanti manīṣiṇaḥ}\thinspace |
 
  \textit{śrūyatāṃ pūrvavṛttāni saṃkṣipya kathayāmy aham}\thinspace ||
  
 
  `Devī spoke: Are one's own flesh and blood and one's son and wife praised as donation,
  O Mahādeva? Tell me the truth please.
  Maheśvara spoke: The wise praise one's own flesh and blood as donation.
  Let's hear the old legends, I shall tell you briefly.'
 }}

  \maintext{kṛcchrātikṛcchraṃ naktaṃ ca taptakṛcchram ayācitam |}%

  \maintext{cāndrāyaṇaṃ parākaṃ ca tapaḥ sāṃtapanādayaḥ }||\thinspace6:27\thinspace||%
\translation{[as also] the `painful penance' and the `extremely paniful one', [eating only] at night, the `hot and painful' and [the one in which only food obtained] without solicitation [can be eaten], the \textit{cāndrāyaṇa} and \textit{parāka} penances, the \textit{sāṃtapana}, etc. \blankfootnote{6.27 \textit{Pāda} a is a \textit{ma-vipulā}s.
 For short descriptions and the loci classici of these penances, see, e.g.,
  \mycitep{KaneHistory}{v. 4, 130--152}.
  For \textit{nakta}/\textit{naktānna}, see \VSS\ 8.22 below and, e.g., 
  \SDHS\ chapter 10 {\rm (}\mycite{SDhS10_ed}{\rm )}, and for \textit{ayācita}, \VSS\ 8.23 below.
 }}

  \maintext{yenedaṃ tapa tapyate sumanasā saṃsāraduḥkhacchidam}%

 \nonanustubhindent \maintext{āśāpāśa vimucya nirmalamatis tyaktvā jaghanyaṃ phalam |}%

  \maintext{svargākāṅkṣyanṛpatvabhogaviṣayaṃ sarvāntikaṃ tatphalaṃ}%

 \nonanustubhindent \maintext{jantuḥ śāśvatajanmamṛtyubhavane tanniṣṭhasādhyaṃ vahet }||\thinspace6:28\thinspace||%
\translation{He who performs with a well-disposed mind this penance that puts an end to the suffering caused by transmigration {\rm (}\textit{saṃsāra}{\rm )}, abandoning the trap of hope, with a spotless mind, giving up the lowest rewards [such as] wishing for heaven and being a king and having enjoyments for the senses, that man will experience the ultimate {\rm (}\textit{sarvāntika}{\rm )} reward that in this home of eternal births and deaths accomplishes their cessation. \blankfootnote{6.28 Note my emendation in \textit{pāda} a {\rm (}\textit{sumanasā} from \textit{sumanasaḥ}{\rm )} and that
  in order to restore the metre, I accepted \Ed's stem form \textit{tapa}.
 Note the stem form °\textit{pāśa} in \textit{pāda} b metri causa.
 }}

\centerline{\maintext{\dbldanda\thinspace iti vṛṣasārasaṃgrahe ṣaṣṭho 'dhyāyaḥ\thinspace\dbldanda}}
\translation{Here ends the sixth chapter in the \textit{Vṛṣasārasaṃgraha}.}

  \chptr{saptamo 'dhyāyaḥ}
\addcontentsline{toc}{section}{Chapter 7}
\fancyhead[CO]{{\footnotesize\textit{Translation of chapter 7}}}%

  \trchptr{ Chapter Seven }%

  \subchptr{niyameṣu dānam {\rm {\rm (}4{\rm )}}}%

  \trsubchptr{Fourth Niyama-rule: donation}%

  \maintext{dānāni ca tathety āhuḥ pañcadhā munibhiḥ purā |}%

  \maintext{annaṃ vastraṃ hiraṇyaṃ ca bhūmi godāna pañcamam }||\thinspace7:1\thinspace||%
\translation{In the past the wise declared that, again, there were five kinds of donation. Donation of food, clothes, gold, land, and the fifth, donation of cows. \blankfootnote{7.1 \textit{tathety} in \textit{pāda} a is suspicious and my translation of it {\rm (}`again'{\rm )} is tentative and
  is supposed to refer back to the fact that all \textit{yama}s so far have been 
  devided into five types. Note how \textit{annaṃ}, \textit{vastraṃ}, \textit{hiraṇyaṃ} and 
  \textit{bhūmi} {\rm (}the latter treated as neuter, or given in stem form{\rm )} 
  are all meant to go with °\textit{dāna} {\rm (}again, in stem form, metri causa{\rm )}.
 }}

  \subsubchptr{annadānam}%

  \trsubsubchptr{Donation of food}%

  \maintext{annāt tejaḥ smṛtiḥ prāṇaḥ annāt puṣṭir vapuḥ sukham |}%

  \maintext{annāc chrīḥ kānti vīryaṃ ca annāt sattvaṃ ca jāyate }||\thinspace7:2\thinspace||%
\translation{From food [come] energy, memory, the vital breath, growth, body, happiness. From food arise grace and beauty, heroism, strength. \blankfootnote{7.2 Note the stem form noun \textit{kānti} metri causa in \textit{pāda} c.
 }}

  \maintext{annāj jīvanti bhūtāni annaṃ tuṣṭikaraṃ sadā |}%

  \maintext{ānnāt kāmo mado darpaḥ annāc chauryaṃ ca jāyate }||\thinspace7:3\thinspace||%
\translation{Living beings live on food. Food always satisfies. From food arise desire, rapture, pride, and valour. }

  \maintext{annaṃ kṣudhātṛṣāvyādhīn sadya eva vināśayet |}%

  \maintext{annadānāc ca saubhāgyaṃ khyātiḥ kīrtiś ca jāyate }||\thinspace7:4\thinspace||%
\translation{Food drives away hunger and thirst and disease instantly. From donations of food arise beauty, fame, and glory. }

  \maintext{annadaḥ prāṇadaś caiva prāṇadaś cāpi sarvadaḥ |}%

  \maintext{tasmād annasamaṃ dānaṃ na bhūtaṃ na bhaviṣyati }||\thinspace7:5\thinspace||%
\translation{He who donates food donates life. He who donates life donates everything. Therefore nothing is equal to the donation of food, nothing was, nothing will be. \blankfootnote{7.5 See some similar verses from the \SDHU, the \MBH, and the \NARADAP\ in
  the apparatus.
 }}

  \subsubchptr{vastradānam}%

  \trsubsubchptr{Donation of clothes}%

  \maintext{vastrābhāvān manuṣyasya śriyād api parityajet |}%

  \maintext{vastrahīno na pūjyeta bhāryāputrasakhādibhiḥ }||\thinspace7:6\thinspace||%
\translation{In the absence of [proper] clothes, a man will also lose his fortunes. A person without clothes may not be respected by his wife, son, friends, etc. \blankfootnote{7.6 \textit{Pāda} b is difficult to interpret securely. I translate it as if reading
  \textit{śrīs tam api parityajet} or \textit{śriyāpi parityajyate}. Consider also \BRAHMAP\ 220.139:
  
 
  \textit{vastrābhāve kriyā nāsti yajñā vedās tapāṃsi ca}\thinspace |
 
  \textit{tasmād vāsāṃsi deyāni śrāddhakāle viśeṣataḥ}\thinspace ||
  
 
  `If one has no clothes, there is no ritual, no worship, no Vedas or penance.
  Therefore clothes should be donated, especially at the time
  of a Śrāddha ritual.'
 }}

  \maintext{vidyāvān sukulīno 'pi jñānavān guṇavān api |}%

  \maintext{vastrahīnaḥ parādhīnaḥ paribhūtaḥ pade pade }||\thinspace7:7\thinspace||%
\translation{Be it a learned person from a good family or an intelligent and virtuous person, without clothes everybody is subdued and humiliated on every occasion }

  \maintext{apamānam avajñāṃ ca vastrahīno hy avāpnuyāt |}%

  \maintext{jugupsati mahātmāpi sabhāstrījanasaṃsadi }||\thinspace7:8\thinspace||%
\translation{because a man without clothes receives contempt and disrespect. Even if he is a great soul, he will wish to avoid the court, women, and the assembly. }

  \maintext{tasmād vastrapradānāni praśaṃsanti manīṣiṇaḥ |}%

  \maintext{na jīrṇaṃ sphuṭitaṃ dadyād vastraṃ kutsitam eva vā }||\thinspace7:9\thinspace||%
\translation{Therefore the wise praise donations of clothes. One should not give away old, torn or dirty clothes. }

  \maintext{navaṃ purāṇarahitaṃ mṛdu sūkṣmaṃ suśobhanam |}%

  \maintext{susaṃskṛtya pradātavyaṃ śraddhābhaktisamanvitam }||\thinspace7:10\thinspace||%
\translation{[Clothes] should be donated [only if they are] new, not worn, soft, delicate and beautiful, nicely ornamented, and in good faith and with devotion. }

  \maintext{śraddhāsattvaviśeṣeṇa deśakālavidhena ca |}%

  \maintext{pātradravyaviśeṣeṇa phalam āhuḥ pṛthak pṛthak }||\thinspace7:11\thinspace||%
\translation{They say that the reward [of donation/generosity] is in every case dependent on the particular [donor's] willingness and character, the choice of place and time, and on the particular recipient and material. \blankfootnote{7.11 It seems that \textit{vidhena ca} stands for \textit{vidhinā ca} or rather \textit{vidhānena} metri causa in \textit{pāda} b.
 }}

  \maintext{yādṛśaṃ dīyate vastraṃ tādṛśaṃ prāpyate phalam |}%

  \maintext{jīrṇavastrapradānena jīrṇavastram avāpnuyāt |}%

  \maintext{śobhanaṃ dīyate vastraṃ śobhanaṃ vastram āpnuyāt }||\thinspace7:12\thinspace||%
\translation{The reward received will be similar to the clothes donated. By donating old clothes, one would receive old clothes [as a reward]. By donating beautiful clothes, one would receive beautiful clothes [as a reward]. }

  \maintext{dadyād vastra suśobhanaṃ dvijavare kāle śubhe sādaraṃ}%

 \nonanustubhindent \maintext{saubhāgyam atulaṃ labheta sa naro rūpaṃ tathā śobhanam |}%

  \maintext{tasmin yāti suvastrakoṭi śataśaḥ prāpnoti niḥsaṃśayaṃ}%

 \nonanustubhindent \maintext{tasmāt tvaṃ kuru vastradānam asakṛt pāratrikotkarṣaṇam }||\thinspace7:13\thinspace||%
\translation{Should one bestow very beautiful clothes on a Brahmin at an auspicious time, respectfully, he [i.e. the donor] will receive unequalled attractiveness and a beautiful appearance. When he departs, he will be given hundreds of millions of items of nice clothes, no doubt about that. Therefore do donate clothes often. It is the way up to the other world. \blankfootnote{7.13 Note the stem form \textit{vastra} in \textit{pāda} a metri causa.
 `on a Brahmin' {\rm (}in \textit{pāda} a{\rm )}: literally, `on a person who is first among the twice-born'
  {\rm (}\textit{dvijavare}{\rm )}.
 The final syllable of \textit{saubhāgyam} in \textit{pāda} b counts as long by licence; see, e.g., 5.20 and 6.18b.
  This time the c\ae sura is not involved.
 Understand \textit{tasmin yāti} in \textit{pāda} c as \textit{tasmin yāte} {\rm (}metri causa{\rm )};
  °\textit{koṭi} is treated as neuter or as a stem form {\rm (}also metri causa{\rm )}.
 }}

  \subsubchptr{suvarṇadānam}%

  \trsubsubchptr{Donation of gold}%

  \maintext{suvarṇadānaṃ viprendra saṃkṣipya kathayāmy aham |}%

  \maintext{pavitraṃ maṅgalaṃ puṇyaṃ sarvapātakanāśanam }||\thinspace7:14\thinspace||%
\translation{O great Brahmin, now I shall teach you about the donation of gold in a concise manner. It is a pure, auspicious and meritorious [act] and it washes off all sins. }

  \maintext{dhārayet satataṃ vipra suvarṇakaṭakāṅgulim |}%

  \maintext{mucyate sarvapāpebhyo rāhuṇā candramā yathā }||\thinspace7:15\thinspace||%
\translation{Should one always wear a golden bracelet or ring, O Brahmin, he will be freed of all sins, just as the moon is freed from [the demon] Rāhu [after an eclipse]. \blankfootnote{7.15 I suspect that \textit{aṅguli} is used in \textit{pāda} b in the sense of \textit{aṅgulīya} {\rm (}`finger-ring'{\rm )}.
 }}

  \maintext{dattvā suvarṇaṃ viprebhyo devebhyaś ca dvijarṣabha |}%

  \maintext{tuṭimātre 'pi yo dadyāt sarvapāpaiḥ pramucyate }||\thinspace7:16\thinspace||%
\translation{If a person donates gold to Brahmins or gods, O excellent Brahmin, even if it is only in a minute quantity, he will be freed of all sins. \blankfootnote{7.16 \textit{Pāda} a is a \textit{ma-vipulā}.
 The form \textit{tuṭi} as a widespread variant of \textit{truṭi}, see, e.g., {\rm (}Old{\rm )} \SKANDAP\ 27.14:
  
 
  \textit{kāñcanaṃ tuṭimātraṃ vā yo dadyād bahu vā mama}\thinspace | 
 
  \textit{tasya haimavate śṛṅge dadāni gṛham uttamam}\thinspace ||
 }}

  \maintext{raktimāṣakakarṣaṃ vā palārdhaṃ palam eva vā |}%

  \maintext{evam eva phalaṃvṛddhir jñeyā dānaviśeṣataḥ }||\thinspace7:17\thinspace||%
\translation{[The amount can be just] one \textit{rakti}, a \textit{māṣaka}, a \textit{karṣa}, half a \textit{pala} or a \textit{pala}: this is exactly how the increase in the [size of the corresponding] reward will be, in proportion to the properties [i.e.\ amount] of the donation. \blankfootnote{7.17 I suspect that \textit{phalaṃ vṛddhir}, or \textit{phalaṃvṛddhir}, stands for 
  \textit{phalavṛddhir} {\rm (}\textit{phalasya vṛddhiḥ}{\rm )} metri causa, meaning `the increase of the reward.'
  \textit{rakti}, \textit{māṣaka}, \textit{karṣa}, and \textit{pala} are units of weight.
 }}

  \subsubchptr{bhūmidānam}%

  \trsubsubchptr{Donation of land}%

  \maintext{sarvādhāraṃ mahīdānaṃ praśaṃsanti manīṣiṇaḥ |}%

  \maintext{annavastrahiraṇyādi sarvaṃ vai bhūmisambhavam }||\thinspace7:18\thinspace||%
\translation{The wise praise the donation of land as the basis of everything [else]. Food, clothes, gold etc., all these originate in land. }

  \maintext{bhūmidānena viprendra sarvadānaphalaṃ labhet |}%

  \maintext{bhūmidānasamaṃ vipra yady asti vada tattvataḥ }||\thinspace7:19\thinspace||%
\translation{O Brahmin, one can obtain all the rewards of donation by donating land. If there is anything that equals the donation of land, O Brahmin, you should definitely tell me. }

  \maintext{mātṛkukṣivimuktas tu dharaṇīśaraṇo bhavet |}%

  \maintext{carācarāṇāṃ sarveṣāṃ bhūmiḥ sādhāraṇā smṛtā }||\thinspace7:20\thinspace||%
\translation{[Humans] have the earth as their abode as soon as they get out of the mother's womb. Land is said to be common to all that are mobile and immobile. \blankfootnote{7.20 I take \textit{sādhāraṇā} as one word, but it is possible that the intention of the author
  was \textit{sā dhāraṇā} in two words, in fact meaning \textit{sādhāraḥ} {\rm (}\textit{sā ādhāraḥ}, `it is the basis'{\rm )}.
 }}

  \maintext{ekahastaṃ dvihastaṃ vā pañcāśac chatam eva vā |}%

  \maintext{sahasrāyutalakṣaṃ vā bhūmidānaṃ praśasyate }||\thinspace7:21\thinspace||%
\translation{Be it [only a land of] one forearm, two forearms, fifty or a hundred, a thousand, ten thousand, a hundred thousand, donation of land is held in great esteem. }

  \maintext{ekahastāṃ ca yo bhūmiṃ dadyād dvijavarāya tu |}%

  \maintext{varṣakoṭiśataṃ divyaṃ svargaloke mahīyate }||\thinspace7:22\thinspace||%
\translation{He who donates [as much as] a piece of land of one forearm to a Brahmin will enjoy a billion divine years in heaven. }

  \maintext{evaṃ bahuṣu hasteṣu guṇāguṇi phalaṃ smṛtam |}%

  \maintext{śraddhādhikaṃ phalaṃ dānaṃ kathitaṃ te dvijottama }||\thinspace7:23\thinspace||%
\translation{Thus in case of [donating] many forearms [of land], the reward is said to be proportional to the properties [of the land]. O Brahmin, I have taught you about the rewards of donation that is made in good faith. \blankfootnote{7.23 I think that \textit{guṇāguṇi}, or perhaps \textit{guṇaguṇi} {\rm (}which would be unmetrical, containing
  two \textit{laghu}s in both the second and third syllables of the \textit{pāda}{\rm )}, should refer to the idea
  that, e.g., the donation of a piece of land of 2 × 2 \textit{hasta}s would result in 
  twice, or four times, \textit{koṭiśata} years in heaven, \textit{guṇa} generally meaning `times.' 
  I take \textit{guṇā}° as referring to the size of the land donated, and °\textit{guṇi}[\textit{n}] as 
  `amounting to that many times,' but this is only a guess, 
  and it would need to be supported by some similar passage, other than 7.17 above.
 
 
  I suspect that \textit{pāda} c is an awkward attempt at saying \textit{śraddhādhikadāna{\rm (}sya{\rm )} phalaṃ}.
 }}

  \maintext{jāmadagnyena rāmeṇa bhūmiṃ dattvā dvijāya vai |}%

  \maintext{āyur akṣayam āptaṃ tu ihaiva ca dvijottama }||\thinspace7:24\thinspace||%
\translation{[Paraśu]rāma, the son of Jamadagni, having donated land to the Brahmin [Kaśyapa], obtained eternal life in this very world, O excellent Brahmin. \blankfootnote{7.24 See a summary of the corresponding episodes in the \MBH\ in 
  \mycitep{PuranicEnc}{570--571}, s.v. Paraśurāma:
  
 
  `To atone for the sin of slaughtering even
  innocent Kṣatriyas, Paraśurāma gave away all his
  riches as gifts to brahmins. He invited all the brahmins
  to Samantapañcaka and conducted a great Yāga there.
  The chief Ṛtvik {\rm (}officiating priest{\rm )} of the Yāga was
  the sage Kaśyapa and Paraśurāma gave all the lands
  he conquered till that time to Kaśyapa. Then a platform 
  of gold ten yards long and nine yards wide was
  made and Kaśyapa was installed there and worshipped.
  After the worship was over according to the instructions
  from Kaśyapa the gold platform was cut into pieces
  and the gold pieces were offered to brahmins.
 
  When Kaśyapa got all the lands from Paraśurāma he
  said thus:---``Oh Rāma, you have given me all your
  land and it is not now proper for you to live in my
  soil. You can go to the south and live somewhere on
  the shores of the ocean there.'' Paraśurāma walked
  south and requested the ocean to give him some land to
  live.'
 
 
  Note that without applying the muta cum liquida licence {\rm (}\textit{ca dvi}°{\rm )}, \textit{pāda} d would be iambic and thus
  metrically problematic.
 }}

  \subsubchptr{godānam}%

  \trsubsubchptr{Donation of cows}%

  \maintext{hemaśṛṅgāṃ raupyakṣurāṃ cailaghaṇṭāṃ dvijottama |}%

  \maintext{viprāya vedaviduṣe dattvānantaphalaṃ smṛtam }||\thinspace7:25\thinspace||%
\translation{[A cow] with golden horns, silver hooves, garment and bell, O Brahmin, when given to a Veda-knowing Brahmin, [produces] rewards that are said to be endless. \blankfootnote{7.25 \textit{kṣura} in \textit{pāda} a is a known variant of the better-attested \textit{khura}.
  \textit{Pāda} a is unmetrical.
 \textit{Pāda} c is a \textit{na-vipulā}.
 }}

  \subsubchptr{dānapraśaṃsā}%

  \trsubsubchptr{Praise of donation}%

  \maintext{dānābhyāsarataḥ pravartanabhavāṃ śakyānurūpaṃ sadā}%

 \nonanustubhindent \maintext{annaṃ vastrahiraṇyaraupyam udakaṃ gāvas tilān medinīm |}%

  \maintext{dadyāt pādukachattrapīṭhakalaśaṃ pātrādyam anyac ca vā}%

 \nonanustubhindent \maintext{śraddhādānam abhinnarāgavadanaṃ kṛtvā mano nirmalam }||\thinspace7:26\thinspace||%
\translation{Always rejoicing in the practice of giving, \dots, as far as one's capacities go, one should give food, clothes, gold and silver, water, cows, sesame seeds, land, sandals, parasols, seats, jars, cups, or anything else. By giving in good faith {\rm (}\textit{śraddhādānaṃ kṛtvā}{\rm )}, with words of unconditioned affection, one's mind [becomes] spotless. \blankfootnote{7.26 I am unable to interpret \textit{pravartanabhavāṃ} in \textit{pāda} a and
  I suspect that \textit{śakyānurūpaṃ} in the same \textit{pāda} stands for \textit{śaktyanurūpaṃ} metri causa.
 \textit{abhinnarāgavadanaṃ} in \textit{pāda} d is suspect. Perhaps °\textit{vandanaṃ} was meant 
  {\rm (}`unconditioned affection and adoration'{\rm )}.
 }}

  \maintext{dānād eva yaśaḥ śriyaḥ sukhakarāḥ khyātim atulyāṃ labhed}%

 \nonanustubhindent \maintext{dānād eva nigarhaṇaṃ ripugaṇe ānandadaṃ saukhyadam |}%

  \maintext{dānād ūrjayatā prasādam atulaṃ saubhāgya dānāl labhed}%

 \nonanustubhindent \maintext{dānād eva anantabhoga niyataṃ svargaṃ ca tasmād bhavet }||\thinspace7:27\thinspace||%
\translation{Glory and fortune that makes us happy come about only by donations, and one can gain unequalled fame. Only from donations will reproach [exercised by] the enemy [turn into] pleasure and happiness. Vigour and unequalled graciousness come from donation. One can reach attractiveness thought donations. Endless enjoyments surely come only from donations, and heaven is [reached] also because of it. \blankfootnote{7.27 I suspect that \textit{khyātiś ca tulyaṃ} in the MSS stands for \textit{khyātim atulyāṃ} {\rm (}`and unequalled fame'{\rm )} and
  that it is not a clumsy attempt to restore the metre, but rather a later correction gone wrong.
  I have emended the phrase believing that the second {\rm (}last{\rm )} syllable of \textit{khyātim} may be treated as \textit{guru}.
  See the same licence applied in non-\textit{anuṣṭubh} verses above,
  e.g., in 5.20a, 6.18b, 7.13b {\rm (}just before \textit{atula}{\rm )}.
 I doubt if \Ed's reading in \textit{pāda} c, \textit{durjayatā} {\rm (}`invincibility'{\rm )} were better than \textit{ūrjayatā} transmitted in
  all the MSS consulted. While \textit{ūrjayatā} is still problematic, it is not inconceivable that it
  stands for \textit{ūrjatā} meaning most probably `being powerful, strength, vigour.' Also, note here
  the stem form noun \textit{saubhāgya} metri causa.
 Note \textit{svargaṃ} as a neuter noun, and the stem form °\textit{bhoga} metri causa in \textit{pāda} d. 
  The lack of sandhi between \textit{eva} and \textit{ananta}° helps restore the metre.
 }}

  \maintext{dānād eva ca śakralokasakalaṃ dānāj janānandanaṃ}%

 \nonanustubhindent \maintext{dānād eva mahīṃ samasta bubhuje samrāḍ mahīmaṇḍale |}%

  \maintext{dānād eva surūpayonisubhagaś candrānano vīkṣyate}%

 \nonanustubhindent \maintext{dānād eva anekasambhavasukhaṃ prāpnoti niḥsaṃśayam }||\thinspace7:28\thinspace||%
\translation{Śakra [conquered] the whole world by donations only. Donations make people happy. Samrāj enjoyed all the land in the world only because of donations. Skanda appears as handsome and fortunate, and has a good family only because of donations. One can reach happiness that lasts countless births only through donations, there is no doubt about that. \blankfootnote{7.28 °\textit{lokasakalaṃ} in \textit{pāda} a is suspect and \Ed's silent emendation {\rm (}°\textit{lokam atulaṃ}{\rm )} is
  not without reason. This line may contain two general statements, the 
  first perhaps saying that by donation even Indra's world can be acquired or reached.
  Nevertheless I suspect that there is a hidden reference to a myth, perhaps
  that of Dadhīca, who gave his bones to Indra to help him defeat Vṛtra. 
  See \VSS\ 17.47:
  
 
  \textit{dadhīciḥ svatanuṃ dattvā vibudhānāṃ varānane}\thinspace | 
 
  \textit{bhuktvā lokān kramāt sarvān śivaloke pratiṣṭhitaḥ}\thinspace ||
  
 
  `Dadhīci gave the gods his own body, O Varānanā. Enjoying all the worlds in due order, 
  he is now living in Śivaloka.'
 
 One could translate \textit{pāda} b as a general statement {\rm (}`A universal monarch\dots{\rm )},
  but again I suspect here a reference to a specific person {\rm (}the son of Citraratha by Ūrṇā?{\rm )} and 
  a specific legend. The perfect form \textit{bubhuje}, and the next \textit{pāda}, at least point to this direction.
 My translation of \textit{pāda} d is also tentative. I take \textit{surūpayonisubhaga} as
  \textit{surūpa-suyoni-subhaga}. Unfortunately, the reference to any specific legend
  escapes me. Perhaps the reference is to Brahmā's boon to Tārakāsura,
  which ultimately was the cause of Skanda's birth.
 }}

\centerline{\maintext{\dbldanda\thinspace iti vṛṣasārasaṃgrahe dānapraśaṃsādhyāyaḥ saptamaḥ\thinspace\dbldanda}}
\translation{Here ends the seventh chapter in the \textit{Vṛṣasārasaṃgraha} called Praise of Donations.}
