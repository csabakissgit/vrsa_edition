
  \chptr{aṣṭādaśamo 'dhyāyaḥ}
\addcontentsline{toc}{section}{Chapter 18}
\fancyhead[CO]{{\footnotesize\textit{Translation of chapter 18}}}%

  \trchptr{Chapter Eighteen}%

  \subchptr{svargān martyam upāgatānāṃ cihnāni}%

  \trsubchptr{Marks of those who return from heaven}%

  \maintext{devy uvāca |}%

  \maintext{bhuktvā tu bhogān suciraṃ yatheṣṭaṃ}%

 \nonanustubhindent \maintext{puṇyakṣayān martyam upāgatānām |}%

  \maintext{cihnāni teṣāṃ kathayasva me 'dya}%

 \nonanustubhindent \maintext{yathākramaṃ karmaphalaṃ viśeṣāt }||\thinspace18:1\thinspace||%
\translation{Devī spoke: Please tell me now about the characteristic marks of those who, after having experienced enjoyable things as they please for a long time, their merits thus having worn away, return to the mortal world, and especially about the fruits of their deeds, one by one. }

  \subchptr{dānāṣṭakam}%

  \trsubchptr{Eight kinds of donation}%

  \maintext{maheśvara uvāca |}%

  \maintext{sadānnadātā kṛpaṇārtidīnāṃ}%

 \nonanustubhindent \maintext{sa varṣakoṭyāyutam īśaloke |}%

  \maintext{bhuktvā ca bhogān samam apsarobhiḥ}%

 \nonanustubhindent \maintext{prakṣīṇapuṇyaḥ punar eti martyam }||\thinspace18:2\thinspace||%
\translation{Maheśvara spoke: He who regularly gives food to the poor and to the ones afflicted by pain will experience enjoyments in Īśaloka together with Apsarases for millions of years, before he returns to the world of mortals, his merits having worn away. \blankfootnote{18.2 Note the variant \textit{bhagavān uvāca} here.
 Note \textit{koṭyāyutam} in \textit{pada} b instead of the expected but unmetrical \textit{koṭyayutam}. 
  Cf.\ 18.4b below.
 }}

  \maintext{jāyanti divyeṣu kuleṣu puṃsaḥ}%

 \nonanustubhindent \maintext{sastrīsamṛddhe bahubhṛtyapūrṇe |}%

  \maintext{gauraśvaratnādidhanākuleṣu}%

 \nonanustubhindent \maintext{rūpojjvalaḥ kāntisamāyutaś ca  }||\thinspace18:3\thinspace||%
\translation{[These] men will be [re-]born in divine families, [later] having a wife and wealth and many servants, into families that are stuffed with wealth that consists of cows, horses, jewels etc., he himself possessing shining beauty and loveliness. \blankfootnote{18.3 Note the change from plural {\rm (}\textit{kuleṣu}{\rm )} to singular {\rm (}°\textit{samṛddhe}, °\textit{pūrṇe}{\rm )}
  in \textit{pāda}s a and b, the slightly irregular plural nominative \textit{puṃsaḥ} in \textit{pāda} a,
  and that \textit{sastrī} might have been meant as a separate word, in the sense of
  \textit{sastrīkaḥ} {\rm (}`married'{\rm )}.
 I take \textit{gaur aśva}° in \textit{pāda} c as if it were part of the compound:
  \textit{go'śva}°. See \NARADAP\ 1.71.77ab for a compound similar to the one here:
  \textit{gajāśvaratharatnaiś ca grāmakṣetradhanādibhiḥ}
 }}

  \maintext{vastraṃ susatkṛtya dvijasya dānāt}%

 \nonanustubhindent \maintext{svargeṣu modanti sa varṣakoṭyaḥ |}%

  \maintext{punaś ca te martyam upāgatāś ca}%

 \nonanustubhindent \maintext{cihnaṃ mahacchrīpadam āpnuvanti }||\thinspace18:4\thinspace||%
\translation{[If one] donates clothes to a Brahmin with utmost respect, he will have fun in the heavens for millions of years. They will return to the world of mortals, and their characteristics mark is that they rise to an extremely glorious rank. \blankfootnote{18.4 Note that \textit{pāda} a can be considered metrical only if 
  \textit{dvi} in \textit{dvijasya} does not make the previous syllable heavy 
  {\rm (}muta cum liquida{\rm )}.
 Note the plural verb \textit{modanti} metri causa and the plural nominative °\textit{koṭyaḥ} in \textit{pāda} b for
  a more standard accusative °\textit{koṭīḥ} {\rm (}from \textit{koṭi} or \textit{koṭī}; cf.\ 18.2b above{\rm )}.
 }}

  \maintext{kūpaprapāpuṣkariṇīpradātā}%

 \nonanustubhindent \maintext{sa lokam āpnoti jaleśvarasya |}%

  \maintext{tataḥ sa tasmāc cyutim āpya lokāt}%

 \nonanustubhindent \maintext{sukhī sutṛpteṣu kuleṣu jāyet }||\thinspace18:5\thinspace||%
\translation{He who donates wells, fountains, or lotus-ponds will reach the world of Jaleśvara [i.e.\ Varuṇa]. Then descending from that world, he will be [re-]born into a very comfortable [well-to-do? happy? tarpaṇa?] family, and will be happy. \blankfootnote{18.5 The phrase \textit{sutṛpteṣu kuleṣu} {\rm (}lit.\ `into satified families'{\rm )} is 
  slightly odd.
 }}

  \maintext{ratnipramāṇād api hemadānāt}%

 \nonanustubhindent \maintext{surendralokaṃ samavāpnuvanti |}%

  \maintext{tasmāc cyuto martyam upāgatānāṃ}%

 \nonanustubhindent \maintext{cihnaṃ samṛddhir dhanadhānyalakṣmyāḥ }||\thinspace18:6\thinspace||%
\translation{By donating as little gold as a cubit[?], people can reach the world of Surendra [i.e.\ Indra]. The characteristic marks of those who descend from there to the world of mortals is prosperity, wealth, crops, and good fortune. \blankfootnote{18.6 \textit{ratni} seems to be too large a measure in this context. Maybe \textit{reṇu} {\rm (}`a grain of dust'{\rm )} was meant?
 I have chosen \textit{lakṣmyāḥ} in \textit{pāda} d against \textit{lakṣyāḥ} as
  a lectio difficilior. It is supposed to stand for a 
  plural nominative.
 }}

  \maintext{adūṣyabhūmīvaravipradānāt}%

 \nonanustubhindent \maintext{sa lokam āpnoti sureśvarasya |}%

  \maintext{bhuktvā tu bhogān cyuta martyaloke}%

 \nonanustubhindent \maintext{cihnaṃ labhed vai viṣayādhipatvam }||\thinspace18:7\thinspace||%
\translation{By donating an excellent piece of land to a Brahmin without corruption[? adūṣita° ?], one will reach the world of Sureśvara [Śiva/Brahmā?]. After experiencing enjoyments, he descends to the world of mortals, And the characteristic mark [will be] that he will obtain the rank of `lord of the land.' \blankfootnote{18.7 Note the stem form \textit{cyuta} in \textit{pāda} c metri causa.
 }}

  \maintext{dvijasya satkṛtya tilapradātā}%

 \nonanustubhindent \maintext{sa lokam āpnoti ca keśavasya |}%

  \maintext{bhraṣṭas tato martyam upāgatas tu}%

 \nonanustubhindent \maintext{cihnaṃ labhed akṣayam arthalābham }||\thinspace18:8\thinspace||%
\translation{He who donates sesame seeds to a Brahmin respectfully will reach the world of Keśava [i.e.\ Viṣṇu]. Then, having fallen and returned to the world of mortals, the characteristic mark [will be] that he will obtain undiminishing acquisition of wealth. }

  \maintext{gavāṃ surūpāṃ vidhivad dvijānāṃ}%

 \nonanustubhindent \maintext{dattvā ca golokam avāpnuvanti |}%

  \maintext{kalpāvasāne samupetya martye}%

 \nonanustubhindent \maintext{cihnaṃ gavāḍhyaṃ śatagoyutaṃ ca }||\thinspace18:9\thinspace||%
\translation{By donating beautiful cows to Brahmins according to rule, people reach Goloka. At the end of the \ae on, they return to the world of mortals. Their characteristic mark will be an abundance of cows, having a hundred cows[?]. \blankfootnote{18.9 It seems that \textit{gavāṃ} is meant to be a singular accusative of \textit{go}.
 }}

  \maintext{svargaṃ gatānāṃ puruṣasya cihnaṃ}%

 \nonanustubhindent \maintext{dhanāḍhyatā śrī sukhabhogalābham |}%

  \maintext{āyuryaśorūpakalatraputraṃ}%

 \nonanustubhindent \maintext{sampadvibhūtikulakīrtim artham }||\thinspace18:10\thinspace||%
\translation{The characteristic marks of those who have been in heaven are: an abundance of wealth, grace, the attainment of happiness and enjoyment, [a long] life, fame, beauty, wife, sons, success, power, family, glory, and riches. \blankfootnote{18.10 Note the discrepancy in grammatical number in \textit{pāda} a.
 Note the seemingly accusative forms °\textit{lābham} and °\textit{kīrtim} {\rm (}for \textit{lābhaḥ}
  and \textit{kīrtir}{\rm )}. The last syllable of \textit{vibhūti} is treated as long.
 }}

  \subchptr{nirayān martyam upāgatānāṃ cihnāni}%

  \trsubchptr{Marks of those who return from hell}%

  \maintext{dānāṣṭakaṃ cottama kīrtitaṃ te}%

 \nonanustubhindent \maintext{cihnaṃ ca lokaṃ ca samāsato me |}%

  \maintext{śṛṇotu devī nirayāgatānāṃ}%

 \nonanustubhindent \maintext{cihnaṃ ca karmaṃ ca vipākatāṃ ca }||\thinspace18:11\thinspace||%
\translation{I have taught you the eight supreme kinds of donation, the characteristic marks, and the [corresponding] worlds in brief. Listen, O Goddess, to the characteristic marks of those who have returned from hell, and to their actions and the fruition [thereof]. \blankfootnote{18.11 For a similar description of the consequences of sins in next lives, see
  \Manu\ 11 and 12, and \YAJNS\ 3.207ff {\rm (}in \mycite{YajnavalkyaOlivelle}{\rm )}, 5 {\rm (}\textit{prāyaścittaprakaraṇa}{\rm )}.
  Note the stem form adjective \textit{uttama} metri causa, in \textit{pāda} a.
 Note \textit{me} for \textit{mayā} in \textit{pāda} b {\rm (}\mycitep{OberliesEpicSkt}{4.1.3 [pp. 102--103]}{\rm )}.
 The slightly odd phrase \textit{śṛṇotu devī}, instead of a vocative with \textit{śṛṇu},
  is metri causa.
 Note the accusative form \textit{karmaṃ}, metri causa, in \textit{pāda} d.
 }}

  \maintext{hatvā ca vipraṃ manasā ca vācā}%

 \nonanustubhindent \maintext{sa yāti pāraṃ nirayasya ghoram |}%

  \maintext{aśītikalpaṃ niraye krameṇa}%

 \nonanustubhindent \maintext{bhuktvā punas tirya śatāyutānām }||\thinspace18:12\thinspace||%
\translation{If one kills a Brahmin, [even if only] mentally or verbally, one goes to the boundaries of terrible hell. Gradually experiencing [his karmas] for eighty \ae on in hell, he will [live] as an animal for millions [of years/lives/\ae ons]. \blankfootnote{18.12 Note the stem form \textit{tirya} in \textit{pāda} d {\rm (}metri causa{\rm )},
  and that the phrase \textit{śatāyutānām} is ambiguous. Perhaps
  \textit{śatāyutābdam} {\rm (}for \textit{śatāyutāny abdāni}{\rm )} or \textit{śatāyutāni janmāni}
  was meant.
 }}

  \maintext{jāyanti te mānuṣa hīnavidyāḥ}%

 \nonanustubhindent \maintext{pratyantavāsāḥ kulavittahīnāḥ |}%

  \maintext{nityaṃ ca tasyākṣayarogapīḍā}%

 \nonanustubhindent \maintext{idaṃ tu cihnaṃ dvijajīvahartuḥ }||\thinspace18:13\thinspace||%
\translation{Those men will be [re-]born as ignorant, will live on the fringes of town, and will lack a good family and wealth. They will always be tormented by incurable diseases. These are the characteristic marks of one who threatens the life of a Brahmin. \blankfootnote{18.13 In \textit{pāda} a, I take \textit{mānuṣa} as a stem form noun {\rm (}metri causa, for {\rm (}\textit{mānuṣā}[\textit{ḥ}]{\rm )}.
 }}

  \maintext{pītvā ca madyaṃ dvija kāmato vā}%

 \nonanustubhindent \maintext{āghrāti gandhaṃ svamanīṣikeṇa |}%

  \maintext{sa yāti ghoraṃ narakam asahyaṃ}%

 \nonanustubhindent \maintext{yāvac ca kalpaṃ daśa atra bhuktvā }||\thinspace18:14\thinspace||%
\translation{If a Brahmin drinks alcohol no doubt intentionally, smells [its] odour on his own accord, he will go to the terrible and unbearable hell for ten \ae ons, experiencing [his karmas] there. \blankfootnote{18.14 I take \textit{dvija} in \textit{pāda} a as a stem form noun {\rm (}for \textit{dvijaḥ}{\rm )}.
  If standard \textit{sandhi} is expected between \textit{pāda}s a and b, then \textit{vā} in \textit{pāda} a
  is to be understood to stand for \textit{vai} {\rm (}`definitely, without a doubt'{\rm )}.
 In \textit{pāda} b, °\textit{manīṣikena} stands for the better attested °\textit{manīṣikayā}.
 Strictly speaking, \textit{pāda} c is unmetrical, the last
  syllable of \textit{narakam} ending in a light syllable.
  Word-ending syllables are often treated as heavy in this text.
 In \textit{pāda} d \textit{atra} probably stands for \textit{tatra {\rm (}narake{\rm )}}. 
  It is not clear why \textit{atra} seemed better to the redactors. Note also
  that the use of the singular with numerals is one of the hallmarks of this text.
 }}

  \maintext{tiryaṃ ca sarvam anubhūya duḥkhaṃ}%

 \nonanustubhindent \maintext{sa kaṣṭakaṣṭena manuṣyajanma |}%

  \maintext{caṇḍālaśaunaśvapacatvam eti}%

 \nonanustubhindent \maintext{śyāmaṃ ca tālu bhavatīha cihnam }||\thinspace18:15\thinspace||%
\translation{Experiencing all the pain of animal existence, he will, with great difficulty, [reach] a human birth. He will go through [states of being] a Caṇḍāla, a butcher, and a dog-cooker. In this case, the characteristic mark is that his palate becomes black. \blankfootnote{18.15 The syntax of \textit{pāda} a is obscure. Either understand \textit{tiryaṃ} as \textit{tiryaś} {\rm (}`being an animal,'
  `in an animal form'{\rm )} or \textit{tiryaṃ} as qualifying \textit{duḥkhaṃ} {\rm (}`the pain of animal existence'{\rm )}.
  The last syllable of \textit{sarvam} in \textit{pāda} a is treated as long.
 One may consider emending °\textit{janma} to °\textit{janmā} in \textit{pāda} b, turning it into a \textit{bahuvrīhi} compound,
  to make it agree with \textit{sa} {\rm (}`he [will become] a human with great difficulty'{\rm )}.
 The two syllables of \textit{tālu} scan as long-long. Note the relevant
  remark in \YAJNS\ 3.210b quoted in the apparatus {\rm (}`'{\rm )}.
 }}

  \maintext{nindanti ye veda {\rm †}sambhūya{\rm †} jihvā}%

 \nonanustubhindent \maintext{yaḥ kūṭasākṣī sa ca khalv alāndhau |}%

  \maintext{suhṛdvadhā mṛtyuśataṃ hi garbhe}%

 \nonanustubhindent \maintext{garhāśanocchiṣṭabhujo bhavanti }||\thinspace18:16\thinspace||%
\translation{Those who despise the Vedas will [be reborn] with their tongues ... He who gives false testimony will [be reborn] blind[?]. [In case of] the murder of a friend, [one will experience] a hundred deaths in the womb. Those who eat forbidden food will eat [only] leftovers [in their next lives]. \blankfootnote{18.16 I take \textit{veda} as a stem form noun in \textit{pāda} a. 
  I suspect that \textit{pāda} a may have contained a reference to \textit{upajihvā}, a disease of the tongue.
  Alternatively, it may speak about one's tongue with a sword.
 Understans \textit{garhāśanocchiṣṭabhujo} in \textit{pāda} d as \textit{garhitāśanā 
  ucchiṣṭabhujo} with double sandhi.
 }}

  \maintext{stainyaṃ tu yaḥ kurvati pāpasattvaṃ}%

 \nonanustubhindent \maintext{te pāpadoṣān narakaṃ vrajanti |}%

  \maintext{manvantarādīny anubhūya duḥkhaṃ}%

 \nonanustubhindent \maintext{punaś ca tiryaṃ śataśo 'nubhūyāt }||\thinspace18:17\thinspace||%
\translation{Those wicked people[?] who steal will, because of this sinful crime, go to hell. Suffering pain for at least[?] a Manu-era, one will again and again, for a hundred times, experience animal existence. \blankfootnote{18.17 Note the discrepancy between \textit{yaḥ kurvati} and \textit{te vrajanti} in \textit{pāda}s a and b, and 
  the corresponding attempt in \msCa\ to correct \textit{yaḥ} to \textit{ye}. One could also emend °\textit{sattvaṃ} to °\textit{sattvaḥ}.
 Note how \Ed\ echoes the reading of \msPaperA\ in \textit{pāda} d.
 }}

  \maintext{mānuṣyajanmeṣu ca duḥkhabhāgī}%

 \nonanustubhindent \maintext{stenatvam āyāti punaś ca mūḍhaḥ |}%

  \maintext{suvarṇacorī kunakhatva cihnaṃ}%

 \nonanustubhindent \maintext{viśīrṇagātro rajatāpahārī }||\thinspace18:18\thinspace||%
\translation{When born as a human, he will suffer, the fool will become a thief again. If one steals gold, the characteristic mark will be that one will have ugly nails. One who steals silver will have broken limbs. \blankfootnote{18.18 \textit{kunakhatva} in \textit{pāda} c is in stem form.
 }}

  \maintext{tāmrāpahārī sphuṭitāgrapāṇir}%

 \nonanustubhindent \maintext{lohāpahārī bhujacheda cihnam |}%

  \maintext{kāṃsāpahārī karabhagna cihnaṃ}%

 \nonanustubhindent \maintext{hṛtvā ca rīti-trapu-sīsakānām }||\thinspace18:19\thinspace||%
\translation{If one steals copper, the fore part of one's hand will be split. If one steals steel, the characteristic mark will be a broken arm. If one steals brass, the characteristic mark will be a broken hand. Stealing bell-metal, tin or lead \blankfootnote{18.19 Note the stem forms °\textit{cheda} and °\textit{bhagna} in \textit{pāda}s b and c.
 Note \textit{kāṃsa} as an alternative form of \textit{kāṃsya}, and °\textit{bhagna} as a stem form in \textit{pāda} c.
 }}

  \maintext{nāsoṣṭhakarṇaśravaṇasya chedaś}%

 \nonanustubhindent \maintext{cihnaṃ nṛṇāṃ vastraharaḥ kucailaḥ |}%

  \maintext{dhānyāpahārī bhavate 'ṅgahīno}%

 \nonanustubhindent \maintext{dīpāpahārī bhavate 'ndha cihnam }||\thinspace18:20\thinspace||%
\translation{will cause clefts in the nose, lips, ears, and problems with hearing[?]. The characteristic mark of one who stole people's clothes is being badly-dressed. Those who steal grain will have missing limbs. If one steals lamps, the characteristic mark is that he will become blind. \blankfootnote{18.20 Note stem form metri causa in \textit{pāda} d {\rm (}\textit{andha}{\rm )}.
 }}

  \maintext{nirvāpahā kāṇa bhaveta cihnaṃ}%

 \nonanustubhindent \maintext{yaḥ strīṃ haret so 'pi jitaḥ striyā syāt |}%

  \maintext{sasyāpahārī bhavate 'nnahīno}%

 \nonanustubhindent \maintext{hṛtvāyudham astrahatatva cihnam }||\thinspace18:21\thinspace||%
\translation{The characteristic mark of one who takes away sacrifical offerings is becoming one-eyed. He who abducts women will himself be overcome by a woman. Somebody who steals corn will lack food. If one steals weapons, the characteristic mark is death by a missile. }

  \maintext{annāpahārī paradattabhoktā}%

 \nonanustubhindent \maintext{hṛtvā tu gāvaḥ sa bhaved daridraḥ |}%

  \maintext{hariṃ haret tad dhariṇā dahanti}%

 \nonanustubhindent \maintext{hṛtvā tu meṣān ajagardabhaṃ vā }||\thinspace18:22\thinspace||%
\translation{One who steals food will live on [food] given by others. One who steals cows will become poor. [If] someone steals a horse, then he will be destroyed by a horse. One who steals sheep, goats, donkeys, \blankfootnote{18.22 Understand \textit{gāvaḥ} in \textit{pāda} c as plural accusative {\rm (}for \textit{gāḥ}{\rm )}.
  {\rm (}\mycitep{OberliesEpicSkt}{2.15 [p.~68]}{\rm )}.
 }}

  \maintext{sa bhārabhṛjjīvya{-}m{-}udāharanti}%

 \nonanustubhindent \maintext{ratnāpahārī anapatyatā ca |}%

  \maintext{chatrāpahārī apavitratā ca}%

 \nonanustubhindent \maintext{hṛtvā ca bījaṃ sa bhaved abījaḥ }||\thinspace18:23\thinspace||%
\translation{will lead[?] a burdened life, they say[?]. One who steals jewels: [the mark is] childlessness. One who steals parasols: [the mark is] impurity. Stealing seeds, one becomes seedless. }

  \maintext{godhūmaśāliyavamudgamāṣān}%

 \nonanustubhindent \maintext{hṛtvā masūraṃ vilayaṃ vrajanti |}%

  \maintext{kāmāturo mātara mātṛputrīṃ}%

 \nonanustubhindent \maintext{mātṛsvasāṃ gacchati mātulānīm }||\thinspace18:24\thinspace||%
\translation{Those who steal wheat, rice, barley, mungo beans, wild beans, or lentils, will die. If somebody, being sick with desire, sexually approaches his mother, his mother's daughter, his mother's sister, or the wife of a maternal uncle, \blankfootnote{18.24 Note \textit{mātara} for \textit{mātaraṃ} metri causa.
 }}

  \maintext{rājāṅganāṃ putrasutāṃ snuṣāṃ ca}%

 \nonanustubhindent \maintext{pravrājinīṃ brāhmaṇim antyajāṃ ca | }%

  \maintext{ajāśvameṣaṃ surabhīsutāṃ ca}%

 \nonanustubhindent \maintext{yat kāmayet teṣu vimūḍhacetāḥ }||\thinspace18:25\thinspace||%
\translation{or if he has sex with a royal concubine, his son's daughter, a daughter-in-law, a female religious mendicant, a Brahmin's wife, or a low-born woman, a goat, horse, sheep, or a cow, with a foolish mind, \blankfootnote{18.25 Note the form \textit{brāhmaṇim} in \textit{pāda} b metri causa.
 }}

  \maintext{sa yāti kṛcchraṃ narakaṃ sughoraṃ}%

 \nonanustubhindent \maintext{sa varṣakoṭīśataśo bhramitvā |}%

  \maintext{tiryaṃ ca bhūyaḥ śataśo vyatītya}%

 \nonanustubhindent \maintext{kaṣṭena vai jāyati mānuṣatvam }||\thinspace18:26\thinspace||%
\translation{he will go to the painful and extremely terrible hell. Wandering [through transmigration] a million times, dying as an animal again and again a hundred times, he will, with great difficulty, be born as a human. }

  \maintext{hīnāṅgatā dīnaśarīratāś ca}%

 \nonanustubhindent \maintext{yo mātṛgāmī sa bhaved aliṅgaḥ |}%

  \maintext{mātṛsvasātalpaga vātaliṅgo}%

 \nonanustubhindent \maintext{liṅgāparodhaḥ sutaputrikāmaḥ }||\thinspace18:27\thinspace||%
\translation{They will lack some limbs and will have a miserable body. He who had sex with his mother will have no penis; one who has sex with his mother's sister will have a damaged[?] penis; he who enjoys his son's daughter will have a non-functional[?] penis. \blankfootnote{18.27 Note \textit{putri} in \textit{pāda} d metri causa for \textit{putrī}.
 }}

  \maintext{snuṣāṃ ca yaḥ sevati raktamehī}%

 \nonanustubhindent \maintext{dauścarmatāṃ ca dvijasundarīṣu |}%

  \maintext{rājāṅganāyāsu ca liṅgacchedaḥ}%

 \nonanustubhindent \maintext{pravrājinīkāmuka mūtrakṛcchram }||\thinspace18:28\thinspace||%
\translation{He who has sex with a daughter-in-law will pass blood with his urine; with a Brahmin's wife: skin disease; with royal concubines: a cut-off penis; longing for a female mendicant: painful discharge of urine; \blankfootnote{18.28 Understand °\textit{āṅganāyāsu} as °\textit{āṅganāsu}?
 Note stem form in \textit{pāda} d.
 }}

  \maintext{savyādhiliṅgaṃ labhate 'ntyajāsu}%

 \nonanustubhindent \maintext{vilīnaliṅgaḥ paśuyonigāmī |}%

  \maintext{jāyanti te mūṣika dhānyacaurī}%

 \nonanustubhindent \maintext{kṣīraṃ hared vāyasatāṃ prayāti }||\thinspace18:29\thinspace||%
\translation{[sex] with low-born women: he will have a penis disease; he who has sex with animals will have his penis disappear. He who steals grain will be born as a rat. If one steals milk, one will become a crow. \blankfootnote{18.29 Note the singular subject with a plural predicate, and the stem
  form noun \textit{mūṣika} in \textit{pāda} c.
 For this and the next verse, compare \MANU\ 12.62 {\rm (}in Olivelle's translation;
  see the Sanskrit in the apparatus; emphasis mine{\rm )}:
  `\textit{By stealing grain, one becomes a rat;} by stealing bronze, a
  ruddy goose; by stealing water, a Plava coot; by stealing honey, a gnat; 
  \textit{by stealing milk, a crow; by stealing sweets, a dog; by stealing ghee, a
  mongoose'.}
 }}

  \maintext{kāṃsāpahārī sa bhavet tu haṃsaḥ}%

 \nonanustubhindent \maintext{śvānatvam āyāti rasāpahārī |}%

  \maintext{hṛtvā ca sūcīṃ tu bhavet sa daṃśaḥ}%

 \nonanustubhindent \maintext{hṛtvā tu sarpir vṛkatāṃ prayāti }||\thinspace18:30\thinspace||%
\translation{He who steals copper will become a goose. He who steals sweet juices will become a dog. By stealing a needle he becomes a gnat. By stealing ghee, he becomes a bull. \blankfootnote{18.30 For my emendation of \textit{haṃsāpahārī} to \textit{kāṃsāpahārī} in \textit{pāda} a,
  see \MANU\ 12.62: \textit{kāṃsyaṃ haṃso...} and 18.19c above. Note how \msM\ is closer to
  to reading \textit{kāṃsā}° than any of the other witnesses.
  Since \textit{nihaṃsaḥ} in the same \textit{pāda} is difficult to interpret,
  and we expect \textit{haṃsaḥ} anyway, I conjectured \textit{tu haṃsaḥ} here.
 }}

  \maintext{māṃsaṃ tu hṛtvā sa bhaveta gṛdhras}%

 \nonanustubhindent \maintext{tailāpahārī khagatāṃ prayāti |}%

  \maintext{guḍaṃ ca hṛtvā guḍikā bhavanti}%

 \nonanustubhindent \maintext{śākāpahārī sa bhaven mayūraḥ }||\thinspace18:31\thinspace||%
\translation{If he steals meat, he will be a vulture. If he steals oil, he will be a bird. If he steals sugar, he will become a flying fox. If he steals vegetables, he will become a peacock. \blankfootnote{18.31 For this verse, see \MANU\ 12.63--65 {\rm (}in Olivelle's translation;
  see the relevant excerpts from the Sanskrit in the apparatus; these in italics here{\rm )}:
  `\textit{by stealing meat, a vulture;} by stealing fat, a Madgu cormorant;
  \textit{by stealing oil, a cockroach;} by stealing salt, a cricket; by stealing curd, a Balāka
  flamingo; by stealing silk, a partridge; by stealing linen, a frog; by stealing cotton
  cloth, a Krauñca crane; by stealing a cow, a monitor lizard; 
  \textit{by stealing molasses, a flying fox;} by stealing fine perfumes, a muskrat; 
  \textit{by stealing leafy vegetables, a peacock;}
  by stealing various kinds of cooked food, a porcupine; by stealing uncooked
  food, a hedgehog'.
 Here in \textit{pāda} c, based on \MANU\ 12.64d {\rm (}\textit{... vāggudo guḍam}{\rm )}, 
  what is expected is \textit{guḍaṃ hṛtvā vāggudā bhavanti}, and \textit{guḍikā} {\rm (}`a ball'{\rm )}
  is out of context. I translate what the original intention may have been.
 }}

  \maintext{hṛtvā paśuṃ paṅgura jāyate ha}%

 \nonanustubhindent \maintext{śvitratvam āyāti suvastrahārī |}%

  \maintext{hṛtvā dukūlaṃ sa ca sārasattvaṃ}%

 \nonanustubhindent \maintext{kṣaumaṃ ca hṛtvā sa ca darduratvam }||\thinspace18:32\thinspace||%
\translation{If someone steals cattle, he will be [re-]born lame. One who steals nice clothes will have white leprosy. If one steals fine cloth {\rm (}\textit{dukūla}{\rm )}, one becomes a crane. If one steals linen, one will become a frog. \blankfootnote{18.32 Note \textit{paṅgura} in \textit{pāda} a in stem form, standing for \textit{paṅgula} or \textit{paṅguka} {\rm (}see \msM{\rm )}.
 }}

  \maintext{aurṇāni vastrāṇy apahṛtya meṣaḥ}%

 \nonanustubhindent \maintext{chucchundarī jāyati gandhahārī |}%

  \maintext{brahmasvam alpam apahṛtya bhoktā}%

 \nonanustubhindent \maintext{sa gṛdhra ucchiṣṭabhujo bhavanti }||\thinspace18:33\thinspace||%
\translation{If one steals woolen clothes, one will become a ram. One who steals perfumes will be [re-]born as a [female?] musk-rat. If one steals and enjoys the property of a Brahmin, [even if it is only] a small amount, one becomes a vulture that eats leftovers. \blankfootnote{18.33 Note how the second syllable of \textit{alpam} is counted as heavy in \textit{pāda} c.
 Note the discrepancy in the use of the singular and the plural in \textit{pāda} d.
 }}

  \maintext{pādena yaḥ sparśayate dvijāṅghriṃ}%

 \nonanustubhindent \maintext{tad vātaraktaṃ caraṇe bhaveta |}%

  \maintext{pādena yaḥ sparśayate ca gāvaḥ}%

 \nonanustubhindent \maintext{sa pādarogān vividhān labheta }||\thinspace18:34\thinspace||%
\translation{[If] one who touches the feet of a Brahmin with his foot, then one will have rheumatism in his feet. He who touches a cow with his feet, will have various kinds of foot-diseases. \blankfootnote{18.34 CHECK Note the use \textit{tad} in \textit{pāda} b.
 Understand \textit{gāvaḥ} in \textit{pāda} c as plural accusative {\rm (}for \textit{gāḥ}{\rm )}.
  {\rm (}\mycitep{OberliesEpicSkt}{2.15 [p.~68]}{\rm )}.
 }}

  \maintext{yo mātaraṃ tāḍayate padena}%

 \nonanustubhindent \maintext{pāde tadīye kṛmayaḥ patanti |}%

  \maintext{padā spṛśed yaḥ pitaraṃ durātmā}%

 \nonanustubhindent \maintext{sūnonnapādaḥ sa bhavet paratra }||\thinspace18:35\thinspace||%
\translation{If someone kicks his mother with his foot, worms will settle in his feet. If a wicked person touches his father with his foot, in a next life {\rm (}\textit{paratra}{\rm )}, he will become ... }

  \maintext{padā spṛśet toyam anādareṇa}%

 \nonanustubhindent \maintext{sa ślīpadī pādayuge bhaveta |}%

  \maintext{pādena yaḥ sparśayate hutāśaṃ}%

 \nonanustubhindent \maintext{sa cāgnipādaḥ satataṃ bhaveta }||\thinspace18:36\thinspace||%
\translation{He who touches water with his foot without paying respect will have elephantiasis in both feet. If someone touches fire with his foot, will always remain `fire-footed.' }

  \maintext{pādena yaś cāryam upaspṛśeta}%

 \nonanustubhindent \maintext{sa pādachedaṃ bahuśo labheta |}%

  \maintext{granthāpahārī sa bhaveta mūkaḥ}%

 \nonanustubhindent \maintext{durgandhavaktraḥ parachidravādī }||\thinspace18:37\thinspace||%
\translation{He who touches his teacher with his foot, will break his foot many times. He who steals a book will become mute. He who talks about others' faults will have a stinking mouth. \blankfootnote{18.37 \textit{cāryam} in \textit{pāda} a most probably stands for \textit{cācāryam}.
 }}

  \maintext{paiśunyavādī sa ca pūtināso}%

 \nonanustubhindent \maintext{nṛ namravaktras tv anṛtāpavādī |}%

  \maintext{pāruṣyavaktā mukhapākarogī}%

 \nonanustubhindent \maintext{asatpralāpī sa ca dantarogaḥ }||\thinspace18:38\thinspace||%
\translation{He who speaks slanderously will have a fetid nose. If a man[?] lies, he will have a curved[? disfigured?] mouth ... He who speaks abusively will have a mouth ill with inflammation. He who spreads false rumours will have tooth-aches. }

  \maintext{tīkṣṇapradāyī sa ca vakranāsaḥ}%

 \nonanustubhindent \maintext{sambhinnavaktā sa ca kaṇṭharogī |}%

  \maintext{kruddhekṣaṇaḥ paśyati yas tu vipraṃ}%

 \nonanustubhindent \maintext{tīvrākṣirogī sa tu jāyate hi }||\thinspace18:39\thinspace||%
\translation{He who is abusive will have a distorted nose. He whose talk is idle will have a sore throat. He who beholds a Brahmin with angry eyes, will be [re-]born with severe eye-diseases. }

  \maintext{pradveṣayālokayate 'tithīn ya}%

 \nonanustubhindent \maintext{utpāṭitākṣiḥ sa bhavet paratra |}%

  \maintext{vairūpyacakṣus tv atisūkṣmacakṣuḥ}%

 \nonanustubhindent \maintext{sa jāyate kekarapiṅgacakṣuḥ }||\thinspace18:40\thinspace||%
\translation{He who looks at his guests with hatred will have his eyes pulled out in his next life, will have ugly eyes, too little eyes, will be squint-eyed, yellow-eyed. }

  \maintext{gartākṣikādīni vipaṇḍulāni}%

 \nonanustubhindent \maintext{netrāmayāny eva ca pāpadoṣāt |}%

  \maintext{śṛṇvanti ye pāpakathāṃ praśastāṃ}%

 \nonanustubhindent \maintext{tān karṇasarpiḥ paripīḍayeta }||\thinspace18:41\thinspace||%
\translation{Eye-diseases such as hollow-eyedness and pale[-eyedness] [will arise] because of this sinful crime. Those who listen to wicked tales that are praised[?] will be tormented by earwax[?]. }

  \maintext{śṛṇoti nindāṃ hariśarvayor yaḥ}%

 \nonanustubhindent \maintext{sa karṇaśūlena tu jīvatīva |}%

  \maintext{mātāpitṝṇāṃ śṛṇute 'pavādaṃ}%

 \nonanustubhindent \maintext{sa karṇaśophena vināśam eti }||\thinspace18:42\thinspace||%
\translation{He who listens to abuse towards Hari or Śarva [i.e.\ Śiva] will live, so to say, with ear-ache. If he listens to abusive words about his parents, he will perish from an ear-tumour. }

  \maintext{śṛṇoti nindāṃ guruviprajāṃ yaḥ}%

 \nonanustubhindent \maintext{sa karṇapūyaṃ sravate saraktam |}%

  \maintext{virūpadāridryakulādhameṣu}%

 \nonanustubhindent \maintext{aniṣṭakarmabhṛtijīvanaṃ ca |}%

  \maintext{akīrtanaṃ darśanavarjanaṃ ca}%

 \nonanustubhindent \maintext{śvapākaḍombādiṣu jāyate saḥ }||\thinspace18:43\thinspace||%
\translation{If he listens to abuse aimed at the guru or Brahmins, he will ooze bloody ear- [puṭaṃ?] ... [The marks will be] ugliness, poverty, [birth] in the lowest families, repulsive work, maintenance, and livelihood, disgrace, loss of eyesight, and he will be born amongst dog-cookers, Ḍombas etc. }

  \maintext{etāni cihnaṃ nirayāgatānāṃ}%

 \nonanustubhindent \maintext{mānuṣyaloke kukṛtasya dṛṣṭam |}%

  \maintext{samāsataḥ kīrtita eva devi}%

 \nonanustubhindent \maintext{yathaiva muktas tv iha karmabhaṅgaḥ }||\thinspace18:44\thinspace||%
\translation{These are the characteristic marks of those sinners who have returned from hell, as seen in the human world. In brief, I have proclaimed, O Devī, how one is liberated in this world, with one's karmas destroyed. [?] }

  \maintext{mātāpitroghatoyā sutaduhitṛvahā bhrātṛgambhīravegā}%

 \nonanustubhindent \maintext{bhāryāvartā vivartā kuṭilagativadhū bāndhavormītaraṅgā |}%

  \maintext{kāmakrodhobhakūlā karimakarajhaṣā grāhakāmā bhayante}%

 \nonanustubhindent \maintext{mṛtyor ākhyārṇave 'smin na śaraṇa vivaśā kāladaṣṭā prayāti }||\thinspace18:45\thinspace||%
\translation{The water of its torrents is mother and father; its flow is son and daughter; its underwater currents are brothers; its revolving whirlpools the wife; its winding currents the daughter-in-laws; its rising waves the relatives; its two banks Desire and Anger; ...: in this [life] that appears as an ocean, there is no refuge from death, [and people] depart helplessly, bitten by Time. }

  \maintext{nityaṃ yena vināśa yāti divasaṃ pañcatvam āpadyate}%

 \nonanustubhindent \maintext{tyaktvā deha vanāntareṣu viṣame śvānaśṛgālākule |}%

  \maintext{bandhuḥ sarva nivartate gatadayā dharmaika tatra sthitaḥ}%

 \nonanustubhindent \maintext{tasmād dharmaparo na cānyasuhṛdaḥ sevet paratrārthinaḥ }||\thinspace18:46\thinspace||%
\translation{[And yet, it is] always [like this:] [when] the day [of] dissolution {\rm (}\textit{pañcatva}{\rm )} comes by which one perishes {\rm (}\textit{vināśaṃ yāti}{\rm )}, abandoning the body [of the deceased person] in a forest, in a rough place filled with dogs and jackals, all the relatives turn back home, with their compassion gone, and only Dharma stays. Therefore one [should] cling on to Dharma and should not serve any other friends if one seeks the other world. \blankfootnote{18.46 In \textit{pāda} a, \textit{vināśa} is a stem form noun.
 Note the stem form noun \textit{deha} in \textit{pāda} b and that this \textit{pāda} is metrical only if we read \textit{śrigālākule}.
 In \textit{pāda} c, understand \textit{bandhuḥ sarva} as \textit{bandhavaḥ sarve} and \textit{dharmaika} as \textit{dharma ekas}.
 I translate \textit{paratrāthinaḥ} in \textit{pāda} d as if it read \textit{paratrārthī}.
 }}
\centerline{\maintext{\dbldanda\thinspace iti vṛṣasārasaṃgrahe pūrvakarmavipākacihnāṣṭādaśamo 'dhyāyaḥ\thinspace\dbldanda}}
\translation{Here ends the eighteenth chapter in the Vṛṣasārasaṃgraha called Marks of the Fruition of Previous Karma.}
