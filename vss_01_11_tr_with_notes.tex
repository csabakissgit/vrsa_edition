\documentclass{article}
\usepackage[utf8x]{inputenx}
\newcommand{\skt}[1]{\textit{#1}}
\newcommand{\danda}{\thinspace$\cal j$ }
\newcommand{\twodanda}{\thinspace$\cal k$ }
\newcommand{\msCa}{{\rm C$_{\scriptscriptstyle 94}$}}
\newcommand{\msCaacorr}{{\rm C$^{\scriptscriptstyle ac}_{\scriptscriptstyle 94}$}}
\newcommand{\msCapcorr}{{\rm C$^{\scriptscriptstyle pc}_{\scriptscriptstyle 94}$}}
\newcommand{\msCb}{{\rm C$_{\scriptscriptstyle 45}$}}
\newcommand{\msCbacorr}{{\rm C$^{\scriptscriptstyle ac}_{\scriptscriptstyle 45}$}}
\newcommand{\msCbpcorr}{{\rm C$^{\scriptscriptstyle pc}_{\scriptscriptstyle 45}$}}
\newcommand{\msCc}{{\rm C$_{\scriptscriptstyle 02}$}}
\newcommand{\msCcacorr}{{\rm C$^{\scriptscriptstyle ac}_{\scriptscriptstyle 02}$}}
\newcommand{\msCcpcorr}{{\rm C$^{\scriptscriptstyle pc}_{\scriptscriptstyle 02}$}}
\newcommand{\msNa}{{\rm K$_{\scriptscriptstyle 82}$}}  
\newcommand{\msNaacorr}{{\rm K$^{\scriptscriptstyle ac}_{\scriptscriptstyle 82}$}}
\newcommand{\msNapcorr}{{\rm K$^{\scriptscriptstyle pc}_{\scriptscriptstyle 82}$}}
\newcommand{\msNb}{{\rm K$_{\scriptscriptstyle 10}$}}  
\newcommand{\msNbacorr}{{\rm K$^{\scriptscriptstyle ac}_{\scriptscriptstyle 10}$}}
\newcommand{\msNbpcorr}{{\rm K$^{\scriptscriptstyle pc}_{\scriptscriptstyle 10}$}}
\newcommand{\msNc}{{\rm K$_{\scriptscriptstyle 7}$}}  
\newcommand{\msNcacorr}{{\rm K$^{\scriptscriptstyle ac}_{\scriptscriptstyle 7}$}}
\newcommand{\msNcpcorr}{{\rm K$^{\scriptscriptstyle pc}_{\scriptscriptstyle 7}$}}
\newcommand\msBod{{\rm B}}
\newcommand\msBodac{{\rm B}$^{\scriptscriptstyle ac}$}
\newcommand\msBodpc{{\rm B}$^{\scriptscriptstyle pc}$}
\newcommand\msP{{\rm P}}
\newcommand\msPac{{\rm P}$^{\scriptscriptstyle ac}$}
\newcommand\msPpc{{\rm P}$^{\scriptscriptstyle pc}$}
\newcommand\Ed{{\rm E$^{\scriptscriptstyle N}$}}
\newcommand{\msCaNa}{{\normalfont C$_\textrm{a}$N$_\textrm{a}$}}
\newcommand{\msCbNa}{{\normalfont C$_\textrm{b}$N$_\textrm{a}$}}
\newcommand{\msCcNa}{{\normalfont C$_\textrm{c}$N$_\textrm{a}$}}
\newcommand{\msCabNa}{{\normalfont C$_\textrm{a}$C$_\textrm{b}$N$_\textrm{a}$}}
\newcommand{\msCbcNa}{{\normalfont C$_\textrm{b}$C$_\textrm{c}$N$_\textrm{a}$}}
\newcommand{\msCabcNa}{{\normalfont C$_\textrm{a}$C$_\textrm{b}$C$_\textrm{c}$N$_\textrm{a}$}}
\newcommand{\msCab}{{\normalfont C$_\textrm{a}$C$_\textrm{b}$}}
\newcommand{\msCac}{{\normalfont C$_\textrm{a}$C$_\textrm{c}$}}
\newcommand{\msCbc}{{\normalfont C$_\textrm{b}$C$_\textrm{c}$}}
\newcommand{\msCabc}{{\normalfont C$_\textrm{a}$C$_\textrm{b}$$_\textrm{c}$}}
\newcommand{\mssCaCbCc}{{\normalfont C}}
\newcommand{\msL}{{\rm L}\allowbreak}
\newcommand{\msLacorr}{{\rm L}$^{\scriptscriptstyle ac}$\allowbreak}
\newcommand{\msLpcorr}{{\rm L}$^{\scriptscriptstyle pc}$\allowbreak}
\newcommand{\msM}{{\rm L}\allowbreak}
\newcommand{\Cod}{\textit{Cod.}}
\newcommand{\Codd}{$\Sigma$}

\newcommand{\msA}{{\rm A}}
\newcommand{\msB}{{\rm A}}
\newcommand{\msC}{{\rm A}}
\newcommand{\msD}{{\rm A}}
\newcommand{\msE}{{\rm A}}
\newcommand{\msF}{{\rm A}}

\newcommand{\KKT}{KKT}
\newcommand{\LP}{{\englishfont Li\.nPu}}
\newcommand{\msNKeightytwo}{{\englishfont N$^{\scriptscriptstyle K}_{\scriptscriptstyle 82}$}\allowbreak}
\newcommand{\msNKeightytwoac}{{\englishfont N$^{\scriptscriptstyle Kac}_{\scriptscriptstyle 82}$}\allowbreak}
\newcommand{\msNKeightytwopc}{{\englishfont N$^{\scriptscriptstyle Kpc}_{\scriptscriptstyle 82}$}\allowbreak}

\newcommand{\msNKtwelve}{{\englishfont N$^{\scriptscriptstyle K}_{\scriptscriptstyle 12}$}\allowbreak}
\newcommand{\msNKtwelveac}{{\englishfont N$^{\scriptscriptstyle Kac}_{\scriptscriptstyle 12}$}\allowbreak}
\newcommand{\msNKtwelvepc}{{\englishfont N$^{\scriptscriptstyle Kpc}_{\scriptscriptstyle 12}$}\allowbreak}

\newcommand{\msNKtwelveb}{{\englishfont N$^{\scriptscriptstyle K}_{\scriptscriptstyle 12b}$}\allowbreak}
\newcommand{\msNKtwelvebac}{{\englishfont N$^{\scriptscriptstyle Kac}_{\scriptscriptstyle 12b}$}\allowbreak}
\newcommand{\msNKtwelvebpc}{{\englishfont N$^{\scriptscriptstyle Kpc}_{\scriptscriptstyle 12b}$}\allowbreak}

\newcommand{\msNCfortyfive}{{\englishfont N$^{\scriptscriptstyle C}_{\scriptscriptstyle 45}$}\allowbreak}
\newcommand{\msNCfortyfiveac}{{\englishfont N$^{\scriptscriptstyle Cac}_{\scriptscriptstyle 45}$}\allowbreak}
\newcommand{\msNCfortyfivepc}{{\englishfont N$^{\scriptscriptstyle Cpc}_{\scriptscriptstyle 45}$}\allowbreak}

\newcommand{\msNCninetyfour}{{\englishfont N$^{\scriptscriptstyle C}_{\scriptscriptstyle 94}$}\allowbreak}
\newcommand{\msNCninetyfourac}{{\englishfont N$^{\scriptscriptstyle Cac}_{\scriptscriptstyle 94}$}\allowbreak}
\newcommand{\msNCninetyfourpc}{{\englishfont N$^{\scriptscriptstyle Cpc}_{\scriptscriptstyle 94}$}\allowbreak}
\newcommand{\msNCninetyfouracorr}{{\englishfont N$^{\scriptscriptstyle Cac}_{\scriptscriptstyle 94}$}\allowbreak}
\newcommand{\msNCninetyfourpcorr}{{\englishfont N$^{\scriptscriptstyle Cpc}_{\scriptscriptstyle 94}$}\allowbreak}

\newcommand{\msNKtwentyeight}{{\englishfont N$^{\scriptscriptstyle K}_{\scriptscriptstyle 28}$}\allowbreak}  
\newcommand{\msNKtwentyeightac}{{\englishfont N$^{\scriptscriptstyle Kac}_{\scriptscriptstyle 28}$}\allowbreak}  
\newcommand{\msNKtwentyeightpc}{{\englishfont N$^{\scriptscriptstyle Kpc}_{\scriptscriptstyle 28}$}\allowbreak}  

\newcommand{\msNKoseventyseven}{{\englishfont N$^{\scriptscriptstyle Ko}_{\scriptscriptstyle 77}$}\allowbreak}  
\newcommand{\msNKoseventysevenac}{{\englishfont N$^{\scriptscriptstyle Koac}_{\scriptscriptstyle 77}$}\allowbreak}  
\newcommand{\msNKoseventysevenpc}{{\englishfont N$^{\scriptscriptstyle Kopc}_{\scriptscriptstyle 77}$}\allowbreak}  





% SDHS10
\newcommand{\msGa}{{\englishfont G$^{\scriptscriptstyle Ki}$}\allowbreak}
\newcommand{\msGaac}{{\englishfont G$^{\scriptscriptstyle Kiac}$}\allowbreak}
\newcommand{\msGapc}{{\englishfont G$^{\scriptscriptstyle Kipc}$}\allowbreak}



\begin{document}
\begin{center}{\Huge \textbf{Vṛṣasārasaṃgraha}}\\ {\Large (translation)}\bigskip\\ {\large\today}\end{center}
\vfill\pagebreak\begin{center}{\large\textbf{  Chapter One 
}}\end{center}


\textbf{1.1}%
\  Having bowed to [Him] whose boundaries are limitless,%
\              who has no beginning, no middle part and no end,%
\              [to Him] who is very subtle and who is the unmanifest and fine essence of the world,%
\ [to Him] who is respected as the foremost by Hari, Indra, Brahmā and the other [gods],%
\              I shall recite [the work called] `A Compendium on the Essence of the Bull [of Dharma]'.%
\footnote{\skt{Pāda} a is reminiscent of, among other famous passages, Bhagavadgītā 11.19:                        
                \skt{anādimadhyāntam anantavīryam 
                anantabāhuṃ śaśisūryanetram / 
                paśyāmi tvāṃ dīptahutāśavaktraṃ 
                svatejasā viśvam idaṃ tapantam //} 
              This faint reference to the Bhagavadgītā seems proper at the              beginning of a work that claims to deliver a teaching               based on, but also to surpass, the Mahābhārata (see following verses).                See also e.g. Kūrmapurāṇa 1.11.237:
                        \skt{rūpaṃ tavośeṣakalāvihīnam [tavā? CHECK]
                        agocaraṃ nirmalam ekarūpam / 
                        anādimadhyāntam anantām [anantam? CHECK] ādyaṃ  
                        namāmi satyaṃ tamasaḥ parastāt} // 
                To say that a god has no beginning and no end in a temporal or spacial                sense is natural (\skt{anādi°...°antam}), but to have no `middle part'                (\skt{°madhya°}) in these senses is slightly less so.                Thus the rather commonly occuring phrase \skt{anādimadhyāntam} is probably not                 much more than a fixed expression meaning `endless and/or eternal'.                As to which god this stanza is referring to, it may be Śiva,                his name not being listed among those who treat him as chief god,                but the phrasing of the verse is vague enough to keep the question somewhat open:                the impersonal Brahman might be another option, even more so if we                look at 1.9--10, two verses nearby verses discussing \skt{brahmavidyā}.                                

                In \skt{pāda} b \skt{jagat-susāraṃ} is most probably not                 to be interpreted as \skt{jagatsu sāraṃ}.                

                Strictly speaking, \skt{pāda} c is unmetrical, but it is better to                 simply acknowledge here the phenomenon of `muta cum liquida', namely                that syllables followed by consonant clusters such as                 \skt{ra, bra, hra, kra, śra, śya, śva, sva, dva} can be treated as short.                Thus \skt{harīndrabrahmā°} can be treated as a regular beginning                of an \skt{upajāti} (. - . - -), the syllable                 \skt{bra} not turning the previous syllable long.                

                The reading \skt{āsamagraṃ} in \skt{pāda} c is difficult to interpret.                 The most tempting of all the possible corrections I have considered                (\skt{arcyam/arhyam/arghyam/īḍyam agraṃ})                seemed to be \skt{āptam agraṃ}, meaning `appointed/received/respected                [by Hari, Indra, Brahmā etc.] as the foremost one'.                The fact that the \skt{akṣara}s \skt{āsam} and \skt{āptam} look similar                in most of the scripts used in our manuscripts supports this                conjecture.                

                Note how we could percieve the end of \skt{pāda}s a and b,                 as well as \skt{pāda}s c and d as rhymes.                

                Is pāda d hypermetrical? It is actually a \skt{vaṃśastha} (\skt{triṣṭubh - jagatī} change).                See Apte App. A p. 4.                 }%


\textbf{1.2}%
\ Having listened to the Bhāratasaṃhitā [i.e. the Mahābhārata], the supreme book of%
\               a hundred thousand [verses], a thousand chapters (\skt{adhyāya})%
\ with all its hundred sections (\skt{parvan}),%
\footnote{The dialouge of Janamejaya and Vaiśampāyana make up the outermost layer of the VSS                 (except for the introductory stanzas 1.1-3), which mostly contains                 general \skt{dharmaśāstric} material.                

                The hundred \skt{parvan}s of the Mahābhārata are listed in MBh 1.2.33--70. }%


\textbf{1.3}%
\ Janamejaya remained unsatisfied and what he asked Vaiśampāyana%
\footnote{For a similar unsatisfaction or dissatisfaction with previous                 teachings, see Niśvāsa mūla 1.9:                

                <skt>vedāntaṃ viditaṃ deva sāṃkhyaṃ vai pañcaviṃśakam \danda                     na ca tṛptiṃ gamiṣyāmo hy ṛte śaivād anugrahāt \twodanda</skt>                

                 and Śivadharmaśāstra... CHECK. }%
\ in the past, listen to that unweariedly.%
\footnote{Vaiśampāyana, a ṛṣi, the disciple of Vyāsa, recited the Mahābhārata at the snake sacrifice of                 Janamejaya. CHECK SOURCE                 

                Note how we are forced to emend \skt{pāda} c to contain a stem form proper noun (\skt{janamejaya})                to maintain the metre, and note how the manuscripts struggle with this \skt{pāda}.         }%


\textbf{1.4}%
\ Janamejaya spoke:%
\ O venerable sir, O knower of the entire Dharma, O you who are well-versed in all the%
\                 sciences (\skt{śāstra})!%
\ Is there a supreme and secret Dharma which liberates [us] from the ocean of%
\                         mundane existence (\skt{saṃsāra})?%
\footnote{Note \skt{dharma} as a neuter noun in \skt{pāda} c and in the next verse. }%


\textbf{1.5}%
\ Teach me the Dharma that emerged from [Vyāsa] Dvaipāyana's mouth, O best of Brahmins.%
\footnote{The majority of the MSS consulted include a \skt{vā} in \skt{pāda} b,                 distinguishing between the `secret Dharma' mentioned in 1.4c and                the one taught by Vyāsa. This may or may not be the better reading.                I decided to follow MS \msCb\ because I suspected that the two Dharmas                hinted at are the same. }%
\ Help me find satisfaction at all cost, O great ascetic!%


\textbf{1.6}%
\ Vaiśampāyana spoke:%
\ Listen with great attention, O king, to this unsurpassed narration of Dharma.%
\ Hear the secret Dharma that I received by Vyāsa's favour.%


\textbf{1.7}%
\ --\textbf{1.8} Viṣṇu, the great Lord, assuming the form of a Brahmin, wanted to%
\                  test the one who performed nonmaterial sacrifices,%
\                  the one who focused on his austerities and observances,%
\                  the one whose conduct was virtuous and pure, and who was intent on%
\                  compassion towards all living beings, and therefore he humbly asked him a question.%
\footnote{Note the odd syntax here: \skt{viṣṇunā... dvijarūpadharo bhūtvā papraccha}.               The agent of the active verb is in the instrumental case. }%


\textbf{1.8}%


\textbf{1.9}%
\ [Vigatarāga spoke:]%
\ How is the knowledge of Brahman to be understood if [this knowledge] is devoid of%
\                 [definitions of the] form and colour [of Brahman]?%
\ [And] the syllable that is devoid of vowels and consonants:%
\                                 is that [its] highest [form]?%
\footnote{The translation of this verse, and the reconstruction and interpretation                        of \skt{pāda} d, which is echoed in 1.10d, is slightly tentative. }%


\textbf{1.10}%
\ Anarthayajña spoke:%
\ That syllable is not to be pronounced, is unquestionable, non-dividable, consistent,%
\ spotless, all-pervading and subtle: what could be higher than that?%
\footnote{I interpret \skt{pāda} d, which is an echo of 1.9d, tentitvely as a compound,                slightly differently from the way I did above. }%


\textbf{1.11}%
\ Vigatarāga spoke:%
\ When the body disintegrates in the ground, in water, in fire or [is torn apart] by jackals%
\                         and other [animals],%
\footnote{The word \skt{°śivā°} in \skt{pāda} b is slightly suspect, and could be the result                of metathesis, from \skt{°viṣā°} (`by poison'). Nevertheless,                 jackals seems appropriate in this context, for they                 are commonly associated with human corpses, death and the cremation ground.                (see e.g. Ohnuma 2019)                 (Reiko Ohnuma 2019 = Ohnuma, R. The Heretical, Heterodox Howl:                         Jackals in Pāli Buddhist Literature. Religions 2019, 10, 221.) }%
\ how is the supportless and spotless soul led [to the netherworld] by Yama's messengers?%


\textbf{1.12}%
\ How is it bound by the nooses of death/time? And if it is bodiless, how can it move?%
\footnote{The word \skt{kāla} has, as usual, a double meaning in this verse: a \skt{kālapāśa}                        is both Yama's noose, and also the limitation caused by time,                         as becomes clear at the discussion                        on the different time units in verses 1.11--31. }%
\ And how does the [soul of a] virtuous [person] (\skt{bahudharmakṛt})%
\                      reach heaven if it has no body?%
\ Teach me about this doubtful matter (\skt{saṃśaya}) that I am raising (\skt{me}).%
\                                         I want to know the truth about it.%


\textbf{1.13}%
\ Anarthayajña spoke:%
\ You are asking me about an extremely doubtful%
\                         and problematic matter, O great Brahmin.%
\ It is difficult to understand by humans, and [even] by gods (\skt{deva}),%
\                         demons (\skt{dānava}) and serpents (\skt{pannaga}).%


\textbf{1.14}%
\ The cause of both the birth and death of the body is karma.%
\ Good and bad deeds are called the two nooses.%


\textbf{1.15}%
\ [Man] goes to hell or heaven accordingly.%
\ Happiness and suffering, both arising from karma, are to be experienced by the body.%


\textbf{1.16}%
\ O great Brahmin, the body is produced for humans for his reason.%
\ Now learn about that which they call the noose of time,%
\                         I shall teach you, O you of great observances.%


\textbf{1.17}%
\ [If] you don't know anything, how could you start%
\                                 your investigation, O Brahmin?%
\footnote{The variant \skt{jijñāsyasi} seems to be the lectio difficilior as opposed to                        \skt{vijñāsyasi}, but the latter could also work fine here. }%
\ O Brahmin king, you should know the noose of time in its entirety.%


\textbf{1.18}%
\ Learn about time which is divided into digits (\skt{kalā}), [i.e. about]%
\                         the division[s] (\skt{kalā}) of the entity [called] Time (\skt{kālatattva}).%
\ Two atomic units of time (\skt{truṭi}) is one twinkling (\skt{nimeṣa}).%
\                      One digit (\skt{kalā}) is twice a twinkling.%
\footnote{1.18d and 1.19a are problematic in the light of 1.19b, which                         redefines \skt{kalā} in harmony with the traditional                        interpretaion, see e.g. Arthaśāstra 2.20.33: \skt{trimśatkāṣṭhāḥ kalāḥ}. }%


\textbf{1.19}%
\ Two digits (\skt{kalā}) form one bit (3.2 seconds; \skt{kāṣṭhā}).%
\                  Thirty bits (\skt{kāṣṭhā}) is one digit (1.6 minutes; \skt{kalā}).%
\ Thirty digits (\skt{kalā}) make up one section (48 minutes; \skt{muhūrta})%
\                                 according to mankind, O great Brahmin.%


\textbf{1.20}%
\ Thirty sections (\skt{muhūrta}) are known to the%
\                         wise as night and day [i.e. a full day].%
\ Thirty days and nights are taught by the wise ones to be one month.%


\textbf{1.21}%
\ One year is twelve months [according to] people who know the entity of time.%
\ The time span of three hundred thousand plus sixty thousand years%


\textbf{1.22}%
\ by human standards is said to be the Kali era.%
\ The Dvāpara era is known to be twice as long as the Kali era.%
\footnote{Note the stem form noun \skt{yuga}. }%


\textbf{1.23}%
\ The Tretā era is thrice [as long], the Kṛta era four [times as long as the Kali].%
\ This is [how to add up] the number[s] related to the Four Yugas [= a \skt{mahāyuga}].%
\                 Taking [this length of four \skt{yuga}s] seventy-one [times],%
\footnote{The element \skt{°yugā°} seems to stand for °yuga° metri causa.                If \skt{°yugā} and \skt{saṃkhyā} are to be separated, \skt{eṣā} becomes                 problematic to interpret. }%


\textbf{1.24}%
\ the knowledge about one time-span of Manu is being taught briefly%
\              [i.e.\ 71 four-fold \skt{mahāyuga}s make up a \skt{manvantara}].%
\footnote{See 21.34ff. }%
\ One Kalpa is fourteen \skt{manvantara}s in total.%


\textbf{1.25}%
\ Brahmā's day is made up of ten thousand Kalpas.%
\ [Brahmā's] night is of the same [length] according to the wise who know the truth.%


\textbf{1.26}%
\ When [Brahmā's] night falls, the whole moving and unmoving universe dissolves.%
\ And when [his] daylight comes, the moving and unmoving [universe] is born.%
\footnote{The plural form \skt{pralīyante} in \skt{pāda} a is metri causa for \skt{pralīyate},                perhaps also influencing \skt{utpadyante} (for \skt{utpadyate}) in \skt{pāda} d,                which in turn is used here to avoid an iambic pattern (- - . - . - . -). }%


\textbf{1.27}%
\ A \skt{para} times \skt{parārdha} [number of, i.e.%
\                two hundred quadrillion times a hundred quadrillion] \skt{kalpas}%
\                 have passed [so far], O great Brahmin.%
\ Bhṛgu and the other sages say that the future is the same [time span].%
\footnote{Note the peculiar compound \skt{bhṛgu-r-ādi-maharṣayaḥ}. }%


\textbf{1.28}%
\ Just as the sun, the planets, the stars and the moon are percieved by us as wandering around,%
\ the wheel of time (\skt{kālacakra}) keeps spinning and we never experience its halting.%


\textbf{1.29}%
\ Time creates living beings and time destroys them again.%
\ Everything is under the control of time. There is nothing that can bring time under control.%


\textbf{1.30}%
\ Fourteen \skt{parārdha}s is [the number of] the kings of the gods [i.e. Indras?], O Brahmin,%
\ who passed by over time, for time is difficult to overcome.%
\footnote{Note that \skt{samatītāni} (neuter) most probably picks up \skt{devarājāḥ}                (masculine) in this verse, or rather \skt{devarājā} stands for                \skt{devarājānāṃ} and \skt{samatītāni} picks up \skt{°parārdhāni}. }%


\textbf{1.31}%
\ Time is [manifest] as a great yogin,%
\                         as Brahmā, Viṣṇu and supreme Śiva,%
\ it is beginningless and endless, it is the creator,%
\                                 the great soul. Pay homage [to Time].%


\textbf{1.32}%
\ Vigatarāga spoke:%
\ I have just heard [the term] `wheel of time' (\skt{kālacakra}) uttered from [your] lotus mouth,%
\ as well as \skt{parārdha} and \skt{para}. You have made these things appear as%
\                              exciting, as things to hear.%
\footnote{The reading of all manuscripts consulted, \skt{vinisṛtam},                 may be considered metrical if we interpret it, loosely, as \skt{vinisritam}.        

                \skt{Pāda} d is suspicious and my translation is tentative. }%


\textbf{1.33}%
\ Anarthayajña spoke:%
\ One, ten, a hundred, a thousand, and ten thousand (\skt{ayuta}),%
\ a hundred thousand (\skt{prayuta}), a million (\skt{niyuta}), ten millions (\skt{koṭi}),%
\              a hundred millions (\skt{arbuda}), and a billion (\skt{vṛnda}, 10${\uparrow}$9),%


\textbf{1.34}%
\ ten billion (\skt{kharva}), a hundred billion (\skt{nikharva}), one trillion (\skt{śaṅku}, 10${\uparrow}$12),%
\              and ten trillion (\skt{padma}),%
\ a hundred trillion (\skt{samudra}), one quadrillion (\skt{madhya} 10${\uparrow}$15), ten quadrillion (\skt{[an]anta}),%
\                          a hundred quadrillion (\skt{parārdha}), and two hundred quadrillion (\skt{para}).%
\footnote{For \skt{anta} meaning \skt{ananta}, see 1.58cd-59ab. }%


\textbf{1.35}%
\ All should be known as powers of ten up to \skt{parārdha}.%
\ The number corresponding to \skt{para} is double the \skt{parārdha}.%


\textbf{1.36}%
\ There is no higher number than \skt{para}. This is my conviction,%
\ which is based on readings of the Purāṇas and the Vedas%
\                 and [which I have now] taught [to you], O great Brahmin.%


\textbf{1.37}%
\ Vigatarāga spoke:%
\ How many eggs of Brahmā are there? And are its measurements available anywhere?%
\footnote{The word \skt{prāpitaṃ} is a conjecture for \skt{cāpitaṃ}, which I find unintelligible.                 Another possibility could be \skt{jñāpitaṃ}. }%
\ From how many finger's breadths high does the sun heat the earth?%
\footnote{The purport of \skt{pāda}s c and d is slightly obscure to me. }%


\textbf{1.38}%
\ Anarthayajña spoke:%
\ How could I enumerate all the eggs of Brahmā, O Brahmin?%
\footnote{One would expect \skt{brahmāṇḍāni} in \skt{pāda} a instead of \skt{brahmāṇḍānāṃ},                but we should probably understand \skt{brahmāṇḍānāṃ viśeṣān prasaṃkhyātuṃ...} }%
\ Even the gods don't know [all the details], not to mention mortals.%


\textbf{1.39}%
\ I shall teach [these details to you] one by one, as far as I can, O great Brahmin,%
\ in the manner in which Brahmā taught Mātariśvan in the past, truthfully.%
\footnote{Note that in \skt{pāda} d \skt{mātariśvan} stands for the accusative \skt{mātariśvānaṃ} or                        the dative \skt{mātariśvane} or the genitive \skt{mātariśvanaḥ}.                        The claim that Brahmā taught Mātariśvan is confirmed in 1.64cd,                        again using the nominative for the accusative, dative or genitive, and                        also e.g. in Brahmāṇḍapurāṇa 3.4.58cd. }%


\textbf{1.40}%
\ The ten names [of cosmic rulers/worlds] associated with each of the%
\                 eight directions in Brahmā's Egg, inside Śiva's Egg, [...], are being taught now, listen.%
\footnote{The cruxed \skt{pāda} may have read \skt{sarveṣām eva pūjitāḥ} originally                                         (`They are worshipped by all').        

        In \skt{pāda} c, understand \skt{diśāṣṭānāṃ} as \skt{diśām aṣṭānāṃ} or \skt{digaṣṭakānāṃ} }%


\textbf{1.41}%
\ [1] Saha, [2] Asaha, [3] Sahas, [4] Sahya, [5] Visaha, [6] Saṃhata, [7] Asabhā,%
\footnote{I chose to supply an \skt{avagraha} before \skt{sabhā} only because all the sources                 consulted read \skt{saṃhato} as the previous word, making the \skt{sandhi}                \skt{o-s} suspicious.                 Note that many of the names here and in the following verses are,                in the absence of any parallel passage, rather insecure.                What is clear here is that the names evoke the name Sahasrākṣa, one of the appellations of                Indra, the guadrian of the eastern direction. }%
\ [8] Prasaha, [9] Aprasaha, [10] Sānu: [these are] the ten Leaders in the East.%


\textbf{1.42}%
\ [1] Prabhāsa, [2] Bhāsana, [3] Bhānu, [4] Pradyota, [5] Dyutima, [6] Dyuti,%
\ [7] Dīptatejas, [8] Tejas, [9] Tejā, [10] Tejavaho: [these are] the ten%


\textbf{1.43}%
\ [leaders] in the direction of Agni [SE]. Now listen to [the names for]%
\                 the direction of Yama [S], O Brahmin.%
\ [1] Yama, [2] Yamunā, [3] Yāma, [4] Saṃyama, [5] Yamuna, [6] Ayama,%


\textbf{1.44}%
\ [7] Saṃyana, [8] Yamanoyāna, [9] Yaniyugmā, [10] Yanoyana.%
\footnote{I have choosen the variant \skt{saṃyano} in \skt{pāda} c only to avoid the repetition of                        the name \skt{saṃyama}, and the variant \skt{yanoyanaḥ} because I suspect that                        most of the names here should begin with \skt{ya}. All the name forms                        in this verse are to be taken as tentative. The only                         guiding light is the presence of \skt{ya}, reinforcing their connection with Yama. }%


\textbf{1.45}%
\ [1] Nagaja, [2] Naganā, [3] Nanda, [4] Nagara, [5] Naga, [6] Nandana,%
\                 [7] Nagarbha, [8] Gahana, [9] Guhyo, [10] Gūḍhaja: [these are] the ten associated with%
\                         [the South-West].%
\footnote{Note that the reconstruction of these names are tentative. What is clear here is that the                initials should be \skt{na} and \skt{ga}, probably suggesting a connection with \skt{nāga}s. }%
\ I shall teach you the [names] in Varuṇa's direction [in the west]. Listen, O Brahmin, learn from me.%


\textbf{1.46}%
\ [1] Babhra, [2] Setu, [3] Bhava, [4] Udbhadra, [5] Prabhava, [6] Udbhava, [7] Bhājana,%
\ [8] Bharaṇa, [9] Bhuvana, and [10] Bhartṛ: these ten dwell in Varuṇa's direction [in the west].%


\textbf{1.47}%
\ [1] Nṛgarbha, [2] Asuragarbha, [3] Devagarbha, [4] Mahīdhara,%
\ [5] Vṛṣabha, [6] Vṛṣagarbha, [7] Vṛṣāṅka, [8] Vṛṣabhadhvaja,%


\textbf{1.48}%
\ and [9] Vṛsaja and [10] Vṛṣanandana: these are to be known properly%
\ as the ten leaders in Vāyu's direction [in the north-west], as I taught them, O Brahmin.%


\textbf{1.49}%
\ [1] Sulabha, [2] Sumana, [3] Saumya, [4] Supraja, [5] Sutanu, [6] Śiva,%
\ [7] Sata, [8] Satya, [9] Laya, [10] Śambhu: [these are] the ten leaders in the north.%


\textbf{1.50}%
\ [1] Indu, [2] Bindu, [3] Bhuva, [4] Vajra, [5] Varada, [6] Vara, [7] Varṣaṇa,%
\ [8] Ilana, [9] Valina, [10] Brahmā: [these are] the ten leaders in the Iśāna direction [in the north-east].%


\textbf{1.51}%
\ [1] Apara, [2] Vimala, [3] Moha, [4] Nirmala, [5] Mana, [6] Mohana,%
\ [7] Akṣaya, [8] Avyaya, [9] Viṣṇu, [10] Varada: [these are] the ten [leaders] in the centre.%


\textbf{1.52}%
\ Each of the ten deities[?] has a retinue of a hundred [deities].%
\ Each one in [these groups of] a hundred [deities] is surrounded by a thousand.%


\textbf{1.53}%
\ Each one in these [groups of] a thousand [deities] is surrounded by ten thousand%
\                                         [deities].%
\ The ten thousand by a multitude of a hundred thousand.%
\                  The hundred thousand is surrounded by a million,%
\footnote{We are forced to follow \Ed's readings here to make sense of this passage.              Note that \skt{vṛnda} is not a number here. Elsewhere in this chapter it is the word that                                        signifies `a billion'. }%


\textbf{1.54}%
\ [that is] each one has a retinue of a million [deities] (\skt{niyuta}).%
\ [Then] each [of those] is surrounded by ten million [deities] (\skt{koṭi}),%
\                 [they] by a hundred million (\skt{daśakoṭi} = \skt{arbuda}).%


\textbf{1.55}%
\ Each one of the hundred million (\skt{daśakoṭi} = \skt{arbuda}) is surrounded by a billion (\skt{vṛnda}) bhṛta???%
\ Each of those billion (\skt{vṛnda}) is surrounded by ten billion (\skt{kharva}) [deities].%


\textbf{1.56}%
\ Each of those ten billion (\skt{kharva}) is surrounded by a hundred billion (\skt{daśakharva} = \skt{nikharva}).%
\ Each of those hundred billion (\skt{daśakharva} = \skt{nikharva}) is surrounded by one trillion (\skt{śaṅku}) [deities].%


\textbf{1.57}%
\ Each of those one trillion (\skt{śaṅku}) is surrounded be ten trillion (\skt{padma}).%
\footnote{Note \skt{śaṅkubhiḥ pṛthag...}: it stands for \skt{śaṅkūṣu pṛthag...} (instrumental for locative). }%
\ Each of those ten trillion (\skt{padma}) is surrounded by a hundred trillion (\skt{samudra}).%


\textbf{1.58}%
\ And each of those hundred trillion (\skt{samudra}) is surrounded by those whose number is one quadrillion (\skt{madhya}).%
\ Each of those quadrillion (\skt{madhya}) is surrounded by ten quadrillion (\skt{ananta}).%


\textbf{1.59}%
\ Each of those ten quadrillion (\skt{ananta}) is surrounded by a hundred quadrillion (\skt{parārdha}).%
\ Each of those hundred quadrillion (\skt{parārdha}) is surrounded by two hundred quadrillion (\skt{para}).%
\ This is how it is taught, O Brahmin. [All] the possible numbers have been taught.%
\footnote{The translation of \skt{pāda}s c and d is tentative. }%


\textbf{1.60}%
\ Hear about the measurements [of the universe] briefly, O Brahmin, from me,%
\              I shall teach [you].%
\ Listen to the extent [of the Brahmāṇḍa], O Brahmin!%
\                 I shall teach it to you in concise manner.%
\         The body of the Egg is like that of the full moon at moonrise.%


\textbf{1.61}%
\ The whole circumference of the Eggs has been declared by Brahmā%
\              to be \skt{koṭi} times a thousand \skt{koṭi} yojanas.%
\footnote{aṇḍānāṃ plural...: a new egg in every mahākalpa? CHECK }%


\textbf{1.62}%
\ The Sun shines from above from seven thousand and seven hundred \skt{koṭi} [height] ... twenty \skt{koṭi} gulma?? mūrdha?%


\textbf{1.63}%
\ In brief the numbers pertaining to the measurements have been taught.%
\footnote{Note the mixture of different grammatical genders and numbers here.                 Understand \skt{pramāṇeṣu saṃkhyāḥ kīrtitāḥ samāsataḥ}. }%
\ The characteristics of the unmeasurable Brahmāṇḍa[s] have been taught.%


\textbf{1.64}%
\ O true Brahmin, the Purāṇa[s of] 8,000,000 [verses]%
\footnote{\skt{Pāda} a should probably be analysed and interpreted as                         \skt{purāṇam (purāṇānām aśītisahasrāṇi śatāni ślokāni) brahmaṇā kathitam}.                

                Compare this list to Viṣṇupurāṇa 3.3.11--19:                

                \skt{dvāpare prathame vyastaḥ svayaṃ vedaḥ svayaṃbhuvā\danda
                dvitīye dvāpare caiva vedavyāsaḥ prajāpati\twodanda
                tṛtīye cośanā vyāsaś caturthe ca bṛhaspatiḥ\danda 
                savitā pañcame vyāsaḥ ṣaṣṭhe mṛtyuḥ smṛtaḥ prabhuḥ\twodanda 
                saptame ca tathaivendro vasiṣṭhaś cāṣṭame smṛtaḥ\danda 
                sārasvataś ca navame tridhāmā daśame smṛtaḥ\twodanda 
                ekādaśe tu triśikho bharadvājas tataḥ paraḥ\danda 
                trayodaśe cāntarikṣo varṇī cāpi caturdaśe\twodanda 
                trayyāruṇaḥ pañcadaśe ṣoḍaśe tu dhanañjayaḥ\danda 
                kratuñjayaḥ saptadaśe tadūrdhvaṃ ca jayaḥ smṛtaḥ\twodanda 
                tato vyāso bharadvājo bharadvājāc ca gautamaḥ\danda 
                gautamād uttaro vyāso haryātmā yo 'bhidhīyate\twodanda 
                atha haryātmanonte ca smṛto vājaśravāmuniḥ\danda 
                somaśuṣkāyaṇas tasmāt tṛṇabindur iti smṛtaḥ\twodanda 
                ṛkṣobhūdbhārgavas tasmād vālmīkir yo 'bhidhīyate\danda 
                tasmād asmatpitā śaktir vyāsas tasmād ahaṃ mune\twodanda 
                jātukarṇo 'bhavan mattaḥ kṛṣṇadvaipāyanas tataḥ\danda 
                aṣṭaviṃśatir ity ete vedavyāsāḥ purātanāḥ\twodanda  
}                

                Another relevant passage is Brahmāṇḍapurāṇa 3.4.58cd--67:                

                \skt{brahmā dadau śāstram idaṃ purāṇaṃ mātariśvane\twodanda 
                tasmāc cośanasā prāptaṃ tasmāc cāpi bṛhaspatiḥ\danda   
                bṛhaspatis tu provāca savitre tadanantaram\twodanda   
                savitā mṛtyave prāha mṛtyuś cendrāya vai punaḥ\danda 
                indraś cāpi vasiṣṭāya so 'pi sārasvatāya cai\twodanda 
                sārasvatas tridhāmne 'tha tridhāmā ca śaradvate\danda 
                śaradvāṃs tu triviṣṭāya so 'ntarikṣāya dattavān\twodanda 
                carṣiṇe cāntarikṣo vai so 'pi trayyāruṇāya ca\danda 
                trayyāruṇād dhanañjayaḥ sa vai prādāt kṛtañjaye\twodanda 
                kṛtañjayāt tṛṇañjayo bharadvājāya so 'py atha\danda   
                gautamāya bharadvājaḥ so 'pi niryyantare punaḥ\twodanda 
                niryyantaras tu provāca tathā vājaśravāya vai\danda   
                sa dadau somaśuṣmāya sa cādāt tṛṇabindave\twodanda 
                tṛṇabindus tu dakṣāya dakṣaḥ provāca śaktaye\danda 
                śakteḥ parāśaraś cāpi garbhasthaḥ śrutavānidam\twodanda 
                parāśarāj jātukarṇyas tasmād dvaipāyanaḥ prabhuḥ\danda   
                dvaipāyanāt punaś cāpi mayā prāptaṃ dvijottama\twodanda 
                mayā caitat punaḥ proktaṃ putrāyāmitabuddhaye\danda   
                ity eva vākyaṃ brahmādiguruṇāṃ samudāhṛtam\twodanda  } 
                 }%
\ were taught by [1] Brahmā to [2] Mātariśvan [= Vāyu] in their entirety,%
\              in their true form.%


\textbf{1.65}%
\ Vāyu abridged the verses and then gave [the Purāṇas] to [3] Uśanas.%
\ He [Uśanas] also abridged the verses, and [4] Bṛhaspati received them.%


\textbf{1.66}%
\ Bṛhaspati taught 30,000 [verses] to [5] Sūrya [the Sun].%
\ Divākara [= the Sun] taught 25,000 [verses] to [6] Mṛtyu [Death].%


\textbf{1.67}%
\ Death taught 21,000 [verses] to [7] Indra.%
\ Indra taught 20,000 verses to [8] Vasiṣṭha.%


\textbf{1.68}%
\ And he[, Vasiṣṭha taught] 18,000 [verses] to [9] Sārasvata.%
\ Sārasvata [taught] 17,000 [verses] to [10] Tridhāman.%


\textbf{1.69}%
\ [Tridhāman] taught 16,000 verses to [11] Bharadvāja.%
\ [Bharadvāja] taught 15,000 verses to [12] Trivṛṣa.%


\textbf{1.70}%
\ [Trivṛṣa] then [taught] 14,000 verses to [13] Antarīkṣa.%
\ [Antarīkṣa] taught 13,000 [verses] to [14] Trayyāruṇi.%


\textbf{1.71}%
\ Trayyāruṇi, the great Brahmin, having abridged them again,%
\ taught 12,000 [verses] to [15] Dhanaṃjaya.%


\textbf{1.72}%
\ Dhanaṃjaya, the great sage, handed [them] over to [16] Kṛtaṃjaya.%
\ [This recension  was transmitted] from Kṛtaṃjaya, O great Brahmin, to [17] noble Ṛṇaṃjaya.%


\textbf{1.73}%
\ Then from Ṛṇaṃjaya it was given to [18] Gautama, the great sage,%
\ from Gautama to [19] Bharadvāja, from him to [20] Dharmadvata.%
\footnote{The name \skt{harmyadvata} is probably a variant or a corrupted form                        of \skt{harmyātman}, who appears in lists of \skt{vedavyāsa}s                        in the Purāṇas (see note to 1.64). }%


\textbf{1.74}%
\ Then [21] Rājaśravas received it, then [22] Somaśuṣma.%
\ Then from Somaśuṣma [23] Tṛṇabindu received it, O Brahmin.%


\textbf{1.75}%
\ Tṛṇabindu taught it to [24] Vṛkṣa, Vṛkṣa to [25] Śakti [the father of Parāśara].%
\ Śakti taught it to [26] Parāśara, then [Parāśara] to [27] Jātūkarṇa.%
\footnote{Perhaps keep jatu°. }%


\textbf{1.76}%
\ Jātukarṇa taught it to [28] [Vyāsa] Dvaipāyana, the great sage.%
\ Dvaipāyana, the great sage, gave it to Romaharṣa.%


\textbf{1.77}%
\ He [Dvaipāyana] taught the Purāṇa[s]%
\ [consisting of] 12,000 [verses] to Romaharṣa, his brilliant son,%
\                 [in the form that] has been revealed [to us]%
\ for the benefit of humankind. What else do you wish to know?%
\vfill\pagebreak\begin{center}{\large\textbf{  Chapter Two 
}}\end{center}


\textbf{2.1}%
\ Vigatarāga spoke:%
\ I, the Brahmin(? phps accept it) [rather:%
\         through you, a Brahmin], have listened to the concise description of the Brahmāṇḍa,%
\         it's extent, colour, form and the numbers associated with it.%


\textbf{2.2}%
\ You mentioned the Śivāṇḍa as taught to be the receptacle of the Brahmāṇḍa [see 1.40ab].%
\              What are its characteristics and how much is its extent?%


\textbf{2.3}%
\ Whose dwelling/resting place is it [phps ālayana for ālaya]%
\               and [what] is the extent/proof of the one who dwells there? [maybe the number of inhabitants Flo]%
\              [Or: what is its extent and [who are its] inhabitants]?%
\              Who are the people there? And who is Prajāpati there?%


\textbf{2.4}%
\ Anarthayajña spoke:%
\ Please don't ask me about the characteristics of the Śivāṇḍa, O Brahmin.%
\              How could even the gods have the power to really know and see...%


\textbf{2.5}%
\ The path leading to it is not to be trodden, it is extremely secret and [...]%
\         There is no master or the opposite there, nobody to be punished and no punisher.%


\textbf{2.6}%
\ There are no truthful or untruthful people there, no moral or immoral people,%
\              no wicked people, no hypocrisy, no thirst or envy.%


\textbf{2.7}%
\ There is no anger or desire, no arrogance or discontent ([a]sūyaka).%
\                 No envy or hatred, no cheaters and no jealousy.%


\textbf{2.8}%
\ There is no disease, no aging, no grief and no agitation there.%
\                 There are no inferior or superior people and there is nobody in-between.%


\textbf{2.9}%
\ There are no privileged men or women there in Śiva's abode,%
\ no reproach or praise, no selfish or treacherous people.%


\textbf{2.10}%
\ There is no pride or arrogance there, no cruelty or trickery and so on.%
\              There are no beggars and no donors there.%


\textbf{2.11}%
\ Go without material desires (\skt{anarthin}), being there you'll be resting under a wishing tree.%
\ There is no karma there and no enemy. The era of strife [the Kali era] is not there and there is%
\                 no fight.%


\textbf{2.12}%
\ There is no Dvāpara era or Tretā or Kṛta.%
\              There are no Manvantaras (1 Manvantara = 1000 Kalpas) there and no Kalpas.%


\textbf{2.13}%
\ No universal floods of destruction come, and there are no days and nights of Brahmā.%
\ There is no birth and death there and one never encounters catastrophes.%


\textbf{2.14}%
\ Nobody is tied to the noose of hope and there is no passion or delusion.%
\ There are no gods and demons there and no Yakṣas, Serpents and Rākṣasas.%


\textbf{2.15}%
\ There are no Ghosts nor Piśācas, no Gandharvas and no Ṛṣis.%
\ There are no asterisms and planets there, no Nāgas, Kiṃnaras or Garuḍa-like creatures.%


\textbf{2.16}%
\ There is no recitation there or daily rituals, nobody performs the Agnihotra and there is no sacrificer.%
\ There are no religious observances and no austerities and no 'animal hell' [or: on animals and no hell].%


\textbf{2.17}%
\ Nobody would be able to tell the extent of the god Īśāna's[??]%
\ powers starting with aiśvarya, not even in a hundred years.%


\textbf{2.18}%
\ [Instead] I shall teach you all that are produced by Hara's wish one by one,%
\ excluding the gods and people, starting with the trees, the bushes and creepers.%


\textbf{2.19}%
\ [Their?] height is two Parārdha, and [their?] width is the same.%
\              There are lovely flowers of different forms [there] and also lovely fruits.%


\textbf{2.20}%
\ There are also golden trees and also gem trees,%
\              coral gem thickets and ruby plants.%


\textbf{2.21}%
\ There are trees with twigs on which creepers with tasty roots reach for the tasty fruits.%
\              [REVISE]%
\              All of them can change their shapes on their own accord [just bending etc.?] and%
\                         they fulfill man's desires and they whisper in a lovely way[?]%
\                         [any language? maybe not].%
\footnote{After kāmarū°, MS \msCc\ has some folios missing and resumes only at 3.XX. CHECK Florinda's pics! }%


\textbf{2.22}%
\ There [in the Śivāṇḍa], O Brahmin, all the subjects are the oceans of endless virtues.%
\              They are all equally beautiful and strong, and they shine like millions of suns.%


\textbf{2.23}%
\ ... is two Parārdha [yojanas] long and two Parārdha [yojanas] wide,%
\              and two Parārdha yojanas is its extension[?], O great Brahmin.%


\textbf{2.24}%
\ Authority is not a number [cannot be expressed by a number?%
\                OR: there is no question of....?]%
\         neither is the Power of strength, O Brahmin.%
\         Down and up are no numbers [no question of going to heaven or hell?],%
\                 and nobody goes to the Tiryañc [hell] [??? OR with iti: there is no horizontal extension?].%


\textbf{2.25}%
\ I do not know the length and width of the Śivāṇḍa.%
\              Enjoyment is undecaying there, and there is no birth or death there.%
\footnote{Pāda c is unmetrical, or rather, a ra-vipulā with licence                 (tatraiva as SHORT-LONG). Note also the gender problem                 (<i>bhogam akṣayas</i>), or rather take <i>-m-</i> as a sandhi-bridge                 (<i>bhoga-m-akṣayas</i>, for <i>bhogo 'kṣayas</i>). }%


\textbf{2.26}%
\ Inside the Śivāṇḍa, there is the dwelling-place of Īśāna's people [= Īśāna's region]%
\              [on] one and a half Para krore [yojanas? or that many people?], who shine like cow's milk%
\                 [or the region shines?].%


\textbf{2.27}%
\ They are all like the rising sun in the House of Tatpuruṣa%
\              [on] one and a half Para krore [yojanas? or that many people?] in the east.%


\textbf{2.28}%
\ All of them are like collyrium in the southern direction, in the House of Aghora,%
\              [on] one and a half Para krore [yojanas?].%
\footnote{Note the Aiśa form <i>diśiṃ</i> in <ms>C<sub>45</sub></ms>. }%


\textbf{2.29}%
\ In the western direction, in Sadyojāta's beloved House, [on] one and a half krore [yojanas?]%
\              they are like jasmine, the moon, like snowy rocks.%
\footnote{Note the Aiśa form <i>diśiṃ</i> in <ms>K<sub>07</sub></ms> in pāda b.                In pāda d, we may suppose the presence of a sandhi-bridge:                <i>sadya-m-iṣṭālayaḥ</i>. }%


\textbf{2.30}%
\ In the northern direction, in Vāmadeva's House of one and a half krore [yojanas?]%
\              they are like saffron and water.%
\footnote{Note the Aiśa form <i>diśiṃ</i> in <ms>C<sub>95</sub></ms> in pāda b. }%


\textbf{2.31}%
\ Īśāna has five parts (kalā), [his Tatpuruṣa] face has four.%
\              Aghora has eight, and there are thirteen Vāmadeva[-kalā]s.%
\footnote{Note how <i>vaktrasya</i> should refer to Śiva's Tatpuruṣa-face,                 given that the text lists Śiva's five faces: Īśāna, Tatpuruṣa, Aghora, Vāmadeva, Sadyojāta. }%


\textbf{2.32}%
\ Sadyojāta has eight parts. These parts, altogether thirty-eight, which%
\              liberate us from the ocean of existence, have been taught, O truest Brahmin.%


\textbf{2.33}%
\ Those who explore the Truth should know the numbers, the colours and%
\              directions associated with each one [of Śiva's faces] in the way taught above.%


\textbf{2.34}%
\ If one has the intention to go to the Śivāṇḍa [if he is 'pulled' towards it],%
\              one should practise Śiva yoga regularly. Without Śiva yoga, O Brahmin, it is impossible to go there.%


\textbf{2.35}%
\ [Even] by [performing] millions of sacrifices such as the Aśvamedha,%
\              or all the difficult austerities, for a hundred Kalpas,%
\               it is impossible to get there even for the gods, O great ascetic.%
\footnote{Understand \skt{kṛcchrāditapa sarvāṇi} as \skt{kṛcchrāditapāṃsi sarvāṇi}. }%


\textbf{2.36}%
\ By [merely] bathing and performing austerities at all the sacred places such as the Gaṅgā,%
\               even the honorable Ṛṣis will not be able to get there.%


\textbf{2.37}%
\ Or by donating the oceans of the seven islands with all their gems to a Veda expert, O Brahmin,%
\              having faith and devotion, one will not be able to go there%
\              without meditation. [This is a] certainty.%


\textbf{2.38}%
\ He who destroys his own body and gives it without hesitation to those who are in need of it,%
\ or gives away his wife, his son and his possessions or his own head to those in need,%
\ or by [performing] other difficult deeds, will not be able to go there [by merely doing these].%


\textbf{2.39}%
\ He who has completed the sacrifices, the pilgrimages,%
\              the austerities, the donations, the study of the Vedas,%
\ will experience those enjoyments that the Brahmāṇḍa offers, still being%
\                              subject to time and death.%


\textbf{2.40}%
\ Dharma decays with time that is sent by...%
\              Like a circle of burning coal, time goes round and round.%
\               Time is called \skt{kāla} because of the waves (kalana)%
\               of the three divisions of time [past, present, future].%
\vfill\pagebreak\begin{center}{\large\textbf{  Chapter Three 
}}\end{center}


\textbf{3.1}%
\ Vigatarāga spoke:%
\ Why do they call [Dharma] Dharma? And how many embodiments (\skt{mūrti}) is he known to have?%
\footnote{For the correct interpretation of \skt{pāda} a, namely to decide whether these questions                focus on the bull of Dharma or Dharma itself/himself, see MBh 12.110.10--11:                

                \skt{prabhāvārthāya bhūtānāṃ dharmapravacanaṃ kṛtam\danda
                 yat syād ahiṃsāsaṃyuktaṃ sa dharma iti niścayaḥ\twodanda
                 dhāraṇād dharma ity āhur dharmeṇa vidhṛtāḥ prajāḥ\danda
                 yat syād dhāraṇasaṃyuktaṃ sa dharma iti niścayaḥ\twodanda}                

                Note the similarities with this chapter: the phrase \skt{dharma ity āhur},                the fact that the present chapter from verse 18 on is actually a chapter on \skt{ahiṃsā},                and that the etimological explanation involves the word [\skt{ā}]\skt{dhāraṇa} in                both cases. These lead me to think that in \skt{pāda}s ab of this verse in the VSS,                it is Dharma that is the focus of the inquiry and not the bull. }%
\ He is known as a bull: how many legs does it/he have? How many are his paths?%
\footnote{Understand \skt{pāda} d as \skt{gatayas tasya kati smṛtāḥ}. I have accepted                        \skt{smṛtāḥ} because this plural signals that \skt{gatis} is meant to be plural,                        similarly to what happens in 3.6cd (\skt{tasya patnī... mahābhāgāḥ}). }%


\textbf{3.2}%
\ I have become curious [about these questions]. Put an end to my doubts for good.%
\                 Whose son is [Dharma], O best of sages? How many children does he have?%


\textbf{3.3}%
\ Anarthayajña spoke:%
\ Well, the root [sic!] \skt{dhṛti} (`resolution') is said to be a synonym [of \skt{dharma}].%
\                  It is called Dharma because it supports (\skt{āDHĀRaṇa}) and because it is great (\skt{MAhattva}).%
\footnote{On a non-verbal stem being a \skt{dhātu}, see e.g.                                                 Vāyupurāṇa 3.17cd:                        \skt{bhāvya ity eṣa dhātur vai bhāvye kāle vibhāvyate};                                                Vāyupurāṇa 3.19cd (= Brahmāṇḍapurāṇa 1.38.21ab):                        \skt{nātha ity eṣa dhātur vai dhātujñaiḥ pālane smṛtaḥ};                                                Liṅgapurāṇa 2.9.19:                        \skt{bhaja ity eṣa dhātur vai sevāyāṃ parikīrtitaḥ}; etc. }%


\textbf{3.4}%
\ The four-legged Bull is the embodiment of both Śruti and Smṛti.%
\                The four \skt{āśrama}s are taught by the wise to be [the four legs of] Dharma.%
\                 [or rather: ... which is Dharma as made up of the four āśramas... kīrtitaḥ!]%
\footnote{A similar image of the legs of the Bull of Dharma being the four \skt{āśrama}s                 is hinted at MBh 12.262.19--21:                 

                    \skt{dharmam{ }ekaṃ catuṣpādam{ }āśritās{ }te nararṣabhāḥ\danda               
taṃ santo vidhivat{ }prāpya gacchanti paramāṃ gatim\twodanda               
gṛhebhya eva niṣkramya vanam{ }anye samāśritāḥ\danda               
gṛham{ }evābhisaṃśritya tato 'nye brahmacāriṇaḥ\twodanda               
dharmam{ }etaṃ catuṣpādam{ }āśramaṃ brāhmaṇā viduḥ\danda               
ānantyaṃ brahmaṇaḥ sthānaṃ brāhmaṇā nāma niścayaḥ\twodanda}        

        On the more frequently quoted interpretation of the four legs, see Olivelle `Āśrama', 235:        ``Dharma and truth possess all four feet and are whole during the Kṛta yuga,         and people did not obtain anything unrighteously (\skt{adharmeṇa}).         By obtaining, however, \skt{dharma} has lost one foot during each of the other \skt{yuga}s         and righteousness (\skt{dharma}) likewise has diminished by one quarter due to theft,         falsehood, and deceit. (MDh 1.81--82)''        

        Understand \skt{pāda}s c and d as \skt{catvāri āśramāṇi kīrtitāni dharmo manīṣibhiḥ} or                \skt{yo dharmaḥ kīrtitaś caturāśramāṇi manīṣibhiḥ} or                 \skt{yo dharmaś caturāśramaḥ kīrtito manīṣibhiḥ}. }%


\textbf{3.5}%
\ And the paths of Dharma are five. Listen, O Brahmin:%
\                   [existence as] gods, men, animals, [existence in] hell%
\                         and [as] immovable things [such as plants and rocks] etc.%
\footnote{Understand \skt{gatiś} as \skt{gatayaś} and note that \skt{vijñeyāḥ} is an emendation from                \skt{vijñeyaḥ} following the logic of 3.1d.                 \skt{tirya} seems to be an acceptable nominal stem in this text for \skt{tiryañc}. See                e.g. 4.6a: \skt{devamānuṣatiryeṣu}.  \skt{°ādayaḥ} in \skt{pāda} d seems superfluous. }%


\textbf{3.6}%
\ Eternal Dharma was born after splitting Brahmā's heart.%
\                  He has beautiful wives, thirteen in number, with nice waists.%
\footnote{Note the use of the singular in \skt{pāda}s c and d. I have left \skt{sumadhyamāḥ} as the        manuscripts transmit it: it signals the presence of the plural. And consider         correcting \skt{mahābhāgā} to \skt{mahābhāgās}. In sum,                 understand \skt{tasya patnyo mahābhāgās trayodaśa sumadhyamāḥ}. }%


\textbf{3.7}%
\ They are Dakṣa's daughters, [called] Śraddhā and so on. They have huge eyes and they are beautiful.%
\footnote{\skt{śraddhāḍhyāḥ} in \skt{pāda} b is an attractive lectio difficilior (`they were rich in faith/devotion'), but I have finally                 decided to accept the easier and better-attested \skt{śraddhādyā}[\skt{ḥ}]. }%
\ and they are charming. Numerous sons and grandsons were born to him.%
\ This is the emergence of Dharma. What more do you wish to hear?%
\footnote{Again, I have chosen/applied the plural forms \skt{°ādyāḥ} and \skt{sumanoharāḥ} in \skt{pāda} b to hint at the fact                        that the presence of the plural is to be preferred here; thus only \skt{viśālākṣī} is                         problematic. As \skt{patnī} in the previous verse, it should be treated as a plural.        Note the use of the singular for the plural also in \skt{pāda}s cd, especially \skt{babhūva ha} for \skt{babhūvuḥ}.         

        MMW on Dakṣa:        ``daughters of whom 27 become the Moon's wives, forming the lunar asterisms, and 13         [or 17 BhP.; or 8 R.] those of Kaśyapa, becoming by this latter the mothers         of gods, demons, men, and animals, while 10 are married to Dharma, Mn. ix, 128f.'' CHECK }%


\textbf{3.8}%
\ Vigatarāga spoke:%
\footnote{Consider emending \skt{tebhyaḥ} to the correct feminine form \skt{tābhyaḥ}.                Note again the use of the singular (nominative) for the plural (accusative) in \skt{pāda}s ab.                Alternatively, emend \skt{dharmapatnī} to \skt{dharmapatnīr} (plural accusative) and                 \skt{putras} to \skt{putrān} to make them work with \skt{śrotum icchāmi}. }%
\ I would like to hear about Dharma's wives according to the truth[?] and about each one of the sons born to them.%
\         Teach me, O great ascetic.%


\textbf{3.9}%
\ Anarthayajña spoke:%
\footnote{For Dharma's thirteen wives and their sons, see Liṅgapurāṇa 1.5.34-37 (note the                        similarity between the first line and VSS 3.6cd--7ab above):        

        \skt{dharmasya patnyaḥ śraddhādyāḥ kīrtitā vai trayodaśa\danda        
tāsu dharmaprajāṃ vakṣye yathākramam{ }anuttamam\twodanda        
kāmo darpo 'tha niyamaḥ saṃtoṣo lobha eva ca\danda        
śrutas{ }tu daṇḍaḥ samayo bodhaś{ }caiva mahādyutiḥ\twodanda        
apramādaś{ }ca vinayo vyavasāyo dvijottamāḥ\danda        
kṣemaṃ sukhaṃ yaśaś{ }caiva dharmaputrāś{ }ca tāsu vai\twodanda         
dharmasya vai kriyāyāṃ tu daṇḍaḥ samaya eva ca\danda        
apramādas{ }tathā bodho buddher{ }dharmasya tau sutau\twodanda}         }%
\ [Dharma's wives are:] [1] Śraddhā (`Faith'), [2] Lakṣmī (`Prosperity'), [3] Dhṛti (`Resolution'),%
\         [4] Tuṣṭi (`Satisfaction'), [5] Puṣṭi (`Growth'), [6] Medhā (`Wisdom'),%
\         [7] Kriyā (`Ritual'), [8] Lajjā (`Modesty'), [9] Buddhi (`Intelligence'),%
\         [10] Śānti (`Tranquillity'), [11] Vapus (`Beauty'), [12] Kīrti (`Fame'), [13] Siddhi (`Success'),%
\                       [all] born to Prasūti [Dakṣa's wife].%
\footnote{

\skt{prasūtisambhavāḥ} is a rather bold conjecture that can be supported by two facts:                        firstly, the readings of the manuscripts are difficult to make sense of and thus are                                         probably corrupt; secondly, a corruption from the name Prasūti, that of Dakṣa's wife, to \skt{ābhūti}                                         is relatively easily to explain, \skt{sū} and \skt{bhū} being close enough in some scripts                                          (e.g. in \msCa) to cause confusion. Another option would be to accept                                          Ābhūti as the name of Dakṣa's wife.                

                For Prasūti being Dakṣa's wife in other sources,                         see e.g. Liṅgapurāṇa 1.5.20--21 (but also note the presence of the name Sambhūti...):                               \skt{prasūtiḥ suṣuve dakṣāc{ }caturviṃśatikanyakāḥ\danda                                śraddhāṃ lakṣmīṃ dhṛtiṃ puṣṭiṃ tuṣṭiṃ medhāṃ kriyāṃ tathā\twodanda                                buddhi lajjāṃ vapuḥ śāntiṃ siddhiṃ kīrtiṃ mahātapāḥ\danda                                khyātiṃ śāntiś{ }ca saṃbhūtiṃ smṛtiṃ prītiṃ kṣamāṃ tathā\twodanda} }%


\textbf{3.10}%
\ Śraddhā's son is Kāma (`Desire'), Darpa (`Pride') is said to be Lakṣmī's son.%
\footnote{Understand \skt{śraddhā} as a stem form noun for \skt{śraddhāyāḥ} (gen./abl.). }%
\ Dhṛti's son is Niyama (`Rule'), Saṃtoṣa (`Satisfaction') is Tuṣṭi's son.%


\textbf{3.11}%
\ To Puṣṭi was born a son [called] Lābha (`Profit'). Medhā's son is Śruta (`Sacred Knowledge').%
\ Kriyā's sons are Abhaya (`Freedom from danger'), Daṇḍa (`Punishment') and Samaya (`Law').%
\footnote{It is tempting to emend \skt{abhayaḥ} to \skt{ubhayaḥ}, thus matching the relevant line in the Kūrmapurāṇa cited above:                        \skt{kriyāyāś cābhavat putro daṇḍaḥ samaya eva ca} and allotting only two sons to Kriyā, but                        in a number of sources Kriyā actually has three sons, see e.g. Viṣṇupurāṇa 1.7.29(ab? CHECK in book),                        where they are named as Daṇḍa, Naya and Vinaya:                                \skt{medhā śrutaṃ kriyā daṇḍaṃ nayaṃ vinayam eva ca}.                         Perhaps read \skt{kriyāyās tu nayaḥ putro} in pāda c? Compare Vāyupurāṇa 1.10.34cd                                        \skt{kriyāyās tu nayaḥ prokto daṇḍaḥ samaya eva ca}                                 with Brahmāṇḍapurāṇa 1.9.60ab:                                        \skt{kriyāyās tanayau proktau damaś ca śama eva ca} }%


\textbf{3.12}%
\ Lajjā's son is Vinaya (`Discipline'), Buddhi's son is Bodha (`Intelligence').%
\ Lajjā has two [more] sons: Sudhiya[/Sudhī] (`Wise') and Apramāda (`Cautiousness').%
\                 [or one more son only: the wise Apramāda?]%
\footnote{In a very similar passages in Kūrmapurāṇa 1.8.20 ff., Apramāda is Buddhi's son and                  Lajjā has only one son, Vinaya. In the above verse (VSS 3.12), \skt{sudhiyaḥ} (for \skt{sudhīḥ}) may only be                         qualifying \skt{apramāda}, thus Lajjā may have two sons: Vinaya and the wise Apramāda. }%


\textbf{3.13}%
\ Kṣema (`Peace') is to be known as Śānti's son, Vyavasāya (`Resolution') is Vapus' son.%
\ Yaśas (`Fame') is Kīrti's son, Sukha (`Joy') was born to Siddhi.%
\ [This is how] the sons of Dharma in the era of Svāyambhuva [Manu] were known.%
\footnote{Note that \skt{sukhaṃ} in \skt{pāda} d is probably meant to be masculine (\skt{sukhaḥ}), but e.g. in the                 Kūrmapurāṇa passage quoted above it is also neuter. For the emendation in \skt{pāda} e,                 see Matsyapurāṇa 9.2cd:                                       \skt{yāmā nāma purā devā āsan svāyambhuvāntare}                and Bhāgavatapurāṇa 6.4.1:                                            \skt{devāsuranṛṇāṃ sargo nāgānāṃ mṛgapakṣiṇām\danda                                      sāmāsikas tvayā prokto yas tu svāyambhuve 'ntare\twodanda}. }%


\textbf{3.14}%
\ Vigatarāga spoke:%
\ How does Dharma have two embodiments? Tell me, O great ascetic.%
\ I am extremely intrigued. Cut my doubts concerning [this] knowledge.%
\footnote{Note \skt{dharma} as a neuter noun and the form \skt{atīvaṃ} for \skt{atīva} metri causa.         My emendation from \skt{kīrtaya} (`declare') to \skt{kartaya} (`cut') was influenced by the combination        of \skt{chindhi} and \skt{saṃśaya}, often with \skt{kautūhala}, elsewhere in the VSS:                3.2b: \skt{saṃśayaṃ chindhi tattvataḥ};                 10.XXcd: \skt{kautūhalaṃ mahaj jātaṃ chindhi saṃśayakārakam};                15.2ab: \skt{etat kautūhalaṃ chindhi saṃśayaṃ parameśvara}.                 The reading \skt{kīrtaya} may have been the result of the influence of \skt{kīrtitā} in 3.13b above                        (De Simini's convinicing observation). }%


\textbf{3.15}%
\ Anarthayajña spoke:%
\ Dharma's embodiment is said to consist of Śruti and Smṛti.%
\footnote{The reading \skt{°dvayī} in \msNc\ in \skt{pāda} a is attractive, but as Judit                         Törzsök has pointed out to me, it is probable that                        the slightly less convincing but widespread variant \skt{°dvayor} is original. }%
\ The characteristics of the Śrauta [tradition] are an association with a wife [i.e.\ marriage] and with%
\                          the fire ritual, and sacrifice.%
\ The Smārta [tradition] [focuses on] the conduct (\skt{ācāra}) of the classes (\skt{varṇa}) and life-stages (\skt{āśrama})%
\                   which is connected to rules and regulations (\skt{yama-niyama}).%
\footnote{To state that the Smārta tradition is connected to \skt{yama}s and \skt{niyama}s and the \skt{āśrama}s and        then to discuss these at length (principally in chapters 3--8 and 11) can be seen                                 as a clear self-identification with the Smārta tradition. }%


\textbf{3.16}%
\footnote{\skt{Pāda} a should be understood as \skt{yamaniyamayoś caiva}, but the author of this line                may have tried to avoid the metrical fault of having two short syllables in the second and third positions. }%
\ Now hear the classification of both the \skt{yama} and \skt{niyama} rules.%
\footnote{Note that this is the beginning of a long section in our text        that describes the \skt{yama-niyama} rules, reaching up to the end of chapter eight.         The title given in the colophon of the next chapter, chapter four, namely \skt{yamavibhāga},        would fit this locus better than the beginning of that chapter, which         commences with a discussion on the second of the \skt{yama}s, \skt{satya}. }%
\ Non-violence, truthfulness, not stealing, kindness, self-restraint, the rule of taboos,%


\textbf{3.17}%
\ virtue, carefulness, charm, honesty: these are the ten \skt{yama}s.%
\footnote{Note how all witnesses read \skt{mādhūrya} instead of \skt{mādhurya}. The former may have been                acceptable originally in this text. }%
\ The wise say that there are five subclasses to each.%


\textbf{3.18}%
\ I shall teach you about non-violence and the other [\skt{yama}-rules]. Listen carefully, O Brahmin.%
\ Frightening and beating [other people], tying [someone] up, killing and the destruction of [other people's] livelihood:%
\ violence is said by the wise who see the truth to be of [these] five types.%


\textbf{3.19}%
\ Cruel people beat [other people] with sticks, clods of earth [understand: they stone them],%
\                         whips and other [objects] in the everyday world.%
\ Their bodies broken by the same blows, they receive the capital punishment.%
\footnote{Note the use of the singular in \skt{pāda}s cd referring back to the agents of the previous sentence.                Most probably, °\skt{vadhyam} is to be understand as °\skt{vadham} and the form                         \skt{vadhyam} serves only to avoid two \skt{laghu} syllables in \skt{pāda} d. }%


\textbf{3.20}%
\ [Others,] tie up [people] at their feet and their arms and chests.%
\                 [These,] bound by [with?] their hair and [on their?] necks,%
\footnote{Understand \skt{bhujoraś ca} in \skt{pāda} a as \skt{bhuje urasi ca}, in this case with an instance of double sandhi:                \skt{bhuje urasi ca} -- \skt{bhuja urasi ca} -- \skt{bhujorasi ca}. Alternatively, understand it as a compound:                                     \skt{bhujorasi}. }%
\ die without being wounded. This is the capital punishment for tying up [other people].%


\textbf{3.21}%
\ He who frightens [other people] with the terrible danger of enemies and thieves,%
\         with lions, tigers, elephants or snakes,%
\ will be destroyed [by the above] or by other horrors.%


\textbf{3.22}%
\ He who robs somebody's money is to be punished by the same person.%
\footnote{Understand \skt{vadhaḥ} in \skt{pāda} b as \skt{vadhyaḥ} metri causa. }%
\ He is [to be] hit by those whose livelihood got damaged by him as many times [as%
\                         the victims are].%


\textbf{3.23}%
\ [Those who kill other people] with poison, fire, arrows, swords, or by the force of%
\                                 magic or yoga%
\ are called murderers by the sages who see the truth, O great Brahmin[, and%
\                 to be killed by the same methods].%
\footnote{\skt{Pāda} a is unmetrical.               Note how elliptical this verse is and that \skt{hiṃsakāni} is neuter although it refers to                 people, perhaps implying \skt{bhūtāni}. Alternatively, take \skt{°ny°} in \skt{hiṃsakāny} as                 rather unusual sandhi-bridge (\skt{hiṃsakā-ny-āhu}).                 Note also that \skt{āhu} stands for \skt{āhur} metri causa. }%


\textbf{3.24}%
\ Non-violence is the highest Dharma. He who abandons it is a wicked person.%
\ It is free of pain and trouble, it yields the fruits of all [other] Dharmic teachings [in itself].%
\footnote{Note \skt{dharma} as a neuter noun in \skt{pāda} a and that \skt{°vinirmuktaṃ} and                        \skt{°pradam} are neuter accordingly. }%


\textbf{3.25}%
\ There isn't a bigger fool than he [who abandons it is]. There is no bigger mental darkness%
\                                 [than the abandonment of non-violence].%
\ There is no greater suffering or greater infamy.%
\footnote{Note that \skt{parataro} is masculine in \skt{pāda} d, picking up a neuter \skt{'yaśaḥ}.        This phenomenon is probably the result of \skt{'yaśaḥ} resembling a masculine noun ending in \skt{-aḥ}                and also of the metrical problem with the grammatically correct                        \skt{nātaḥ parataram ayaśaḥ}. }%


\textbf{3.26}%
\ There is no greater sin or a more effective poison.%
\ There is no greater ignorance, there is nothing worse, O great ascetic.%
\footnote{\skt{Pāda} d (\skt{nātaḥ paraṃ tapodhana}) is slightly suspicious. The text may have read \skt{nātaḥ paratamo 'dhanaḥ}         (`There is no bigger loss of wealth') or possibly something starting with                \skt{nātaḥ paraṃ tapo ...} (`There is no greater austerity...'). }%


\textbf{3.27}%
\ He who does not harm the four types of living beings beginning with plants%
\ is the best person, having compassion for all creatures.%


\textbf{3.28}%
\ He who always has compassion for all creatures is the [true] Pandit.%
\ He is the [true] sacrificer, the [true] ascetic, he is the donor, the one with a firm vow CHECK.%


\textbf{3.29}%
\ Non-violence is the supreme sacred place. Non-violence is the highest austerity.%
\ Non-violence is the highest donation. Non-violence is the highest joy.%


\textbf{3.30}%
\ Non-violence is the supreme sacrifice. Non-violence is the supreme religious observance.%
\ Non-violence is supreme knowledge. Non-violence is the supreme ritual.%


\textbf{3.31}%
\ Non-violence is the highest purity. Non-violence is the highest self-restraint.%
\ Non-violence is the highest profit. Non-violence is the greatest fame.%


\textbf{3.32}%
\ Non-violence is the supreme Dharma. Non-violence is the supreme path.%
\ Non-violence is the supreme Brahman. Non-violence is supreme Śiva.%


\textbf{3.33}%
\ One should refrain from meat-consumption. One should not even desire it mentally.%
\ He who abandons meat will receive a great reward.%


\textbf{3.34}%
\ He who wishes to nourish his own flesh with the flesh of other [beings],%
\footnote{See Uttarottara chapter two for a similar section on meat-consumption. }%
\ outside of worshipping the ancestors and the gods, is the biggest sinner of all.%


\textbf{3.35}%
\ During the \skt{madhuparka} offering and during a sacrifice, during rituals%
\                                                 for the ancestors and the gods:%
\ only in these cases are animals to be slaughtered and not in any other case.%
\                                                 [This is what] Manu taught.%


\textbf{3.36}%
\ Should he buy it or procure it himself or should it be offered by others,%
\ if he eats meat, he will not sin if he first worships the gods and the ancestors.%


\textbf{3.37}%
\ [People who know] the Vedas and [perform] sacrifices and austerities and [visit] sacred places,%
\         donate, [are of] good conduct, [perform] rituals and [keep] religious vows [but eat meat]%
\         will not [be able to] enjoy even a tiny portion of [such rewards that]%
\         [those] people [receive] who have given up meat.%
\footnote{See a similarly phrased comparison in Manu 2.86:                

                       \skt{ye pākayajñās catvāro vidhiyajñasamanvitāḥ \danda
                        sarve te japayajñasya kalāṃ nārhanti ṣoḍaśīm \twodanda} }%


\textbf{3.38}%
\ The deer and the goats, the sheep, the cows and other [animals] wander in the world%
\                  happily and in great strength [just] from eating leaves and grass.%


\textbf{3.39}%
\ Monkeys eat fruits, Rākṣasas prefer blood.%
\footnote{Understand \skt{phalam āhārā} as \skt{phalāhārā} (-m- is a sandhi-bridge). }%
\ The fruit-eating monkeys defeated all the Rākṣasas [as the Rāmāyaṇa tells us].%


\textbf{3.40}%
\ Therefore one should not crave meat in the hope of gaining strength, O Brahmin,%
\  in order to be able to draw a bow with force,%
\                         or out of fear of the danger coming from the enemy.%
\footnote{\skt{guṇākāśāt} in pāda c is difficult to interpret and                 \skt{guṇākarṣāt} is a conjecture by Judit Törzsök which fits the context well,                although the polysemy of \skt{guṇa} may allow for other solutions.        

        Verses 3.40--42 may be echoing Brahmapurāṇa 216.64--66:        

                      \skt{ māṃsān miṣṭataraṃ nāsti bhakṣyabhojyādikeṣu ca \danda
                        tasmān māṃsaṃ na bhuñjīta nāsti miṣṭaiḥ sukhodayaḥ \twodanda 
                        gosahasraṃ tu yo dadyād yas tu māṃsaṃ na bhakṣayet \danda
                        samāv etau purā prāha brahmā vedavidāṃ varaḥ \twodanda
                        sarvatīrtheṣu yat puṇyaṃ sarvayajñeṣu yat phalam \danda
                        amāṃsabhakṣaṇe viprās tac ca tac ca ca tatsamam \twodanda }         }%


\textbf{3.41}%
\ One cannot be equal to someone who refrains from violence%
\                 by [merely] wishing to make donations and perform sacrifices.%
\footnote{Pādas ab probably stand for \skt{ahiṃsako nāsti samo dānayajñasamīhaiḥ puruṣaiḥ} CHECK                        and are reminescent of Śivadharmaśāstra 11.92:

                              \skt{  ahiṃsaikā paro dharmaḥ śaktānāṃ parikīrtitam\danda
                                aśaktānām ayaṃ dharmo dānayajñādipūrvakaḥ \twodanda }                

                Note the variant \skt{°dharma°} in both \msCc\ and \Ed\ in \skt{pāda} b. }%
\ [He will have] fame and glory in this world and the supreme path in the other.%


\textbf{3.42}%
\ A person who refrains from violence will gain,%
\                  no doubt about it, the [same] meritorious rewards%
\                  that others would get by donating the three worlds filled with jewels and%
\                  gems in their entirety to an excellent Brahmin,%
\                  by performing a thousand [times] ten trillion (\skt{padma})%
\                  [times] ten thousand (\skt{ayuta}) \skt{koṭīyajña} (= koṭihoma?) sacrifices,%
\                  by donating the earth [to a priest] as sacrificial fee,%
\                  and by bathing [at] a thousand times ten million times a million (\skt{niyuta})%
\                                                 sacred places at once,%
\footnote{On \skt{padma} meaning `ten trillion', and on other words for numbers, see 1.32--35.         

                \skt{koṭīyajña} in pāda d may refer to a special kind of sacrifice,                 mostly known as \skt{koṭihoma} in the Purāṇas and in inscriptions                 (see e.g. Fleming 2010 and 2013)                It probably involves a hundred fire-pits                 and a hundred times one thousand brāhmaṇas (hence the name `the ten-million sacrifice').                See e.g. Bhaviṣyapurāṇa uttaraparvan 4.142.54--58:                        

                             \skt{  śatānano daśamukho dvimukhaikamukhas tathā \danda                                caturvidho mahārāja koṭihomo vidhīyate \twodanda                                 kāryasya gurutāṃ jñātvā naiva kuryād aparvaṇi \danda                                yathā saṃkṣepataḥ kāryaḥ koṭihomas tathā śṛṇu \twodanda                                kṛtvā kuṇḍaśataṃ divyaṃ yathoktaṃ hastasaṃmitam \danda                                ekaikasmiṃs tataḥ kuṇḍe śataṃ viprān niyojayet \twodanda                                sadyaḥ pakṣe tu viprāṇāṃ sahasraṃ parikīrtitam \danda                                ekasthānapraṇīte 'gnau sarvataḥ paribhāvite \twodanda                                 homaṃ kuryur dvijāḥ sarve kuṇḍe kuṇḍe yathoditam \danda                                yathā kuṇḍabahutve 'pi rājasūye mahākratau \twodanda  }                        

                Note that the second syllable of \skt{phalam} in \skt{pāda} d is treated as a long syllable: this                happens often at word-boundaries in this text; and                 note how \msNc\ aims to restore the metre by inserting \skt{tv} after its \skt{phalaṃ}. }%
\vfill\pagebreak\begin{center}{\large\textbf{ Chapter Four 
}}\end{center}


\textbf{4.1}%
\ Anarthayajña spoke:%
\ The state of being real (\skt{sad-bhāva}) is called Truth (\skt{sat-ya}).%
\          Alternatively, it is also a notion that originates in perception.%
\          [Also, it is] relating things that correspond to reality. This is how Truth is discussed. REVISE%
\footnote{Should we read \skt{satyalakṣaṇaṃ} in pāda d, following the rather similar                         Śivadharmaśāstra 11.105cd? }%


\textbf{4.2}%
\ He who endures severe abuse and beating etc.%
\ but keeps quiet, his self being conquered, is said to be [an example of] truth.%
\footnote{\skt{suduḥsaham} (singular) in \skt{pāda} b picks up \skt{°ādīni} (plural) in \skt{pāda} a.        The \skt{-m} in \skt{satyam} may be a sandhi-bridge and the phrase may refer to a        masculine subject thus: \skt{sa ca satya -m- udāhṛtaḥ}. }%


\textbf{4.3}%
\ If one is being interrogated any time with a sword lifted to strike him down,%
\         then it is not the truth that is to be spoken. [In this case,] a lie is called truth.%


\textbf{4.4}%
\ A person who is walking on the road and is afraid of being killed,%
\         should not reply [to people who are potentially dangerous] even if they ask him.%
\                         That is also called Truth.%


\textbf{4.5}%
\ A lie does not hurt when it is connected with joking,%
\         with women, O king[!], at the time of marriage,%
\         at the departure from life and when one's entire wealth is about to be taken away.%
\         They call these five kinds of lies Truth.%


\textbf{4.6}%
\ Since Truth is the supreme Dharma with respect to gods, humans and animals[?],%
\         Truth is the best, the most preferable. Truth is the eternal Dharma.%


\textbf{4.7}%
\ Truth is an unmanifest ocean. Truth yields imperishable pleasures.%
\         Truth is the ship that carries you to the other world. Truth is the wide path.%
\footnote{\skt{Pāda} d is slightly problematic because it is difficult to ascertain if some of the                MSS actually read \skt{panthāna} or \skt{pasthāna} (or \skt{yasthāna}).                I suspect that \skt{panthāna} is a stem form noun formed (metri causa) to stand for an irregular nominative                of \skt{pathin}. }%


\textbf{4.8}%
\ Truth is said to be the desired path. Truth is the supreme sacrifice.%
\         Truth is a pilgrimage place, a supreme pilgrimage place. Truth is an endless donation.%


\textbf{4.9}%
\ Truth is morality, austerity, knowledge. Truth is purity, self-control and tranquillity.%
\                  Truth is the ladder upwards. Truth is fame and glory and happiness.%


\textbf{4.10}%
\ [When] a thousand Aśvamedha sacrifices and Truth are measured on a pair of scales,%
\                         Truth indeed surpasses a thousand Aśvamedha sacrifices.%


\textbf{4.11}%
\ The Sun shines because of Truth. The Earth stays in place by Truth.%
\                 The winds blow because of Truth. Water is cooling through Truth.%
\footnote{Here and several times below, \skt{satye} is probably to be taken as standing for \skt{satyena}. }%


\textbf{4.12}%
\ The oceans dwell in Truth because of their encounter[?] with Priyavrata [Manu's son].%
\footnote{\skt{Pāda} b, \skt{samayena priyavrataḥ}, probably stand for \skt{samayena priyavratasya} although        it is unclear to me what exactly \skt{samaya} refers to here.        

        For Priyavrata's story, in which he wanted to turn nights into days by         circling aroung Mount Meru in a chariot, and by this produced the seven oceans,         see e.g. Bhāgavatapurāṇa 5.1.30--31:
         \skt{yāvad avabhāsayati suragirim anuparikrāman bhagavān ādityo         vasudhātalam ardhenaiva pratapaty ardhenāvacchādayati, tadā hi [priyavrataḥ]         bhagavadupāsanopacitātipuruṣaprabhāvas tad anabhinandan samajavena         rathena jyotirmayena rajanīm api dinaṃ kariṣyāmīti saptakṛtvas          taraṇim anuparyakrāmad dvitīya iva pataṅgaḥ\danda         ye vā u ha tadrathacaraṇanemikṛtaparikhātās te sapta sindhava āsan         yata eva kṛtāḥ sapta bhuvo dvīpāḥ\danda}          }%
\ Govinda abides in Truth because He [as Vāmana] stopped [Mahā]Bali [in spite of the%
\                         fact that this was achieved by a trick].%
\footnote{

Pādas cd: for a somewhat similar reference to the story of Mahābali, see e.g. Vāmanapurāṇa 65.66:        \skt{evaṃ purā cakradhareṇa viṣṇunā baddho balir vāmanarūpadhāriṇā \danda        śakrapriyārthaṃ surakāryasiddhaye hitāya viprarṣabhagodvijānām \twodanda}  }%


\textbf{4.13}%
\ Fire burns with Truth. The Moon rises by Truth.%
\footnote{Since \skt{śaśi} (instead of \skt{śaśin}) is a possible stem in this text,                 \skt{śaśir ācaraḥ} could also be possible here in pāda b (see \msNa\msNb\msNc), perhaps standing for                 \skt{śaśinaś caraṇam} or \skt{śaśiś carati}. My emendation (\skt{śaśinācaraḥ})                 could stand for \skt{śaśinā/śaśinaś cāraḥ} metri causa. }%
\ It is because of Truth that the Vindhya mountain stands in place and that%
\                 although is was growing it is not growing [anymore].%
\footnote{

Pādas cd refer to the story of Agastya and the Vindhya mountain:        Vindhya became jealous of the Sun's revolving around Mount Meru and when the Sun         refused to him the same favour, he decided to grow higher and obstruct the Sun's movement.        As a solution to this situation,         Agastya asked Vidhya to bend down to make it easier for him to reach the south and        to remain thus until he retured. Vindhya agreed to do what Agastya asked him to do         but Agastya never returned. See Mahābhārata 3.102.1--14 (see in the word \skt{samaya} in verse 13                and compare it to VSS 4.12b):                

        \skt{                yudhiṣṭhira uvāca \danda
                         kimarthaṃ sahasā vindhyaḥ pravṛddhaḥ krodhamūrchitaḥ \danda
                         etad icchāmy ahaṃ śrotuṃ vistareṇa mahāmune \twodanda
                         lomaśa uvāca \danda
                         adrirājaṃ mahāśailaṃ meruṃ kanakaparvatam \danda
                         udayāstamaye bhānuḥ pradakṣiṇam avartata \twodanda
                         taṃ tu dṛṣṭvā tathā vindhyaḥ śailaḥ sūryam athābravīt \danda
                         yathā hi merur bhavatā nityaśaḥ parigamyate \twodanda
                         pradakṣiṇaṃ ca kriyate mām evaṃ kuru bhāskara \danda
                         evam uktas tataḥ sūryaḥ śailendraṃ pratyabhāṣata \twodanda
                         nāham ātmecchayā śaila karomy enaṃ pradakṣiṇam \danda
                         eṣa mārgaḥ pradiṣṭo me yenedaṃ nirmitaṃ jagat \twodanda
                         evam uktas tataḥ krodhāt pravṛddhaḥ sahasācalaḥ \danda
                         sūryācandramasor mārgaṃ roddhum icchan paraṃtapa \twodanda
                         tato devāḥ sahitāḥ sarva eva; sendrāḥ samāgamya mahādrirājam \danda
                         nivārayām āsur upāyatas taṃ; na ca sma teṣāṃ vacanaṃ cakāra \twodanda
                         athābhijagmur munim āśramasthaṃ; tapasvinaṃ dharmabhṛtāṃ variṣṭham \danda
                         agastyam atyadbhutavīryadīptaṃ; taṃ cārtham ūcuḥ sahitāḥ surās te \twodanda
                         devā ūcuḥ \danda
                         sūryācandramasor mārgaṃ nakṣatrāṇāṃ gatiṃ tathā \danda
                         śailarājo vṛṇoty eṣa vindhyaḥ krodhavaśānugaḥ \twodanda
                         taṃ nivārayituṃ śakto nānyaḥ kaś cid dvijottama \danda
                         ṛte tvāṃ hi mahābhāga tasmād enaṃ nivāraya \twodanda
                         lomaśa uvāca \danda
                         tac chrutvā vacanaṃ vipraḥ surāṇāṃ śailam abhyagāt \danda
                         so 'bhigamyābravīd vindhyaṃ sadāraḥ samupasthitaḥ \twodanda
                         mārgam icchāmy ahaṃ dattaṃ bhavatā parvatottama \danda
                         dakṣiṇām abhigantāsmi diśaṃ kāryeṇa kena cit \twodanda
                         yāvadāgamanaṃ mahyaṃ tāvat tvaṃ pratipālaya \danda
                         nivṛtte mayi śailendra tato vardhasva kāmataḥ \twodanda
                         evaṃ sa samayaṃ kṛtvā vindhyenāmitrakarśana \danda
                         adyāpi dakṣiṇād deśād vāruṇir na nivartate \twodanda
                         etat te sarvam ākhyātaṃ yathā vindhyo na vardhate \danda
                         agastyasya prabhāvena yan māṃ tvaṃ paripṛcchasi \twodanda } }%


\textbf{4.14}%
\ The [mythical] Lokāloka mountains are located in Truth. Mount Meru stands by Truth.%
\ The Vedas abide in Truth. Dharma is rooted in Truth.%


\textbf{4.15}%
\ The milk a cow yields is Truth. Ghee in milk is there as Truth.%
\ The soul dwells in the body in Truth. The eternal soul is Truth.%


\textbf{4.16}%
\  If Truth alone (ekena) is obtained, Dharma is surely accomplished.%
\footnote{Another way to translate \skt{ekena} in pāda a would turn the sentence into this:               `If Truth is obtained by somebody, he will be one for whom Dharma is surely accomplished.' }%
\ By the heroism of Rāma Rāghava, Truthfulness was well-guarded, more than anything else.%


\textbf{4.17}%
\ This is how [I] taught the rules of Truth to you, O virtuous one,%
\ to favour the whole world. What else do you wish to hear?%


\textbf{4.18}%
\ Vigatarāga spoke:%
\ I can't have enough of learning about [this teaching of] your[s on] Dharma.%
\footnote{It is not inconceivable that \skt{tava} is meant to carry the sense of an ablative,                as Kenji Takahashi has suggested to me:                `I can't have enough of learning about Dharma from you.'  }%
\ Teach me further than this, O great ascetic.%


\textbf{4.19}%
\ Anarthayajña spoke:%
\ Now listen to [my teaching about] stealing, O great Brahmin,%
\                                 which is taught to be of five kinds.%
\ Firstly, [listen to] theft [lit. `taking what has not been given'], then bribery,%
\ cheating with weights, cheating with scales, and the fifth kind, robbery.%


\textbf{4.20}%
\ Theft is when somebody else's wealth is taken away through a bold/impudent crime.%
\ [A person who commits such a crime] is foolish even if%
\                 he remains unnoticed [or: kept back from the crime?].%


\textbf{4.21}%
\ O great Brahmin, listen to bribery, which defiles Dharma.%
\ A sum of money taken in order to annul a punishment [or something that is%
\                                         to be done, in order to become exempt from a duty] is a bribe.%
\ Therefore this [also] should be considered as such%
\                 [i.e.\ as stealing because] it is committed out of greed.%
\footnote{Note \skt{asau} in pāda c as an accusative form. }%


\textbf{4.22}%
\ [Even if] somebody wants to protect families by the method of cheating with weights,%
\ that person should be considered a thief, because he takes away other people's wealth.%


\textbf{4.23}%
\ [The case is similar] if somebody takes away somebody else's belongings%
\                                          by the method of cheating with scales.%
\footnote{A line may have dropped out after pāda b, perhaps because a line                 similar to 4.22cd caused an eyeskip. Alternatively, this line may simply be                elliptical. }%
\ Other people, deceitful swindlers (\skt{kūṭa-kāpaṭika}) [can also] have the characteristics of thieves.%


\textbf{4.24}%
\ [If] someone, by deceit or by force, snatches away the wealth%
\ of weak and honest people or children [and women and simpletons?],%
\                         that morally corrupt thief is [rightly] called a thief.%


\textbf{4.25}%
\ There is no sin equal to stealing. There is no crime (\skt{adharma}) equal to it.%
\ There is no ill-fame comparable to that of being a thief.%
\                 There is no bad-conduct comparable to being a thief.%


\textbf{4.26}%
\ There is no such ignorance as stealing. There are no bigger rouges than thieves.%
\ There is nobody as ignorant as a thief. There is not a lazy person who is comparable to a thief.%


\textbf{4.27}%
\ There is nobody as detestable as a thief. There is nobody as much of an enemy as a thief.%
\ There is no such suffering as stealing. There is nobody more disgraced than a thief.%
\footnote{Note how \skt{stena} and \skt{steya} are used interchangeably (or chaotically)                        in the above passages in the MSS to denote both `thief' and 'theft/stealing'.                         The scribe of \msNc\ ends up writing \skt{stenya} in 4.27e. }%


\textbf{4.28}%
\ Some [thieves] take away [other people's] wealth in disguise, some in broad daylight.%
\footnote{It appears that \skt{hriyate} in pāda a is to be taken as an active verb (\skt{harate}).                 Note also how \msCb\ and \msNc\ read the same here. }%
\footnote{Take \skt{°hariṇo} in pāda b as singular and \skt{m} in \skt{'nya-m-adhamo} as a sandhi-bridge. }%
\ Other wicked people take money from deposits, and some people steal through fraud.%
\ Some gather wealth by forging documents, others steal from stolen money???%
\ Some people's wealth is from a purchased [child?? (\skt{krīta})].%
\                  ..............%
\                  These are considered the vilest.%


\textbf{4.29}%
\footnote{Understand \skt{stenastulya na mūḍham{ }asti} (the reading of \Ed!) as a `metri causa' version of                \skt{stenatulyo na mūḍho 'sti}, and see a similar case of a nominative ending                inside of compound in pāda c below. One major concern remains here:                the accepted reading here is that of \Ed, an edition that rarely emerges as                 the sole transmitter of the best reading. A solution could be                 to emend to \skt{stenaṃtulya...}, meaning `There is no bigger foolishness than theft',                but then the second part of pāda a is difficult to connect.                 

  }%
\ There are no bigger idiots than thieves, who are wicked people without Dharma and Artha.%
\ As long as he lives, he trembles in fear of the king, wailing.%
\ Having received his punishment, he gets into severe and [in]tolerable%
\                                  difficulties, propelled by [his] karma.%
\footnote{Understand \skt{prāptaḥśāsana tīvrasahyaviṣamaṃ} in pāda c as \skt{prāptaśāsanas tīvram asahyaṃ ca viṣamaṃ prāpnoti}.                Alternatively, understand \skt{tīvrasahya°} as \skt{duḥsahya°} (suggested by Törzsök).                

                The actual reading of \msCa, \skt{prāptaś} (lost in the process of normalization and standing                        in contrast with that of all other MSS that read \skt{prāptaḥ}) may suggest                        a doubling of the \skt{ś} of \skt{śāsana} metri causa (suggestion by Törzsök).                        More likely is that a licence of having a nominative ending inside of a compound                        is applied here, as probably above in pāda a (also remarked by Törzsök). }%
\ When his time comes, he dies and goes to hell, weeping vehemently.%


\textbf{4.30}%
\ Having spent ten million aeons of suffering,%
\                                  they emerge from hell to the state of animal existence.%
\footnote{Note °\skt{kalpa} for °\skt{kalpaṃ} metri causa. }%
\ Similarly [CHECK eka],%
\                 after roaming about in animal existence for a hundred and one times ten million years,%
\ then they reach the status of human existence on earth which is full of poverty and disease.%
\footnote{I understand \skt{vipule} as \skt{vipulāyāṃ}, \skt{vipulā} appearing in Amarakośa 2.1.7 as a synonym of                        \skt{dhātrī}, `earth'. }%
\ Then abandoning all one's karmas, the causes of suffering, one seeks refuge in Śiva.%
\footnote{Note the switch from plural to singular in pāda d. }%


\textbf{4.31}%
\ The one who is hostile towards the eight-formed Śiva, he who hurts his mother or father,%
\ he who is hostile towards cows or guests: these are the five types of cruel people.%
\footnote{Note \skt{pitur} and \skt{mātur} used as accusative forms in \skt{pāda} b, or alternatively                        understand: `who are hateful towards their fathers and mothers'. }%


\textbf{4.32}%
\ Śiva in his manifest form (\skt{sākṣāt}) is of eight forms, with the five elements (vyoman! NOTE),%
\                 and the Sun, the Moon, and the sacrificer. [He who] disgraces [any of these] is a cruel person.%
\footnote{See Śakuntalā 1.1:        

                        \skt{yā sṛṣṭiḥ sraṣṭur ādyā [1] vahati vidhihutaṃ yā havir [2] yā ca hotrī [3]
                        ye dve kālaṃ vidhattaḥ [4,5] śruti-viṣaya-guṇā yā [6] sthitā vyāpya viśvam \danda
                        yām āhuḥ sarva-bīja-prakṛtir [7] iti yayā prāṇinaḥ prāṇavantaḥ [8]
                        pratyakṣābhiḥ prapannas tanubhir avatu vas tābhir aṣṭābhir īśaḥ \twodanda}
                        
                 The eight \skt{tanu}s here are: [1] jala [2] agni [3] yajamāna [4,5] sūrya + candra [6] ākāśa [7] bhūmi [8] vāyu                

                For a similar interpretation of \skt{aṣṭamūrti}, see e.g. Īśānaśivagurudevapaddhati 2.29.34 (\skt{mantrapāda};                                note \skt{yajamāna} for our \skt{dīkṣa}):                                        \skt{kṣmā-vahni-yajamānārka-jala-vāyv-indu-puṣkaraiḥ}\danda                                        \skt{aṣṭābhir mūrtibhiḥ śambhor dvitīyāvaraṇaṃ smṛtam}\twodanda                (For \skt{puṣkara} as `sky, atmosphere', see e.g. Amarakośa 1.2.167:                        \skt{dyodivau dve striyām abhraṃ vyoma puṣkaram ambaram}.)                A closely related Aṣṭamūrti-hymn appears in Niśv mukha 1.30--41 (I owe thanks to Nirajan Kafle                        for drawing my attention to this); see Kafle 2018: 62, 63, 116, 119. Kafle notes                        that this hymn is closely parallel to some passages in the Prayogamañjarī (1.19--26),                        the Tantrasamuccaya (1.16--23), and the Īśānaśivagurudevapaddhati (kriyāpāda 26.56--63).                                        See also TAK I s.v. \skt{aṣṭamūrti}. }%


\textbf{4.33}%
\ The father is to be considered similar to the sky, he is the cause of one's birth.%
\                ....%


\textbf{4.34}%
\ The mother is more venerable than the earth. Who would not praise a mother?%
\ By that [praise], sacrifices, donations, austerities and [the study of] the Vedas, all will%
\                         be completed.%


\textbf{4.35}%
\ Cows are a sacred [auspicious/purifying Judit] blessing, they are the gods of the gods.%
\ Cows contain in themselves all the gods. That's exactly why one should not hurt them.%


\textbf{4.36}%
\ Cows are the protectors of the world as if the world were their new-born [calf],%
\                                         there is no doubt about it.%
\ The collection of [the five products of the cow, the \skt{pañcagavya},]%
\                         ghee, milk, curd, and [the cow's] urine and dung [is auspicious].%


\textbf{4.37}%
\ People who drink the five products of the cow,%
\                  the five nectars, the five holy and pure [substances] [or: clarified with a strainer??],%
\ will obtain the fruits of a horse sacrifice,%
\ and then reach the undecaying heavens.%


\textbf{4.38}%
\ There is no wealth comparable to [having] a cow.%
\ They yield milk, they draw [a plough etc.]. [As] they roam under the sky,%
\ feeding on grass, they issue nectar.%
\ When given to Brahmins, they deliver the family [from \skt{saṃsāra}/the suffering experienced in hell].%


\textbf{4.39}%
\ He who never fails to serve the cow daily [e.g. with a handful of grass],%
\ and he who tends to the cows' service,%
\ will obtain the merits of all sacrifices, austerities and donation [because]%
\                         he is one who is kind to it (\skt{tām}?) [i.e. to the cow].%


\textbf{4.40}%
\ He who looks after a guest, he who respects a guest,%
\ he who worships a guest, he who praises a guest,%
\footnote{Not the peculiar verb forms \skt{anugaccheta} and \skt{anupūjyeta}) in this verse. }%


\textbf{4.41}%
\ he who does not harm a guest, he who does not commit a fault towards a guest,%
\ he who does kind things to a guest, he who attends to the needs of a guest,%
\ he who makes a guest satisfied: his merits are endless.%


\textbf{4.42}%
\ He should offer [the guest] a seat, water-offering, feet-washing water [or: °pātreṇa?],%
\                 water for washing his feet[?],%
\footnote{Pāda b seems to awkwardly repeat what \skt{arghapādyena} in pāda a signifies.                Some emendation may be required here, perhaps taking into account bathing (\skt{snāna}) or                         an unguent (\skt{abhyaṅga}). }%
\ or gifts of food and clothes, or all [of these].%


\textbf{4.43}%
\ He who worships the guest by [offering him] his own son, wife or himself%
\footnote{For the requirement that one could part with his wife or son, or his own life,                for the benefit of someone else, see VSS 2.38 and the narrative in VSS chapter 12;                these influenced my decision to emend \skt{°ātmano} to \skt{°ātmanā} in pāda a. }%
\ with willingness and with a brave and non-hesitating mind,%


\textbf{4.44}%
\ and does not ask [the guests about their] lineage, Vedic affiliation (\skt{caraṇa}),%
\                                         studies, country or birth,%
\ and imagines mentally, with devotion, that it is Dharma himself who has arrived,%


\textbf{4.45}%
\ [will obtain all the fruits of] thousands of Aśvamedha sacrifices and hundreds%
\                                         of Rājasūya sacrifices,%
\ a thousand Puṇḍarīka sacrifices and the fruit of [visiting] all the pilgrimage places and%
\                 [performing] all the austerities;%


\textbf{4.46}%
\ he whose guest is satisfied [and] he who can abandon the sentiment of cruelty,%
\footnote{The demonstrative pronoun \skt{tasya} in pāda c may refer to the guest:              `he will obtain all his [i.e. the guest's] merits', hinting at some sort of karmic exchange.                Nevertheless, I think that \skt{tasya} points at the merits one can obtain by rituals listed                 in the previous verse. This is suggested by passages such as the following:                

                Mahābhārata Supp. 13.14.379 ff.:
                   <skt>ahany ahani yo dadyāt kapilāṃ dvādaśīḥ samāḥi\danda
                        māsi māsi ca satreṇa yo yajeta sadā naraḥ\twodanda 
                        gavāṃ śatasahasraṃ ca yo dadyāj jyeṣṭhapuṣkare\danda 
                        na taddharmaphalaṃ tulyam <b>atithir yasya tuṣyati</b>\twodanda</skt>                

                 Brahmavaivartapurāṇa 3.44--46:
                   <skt>atithiḥ pūjito yena pūjitāḥ sarvadevatāḥ\danda
                        <b>atithir yasya santuṣṭas</b> tasya tuṣṭo hariḥ svayam\twodanda
                        snānena sarvatīrtheṣu sarvadānena yat phalam\danda
                          sarvavratopavāsena sarvayajñeṣu dīkṣayā\twodanda 
                        sarvais tapobhir vividhair nityair naimittikādibhiḥ\danda  
                        tad evātithisevāyāḥ kalāṃ nārhanti ṣoḍaśīm\twodanda</skt>          }%
\ will obtain all the merits of [the above], there is no doubt about it.%


\textbf{4.47}%
\ ... he who [does not] know [how to greet his] guests ... will never reach the path ... ?%
\ Therefore one should go up to the arriving guest with respectfully joined palms.%


\textbf{4.48}%
\ By one \skt{prastha} of coarsely ground grains%
\footnote{This verse is a reference to the story related by a mongoose in MBh 14.92--93:                 A Brahmin who practises the vow of gleaning (\skt{uñcha}) and his family                receive a guest. They feed the guest with the last morsels of the little food                they have. In the end, the guest reveals that he is in fact Dharma (14.93.80cd) and as                 a reward the family departs to heaven. The noble act of the poor Brahmin and his family                is depicted as yielding greater rewards than Yudhiṣṭhira's grandiose horse-sacrifice.                 (See some remarks on this story in Takahashi 2021.) }%
\ given to a guest, an extremely great sacrifice was performed [so to say],%
\         and his [the Brahmin's and his family members'] bodies (\skt{svaśarīraṃ}) reached heaven.%
\footnote{

We would be forced to accept the reading of \Ed\ in pāda d if the expression                were in the masculine (\skt{saśarīro divaṃ gataḥ}). This would make sense                and it would also echo expressions occuring e.g. in the Mahābhārata:                3.164.33cd: <skt>paśya puṇyakṛtāṃ lokān saśarīro divaṃ vraja</skt>;                14.5.10cd:  <skt>saṃjīvya kālam iṣṭaṃ ca saśarīro divaṃ gataḥ</skt>.                It is tempting to emend the pāda accordingly, but I have retained                         \skt{svaśarīraṃ divaṃ gatam} and I interpret it as                         referring to the Brahmin's whole family (\skt{sva}).                 }%


\textbf{4.49}%
\ The mongoose related [this story in the Mahābhārata] in the past in detail, O great Brahmin,%
\ and you've known it already. The story of the \skt{prastha} is well-known.%


\textbf{4.50}%
\ Self-restraint of humans is in itself the collected essence of Dharma.%
\ Self-restraint is Dharma, Self-restraint is heaven,%
\                  Self-restraint is fame, Self-restraint is happiness.%


\textbf{4.51}%
\  Self-restraint is sacrifice, Self-restraint is a pilgrimage-place,%
\                   Self-restraint is merit, Self-restraint is religious austerity.%
\ If one has no Self-restraint, there is no Dharma,%
\                  [while] Self-restraint yields a multitude of desired objects.%


\textbf{4.52}%
\ The elephant, the fish, the moth, the bee and the deer are without Self-restraint.%
\footnote{Note \skt{kari} for \skt{karī} metri causa, and the end of pāda b (\skt{°mṛgāḥ}), which                         should be treated metrically as if it read \skt{°mrigāḥ}. }%
\ The senses are the skin, the tongue, the nose, the eye and the ear.%


\textbf{4.53}%
\ Each of these sense faculties are hard to conquer and%
\                         all are known to be fatal [if unconquered].%
\ If one masters Self-restraint, the [one with a?] lack of Self-restraint will die.????%


\textbf{4.54}%
\ In the case of the deer, death comes about because of hearing [when hunters use buck grunts].%
\                  Moths die because[?] of their eyes [as they are attracted to the light of a lamp].%
\ Bees perish because of their smelling, fish because of their tongues.%


\textbf{4.55}%
\ The elephant perishes because of touch, not being able to tolerate being in fetters [?].%
\ How much more true it is for those who enjoy all five [senses]!%
\                  Why should death come as a surprise for them?%


\textbf{4.56}%
\footnote{Purūravas (double sandhi originally? purūravās ati° -- purūravā ati° -- purūravāti°).        Pāda a may refer to the following passage in the Mahābhārata (1.70.16--18, 20ab):        
                                 <skt>purūravās tato vidvān ilāyāṃ samapadyata\danda
                                 sā vai tasyābhavan mātā pitā ceti hi naḥ śrutam\twodanda
                                 trayodaśa samudrasya dvīpān aśnan purūravāḥ\danda
                                 amānuṣair vṛtaḥ sattvair mānuṣaḥ san mahāyaśāḥ\twodanda
                                 vipraiḥ sa vigrahaṃ cakre vīryonmattaḥ purūravāḥ\danda
                                 jahāra ca sa viprāṇāṃ ratnāny utkrośatām api\twodanda
                                        ... 
                                 tato maharṣibhiḥ kruddhaiḥ śaptaḥ sadyo vyanaśyata\danda</skt>        

                        (``The wise Purūravas was born to Ilā. We heard that Ilā                                 was both his mother and his father.                            The great Purūravas ruled over thirteen islands of the ocean                           and, though human, he was always surrounded by superhuman beings.                           Intoxicated with his power, Purūravas quarrelled with some Brahmins                             and robbed them of their wealth even though they were protesting. [...]                           Therefore, cursed be the great Ṛṣis, he perished.'')        

        See also Buddhacarita 11.15 (Aiḍa = Purūravas):

                <skt> aiḍaś ca rājā tridivaṃ vigāhya
                         nītvāpi devīṃ vaśam urvaśīṃ tām\danda
                      lobhād ṛṣibhyaḥ kanakaṃ jihīrṣur  
                        jagāma nāśaṃ viṣayeṣv atṛptaḥ\twodanda</skt>        

                For Daṇḍa(ka)'s story, see Rāmāyaṇa 7.71.31 ff.:                Daṇḍa meets Arajā, a beautiful girl, in a forest and rapes her. As a consequence, her father, Śukra/Bhārgava,                destroyes Daṇḍa's kingdom, which thus becomes the desolate Daṇḍaka-forest.                   }%
\ Purūravas [perished] by excessive greed, Daṇḍaka by excessive desire,%
\ Sagara's sons by excessive pride, Rāvaṇa by excessive haughtiness,%
\footnote{

For two versions of the destruction of                Sagara's sons, who were chasing the sacrificial horse of their father's Aśvamedha sacrifice,                and by doing so disturbed Kapila's meditation, and who in turn burnt them to ashes,                see Mahābhārata 3.105.9 ff. and Brahmāṇḍapurāṇa 2.52--53.                

                As for Rāvaṇa's haughtiness,                especially the fact that he chose to be invincible by all creatures except humans,                and its consequences,                one should recall the story of the Rāmāyaṇa and Rāvaṇa's destruction brought about by Rāma therein. }%


\textbf{4.57}%
\ Saudāsa by excessive anger, the Yādavas by excessive drinking,%
\footnote{Saudāsa, also known as Kalmāṣapāda, hit Śakti, Vasiṣṭha's son, with a whip because                the latter did not give way to him, and as a consequence Śakti cursed Saudāsa:                Saudāsa had to roam the world as a Rākṣasa for twelve years.                 See Mahābhārata 1.166.1 ff.                

                As for the end of the Yādavas, see the short Mausalaparvan of the Mahābhārata (canto 16):                cursed by the sages Viśvāmitra, Kaṇva and Nārada, and seeing menacing omens,                the Yādavas take to drinking in Prabhāsa and destroy each other.         }%
\ Māndhātṛ by excessive desire, Nahuṣa by contempt for Brahmins,%
\footnote{

Most probably, \skt{atitṛṣṇā} in the MSS stand for \skt{atitṛṣṇāt} (intending \skt{atitṛṣṇayā}).                The form \skt{māndhāto} in \msCb\ stands for \skt{māndhātā} (nominative of \skt{māndhātṛ}).                I have corrected it in spite of the fact that the authors' knowledge about his story may                come from Divyāvadāna 17, where it sometimes appears to be an a-stem noun (\skt{māndāta}).                \skt{dvijavajñayā} in \skt{pāda} d stands for \skt{dvijāvajñayā} metri causa.                

                Māndhātṛ was born from his father's body who, being excessively thirsty once,                had drank some decoction prepared for ritual purposes and as a result become pregnant with him.                Nevertheless, Buddhacarita 11.13 suggests that Māndhātṛ himself was still unsatisfied                with wordly objects even after he had obtained half of Indra's throne:                

                <skt>                devena vṛṣṭe 'pi hiraṇyavarṣe  
                                        dvīpān samagrāṃś caturo 'pi jitvā\danda  
                                     śakrasya cārdhāsanam apy avāpya   
                                        māndhātur āsīd viṣayeṣv atṛptiḥ\twodanda</skt> 

                In fact, as Monika Zin points out (2012: 149) Māndhātṛ/Māndhāta's rise and fall is a very popular theme                in the `Narrative Art of the Amaravati School':                 ``Statistics show that in the Amaravati School the most frequently represented narrative is                  the story of King Māndhātar, which appears 47 times.''        See ibid. p. 151:        ''The story [e.g. <i>Divyāvadāna</i> XVII, see more sources in fn. 17 of this article]         relates that Māndhātar was a miraculously born <i>cakravartin</i> with Seven        Jewels who could cause rain to fall so that his subjects could prosper; not usual rain, but rain        of coins, of grain or of cloth. By virtue of his moral strength alone, Māndhātar conquered the world -        without any weapons. He conquered all the countries on earth, then Uttarakuru,        Pūrvavideha and Aparagodānīya, after which he set out to conquer the heavens. When he was        traversing from one abode of the gods to the next (Nāgas, Sadāmattas, Mālādharas, etc.)        groups of gods pledged obeisance to him and immediately marched in front of his troops.        Māndhātar reached the splendid city of the Trayastriṃśa gods atop Sumeru, where Indra, in        the meeting-hall, bequeathed to him half of his own seat and half of his heavenly realm.        Māndhātar then ruled together with Indra for an unimaginable period of time during which 36        Indras changed. One day, shortly after he won a battle against the Asuras, a sinful thought        came to his mind: why should he rule alongside Indra? It was he, after all, who won the war,        not Indra - he was better and should, therefore, rule alone.        At that very moment Māndhatar fell from heaven, down to his former realm, became sick and died.        Shortly before his death, he preached a sermon to his subjects in which <i>gātha</i>s         from the <i>Dhammapada</i> (186--187) appear...''                 

                Nahuṣa was elevated to the position of Indra for a period of time and he also wanted                to take Śacī, Indra's wife. Indra instructed Śacī to tell Nahuṣa to                 harness some Ṛsis to a vehicle and use this vehicle to take Śacī.                 Agastya, one of the Ṛṣis, was insulted even further by Nahuṣa, therefore                he cursed Nahuṣa, who then fell from the vehicle. See Mahābhārata 12.329.35 ff. and                the verse in the Buddhacarita (11.14) that follows the one about Māndhātṛ:        

                                <skt>        bhuktvāpi rājyaṃ divi devatānāṃ   
                                                śatakratau vṛtrabhayāt pranaṣṭe\danda
                                             darpān maharṣīn api vāhayitvā  
                                                kāmeṣv atṛpto nahuṣaḥ papāta\twodanda</skt>                 }%


\textbf{4.58}%
\ [Mahā]bali perished by excessive donations, Arjuna by excessive heroism,%
\footnote{Pāda a is most probably a reference to Mahābali's promises made to Vāmana that caused his fall.                 Arjuna: the exile? Flo Kirāṭārjunīya?? he killed Bhīṣma? Flo   }%
\ King Nala by excessive gambling, Nṛga by taking a cow.%
\footnote{

 King Nala was an expert in the game of dice and lost his kingdom to Puṣkara in                             a game. See e.g. Mahābhārata 3.56.1 ff.                 

                       As for Nṛga, see Mahābhārata 14.93.74: 
                       <skt>                       gopradānasahasrāṇi dvijebhyo 'dān nṛgo nṛpaḥ\danda
                       ekāṃ dattvā sa pārakyāṃ narakaṃ samavāptavān\twodanda
                                               </skt>                               (``King Nṛga had made gifts of thousands of cows for the twice-born.                                 By giving away one single cow that belonged to someone else,                                  he fell into hell.'')                 }%


\textbf{4.59}%
\ [For] a person who is without Self-restraint, O great Brahmin,%
\ there is no heaven, liberation or happiness.%
\footnote{Note how flexible the gender of most nouns is in pāda b:                         \skt{svarga}, \skt{mokṣa} and \skt{dama} are usually masculine in standard Sanskrit. }%
\footnote{The majority of the witnesses suggest that pāda c ends in a stem form noun (\skt{°nāśa}).                This pāda is unmetrical, or rather it applies the licence of a word-final                short syllable being counted as potentially long (\skt{°dharMA°}).  }%
\ O Brahmin, people without Self-restraint%
\                         are the destruction of knowledge, Dharma, family and fame.%
\footnote{Note how \skt{viprā} in pāda d is probably an attempt in some MSS to restore the metre.                This pāda is also unmetrical, or rather it applies the licence of a word-final                short syllable being counted as potentially long (\skt{viPRA}). }%


\textbf{4.60}%
\ [For] a person without taboos there is neither the other world, nor this life.%
\ In the case of a person without taboos there is no Dharma or religious austerity.%


\textbf{4.61}%
\ These five are taboo:%
\                  women who are not depending on oneself, others' wealth, taking away others' lives,%
\                  hurting others and [consuming] others' food.%


\textbf{4.62}%
\ Listen, O great Brahmin, the wise should always treat women who are not%
\                         dependent on oneself as taboo,%
\ [be she] a queen, a Brahmin's wife, a wandering religious mendicant,%
\                         a relative or of another family.%


\textbf{4.63}%
\ Listen further to something else with regards to others' wealth.%
\                  [It may include] gaining wealth through unlawful means,%
\ when somebody takes away other people's wealth by cheating with%
\                 [small] weights of an \skt{āḍha[ka]} or a \skt{prastha} and with scales.%


\textbf{4.64}%
\ O Brahmin, the wise should regard the taking away [of others'] lives as taboo.%
\ Wild and domesticated animals, [serpents] that live in holes and those that%
\                 walk on their feet [are examples of life forms not to destroy].%
\footnote{In pāda d, understand \skt{caraṇācara} as \skt{caraṇacara} (metri causa). }%


\textbf{4.65}%
\ And what is the hurting of others? Listen, O Brahmin, I'll tell you briefly.%
\ He who is hostile to the gods, Brahmins, gurus, mothers and guests [hurts others].%
\footnote{Note \skt{mātā} as a stem form. }%


\textbf{4.66}%
\ As regards other people's food, eating together with people whose food%
\                         is not to be accepted (\skt{abhojyeṣu}) is taboo,%
\ [e.g.] after birth or death [in the family], in case there are vendors of alcohol,%
\                 in the case of a family having lost their caste, and in the case of a Naṭa [dancer caste?].%
\footnote{One should probably understand \skt{śauṇḍe} in pāda c as \skt{śauṇḍike} (alternatively,                it may be corrupted from \skt{ṣaṇḍhe}); see both in Vāsiṣṭhadharmaśāstra 14.1--3:                

                <skt>athāto bhojyābhojyaṃ ca varṇayiṣyāmaḥ\danda                cikitsaka-mṛgayu-puṃścalī-ḍaṇḍika-stenābhiśastar-ṣaṇḍha-patitānām annam abhojyam\danda                kadarya-dīkṣita-baddhātura-somavikrayi-takṣa-rajaka-śauṇḍika-sūcaka-vārdhuṣika-carmāvakṛntānām\twodanda</skt> etc.                

                In Olivelle's translation (DhSūtras 1999: 285):                        ``Next we will describe food that is fit and food that is                        unfit to be eaten [...] The following are unfit                        to be eaten: food given by a physician, a hunter, a harlot, a law                        enforcement agent, a thief, a heinous sinner [...] a                        eunuch, or an outcaste; as also that given by a miser, a man                        consecrated for a sacrifice, a prisoner, a sick person, a man who                        sells Soma, a carpenter, a washerman, a liquor dealer, a spy, an                        usurer, a leather worker...''                

                In support of reading \skt{ṣaṇḍhe}, see Manu 3.239:                

                        <skt>cāṇḍālaś ca varāhaś ca kukkuṭaḥ śvā tathaiva ca\danda
                        rajasvalā ca ṣaṇḍhaś ca nekṣerann aśnato dvijān\twodanda</skt> }%


\textbf{4.67}%
\ Those people who cling to [the prohibition of] the five kinds of taboo [and thus]%
\                         seek heaven, wealth and liberation,%
\ will reach eternal faultlessness in this world, embellished with fame and glory.%
\footnote{Understand \skt{kīrtir yaśo°} as \skt{kīrtiyaśo°} ('r' being an intrusive consonant here metri causa). }%
\ [A person like that] will obtain wisdom, intelligence, [knowledge of] the Śruti and Smṛti traditions,%
\                  and honour forever.%
\ He will be kindness itself[?] and he will obtain an extra long life, no doubt.%
\footnote{Understand \skt{āyuṣa} as \skt{āyuṣaṃ} (metri causa). }%


\textbf{4.68}%
\ The four cases of observing silence, [victory over] the four enemies, the four%
\                         sanctuaries/planes,%
\ the four meditations, and the four legged [Dharma] are called the five ways of being virtuous[?].%


\textbf{4.69}%
\ I shall tell you about the four cases of observing silence. Listen, be attentive.%
\ One should avoid [1] violent [words], [2] slanderous [words], [3] lies, and [4] idle [talk].%
\footnote{Is \skt{sambhinna} a Buddhist term? See also Dharmaputrikā 1.31. }%


\textbf{4.70}%
\ The fourfold enemy, desire, anger, greed and delusion,%
\ is to be destroyed. He who destroys [these] enemies will become sinless.%
\footnote{Possible direct sources for the idea that \skt{kāma} is an enemy to be defeated include                Buddhacarita 11.17:

        <skt>cīrāmbarā mūlaphalāmbubhakṣā
              jaṭā vahanto 'pi bhujaṃgadīrghāḥ\danda
              yair nānyakāryā munayo 'pi bhagnāḥ
              kaḥ kāmasaṃjñān mṛgayeta śatrūn\twodanda</skt>

         and Bhagavadītā 3.43:        

        <skt>evaṃ buddheḥ paraṃ buddhvā saṃstabhyātmānam ātmanā\danda
             jahi śatruṃ mahābāho kāmarūpaṃ durāsadam\twodanda</skt> }%


\textbf{4.71}%
\ I shall teach you the four sanctuaries/planes. Listen, O Brahmin.%
\ Compassion, sympathy in joy, indifference, and benevolence are the four sanctuaries/planes.%
\footnote{Is \skt{āyatana} just a synonym of \skt{vihāra} here or                 could this use of the term \skt{āyatana} for the four Buddhist                 \skt{brahmavihāra}s have been influenced by the following passage in the Dharmasamuccaya (date?)?        

                        <skt>mokṣasyāyatanāni ṣaṭ\danda
                        apramādas tathā śraddhā vīryārambhas tathā dhṛtiḥ\danda
                        jñānābhyāsaḥ saṃtāśleṣo mokṣasyāyatanāni ṣaṭ\twodanda1.3\twodanda
                        nava śāntisamprāptihetavaḥ\danda</br>                        dānaṃ śīlaṃ damaḥ kṣāntir maitrī bhūteṣv ahiṃsatā\danda
                        karuṇāmuditopekṣā śāntisamprāptihetavaḥ\twodanda1.4\twodanda
</skt> }%


\textbf{4.72}%
\ I shall now teach you the four meditations,%
\                 which will liberate you from mundane existence (\skt{saṃsāra}).%
\footnote{Note the stem form \skt{dhyāna} in \skt{°dhyānādhunā} (for \skt{°dhyānam adhunā}) in pāda a. }%
\ Meditation is taught to be fourfold: of the Self, \skt{vidyā}, \skt{bhava} [= Śiva?]%
\                                         and the subtle one.%
\footnote{For contrast, see VSS 6.8:                

                <skt>dhyānaṃ pañcavidhaṃ caiva kīrtitaṃ hariṇā purā\danda
                     sūryaḥ somo 'gni sphaṭikaḥ sūkṣmaṃ tattvaṃ ca pañcamam\twodanda</skt> }%


\textbf{4.73}%
\ The \skt{tattva} of the Self is the \skt{ātman}. \skt{Vidyā} in the five in a fivefold way[??].%
\ They call the thirty-sixth the imperishable one,%
\                         [and] the subtle \skt{tattva} has no attributes.%
\footnote{If pāda c is indeed a reference to a 36-tattva philosophical system,                it is in striking contrast with the 25-tattva system described in VSS chapter 20. }%


\textbf{4.74}%
\ Dharma is said to be four-legged [as] it rests on the four \skt{āśrama}s,%
\ [those of] the householder, the chaste one, the forest-dweller and the mendicant.%


\textbf{4.75}%
\ Virtuous are those who know these thoroughly, O great Brahmin.%
\footnote{Note the plural instrumental (\skt{yair}) with a singular active verb (\skt{vetti}). }%
\ [They will experience] the purification of all sins and the growth of merits.%


\textbf{4.76}%
\ One's life-span, fame and glory and happiness grow only through virtue (\skt{dhanya}).%
\ [In] a virtuous person piece, prosperity, memory/tradition? and intelligence%
\                         will arise.%


\textbf{4.77}%
\ There are five areas of negligence. I shall teach them to you, listen.%
\footnote{Note the stem form noun in pāda a (\skt{°sthāna}) metri causa, and also                         that this stem form noun may function as a singular noun                        next to a number (\skt{pañca}), a frequently seen phenomenon in this text. }%
\ Murdering a Brahmin, drinking alcohol, stealing, having sex with the guru's wife:%
\ they call these Grievous Sins. The fifth is when one is connected with them%
\                 [i.e. with these sins or with people involved in these sinful acts].%
\footnote{Note how \skt{pāda} f deviates from Manu. }%


\textbf{4.78}%
\ A lie concerning one's superiority, a slander that reaches the king's ear,%
\ and false accusations against an elder are equal to killing a Brahmin.%
\footnote{The translation of this verse is based on Olivelle's (Olivelle Crit Ed. p. 218). }%


\textbf{4.79}%
\ Defaming a Brahmin or the Ṛgveda, being a false witness, murdering a friend,%
\ eating unfit or forbidden food are six [deeds that are] equal to drinking alcohol.%


\textbf{4.80}%
\ Sexual intercourse with a female relative, with an unmarried girl,%
\                  with women of the lowest castes,%
\ with the wife of a friend or of one's own son are said to be equal to violating the guru's bed.%


\textbf{4.81}%
\ Stealing/taking away deposits, people, horses, silver,%
\ land, diamonds, or gems are said to be equal to stealing gold.%


\textbf{4.82}%
\ If a man takes parts in these four [i.e. \skt{brahmahatyā, surāpāna, stena, gurvaṅganāgama}],%
\ that is the fifth Grievous Sin. By this all [of them] have been explained.%
\ These five kinds of negligence are to be avoided, O great Brahmin.%
\footnote{Note syntax. }%


\textbf{4.83}%
\ [Charm has five types:] bodily, verbal and mental charm,%
\                         [charm of] the eyes and [of one's] thoughts pañcamaḥ.%
\footnote{My emendation from \skt{°manasā dhūryaś} to \skt{°mana-mādhuryaś} is based on the fact that following the list                of \skt{yama}s in 3.16cd--17ab, we need some reference to \skt{mādhurya} here and that it is easy to see how this                 corruption came about: \skt{°mano-mādhurya°} would be unmetrical, thus the form \skt{°mana-mādhurya};                        \skt{°mana-mā°} is easily corrupted to \skt{°manasā°} (not to mention the fact                         that \skt{manasā} comes up in the next verse);                         in addition we need five items in this line because of \skt{pañcamaḥ}.                        As always, I correct \skt{mādhūrya} to \skt{mādhurya}, although it seems that                         the former is acceptable in this text.                         I did not correct \skt{mādhuryaś} to \skt{mādhuryaṃ} because of the corresponding                        \skt{pañcamaḥ}. }%
\ Giving [others] a friendly glance [is commendable] and one should avoid cruel thoughts.%


\textbf{4.84}%
\ One should meditate with a tranquil mind and should speak [to other people using] gentle words.%
\ [When] respectable people arrive at one's own hermitage,%
\                        [one should] present them with as many gifts as one can,%


\textbf{4.85}%
\ with gifts of fire-wood, water and fire.%
\footnote{Understand \skt{jātavedam} in pāda b as \skt{jātavedasam} or \skt{jātavedāḥ},                or rather as belonging to the compound \skt{°dānaṃ}: \skt{jātavedodānaṃ}. }%
\ [If] fire-wood, fire and water are easily available [but] are not given [as gift]%
\ or [if the phrase] `Live [for a hundred years]!' is not uttered [by him] when [somebody else]%
\                 sneezes, what reward could there be for him in the afterlife?%
\footnote{For pāda e, see Mahāsubhāṣitasaṃgraha 2558:                 <skt>amṛtāyatām iti vadet pīte bhukte kṣute ca śataṃ jīva</skt>                (`When eating or drinking, one should say: "Let it turn into nectar!";                  and after sneezing: "Live for a hundred years!".')  }%


\textbf{4.86}%
\ The sages who see the truth praise five types of sincerity.%
\ [Sincerity] in action, in livelihood, in prosperity, in gratifying others [and ...?].%
\ A sincere person does not rejoice in women, wealth, bribery and property.%


\textbf{4.87}%
\ Sincerity [means] no sacrifice [performed] idly. Sincerity [means] no austerity [performed] idly.%
\ Sincerity [means] no donation [given] idly. Sincerity [means] no fires [kindled] idly.%


\textbf{4.88}%
\ The sense faculties of a sincere person are firm even when he is delighted.%
\ The gods always live inside the body of a sincere person.%


\textbf{4.89}%
\ Thus has been taught this section on the \skt{yama}-rules, O great Brahmin.%
\footnote{In pāda a \skt{°pra°} does not make the previous syllable long: this is the phenomenon of                `muta cum liquida', one of the hallmarks of the \skt{Vṛṣasārasaṃgraha},                 that is, syllables such as \skt{tra, pra, bra, dra} do not necessarily make the                 previous syllable long. }%
\ Humans should follow them to reach happiness here and in the other world.%
\footnote{In pāda b, \skt{parata} most probably stands for \skt{paratra} or \skt{parataḥ} metri causa.         We may correct it to \skt{paratra} (`muta cum liquida'). }%
\ He'll live by Śaṅkara's command with his filth of sins destroyed.%
\footnote{\skt{°malapahārī} in the MSS stands either for \skt{°malāpahārī} or \skt{°malaprahārī} metri causa.                 I could have choosen to emend it to \skt{°malaprahārī} (`muta cum liquida' again),                but I decided not to because \skt{apahārin}, \skt{apahāra}                \skt{apahāraka} are used in the text very frequently. See also 8.XX, which contains a very similar expression:                        \skt{sakalamalapahāre dharmapañcāśad etat}. }%
\ He'll become a ruler of the world [that he subjugates] under one royal umbrella.%
\vfill\pagebreak\begin{center}{\large\textbf{ Chapter Five 
}}\end{center}


\textbf{5.1}%
\ Vigatarāga spoke:%
\ [Please] now teach me the true nature of the Niyama-rules in detail.%
\ It is comparable to a speech of ambrosia. I have become curious to hear [it].%
\ [It was?] burnt by the fire of Prakṛti, sprinkled with the water of knowledge[?!].%
\ There is no satisfaction [yet] in the Dharmas [for me]. ...[perhaph%
\                         \skt{apara-vadam ataj-jñā... or apara[ṃ] vada me tajjña? mata-jña?}].%


\textbf{5.2}%
\ Anarthayajña spoke:%
\ I shall teach you something else that is nice to hear, O great Brahmin:%
\footnote{In \skt{pāda} a, \skt{anyat} is a bit strange, but it could be echoing \skt{apara} above in 5.1d. }%
\ the particular part[s, for kalā; or for kalpa?] of Niyama are of five types [each].%
\ It is the essence of Dharma, dear to Hari, Hara and the sages, O great Brahmin,%
\ the destruction of the impurity of the Kali age, generally[?] known as liberation.%


\textbf{5.3}%
\ Purification, sacrifice, penance, donation, Vedic study and the restraint of sexual desire,%
\ religious observances, fasting, taciturnity, and bathing: these are the ten Niyamas.%


\textbf{5.4}%
\ From among these, now I shall tell you the particulars of purification [first], and [then] the others.%
\ [1] Bodily purity, [2] [purity of] food, [3] [purity of] property[?],%
\                  [4] [purity of] conduct[?], and the fifth, [5]...?%


\textbf{5.5}%
\ He should not beat or tie or kill [any living being].%
\ When this concerns others' wives and property, it is called bodily purity.%


\textbf{5.6}%
\ The cleanliness of the ears, O great Brahmin, and of the anus, the loins, the mouth etc.%
\                 [is also bodily purity].%
\ The purity of the mouth [comes from] sipping water when eating, speaking,%


\textbf{5.7}%
\ [after] the emission of urine and faeces, and [before] the worship of gods.%
\ The wise one should clean his anus and his loins with clay and water.%
\footnote{Note [or emend?] the form \skt{śaucayīta}.  }%


\textbf{5.8}%
\ One [portion of clay] for the loins, five for the anus, and ten for one [the left] hand.%
\ [Then] seven is to be applied for both [hands] by him who wishes cleanliness with clay.%


\textbf{5.9}%
\ This is the purification for the householder (\skt{gṛhastha}), twice as much for the chaste one (\skt{brahmacārin}),%
\                  three times as much for the forest-dweller (\skt{vānaprastha}), four times as much for the ascetic (\skt{yati}).%


\textbf{5.10}%
\ I shall teach you the rules of purity with food. Listen, pay great attention.%
\ He should eat [as much] food [that fills] two quarters [of the stomach] and drink water [that fills] one quarter.%
\                  In order to be able to practise breath-control, he should save the remaining quarter.%
\footnote{For similar instructions, see a verse cited in Śaṅkara's commentary ad BhG 6.16:                                \skt{uktaṃ hi\danda                                 ardhaṃ savyañjanānnasya tṛtīyam{ }udakasya ca\danda                                 vāyoḥ saṃcaraṇārthaṃ tu caturtham{ }avaśeṣayet\twodanda}                               (``Half is for food with sauce, the third part for water,                               but in order to be able to move the air, he should leave the fourth part [empty].'')                See also e.g. Aṣṭāṅgahṛdaya 8.46cd-47ab:                                        \skt{annena kukṣer dvāv aṃśau pānenaikaṃ prapūrayet\twodanda                                         āśrayaṃ pavanādīnāṃ caturtham avaśeṣayet\danda}                and Sannyāsopaniṣad 59:                                         \skt{āhārasya ca bhāgau dvau tṛtīyam udakasya ca\danda                                         vāyoḥ saṃcaraṇārthāya caturtham avaśeṣayet\twodanda}  }%


\textbf{5.11}%
\ [By] the wise one['s applying] the six soft and sweet juices, [which are] the six juices in food,%
\                 the disturbances of the \skt{dhātu}s and the terrible illnesses will disappear.%


\textbf{5.12}%
\ He should not eat foods that are forbidden and he should not drink drinks that are forbidden.%
\                  He should not go where he is not allowed to and he should not say what is improper.%


\textbf{5.13}%
\ He should avoid garlic, onion, \skt{gṛñjana} onion, mushrooms,%
\                 buffalo meat? and pork, following the rules.%


\textbf{5.14}%
\ He should not eat \skt{chattrāka} mushrooms, village hog, and cow flesh.%
\                 He should also avoid sparrows, pigeons, and water-birds.%


\textbf{5.15}%
\ He should also avoid [eating] swans, cranes, \skt{cakravāka} birds, dogs, parrots and hawks,%
\                 crows, owls, \skt{balāka} cranes, fish etc.%


\textbf{5.16}%
\ He should avoid everything that is ritually impure or polluted.%
\                He should also completely avoid those vegetables, roots and fruits that are prohibited.%


\textbf{5.17}%
\footnote{Understant \skt{°śaivabhāratasaṃhite} as  \skt{śaive bhāratasaṃhitāyāṃ}.  }%
\ In the books of Manu, in the Purāṇas, in Śaiva texts, and in the Bhāratasaṃhitā (= the Mahābhārata),%
\                  the practice of purity is definitely expanded in full.%


\textbf{5.18}%
\ Now you have asked me [? about it], and I taught it [to you] in a condensed form.%
\                  He who speaks the truth is pure. He who engages in yogic meditation is pure.%


\textbf{5.19}%
\ He who avoids violence and is restrained is pure.%
\                 He whose patience has become compassion is pure[???].%
\ Of all the [ways of] purification, material purification is taught to be the highest.%


\textbf{5.20}%
\ For he who is pure with regards to material things is truly pure,%
\                 and not he who [only] uses clay and water [i.e. who performs only ordinary baths].%
\                 When purification pertains to the body, to speech and to the mind, that is purity%
\                 of all things.%


\textbf{5.21}%
\ If a person knows the rules of purity and impurity, he will surely (niścayaṃ?)%
\                 gain happiness at the end of time, eternally embellished with glory and fame.%
\                 He has reached here in this world all the merits that the books on true Dharma teach,%
\         i       and at the end of his life he will undoubtedly reach the desired path in the other world.%
\footnote{Note the stem form adjective \skt{°jña} and noun \skt{°mānava} metri causa,                 the second syllable of \skt{yadi} as a long syllable at the caesure, the plural \skt{āpnuvanti} where one would expect a verb                    in the singular, \skt{kīrtir} metri causa for a compounded stem form (\skt{kīrti°}),                and the sandhi-bridge \skt{-m-} in \skt{paratra-m-īhita°}.  }%
\vfill\pagebreak\begin{center}{\large\textbf{ Chapter Six 
}}\end{center}


\textbf{6.1}%
\ [Anarthayajña spoke:]%
\footnote{Maybe ījyāṃ is to be accepted. No, see 5.3a. }%
\ Now I shall teach you the five types of sacrifice, O excellent Brahmin,%
\                       for [your] success in Dharma and liberation. Listen carefully, O Brahmin!%


\textbf{6.2}%
\ Material sacrifice, sacrifice through work, sacrifice through recitation,%
\                         knowledge and meditation: I shall teach you these five one by one.%
\footnote{Note pañcaitat for pañcaitāni or pañcete. }%


\textbf{6.3}%
\ Material sacrifice includes the following:%
\                 the worship of fire etc., the performance of the ritual of Agnihotra,%
\                 oblations on the eight day after full moon,%
\                 oblations offered at new and full moons, and%
\                 the rituals for the ancestors.%
\footnote{See Dharmasūtras, Niśv book, Kiraṇa, Svacchanda, Tantrāloka etc. }%


\textbf{6.4}%
\ The sacrifice through work is the construction of%
\              a grove, a park, a pond or a temple with one's own hands.%


\textbf{6.5}%
\ Next I shall teach you the sacrifice with recitation,%
\                 the bestower of the fruits of heaven and liberation.%
\                 One should recite the Vedas, the Śivasaṃhitā [= Śivasaṃkalpa? or rather śaivaṃ bhāratasaṃhitaṃ ca?],%
\footnote{Note vedādhyayana (stem form) and °saṃhitam for saṃhitāṃ metri causa. }%


\textbf{6.6}%
\ the epics and the Purāṇas: this is called sacrifice with recitation.%
\ He who is knowledgeable about inference CHECK and reasoning,%
\              [and knows that] ``this is [proper] action; the other is improper action'',%


\textbf{6.7}%
\ and views [things through?] the eyes of science is called%
\                         [a person performing] sacrifice through knowledge.%
\         I shall teach you concisely about sacrifice through meditation. Listen to me.%


\textbf{6.8}%
\ Meditation was taught by Hari in the past as of five kinds.%
\                 [Meditation of] the Sun, the Moon, Fire, Crystal and the subtle Tattva as fifth.%


\textbf{6.9}%
\ First it is the Sun [that should be meditated upon],%
\                 which is said to be Prakṛti Tattva.%
\         He should visualize the Moon in its centre: that is said to be Puruṣa [Tattva].%
\footnote{Note śaśiṃ for śaśinaṃ. }%


\textbf{6.10}%
\ In the centre of the Moon disk, he should visualise a flame, a fire.%
\                 That is said to be Prabhu Tattva, the destroyer of birth and death.%


\textbf{6.11}%
\ In the centre of the ring of fire, he should visualize a spottless crystal.%
\                 That is said to be Vidyā Tattva, the never-born, imperishable Cause.%


\textbf{6.12}%
\ In the centre of the disk of Vidyā, he should visualize the highest Tattva,%
\              never-heard, unparalleled one, undecaying and imperishable Śiva.%
\                 The fifth Tattva of the sacrifice through meditation has been taught in short.%


\textbf{6.13}%
\ Vigatarāga spoke:%
\footnote{\skt{tri°} in the MSS is a problem. }%
\ Teach me: what are the fruits of [reaching] each Tattva?%
\                Which worlds can be attained and how much time [can one spend there], O great ascetic?%


\textbf{6.14}%
\ Anarthayajña spoke:%
\ The first [world to reach] is Brahmaloka,%
\              through the meditation on the first Tattva, Prakṛti.%
\              He will rejoice [there] happily like Śiva for millions of aeons.%
\footnote{Odd syntax plus gender. }%


\textbf{6.15}%
\ If one dies while meditating on the second Tattva,%
\              Puruṣa, one goes to Viṣṇuloka from this world, [and will live there] happily%
\                 for billions of aeons.%


\textbf{6.16}%
\ Should one die while meditating on the third Tattva,%
\              Prabhu, one can live in Śivaloka continuously for a hundred billion aeons.%


\textbf{6.17}%
\ If he visualizes Vidyā Tattva, [i.e.] Sadāśiva [or sadā śivam?]%
\              he can reach [His] immortal, diseaseless, imperishable world [and can live there]%
\                 well beyond endless aeons[?].%


\textbf{6.18}%
\ The fifth one, the subtle Śivatattva dwells in the Self.%
\              There is no counting of time there and he will be rejoicing [there] together with Śiva.%


\textbf{6.19}%
\ [If] he practises the five meditations, there is no%
\                 rebirth and no more fetters of transmigration.%
\ O excellent Brahmin, [the Lord] should be seeked,%
\                  a wishing tree of desires, [as] he burns away existence.%
\footnote{Note how a plural imperative ātmanepada form (jijñāsyantāṃ) stands for the singular                (jijñāsyatāṃ) metri causa. Note also that the last syllable of                dvijendra counts here as long: this phenomenon of a word-ending                syllable becoming long by position is common in the VSS. }%
\ Liberation comes within one single birth!%
\              People, why should you not strive [for it]!%
\footnote{Note the form janmena. }%
\ [This is known] as the destroyer of all impurity. [It's ascertainable] by direct perception.%
\              It is not inference. It is to be experienced by your own self.%


\textbf{6.20}%
\ The first [type of penance] is mental penance, the second is verbal penance,%
\              the third is the bodily one, the next one[??] is the one which is both mental and verbal action.%
\              The fifth type of penance is a mixture of the bodily and the verbal.%


\textbf{6.21}%
\footnote{Note that miśraka in pāda b stands for miśrakaṃ metri causa. }%
\ Gentleness of the mind, calmness, self-control,%
\                  taciturnity and the purification of one's state of mind:%
\                         mental penance comprises these five.%
\footnote{ete would be better for etāni? phps no, see 6.24c. }%


\textbf{6.22}%
\ Verbal penance is taught as speech that causes no anxiety,%
\                 which is kind, true and useful, and [it include] also the practice of recitation.%


\textbf{6.23}%
\ Bodily penance is taught as the following:%
\                 honesty, harmlessness, chastity, the worship of gods, and purity as the fifth.%


\textbf{6.24}%
\ [Penance] which is a mixture of the mental [and the verbal] is%
\                 taught by the great Ṛṣis to be these five:%
\                 He should speak [about things that are] agreeable, virtuous [bhāva?], auspicious,%
\                 salutary and useful.%


\textbf{6.25}%
\ [Penance] in which bodily [and verbal things] are mixed is%
\              taught by the great Ṛṣis to be these five:%
\                 the worship of the guest and the guru by%
\                 asking about their well-being, celebrating them and blessing them.[??]%


\textbf{6.26}%
\ [Being] a [so-called] frog-yogin in the winter,%
\                         or one with the five fires, or one who has the clouds [i.e. the open sky]%
\                         for shelter in the rainy season: this kind of penance is called \skt{sādhana}.%
\footnote{CHECK abhrāvakāśa in MBh, Manu and Śivadharmasaṃgraha. }%


\textbf{6.27}%
\ Carving out his own flesh as a donation, or%
\                 [offering his own] hand, feet and head, ... puṣpa as blood?%
\                 All these kinds of penance is \skt{sādhana},%


\textbf{6.28}%
\ [such as also] the Painful penance and the Extremely Paniful one, [eating only] at night,%
\                  the Hot and Painful and [the one in which only food obtained] without%
\                  solicitation [can be eaten], the Cāndrāyaṇa and Parāka penances,%
\                  the Sāṃtapana etc.%


\textbf{6.29}%
\ A person who performs with a well-disposed mind%
\                 this penance that puts an end to the suffering caused by mundane existence,%
\                 abandoning the trap of hope, with a spotless mind, giving up the lowest rewards%
\                 [such as] wishing for heaven, being a king and having enjoyments for the senses,%
\                 can bring that ultimate [? \skt{sarvāntika}] reward which stems from it [i.e. from \skt{tapas}]%
\                 to [this] home of eternal births and deaths.%
\footnote{Note the stem form \skt{°pāśa} in \skt{pāda} b metri causa. }%
\vfill\pagebreak\begin{center}{\large\textbf{ Chapter Seven 
}}\end{center}


\textbf{7.1}%
\ In the past the wise declared that there were five kinds of donation ... CHECK%
\ Donation of food, clothes, gold, land and the fifth, donation of cows.%
\footnote{\skt{tathety} is suspicious. Note how \skt{annaṃ}, \skt{vastraṃ}, \skt{hiraṇyaṃ} and \skt{bhūmi} (the latter treated as neuter, or given in                                stem form) are all meant to go with -\skt{dāna} (again, in stem form, metri causa). }%


\textbf{7.2}%
\ From food [comes] energy, memory, the vital breath, growth, body, happiness.%
\ From food arise grace and beauty, heroism, strength.%


\textbf{7.3}%
\ Living beings live on food. Food always satisfies.%
\ From food arise desire, rapture, pride and valour.%


\textbf{7.4}%
\ Food drives away hunger and thirst and disease instantly.%
\ From donations of food arise happiness, fame and glory.%


\textbf{7.5}%
\ He who donates food donates life. He who donates life donates everything.%
\ Therefore nothing is equal to the donation of food, nothing was, nothing will be.%


\textbf{7.6}%
\ ...  ?%
\ A person without clothes may not be respected by his wife, son, friends etc.%


\textbf{7.7}%
\ Be it a learned person from a good family or an intelligent and virtuous one,%
\ a person without clothes is subdued and humiliated on every occasion%


\textbf{7.8}%
\ because a person without clothes receives contempt and disrespect.%
\ Even a great soul will try to avoid [him] at the court, among women, in an assembly.%
\footnote{The intention originally may have been this: ``Even if he is a great soul, he will be avoided...'' }%


\textbf{7.9}%
\ Therefore the wise praise donations of clothes.%
\ One should not give away old, torn or dirty clothes.%


\textbf{7.10}%
\ [Clothes] should be donated [only if they are] new, not worn,%
\                  soft, delicate and beautiful,%
\ well-washed, and [if] accompanied by willingness and devotion.%


\textbf{7.11}%
\ They say that the reward [of donation/generosity] is in every case dependent on the%
\                 particular [donor's] willingness and character, the choice of place and time, and on%
\                 the particular recipient and material.%
\footnote{It seems that \skt{vidhena ca} stands for \skt{vidhinā ca} or rather \skt{vidhānena} metri causa in \skt{pāda} b. }%


\textbf{7.12}%
\ The reward received will be similar to the clothes donated.%
\ By donating old clothes, one would receive old clothes [as a reward].%
\ By donating beautiful clothes, one would receive beautiful clothes [as a reward].%


\textbf{7.13}%
\ Should one bestow very beautiful clothes on a Brahmin [lit. on a person who is first among the twice-born]%
\                         in an auspicious time, respectfully.%
\ he [i.e. the donor] will receive unequalled happiness and a beautiful appearance.%
\ When he departs, he will be given hundreds of millions of items of nice clothes, no doubt about that.%
\ Therefore do donate clothes often. It is the way up to the other world.%


\textbf{7.14}%
\ O great Brahmin, now I shall teach you about the donation of gold in a concise manner.%
\ It is pure, auspicious and meritorious [act] and it washes off all sins.%


\textbf{7.15}%
\ Should one hand over [to someone] a golden bracelet or ring, O Brahmin,%
\footnote{I suspect that \skt{aṅguli} is used here in the sense of \skt{aṅgulīya} (`finger-ring'). }%
\ he will be freed of all sins, just as the moon is freed from [the demon] Rāhu.%


\textbf{7.16}%
\ If a person donates  gold to Brahmins or gods, O excellent Brahmin,%
\ even if it is only in a minute quantity, he will be freed of all sins.%
\footnote{The form \skt{tuṭi} as a widespread variant of \skt{tuṭi}, see e.g. CHECK. }%


\textbf{7.17}%
\ [The amount can be just] one \skt{rakti}, a \skt{māṣaka}, a \skt{karṣa}, half a \skt{pala} or a \skt{pala}:%
\ this is exactly how the increase in the [size of the corresponding] reward will be,%
\                                 in proportion to the kind [i.e.\ amount] of the donation.%
\footnote{I suspect that \skt{phalaṃ vṛddhir} stands for \skt{phalavṛddhir} (\skt{phalasya vṛddhiḥ}) metri causa, meaning                                `the increase of the reward'. }%


\textbf{7.18}%
\ The wise praise the donation of land as the basis of everything [else].%
\ Food, clothes, gold etc.: all of these originate in the land.%


\textbf{7.19}%
\ O Brahmin, one can obtain all the rewards of donation be donating land.%
\ If there is anything that equals the donation of land, O Brahmin, you should really tell me.%


\textbf{7.20}%
\ [Humans] have the earth as their abode as soon as they get out of their mother's womb.%
\ Land is taught as common to all that is mobile and immobile.%
\footnote{I take \skt{sādhāraṇā} as one word, but it is possible that the intention of the author                        was \skt{sā dhāraṇā} in two words, in fact meaning \skt{sādhāraḥ} (\skt{sā ādhāraḥ}, `it is the basis'). }%


\textbf{7.21}%
\ Be it [only a land of] one forearm, two forearms, fifty or a hundred,%
\ a thousand, ten thousand, a hundred thousand, donations of land are held in great esteem.%


\textbf{7.22}%
\ Should he donate a piece of land of [only] one forearm to an excellent Brahmin,%
\ he will enjoy a billion divine years in heaven.%


\textbf{7.23}%
\ Thus in case of many forearms [of land], the reward is said to be%
\                         [proportional to the dimensions of the land, i.e.] ...%
\ O Brahmin, I have taught you about the rewards of donation that is made willingly.%
\footnote{I think that \skt{guṇāguṇi}, or perhaps \skt{guṇaguṇi} (which would be unmetrical), should refer to the idea                        that e.g. the donation of a piece of land of 2 x 2 \skt{hasta}s would result in                         4 x \skt{koṭiśata} years in heaven, \skt{guṇa} generally meaning `times'. But this is only a guess, and                        it needs to be supported by some similar passage.                        I suspect that \skt{pāda} c is an awkward attempt at saying \skt{śraddhādhikadāna(sya) phalaṃ}. }%


\textbf{7.24}%
\ [Paraśu]rāma, the son of Jamadagni, having donated land to the Brahmin [Kaśyapa],%
\footnote{See entry `Paraśurāma' in Purāṇic Enc.:                

                                To atone for the sin of slaughtering even                                innocent Kṣatriyas, Paraśurāma gave away all his                                riches as gifts to brahmins. He invited all the brahmins                                to Samantapañcaka and conducted a great Yāga there.                                The chief Ṛtvik (officiating priest) of the Yāga was                                the sage Kaśyapa and Paraśurāma gave all the lands                                he conquered till that time to Kaśyapa. Then a plat-                                form of gold ten yards long and nine yards wide was                                made and Kaśyapa was installed there and worshipped.                                After the worship was over according to the instructions                                from Kaśyapa the gold platform was cut into pieces                                and the gold pieces were offered to brahmins.                                When Kaśyapa got all the lands from Paraśurāma he                                said thus:—“Oh Rāma, you have given me all your                                land and it is not now proper for you to live in my                                soil. You can go to the south and live somewhere on                                the shores of the ocean there.” Paraśurāma walked                                south and requested the ocean to give him some land to                                live.  }%
\ obtained eternal life in this very world, O excellent Brahmin.%


\textbf{7.25}%
\ [A cow] with golden horns, silver hooves, garment and bell, O Brahmin,%
\ when given to a Veda-knowing Brahmin, [produces] rewards that are said to be endless.%


\textbf{7.26}%
\ Always rejoicing in the practice of giving as far as his capacities go ... ?%
\ one should give food, clothes, gold and silver, water, cows, sesamum [oil?], land,%
\ sandals, parasols, seats, jars, cups or anything else.%
\ Making the [deed of] giving willingly (\skt{śraddhādāna}) something done with an%
\                  uninterrupted facial expression of affection, one's mind becomes spotless.%
\footnote{For \skt{śakyānurūpaṃ} in \skt{pāda} a understand \skt{śakyatānurūpaṃ}. }%


\textbf{7.27}%
\ Glory and fortune that makes us happy come about only by donations, and one can gain unequalled fame.%
\footnote{I suspect that \skt{khyātiś ca tulyaṃ} in the MSS stands for \skt{khyātim atulyāṃ} (`and unequalled fame')                 metri causa. I have corrected those parts of this phrase that could be                                 corrected without violating the metre. }%
\ The reproach of the enemy will give pleasure and happiness only because of donations[?].%
\ Being invincible comes from donation and also unequalled graciousness.%
\                 One can reach happiness thought donations.%
\footnote{REVISE! ūrja? }%
\ Endless enjoyments surely come only from donations, and heaven is [reached] also because of it.%
\footnote{Note \skt{svargaṃ} as a neuter in \skt{pāda} d. }%


\textbf{7.28}%
\ The unequalled world of Śakra [i.e. Indra] [can be reached] only by donations.%
\                                 Donations make people happy.%
\ Samrāj enjoyed the whole earth in the world only because of donations. CHECK%
\footnote{Revise. }%
\ Skanda (\skt{candrānana}) is seen as a handsome and fortunate one with a [good] family[? CHECK] only%
\                         because of donations.%
\ One can reach happiness that lasts countless births only through donations,%
\                                                 there is no doubt about that.%
\vfill\pagebreak\begin{center}{\large\textbf{ Chapter Eight
}}\end{center}


\textbf{8.1}%
\ Five kinds of study are to be pursued by those who wish to%
\                         be happy in this life and in the other:%
\ [one has to study the] Śaiva [teachings], Sāṃkhya [philosophy], the Purāṇa[s],%
\                 the Smārta [tradition] and the \skt{Bhāratasaṃhitā} [i.e. the \skt{Mahābhārata}].%
\footnote{Note the accusative ending of \skt{°saṃhitām} after a list consisting of words probably in the                        nominative. One may correct it to \skt{°saṃhitā}. }%


\textbf{8.2}%
\ He should reflect on the Śaiva truth in both Śaiva and%
\                 Pāśupata [teachings].%
\footnote{Note that \skt{śaivatattvaṃ} in pāda a is the result of a conjecture and that the reading \skt{śaivapāśupatadvaye}                       in pāda b is based on one single manuscript (\msP). In spite of this uncertainty,                       I think that this form of the current half-verse is the only one that yields an appropriate meaning. }%
\ In those teachings the whole essence of truth is taught extensively.%


\textbf{8.3}%
\ Those who reflect on the truth (\skt{tattva}) can grasp the truth (\skt{tattva}) of enumeration (\skt{saṃkhyā})%
\                 [of ontological principles/reality levels] from Sāṃkhya [texts].%
\ The great sages taught [those twenty-five] \skt{tattva}s [of Sāṃkhya] as being in groups of five.%
\footnote{In pāda d, \skt{kīrtitāni} pick up an implied \skt{tattvāni}. }%


\textbf{8.4}%
\ In the Purāṇas it is the sheaths of the world that are described extensively.%
\ One can definitely enter [the realm] of the lower [world, i.e. hell],%
\                  the upper [world, i.e. heaven], and middle [world, i.e. the human world],%
\                  and the horizontal [world, i.e. of animals by studying the Purāṇas].%
\footnote{Note that \skt{tirya} seems to be an acceptable nominal stem in this text for \skt{tiryañc}.               I understand the causative form \skt{sampraveśayet} as non-causative, and               interpret °madhya° as the `human world' tentatively. }%


\textbf{8.5}%
\ The Smārta [tradition] deals with the conduct of the classes (\skt{varṇa}) and%
\                 the conduct in the life-stages (\skt{āśrama}), and with the activities of Dharma and legal proceedings.%
\footnote{Compare pāda a with 3.15c. }%
\ Good conduct is to be gathered from that [source] without hesitation, with trust.%


\textbf{8.6}%
\ A man who studies the epics (\skt{itihāsa}) will become omniscient.%
\ [All his] doubts about Dharma, Artha, Kāma and Mokṣa will be eliminated.%


\textbf{8.7}%
\ Listen with great attention, O Brahmin, to%
\                         the five types of sexual restraint [concerning the following:]%
\ women, forbidden ejaculation, and masturbation are mentioned [in this context, as well as]%
\ offence while sleeping, O Brahmin, and daydreaming as the fifth.%


\textbf{8.8}%
\ A woman is not to be approached sexually in daytime and%
\                 on the four days of the changes of the Moon (\skt{parvan}), even if she is one's lawful wife.%
\footnote{Understand \skt{parve} as \skt{parvani} (thematisation of the stem in \skt{-an}). }%
\ One should not have sex with a woman who is taboo or with one of those who have lost their class (\skt{varṇa}) or%
\                         are [of a] superior [\skt{jāti} than oneself].%


\textbf{8.9}%
\ Intercourse with goats, sheep, cows, mares, buffaloes%
\footnote{Understand \skt{°ādīnāṃ} in pāda a as standing for the locative case. }%
\ is called forbidden ejaculation, which is to be avoided at all cost.%
\footnote{Understand \skt{°sargam} as neuter nominative (instead of \skt{°sargaḥ}) or alternatively               understand pāda c with a hiatus bridge: \skt{garhitotsarga-m-ity etad}. }%


\textbf{8.10}%
\ Rubbing himself against something else than a female sexual organ or rubbing his anus,%
\footnote{The conjecture that changes \skt{anyonya°} to \skt{ayonya°} in pāda a involves                 minimal intervention and makes the sentence much more meaningful than the                 version transmitted. Also consider \skt{ayoni°}. }%
\ are called masturbation, therefore these are to be avoided.%
\footnote{The variant \skt{strī} for \skt{tāṃ} in pāda d in the \Ed\ may be an example of Naraharināth, the editor's                         conscious interventions. }%


\textbf{8.11}%
\ Offence while sleeping, O best of Brahmins, has always been [considered] undesirable by%
\                                 the learned.%
\ [If] one enjoys women while sleeping, his semen gets spilt.%


\textbf{8.12}%
\ Daydreaming [about women] should always be avoided by those who are intent on Dharma.%
\ Women are called `the bolts [that block the gate to] the path to heaven'.%


\textbf{8.13}%
\ [Hear about] the five religious observances [called] the cat, the crane, the dog, the cow, and%
\                                 the earth.%
\ <sep/>He buries his own urine and faeces in the ground, O truest Brahmin.%
\footnote{Note \skt{°viṣṭha°} for \skt{viṣṭhā} metri causa in pāda c (\skt{ma-vipulā}).                Alternatively, read \skt{svaviṣṭhāmūtra bhūmīṣu}. }%
\ He rejoices [seeing] the sun and the moon when performing the cat observance.%
\footnote{Note the stem form \skt{sūryasoma} for \skt{sūryasomau} in pāda e.                 It is not entirely clear why cats would rejoice seeing the Sun and the Moon.                Perhaps this remark refers to the fact that cats can be active both                in the daytime and at night. }%


\textbf{8.14}%
\ O great ascetic, one should suppress all of his senses like a crane,%
\ and should cultivate the peace of the mind, focusing on achieving liberation.%
\footnote{Cranes are compared to ascetics here probably because of the similarity of                their tendency of relaxing standing on one leg to ascetics performing penance                 standing on one leg (such as the ascetic depicted on the famous relief in Mahabalipuram).  }%


\textbf{8.15}%
\ He does not bury his urine and faeces in the ground, and he barks constantly.%
\ Lord Śarva [i.e. Śiva] is satisfied when one practises the dog observance.%
\footnote{CITE source on dog being Bhairava's vāhana... }%


\textbf{8.16}%
\ A person practising the Cow Vow should never hold back his urine and faeces.%
\ He is terrifying and he gives satisfaction, [as] stated in the Purāṇas.%
\footnote{I prefer reading \skt{bhīma tuṣṭi°} as two separate words, the first                one in stem form, to reading it as a compound because                of the following \skt{caiva}, and to the reading \skt{bhīmas tuṣṭi°}                 because the corresponding witnesses are the ones that usually give inferior readings. }%


\textbf{8.17}%
\ CHECK Digging [the earth] with spades and collecting [? the soil] with wedges:%
\footnote{While \skt{dārayanto} as an active participle in the masculine nominative is acceptable                as an irregular form, the precise interpretation of pādas a and b is still problematic. }%
\ Goddess Earth bears [this] patiently. This is exactly how one can practise the earth vow.%


\textbf{8.18}%
\ He who practises these five religious observances with his senses subdued%
\ will, without doubt, reach this superior world (i.e. Śiva's heaven).%
\footnote{Note the neuter \skt{idaṃ} picking up the normally masculine \skt{lokaṃ} in pāda c. }%


\textbf{8.19}%
\ Eating leftovers, [not] eating in-between [breakfast and dinner],%
\                         eating [only] at night, eating food obtained without solicitation,%
\ and fasting: listen, I shall teach you these five.%


\textbf{8.20}%
\ [He who eats] the leftovers belonging to all the gods, to guests, and to the ancestors,%
\ he who eats the leftovers (śeṣāśin) of servants, sons and wives is the one who%
\                         consumes the remains of food (\skt{vighasāśana}).%


\textbf{8.21}%
\ He will be regarded as one that is always fasting%
\                         if he never eats between breakfast and dinner.%
\footnote{My translation here follows the parallel verse in the MBh and                       is based on that of Kisari Mohan Ganguli. The syntax of the version here in the VSS is less                      smooth than that in the MBh, and the VSS's reading \skt{prāntarāśī} definitely required an emendation. }%


\textbf{8.22}%
\ One should not eat in the daytime or in the evening,%
\ and should eat [only] at midnight if he wishes to follow the order of [eating only at] night.%
\footnote{Note \skt{°vele} for \skt{°velāyāṃ} in pāda c. }%


\textbf{8.23}%
\ He should eat only the unsolicited food of someone who has not yet started eating [this food].%
\footnote{The translation of \skt{anārambhasya} (`of someone who has not yet started eating') is tentative. }%
\ He who eats [only] that which has been given by others [without asking them for it]%
\                                         is called [one who eats] unsolicited [food].%


\textbf{8.24}%
\ Chewable and unchewable food (\skt{bhakṣyaṃ bhojyaṃ ca}),%
\                         food to be sipped or sucked or drunk, as the fifth [category]:%
\footnote{For a detailed discussion of the categories \skt{bhakṣya, bhojya, lehya} and \skt{coṣya},                        see Kafle 2020:245, n. 534. See also Śivadharmottara 8.13:

                        \skt{bhakṣyaṃ bhojyaṃ ca peyaṃ ca lehyaṃ coṣyaṃ ca picchilam} \danda<br>                        \skt{iti bhedāḥ ṣaḍannasya madhurādyāś ca ṣaḍguṇāḥ} \twodanda }%
\ if one does not long for and does not consume [any of the above], that is called fasting (\skt{upavāsa}).%


\textbf{8.25}%
\ One should keep these five types of taciturnity,%
\                    always dwelling in religious observances:%
\footnote{\skt{pāruṣya} seems to be the good reading in pāda a because                 in the following a short section on this category is coming up.                As far as the readings \skt{spṛṣṭavāg} and \skt{pṛṣṭavāg} are concerned, I suppose                 \skt{pṛṣṭavāg} is not inconceivable (as suggested by Judit Törzsök),                 for in 8.29 it is questions that are given as relevant examples.                 Nevertheless I conjectured \skt{tīkṣṇavāg} here, relying on the same verse, 8.29. }%
\ [in situations where silence is best instead of] deceitful speech, envious speech, abuse, harsh speech, bragging.%


\textbf{8.26}%
\ Fictitious [speech], [speech on] unknown [things], [speech about things] outside the range of Dharma,%
\ meaningless and unfriendly speech: these are called lying.%


\textbf{8.27}%
\ One who does not rejoice in others' fortune or in others' power,%
\ one who would like to see something disadvantageous [for others] is called envious%
\                 [and he should rather remain silent].%


\textbf{8.28}%
\ [May your] mother and father be dead! [This is] how a ruined state will befall [you]!%
\ Enjoy the love of unclean [women]! [These are] called abuse.%
\footnote{My translation of pāda b, or rather of the whole verse, is tentative. }%


\textbf{8.29}%
\ Won't you burst in your heart, stupid? Will your head not split into two?%
\footnote{Understand \skt{śiro} as standing for the locative (\skt{śirasi}). }%
\ [If one utters] these or similar [curses], he is said to be one of harsh speech.%


\textbf{8.30}%
\ Relating fancy stories about gambling, enjoyments, fights, drinking and women%
\footnote{I take \skt{°katham} in pāda b as an alternative nominative form of \skt{°kathā} metri causa and as                 belonging to all the categories here thus: \skt{dyūtakathā, bhojanakathā, yuddhakathā, madyakathā,                strīkathā}. }%
\ are the five types of bragging, as I teach them, O excellent Brahmin.%
\footnote{Understand \skt{me} in pāda d as \skt{mayā}. }%


\textbf{8.31}%
\ Taciturnity should always be practised by those who prefer the beauty of speech.%
\ One should always speak without abuse and without idle talk.%


\textbf{8.32}%
\ He who does not practise taciturnity is defiled and he is the black sheep of the family.%
\ For a number of rebirths, [his mouth] will stink and he will become mute.%
\footnote{The form \skt{janme} for \skt{janmani} often occurs in Śaiva tantras as a tipically Aiśa phenomenon.                See XXXXX  }%


\textbf{8.33}%
\ Therefore the speech of a person who always keeps the observance of taciturnity firmly, with resolution,%
\ will be impossible to ignore and he will make the community rejoice.%
\ The fragrance of lotuses and [other kinds of] strong fragrances will blow from his mouth.%
\ Thousands of faultless \skt{śāstra}s will be declared in the words of this person.%
\footnote{To make sense of pāda d, we are forced to take \skt{śāstra} as a stem form noun and                 \skt{naraḥ} as a (regular) genitive from \skt{nṛ}. (I thank Judit Törzsök for this interpretation.)                Another way of understanding the beginning of this sentence would be to separate \skt{śāstrāneka°} as                \skt{śāstrān eka°}, treating the word \skt{śāstra} as masculine. }%


\textbf{8.34}%
\ I shall teach you the five kinds of bathing as they really are:%
\ Fire bath, water bath, Vedic bath, wind bath and divine bath.%


\textbf{8.35}%
\ Fire bath is [performed] with ashes. Its fruits are a hundred times bigger than [those of] a water [bath].%
\ [Things] purified with ashes are holy. Ashes destroy sin.%


\textbf{8.36}%
\ Therefore one should use ashes for it purifies humans of their defilement.%
\ Ashes produce peace for everyone. Ashes are the ultimate protectors.%


\textbf{8.37}%
\footnote{Note \skt{tryāyuṣa} in the sense of the three \skt{puṇḍra}-lines on the                forehead and compare with 11.28c. Understand \skt{sthitam} as                 \skt{sthitaḥ} or rather \skt{sthitāḥ} if we are to connect this line                to the next (8.37cd). }%
\ Drawing [the sectarian marks on their foreheads while reciting] the Tryāyuṣa [mantra],%
\                                 remaining in chastity,%
\ all the Ṛṣis purified themselves with ashes.%
\footnote{Grammatical notes on kṛtam and ātmanaḥ }%


\textbf{8.38}%
\ The gods, afflicted by their fear of Vīrabhadra, were set free with the help of ashes.%
\footnote{It is not clear which story concerning Vīrabhadra is referred to here.                 Is it the destruction of Dakṣa's sacrifice, after which the gods were relieved?                 Or, which is a less likely possibility, another in which                 Kaśyapa and other Ṛṣis were burnt to ashes then reanimated by Vīrabhadra in the Śokara forest?                 For the latter, less well-known story, see Padmapurāṇa 5.107.1--14ff:

                        \skt{śucismitovāca <br>                         kaśyapaṃ jamadagniṃ ca devānāṃ ca purā katham \danda <br>                        rarakṣa bhasma tad brahman samācakṣva mune mama \twodanda1  <br>                        dadhīca uvāca  <br>                        kaśyapādiyutā devāḥ pūrvam abhyāgaman girim  \danda<br>                        śokaraṃ nāma vikhyātaṃ girimadhye suśobhanam \twodanda2  <br>                        nānāvihaṃgasaṃkīrṇaṃ nānāmunigaṇāśrayam \danda <br>                        vāsudevāśrayaṃ ramyam apsarogaṇasevitam \twodanda3  <br>                        vicitravṛkṣasaṃvītaṃ sarvartukusumojjvalam  \danda<br>                        tathāvidhaṃ praviśyaite giriṃ vayam athāpare \twodanda4  <br>                        stuvaṃtaḥ keśavaṃ tatra gatāḥ sma giriśeśvaram  \danda<br>                        dṛṣṭvā tatra mahājvālāṃ praviṣṭāśca vayaṃ ca tām \twodanda5  <br>                        māmekaṃ tu tiraskṛtya hy adahad devatā munīn  \danda<br>                        māṃ dadāha tataḥ paścād bhasmībhūtā vayaṃ śubhe \twodanda6<br>                          asmān etādṛśān dṛṣṭvā vīrabhadraḥ pratāpavān \danda <br>                        kenāpikāraṇenāsau gatavān parvataṃ ca tam \twodanda7  <br>                        bhasmoddhūlitasarvāṃgo mastakasthaśivaḥ śuciḥ  \danda<br>                        ekākī niḥspṛhaḥ śānto hāhāśabdam athāśṛṇot \twodanda8  <br>                        atha ciṃtāparaś cāsīn mriyamāṇa śavadhvaniḥ \danda <br>                        śavānām iva gaṃdhaś ca dṛśyate tannirīkṣaṇe \twodanda9  <br>                        iti niścitya manasā jagāmāgnim atiprabham  \danda<br>                        sa vahnir vīrabhadraṃ ca dagdhum ārabdhavān atha \twodanda10 <br>                         tṛṇāgnir iva śāṃto 'bhūd āsādya salilaṃ yathā \danda <br>                        tato 'parāṃ mahājvālāṃ vīrabhadras tu dṛṣṭavān \twodanda11 <br>                         khaṃ gacchaṃtīṃ mahākālo jvālāṃ nipatitām api  \danda<br>                        manasā ciṃtayac cāpi vīrabhadraḥ pratāpavān \twodanda12  <br>                        sarveṣāṃ nāśinī jvālā prāṇināṃ śatakoṭiśaḥ \danda <br>                        tat sarvaṃ rakṣaṇārthaṃ hi pipāsuś cāpy ahaṃ tv imām \twodanda13  <br>                        prāśnāmi mahatīṃ jvālāṃ salilaṃ tṛṣito yathā \danda <br>                        etasminn aṃtare vīraṃ vāg āha cāśarīriṇī \twodanda14} 

``Śucismitā said:<br>1. O brāhmaṇa, O sage, tell me how formerly the sacred ash protected Kaśyapa, Jamadagni of the gods?Dadhīca said:<br>2--6. Formerly gods accompanied by Kaśyapa and others went to a well-known mountain named Śokara. In the middle of the mountain was a very beautiful (forest) which was full of many birds, which was resorted to by various hosts of sages, which was the resort of Vāsudeva, which was charming, which was resorted to by bevies of celestial nymphs, which was crowded with strange trees, which was bright with flowers of all seasons. We and others entered the best mountain (forest) like that and praising Viṣṇu went there to lord Śiva. We saw a great flame there and we entered it. Excepting me that deity (i.e. that flame) burnt (other) sages. After that it (also) burnt me. O auspicious one, we were reduced to ash.<br>7--14. Seeing us like this, that brave Vīrabhadra went to that mountain for some reason. With his entire body smeared with sacred ash, he remaining at the top, auspicious and pure, all alone, desireless and tranquil, heard the sound of wailing. Then he was full of thought: ‘The sound of the bodies of dead men and the smell as it were of dead bodies, are being perceived.’ Deciding like this in his mind, he went to the fire of great brilliance. Then that fire also started to burn Vīrabhadra. But it went out as the fire of (i.e. burning) grass (i.e. hay) would go out on receiving (i.e. being sprinkled over with) water. Then Vīrabhadra saw a great, mighty flame, which went (up) to the sky even (like) flame falling (i.e. dropped by) Śiva (obscure!). The brave Vīrabhadra thought in his mind: ‘(This) flame is the destroyer of hundreds of crores of beings. So for the protection of all I desire to drink it. As a thirsty man drinks water, I shall consume this great flame.’ In the meanwhile a divine voice said to (Vīrabhadra) the hero [...] (translation by N.A. Deshpande, in: Padma-purāna, Delhi: MLBD, 1951)''  }%
\ Seeing the glory of ashes, Brahmā consented [to the use of this otherwise impure substance].%


\textbf{8.39}%
\ [Thus] the Pāśupata observance was created, which is above [the system of] the four \skt{āśrama}s.%
\ Therefor the Pāśupata [observance] is the best because it involves carrying ashes [on one's body].%
\footnote{One could simply accept the reading of \msCc (\skt{°hetunā}) in pāda d, but all other rejected                 readings hint at an original \skt{hetutaḥ} (as pointed out by Judit Törzsök). }%


\textbf{8.40}%
\ A water bath (\skt{vāruṇa}) is to be performed with water by people in various ways%
\footnote{The reading \skt{vvidhaṃ} in pāda b seems to be the lectio difficilior as opposed to                the rejected \skt{vidhivat}. }%
\ in the water of rivers, water tanks, streams and ponds.%


\textbf{8.41}%
\ The wise know the Vedic bath as [the one performed with the Vedic mantra beginning] \skt{āpo hi ṣṭhā} [ṚV 10.9.1--3],%
\                         O excellent Brahmin.%
\footnote{The Ṛgvedic mantra starting with \skt{āpo hi ṣṭhā} (ṚV 10.9) is traditionally associated with                 \skt{mārjana} (`cleaning, wiping'). According to Kane (A History of Dharmaśāstra, vol. 4, p. 120),                a Brahmin ``should bathe thrice in the day, should perform \skt{mārjana} (splashing                or sprinkling water on the head and other limbs by means of \skt{kuśas}                 dipped in water after repeating sacred mantras) with the three verses `apo hi sthā' [sic] (Ṛg. X.9.1--3) [...]''                This suggests a method of bathing that is more of a ritual than an actual bath. }%
\ It is to be performed at the three junctures of the day (dawn, noon, evening).%
\                 It is called the Vedic bath.%


\textbf{8.42}%
\ He should go where, on the paths where cows roam, dust is rising,%
\ and he should sit down there. This is called [a kind of] bath,%
\                        [namely the \skt{vāyavya} or wind-bath].%
\footnote{This version of bathing seems to be rather a kind of bathing                 in the holy dust raising from under the hooves of cows. }%


\textbf{8.43}%
\ One should immerse his own body in the water-showers of rain water.%
\ The one and only great Lord (\skt{maheśvara}) of the universe calls it heavenly bath.%


\textbf{8.44}%
\ Thus have I taught you the section on the Niyama-rules [see Chapters 5--8]%
\                         in divisions of five [sub-categories]%
\ because you asked me to, favouring the whole world.%
\footnote{Understand \skt{sarvalokānukampya} in pāda b as \skt{sarvalokān anukampya}. }%
\ [These Niyama-rules] wipe off all the defilement, these fifty Dharma [teachings,%
\                         i.e. 10 main topics/rules × 5 subcategories].%
\footnote{Understand \skt{sakalamalapahārī} in pāda c as \skt{sakala-mala-apahārī}, which would be unmetrical.                       Understand \skt{etan/etad} as either picking up °\skt{pahārī} or                        a plural corresponding to °\skt{pañcāśad}. }%
\ There will not be rebirth [for one who keeps these rules], not even in%
\                                millions of aeons.%
\vfill\pagebreak\begin{center}{\large\textbf{ Chapter Nine
}}\end{center}


\textbf{9.1}%
\ The whole universe with its moving and unmoving elements is divided%
\                         by the three [divisions of] time and the [three] \skt{guṇa}s%
\                         [or guṇa not tech term here?].%
\ Therefore the whole world is bound by the fetters of the three \skt{guṇa}s.%


\textbf{9.2}%
\ Vigatarāga spoke:%
\ What does the term `the three divisions of time' mean for the soul in the three worlds[?]?%
\footnote{I have included the element \skt{trai°} in the lemma in pādas ab only because \msCc\                 has a slightly unusual ligature there (\skt{mtrai}) }%
\ Talk about it in a somewhat more extended manner, O great ascetic.%


\textbf{9.3}%
\ Anarthayajña spoke:%
\ The three [divisions of] time are the three \skt{guṇa}s.%
\                 It[?] is pervading and born from Prakṛti.%
\ They support each other, they serve each other.%


\textbf{9.4}%
\ Sattva, Rajas and Tamas; Rajas, Sattva and Tamas;%
\ Tamas, Sattva and Rajas; they are each other's pairs.%


\textbf{9.5}%
\ Lord Viṣṇu is Sattvic. [Brahmā], the one who was born on a lotus, is Rājasa.%
\ Lord Īśa is Tāmasa, the limbless is all ... [?]%


\textbf{9.6}%
\ Sattva is of the colour of jasmine and the moon.%
\                  Rajas is of the colour of ruby.%
\ Tamas is of the colour of lamp-black ... śaila. [This is what] the wise teach.%


\textbf{9.7}%
\ Sattva is water, Rajas is charcoal, Tamas is full of smoke.%
\ All souls are constructed/suffer (\skt{pacyante}) as bound by%
\                 these \skt{guṇa}s.%


\textbf{9.8}%
\ Vigatarāga spoke:%
\ By what sorts of noose of \skt{guṇa}s is [the soul] bound?%
\ Teach me the signs connected to them one by one, O great ascetic.%


\textbf{9.9}%
\ Anarthayajña spoke:%
\ The souls are bound in many ways and by many conditions by the fetters of the \skt{guṇa}s.%
\ Those who are deluded do not recognize [them]. The Śivayogins do recognize [them].%


\textbf{9.10}%
\ He who is always established in Sattva goes upwards.%
\                  He who is covered with Rajas goes in the middle.%
\ Those lowest of men in the state of Tamas go downward.%
\footnote{Understand \skt{adhogatis} in pāda c as a bahuvrīhi in plural (\skt{adhogatayas}). }%


\textbf{9.11}%
\ These three kinds of \skt{guṇa}s are to be acknowledged even in heaven, O great ascetic,%
\ and among humans and also among animals.%


\textbf{9.12}%
\ The ten superior Sattva [beings] are:%
\                  Brahmā, Viṣṇu, Rudra, Dharma, Indra, Prajāpati,%
\ Soma, Agni, Varuṇa and Sūrya.%


\textbf{9.13}%
\ ...%


\textbf{9.14}%
\ ...%


\textbf{9.15}%
\ ...%
\ ...%


\textbf{9.16}%
\ ...%
\ ...%


\textbf{9.17}%
\ ...%
\ ...%


\textbf{9.18}%
\ These are the ten superior Tāmasa [animals]:%
\                         cows, elephants, Gayal oxen, horses, deer, Yaks, Kiṃnaras,%
\ lions, tigers, wild boar.%


\textbf{9.19}%
\ The ten middle ranking Tāmasa [beings] are: rams, sheep, buffaloes, mice, mongooses etc.,%
\footnote{\skt{°mahiṣyāś} seems to be an equivalent of \skt{°mahiṣāś} metri causa. }%
\ camels, Raṅku  deer, hares, rhinoceroses. [only 9!]%


\textbf{9.20}%
\ The ten low-ranking Tāmasa [beings] are: bears, alligators, deer, horned animals[?],%
\                         cranes, apes, donkeys,%
\ boar, dogs and frogs.%


\textbf{9.21}%
\ The ten Tāmasa-Sāttvika [beings] are:%
\                 curlews, swans, parrots, falcons, vultures, B[h]āruṇḍa birds, cranes,%
\footnote{Although all the manuscripts consulted read \skt{kroñca°} in pāda a, I decided                to accept \Ed's standard spelling in this case. In pāda b, I left \skt{°bāruṇḍa°}                thus, although what is really meant is probably \skt{bhāraṇḍa}, \skt{bhāruṇḍa} or \skt{bhuruṇḍa}. }%
\ Cakra[vāka] birds, parrots, and peacocks.%
\footnote{Note the repetition of \skt{śuka} in this stanza. }%


\textbf{9.22}%
\ The ten Tāmasa-Rājasa [beings] are: Balāka-cranes, cocks,%
\                         crows, Bengal kites, Lāvakas, partridges,%
\ vultures, herons, Bakas and hawks.%


\textbf{9.23}%
\ The ten lowest Tāmasa [beings] are:%
\                         cuckoos, owls, Kiñjalkas[?], doves,%
\ Śārika birds and sparrows.%
\footnote{This list is problematic for it has only six elements instead of the expected ten                 and \skt{kiñjalka} is difficult to interpret. }%


\textbf{9.24}%
\ Makaras crocodiles, cow-killing alligators and bears are of Tamas-Sattva.%
\footnote{Note that the reading that yields `and bears' (\skt{ṛkṣāś ca}) is my conjecture                        for a problematic \skt{ṛṣā ca}. It is far from satisfactory since bears have already appeared in                         verse 9.20 above. }%
\ Tortoises, Śuśus[?], crocodiles of the Ganges and frogs are of Tamas-Rajas.%
\footnote{I have not been able to identify the probably aquatic animal behind the                         word \skt{śuśu} here. }%
\ Conch-shells, pearl-oysters, shells and [...] are Tamas-Tāmasa.%


\textbf{9.25}%
\ ...%
\ ...%


\textbf{9.26}%
\ The ten Tamas-Rajas [trees] are:%
\                 Citron trees, bread-fruit trees, hog-plum trees, pomegranate trees, jujube trees, ratan trees,%
\ Neemb trees, Kadamba trees and ...%


\textbf{9.27}%
\ ...%
\ ...%


\textbf{9.28}%
\ ...%
\ ...%


\textbf{9.29}%
\ [These words describe] the people who are the best among the Sāttvika [type]:%
\                compassion, truthfulness, self-control, purity, knowledge, taciturnity, penance, patience,%
\ integrity, lack of self-conceit.%


\textbf{9.30}%
\ [These words describe] the people who are the best among the Rājasa [type]:%
\                   desire, thirst, pleasure, gambling, arrogance, fight, intoxication, delight,%
\ cruel, quarrelling.%


\textbf{9.31}%
\ [These words describe] people who are the best among the Tāmasa [type]:%
\                 harming, envious, incompassionate, stupid, sleepy, lazy, cowardly, idle,%
\ angry, greedy, cheating.%


\textbf{9.32}%
\ The Sāttvika can be characterised as follows:%
\               light, joyful, bright, always eager for yoga meditation,%
\ wise, intelligent and dispassionate.%


\textbf{9.33}%
\ The Rājasa can be characterised as follows:%
\               childish, skilful, passionate, proud, arrogant, greedy,%
\ desirous, jealous and chattering.%


\textbf{9.34}%
\ The Tāmasa can be characterised as follows:%
\                 anxious, lazy, deluded, cruel, a pitiless robber,%
\ angry, wicked and sleepy.%
\footnote{In pāda a, \skt{piśuno} might be the right choice: it is a ra-vipulā                         if \skt{dr} in \skt{nidrā} does not make the previous syllable long, a licence                        often occuring in this text (`muta cum liquida'). }%


\textbf{9.35}%
\ Vigatarāga spoke:%
\ By what signs can the food of all humans be recognized? [?]%
\ Teach me about the three \skt{guṇa}s, O great ascetic.%


\textbf{9.36}%
\ Anarthayajña spoke:%
\ The Sāttvikas prefer food that yields [long] life,%
\                  fame, happiness, joy, which increases strength and health,%
\ which is savoury and which tastes nice, and which is soft.%


\textbf{9.37}%
\ The best food for the Rājasas is rather warm, acidic, salty, hard, hot and pungent.%
\ It gives you pain, a burning sensation and indigestion.%


\textbf{9.38}%
\ Tāmasas prefer food that is prohibited, impure and foul-smelling, ... stale%
\footnote{Understand \skt{°pūtī} in pāda a as standing for \skt{°pūti} metri causa, and                 note that °āmedhya° in the same pāda is an emendation (correcting \msNc's reading). }%
\  ... and tasteless.%
\footnote{Read \skt{āmayārasa} in pāda c? }%


\textbf{9.39}%
\ Vigatarāga spoke:%
\ How can one recognize [the state of getting] beyond the \skt{guṇa}s,%
\                 which leads one to the other shore of [the ocean] of mundane existence?%
\ Tell me truly about the liberation of those who are [initially] bound%
\                        by the noose of the \skt{guṇa}s.%


\textbf{9.40}%
\ Anarthayajña spoke:%
\ Well, he who looks at all living beings in the correct way, as his own Self,%
\                         O Brahmin,%
\ is to be known as one beyond the \skt{guṇa}s, as one who%
\                 has departed to the other shore of [the ocean of] mundane existence.%


\textbf{9.41}%
\ He who treats envy and hate[?], happiness and sorrow,%
\ praise and reproach as equal is called `one who is beyond the \skt{guṇa}s'.%


\textbf{9.42}%
\ He who is indifferent to pleasant and unpleasant things,%
\                         to enemy or friend,%
\ to respect and contempt is called `one who is beyond the \skt{guṇa}s'.%


\textbf{9.43}%
\ O Brahmin, thus has the exposition of the essence of the \skt{guṇa}s been%
\                         taught to you.%
\ Those who are connected with the \skt{guṇa}s are mundane (\skt{saṃsārin}),%
\                         those beyond the \skt{guṇa}s are on the supreme path.%
\vfill\pagebreak\begin{center}{\large\textbf{ Chapter Seven 
}}\end{center}


\textbf{10.1}%
\ Vigatarāga spoke:%
\ Which pilgrimage place do the wise consider the best of all?%
\ Tell me, O best of sages, if there is one in the world that fulfills [all] desires.%


\textbf{10.2}%
\ Anarthayajña spoke:%
\ This question [that I have been] asked is an extremely deep secret.%
\                  Out of fondness, O excellent Brahmin,%
\ I'll teach you an ancient legend that Nandi told me.%


\textbf{10.3}%
\ Nandikeśvara spoke:%
\ On a beautiful peak of Mount Kailāsa, which is%
\                  frequented by Siddhas and celestial singers (\skt{cāraṇa}),%
\ there was Śiva himself there, seated, and Devī spoke to him thus:%


\textbf{10.4}%
\ Devī spoke:%
\ O Lord, Lord of the chiefs of the gods, O ruler of all beings and all the world,%
\ I would like to ask you about one thing that concerns the eternal and secret Dharma,%


\textbf{10.5}%
\ the transcendental and highly secret pilgrimage place by which one%
\                  can be liberated from Saṃsāra.%
\ O Maheśvara, teach me the truth for the benefit of mankind.%


\textbf{10.6}%
\ Maheśvara spoke:%
\ Who else would ask me that question if not you, O Sundarī?%
\ Listen, I'll expound that question which is difficult to grasp even for the gods.%


\textbf{10.7}%
\ [If one] gets to know Kurukṣetra, Prayāga, Vārāṇasī,%
\ Gaṅgā, Agni, Somatīrtha, Sūrya, Puṣkara, Mānasa,%


\textbf{10.8}%
\ Naimiṣa, Bindusaras, Setubandha, Surahrada,%
\footnote{Note \skt{bindusāraṃ} for \skt{bindusaras/°saraṃ/°sarasaṃ} metri causa. }%
\ Ghaṇṭikeśvara, and Vāgīśa, he'll certainly be able to destroy his sins.%


\textbf{10.9}%
\ Umā spoke:%
\ This and other [related] things, O Mahādeva,%
\                                 have been [just] taught to me [by you] as previously.%
\footnote{Is perhaps \skt{pūrvavat} used in the sense of \skt{pūrvaṃ} here? }%
\ Among these[?] the pilgrimage place that yields all enjoyments, O Suranāyaka.%


\textbf{10.10}%
\ [But] how is one liberated from mundane existence merely be knowledge, O Īśvara?%
\ Cut [this] great curiosity arising [in me] that causes doubt.%


\textbf{10.11}%
\ Rudra spoke:%
\ How could I not know that pilgrimage place which is both easy and difficult to reach?%
\ It is easy to reach for those who serve their guru and difficult to reach should one abandon it%
\                                 [i.e. the service of the guru].%


\textbf{10.12}%
\ \skt{Kuru} [in \skt{kurukṣetra}] is to be known as the soul (\skt{puruṣa}),%
\                         \skt{kṣetra} as the body.%
\ Kurukṣetra [which] is in the body yields the fruits of all pilgrimage places.%


\textbf{10.13}%
\ [And there will be] the obtaining of the fruits of all sacrifices,%
\                         the fruits of all [possible] donations,%
\ and all the fruits of all religious observances and penance observed.%


\textbf{10.14}%
\ In the same manner [one will obtain] the fruits of those fifteen%
\                         pilgrimage places [from Kurukṣetra to Vāgīśa, cf. 10.7--8,%
\                          by only knowing the bodily Kurukṣetra].%
\ ... [this] great pilgrimage place is extremely auspicious and pleasant.%


\textbf{10.15}%
\ Devī spoke:%
\ I am extremely thrilled, O Tridaśeśvara.%
\ Hearing this which is easy to obtain, easy to perform and is subtle,%
\                         I am filled with satisfaction.%


\textbf{10.16}%
\ Teach me on, teach me the remaining fourteen pleasant [pilgrimage places],%
\ Prayāga and the others, one by one, as they are, O Sureśvara.%


\textbf{10.17}%
\footnote{There seems to be only two yogic tunnel here (and in 10.20--21): Suṣumṇā and Iḍā, instead of                the usual three (Iḍā, Piṅgalā, Suṣumnā). This is strikingly similar to                what we see in the archaic yoga of the Niśvāsa Naya, see Goodall et al. pp. 33--34.                

                Note \Ed's attempt to make pāda a metrical, but also note how some                 similar passages in other texts have the same hypermetrical reading as all our manusctipts; }%
\ The Suṣumnā[-tube] is the Honourable Gaṅgā, Iḍā[-tube] is the river Yamunā.%
\footnote{MBh Indices 6.3A.41--44:
                 \skt{iḍā bhagavatī gaṅgā piṅgalā yamunā nadī \danda
                 tayor madhye tṛtīyā tu tat prayāgam anusmaret \twodanda
                 iḍā vai vaiṣṇavī nāḍī brahmanāḍī tu piṅgalā \danda
                 suṣumṇā caiśvarī nāḍī tridhā prāṇavahā smṛtā} \danda
        

        See also \skt{Haṭhayogapradīpikā} 3.110:        
        iḍā bhagavatī gaṅgā piṅgalā yamunā nadī \danda
        iḍāpiṅgalayor madhye bālaraṇḍā ca kuṇḍalī \twodanda                                                         }%


\textbf{10.18}%
\ The right nostril is [the river] Vāruṇī, the left nostril is known as [the river] Asi.%
\ Because [it is] at the confluence of Vāruṇā and Asi,%
\                         [the city there] is known as Vārāṇasī.%


\textbf{10.19}%
\ She is called the ethereal Gaṅgā [because] the nectar of immortality issues from her%
\ day and night uninterruptedly. That's why she is called Gaṅgā (perhaps: `ever-goer').%


\textbf{10.20}%
\ Somatīrtha is the tube Iḍā. It is characterised by the ringing of small bells.%
\ Upon hearing that [ringing], all of one's sins will be destroyed.%


\textbf{10.21}%
\ Somatīrtha is the [tube] Suṣumnā ....%
\ By merely hearing about it one is liberated, even if%
\                         one has a huge heap of sins.%


\textbf{10.22}%
\ Agnitīrtha is the Arjuna tube[??]. It is charming because of the hum of%
\                         Veda recitation.%
\ Upon hearing this or that syllable, one will become immortal.%


\textbf{10.23}%
\ Puṣkara is [a lotus] with eight petals and a pericarp%
\                                          in the centre of the heart.%
\footnote{\skt{hṛdi} might be meant to be a nominative, as in 12.17, here compounded with \skt{madhyastham}. }%
\ One should visualize the Subtle One in its centre [and] it'll%
\                         destroy birth and death.%


\textbf{10.24}%
\ In the centre of the Mānasa lake on a lotus with [the syllables] HAṂ-SA,%
\footnote{Understand \skt{mānasasara°} in pāda a as \skt{mānasasaro} (metri causa). }%
\ ...%


\textbf{10.25}%
\ Listen to Naimiṣa, O Deveśī. It presents proof in a moment.%
\ One can observe one's own or others' shadow properly[?].%


\textbf{10.26}%
\ ...%
\ When he has seen the proof thus, he is called the knower of Naimiṣa.%


\textbf{10.27}%
\ Listen O Sundarī, I shall teach you the pilgrimage place called Bindusaras.%
\ The heart is to be known to be located in the centre of the body.%
\                  In the centre of the heart, there is a lotus.%
\footnote{Note \skt{hṛdi} as a nominative in pāda c and possibly also in pāda d (and see 10.23a). }%


\textbf{10.28}%
\ There is a pericarp in the centre of the lotus, and the subtle sonic matter (\skt{bindu})%
\                                 in the centre of the pericarp.%
\ In the centre of the subtle sonic matter (\skt{bindu}), there is the subtle sound (\skt{nāda}).%
\                 How is that subtle sound (\skt{nāda}) divided?%


\textbf{10.29}%
\ Divided by the sound U and the sound MA, the subtle sound (\skt{nāda}) departs.%
\ Realizing that [subtle sound], O Viśālākṣi, one can obtain immortality.%


\textbf{10.30}%
\ I shall teach you Setubandha, [which sports] a current whose water of subtle sound (\skt{nāda})%
\                                 cleanses you of the dirt of your sins.%
\ The banks [of this river] are the tongue, the throat and the chest,%
\                         its sandy beaches are the host of gods, it roars with whirlpools and is wavy.%
\footnote{Note that \skt{°kaṇṭhora} is a conjecture based on the context: this line                        probably talks about sounds and the production of sounds. For this                         \skt{uraḥ}/\skt{ura} (`chest') seems better that \skt{ūru} (`thigh'). }%
\ It's full of the roar of Ganges crocodiles and full of fish, ten types of sea-monsters [also: makāra?],%
\                          terrifying alligators and with \skt{visarga}[?]%
\ Go to Setubandha, [the pilgrimage place that] tastes like the pleasure of intoxication%
\                         in the deep ...%


\textbf{10.31}%
\ O Moon-faced goddess, listen to [Surahrada], the reaching of the cessation of%
\                         all sorrow, located in the centre of the seven islands.%
\ It is frequented by Īśāna, it's a spotless lake in the heart full of%
\                         the cool water of sound (\skt{nāda}).%
\ There is a lotus arising, with Prakṛti as its petals, and divided by its Śakti filaments.%
\ It is praised by the five voids, it is the path to the supreme level, and%
\                         it is to be served if one wishes to obtain [heaven].%


\textbf{10.32}%


\textbf{10.33}%


\textbf{10.34}%
\vfill\pagebreak\begin{center}{\large\textbf{ Chapter Eleven 
}}\end{center}


\textbf{11.1}%
\ The Goddess spoke:%
\ O Paraśreṣṭha, O Surottama! Is there another [form of] universal sacrifice,%
\ which is free of pain, which is easy, and which does not require an abundance of materials, O Īśvara?%
\footnote{alpakleśa -m- anāyāsa (sandhi bridge) }%


\textbf{11.2}%
\ For the benefit of mankind, teach me, O Suraśreṣṭha, how one obtains the fruits%
\                      of [this] universal sacrifice, which [process] is praised even by the gods.%


\textbf{11.3}%
\ Maheśvara spoke:%
\ I cannot see anything comparable to your compassion towards living beings, O Bhāminī.%
\footnote{Understand dayā as instrumental: tava dayayā bhūteṣu na tulyaṃ paśyāmi. }%
\ What else could I teach concerning which there is no compassion [in you towards living beings]?%


\textbf{11.4}%
\ I heard [this] previously from Sadāśiva's mouth, O Varasundarī.%
\ Listen, O Goddess, I shall teach you the ultimate essence of Dharma.%


\textbf{11.5}%
\ Immaterial sacrifice satisfies all desires.%
\ It is undecaying and imperishable, and it removes all sins.%


\textbf{11.6}%
\ Material things present many kinds of obstacle and [their acquisition causes] great%
\                      fatigue,%
\ similarly to Indra's murder of the Brahmin [Viśvarūpa],%
\              which yielded fruits that were distributed [among trees, lands etc.].%
\footnote{Context: Viśvarūpa was a son of Tvaṣṭṛ. Viśvarūpa's heads were struck off by Indra.                          In the Bhāgavatapurāṇa, Indra's sin are distributed among the ground,                          water, trees and women. }%


\textbf{11.7}%
\ Material sacrifice can be purified by the five purifications, O Varānanā.%
\ If it is purified, then the fruits will also be pure.%
\              If it is not purified, there is no fruit.%


\textbf{11.8}%
\ The Goddess spoke:%
\ I am not sure about the five purifications, O Suraśreṣṭha.%
\ Please teach [them to] me one by one, I want to hear [them] as [they] really [are].%


\textbf{11.9}%
\ Rudra spoke:%
\ The first is the purification of the mind, then comes the purification of the substances.%
\ The third is the purification of the mantras. The next one is the purification of the ritual.%
\ The fifth is the purification of Sattva. The purification of the sacrifice is [thus] fivefold.%


\textbf{11.10}%
\ The purification of the mind is [achived] by mentally creating what is not wrong.%
\ The purification of the substances is [achieved]%
\                         by [using] substances that were not obtained by unlawful means.%


\textbf{11.11}%
\ The purification of the mantras is [achived] by [properly] joining vowels to consonants.%
\ The purification of the ritual is [achived] by not altering the proper sequence.%
\ The purification of Sattva is [achived] by the non-prevalence of Rajas and Tamas.%


\textbf{11.12}%
\ When he has purified the ritual (\skt{vidhi}) thus and performs the sacrifice,%
\ he will obtain the fruits of the sacrifice, and will not experience birth and death [again].%


\textbf{11.13}%
\ But he who performs immaterial sacrifice, O Varasundarī,%
\ will not obtain [only] its fruits, [but] of all sacrifices, without exception.%


\textbf{11.14}%
\ His sacrificial ground is Kurukṣetra, he has made his abode in the house of Truth/Sattva.%
\ His great altar is the withdrawal of the senses. His seat of kuśa grass is self-control.%


\textbf{11.15}%
\ The injunction is the various .. . He lights the fire of meditation%
\ which is flaring up by the fuel of the firewood of yoga and%
\              is abounding in the smoke of penance.%
\footnote{Or emend to °indhana-samujjvāla°, where °samujjvāla° is metri causa for °samujjvala°? }%


\textbf{11.16}%
\ The placing down of the chalice is knowledge about Śiva.%
\                                 [The oblation of] boiled rice is [when] he becomes Śiva [?!].%
\ The continuous oblation of clarified butter is poured%
\                                 with the ladle of Lambaka [uvula, lambikā?].%


\textbf{11.17}%
\ Transforming concentration into an Adhvaryu [priest],%
\              breath control will be the [other] priests.%
\footnote{Understand: dhāraṇām adhvaryuvat kṛtvā (dhāraṇā is a stem form noun). }%
\ Samādhi which involves Tarka and which is long is%
\                  the burning of the oblation[? vayas-tāpana?].%


\textbf{11.18}%
\ The sacrificial post is made up of the knowledge about Brahman.%
\              The tying of the sacrificial animal is [the mental state called] Manonmanas.%
\ His wife is Faith, O Viśālākṣī. His sacrificial ritual intention/declaration is the eternal abode.%
\footnote{Understand: padaṃ śāśvatam (pada is a stem form noun metri causa). }%


\textbf{11.19}%
\ Rice oblation is the consumption of the nectar of immortality%
\                      that arises from the victory over the five senses.%
\ The great mantra is Brahmā's sound. Expiation is the victory over breath.%


\textbf{11.20}%
\ The consumption of Soma is complete knowledge.%
\              The commencement [of the reading of the Veda] is the four yama-rules[?].%
\ The ritual water-bath is [the reading of] the epics.%
\              His garment is made of [his readings of] the Purāṇas.%


\textbf{11.21}%
\ Ritual bathing and sipping water once are [to be performed]%
\              at the confluence of the Iḍā and the Suṣumnā.%
\ Having honoured Contentment as a guest, he salutes the Brahmin as Compassion.%


\textbf{11.22}%
\ The Brahmakūrca [penance] is the Guṇātīta [state of mind], the scent of the sacrifice is%
\                 the Nirañjana [state of mind].%
\footnote{On the guṇātīta state of mind, see 9.39--43.                 Understand guṇātītatvaṃ and nirañjanatvaṃ? }%
\ [His] sacred thread is the three Tattvas.%
\                      For a shaven head he has enlightenment/teaching.%


\textbf{11.23}%
\ The four Vedas are Nivṛtti etc. His seat is the four Prakaraṇas.%
\ He should always perform a sacrifice donating the priestly fee of%
\              providing being[s] with freedom from danger.%


\textbf{11.24}%
\ The attainment of immaterial sacrifice has been taught to you, O Varānanā.%
\ [The sacrificer] will in any case obtain the fruits of upto a thousand [ordinary] sacrifices.%


\textbf{11.25}%
\ The first life-stage [life option] has been taught to you, O Varānanā,%
\ the true Dharma, which is revered by Sadāśiva and also by the [other] gods.%


\textbf{11.26}%
\ [Now] learn about brahmacarya. Listen with attention, O Śubhā.%
\ [This is] the second life-stage, O Devī, the destroyer of all sins.%


\textbf{11.27}%
\ [Here] religious observance is [now] meditation on Brahman.%
\              The Sāvitrī [hymn] is absorption in Prakṛti.%
\ The Brahmanical cord is the subtle syllable.%
\              His girdle is now contained in the three guṇas.%


\textbf{11.28}%
\ His staff is self-restraint, his bowl compassion.%
\                         Begging/alms? is liberation from saṃsāra.%
\ The tryāyuṣa [mantra] is the one beyond the two syllables[?].%
\              It[?] is embellished with the ashes of knowledge.%


\textbf{11.29}%
\ The bath-vow is speaking the truth always.%
\              It is accompanied by the purity of moral conduct.%
\ Sacrifice to Agni is the three tattvas[?].%
\                 Recitation is the sound at the aperture of Brahmā.%


\textbf{11.30}%
\ [This is] the second life-stage as Lord Śiva taught it, O Devī.%
\ I have also taught [it to] you[,] the destruction of birth and death.%


\textbf{11.31}%
\ Listen, O Long-eyed goddess, I shall teach you the forest-dweller's way of life,%
\ which is revered by the Ṛṣis and the gods, as I heard it, as it [really] is.%


\textbf{11.32}%
\ Having taken to the forest of indifference,%
\                         he should take residence in the Āśrama of niyama-rules,%
\ within walls that have the stone-strong gate of moral conduct,%
\                  with his sense faculties conquered.%


\textbf{11.33}%
\ The spiritual substratum of material objects [adhibhūta?] is his mother,%
\                      the supreme spirit is his father.%
\ the divine realm is his teacher, determination his brothers.%


\textbf{11.34}%
\ His wives are Śruti and Smṛti, his son is Wisdom,%
\              his younger brother Patience.%
\ His relative is Benevolence, his twisted hair is his bow,%
\                                  Compassion his sacred thread.%


\textbf{11.35}%
\ Sympathy is the four ways of taciturnity. All his teendők... are Endurance.%
\ He has the yama-rules for a garment made of bark, and he wears%
\               Penance for the skin of a black antelope.%


\textbf{11.36}%
\ He is seated on the highest level of non-attachment,%
\               and the firm observance is his yoga-belt.%
\ The sound of murmuring the Vedas is noise[??]. Fire sacrifice is breath-control.%
\footnote{hāvana = havana metri causa }%


\textbf{11.37}%
\ He is full of[??] conquered breaths for a deer[??].%
\              [For him] sacrifice is resolution, ritual is recitation.%
\footnote{°mṛgākūla for °mṛgākulaḥ metri causa? }%
\ His companion from among all the collected teachings[??!]%
\              of the Śāstras is self-control, compassion etc.%
\footnote{Or: [For him] the gist of the Śāstras is friendship[?], self-control, compassion etc. }%


\textbf{11.38}%
\ He should perform sacrifice to Śiva [with/as?] the worship of the eight [yogic?] practices.%
\ He is purified by the water of the five Brahma[-mantras]%
\              in the auspicious pool on the sacred banks of truthfulness.%


\textbf{11.39}%
\ Having bathed and having sipped water [there],%
\              he should take refuge at [or rather upāsayet?] the three junctures of the day.%
\ His rosary is the meaning of the Purāṇas.%
\              The pacification of mantras? is? recitation day and night.%


\textbf{11.40}%
\ His jar of epics is filled with the water of knowledge.%
\footnote{pūrṇa-m-itihāsa°: -m- is a filler. }%
\ [Tentatively:] The actions of the five [medical] procedures are suicide.%
\                                 The five kinds of pleasure are recitation.[?]%


\textbf{11.41}%
\ The Śivasaṃkalpa [hymn] is practice (sādhana),%
\                      which yields fruits of yoga accomplishments.%
\footnote{The Śivasaṃkalpa is Ṛgvedakhila 4.11 ff:        yenedam bhūtaṃ bhuvanaṃ bhaviṣyat parigṛhītam amṛtena sarvam,        yena yajñas tāyate saptahotā tan me manaś śivasaṅkalpam astu, etc.         See also Manu 11.251ab: sakṛt japtvāsyavāmīyaṃ śivasaṃkalpam eva ca. }%
\ His food is the fruit of Contentment. He conquered lust and anger.%


\textbf{11.42}%
\ His practice is the victory over the trap of hope.%
\              He prefers the joy of yoga meditation.%
\ The forest-dweller should observe his vow%
\              by providing his guests with fearlessness.%
\ This is how the Dharma of the forest-dweller has been taught in the past.%
\footnote{Gender! }%


\textbf{11.43}%
\ [If the yogin] follows, with faith and self-control, the supreme Dharma,%
\              which delivers him from Saṃsāra, removes transient existence, uproots ignorance,%
\footnote{\msNa\ only corrects °haraṇamanitya° to °haraṇam anitya° (CHECK this),                but its scribe probably meant an anusvāra at the end of °haraṇaṃ,                perhaps trying to correct the metre. He tries to correct the metre               also with anityaharaṇan tajñā°. }%
\ increases wisdom, which is fruitful, which delivers cross him from the flood of affliction,%
\ removes birth, disease and burns bad karma,%
\ he will really become a living Śiva.%
\footnote{The fourth line of this verse could be Naraharinātha's invention. }%


\textbf{11.44}%
\ Here follows the a wandering religious mendicant's Dharma.%
\              Listen, I shall teach you about it.%
\ Making joy and pain equal, he gets rid of greed and folly.%


\textbf{11.45}%
\ He should avoid honey and meat, as well as others' wives.%
\ He should avoid staying [in a place] for long and also staying at others' places.%


\textbf{11.46}%
\ He should avoid food that has been thrown away and%
\                  he should avoid a single alms round[?? the same food?].%
\ He should always refrain from accumulating wealth and from self-conceit.%


\textbf{11.47}%
\ Meditating on the subtle he can put his feet into the pure.[??]%
\ He should not get angry when [food] in not available, and when it is, he should not rejoice.%


\textbf{11.48}%
\ He should not be agitated with regards to thirst for material things or to violent anger.%
\ He should take praise and reproach equal, as well as pleasant and unpleasant things.%


\textbf{11.49}%
\ His garment is the Niyama-rules, and he is girded by the girdle of self-control.%
\footnote{Check if saṃyama is a technical term here. }%
\ He makes his mind supportless, his intellect spotless,%


\textbf{11.50}%
\ his self Earth, the Manonmana ether[?],%
\ his three staffs the three guṇas, his bowl the imperishable syllable.%
\footnote{tridaṇḍa = the three staves of the Parivrājaka MMW, check.                 Olivelle p. 173: ``There are numerous scriptural passages cited                  by the Vaisnavas that prescribe the carrying of a triple staff---that is,                 three bamboos tied together---by renouncers.''              °kṣaram avyayam would be unmetrical, so the nominative is used here. }%


\textbf{11.51}%
\ He should throw away [the distinction between?] Dharma and Adharma,%
\                      and should avoid envy and hatred.%
\ He is indifferent to the opposites [such as cold and heat, good and bad],%
\                  dwells always in truthfulness, unselfish, humble.%


\textbf{11.52}%
\ He should go on his alms round visiting seven houses at the eighth part of the day.%
\ He should not sit down, he should not stay, and he should not say `Give me!'.%


\textbf{11.53}%
\ He should live on what is available, on[?] eight bites a day.%
\ He should not stick to items of clothes, food or a bed for long.%


\textbf{11.54}%
\ He should nor rejoice in death, he should not rejoice in life.%
\ Having conquered his senses, having killed his desire, firm in his observances,%


\textbf{11.55}%
\ the Bhikṣu should never think about the past or the future.%
\ The wandering mendicant should always avoid anger, self-conceit, intoxication and pride.%


\textbf{11.56}%
\ Making indifference a bow which is strung by the strings of breath-control,%
\ he should kill the beast the sense-faculty which is%
\                         the mind with the sharp-pointed arrow of concentration.%


\textbf{11.57}%
\ He should stab the enemy that is Saṃsāra with the extremely sharp%
\                  knife of friendliness.%
\footnote{Buddhist terms. }%
\ He should defeat the rutting elephant of anger with the whirling discus of compassion.%


\textbf{11.58}%
\ His body is clad in the armour of sympathy, his quiver is full of indifference.%
\ He should constantly recall the unutterable syllable which is supreme Brahman, O Brahmin.%


\textbf{11.59}%
\ Brahmā's heart is Viṣṇu. Viṣṇu's heart is Śiva.%
\ Śiva's heart is the Junctures of the day. Therefore he should worship the Junctures.%


\textbf{11.60}%
\ [Śiva] is deliverance from the ocean of mundane existence, the path to happiness, the Brahman,%
\                  the junctures, the [sacred] syllable.%
\ [the yogin] should always, unweariedly, meditate on matchless Śiva,%
\              who is to be recognized as the manifested soul.%
\ He should take refuge in Hara, who is devoid[!] of form, colour, qualities etc.,%
\              who is the supreme aim which is difficult to aim at,%
\footnote{vihita here in the sense of `devoid'. }%
\ ... , the divine guru, who removes all pain.%


\textbf{12.1}%
\ The Goddess spoke:%
\ Harmlessness is always praised as the highest Dharma.%
\ Tell me the ultimate Dharma of those who practise hospitality.%


\textbf{12.2}%
\ Maheśvara spoke:%
\ Hear the ultimate Dharma of harmlessness and[?] that of the ones who practise hospitality.%
\ O beautiful-eyed goddess, [if] all the three worlds, full of wealth,%


\textbf{12.3}%
\ [were handed over as] a gift to [a Brahmin who] knows%
\                 the four Vedas, [that] cannot be compared to somebody who avoids doing harm.%
\ Hear [now] the Dharma of the hospitable ones. I'll teach it [to you], O beautiful one.%


\textbf{12.4}%
\ This is an old story of what happened once in a city called Kusuma [i.e.\ Pāṭaliputra].%
\ There was a famous and wise man called Vipula, Kapila's son.%


\textbf{12.5}%
\ He always followed his Dharma, he conquered anger, he spoke only the truth and he also conquered his senses.%
\ He was favourable to Brahmin. He was not ungrateful and he was my determined devotee.%


\textbf{12.6}%
\ He was rich and he worshipped[?] his guests. He was generous, restrained, and merciful.%
\ He wealth always came through just means. He always stayed away from illegal actions.%


\textbf{12.7}%
\ He had a beautiful wife whose face was as pure as the disk of the moon.%
\ Her breasts were round and elevated, she was lovely, a source of all pleasure.%
\ She was faithful, devoted to her husband and his needs.%


\textbf{12.8}%
\ Now, once there was an eclipse of the sun.%
\ Three quarters [of the sun] were eclipsed, and it was in the dark half of the month Mādhava.%


\textbf{12.9}%
\ Eager to take a ritual bath, the king and all citizens went down [to the river].%
\ They were worshipping the gods and the deceased ancestors according to rule.%


\textbf{12.10}%
\ Some sacrificed in the fire, some fed the Brahmins,%
\ some gave donations, others praised the deity.%


\textbf{12.11}%
\ Some people practised yoga meditation, others were engrossed in five-fire penance.%
\ While all the royals and other people were doing this all around,%


\textbf{12.12}%
\ Vipula too, there at the confluence of the Gaṅgā and the Gaṇḍakī,%
\footnote{Note \skt{gaṇḍakī} metri causa for \skt{gaṇḍakī} in pāda b. }%
\ together with his wife, performed a bath, and, attired in linen clothes,%


\textbf{12.13}%
\ was satiating the deities, the gurus, the Brahmins and others.%
\ Then, jumping on the possibility, a Brahmin came up [to them] as a guest.%


\textbf{12.14}%
\ The wife got infatuated with that Brahmin's extreme beauty.%
\ The Brahmin [felt] the same. His beauty was unparalleled.[?]%
\footnote{Pāda d is slightly suspect. The expression \skt{rūpeṇāpratimo/°pratimā bhuvi} is                         common in the Mahābhārata and in the Purāṇas. Is that what was meant here?                        May a dual have been intended? }%


\textbf{12.15}%
\ Their gaze got fixed on each other mutually.%
\ Vipula joined his hands [and said:] ``O virtuous Brahmin,%


\textbf{12.16}%
\ I am at your service, be gracious to me now, O great Brahmin.''%
\ [He offered him?] his wife, servants, cattle, village and all kinds of jewels,%


\textbf{12.17}%
\ Having been addressed and greeted hospitably by Vipula, the Brahmin spoke:%
\ ``If you really mean to give, your heart is very generous.''%


\textbf{12.18}%
\ Vipula spoke:%
\ ``My heart is generous, generousity is the fruit of auserity.%
\ Just command me quickly, O Brahmin. What is your desire?%
\ There is nothing that should not be donated to a Brahmin, beginning with one's own head, O Brahmin.''%


\textbf{12.19}%
\ The Brahmin spoke:%
\ ``If you talk like this, my dear, give me your beautiful wife.%
\ Be happy, may you be fortunate, and may you prosper eternally!''%


\textbf{12.20}%
\ Vipula spoke:%
\ ``Accept my wife who has nice buttocks, and is young and beautiful,%
\ unreproached, large-eyed and whose face resembles the full-moon.''%


\textbf{12.21}%
\ The wife spoke:%
\ ``How can you abandon me, my lord? How can you leave me who am sinless?%
\ How can somebody leave a wife who is extremely kind and faultless?%


\textbf{12.22}%
\ A wife is a man's friend in this world and in the other world.%
\ [Even if] a man gives enormous donations or performs numerous sacrifices,%


\textbf{12.23}%
\ or performs hard penance, he cannot get to heaven without having a son.%
\ I have heard that this was taught by the ancestors, and by Brahmins in my presence.%


\textbf{12.24}%
\ The sonless cannot obtain heaven. I have heard this so many times!%
\ Mandapāla, the great Brahmin went to heaven as a reward of his austerities,%


\textbf{12.25}%
\ and by his numerous donations and various sacrifices,%
\ and by reciting the Vedas, and performing sacrifices, that[?] great Brahmin.%


\textbf{12.26}%
\ But when he reached the gate [of heaven], it was blocked by the celestial messengers:%
\ ``The sonless cannot get to heaven, not even by hundreds of sacrifices.''%


\textbf{12.27}%
\ Mandapāla, the great sage was thus informed and he fell from heaven.%
\ The clever Brahmin begot sons who were born from a Śāraṅga-bird.%


\textbf{12.28}%
\ By the virtue of this, he reached heaven unobstructed.%
\ I am a wife, who is for the protectors of the race, and a wife because I bear [sons].%


\textbf{12.29}%
\ Taking a wife is for the sake of having sons according to the Śāstras.%
\ You can give the Brahmin all the wealth at home,%
\      all the villages, the stations of herdsmen and the houses,%


\textbf{12.30}%
\ but please don't give me away!''%
\ Having heard his wife's speech, Vipula spoke again.%


\textbf{12.31}%
\ Vipula spoke:%
\ ``Alright, my beautiful wife, I know! Good, good, my faithful wife!%
\ I am beaten by this speach and I am satisfied with it.%


\textbf{12.32}%
\ Today the Brahmin came up to me at the time of eclipse, and asked me.%
\ I promised him that I would give [you away]. If I don't give [you to him], I'll go to hell.%


\textbf{12.33}%
\ If I go to hell along with my family/decendants,%
\ I will not be freed from hell, O Yaśasvinī, for millions of eons,%


\textbf{12.34}%
\ as long as millions of births.%
\ I can see something bad, O Devī, from not giving, Varavarṇinī%


\textbf{12.35}%
\ and I can see eternal good in heaven from giving.%
\ I have never ever lied, observing the vow of truthfulness.%


\textbf{12.36}%
\ Transgressing the law of truth, I would not follow any other law.%
\ You mentioned earlier that the wife is one's Dharmic friend.%


\textbf{12.37}%
\ If you are indeed my Dharmic friend, then the time has come CHECK%
\ was Dharma himself coming to me in the form of a Brahmin%


\textbf{12.38}%
\ to test me. O my dear, please don't cause me trouble.%
\ The Unmenifest is my mother, Brahmā is my father, Intelligence is my wife, self-control is my friend.%


\textbf{12.39}%
\ Dharma is my son, Ritual is my guru. These are my relatives.%
\ The best planet... the Sun, the best one among the rivers is the Gaṅgā.%


\textbf{12.40}%
\ The best day is at waning moon, the best man is the Brahmin.%
\ I have given you to the Brahmin to serve him.%
\ Having given everything to the Brahmin, I'll resort to the forest.''%


\textbf{12.41}%
\ Śaṅkara [i.e.\ Śiva] spoke:%
\ The wife remained silent, her eyes filled with tears.%
\ [Vipula] took her hand and the long-eyed one woman was presented to the Brahmin.%


\textbf{12.42}%
\ I am ready to give all the wealth I have at home, all the gold and the cattle,%
\ O great Brahmin, and the .....%


...


\textbf{12.108}%
\ I went to Soma's world, and he gave me the magical fruit.%


\textbf{12.109}%
\ I cannot give you another one. Go now to Soma's city.%


...


\textbf{12.136}%
\ Gāyatrī, who is the mother of the Vedas, and beautiful Sāvitrī.%


...


\textbf{13.17}%
\ From ritual arises sacrifice. From sacrifice arises smoke.%


\textbf{13.18}%
\ From the clouds food arises. From food living beings arise.%
\ From food arises rasa. From rasa arises blood.%


\textbf{13.19}%
\ From blood flesh arises. From flesh fat arises.%
\ From fat bones arise. From the bones marrow arises.%


\textbf{13.20}%
\ From marrow arises semen. And man is born from semen.%


...



\textbf{15.1}%
\ The goddess spoke:%
\ A certain `soul being' was mentioned. What are [its] characteristics?%
\                  I do not know about its location or form or colour, O Īśvara.%


\textbf{15.2}%
\ This is what I'm  curious about. Drive away my doubts, Parameśvara.%
\                  I cannot see anything else [as vitally important?]. Teach me the details of the soul.%
\footnote{\skt{Pāda} c is suspicious. It may have originally read \skt{na cānyad eva yācāmi} (``I am not asking for anything else''). }%


\textbf{15.3}%
\ Īśvara spoke:%
\ O Goddess, who would be able to talk about the characteristics of the soul?%
\                  There is nothing like the form and colour of the soul or its location.%


\textbf{15.4}%
\ Its pervasive, omnipresent, subtle, it exists dwelling in everything.%
\                  It is supportless, it is not contained in anything, it is unparalleled and spotless.%


\textbf{15.5}%
\ As fire [hidden] in fire-kindling sticks[?] is not perceivable in the wood,%
\                 similarly the soul cannot be seen although it is dwelling in the body, O Sundarī.%
\footnote{Note \skt{paśyeta} as a passive form. }%


\textbf{15.6}%
\ Just as ghee can and cannot be seen in[?] curd[?],%
\                  in the same way the soul in the body can and cannot be seen.%


\textbf{15.7}%
\ The goddess spoke:%
\ Is it without any direct proof? Is there no way to directly see any proof [of its existence]?%
\                  How is it pervasive, O Mahādeva? How can it be omnipresent?%


\textbf{15.8}%
\ Maheśvara spoke:%
\ It is doubtlessly pervasive, omnipresent, it is Śiva.%
\                   It can be perceived through its contact with the senses. [That is] the direct perception of the evidence of%
\                         [the existence of] the soul.%


\textbf{15.9}%
\ As air in the sky is endowed with the qualities of sound and touch,%
\                  similarly one can perceive the soul through the functioning of its qualities and in no other way.%
\footnote{Although it is difficult to be certain whether the majority of the MSS read °\skt{ceṣṭena} or \skt{°veṣṭena},                        I suppose that the somewhat irregular \skt{°ceṣṭena} is the right reading                         and that it stands for a more standard \skt{°ceṣṭayā}. }%


\textbf{15.10}%
\ The goddess spoke:%
\ The soul was mentioned earlier as being pervasive and also omnipresent.%
\                  [I suppose] you said that [only] idly. In this case, why does [the soul] die?%
\footnote{pāda c strange structure, but frequent in this text. }%


\textbf{15.11}%
\ Īśvara spoke:%
\ O Goddess, nobody's soul ever dies, O Surasundarī.%
\                  As [in the case of] space inside a pot, and space outside it,%


\textbf{15.12}%
\ there is no perceivable difference when the pot is broken to pieces, O Viśālākṣī.%
\                   [Similarly,] when the body perishes, O goddess, there is no perceivable destruction [of the soul].%


\textbf{15.13}%
\ It is extremely subtle, omnipresent, pervasive, it is the supreme soul, it is imperishable.%
\              It is outside and inside the living beings. It is immovable and moving.%
\footnote{Note \skt{paramātmānam} for \skt{paramātmā}. }%


\textbf{15.14}%
\ It is immeasurable, imperishable, unmanifest and manifest.%
\                  It appears as having the qualities of all the senses but is devoid of senses.%


\textbf{15.15}%
\ Thus have I briefly described to you, O Mahādevī, the soul. O Varavarṇinī,%
\                 what else would you like to hear?%


\textbf{15.16}%
\ The goddess spoke:%
\ O Mahādeva, O Īśāna, Īśvara! Tell me what is the best in essence.%
\                 I would like to hear it, O Deveśa. Tell me for the benefit of mankind.%
\footnote{Pāda d is a clumsy paraphrase of the common \skt{mānuṣāṇāṃ hitāya ca} or similar phrases. }%


\textbf{15.17}%
\ Īśvara spoke:%
\ The best life-stage is that of the householder. The best caste is the Brahmin.%
\                   The best ritual is the \skt{aśvamedha}. The best recitation is the \skt{aghamarṣaṇa}.%


\textbf{15.18}%
\ The best god is Hari. The best river is the Ganges.%
\                   The best austerity is fasting. The best pilgrimage-place is Surahrada.%
\footnote{\skt{anāśanas} (or \skt{anāsanas} in most MSS) stands for \skt{anaśanas} (found only in \msNc) but the latter would cause                 a metrical problem, namely both the second and third syllables would be two short. This is why I retained                        the non-standard form \skt{anāśanas}. }%


\textbf{15.19}%
\ The best cloth is linen. The best ornament is fame.%
\                  The best Śruti is the Mahābhārata. The best of vows is compassion.%


\textbf{15.20}%
\ The best donation is the freedom from danger. The best sense-faculty is the mind.%
\                  The best perception is knowledge. The best word is the truth.%
\footnote{Note the form \skt{saṃgrahaṣu} in \skt{pāda} c for \skt{saṃgraheṣu} (as in \msNc) metri causa }%


\textbf{15.21}%
\ The best weapon is the bow. The best relatives are the mothers.%
\                  The best medicine is knowledge. The best doctor is Śiva's syllable.%


\textbf{15.22}%
\ The best letter is `a'. The best Dharma is non-violence.%
\                  The best domestic animal is the cow. The best person is the king.%


\textbf{15.23}%
\ The best month is Mārgaśiras. The best of the four aeons is the Kṛta.%
\                 The best season is spring. The best path of the Sun is the northern one.%
\footnote{Understand \skt{māsi} in \skt{pāda} a as \skt{māseṣu}. }%


\textbf{15.24}%
\ The best day is the day of the new-moon. The best planet is the Sun.%
\                   The best among women are Lakṣmī and Dhṛti [two of Dharma's thirteen wives]. The best Vasu is Agni.%


\textbf{15.25}%
\ The best Ṛṣi is Uśanas. The best brightness is the Moon['s].%
\                  The best constellation is Abhijit. The best CHECK is time.%


\textbf{15.26}%
\ The best Veda is the Sāmaveda. The best mountain is the Himalayas.%
\                  The [best] among trees are the Aśvattha and Vaṭa. The best being is consciousness??%


\textbf{15.27}%
\ The [best] of all knowledge is the spiritual one (Sāṃkhya?). The best speech is the truthful one.%
\                   The best demon is Prahlāda. The guard? of the Yakṣas is Kubera.%


\textbf{15.28}%
\ The best wind is Marīci?.  The best among the deer is the reddish one.%
\         The best deity to be propitiated is Nārāyaṇa. The best ancestor is Brahmā.%


\textbf{15.29}%
\ O goddess, having told you this summary of the essence of everything in an extracted form, O Varānanā,%
\                         what shall I tell you further?%
\vfill\pagebreak\begin{center}{\large\textbf{ Chapter Sixteen 
}}\end{center}


\textbf{16.1}%
\ The goddess spoke:%
\ Now I would like to hear the exposition of the essence of yoga.%
\ Furthermore teach me about the Karaṇa [exercises, practice?], according to the rules, O Sureśvara.%


\textbf{16.2}%
\ Īśvara spoke:%
\ Listen, o Devī, I shall teach you the supreme essence of yoga,%
\ by knowing which people will cease to experience the fetter of mundane existence.%


\textbf{16.3}%
\ [One can be] a Brahmin-slayer, a violator of his teacher's bed, a drunkard, a thief%
\ or can be born into a mixed caste: it [i.e. yoga] will eliminate all [of his sins].%


\textbf{16.4}%
\ He who engages in Prāṇāyāma for [just] half a moment or for a moment,%
\ [and] focuses on the object to be visualized (dhyeya) will have those sins destroyed ... [kṣaṇāt? cf. parallel]%


\textbf{16.5}%
\ Mighty [balavat] Yama, the cruel Ender, frightening-looking death%
\ will not take possession of the brave yogin.%


\textbf{16.6}%
\ Just as the faults of all metals are burnt out by blowing [the fire that heats] them,%
\ in the same way sins are surely burnt away by the control of the breath.%


\textbf{16.7}%
\ There is nothing like a thousand Aśvamedha sacrifices, a hundred Rājasūya rituals%
\ or a hundred [rounds of] prāṇāyāma.%


\textbf{16.8}%
\ By sacrifice, one can reach the gods [Veda?]. The result of austerities is sovereignty [in yoga?].%
\                  By renunciation, one reaches Brahmā's place, and by indifference, Prakṛti's abode.%


\textbf{16.9}%
\ By knowledge, one attains kaivalya and the supreme and eternal Brahman [Sāṃkhya?].%
\                  These are taught to be the five paths according to the rules.%


\textbf{16.10}%
\ he will get beyond all sins and will attain immortality.%
\                  If the knower of yoga practises yoga for half a moment or for a moment,%


\textbf{16.11}%
\ Even if he practises diligently, until he knows the Truth,%
\                  he will surely abide in Brahmā's and Viṣṇu's homes, O Sundarī.%


\textbf{16.12}%
\ Then when his merits are exhausted, he will be born in the world of mortals, in a noble family.%
\                  He will experience thousands of karmas, while he has all possible desires.%


\textbf{16.13}%
\ He should practise only yoga, and he will be a man who remembers his own previous births.%
\                  Crossing the ocean of mundane existence, he will obtain Śivaness.%


\textbf{16.14}%
\ The goddess spoke:%
\ I wish to hear about the method of yoga. Teach me, O Puruṣottama, O Sureśvara, about%
\ meditation, concentration and the Powers.%


\textbf{16.15}%
\ Maheśvara spoke:%
\ Listen, I shall teach you the method of yoga, the destroyer of the noose of existence.%
\ [With his body] purified and his mind concentrated, the yogin should sit%
\                 down assuming a sitting posture (āsana) in a place which is devoid of humans and noise,%
\ and he should think of the Supreme Soul.%


\textbf{16.16}%
\ [The āsanas are:] padmaka, svastika, niṣkala, añjali, ardhacandra, daṇḍa, paryaṅka, and bhadra.%


\textbf{16.17}%
\ He should practise yoga by assuming [any one of] these āsanas.%


\textbf{16.18}%
\ Withdrawal of the senses (pratyāhāra), meditation (dhyāna), breat-controll (prāṇayāma), concentration (dhāraṇā),%
\ reflection (tarka), and samādhi: these are called the six-limbed yoga/ yoga with six ancillaries.%


\textbf{16.19}%
\ That [method] which draws in the senses that are clinging on to the objects again and again [see DhP]%
\ with the help of the mind is called withdrawal of the senses.%


\textbf{16.20}%
\ O Devī, [the yogin] should concentrate on the [five]%
\                 sense-objects beginning with sound after he has made them into a ball.%
\ His passions gone, dwelling in samādhi, he should join the object of meditation with the object[?].%


\textbf{16.21}%
\ The Self is the meditator (dhyātṛ), the mind is meditation (dhyāna),%
\                         the object of meditation (dhyeya) is Pure Supreme Śiva%
\ As regards supreme sovereignty, [that] is the only aim in it [i.e. in dhyāna].%


\textbf{16.22}%
\ Inhalation, breath retention, then exhalation,%
\                  and the tranquillized one: prāṇāyāma is fourfold.%


\textbf{16.23}%
\ During inhalation, the wise one should establish the fire through his great-toe.%
\                  By breath retention he should stop it [i.e. the fire] and visualize it [i.e. himself] as being burnt.%


\textbf{16.24}%
\ Then, while exhaling, he should imagine himself as reduced to ashes.%
\                  Now his Self has a purified body, one which is as spotless as a clear crystal.%


\textbf{16.25}%
\ [When this is maintained for] twelve measures of time, that is called nirvāṇa/exhalation?%
\                   Concentration (dhāraṇā) is twice as long as breath-control (prāṇāyāma), there is no doubt about it.%


\textbf{16.26}%
\ As regards yoga, it (dhāraṇā?) is said to be three times as long,%
\              in saṃkrama it is four times longer.%
\              In case of ritual suicide (utkrānti) is concerned,%
\              it is five times longer. [To reach] yogic Powers (yogasiddhi) [it takes] six times longer.%


\textbf{16.27}%
\ [The yogin should] always be practising yoga with the six ancillaries.%
\                   Yoga is taught as having two forms: mental (mānasa) and simultaneous?? (yaugapadya).%


\textbf{16.28}%
\ [The yogin] can meditate on the supreme subtle one only mentally, without performing breath-control:%
\                   that type of yoga is called mental [yoga] (mānasa).%


\textbf{16.29}%
\ [If the yogin] controls his breath with his mind, and his mind with breath-control,%
\                 and thus meditates on the supreme subtle one, that is called simultaneous [yoga] (yaugapadya).%


\textbf{16.30}%
\ I shall teach you the signs of success in yoga, listen, O Sundarī.%
\ When a conch-shell, kettle-drum, mṛdaṅga-drum, flute or dundubhi-drum is beaten,%
\                  he will not perceive [the sound] when he has reached such-ness [i.e.\ Śivaness].%


\textbf{16.31}%
\ Similarly, he will not be able to tell cold from heat, joy from sadness, he will not experience thirst or hunger%
\  or pain, when he attains success in yoga, O Sundarī.%


\textbf{16.32}%
\ This is how I taught the technique of yoga in a nutshell, O Devī, as a reply to your question, O Sundarī.%
\ What else shall I teach you?%
\footnote{Note 'smi. }%


\textbf{16.33}%
\ The goddess spoke:%
\ Tell me about the liberation from mundane existence without yoga, O Deveśa!%
\ O Mahādeva, [that could] free [one's] mind of doubts/hesitation.%
\footnote{Understand \skt{nirvikalpakaraṃ manaḥ} as \skt{manonirvikalpakaraṃ} }%


\textbf{16.34}%
\ Maheśvara spoke:%
\ Sighing is Sadāśiva, a deep breath is supreme Śiva.%
\ In between the two, there is Śiva the supreme and imperishable Self.%


\textbf{16.35}%
\ For one [who knows this], there is neither yoga meditation and nor karaṇa.%
\ He is liberated by [the] mere knowledge [of this]. What else do you want to ask?%


\textbf{16.36}%
\ I shall teach you another kind of knowledge. Listen, O Devī, listen to me.%
\ Listen in short to the [its] exposition as constructed in the five śāstras,%
\ in Sāṃkhya, in yoga, in the Pañcarātra, in Śaivism and in the Vedas.%
\footnote{Note how there is no question from Devī after \skt{kim anyat paripṛcchasi} and how we begin a new topic instead.                Note also how \skt{saṃkṣepa} stands either for \skt{saṃkṣipta°} or \skt{saṃkṣepena/saṃkṣepataḥ}. }%


\textbf{16.37}%
\ The quintessential yoga which is established in Sāṃkhya, and which%
\                 is for [liberation from] the terrible ocean of mundane existence,%
\ and which I am teaching you now,%
\ is certainly there for you as a certainty in essential yoga [teachings], and in the Pañcarātra,%
\ in the Vedas, and in Śaivism.%


\textbf{16.38}%
\ If all of [his] senses beginning with smelling,%
\ and also his mind, are dissolved, so to say,%
\ and if he suppresses all sensations (bhāva) with his mind,%
\ he will attain his aim and will find refuge in Śiva.%


\textbf{16.39}%
\ [When there is] motionlessness of all senses beginning with hearing,%
\ and his attention (cittaṃ), controlled by his mind, becomes focused (ekāgra),%
\ his body will slowly disappear.%
\ This is called `success in union' by the experts.%


\textbf{16.40}%
\ First, he should slowly stop his mind [or subj. = manas?], subduing the sense[s]%
\ until it [the senses] dissolve[s] together with the mind [see above].%
\                                 Thus [the yogin's] body is rendered unconscious/senseless.%
\ [The yogin] certainly attains this yoga in its entirety, namely meditation and samādhi.%
\ Why is it [if] somebody does not seek [this] essence extracted[? mathitam might be better]%
\                                 from ten thousand million books?%


\textbf{16.41}%
\ He has conquered his joy in his Self, [instead] he rejoices in samādhi%
\                         and he has also taken refuge in indifference (vairāgya).%
\ When the end comes, his mind will remain [in] corporal form???%
\ That is to be known as the highest path, Śiva's abode, which puts an end to mundane suffering.%
\ And this is taught as `completion' [niṣṭhā!] in the Vedānta (in the Upaniṣads?). Why should anyone resort to any other teaching?%


\textbf{16.42}%
\ On [upari, here with loc.] the pericarp of the heart-lotus, there is a sun, illuminating the intermediate space.%
\ There is a lamp lit by the shining of the most dense mass of rays of its light,%
\ which having pierced the mouth at the soft palate, goes upwards through the soft palate towards the top of the head.%
\ Those practising yoga leave for Śiva's supreme abode through the door on the top of their heads.%


\textbf{16.43}%
\ [Tentative:] Kṛṣṇa, the highest of the darkest ones, the extremely great one,%
\                         who is essentially the splendour of light/who shines/is sharp, the one who has never been born,%
\ the supporter of the world and the non-world and of the earth, husband to Śrī, abiding in the breath,%
\ the imperishable creator, the imperishable cause,[?] he the [all-]pervading, the arranger/distributor?, ...?%
\ Viṣṇu, ..., the lord of the universe, the omniscient one.%


\textbf{16.44}%
\ is viewing the senses and the mind inside the body through the Buddhi which is transformed by meditation.%
\ That Puruṣa is located in the abode in the heart-lotus, he who gives us exhalation and inhalation.%


\textbf{16.45}%
\ He who is intensifying energy, the unborn one, who is a very dense mass, who is hidden in the garland of knots,%
\ the embodiment, who follows the embodied form, ...%
\ piercing the knot together with the bond, abandoning the%
\                 objects of the senses and attachment like poison, focusing their states of mind,%
\ they can see him, the God, who is devoid of [even] a small portion of the Guṇas, and who is formless light.%


\textbf{16.46}%
\ He whose inner self is energy, and who is hiding in the contracted place%
\                         in the abode which is the hollow of the lotus [in the heart],%
\ who resembles the Moon's light, who is always hidden in the pericarp among the spotless petals,%
\ is, while remaining in that place, the abode of the three worlds and the home of all beings,%
\ free of bondage, the one with the crescent moon, being at the Tattva above space.%


\textbf{16.47}%
\ These are all the Tattvas, O Devī.%
\ The five-fold classification has been taught in short.%
\ What other topic do you wish to hear,%
\ [something] related to liberation from saṃsāra?%


\textbf{16.48}%
\ O God, I am satisfied. Now my doubts have been removed.%
\ Now you are a gracious supreme Lord, O Īśvara!%
\ Now I have heard FROM you the power of the fruits of merit.%

...



\textbf{20.1}%
\ Vigatarāga spoke:%
\ I would like to learn about the twenty-five Tattvas truely.%
\                  Teach me now so that my doubts could be dispelled.%
\footnote{ This chapter echoes and is partly based on MBh 12.247.1-10 (Mokṣadharma, see                        parallel passages in the apparatus):                

                        \skt{bhīṣma uvāca \danda
                        bhūtānāṃ guṇasaṃkhyānaṃ bhūyaḥ putra niśāmaya \danda
                        dvaipāyanamukhād bhraṣṭaṃ ślāghayā parayānagha \twodanda}1
                        \skt{dīptānalanibhaḥ prāha bhagavān dhūmravarcase \danda
                        tato 'ham api vakṣyāmi bhūyaḥ putra nidarśanam \twodanda}2
                        \skt{bhūmeḥ sthairyaṃ pṛthutvaṃ ca kāṭhinyaṃ prasavātmatā \danda
                        gandho gurutvaṃ śaktiś ca saṃghātaḥ sthāpanā dhṛtiḥ \twodanda}3
                        \skt{apāṃ śaityaṃ rasaḥ kledo dravatvaṃ snehasaumyatā \danda
                        jihvā viṣyandinī caiva bhaumāpyāsravaṇaṃ tathā \twodanda}4
                        \skt{agner durdharṣatā tejas tāpaḥ pākaḥ prakāśanam \danda
                        śaucaṃ rāgo laghus taikṣṇyaṃ daśamaṃ cordhvabhāgitā \twodanda}5
                        \skt{vāyor aniyamaḥ sparśo vādasthānaṃ svatantratā \danda
                        balaṃ śaighryaṃ ca mohaś ca ceṣṭā karmakṛtā bhavaḥ \twodanda}6
                        \skt{ākāśasya guṇaḥ śabdo vyāpitvaṃ chidratāpi ca \danda
                        anāśrayam anālambam avyaktam avikāritā \twodanda}7
                        \skt{apratīghātatā caiva bhūtatvaṃ vikṛtāni ca \danda
                        guṇāḥ pañcāśataṃ proktāḥ pañcabhūtātmabhāvitāḥ \twodanda}8
                        \skt{calopapattir vyaktiś ca visargaḥ kalpanā kṣamā \danda
                        sad asac cāśutā caiva manaso nava vai guṇāḥ \twodanda}9
                        \skt{iṣṭāniṣṭavikalpaś ca vyavasāyaḥ samādhitā \danda
                        saṃśayaḥ pratipattiś ca buddhau pañceha ye guṇāḥ \twodanda}10
         }%


\textbf{20.2}%
\ Anarthayajña spoke:%
\ How can you possibly ask me to reveal everything to be visible directly?%
\ [But] I made a decision that [whenever being] questioned, I am to speak.%
\ Listen, I shall teach you the supreme essence of the reality levels/principles (\skt{tattva}).%


\textbf{20.3}%
\ That which has no beginning, no middle part and no end,%
\                 and is not to be known even by the gods,%
\ that which is extremely subtle and extremely large, supportless and spotless,%
\footnote{Note that the key terms (\skt{ādi, madhya, anta, sūkṣma}) in this verse are to be found in VSS 1.1ab:                

               \skt{ anādimadhyāntam anantapāraṃ
                 susūkṣmam avyaktajagatsusāram} }%


\textbf{20.4}%
\ inconceivable, immeasurable, imperishable, devoid of syllables,%
\ that which is everything and everywhere and that which is pervasive,%
\                         exists covering everything.%


\textbf{20.5}%
\ It appears to have the qualities of all the sense faculties%
\                          but is devoid of all sense faculties.%
\ It is not subject to ageing, it is immortal and unborn. It is peaceful, it is%
\                 the supreme soul, it is undecaying Śiva.%
\footnote{I take \skt{ajarāmarajaḥ} in \skt{pāda} a as \skt{ajaro 'maro 'ajaś ca}. }%


\textbf{20.6}%
\ It is characterised by being unobservable, it is self-abiding, it is Brahmā,%
\                         it is called Puruṣa.%
\ It is to be known as the twenty-fifth [Tattva], the Lord (\skt{prabhu}) who destroys death and rebirth.%


\textbf{20.7}%
\ He is free of the stain of having parts[?], and is devoid of the fifty voids.%
\ As a waterbird is not stained by the water while swimming in it,%
\ similarly [the Puruṣa] is not stained even by hundreds of sinful acts.%


\textbf{20.8}%
\ The twenty-fourth Tattva is Prakṛti ...%
\footnote{Understand \skt{caturviṃśati} as \skt{caturviṃśaṃ}. }%
\ It is in fact to be known as Vikṛti (`Modification') by the wise.%


\textbf{20.9}%
\ All [the other Tattvas below Prakṛti], Buddhi, Ahaṃkāra etc.%
\                         originate in Prakṛti.%
\ Earth etc. [up to Buddhi?] dissolve in Vikṛti one by one.%


\textbf{20.10}%
\ The Mati Tattva [= Buddhi] is the twenty-third.%
\                        It possesses qualities such as Dharma [dharmic?].%
\ Know it as the perceiver of the soul, produced by an abundance of Sattva.[?]%
\footnote{For my emendation of \skt{vidhi} to \skt{viddhi}, see \skt{viddhi} in 20.12b below. }%


\textbf{20.11}%
\ The twenty-second Tattva is Ahaṃkāra according to the wise.%
\ [This is the Tattva that] says: CHECK Parākhya...  are mine!%
\                        It is produced by an abundance of Rajas.%


\textbf{20.12}%
\ Know the twenty-first Tattva as Hollowness (\skt{suṣira}) [= \skt{ākāśa}], O Brahmin.%
\ Hollowness is beyond Sound [but] it is characterised by the quality of Sound.%
\footnote{Note that from now on in this chapter, \skt{guṇa} is used in the sense of the \skt{tanmātra} of Sāṃkhya philosophy        and that the word \skt{tanmātra} does not occur in the VSS. }%


\textbf{20.13}%
\ The seven [diatonic musical] notes (\skt{svara}), the three basic scales (\skt{grāma}),%
\                and the twenty-one modal scales (\skt{mūrchana}).%
\ The forty-nine hexatonic and pentatonic scales (\skt{tāna}). The classification of Sound%
\                                 includes these and other [classes].%


\textbf{20.14}%
\ These and many other are the classes of sounds, O Brahmin.%
\ [This] has been declared by the experts on musical notes:%


\textbf{20.15}%
\ the sounds of flutes, tambourines, lutes, kettle-drums,%
\ conch-shells, bass-drums and gongs.%


\textbf{20.16}%
\ Listen, O excellent Brahmin, I shall teach you the element (\skt{dhātu}) of space.%
\ [Space is present in the following ten bodily locations:]%
\                 the anus, the sexual organs, the stomach, the neck, the two ears[?], the mouth, the two%
\                                         nostrils,%


\textbf{20.17}%
\ and the tenth, the [cavity of the] heart. The body originates in space.%
\ Next I shall teach you something else. Listen to it, O excellent Brahmin.%


\textbf{20.18}%
\ Ten element-guṇas (\skt{dhātuguṇa}) are to be known for each of the five elements[?] (\skt{bhūta}).%
\ The qualities of Space are Sound, pervasion and `perforatedness' [being pervaded],%


\textbf{20.19}%
\ its being supportless, independent and unmanifest, invariableness,%
\ not being restrainable, being an element, and ...,  [?].%


\textbf{20.20}%
\ The birth of Wind is then from the Space \skt{dhātu}.%
\ Together with the previous Sound \skt{guṇa}, Wind has Touch as its guṇa.%


\textbf{20.21}%
\ I have already described Sound, listen to Touch, O excellent Brahmin.%
\ Hard, smooth, slippery, soft, sticky, sharp, fluid,%


\textbf{20.22}%
\ rough, rugged, pointed[?], cold, hot: these are [the] twelve [\skt{vāyuguṇa}s].%
\ It is the body that senses both pleasant and unpleasant touches.%


\textbf{20.23}%
\ Prāṇa, Apāna, Samāna, Udāna and Vyāna,%
\ Nāga, Kūrma, Kṛkara, Devadatta, Dhanaṃjaya:%


\textbf{20.24}%
\ These are said to be the ten main Winds, O excellent Brahmin.%
\ Dhanaṃjaya is [responsible for] noise, Devadatta [for] yawning,%


\textbf{20.25}%
\ Kṛkara constantly causes hunger, Kūrma is [is responsible for] the opening of the eyes.%
\footnote{Kṛkara in other texts usually performs sneezing (\skt{kṣut}), here                it seems that \skt{kṣudha}° stands for \skt{kṣudhā}° metri causa. }%
\ Nāga constantly opens [up things] and nourishes, O Brahmin.%


\textbf{20.26}%
\ Prāṇa makes living beings inhale and exhale.%
\ It is called Prāṇa because it sets [beings] in motion (\skt{prayāṇaṃ kurute}).%


\textbf{20.27}%
\ Apāna takes people's[?] food down.%
\ It gets rid of semen and urine, that is why it is called%
\                         Apāna [the 'down and out' Wind].%


\textbf{20.28}%
\ The Wind called Samāna brings into equilibrium that which has been drunk,%
\                 the food that has been eaten, the blood [and the three humours] Pitta,%
\                         Kapha and Anila.%


\textbf{20.29}%
\ The Wind called Udāna causes the lower lip and the mouth to tremble,%
\                  it irritates the eyes and the limbs and%
\                  it disturbs the vital organs.%


\textbf{20.30}%
\ Vyāna bends the limbs, [it makes the body] deformed, [it causes] illness and irritaion.%
\ It is said to destroy pleasure ... is called Vyāna.%


\textbf{20.31}%
\ [Everything] concerning the section on the ten Winds has been%
\                         taught by me, O excellent Brahmin.%
\ [Now] listen as I teach you the ten other \skt{guṇa}s of Wind.%


\textbf{20.32}%
\ The Wind has [these \skt{guṇa}s]: unsettledness, touch, presence in speech,%
\                         independence,%
\ strength, quickness, release, movement and performing actions and existence.%
\footnote{While I hesitate to emend this verse to fully correspond to the                         very similar one in the MBh, my translation partly reflects the                        latter. }%


\textbf{20.33}%
\ Fire is created by Wind. Its \skt{guṇa} is Form.%
\footnote{Understand \skt{sṛjas} as \skt{sṛṣṭas} and \skt{tadrūpaṃ guṇam ucyate} as \skt{tadguṇaṃ rūpam ucyate} }%
\ There are three guṇas of Fire together with Sound and Touch.%
\footnote{I understand \skt{śabdasparśasama jyotis triguṇaṃ} as \skt{śabdasparśena saha jyotis triguṇaṃ}. }%


\textbf{20.34}%
\ Sound and Touch have been discussed before, therefore [now] hear about the Form guṇa.%
\ Short, tall, minute, gross and circular,%


\textbf{20.35}%
\ square, ???, triangle and hexagon.%
\ Light, dark, red, blue, yellow, brown,%


\textbf{20.36}%
\ dark-blue, golden, deep-brown: these are the nine colours.%
\ The ninefold guṇas of the nine colours make up 81.%


\textbf{20.37}%
\ I am telling you the ten Fire dhātus, listen and be attentive.%
\ Desire, heat, sight, anger, the digestive fire as the fifth,%


\textbf{20.38}%
\ knowledge, yoga, penance, meditation, the fire of the universe[?] as the tenth.%
\ I shall teach you the other ten \skt{guṇa}s of Fire, O excellent Brahmin.%


\textbf{20.39}%
\ Fire has [the following qualities:] unconquerable, ..., splendour, heat, cooking,%
\                         illuminating, purity, passion, lightness, sharpness and the tenth, tending upwards.%


...


\textbf{20.54}%
\ O Brahmin, I have taught you in a complete form the various \skt{guṇa}s and \skt{dhātu}s%
\                         and [their] origin as I heard it before.%


...


\textbf{21.1}%
\ Vigatarāga spoke:%
\ Bravo, O best of the wise, bravo, O best of the ones who maintain Dharma!%
\ Bravo self-control, bravo tranquillity! Bravo sacrifice, bravo penance!%


\textbf{21.2}%
\ By this nectar-like speech [of yours], my amazement has risen considerably.%
\ And I am pleased with the extraordinary flavour of knowledge based on penance.%


\textbf{21.3}%
\ What kind of boon  shall I give you? Tell me. I'll give you anything you desire.%
\ Having heard this, [Anarthayajña] then replied with appropriate words.%


\textbf{21.4}%
\ [Anarthayajña spoke:]%
\ Who are you, O best of benefactors? Are you a god, a Dānava-demon or a Rākṣasa?%
\ Or rather [you must be] Lord Viṣṇu, who has come to test me.%


\textbf{21.5}%
\ I recognize you clearly, O best of men, O highest person!%
\ Display your [true] Form, O Govinda, if penance can yield fruit.%


\textbf{21.6}%
\ Then lotus-eyed Hari displayed his own [true] body,%
\ holding in his hands a conch-shell, a discus and a mace, wearing yellow garments.%


\textbf{21.7}%
\ Seeing him, Anarthayajña was truly amazed.%
\ Thrilled by unequalled delight, his eyes filled with tears,%


\textbf{21.8}%
\ his voice trembling, he began speaking to Janārdana [i.e. Viṣṇu].%
\ My birth and my austerities have now borne their fruits.%


\textbf{21.9}%
\ Obeisance to you who are the origin of man and other [living beings]! [?]%
\ Obeisance to you who are the universe!%
\ Obeisance to you who [transforming into a person? DG]%
\ Obeisance to you from whom Brahmā was born!%


\textbf{21.10}%
\ Obeisance to you who have a thousand heads!%
\ Obeisance to you who have a thousand eyes!%
\ Obeisance to you who have a thousand liṅgas!%
\ Obeisance to you who have a thousand chests!%


\textbf{21.11}%
\ Obeisance to you who have a thousand embodiments!%
\ Obeisance to you who have a thousand arms!%
\ Obeisance to you who have a thousand faces!%
\ Obeisance to you who have a thousand supernatural powers!%


\textbf{21.12}%
\ Obeisance to you who assumed the form of a boar!%
\ Obeisance to you who [in that form] dug out and saved the Earth!%
\ Obeisance to you who create all living beings!%
\ Obeisance to you on whom the four life-stages depend [seat of...]!%


\textbf{21.13}%
\ Obeisance to you who assumed the form of the Man-lion!%
\ Obeisance to you who [in that form] tore asunder the chest of Diti's son [Hiraṇyakaśipu]!%
\ Obeisance to you who destroyed the armies [conj.] of the Asuras!%
\ Obeisance to you who destroyed the Asuras' haughtiness!%


\textbf{21.14}%
\ Obeisance to you who tamed Diti's son [Bali?]!%
\ Obeisance to you who destroyed Bali's sacrifice!%
\ Obeisance to you of the three steps/Trivikrama!%
\ Obeisance to you who drove away the pain of the thirty gods!%


\textbf{21.15}%
\ Obeisance to you who are imperishable, O endless one!%
\ Obeisance to you who drive away the pain of the world!%
\ Obeisance to you who killed [the Asuras] Madhu and Kaiṭa[bha]!%
\ Obeisance to you who are the friend of the three worlds!%


\textbf{21.16}%
\ Obeisance to you who are the delight of the thirty gods!%
\ Obeisance to you who possess divine vision!%
\ Obeisance to you who have gone beyond the limits of existence!%
\ Obeisance to you who are worshipped by the three worlds!%


\textbf{21.17}%
\ Obeisance to you who hold a mace in [one of] your right[?] hand[s]!%
\ Obeisance to you who hold an excellent discus in your hand!%
\ Obeisance to you who hold a conch-shell in your hand!%
\ Obeisance to you who hold a conch-shell[? rather: lotus] in your hand!%


\textbf{21.18}%
\ Obeisance to you who recline on the ocean!%
\ Obeisance to you who have the form that crushed Hara [the Dānava?]!%
\ Obeisance to you whose banner has the King of Birds [Garuḍa] [on it]!%
\ Obeisance to you whose eyes are the Sun and the Moon!%


\textbf{21.19}%
\ Obeisance to you whose vehicle is the Enemy of Serpents [i.e. Garuḍa]!%
\ Obeisance to you who display your extraordinary form!%
\ Obeisance to you whose splendour is that of a hundred thousand suns!%
\ Obeisance to you who was, [in your Kūrma-avatāra] the firm support at the churning out of the divine nectar!%


\textbf{21.20}%
\ Obeisance to you who are praised in the world of immortals!%
\ Obeisance to you who are the seat of the temple of the world!%
\ Obeisance to you, the only one affectionate towards the world!%
\ Obeisance to you who bestow happiness on everyone, obeisance!%


\textbf{21.21}%
\ O Govinda, forgive my sin.%
\ As you were asking me very actively, I, being a wicked person,%
\ told you all this out of arrogance.%
\ Have pity on me, Lord of the thirty gods [instr.?].%


\textbf{21.22}%
\ Vaiśampāyana spoke:%
\ Keśava, the destroyer of the heroes of the enemy, was satisfied by this hymn of praise.%
\ He, the great general, replied in a ... [nirupaspṛhā/spṛhayā] voice.%


\textbf{21.23}%
\ I am satisfied by this hymn of praise of me, dear Sir. I am vehemently trembling [with joy].%
\ I'll grant you any boon you desire even if it is something difficult to obtain in the three worlds.%


\textbf{21.24}%
\ [He who] praises me with this ....?  [hymn]%
\ that you recited and which is fascinating because it contains the meaning of the Vedas,%
\ will dwell in heaven for as many aeons%
\ as the number of syllables in it.%


\textbf{21.25}%
\ And you should choose a boon at your pleasure,%
\ fearlessly, beginning from sovereignty over the three worlds.%
\ Shall I grant you sovereignty over the seven-fold[?] world?%
\ Or a heap of gold? Or many girls?%


\textbf{21.26}%
\ Hearing the divine boons [offered] [em. to vacam?] by the imperishable one,%
\ he bowed down to his lotus-feet.%
\ Having recognized that Viṣṇu was being most generous,%
\ with a delighted heart....[to be reconstructed]%


\textbf{21.27}%
\ Anarthayajña spoke:%
\ I do not desire anything else as a gift, O God.%
\ Only (\skt{eka}) the essence of bondage is certain [??].%
\ I have been freed from this bondage by your Lordship's grace,%
\ and, O Govinda, I am delighting in Dharma.%


\textbf{21.28}%
\ The Lord spoke:%
\ The extent to which your mind has been enlightened%
\ is something even the great sages and the gods have never seen,%
\ [this] spotless freedom from suffering.%
\ The ocean of existence has certaily been crossed.%


\textbf{21.29}%
\ Well, let's go now to the White Island,%
\ which is hidden and is inaccessible even for the gods.%
\ He who dies after his mind has been purified by devotion towards me,%
\ will never again enter the dreadful ocean [of existence].%


\textbf{21.30}%
\ Vaiśampāyana spoke:%
\ Having spoken thus, then Hari took the great ascetic by the hand,%
\ who disappeared in that moment, and with him Keśava, too.%


\textbf{21.31}%
\ Thus, as a consequence of the abundance of Dharma[?? in him?],%
\ he [Anarthayajña] reached world of the Highest Person,%
\ of the one who is the origin of all living beings, and who is imperishable,%
\ the eternal and never-ending [world] of the never-decaying.%


\textbf{21.32}%
\ You yourself should be loyal to Keśava,%
\ to Janārdana of unmeasurable heroism,%
\ so that you can tread the path of that excellent Brahmin,%
\ that excellent person.%


\textbf{21.33}%
\ What else should I teach you further, O king?%
\ If you have any curiosity remaining,%
\ ask me,  Sir, whatever you want%
\ regarding the future or the past, anything you wish, Sir.%


\textbf{21.34}%
\ Janamejaya spoke:%
\ How many kalpas have passed until now?%
\ How many are the future kalpas?%
\ How many Indras are taught to exist with regard to each aeon?%
\ Tell me one by one[???].%


\textbf{21.35}%
\ Vaiśampāyana spoke:%
\ 100,000 billions of Kalpas have passed so far [rājyam? / rājan?].%
\ There are fourteen Indras in one Kalpa, O king.%
\ The same [number applies to] Manvantaras per Kalpa.%
\ The future Kalpas are again 100,000 billion.%


\textbf{21.36}%
\ The first Kalpa was the Varāhakalpa.%
\ Six Manvantaras have passed, O King.%
\ Seventy-one four-fold [Mahā]yugas%
\ is the number that applies to a Manvantara.%


\textbf{21.37}%
\ Fourteen Manvantara-periods%
\ is one Kalpa, according to the sages.%
\ Ten thousand Kalpas is Brahmā's day.%
\ His night is [of] the same [length] according to the experts.%


\textbf{21.38}%
\ Six hundred-thousand Kalpas is called a [cosmic] month.%
\ Twelve of them is called a year.%


\textbf{21.39}%
\ Brahmā's life is said to be that year multiplied by 100,000 billions of Kalpas[?].%
\ But even Brahmā, the Lord of the three worlds, the supreme person, is taught to be transient.%
\ Why should we grieve over the rest of the four kinds of living beings and the fate[?] of the soul?%
\ Therefore there is nothing that is untouched by the fine[?] essence of the world except for eternal Śiva.%
\vfill\pagebreak\begin{center}{\large\textbf{ Chapter Twenty-two 
}}\end{center}


\textbf{22.1}%
\ Janamejaya spoke:%
\ I have heard from [your] lotus-mouth the ultimate compendium on the essence of Dharma,%
\footnote{Gender problem or śruto = I heard. }%
\ in the proper way, together with the meaning of the Vedas, conveyed by sweet and polished speech.%


\textbf{22.2}%
\ This great essence is systematic (nyāyayukta), and it is the supreme secret knowledge.%
\ I am satisfied now having drunk the nectar of immortality that removes birth, death and disease.%


\textbf{22.3}%
\ I want to ask you another question, O great ascetic, which concerns the name [Anarthayajña].%
\ I would like to hear about your [rather: his] Varṇa, Gotra and Āśrama.%


\textbf{22.4}%
\ Vaiśampāyana spoke:%
\ Listen, O King, attentively.%
\ I shall tell you about the Āśrama, the Varṇa and the Jāti of the great and noble yogin, O king.%


\textbf{22.5}%
\ In the southern region of the Himālaya, on the Mṛgendra peak, O king,%
\ on the banks of the river Mahendrapathagā[?], O King,%


\textbf{22.6}%
\ there was his hermitage.%
\ The illustrous one lived on the beautiful banks [of the river],%
\              having reached the other shore of Truth, free from desire,%


\textbf{22.7}%
\ leading a moral and pure life, with all opposites [such as happiness and pain]%
\                                 and weakness conquered,%
\ with all arrogance, fear, anger and greediness conquered.%


\textbf{22.8}%
\ The Kṣatriyas born in the Soma clan became Brāhmaṇas.%
\ Because of [the] penance [they performed],%
\              [and because of their] discipline and good conduct, Viṣṇu made them Brāhmaṇas,%


\textbf{22.9}%
\ [them who] before that were called Ajitas,%
\                      the one who has conquered lust and anger.%
\ O great King, I shall tell you about his vow, listen.%


\textbf{22.10}%
\ He flourished[?] in the town of the Spirit,%
\              which is[?] populated by Matter[?],%
\ in the vicinity of the divine realm[?],%
\              the ten abodes and the five [phps Sāṃkhyatattvas??].%


\textbf{22.11}%
\ The vow of ten sacrifices was observed and he conquered the ten desires.%
\ He followed the ten Niyama-rules. The ten winds were his priest.%


\textbf{22.12}%
\ With the ten-syllable mantra, he was at the level of the ten Dharmic rituals[???].%
\ His ten sense faculties had the energy of the flames%
\                         in the [sacrificial] fire lit by the ten saṃyamas.%


\textbf{22.13}%
\ He practised the ten yogic sitting positions and focused on the ten ways of meditation.%
\ His intellect was his altar, his mind the sacrificial post,%
\                         and Soma consumption was the immortal syllable.%


\textbf{22.14}%
\ The priestly fee was fearlessness offered to living beings,%
\                      the tying of the sacrificial animal was performed on[?] himself.%
\ He spent his time performing immaterial sacrifice [thus].%
\ The sages, who know the truth, call him Anarthayajña.%


\textbf{22.15}%
\ Janamejaya spoke:%
\ Please let me hear about the ten sacrifices, O best of Brahmins,%
\ and about the ten desires and the ten kinds of meditation,%
\              the ten yogas and the ten-syllable [mantra].%


\textbf{22.16}%
\ Vaiśampāyana spoke:%
\ Sacrifice to/with the Brahman [?; = Vedic offering at <i>saṃdhyā</i>], to the Devas,%
\                          the Ancestors, the Ghosts, the Guests,%
\ recitation, yoga, penance, meditation and [Vedic?] study: these are the ten [sacrifices].%
\footnote{The missing bit is broken off in \msNa. \msL\ seems to copy \msNa\ here directly. }%


\textbf{22.17}%
\ Wife, son, cattle, servant, wealth, grain, fame, beauty,%
\ respect, and enjoyment as the tenth, O king: the ten desires have been taught.%


\textbf{22.18}%
\ Mental, simultaneous and condensed [yoga], O king,%
\ and the yoga named Viśālā, and also the one known as Dvikaraṇa,%


\textbf{22.19}%
\ sun, moon, fire, crystal and sky.%
\ Always sitting in [one of] the ten yoga positions, the great ascetic,%


\textbf{22.20}%
\ when his mind is still not under control, should visualize the subtle one.%
\                                  This is mental yoga.%
\ When he can control his mind with breath-control, that is called simultaneous [yoga].%


\textbf{22.21}%
\ He should visualize the universe with all its moving and motionless [animate and inanimate]%
\                         parts, from Brahmā to a tuft of grass,%
\ as gradually dissolving, and should reflect upon the subtle one:%


\textbf{22.22}%
\ this is called condensed [yoga]. Now listen to the Viśālā.%
\ The wise one should call to mind [everything] from Brahmā to the subtle.%


\textbf{22.23}%
\ He should visualize both the condensed and the Viśālā mutually [one after the other? DG]:%
\ this is the yoga method called Dvikaraṇī.%


\textbf{22.24}%
\ He should imagine his heart in the center of his body,%
\                         and that there is a lotus in his heart.%
\footnote{hṛdi as nominative... }%
\ In the center of the lotus, know that there is a pericarp, O king.%
\footnote{gopate is slightly odd for `king'. }%


\textbf{22.25}%
\ The wise ones know that there are five dots in the center of the pericarp:%
\ the sun, the moon, the flame, the crystal and the sky.%


\textbf{22.26}%
\ He should visualize the disk of the moon in the centre of the sun.%
\ In the centre of that [i.e.\ the moon], he should visualize fire that blazes without smoke.%


\textbf{22.27}%
\ In the centre of the fire, he should visualize a gem which has the splendour of a jet of clear water.%
\ In its center, he should visualize the sky, subtle and imperishable Śiva.%


\textbf{22.28}%
\ This is how I taught you the ten yogas, O king.%
\ The ten ways of meditation are taught in short as here follows, listen.%


\textbf{22.29}%
\ Sound, yellow, lightning, Candramālinī,%
\ moon, pleasing, well-done,%


\textbf{22.30}%
\ Saumyā, spotless and supportless.%
\ [1] Putting[?] two fingers in his ears, one can hear[!] sounds.%


\textbf{22.31}%
\ Having heard this and that syllable, he is fit for immortality.%
\ [2] He should continuously visualize yellow, smokeless[?] flames tirelessly.%
\footnote{Stem forms? śikhām adhūmāṃ ? }%


\textbf{22.32}%
\ He will be freed of all his sins and will reach the level without opposites.%
\ [3] The lightning in the middle of the night marks the unborn of no diseases.%
\footnote{OR: lakṣyateja a°: the visible energy?? }%


\textbf{22.33}%
\ After five months of continuous practice, men will develop divine sight.%
\ [4] Then he should visualize the Bindumālā [Candramālā??] which rests in the shadow of a tree.%


\textbf{22.34}%
\ [5] Seeing it as genuine crystal, he is liberated from the fetters.%
\ [6] He should visualize the Pleasing one ... ? pressing it in the eye.%


\textbf{22.35}%
\ When he sees the white and yellow and red drop, he will not be born again.%
\ These are the six ways of meditation, the Pleasing one and the others, as I taught them to you.%


\textbf{22.36}%
\ Now I shall teach you another thing: the fourfold supreme atom.%
\                  O supreme sovereign, listen to the characteristics of that[?]%
\                  by which the whole world, made up of the four elements (\skt{bhūta}), Earth etc., is pervaded.%
\                  I shall tell you [about them] now.%


\textbf{22.37}%
\ The subtle atom of Earth tends upwards, O king.%
\                 The pure one should observe the meditation that is direct perception firmly.%


\textbf{22.38}%
\ He will be freed from all his sins, as the Moon is from Rāhu.%
\                  The yogin who constantly practises by this [method] is the lord of the world.%


\textbf{22.39}%
\ The atom of Water tends downwards, O great king.%
\                  If one practises this, O king, there will be a destruction of all his sins.%


\textbf{22.40}%
\ The atoms? of Fire tend upwards and horizontally.%
\                 He who constantly meditates upon this will reach the supreme path.%
\footnote{Note how a neuter ending in \skt{pāda} a governs a seemingly feminine ending in \skt{pāda} b REVISE. See the                 same in 22.41ab. CHECK phenomenon and note gatiḥ. }%


\textbf{22.41}%
\ The atoms of Air tend downwards and horizontally.%
\                [If] he is not perplexed when seeing this, he is Hanumān, O king.%
\footnote{Note \skt{°sambhava} instead of the more correct \skt{°sambhavo} metri causa. }%


\textbf{22.42}%
\ [If] he perceives these four [types of] atoms, O king,%
\          by this he has sacrificed with all sacrifices, by this penance is completed.%


\textbf{22.43}%
\ By this the whole Earth with its surrounding seven seas is given [as a sacrificial fee],%
\                 and he will have received all consecrations at the sacred places and all the religious vows and rituals%
\                                         will have been completed.%


\textbf{22.44}%
\ If one practises the ten meditations by this method, O king,%
\                  uninterruptedly, it will yield all the desired fruits.%


...


\textbf{24.1}%
\ I have heard about the conflicts between%
\               the gods and the demons,%
\ and learnt about the miracle that is produced by sleep%
\                         by your kindness.%


\textbf{24.2}%
\ Now, I would like to hear about the breadth and length%
\              of the three worlds, O Brahmin.%
\ Where are the hells and the Pātāla located, O excellent Brahmin.%
\footnote{In the light of the following verses, kasmiṃścid seems to carry         the function of an interrogative (kasmin) here, and the form narake         in the MSS might be taken as an Aiśa neuter plural (for narakāṇi),         but I have decided to except Naraharinātha's narakaṃ. }%


\textbf{24.3}%
\ And I want [to learn about] the seven islands and the seven oceans.%
\footnote{CHECK samicchāmi }%
\ And teach me about the peak of Mount Meru, O best of Brahmins,%
\              the abode of the gods.%
\footnote{CHECK mūrdhaṃ as an acc. Or ūrdhvaṃ is meant? }%


\textbf{24.4}%
\ Hear about, O king, the breadth and length%
\              of the three worlds.%
\ The first [level of the universe], beneath everything, is to be known as the fire of%
\              [the end of] time (kālāgni), O king of the people.%


\textbf{24.5}%
\ Above that, O best of kings, are the divisions of hell to be found.%
\footnote{On \skt{koṭi}s as divisions of hell, see e.g.\ ŚDhU 7, and also Bhṛgusaṃhitā 36.40 ff. }%
\ They start with Raurava and end with Avīcī, and%
\              they are called the places of torment.%
\footnote{See Mitākṣara: \skt{evaṃ rauravādinarakeṣu}... }%


\textbf{24.6}%
\ Above them are the Pātālas, which are only seven.%
\ The first is Ābhāsatāla, the next one is Svatāla,%
\footnote{CHECK Niśv book p. 209 and various lists in Goodall 2004:289-291, fn. 522 (Prākhya). }%


\textbf{24.7}%
\ [then] Śītala, Gabhasti, Śarkara and Śilātala.%
\ The seventh is Mahātāla, the abode of the serpent Śeṣa,%
\footnote{Monier-Williams: mahātala; emend? }%


\textbf{24.8}%
\ [and also of] Bali the Daitya prince and Viśaṃkhaṇa[?] the Rākṣasa.%
\ These and all the other Nāgas, Dānavas and Rākṣasas%
\              [live in the seven Pātālas].%


\textbf{24.9}%
\ Then one should learn about the seven islands, which%
\              are surrounded by seven oceans.%
\footnote{See VSS 4.12 for a reference to the myth of Priyavrata dividing the earth into seven parts,                thus producing the seven seas and the seven islands. }%
\ Ten sons of kingly heroism were born to Priyavrata [Manu's son]:%
\footnote{Note putro for plural. Perhaps the original read putrābhūd (with double sandhi)? }%


\textbf{24.10}%
\ Agnīdhra, Agnibāhu, Medhas, Medhātithi, Vasu,%
\footnote{Agnīdhra is a variant of the form given in Monier-Williams as Āgnīndhra. }%
\ Jyotiṣmat, Dyutimat, Havya, Savana, and Patra.%


\textbf{24.11}%
\ The three men Agnibāhu, Medhas and Patra%
\ resorted to the path of liberation out of their%
\              fear of transmigration (saṃsāra).%


\textbf{24.12}%
\ Priyavrata consecrated Agnidhra [as king of] the first island%
\              [Jambudvīpa],%
\ and named Medhātithi to be `King of Plakṣadvīpa'.%


\textbf{24.13}%
\ Vasu was consecrated as king in Śālmalīdvīpa.%
\footnote{Note that mahīpatiḥ was probably meant to be the agent of the action, i.e. Priyavrata. }%
\ He [Priyavrata] consecrated Jyotiṣmat as king in Kuśadvīpa,%


\textbf{24.14}%
\ and made Dyutimat the king of Krauñcadvīpa, O king,%
\footnote{Note that nareśvara[ḥ] might be the agent of the action, i.e. Priyavrata. }%
\ Havya the king of Śākadvīpa and Savana is said to have been%
\                                 [the king] in Puṣkara[dvīpa].%


\textbf{24.15}%
\ On the island of Puṣkara, there is a mountain called Mānasottara.%
\ There are ... Lokapālas there ...%


\textbf{24.16}%
\ There is Mahāvīta country there, and Dhātaki[n?], O king!%
\footnote{dhātaki: N. of one of the 2 sons of Vītihotra Praiyavrata (king of a Varṣa of Puṣkara-dvīpa), Pur. }%
\ Outside of [Puṣkaradvīpa], an ocean called Sweet-water (svādudaka) emerged.%


\textbf{24.17}%
\ [The extension of this ocean is] 64 lakh yojanas, O king!%
\ Within the Puṣkara island, there is an ocean called the Ocean of milk (kṣīroda).%


\textbf{24.18}%
\ [Its extension is] 32 [yojanas] and it is located around[?] the Śāka island.%
\footnote{bahirvahaḥ??  }%
\ Jalada, Kumāra, Sukumāra, Maṇīcaka,%


\textbf{24.19}%
\ Kusuma, Uttaramoda, and Mahādruma are%
\ the seven sons of Havya, and the country names [in Śākadvīpa] are the same.%


\textbf{24.20}%
\ At the [inner] shores of the island, one should%
\                         point out a half-whey, half-milk ocean.%
\ On the seashore of Krauñcadvīpa, these are the seven countries:%


\textbf{24.21}%
\ Kuśala, Manonuga, Uṣṇa, Yāvana, Andhakāraka,%
\footnote{Note that pāda a is hypermetrical. }%
\ Muni, and Dundubhi, and [these are also the names of] Dyutimat's sons.%


\textbf{24.22}%
\ An ocean of half curd, half-scum-of-melted-butter is around%
\                 the Kuśa island.%
\ Hear also the seven counties that are located there%
\              by name, O Bhārata!%
\footnote{Note °varṣe as neuter plural nominative/accusative. }%


\textbf{24.23}%
\ Udbhimat?, Dhenumat, Svairanna (/Svairatha), Ālambana, Dhṛti,%
\ the sixth is Prabhākara, and the seventh Kapila.%


\textbf{24.24}%
\footnote{The term madirodadhi for this ocean seems unique in the VSS. }%
\ ... the Ocean of alcohol (madirodadhi).%
\ around[!] Śālmalīdvīpa[, where] there are said to be seven countries:%


\textbf{24.25}%
\ Śveta, Harita, Jīmūta, Rohita,%
\ Vaidyuta, Mānasa, and the seventh, Suprabhaḥ.%


\textbf{24.26}%
\ ... the Ocean of sugar-cane.%
\ Plakṣadvīpa with its seven countries is surrounded by it.%


\textbf{24.27}%
\ Śānta, Śiśira, Sukhada, Ānanda,%
\ Śiva, Kṣema and Dhruva: these are Medhātithi's seven sons%
\              [and the names of their countries].%
\footnote{Purāṇic Encyclopedia p. 499        ( https://www.sanskrit-lexicon.uni-koeln.de/scans/PEScan/2020/web/webtc/servepdf.php?page=499-b ):                                ``MEDHĀTITHI I . Grandson of Svāyambhuva Manu.                                Svāyambhuva Manu had two sons named Priyavrata and                                Uttānapāda. Of these Priyavrata married Sarūpā and                                Barhiṣmatī, daughters of Viśvakarmaprajāpati. Medhā-                                tithi was the son born to Priyavrata of Sarūpā. Agnī-                                dhra, and others were the brothers of Medhātithi.                                Medhātithi became the King of Plakṣadvīpa after the                                death of Priyavrata. (8th Skandha, Devī Bhāgavata).                                Medhātithi got seven sons named Śāntahaya, Śiśira,                                Sukhodaya, Ānanda, Śiva. Kṣemaka and Dhruva. They                                all became Kings of Plakṣadvīpa. The countries they                                ruled were named after them as Śāntahayavarṣa, Śiśira-                                varṣa, Sukhodayavarṣa, Ānandavarṣa, Śivavarṣa, Kṣema-                                kavarṣa and Dhruvavarṣa. There are seven mountains                                showing the boundaries of these states and they are                                called Gomeda, Cāndra, Nārada, Dundubhi, Somaka,                                Sumana and Vaibhrāja. In these beautiful countries                                and grand mountains live a great many Devas,                                Gandharvas and virtuous men. (Chapter 4, Aṃśa 2,                                Viṣṇu Purāṇa).'' }%


\textbf{24.28}%
\ At its shores, there is the Salty ocean (lavaṇoda), which%
\              surrounds[!] Jambudvīpa.%
\ Its territory is one lakh yojanas and its contains the following%
\              minor islands:%


\textbf{24.29}%
\ Aṅgadvīpa, Yavadvīpa, Malayadvīpa,%
\ Śaṅkhadvīpa, Kamudvīpa[?] and Varāhadvīpa,%


\textbf{24.30}%
\ Siṃha, Barhiṇadvīpa, Padma, Cakra,%
\ Vajraratnākaradvīpa, Haṃsaka, Kumuda,%


\textbf{24.31}%
\ Lāṅgala, Vṛṣadvīpa, Bhadrākāra,%
\ Candradvīpa, Sindhu, Candanadvīpa,%
\ and so on so forth. There are said to be thousands of minor islands.%
\footnote{Note the discrepancy in the numbers: °sahasrāṇi... °ādīni kīrtitam  }%


\textbf{24.32}%
\ Agnīdhra consecrated [his] nine sons in nine countries.%
\ [The names of the countries/sons are:]%
\              Nābhi, Kiṃpuruṣa, Hari, Ilāvṛta,%


\textbf{24.33}%
\ the fifth, Ramyaka country and the sixth, Hiraṇmaya,%


\textbf{24.34}%
\ the seventh, Kurava, the eighth Bhadrāśva, and%
\                 the ninth was Ketumāla. The nine countries have been taught.%


\textbf{24.35}%
\ South of the Himālaya, there is the country called Bhārata.%
\                Again, there emerged a ninefold division there due to%
\                Bhārata's sons:%


\textbf{24.36}%
\ Indradvīpa, Kaśeru, Tāmravarṇa, Gabhastimat,%
\              Nāgadvīpa, Saumya, Gāndharva, Vāruṇa,%
\ and the ninth island, called Kumāradvīpa.%
\             South of Hemakūṭa[?] there is the country called Kiṃpuruṣa.%


\textbf{24.37}%


\textbf{24.38}%


\textbf{24.39}%


\textbf{24.40}%


\textbf{24.41}%


\textbf{24.42}%


\textbf{24.43}%


\textbf{24.44}%


\textbf{24.46}%


\textbf{24.47}%


\textbf{24.48}%


\textbf{24.49}%


\textbf{24.50}%


\textbf{24.51}%


\textbf{24.52}%


\textbf{24.53}%


\textbf{24.54}%


\textbf{24.55}%


\textbf{24.56}%


\textbf{24.57}%


\textbf{24.58}%


\textbf{24.59}%


\textbf{24.60}%


\textbf{24.61}%
\ This is the extension of Bhūrloka. Above it, there is Bhuvaḥ,%
\ Svarloka, ... and above it, Maharloka,%


\textbf{24.62}%
\ Janaloka, Tapoloka, Satyaloka, in due order.%
\ Satya[loka] is said to be Brahmaloka and above it is located%
\                         Viṣṇuloka.%


\textbf{24.63}%
\ Beyond that, the great city of divine visions is to be recognized as%
\ a thousand-story palace with gates [decorated] with cat's-eye gems and%


\textbf{24.64}%
\ coloured with different kinds of precious stones, inhabited by%
\                         different troops of beings.%
\ That place is full of charming riches of all desires.%
\footnote{Note samṛddhāni as a plural instrumental. }%


\textbf{24.65}%
\ There, on a divine throne which is ornamented with all%
\                 kinds of precious stones,%
\ the Lord Rudra is sitting, the one who wears his matted hair%
\              marked with the Moon,%


\textbf{24.66}%
\ the one with three eyes, the best of the three worlds,%
\                 holding a trident, the ruler of the thirty [gods],%
\ together with Devī, he the illustrious one, surrounded by%
\                                 the Gaṇas,%


\textbf{24.67}%
\ with Skanda and Nandi standing in front, in the crowd of%
\                       a hundred lakhs of Gaṇas,%
\ embellished CHECK with many beautiful Rudra girls.%


\ enjoys the food all the time, O king,%
\ a ruler will fight for[?] the possession (bhoga) of [the elephant's]%
\                 outer teeth [i.e. the tusk],%
\ The difference between the two, see, of a similar kind[???].%


\textbf{24.78}%
\ There is nothing like donations for somebody who%
\                 offers freedom from danger.%
\ There is nothing like sacrifices for him who%
\                         has conquered his senses.%
\ There is nothing like wealth for him who%
\                         has conquered his senses.%
\ There is nothing like Dharma[?] for him whose%
\                         desires are tamed[?].%


\textbf{24.79}%
\ For there are no big differences between Dharma and a-Dharma.%


\textbf{24.80}%
\ As this sacred and superior true [sat-?] Dharma%
\                 was in the past taught by Parameśvara,%
\ I too have taught it in the same form, as it is,%
\                 the very essence of the Purāṇas, the Vedas and the Upaniṣads.%


...


\textbf{24.86}%
\end{document}
