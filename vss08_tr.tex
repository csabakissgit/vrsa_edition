\documentclass{article}
\usepackage[utf8x]{inputenx}
\newcommand{\vsnum}[1]{\textbf{#1}}
\newcommand{\skt}[1]{\textit{#1}}
\begin{document}


%%%%%%%%%
\vsnum{8.1:} Five kinds of study are to be pursued by those who wish to be happy in this life and in the other: [one has to study the] Śaiva [teachings], Sāṃkhya [philosophy], the Purāṇa[s], the Smārta [tradition] and the \skt{Bhāratasaṃhitā} [i.e. the \skt{Mahābhārata}].

\vsnum{8.2: }He should reflect on the Śaiva truth in both Śaiva and Pāśupata [teachings]. In those teachings the whole essence of truth is taught extensively.

\vsnum{8.3: }Those who reflect on the truth (\skt{tattva}) can grasp the truth (\skt{tattva}) of enumeration (\skt{saṃkhyā}) [of ontological principles/reality levels] from Sāṃkhya [texts]. The great sages taught [those twenty-five] \skt{tattva}s [of Sāṃkhya] as being in groups of five.

\vsnum{8.4: }In the Purāṇas it is the sheaths of the world that are described extensively. One can definitely enter [the realm] of the lower [world, i.e. hell], the upper [world, i.e. heaven], and middle [world, i.e. the human world], and the horizontal [world, i.e. of animals by studying the Purāṇas].

\vsnum{8.5: }The Smārta [tradition] deals with the conduct of the classes (\skt{varṇa}) and the conduct in the life-stages (\skt{āśrama}), and with the activities of Dharma and legal proceedings. Good conduct is to be gathered from that [source] without hesitation, with trust.

\vsnum{8.6: }A man who studies the epics (\skt{itihāsa}) will become omniscient. [All his] doubts about Dharma, Artha, Kāma and Mokṣa will be eliminated.

\vsnum{8.7: }Listen with great attention, O Brahmin, to the five types of sexual restraint [concerning the following:] women, forbidden ejaculation, and masturbation are mentioned [in this context, as well as] offence while sleeping, O Brahmin, and daydreaming as the fifth.

\vsnum{8.8: }A woman is not to be approached sexually in daytime and on the four days of the changes of the Moon (\skt{parvan}), even if she is one's lawful wife. One should not have sex with a woman who is taboo or with one of those who have lost their class (\skt{varṇa}) or are [of a] superior [\skt{jāti} than oneself].

\vsnum{8.9: }Intercourse with goats, sheep, cows, mares, buffaloes is called forbidden ejaculation, which is to be avoided at all cost.

\vsnum{8.10: }Rubbing himself against something else than a female sexual organ or rubbing his anus, are called masturbation, therefore these are to be avoided.

\vsnum{8.11: }Offence while sleeping, O best of Brahmins, has always been [considered] undesirable by the learned. [If] one enjoys women while sleeping, his semen gets spilt.

\vsnum{8.12: }Daydreaming [about women] should always be avoided by those who are intent on Dharma. Women are called `the bolts [that block the gate to] the path to heaven'.

\vsnum{8.13: }[Hear about] the five religious observances [called] the cat, the crane, the dog, the cow, and the earth. <sep/>He buries his own urine and faeces in the ground, O truest Brahmin. He rejoices [seeing] the sun and the moon when performing the cat observance.

\vsnum{8.14: }O great ascetic, one should suppress all of his senses like a crane, and should cultivate the peace of the mind, focusing on achieving liberation.

\vsnum{8.15: }He does not bury his urine and faeces in the ground, and he barks constantly. Lord Śarva [i.e. Śiva] is satisfied when one practises the dog observance.

\vsnum{8.16: }A person practising the Cow Vow should never hold back his urine and faeces. He is terrifying and he gives satisfaction, [as] stated in the Purāṇas.

\vsnum{8.17: }CHECK Digging [the earth] with spades and collecting [? the soil] with wedges: Goddess Earth bears [this] patiently. This is exactly how one can practise the earth vow.

\vsnum{8.18: }He who practises these five religious observances with his senses subdued will, without doubt, reach this superior world (i.e. Śiva's heaven).

\vsnum{8.19: }Eating leftovers, [not] eating in-between [breakfast and dinner], eating [only] at night, eating food obtained without solicitation, and fasting: listen, I shall teach you these five.

\vsnum{8.20: }[He who eats] the leftovers belonging to all the gods, to guests, and to the ancestors, he who eats the leftovers (śeṣāśin) of servants, sons and wives is the one who consumes the remains of food (\skt{vighasāśana}).

\vsnum{8.21: }He will be regarded as one that is always fasting if he never eats between breakfast and dinner.

\vsnum{8.22: }One should not eat in the daytime or in the evening, and should eat [only] at midnight if he wishes to follow the order of [eating only at] night.

\vsnum{8.23: }He should eat only the unsolicited food of someone who has not yet started eating [this food]. He who eats [only] that which has been given by others [without asking them for it] is called [one who eats] unsolicited [food].

\vsnum{8.24: }Chewable and unchewable food (\skt{bhakṣyaṃ bhojyaṃ ca}), food to be sipped or sucked or drunk, as the fifth [category]: if one does not long for and does not consume [any of the above], that is called fasting (\skt{upavāsa}).

\vsnum{8.25: }One should keep these five types of taciturnity, always dwelling in religious observances: [in situations where silence is best instead of] deceitful speech, envious speech, abuse, harsh speech, bragging.

\vsnum{8.26: }Fictitious [speech], [speech on] unknown [things], [speech about things] outside the range of Dharma, meaningless and unfriendly speech: these are called lying.

\vsnum{8.27: }One who does not rejoice in others' fortune or in others' power, one who would like to see something disadvantageous [for others] is called envious [and he should rather remain silent].

\vsnum{8.28: }[May your] mother and father be dead! [This is] how a ruined state will befall [you]! Enjoy the love of unclean [women]! [These are] called abuse.

\vsnum{8.29: }Won't you burst in your heart, stupid? Will your head not split into two? [If one utters] these or similar [curses], he is said to be one of harsh speech.

\vsnum{8.30: }Relating fancy stories about gambling, enjoyments, fights, drinking and women are the five types of bragging, as I teach them, O excellent Brahmin.

\vsnum{8.31: }Taciturnity should always be practised by those who prefer the beauty of speech. One should always speak without abuse and without idle talk.

\vsnum{8.32: }He who does not practise taciturnity is defiled and he is the black sheep of the family. For a number of rebirths, [his mouth] will stink and he will become mute.

\vsnum{8.33: }Therefore the speech of a person who always keeps the observance of taciturnity firmly, with resolution, will be impossible to ignore and he will make the community rejoice. The fragrance of lotuses and [other kinds of] strong fragrances will blow from his mouth. Thousands of faultless \skt{śāstra}s will be declared in the words of this person.

\vsnum{8.34: }I shall teach you the five kinds of bathing as they really are: Fire bath, water bath, Vedic bath, wind bath and divine bath.

\vsnum{8.35: }Fire bath is [performed] with ashes. Its fruits are a hundred times bigger than [those of] a water [bath]. [Things] purified with ashes are holy. Ashes destroy sin.

\vsnum{8.36: }Therefore one should use ashes for it purifies humans of their defilement. Ashes produce peace for everyone. Ashes are the ultimate protectors.

\vsnum{8.37: }Drawing [the sectarian marks on their foreheads while reciting] the Tryāyuṣa [mantra], remaining in chastity, all the Ṛṣis purified themselves with ashes.

\vsnum{8.38: }The gods, afflicted by their fear of Vīrabhadra, were set free with the help of ashes. Seeing the glory of ashes, Brahmā consented [to the use of this otherwise impure substance].

\vsnum{8.39: }[Thus] the Pāśupata observance was created, which is above [the system of] the four \skt{āśrama}s. Therefor the Pāśupata [observance] is the best because it involves carrying ashes [on one's body].

\vsnum{8.40: }A water bath (\skt{vāruṇa}) is to be performed with water by people in various ways in the water of rivers, water tanks, streams and ponds.

\vsnum{8.41: }The wise know the Vedic bath as [the one performed with the Vedic mantra beginning] \skt{āpo hi ṣṭhā} [ṚV 10.9.1--3], O excellent Brahmin. It is to be performed at the three junctures of the day (dawn, noon, evening). It is called the Vedic bath.

\vsnum{8.42: }He should go where, on the paths where cows roam, dust is rising, and he should sit down there. This is called [a kind of] bath, [namely the \skt{vāyavya} or wind-bath].

\vsnum{8.43: }One should immerse his own body in the water-showers of rain water. The one and only great Lord (\skt{maheśvara}) of the universe calls it heavenly bath.

\vsnum{8.44: }Thus have I taught you the section on the Niyama-rules [see Chapters 5--8] in divisions of five [sub-categories] because you asked me to, favouring the whole world. [These Niyama-rules] wipe off all the defilement, these fifty Dharma [teachings, i.e. 10 main topics/rules × 5 subcategories]. There will not be rebirth [for one who keeps these rules], not even in millions of aeons.


\end{document}