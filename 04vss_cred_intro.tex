\section{Preliminary remarks}

%While it is probably unnecessary to argue in favour of
%producing a high-quality edition of any of the texts in the Śivadharma
%corpus---given its importance for our understanding
%of the history of Śaivism---
It is perhaps worth clarifying why the versions of the \VSS\ and other texts of
the Śivadharma corpus as printed in \mycite{NaraharinathSivadharma} are not satisfactory,%
		\footnote{As \citeauthor{WestTextual} (\citeyear{WestTextual}, 61) 
			puts it, following a long tradition of philologists:
			`Is your edition really necessary? That is the first question.'}
and why there is a need to produce high-quality critical editions of them.
One could simply refer the reader to the apparatus in
this new edition: the readings given in \citeauthor{NaraharinathSivadharma}'s 
\emph{editio princeps} rarely prove useful or are 
accepted against the manuscript evidence.
One could also point out further problems in 
\citeauthor{NaraharinathSivadharma}'s edition, such
as countless typos, misreadings, and readings and omissions that 
may come from his law-quality sources,%
		\footnote{Just to quote a few from the first few verses:
								\skt{sahasrādhyāyar uttamam} for \skt{sahasrādhyāyam uttamam} (1.2b),
								\skt{nāradasaṃhitāṃ} for \skt{bhāratasaṃhitām} (1.2d),
								\skt{śaṃkha} for \skt{śaṅkuḥ} (1.34b), omissions in 1.34cd--35, etc.}
and a lack of any critical apparatus or any documentation of the witness(es) used.%
		\footnote{He must have worked from paper manuscripts,
								see p.~\pageref{narahari_paperms}.}
In addition to this, although it does not affect this volume,
a great chunk of the text, \VSS\ 17.38--18.16, is
missing in \citeauthor{NaraharinathSivadharma}.

It would be more difficult than this to vindicate in detail the methology
I have applied. I find \citeauthor{HannederIntro}'s 
words on textual criticism comforting:

\begin{quote}
[T]extual criticism is often viewed as something to be learned by practice rather from reading about it.
\dots\ In fact, both translating and editing are something most Indologists have learned in a pragmatic
way through examples from within the field, and some have managed to become quite good at it.
\dots\ [I]n most cases this approach is sufficient \dots%
		\footnote{\mycitep{HannederIntro}{5}.}
\end{quote}

\noindent
My experience is that when preparing critical editions, each text, 
and sometimes each manuscript or each chapter, \textit{horribile dictu},
each verse, requires a slightly different approach, and these approaches 
keep changing during the editorial process. For example, the idea that 
there could be a connection between the linguistic oddities of 
the \VSS\ and classical Newar
%		\footnote{See p.~\pageref{newar}.}
arose relatively late, and it did change my views on some textual
problems and some of the solutions thereof, and led me to change some
of my previously proposed emendations.
Thus editing is always subjective in the sense that the method
applied is influenced by the editor's knowledge of the text, the genre,
the milieu, etc., or in the case of this edition, the collective knowledge
of all my colleagues who took part in \VSS\ reading session and brain-storming
meetings throughout the years.

Since it is not unlikely that originally the \VSS\ had multiple authors and redactors,
the text itself is also unlikely to be homogenous: each chapter may
have its own style and its own types of textual problems. In addition 
to this, all MSS we have access to surely trasmit a highly contaminated
version of the text. This makes the construction of a stemma codicum more or less
useless in this case.
