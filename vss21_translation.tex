\documentclass[12pt]{article} % use larger type; default would be 10pt

\usepackage[utf8x]{inputenx} % set input encoding (not needed with XeLaTeX)

\usepackage{geometry} % to change the page dimensions
\geometry{a4paper} % or letterpaper (US) or a5paper or....
% \geometry{margin=2in} % for example, change the margins to 2 inches all round
% \geometry{landscape} % set up the page for landscape
%   read geometry.pdf for detailed page layout information

\usepackage{graphicx} % support the \includegraphics command and options

\usepackage{xcolor}
\newcommand{\epic}[1]{\textsc{#1}}
\newcommand{\saiva}[1]{\textbf{#1}}
\newcommand{\vaisnava}[1]{\textit{#1}}
\newcommand{\brahma}[1]{{\textrm{#1}}}



\begin{document}



\section{Dramatis Person\ae\ in the Vṛṣasārasaṃgraha}  %(\LaTeX\ error: `Too deeply nested'.)
		
		[\epic{epic/dharmaśāstric 4 chs}, \vaisnava{Vaiṣṇava 12}, \saiva{Śaiva 8}, \brahma{Brāhmya} 1?]
		\bigskip

\noindent
$\textcolor{blue}{\triangleright}$ \epic{Ch.\ 1 (1.1--8): Janamejaya asks Vaiśampāyana, Vaiśampāyana relates a dialogue:}
				\begin{itemize}
				\item  \vaisnava{Chs.\ 1--9 (1.9--9.43): 
						Vigatarāga asks Anarthayajña}
					\begin{itemize}
						\item   \vaisnava{Ch.\ 10 (10.1--3):
						Vigatarāga is still asking Anarthayajña; 
						the latter recites a text in which} 
							\saiva{Nandikeśvara relates a dialogue}\dots
							\begin{itemize}
							\item \saiva{Ch.\ 10--11 (10.4--11.60) 
													\dots between Devī/Umā and 
													Mahe\-śva\-ra/Rudra}
							\item  \saiva{Ch.\ 12: 
							Mahe\-ś\-va\-ra tells Devī the story of \brahma{Vipula, who 
							gave his wife to a Brahmin selflessly; story 
							culminates in the praise of Brahmā} and ends
							somewhat abruptly}
							\item  \saiva{Chs.\ 13--18: Back to Maheśvara and Devī}
							\end{itemize}
					\end{itemize}
					\item  \vaisnava{Ch.\ 19--21: 
					Back to Vigatarāga and Anarthayajña;\linebreak
					(ch.\ 21: revelation $\rightarrow$ Vigatarāga = Viṣṇu !)} 
					\epic{Back to\dots}
				\end{itemize}
$\textcolor{blue}{\triangleright}$  \epic{Chs.\ 22--24: \dots Janamejaya and Vaiśampāyana}



\section{Draft raw translation of Vṛṣasārasaṃgraha 21}

\parindent0em
\textbf{21.1} 

Vigatarāga spoke: 
    
    Bravo, O best of the wise, bravo, O best of the ones who maintain Dharma!  

     Bravo self-control, bravo tranquillity!  Bravo sacrifice, bravo penance! 

\textbf{21.2} 

By this nectar-like speech [of yours], my amazement has risen 
considerably. 

     And I am pleased with the extraordinary flavour of knowledge based on penance. 

\textbf{21.3}

 What kind of boon [shd be m.!] shall I give you? Tell me. I'll give you anything you desire. 

     Having heard this, he replied with appropriate words. 

\textbf{21.4}

 [Anarthayajña spoke:] 

     Who are you, O best of benefactors? Are you a god, a Dānava-demon or a Rākṣasa? 

     Or rather [you must be] Lord Viṣṇu, who has come to test me. 

\textbf{21.5}

 I recognize you clearly, O best of men, O highest person! 

     Display your [true] Form, O Govinda, if penance can yield fruit. 

\textbf{21.6}

 Then lotus-eyed Hari displayed his own [true] body, 

     holding in his hands a conch-shell, a discus and a mace, wearing yellow garments. 

\textbf{21.7}

 Seeing him, Anarthayajña was truly amazed. 

     Thrilled by unequalled delight, his eyes filled with tears, 

\textbf{21.8}

 his voice trembling, he began speaking to Janārdana [i.e. Viṣṇu]. 

     My birth and my austerities have now borne their fruits. 

\textbf{21.9}

 Obeisance to you who are the origin of man and other [living beings]! [?] 
    
     Obeisance to you who are the universe! 
    
     Obeisance to you who ???  
    
      Obeisance to you from whom Brahmā was born! 

\textbf{21.10}

 Obeisance to you who have a thousand heads! 

      Obeisance to you who have a thousand eyes! 

      Obeisance to you who have a thousand liṅgas! 

      Obeisance to you who have a thousand chests! 

\textbf{21.11}

 Obeisance to you who have a thousand embodiments! 

      Obeisance to you who have a thousand arms! 

      Obeisance to you who have a thousand faces! 

      Obeisance to you who have a thousand supernatural powers! 

\textbf{21.12}

 Obeisance to you who assumed the form of a boar! 

      Obeisance to you who [in that form] dug out and saved the Earth! 

      Obeisance to you who create all living beings! 

      Obeisance to you on whom the four life-stages depend! 

\textbf{21.13}

 Obeisance to you who assumed the form of the Man-lion! 

      Obeisance to you who [in that form] tore asunder the chest of Diti's son 
      [Hiraṇyakaśipu]! 
      
      Obeisance to you who destroyed the armies [conj.] of the Asuras! 
      
      Obeisance to you who destroyed the Asuras' haughtiness! 

\textbf{21.14}

 Obeisance to you who tamed Diti's son! 

      Obeisance to you who destroyed Bali's sacrifice! 

      Obeisance to you of the three steps/Trivikrama! 

      Obeisance to you who drove away the pain of the thirty gods! 

\textbf{21.15}

 Obeisance to you who are imperishable, O endless one! 

      Obeisance to you who drive away the pain of the world! 

      Obeisance to you who killed [the Asuras] Madhu and Kaiṭa[bha]! 

      Obeisance to you who are the friend of the three worlds! 

\textbf{21.16}

 Obeisance to you who are the delight of the thirty gods! 

      Obeisance to you who possess divine vision! 

      Obeisance to you who have gone beyond the limits of existence! 

      Obeisance to you who are worshipped by the three worlds! 

\textbf{21.17}

 Obeisance to you who hold a mace in [one of] your right[?] hand[s]! 

      Obeisance to you who hold an excellent discus in your hand! 

      Obeisance to you who hold a conch-shell in your hand! 

      Obeisance to you who hold a conch-shell[? rather: lotus] in your hand! 

\textbf{21.18}

 Obeisance to you who recline on the ocean! 

      Obeisance to you who have the form that crushed Hara [the Dānava?]! 

      Obeisance to you whose banner has the King of Birds [Garuḍa] [on it]! 

      Obeisance to you whose eyes are the Sun and the Moon! 

\textbf{21.19}

 Obeisance to you whose vehicle is the Enemy of Serpents [i.e. Garuḍa]! 

      Obeisance to you who display your extraordinary form! 

      Obeisance to you whose splendour is that of a hundred thousand suns! 

      Obeisance to you who was, [in your Kūrma-avatāra] the firm support at the 
      churning out of the divine nectar! 

\textbf{21.20}

 Obeisance to you who are praised in the world of immortals! 

      Obeisance to you who are the seat of the temple of the world! 

      Obeisance to you, the only one affectionate towards the world! 

      Obeisance to you who bestow happiness on everyone, obeisance! 

\textbf{21.21}

 O Govinda, forgive my sin. 

      As you were asking me very actively, I, being a wicked person, 
      told you all this out of arrogance. 

      Have pity on me, Lord of the thirty gods [instr.?]. 

\textbf{21.22}

 Vaiśampāyana spoke: 

      Keśava, the destroyer of the heroes of the enemy, was satisfied by this hymn of praise. 

      He, the great general, replied in a ... [nirupaspṛhā/spṛhayā] voice.  

\textbf{21.23}

 I am satisfied by this hymn of praise of me, dear Sir. I am vehemently trembling [with joy]. 

      I'll grant you any boon you desire even if it is something difficult to obtain in the three worlds. 

\textbf{21.24}

 [He who] praises me with this ....?  [hymn] 

      that you recited and which is fascinating because it contains the meaning of the Vedas, 

      will dwell in heaven for as many aeons 

      as the number of syllables in it. 
\textbf{21.25}

 And you should choose a boon at your pleasure, 

fearlessly, beginning from sovereignty over the three worlds. 

      Shall I grant you sovereignty over the seven-fold[?] world? 

      Or a heap of gold? Or many girls? 

\textbf{21.26}

 Hearing  the divine boons [offered] [em. to vacam?] by the imperishable one, 

      he bowed down to his lotus-feet. 

      Having recognized that Viṣṇu was being most generous, 

      with a delighted heart....[to be reconstructed] 

\textbf{21.27}

 Anarthayajña spoke: 

      I do not desire anything else as a gift, O God. 

      The  essence of bondage is without doubt one [thing??]. 

      I have been freed from this bondage by your Lordship's grace, 

      and, O Govinda, I am delighting in Dharma. 

\textbf{21.28}

 The Lord spoke: 

      The extent to which your mind has been enlightened 

      O great sage, is something even the gods have never seen, 

      [this] spotless freedom from suffering. 

      The ocean of existence has certaily been crossed. 

\textbf{21.29}

 Well, let's go now to the White Island, 

      which is hidden and is inaccessible even for the gods. 

      He who dies after his mind has been purified by devotion towards me, 

      will never again enter the dreadful ocean [of existence]. 

\textbf{21.30}

 Vaiśampāyana spoke: 

      Having spoken thus, then Hari took the great ascetic by the hand, 

      who disappeared in that moment, and with him Keśava, too. 

\textbf{21.31}

 Thus, as a consequence of the abundance of Dharma[?? in him?], 

      he [Anarthayajña] reached world of the Highest Person, 

      of the one who is the origin of all living beings, and who is imperishable, 

      the eternal and never-ending [world] of the never-decaying. 

\textbf{21.32}

 You yourself should be loyal to Keśava, 

      to Janārdana of unmeasurable heroism, 

      so that you can tread the path of that excellent Brahmin, 

      that excellent person. 

\textbf{21.33}

 What else should I teach you further, O king?

      If you have any more interest remaining,

      ask me, Sir, whatever you want 

      regarding the future or the past, anything you wish, Sir. 

\textbf{21.34}

 Janamejaya spoke: 

      How many past kalpas have passed until now? 

      How many are the future kalpas? 

      How many Indras are taught to exist with regard to each aeon? 

	Tell me one by one[???]
	  
\textbf{21.35}

 Vaiśampāyana spoke: 

      100,000 billions of Kalpas have passed so far [rājyam / rājan?]. 

	There are fourteen Indras in one Kalpa, O king

	The same [number applies to] Manvantaras per Kalpa.

      The future Kalpas are again 100,000 billion. 

\textbf{21.36}

 The first Kalpa was the Varāhakalpa. 

      Six Manvantaras have passed, O King. 

      Seventy-one four-fold [Mahā]yugas

	is the number that applies to a Manvantara.

\textbf{21.37}

 Fourteen Manvantara-periods 

      is one Kalpa, according to the sages. 

      Ten thousand Kalpas is Brahmā's day. 

      His night is the same according to the experts. 

\textbf{21.38}

 Six hundred-thousand Kalpas is called a [cosmic] month. 

      Twelve of them is called a year. 

\textbf{21.39}

 Brahmā's life is said to be that year multiplied by 100,000 billions of 
Kalpas[?].

      But even Brahmā, the Lord of the three worlds, the supreme person, is 
      taught to be transient. 
      
      Why should we grieve over the rest of the four kinds of living beings and the fate[?] of the soul?
      
      Therefore there is nothing that is untouched by the final[?] essence of the world except for eternal Śiva. 



\end{document}


